\documentclass[reqno, a4paper,11pt]{scrbook}
\usepackage[ngerman]{babel}
\usepackage{amssymb}
\usepackage{amsmath}
\usepackage{amsfonts}
\usepackage{stmaryrd}

% Move sections to new page if there is not enough space
\usepackage[nobottomtitles]{titlesec}

% smaller border
\usepackage{geometry}
\geometry{a4paper,tmargin=2cm,lmargin=2cm,rmargin=2cm}
\setlength\parskip{\smallskipamount}
\setlength\parindent{0pt}
\tolerance=900

% theorem environment
\usepackage[hyperref,amsmath,thmmarks,thref]{ntheorem}

% no italics in theorems
\theorembodyfont{}

% create index
\usepackage{index}
\newindex{default}{idx}{ind}{Vokabeln}

% define theorem environments


\theoremstyle{break}
	\newtheorem{thm}{theorem}
	\newtheorem{example}{Example}

\theoremstyle{nonumberbreak}
	\newtheorem{nnexam}{Example}

\theoremstyle{nonumberplain}
\theoremsymbol{\ensuremath{\Box}}
    \newtheorem{proof}{Proof}

%aliases for commands
\newcommand{\boundary}{\partial}

\newcommand{\RR}{% real numbers
	\ensuremath{\mathbb{R}}%
}

\newcommand{\CC}{% complex numbers
	\ensuremath{\mathbb{C}}%
}

\newcommand{\NN}{% natural numbers
	\ensuremath{\mathbb{N}}%
}

\newcommand{\ZZ}{% integers
	\ensuremath{\mathbb{Z}}%
}

\renewcommand{\phi}{% use the nice phi
	\ensuremath{\varphi}%
}
\renewcommand{\theta}{% use the nice theta
	\ensuremath{\vartheta}%
}
\newcommand{\define}{% nice Def symbol
	\ensuremath{\mathrel{\mathop:}=}%
}

\author{Jonas Fietz}

\begin{document}

\chapter{Introduction}
Setup: Let $(X, \|.\|)$ be a Banach space and $L: X \rightarrow \RR$ a functional. Furthermore, let $\phi \neg M \subset X$ be a given set or set of admissable elements. \\
Goal: Minimize $L$ over $M$. \\
Find $u_0 \in M$ such that $L[u_0] \subseteq L[u]$ $\forall u \in M$. This is equivalent to $L[u_0] = \inf_M L$. $u_0$ is called a minimum. \\
Two basic questions come to mind:
\begin{enumerate}
\item Does a minimizer exist? (not always: $M=[0,\infty), L(x) = e^{-x})$
\item If it exists, can one determine it explicitly?
\end{enumerate}

\section{Examples}
\begin{example}[Straight line]
Given $A = (a, y_a), B=(b, y_b)$. Find the curve $\gamma: [a,b] \rightarrow \RR$, $\gamma(a) = y_a$, $\gamma(b) = y_b$ of shortest length. \\

$X = \{ \gamma \in C^1[a,b] \}, \left|\gamma\right| = \max_{t \in [a,b]}{\left| \gamma(t) \right|} + \max_{t \in [a,b]}{\left| \gamma^\prime(t) \right|} $
$M = \{ \gamma \in X: \gamma(u) = y_a, \gamma(b) = y_b \}$ \\
$L[\gamma] = \text{ length of curve} \Gamma = \{ (t, \gamma(t)): t \in [a,b]\} \subset \RR^2$ $= \int_{\Gamma}{1 ds} = \int_a^b{\sqrt{1 + \gamma^{\prime 2}} dt}$ \\
Find $\gamma_0: L[\gamma_0] = \inf_M L$. \\
Let us assume that a minimizer $\gamma_0$ exists. Can we determine it? \\
Consider curves $\gamma_\epsilon (t) \define \gamma_0(t) + \epsilon \delta(t)$ where  $\delta \in X$, $\delta(a) = \delta(b) = 0$ and $\epsilon \in \RR$. Notice $\gamma_\epsilon \in M$. Define $l(\epsilon) \define L[\gamma_\epsilon]$. Then $l: \RR \rightarrow \RR$, $l$ has a minimum at $\epsilon = 0$. \\
\begin{displaymath}
l(0) = L[\gamma_0] \leq L[\gamma_\epsilon] = l(\epsilon), \epsilon \in \RR 
\end{displaymath}
Necessary condition: 
\begin{eqnarray*}
0 &=& \frac{d}{d \epsilon} l(\epsilon)|_{\epsilon = 0} = \frac{d}{d \epsilon} \int_a^b{\sqrt{1+ \gamma_\epsilon^{\prime 2}(t)} dt}  \\
 &=&  \int_a^b{\frac{d}{d \epsilon} \sqrt{1+ (\gamma_0^{\prime}(t) + \epsilon \delta^\prime (t))^2} dt} |_{\epsilon=0} \\
 &=& \int_a^b{ \frac{2 \gamma_0^{\prime}(t) \delta^\prime (t)}{2 \sqrt{1+ \gamma_0^{\prime 2}(t)}} dt} = \int_a^b{\rho (t) \delta^\prime(t) dt} \text{, where } \rho(t) \define \frac{\gamma_0^{\prime}(t)}{\sqrt{1+ \gamma_0^{\prime 2}(t)}} \\
\end{eqnarray*}
Choose $\delta(t) = \int_a^t{\delta(s)ds}- \frac{t-a}{b-a} \int_a^b{\rho(s)ds}$, $\delta \in X$, $\delta(a)=0=\delta(b)$ \\

Plug in: $0 = \int_a^b{\rho(t)(\rho(t)-c)dt}$, $c=\frac{1}{b-a} \int_a^b{\rho(s)ds} = \int_a^b{(\rho(t)-c)^2}dt$ because: \\
$c*\int_a^b{(\rho(t)-c)dt} = 0$ \\
Conclusion: $\rho(t)=c$
This implies $\gamma_0'(t)\equiv \text{const}$

So $\gamma_0$ is a straight line. Warning: We did not prove that a minimizer exists. What we proved is that if a minimizer exists, then it is the straight line.
\end{example}
\begin{example}[The law of refraction]
Two names: Fermat (1662) and Snellius (1618). Ray of light from A to B. How does it look? 
$v(x,y)$ = speed of light at $(x,y) \in \mathbb{R}$ \\
Fermat: the ray of light from A to B is such, that among all paths from A to B, it minimizes the time. \\

Paths: $\{(x,y(x)): x \in [a,b]\} = \Gamma_y$
$y \in X = C^1 [a,b ]$. $M=\{\gamma \in X: \gamma(a) = y_a, \gamma(b) = y_b\}$

$T[y] = \int^{\Gamma}{\frac{1}{v} ds} = \int_a^b{\frac{1}{v(x, y(x)} \sqrt{1+y^{\prime 2}(x)}dx}  $

Special case: \\
Change definition of $X = \{y \in ([a,b], \text{piecewise} C^1\}$

$T[y] = \int_a^b{n(x) \sqrt{1+{y^{\prime 2}(x)}dx}}$
$t(\epsilon) := T[y_0+\epsilon z]$, $y_0$ is the minimizer (assumed to exist). $z \in X$, $z(a)=z(b)=0$

$0 = \frac{d}{d\epsilon} t(\epsilon) |_{\epsilon = 0}  = \int_a^b{n(x) \frac{y_0^\prime (x) z^\prime}{\sqrt{1+{y^\prime}^2(x)}} dx}$ Define $\rho$ as before.

Choice of $z(x): z(x):=\int_a^x{\rho(\xi)d\xi} - \frac{x-a}{b-a}\int_a^b{\rho(\xi)d\xi}$ as before 
$0 = \int_a^b{\rho (x)-c}$, $\rho = \text{const}$


Therefore: $\delta_0^\prime (x) = \{ {c_1, a < x < \nu}{c_2, \nu < x < b}$ TODO FIX BIG \{
$\delta_0^\prime (x) = \{ {tan(\alpha), x\in(a,\nu)}{tan(\beta), x \in (\nu, b)}$

$\rho = \text(const) \Leftrightarrow n_1 sin(\alpha) = n_2 sin(\beta)$
Snellius-Fermat law of refraction.
\end{example}
\begin{example}[Dirichlet's principle (Dirichlet, 1805-1859)]
Dirichlet's attempt to solve: * TODO BIG \{$\Delta laplace u(x) = 0$ for $x \in \Omega$ $u(x) = f(x)$ for $x \in \delta \Omega$

$\Omega \subset \mathbb{R}^n$ open, bounded
$f: \boundary \Omega \rightarrow \mathbb{R}$ continuous

$\Delta u(x) = \sum_{i=1}^n{\frac{\partial^2 u}{\partial x^2} (x)}$

Dirichlet's idea: Find a minimizer $u_0$ of the functional 

$L[u] = \int_{\Omega}{|\nabla u(x)|^2 dx}$, $u \in M=\{v \ in C^1(\Omega-quer), v(x) = f(x) \forall (x) = f(x) \forall x \in \boundary \Omega\}$

Claim: if a minimizer exists, then it solves (*) $X = C^1(\Omega-quer)$

$\frac{d}{d \epsilon} L[u_0+\epsilon w] |_{\epsilon=0} = 0 \forall w \in X with w(x) = 0 \forall x \in \boundary \Omega$

$0 = \frac{d}{d\epsilon} L [u_0 + \epsilon w] | _{\epsilon = 0} = \frac{d}{d\epsilon} \int_{\Omega}{|\nabla u(x)|^2 dx}|_{\epsilon=0}$

$ = TODO Equations 1$

Conclusion: Solving (*) is reduced to proving existence of a minimizer of $L$ over the set $M$. In Dirichlet's time the existence of a minimizer was supposed to be ''self evident''. \\
\end{example}
\subsection{crisis in the calculus of variations}
Weierstra\ss (approx. 1870) gave an example of a well-behaved functional which did not have a minimizer. 

$X=C^1 [0,1], M=\{u\in X: u(0)=0, u(1) = 1\}$ $L [u] = \int_0^1{x^4 u'^2(x) dx}$. $L[u] >= 0 \forall u$

It exists a sequence $u_n \in M$ such that $L[u_n] \rightarrow^{n\rightarrow \infty} 0$. Hence $inf_M L = 0$.
But $\nexists u_0 ¸in M$ such that $L[u_0] = 0$. ($Ö[u_0]= 0 \Rightarrow u'_0 \equiv 0$, contradicting $u_0(0)=0$, $u_0(1)=1$

TODO Equations 2

\end{document}
