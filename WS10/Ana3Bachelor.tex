\documentclass[a4paper,twoside,DIV15,BCOR12mm,chapterprefix=true,headings=onelinechapter]{scrbook}
\usepackage{ana}

\author{Die Mitarbeiter von \url{http://mitschriebwiki.nomeata.de/}}
\title{Analysis III - Bachelorversion}
\makeindex

\begin{document}
\maketitle

\renewcommand{\thechapter}{\Roman{chapter}}
%\chapter{Inhaltsverzeichnis}
\addcontentsline{toc}{chapter}{Inhaltsverzeichnis}
\tableofcontents

\chapter{Vorwort}

\section{Über dieses Skriptum}
Dies ist ein Mitschrieb der Vorlesung \glqq Analysis III\grqq\ von Herrn Schmoeger im
Wintersemester 2010 an der Universität Karlsruhe (KIT). Die Mitschriebe der Vorlesung werden mit ausdrücklicher Genehmigung 
von Herrn Schmoeger hier veröffentlicht, Herr Schmoeger ist für den Inhalt nicht 
verantwortlich.

\section{Wer}
Gestartet wurde das Projekt von Joachim Breitner. Beteiligt an diesem Mitschrieb sind \ldots na mal schaun.

\section{Wo}
Alle Kapitel inklusive \LaTeX-Quellen können unter \url{http://mitschriebwiki.nomeata.de} abgerufen werden.
Dort ist ein \emph{Wiki} eingerichtet und von Joachim Breitner um die \LaTeX-Funktionen erweitert.
Das heißt, jeder kann Fehler nachbessern und sich an der Entwicklung
beteiligen. Auf Wunsch ist auch ein Zugang über \emph{Subversion} möglich.


\renewcommand{\thechapter}{\arabic{chapter}}
\renewcommand{\chaptername}{§}
\renewcommand*{\chapterformat}{§\,\thechapter \enskip}
\setcounter{chapter}{-1}

\chapter{Vorbereitungen}
In diesem Paragraphen seien $X,Y,Z$ Mengen ($\ne\varnothing$) und $f:X\to Y, g:Y\to Z$ Abbildungen.
\begin{enumerate}
\index{Potenzmenge}
\index{Disjunktheit}
\item 
\begin{enumerate}
\item $\mathcal{P}(X):=\{A:A\subseteq X\}$ heißt \textbf{Potenzmenge} von $X$.
\item Sei $\fm\subseteq\mathcal{P}(X)$, so heißt $\fm$ \textbf{disjunkt}, genau dann wenn $A\cap B=\varnothing$ für $A,B\in\fm$ mit $A\ne B$.
\item Sei $(A_j)$ eine Folge in $\mathcal{P}(X)$ (also $A_j\subseteq X$), so heißt $(A_j)$ \textbf{disjunkt}, genau dann wenn $\{A_1,A_2,\ldots\}$ disjunkt ist. In diesem Fall schreibe: $\dot{\bigcup}_{j=1}^\infty:=\bigcup_{j=1}^\infty A_j$\\
Allgemein sei $\bigcup_{j=1}^\infty A_j:=\bigcup A_j$ und $\bigcap_{j=1}^\infty A_j:=\bigcap A_j$.
\end{enumerate}
\item Sei $A\subseteq X$, für $x\in X$ definiere
\[\mathds{1}_A(x):=\begin{cases}1, x\in A\\ 0, x\in A^c\end{cases}\]
wobei $A^c:=X\setminus A$.
\item Sei $B\subseteq Y$ dann ist $f^{-1}(B):=\{x\in X: f(x)\in B\}$ und es gelten folgende Eigenschaften:
\begin{enumerate}
\item $f^{-1}(B^c)=f^{-1}(B)^c$
\item Ist $B_j$ eine Folge in $\mathcal{P}(Y)$, so gilt:
\begin{align*}
f^{-1}(\bigcup B_j)=\bigcup f^{-1}(B_j)\\
f^{-1}(\bigcap B_j)=\bigcap f^{-1}(B_j)\\
\end{align*}
\item Ist $C\subseteq Z$, so gilt:
\[(g\circ f)^{-1}(C)=f^{-1}(g^{-1}(C))\]
\end{enumerate}
\item $\sum_{j=1}^\infty a_j =: \sum a_j$
\end{enumerate}

\chapter{$\sigma$-Algebren und Maße}
In diesem Paragraphen sei $\varnothing\ne X$ eine Menge.

\begin{definition}
\index{$\sigma$-!Algebra}
Sei $\fa\subseteq\mathcal{P}(X)$, $\fa$ heißt eine \textbf{$\sigma$-Algebra} auf $X$, wenn gilt:
\begin{enumerate}
\item[($\sigma_1$)] $X\in\fa$
\item[($\sigma_2$)] Ist $A\in\fa$, so ist auch $A^c\in\fa$.
\item[($\sigma_3$)] Ist $(A_j)$ eine Folge in $\fa$, so ist $\bigcup A_j\in\fa$.
\end{enumerate}
\end{definition}

\begin{beispiel}
\begin{enumerate}
\item $\{X,\varnothing\}$ und $\mathcal{P}(X)$ sind $\sigma$-Algebren auf $X$.
\item Sei $A\subseteq X$, dann ist $\{X,\varnothing, A, A^c\}$ eine $\sigma$-Algebra auf $X$.
\item $\fa:=\{A\subseteq X: A$ abzählbar oder $A^c$ abzählbar$\}$ ist eine $\sigma$-Algebra auf $X$.
\end{enumerate}
\end{beispiel}

\begin{lemma}
\label{Lemma 1.1}
Sei $\fa$ eine $\sigma$-Algebra auf $X$, dann:
\begin{enumerate}
\item $\varnothing\in\fa$
\item Ist $(A_j)$ eine Folge in $\fa$, so ist $\bigcap A_j\in\fa$.
\item Sind $A_1,\ldots,A_n\in\fa$, so gilt:
\begin{enumerate}
\item $A_1\cup\cdots\cup A_n\in\fa$
\item $A_1\cap\cdots\cap A_n\in\fa$
\item $A_1\setminus A_2\in\fa$
\end{enumerate}
\end{enumerate}
\end{lemma}

\begin{beweis}
\begin{enumerate}
\item $\varnothing=X^c\in\fa$ (nach ($\sigma_2$)).
\item $D:=\bigcap A_j$. $D^c=\bigcup A_j^c\in\fa$ (nach ($\sigma_2$) und ($\sigma_3$)), also gilt auch $D=(D^c)^c\in\fa$.
\item \begin{enumerate}
\item $A_1\cup\cdots\cup A_n\in\fa$ folgt aus ($\sigma_3$) mit $A_{n+j}:=\varnothing$ ($j\ge 1$).
\item $A_1\cap\cdots\cap A_n\in\fa$ folgt aus (2) mit $A_{n+j}:=X$ ($j\ge 1$).
\item $A_1\setminus A_2=A_1\cap A_2^c\in\fa$
\end{enumerate}
\end{enumerate}
\end{beweis}

\begin{lemma}
\label{Lemma 1.2}
Sei $\varnothing\ne\cf$ eine Menge von $\sigma$-Algebren auf $X$. Dann ist 
\[\fa_0:=\bigcap_{\fa\in\cf}\fa\]
eine $\sigma$-Algebra auf $X$.
\end{lemma}

\begin{beweis}
\begin{enumerate}
\item[($\sigma_1$)] $\forall\fa\in\cf:X\in\fa\implies X\in\fa_0$.
\item[($\sigma_2$)] Sei $A\in\fa_0$, dann gilt:
\begin{align*}
\forall\fa\in\cf:A\in\fa &\implies \forall\fa\in\cf:A^c\in\fa\\
&\implies A^c\in\fa_0
\end{align*}
\item[($\sigma_3$)] Sei $(A_j)$ eine Folge in $\fa_0$, dann ist $(A_j)$ Folge in $\fa$ für alle $\fa\in\cf$, dann gilt:
\begin{align*}
\forall\fa\in\cf:\bigcap A_j\in\fa \implies \bigcap A_j\in\fa_0
\end{align*}
\end{enumerate}
\end{beweis}

\begin{definition}
\index{Erzeuger}
Sei $\varnothing\ne\mathcal{E}\subseteq\mathcal{P}(X)$ und $\cf:=\{\fa:\fa$ ist $\sigma$-Algebra auf $X$ mit $\mathcal{E}\subseteq\fa\}$. Definiere
\[\sigma(\mathcal{E}):=\bigcap_{\fa\in\cf}\fa\]
Dann ist wegen 1.2 $\sigma(\mathcal{E})$ eine $\sigma$-Algebra auf $X$. $\sigma(\mathcal{E})$ heißt die \textbf{von $\mathcal{E}$ erzeugte $\sigma$-Algebra}. $\mathcal{E}$ heißt ein \textbf{Erzeuger} von $\sigma(\mathcal{E})$.
\end{definition}

\begin{lemma}
\label{Lemma 1.3}
Sei $\varnothing\ne\mathcal{E}\subseteq\mathcal{P}(X)$.
\begin{enumerate}
\item $\mathcal{E}\subseteq\sigma(\mathcal{E})$. $\sigma(\mathcal{E})$ ist die "kleinste" $\sigma$-Algebra auf $X$, die $\mathcal{E}$ enthält.
\item Ist $\mathcal{E}$ eine $\sigma$-Algebra, so ist $\sigma(\mathcal{E})=\mathcal{E}$.
\item Ist $\mathcal{E}\subseteq\mathcal{E}'$, so ist $\sigma(\mathcal{E})\subseteq\sigma(\mathcal{E}')$.
\end{enumerate}
\end{lemma}

\begin{beweis}
\begin{enumerate}
\item Klar nach Definition.
\item $\fa:=\mathcal{E}$, dann gilt $\fa\subseteq\sigma(\mathcal{E})\subseteq\fa$.
\item $\mathcal{E}\subseteq\mathcal{E}'\subseteq\sigma(\mathcal{E}')$, also folgt nach Definition $\sigma(\mathcal{E})\subseteq\sigma(\mathcal{E}')$.
\end{enumerate}
\end{beweis}

\begin{beispiel}
\begin{enumerate}
\item Sei $A\subseteq X$ und $\mathcal{E}:=\{A\}$. Dann ist $\sigma(\mathcal{E})=\{X,\varnothing,A,A^c\}$.
\item $X:=\{1,2,3,4,5\}, \mathcal{E}:=\{\{1\},\{1,2\}\}$. Dann gilt:
\[\sigma(\mathcal{E}):=\{X,\varnothing, \{1\},\{2\},\{1,2\},\{3,4,5\},\{1,3,4,5\},\{2,3,4,5\}\}\]
\end{enumerate}
\end{beispiel}

\begin{erinnerung}
\index{Offenheit}\index{Abgeschlossenheit}
Sei $d\in\mdn, X\subseteq\mdr^d$. $A\subseteq X$ heißt \textbf{offen} (\textbf{abgeschlossen}) in $X$, genau dann wenn ein offenes (abgeschlossenes) $G\subseteq\mdr^d$ existiert mit $A=X\cap G$.\\
Beachte: $A$ abgeschlossen in $X$ $\iff$ $X\setminus A$ offen in $X$.
\end{erinnerung}

\begin{definition}
\index{Borel!$\sigma$-Algebra}\index{$\sigma$-!Algebra, Borelsche}
\index{Borel!Mengen}
Sei $X\subseteq\mdr^d$.
\begin{enumerate}
\item $\mathcal{O}(X):=\{A\subseteq X:A$ ist offen in $X\}$
\item $\fb(X):=\sigma(\mathcal{O}(X))$ heißt \textbf{Borelsche $\sigma$-Algebra} auf $X$.
\item $\fb_d:=\fb(\mdr^d)$. Die Elemente von $\fb_d$ heißen \textbf{Borelsche Mengen} oder \textbf{Borel-Mengen}.
\end{enumerate}
\end{definition}

\begin{beispiel}
\begin{enumerate}
\item Sei $X\subseteq\mdr^d$. Ist $A\subseteq$ offen (abgeschlossen) in $X$, so ist $A\in\fb(X)$.
\item Ist $A\subseteq\mdr^d$ offen (abgeschlossen) so ist $A\in\fb_d$.
\item Sei $d=1, A=\mdq$. $\mdq$ ist abzählbar, also $\mdq=\{r_1,r_2,\ldots\}$ (mit $r_i\ne r_j$ für $i\ne j$). Also ist $\mdq=\bigcup \{r_j\}$. Sei nun $r\in\mdq$, dann ist $B:=(-\infty,r)\cup(r,\infty)\in\fb_1$. Daraus folgt $\{r_j\}\in\fb_1$, also auch $\mdq\in\fb_1$.\\
Allgemeiner lässt sich zeigen: $\mdq^d:=\{(x_1,\ldots,x_n):x_j\in\mdq (j=1,\ldots,n)\}\in\fb_d$.
\end{enumerate}
\end{beispiel}

\begin{definition}
\index{Intervall}
\index{Halbraum}
\begin{enumerate}
\item Seien $I_1,\ldots,I_d$ Intervalle in $\mdr$. $I_1\times\cdots\times I_d$ heißt ein \textbf{Intervall} in $\mdr^d$.
\item Seien $a=(a_1,\ldots,a_d), b=(b_1,\ldots,b_d)\in\mdr^d$.
\[a\le b:\iff a_j\le b_j\quad (j=1,\ldots,d)\]
\item Seien $a,b\in\mdr^d$ und $a\le b$.
\begin{align*}
(a,b)&:=(a_1,b_1)\times\cdots\times(a_d,b_d)\\
(a,b]&:=(a_1,b_1]\times\cdots\times(a_d,b_d]\\
[a,b)&:=[a_1,b_1)\times\cdots\times[a_d,b_d)\\
[a,b]&:=[a_1,b_1]\times\cdots\times[a_d,b_d]
\end{align*}
mit der Festlegung $(a,b):=(a,b]:=[a,b):=\varnothing$, falls $a_j=b_j$ für ein $j\in\{1,\ldots,d\}$.
\item Für $k\in\{1,\ldots,d\}$ und $\alpha\in\mdr$ definiere die folgenden \textbf{Halbräume}:
\begin{align*}
H_k^-(\alpha):=\{(x_1,\ldots,x_d)\in\mdr^d:x_k\le\alpha\}\\
H_k^+(\alpha):=\{(x_1,\ldots,x_d)\in\mdr^d:x_k\ge\alpha\}
\end{align*}
\end{enumerate}
\end{definition}

\begin{satz}[Erzeuger der Borelschen $\sigma$-Algebra auf $\mdr$]
\label{Satz 1.4}
Es seien $\ce_1,\ce_2,\ce_3$ wie folgt definiert:
\begin{align*}
\ce_1&:=\{(a,b):a,b\in\mdq^d,a\le b\}\\
\ce_2&:=\{(a,b]:a,b\in\mdq^d, a\le b\}\\
\ce_3&:=\{H^-_k(\alpha):\alpha\in\mdq, k=1,\ldots,d\}
\end{align*}
Dann gilt:
\[\fb_d=\sigma(\ce_1)=\sigma(\ce_2)=\sigma(\ce_3)\]
Entsprechendes gilt für die anderen Typen von Intervallen und Halbräumen.
\end{satz}

\begin{beweis}
\begin{enumerate}
\item Sei $G\in\co(\mdr^d), \fm:=\{(a,b):a,b\in\mdq^d,a\le b, (a,b)\subseteq G\}$. Dann ist $\fm$ abzählbar und $G=\bigcup_{I\in\fm}I$. also gilt:
\[G\in\sigma(\ce_1)\implies \fb_d=\sigma(\co(\mdr^d))\subseteq\sigma(\ce_1)\]
\item Sei $(a,b)\in\ce_1$.\\
\textbf{Fall 1:} $(a,b)=\varnothing\in\ce_2\subseteq\sigma(\ce_2)$\\
\textbf{Fall 2:} $(a,b)\ne\varnothing, a=(a_1\ldots,a_d), b=(b_1\ldots,b_d)$. Dann gilt für alle $j\in\{1,\ldots,d\}:a_j<b_j$, also gilt auch:
\[\exists N\in\mdn:\forall n\ge N: \forall j\in\{1,\ldots,d\}:a_j<b_j-\frac1n\]
Definiere $c_n:=(\frac1n,\ldots,\frac1n)\in\mdq^d$. Dann gilt:
\[(a,b)=\bigcup_{n\ge N}(a,b-c_n]\in\sigma(\ce_2)\]
Also auch $\ce_1\subseteq\sigma(\ce_2)$ und damit $\sigma(\ce_1)\subseteq\sigma(\ce_2)$.
\item Seien $a = (a_1,\ldots,a_d), b=(b_1,\ldots,b_d) \in \mdq^d$ mit $a \leq b$. Nachrechnen:
\[(a,b] = \bigcap_{k=1}^d (H^-_k(b_k) \cap H^-_k(a_k)^c) \in \sigma(\ce_3). \]
Das heißt $\ce_2 \subseteq \sigma(\ce_3)$ und damit auch $\sigma(\ce_2) \subseteq \sigma(\ce_3)$. 
\item $H^-_k(\alpha)$ ist abgeschlossen, somit ist $H^-_k(\alpha)^c$ offen und damit $H^-_k(\alpha)^c \in \fb_d$, also auch $H^-_k(\alpha) \in \fb_d$. Damit ist $\ce_3 \subseteq \fb_d \implies \sigma(\ce_3) \subseteq \fb_d$. 
\end{enumerate}
\end{beweis}

\begin{definition}
\index{Spur}
Sei $\varnothing \neq \fm \subseteq \mathcal{P}(X)$ und $\varnothing \neq Y \subseteq X$. 
\[\fm_Y := \{A \cap Y : A \in \fm\}\] 
heißt die \textbf{Spur von $\fm$ in $Y$}.
\end{definition}

\begin{satz}[Spuren und $\sigma$-Algebren]
\label{Satz 1.5}
Sei $\varnothing \neq Y \subseteq X$ und $\fa$ eine $\sigma$-Algebra auf $X$.
\begin{enumerate}
\item $\fa_Y$ ist eine $\sigma$-Algebra auf $Y$.
\item $\fa_Y \subseteq \fa \iff Y \in \fa$
\item Ist $\varnothing \neq \ce \subseteq \mathcal{P}(Y)$, so ist $\sigma(\ce_Y) = \sigma(\ce)_Y$.
\end{enumerate}
\end{satz}

\begin{beweis}
\begin{enumerate}
\item \begin{enumerate}
\item[($\sigma_1$)] Es ist $Y=Y\cap X\in\fa_Y$, da $X\in\fa$.
\item[($\sigma_2$)] Sei $B\in\fa_Y$, dann existiert ein $A\in\fa$ mit $B=A\cap Y$. Also ist $Y\setminus B=(X\setminus A)\cap Y\in\fa_Y$, da $X\setminus A\in\fa$ ist.
\item[($\sigma_3$)] Sei $(B_j)$ eine Folge in $\fa_Y$, dann existiert eine Folge $(A_j)\in\fa^\mdn$ mit $B_j=A_j\cap Y$. Es gilt:
\[\bigcup B_j=\bigcup(A_j\cap Y)=(\bigcup A_j)\cap Y\in\fa_Y\]
\end{enumerate}
\item Der Beweis erfolgt durch Implikation in beiden Richtungen:
\begin{enumerate}
\item["`$\implies$"'] Es gilt $Y\in\fa_Y\subseteq\fa$.
\item["`$\impliedby$"'] Sei $B\in\fa_Y$, dann existiert ein $A\in\fa$ mit $B=A\cap Y\in\fa$.
\end{enumerate}
\item Es gilt:
\begin{align*}
\ce\subseteq\sigma(\ce)&\implies\ce_Y\subseteq\sigma(\ce)_Y\\
&\implies\sigma(\ce_Y)\subseteq\sigma(\ce)_Y
\end{align*}
Sei nun:
\[\cd:=\{A\subseteq X:A\cap Y\in\sigma(\ce_Y)\}\]
Übung: $\cd$ ist eine $\sigma$-Algebra auf $X$.\\
Sei $E\in\ce$ dann ist $E\cap Y\in\ce_Y\subseteq\sigma(\ce_Y)$ also $E\in\cd$ und damit $\ce\subseteq\cd$. Daraus folgt:
\begin{align*}
\sigma(\ce)_Y&\subseteq\sigma(\cd)_Y=\cd_Y=\{A\cap Y:A\in\cd\}\\
&\subseteq\sigma(\ce_Y)
\end{align*}
\end{enumerate}
\end{beweis}

\begin{folgerungen}
Sei $X\subseteq\mdr^d$. Dann gilt:
\begin{enumerate}
\item $\fb(X)=(\fb_d)_X$
\item Ist $X\in\fb_d$, so ist $\fb(X)=\{A\in\fb_d:A\subseteq X\}\subseteq\fb_d$.
\end{enumerate}
\end{folgerungen}

\begin{definition}
Wir fügen $\mdr$ das Symbol $+\infty$ hinzu. Es soll gelten:
\begin{enumerate}
\item $\forall a\in\mdr:a<+\infty$
\item $\pm a+(+\infty):=+\infty=:(+\infty)\pm a$
\item $(+\infty)+(+\infty):=+\infty$
\end{enumerate}
Sei etwa $[0,+\infty]:=[0,\infty)\cup\{+\infty\}$.
\begin{enumerate}
\item Sei $(x_n)$ eine Folge in $[0,+\infty]$. Es gilt:
\[x_n\stackrel{n\to\infty}{\to}\infty:\iff \forall c>0\exists n_c\in\mdn:\forall n\ge n_c: x_n> c\]
\item Sei $(a_n)$ eine Folge in $[0,+\infty]$. Es gilt
\[\sum_{n=1}^\infty a_n=\sum a_n = +\infty\]
genau dann wenn $a_j=+\infty$ für ein $j\in\mdn$ oder, falls alle $a_j<+\infty$, wenn $\sum a_n$ divergiert.
\end{enumerate} 
Wegen 13.1 Ana I können Reihen der obigen Form beliebig umgeordnet werden, ohne dass sich ihr Wert verändert.
\end{definition}

\begin{definition}
\index{Maß}
\index{$\sigma$-!Additivität}
\index{Maßraum}
\index{Maß!endliches}
\index{Wahrscheinlichkeitsmaß}\index{Maß!Wahrscheinlichkeits-}
Sei $\fa$ eine $\sigma$-Algebra auf $X$ und $\mu:\fa\to[0,+\infty]$ eine Abbildung. $\mu$ heißt ein \textbf{Maß} auf $\fa$, genau dann wenn gilt:
\begin{enumerate}
\item[$(M_1)$] $\mu(\varnothing)=0$
\item[$(M_2)$] Ist $(A_j)$ eine disjunkte Folge in $\fa$, so ist $\mu(\bigcup A_j)=\sum\mu(A_j)$. Diese Eigenschaft heißt \textbf{$\sigma$-Additivität}.
\end{enumerate}
Ist $\mu$ ein Maß auf $\fa$, so heißt $(X,\fa,\mu)$ ein \textbf{Maßraum}.\\
Ein Maß $\mu$ heißt \textbf{endlich}, genau dann wenn $\mu(X)<\infty$. Ein Maß $\mu$ heißt ein \textbf{Wahrscheinlichkeitsmaß}, genau dann wenn $\mu(X)=1$ ist.
\end{definition}

\begin{beispiel}
\index{Punktmaß}\index{Maß!Punkt-}
\index{Dirac-Maß}\index{Maß!Dirac-}
\index{Zählmaß}\index{Maß!Zähl-}
\begin{enumerate}
\item Sei $\fa=\cp(X)$ und $x_0\in X$. $\delta_{x_0}:\fa\to[0,+\infty]$ sei definiert durch:
\[\delta_{x_0}(A):=
\begin{cases}
1,\ x_0\in A\\
0,\ x_0\not\in A
\end{cases}\]
Klar ist, dass $\delta_{x_0}(\varnothing)=0$ ist.\\
Sei $(A_j)$ eine disjunkte Folge in $\fa$.
\[\delta_{x_0}(\bigcup A_j)=
\left.\begin{cases}
1,\ x_0\in\bigcup A_j\\
0,\ x_0\not\in\bigcup A_j
\end{cases}\right\}=\sum\mu(A_j)\]
$\delta_{x_0}$ ist ein Maß auf $\cp(X)$ und heißt \textbf{Punktmaß} oder \textbf{Dirac-Maß}.
\item Sei $X:=\mdn$, $\fa:=\cp(X)$ und $(p_j)$ eine Folge in $[0,+\infty]$. Definiere $\mu:\fa\to[0,+\infty]$ durch:
\begin{align*}
\mu(A):=
\begin{cases}
0&,A=\varnothing\\
\sum_{j\in A}p_j&,A\ne\varnothing
\end{cases}
\end{align*}
Übung: $\mu$ ist ein Maß auf $\fa=\cp(\mdn)$ und heißt ein \textbf{Zählmaß}. Sind alle $p_j=1$, so ist $\mu(A)$ gerade die Anzahl der Elemente von $A$.
\item Sei $(X,\fa,\mu)$ ein Maßraum, $\varnothing\ne Y\subseteq X$ und $\fa_0\subseteq\fa$ eine $\sigma$-Algebra auf $Y$. Definiere $\mu_0:\fa_0\to[0,+\infty]$ durch $\mu_0(A):=\mu(A)$ ($A\in\fa_0$). Dann ist $(Y,\fa_0,\mu_0)$ ein Maßraum.\\
Ist spezieller $Y\in\fa$, so ist $\fa_0:=\fa_Y\subseteq\fa$ und man definiert $\mu_{|Y}:\fa_Y\to[0,+\infty]$ durch $\mu_{|Y}(A)=\mu(A)$.
\end{enumerate}
\end{beispiel}

\begin{satz}
\label{Satz 1.7}
\((X,\fa,\mu)\) sei ein Ma\ss raum, es seien \(A,B\in\fa\) und \((A_{j})\) sei eine Folge in \(\fa\). Dann:
\begin{enumerate}
\item \(A\subseteq B\,\implies\,\mu(A)\leq\mu(B)\)
\item Ist \(\mu(A)<\infty\) und \(A\subseteq B,\implies\,\mu(B\setminus A)=\mu(B)-\mu(A)\)
\item Ist \(\mu\) endlich, dann ist \(\mu(A)<\infty\) und \(\mu(A^{c})=\mu(X)-\mu(A)\)
\item \(\mu\left(\bigcup A_{j}\right)\leq\sum{\mu(A_{j})}\) (\(\sigma\)-Subadditivit\"at)
\item Ist \(A_{1}\subseteq A_{2}\subseteq A_{3}\subseteq\cdots\), so ist \(\mu(\bigcup A_{j})=\lim_{n\to\infty}{\mu(A_{n})}\)
\item Ist \(A_{1}\supseteq A_{2}\supseteq A_{3}\supseteq\cdots\) und \(\mu(A)<\infty\), so ist
	\(\mu(\bigcap A_{j})=\lim_{n\to\infty}{\mu(A_{n})}\)
\end{enumerate}
\end{satz}
\begin{beweis}
\begin{enumerate}
% Eigentlich muesste es in folgender Zeile statt B=(B\setminus A)\cup A korrekt 
% heissen: B=(B\setminus A)\cupdot A -- Spaeter...
\item[(1)-(3)] \(B=(B\setminus A)\cup A\). Dann: \(\mu(B)=\underbrace{\mu(B\setminus A)}_{\geq0}+\mu(A)\geq\mu(A)\)
\item[(4)] % Das muesste jetzt eigentlich Punkt 4 sein...
\(B_{1}=A_{1},\,B_{k}:=A_{k}\setminus\bigcup_{j=1}^{k-1}{A_{j}}\quad(k\geq 2)\)

Dann: \(B_{j}\in\fa,\,B_{j}\subseteq A_{j}\,(j\in\MdN);\,(B_{j})\) disjunkt und \(\bigcup A_{j}=\bigcup B_{j}\). Dann:
\[
\mu\left(\bigcup A_{j}\right)=\mu\left(\bigcup B_{j}\right)=\sum{\underbrace{\mu(B_{j})}_{\leq\mu(A_{j})}}\leq\sum{\mu(A_{j})}
\]
\item[(5)] % Das muesste jetzt eigentlich Punkt 5 sein...
\(B_{1}=A_{1},\,B_{k}=A_{k}\setminus A_{k-1}\,(k\geq 2)\)

Dann: \(B_{j}\subseteq\fa;\,B_{j}\subseteq A_{j}\,(j\in\MdN);\,\bigcup A_{j}=\bigcup B_{j}\) und \(A_{n}=\bigcup_{j=1}^{n}{B_{j}}\)%\bigcupdot_{j=1}^{n}{B_{j}}\)

Dann: \(\mu(\bigcup A_{j})=\mu(\bigcup B_{j})=\sum{\mu(B_{j})}=\lim_{n\to\infty}{\underbrace{\sum_{j=1}^{n}{\mu(B_{j})}}_{=\mu\left(\bigcup_{j=1}^{n}{B_{j}}\right)=\mu(A_{n})}}\)
\item[(6)] \"Ubung
\end{enumerate}
\end{beweis}

\chapter{Das Lebesguema\ss}
\index{Lebesguemaß}
In diesem Kapitel sei \(X\) eine Menge, \(X\neq\varnothing\).
\begin{definition}
Sei \(\varnothing\neq\mathfrak{R}\subseteq\mathcal{P}(X)\). \(\mathfrak{R}\)
hei\ss t ein Ring (auf \(X\)), genau dann wenn gilt:
\begin{enumerate}
\item \(\varnothing\in\mathfrak{R}\)
\item \(A,B\in\mathfrak{R}\,\implies\,A\cup B,\,B\setminus A\in\mathfrak{R}\)
\end{enumerate}
\end{definition}
\begin{definition}
Sei \(d\in\MdN\).
\begin{enumerate}
\item \(I_{d}:=\{(a,b]\mid a,b\in\MdR^{d},\,a\leq b\}\).
Seien \(a=(a_{1},\ldots,a_{d}),\,b=(b_{1},\ldots,b_{d})\in\MdR^d\) und \(I:=(a,b]\in I_{d}\)
\[
\lambda_{d}(I)=\begin{cases}0&\text{falls }I=\varnothing\\(b_{1}-a_{1})(b_{2}-a_{2})\cdots(b_{d}-a_{d})&\text{falls }I\neq\varnothing\end{cases}\quad\text{(Elementarvolumen)}
\]
\item \(\cf_d:=\left\{\bigcup_{j=1}^{n}I_{j}\mid n\in\MdN,\,I_{1},\ldots,I_{n}\in I_{d}\right\}\) (Menge der Figuren)
\end{enumerate}
\end{definition}
Ziel dieses Kapitels: Fortsetzung von \(\lambda_{d}\) auf \(\cf_{d}\) und dann auf \(\fb_d\) (\(\leadsto\) Lebesguema\ss)

Beachte: \(\ci_{d}\subseteq\cf_{d}\subseteq\fb_{d}\overset{1.4}{\implies}\fb_{d}=\sigma(\ci_{d})=\sigma(\cf_{d})\)
\begin{lemma}
\label{Lemma 2.1}
Seien \(I,I'\in\ci_{d}\) und \(A\in\cf_{d}\). Dann:
\begin{enumerate}
\item \(I\cap I'\in\ci_{d}\)
\item \(I\setminus I'\in\cf_{d}.\) Genauer: \(\exists\left\{I_{1}',\ldots,I_{l}'\right\}\subseteq\ci_{d}\) disjunkt:
\(I\setminus I'=\bigcup_{j=1}^{l}{I_{j}'}\) % \bigcupdot
\item \(\exists\left\{I_{1}',\ldots,I_{l}'\right\}\subseteq\ci_{d}\) disjunkt: \(A=\bigcup_{j=1}^{l}{I_{j}'}\)
\item \(\cf_d\) ist ein Ring.
\end{enumerate}
\end{lemma}
\begin{beweis}
\begin{enumerate}
\item Sei \(I=\prod_{k=1}^{d}{(a_{k},b_{k}]},\,I'=\prod_{k=1}^{d}{(\alpha_{k},\beta_{k}]};\,\alpha_{k}':=\max\{\alpha_{k},a_{k}\},\,\beta_{k}':=\min\{\beta_{k},b_{k}\}\)

Ist \(\alpha_{k}'\geq\beta_{k}'\) f\"ur ein \(k\in\{1,\ldots,d\}\), so ist \(I\cap I'=\varnothing\in\ci_{d}\).
Sei \(\alpha_{k}'<\beta_{k}'\forall k\in\{1,\ldots,d\}\), so ist \(I\cap I'=\prod_{k=1}^{d}{(\alpha_{k}',\beta_{k}']\in\ci_{d}}\)
\item Induktion nach \(n\):
\begin{itemize}
\item[I.A.] Klar \checkmark % hier fehlt noch eine Graphik
\item[I.V.] Die Behauptung gelte f\"ur ein \(d\geq 1\)
\item[I.S.] Seien \(I,I'\in\ci_{d+1}\). Es existieren \(I_{1},I_{1}'\in\ci_{1}\) und \(I_{2},I_{2}'\in\ci_{d}\) mit:
\(I=I_{1}\times I_{2},\,I'=I_{1}'\times I_{2}'\)
% Graphik einfuegen!

Nachrechnen: 
\[
I\setminus I'=(I_{1}\setminus I_{1}')\times I_{2}\cup(I_{1}\cap I_{1}')\times(I_{2}\setminus I_{2}')
\]
I.A.\(\implies\,I_{1}\setminus I_{1}'=\) endliche disjunkte Vereinigung von Elementen aus \(\ci_{1}\)\\
I.V.\(\implies\,I_{2}\setminus I_{2}'=\) endliche disjunkte Vereinigung von Elementen aus \(\ci_{d}\)\\
Daraus folgt die Behauptung f\"ur \(d+1\)
\end{itemize}
\item Wir zeigen mit Induktion nach \(n\): ist \(A=\bigcup_{j=1}^{n}{I_{j}}\) mit \(I_{1},\ldots,I_{d}\in\ci_{d}\), so
existiert \(\{I_{1}',\ldots,I_{l}'\}\subseteq\ci_{d}\) disjunkt: 
\(A=\bigcup_{j=1}^{l}{I_{j}'}\)
\begin{itemize}
\item[I.A.] \(n=1:\,A=I_{1}\)\checkmark
\item[I.V.] Die Behauptung gelte f\"ur ein \(n\geq 1\)
\item[I.S.] Sei \(A=\bigcup_{j=1}^{n+1}{I_{j}}\quad(I_{1},\ldots,I_{n+1}\in\ci_{d})\)

IV\(\,\implies\,\exists\{I_{1}',\ldots,I_{l}'\}\subseteq\ci_{d}\) disjunkt:
\(\bigcup_{j=1}^{n}{I_{j}}=\bigcup_{j=1}^{l}{I_{j}'}\)	% \bigcupdot...

Dann: \(A=I_{n+1}\cup\bigcup_{j=1}^{l}{I_{j}'}=I_{n+1}\cup\bigcup_{j=1}^{l}{(I_{j}'\setminus I_{n+1})}\) % \cupdot...

Wende (2) auf jedes \(I_{j}'\setminus I_{n+1}\) an \((j=1,\ldots,l)\): 
\(I_{j}'\setminus I_{n+1}=\bigcup_{j=1}^{l_{j}}{I_{j}''}\quad(I_{j}''\in\ci_{d})\)

Damit folgt:
\[
A=I_{n+1}\cup\bigcup_{j=1}^{l}{\left(\bigcup_{j=1}^{l_{j}}{I_{j}''}\right)}
\]
Daraus folgt die Behauptung f\"ur \(n+1\).
\end{itemize}
\item \((a,a]=\varnothing\implies\varnothing\in\cf_{d}\)

Seien \(A,B\in\cf_{d}\). Klar: \(A\cup B\in\cf_{d}\)

Sei \(A=\bigcup_{j=1}^{n}{I_{j}},\,B=\bigcup_{j=1}^{n}{I_{j}'}\quad(I_{j},I_{j}'\in\ci_{d})\). Zu zeigen: \(B\setminus A\in\cf_{d}\)
\begin{itemize}
\item[I.A.] \(n=1:\,A=I_{1}\implies B\setminus A=\bigcup_{j=1}^{n}(\underbrace{I_{j}'\setminus I_{j}}_{\in\cf_{d}}\). Wende
(2) auf jedes \(I_{j}'\setminus I_{1}\) an. Aus (2) folgt dann \(B\setminus A\in\cf_{d}\).
\item[I.V.] Die Behauptung gelte f\"ur ein \(n\in\MdN\)
\item[I.S.] Sei \(A'=A\cup I_{n+1}\quad(I_{n+1}\in\ci_{d})\). Dann:
\[
B\setminus A'=\underbrace{(B\setminus A)}_{\in\cf_{d}}\setminus\underbrace{I_{n+1}}_{\in\cf_{d}}\in\cf_{d}
\]
\end{itemize}
\end{enumerate}
\end{beweis}

\begin{lemma}
\label{Lemma 2.2}
Sei \(A\in\cf_{d}\) und \(\{I_{1},\ldots,I_{n}\}\subseteq\ci_{d}\) disjunkt und
\(\{I_{1}',\ldots,I_{m}'\}\subseteq\ci_{d}\) und 
\(\bigcup_{j=1}^{n}{I_{j}}=\bigcup_{j=1}^{m}{I_{j}'}\). Dann:
\[
\sum_{j=1}^{n}{\lambda_{d}(I_{j})}=\sum_{j=1}^{m}{\lambda_{d}(I_{j}')}
\]
\end{lemma}
\begin{definition}
Sei \(A\in\cf_{d}\) und \(A=\bigcup_{j=1}^{n}{I_{j}}\) mit \(\{I_{1},\ldots,I_{n}\}\subseteq\ci_{d}\)
disjunkt (beachte Lemma \ref{Lemma 2.1}, Punkt 3).
\[
\lambda_{d}(A):=\sum_{j=1}^{n}{\lambda_{d}(I_{j})}
\]
Wegen Lemma \ref{Lemma 2.2} ist \(\lambda_{d}:\cf_{d}\to[0,\infty)\)
wohldefiniert.
\end{definition}
\begin{satz}
\label{Satz 2.3}
Seien \(A,B\in\cf_{d}\) und \((B_{n})\) sei eine Folge in \(\cf_{d}\).
\begin{enumerate}
\item \(A\cap B=\varnothing\implies\lambda_{d}(A\cup B)=\lambda_{d}(A)+\lambda_{d}(B)\)
\item \(A\subseteq B\implies\lambda_{d}(A)\leq\lambda_{d}(B)\)
\item \(\lambda_{d}(A\cup B)\leq\lambda_{d}(A)+\lambda_{d}(B)\)
\item Sei \(\delta>0\). Es existiert \(C\in\cf_{d}:\overline{C}\subseteq B\) und
\(\lambda_{d}(B\setminus C)\leq\delta\).
\item Ist \(B_{n+1}\subseteq B_{n}\forall n\in\mdn\) und \(\bigcap B_{n}=\varnothing\), so gilt: \(\lambda_{d}(B_{n})\to 0\,(n\to \infty)\)
\end{enumerate}
\end{satz}

\begin{beweis}
\begin{enumerate}
\item Aus Lemma \ref{Lemma 2.1} folgt: Es existiert 
\(\{I_{1},\ldots,I_{n}\}\subseteq\ci_{d}\)
disjunkt und es existiert \(\{I_{1}',\ldots,I_{m}'\}\subseteq\ci_{d}\) disjunkt:
\(A=\bigcup_{j=1}^{n}{I_{j}},\,B=\bigcup_{j=1}^{m}{I_{j}'}\).

\(J:=\{I_{1},\ldots,I_{n},I_{1}',\ldots,I_{m}'\}\subseteq\ci_{d}\). Aus 
\(A\cap B=\varnothing\) folgt: \(J\) ist disjunkt. Dann: 
\(A\cup B=\bigcup_{I\in J}{I}\)	% Hier auch wieder: \bigcupdot

Also:
\begin{align*}
\lambda_{d}(A\cup B)&=\sum_{I\in J}{\lambda_{d}(I)}\\
    &=\sum_{j=1}^{n}{\lambda_{d}(I_{j})}+\sum_{j=1}^{m}{\lambda_{d}(I_{j}')}\\
    &=\lambda_{d}(A)+\lambda_{d}(B)
\end{align*}
\item wie bei Satz \ref{Satz 1.7}
\item \(\lambda_{d}(A\cup B)=\lambda(A\cup(B\setminus A))\overset{(1)}{=}\lambda_{d}(A)+\lambda_{d}(B\setminus A)\overset{(2)}{\leq}\lambda_{d}(A)+\lambda_{d}(B)\) % \cupdot...
\item \"Ubung; es gen\"ugt zu betrachten: \(B\in\ci_{d}\) % Graphik einfuegen
\item Sei \(\varepsilon>0\). Aus (4) folgt: Zu jedem \(B_{n}\) existiert ein
\(C_{n}\in\cf_{d}:\overline{C}_{n}\subseteq B_{n}\) und
\begin{equation}
\label{eq: Abschaetzung Mass -- Beweis Satz 2.3.(5)}
\lambda_{d}(B_{n}\setminus C_{n})\leq\frac{\varepsilon}{2^{n}}
\end{equation}
Dann:
\(\bigcap{\overline{C}_{n}}\subseteq\bigcap{B_{n}}=\varnothing\implies\bigcup{\overline{C}_{n}^{c}}=\mdr^{d}\implies\underbrace{\overline{B}_{1}}_{\text{kompakt}}\subseteq\bigcup{\underbrace{\overline{C}_{n}^{c}}_{\text{offen}}}\)

Aus der Definition von Kompaktheit (Analysis II, \S 2) folgt:
\(\exists m\in\mdn:\,\bigcup_{j=1}^{m}{\overline{C}_{j}^{c}}\supseteq\overline{B}_{1}\)
Dann: \(\bigcap_{j=1}^{m}{\overline{C}_{j}}\subseteq\overline{B}_{1}^{c}\).
Andererseits: \(\bigcap_{j=1}^{m}{\overline{C}_{j}}\subseteq\bigcap_{j=1}^{m}{B_{j}}\subseteq B_{1}\subseteq\overline{B}_{1}\). 

Also: \(\bigcap_{j=1}^{m}{\overline{C}_{j}}=\varnothing\). Das hei\ss t:
\(\bigcap_{j=1}^{n}{\overline{C}_{j}}=\varnothing\,\forall n\geq m\)

\(D_{n}:=\bigcap_{j=1}^{n}{C_{j}}\). Dann: \(D_{n}=\varnothing\,\forall n\geq m\)

\textbf{Behauptung:} \(\lambda_{d}(B_{n}\setminus D_{n})\leq\left(1-\frac{1}{2^{n}}\right)\ep\,\forall n\in\mdn\)
\begin{beweis}
\begin{itemize}
\item[I.A.] \(\lambda_{d}(B_{1}\setminus D_{1})=\lambda_{d}(B_{1}\setminus C_{1})\overset{\eqref{eq: Abschaetzung Mass -- Beweis Satz 2.3.(5)}}{\leq}\frac{\ep}{2}=\left(1-\frac{1}{2}\right)\ep\) \checkmark
\item[I.V.] Die Behauptung gelte f\"ur ein \(n\in\mdn\).
\item[I.S.] \begin{align*}
    \lambda_{d}(B_{n+1}\setminus D_{n+1})&=\lambda_{d}\left((B_{n+1}\setminus D_{n})\cup(B_{n+1}\setminus C_{n+1})\right)\\
    &\overset{(3)}{\leq}\lambda_{d}(\underbrace{B_{n+1}\setminus D}_{\subseteq B_{n}\setminus D_{n}}+\underbrace{\lambda_{d}(B_{n+1}\setminus C_{n+1})}_{\overset{\eqref{eq: Abschaetzung Mass -- Beweis Satz 2.3.(5)}}{\leq}\frac{\ep}{2^{n+1}}}\\
    &\overset{(2)}{\leq}\lambda_{d}(B_{n}\setminus D_{n})+\frac{\ep}{2^{n+1}}\\
    &\overset{\text{I.V.}}{\leq}(\left(1-\frac{1}{2^{n}}\right)+\frac{\ep}{2^{n+1}}\\
&=\left(1-\frac{1}{2^{n+1}}\right)\ep
    \end{align*}
\end{itemize}
\end{beweis}

F\"ur \(n\geq m:\,D_{n}=\varnothing\,\implies\,\lambda_{d}(B_{n})=\lambda_{d}(B_{n}\setminus D_{n})\leq\left(1-\frac{1}{2^{n}}\right)\varepsilon\leq\varepsilon\)
\end{enumerate}
\end{beweis}

\begin{definition}
%\index{Prämass}
Es sei \(\fr\) ein Ring auf \(X\). Eine Abbildung \(\mu:\fr\to[0,\infty]\) 
hei\ss t ein Pr\"ama\ss \ auf \(\fr\), wenn gilt:
\begin{enumerate}
\item \(\mu(\varnothing)=0\)
\item Ist \(A_{j}\) eine disjunkte Folge in \(\fr\) und \(\bigcup{A_{j}}\in\fr\), so ist \(\mu\left(\bigcup{A_{j}}\right)=\sum{\mu(A_{j})}\).
\end{enumerate}
\end{definition}

\begin{satz}
\label{Satz 2.4}
\(\lambda_{d}:\cf_{d}\to[0,\infty]\) ist ein Pr\"ama\ss .
\end{satz}
\begin{beweis}
\begin{enumerate}
\item Klar: \(\lambda_{d}(\varnothing)=0\)
\item Sei \(A_{j}\) eine disjunkte Folge in \(\cf_{d}\) und \(A:=\bigcup{A_{j}}\in\cf_{d}\).

\(B_{n}:=\bigcup_{j=1}^{\infty}{A_{j}}\,(n\in\mdn)\); \((B_{n})\) hat die
Eigenschaften aus \ref{Satz 2.3}, Punkt 5. Also: \(\lambda_{d}(B_{n})\to 0\).

F\"ur \(n\geq 2\):
\[
\lambda_{d}(A)=\lambda_{d}(A_{1}\cup\cdots\cup A_{n-1}\cup B_{n})\overset{\ref{Satz 2.3}.(1)}{=}\sum_{j=1}^{n-1}{\lambda_{d}(A_{j})}+\lambda_{d}(B_{n})
\]
Daraus folgt: 
\[
\sum_{j=1}^{n-1}{\lambda_{d}(A_{j})}=\lambda_{d}(A)-\lambda_{d}(B_{n})\quad\forall n\geq 2
\]
Mit \(n\to\infty\) folgt die Behauptung.
\end{enumerate}
\end{beweis}

\begin{satz}[Fortsetzungssatz von Carath\'eodory]
\label{Satz 2.5}
Sei \(\fr\) ein Ring auf \(X\) und \(\mu:\fr\to[0,\infty]\) ein Pr\"ama\ss. Dann
existiert ein Ma\ss raum \((X,\fa(\mu),\overline{\mu})\) mit
\begin{enumerate}
\item \(\sigma(\fr)\subseteq\fa(\mu)\)
\item \(\overline{\mu}(A)=\mu(A)\,\forall A\in\fr\)
\end{enumerate}
Insbesondere: \(\overline{\mu}\) ist ein Ma\ss \ auf \(\sigma(\fr)\).
\end{satz}

\begin{satz}[Eindeutigkeitssatz]
\label{Satz 2.6}
Sei \(\varnothing\neq\ce\subseteq\cp(X)\), es seien \(\nu,\,\mu\) Ma\ss e auf
\(\sigma(\ce)\) und es gelte: \(\mu(E)=\nu(E)\,\forall E\in\ce\).

Weiter gelten:
\begin{enumerate}
\item \(E,F\in\ce\implies E\cap F\in\ce\quad\text{(durchschnittstabil)}\)
\item Es existiert eine Folge \((E_{n})\) in \(\ce\): \(\bigcup{E_{n}}=X\) und
\(\mu(E_{n})<\infty\forall n\in\mdn\).
\end{enumerate}
Dann: \(\mu=\nu\) auf \(\sigma(\ce)\).
\end{satz}

\begin{satz}%[Lebesguema\ss]
\label{Satz 2.7}
\index{Lebesguemaß}
Es gibt genau eine Fortsetzung von \(\lambda_{d}:\cf_{d}\to[0,\infty]\) auf
\(\fb_{d}\) zu einem Ma\ss. Diese Fortsetzung hei\ss t \textbf{Lebesguemaß} \ (L-Ma\ss)
und wird ebenfalls mit \(\lambda_{d}\) bezeichnet.
\end{satz}
\begin{beweis}
Aus Lemma \ref{Lemma 2.1} und Satz \ref{Satz 2.4} folgt: \(\lambda_{d}\) ist ein
Pr\"ama\ss \ auf \(\fr:=\cf_{d}\); es ist \(\sigma(\fr)=\fb_{d}\).

Aus Satz \ref{Satz 2.5} folgt: \(\lambda_{d}\) kann zu einem Ma\ss \ auf 
\(\fb_{d}\) fortgesetzt werden.

Sei \(\nu\) ein weiteres Ma\ss \ auf \(\fb_{d}\) mit: 
\(\nu(A)=\lambda_{d}(A)\,\forall A\in\cf_{d}\). \(\ce:=\ci_{d}\). Dann:
\(\sigma(\ce)\overset{\ref{Satz 1.4}}{=}\fb_{d}\).
\begin{enumerate}
\item \(E,F\in\ce\overset{\ref{Lemma 2.1}}{\implies}E\cap F\in\ce\)
\item \(E_{n}:=(-n,n]^{d}\)

Klar: 
\begin{align*}
\bigcup E_{n}&=\mdr^{d}\\
\lambda_{d}(E_{n})&=(2n)^{d}<\infty
\end{align*}
\end{enumerate}
Klar: \(\nu(E)=\lambda_{d}(E)\,\forall E\in\ce\). Mit Satz \ref{Satz 2.6} folgt
dann: \(\nu=\lambda_{d}\) auf \(\fb_{d}\).
\end{beweis}

\begin{bemerkung}
Sei \(X\in\fb_{d}\). Aus 1.6 folgt: \(\fb(X)=\{A\in\fb_{d}\mid A\subseteq X\}\).
Die Einschr\"ankung von \(\lambda_{d}\) auf \(\fb(X)\) hei\ss t ebenfalls
L-Ma\ss \ und wird mit \(\lambda_{d}\) bezeichnet.
\end{bemerkung}

\begin{beispieleX}
\begin{enumerate}
\item Seien \(a=(a_{1},\ldots,a_{d}),\,b=(b_{1},\ldots,b_{d})\in\mdr^{d},\,a\leq b\) und \(I=[a,b]\).\\
\textbf{Behauptung}\\\(\lambda_{d}([a,b])=(b_{1}-a_{1})\cdots(b_{d}-a_{d})\) (Entsprechendes gilt f\"ur \((a,b)\) und \([a,b)\))
\begin{beweis}
\(I_{n}:=(a_{1}-\frac{1}{n},b_{1}]\times\cdots\times(a_{d}-\frac{1}{n},b_{d}];\,I_{1}\supset I_{2}\supset\cdots;\,\bigcap I_{n}=I,\,\lambda_{d}(I_{1})<\infty\)

Aus Satz \ref{Satz 1.7}, Punkt 5, folgt:
\begin{align*}
\lambda_{d}(I)&=\lim_{n\to\infty}{\lambda_{d}(I_{n})}\\
&=\lim_{n\to\infty}{(b_{1}-a_{1}+\frac{1}{n})\cdots(b_{d}-a_{d}+\frac{1}{n})}\\
&=(b_{1}-a_{1})\cdots(b_{d}-a_{d})
\end{align*}
\end{beweis}
\item Sei \(a\in\mdr^{d},\,\{a\}=[a,a]\in\fb_{d}\). Aus obigem Beispiel (1)
folgt: \(\lambda_{d}(\{a\})=0\).
\item \(\mdq^{d}\) ist abz\"ahlbar, also: \(\mdq^{d}=\{a_{1},a_{2},\ldots\}\)
mit \(a_{j}\neq a_{i}\,(i\neq j)\). Dann: \(\mdq^{d}=\bigcup\{a_{j}\}\) %\bigcupdot...

Dann gilt: \(\mdq^{d}\in\fb_{d}\) und \(\lambda_{d}(\mdq^{d})=\sum{\lambda_{d}(\{a_{j}\})}=0\).
\item Wie in Beispiel (3): Ist \(A\subseteq\mdr^{d}\) abz\"ahlbar, so ist
\(A\in\fb_{d}\) und \(\lambda_{d}(A)=0\).
\item Sei \(j\in\{1,\ldots,d\}\) und \(H_{j}:=\{(x_{1},\ldots,x_{d})\in\mdr^{d}\mid x_{j}=0\}\). \(H_{j}\) ist abgeschlossen, damit folgt: \(H_{j}\in\fb_{d}\).

Ohne Beschr\"ankung der Allgemeinheit sei \(j=d\). Dann:
\(I_{n}:=\underbrace{[-n,n]\times\cdots\times[-n,n]}_{(d-1)-\text{mal}}\times\{0\}\).
% Hier fehlt noch eine Graphik
Aus Beispiel (1) folgt: \(\lambda_{d}(I_{n})=0\).

Aus \(H_{d}=\bigcup{I_{n}}\) folgt: \(\lambda_{d}(H_{d})\leq\sum{\lambda_{d}(I_{n})}=0\). Also: \(\lambda_{d}(H_{j})=0\).
\end{enumerate}
\end{beispieleX}

\begin{definition}
Sei $x\in\mdr^d, B\subseteq\mdr^d$. Definiere:
\[x+B:=\{x+b\mid b\in B\}\]
\end{definition}

\begin{beispiel}
Ist $I\in\ci_d$, so gilt $x+I\in\ci_d$ und $\lambda_d(x+I)=\lambda_d(I)$.
\end{beispiel}

\begin{satz}
\label{Satz 2.8}
Sei $x\in\mdr^d, \fa:=\{B\in\fb_d:x+B\in\fb_d\}$ und $\mu:\fa\to[0,\infty]$ sei definiert durch $\mu(A):=\lambda_d(x+A)$. Dann gilt:
\begin{enumerate}
\item $(\mdr^d,\fa,\mu)$ ist ein Maßraum.
\item Es ist $\fa=\fb_d$ und $\mu=\lambda_d$ auf $\fb_d$. D.h. für alle $A\in\fb_d$ ist $x+A\in\fb_d$ und $\lambda_d(x+A)=\lambda_d(A)$ (Translationsinvarianz des Lebesgue-Maßes).
\end{enumerate}
\end{satz}

\begin{beweis}
\begin{enumerate}
\item Leichte Übung!
\item Es ist klar, dass $\fb_d\supseteq\fa$. Nach dem Beispiel von oben gilt:
\[\ci_d\subseteq\fa=\fb_d=\sigma(\ci_d)\subseteq\sigma(\fa)=\fa\]
Setze $\ce:=ci_d$, dann ist $\sigma(\ce)=\fb_d$ und es gilt nach dem Beispiel von oben:
\[\forall E\in\ce:\mu(E)=\lambda_d(E)\]
$\ce$ hat die Eigenschaften (1) und (2) aus Satz \ref{Satz 2.6}, daraus folgt dann, dass $\mu=\lambda_d$ auf $\fb_d$ ist.
\end{enumerate}
\end{beweis}

\begin{satz}
\label{Satz 2.9}
Sei $\mu$ ein Maß auf $\fb_d$ mit der Eigenschaft:
\[\forall x\in\mdr^d, A\in\fb_d:\mu(A)=\mu(x+A)\]
Weiter sei $c:=\mu((0,1]^d)<\infty$. Dann gilt:
\[\mu=c\cdot\lambda_d\]
\end{satz}

\begin{satz}[Regularität des Lebesgue-Maßes]
\label{Satz 2.10}
Sei $A \in\fb_d$, dann gilt:
\begin{enumerate}
\item
$\lambda_d(A)=\inf\left\{\lambda_d(G)\mid G\subseteq\mdr^d\text{ offen und }A \subseteq G\right\}\\
 =\inf\left\{\lambda_d(V)\mid V=\bigcup_{j=1}^\infty I_j, I_j\subseteq\mdr^d\text{ offenes Intervall }, A\subseteq V\right\}$
\item $\lambda_d(A)=\sup\{\lambda_d(K)\mid K\subseteq\mdr^d\text{ kompakt }, K\subseteq A\}$
\end{enumerate}
\end{satz}

\begin{beweis}
\begin{enumerate}
\item Ohne Beweis.
\item Setze $\beta:=\sup\{\lambda_d(K)\mid K\subseteq\mdr^d\text{ kompakt }, K\subseteq A\}$. Sei $K$ kompakt und $K\subseteq A$, dann gilt $\lambda_d(K)\le\lambda_d(A)$, also ist auch $\beta\le\lambda_d(A)$.

\textbf{Fall 1:} Sei $A$ zusätzlich beschränkt.\\
Sei $\ep>0$. Es existiert ein $r>0$, sodass $A\subseteq B:=\overline{U_r(0)}\subseteq[-r,r]^d$ ist, dann gilt:
\[\lambda_d(A)\le\lambda_d([-r,r]^d)=(2r)^d<\infty\]
Aus (1) folgt, dass eine offene Menge $G\supseteq B\setminus A$ existiert mit $\lambda_d(G)\le\lambda_d(B\setminus A)+\ep$. Dann gilt nach \ref{Satz 1.7}:
\[\lambda_d(B\setminus A)=\lambda_d(B)-\lambda_d(A)\]
Setze nun $K:=B\setminus G=B\cap G^c$, dann ist $K$ kompakt und $K\subseteq B\setminus(B\setminus A)=A$. Da $B\subseteq G\cup K$ ist, gilt:
\[\lambda_d(B)\le\lambda_d(G\cup K)\le \lambda_d(B)-\lambda_d(A)+\ep+\lambda_d(K)\]
Woraus folgt:
\[\lambda_d(A)\le\lambda_d(K)+\ep\]

\textbf{Fall 2:} Sei $A\in\fb_d$ beliebig.\\
Setze $A_n:=A\cap\overline{U_n(0)}$. Dann ist $A_n$ für alle $n\in\mdn$ beschränkt, $A_n\subseteq A_{n+1}$ und $A=\bigcup_{n\in\mdn} A_n$. Nach \ref{Satz 1.7} gilt:
\[\lambda_d(A)=\lim\lambda_d(A_n)\]
Aus Fall 1 folgt, dass für alle $n\in\mdn$ ein kompaktes $K_n\subseteq A_n$ mit $\lambda_d(A_n)\le\lambda_d(K_n)+\frac1n$ existiert. Dann gilt:
\[\lambda_d(A_n)\le\lambda_d(K_n)+\frac1n\le\lambda_d(A)+\frac1n\]
Also auch:
\[\lambda_d(A)=\lim\lambda(K_n)\le\beta\]
\end{enumerate}
\end{beweis}

\textbf{Auswahlaxiom:}\\
Sei $\varnothing\ne\Omega$ Indexmenge, es sei $\{X_\omega\mid \omega\in\Omega\}$ ein disjunktes System von nichtleeren Mengen $X_\omega$. Dann existiert ein $C\subseteq\bigcup_{\omega\in\Omega}X_\omega$, sodass $C$ mit jedem $X_j$ genau ein Element gemeinsam hat.

\begin{satz}[Satz von Vitali]
\label{Satz 2.11}
Es existiert ein $C\subseteq\mdr^d$ sodass $C\not\in\fb_d$.
\end{satz}

\begin{beweis}
Wir definieren auf $[0,1]^d$ eine Äquivalenzrelation $\sim$, durch:
\begin{align*}
\forall x,y\in[0,1]^d: x \sim y\iff x-y\in\mdq^d\\
\forall x\in[0,1]^d:[x]:=\{y\in[0,1]^d\mid x\sim y\}
\end{align*}
Nach dem Auswahlaxiom existiert ein $C\subseteq[0,1]^d$, sodass $C$ mit jedem $[x]$ genau ein Element gemeinsam hat.
Es ist $\mdq^d\cap[-1,1]^d=\{q_1,q_2,\ldots\}$ mit $q_i\ne q_j$ für $(i\ne j)$. Dann gilt:
\begin{align*}
\tag{1} \bigcup_{n=1}^\infty(q_n+C)\subseteq[-1,2]^d\\
\tag{2} [0,1]^d\subseteq\bigcup_{n=1}^\infty(q_n+C)
\end{align*}
\begin{beweis}
Sei $x\in[0,1]^d$. Wähle $y\in C$ mit $y\in[x]$, dann ist $x\sim y$, also $x-y\in\mdq^d\cap[-1,1]^d$. D.h.:
\[\exists n\in\mdn: x-y=q_n\implies x=q_n+y\in q_n+C\] 
\end{beweis}
Außerdem ist $\{q_n+C\mid n\in\mdn\}$ disjunkt.
\begin{beweis}
Sei $z\in(q_n+C)\cap(q_m+C)$, dann existieren $a,b\in\mdq^d$, sodass gilt:
\begin{align*}
(q_n+a=z=q_m+b) &\implies (b-a=q_m-q_n\in\mdq^d)\\
&\implies (a\sim b) \implies([a]=[b])\\
&\implies (a=b)\implies (q_n,q_m)
\end{align*}
\end{beweis}
\textbf{Annahme:} $C\in\fb_d$, dann gilt nach (1):
\begin{align*}
3^d&=\lambda_d([-2,1]^d)\\
&\ge\lambda_d(\bigcup(q_n+C))\\
&=\sum \lambda_d(q_n+C)\\
&=\sum \lambda_d(C)
\end{align*}
Also ist $\lambda_d(C)=0$. Damit folgt aus (2):
\begin{align*}
1&=\lambda_d([0,1]^d)\\
&\le \lambda_d(\bigcup (q_n+C))\\
&=\sum \lambda_d(C)\\
&=0
\end{align*}
\end{beweis}

\chapter{Messbare Funktionen}

In diesem Paragraphen seien $\varnothing\ne X,Y,Z$ Mengen.

\begin{definition}
\index{messbar!Raum}\index{Raum!messbarer}
Ist $\fa$ eine $\sigma$-Algebra auf $X$, so heißt $(X,\fa)$ ein \textbf{messbarer Raum}.
\end{definition}

\begin{definition}
\index{$\fa$-$\fb$-messbar}
\index{messbar!Funktion}
Sei $\fa$ eine $\sigma$-Algebra auf $X$, $\fb$ eine $\sigma$-Algebra auf $Y$ und $f:X\to Y$ eine Funktion. $f$ heißt genau dann \textbf{$\fa$-$\fb$-messbar}, wenn gilt:
\[\forall B\in\fb: f^{-1}(B)\in\fa\]
\end{definition}

\begin{bemerkung}
Seien die Bezeichnungen wie in obiger Definition, dann gilt:
\begin{enumerate}
\item $f$ sei $\fa$-$\fb$-messbar, $\fa'$ eine weitere $\sigma$-Algebra auf $X$ mit $\fa\subseteq\fa'$ und $\fb'$ sei eine $\sigma$-Algebra auf $Y$ mit $\fb'\subseteq\fb$.\\
Dann ist $f$ $\fa'$-$\fb'$-messbar.
\item Sei $X_0\in\fa$, dann gilt $\fa_{X_0}\subseteq\fa$ nach \ref{Satz 1.5}. Nun sei $f:X\to Y$ $\fa$-$\fb$-messbar, dann ist $f_{\mid X_0}:X_0\to Y$ $\fa_{X_0}$-$\fb$-messbar.
\end{enumerate}
\end{bemerkung}

\begin{beispiel}
\begin{enumerate}
\item Sei $\fa$ eine $\sigma$-Algebra auf $X$ und $A\subseteq X$. $\mathds{1}_A:X\to\mdr$ ist genau dann $\fa$-$\fb_1$-messbar, wenn $A\in\fa$ ist.
\item Sei $X=\mdr^d$. Ist $A\in\fb_d$, so ist $\mathds{1}_A$ $\fb_d$-$\fb_1$-messbar.
\item Ist $C$ wie in \ref{Satz 2.11}, so ist $\mathds{1}_A$ nicht $\fb_d$-$\fb_1$-messbar.
\item Es sei $f:X\to Y$ eine Funktion und $\fb$ ($\fa$) eine $\sigma$-Algebra auf $Y$ ($X$), dann ist $f$ $\cp(X)$-$\fb$-messbar ($\fa$-$\{Y,\varnothing\}$-messbar).
\end{enumerate}
\end{beispiel}

\begin{satz}
\label{Satz 3.1}
Seien \(\fa,\,\fb\,\fc\) \(\sigma\)-Algebren auf \(X,\,Y\) bzw. \(Z\). Weiter seien \(f:\,X\to Y\) und \(g:\,Y\to Z\)
Funktionen.
\begin{enumerate}
\item Ist \(f\) \(\fa-\fb-\)messbar und ist \(g\) \(\fb-\fc-\)messbar, so ist \(g\circ f:\,X\to Z\) \(\fa-\fc-\)messbar.
\item Sei \(\varnothing\neq\ce\subseteq\cp(Y)\) und \(\sigma(\ce)=\fb\). Dann:
\begin{center}
\(f\) ist \(\fa-\fb-\)messbar, genau dann, wenn gilt: \(\forall E\in\ce:\,f^{-1}(\ce)\in\fa\)
\end{center}
\end{enumerate}
\end{satz}

\begin{beweis}
\begin{enumerate}
\item Sei \(C\in\fc\); \(g\) ist messbar, daraus folgt \(g^{-1}(C)\in\fb\);
\(f\) ist messbar, daraus folgt \(f^{-1}(g^{-1}(C))=(g\circ f)^{-1}(C)\in\fa\)
\item \begin{itemize}
\item[\(\Rightarrow\)] \checkmark
\item[\(\Leftarrow\)] \(\fd:=\{B\subseteq Y\mid f^{-1}(B)\in\fa\}\)
\"Ubung: \(\fd\) ist eine \(\sigma\)-Algebra auf \(Y\).

Aus der Voraussetzung folgt: \(\fe\subseteq\fd\).
Dann: \(\fb=\sigma(\fe)\subseteq\fd\). Ist \(B\in\fb\), so ist \(B\in\fb\), also
\(f^{-1}(B)\in\fd\).
\end{itemize}
\end{enumerate}
\end{beweis}

\begin{definition}
\index{messbar!Borel}\index{messbar}
Sei \(X\in\fb_{d}\). Ist \(f:\,X\to\mdr^{k}\) \(\fb(X)-\fb_{k}-\)messbar, so hei\ss t \(f\) (Borel-)messbar.
\end{definition}
Ab jetzt sei stets \(X\in\fb_{d}\). (Erinnerung: \(\fb(X)=\{A\in\fb_{d}\mid A\subseteq X\}\))

\begin{satz}
\label{Satz 3.2}
Seien \(f,\,g:\,X\to\mdr^{k}\) und \(\alpha,\beta\in\mdr\).
\begin{enumerate}
\item Ist \(f\) auf \(X\) stetig, so ist \(f\) messbar.
\item Ist \(f=(f_{1},\ldots,f_{k})\), so gilt: \(f\) ist messbar \(\Leftrightarrow\) alle \(f_{j}\) sind messbar.
\item Sind \(f\) und \(g\) messbar, so ist \(\alpha f+\beta g\) messbar.
\item Sei \(k=1\) und \(f\) und \(g\) seien messbar. Dann:
\begin{enumerate}
\item \(fg\) ist messbar
\item Ist \(f(x)\neq0\forall x\in X\), so ist \(\frac{1}{f}\) messbar
\item \(\{x\in X\mid f(x)\geq g(x)\}\in\fb(X)\)
\end{enumerate}
\end{enumerate}
\end{satz}

\begin{beweis}
Kommt noch\ldots
\end{beweis}

\begin{folgerungen}
\label{Folgerung 3.3}
\begin{enumerate}
\item Seien \(A,\,B\in\fb(X),\,A\cap B=\varnothing\) und \(X=A\cup B\). Weiter seien \(f:A\to\mdr^{k}\) und
\(g:B\to\mdr^{k}\) seien messbar. Dann ist \(h:X\to\mdr^{k}\), definiert durch 
\[
h(x):=\begin{cases}f(x)&x\in A\\g(x)&x\in B\end{cases},
\]
messbar.
\item Ist \(f:X\to\mdr^{k}\) messbar und \(g(x):=\lVert f(x)\rVert\,(x\in X)\), so ist \(g\) messbar.
\end{enumerate}
\end{folgerungen}

\begin{beweis}
\begin{enumerate}
\item Sei \(C\in\fb_{k}\). Dann:
\[
h^{-1}(C)=\underbrace{f^{-1}(C)}_{\in\fb(A)\subseteq\fb(X)}\cup\underbrace{g^{-1}(C)}_{\in\fb(B)\subseteq\fb(X)}\in\fb(X)
\]
\item Definiere \(\vp(z)=\lVert z\rVert\quad(z\in\mdr^{k})\); \(\vp\) ist
stetig, also messbar.

Es ist \(g=\vp\circ f\). Mit \ref{Satz 3.1} folgt: \(g\) ist messbar.
\end{enumerate}
\end{beweis}

\begin{beispiel}
Kommt noch\ldots
\end{beispiel}

\textbf{Ein neues Symbol kommt hinzu:} \(-\infty\){

\(\imdr:=[-\infty,+\infty]:=\mdr\cup\{-\infty,+\infty\}\)

In \(\imdr\) gelten folgende Regeln, wobei \(a\in\mdr\):
\begin{enumerate}
\item \(-\infty<a<+\infty\)
\item \(\pm\infty+(\pm\infty)=\pm\infty\)
\item \(\pm\infty+a:=a+(\pm\infty):=\pm\infty\)
\item \(a\cdot(\pm\infty):=(\pm\infty)\cdot a=\begin{cases}\pm\infty&a>0\\
    0&a=0\\\mp\infty&a<0\end{cases}\)
\item \(\frac{a}{\pm\infty}:=0\)
\end{enumerate}
}

\begin{definition}
\begin{enumerate}
\item Sei \((x_{n})\) eine Folge in \(\imdr\). \(x_{n}\rightarrow+\infty:\Leftrightarrow\forall c\in\mdr\exists n_{c}\in\mdn:x_{n}\geq c\forall n\geq n_{c}\)\\
Analog f\"ur \(-\infty\).
\item Seien \(f,g: X\to\imdr\). Dann:
\begin{align*}
    \{f\leq g\}&:=\{x\in X\mid f(x)\leq g(x)\}\\
    \{f\geq g\}&:=\{x\in X\mid f(x)\geq g(x)\}\\
    \{f\neq g\}&:=\{x\in X\mid f(x)\neq g(x)\}\\
    \{f<g\}&:=\{x\in X\mid f(x)<g(x)\}\\
    \{f>g\}&:=\{x\in X\mid f(x)>g(x)\}
\end{align*}
\item Sei \(a\in\imdr\) und \(f:\,X\to\imdr\). Dann:
\begin{align*}
    \{f\leq a\}&:=\{x\in X\mid f(x)\leq a\}\\
    \{f\geq a\}&:=\{x\in X\mid f(x)\geq a\}\\
    \{f\neq a\}&:=\{x\in X\mid f(x)\neq a\}\\
    \{f<a\}&:=\{x\in X\mid f(x)<a\}\\
    \{f>a\}&:=\{x\in X\mid f(x)>a\}
\end{align*}
\end{enumerate}
\end{definition}

\begin{definition}
\index{Borel!$\sigma$-Algebra}\index{messbar}
\(\ifb_{1}:=\{B\cup E\mid B\in\fb_{1},\,E\subseteq\{-\infty,+\infty\}\}\). Dann: \(\fb_{1}\subseteq\ifb_{1}\)\\
Übung: \(\ifb_{1}\) ist eine \(\sigma\)-Algebra auf \(\imdr\).\\
\(\ifb_{1}\) hei\ss t \textbf{Borelsche \(\sigma\)-Algebra} auf \(\imdr\).
Sei \(f:\,X\to\imdr\). \(f\) hei\ss t \textbf{(Borel-)messbar} (mb) \(:\Leftrightarrow\,f\) ist \(\fb(X)-\ifb_{1}-\) messbar.
\end{definition}

\begin{beispiel}
Kommt noch\ldots
\end{beispiel}

\begin{satz}
\label{Satz 3.4}
\begin{enumerate}
\item Definiere die Mengen:
\begin{align*}
\ce_1&:=\{[-\infty,a]\mid a\in\mdq\} & \ce_2&:=\{[-\infty,a)\mid a\in\mdq\}\\
\ce_3&:=\{(a,\infty]\mid a\in\mdq\} & \ce_4&:=\{[a,\infty]\mid a\in\mdq\}
\end{align*}
Dann gilt:
\[\overline{\fb_1}=\sigma(\ce_j)\quad \text{ für }j\in\{1,2,3,4\}\]
\item Für $f:X\to\imdr$ sind die folgenden Aussagen äquivalent:
\begin{enumerate}
\item $f$ ist messbar.
\item $\forall a\in\mdq: \{f\le a\}\in\fb(X)$.
\item $\forall a\in\mdq: \{f\ge a\}\in\fb(X)$.
\item $\forall a\in\mdq: \{f< a\}\in\fb(X)$.
\item $\forall a\in\mdq: \{f> a\}\in\fb(X)$.
\end{enumerate}
\item Die Äquivalenzen in (2) gelten auch für Funktionen $f:X\to\mdr$.
\end{enumerate}
\end{satz}

\begin{beweis}
Die folgenden Beweise erfolgen exemplarisch für einen der Unterpunkte und funktionieren fast analog für die anderen.
\begin{enumerate}
\item Für $a\in\mdq$ gilt:
\[[-\infty,a]^c=(a,\infty]\in\sigma(\ce_1)\]
D.h. es gilt $\ce_3\subseteq\sigma(\ce_1)$ und damit auch $\sigma(\ce_3)\subseteq\sigma(\ce_1)$.
\item Es gilt:
\[\{f\le a\}=\{x\in X\mid f(x)\le a\}=f^{-1}([-\infty,a])\]
Die Äquivalenz folgt dann aus (1) und \ref{Satz 3.1}.
\item Die Funktion $f:X\to\imdr$ kann aufgefasst werden als Funktion $\overline{f}:X\to\imdr$. Es ist $f$ genau dann $\fb(X)$-$\fb_1$-messbar wenn $\overline{f}$ $\fb(X)$-$\overline{\fb_1}$-messbar ist. 
\end{enumerate}
\end{beweis}

\begin{definition}
Sei $M\subseteq\imdr$.
\begin{enumerate}
\item Ist $M=\varnothing$ oder $M=\{-\infty\}$, so sei 
\[\sup M:=-\infty\]
\item Ist $M\setminus\{-\infty\}\ne\varnothing$ und nach oben beschränkt (also insbesondere $\infty\not\in M$), so sei 
\[\sup M:= \sup (M\setminus\{-\infty\})\]
\item Ist $M\setminus\{-\infty\}$ nicht nach oben beschränkt oder $\infty\in M$, so sei 
\[\sup M:=\infty\]
\item Es sei $\inf M:=-\sup(-M)$, wobei $-M:=\{-m\mid m\in M\}$.
\end{enumerate}
\end{definition}

\begin{definition}
Sei $(f_n)$ eine Folge von Funktionen $f_n:X\to\imdr$.
\begin{enumerate}
\item Die Funktion $\sup_{n\in\mdn}(f_n):X\to\imdr$  $\left(\sup_{n\in\mdn}(f_n):X\to\imdr\right)$ ist definiert durch:
\[(\sup_{n\in\mdn} f_n)(x):=\sup\{f_n(x)\mid n\in\mdn\}\quad x\in X\]
\[\left((\inf_{n\in\mdn} f_n)(x):=\inf\{f_n(x)\mid n\in\mdn\}\quad x\in X\right)\]
\item Die Funktion $\limsup_{n\to\infty} f_n:X\to\imdr$ $\left(\liminf_{n\to\infty} f_n:X\to\imdr\right)$ ist definiert durch:
\begin{align*}
\tag{$*$} \limsup_{n\to\infty} f_n &:= \inf_{j\in\mdn}(\sup_{n\ge j} f_n)\\
\liminf_{n\to\infty} f_n &:= \sup_{j\in\mdn}(\inf_{n\ge j} f_n)
\end{align*}
\textbf{Erinnerung:} Für eine beschränkte Folge $(a_n)$ in $\mdr$ war
\[\limsup_{n\to\infty} a_n:=\inf\{\sup\{a_n\mid n\ge j\}\mid j\in\mdn\}\]
\item Sei $N\in\mdn$ und $g_j:=f_j$ (für $j=1,\ldots,N$), $g_j:=f_N$ (für $j>N$). Definiere:
\begin{align*}
\max_{1\le n\le N} f_n &:=\sup_{j\in\mdn} g_n\\
\min_{1\le n\le N} f_n &:=\inf_{j\in\mdn} g_n
\end{align*}
\item Ist $f_n(x)$ für jedes $x\in\imdr$ konvergent, so ist $\lim_{n\to\infty} f_n:X\to\imdr$ definiert durch:
\[(\lim_{n\to\infty} f_n)(x):=\lim_{n\to\infty} f_n(x)\]
(In diesem Fall gilt $\lim_{n\to\infty} f_n = \limsup_{n\to\infty} f_n = \liminf_{n\to\infty} f_n$.)
\end{enumerate}
\end{definition}

\begin{satz}
\label{Satz 3.5}
Sei $f_n$ eine Folge von Funktionen $f_n:X\to\imdr$ und jedes $f_n$ messbar.
\begin{enumerate}
\item Dann sind ebenfalls messbar:
\begin{align*}
&\sup_{n\in\mdn} f_n  &&\inf_{n\in\mdn} f_n &&\limsup_{n\in\mdn} f_n &&\liminf_{n\in\mdn} f_n
\end{align*}
\item Ist $(f_n(x))$ für jedes $x\in X$ ind $\imdr$ konvergent, so ist $\lim_{n\to\infty} f_n$ messbar.
\end{enumerate}
\end{satz}

\begin{beweis}
\begin{enumerate}
\item Sei $a\in\mdq$, dann gilt (nach \ref{Satz 3.4}(2)):
\[\{\sup_{n\in\mdn} f_n\le a\}=\bigcap_{n\in\mdn}\{f_n\le a\}\in\fb(X)\]
Also ist $\sup_{n\in\mdn} f_n$ messbar. Analog lässt sich die Messbarkeit von $\inf_{n\in\mdn} f_n$ zeigen, der Rest folgt dann aus ($*$).
\item Folgt aus (1) und obiger Bemerkung in der Definition.
\end{enumerate}
\end{beweis}

\begin{beispiel}
Sei $X=I$ ein Intervall in $\mdr$ und $f:I\to\mdr$ sei auf $I$ differenzierbar.\\
Für $x\in I,n\in\mdn$ sei $f_n:= n(f(x+\frac1n)-f(x))$. Da $f$ stetig ist, ist auch jedes $f_n$ stetig, also insbesondere mess bar und es gilt:
\[f_n(x)=\frac{f(x-\frac1n)-f(x)}{\frac1n}\stackrel{n\to\infty}{\to}f'(x)\]
Aus \ref{Satz 3.5}(2) folgt, dass $f'$ messbar ist. 
\end{beispiel}

\begin{definition}
\index{Positivteil}\index{Negativteil}
Sei $f:X\to\imdr$ eine Funktion.
\begin{enumerate}
\item $f_+:=\max\{f,0\}$ heißt \textbf{Positivteil} von $f$.
\item $f_-:=\max\{-f,0\}$ heißt \textbf{Negativteil} von $f$.
\end{enumerate}
Es gilt $f_+,f_-\ge 0$, $f=f_+-f_-$ und $|f|=f_++f_-$.
\end{definition}

\begin{satz}
\label{Satz 3.6}
Seien $f,g:X\to\imdr$ und $\alpha,\beta\in\mdr$.
\begin{enumerate}
\item Sind $f,g$ messbar und ist $\alpha f(x)+\beta g(x)$ für jedes $x\in X$ definiert, so ist $\alpha f+\beta g$ messbar.
\item Sind $f,g$ messbar und ist $f(x)g(x)$ für jedes $x\in X$ definiert, so ist $fg$ messbar.
\item $f$ ist genau dann messbar, wenn $f_+$ und $f_-$ messbar sind. In diesem Fall ist auch $|f|$ messbar.
\end{enumerate}
\end{satz}

\begin{beweis}
\begin{enumerate}
\item[(1)+(2)] Für alle $n\in\mdn, x\in X$ seien $f_n$ und $g_n$ wie folgt definiert:
\begin{align*}
f_n(x)&:=\max\{-n,\min\{f(x),n\}\}\\
g_n(x)&:=\max\{-n,\min\{g(x),n\}\}
\end{align*}
Dann sind $f_n(x),g_n(x)\in[-n,n]$ für alle $n\in\mdn,x\in X$. Nach \ref{Satz 3.2}(3) sind also $\alpha f_n+\beta g_n$ und $f_ng_n$ messbar. Außerdem gilt:
\begin{align*}
\alpha f_n(x)+\beta g_n(x)&\stackrel{n\to\infty}\to \alpha f(x)+\beta g(x)\\
f_n(x)g_n(x)&\stackrel{n\to\infty}\to f(x)g(x)
\end{align*}
Die Behauptung folgt aus \ref{Satz 3.5}(2).
\item[(3)] Nach \ref{Satz 3.5}(1) sind $f_+$ und $f_-$ messbar, wenn $f$ messbar ist. Die umgekehrte Implikation folgt aus \ref{Satz 3.6}(1). Sind $f_+$ und $f_-$ messbar, so folgt ebenfalls aus \ref{Satz 3.6}(1), dass $|f|=f_++f_-$ messbar ist.
\end{enumerate}
\end{beweis}

\begin{beispiel}
Sei $C\subseteq\mdr^d$ wie in \ref{Satz 2.11}, also $C\not\in\fb_d$. Definiere $f:\mdr^d\to\mdr$ wie folgt:
\[f(x):=\begin{cases} 1&,x\in C\\ -1&,x\not\in C\end{cases}\]
Dann ist $\{f\ge 1\}=C$, also $f$ \textbf{nicht} messbar. Aber für alle $x\in\mdr^d$ ist $|f(x)|=1$, also $|f|=\mathds{1}_{\mdr^d}$ und damit messbar.
\end{beispiel}

\begin{definition}
\index{einfach}
\index{Treppenfunktion}
\index{Normalform}
$f:X\to\mdr$ sei messbar.
\begin{enumerate}
\item $f$ heißt \textbf{einfach} oder \textbf{Treppenfunktion}, genau dann wenn $f(X)$ endlich ist.
\item $f$ sei einfach und $f(x)=\{y_1,\ldots,y_m\}$ mit $y_i\ne y_j$ für $i\ne j$. Sei weiter $A_j:=f^{-1}(\{y_j\})$ für $j=1,\ldots,m$. Dann sind $A_1,\ldots,A_m\in\fb(X)$ und $X=\bigcup_{j=1}^m A_j$ disjunkte Vereinigung.
\[f=\sum_{j=1}^m y_j \mathds{1}_{A_j}\]
heißt \textbf{Normalform} von $f$.
\end{enumerate}
\end{definition}

\begin{beispiel}
\end{beispiel}

\begin{satz}
\label{Satz 3.7}
Linearkombinationen und Produkte, sowie endliche Maxima und Minima einfacher Funktionen sind einfach.
\end{satz}

\begin{satz}
\label{Satz 3.8}
\index{zulässig}
Sei $f:X\to\imdr$ messbar.
\begin{enumerate}
\item Ist $f\ge 0$ auf $X$, so existiert eine Folge $(f_n)$ von einfachen Funktionen $f_n:X\to[0,\infty)$, sodass $0\le f_n\le f_{n+1}$ auf $X$ ($\forall n\in\mdn$) und $f_n(x)\stackrel{n\to\infty}{\to}f(x)$ ($\forall x\in X$). In diesem Fall heißt $(f_n)$ \textbf{zulässig} für $f$.
\item Es existiert eine Folge $(f_n)$ von einfachen Funktionen $f_n:X\to\mdr$, sodass $|f_n|\le |f|$ auf $X$ ($\forall n\in\mdn$) und $f_n(x)\stackrel{n\to\infty}{\to}f(x)$ ($\forall x\in X$).
\item Ist $f$ beschränkt auf $X$ (also insbesondere $\pm\infty\not\in f(X)$), so kommt in (2) noch hinzu, dass $(f_n)$ auf $X$ gleichmäßig gegen $f$ konvergiert.
\end{enumerate}
\end{satz}

\begin{folgerungen}
\end{folgerungen}

\begin{beweis}
\end{beweis}

\chapter{Konstruktion des Lebesgueintegrals}

In diesem Paragraphen sei $\varnothing\ne X\in\fb_d$. Wir schreiben außerdem $\lambda$ statt $\lambda_d$.

\begin{definition}
\index{Lebesgueintegral}
Sei $f:X\to [0,\infty)$ eine einfache Funktion mit der Normalform $f=\sum_{j=1}^m y_j\mathds{1}_{A_j}$.\\
Das \textbf{Lebesgueintegral} von $f$ ist definiert durch:
\[\int_X f(x)\text{ d}x:=\sum_{j=1}^m y_j\lambda(A_j)\]
\end{definition}

\begin{satz}
\label{Satz 4.1}
Sei $f:X\to[0,\infty)$ einfach, $z_1,\ldots,z_k\in[0,\infty)$ und $B_1,\ldots,B_k\in\fb(X)$ mit $\bigcup B_j=X$ und $f=\sum_{j=1}^k z_j\mathds{1}_{B_j}$. Dann gilt:
\[\int_X f(x)\text{ d}x=\sum_{j=1}^k z_j\lambda(B_j)\]
\end{satz}

\begin{beweis}
In der großen Übung.
\end{beweis}

\begin{satz}
\label{Satz 4.2}
Seien $f,g:X\to[0,\infty)$ einfach, $\alpha, \beta\in[0,\infty)$ und $A\in\fb(X)$.
\begin{enumerate}
\item $\int_X \mathds{1}_A(x)\text{ d}x=\lambda(A)$
\item $\int_X (\alpha f+\beta g)(x)\text{ d}x = \alpha\int_X f(x)\text{ d}x + \beta\int_X g(x)\text{ d}x$
\item Ist $f\le g$ auf $X$, so ist $\int_X f(x)\text{ d}x\le \int_X g(x)\text{ d}x$.
\end{enumerate}
\end{satz}

\begin{beweis}
\end{beweis}

\begin{definition}
\index{Lebesgueintegral}
Sei $f:X\to[0,\infty]$ messbar. $(f_n)$ sei eine für $f$ zulässige Folge. Das \textbf{Lebesgueintegral} von $f$ ist definiert als:
\begin{align*}
\tag{$*$}\int_X f(x)\text{ d}x:=\lim_{n\to\infty}\int_X f_n(x)\text{ d}x
\end{align*}
\end{definition}

\begin{bemerkung}
\end{bemerkung}

\textbf{Bezeichnung:}\\
Für messbare Funktionen $f:X\to[0,\infty]$ definiere
\[M(f):=\{\int_X g\text{ d}x\mid g:X\to[0,\infty) \text{ einfach und }g\le f\text{ auf }X\}\]

\begin{satz}
\label{Satz 4.3}
Ist $f:X\to[0,\infty]$ messbar und $(f_n)$ zulässig für $f$, so gilt:
\[L:=\lim_{n\to\infty}\int_X f_n\text{ d}x=\sup M(f)\]
Insbesondere ist $\int_X f(x) \text{ d}x$ wohldefiniert.
\end{satz}

\begin{folgerungen}
\label{Folgerung 4.4}
Ist $f:X\to[0,\infty]$ messbar, so ist $\int_X f(x) \text{ d}x=\sup M(f)$.
\end{folgerungen}

\begin{beweis}
Sei \(\int_Xf_n\,dx\in M(f) \,\forall\natn \). Dann ist \[L = \sup\left\{\int_Xf_n\,dx\mid\natn\right\} \leq \sup M(f)\]\\
Sei nun $g$ einfach und \(0\leq g\leq f\). Sei weiter \[g=\sum^m_{j=1}y_j\mathds{1}_{A_j}\] die Normalform von $g$.\\
Sei \(\alpha>1\) und \(B_n:=\{\alpha f_n\geq g\}\). Dann ist \[B_n\in\fb(X) \text{ und }(B_n\subseteq B_{n+1}\text{, sowie } \mathds{1}_{B_n}g\leq\alpha f_n.\]
Sei \(x\in X\).\\
\textbf{Fall 1:} Ist \(f(x)=0\), so ist wegen \(0\leq g\leq f\) auch \(g(x)=0\). Somit ist \(x\in B_n\) für jedes \(\natn\).\\
\textbf{Fall 2:} Ist  \(f(x)>0\), so ist \[\frac{1}{\alpha}g(x)<f(x)\] (Dies ist klar für \(g(x)=0\) und falls gilt: \(g(x)>0\), so ist \(\frac{1}{\alpha}g(x)<g(x)\leq f(x) \) )\\
Da $f_n$ zulässig für $f$ ist, gilt: \(f_n(x)\to f(x)\  (n\to\infty)\), weshalb ein \(n(x)\in\mdn\) existiert mit:
\[\frac{1}{\alpha}g(x)<f(x)\text{für jedes } n\geq n(x)\]
Es folgt \(x\in B_n\) für jedes \(n\geq n(x)\).\\
\textbf{Fazit:} \(X=\bigcup B_n\). \[A_j=A_j\cap X=A_j\cap\left(\bigcup B_n\right) = \bigcup(A_j\cap B_n) \text{ und } A_j\cap B_n\subseteq A_j\cap B_{n+1} \]
Aus \ref{Satz 1.7} folgt \(\lambda(A_j)=\lim\limits_{n\to\infty}\lambda(A_j\cap B_n)\). Das liefert:
\begin{align*}
   \int\limits_Xg\,dx &= \sum\limits_{j=1}^m y_j\lambda(A_j) 
   = \sum\limits_{j=1}^m y_j\lim\limits_{n\to\infty}\lambda(A_j\cap B_n)\\ 
   &=\lim\limits_{n\to\infty}\sum\limits_{j=1}^m y_j\lambda(A_j\cap B_n)
   \overset{\ref{Satz 4.1}}= \lim\limits_{n\to\infty} \int\limits_X \mathds{1}_{B_n}g\,dx\\
   &\leq  \lim\limits_{n\to\infty} \int\limits_X \alpha f_n\,dx
   =\alpha L
\end{align*}
g war einfach und \(0\leq g\leq f\) beliebig, sodass \[\sup M(f)\leq\alpha L \overset{\alpha\to 1}\implies \sup M(f)\leq L \]
\end{beweis}

\begin{satz}
\label{Satz 4.5}
Seien $f,g:X\to[0,\infty]$ messbar und $\alpha,\beta\ge0$.
\begin{enumerate}
\item $\int_X (\alpha f+\beta g)(x) \text{ d}x=\alpha\int_X f(x) \text{ d}x+\beta\int_X g(x) \text{ d}x$
\item Ist $f\le g$ auf $X$, so gilt $\int_X f(x) \text{ d}x\le \int_X g(x) \text{ d}x$
\item $\int_X f(x) \text{ d}x=0 \iff \lambda(\{f>0\})=0$
\end{enumerate}
\end{satz}

\begin{beweis}
\begin{enumerate}
\item \((f_n)\) und \((g_n)\) seien zulässig für $f$ bzw. $g$. Weiter sei \((h_n):=\alpha (f_n)+\beta (g_n) \).
Dann ist wegen \ref{Satz 3.7} und \(\alpha , \beta \geq 0\), dass \((h_n)\) zulässig für \(\alpha f+\beta g\) ist. Dann:
\begin{align*}
\int_X(\alpha f + \beta g)\,dx
&= \lim\limits_{n\to\infty}\int_X \left( \alpha (f_n)+\beta (g_n) \right)\,dx\\
&\overset{\ref{Satz 4.2}}= \alpha\lim\limits_{n\to\infty}\int_X(f_n)\,dx + \beta\lim\limits_{n\to\infty}\int_X(g_n)\,dx\\
&=\alpha\int_Xf\,dx + \beta\int_Xg\,dx
\end{align*}
\item Wegen \(f\leq g\) auf $X$ ist \(M(f)\subseteq M(g)\) und somit auch \(\sup M(f)\leq\sup M(g)\). Aus \ref{Folgerung 4.4} folgt nun die Behauptung.
\item Setze \(A:=\{f>0\}=\{x\in X:f(x)>0\}\).
\begin{enumerate}
\item["'$\implies$"'] Sei \(\int_Xf\,dx=0\) und \(A_n:=\{f>\frac{1}{n}\}\). Dann ist \(A=\bigcup A_n\) und \(f\geq\frac{1}{n}\mathds{1}_{A_n}\). Damit folgt:
\begin{align*}
0 = \int_Xf\,dx 
\overset{\text{(2)}}\geq \int_X\frac1{n}\mathds{1}_{A_n}\,dx
=\frac1{n}\lambda(A_n)
\intertext{Es ist also \(\lambda(A_n)=0\) und damit gilt weiter}
\lambda(A)=\lambda(\bigcup A_n) \overset{\ref{Satz 1.7}}\leq \sum\lambda(A_n)=0
\end{align*}
Also ist auch \(\lambda(A)=0\).
\item["'$\impliedby$"'] Sei \(\lambda(A)=0\), \((f_n)\) zulässig für $f$ und \(c_n:=\max\{f_n(x):x\in X\}\). Dann ist \(f_n\leq c_n\mathds{1}_A\) und es gilt:
\[0 \leq \int_Xf_n\,dx\overset{\text{(2)}} \leq \int_Xc_n\mathds{1}_A\,dx = c_n\lambda(A) \overset{\text{Vor.}} = 0 \]
Es ist also  \(\int_Xf_n\,dx=0\) für jedes $\natn$ und somit auch \(\int_Xf\,dx=0\)
\end{enumerate}
\end{enumerate}
\end{beweis}

\begin{satz}[Satz von Beppo Levi]
\label{Satz 4.6}
Sei $(f_n)$ eine Folge messbarer Funktionen $f_n:X\to[0,\infty]$ und es gelte $f_n\le f_{n+1}$ auf $X$ für jedes $n\in\mdn$.
\begin{enumerate}
\item Für alle $x\in X$ existiert $\lim_{n\to\infty} f_n(x)$.
\item Die Funktion $f:x\to[0,\infty]$ definiert durch:
\[f(x):=\lim_{n\to\infty} f_n(x)\]
ist messbar.
\item $\int_X \lim\limits_{n\to\infty}f_n(x) \text{ d}x=\int_X f(x) \text{ d}x=\lim\limits_{n\to\infty}\int_X f_n(x) \text{ d}x$
\end{enumerate}
\end{satz}

\begin{beweis}
\begin{enumerate}
\item Für alle $x\in X$ ist \(\left(f_n(x)\right)\) wachsend, also konvergent in \([0,+\infty]\).
\item folgt aus \ref{Satz 3.5}.
\item Sei \( \left(u_j^{(n)}\right)_{j\in\mdn} \) zulässig für $f_n$ und \(v_j:=\max\left\{u_j^{(1)}, u_j^{(2)}, \ldots , u_j^{(j)} \right\} \).
Aus \ref{Satz 3.7} folgt, dass $v_j$ einfach ist und aus der Konstruktion lässt sich nachrechnen, dass gilt:
 \[0\leq v_j\leq v_{j+1} \text{ und } v_j\leq f_n\leq f \text{ und } f_n=\sup\limits_{j\in\mdn}u_j^{(n)} \leq \sup\limits_{j\in\mdn}v_j \text{ (auf $X$)}\]
Damit ist $(v_j)$ zulässig für $f$ und es gilt:
\[ \int_Xf\,dx=\lim\limits_{j\to\infty}\int_Xv_j\,dx\leq\lim\limits_{j\to\infty}\int_Xf_j\,dx\leq\int_Xf\,dx \]
\end{enumerate}
\end{beweis}

\begin{satz}[Satz von Beppo Levi (Version II)]
\label{Satz 4.7}
Sei $(f_n)$ eine Folge messbarer Funktionen $f_n:X\to[0,\infty]$.
\begin{enumerate}
\item Für alle $x\in X$ existiert $s(x):=\sum_{j=1}^\infty f_j(x)$.
\item $s:X\to[0,\infty]$ ist messbar.
\item $\int_X \sum_{j=1}^\infty f_j(x) \text{ d}x= \sum_{j=1}^\infty \int_X f_j(x) \text{ d}x$
\end{enumerate}
\end{satz}

\begin{beweis}
Setze \[s_n:=\sum\limits_{j=1}^nf_j\]
Dann erfüllt \((s_n)\) die Voraussetzungen von \ref{Satz 4.6}. Aus 4.6 und \ref{Satz 4.5}(1) folgt die Behauptung.
\end{beweis}

\begin{satz}
\label{Satz 4.8}
Sei $f:X\to[0,\infty]$ messbar und es sei $\varnothing\ne Y\in\fb(X)$ (also $Y\subseteq X$ und $Y\in\fb_d$). Dann sind die Funktionen $f_{|Y}:Y\to[0,\infty]$ und $\mathds{1}_Y\cdot f:X\to[0,\infty]$ messbar und es gilt:
\[\int_Y f(x) \text{ d}x:=\int_X f_{|Y}(x) \text{ d}x=\int_X (\mathds{1}_Y\cdot f)(x) \text{ d}x\]
\end{satz}

\begin{beweis}
\textbf{Fall 1:} Die Behauptung ist klar, falls $f$ einfach ist. (Übung!)\\
\textbf{Fall 2:} Sei \((f_n)\) zulässig für $f$ und \(g_n:=f_{n|Y} , h_n:=\mathds{1}_Y f_n\)
Dann ist \((g_n)\) zulässig für \(f_{n|Y}\) und \((h_n)\) ist zulässig für \(mathds{1}_Y f_n\).
Insbesondere sind  \(f_{n|Y}\) und \(mathds{1}_Y f_n\) nach \ref{Satz 3.5} messbar.
Weiter gilt:
\[ \int_Y f_{|Y}\,dx \overset{n\to\infty}\longleftarrow \int_Yg_n\,dx \overset{Fall 1}=\int_Xh_n\,dx\overset{n\to\infty}\longrightarrow \int_X\mathds{1}_Yf\,dx   \]
\end{beweis}

\begin{definition}
\index{integrierbar}\index{Integral}\index{Lebesgueintegral}
Sei $f:X\to\imdr$ messbar. $f$ heißt (Lebesgue-)\textbf{integrierbar} (über $X$), genau dann wenn $\int_X f_+(x) \text{ d}x<\infty$ \textbf{und} $\int_X f_-(x) \text{ d}x<\infty$.\\
In diesem Fall heißt:
\[\int_X f(x) \text{ d}x:=\int_X f_+(x) \text{ d}x-\int_X f_-(x) \text{ d}x\]
das (Lebesgue-)\textbf{Integral} von $f$ (über $X$).
\end{definition}

\textbf{Beachte:}\\
Ist $f:X\to[0,\infty]$ messbar, so ist $f$ genau dann integrierbar, wenn gilt:
\[\int_X f(x) \text{ d}x<\infty\]

\begin{beispiel}
Sei $X \in \fb_1$, $f(x) := \begin{cases} 1&,x\in X\cap\MdQ\\ 0&,x\in X\setminus\MdQ\end{cases} = \mathds{1}_{X\cap\MdQ}$.
$X, \MdQ \in \fb_1 \implies X \cap \MdQ \in \fb_1 \implies f$ ist messbar.
\[0 \leq \int_X f(x) \text{ d}x = \int_X \mathds{1}_{X\cap\MdQ} \text{ d}x = \lambda(X\cap\MdQ) \leq \lambda(\MdQ) = 0\]
\textbf{Das heißt:} $f \in \fl^1(X)$, $\int_X f \text{ d}x = 0$.
Ist speziell $X = [a,b]\quad (a<b)$, so gilt: $f \in \fl^1([a,b])$, aber $f \not\in R([a,b])$. 
\end{beispiel}

\begin{satz}[Charakterisierung der Integrierbarkeit]
\label{Satz 4.9}
Sei $f: X \to \imdr$ messbar. Die folgenden Aussagen sind äquivalent:
\begin{enumerate}
 \item $f$ ist integrierbar.
 \item Es existieren integrierbare Funktionen $u, v: X \to [0,+\infty]$ mit $u(x)=v(x)=\infty$ für \textbf{kein} $x \in X$ und $f=u-v$ auf $X$.
 \item Es existiert eine integrierbare Funktion $g: X \to [0,+\infty]$ mit $\lvert f \rvert \leq g$ auf $X$.
 \item $\lvert f \rvert$ ist integrierbar.
\end{enumerate}
\end{satz}

\textbf{Zusatz:}
\begin{enumerate}
 \item $\fl^1(X) = \{f: X \to \mdr \mid f$ ist messbar und $\int_X \lvert f \rvert \text{ d}x < \infty\}$ (folgt aus (1)-(4)).
 \item Sind $u,v$ wie in (2), so gilt: $ \int_X f \text{ d}x = \int_X u \text{ d}x - \int_X v \text{ d}x$.
\end{enumerate}


\begin{beweis}[des Satzes]
\begin{enumerate}
 \item[(1) $\Rightarrow$ (2)] $u:= f_+$, $v := f_-$.
 \item[(2) $\Rightarrow$ (3)] $g := u+v$, dann ist $u,v \geq 0$, $g \geq 0$, $\int_X g \text{ d}x \stackrel{4.5}{=} \int_X u \text{ d}x + \int_X v \text{ d}x < \infty$. $\implies g$ ist integrierbar und: $|f| = |u-v| \leq |u| + |v| = u+v = g$ auf $X$.
 \item[(3) $\Rightarrow$ (4)] \ref{Satz 4.5} $\implies \int_X |f| \text{ d}x \leq \int_X g \text{ d}x < \infty \implies f$ ist integrierbar.
 \item[(4) $\Rightarrow$ (1)] $f_+, f_- \leq |f|$ auf $X$. $\implies 0 \leq \int_X f_\pm \text{ d}x \leq \int_X |f| \text{ d}x < \infty \stackrel{Def.}{\implies} f$ ist integrierbar.
\end{enumerate}
\end{beweis}

\begin{beweis}[des Zusatzes]
\begin{enumerate}
 \item \checkmark
 \item Es ist $f = u-v = f_+ - f_- \implies u+f_- = f_+ + v$.
\[\implies \int_X u \text{ d}x + \int_X f_- \text{ d}x \stackrel{4.5}{=} \int_X (u+ f_-) \text{ d}x = \int_X (f_+ + v) \text{ d}x \stackrel{4.5}{=} \int_X f_+ \text{ d}x + \int_X v \text{ d}x\]
\[\implies \int_X u \text{ d}x - \int_X v \text{ d}x = \int_X f_+ \text{ d}x - \int_X f_- \text{ d}x \stackrel{Def.}{=} \int_X f \text{ d}x. \]
\end{enumerate}
\end{beweis}

\begin{satz}[Folgerung]
\label{Folgerung 4.10}
\label{Satz 4.10}
Sei $f:X\to\imdr$ integrierbar und $N := \{\lvert f \rvert = +\infty\} = \{x\in X : \lvert f(x) \rvert = + \infty\}$. Dann ist $N\in \fb(X)$ und $\lambda(N) = 0$.
\end{satz}

\begin{beweis}
 $\ref{Satz 3.4} \implies N \in \fb(X).$ $n\mathds{1}_N \leq \lvert f \rvert$ für alle $n\in N$. Dann: 
\[n \cdot \lambda(N) = \int_X n\mathds{1}_N \text{ d}x \stackrel{4.5}{\leq} \int_X \lvert f \rvert \text{ d}x \stackrel{4.9}{<} \infty \text{  für alle } n \in \mdn\]
Also: $0 \leq n\lambda(N) \leq \int_X \lvert f \rvert \text{ d}x \quad \forall n \in \mdn \implies \lambda(N) = 0$ 
\end{beweis}

\begin{satz}
\label{Satz 4.11}
$f, g: X \to \imdr$ seien integrierbar und es sei $\alpha \in \mdr$.
\begin{enumerate}
 \item $\alpha f$ ist integrierbar und $\int_X (\alpha f) \text{ d}x = \alpha \int_X f \text{ d}x$.
 \item Ist $f+g:X\to\imdr$ auf $X$ definiert, so ist $f+g$ integrierbar und es gilt:
 \[\int_X (f+g)\text{ d}x = \int_X f \text{ d}x + \int_X g \text{ d}x\]
(Für $f=+\infty$ und $g=-\infty$ ist $f+g$ beispielsweise nicht definiert.)
 \item $\fl^1(X)$ ist ein reeller Vektorraum und die Abbildung $f \mapsto \int_X f \text{ d}x$ ist linear auf $\fl^1(X)$.
 \item $\max\{f,g\}$ und $\min\{f,g\}$ sind integrierbar.
 \item Ist $f\leq g$ auf $X$, so ist $\int_X f \text{ d}x \leq \int_X g \text{ d}x$.
 \item $\lvert \int_X f \text{ d}x \rvert \leq \int_X \lvert f \rvert \text{ d}x$. (Dreiecksungleichung für Integrale)
 \item Sei $\varnothing\ne Y \in \fb(X)$. Dann sind die Funktionen $f_{|Y}: Y \to \imdr$ und $\mathds{1}_Y\cdot f: X \to \imdr$ integrierbar und
\[\int_Y f(x) \text{ d}x := \int_Y f_{|Y} (x) \text{ d}x = \int_X(\mathds{1}_Y \cdot f)(x) \text{ d}x\]
 \item Sei $\lambda(X) < \infty$ und $h: X \to \mdr$ sei messbar und beschr\"ankt. Dann: $h \in \fl^1(X)$ und $\lvert \int_X h \text{ d}x\rvert \leq \|h\|_\infty \lambda(X) \quad$ (mit $\|h\|_\infty := \sup\{|h(x)| : x\in X\}$) 
\end{enumerate}

\end{satz}

\begin{beweis}
 
\end{beweis}

\begin{satz}
\label{Folgerung 4.12}
\begin{enumerate}
 \item Sind $\varnothing\ne A,B \in \fb(X)$ disjunkt, $X = A \cup B$ und ist $f: X \to \imdr$ integrierbar (über $X$), so ist $f$ integrierbar über $A$ und integrierbar über $B$ und es gilt:
 \[\int_X f \text{ d}x = \int_A f \text{ d}x + \int_B f \text{ d}x\]
 \item Ist $\varnothing \neq K \subseteq \mdr^d $ kompakt und $f:K\to\mdr$ stetig, so ist $f \in \fl^1(K)$.
\end{enumerate}

\end{satz}

\begin{beweis}
\begin{enumerate}
 \item Aus \ref{Satz 4.11}(7) folgt: $f$ ist integrierbar \"uber $A$ und integrierbar \"uber $B$. Es ist 
\[ \int_X f(x) \text{ d}x = \int_X \left( \mathds{1}_{A\cup B} \cdot f \right)(x) \text{ d}x = \int_X \left( \left( \mathds{1}_A + \mathds{1}_B \right) f\right)(x) \text{ d}x \]
\[= \int_X \left(\mathds{1}_A f + \mathds{1}_B f \right)(x) \text{ d}x \stackrel{4.11(2)}{=} \int_X \mathds{1}_A f \text{ d}x + \int_X \mathds{1}_B f \text{ d}x \stackrel{4.11(7)}{=} \int_A f \text{ d}x + \int_B f \text{ d}x.\]

 \item $K$ ist kompakt, also gilt: $\lambda(K) < \infty$. Aus \ref{Satz 3.2}(1) folgt, dass $f$ messbar ist. Analysis II (\glqq stetige Funktionen auf kompakten Mengen nehmen Minimum und Maximum an\grqq ) liefert: $f$ ist beschr\"ankt. Insgesamt folgt mit \ref{Satz 4.11}(8) schlie"slich: $f \in \fl^1(K)$.
\end{enumerate}
\end{beweis}

\begin{satz}
\label{Satz 4.13}
Seien $a,b\in\mdr$, $a<b$, $X:=[a,b]$ und $f\in C(X)$. Dann ist $f\in\fl^1(X)$ und es gilt:
\[L-\int_X f(x) \text{ d}x=R-\int_a^b f(x) \text{ d}x\]
\end{satz}

\begin{beweis}
Sei $\natn$, $t_j^{(n)}:=a+j\frac{b-a}{n}$ ($j=1,\dots,n$) und $I_j^{(n)}:=\left[t_{j-1}^{(n)},t_j^{(n)}\right]$ ($j=1,\dots,n$).
\begin{align*}
S_n:=\sum^n_{j=1} f \left(t_j^{(n)}\right) \underbrace{ \frac{b-a}{n}}_{= \lambda_1 \left(I_j^{(n)}\right)} \text{ ist Riemannsche Zwischensumme für R-} \int_a^bf(x)\,dx.
\end{align*}
Aus Analysis I folgt $S_n\to\text{R-}\int_a^bf(x)\,dx$ ($n\to\infty$). 
Definiere $f_n:=\sum^n_{j=1}f \left(t_j^{(n)} \right) \mathds{1}_{I_j^{(n)}} $. Dann ist $f_n$ einfach und 
\[\int_X f_n(x)\,dx=\sum_{j=1}^n f \left(t_j^{(n)} \right) \lambda_1 \left(I_j^{(n)}\right)=S_n\]
$f$ ist auf $X$ gleichmäßig stetig also konvergiert $f_n$ auf $X$ gleichmäßig gegen $f$ (Übung!), also gilt:
\[\lVert f_n-f \rVert_{\infty}=\text{sup} \left \{ \lvert f_n(x)-f(x) \rvert : x\in X \right\} \to 0 \  (n\to \infty)\]
Aus \ref{Folgerung 4.12}(2) folgt $f\in \mathfrak{L}^1(X)$
\begin{align*}
\left\lvert \text{L-} \int \limits_X f(x)\,dx -S_n \right\rvert = \left\lvert \text{L-} \int \limits_X (f-f_n)\,dx \right\rvert \stackrel{\text{4.11}}\leq \int \limits_X(f-f_n)\,dx \stackrel{\text{4.11}}\leq \lVert f-f_n \rVert_{\infty} \underbrace{\lambda(X)}_{=b-a} \to 0
\end{align*}
Daraus folgt $S_n \to$ L- $\int_X f\,dx$
\end{beweis}

\begin{satz}
\label{Satz 4.14}
Sei $a\in\mdr, X:=[a,\infty)$ und $f\in C(X)$. Dann gilt:
\begin{enumerate}
\item $f$ ist messbar.
\item $f\in\fl^1(X)$ genau dann wenn das uneigentliche Riemann-Integral $\int_a^\infty f(x) \text{ d}x$ \textbf{absolut} konvergent ist. In diesem Fall gilt:
\[L-\int_X f(x) \text{ d}x=R-\int_a^\infty f(x) \text{ d}x\]
Entsprechendes gilt für die anderen Typen uneigentlicher Riemann-Integrale.
\end{enumerate}
\end{satz}

\begin{beweis}
Eine Hälfte des Beweises folgt in §6.
\end{beweis}

\begin{beispiel}
\begin{enumerate}
\item Sei $X=(0,1]$, $f(x)=\frac{1}{\sqrt{x}}$. Aus Analysis I wissen wir, dass R-$\int^1_0\frac{1}{\sqrt{x}}\,dx$ (absolut) konvergent ist. Also ist $f\in\mathfrak{L}^1(X)$.\\
Außerdem wissen wir aus Analysis I, dass R-$\int_0^1\frac{1}{x}$ divergent ist. Also ist $f^2\notin\mathfrak{L}^1(X)$.
\item Sei $X=[0,\infty)$, $f(x)=\frac{\sin(x)}{x}$. Aus Analysis I wissen wir, dass R-$\int^{\infty}_1f(x)\,dx$ konvergent, aber nicht absolut konvergent ist. Also ist $f\notin\mathfrak{L}^1(X)$.
\end{enumerate}
\end{beispiel}

\chapter{Nullmengen}
In diesem Paragraphen sei stets $\varnothing\ne X\in\fb_d$. Wir schreiben wieder $\lambda$ statt $\lambda_d$.

\begin{definition}
\index{Nullmenge}\index{Borel!Nullmenge}
Sei $N\in\fb_d$. $N$ heißt eine \textbf{(Borel-)Nullmenge}, genau dann wenn $\lambda(N)=0$ ist.
\end{definition}

\begin{beispiel}
\begin{enumerate}
\item Ist $N\subseteq\mdr^d$ höchstens abzählbar, so ist $N\in\fb_d$ und $\lambda(N)=0$.
\item Sei $j\in\{1,\dots,d\}$ und $H_j:=\left\{(x_1,\dots,x_d) \in\mdr^d : x_j=0 \right\}$. Aus Beispiel (5) nach \ref{Satz 2.7} folgt, dass $H_j$ eine Nullmenge ist.
\end{enumerate}
\end{beispiel}

\begin{lemma}
\label{Lemma 5.1}
Seien $M,N,N_1,N_2,\ldots\in\fb_d$.
\begin{enumerate}
\item Ist $M\subseteq N$ und $N$ Nullmenge, dann ist $M$ Nullmenge.
\item Sind alle $N_j$ Nullmengen, so ist auch $\bigcup N_j$ eine Nullmenge.
\item $N$ ist genau dann eine Nullmenge, wenn für alle $\ep>0$ offene Intervalle $I_1,I_2,\ldots\subseteq\mdr^d$ existieren mit $N\subseteq\bigcup I_j$ und $\sum_{j=1}^\infty I_j\le\ep$.
\end{enumerate}
\end{lemma}

\begin{beweis}
\begin{enumerate}
\item $0\le\lambda(M)\le\lambda(N)=0$
\item $0\le\lambda(\bigcup N_j)\le\sum\lambda(N_j)=0$
\item Folgt aus \ref{Satz 2.10}.
\end{enumerate}
\end{beweis}

\begin{bemerkung}
$\ $
\begin{enumerate}
\item $\mdq$ ist "`klein"': $\mdq$ ist "`nur"' abzählbar.
\item $\mdq$ ist "`groß"': $\overline\mdq=\mdr$
\item $\mdq$ ist "`klein"': $\lambda(\mdq)=0$
\end{enumerate}
\end{bemerkung}

\begin{definition}
\index{für fast alle}
\index{fast überall}
\begin{enumerate}
\item Sei $(E)$ eine Eigenschaft für Elemente in $X$.\\
$(E)$ gilt \textbf{für fast alle} (ffa) $x\in X$, genau dann wenn $(E)$ \textbf{fast überall} (fü) (auf $X$) gilt, genau dann wenn eine Nullmenge $N\subseteq X$ existiert, sodass $(E)$ für alle $x\in X\setminus N$ gilt.
\item $\int_\varnothing f(x) \text{ d}x:=0$
\end{enumerate}
\end{definition}

\begin{satz}
\label{Satz 5.2}
Seien $f,g:X\to\imdr$ messbare Funktionen.
\begin{enumerate}
\item Ist $f$ integrierbar, so ist $f$ fast überall endlich.
\item Ist $f \ge0$ auf $X$, so ist $\int_X f(x)\text{ d}x=0$ genau dann wenn fast überall $f=0$.
\item Ist $f$ integrierbar und $N\subseteq X$ eine Nullmenge, so gilt:
\[\int_N f(x)\text{ d}x=0\] 
\end{enumerate}
\end{satz}

\begin{beweis}
\begin{enumerate}
\item ist gerade \ref{Folgerung 4.10}.
\item ist gerade \ref{Satz 4.5}(3)
\item Setze $g:=\mathds{1}_N f$. Aus \ref{Satz 4.11} folgt, dass g integrierbar ist, also ist nach \ref{Satz 4.9} auch $\lvert g \rvert$ integrierbar. Für $x\in X\setminus N$ gilt: 
\[g(x)=\lvert g(x) \rvert =0\]
D.h. $\lvert g \rvert =0$ fast überall. Aus (2) folgt damit $\int_X \lvert g \rvert \,dx = 0$. Dann ist mit \ref{Satz 4.11}: \[\left\lvert\int_X g\,dx \right\rvert \leq \int_X \lvert g \rvert \,dx =0\] 
und somit $\int_X g\,dx=0$.
\end{enumerate}
\end{beweis}

\begin{satz}
\label{Satz 5.3}
$f,g:X\to\imdr$ seien messbar.
\begin{enumerate}
\item Ist $f$ integrierbar und gilt fast überall $f=g$, so ist $g$ integrierbar und es gilt:
\[\int_Xf\,dx=\int_Xg\,dx\]
\item Ist $f$ integrierbar und $g:=\mathds{1}_{\{ \lvert f \rvert <\infty \}}\cdot f$, so ist $g$ integrierbar und es gilt: \[\int_Xf\,dx=\int_Xg\,dx\]
\item Sind $f$ und $g$ beide $\geq0$ auf $X$, und ist fast überall $f=g$, so ist 
\[\int_Xf\,dx=\int_Xg\,dx\]
\end{enumerate}
\end{satz}

\begin{beweis}
\begin{enumerate}
\item Nach Voraussetzung existiert eine Nullmenge $N\subseteq X$, sodass gilt:
\[\forall x\in X\setminus N:f(x)=g(x)\] 
Aus \ref{Satz 5.2}(3) folgt dann $\int_N f\,dx=0$.
Sei $x\in X\setminus N$ Dann gilt: 
\[\left( \mathds{1}_N \lvert g \rvert \right)(x)=\mathds{1}_N(x)\cdot \lvert g(x) \rvert=0\] 
D.h.: Fast überall ist $\mathds{1}_N \lvert g \rvert =0$. Aus \ref{Satz 5.2}(2) folgt $\int_N \lvert g \rvert\,dx=\int_X\mathds{1}_N\cdot \lvert g \rvert\,dx=0$.
Dann gilt:
\begin{align*}
\int_X \lvert g\rvert\,dx & = \int_X \left(\mathds{1}_N \lvert g\rvert + \mathds{1}_{X\setminus N} \lvert g\rvert \right)\,dx\\ 
 &= \int_X\mathds{1}_N \lvert g\rvert\,dx + \int _X\mathds{1}_{X\setminus N} \lvert g\rvert\,dx\\
 &= \int_X \mathds{1}_{X\setminus N} \lvert g \rvert\,dx\\
& \leq\int_X \lvert f\rvert\,dx \overset{\ref{Satz 4.9}}< \infty
%hier soll eigentlich das kleinergleich unter das erste gleichzeichen...
\end{align*}
\ref{Satz 4.9} liefert nun, dass $\lvert g\rvert$ und damit auch $g$ integrierbar ist. Weiter gilt:
\begin{align*}
\int_Xg\,dx &\overset{\ref{Folgerung 4.12}} = \int_N g\,dx+ \int_{X\setminus N}g\,dx = \int_{X\setminus N}g\,dx\\
&= \int_{X\setminus N}f\,dx \overset{\ref{Satz 5.2}(3)}= \int_N f\,dx +\int_{X\setminus N}f\,dx\\
&\overset{\ref{Folgerung 4.12}}= \int_X f\,dx.
\end{align*}

\item Setze $N:=\left\{\lvert f\rvert =\infty \right\}$. Aus \ref{Satz 5.2}(1) folgt, dass $N$ eine Nullmenge ist. Sei $x\in X\setminus N$, so ist $x\in \left\{\lvert f\rvert <\infty \right\}$ und $g(x)=f(x)$.
D.h. fast überall ist $f=g$. (Klar: $g$ ist mb). Dann folgt die Behauptung aus (1).
\item \textbf{Fall 1:} $\int_Xf\,dx<\infty$\\
Dann ist $f$ integrierbar, damit ist nach (1) auch $g$ integrierbar und es gilt:
\[\int_Xf\,dx=\int_Xg\,dx\]
\textbf{Fall 2:} $\int_Xf\,dx=\infty$.\\
Annahme: $\int_Xg\,dx<\infty$. Dann gilt nach Fall 1: $\int_Xf\,dx<\infty$. $\lightning$
\end{enumerate}
\end{beweis}

\begin{definition}
$(f_n)$ sei eine Folge von Funktionen $f_n:X\to\imdr$.
\begin{enumerate}
\item $(f_n)$ konvergiert fast überall (auf $X$) genau dann, wenn eine Nullmenge $N\subseteq X$ existiert, sodass für alle  $x\in X\setminus N$ $\left(f_n(x)\right)$ in $\imdr$ konvergiert.
\item Sei $f:X\to\imdr$. $(f_n)$ konvergiert fast überall (auf $X$) gegen $f$ genau dann, wenn eine Nullmenge $N\subseteq X$ existiert mit: $f_n(x)\to f(x) \forall x\in X\setminus N$\\
In diesem Fall schreiben wir: $f_n\to f$ fast überall.
\end{enumerate}
\end{definition}

\begin{satz}
\label{Satz 5.4}
Sei \((f_{n})\) eine Folge messbarer Funktionen \(f_{n}: X\to\imdr\) und \((f_{n})\) konvergiere fast \"uberall (auf \(X\)).
Dann:
\begin{enumerate}
\item Es existiert \(f: X\to\imdr\) messbar mit \(f_{n}\to f\) fast \"uberall.
\item Ist \(g: X\to\imdr\) eine Funktion mit \(f_{n}\to g\) fast \"uberall, so gilt \(f=g\) fast \"uberall.
\end{enumerate}
\end{satz}

\begin{bemerkung}
Ist \(g\) wie in (2), so muss \(g\) nicht messbar sein (ein Beispiel gibt es in der \"Ubung).
\end{bemerkung}

\begin{beweis}
\begin{enumerate}
\item Es existiert eine Nullmenge \(N_{1}\subseteq X:\,(f_{n}(x))\) konvergiert in \(\imdr\) für alle 
\(x\in X\setminus N_{1}\).
\[
f(x)=\begin{cases}0&x\in N_{1}\\\lim_{n\to\infty}{f_{n}(x)}&x\in X\setminus N_{1}\end{cases}
\]
\(g_{n}:=\mathds{1}_{X\setminus N}\cdot f_{n}\), \(g_{n}\) ist messbar und \(g_{n}(x)\to f(x)\) für alle \(x\in X\).
Mit \ref{Satz 3.5} folgt: \(f\) ist messbar.
\item Es existiert eine Nullmenge \(N_{2}\subseteq X:\,f_{n}(x)\to g(x)\,\forall x\in X\setminus N_{2}\). 
\(N=N_{1}\cup N_{2}\). Aus \ref{Lemma 5.1} folgt: \(N\) ist eine Nullmenge. 

Für \(x\in X\setminus N:\,f(x)=g(x)\).
\end{enumerate}
\end{beweis}

\begin{satz}[Satz von Beppo Levi (Version III)]
\label{Satz 5.5}
Sei \((f_{n})\) eine Folge messbarer Funktionen \(f_{n}:\,X\to[0,+\infty]\) und für jedes \(n\in\mdn\) gelte:
\(f_{n}\leq f_{n+1}\) fast überall.  Dann existiert eine messbare Funktion
\(f:X\to[0,+\infty]\) mit: \(f_{n}\to f\) fast \"uberall und \(\int_{X}{f\mathrm{d}x}=\lim_{n\to\infty}{\int_{X}{f_{n}\mathrm{d}x}}\).
\end{satz}

\begin{beweis}
Zu jedem \(n\in\mdn\) existiert eine Nullmenge 
\(N_{n}:\,f_{n}(x)\leq f_{n+1}(x)\forall x\in X\setminus N_{n}\). 
\(N:=\bigcup_{n=1}^{\infty}{N_{n}}\); Mit \ref{Lemma 5.1} folgt: \(N\) ist eine
Nullmenge.

Dann: \(f_{n}(x)\leq f_{n+1}(x)\forall x\in X\setminus N\forall n\in\mdn\).

\(\hat{f}_{n}:=\mathds{1}_{X\setminus N}\cdot f_{n}\), \(\hat{f}_{n}\) ist 
messbar, \(\hat{f}_{n}\leq\hat{f}_{n+1}\) auf \(X\) f\"ur alle \(n\in\mdn\).

\(f(x):=\lim_{n\to\infty}{\hat{f}_{n}(x)}\,(x\in X)\); \ref{Satz 3.5} liefert:
\(f\) ist messbar. Weiter: \(\hat{f}_{n}\to f\).
\[
\int_{X}{f\mathrm{d}x}\overset{\text{\ref{Satz 4.6}}}{=}\lim_{n\to\infty}{\int_{X}{\hat{f}_{n}\mathrm{d}x}}\overset{\text{\ref{Satz 5.3}.(2)}}{=}\lim_{n\to\infty}{\int_{X}{f_{n}\mathrm{d}x}}
\]
\end{beweis}

\chapter{Der Konvergenzsatz von Lebesgue}
Stets in diesem Paragraphen: \(\varnothing\neq X\in\fb_{d}\)

\begin{lemma}[Lemma von Fatou]
\label{Lemma 6.1}
\((f_{n})\) sei eine Folge messbarer Funktionen \(f_{n}:\,X\to[0,+\infty]\).
\begin{enumerate}
\item \(\int_{X}{(\liminf_{n\to\infty}f_{n})(x)\mathrm{d}x}\leq\liminf_{n\to\infty}{\int_{X}{f_{n}(x)\mathrm{d}x}}\)
\item Ist \(f: X\to[0,+\infty]\) messbar und gilt \(f_{n}\to f\) fast \"uberall,
so ist
\[
\int_{X}{f\mathrm{d}x}\leq\liminf_{n\to\infty}{\int_{X}{f_{n}\mathrm{d}x}}
\]
\item Ist \(f\) wie in (2) und ist \(\left(\int_{X}{f_{n}\mathrm{d}x}\right)\)
beschr\"ankt, so ist \(f\) integrierbar.
\end{enumerate}
\end{lemma}

\begin{beweis}
Kommt noch\ldots
\end{beweis}

\begin{satz}[Konvergenzsatz von Lebesgue (Majorisierte Konvergenz)]
\label{Satz 6.2}
\((f_{n})\) sei eine Folge messbarer Funktionen \(f_{n}:X\to\imdr\), \((f_{n})\)
konvergiere fast \"uberall und es sei \(g:X\to[0,+\infty]\) integrierbar. F\"ur
jedes \(n\in\mdn\) gelte \(\lvert f_{n}\rvert\leq g\) fast \"uberall. Dann sind
alle \(f_{n}\) integrierbar und es existiert ein \(f\in\fl^{1}(X)\) mit:
\begin{enumerate}
\item \(f_{n}\to f\) fast \"uberall
\item \(\int_{X}{f_{n}\mathrm{d}x}\to\int_{X}{f\mathrm{d}x}\)
\item \(\int_{X}{\lvert f_{n}-f\rvert\mathrm{d}x}\to 0\)
\end{enumerate}
\end{satz}

\begin{beispiel}
Kommt noch\ldots
\end{beispiel}

\begin{beweis}
Kommt noch\ldots
\end{beweis}

\begin{beispiel}
Sei \(X:=[1,\infty)\) und \(f_n(x):=\frac1{x^\frac32}\sin\left(\frac xn \right) \) für alle \(x\in X, n\in\mdn\) mit \(f_n(x)\to f(x)\equiv 0\) für jedes \(x\in X\).
Dann ist \(\lvert f_n(x) \rvert\leq \frac1{x^\frac32}\) für jedes \(x\in X\) und $\natn$. 
Definiere nun \[g(x):=\frac1{x^\frac32}\]
Aus Analysis I ist bekannt, dass \(\int^\infty_1 g(x)\,dx\) (absolut) konvergent ist 
und aus \ref{Satz 4.14} folgt \[g\in\mathfrak{L}^1(X) \text{ sowie } \int_X g(x)\,dx = \text{R-}\int^\infty_1 g(x)\,dx\]
Weiter folgen aus \ref{Satz 6.2}:
\[\int_X f_n\,dx\to 0 \text{ und } \int_X\lvert fn\rvert\,dx\to 0 \ (n\to\infty) \]
\end{beispiel}

\begin{folgerung}[aus \ref{Satz 6.2}]
\label{Folgerung 6.3}
\begin{enumerate}
	\item 	Sei \(f:X\to\imdr\) messbar und \((A_n)\) sei eine Folge in \(\fb(X)\) mit \(A_n\subseteq A_{n+1}\) für jedes $\natn$ und \(X=\bigcup A_n\). Weiter sei
		\begin{align*}
		f_n:=\mathds{1}_{A_n}\cdot f \text{ integrierbar für alle } \natn \intertext{und} \left(\int_{A_n}\lvert f\rvert\,dx\right) \text{ sei beschränkt. }
		\end{align*}
		Dann ist $f$ integrierbar und es gilt: \[\int_{A_n}f\,dx = \int_Xf\,dx\]
	\item 	Sei \(a\in\mdr\), \(X:=[a,\infty]\) und \(f:X\to\mdr\) sei stetig. Weiter sei R-\(\int_a^\infty f\,dx\) \textbf{absolut} konvergent. Dann ist \(f\in\mathfrak{L}^1(X)\) und wie in 					\ref{Satz 4.14}:
		\[\text{L-}\int_Xf\,dx=\text{R-}\int^\infty_a f\,dx \]
\end{enumerate}
\end{folgerung}

\begin{beweis}
\begin{enumerate}
	\item 	Sei \(x\in X\). Es exisitert ein $m\in\mdn$, für das \(x\in A_m\) ist und somit auch \(x\in A_n \) für jedes \(n\geq m\). Nach der Definition von $f_n$ gilt dann \(f_n(x)=f(x)\) für jedes 				\(n\geq m\) und somit \(f_n\to f\) auf $X$. Damit gilt auch \[\lvert f_n\rvert\to\lvert f\rvert \text{ auf } X\] Durch die Konstruktion der $f_n$ ergibt sich: 
		\[ \lvert f_n\rvert=\lvert \mathds{1}_{A_n}f\rvert=\mathds{1}_{A_n}\lvert f\rvert \leq \mathds{1}_{A_{n+1}}\lvert f\rvert=\lvert f_{n+1}\rvert \]
		Dann gilt:
		\[ \int_X \lvert f\rvert\,dx \overset{\ref{Satz 4.6}}=\lim\int_X \lvert f_n\rvert\,dx = \lim\int_{A_n} \lvert f\rvert\,dx \overset{Vor.}<\infty \]
		Es folgt, dass \(\lvert f\rvert\) integrierbar ist und somit ist nach \ref{Satz 4.9} auch $f$ integrierbar. Da \(\lvert f_n\rvert \leq \lvert f\rvert\) auf $X$ für jedes \(\natn\) gilt, ist $f$ eine 		
		integrierbare Majorante und es folgt mit \ref{Satz 6.2}:
		\[ \int_Xf\,dx = \lim\int_Xf_n\,dx = \lim\int_{A_n}f\,dx \]
	\item 	Setze \(A_n:=[a,n]\ (\natn)\) und es gelte o.B.d.A.: \(a\leq 1\). Dann gilt:
		\[ \int_{A_n}\lvert f\rvert\,dx \overset{\ref{Satz 4.13}}= \text{R-}\int^n_a \lvert f\rvert\,dx \overset{Vor.}\longrightarrow \text{R-}\int^\infty_a \lvert f\rvert\,dx \]
		D.h.\(\left(\int_{A_n}\lvert f\rvert\,dx\right)\) ist beschränkt. Definiere \(f_n:=\mathds{1}_{A_n}f\) mit \ref{Satz 4.13} folgt daraus, dass $f_n$ integrierbar ist. Weiter folgt
		aus (1) \(f\in\mathfrak{L}^1(X)\). Es ist \(f(X)\subseteq\mdr\) und 
		\[ \text{L-}\int_Xf\,dx = \lim\int_{A_n}f\,dx \overset{\ref{Satz 4.13}}= \lim\left(\text{R-}\int^n_a f\,dx \right) = \text{R-}\int^\infty_a f\,dx. \]
\end{enumerate}
\end{beweis}

\begin{bemerkung}
\ref{Folgerung 6.3}(2) gilt entsprechend für die anderen Typen uneigentlicher Riemann-Integrale.
\end{bemerkung}

\begin{folgerung}
\label{Folgerung 6.4}
\begin{enumerate}
	\item 	\((f_n)\) sei eine Folge integrierbarer Funktionen \(f_n\colon X\to\imdr\), \(g\colon X\to[0,+\infty]\) sei ebenfalls integrierbar und 
		\[g_n:=f_1+f_2+\ldots+f_n \ (\natn)\]
		Weiter sei $N$ eine Nullmenge in $X$ so, dass \((g_n(x))\) für jedes \(x\in X\setminus N\) in $\imdr$  konvergiert und 
		\[\lvert g_n(x)\rvert \leq g(x) \text{ für jedes } \natn \text{ und } x\in X\setminus N\]
		Setzt man
		\[f(x):=\sum^\infty_{j=1}f_j(x):=	
		\begin{cases}
			0, 				& \text{falls } x\in N 			\\
			\lim\limits_{n\to\infty}g_n(x), & \text{falls } x\in X\setminus N
		\end{cases},\]
		so gilt, dass $f$ integrierbar ist und
		\[\int_X \left( \sum^\infty_{j=1}f_j(x) \right)\,dx = \sum^\infty_{j=1}\left( \int_Xf_j(x)\,dx \right) \]
	\item 	Sei \(f\in\mathfrak{L}^1(X)\) und \((A_n)\) eine \textbf{disjunkte} Folge in \(\fb(X)\) mit \(X=\dot\bigcup A_n\). Dann gilt
		\[\int_Xf\,dx = \sum^\infty_{j=1}\int_{A_j}f\,dx \]
\end{enumerate}
\end{folgerung}

\begin{beweis}
\begin{enumerate}
	\item 	Fast überall gelten \(g_n\to f\) und für jedes \(\natn\) auch \(\lvert g_n\rvert \leq g\). Aus \ref{Satz 6.2} folgt
		\begin{align*}
			\int_X \left(\sum^\infty_{j=1}f_j(x)\right) \,dx 
			&= \int_Xf\,dx  					\\
			&\overset{\ref{Satz 6.2}}= \lim\int_Xg_n\,dx 	\\
			&= \lim\int_X\left(\sum^n_{j=1}f_j\right)\,dx 	\\
			&=\lim\sum^n_{j=1}\int_Xf_j(x)\,dx 			\\
			&=\sum^\infty_{j=1}\int_Xf_j\,dx 			\\
		\end{align*}
	\item 	Setze \(f_j:=\mathds{1}_{A_j}f\), \(g:=\lvert f\rvert\), \(g_n:=f_1+\ldots+f_n\). Dann ist
		\[\lvert g_n\rvert = \lvert \mathds{1}_{A_1\cup\ldots\cup A_n}\cdot f\rvert \leq \lvert f\rvert =g \]
		Es gilt: \(g_n\to f\) auf $X$. Aus (1) folgt
		\[ \int_Xf\,dx = \sum^\infty_{j=1}\int_Xf_j\,dx = \sum^\infty_{j=1}\int_{A_j}f\,dx \]
\end{enumerate}
\end{beweis}


\chapter{Parameterintegrale}

In diesem Paragraphen sei stets \(\varnothing\neq X\in \fb_d\).

\appendix
\chapter{Satz um Satz (hüpft der Has)}
\listtheorems{satz,wichtigedefinition}

\renewcommand{\indexname}{Stichwortverzeichnis}
\addcontentsline{toc}{chapter}{Stichwortverzeichnis}
\printindex

\chapter{Credits für Analysis III} Abgetippt haben die folgenden Paragraphen:\\% no data in Ana2Vorwort.tex
\textbf{§ 1: $\sigma$-Algebren und Maße}: Rebecca Schwerdt, Peter Pan, Philipp Ost\\
\textbf{§ 2: Das Lebesguemaß}: Rebecca Schwerdt, Philipp Ost\\
\textbf{§ 3: Messbare Funktionen}: Rebecca Schwerdt, Philipp Ost\\
\textbf{§ 4: Konstruktion des Lebesgueintegrals}: Rebecca Schwerdt, Philipp Ost, Peter Pan\\
\textbf{§ 5: Nullmengen}: Rebecca Schwerdt, Jan Ihrens, Philipp Ost\\


\end{document}
