\documentclass[a4paper,twoside,DIV15,BCOR12mm,chapterprefix=true,headings=onelinechapter]{scrbook}
\usepackage{ana}

\author{Die Mitarbeiter von \url{http://mitschriebwiki.nomeata.de/}}
\title{Analysis III - Bachelorversion}
\makeindex

\begin{document}
\maketitle

\renewcommand{\thechapter}{\Roman{chapter}}
%\chapter{Inhaltsverzeichnis}
\addcontentsline{toc}{chapter}{Inhaltsverzeichnis}
\tableofcontents

\chapter{Vorwort}

\section{Über dieses Skriptum}
Dies ist ein Mitschrieb der Vorlesung \glqq Analysis III\grqq\ von Herrn Schmoeger im
Wintersemester 2010 an der Universität Karlsruhe (KIT). Die Mitschriebe der Vorlesung werden mit ausdrücklicher Genehmigung 
von Herrn Schmoeger hier veröffentlicht, Herr Schmoeger ist für den Inhalt nicht 
verantwortlich.

\section{Wer}
Gestartet wurde das Projekt von Joachim Breitner. Beteiligt an diesem Mitschrieb sind \ldots na mal schaun.

\section{Wo}
Alle Kapitel inklusive \LaTeX-Quellen können unter \url{http://mitschriebwiki.nomeata.de} abgerufen werden.
Dort ist ein \emph{Wiki} eingerichtet und von Joachim Breitner um die \LaTeX-Funktionen erweitert.
Das heißt, jeder kann Fehler nachbessern und sich an der Entwicklung
beteiligen. Auf Wunsch ist auch ein Zugang über \emph{Subversion} möglich.


\renewcommand{\thechapter}{\arabic{chapter}}
\renewcommand{\chaptername}{§}
\renewcommand*{\chapterformat}{§\,\thechapter \enskip}
\setcounter{chapter}{-1}

\chapter{Vorbereitungen}
In diesem Paragraphen seien $X,Y,Z$ Mengen ($\ne\emptyset$) und $f:X\to Y, g:Y\to Z$ Abbildungen.
\begin{enumerate}
\index{Potenzmenge}
\index{Disjunktheit}
\item 
\begin{enumerate}
\item $\mathcal{P}(X):=\{A:A\subseteq X\}$ heißt \textbf{Potenzmenge} von $X$.
\item Sei $\fm\subseteq\mathcal{P}(X)$, so heißt $\fm$ \textbf{disjunkt}, genau dann wenn $A\cap B=\emptyset$ für $A,B\in\fm$ mit $A\ne B$.
\item Sei $(A_j)$ eine Folge in $\mathcal{P}(X)$ (also $A_j\subseteq X$), so heißt $(A_j)$ \textbf{disjunkt}, genau dann wenn $\{A_1,A_2,\ldots\}$ disjunkt ist. In diesem Fall schreibe: $\dot{\bigcup}_{j=1}^\infty:=\bigcup_{j=1}^\infty A_j$\\
Allgemein sei $\bigcup_{j=1}^\infty A_j:=\bigcup A_j$ und $\bigcap_{j=1}^\infty A_j:=\bigcap A_j$.
\end{enumerate}
\item Sei $A\subseteq X$, für $x\in X$ definiere
\[\mathds{1}_A(x):=\begin{cases}1, x\in A\\ 0, x\in A^C\end{cases}\]
wobei $A^C:=X\setminus A$.
\item Sei $B\subseteq Y$ dann ist $f^{-1}(B):=\{x\in X: f(x)\in B\}$ und es gelten folgende Eigenschaften:
\begin{enumerate}
\item $f^{-1}(B^C)=f^{-1}(B)^C$
\item Ist $B_j$ eine Folge in $\mathcal{P}(Y)$, so gilt:
\begin{align*}
f^{-1}(\bigcup B_j)=\bigcup f^{-1}(B_j)\\
f^{-1}(\bigcap B_j)=\bigcap f^{-1}(B_j)\\
\end{align*}
\item Ist $C\subseteq Z$, so gilt:
\[(g\circ f)^{-1}(C)=f^{-1}(g^{-1}(C))\]
\end{enumerate}
\item $\sum_{j=1}^\infty a_j =: \sum a_j$
\end{enumerate}

\chapter{$\sigma$-Algebren und Maße}
In diesem Paragraphen sei $\emptyset\ne X$ eine Menge.

\begin{definition}
\index{$\sigma$-Algebra}
Sei $\fa\subseteq\mathcal{P}(X)$, $\fa$ heißt eine \textbf{$\sigma$-Algebra} auf $X$, wenn gilt:
\begin{enumerate}
\item[($\sigma_1$)] $X\in\fa$
\item[($\sigma_2$)] Ist $A\in\fa$, so ist auch $A^C\in\fa$.
\item[($\sigma_3$)] Ist $(A_j)$ eine Folge in $\fa$, so ist $\bigcup A_j\in\fa$.
\end{enumerate}
\end{definition}

\begin{beispiel}
\begin{enumerate}
\item $\{X,\emptyset\}$ und $\mathcal{P}(X)$ sind $\sigma$-Algebren auf $X$.
\item Sei $A\subseteq X$, dann ist $\{X,\emptyset, A, A^C\}$ eine $\sigma$-Algebra auf $X$.
\item $\fa:=\{A\subseteq X: A$ abzählbar oder $A^C$ abzählbar$\}$ ist eine $\sigma$-Algebra auf $X$.
\end{enumerate}
\end{beispiel}

\begin{lemma}
Sei $\fa$ eine $\sigma$-Algebra auf $X$, dann:
\begin{enumerate}
\item $\emptyset\in\fa$
\item Ist $(A_j)$ eine Folge in $\fa$, so ist $\bigcap A_j\in\fa$.
\item Sind $A_1,\ldots,A_n\in\fa$, so gilt:
\begin{enumerate}
\item $A_1\cup\cdots\cup A_n\in\fa$
\item $A_1\cap\cdots\cap A_n\in\fa$
\item $A_1\setminus A_2\in\fa$
\end{enumerate}
\end{enumerate}
\end{lemma}

\begin{beweis}
\begin{enumerate}
\item $\emptyset=X^C\in\fa$ (nach ($\sigma_2$)).
\item $D:=\bigcap A_j$. $D^C=\bigcup A_j^C\in\fa$ (nach ($\sigma_2$) und ($\sigma_3$)), also gilt auch $D=(D^C)^C\in\fa$.
\item \begin{enumerate}
\item $A_1\cup\cdots\cup A_n\in\fa$ folgt aus ($\sigma_3$) mit $A_{n+j}:=\emptyset$ ($j\ge 1$).
\item $A_1\cap\cdots\cap A_n\in\fa$ folgt aus (2) mit $A_{n+j}:=X$ ($j\ge 1$).
\item $A_1\setminus A_2=A_1\cap A_2^C\in\fa$
\end{enumerate}
\end{enumerate}
\end{beweis}

\begin{lemma}
Sei $\emptyset\ne\cf$ eine Menge von $\sigma$-Algebren auf $X$. Dann ist 
\[\fa_0:=\bigcap_{\fa\in\cf}\fa\]
eine $\sigma$-Algebra auf $X$.
\end{lemma}

\begin{beweis}
\begin{enumerate}
\item[($\sigma_1$)] $\forall\fa\in\cf:X\in\fa\implies X\in\fa_0$.
\item[($\sigma_2$)] Sei $A\in\fa_0$, dann gilt:
\begin{align*}
\forall\fa\in\cf:A\in\fa &\implies \forall\fa\in\cf:A^C\in\fa\\
&\implies A^C\in\fa_0
\end{align*}
\item[($\sigma_3$)] Sei $(A_j)$ eine Folge in $\fa_0$, dann ist $(A_j)$ Folge in $\fa$ für alle $\fa\in\cf$, dann gilt:
\begin{align*}
\forall\fa\in\cf:\bigcap A_j\in\fa \implies \bigcap A_j\in\fa_0
\end{align*}
\end{enumerate}
\end{beweis}

\begin{definition}
\index{Erzeuger}
Sei $\emptyset\ne\mathcal{E}\subseteq\mathcal{P}(X)$ und $\cf:=\{\fa:\fa$ ist $\sigma$-Algebra auf $X$ mit $\mathcal{E}\subseteq\fa\}$. Definiere
\[\sigma(\mathcal{E}):=\bigcap_{\fa\in\cf}\fa\]
Dann ist wegen 1.2 $\sigma(\mathcal{E})$ eine $\sigma$-Algebra auf $X$. $\sigma(\mathcal{E})$ heißt die \textbf{von $\mathcal{E}$ erzeugte $\sigma$-Algebra}. $\mathcal{E}$ heißt ein \textbf{Erzeuger} von $\sigma(\mathcal{E})$.
\end{definition}

\begin{lemma}
Sei $\emptyset\ne\mathcal{E}\subseteq\mathcal{P}(X)$.
\begin{enumerate}
\item $\mathcal{E}\subseteq\sigma(\mathcal{E})$. $\sigma(\mathcal{E})$ ist die "kleinste" $\sigma$-Algebra auf $X$, die $\mathcal{E}$ enthält.
\item Ist $\mathcal{E}$ eine $\sigma$-Algebra, so ist $\sigma(\mathcal{E})=\mathcal{E}$.
\item Ist $\mathcal{E}\subseteq\mathcal{E}'$, so ist $\sigma(\mathcal{E})\subseteq\sigma(\mathcal{E}')$.
\end{enumerate}
\end{lemma}

\begin{beweis}
\begin{enumerate}
\item Klar nach Definition.
\item $\fa:=\mathcal{E}$, dann gilt $\fa\subseteq\sigma(\mathcal{E})\subseteq\fa$.
\item $\mathcal{E}\subseteq\mathcal{E}'\subseteq\sigma(\mathcal{E}')$, also folgt nach Definition $\sigma(\mathcal{E})\subseteq\sigma(\mathcal{E}')$.
\end{enumerate}
\end{beweis}

\begin{beispiel}
\begin{enumerate}
\item Sei $A\subseteq X$ und $\mathcal{E}:=\{A\}$. Dann ist $\sigma(\mathcal{E})=\{X,\emptyset,A,A^C\}$.
\item $X:=\{1,2,3,4,5\}, \mathcal{E}:=\{\{1\},\{1,2\}\}$. Dann gilt:
\[\sigma(\mathcal{E}):=\{X,\emptyset, \{1\},\{2\},\{1,2\},\{3,4,5\},\{1,3,4,5\},\{2,3,4,5\}\}\]
\end{enumerate}
\end{beispiel}

\begin{erinnerung}
\index{Offenheit}\index{Abgeschlossenheit}
Sei $d\in\mdn, X\subseteq\mdr^d$. $A\subseteq X$ heißt \textbf{offen} (\textbf{abgeschlossen}) in $X$, genau dann wenn ein offenes (abgeschlossenes) $G\subseteq\mdr^d$ existiert mit $A=X\cap G$.\\
Beachte: $A$ abgeschlossen in $X$ $\iff$ $X\setminus A$ offen in $X$.
\end{erinnerung}

\begin{definition}
\index{Borel!$\sigma$-Algebra}\index{$\sigma$-Algebra!Borelsche}
\index{Borel!Mengen}
Sei $X\subseteq\mdr^d$.
\begin{enumerate}
\item $\mathcal{O}(X):=\{A\subseteq X:A$ ist offen in $X\}$
\item $\fb(X):=\sigma(\mathcal{O}(X))$ heißt \textbf{Borelsche $\sigma$-Algebra} auf $X$.
\item $\fb_d:=\fb(\mdr^d)$. Die Elemente von $\fb_d$ heißen \textbf{Borelsche Mengen} oder \textbf{Borel-Mengen}.
\end{enumerate}
\end{definition}

\begin{beispiel}
\begin{enumerate}
\item Sei $X\subseteq\mdr^d$. Ist $A\subseteq$ offen (abgeschlossen) in $X$, so ist $A\in\fb(X)$.
\item Ist $A\subseteq\mdr^d$ offen (abgeschlossen) so ist $A\in\fb_d$.
\item Sei $d=1, A=\mdq$. $\mdq$ ist abzählbar, also $\mdq=\{r_1,r_2,\ldots\}$ (mit $r_i\ne r_j$ für $i\ne j$). Also ist $\mdq=\bigcup \{r_j\}$. Sei nun $r\in\mdq$, dann ist $B:=(-\infty,r)\cup(r,\infty)\in\fb_1$. Daraus folgt $\{r_j\}\in\fb_1$, also auch $\mdq\in\fb_1$.\\
Allgemeiner lässt sich zeigen: $\mdq^d:=\{(x_1,\ldots,x_n):x_j\in\mdq (j=1,\ldots,n)\}\in\fb_d$.
\end{enumerate}
\end{beispiel}

\begin{definition}
\index{Intervall}
\index{Halbraum}
\begin{enumerate}
\item Seien $I_1,\ldots,I_d$ Intervalle in $\mdr$. $I_1\times\cdots\times I_d$ heißt ein \textbf{Intervall} in $\mdr^d$.
\item Seien $a=(a_1,\ldots,a_d), b=(b_1,\ldots,b_d)\in\mdr^d$.
\[a\le b:\iff a_j\le b_j\quad (j=1,\ldots,d)\]
\item Seien $a,b\in\mdr^d$ und $a\le b$.
\begin{align*}
(a,b)&:=(a_1,b_1)\times\cdots\times(a_d,b_d)\\
(a,b]&:=(a_1,b_1]\times\cdots\times(a_d,b_d]\\
[a,b)&:=[a_1,b_1)\times\cdots\times[a_d,b_d)\\
[a,b]&:=[a_1,b_1]\times\cdots\times[a_d,b_d]
\end{align*}
mit der Festlegung $(a,b):=(a,b]:=[a,b):=\emptyset$, falls $a_j=b_j$ für ein $j\in\{1,\ldots,d\}$.
\item Für $k\in\{1,\ldots,d\}$ und $\alpha\in\mdr$ definiere die folgenden \textbf{Halbräume}:
\begin{align*}
H_k^-(\alpha):=\{(x_1,\ldots,x_d)\in\mdr^d:x_k\le\alpha\}\\
H_k^+(\alpha):=\{(x_1,\ldots,x_d)\in\mdr^d:x_k\ge\alpha\}
\end{align*}
\end{enumerate}
\end{definition}

\begin{satz}
Es seien $\ce_1,\ce_2,\ce_3$ wie folgt definiert:
\begin{align*}
\ce_1&:=\{(a,b):a,b\in\mdq^d,a\le b\}\\
\ce_2&:=\{(a,b]:a,b\in\mdq^d, a\le b\}\\
\ce_3&:=\{H^-_k(\alpha):\alpha\in\mdq, k=1,\ldots,d\}
\end{align*}
Dann gilt:
\[\fb_d=\sigma(\ce_1)=\sigma(\ce_2)=\sigma(\ce_3)\]
Entsprechendes gilt für die anderen Typen von Intervallen und Haupträumen.
\end{satz}

\begin{beweis}
\begin{enumerate}
\item Sei $G\in\co(\mdr^d), \fm:=\{(a,b):a,b\in\mdq^d,a\le b, (a,b)\subseteq G\}$. Dann ist $\fm$ abzählbar und $G=\bigcup_{I\in\fm}I$. also gilt:
\[G\in\sigma(\ce_1)\implies \fb_d=\sigma(\co(\mdr^d))\subseteq\sigma(\ce_1)\]
\item Sei $(a,b)\in\ce_1$.\\
\textbf{Fall 1:} $(a,b)=\emptyset\in\ce_2\subseteq\sigma(\ce_2)$\\
\textbf{Fall 2:} $(a,b)\ne\emptyset, a=(a_1\ldots,a_d), b=(b_1\ldots,b_d)$. Dann gilt für alle $j\in\{1,\ldots,d\}:a_j<b_j$, also gilt auch:
\[\exists N\in\mdn:\forall n\ge N: \forall j\in\{1,\ldots,d\}:a_j<b_j-\frac1n\]
Definiere $c_n:=(\frac1n,\ldots,\frac1n)\in\mdq^d$. Dann gilt:
\[(a,b)=\bigcup_{n\ge N}(a,b-c_n]\in\sigma(\ce_2)\]
Also auch $\ce_1\subseteq\sigma(\ce_2)$ und damit $\sigma(\ce_1)\subseteq\sigma(\ce_2)$.
\item Seien $a = (a_1,\ldots,a_d), b=(b_1,\ldots,b_d) \in \mdq^d$ mit $a \leq b$. Nachrechnen:
\[(a,b] = \bigcap_{k=1}^d (\underbrace{H^-_k(b_k)}{\in \ce_3} \cap \underbrace{\underbrace{H^-_k(a_k)}{\in \ce_3}^C}{\in \sigma(\ce_3)}$) \in \sigma(\ce_3). \]
Das heißt: $\ce_2 \subseteq \sigma(\ce_3) \implies \sigma(\ce_2) \subseteq \sigma(\ce_3)$. %bin mir nicht sicher ob statt dem letzten subseteq ein = stehen sollte?
\item $H^-_k(\alpha)$ ist abgeschlossen $\implies H^-_k(\alpha)^C$ ist offen $\implies H^-_k(\alpha)^C \in \fb_d \implies H^-_k(\alpha) \in \fb_d$. Also: $\ce_3 \subseteq \fb_d \implies \sigma(\ce_3) \subseteq \fb_d$. 
\end{enumerate}
\end{beweis}

\begin{definition}
\index{Spur}
Sei $\emptyset \neq \fm \subseteq \mathcal{P}(X)$ und $\emptyset \neq Y \subseteq X$. $\fm_Y := \{A \cap Y : A \in \fm\}$ heißt die \textbf{Spur von $\fm$ in $Y$}.
\end{definition}

\begin{satz}
Sei $\emptyset \neq Y \subseteq X$ und $\fa$ sei eine $\sigma$-Algebra auf $X$.
\begin{enumerate}
\item $\fa_Y$ ist eine $\sigma$-Algebra auf $Y$.
\item $\fa_Y \subseteq \fa \iff Y \in \fa$
\item Ist $\emptyset \neq \ce \subseteq \mathcal{P}(Y)$, so ist $\sigma(\ce_Y) = \sigma(\ce)_Y$.
\end{enumerate}
\end{satz}

\appendix
\chapter{Satz um Satz (hüpft der Has)}
\listtheorems{satz,wichtigedefinition}

\renewcommand{\indexname}{Stichwortverzeichnis}
\addcontentsline{toc}{chapter}{Stichwortverzeichnis}
\printindex

\chapter{Credits für Analysis III} Abgetippt haben die folgenden Paragraphen:\\% no data in Ana2Vorwort.tex
\textbf{§ 1: $\sigma$-Algebren und Maße}: Rebecca Schwerdt\\


\end{document}
