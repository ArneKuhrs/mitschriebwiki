\chapter{Gruppen}

\section{Grundlegende Definitionen}

\begin{Def} 
    Sei $M$ eine Menge.

    \begin{enum}
        \item Eine \emp{Verknüpfung} auf $M$ ist eine Abbildung $\cd: M
              \times M \ra M $

        \item Eine Menge $M$ zusammen mit einer Verknüpfung $\cd$ heißt
              \emp{Magma}.

        \item Eine Verknüpfung $\cd : M \times M \ra M$ heißt \emp{assoziativ},
        wenn \[\forall x,y,z \in M: (x\cd y)\cd z=x\cd(y\cd z)\]

        \item Eine \emp{Halbgruppe} ist ein assoziatives Magma.
    
        \item $e \in M$ heißt \emp{neutrales Element} für die Verknüpfung $\cd$,
        wenn \[\forall x \in M: x \cd e = e \cd x = x\]
          
        \item Eine Halbgruppe mit neutralem Element heißt \emp{Monoid}.

        \item Eine \emp{Gruppe} ist ein Monoid $(G,\cd)$, in dem es zu jedem $x
        \in G$ ein $x' \in G$ gibt mit \[ x \cd x' = x' \cd x = e \] $x'$ heißt
        dann \emp{zu $x$ inverses Element.}
    \end{enum}
\end{Def}

\begin{Bem}
    Sei $(M,\cd)$ ein Magma.
    
    \begin{enum}
        \item In $M$ gibt es höchstens ein neutrales Element. \newline
        \sbew{0.9}{Sind $e, e'$ neutrale Elemente, so ist $e=e\cd e' = e'$}

        \item Ist $M$ Monoid, so gibt es zu $x \in M$ höchstens ein inverses 
        Element.\newline
        \sbew{0.9}{Seien $x',x''$ zu $x$ invers, so ist $x'=(x'' \cd x)\cd x' =
        x'' \cd (x \cd x')=x''$}
    \end{enum}
\end{Bem}

\begin{DefBem} 
    Sei $(M,\cd)$ ein(e)
    \begin{small} $\left\{ \begin{array}{l}
            Magma \\
            Halbgruppe \\
            Monoid \\
            Gruppe \end{array}
    \right\}$
    \end{small}

    \begin{enum}
        \item $U \subseteq M$ heißt Unter-\bla, wenn $U \cd U \subseteq U$ und
        $(U,\cd)$ selbst ein(e) \bla ist.
        
        \item $U \subseteq M$ Unterhalbgruppe $\Leftrightarrow U \cd U \subseteq
        U$
        
        \item $U \subseteq M$ Untermonoid $\lra U \cd U \subseteq U$ und $e \in
        U$
        
        \item $U \subseteq M$ Untergruppe $\lra U \neq \emptyset$ und $\forall
        x,y \in U: x \cd y^{-1} \in U$ \newline \sbew{0.9}{\newline
        ''$\Leftarrow$'': \newline Sei $x \in U \> \Rightarrow e=x \cd x^{-1}
        \in U \Rightarrow $ mit $x$ ist auch $x^{-1}$ in $U \Rightarrow$ mit
        $x,y$ ist auch $xy = x(y^{-1})^{-1} \in U$}
    \end{enum}
\end{DefBem}

\begin{Bem} 
    Sei $ (M,\cd)$ Monoid. Dann ist $M^x \defeqr \{x \in M:$ es gibt inverses
    $x^{-1}$ zu $x \in M\}$ eine Gruppe. \newline
    \sbew{1.0}{ $\\e \in M^x$, da $e\cd e = e$, also $M^x \neq \emptyset$.
    Sind $x,y \in M^x$, so ist $x \cd y \in M^x$, da $xy \cd
    (y^{-1}x^{-1})=e \Rightarrow \cd$ ist Verknüpfung auf $M^x \Rightarrow
    (M^x, \cd)$ ist Gruppe.}
\end{Bem}

\begin{DefBem}
    Seien $(M, \cd), (M',*)$ \bla
    \begin{enum}
        \item Eine Abbildung $f:M\ra M'$ heißt \emp{Homomorphismus}, wenn
        $\forall \; x,y \in M:\\\\$ \bigskip $ f(x\cd y) = f(x)*f(y)$ \hfill
        \textmd{(i)} \newline Hat M ein neutrales Element, so muß außerdem
        gelten: $\\\\f(e) = e'$ \hfill \textmd{(ii)}
        
        \item Ist $f:G \ra G'$ Abbildung von Gruppen, die \textmd{(i)} erfüllt,
        so ist $f$ Homomorphismus. \newline \sbew{0.9}{ $f(e) = f(e \cd e) =
        f(e) * f(e) \overset{\cd f(e)^{-1}}{\Rightarrow} e' = f(e)$} 
        
        \item Ein Homomorphismus $f:M \ra M'$ heißt \emp{Isomorphismus}, wenn es
        einen Homomorphismus $g:M'\ra M$ gibt, mit $f \circ g = id_{M'}$ und $g
        \circ f = id_M$

        \item Jeder bijektive Homomorphismus ist Isomorphismus. \newline
        \sbew{0.9}{Sei $f:M\ra M'$ bijektiver Homomorphismus und $g:M' \ra M$
        die Umkehrabbildung. z.z.: $g$ ist Homomorphismus. \newline Seien $x,y
        \in M'$. Schreibe $x=f(\hat{x}), y=f(\hat{y})$ für passende $\hat{x},\hat{y} \in M 
        \Rightarrow g(x\cd y) = g(f(\hat{x}) \cd f(\hat{y})) = g(f(\hat{x} \cd \hat{y})) = 
        \hat{x} \cd \hat{y} = g(f(\hat{x})) \cd g(f(\hat{y})) = g(x) \cd g(y) $}

        \item Die Komposition von Homomorphismen ist wieder ein Homomorphismus.
    \end{enum}
\end{DefBem}

\begin{Def} 
    Sei $f:M \ra M'$ Hom von \bla.
    
    \begin{enum}
        \item $Bild(f) \defeqr \{f(x):x \in M\} \subseteq M'$ ist ein
        Unter-\bla. \newline
        \sbew{0.9}{Sind $x,x' \in M$, so ist $f(x)*f(x')=f(x \cd x') \in 
        Bild(f)$. Sind $M,M'$ Monoide, so gilt: $f(e) = e' \in$ Bild$(f)$.
        Sind $M, M'$ Gruppen, so gilt: $f(x)^{-1} = f(x^{-1}) \in$ Bild$(f)$, da
        $f(x\cd x^{-1}) = f(e) = e' = f(x) * f(x^{-1})$}
        
        \item Sind $M, M'$ Monoide/Gruppen, so ist Kern$(f) \defeqr \{x \in M :
        f(x) = e'\}$ Untermonoid/-gruppe von $M$. \newline
        \sbew{0.9}{$x,y \in$ Kern$(f) \Rightarrow f(xy) = f(x) * f(y) = e'*e' =
        e' \Rightarrow xy \in$ Kern$(f), e \in$ Kern$(f) \;\chk$ \\
        $x\in$ Kern$(f) \Rightarrow f(x^{-1}) = f(x)^{-1} = (e')^{-1} = e'
        \Rightarrow x^{-1} \in$ Kern$(f)$}

        \item Sind $G,G'$ Gruppen, so ist $f$ genau dann injektiv, wenn Kern$(f)
        = \{e\}$
    \end{enum}
\end{Def}
