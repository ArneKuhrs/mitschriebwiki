\documentclass[12pt]{scrbook}   %12pt --> art10

\oddsidemargin=0cm
\evensidemargin=0cm
\topmargin=-2cm
\textwidth=18cm
\textheight=25cm
\mathsurround=1.5pt
\parskip=5pt
\parindent=0pt
%\setcounter{page}{0}

%\input pictex
%\input prepictex
%\input postpictex

%\usepackage{color}
\usepackage{a4}
\usepackage{amsmath,amssymb,amsfonts}
\usepackage{mathtools}
\usepackage[utf8]{inputenc}
\usepackage{fontenc}[german]
\usepackage[all]{xy}
\usepackage{german}
\usepackage{faktor}
\usepackage{nicefrac}
\usepackage{enumerate}
\usepackage{url}
\selectlanguage{german}
%\renewcommand{\thefootnote}{\fnsymbol{footnote}}
%\pagestyle{empty}

\usepackage{hyperref}

\newtheorem{alles}{alles}[section]
\newtheorem{bemX}[alles]{Bemerkung}
\newenvironment{bem}[1]{\begin{bemX}{\bf #1}\par\rm}{\end{bemX}}
\newtheorem{satzX}[alles]{Satz}
\newenvironment{satz}[1]{\begin{satzX}{\bf #1}\nopagebreak\par}{\end{satzX}}
\newtheorem{hilfsX}[alles]{Hilfssatz}
\newenvironment{hilfs}[1]{\begin{hilfsX}{\bf #1}\par}{\end{hilfsX}}
\newtheorem{definiX}[alles]{Definition}
\newenvironment{defini}[1]{\begin{definiX}{\bf #1}\par\rm}{\end{definiX}}
\newtheorem{bspX}[alles]{Beispiel}
\newenvironment{bsp}[1]{\begin{bspX}{\bf #1}\par\rm}{\end{bspX}}
\newtheorem{bspX*}{Beispiel}
\newenvironment{bsp*}[1]{\begin{bspX*}{\bf #1}\par\rm}{\end{bspX}}
\newtheorem{fazit}[alles]{Fazit}
\newtheorem{Folgerung}[alles]{Folgerung}

\newcommand{\qed}{\phantom{.}\hfill{$\blacksquare$}}

\newcommand{\da}{\coloneqq}

\author{Dr. Stefan Kühnlein und \url{http://mitschriebwiki.nomeata.de/}}
\title{Topologie}

%\makeindex

\begin{document}
\maketitle

\renewcommand\theHchapter{\arabic{chapter}}
%\renewcommand{\thechapter}{\Roman{chapter}}
%\chapter{Inhaltsverzeichnis}
\stepcounter{chapter}
\addcontentsline{toc}{chapter}{\protect\numberline {\thechapter}Inhaltsverzeichnis}
\tableofcontents


\chapter{Vorwort}

\section*{Über dieses Skriptum}

Das Ziel diese Skripts ist es, einen ersten Einblick in die Topologie zu geben.
An keiner Stelle wird versucht, Ergebnisse bis in die letzten Winkel und 
Spitzen zu treiben; wir erlauben uns auch bisweilen, nicht geringstmögliche 
Voraussetzungen in Aussagen zu machen, sondern hoffen, durch eine 
Beschränkung auf einfachere Situationen bisweilen den Inhalt der Sätze
(von denen es ohenhin nicht so viele gibt) deutlich zu machen. Die Vorlesung
ist nicht für Spezialisten gedacht - das verbietet sich schon angesichts des
Dozenten, der ja auch selbst kein Spezialist ist. Hiermit sei seiner Hoffnung 
der Ausdruck verliehen, dass die subjektive Stoffauswahl nicht zu sehr zu 
Lasten der Allgemeinheit geht und ein Verständnis trotz allem zustande
kommen kann.

Jedenfalls werden in dieser Vorlesung nicht alle erlaubten Implikationen 
zwischen allen möglichen Aussagen vorgeführt werden. 

Ich habe auf eine umfangreiche Illustration verzichtet, zum einen weil dies in 
der Vorlesung passieren soll, zum 
anderen, weil es vielleicht auch für Leser eine instruktive Übung ist, sich
selbst ein Bild von dem zu machen, wovon die Rede ist. Der begriffliche 
Apparat ist das präzise Werkzeug, die Bilder sind ja „nur“ ein 
Hilfsmittel, das uns helfen soll zu sehen, wo die Werkzeuge angesetzt werden 
können. Außerdem sind manche Bilder sehr irreführend, zumal wenn es um
Sachverhalte geht, die sich definitiv nicht mehr in unserem Anschauungsraum
abspielen können.

\subsection*{Online-Version}
Auf \url{http://mitschriebwiki.nomeata.de} finden sich die \LaTeX-Quellen zu diesem Skript, sowie die Möglichkeit, dort direkt am Skript mitzuarbeiten, etwa um Fehler zu beseitigen. Dies Basiert auf latexki\footnote{\url{http://latexki.nomeata.de/}}, einem von Joachim Breitner programmiertem Wiki für \LaTeX-Dokumente.

\bigskip

Nun zum Inhalt selbst.

\setcounter{chapter}{0}
\renewcommand{\thechapter}{\Roman{chapter}}
\renewcommand{\theHchapter}{\Roman{chapter}}

\renewcommand{\thesection}{{\rm\bfseries §}\,\thechapter.\arabic{section}}
\renewcommand{\thealles}{\thechapter.\arabic{section}.\arabic{alles}}

\chapter{Einstieg}

\section{Kontext}

Die Topologie ist eine mathematische Grundlagendisziplin die sich verstärkt
seit dem Ende des 19.\ Jahrhunderts eigenständig entwickelt hat. Vorher waren
einige topologische Ideen im Zusammenhang mit geometrischen und analytischen 
Fragestellungen entstanden. Um Topologie handelt es sich zunächst immer dann,
wenn geometrische Objekte deformiert werden und solche Eigenschaften der 
Objekte in den Vordergrund treten die sich dabei nicht ändern. 

Topologisch ist eine Kugel dasselbe wie ein Würfel - geometrisch zwar 
völlig unterschiedlich, aber doch gibt es einige Gemeinsamkeiten. 
Es wäre vielleicht einmal interessant zu verfolgen, ob der Kubismus am Ende
des 19.\ Jhdts.\ und die topologische Frage nach „simplizialen 
Zerlegungen“ geometrischer Objekte sich gegenseitig beeinflusst haben\dots

Der Begriff der Nähe spielt in der Topologie eine gewisse Rolle, mehr als der
Begriff des Abstands, der für die Geometrie immerhin namensgebend war.
Die topologischen 
Mechanismen, die so entwickelt wurden, wurden nach und nach von ihren 
geometrischen Eltern entfernt; dafür sind die Eltern ja da: sich 
überflüssig zu machen. Und so konnten topologische Ideen sich auch auf 
andere Bereiche der Mathematik ausdehnen und diese geometrisch durchdringen.

Auch außerhalb der Mathematik ist die Topologie längst keine unbekannte
mehr. So gab es in der ersten Hälfte des 20.\ Jhdts.\ die topologische 
Psychologie von Kurt Lewin, die allerdings nur die Terminologie von der 
Topologie übernahm, und nicht etwa mithilfe topologischer Argumente neue
Einsichten produzierte. Etwas anders sieht es natürlich mit den 
„richtigen“Naturwissenschaften aus. In der Physik taucht
die Topologie zum Beispiel in der Form von Modulräumen in der Stringtheorie 
auf, und in der Molekularchemie kann man zum Beispiel Chiralität als
topologisches Phänomen verstehen.



\section{Beispiele -- was macht die Topologie?}

\begin{bsp}{\bf Nullstellenfang mit dem Lasso} \label{Lasso}

{\rm Es sei $f:\mathbb C\longrightarrow \mathbb C$ eine nichtkonstante 
Polynomabbildung, d.h.\ $f(z) = \sum_{i=0}^d a_iz^i$ mit $d>0$ und $a_d\neq 0.$

Dann hat $f$ eine Nullstelle in $\mathbb C.$ Das kann man zum Beispiel so 
plausibel machen:

Wenn $a_0=0$ gilt, dann ist $z=0$ eine Nullstelle. Wenn $a_0\neq 0$, dann 
brauchen wir ein Argument. Wir betrachten den Kreis vom Radius $R$ um den 
Nullpunkt: $RS^1 = \{z\in \mathbb C \mid |z|=R\}.$ Aus der Gleichung
$$f(z) = a_dz^d \cdot (1 + \frac{a_{d-1}}{a_d z} + \dots + 
\frac{a_{0}}{a_d z^d})$$ 
folgt, dass das Bild von $RS^1$ unter $f$ jedenfalls für großes $R$ im 
Wesentlichen der $d$-fach 
durchlaufene Kreis vom Radius $|a_d|R^d$ ist. Im Inneren dieser Schlaufe liegen
für großes $R$ sowohl die 0 als auch $a_0.$ Wenn man nun den Radius 
kleiner macht, so wird diese Schlaufe für $R\searrow 0$ zu einer Schlaufe um 
$a_0$
zusammengezogen -- das ist die Stetigkeit von $f.$ Für kleines $R$ liegt 
insbesondere $0$ nicht im Inneren der Schlaufe. Das aber heißt, dass beim
Prozess des Zusammenziehens die Schlaufe irgendwann mindestens einmal die 0
trifft. Dann hat man eine Nullstelle von $f$ gefunden.

Einen anders gelagerten und präzisen topologischen Beweis des 
Fundamentalsatzes werden wir in \ref{Fundamentalsatz} führen.
}
\end{bsp}

In diesem Argument -- das man streng durchziehen kann -- wird ein topologisches
Phänomen benutzt, um den Fundamentalsatz der Algebra zu beweisen. Das 
Zusammenziehen der Kurve durch Variation des Parameters $R$ werden wir später
allgemeiner als Spezialfall einer Homotopie verstehen.

\begin{bsp}{\bf Fahrradpanne}

{\rm Es gibt keine stetige Bijektion von einem Torus $T$ 
(„Fahrradschlauch“) auf eine Kugeloberfläche $S.$ 


Denn:  Auf dem Torus gibt es eine geschlossene Kurve $\gamma$, die ihn nicht 
in zwei 
Teile zerlegt. Ihr Bild unter einer stetigen Bijektion von $T$ nach $S$ 
würde dann $S$ auch nicht in zwei Teile zerlegen, da das stetige Bild des
Komplements $T\setminus \gamma$ gleich 
$S\setminus{\rm Bild\ von\ }\gamma$ zusamenhängend sein müsste, aber
das stimmt für keine geschlossene Kurve auf $S$.}
\end{bsp}

Auch hier sieht man ein topologisches Prinzip am Werk. Es ist oft sehr schwer
zu zeigen, dass es zwischen zwei topologischen Räumen (siehe später) keine
stetige Bijektion gibt. Dass ich keine solche finde sagt ja noch nicht wirklich
etwas aus\dots

In der linearen Algebra wei\ss\  man sehr genau, wann es 
zwischen zwei Vektorräumen einen Isomorphismus gibt, das hängt ja nur an 
der Dimension. Ähnlich versucht man in der Topologie, zu topologischen 
Räumen zugeordnete Strukturen zu finden, die nur vom Isomorphietyp 
abhängen, und deren Isomorphieklassen man besser versteht als die der 
topologischen Räume.

\begin{bsp} {\bf Eulers \footnote{Leonhard Euler, 1707-1783}Polyederformel}

{\rm Für die Anzahl $E$ der Ecken, $K$ der Kanten und $F$ der Flächen eines
(konvexen) Polyeders gilt die Beziehung $E-K+F=2.$

Das kann man zum Beispiel einsehen, indem man das Polyeder zu einer Kugel 
aufbläst, auf der man dann einen Graphen aufgemalt hat (Ecken und Kanten des 
Polyeders), und dann für je zwei solche zusammenhängenden Graphen zeigt, 
dass sie eine gemeinsame Verfeinerung haben. Beim Verfeinern ändert sich
aber $E-K+F$ nicht, und so muss man nur noch für ein Polyeder die 
alternierende Summe auswerten, zum Beispiel für das Tetraeder, bei dem 
$E=F=4, K=6$ gilt.
}
\end{bsp}

\begin{bsp} {\bf Reelle Divisionsalgebren}

{\rm Eine reelle Divisionsalgebra ist ein $\mathbb R$-Vektorraum $A$ mit einer
bilinearen Multiplikation, für die es ein neutrales Element gibt und jedes
$a\in A\setminus\{0\}$ invertierbar ist. 

Beispiele hierür sind $\mathbb R,\mathbb C, \mathbb H$ 
(Hamilton\footnote{William Hamilton, 1788-1856}-Quaternionen) und -- wenn man 
die Assoziativität wirklich nicht haben will -- $\mathbb O$ (die 
Cayley\footnote{Arthur Cayley, 1821-1895}-Oktaven). Die Dimensionen dieser 
Vektorräume sind $1,2,4,8.$ Tatsächlich ist es so, dass es keine weiteren
endlichdimensionalen reellen Divisionsalgebren gibt. Dies
hat letztlich einen topologischen Grund. 

Zunächst überlegt man sich, dass die Struktur einer Divisionsalgebra auf 
$\mathbb R^n$ auf der $n-1$-dimensionalen Sphäre eine Verknüpfung
induziert, die fast eine Gruppenstruktur ist.

Dann kann man im wesentlichen topologisch zeigen, dass solch eine Struktur
auf der Späre nur für
$n\in\{1,2,4,8\}$ existieren kann. Solch eine Gruppenstruktur stellt nämlich 
topologische Bedingungen, die für die anderen Sphären nicht erfüllt sind.
}\end{bsp}



Eng damit zusammen hängt der

\begin{bsp} {\bf Satz vom Igel\footnote{Frans Ferdinand Igel, ???}}

{\rm Dieser Satz sagt, dass jeder stetig gekämmte Igel mindestens einen 
Glatzpunkt besitzt. Die Richtigkeit dieses Satzes gründet sich nicht darauf, 
dass es bisher noch niemanden gelingen ist, einen Igel zu kämmen. Sie hat
handfeste mathematische Gründe, die in einer etwas präziseren Formulierung 
klarer werden:

Etwas weniger prosaisch besagt der Satz „eigentlich\grqq, dass ein 
stetiges Vektorfeld auf der zweidimensionalen Sphäre mindestens eine
Nullstelle besitzt.
}
\end{bsp}

\begin{bsp} {\bf Brouwers\footnote{Luitzen Egbertus Jan Brouwer, 1881-1966}
Fixpunktsatz}

{\rm Jede stetige Abbildung des $n$-dimensionalen Einheitswürfels 
$W=[0,1]^n$ in sich selbst hat einen Fixpunkt.

Für $n=1$ ist das im Wesentlichen der Zwischenwertsatz. Ist $f:[0,1]
\longrightarrow [0,1]$ stetig, so ist auch $g(x) \da f(x)-x$ eine stetige 
Abbildung von $[0,1]$ nach $\mathbb R$, und es gilt $g(0)\geq 0, g(1)\leq 0.$

Also hat $g$ auf jeden Fall eine Nullstelle $x_0$, aber das heißt dann 
$f(x_0) = x_0.$

Für $n\geq 2$ ist der Beweis so einfach nicht möglich, wir werden ihm 
eventuell auch nur für $n=2$ später noch begegnen.
}
\end{bsp}

\section{Mengen, Abbildungen, usw.}

Wir werden für eine Menge $M$ mit ${\cal P}(M)$ immer die Potenzmenge 
bezeichnen: 
$${\cal P}(M) = \{A \mid A\subseteq M\}.$$
Für eine Abbildung $f:M\longrightarrow N$ nennen wir das Urbild
$f^{-1}(n)$ eines Elements $n\in f(M)\subseteq N$ auch eine 
\index{Faser}{\it Faser} von $f.$

Eine Abbildung ist also injektiv, wenn alle Fasern  einelementig sind.

Ist $f$ surjektiv, so gibt es eine Abbildung $s:N\longrightarrow M$ mit
$f\circ s = {\rm Id}_N$ -- die identische Abbildung auf $N.$ Jede solche 
Abbildung $s$ heißt ein \index{Schnitt}{\it Schnitt} zu $f$. Er wählt zu 
jedem $n\in N$ ein $s(n)\in f^{-1}(n)$ aus. Wenn man also $M$ als Vereinigung
der Fasern von $f$ über den Blumentopf $N$ malt, so erhält der Name Schnitt
eine gewisse Berechtigung.

Eine \index{Partition}{\it Partition} von $M$ ist eine Zerlegung von $M$ in 
disjunkte, nichtleere Mengen $M_i,i\in I,$ wobei $I$ eine Indexmenge ist:
$$M=\bigcup_{i\in I} M_i,\ \ \forall i\neq j: M_i\cap M_j = \emptyset, 
M_i\neq\emptyset.$$

Hand in Hand mit solchen Partitionen gehen Äquivalenzrelationen auf $M.$
Die Relation zur Partition $M_i,i\in I$ wird gegeben durch
$$m\sim \tilde m \iff \exists i\in I: m,\tilde m\in M_i.$$
Umgekehrt sind die Äquivalenzklassen zu einer Äquivalenzrelation $\sim $ 
eine Partition von $M$. Die Menge der Äquivalenzklassen nennen wir auch den
\index{Faktorraum}{\it Faktorraum} $\faktor M\sim$:
$$\faktor M \sim = \{M_i\mid i\in I\}.$$
Die Abbildung $\pi_\sim: M\longrightarrow \faktor{M}{\sim},\ \pi_\sim(m)\da [m] = $ 
Äquivalenzklasse von $m$ heißt die {\it kanonische Projektion von } $M$ 
auf $\faktor M \sim$.

Ist $\sim$ eine Äquivalenzrelation auf $M$ und $f:M\longrightarrow N$ eine
Abbildung, so dass
jede Äquivalenzklasse von $\sim$ in einer Faser von $f$ enthalten ist (d.h.
$f$ ist konstant auf den Klassen), so wird durch 
$$\tilde f:\faktor M \sim \longrightarrow N,\ \tilde f([m]) \da f(m),$$
eine Abbildung definiert, für die $f=\tilde f \circ \pi_\sim$ gilt. Das ist
die mengentheoretische Variante des Homomorphiesatzes.

\begin{bsp}{\bf Gruppenaktionen}

{\rm Ein auch in der Topologie wichtiges Beispiel, wie Äquivalenzrelationen 
bisweilen entstehen, ist das der \index{Gruppenoperation}{\it Operation} 
einer Gruppe $G$ auf der Menge $X$.

Solch eine Gruppenaktion ist eine Abbildung
$$\bullet:G\times X\longrightarrow X,$$
die die folgenden Bedingungen erfüllt:
$$\begin{array}{rl}\forall x\in X:& e_G\bullet x = x\\
\forall g,h\in G, x\in X:& g\bullet (h\bullet x) = (gh)\bullet x.\\
\end{array}$$
Hierbei ist $e_G$ das neutrale Element von $G$ und $gh$ ist das Produkt von 
$g$ und $h$ in $G$.

\paragraph{Beispiel zum Beispiel:} Für jede Menge $X$ gibt es die symmetrische Gruppe $S_X \da {\rm Sym}(X) \da \{ f : X\longrightarrow X \mid f \text{ bijektiv}\}$, als Verknüpfung benutzen wir die Hintereinanderausführung von Abbildungen. Das neutrale Element ist ${\rm Id}_X : X\longrightarrow X$, ${\rm Id}_X(x) = x$ für alle $x\in X$.
Betrachte nun: $\bullet:{\rm Sym}(X) \times X \longrightarrow X$, $(f,x)\mapsto f(x)$. Dies ist eine Grupenoperation.

Für jedes $g\in G$ ist die Abbildung 
$$\rho_g:X\longrightarrow X,\ \rho_g(x) \da g\bullet x,$$
eine Bijektion von $X$ nach $X$, die Inverse ist $\rho_{g^{-1}},$ und 
$$g\mapsto \rho_g$$
ist ein Gruppenhomomorphismus von $G$ in die symmetrische Gruppe von $X$.

Die \index{Bahn}{\it Bahn} (oder der Orbit) von $x\in X$ unter der Operation von $G$ ist 
$$G\bullet x \da \{g\bullet x\mid g\in G\}.$$
Man kann leicht verifizieren, dass die Bahnen einer Gruppenoperation eine
Partition von $X$ bilden. Mit $\faktor{X}{G} \da \{G\bullet x \mid x \in X\}$ bezeichnen wir den Bahnenraum von $X$ unter der Operation von $G$.

\paragraph{Weiteres Beispiel zum Beispiel:} Betrachte $\bullet : \mathbb Z ^2 \times \mathbb R^2 \longrightarrow \mathbb R^2$, $(k,v)\mapsto k+v$. Dann ist etwa der Orbit des Nullpunktes die Menge aller ganzzahligen Vektoren. 
Der Bahnenraum $\faktor{\mathbb R^2}{\mathbb Z^2} \cong [0,1)^2$ entspricht dann topologisch dem Torus.

Umgekehrt ist es so, dass jede Partition $(X_i)_{i\in I}$ von $X$ von der 
natürlichen Aktion einer geeigneten Untergruppe $G$ von ${\rm Sym}(X)$ 
herkommt. Hierzu wähle man einfach
$$G\da\{\sigma\in {\rm Sym}(X) \mid \forall i\in I: \sigma(X_i)= X_i\}$$
und verifiziere was zu verifizieren ist.

Ist $\bullet: G\times X \longrightarrow X$ eine Gruppenaktion, so heißt für $x\in X$ die Menge ${\rm Stab}_G(x) \da \{g\in G\mid g\bullet x = x \}$ der \textit{Stabilisator}\index{Stabilisator} oder „Fixgruppe“ von $x$. ${\rm Stab}_G(x) \subseteq G$ ist eine Untergruppe.

Die Gruppenaktion heißt \textit{transitiv}\index{transitiv}, wenn es nur eine Bahn gibt: $\exists x \in X: X = G\bullet x$.
}
\end{bsp}
\begin{bsp} {\bf Der projektive Raum $\mathbb P^nK$}

{\rm Es seien $K$ ein Körper und $n$ eine natürliche Zahl.

\paragraph{Problemstellung:} In $\mathbb A ^2(K)$, der affinen Ebene, gibt es Geraden. Zwei Geraden $g$, $h$ in $\mathbb A^2(K)$ schneiden sich entweder in einem Punkt oder sie sind parallel. Das führt oft zu Fallunterscheidungen in Beweisen, was lästig ist. Wir vergrößern daher $\mathbb A^2(K)$ durch Hinzunahme von „unendlich fernen Punkten“, wobei jeder Richtung von Geraden ein solcher Punkt entspricht: Zwei parallele Geraden schneiden sich dann in dem ihrer Richtung entsprechenden Punkt. Eine Konstruktion, die diesen Raum schön beschreibt, ist die folgende:

Auf $X\da K^{n+1}\setminus\{0\}$ operiert die Gruppe $K^\times$ durch
die skalare Multiplikation
$$a\bullet v \da a\cdot v.$$
Die Bahn von $v\in X$ unter dieser Operation ist $Kv\setminus \{0\}.$
Da die $0$ ohnehin zu jeder Geraden durch den Ursprung gehört, kann man
den Bahnenraum $\faktor{X}{K^\times}$ mit der Menge aller Geraden durch den Ursprung
identifizieren. Dieser Raum heißt der {\it $n$-dimensionale projektive Raum
über $K$}, \index{projektiver Raum}  in Zeichen $\mathbb P^n(K).$

Speziell für $n=1$ gilt:
$$\mathbb P^1(K) = \{[{a\choose 1}] \mid a\in K \} \cup \{[{1\choose 0}]\}.$$
Oft identifiziert man den ersten großen Brocken hier mit $K,$ den 
hinzukommenden Punkt nennt man suggestiver Weise $\infty,$ also $\mathbb P^1(K) = K \cup \{\infty\}$.

Genauso haben wir für beliebiges $n$ eine Zerlegung
$$\mathbb P^n(K) = \{[ {v\choose 1}] \mid v\in K^n \} \cup 
\{[{w\choose 0}] \mid w\in K^n,w\neq 0\} = K^n \cup \mathbb P^{n-1}(K),$$
wobei die Auswahl des {\it affinen Teils} $K^n$ durch die Bedingung, dass die 
letzte Koordinate nicht null ist, relativ willkürlich ist. 

Was ist also $\mathbb P^2(\mathbb R)$? Da jede Gerade durch $0$ in $\mathbb R^3$ durch einen Vektor der Länge 1 erzeugt wird, gilt: 
$$ \mathbb P^2(\mathbb R) = \faktor{S^2}{\{\pm1\}}$$
wobei $S^2 \da \{ v\in\mathbb R^3\mid |v|=1\}$. Diesen Raum nennt man auch {\it Kreuzhaube}\index{Kreuzhaube}.
}
\end{bsp}


\begin{defini}{Faserprodukte}
Es seien $A,B,S$ Mengen und 
$f_A:A\longrightarrow S,\ f_B:B\longrightarrow S$
zwei Abbildungen.

Weiter sei $F$ eine Menge mit Abbildungen $\pi_A,\pi_B$ von $F$ nach $A$ bzw.\ 
$B$, so dass $f_A\circ\pi_A = f_B\circ \pi_B.$

$F$ heißt ein {\it Faserprodukt}\index{Faserprodukt} von $A$ und $B$ über
$S$, wenn für jede Menge $M$ und jedes Paar von Abbildungen $g_A,g_B$ von
$M$ nach $A$ bzw.\ $B$ mit $f_A\circ g_A = f_B\circ g_B$ genau eine Abbildung
$h:M\longrightarrow F$ existiert, so dass 
$$g_A = \pi_A\circ h,\ \ g_B = \pi_B\circ h.$$
\end{defini}

Insbesondere impliziert das, dass es zwischen zwei Faserprodukten von $A$ und 
$B$ über $S$ genau einen sinnvollen Isomorphismus gibt. Denn nach Definition
gibt es für ein zweites Faserprodukt $(\widetilde F,\widetilde{\pi_A},
\widetilde{\pi_B})$ genau eine Abbildung $h$ von $\widetilde F$ nach $F$ mit
$$\widetilde{\pi_A} = \pi_A\circ h,\ \ \widetilde{\pi_B} = \pi_B\circ h$$
und auch genau eine Abbildung $\tilde h:F\longrightarrow  \widetilde F$ mit
$$\pi_A = \widetilde{\pi_A} \circ \tilde h,\ \ \pi_B = \widetilde{\pi_B}\circ 
\tilde h.$$
Dann ist aber $h\circ\tilde h$ eine Abbildung von $F$ nach $F$ mit
$$\pi_A = pi_A\circ (h\circ\tilde h),\ \ \pi_B = \pi_B\circ (h\circ\tilde h),$$
was wegen der Eindeutigkeit aus der Definition zwangsläufig
$$h\circ\tilde h ={\rm Id}_F$$ 
nach sich zieht. Analog gilt auch
$$\tilde h\circ h = {\rm Id}_{\widetilde F}.$$

{\bf Schreibweise:} Für das Faserprodukt schreibt man meistens $A\times_SB,$
wobei in der Notation die Abbildungen $f_A$ unf $f_B$ unterdrückt werden.

\medskip

Ein Faserprodukt existiert immer. Wir können nämlich 
$$F\da\{(a,b)\in A\times B \mid f_A(a) = f_B(b)\} $$
wählen und für $\pi_A,\pi_B$ die Projektion auf den ersten beziehungsweise
zweiten Eintrag. 

Die Abbildung $h$ aus der Definition ist dann einfach $h(m) = (g_A(m), g_B(m)).$

Wir können $F$ auch hinschreiben als 
$$F=\bigcup_{s\in S} \left(f_A^{-1}(s) \times f_B^{-1}(s)\right),$$
also als Vereinigung der Produkte der Fasern von $f_A$ und $f_B$ über jeweils
demselben Element von $S.$ Das erklärt den Namen.

\begin{bsp} {\bf Spezialfälle}

{\rm 
\begin{itemize}
\item[a)] Wenn $S$ nur aus einem Element $s$ besteht, dann sind $f_A$ und $f_B$
konstant, und damit $A\times_SB = A\times B$ das mengentheoretische Produkt.
\item[b)] Wenn $A,B$ Teilmengen von $S$ sind und die Abbildungen $f_A,f_B$ 
einfach die Inklusionen, dann gilt
$$A\times_SB = \{(a,b)\in A\times B\mid a=b\} = \{(s,s)\mid s\in A\cap B\} 
\simeq A\cap B.$$
\end{itemize}
}
\end{bsp}

\section{Metrische Räume}

\begin{defini} {\bf Metrischer Raum} 

{\rm Ein {\it metrischer Raum}\index{metrischer Raum} ist eine Menge $X$ 
zusammen mit einer Abbildung $$d:X\times X\longrightarrow \mathbb R_{\geq 0},$$
so dass die folgenden Bedingungen erfüllt sind:
\begin{itemize}
\item $\forall x,y\in X: d(x,y) = d(y,x)$ (Symmetrie)
\item $\forall x\in X: d(x,x) = 0.$
\item $\forall x,y\in X: x\neq y\Rightarrow d(x,y) >0.$
\item $\forall x,y,z\in X: d(x,y) + d(y,z) \geq d(x,z).$ (Dreiecksungleichung)
\end{itemize}
Die Abbildung $d$ heißt dabei die {\it Metrik}.

Penibler Weise sollte man einen metrischen Raum als Paar $(X,d)$ schreiben.
Meistens wird das micht gemacht, aber Sie kennen diese Art der Schlamperei ja
schon zur Genüge\ldots
}\end{defini}

\begin{bsp}{\bf LA und Ana lassen grüßen} \label{L_unendlich}

{\rm 
\begin{itemize}
\item [a)] Ein reeller Vektorraum mit einem Skalarprodukt 
$\langle\cdot,\cdot\rangle$ wird bekanntlich mit 
$$d(v,w) \da \sqrt{\langle v-w,v-w\rangle} = |v-w|$$
zu einem metrischen Raum. Im folgenden heißt $\mathbb R^n$ als metrischen Raum immer: mit dem Standardskalarprodukt.
\item[b)] Jede Menge $X$ wird notfalls durch 
$$d(x,y) = \left\{\begin{array}{rl} 1, & {\rm falls}\ x\neq y,\\
                                    0, & {\rm falls}\ x=y,\\
\end{array}\right.$$
zu einem metrischen Raum. Diese Metrik heißt die {\it diskrete Metrik}
\index{diskrete Metrik} auf $X.$
\item[c)] Es sei $X$ eine Menge und ${\cal B}(X)$ der Vektorraum der 
beschränkten reellwertigen Funktionen auf $X.$ Dann wird $X$ vermöge
$$d(f,g) \da \sup\{|f(x)-g(x)|\mid x\in X\}$$
zu einem metrischen Raum. 

Diese Metrik ist stets endlich, denn $f$ und $g$ sind beschränkt, also gibt es ein $R>0$ so dass $|f(x)|,|g(x)|<R$ für jedes $x\in X$, und damit $|f(x) - g(x)|<2R$.

% Anstelle der Norm aus einem Skalarprodukt benutzt man hier also die sogenannte
% Maximumsnorm 
% $$|f|_\infty \da \sup\{|f(x)|\mid x\in X\},$$
% um eine Metrik zu konstruieren. Diese kommt nicht von einem Skalarprodukt her,
% wenn $X$ mindestens 2 Elemente hat.

Allgemeiner sei für eine Menge $X$ und einen metrischen Raum $(Y,e)$ die 
Menge
${\cal B}(X,Y)$ definiert als die Menge aller beschränkten Abbildungen von 
$X$ nach $Y$. Dabei heißt $f$ beschränkt, wenn ein $y \in Y$ und ein $R\in \mathbb R$ gibt mit
$$\forall x\in X: e(f(x),y) < R.$$
Dann wird ${\cal B}(X,Y)$ zu einem metrischen Raum vermöge
$$d(f,g) \da \sup\{e(f(x),g(x))\mid x\in X\}.$$
\item[d)] Auf den rationalen Zahlen lässt sich für eine Primzahl $p$
auf folgende Art eine Metrik konstruieren:

Jede rationale Zahl $q\neq 0$ kann man schreiben als 
$p^{{\rm v}_p(q)} \cdot \frac ab,$ wobei $a,b\in \mathbb Z$ keine Vielfachen von 
$p$ sind. Dann ist $v_p(q)$ eindeutig bestimmt.

Wir setzen für zwei rationale Zahlen $x,y$
$$d_p(x,y) \da
\begin{cases}
0, & \text{falls }x=y,\\
p^{-{\rm v}_p(x-y)}, & \text{sonst}\,.\end{cases}$$
Dies ist die sogenannte $p$-adische Metrik auf $\mathbb Q.$
\end{itemize}
}\end{bsp}

\begin{defini} {\bf Folgen und Grenzwerte}

{\rm 
\begin{itemize}
Es sei $(X,d)$ ein metrischer Raum. 
\item[a)]
Eine Folge $(x_n)_{n\in \mathbb N}$
in $X$ heißt {\it konvergent gegen den Grenzwert} $y\in X,$ falls
$$\forall \varepsilon > 0 : \exists k\in\mathbb N : \forall n \in N: \ge k: d(x_n,y)<\varepsilon.$$
Natürlich ist $y$ hierbei eindeutig bestimmt.
\item[b)] Eine Cauchyfolge\footnote{Augustin-Louis Cauchy, 1789-1857} in $X$ 
ist eine Folge $(x_n)_{n\in \mathbb N}$ mit
$$\forall \varepsilon>0 : \exists N\in \mathbb N: \forall m,n>N:
d(x_m,x_n) < \varepsilon.$$
Jede konvergente Folge ist eine Cauchyfolge.
\item[c)] $X$ heißt {\it vollständig}\index{vollständig}, wenn jede
Cauchyfolge in $X$ einen Grenzwert in $X$ hat.
\end{itemize}
}
\end{defini}

\begin{bsp} {\bf Schatten der Vergangenheit}

{\rm 
Für einen vollständigen metrischen Raum $(Y,e)$ und jede Menge $M$ ist auch $\mathcal B(M,Y)$: Ist $(f_i)_{i\in\mathbb N}$ eine Cauchy-Folge in $\mathcal B(M,Y)$, dann ist für jedes $m\in M$ auch $(f_i(m))_i$ eine Cauchy-Folge in $Y$. Da $Y$ vollständig ist, gibt es eine Funktion $f:M\to Y$ so dass für jedes $m\in M$ gilt: $f_i(m)\to f(m)$ für $i\to\infty$.

$f$ ist beschränkt, denn es gibt ein $\varepsilon > 0$ und ein $k>0$ so dass für jedes $i,j \ge k$ gilt: $d(f_i,f_j)< \varepsilon$. Sei $m_0\in M$. Für $m\in M$ gilt dann:
\begin{align*}
e(f(m), f(m_0)) &\le e(f(m),f_k(m_0)) + e(f_k(m_0), f(m_0)) \\
&\le e(f(m),f_k(m)) + e(f_k(m),f_k(m_0)) + e(f_k(m_0), f(m_0)) \\
&\le \varepsilon + R + \varepsilon 
\end{align*}
Zur Übung: Der Rest.

Jeder endlichdimensionale euklidische Vektorraum ist vollständig. 

% Die Konvergenz einer Folge in ${\cal B}(X)$ mit der $\infty$-Norm ist einfach 
% die gleichmäßige Konvergenz im Sinne der Analysis. Insbesondere ist
% ${\cal B}(X)$ mit dieser Metrik vollständig.

$\mathbb Q$ mit der $p$-adischen Metrik ist nicht vollständig. Man kann einen
Körper $\mathbb Q_p$ konstruieren, der $\mathbb Q$ enthält, auf dem eine
Metrik definiert ist, die die $p$-adische fortsetzt, und in dem sich jedes 
Element durch eine $p$-adische Cauchyfolge in $\mathbb Q$ approximieren 
lässt. Damit wird ein arithmetisch wichtiges Pendant zu den reellen Zahlen 
geschaffen, die sich ja auch konstruieren lassen als (archimedische) 
Cauchyfolgen in $\mathbb Q$ modulo Nullfolgen. 
}
\end{bsp}
\begin{defini}{Isometrien}
Es seien $(X,d)$ und $(Y,e)$ zwei metrische Räume. Eine 
{\it abstandserhaltende Abbildung} von $X$ nach $Y$  ist eine Abbildung
$f:X\longrightarrow Y,$ für die gilt:
$$\forall x_1,x_2 \in X: d(x_1,x_2) = e(f(x_1),f(x_2)).$$
Solche Abbildungen sind immer injektiv. Eine surjektive abstandserhaltende
Abbildung heißt eine {\it Isometrie}.
\footnote{Das Erlanger Programm (Felix Klein, ewa 1871) ist der Versuch, interessante Klassen von geometrischen Objekten durch das Studium ihrer Isometriegruppen zu untersuchen.}

Für eine Isometrie $f:X\to Y$ ist auch $f^{-1}$ eine Isometrie.
Die Menge der Isometrien von $X$ nach $X$ ist eine Untergruppe der { 
symmetrischen Gruppe} 
${\rm Sym}(M)$ aller Bijektionen von $X$ nach $X$.
\end{defini}

Aber eigentlich ist das momentan kein Begriff, der unsere Aufmerksamkeit zu 
stark in Anspruch nehmen sollte. 

\begin{defini}{Kugeln}
Es seien $(X,d)$ ein metrischer Raum und $x\in X$ sowie $r>0$ eine reelle 
Zahl. Dann heißt
$$B_r(x) \da \{y\in X \mid d(x,y)<r\}$$
die {\it offene Kugel} vom Radius $r$ um den Mittelpunkt $x.$ 

{\bf Vorsicht:} Weder $x$ noch $r$ müssen durch die Menge $B_r(x)$ eindeutig 
bestimmt sein.
Wenn zum Beispiel $X$ mit der diskreten Metrik ausgestattet ist, so ist 
$X=B_2(x) = B_3(y)$ für alle $x,y\in X.$
\end{defini}

\begin{defini}{Stetigkeit}
\label{stetig1}
Es seien $(X,d)$ und $(Y,e)$ zwei metrische Räume. Dann heißt eine
Abbildung $f:X\longrightarrow Y$ {\rm stetig}, falls für jedes 
$x\in X$ und jedes $\varepsilon>0$ ein $\delta>0$ existiert, so dass
$$f(B_\delta(x)) \subseteq B_\varepsilon(f(x)).$$
In Worten: Jede offene Kugel um $f(x)$ enthält das Bild einer offenen Kugel 
um $x$.
\end{defini}

So ist zum Beispiel jede Abbildung von $X$ nach $Y$ stetig, wenn auf $X$ die 
diskrete Metrik vorliegt. Denn dann ist ja $\{x\} = B_{\frac12}(x)$ im Urbild
jeder offenen Kugel um $f(x)$ enthalten.

\begin{defini}{Noch einmal die Analysis}
Es seien $X$, $Y$ metrische Räume. Dann bezeichnen wir mit 
${\cal C}(X,Y)$ die Menge aller stetigen Abbildungen von $X$ nach $Y,$ und 
mit ${\cal C}_0(X,Y)$ die Menge aller beschränkten stetigen Abbildungen
von $X$ nach $Y.$ 

Wenn $Y$ vollständig ist, dann ist auch ${\cal C}_0(X,Y)$ (als Teilraum von
${\cal B}(X,Y)$) vollständig. 

Im Fall $Y=\mathbb R$ lässt man das $Y$ auch häufig weg und schreibt nur 
${\cal C}(X)$ bzw.\ ${\cal C}_0(X).$
\end{defini}

\begin{hilfs}{Normen}\label{Normen}
Es sei $V=\mathbb R^n$ mit dem Standardskalarprodukt versehen und $N$ eine
Norm auf $V,$ d.h. $N:V\longrightarrow \mathbb R_{\geq 0}$ erfüllt 
\begin{itemize}
\item $\forall v\in V:N(v) = 0 \iff v=0$ (Positivität),
\item $\forall a\in \mathbb R,v\in V: N(av) = |a|N(v)$ (Homogenität),
\item $\forall v,w\in V: N(v+w) \leq N(v) + N(w)$ (Dreicksungleichung).
\end{itemize}
Dann ist $N$ stetig bezüglich der Standardmetrik.
\end{hilfs}
{\it Beweis.} Das Urbild von $(-\varepsilon , \varepsilon)$ unter $N$ ist 
konvex, d.h. 
$$\forall v,w\in V: [N(v),N(w)\leq \varepsilon\Rightarrow 
\forall \lambda\in [0,1]: N(\lambda v + (1-\lambda) w) \leq \varepsilon].$$
Das folgt sofort aus den drei aufgelisteten Eigenschaften der Normabbildung. 

Wegen der Positivität und der Homogenität gibt es eine Konstante $c>0$ 
(abhän\-gig von $\varepsilon$), so dass die Vektoren 
$\pm c e_i,\ 1\leq i\leq n,$
in $N^{-1}(-\varepsilon , \varepsilon)$ liegen. Dabei ist $\{e_1,\dots ,e_n\}$ 
die Standardbasis von $\mathbb R^n.$

Es sei $v\in B_{c/\sqrt n}(0)\subseteq V,$ d.h. $v= \sum_{i=1}^n a_i c e_i, 
\sum_i a_i^2 < 1/n.$ Dann ist aber die Summe $\sum_i|a_i| < 1.$ Für
$\alpha = \sum |a_i|$ ist also $v=\alpha\cdot \sum_i \frac{a_i}{\alpha}
ce_i$ das $\alpha$-fache einer Konvexkombination 
von $\pm ce_1,\dots \pm ce_n.$ Wegen  $|\alpha|<1$ liegt wegen der 
Homogenität von $N$ auch $v$ im Urbild von $(-\varepsilon, \varepsilon)$, und
damit liegt die offene Kugel $B_{c\sqrt n}(0)$ im Urbild: $N$ ist stetig im 
Ursprung. 

Nun seien $x\in V$ beliebig und $\delta>0$ vorgegeben. Dann gibt es nach dem 
eben gesehenen ein $\varepsilon>0$ mit 
$$\forall y\in B_\varepsilon (0) : |N(y)| < \delta.$$
Für $y\in B_\varepsilon (0)$ gilt demnach wegen
$N(x) = N(x+y-y)\geq N(x+y) + N(y):$
$$-N(y) \leq  N(x+y) - N(x) \leq N(y),$$
und daher $N(B_\varepsilon(x)) \subseteq B_\delta(N(x)).$ 
\nopagebreak

Das zeigt die Stetigkeit von $N.$ \qed

{\bf Bemerkung:} Es sei $N$ eine Norm auf $\mathbb R^n$, dann kann dazu eine Metrik definiert werden: $d_N(x,y) \da N(x-y)$. Der Hilfssatz zeigt, dass in jeder Kugel $B_\varepsilon(x)_{d_N}$ bezüglich $d_N$ auch eine offene Kugel um $x$ bezüglich der Standardmetrik liegt. Hier nicht gezeigt: Das gilt auch umgekehrt. Insbesondere liefert jede Norm auf $\mathbb R^n$ den selben Konvergenzbegriff.

\begin{defini}{Die Topologie eines metrischen Raums}
Es sei $(X,d)$ ein metrischer Raum. Eine Teilmenge $A\subseteq X$ heißt
{\it offen} \index{offen}, falls für jedes $x\in A$ eine reelle Zahl $r>0$
existiert, so dass $B_r(x)\subseteq A$ gilt. 

Die Gesamtheit aller offenen Mengen in $X$ heißt die {\it Topologie} von 
$(X,d).$
\end{defini}

{\bf Beispiel:} Für die diskrete Metrik $d$ auf einer Menge $X$ ist die Topologie gleich $\mathcal P(x)$: Für $M\subseteq X$, $m\in M$ gilt: $B_{\frac{1}{2}}(m) = \{m\} \subseteq M$.

\begin{bem}{Eigenschaften}
Für einen metrischen Raum $(X,d)$ mit Topologie $\mathcal T$ gilt:
\begin{itemize}
\item Sowohl $\emptyset$ als auch $X$ sind offen.
\item Beliebige Vereinigungen und endliche Durchschnitte von offenen
Mengen sind wieder offen.

\item Eine Abbildung $f:X\longrightarrow Y$ zwischen metrischen Räumen ist genau 
dann stetig, wenn für jede offene Teilmenge $U\subseteq Y$ das Urbild
$f^{-1}(U)$ offen in $X$ ist.
\end{itemize}

Es kann sehr viele verschiedene Metriken auf $X$ geben, die zur selben 
Topologie führen. So stimmen zum Beispiel für zwei Normen auf $\mathbb R^n$
die zugehörigen Topologien überein, was im Wesentlichen aus Hilfssatz 
\ref{Normen} folgt. Dieser Hilfssatz sagt nämlich, dass die Identität auf
$\mathbb R^n$ stetig ist, wenn wir auf Seiten des Definitionsbereichs die
euklidische Standardlänge als Norm benutzen, und auf Bildseite die Norm $N.$
Auch in der anderen Richtung ist die Identität stetig (das muss man noch 
beweisen!), und das impliziert die Gleichheit der zugehörigen Topologien.   
\end{bem}




\chapter{Topologische Grundbegriffe}

\section{Topologische Räume und ein paar Konstruktionen}

\begin{defini}{Topologischer Raum}
Ein {\it topologischer Raum} \index{topologischer Raum} ist eine Menge 
$X$, für die eine Teilmenge 
${\cal T} \subseteq {\cal P} (X)$ mit folgenden Eigenschaften ausgewählt 
wurde:
\begin{itemize}
\item $\emptyset,\,X\in {\cal T}$
\item $\forall A,B\in {\cal T}: A\cap B\in {\cal T}$
\item $\forall {\cal S}\subseteq{\cal T}: \bigcup_{A\in {\cal S}}A \in {\cal T}.$
\end{itemize}
Hierbei heißt $\cal T$ die Topologie auf $X$, und die Elemente von 
$\cal T$ sind die \index{offen} {\it offenen Mengen} des 
topologischen Raums $(X,{\cal T}).$ 

Die Mengen $X\setminus A, A\in {\cal T},$ heißen {\it abgeschlossene
Teilmengen}\index{abgeschlossen} von $X$.
\end{defini}

\begin{bsp}{Alte Bekannte}
Die offenen Mengen eines metrischen Raums $X$ bilden eine Topologie auf 
$X$.

Ist $X$ eine beliebige Menge, so ist die Potenzmenge ${\cal P}(X)$  eine
Topologie auf $X.$ Sie ist sogar die Topologie zu einer Metrik auf $X$, zur 
diskreten Metrik nämlich. Sie heißt die {\it diskrete Topologie}.

Auch $\{\emptyset, X\}$ ist eine Topologie auf $X.$

Auf $\{0,1\}$ gibt es die Topologie $\{\emptyset, \{0\}, \{0,1\}\}.$
Diese kommt nicht von einer Metrik her.
\end{bsp}

\begin{defini}{Inneres, Abschluss und der zu schmale Rand}
Es seien $X$ ein topologischer Raum und $A\subseteq X$ eine offene 
Teilmenge. Das {\it Innere} $\mathring A$ von $A$ ist definiert als die
Vereinigung
$$\mathring A \da \bigcup_{\mathclap{U\subseteq A,\,U\>\text{offen}}} U\,.$$

$\mathring A$ ist offen.

Der {\it Abschluss} $\bar A$ von $A$ ist definiert als der Durchschnitt aller 
abgeschlossenen Teilmengen von $X$, die $A$ enthalten.

$\bar B$ ist abgeschlossen.

Der {\it Rand} von $A$ ist die Menge 
$\partial A\da\bar A\setminus \mathring A$. Sie ist abgeschlossen.
\end{defini}

{\bf Warnung:} Im metrischen Raum $(X,d)$ muss nicht gelten, dass für $r>0$, $x\in X$ die Menge $\overline{B_r(x)} = \{ y \in X \mid d(x,y) \le r\}$. Ein Gegenbeispiel ist etwa die diskrete Metrik $D$, also ist $B_1(x) = \{x\}$ abgeschlossen, und damit $\overline{B_1(x)} = B_1(x) \ne \{y\in X\mid d(x,y) \le 1\} = X$.

\begin{defini}{Dichtheit, Diskretheit} \label{dicht}
Sei $(X,\mathcal T)$ ein topologischer Raum.
\begin{enumerate}[a)]
\item Eine Teilmenge $A\subseteq X$ heißt 
{\it dicht}\index{dicht}, wenn ihr Abschluss ganz $X$ ist. 

Jede Teilmenge ist also dicht in ihrem Abschluss, wenn man diesen wie in 
\ref{Spurtopologie} als topologischen Raum betrachtet.

$X$ heißt {\it separabel}\index{separabel}, wenn es eine abzählbare dichte Teilmenge von $X$ gibt.

Ein Punkt $x\in X$ heißt {\it generisch}\index{generisch}, wenn $\overline{\{x\}}=X$.

{\bf Beispiel:} Sei $X = \mathbb C$. Eine Teilmenge $V\subseteq \mathbb C$ sei abgeschlossen, wenn es Polynome $f_i \in \mathbb Q[X]$, $i\in I$, gibt so dass $V = \{ z \in \mathbb C \mid \forall i \in I: f_i(z) = 0\}$. Für $z\in \mathbb C$ ist $\overline{\{z\}} = \{ y \in \mathbb C \mid \forall f \in \mathbb Q[X]: f(z) = 0 \implies f(y) = 0\}$. So ist etwa $\overline{\{0\}} = \{0\}$, $\overline{\{\sqrt 2\}} = \{ \sqrt 2 , - \sqrt 2\}$ und $\overline{\{\pi\}} = \mathbb C$, denn $\pi$ ist transzendent.

\item Eine Teilmenge $D\subseteq X$ heißt \index{diskret}{\it diskret}, wenn
jeder Punkt $x\in X$ in einer offenen Menge liegt, die mit $D$ endlichen Durchschnitt 
hat.

Für metrische Räume heißt das gerade, dass $D$ keinen Häufungspunkt 
besitzt. 
\end{enumerate}
\end{defini}


\begin{defini}{Umgebungen, Basis einer Topologie}
Es sei $(X,{\cal T})$ ein topologischer Raum.
\begin{enumerate}[a)]
\item Für $x\in X$ heißt eine Teilmenge $A\subseteq X$ eine 
{\it Umgebung}\index{Umgebung} von $x$, falls eine offene Teilmenge 
$U\subseteq X$ existiert mit $x\in U\subseteq A.$ Ist $A$ selbst schon offen, 
so heißt es eine {\it offene Umgebung} von $x$ (falls $x\in A$).

Insbesondere ist $A\subseteq X$ genau dann offen, wenn sie für jedes $a\in A$ eine Umgebung ist.

\item Eine Teilmenge ${\cal B}\subseteq {\cal T}$ heißt eine {\it Basis}
von ${\cal T},$ falls jedes Element von $\cal T$ sich schreiben lässt als
Vereinigung von Elementen aus ${\cal B}.$

(So sind zum Beispiel die offenen Kugeln $B_r(x)$ eine Basis der Topologie auf 
einem metrischen Raum.)

$\cal S$ heißt eine {\it Subbasis} von $\cal T$, wenn die endlichen Durchschnitte von Elementen aus ${\cal S}$ eine Basis der Topologie ist.

Für jede Teilmenge $\cal S \subseteq \cal P(X)$ gibt es eine Topologie, die $\cal S$ als Subbasis besitzt. Diese heißt die von $\cal S$ erzeugte Topologie und hat die Basis $\mathcal B = \{S_1\cap\cdots\cap S_n \mid S_i \in \mathcal S,\, k\in \mathbb N\}$ und die offenen Mengen $\mathcal T = \{ \bigcup_{i\in I} U_i \mid U_i \in B \text{ und Indexmenge } I\}$.

\item Für $x\in X$ heißt eine Menge $\mathcal B_x\subseteq \mathcal T$ von Umgebungen von $x$ 
eine {\it Umgebungsbasis} von $x$, wenn jede Menge in $B_x$ den Punkt $x$ enthält und jede Umgebung von $x$ ein Element von $\mathcal B_x$ als Teilmenge enthält.

{\bf Nebenbemerkung:} Eine Teilmenge $\mathcal B \subseteq \mathcal T$ ist genau dann eine Basis der Topologie, wenn für alle $x\in X$ gilt: $\mathcal B_x \da \{ U\in \mathcal B \mid x\in U\}$ ist Umgebungsbasis von $x$.
\end{enumerate}
\end{defini}

\begin{bem}{Einsichtig}
Eine Teilmenge ${\cal B }\subseteq {\cal T}$ ist genau dann eine Basis 
der Topologie $\cal T$, wenn sie für jedes $x\in X$ eine Umgebungsbasis 
enthält.

Für jede Teilmenge $\cal B$ von ${\cal P}(X)$ gibt es genau eine Topologie,
die $\cal B$ als Subbasis besitzt. Sie ist die {\it von ${\cal B}$ erzeugte } 
Topologie, und besitzt 
$$\{U_1\cap \dots \cap U_n \mid n\in\mathbb N, U_i\in {\cal B}\}$$
als Basis.
\end{bem}

\begin{defini} {\bf Feinheiten}
Wenn ${\cal T}_1$, ${\cal T}_2$ zwei Topologien auf einer Menge $X$ sind, so
heißt ${\cal T}_1$ {\it feiner} als ${\cal T}_2,$ wenn ${\cal T}_2\subseteq
{\cal T}_1,$ also wenn ${\cal T}_1$ mehr offene Mengen besitzt als ${\cal T}_2.$

Die feinste Topologie auf $X$ ist also die diskrete $\mathcal P(X)$, während 
$\{\emptyset, X\}$ die gröbste Topologie auf $X$ ist.

Zu je zwei Topologien gibt es eine gemeinsame Verfeinerung. Die gröbste 
gemeinsame Verfeinerung ist die Topologie, die die Vereinigung der beiden 
gegebenen als Subbasis besitzt.
\end{defini}

\begin{defini} {\bf Teilräume und Produkte}\label{Spurtopologie}
\begin{enumerate}[a)]
\item Es seien $X$ eine Menge und $(Y,{\cal S})$ ein Topologischer Raum. 
Weiter sei $f:X\longrightarrow Y$ eine Abbildung. Für zwei Teilmegen
$A,B\subseteq Y$ gilt
$$f^{-1}(A\cup B) = f^{-1}(A)\cup f^{-1}(B), \quad
f^{-1}(A\cap B) = f^{-1}(A)\cap f^{-1}(B).$$
Das zeigt im Wesentlichen bereits, dass 
$${\cal T}\da \{ f^{-1}(U) \mid U\in {\cal S}\} \subseteq \mathcal P(X)$$
eine Topologie auf $X$ ist. Man nennt sie die \index{Spurtopologie}
{\it Spurtopologie} auf $X$ (bezüglich der Abbildung $f$).

Damit können wir unheimlich viele neue topologische Räume konstruieren.
(Tun Sie das!) 
\item Ist speziell $X\subseteq Y$ und $f$ die Einbettung dieser Teilmenge,
so nennt man $X$ (mit der Spurtopologie) einen {\it Teilraum} von $Y.$

Eine Teilmenge $A$ von $X$ ist genau dann offen bezüglich der Spurtopologie, 
wenn es eine offene Teilmenge $U$ von $Y$ gibt mit $A= U\cap X.$

{\bf Vorgriff:} Eine stetige Abbildung $f: X\to Y$ zwischen topologischen Räumen is teine Abbildung $f$, so dass für alle offenen $U\subseteq Y$ gilt: $f^{-1}(U) \subseteq X$ ist offen. Ist $X$ eine Menge, $(Y,\mathcal T)$ ein topologischer Raum und $f: X\to Y$, dann ist die Spurtopologie bezüglich $f$ die gröbste Topologie auf $X$, für die $f$ stetig ist.

\item Sind $X,Y$ zwei topologische Räume, so definieren wir auf 
$X\times Y$ die {\it Produkttopologie}\index{Produkttopologie}, indem wir 
als Basis die Produkte $U\times V$ für offene $U\subseteq X$ und
$V\subseteq Y$ verwenden. 
\end{enumerate}
\end{defini}

\begin{defini}{Quotiententopologie} \label{Quotienten}
% Anders definiert, nicht das gleiche!
%
%Es sei $X$ ein topologischer Raum und $\equiv$ eine Äquivalenzrelation
%auf $X.$ Dann wird auf dem Raum $\faktor X \equiv$ der Äquivalenzklassen von $X$ 
%eine Topologie eingeführt, indem man für offenes $U\subseteq X$ die
%Menge 
%$$\{[u]\mid u\in U\}$$
%aller Äquivalenzklassen von Elementen aus $U$ zur offenen Menge erklärt
%und die davon erzeugte Topologie verwendet. Diese Topologie heißt die
%

Es seien $(X,\mathcal T)$ ein topologischer Raum und $\sim$ eine Äquivalenzrelation auf $X$. Dann definieren wir auf $\faktor X \sim$ eine Topologie $\mathcal {\tilde T}$ durch die Vorschrift: $M \subseteq \faktor X \sim$ ist offen, wenn $\bigcup_{m\in M} m \subseteq X$ offen ist.

Für die kanonische Projektion $\pi :X \to \faktor X\sim$, $\pi(x) = [x]$, heißt das: $M\subseteq \faktor M \sim $ ist genau dann offen, wenn $\pi^{-1}(M) \subseteq M$ offen ist. $\mathcal{\tilde T}$ ist die feinste Topologie, für die $\pi$ stetig ist. Sie heißt {\it Quotiententopologie}\index{Quotiententopologie} (von $\mathcal T$ unter $\sim$).

Damit bekommen wir zum Beispiel eine Topologie auf dem projektiven Raum
$\mathbb P^n(\mathbb R)$ oder $\mathbb P^n(\mathbb C).$ Die Topologie auf 
$\mathbb P^1(\mathbb C)$ verdient hier historisch und didaktisch besondere
Aufmerksamkeit. Eine Teilmenge $A\subseteq\mathbb P^1(\mathbb C) = \mathbb C
\cup\{\infty\}$ ist genau dann offen, wenn $A\cap \mathbb C$ offen ist und wenn
zusätzlich im Fall $\infty\in A$ ein $R>0$ existiert mit 
$$\{z\in \mathbb C \mid |z|>R \} \subseteq A.$$
\end{defini}

{\bf Vorsicht:} Ein Quotientenraum $(\faktor X \sim, \mathcal {\tilde T})$ muss nicht immer schön sein, auch wenn $X$ das war. So hat etwa der Raum $\faktor {\mathbb R}{\mathbb Q}$ die Quotiententopologie $$\mathcal{\tilde T} = \{ \emptyset,\, \faktor{\mathbb R}{\mathbb Q}\}\,.$$

{\bf Warnung:} Im Allgemeinen ist es falsch, dass für offenes $O\subseteq X$ die Menge $\pi(O) = \{ [x] \mid x\in O\} \subseteq \faktor X\sim$ offen ist. Ist beispielsweise $X=\mathbb R$ und betrachte die Äquivalenzrelation mit Äquivalenzklassen $[0,1]$, $\{y\}$ für $y\notin [0,1]$. Dann ist $\{[0]\} = \pi( (0,1) )$, aber $\pi^{-1}(\{0\}) = [0,1]$ ist nicht offen.

\section{Wichtige Eigenschaften topologischer Räume}
\begin{defini}{Kompaktheit}
Ein topologischer Raum $X$ heißt \index{kompakt}{\it kompakt}, wenn
jede Überdeckung $X=\bigcup_{i\in I}U_i$ von $X$ durch offene Mengen eine
endliche Teilüberdeckung enthält: 
$$\exists n\in \mathbb N, i_1,\dots i_n\in I: X=\bigcup_{k=1}^n U_{i_k}.$$
Genauso heißt eine Teilmenge von $X$ kompakt, wenn sie bezüglich der 
Spurtopologie (der Inklusion) kompakt ist.

Anstelle des Begriffs „kompakt“ wird auch gelegentlich 
„überdeckungsendlich“ verwendet. Es ist nicht ganz einheitlich, ob 
zur Kompaktheit auch die Eigenschaft, hausdorff'sch zu sein (siehe später), 
gehört oder 
nicht. Wir wollen hier Kompaktheit so verstehen wie gesagt.
\end{defini}
\begin{hilfs}{Kompakta in metrischen Räumen}
Es sei $(X,d)$ ein metrischer Raum. Dann gilt:
\begin{enumerate}[a)]
\item Ist $X$ kompakt, dann ist $X$ beschränkt.
\item Ist $K\subseteq X$ kompakt, dann ist $K$ abgeschlossen in $X$.
\item Ist $X$ kompakt, so ist $X$ auch vollständig.
\end{enumerate}
\end{hilfs}

{\it Beweis.} 
\begin{enumerate}[a)]
\item Sei ohne Einschränkung $X$ nicht leer und $x\in X$. 
Es ist $$X=\bigcup_{n\in \mathbb N} B_n(x),$$ also gibt es $n_1 \le \cdots \le n_k\in \mathbb N: X= \bigcup_{i=1}^kB_{k_i}(x)= B_{n_k}(x)$, da $B_{n_1}(x)\subseteq B_{n_2}(x) \subseteq \cdots \subseteq B_{n_k}(x)$, also hat $X$ einen Durchmesser kleiner $2n_k$.

\item Sei $K\subseteq X$ kompakt, $x\in X\setminus K$. Zu zeigen: $\exists \varepsilon > 0 : B_\varepsilon(x) \cap K = \emptyset$. Betrachte für $\varepsilon>0$ die abgeschlossene Menge $D_\varepsilon(x) = \{y\in X \mid d(x,y) \le \varepsilon\}$. Es gilt
$$ \bigcap_{\varepsilon >0} D_{\varepsilon}(x) = \{x\} \implies K\subseteq \bigcap_{\varepsilon>0} \underbrace{X\setminus D_{\varepsilon}(x)}_{\text {offen}}.$$
$X$ ist kompakt, also gibt es endlich viele $\varepsilon_1 > \varepsilon_2 > \cdots >\varepsilon_n >0$ mit $K\subseteq \bigcup _{i=1}^n X\setminus D_{\varepsilon_i}(x) = X \setminus D_{\varepsilon_n}(x) \implies B_{\varepsilon_n}(x) \subseteq X\setminus K$. $x\in X\setminus K$ war beliebig, also ist $K$ abgeschlossen.

\item Sei $X$ kompakt und $(x_n)_{n\in\mathbb N}$ eine Cauchy-Folge in $X$. Wenn diese Folge nicht konvergiert, dann existiert für jedes $x\in X$ ein $\varepsilon>0$ derart, dass für unendlich viele $n\in \mathbb N$ gilt: $x_n \notin B_\varepsilon(x)$. Es gilt: $\exists N > 0 \forall k, l \in \mathbb l,\, k,l\ge N: d(x_k,x_l) < \frac\varepsilon2$. Wäre hier $x_k \in B_{\frac \varepsilon 2}$, so wäre $x_l \in B_\varepsilon(x)$ für alle $l\ge n$, im Widerspruch zur Wahl von $\varepsilon$.

Also liegen in $B_{\frac\varepsilon2}$ liegen nur Folgenglieder $x_n$ mit $n<N$, also endlich viele. Das heißt: für jedes $x\in X$ existiert eine offene Umgebung $U_x$ von $x$, in der nur endlich viele Folgenglieder liegen. Diese offenen Mengen überdecken $X$, und aus der Kompaktheit von $X$ folgt: $\exists y_1,\ldots,y_n \in X: X= \bigcup_{i=1}^n U_{y_i}$. In jedem $U_{y_i}$ liegen nur endlich viele Folgenglieder, also liegen auch in $X$ nur endlich viele, was einen Widerspruch darstellt.  \qed
\end{enumerate}


% Ein kompakter metrischer Raum $X$ ist sicher beschränkt, denn 
% gilt für jedes $x\in X,$ und das ist eine offene Überdeckung von $X.$
% 
% Eine kompakte Teilmenge $A$ eines metrischen Raums $X$ ist 
% auch abgeschlossen. Ist nämlich $x\in X\setminus A$ im Komplement von 
% $A,$ so ist 
% $$A\subseteq \bigcup_{n\in \mathbb N} \{y\in X\mid d(y,x)> \nicefrac1n \}$$
% eine offene Überdeckung von $A,$ und damit langen endlich viele dieser
% Mengen, um $A$ zu überdecken. Es ist also 
% $$A\subseteq \{y\in X\mid d(y,x)> \nicefrac1n \}$$
% für ein festes $n\in \mathbb N,$ und daher ist $B_{\nicefrac1n}(x)$ in 
% $X\setminus A$ enthalten.
% 
% Ein etwas feineres Argument zeigt, dass ein kompakter metrischer Raum sogar
% vollständig ist.
% 
% Eine abgeschlossene Teilmenge $A$ eines kompakten Raums $X$ ist kompakt, denn
% für jede Überdeckung $\ddot U$ von $A$ durch offene Teilmengen von $X$ 
% ist $\ddot U\cup\{X\setminus A\}$ eine offene Überdeckung von $X$, also
% langen endlich viele davon, um $X$ zu überdecken, und von diesen endlich
% vielen kann man notfalls $X\setminus A$ weglassen, um eine endliche 
% Teilüberdeckung von $A$ zu erhalten.
% 
% \end{hilfs}

\begin{bem}{Kompakta in sonstigen topologischen Räumen}
Es sei $X$ ein topologischer Raum.
\begin{enumerate}[a)]
\item Nicht jeder kompakte Teilraum von $X$ muss abgeschlossen sein.

Betrachte zum Beispiel den Raum $X=\mathbb Z$ mit der „koendlichen“ Topologie: Die offenen Mengen sind $\emptyset$, $\mathbb Z$, sowie alle Komplemente von endlichen Teilmengen von $\mathbb Z$. Hier ist jede Teilmenge kompakt. Ist nämlich $\emptyset \ne A\subseteq \mathbb Z$ mit $A\subseteq \bigcup_{i\in I} U_i$, $U_i$ offen. Wähle ein $i_1 \in I$ mit $U_{i_1} \ne \emptyset$. Dann existieren $z_1,\ldots, z_k\in \mathbb Z$ so dass $U_{i_1} =\mathbb Z\setminus \{z_1,\ldots, z_k\}$. Seien $z_1,\ldots,z_d\in A$ und $z_{d+1},\ldots,z_k\notin A$.Dann gilt für jedes $j=1,\ldots,d$, daas es ein $i_{j+1}\in I$ gibt mit $z_j\in U_{i_{j+1}}$. Also ist $A\subseteq \bigcup_{j=1}^{d+1} U_{i_j}$. Aber es gibt nicht abgeschlossene Mengen, etwa $\mathbb N$. 

\item Ist $X$ kompakt und $A\subseteq X$ abgeschlossen, dann ist $A$ kompakt:

Sei $(U_i)_{i\in I}$ eine offene Überdeckung von $A$, also $A\subseteq \bigcup_{i\in I} U_i$. Dann ist 
$$ X = \bigcup_{i\in I} U_i \cup \underbrace{(X\setminus A)}_{\text{offen}} $$
eine offene Überdeckung von $X$. Also gibt es $i_1,\ldots,i_n \in I$ so dass $X= \bigcup_{j=1}^n U_{i_j} \cup (X\setminus A)$ und damit $A \subseteq \bigcup_{j=1}^n U_{i_j} $. Das zeigt, dass $A$ kompakt ist.

\item Es seien $K\subseteq X$ kompakt und $D\subseteq X$ diskret. Dann ist $D\cap K$ endlich:

Für jedes $x\in X$ existiert eine offene Umgebung $U_x$ von $x$, die mit $D$ endlichen Durchschnitt hat, also kann man $K$ mit den $U_x$, $x\in K$ überdecken. Da $K$ kompakt ist, kann man daraus eine endliche Überdeckung wählen und es gilt
$$ K\cap D \subseteq (\bigcup _{i=1}^n U_{x_i}) \cap D = \bigcup _{i=1}^n (U_{x_i} \cap D ),$$
was endlich ist.
\end{enumerate}
\end{bem}

\begin{satz}{\`a la Heine\footnote{Heinrich-Eduard Heine, 1821-1881}-Borel\footnote{Emile Borel, 1871-1956}}
\label{heineborel}

Es sei $X$ ein vollständiger metrischer Raum. Dann sind äquivalent:

\begin{enumerate}[a)]
\item Für jede beschränkte
Menge $B\subseteq X$ und jedes $\varepsilon>0$ lässt sich $B$ durch endlich viele
Mengen von Durchmesser $\leq \varepsilon$ überdecken.

\item Die folgenden zwei Aussagen für Teilmengen $A\subseteq X$ sind äquivalent:
\begin{enumerate}[i)]
\item $A$ ist kompakt.
\item $A$ ist beschränkt und abgeschlossen.
\end{enumerate}

\end{enumerate}

\end{satz}

{\it Beweis.} 
„a) $\Longrightarrow$ b)“:
Wir haben schon gesehen: ein Kompaktum 
in einem metrischen Raum ist abgeschlossen und beschränkt, also i) $\implies$ ii).

Zu zeigen: ii) $\implies$ i). Sei $A\subseteq X$ abgeschlossen und beschränkt. Weiter sei 
$\ddot U = \{U_i \mid i\in I\}$ eine offene Überdeckung von $A$. Nehmen wir an, es gebe in 
$\ddot U$ keine endliche Teilüberdeckung von $A$.

Wegen Bedingung a) gibt es endlich viele Reilmengen $T_1,\ldots,T_n$ von $A$ von Durchmesser $\le 1$, so dass $A=T_1\cup\cdots\cup T_n$. Ohne Beschränkung der Allgemeinheit seien $T_1,\ldots,T_n$ abgeschlossen, denn $A$ ist Abgeschlossen und wenn $T$ Durchmesser $\le1$, dann auch $\bar T$.

Die Annahme an $A$ erzwingt, dass wenigstens ein $T_j$ sich nicht durch endlich viele der $U_i$ überdecken lässt. Sei $A_1 \in \{T_1,\ldots,T_n\}$ ein solches. $A_1$ lässt sich durch endlich viele abgeschlossene Mengen $S_1,\ldots,S_n\subseteq A_1$ von Durchmesser $\le \frac12$ überdecken. Wähle analog $A_2$, als eines der $S_j$, so dass es sich nicht durch endlich viele $U_i$ überdecken lässt.

Konstruiere rekursiv
$$A\supseteq A_1\supseteq A_2\cdots \supseteq A_k\supseteq \cdots$$
derart, dass $A_k$ Durchmesser $\leq 1/2^k$ hat, abgeschlossen ist und sich nicht durch endlich 
viele $U\in \ddot U$ überdecken lässt. 

$$D \da \bigcap_{k\in \mathbb N} A_k \subseteq A$$ ist abgeschlossen und enthält höchstens ein Element. Um zu zeigen, dass $D$ nicht leer ist, wählen wir sukzessive für $k\in \mathbb N$ ein  $a_k \in A_k$. Dann ist $(x_i)_{i\in \mathbb N}$ eine Cauchy-Folge, denn 
$$d(x_i,x_k) \leq 1/2^{\max (i,k)}.$$

Also konvergiert die Folge gegen ein $a\in A,$ da $X$ vollständig und $A$
abgeschlossen ist.

Wäre $a\notin D$, dann gäbe es ein $k$ mit $a\notin A_k$, und da $A_k$ abgeschlossen ist, existierte ein $\varepsilon >0$ mit $B_\varepsilon\cap A_k =\emptyset$, also für alle $x\in A_k$ gelte $d(x,a) > \varepsilon$, also wäre für alle $l\ge k$: $a_k \in A_l \subseteq A_k$, also $d(a_k, a) > \varepsilon$.

Also gilt $D=\{a\}$.

Da $a\in A$, liegt es in einem der $U_i$, und da $U_i$ offen ist, gibt es ein $r>0$, so dass $B_r(A)\subseteq U_i$. Für $\frac1{2^k}<\frac 1r$ folgt dann aber wegen $a\in A_k$, dass $A_k \subseteq U_i$, im Widerspruch zur Konstruktion der Mengen $A_k$. Also ist die Annahme falsch und daher $A$ kompakt.

„b) $\Longrightarrow$ a)“: Es sei $B\subseteq X$ beschränkt und abgeschlossen, also nach Vorbedingung b) kompakt. Sei $\varepsilon > 0$, dann $$\bar B \subseteq \bigcup_{x\in\bar B} B_{\frac \varepsilon2}(x)$$
Es ist $\bar B$ kompakt, also existieren $x_1,\ldots,x_n \in \bar B$, so dass gilt:
$$ B\subseteq \bar B \subseteq  \bigcup_{i=1}^n B_{\frac \varepsilon2}(x_i)$$
\qed


\begin{bem}{Beispielmaterial} \label{einige Kompakta} 

\begin{enumerate}[a)]
\item Als Spezialfall erhalten wir den klassischen Satz von Heine-Borel,
der sagt, dass eine Teilmenge von $\mathbb R^n$ genau dann überdeckungsendlich
ist, wenn sie abgeschlossen und beschränkt ist. 

Heine hat diesen Satz 1872 
für Intervalle in $\mathbb R$ benutzt, um zu zeigen, dass eine stetige 
Funktion auf einem beschränkten und abgeschlossenen Intervall gleichmäßig stetig ist.

\item $\mathbb R^2$ mit der SNCF-Metrik (siehe Übungsblatt 2) stimmt Bedingung \ref{heineborel}~a) nicht.

\item
Es sei $X\da{\cal C}_0(\mathbb N)$ der Raum der beschränkten Funktionen 
auf $\mathbb N$ (siehe \ref{L_unendlich}). 

Für $n\in\mathbb N$ sei $\delta_n$ die Funktion auf $\mathbb N,$ die
auf $n$ den Wert 1 annimmt, und sonst den Wert 0. Die Menge
$$D\da\{ \delta_n \mid n\in \mathbb N\}$$
ist eine beschränkte, abgeschlossene Teilmenge von $X.$ Aber kompakt ist sie 
nicht, denn in $B_{1/2}(\delta_n)$ liegt kein weiteres 
$\delta_k, k\in \mathbb N,$ und so ist
$$D = \bigcup_{n\in \mathbb N} B_{1/2}(\delta_n)$$
eine offene Überdeckung von $D$ ohne endliche Teilüberdeckung.

In der Funktionalanalysis spielen ähnliche Räume eine wichtige Rolle, und
insbesondere die Frage, wann die abgeschlossene Einheitskugel in einem 
normierten Vektorraum kompakt ist.

\item Es gibt auch topologische Räume, in denen {\it jede Teilmenge} 
kompakt ist, egal ob offen, abgeschlossen, keins von beiden\dots

Als Beispiel hierfür nehme ich eine (beliebige!) Menge $X$ und versehe sie
mit der {\it koendlichen} Topologie. Dies heißt, dass neben der leeren Menge
genau die Mengen offen sind, deren Komplement in $X$ endlich ist.

Klar: hier ist alles kompakt. Denn für $A\subseteq X$ und offenes 
$U\neq\emptyset$ 
überdeckt $U$ bereits alles bis auf endlich viele Elemente von $A.$

\item Eine wichtige Beispielklasse für kompakte Räume sind die
projektiven Räume $\mathbb P^n(\mathbb R)$ und $\mathbb P^n(\mathbb C)$.

Im Fall $n=1$ sieht man die Kompaktheit sehr schön wie folgt: Ist $\ddot U$ 
eine offene Überdeckung von $X=\mathbb P^1(K)$ (mit $K=\mathbb R$ oder 
$\mathbb C$), so gibt es darin eine Menge $U_\infty\in \ddot U,$ so dass
$\infty\in U_\infty.$ Das Komplement von $U_\infty$ ist nach Konstruktion der
Topologie auf $X=K\cup\{\infty\}$ abgeschlossen und beschränkt (siehe 
\ref{Quotienten}~d)) und daher kompakt wegen Heine-Borel. Also reichen endlich viele
weitere Elemente aus $\ddot U$, um $K\setminus U_\infty$ zu überdecken.

\end{enumerate}

\end{bem}

\begin{defini}{zusammenhängend}\label{Zusammenhang} 
Es sei $X$ ein topologischer Raum. Dann heißt $X$ {\it 
zusammenhängend}\index{zusammenhängend}, wenn $\emptyset$ und $X$ die 
einzigen Teilmengen von $X$ sind, die sowohl offen als auch abgeschlossen sind.
Das ist äquivalent dazu, dass es keine Zerlegung von $X$ in zwei nichtleere,
disjunkte und offenen Teilmengen gibt.

Eine Teilmenge $A\subseteq X$ heißt zusammenhängend, wenn sie bezüglich 
der
Teilraumtopologie zusammenhängend ist, also genau dann, wenn sie nicht 
in der Vereinigung zweier offener Teilmengen von $X$ liegt, deren Schnitte 
mit $A$ nichtleer und disjunkt sind.

Die Vereinigung zweier zusammenhängender Teilmengen mit nichtleerem
Durchschnitt ist wieder zusammenhängend.
\end{defini}

\begin{bsp}{Intervalle}
Die zusammenhängenden Teilmengen von $\mathbb R$ sind gerade die 
Intervalle, egal ob offen oder abgeschlossen oder halboffen. Dabei werden auch 
die leere Menge und einelementige Mengen als Intervalle gesehen.

Ist nämlich $A\subseteq \mathbb R$ zusammenhängend und sind 
$x<y$ beide in $A$, so liegt auch jeder Punkt $z$ zwischen $x$ und $y$ in $A$,
da sonst 
$$A = (A\cap(-\infty,z)) \cup (A\cap (z,\infty))$$
eine disjunkte, nichttriviale, offene Zerlegung von $A$ wäre.

Ist umgekehrt $A$ ein Intervall, so sei $A=B\cup C$ eine nichttriviale
disjunkte Zerlegung. Ohne Einschränkung gebe es ein $b_0\in B$ und ein 
$c_0\in C$ mit $b_0<c_0.$ 

Es sei $z\da \sup\{b\in B\mid b<c_0\}.$ Dies liegt in $A,$ und damit auch in 
$B$ oder $C.$ Wäre $z\in B$ und $B$ offen in $A$, so müsste es ein $r>0$ 
geben mit
$$\forall a\in A: |z-a| < r \Rightarrow z\in B.$$
Also kann $z$ nicht zu $B$ gehören, wenn dies offen in $A$ ist, denn es gibt 
Elemente $c\in C, c>z,$ die beliebig nahe an $z$ dran liegen. 

Wäre $z\in C$ und $C$ offen in $A$, so gäbe es ein $r>0$, so dass
$$\forall a\in A: |z-a| < r \Rightarrow z\in C.$$
Das wiederum geht nicht, denn $z$ ist das Supremum einer Teilmenge von $B.$

Also sind weder $B$ noch $C$ offen in $A$, und das zeigt, dass $A$ 
zusammenhängend ist.

\end{bsp}

\begin{defini}{Zusammenhangskomponenten}
Es sei $X$ ein topologischer Raum. Wir nennen zwei Punkte $x,y$ in $X$
{\it äquivalent}, falls es eine zusammenhängende Teilmenge von $X$ gibt, 
die beide enthält. Dies ist tatsächlich eine Äquivalenzrelation:
\begin{itemize}
\item $x\simeq x$ ist klar für alle $x\in X,$ denn $\{x\}$ ist 
zusammenhängend. 
\item Symmetrie ist auch klar, nicht wahr?
\item Transitivität: Es seien $x\simeq y$ und $y\simeq z$, dann gibt es 
zusammenhängende $A,B\subseteq X,$ so dass $x,y\in A$ und $y,x\in B$. 
Aber $A\cup B$ ist auch zusammenhängend, denn aus $A\cup B = U\cup V$ (offene
Zerlegung) folgt
$A=(A\cap U) \cup (A\cap V)$ und analog für $B$, wir hätten also 
disjunkte offene Überdeckungen von $A$ und $B$, und damit folgt OBdA 
$A\subseteq U, A\cap V = \emptyset.$ Genauso ist auch $B$ in einer der beiden 
Mengen enthalten und hat mit der anderen leeren Schnitt. Aus 
$B\subseteq U$ folgt $V=\emptyset,$ während aus $B\subseteq V$ folgt, dass
$U$ und $V$ nicht disjunkt sind: beide enthalten $y.$
\end{itemize}
Die Äquivalenzklasse von $x$ heißt die {\it Zusammenhangskomponente} von 
$x$ und heißt $\pi_0(x)$. Diese ist weder zwangsläufig offen noch zwangsläufig abgeschlossen. Die Menge aller Äquivalenzklassen notiert man als $\pi_0(X)$.
\end{defini}

\begin{defini}{Trennungsaxiom}
Es sei $X$ ein topologischer Raum.
\begin{enumerate}[a)]
\item $X$ heißt ein $T_1$-Raum, wenn die einelementige Mengen abgeschlossen sind. Äquivalent: Für $x\ne y \in X$ gibt es eine Umgebung $U$ von $y$, die $x$ nicht enthält.

{\bf Beispiel:} $\mathbb Z$ mit der koendlichen Topologie, denn hier sind alle endlichen Mengen abgeschlossen.

\item 
$X$ heißt $T_2$-Raum oder \index{hausdorff'sch}
{\it hausdorff'sch}\footnote{Felix Hausdorff, 1868-1942}, wenn je zwei
Punkte $x\neq y$ in $X$ disjunkte Umgebungen haben.

{\bf Beispiel:} $\mathbb Z$ mit der koendlichen Topologie is \emph{kein} Hausdorff-Raum.
\end{enumerate}
\end{defini}

\begin{bsp}{Vererbung}

\begin{enumerate}[a)]
\item Metrische Räume sind hausdorff’sch: für $x,y\in X$, $x\ne y$ sei $r=d(x,y)$. Dann gilt: $B_{\frac r2}(x) \cap B_{\frac r2}(y) = \emptyset$.

\item Wenn $X$ hausdorff’sch ist und $U\subseteq X$, dann ist $U$ in der Teilraumtopologie hausdorff’sch.

\item Wenn $X$, $Y$ zwei Hausdorff-Räume  sind, dann auch $X\times Y$.

{\it Beweis.} 
Seien $(x,y), (\tilde x, \tilde y) \in X\times Y$ verschieden. Dann gilt ohne Beschränkung der Allgemeinheit: $x \ne \tilde x$. $X$ ist Hausdorff’sch, also gibt es offene Umgeungen $U$ um $x$ und $V$ um $\tilde x$ mit leerem Schnitt. Damit sind auch $U\times Y$ und $V\times Y \subseteq X\times Y$ offen und disjunkt. $U\times Y$ ist Umgebung von $(x,y)$, $V\times Y$ ist Umgebung von $(\tilde x, \tilde y)$, also ist $H\times Y$ hausdorff’sch.

\item Ein Quotientenraum eines Hausdorff-Raums muss nicht hausdorff’sch sein.

{\bf Beispiel:} $\faktor{\mathbb R}{\mathbb Q}$ hat mit der Quotiententopologie nur $\faktor{\mathbb R}{\mathbb Q}$ und $\emptyset$ als offene Mengen, und ist somit nicht einmal $T_1$.
\end{enumerate}
\end{bsp}
 
\begin{bem}{Kompakta in Hausdorffräumen}
\label{fuer Liouville}
Jedes Kompaktum $K$ in einem Hausdorffraum $X$ ist abgeschlossen.
\end{bem}

{\it Beweis.} 
Ist $x\in X\setminus K$, so gibt es für jedes $k\in K$ disjunkte
offene Umgebungen $U_k$ von $k$ und $V_k$ von $x$. Es ist $\ddot U \da \{ U_k\mid k\in K\}$ eine offene Überdeckung von $K,$ und wegen der Kompaktheit 
gibt es endlich viele $k_1,\dots ,k_n$ in $K,$ so dass 
$$K \subseteq \bigcup_{i=1}^n U_{k_i}.$$
Dazu disjunkt ist $\bigcap_{i=1}^n V_{k_i},$ aber das ist eine offene Umgebung
von $x$. Also liegt $x$ nicht im Abschluss von $K.$
\qed

\section{Stetigkeit}

\begin{defini}{Stetige Abbildungen}
Eine Abbildung $f:X\longrightarrow Y$ zwischen zwei topologischen Räumen
heißt {\it stetig}\index{stetig}, falls für jede offene Teilmenge $V$
von $Y$ das Urbild $f^{-1}(V)$ in $X$ offen ist.
\end{defini}

Es genügt, diese Bedingung für eine (Sub-)Basis der Topologie von $Y$ zu testen.

Wie bei metrischen Räumen werden wir mit ${\cal C}(X,Y)$ die Menge aller 
stetigen Abbildung zwischen den topologischen Räumen $X$ und $Y$ bezeichnen. Mit $\mathcal C(X)$ bezeichen wir $\mathbb C(X,\mathbb R)$.

Eine stetige Abbildung, die bijektiv ist, und deren Umkehrabbildung auch
stetig ist, heißt ein {\it Homöomorphismus}. Zwei topologische Räume,
zwischen denen es einen Homöomorphismus gibt, heißen kreativer Weise
{\it homöomorph}.

\begin{bem}{Sysiphos\footnote{Ignatz Sysiphos, -683 -- -651}}

\begin{enumerate}[a)]
\item Für metrische Räume $X,Y$ liefert das denselben Begriff der Stetigkeit wie unsere alte 
$\delta$-$\varepsilon$-Definition gesehen.

\item Homöomorph zu sein ist eine Äquivalenzrelation auf jeder Menge von topologischen Räumen.

In der Topologie betrachtet man zwei homöomorphe topologische Räume 
als
im Wesentlichen gleich. Eine Eigenschaft eines topologischen Raums $X$ 
heißt eine {\it topologische Eigenschaft}, wenn jeder zu $X$ homöomorphe
Raum diese Eigenschaft auch hat. Kompaktheit, Zusammenhang, Hausdorffizität
sind solche Eigenschaften. Beschränktheit oder Vollständigkeit eines 
metrischen Raums ist keine topologische Eigenschaft.

Natürlich möchte man eine Übersicht gewinnen,
wann zwei topologische Räume homöomorph sind, oder welche 
Homöomorphieklassen es insgesamt gibt. Das ist in dieser Allgemeinheit ein
aussichtsloses Unterfangen. Es gibt (mindestens) zwei Möglichkeiten, die
Wünsche etwas abzuschwächen: man kann sich entweder auf etwas speziellere
topologische Räume einschränken oder den Begriff des Homöomorphismus
ersetzen.

Das erstere passiert zum Beispiel bei der Klassifikation der topologischen
Flächen.

Für das zweitere bietet sich der Begriff der Homotopie an.

Auf beides kommen wir später noch zu sprechen.

Oft genug ist es sehr schwer nachzuweisen, dass zwei gegebene Räume nicht 
zueinander homöomorph sind. Wenn ich keine bistetige Bijektion finde, sagt 
das vielleicht mehr über mich aus als über die Räume. Hier ist es 
manchmal hilfreich, topologischen Räumen besser greifbare Objekte aus anderen
Bereichen der Mathematik zuordnen zu können, die für homöomorphe Räume 
isomorph sind, und wo dies besser entschieden werden kann. Das ist eine 
Motivation dafür, algebraische Topologie zu betreiben oder allgemeiner
eben Funktoren von der Kategorie der topologischen Räume in andere
Kategorien zu untersuchen. 

\end{enumerate}
\end{bem}

\begin{bem}{Ringkampf}
\begin{enumerate}[a)]

\item Die Identität auf $X$ ist stets ein Homöomorphismus (wenn man 
nicht zwei verschiedene Topologien benutzt\dots). Eine konstante Abbildung ist 
immer stetig. 

\item
Die Verknüpfung zweier stetiger Abbildungen $f:X\longrightarrow  Y, 
g:Y\longrightarrow Z$ ist wieder stetig. 

Insbesondere zeigt das, dass homöomorph zu sein eine Äquivalenzrelation
auf jeder Menge von topologischen Räumen ist.

\item
Sind $f:X\longrightarrow Y$ und $g:X\longrightarrow Z$ stetig, so ist auch 
$f\times g:X\longrightarrow Y\times Z$, $x\mapsto (f(x),g(x))$ stetig bezüglich der Produkttopologie.
Diese ist die feinste Topologie auf $Y\times Z$ mit dieser Eigenschaft.

\item

${\cal C}(X)$ ist wieder der Raum  der stetigen reellwertigen Funktionen auf 
$X$ (wobei $\mathbb R$ bei so etwas immer mit der Standardtopologie versehen 
ist!). Dies ist wieder ein Ring (bezüglich der üblichen Verknüpfungen), 
denn die Addition und Multiplikation sind stetige Abbildungen von 
$\mathbb R^2$ nach $\mathbb R,$ und wir können  b) und c) 
anwenden.
\end{enumerate}
\end{bem}



\begin{hilfs} {\bf Ein Erhaltungssatz}\label{Erhaltungssatz}

Es sei $f:X\longrightarrow Y$ stetig. Dann gelten:
\begin{enumerate}[a)]
\item Wenn $X$ kompakt ist und $f$ surjektiv, dann auch $f(X)$.
\item Wenn $X$ zusammenhängend ist und $f$ surjektiv, dann auch $f(X)$.
\item Wenn $Y$ hausdorff'sch ist und $f$ injektiv, dann ist $X$ 
hausdorff'sch.
\end{enumerate}
\end{hilfs}

{\it Beweis.} 
\begin{enumerate}[a)]
\item Es sei $\ddot V$ eine offene Überdeckung von $f(X)$ in $V.$ Dann
ist $\ddot U\da \{f^{-1}(U)\mid U\in \ddot U\}$ eine offene Überdeckung 
von $X.$ Da $X$ kompakt ist, gibt es endlich viele $V_1,\dots,V_n\in \ddot V,$
so dass bereits $\{f^{-1}(V_i)\mid 1\leq i\leq n\}$ eine Überdeckung von $X$ 
ist. 

Aus $f(f^{-1}(V_i)) = f(X)\cap V_i$ folgt, dass $\{V_1,\dots ,V_n\}$ das Bild
von $f$ überdecken.
\item Es sei $f(X) = A\cup B$ eine disjunkte Zerlegung von $f(X)$ in nicht 
leere Teilmengen. Wenn $A,B$ in der Spurtopologie offen wären, dann gäbe 
es offene Teilmengen $V,W$ von $Y$ mit $A=V\cap f(X), B=W\cap f(X).$

Mithin wäre $f^{-1}(V),f^{-1}(W)$ eine offene Überdeckung von $X$, die 
noch dazu disjunkt ist, da sich $V$ und $W$ nicht in $f(X)$ schneiden.

Andererseits wäre diese Teilmengen von $X$ nicht leer (weil $A$ und $B$ nicht
leer sind), und das widerspricht der Definition von Zusammenhang.

\item Es seien $x_1\neq x_2$ Punkte in $X.$ Dann sind ihre Bilder in $Y$ 
verschieden, denn $f$ soll injektiv sein. Daher haben $f(x_1)$, $f(x_2)$ in $Y$ 
disjunkte Umgebungen, und deren Urbilder sind disjunkte Umgebungen von $x_1$ 
und $x_2.$ \qed
\end{enumerate}

Die Umkehrungen gelten jeweils natürlich nicht, wie einfache Gegenbeispiele 
lehren. 

\begin{bem}{alte Bekannte}

\begin{enumerate}[a)]
\item Man sieht hier insbesondere, dass Kompaktheit, Zusammenhang und hausdorff’sch topologische Eigenschaften sind.

\item Wenn $X$ kompakt ist, und $f:X\to \mathbb R$ stetig, dann  ist $f(X) \subseteq \mathbb R$ kompakt, das heißt beschränkt und abgeschlossen. $f$ nimmt ein Maxmum und ein Minimum an.

Als Spezialfall hiervon erinnnern wir an \ref{Normen}: eine Norm $N$ auf dem 
$\mathbb R^n$ ist immer stetig bezüglich der Standardmetrik. Daher nimmt sie 
auf der (kompakten) Einheitssphäre ein positives Minimum $m$ und ein Maximum
$M$ an, und das führt wegen der Homogenität der Norm zu
$$\forall x\in \mathbb R^n : m|x| \leq N(x) \leq M|x|.$$
Dies zeigt, dass $N$ und die Standardmetrik dieselbe Topologie liefern.
\item
Wenn $X\subseteq \mathbb R$ ein Intervall ist und $f:X\to \mathbb R$ stetig, dann ist $auch f(x)$  ein 
Intervall -- das ist der Zwischenwertsatz.\index{Zwischenwertsatz}

\item Eine Teilmenge $\mathcal A \subseteq \mathcal C(X,Y)$ {\it trennt die Punkte} von $X$, wenn für alle $x_1\ne x_2$ in $X$ ein $f\in \mathcal A$ existiert, so dass $f(x_1) \ne f(x_2)$. Wenn $Y$ hausdorff’sch ist, und $\mathcal A \subseteq \mathcal C(X,Y)$ die Punkte trennt, dann ist $X$ hausdorff’sch.

{\it Beweis.}
Wenn $x_1\ne x_2$ in $X$, so gibt es ein $f\in \mathcal A$ mit $f(x_1) \ne f(x_2)$. Argumentiere weiter wie im Beweis von \ref{Erhaltungssatz} c).
\end{enumerate}
\end{bem}

Insbesondere ist also ${\cal C}(X) = {\cal C}_0(X),$ wenn $X$ kompakt ist, und 
dies ist als Teilraum von ${\cal B}(X)$ ein metrischer Raum.

%Hier gibt es nun den wichtigen Satz 

\begin{satz}{von Dini\footnote{Ulisse Dini, 1845-1918}}
\label{satzvondini}
Es sei $X$ ein kompakter Raum, $Y$ ein metrischer Raum, $(f_n)$ eine Folge in $\mathcal C(X,Y)$, die punktweise gegen $f\in \mathcal C(X,y)$ konvergiert. Weiterhin gelte für alle $x\in X$, $n\in \mathbb N$:
\[
d(f_n(x), f(x)) \ge d(f_{n+1}(x), f(x))
\]
Dann konvergiert $(f_n)$ gleichmäßig gegen $f$, das heißt in der $\infty$-Norm auf $\mathcal C(X,Y) \subseteq B(X,Y)$
\end{satz}

{\textit Beweis.}
Sei $\varepsilon >0$. Für jedes $x\in X$ gibt esein $n(x)$ derart, dass
\[
\forall m \ge n(x): d(f_m(x),f(X)) < \frac\varepsilon3
\]
Weiter gibt es eine Umgebung $U_x$ von $x$, so dass für jedes $y\in U_x$ gilt: $d(f(y),f(x)) < \frac\varepsilon3$ und $d(f_{n(X)}(y), f_{n(x)}(x)) < \frac\varepsilon3$. Also gilt für jedes $m\ge n(x)$ und $y\in U_x$:
\begin{align*}
d(f_m(y), f(y)) &\le d(f_{n(x)}(y), f(y)) \\&\le d(f_{n(x)}(y), f_{n(x)}(x)) + d(f_{n(x)}(x), f(x)) +  d(f(x), f(y)) \\&< \varepsilon
\end{align*}
Aus $X=\bigcup_{x\in X}U_x$ folgt, dass es $x_1,\ldots,x_k\in X$ gibt mit $X=\bigcup_{i=1}^k U_{x_i}$. Sei $N=\max\{n(x_i)\mid 1\le i \le k\}$, also gilt $\forall y\in X, m\ge N: d(f_n(y),f(y))<\varepsilon$.\qed


\textbf{Anwendung:} Sei $X=[0,1]$, $Y=\mathbb R$. Für $n\in \mathbb N_0$ sei rekursiv die Funktion $w_k$ definiert durch $w_0=0$, $w_{n+1}(x) = w_n(x) + \frac12(x-w_n(x)^2)$. Das sind alles Polynome. Dann gilt: $w_n\to \sqrt\cdot$ gleichmäßig auf $[0,1]$:

Punkteweise Konvergenz: Sei $x\in [0,1]$. Für die Funktion $f:[0,\sqrt x] \to \mathbb R$, $w\mapsto w + \frac12 (x-w^2)$ gilt: $f(0) = \frac12x \le f(\sqrt x) = \sqrt x$. $f'(w) = 1-w\ge 0$, also ist $f$ monoton wachsend. $f(w) \in [0,\sqrt x]$, also $w\le f(w)$. Damit gilt: $w_0=0$, $w_{n+1} = f(w_n)$ ist monoton wachsend, also konvergiert $(w_n)$ gegen ein $y\in [0,1]$.

$f$ ist stetig, also $f(y) = \lim_{n\to\infty} f(w_n) = \lim_{k\to \infty}w_{n+1} = y$, also $f(y) = y + \frac12(x-y^2) = y \implies y ^2 = x$, also $y=\sqrt x$. Nach Satz von Dini gilt die Behauptung.

\begin{satz}{von Stone\footnote{Marshall Harvey Stone, 1903-1989}-Weierstra\ss\footnote{Karl Theodor Wilhelm Weierstra\ss, 1815-1897}}
Es sei $K$ ein kompakter topologischer Raum und ${\cal A}\subseteq {\cal C}(K)$
ein Teilring, der die konstanten Funktionen enthält und die Punkte von $X$ trennt, d.h.:
$$\forall x\neq y\in K:\exists f\in {\cal A}:f(x) \ne f(y).$$
Dann ist $\cal A$ dicht in ${\cal C}(K)$.
\end{satz}

%{\bf Das heißt:} Jede stetige Funktion auf $X$ lässt sich gleichmäßig
%durch eine Folge in ${\cal A}$ approximieren. 

%Die Bedingung an ${\cal A}$ hat einen Namen: man sagt, ${\cal A} $ 
%{\it trenne die Punkte} von $X.$
%Insbesondere impliziert dies, dass $K$ hausdorff'sch ist.

%Ein beliebter Spezialfall des
%Satzes ist der eines Kompaktums $X\subseteq \mathbb R^n$, wobei man dann für
%${\cal A}$ gerne den Ring der Polynomfunktionen (in $n$ Variablen) auf $X$ 
%wählt. Klar: Schon die linearen Abbildungen langen, um Punkte zu trennen.

{\it Beweis.}  Hier folge ich den Grundzügen der modernen Analysis
von Dieudonn\'e\footnote{Jean Alexandre Eug\`ene Dieudonn\'e, 1906-1992}.

%Wir bezeichnen mit $\bar{\mathcal A}$ den Abschluss von $\cal A$ in 
%${\cal C}(K).$


%\item[1.] Es gibt eine Folge von reellen Polynomen $u_n\in \mathbb R[X],$
%die auf dem Intervall $[0,1]$ gleichmäßig gegen die Wurzelfunktion 
%konvergiert.
%
%Um dies einzusehen setzen wir $u_1\equiv 0$ und definieren rekursiv
%$$u_{n+1} (t) \da u_n(t) + \frac12(t-u_n(t)^2), n\geq 1.$$
%Dann ist $(u_n)$ punktweise monoton steigend (auf $[0,1]$ wohlgemerkt, nur dort
%betrachten wir das) und beschränkt. Punktweise gilt also (wegen des 
%Monotoniekriteriums) $\lim_{n\to\infty} u_n(t) = \sqrt(t).$ Dann impliziert der 
%Satz von Dini, was in 1.\ behauptet wird.

Für jedes $f\in {\cal A}$ gehört $|f|$ zu $\bar{\mathcal A}$:

Denn: Sei $m=\max\{|f(x)|\mid x\in X\}$, $\frac fm\in \mathcal A$, genauso $\frac{f^2}{m^2} \in \mathcal A$. Verwende die Folge $(w_n)$ aus der Anwedung zum Satz von Dini und betrachte rekursiv: $f_0=0$, $f_{n+1}(x) = f_n(x) + \frac12(\frac {f^2}{m^2}(x) - f_n^2(x))$. $(f_n)$ konvergiert gleichmäßig gegen $\sqrt {\frac{f^2}{m^2}} = |\frac fm|$, also konverigiert $(mf_n)$ gleichmäßig gegen $|f|$.

Mit $f,g \in \mathcal A$ liegen auch die Funktionen $x\mapsto \max(f(x),g(x))$ und $x\mapsto \min(f(x),g(x))$ in $\mathcal A$, denn $\max(f,g)=\frac 12(f+g+|f-g|)$ beziehungsweise $\min(f,g) = \frac12 (f + g -|f-g|)$.


%\item Für $x\neq y\in K$ und $a,b\in \mathbb R$ gibt es $f\in {\cal A}$ 
%mit $f(x)=a, f(y) = b.$
%
%Denn: Es gibt ja nach Voraussetzung in ${\cal A}$ eine Funktion $g$ mit
%$g(x)\neq g(y).$ Setze nun 
%$$f\da \frac a{g(x)-g(y)}(g-g(y)) + \frac b{g(y)-g(x)}(g-g(x)).$$
%NB: $x,y$ sind fest, die Variable versteckt sich hinter dem nackten $g.$

Sei $f\in {\cal C}(K)$, und $\varepsilon >0$. Zeige: Es existiert ein $g\in \bar{\mathcal A}$ mit $d(f,g)<\varepsilon$.

Zunächst zeige: Es gibt für jedex $x\in X$ ein $g\in \bar{\mathcal A}$, so dass $g(x)=f(x)$ und für jedes $y\in X$ gilt: $g(y)<f(y)+\varepsilon$.

Sei $x\in X$ fest. Für $y\in X$ existiert dann eine Funktion $g_y\in \mathcal A$ mit $f(x) = g_y(x)$, $f(y) = g_y(y)$, denn wenn $x=y$, so wähle $g_y$ konstant gleich $f(x)$, und wenn $yx\ne y$, so gibt es ein $h$ mit $h(x)\ne h(y)$. Setzte für $z\in X$:
\[
g_y(z) \da \frac{h(z) -h(y)}{h(x)-h(y)}\cdot f(x) + \frac{h(z) - h(x)}{h(y) - h(x)}\cdot f(y)
\]
$g_y$ ist stetig, also gibt es eine Umgebung $U_y$ von $y$ mit $\forall y\in U_y: |f(z)-g(y)|<\varepsilon$.

$X = \bigcup_{y\in X}U_y$, also existierten $y_1,\ldots,y_k$ mit $X=\bigcup_{i=1}^k U_{y_i}$. Setze $g\da \min_{1\le i\le k}(g_{y_i})\in \bar{\mathcal A}$. $g(x) = f(X)$, da $g_y(x)=f(x)$ für jedes $y\in X$ gilt. Für jedes $x\in X$ gib es ein $i=1,\ldots,k$, so dass $z\in U_{y_i}$, also gilt $g(z) \le g_{y_i}(z)< f(z) + \varepsilon$, womit gezeigt ist, was zunächst zu zeigen war.

Um die Existenz von $g\in \bar{\mathcal A}$ mit $d(f,g)<\varepsilon$ einzusehen, wählen wir für jedes $x\in X$ eine Funktion $h_x\in \bar{\mathcal A}$ mit $h_x(x)=f(x)$ und $\forall y\in X: h_x(y) < f(y) + \varepsilon$.

Dies geht, denn für alle $x\in X$ gibt es eine Umgebung $V_x$ von $x$ mit $\forall y\in V_x$: $h_x(y) > f(y) -\varepsilon$, den $h_x - f$ ist stetig und hat bei $x$ den Funktionswert $0$.

$X=0\bigcup_{x\in X}V_x$. Wähle $x_1,\ldots,x_n\in x$ mit $X=\bigcup_{i=1}^r V_{x_i}$ und setzte $g\da \max_{1\le i \le r}(h_{x_i}) \in \bar{\mathcal A}$. Dann gilt $d(g,f)<\varepsilon$.
\qed

\begin{bsp}

\begin{enumerate}
\item Jede stetige Funktion $f$ auf $[a,b]$ lässt sich gleichmäßig durch Polynomfunktionen approximieren.

$\bar {\rm A}$llgemeiner: Für kompaktes $X\subseteq \mathbb R^n$ lässt sich jede stetige Funktion gleichmäßig durch Polynomfunktionen in $n$ Variablen approximieren.

\item Funktionen auf $\faktor{\mathbb R}{\mathbb Z}$. Die stetigen Abbildungen von $\faktor{\mathbb R}{\mathbb Z}$ nach $Y$ entsprechen den periodischen stetigen Abbildungen von $\mathbb R$ nach $Y$ mit Periode $1$. Für $n\in \mathbb N$ sind $x\mapsto \sin(2\pi n x)$ und $x\mapsto \cos(2\pi n x)$ in $\mathcal C(\faktor{\mathbb R}{\mathbb Z})$. Die Additionstheoreme für Sinus und Kosinus implizieren, dass $\mathcal A=\{x\mapsto \sum_{k=1}^\infty a_i \sin 2\pi kx + \sum_{l=0}^n b_l \sin2\pi lk\mid a_k, b_k \in \mathbb R\}$ ein Teilring von $\mathcal C(\faktor{\mathbb R}{\mathbb Z})$ ist. Insbesondere lässt sich jede stetige Funktion auf $\faktor{\mathbb R}{\mathbb Z}$ gleichmäßig durch Elemente aus $\mathcal A$ approximieren.

Sei $f\in \mathcal C(\faktor{\mathbb R}{\mathbb Z})$, $g\in \mathcal A$, $|f-g|<\varepsilon$. Dann ist $\int_0^1|f-g|^2 dx< \varepsilon^2$, also $d_2(f,g)=\sqrt{\int_0^1|f-g|^2 dx}<\varepsilon$. Das heißt: Zu jedem $\varepsilon >0$ gibt es ein $g\in \mathcal A$: $d(f,g)<\varepsilon$. Schreibe $g=\sum_{k=1}^n a_k\sin 2\pi kx + \sum_{l=0}^n k_l\cos 2\pi lx \in \mathcal A_n = \{\sum_{k=1}^n a_k\sin 2\pi kx + \sum_{l=0}^n k_l\cos 2\pi lx \mid n\text{ fest}\}$.

Auf $\mathcal C(\faktor{\mathbb R}{\mathbb Z})$ habne wir ein Skalarprodukt
\[
\langle p,q\rangle \da \int_0^1p(x)\cdot q(g) dx
\]
Aufgabe: Bestimme den Abstand $d_2(f,\mathcal A_n)$. Benutze dazu eine Orthonormalbasis in $\mathcal A_n$: $S_k(x) = \frac 1{\sqrt 2} \sin 2\pi k x$, $C_0(x) = 1$, $C_l(x) = \frac1{\sqrt 2} \cos 2\pi lx$, $l>0$. Dann minimiert
\[
g = \sum_{k=1}^n \langle f,S_k\rangle S_k + \sum_{l=0}^n \langle f,C_l\rangle C_l
\]
den Abstand $d_2(f,h)$ für $h\in \mathcal A_n$.


\item Der Satz von Stone-Weierstraß wird z.\,B. benutzt beim Beweis des Satzes von Peter und Weyl über Darstellungen von kompakten topologischen Gruppen.

Eng verwandt ist die Möglichkeit, kompakte Symmetriegruppen von Differentialgleichungen bei deren Lösung zu benutzten. Prominentestes Beispiel: Spektrum von Wasserstoffatomen.

\end{enumerate}
\end{bsp}

\begin{defini} {Wo ein Weg ist\dots}
Es sei $X$ ein topologischer Raum. Ein {\it Weg} ist eine stetige 
Abbildung eines kompakten reellen Intervalls $[a,b]\subseteq \mathbb R$ mit $a<b$ nach $X.$

Zwei Wege $\gamma:[a,b]\to X$, $\delta:[c,d] \to X$ heißen äquivalent, wenn es einen Homöomorphismus $\varphi:[a,b]\to [c,d]$ mit $\varphi(a) = c$, $\varphi(b)=d$ gibt mit $\gamma = \delta\circ \varphi$.

Sind $f:[a,b]\longrightarrow X$ und $g:[b,c]\longrightarrow X$ zwei Wege mit
$f(b) = g(b),$ so ist $g\ast f: [a,c]\longrightarrow X$ ein Weg, wenn wir
\[g\ast f(t) = \left\{ \begin{array}{ll}f(t), & t\in [a,b]\\
                                   g(t), & t\in [b,c]\\
\end{array}\right.\]
definieren. 

Für einen Weg $\gamma:[a,b] \to X$ ist ${\rm Bild}(\gamma)$ zusammenhängend.

Wir betrachten auf $X$ die Äquivalenzrelation $\sim$, die gegeben ist durch 
\[ x\sim y \iff \exists \gamma : [a,b] \to X: \gamma(a) =x,\  \gamma(b)=y\,. \]
Diese ist offensichtlich symmetrisch und reflexiv. Sie ist transitiv, da  wenn $\gamma:[a,b] \to X$, $\gamma(a)=x$, $\gamma(b)=y$ ein Weg ist und $\delta:[b,d]\to X$, $\delta(b)=y$, $\delta(d)=z$ auch, dann ist auch $g \ast f$ ein Weg.

Die Äquivalenzklassen von $\sim$ sind alle zusammenhängend, denn wäre $[x]_\sim$ nicht zusammenhängend, dann gäbe es zwei offene Mengen $U,V\in X$ mit $[x]_\sim$ mit $[x]_\sim = ([x]_\sim \cap U) \cup ([x]_\sim \cap V)$, $[x]_\sim \cap U\ne \emptyset$, $[x]_\sim \cap V\ne \emptyset$ $[x]_\sim \cap U\cap V = \emptyset$. Sei ohne Beschränkung der Allgemeinheit sei $x\in U$. Zu $y\in V\cap [x]_\sim$, gibt es einen Weg $\gamma:[0,1]\to X$ mit $\gamma(0)=x$, $\gamma(1)=y$. Offensichtlich ist ${\rm Bild}(\gamma)\subseteq [x]_\sim$, aber $({\rm Bild}(\gamma) \cap U) \cap ({\rm Bild}(\gamma) \cap V) = \emptyset$, im Widerspruch zu


Die Äquivalenzklassen heißen die {\it Wegzusammenhangskomponenten}\index{Wegzusammenhangskomponenten} von $X$. Ein wegzusammenhängender Raum, also einer mit nur einer Wegzusammenhangskomponente, ist zusammenhängend. (Beispiel: $SO(n)$)

Ein Beispiel für einen zusammenhängenden Raum, der nicht wegzusammenhängend ist $\mathbb Z$ mit der koendlichen Topologie.

{\bf Vorsicht:} Es gibt einen Weg $P:[0,1] \to [0,1]\times [0,1]$, der surjektiv ist.\footnote{z.\,B. die Peano-Kurve, siehe \url{http://de.wikipedia.org/wiki/Peano-Kurve}.}
\end{defini}

\begin{defini} {Offenheit}
Eine Abbildung $f:X\longrightarrow Y$ zwischen zwei topologischen 
Räumen heißt {\it offen}, wenn für jede offene Teilmenge $A\subseteq X$
das Bild $f(A)$ in $Y$ offen ist.

$f$ heißt {\it offen in } $x\in X,$ falls jede Umgebung von $x$ unter $f$ 
auf eine Umgebung von $f(x)$ abgebildet wird. Es genügt, dies für eine Umgebungsbasis zu zeigen. Beispielsweise in $\mathbb R$ ist $f;\mathbb R\to \mathbb R$ offen in $x\in \mathbb R$, wenn $\forall \varepsilon >0 \ \exists \delta >0: f( (x-\varepsilon, x+\varepsilon) ) \supseteq (f(x)-\delta, f(x)+\delta)$.

$f$ ist genau dann offen, wenn $f$ in jedem $x\in X$ offen ist.

\end{defini}

\begin{bsp}{Offenheit}
\label{kte Wurzel}
\begin{enumerate}
\item 
Ein Homöomorphismus ist also eine stetige und offene Bijektion.
Sei nämlich $f:X\to Y$ ein Homöomorphismus, also sind $f, f^{-1}$ stetig. Für $U\subseteq X$ ist $f(U) = (f^{-1}){-1}(U)$. Das ist offen für ein offenes $U$.

\item Es sei $U\subseteq \mathbb R^n$ offen und $\varphi : U \to \mathbb R^n$ stetig differenzierbar. Wenn in $x\in U$ die Ableitung $D_{\varphi}(x)$ invertierbar ist, dann ist $\varphi$ offen im Punkt $x$.

Nach dem Satz der impliziten Funktion gibt es $U_o\ni x$, $V_0\ni f(x)$ mit $\varphi|_{U_0} \to V_0$ bijektiv und $\varphi|_{U_0}^{-1}$ stetig differenzierbar. Also ist $\varphi|_{U_0}:U_o\to V_0$ ein Homöomorphismus, also ist $\varphi$ offen in $x$.

\item Für eine natürliche Zahl $k$ ist die Abbildung $\mathbb C\ni z \to z^k$ offen.

Dazu schreiben wir $z=r e^{i\alpha}$, $z\ne 0$, also ist $z^k = r^k e^{ik\alpha}$.

Die Abbildung ist offen in Punkt $z_0 = r_0e^{i\alpha_0}$: Sei $U$ eine offene Umgebung von $z_0$. In $U$ liegt eine Umgebung von $z_0$ von der Gestalt 
\[V=\{re^{i\alpha} \mid r\in (r_0-\delta,r_0+\delta), \alpha\in[\alpha_0-\varepsilon,\alpha_0+\varepsilon]\} \]
für geeignete $\varepsilon,\delta>0$, $\delta < r_0$. Das Bild von $V$ ist
\begin{align*}
&\{r^ke^{ik\alpha} \mid r\in (r_0-\delta,r_0+\delta), \alpha\in[\alpha_0-\varepsilon,\alpha_0+\varepsilon]\} \\
=\ &\{\rho e^{i\beta} \mid \rho\in((r_0-\delta)^k,(r_0+\delta)^k)),\ \beta \in (k(\alpha_0-\varepsilon), k(\alpha_0+\varepsilon))\} 
\end{align*}
Dies ist eine offene Umgebung von $z_0^k$.

Für $z_0=0$ gilt: $\{z^k\mid z\in B_r(0)\} = B_{r^k}(0)$, also ist die Abbildung auch im Nullpunkt offen.

Am Argument für $z_0\ne 0$ sieht man: $z\mapsto z^k$ ist in einer Umgebung von $z_0\ne 0$ injektiv und die Umkehrabbildung $z\mapsto z^{\frac 1k}$ ist stetig und offen.
\end{enumerate}
\end{bsp}

\begin{hilfs}{komplexe Polynome} \label{Polynom-offen}

Es sei $f:\mathbb C\longrightarrow \mathbb C$ eine polynomiale Abbildung,
die nicht konstant ist. Dann ist $f$ offen.
\end{hilfs}

{\it Beweis.}
Wir zeigen, dass das Bild einer Umgebung der $0$ unter $f$ eine Umgebung von 
$f(0)$ ist. Da Translationen in $\mathbb C$ Homöomorphismen sind und aus 
Polynomen wieder Polynome machen, zeigt das, dass für jedes $z\in \mathbb C$
und jede Umgebung $U$ von $z$ die Menge $f(U)$ eine Umgebung von $f(z)$ ist,
und das ist gerade die Behauptung.

Hierbei dürfen wir uns auf den Fall zurückziehen, dass $f(0)=0$ gilt.

Es sei also $f(z) = \sum_{i=1}^d a_iz^i.$

Wir bemerken zunächst, dass $f$ im Nullpunkt reell differenzierbar ist. 
Die Ableitung im Nullpunkt ist die $\mathbb R$-lineare Abbildung, die durch 
Multiplikation mit $a_1$ zustande kommt. Wegen der 
binomischen Formeln gilt hier ja
$$\lim_{|h|\to 0} \frac{f(z_0+h) - f(z_0) - a_1h }{|h|} = 0.$$

Wenn $a_1\neq 0$ gilt, dann ist die Ableitung ein Isomorphismus, und der 
Satz von der impliziten Funktion sagt, dass
es eine Umgebung $U$ von $0$ und eine Umgebung $V$ von $f(0)=0$ gibt, so dass
$f$ auf $U$ injektiv ist, $f(U) = V,$ und die lokale Umkehrabbildung zu $f$
auf $V$ differenzierbar. Das heißt, dass auch $f^{-1}$ in $0$ stetig ist, 
$f$ also offen.

Es bleibt der Fall $a_1=0.$ Es sei $k=\min\{n\in \mathbb N \mid a_k\neq 0\}.$
Dann ist $k>1,$ da $a_0=a_1=0.$ Wir wollen zeigen, dass $f$ in einer Umgebung
der $0$ eine $k$-te Wurzel hat: $f(z) = g(z)^k,$ und dass $g$ offen gewählt 
werden kann. Dann sagt uns die Offenheit von $z\mapsto z^k,$ dass auch $f$ im
Nullpunkt offen ist, und wir sind fertig.

Dazu schreiben wir $f(z) = z^k\cdot h(z)$ mit 
$h(z) = \sum_{i=k}^d a_i z^{i-k}.$
Das Polynom $\tilde h$ hat also im Nullpunkt den Wert $a_k\neq 0.$
In einer Umgebung von $a_k$ gibt es wegen \ref{kte Wurzel} eine stetige, offene 
$k$-te Wurzel. Die $k$-te Wurzel $h(z)^{1/k}$ ist also in einer Umgebung der $0$
definiert, und bei näherem Hinsehen sieht man, dass die Ableitung im 
Nullpunkt regulär ist.

Daher gilt in einer Umgebung der $0:$ 
$$f(z) = z^k (h(z)^{1/k})^k = (z h(z)^{1/k})^k.$$ 
Das ist die $k$-te Potenz einer bei $0$ offenen Abbildung, und damit ist $f$ 
selbst im Ursprung offen.
\qed

\begin{hilfs}{\`a la Liouville}\label{Liouville}

Es seien $f:X\longrightarrow Y$ eine stetige und offene Abbildung, $X$ sei 
nichtleer und kompakt, $Y$ sei zusammenhängend und hausdorff'sch.

Dann ist $f$ surjektiv und insbesondere ist $Y$ auch kompakt.

\end{hilfs}

{\it Beweis.} Das Bild von $f$ ist offen nach Definition der Offenheit und
kompakt wegen \ref{Erhaltungssatz}. Als Kompaktum in $Y$ ist $f(X)$ 
abgeschlossen, siehe \ref{fuer Liouville}. Es ist mithin 
$Y= f(X) \cup (Y\setminus f(X))$ eine Zerlegung von $Y$ als Vereinigung
zweier offener disjunkter Teilmengen. Da $f(X)$ nicht leer ist und $Y$
zusammenhängend ist, muss $Y\setminus f(X)$ leer sein: $f$ ist 
surjektiv. \qed

\paragraph{Bemerkung:} Dies ist ein topologisches Pendant zum Satz von Liouville aus der Funktionentheorie. Dieser besagt: Eine beschränkte holomorphe Funktion $f:\mathbb C\to\mathbb C$ ist konstant. Das folgt aus \ref{Liouville} unter Benutzung des Satzes von der Gebietstreue (nicht konstante holomorphe Funktionen sind offen) und des Riemann’schen Hebbarkeitssatzes: Sei $g(z)\da f(\frac 1z)$, $z\ne 0$ ist beschärnkt, also lässt sich $g$ ei $z=0$ holomorph fortsetzbar. $f$ lässt sich zu einer holomorphen Abbildung $\hat f :\mathbb {\hat C}\to \mathbb C$ fortsetzen.

\begin{satz}{Fundamentalsatz der Algebra}
 \index{Fundamentalsatz der Algebra}\label{Fundamentalsatz} 

Es sei $f:\mathbb C\longrightarrow \mathbb C$ ein nichtkonstantes Polynom. 
Dann besitzt $f$ eine komplexe Nullstelle.

\end{satz}

{\it Beweis.} Nach Hilfssatz \ref{Polynom-offen} ist $f$ offen. Außerdem 
gilt (siehe \ref{Lasso}), dass $|f(z)|$ mit $|z|$ gegen unendlich geht. 

Wir können demnach $f$ zu einer stetigen Abbildung  $\hat f$ von 
$\mathbb P^1(\mathbb C)$ auf sich selbst, mit $\hat f(\infty) = \infty$ und $\hat f(z)=f(z), z\in \mathbb C$, fortsetzen, und man verifiziert, dass
auch die Fortsetzung offen ist. Also ist die Fortsetzung von $f$ surjektiv nach
Liouville, und es gibt ein $z\in\mathbb P^1(\mathbb C)$ mit $f(z) = 0.$
Da $z$ nicht $\infty$ sein kann (hier wird $f$ ja unendlich) ist $z\in 
\mathbb C$ wie behauptet. 
\qed

\section{Topologische Mannigfaltigkeiten}

\begin{defini}{Atlas}
Es sei $X$ ein topologischer Raum. Ein $n$-dimensionaler 
\index{Atlas}{\it Atlas} auf $X$ 
besteht aus einer offenen Überdeckung $\ddot U$ von $X,$ so dass für jedes 
$U\in \ddot U$ ein Homöomorphismus 
\[\varphi_U: U \longrightarrow Z(U)\subseteq \mathbb R^n\]
existiert, wobei $Z(U)$ in $\mathbb R^n$ offen ist.

Jeder Atlas liegt in einem {\it maximalen Atlas}
\[
\{ (U,\varphi) \mid U \subseteq X \text{ offen, } \varphi : U \to Z\subset \mathbb R^n\text{ ein Homöomorphismus} \}\,.
\]
\end{defini}

Zum Beispiel besitzt jede offene Teilmenge $U$ des $\mathbb R^n$ einen Atlas; 
wir nehmen einfach $U$ selbst als Überdeckung und die Identität als
Kartenabbildung. 

\begin{defini} {topologische Mannigfaltigkeit}
Ein topologischer Raum $X$ ist eine $n$-dimensionale {\it topologische
Mannigfaltigkeit}\index{topologische Mannigfaltigkeit}, wenn er hausdorff'sch
ist, mit einem Atlas
ausgerüstet werden kann und eine abzählbare Basis der Topologie besitzt.
\end{defini}


\begin{bem}{Eindeutigkeitsunterstellung und Abzählbarkeitsaxiome}
\begin{enumerate}[a)]
\item 
{\bf Vorsicht:} Wir halten im Vorübergehen fest, dass es nicht {\it a
priori} klar ist, dass die Dimension eines Atlas durch die Topologie auf $X$
festliegt. Das ist so, aber der Beweis ist nicht so offensichtlich.
Schließlich muss man so etwas zeigen, wie dass es für $m\neq n$ keine 
offene stetige Abbildung einer $m$-dimensionalen Kugel in eine 
$n$-dimensionale gibt.

\item
Die letzte Bedingung (abzählbare Basis) ermöglicht einige Konstruktionen mit topologischen 
Mannigfaltigkeiten, die sich als sehr hilfreich erweisen. 
Sie impliziert zum Beispiel, dass jede offene Überdeckung von $X$ eine
abzählbare Teilüberdeckung hat. 

Man nennt sie auch das {\it zweite Abzählbarkeitsaxiom}.


Der Name schreit nach einem Vorgänger: ein topologischer Raum erfüllt das 
{\it erste Abzählbarkeitsaxiom}, wenn jeder Punkt eine abzählbare 
Umgebungsbasis besitzt. Metrische Räume haben diese Eigenschaft zum Beispiel,
sie ist eine {\bf lokale} Bedingung, sagt sie doch nur etwas über Umgebungen 
von einem jeden Punkt aus. Das werden wir im nächsten Hilfssatz einmal
austesten.

Das zweite Abzählbarkeitsaxiom impliziert offensichtlich das erste.

\item Ein topologischer Raum, in dem das erste, aber nicht das zweite Abzählbarkeitsaxiom erfüllt ist, ist $\mathbb R$ mit der diskreten Topologie, denn für $x\in \mathbb R$ ist $\{\{x\}\}$ eine Umgebungsbasis bei $x$. Für eine Basis $\mathcal B$ der diskreten Topologie jedoch muss es für $x\in \mathbb R$ auch eine Teilmenge $U$ mit $x\in U\subseteq\{x\}$ geben, also $\{\{x\}\mid x\in \mathbb R\}\subseteq \mathcal B$ und damit $\mathcal B$ nicht abzählbar.

\item Ein metrischer Raum $(X,d)$ erfüllt immer das erste Abzählbarkeitsaxiom, denn für $x\in X$ ist $\{B_{\frac1n}(x)\mid n\in \mathbb N_{>0}\}$ eine Umgebungsbasis von $x$. Ist nämlich $x\in U$, $U$ offen, so existiert ein $\varepsilon >0$ so dass $B_\varepsilon(x)\subseteq U$. Wähle $n\in \mathbb N$, so dass $n>\frac1\varepsilon$, dann ist $B_{\frac1n}(x)\subseteq B_\varepsilon(x)\subseteq U$.

\item Wenn $x\in X$ eine abzählbare Umgebungsbasis besitzt, so zähle sie ab:
\[
B(x) = \{U_1,U_2,\ldots\}
\]
durch Schneiden erhält man dann die Umgebungsbasis
\[
\tilde B(x) = \{U_1,U_1\cap U_2,\ldots,\bigcap_{i=1}^kU_i \mid k\in\mathbb N\}
\]
die aus immer kleiner werdenden Umgebungen von $x$ besteht.


\end{enumerate}
\end{bem}

\begin{defini} {schon wieder Folgen}
Eine Folge $(x_n)$ in einem topologischen Raum $X$ {\it konvergiert gegen}
$x\in X,$ falls in jeder Umgebung von $x$ alle bis auf endlich viele 
Folgenglieder liegen.

{\bf Vorsicht:} \underline{Der} Grenzwert ist im Allgemeinen nicht mehr 
eindeutig, also eigentlich der bestimmter Singular verboten. Für die
Eindeutigkeit des Grenzwerts braucht man ein Trenungsaxiom, zum Beispiel ist 
hausdorff'sch hinreichend.

$X$ heißt {\it folgenkompakt}, wenn
jede Folge in $X$ eine konvergente Teilfolge besitzt.

\end{defini}

\begin{hilfs} {Folgen für die Folgenkompaktheit}
Es sei $X$ ein topologischer Raum.
\begin{itemize}
\item[a)] Ist $X$ kompakt und erfüllt das erste Abzählbarkeitsaxiom, so
ist $X$ folgenkompakt.
\item[b)] Ist $X$ folgenkompakt und metrisch, so ist $X$ auch kompakt.
\end{itemize}
\end{hilfs}

{\it Beweis.} 
\begin{itemize}
\item[a)] Es sei $(x_n)$ eine Folge in $X.$ Dann gibt es ein $x\in X$, so dass
in jeder Umgebung $U$ von $x$ für unendlich viele $n\in \mathbb N$ der Punkt
$x_n$ liegt. 

Anderenfalls ließe sich für alle $x\in X$ eine Umgebung $U_x$ finden, die
nur endlich viele Folgenglieder enthält, und weil 
$$X=\bigcup_{x\in X} U_x$$
eine endliche Teilüberdeckung hat, hätte man einen Widerspruch.

Nun haben wir so ein $x.$ Dieses besitzt eine abzählbare Umgebungsbasis
$$U_1\supseteq U_2 \supseteq U_3,\dots$$
und wir können bequem eine Teilfolge $x_{n_k}$ wählen mit 
$$\forall k\in \mathbb N:n_{k+1} > n_{k} \ \ {\rm und }\ \ x_{n_k}\in U_k.$$
\item[b)] Es sei $\ddot U$ eine offene Überdeckung des folgenkompakten 
metrischen Raums $X.$ 

Für jedes $x\in X$ wählen wir ein $U_x\in \ddot U$ derart, dass eine der
beiden folgenden Bedingungen erfüllt ist:
$$B_1(x)\subseteq U_x\ \ {\rm  oder }\ \ 
\exists r(x)>0 : B_{r(x)}(x)\subseteq U_x, \forall U\in \ddot U: 
B_{2r(x)}(x)\not\subseteq U.$$
Jetzt nehmen wir an, dass $\ddot U$ keine endliche Teilüberdeckung besitze.
Wir starten mit einem beliebigen $x_1\in X$ und wählen 
$$x_2\in X\setminus U_{x_1},\ 
x_3\in X\setminus(U_{x_1}\cup U_{x_2}),\dots$$
Da $X$ folgenkompakt ist, gibt es eine Teilfolge $(x_{n_k})$, die gegen ein
$a\in X$ konvergiert. Wir wählen ein $r\in (0,1)$ derart, dass
$B_r(a)\subseteq U_a.$ Dann liegt $x_{n_k}$  für 
großes $k$ in $B_{r/5}(a),$ und es gilt 
$$d(x_{n_k},x_{n_{k+1}}) < 2r/5.$$
Andererseits zeigt
$$x_{n_k}\in B_{4r/5}(x_{n_k})\subseteq B_{r}(a)\subseteq U_a\in \ddot U,$$
dass $r(x_{n_k})\geq 2r/5,$ und damit auch 
$$d(x_{n_k},x_{n_{k+1}}) \geq 2r/5.$$
Dieser Widerspruch besiegelt das Schicksal unserer irrigen Annahme, $\ddot U$
habe keine endliche Teilüberdeckung. 

Also ist $X$ kompakt, da $\ddot U$ beliebig war.\qed
\end{itemize}

\begin{bsp}{Schönheiten des Abendlandes}
Nach diesem Grundlagenexkurs kehren wir nun zu den topologischen 
Mannigfaltigkeiten zurück. Wir kennen noch keine Beispiele. Oder doch?

\begin{enumerate}[a)]
\item Jede offene, nichtleere Teilmenge von $\mathbb R^n$ ist eine
$n$-dimensionale topologische Mannigfaltigkeit. Hier muss man vor allem das
zweite Abzählbarkeitsaxiom testen: 

Es gibt eine abzählbare Subbasis der Topologie, z.\,B.
\[
B \da \{B_r(x) \mid x \in \mathbb Q^n\cap U, r\in \mathbb Q, r>0, B_r(x) \subseteq U\}
\]
Ist nämlich $V\subseteq U$ offen und $x\in V$, so gibt es ein $r>0$ so dass $B_r(x)\subseteq V$. Da $\mathbb Q^n$ dicht in $\mathbb R^n$ liegt, gibt es ein $q\in \mathbb Q^n$ mit $d(q,x)<\frac r4$. Wähle $t\in \mathbb Q$ mit $\frac r4<t<\frac r2$, dann ist $x\in B_t(q)\subseteq B_r(x)$. Also liegt jedes $x\in V$ in einem $B_x \in \mathcal B$, so dass $B_x \subseteq V$, also ist $V= \bigcup_{x\in V}B_x$.

\item
Jeder Hausdorffraum mit einem endlichen Atlas ist dann auch eine topologische 
Mannigfaltigkeit.

\item Es sei $K=\mathbb R$ oder $\mathbb C.$ Dann ist der projektive Raum
$\mathbb P^n(K)$ mit der früher eingeführten Quotiententopologie eine 
topologische Mannigfaltigkeit.

Denn er lässt sich überdecken durch die offenen Mengen 
$$U_k\da\{(x_i)_{1\leq i\leq n+1} \mid x_k = 1\},\ 1\leq k\leq n+1,$$
und diese werden beim Quotientenbilden mit ihrem Bild auch topologisch 
identifiziert, liefern also einen endlichen Atlas von $\mathbb P^n(K).$

\item Das Raum-Zeit-Kontinuum ist eine vierdimensionale (kompakte?) Mannigfaltigkeit.

\item Keine Mannigfaltigkeit ist der folgende Raum: Ausgehend von $Y=\mathbb R \times \{0,1\} \subseteq \mathbb R^2$ mit der Äquivalenzrelation 
\[
(y_1,t) \sim (y_2,s) \iff y_1 = y_2 \ne 0
\]
betrachten wir $X\da \faktor Y \sim$. $X$ sieht abgesehen vom Nullpunkt aus wie $\mathbb R$, nur dass er „zwei Nullpunkte“ hat.

Die Abbildungen
\begin{align*}
\mathbb R\ni x \mapsto [(x,1)] \in X
&&
\mathbb R\ni x \mapsto [(x,0)] \in X
\end{align*}
sind Homöomorphismen von $\mathbb R$ auf einen offenen Teil von $X$, also bilden die Umkehrabbildungen einen Atlas auf $X$.

Aber: $X$ ist nicht hausdorff’sch!. Eine offene Umgebung von $[(0,0)]$ enthält $[(x,0)]$ für $|x|<\delta$ für ein $\delta > 0$. Ebenso enthält eine offene Umgebung von $[(0,1)]$ die Punkte $[(x,0)]$ für $|x|<\varepsilon$ für ein $\varepsilon > 0$. Für $x\ne 0$ ist aber $[(x,1)]= [(x,0)]$, also sind diese beiden offenen Umgebungen nicht disjunkt.

\item Keine topologische Mannigfaltigkeit ist zum Beispiel der folgende
Raum, obwohl er einen endlichen Atlas hat: Wir nehmen die Einheitssphäre 
$S^1\subseteq \mathbb C$ und definieren $X=S^1/\simeq,$
wobei die Äquivalenzrelation $\simeq$ durch $x\simeq -x$ für $x\neq 
\pm1$ definiert ist. 

Ein offener Halbkreis wird hierbei injektiv nach $X$ abgebildet, und wir 
erhalten einen schönen Atlas, von dem sogar zwei Karten genügen. Aber $X$ 
ist nicht hausdorff'sch, weil die Klassen von $\pm 1$ sich nicht durch offene
Umgebungen trennen lassen.
\end{enumerate}
\end{bsp}

\begin{hilfs}{$M$ ist regul\"ar}

Es sei $M$ eine $n$-dimensionale Mannigfaltigkeit und $x\in M$
ein Punkt.

%Dann gibt es zwei disjunkte offene Teilmengen $U_A, U_x$ in $M$, sodass
%$$x\in U_x,\ \ A\subseteq U_A.$$

Dann enth\"alt jede offene Umgebung von $x$ den Abschluss einer offenen 
Umgebung von $x.$

\end{hilfs}

{\it Beweis.} Es sei $U$ eine offene Umgebung von $x.$ Wir d\"urfen annehmen, 
dass $U$ der Definitionbereich einer Karte aus dem Atlas von $M$ ist.
Es sei $\varphi_U:U\longrightarrow Z(U)\subseteq \mathbb R^n$ die zugeh\"orige
Karte und ohne Einschr\"ankung $\varphi_U(x) = 0.$

Weiter sei $r>0$ gew\"ahlt, sodass $B_r(0)\subseteq Z(U)$ gilt: diese Menge ist
ja offen. Dann liegt der Abschluss $A\da\overline{B_{r/2}(0)}$ in $Z(U)$ und 
dies ist der Abschluss der Umgebung $B_{r/2}(0)$ von $0$. 

Das zeigt, dass $\varphi_U^{-1}(A)\subseteq U$ der Abschluss in $U$ von einer
Umgebung von $x$ ist. Wir m\"ussen noch \"uberlegen, dass $\varphi_U^{-1}(A)$
tats\"achlich auch in $M$ abgeschlossen ist. Aber das liegt daran, dass
$\varphi_U^{-1} : Z(U) \longrightarrow U\subseteq M$ stetig ist, und daher 
den kompakten Abschluss $\overline{B_{r/2}(0)}$ auf ein kompaktes und daher
abgeschlossenes Bild (wegen \ref{fuer Liouville}) abbildet.\qed

\begin{defini} {regul\"ar, normal}
Es sei $X$ ein topologischer Raum, in dem die Punkte (d.h.: die 
einelementigen Teilmengen) abgeschlossen sind (ein sogenannter $T_1$-Raum also).

\begin{enumerate}[a)]
\item
Dann heißt $X$ {\it regul\"ar}\index{regul\"ar}, falls f\"ur jeden Punkt
$x\in X$ jede Umgebung $U$ von $x$ den Abschluss einer offenen Umgebung von $x$
enth\"alt. 

Das ist \"aquivalent dazu, dass f\"ur jeden Punkt $x\in X$ und jede 
abgeschlossene Teilmenge $A\subseteq X$, $x\notin A$, offene Mengen $U,V$ existieren mit
$$x\in U,\ A\subseteq V,\ U\cap V = \emptyset.$$

Wir haben also gerade gezeigt, dass eine Mannigfaltigkeit regul\"ar
ist.

\item
$X$ heißt {\it normal}\index{normal}, falls es f\"ur je zwei disjunkte
abgeschlossene Mengen $A,B$ in $X$ disjunkte offene Mengen $U,V$ gibt mit
$A\subseteq U$, $B\subseteq V.$ Man sagt auch: $A$ und $B$ haben disjunkte 
offene Umgebungen.
\end{enumerate}
\end{defini}
Es ist klar, dass normal regul\"ar impliziert (denn Punkte sind abgeschlossen),
und dass regul\"ar hausdorffsch impliziert (dito).

Als n\"achstes wollen wir sehen, dass Mannigfaltigkeiten auch normal sind, und 
zeigen sogar:

\begin{hilfs} {Mannigfaltigkeiten sind normal}
Es sei $X$ ein regul\"arer topologischer Raum, der das zweite 
Abz\"ahlbarkeitsaxiom (abszählbare Basis der Topologie) erf\"ullt.

Dann ist $X$ normal.
\end{hilfs}

{\it Beweis.}
Es seien $A,B$ zwei disjunkte abgeschlossene Teilmengen von $X.$ Aufgrund der
Regularit\"at gibt es f\"ur jedes $a\in A$ eine Umgebung $U_a$, deren 
Abschluss zu $B$ disjunkt ist. Wir \"uberdecken $A$ mit diesen $U_a.$ Da das
zweite Abz\"ahlbarkeitsaxiom erf\"ullt ist, gibt es eine abz\"ahlbare 
\"Uberdeckung $\ddot U$ von $A,$ sodass alle $U\in\ddot U$ einen zu $B$ 
disjunkten Abschluss haben. Dasselbe k\"onnen wir auch f\"ur $B$ machen: es 
gibt eine abz\"ahlbare offene \"Uberdeckung $\ddot V$ von $B,$ sodass alle
$V\in \ddot V$ einen zu $A$ disjunkten Abschluss haben.

Nun w\"ahlen wir eine Abz\"ahlung von $\ddot U$ und von $\ddot V:$
$$\ddot U = \{U_1,U_2,U_3,\dots\},\ \ \ddot V=\{V_1,V_2,V_3,\dots\}.$$
Nun definieren wir f\"ur alle $n\in \mathbb N:$
$$\widetilde U_n \da U_n\setminus\bigcup_{i=1}^n \overline{V_i}\quad\text{und}\quad
\widetilde V_n \da V_n\setminus\bigcup_{i=1}^n \overline{U_i}.$$
Diese Mengen sind alle offen, und wir entnehmen nur Punkte, die nicht zu $A$ 
beziehungsweise $B$ geh\"oren. Demnach sind
$$A\subseteq \bigcup_{n\in \mathbb N}\widetilde{U_n}\quad\text{und}\quad 
B\subseteq \bigcup_{n\in \mathbb N}\widetilde{V_n}$$
offene und offensichtlich disjunkte Umgebungen von $A$ und $B$. 
\qed

\begin{bem}{Andere Strukturen}
\begin{enumerate}[a)]
\item Es sei $M$ eine topologische Mannigfaltigkeit. Wenn zwei Karten
$\varphi_U : U\longrightarrow Z(U), \ \varphi_V:V\longrightarrow Z(V)$
auf offenen Mengen mit nichtleerem Schnitt gegeben sind, dann liefert das
nat\"urlich insbesondere einen Hom\"oomorphismus
$$\psi_{U,V} : \varphi_U(U\cap V) \longrightarrow \varphi_V(U\cap V),$$
indem wir erst mit $\varphi_U^{-1}$ zur\"uckgehen und dann mit $\varphi_V$ 
absteigen.

Diese Abbildungen $\psi_{U,V}$ heißen die {\it Kartenwechsel} des Atlanten. 

\item Wenn wir von den Kartenwechseln des Atlanten verlangen, dass sie 
differenzierbar sind, so k\"onnen wir unter R\"uckgriff auf den Atlas 
definieren, wann eine reellwertige Funktion $f$ auf $M$ differenzierbar ist. 
Das ist sie nämlich genau dann, wenn f\"ur alle Karten gilt, dass
$$f\circ \varphi_U^{-1}$$ auf $Z(U)$ differenzierbar ist. Dies ist dann eine
konsistente Bedingung, wenn die Kartenwechsel differenzierbar sind, und die
Differenzierbarkeit in einem Punkt $x\in M$ kann durch Blick auf eine einzige
Karte getestet werden.

Wenn die Kartenwechsel $d$ differenzierbar sind, so kann man auch sagen, wann
eine Funktion $d$ mal differenzierbar ist. Und hier kann $d$ auch $\infty$ 
sein.

So kommt man zum Begriff der Differenzierbaren Mannigfaltigkeit, dem 
Hauptgegenstand der Differentialtopologie.

\item In $\mathbb R^n$ ist die Länge einer (stückweise) stetig differenzierbaren Kurve $\gamma:[0,1]\to \mathbb R^n$ das Integral $\int_0^1 |\gamma'(0)|dt$.

Sei nun $M$ eine differenzierbare Mannigfaltigkeit, $\gamma:[0,1]\to M$ ein Weg. $\gamma$ hießt differenzierbar in $t\in (0,1)$, falls es ein $\varepsilon>0$ gibt mit $\gamma(t-\varepsilon, t+\varepsilon)\subseteq U$, $U$ eine Kartenumgebung von $\gamma(t)$, so dass $\varphi_U\circ\gamma|_{(t-\varepsilon, t+\varepsilon)}:(t-\varepsilon,t+\varepsilon) \to Z(U)\subseteq \mathbb R^n$ in $t$ differenzierbar ist. Das ist unabhängig von der gewählten Umgebung von $\gamma(t)$.

Was ist nun die Länge der Kurve $\gamma$? Wähle dazu $0=t_0< t_1<\cdots t_n=1$, so dass $\gamma([t_1,t_{i+1}])\subseteq U_i$ wobei $(U_i,\varphi_{U_i})$ zum Atlas gehört. Ein Ansatz für die Länge der Kurve ist
\[
L(\gamma) = \sum_{i=1}^n L(\varphi_{U_i} \circ \gamma|_{[t_{i-1},t-i]})\,.
\]
Ist dies abhängig von der gewähnten Zerlegung in $t_i$? Als Ausweg fordern wir, dass der Atlas Riemann’sch ist, also alle Kartenwechsel $\psi_{U,V} : \varphi_U(U\cap V) \to \varphi_V(U\cap V)$ längenerhaltend sind ($\forall x\in \varphi_U(U\cap V): D\psi_{U,V}(X)\in O(n)$). Wenn der Atlas Riemann’sch ist, dann hängt die Länge $L(\gamma)$ nicht von den $t_i$ und $U_i$ ab (was hier nicht bewiesen werden soll).

Wenn dann die Mannigfaltigkeit zusammenhängend ist, ist sie wegzusammenhängend und zwischen je zwei Punkten gibt es stückweise differenzierbare Wege. Durch $d(p,q) \da \inf_{\gamma\in\Omega_{pq}}(L(\gamma))$ mit $\Omega_{pq}\da\{\gamma:[0,1]\to M$, $\gamma(0)=p$, $\gamma(1)=q$ und $\gamma$ stückweise stetig differenzierbar$\}$ definiert eine Metrik auf $M$. Damit versehen nennt man $M$ eine Riemann'sche\footnote{Bernhard Riemann, 1826-1866}
Mannigfaltigkeit. Dies ist der Hauptgegenstand der Riemann'schen Geometrie

\item Es sei $M$ eine differenzierbare Mannigfaltigkeit und $f:M\to \mathbb R$ differenzierbare. Es ist nicht klar, was dann die Ableitung von $f$ ist, denn der Funktionswert der Ableitung 
h\"angt von der benutzten Karte ab:

{\bf Beispiel:} $M=\mathbb R$ ist mit den beiden Karten $id:M \to \mathbb R$, $m:M\to \mathbb R$, $m(x)=2\cdot x$, eine differenzierbare Mannigfaltigkeit. Sei $f:M\to \mathbb R$, $f(x)=x$. $f$ ist differenzierbar und $f\circ id^{-1} = id$ von $f\circ m^{-1}=[x\mapsto \frac12 x]$ verschieden, also hängt der Wert von $f'$ an der Stelle $o\in M$ von der Wahl der Karte ab.

Diese Inkonsistenz muss man buchhalterisch bewältigen. Das dabei benutze Konzept ist das der Differentialform.
\end{enumerate}
\end{bem}

Als n\"achstes wollen wir einen interessanten Existenzsatz \"uber stetige
Funktionen auf normalen topologischen R\"aumen -- wie etwa Mannigfaltigkeiten 
-- zeigen. 

\begin{satz}{Existenzsatz von Urysohn\footnote{Pawel Samuilowitsch Urysohn, 1898-1924}}
Es seien $X$ ein normaler topologischer Raum und $A,B\subseteq X$ zwei 
disjunkte, abgeschlossene Teilmengen. 

Dann gibt es eine stetige Funktion $f\in{\cal C}(X),$ die auf $A$ konstant 
gleich $0$ und auf $B$ konstant gleich $1$ ist und nur Funktionswerte zwischen 
$0$ und $1$ annimmt.
\end{satz}


{\it Beweis.} Nat\"urlich d\"urfen wir $A$ und $B$ als nichtleer voraussetzen, 
sonst nehmen wir einfach eine konstante Funktion.

Wir konstruieren zun\"achst eine Familie von offenen Mengen, die durch die
Zahlen 
$$D\da\{a/2^m \mid 0\leq a\leq 2^m, a,m\in \mathbb N_0\}$$ 
parametrisiert werden und die Bedingung
$$\forall p,q\in D, p<q: \overline{U(p)} \subseteq U(q) $$
erf\"ullen und etwas mit $A$ und $B$ zu tun haben. Diese Definition geht
rekursiv nach dem ben\"otigten Exponenten bei der Potenz von $2$ im Nenner.

Wir setzen $U(1) = U(1/2^0) \da X\setminus B$ und
w\"ahlen weiter zwei disjunkte Umgebungen $U$ bzw.\ $V$ von $A$ bzw.\  $B.$

Der Abschluss von $U$ ist dann immer noch disjunkt zu $B,$ und wir setzen 
$U(0)=U(0/2^0) = U.$

Sind nun alle $U(a/2^n)$ f\"ur $n\leq N$ und alle erlaubten $a$ definiert, so 
m\"ussen wir $U(a/2^{N+1})$ f\"ur ungerade Zahlen $1\leq a\leq 2^{N+1}-1$ 
definieren.

Dazu w\"ahlen wir disjunkte offene Umgebungen $U,V$ von 
$\overline{U((a-1)/2^{N+1})}$ und $X\setminus U((a+1)/2^{N+1})$, die es 
wegen der Disjunktheit und der Normalit\"at von $X$ gibt.
Dann setzen wir $U(a/2^{N+1}) \da U.$

Diese Mengen $U(p)$ tun offensichtlich das, was wir wollen. Wir benutzen sie
nun, um $f$ zu definieren. Wir setzen n\"amlich

$$\forall x\in X: f(x)\da\left\{\begin{array}{rl}
\inf\{p\in D \mid x\in U(p)\}, & {\rm falls}\ x\in U(1),\\
1                            , & x\in B.\\
\end{array}\right.$$

Auf $A \subseteq U(0)$ ist $f$ 0, auf $B$ ist es 1. Wir m\"ussen die 
Stetigkeit von $f$ zeigen.

Dazu sei $x\in X$ mit $f(x)=r\in [0,1].$ 

F\"ur $r<1$ ist f\"ur alle 
$\varepsilon >0$ die Menge 
$$\bigcup_{p\in D,p< r+\varepsilon/2} U(p) \setminus
  \overline{\bigcup_{{q\in D,q< r-\varepsilon/2}} {U(q)}}$$
eine offene Umgebung von $x,$ auf der nur Funktionswerte in 
$(r-\varepsilon, r+\varepsilon)$ angenommen werden. Das zeigt die Stetigkeit 
in $x.$

F\"ur $r=1$ gilt \"ahnliches f\"ur
$$X\setminus\overline{\bigcup_{\mathclap{q\in D,q< r-\varepsilon/2}} {U(q)}},$$
und wir sind fertig.\qed


Dieser Satz sagt also insbesondere, dass die Punkte eines normalen Raumes von 
den stetigen Funktionen getrennt werden -- so wie wir es uns sicher vorstellen.
Der Satz hat einen Bruder, der mit ihm oft in einem Atemzug genannt wird.

\begin{satz}{Erweiterungssatz von Tietze\footnote{Heinrich Franz Friedrich 
Tietze, 1880-1964}}
Es sei $X$ ein normaler topologischer Raum. Weiter sei $A\subseteq X$ eine 
abgeschlossene Teilmenge und $f:A\longrightarrow [-1,1]$ eine stetige Funktion.

Dann l\"asst sich $f$ zu einer stetigen Funktion $F:X\longrightarrow [-1,1]$ fortsetzen.
\end{satz}

Das geht nat\"urlich dann auch f\"ur jedes andere Intervall $[a,b]$ anstelle 
von $[0,1],$ aber der Beweis ist so etwas weniger notationslastig.

{\it Beweis.} Es sei 
$$A_0\da\{a\in A\mid f(a)\leq -1/3\},\ A_1\da \{a\in A\mid f(a)\geq 1/3\}.$$ 
Das sind disjunkte, abgeschlossene 
Teilmengen von $X$ (denn $A$ ist abgeschlossen), und so finden wir eine stetige
Funktion $F_1:X\longrightarrow [-1/3,1/3],$ 
die auf $A_0$ den Wert $-\frac13$ und auf $A_1$ den Wert $\frac13$ annimmt.

Die Funktionswerte von $f-F_1$ liegen also zwischen $-\frac23$ und $\frac23$.

Wir definieren neue Mengen $A_0^{(1)}$ und $A_1^{(1)}$ durch 
$$A_0^{(1)}\da\{a\in A\mid f(a)-F_1(a)\leq -2/9\},\ 
A_1^{(1)}\da \{a\in A\mid f(a)-F_1(a)\geq 2/9\}.$$ 
Dann gibt es eine Funktion $F_2$ auf $X$ mit Werten in $[-2/9,2/9],$ die auf 
$A_0^{(1)}$ den Wert $-2/9$ annimmt und auf $A_1^{(1)}$ den Wert $2/9.$
Die Funktionswerte $(f(a)-F_1(a))-F_2(a)$ liegen also alle zwischen $-4/9$ und
$4/9.$

Wenn nun $F_1,\dots F_n$ sukzessive so definiert sind, dass auf $A$
die Funktionen $f$ und $F_1+\dots +F_n$ sich betragsm\"a\ss ig um h\"ochstens 
$(\frac23)^n$ unterscheiden und $F_i$ f\"ur $1\leq i\leq n$ betragsm\"a\ss ig
nicht gr\"o\ss er ist als $\frac13(\frac23)^{i-1}),$ so definieren wir
$$\begin{array}{ll}
A_0^{(n)}&\da\{a\in A\mid f(a)-(F_1(a)+\dots +F_n(a))\leq -2^{n-1}/3^n\},\\
A_1^{(n)}&\da \{a\in A\mid f(a)-(F_1(a)+\dots +F_n(a))\geq 2^{n-1}/3^n\}.\\
\end{array}$$ 
Wie vorher gibt es eine stetige Funktion $F_{n+1}:X\longrightarrow [-2^{n-1}/3^n,
2^{n-1}/3^n]$, die auf $A_0^{(n)}$ den linken und auf $A_1^{(n)}$ den rechten
Randpunkt des Intervalls annimmt. Sie unterscheidet sich also von 
$f(a)-(F_1(a)+\dots +F_n(a))$ betragsm\"a\ss ig um h\"ochstens 
$(\frac23)^{n+1}.$

Damit stellen wir sicher, dass es eine nicht abbrechende Folge von Funktionen
$F_i,i\in \mathbb N,$ gibt, die die obigen Absch\"atzungen erf\"ullen.

Das zeigt, dass die Funktionenfolge $S_n\da F_1+\dots +F_n$ gleichm\"a\ss ig
konvergiert, der Limes mithin stetig ist, und dass sie auf $A$ gegen
$f$ konvergiert. Au\ss erdem sind die Funktionswerte wegen der geometrischen 
Reihe allesamt betragsm\"a\ss ig zwischen $-1$ und 1.\qed


\begin{bem} {Neue Ziele}
Wenn man im Fortsetzungssatz von Tietze anstelle einer reellwertigen Funktion
eine Vektorwertige Funktion mit Werten in $[-1,1]^d$ auf $A$ vorgibt, so lassen 
sich die Komponenten (die ja alle stetige Funktionen sind) alle nach $X$ 
fortsetzen und zu einer Fortsetzung von $f$ zu einer Funktion 
$F:X\longrightarrow [-1,1]^d$ kombinieren.

Wenn anstelle eines solchen \"ubersichtlichen Raums ein Zielraum verwendet 
wird, von dem man wei\ss , dass er zu $[-1,1]^d$ f\"ur ein $d\in \mathbb N$ 
hom\"oomorph ist, so gilt der Fortsetzungssatz immer noch.
\end{bem}

\begin{bsp}{Kein Ziel}
Es sei $K = \overline{ B_1(0)} \subseteq \mathbb R^2$ die abgeschlossene Einheitskreisschreibe und $S^1=\partial K = \{(x,y)\in \mathbb R^2\mid x^2 +y^2=1\}$. Dann gibt es keine stetige Abbildung $F:K\to S^1$ mit $F|_{S^1} = \operatorname{Id}_{S^1}$.

{\it Beweis.} Zur Vorbereitung betrachten wir $\pi:\mathbb R\to S^1$ mit $\pi(x)=(\cos 2\pi x, \sin 2\pi x)$. Weiter sei $\gamma[0,1]\to S^1$ ein stetiger Weg und $\alpha_0\in \mathbb R$, so dass $\gamma(0)=(\cos 2\pi \alpha_0, \sin 2\pi \alpha_0)$. Dann gibt es genau eine stetige Abbildung $[0,1]\to \mathbb R$, so dass $\lambda(0)=\alpha_0$ und $\gamma = \pi \circ \lambda$.

Um die Eindeutigkeit zu zeigen nehmen wir an, $\lambda_1$ und $\lambda_2$ seien Funktionen mit den gewünschten Eigenschaften. Dann ist $\lambda_1(0)-\lambda_2(0)=\alpha_0-\alpha_0=0$ und $\lambda_1-\lambda_2: [0,1]\to \mathbb R$ ist stetig und nimmt nur Werte in $\mathbb Z$ an, denn für jedes $t\in[0,1]$ gilt $(\pi\circ\lambda_1)(t) = (\pi\circ\lambda_2)(t)$, also $\cos2\pi\lambda_1(t) = \cos2\pi\lambda_2(t)$ und $\sin2\pi\lambda_1(t) = \sin2\pi\lambda_2(t)$. Also ist $\lambda_1-\lambda_2$ konstant gleich $0$, und damit $\lambda_1=\lambda_2$.

Weiter müssen wir die Existenz zeigen: Es sei $t_0=\sup\{\tau\in[0,1]\mid$ es gibt ein $\lambda_\tau:[0,\tau]\to \mathbb R$ mit $\lambda_\tau(0)=\alpha_0$, $\lambda_\tau$ stetig und $\gamma|_{[0,\tau]} = \pi \circ \lambda_\tau\}$. Für $\tau=0$ wird dies durch $\lambda_0:[0,0]\to\mathbb R$, $0\mapsto\alpha_0$ erfüllt. Betrachte dann $\gamma(t_0)$ und eine offene Umgebung $U$ von $\gamma(t_0)$ in $S^1$, so dass ein offenes $V\in \mathbb R$ existiert mit einem Homöomorphismus $\pi|_V : V\to U$, etwa etwa mit $\beta\in \mathbb R$ mit $\gamma(t_0)=\pi(\beta)$, $V=(\beta-\frac13,\beta+\frac13)$ und $U=\pi(V)=\{(\cos2\pi\varphi, \sin2\pi\varphi)\mid \varphi \in (\beta-\frac13,\beta+\frac13)\}$. Wähle $\varepsilon>0$, so dass $D\da \gamma((t_0-\varepsilon,t_0+\varepsilon)\cap[0,1])\subseteq U$. Dann ist $\pi^{-1}|_V \circ \gamma|_D: D\to \mathbb R$ derart, dass $\gamma|_D=\pi \circ \underbrace{(\pi^{-1}|_V\circ \gamma|_D)}_{l\da}$. Wähle nun ein $\tau\in[t_0-\varepsilon,t_0]$, so dass $\lambda_\tau$ definiert ist. Wegen $\pi(\lambda_\tau))=\gamma(\tau)=\pi(l(\tau))$ gibt es ein $k\in\mathbb Z$ mit $\lambda_\tau(\tau)=l(\tau)+k$. Definiere $\tilde\lambda:([0,t_0+\varepsilon]\cap[0,1]) \to \mathbb R$ mit 
\[\tilde\lambda(t) =
\begin{cases}
\lambda_\tau(t),&t\le \tau\\
l(t), &t>\tau
\end{cases}
\]
$\tilde\lambda$ ist stetig, auf der Definitionsmenge von $\tilde\lambda$ gilt $\lambda = \pi \circ \tilde\lambda$, hat den richtigen Anfangswert und ist bei $t_0$ definiert. Also ist $t_0=1$.

Damit ist die Vorbereitung abgeschlossen, und wir wenden uns der Aussage zu. Sei $F:K\to S^1$ eine stetige Abbildung. Dazu betrachten wir den stetigen Weg $\gamma:[0,1]\to S^1$, $t\mapsto F( (t,0))$. Die Vorbereitung sagt uns un, dass es eine Funktion $\lambda:[0,1]\to \mathbb R$ mit $\pi\circ\lambda=\gamma$ gibt. Für $0\le r\le1$ sind $g_r:[0,1] \to S^1$, $x\mapsto F(r\cdot(\cos2\pi x, \sin2\pi x))$ auch stetige Funktionen mit $g_r(0)-g_r(1)=\in \mathbb Z$. Für jedes $r$ gibt es also $l_r:[0,1]\to\mathbb R$ mit $\pi\circ l_r=g_r$ und $l_r(0)=\lambda(r)$.

Daraus konstruieren wir die Abbildung $H:[0,1]\times[0,1]\to \mathbb R$, $(r,x)\mapsto l_r(x)$, die (ohne Beweis) stetig ist. $H(r,1)-H(r,0)$ ist eine stetige Abbildung von $[0,1]\to \mathbb R$, die nur Werte aus $\mathbb Z$ annimmt, also konstant ist und für $r=0$ den Wert $0$ annimmt, also gilt $H(r,1)-H(r_0)$ für alle $r$.

Wäre $F|_{S^1}=\operatorname{Id}|_{S^1}$, dann wäre $g_1(x)=(\sin2\pi x, \cos2\pi x)$ und $l_1(x)=x+c$, $c\in \mathbb Z$. Damit gälte $H(1,0)-H(1,1)=-1$, aber $H(0,0)-H(0,1)=0$, Wid!\qed
\end{bsp}

\begin{bem}[Umlaufzahl]
\begin{enumerate}[a)]
\item 
Im letzten Beispiel betrachteten wir die stetige Kurve $\gamma[0,1]\to S^1$ und draus konstruiert die stetige Abbildung $\lambda : [0,1]\to \mathbb R$ mit $\gamma = \pi \circ \lambda$. $\gamma$ ist gleichmäßig stetig, also gibt es ein $\varepsilon>0$ so dass für alle $x,y\in[0,1]$ mit $|x-y|<\varepsilon$ gilt: $|\gamma(x)-\gamma(y)|<2$.

Wähle $N\in \mathbb N$, so dass $\frac1N<\varepsilon$, und Unterteile das Interval $[0,1]$ in $N$ gleich große Teilintervalle $J_k=[\frac{k-1}N,\frac kN]$, $1\le k\le N$. Die Einschränkung $\gamma|_{J_k} : J_k \to S^1 \setminus \{-\gamma(\frac{k-1}N)\}$ enthält demnach einen der Gegenpunkte nicht. Weiter ist $\pi^{-1}(S^1\setminus \{-\gamma(\frac{k-1}N)\})=\mathbb R \setminus \{\alpha_0 + l \mid l\in \mathbb Z\}$ mit $\pi(\alpha_0) = -\gamma(\frac{k-1}N)$. Das heißt, dass $\pi|_{(\alpha_0,\alpha_0+1)}$ eine Bijektion zwischen $(\alpha_0,\alpha_0+1)$ und $S^1 \setminus \{-\gamma(\frac{k-1}N)\}$, und sogar ein Homöomorphismus.

Sei $\psi:S^1\setminus \{-\gamma(\frac{k-1}N)\}\to \mathbb R$ die stetige Abbildung, für die gilt: $\pi\circ \phi = \operatorname{id}|_{S^1\setminus \{-\gamma(\frac{k-1}N)\}}$. Definiere nun $\lambda_:J_k\to \mathbb R$, $\lambda_k = \psi\circ\gamma|_{J_k}$, also $\pi \circ \lambda_k = \gamma|_{J_k}$. Es ist $a_k\da \lambda_{k+1}(\frac kN) - \lambda_k(\frac kN) \in \mathbb Z$, denn $\pi(\lambda_k(\frac kN)) = \gamma(\frac kN) = \pi(\lambda_{k+1}(\frac kN))$. Daraus konstruieren wir $\lambda : [0,1]\to \mathbb R$ durch $\lambda(x)= \lambda_k(x)+a_1+\ldots+a_{k-1}$, wobei $\frac {k-1}N\le x\le \frac{k}N$.

Nun sei $\gamma:[0,1]\to S^1$ ein geschlossener Weg, also $\gamma(0)=\gamma(1)$. Wie gerade gezeigt, gibt es ein stetiges $\lambda:[0,1]\to \mathbb R$ mit $\gamma = \pi \circ \lambda$. Dann heißt $\lambda(1)-\lambda(0)\in\mathbb Z$ die \emph{Umlaufzahl} von $\gamma$ um $0$

Allgemeiner sei $\gamma:[0,1]\to \mathbb R^2\setminus\{0\}$ ein geschlossener Weg, dann gilt für alle $t\in [0,1]$:
\[
\gamma(t) = \underbrace{\|\gamma(t)\|}_{\in \mathbb R_{>0}} \cdot \underbrace{\frac{\gamma(t)}{\|\gamma(t)\|}}_{\in S^1}
\]
Es ist $t\mapsto \frac{\gamma(x)}{\|\gamma(t)\|}$ ein geschlossener Weg in $S^1$, und wir nennen seine Umlaufzahl auch die \emph{Umlaufzahl} von $\gamma$ um $0$, geschrieben $\mathcal X(\gamma,0)$.

\item 
Dies entspricht der Umlaufzahl im funktionentheoretischen Sinne. Sei $\gamma:[0,1]\to \mathbb C\setminus\{0\}$ stetig differenzierbar und geschlossenen. Diese Kurve hat die funktionentheoretische Umlaufzahl $\mathcal X(\gamma,0)=\frac1{2\pi i}\int_0^1 \frac{\gamma'(t)}{\gamma(t)}dt$. Wenn $\gamma$ wie oben zerlegt ist in $\gamma(t) = r(t)\cdot \tilde\gamma(t)$, dann wähle $\tilde\lambda:[0,1]\to \mathbb R$ differenzierbar, so dass $\tilde\gamma(t)=\pi(\tilde\lambda(t)) = \exp(2\pi i\tilde\lambda(t))$. Die folgende Rechnung zeigt, dass die beiden Umlaufzahlbegriff gleich sind:
\begin{align*}
\int_0^1 \frac{\gamma'(t)}{\gamma(T)}dt &= \int_0^{1} \frac{r(t)'\cdot \exp(2\pi i\tilde\lambda(t)) + r(t) \cdot 2\pi i\tilde\lambda(t)\exp(2\pi i\tilde\lambda'(t))}{r(t)\cdot \exp(2\pi i \tilde \lambda(t))}dt \\
&= \int_0^1 \frac{r'(t)}{r(t)} dt + \int_0^1 2\pi i {\tilde\lambda'(t)}dt \\
&= \ln r(t)|_0^1 + 2\pi i\tilde \lambda(t)|_0^1 \\
&= 0+ 2\pi i(\tilde\lambda(1)-\tilde\lambda(0))
\end{align*}

\item 
Sei $\gamma:[0,1]\to \mathbb R^2$ eine stetige, geschlossene Kurve und $x\in \mathbb R^1\setminus \operatorname{Bild}(\gamma)$. Dann heißt $\mathcal X(\gamma,x)\da \mathcal X(\gamma-x,0)$ die \emph{Umlaufzahl} von $\gamma$ um $x$.

\item Die Umlaufzahl $\mathcal X(\gamma,0)$ bleibt konstant, wenn an $\gamma$ nur wenig gewackelt wird. Präziser:

Es sei $\Gamma:[0,1]\times[0,1]\to\mathbb R^1\setminus\{0\}$ stetig und so, dass für alle $t\in [0,1]$ die Kurve $\gamma_t:[0,1]\to \mathbb R^1\setminus\{0\}$, $x\mapsto \Gamma(t,x)$ geschlossen ist. Dann haben alle Kurven $\gamma_t$ die selbe Umlaufzahl um $0$: $\forall t\in \mathcal X(\gamma_t,0)=\mathcal X(\gamma_0,t)$.

Denn es gibt eine Abbildung $H:[0,1]\times[0,1]\to\mathbb R$ stetig, so dass für alle $t$ und $x$ gilt: $\gamma(t,x) =\|\Gamma(t,x)\|\cdot \pi(H(t,x))$, $\pi : \mathbb R \to S^1$, was man über eine ähnliche Weise zeigen kann, wie weiter oben in dieser Bemerkung benutzt. Weil $\mathcal X(\gamma_t,0)=H(t,1)-H(t,0)\in \mathbb Z$ und stetig ist, ist sie konstant.

%Das Argument vom letzten Beispiel lässt sich nun auch so zeigen: Sei $F: K\to S^1$, $K=\overline{\B_(0)}$. Dann gilt für die Abbildung $[0,1]\times[0,1]\to K \to S^1$, $(t,x)\mapsto t\cdot\pi (x)$
% Blubb\dots
\end{enumerate}
\end{bem}

\section{Mehr von den Kompakta}

Wir haben jetzt schon ein paar Mal gesehen, dass der Begriff der Kompaktheit
ein wichtiger Begriff in der Topologie ist. Daher wollen wir jetzt diesen 
Begriff noch etwas mehr beleuchten.

Unser erstes Ziel ist der Satz von Tikhonow\footnote{Andrej Nikolajewitsch 
Tikhonow, 1906-1993; oft auch als Tychonoff transkibiert}.

Zun\"achst m\"ussen wir aber sagen was ein unendliches Produkt von 
topologischen R\"aumen ist.

\begin{defini}{unendliche Produkte}
Es seien $I$ eine Indexmenge und $X_i,i\in I,$ topologische R\"aume. Das
{\it (unendliche) Produkt} $\prod_{i\in I}X_i$ ist definiert als die Menge aller
Abbildungen 
$$f: I\longrightarrow \bigcup_{i\in I}X_i,\ {\rm sodass}\ \forall i\in I: 
f(i)\in X_i.$$

Die Elemente dieses Produkts werden wir oft auch suggestiv als $(x_i)_{i\in I}$
notieren.

Auf dem Produktraum wollen wir eine Topologie einf\"uhren, n\"amlich die 
gr\"obste, f\"ur die alle Projektionen auf die Komponenten stetig sind. 

F\"ur festes $i_0\in I$ sei $U\subseteq X_{i_0}$ offen. Dann muss also die Menge
$S(i_0,U)$ aller Funktionen $f$ aus dem Produkt mit der Eigenschaft
$$f(i_0)\in U$$
im Produktraum offen sein.

Die Topologie ist also diejenige, die die Mengen $S(i,U)$ mit $i\in I, 
U\subseteq X_i$ offen, als Erzeuger (Subbasis) besitzt.

Eine Teilmenge $T\subseteq \prod_{i\in I}X_i$ ist also genau dann offen, wenn 
f\"ur jedes $f\in T$ endlich viele Indizes $i_1,\dots ,i_k\in I$ und offene 
Mengen $U_{j}\subseteq X_{i_j},\ j =1,\dots ,k,$ existieren, sodass
$$f\in S(i_1,U_1)\cap \dots S(i_k,U_k)\subseteq T$$

\end{defini}

Der Satz von Tikhonow wird nachher sagen, dass ein Produkt von kompakten 
R\"aumen wieder kompakt ist. Als Vorbereitung zeigen wir schon einmal das
folgende.

\begin{hilfs}{spezielle \"Uberdeckungen}\label{Spezialfall} 
Es seien $X_i, i\in I,$ kompakte topologische R\"aume und es gebe eine 
\"Uberdeckung $\ddot U$ von $\prod_{i\in I} X_i$ durch offene Teilmengen der Form
$S(i,U),$ wie wir sie in der Definition der Produkttopologie als Subbasis 
benutzt haben.

Dann besitzt $\ddot U$ eine endliche Teil\"uberdeckung.
\end{hilfs}

{\it Beweis.}

F\"ur den Beweis muss man tats\"achlich alles vor allem sauber hinschreiben.
Wir w\"ahlen also eine Indexmenge $J$ und schreiben uns die \"Uberdeckung
$\ddot U$ als
$$\ddot U = \{S(i_j,U_j)\mid j\in J, i_j\in I, U_j\subseteq X_{i_j}\ {\rm offen
}\}.$$
{\bf Wir nehmen an,} $\ddot U$ besitze keine endliche Teil\"uberdeckung. 

Wir w\"ahlen zun\"achst ein $i_0\in I$ beliebig. 

Weiter w\"ahlen wir ein $x_0\in X_{i_0},$ und bezeichnen mit $H(x_0)$ die Menge 
$$H(x_0):=\{(x_i)_{i\in I} \mid x_{i_0} = x_0 \}.$$
Wenn es eine endliche Teilmenge $K\subseteq J$ gibt mit
$$H(x_0)\subseteq \cup_{k\in K}S(i_k, U_k),$$
so k\"onnten wir f\"ur jedes $i\in \{i_k\mid k\in K\}$ ein 
$$x_i\in X_i\smallsetminus \cup_{k\in K, i_k=i} U_k$$
finden. Ansonsten w\"are ja die Vereinigung dieser $U_k$ ganz $X_i,$ und
die Vereinigung dieser $S(k,U_k), i_k=i,$ w\"are der ganze Produktraum -- 
im Widerspruch zu unserer Annahme.

Nun definieren wir ein Element des Produktraums durch 
$$x_i := \left\{\begin{array}{rl}
x_0,&\ i = i_0,\\
x_i,&\ i_0\neq i\in \{i_k\mid k\in K\},\\
{\rm beliebig},&\ {\rm sonst}.\\
\end{array}\right.$$
Dieses $(x_i)_{i\in I}$ liegt in einem der $S(i_k,U_k), k\in K.$ aber wenn
$i_k$ hier nicht unser altes $i_0$ w\"are, dann w\"are $x_{i_k}$ ja nach 
Konstruktion gerade nicht in $U_k.$ Also muss $i_k=i_0$ sein, und es gilt
$x_0\in U_k.$ Daher folgt $H(x_0)\subseteq S(i_k,U_k)$ f\"ur ein $k\in K$ aus 
der Voraussetzung, dass sich $H(x_0)$ durch endlich viele der $S(i_j,U_j)$
\"uberdecken l\"asst. 

W\"are das f\"ur alle $x_0\in X_{i_0}$ der Fall, so h\"atte man f\"ur jedes
$x_0\in X_{i_0}$ einen Index $j\in J$ mit $i_j = i_0$ und $x_0\in U_j.$ 
Damit ist $X_{i_0} = \cup_{{\rm diese}\ j} U_j,$ und wegen der Kompaktheit von 
$X_{i_0}$ g\"abe es endlich viele $j$ mit $i_0=i_j,$ und so dass die 
zugeh\"origen $U_j$ eine \"uebrdeckung von $X_{i_0}$ sind. Dann \"uberdecken
die zugeh\"origen $S(i_j,U_j)$ aber auch den ganzen Produktraum, und das
widerspricht unserer Grundannahme.

{\bf Fazit:} F\"ur jedes $i\in I$ gibt es ein $y_i\in X_i,$ sodass 
$H(y_i)$ nicht von endlich vielen der $S(i_j,U_j)$ unserer \"Uberdeckung 
$\ddot U$ \"uberdeckt wird.

Aber das hierdurch gefundene Element $(y_i)_{i\in I}$ liegt auch in einem
$S(i_j,U_j)$ f\"ur ein $j\in J.$ Insbesondere liegt also $y_{i_j}\in U_j,$
und damit $H(y_{i_j})$ in $S(i_j,U_j).$ 

Das widerspricht unserer Konstruktion
von $y_{i_j},$ und bringt damit die Annahme zum Einsturz.
\qed

Um nun weiter zu machen brauchen wir eine neue Definition.

\begin{defini} {Filter, Ultrafilter}
\begin{enumerate}[a)]
\item Es sei $X$ eine Menge. Eine Teilmenge $\cal F\subseteq {\cal P}(X)$ 
hei\ss t ein {\it Filter}\index{Filter}, falls folgende Bedingungen erf\"ullt 
sind:

\begin{itemize}
\item $\forall U,V\in {\cal F}: U\cap V\in {\cal F}.$
\item $\forall U\in {\cal F}: \forall V\subseteq X: [U\subseteq V]\Rightarrow 
V\in {\cal F}.$
\item $\emptyset\not\in {\cal F}.$
\end{itemize}

{\bf Beispiel:} Ist $X$ ein topologischer Raum, $x\in X$, so sind die Umgebungen von $x$ ein Filter auf $X$.

\item Ein Filter $\cal F$ auf $X$ hei\ss t ein Ultrafilter, wenn er in keinem
gr\"o\ss eren Filter enthalten ist.

\item Ist $X$ ein topologischer Raum und $\cal F$ ein Filter auf $X$, so 
{\it konvergiert der Filter gegen}\ $a\in X,$ falls jede Umgebung von $a$ zu
$\cal F$ geh\"ort.
\end{enumerate}
\end{defini}


\begin{center}
Die offizielle Fortführung des Skriptes finden Sie auf \url{http://www.mathematik.uni-karlsruhe.de/iag3/lehre/top2007w/}.
\end{center}

\end{document}
% Things to do when pasting stuff from upstream:
% * \hfill{$\bigcirc$} → \qed
% * : = → \da
% * \bf und \rm weg von defini etc.

%\input Stichworte.tex

%%%%%%%%%%%%%%%%%%%%%%%%%%%%%%%%%%%%%%%%
%\hrule\textsc{Stand der Vorlesung}\hrule
%%%%%%%%%%%%%%%%%%%%%%%%%%%%%%%%%%%%%%%%
