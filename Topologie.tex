\documentclass[12pt]{scrbook}   %12pt --> art10

\oddsidemargin=0cm
\evensidemargin=0cm
\topmargin=-2cm
\textwidth=18cm
\textheight=25cm
\mathsurround=1.5pt
\parskip=5pt
\parindent=0pt
%\setcounter{page}{0}

%\input pictex
%\input prepictex
%\input postpictex

%\usepackage{color}
\usepackage{a4}
\usepackage{amsmath,amssymb,amsfonts}
\usepackage[utf8]{inputenc}
\usepackage{fontenc}[german]
\usepackage[all]{xy}
\usepackage{german}
\selectlanguage{german}
%\renewcommand{\thefootnote}{\fnsymbol{footnote}}
%\pagestyle{empty}

\usepackage{hyperref}

\newtheorem{alles}{alles}[section]
\newtheorem{bem}[alles]{Bemerkung}
\newtheorem{satz}[alles]{Satz}
\newtheorem{hilfs}[alles]{Hilfssatz}
\newtheorem{defini}[alles]{Definition}
\newtheorem{bsp}[alles]{Beispiel}
\newtheorem{fazit}[alles]{Fazit}
\newtheorem{Folgerung}[alles]{Folgerung}

\author{Dr. Stefan Kühnlein und \url{http://mitschriebwiki.nomeata.de/}}
\title{Topologie}

%\makeindex

\begin{document}
\maketitle

%\renewcommand{\thechapter}{\Roman{chapter}}
%\chapter{Inhaltsverzeichnis}
\stepcounter{chapter}
\addcontentsline{toc}{chapter}{\protect\numberline {\thechapter}Inhaltsverzeichnis}
\tableofcontents


\chapter{Vorwort}

\section*{Über dieses Skriptum}

Das Ziel diese Skripts ist es, einen ersten Einblick in die Topologie zu geben.
An keiner Stelle wird versucht, Ergebnisse bis in die letzten Winkel und 
Spitzen zu treiben; wir erlauben uns auch bisweilen, nicht geringstmögliche 
Voraussetzungen in Aussagen zu machen, sondern hoffen, durch eine 
Beschränkung auf einfachere Situationen bisweilen den Inhalt der Sätze
(von denen es ohenhin nicht so viele gibt) deutlich zu machen. Die Vorlesung
ist nicht für Spezialisten gedacht - das verbietet sich schon angesichts des
Dozenten, der ja auch selbst kein Spezialist ist. Hiermit sei seiner Hoffnung 
der Ausdruck verliehen, dass die subjektive Stoffauswahl nicht zu sehr zu 
Lasten der Allgemeinheit geht und ein Verständnis trotz allem zustande
kommen kann.

Jedenfalls werden in dieser Vorlesung nicht alle erlaubten Implikationen 
zwischen allen möglichen Aussagen vorgeführt werden. 

Ich habe auf eine umfangreiche Illustration verzichtet, zum einen weil dies in 
der Vorlesung passieren soll, zum 
anderen, weil es vielleicht auch für Leser eine instruktive Übung ist, sich
selbst ein Bild von dem zu machen, wovon die Rede ist. Der begriffliche 
Apparat ist das präzise Werkzeug, die Bilder sind ja „nur“ ein 
Hilfsmittel, das uns helfen soll zu sehen, wo die Werkzeuge angesetzt werden 
können. Außerdem sind manche Bilder sehr irreführend, zumal wenn es um
Sachverhalte geht, die sich definitiv nicht mehr in unserem Anschauungsraum
abspielen können.

\subsection*{Online-Version}
Auf \url{http://mitschriebwiki.nomeata.de} finden sich die \LaTeX-Quellen zu diesem Skript, sowie die Möglichkeit, dort direkt am Skript mitzuarbeiten, etwa um Fehler zu beseitigen. Dies Basiert auf latexki\footnote{\url{http://latexi.nomeata.de/}}, einem von Joachim Breitner programmiertem Wiki für \LaTeX-Dokumente.

\bigskip

Nun zum Inhalt selbst.

\setcounter{chapter}{0}
\renewcommand{\thechapter}{\Roman{chapter}}

\renewcommand{\thesection}{§\,\thechapter.\arabic{section}}
\renewcommand{\thealles}{\thechapter.\arabic{section}.\arabic{alles}}

\chapter{Einstieg}

\section{Kontext}




Die Topologie ist eine mathematische Grundlagendisziplin die sich verstärkt
seit dem Ende des 19.\ Jahrhunderts eigenständig entwickelt hat. Vorher waren
einige topologische Ideen im Zusammenhang mit geometrischen und analytischen 
Fragestellungen entstanden. Um Topologie handelt es sich zunächst immer dann,
wenn geometrische Objekte deformiert werden und solche Eigenschaften der 
Objekte in den Vordergrund treten die sich dabei nicht ändern. 

Topologisch ist eine Kugel dasselbe wie ein Würfel - geometrisch zwar 
völlig unterschiedlich, aber doch gibt es einige Gemeinsamkeiten. 
Es wäre vielleicht einmal interessant zu verfolgen, ob der Kubismus am Ende
des 19.\ Jhdts.\ und die topologische Frage nach „simplizialen 
Zerlegungen“ geometrischer Objekte sich gegenseitig beeinflusst haben\dots

Der Begriff der Nähe spielt in der Topologie eine gewisse Rolle, mehr als der
Begriff des Abstands, der für die Geometrie immerhin namensgebend war.
Die topologischen 
Mechanismen, die so entwickelt wurden, wurden nach und nach von ihren 
geometrischen Eltern entfernt; dafür sind die Eltern ja da: sich 
überflüssig zu machen. Und so konnten topologische Ideen sich auch auf 
andere Bereiche der Mathematik ausdehnen und diese geometrisch durchdringen.

Auch außerhalb der Mathematik ist die Topologie längst keine unbekannte
mehr. So gab es in der ersten Hälfte des 20.\ Jhdts.\ die topologische 
Psychologie von Kurt Lewin, die allerdings nur die Terminologie von der 
Topologie übernahm, und nicht etwa mithilfe topologischer Argumente neue
Einsichten produzierte. Etwas anders sieht es natürlich mit den 
„richtigen“Naturwissenschaften aus. In der Physik taucht
die Topologie zum Beispiel in der Form von Modulräumen in der Stringtheorie 
auf, und in der Molekularchemie kann man zum Beispiel Chiralität als
topologisches Phänomen verstehen.



\section{Beispiele - was macht die Topologie?}

\begin{bsp} \label{Lasso}{\bf Nullstellenfang mit dem Lasso}

{\rm Es sei $f:\mathbb C\longrightarrow \mathbb C$ eine nichtkonstante 
Polynomabbildung, d.h.\ $f(z) = \sum_{i=0}^d a_iz^i$ mit $d>0$ und $a_d\neq 0.$

Dann hat $f$ eine Nullstelle in $\mathbb C.$ Das kann man zum Beispiel so 
plausibel machen:

Wenn $a_0=0$ gilt, dann ist $z=0$ eine Nullstelle. Wenn $a_0\neq 0$, dann 
brauchen wir ein Argument. Wir betrachten den Kreis vom Radius $R$ um den 
Nullpunkt: $RS^1 = \{z\in \mathbb C \mid |z|=R\}.$ Aus der Gleichung
$$f(z) = a_dz^d \cdot (1 + \frac{a_{d-1}}{a_d z} + \dots + 
\frac{a_{0}}{a_d z^d})$$ 
folgt, dass das Bild von $RS^1$ unter $f$ jedenfalls für großes $R$ im 
Wesentlichen der $d$-fach 
durchlaufene Kreis vom Radius $|a_d|R^d$ ist. Im Inneren dieser Schlaufe liegen
für großes $R$ sowohl die 0 als auch $a_0.$ Wenn man nun den Radius 
kleiner macht, so wird diese Schlaufe für $R\searrow 0$ zu einer Schlaufe um 
$a_0$
zusammengezogen -- das ist die Stetigkeit von $f.$ Für kleines $R$ liegt 
insbesondere $0$ nicht im Inneren der Schlaufe. Das aber heißt, dass beim
Prozess des Zusammenziehens die Schlaufe irgendwann mindestens einmal die 0
trifft. Dann hat man eine Nullstelle von $f$ gefunden.

Einen anders gelagerten und präzisen topologischen Beweis des 
Fundamentalsatzes werden wir in \ref{Fundamentalsatz} führen.
}
\end{bsp}

In diesem Argument -- das man streng durchziehen kann -- wird ein topologisches
Phänomen benutzt, um den Fundamentalsatz der Algebra zu beweisen. Das 
Zusammenziehen der Kurve durch Variation des Parameters $R$ werden wir später
allgemeiner als Spezialfall einer Homotopie verstehen.

\begin{bsp}{\bf Fahrradpanne}

{\rm Es gibt keine stetige Bijektion von einem Torus $T$ 
(„Fahrradschlauch“) auf eine Kugeloberfläche $S.$ 


Denn:  Auf dem Torus gibt es eine geschlossene Kurve $\gamma$, die ihn nicht 
in zwei 
Teile zerlegt. Ihr Bild unter einer stetigen Bijektion von $T$ nach $S$ 
würde dann $S$ auch nicht in zwei Teile zerlegen, da das stetige Bild des
Komplements $T\smallsetminus \gamma$ gleich 
$S\smallsetminus{\rm Bild\ von\ }\gamma$ zusamenhängend sein müsste, aber
das stimmt für keine geschlossene Kurve auf $S$.}
\end{bsp}

Auch hier sieht man ein topologisches Prinzip am Werk. Es ist oft sehr schwer
zu zeigen, dass es zwischen zwei topologischen Räumen (siehe später) keine
stetige Bijektion gibt. Dass ich keine solche finde sagt ja noch nicht wirklich
etwas aus\dots

In der linearen Algebra wei\ss\  man sehr genau, wann es 
zwischen zwei Vektorräumen einen Isomorphismus gibt, das hängt ja nur an 
der Dimension. \"Ahnlich versucht man in der Topologie, zu topologischen 
Räumen zugeordnete Strukturen zu finden, die nur vom Isomorphietyp 
abhängen, und deren Isomorphieklassen man besser versteht als die der 
topologischen Räume.

\begin{bsp} {\bf Eulers \footnote{Leonhard Euler, 1707-1783}Polyederformel}

{\rm Für die Anzahl $E$ der Ecken, $K$ der Kanten und $F$ der Flächen eines
(konvexen) Polyeders gilt die Beziehung $E-K+F=2.$

Das kann man zum Beispiel einsehen, indem man das Polyeder zu einer Kugel 
aufbläst, auf der man dann einen Graphen aufgemalt hat (Ecken und Kanten des 
Polyeders), und dann für je zwei solche zusammenhängenden Graphen zeigt, 
dass sie eine gemeinsame Verfeinerung haben. Beim Verfeinern ändert sich
aber $E-K+F$ nicht, und so muss man nur noch für ein Polyeder die 
alternierende Summe auswerten, zum Beispiel für das Tetraeder, bei dem 
$E=F=4, K=6$ gilt.
}
\end{bsp}

\begin{bsp} {\bf Reelle Divisionsalgebren}

{\rm Eine reelle Divisionsalgebra ist ein $\mathbb R$-Vektorraum $A$ mit einer
bilinearen Multiplikation, für die es ein neutrales Element gibt und jedes
$a\in A\smallsetminus\{0\}$ invertierbar ist. 

Beispiele hierür sind $\mathbb R,\mathbb C, \mathbb H$ 
(Hamilton\footnote{William Hamilton, 1788-1856}-Quaternionen) und -- wenn man 
die Assoziativität wirklich nicht haben will -- $\mathbb O$ (die 
Cayley\footnote{Arthur Cayley, 1821-1895}-Oktaven). Die Dimensionen dieser 
Vektorräume sind $1,2,4,8.$ Tatsächlich ist es so, dass es keine weiteren
endlichdimensionalen reellen Divisionsalgebren gibt. Dies
hat letztlich einen topologischen Grund. 

Zunächst überlegt man sich, dass die Struktur einer Divisionsalgebra auf 
$\mathbb R^n$ auf der $n-1$-dimensionalen Sphäre eine Verknüpfung
induziert, die fast eine Gruppenstruktur ist.

Dann kann man im wesentlichen topologisch zeigen, dass solch eine Struktur
auf der Späre nur für
$n\in\{1,2,4,8\}$ existieren kann. Solch eine Gruppenstruktur stellt nämlich 
topologische Bedingungen, die für die anderen Sphären nicht erfüllt sind.
}\end{bsp}



Eng damit zusammen hängt der

\begin{bsp} {\bf Satz vom Igel\footnote{Frans Ferdinand Igel, ???}}

{\rm Dieser Satz sagt, dass jeder stetig gekämmte Igel mindestens einen 
Glatzpunkt besitzt. Die Richtigkeit dieses Satzes gründet sich nicht darauf, 
dass es bisher noch niemanden gelingen ist, einen Igel zu kämmen. Sie hat
handfeste mathematische Gründe, die in einer etwas präziseren Formulierung 
klarer werden:

Etwas weniger prosaisch besagt der Satz „eigentlich\grqq, dass ein 
stetiges Vektorfeld auf der zweidimensionalen Sphäre mindestens eine
Nullstelle besitzt.
}
\end{bsp}

\begin{bsp} {\bf Brouwers\footnote{Luitzen Egbertus Jan Brouwer, 1881-1966}
Fixpunktsatz}

{\rm Jede stetige Abbildung des $n$-dimensionalen Einheitswürfels 
$W=[0,1]^n$ in sich selbst hat einen Fixpunkt.

Für $n=1$ ist das im Wesentlichen der Zwischenwertsatz. Ist $f:[0,1]
\longrightarrow [0,1]$ stetig, so ist auch $g(x) := f(x)-x$ eine stetige 
Abbildung von $[0,1]$ nach $\mathbb R$, und es gilt $g(0)\geq 0, g(1)\leq 0.$

Also hat $g$ auf jeden Fall eine Nullstelle $x_0$, aber das heißt dann 
$f(x_0) = x_0.$

Für $n\geq 2$ ist der Beweis so einfach nicht möglich, wir werden ihm 
eventuell auch nur für $n=2$ später noch begegnen.
}
\end{bsp}

\section{Mengen, Abbildungen, usw.}

Wir werden für eine Menge $M$ mit ${\cal P}(M)$ immer die Potenzmenge 
bezeichnen: 
$${\cal P}(M) = \{A \mid A\subseteq M\}.$$
Für eine Abbildung $f:M\longrightarrow N$ nennen wir das Urbild
$f^{-1}(n)$ eines Elements $n\in f(M)\subseteq N$ auch eine 
\index{Faser}{\it Faser} von $f.$

Eine Abbildung ist also injektiv, wenn alle Fasern  einelementig sind.

Ist $f$ surjektiv, so gibt es eine Abbildung $s:N\longrightarrow M$ mit
$f\circ s = {\rm Id}_N$ -- die identische Abbildung auf $N.$ Jede solche 
Abbildung $s$ heißt ein \index{Schnitt}{\it Schnitt} zu $f$. Er wählt zu 
jedem $n\in N$ ein $s(n)\in f^{-1}(n)$ aus. Wenn man also $M$ als Vereinigung
der Fasern von $f$ über den Blumentopf $N$ malt, so erhält der Name Schnitt
eine gewisse Berechtigung.

Eine \index{Partition}{\it Partition} von $M$ ist eine Zerlegung von $M$ in 
disjunkte, nichtleere Mengen $M_i,i\in I,$ wobei $I$ eine Indexmenge ist:
$$M=\bigcup_{i\in I} M_i,\ \ \forall i\neq j: M_i\cap M_j = \emptyset, 
M_i\neq\emptyset.$$

Hand in Hand mit solchen Partitionen gehen \"Aquivalenzrelationen auf $M.$
Die Relation zur Partition $M_i,i\in I$ wird gegeben durch
$$m\sim \tilde m \iff \exists i\in I: m,\tilde m\in M_i.$$
Umgekehrt sind die \"Aquivalenzklassen zu einer \"Aquivalenzrelation $\sim $ 
eine Partition von $M.$ Die Menge der \"Aquivalenzklassen nennen wir auch den
\index{Faktorraum}{\it Faktorraum} $M/\sim$:
$$M/\sim = \{M_i\mid i\in I\}.$$
Die Abbildung $\pi_\sim: M\longrightarrow M/\sim, \pi_\sim(m):= [m] = $ 
\"Aquivalenzklasse von $m$ heißt die {\it kanonische Projektion von } $M$ 
auf $M/\sim$.

Ist $\sim$ eine \"Aquivalenzrelation auf $M$ und $f:M\longrightarrow N$ eine
Abbildung, sodass
jede \"Aquivalenzklasse von $\sim$ in einer Faser von $f$ enthalten ist (d.h.
$f$ ist konstant auf den Klassen), so wird durch 

$$\tilde f:M/\sim \longrightarrow N, \tilde f([m]) := f(m),$$
eine Abbildung definiert, für die $f=\tilde f \circ \pi_\sim$ gilt. Das ist
die mengentheoretische Variante des Homomorphiesatzes.

\begin{bsp}{\bf Gruppenaktionen}

{\rm Ein auch in der Topologie wichtiges Beispiel, wie \"Aquivalenzrelationen 
bisweilen entstehen, ist das der \index{Gruppenoperation}{\it Operation} 
einer Gruppe $G$ auf der Menge $M$.

Solch eine Gruppenaktion ist eine Abbildung
$$\bullet:G\times M\longrightarrow M,$$
die die folgenden Bedingungen erfüllt:

$$\begin{array}{rl}\forall m\in M:& e_G\bullet m = m\\
\forall g,h\in G, m\in M:& g\bullet (h\bullet m) = (gh)\bullet m.\\
\end{array}$$
Hierbei ist $e_G$ das neutrale Element von $G$ und $gh$ ist das Produkt von 
$g$ und $h$ in $G$.

Für jedes $g\in G$ ist die Abbildung 
$$\rho_g:M\longrightarrow M,\rho_g(m) := g\bullet m,$$
eine Bijektion von $M$ nach $M$, die Inverse ist $\rho_{g^{-1}},$ und 
$$g\mapsto \rho_g$$
ist ein Gruppenhomomorphismus von $G$ in die symmetrische Gruppe von $M$.

Die \index{Bahn}{\it Bahn} von $m\in M$ unter der Operation von $G$ ist 
$$G\bullet m := \{g\bullet m\mid g\in G\}.$$
Man kann leicht verifizieren, dass die Bahnen einer Gruppenoperation eine
Partition von $M$ bilden.

Umgekehrt ist es so, dass jede Partition $(M_i)_{i\in I}$ von $M$ von der 
natürlichen Aktion einer geeigneten Untergruppe $G$ von ${\rm Sym}(M)$ 
herkommt. Hierzu wähle man einfach
$$G:=\{\sigma\in {\rm Sym}(M) \mid \forall i\in I: \sigma(M_i)= M_i\}$$
und verifiziere was zu verifizieren ist.
}
\end{bsp}
\begin{bsp} {\bf projektive Räume}

{\rm Es seien $K$ ein Körper und $n$ eine natürliche Zahl.

Auf $X:=K^{n+1}\smallsetminus\{0\}$ operiert die Gruppe $K^\times$ durch
die skalare Multiplikation
$$a\bullet v := a\cdot v.$$
Die Bahn von $v\in X$ unter dieser Operation ist $Kv\smallsetminus \{0\}.$
Da die $0$ ohnehin zu jeder Geraden durch den Ursprung gehört, kann man
den Bahnenraum $X / K^\times$ mit der Menge aller Geraden durch den Ursprung
identifizieren. Dieser Raum heißt der {\it $n$-dimensionale projektive Raum
über $K$}, \index{projektiver Raum}  in Zeichen $\mathbb P^n(K).$

Speziell für $n=1$ gilt:
$$\mathbb P^1(K) = \{[{a\choose 1}] \mid a\in K \} \bigcup \{[{1\choose 0}]\}.$$
Oft identifiziert man den ersten großen Brocken hier mit $K,$ den 
hinzukommenden Punkt nennt man suggestiver Weise $\infty.$

Genauso haben wir für beliebiges $n$ eine Zerlegung
$$\mathbb P^n(K) = \{[ {v\choose 1}] \mid v\in K^n \} \bigcup 
\{[{w\choose 0}] \mid w\in K^n,w\neq 0\} = K^n \bigcup \mathbb P^{n-1}(K),$$
wobei die Auswahl des {\it affinen Teils} $K^n$ durch die Bedingung, dass die 
letzte Koordinate nicht null ist, relativ willkürlich ist. 
}
\end{bsp}


\begin{defini} {\bf Faserprodukte}

{\rm Es seien $A,B,S$ Mengen und 
$f_A:A\longrightarrow S,\ f_B:B\longrightarrow S$
zwei Abbildungen.

Weiter sei $F$ eine Menge mit Abbildungen $\pi_A,\pi_B$ von $F$ nach $A$ bzw.\ 
$B$, sodass $f_A\circ\pi_A = f_B\circ \pi_B.$

$F$ heißt ein {\it Faserprodukt}\index{Faserprodukt} von $A$ und $B$ über
$S$, wenn für jede Menge $M$ und jedes Paar von Abbildungen $g_A,g_B$ von
$M$ nach $A$ bzw.\ $B$ mit $f_A\circ g_A = f_B\circ g_B$ genau eine Abbildung
$h:M\longrightarrow F$ existiert, sodass 
$$g_A = \pi_A\circ h,\ \ g_B = \pi_B\circ h.$$
}

\end{defini}

Insbesondere impliziert das, dass es zwischen zwei Faserprodukten von $A$ und 
$B$ über $S$ genau einen sinnvollen Isomorphismus gibt. Denn nach Definition
gibt es für ein zweites Faserprodukt $(\widetilde F,\widetilde{\pi_A},
\widetilde{\pi_B})$ genau eine Abbildung $h$ von $\widetilde F$ nach $F$ mit
$$\widetilde{\pi_A} = \pi_A\circ h,\ \ \widetilde{\pi_B} = \pi_B\circ h$$
und auch genau eine Abbildung $\tilde h:F\longrightarrow  \widetilde F$ mit
$$\pi_A = \widetilde{\pi_A} \circ \tilde h,\ \ \pi_B = \widetilde{\pi_B}\circ 
\tilde h.$$
Dann ist aber $h\circ\tilde h$ eine Abbildung von $F$ nach $F$ mit
$$\pi_A = pi_A\circ (h\circ\tilde h),\ \ \pi_B = \pi_B\circ (h\circ\tilde h),$$
was wegen der Eindeutigkeit aus der Definition zwangsläufig
$$h\circ\tilde h ={\rm Id}_F$$ 
nach sich zieht. Analog gilt auch
$$\tilde h\circ h = {\rm Id}_{\widetilde F}.$$

{\bf Schreibweise:} Für das Faserprodukt schreibt man meistens $A\times_SB,$
wobei in der Notation die Abbildungen $f_A$ unf $f_B$ unterdrückt werden.

\medskip

Ein Faserprodukt existiert immer. Wir können nämlich 
$$F:=\{(a,b)\in A\times B \mid f_A(a) = f_B(b)\} $$
wählen und für $\pi_A,\pi_B$ die Projektion auf den ersten beziehungsweise
zweiten Eintrag. 

Die Abbildung $h$ aus der Definition ist dann einfach $h(m) = (g_A(m), g_B(m)).$

Wir können $F$ auch hinschreiben als 
$$F=\bigcup_{s\in S} \left(f_A^{-1}(s) \times f_B^{-1}(s)\right),$$
also als Vereinigung der Produkte der Fasern von $f_A$ und $f_B$ über jeweils
demselben Element von $S.$ Das erklärt den Namen.

\begin{bsp} {\bf Spezialfälle}

{\rm 
\begin{itemize}
\item[a)] Wenn $S$ nur aus einem Element $s$ besteht, dann sind $f_A$ und $f_B$
konstant, und damit $A\times_SB = A\times B$ das mengentheoretische Produkt.
\item[b)] Wenn $A,B$ Teilmengen von $S$ sind und die Abbildungen $f_A,f_B$ 
einfach die Inklusionen, dann gilt
$$A\times_SB = \{(a,b)\in A\times B\mid a=b\} = \{(s,s)\mid s\in A\cap B\} 
\simeq A\cap B.$$
\end{itemize}
}
\end{bsp}

\section{Metrische Räume}

\begin{defini} {\bf Metrischer Raum} 

{\rm Ein {\it metrischer Raum}\index{metrischer Raum} ist eine Menge $X$ 
zusammen mit einer Abbildung $$d:X\times X\longrightarrow \mathbb R_{\geq 0},$$
sodass die folgenden Bedingungen erfüllt sind:
\begin{itemize}
\item $\forall x,y\in X: d(x,y) = d(y,x)$ (Symmetrie)
\item $\forall x\in X: d(x,x) = 0.$
\item $\forall x,y\in X: x\neq y\Rightarrow d(x,y) >0.$
\item $\forall x,y,z\in X: d(x,y) + d(y,z) \geq d(x,z).$ (Dreiecksungleichung)
\end{itemize}
Die Abbildung $d$ heißt dabei die {\it Metrik}.

Penibler Weise sollte man einen metrischen Raum als Paar $(X,d)$ schreiben.
Meistens wird das micht gemacht, aber Sie kennen diese Art der Schlamperei ja
schon zur Genüge\dots
}\end{defini}

\begin{bsp} \label{L_unendlich}{\bf LA und ANA lassen grüßen}

{\rm 
\begin{itemize}
\item [a)] Ein reeller Vektorraum mit einem Skalarprodukt 
$\langle\cdot,\cdot\rangle$ wird bekanntlich mit 
$$d(v,w) := \sqrt{\langle v-w,v-w\rangle} = |v-w|$$
zu einem metrischen Raum
\item[b)] Jede Menge $X$ wird notfalls durch 
$$d(x,y) = \left\{\begin{array}{rl} 1, & {\rm falls}\ x\neq y,\\
                                    0, & {\rm falls}\ x=y,\\
\end{array}\right.$$
zu einem metrischen Raum. Diese Metrik heißt die {\it diskrete Metrik}
\index{diskrete Metrik} auf $X.$
\item[c)] Es sei $X$ eine Menge und ${\cal B}(X)$ der Vektorraum der 
beschränkten reellwertigen Funktionen auf $X.$ Dann wird $X$ vermöge
$$d(f,g) := \sup\{|f(x)-g(x)|\mid x\in X\}$$
zu einem metrischen Raum. 

Anstelle der Norm aus einem Skalarprodukt benutzt man hier also die sogenannte
Maximumsnorm 
$$|f|_\infty := \sup\{|f(x)|\mid x\in X\},$$
um eine Metrik zu konstruieren. Diese kommt nicht von einem Skalarprodukt her,
wenn $X$ mindestens 2 Elemente hat.

Allgemeiner sei für eine Menge $X$ und einen metrischen Raum $(Y,e)$ die 
Menge
${\cal B}(X,Y)$ definiert als die Menge aller beschränkten Abbildungen von 
$X$ nach $Y$. Dabei heißt $f$ beschränkt, wenn ein $R\in \mathbb R$ 
existiert mit
$$\forall x_1,x_2\in X: e(f(x_1),f(x_2)) < R.$$
Dann wird ${\cal B}(X,Y)$ zu einem metrischen Raum vermöge
$$d(f,g) := \sup\{e(f(x),g(x))\mid x\in X\}.$$
\item[d)] Auf den rationalen Zahlen lässt sich für eine Primzahl $p$
auf folgende Art eine Metrik konstruieren:

Jede rationale Zahl $q\neq 0$ kann man schreiben als 
$p^{{\rm v}_p(q)} \cdot \frac ab,$ wobei $a,b\in \mathbb Z$ keine Vielfachen von 
$p$ sind. Dann ist $v_p(q)$ eindeutig bestimmt.

Wir setzen für zwei rationale Zahlen $x,y$
$$d_p(x,y) :=\left\{\begin{array}{rl}
0, & {\rm falls}\ x=y,\\
p^{-{\rm v}_p(x-y)}, & {\rm sonst}.\\ \end{array}\right.$$
Dies ist die sogenannte $p$-adische Metrik auf $\mathbb Q.$
\end{itemize}
}\end{bsp}

\begin{defini} {\bf Folgen und Grenzwerte}

{\rm 
\begin{itemize}
Es sei $(X,d)$ ein metrischer Raum. 
\item[a)]
Eine Folge $(x_n)_{n\in \mathbb N}$
in $X$ heißt {\it konvergent gegen den Grenzwert} $y\in X,$ falls
$$\lim_{n\to\infty} d(x_n,y) = 0.$$
Natürlich ist $y$ hierbei eindeutig bestimmt.
\item[b)] Eine Cauchyfolge\footnote{Augustin-Louis Cauchy, 1789-1857} in $X$ 
ist eine Folge $(x_n)_{n\in \mathbb N}$ mit
$$\forall \varepsilon>0 : \exists N\in \mathbb N: \forall m,n>N:
d(x_m,x_n) < \varepsilon.$$
Jede konvergente Folge ist eine Cauchyfolge.
\item[c)] $X$ heißt {\it vollständig}\index{vollständig}, wenn jede
Cauchyfolge in $X$ einen Grenzwert in $X$ hat.
\end{itemize}
}
\end{defini}

\begin{bsp} {\bf Schatten der Vergangenheit}

{\rm 
Jeder endlichdimensionale euklidische Vektorraum ist vollständig. 

Die Konvergenz einer Folge in ${\cal B}(X)$ mit der $\infty$-Norm ist einfach 
die gleichmäßige Konvergenz im Sinne der Analysis. Insbesondere ist
${\cal B}(X)$ mit dieser Metrik vollständig.

$\mathbb Q$ mit der $p$-adischen Metrik ist nicht vollständig. Man kann einen
Körper $\mathbb Q_p$ konstruieren, der $\mathbb Q$ enthält, auf dem eine
Metrik definiert ist, die die $p$-adische fortsetzt, und in dem sich jedes 
Element durch eine $p$-adische Cauchyfolge in $\mathbb Q$ approximieren 
lässt. Damit wird ein arithmetisch wichtiges Pendant zu den reellen Zahlen 
geschaffen, die sich ja auch konstruieren lassen als (archimedische) 
Cauchyfolgen in $\mathbb Q$ modulo Nullfolgen. 
}
\end{bsp}
\begin{defini} {\bf Isometrien}

{\rm Es seien $(X,d)$ und $(Y,e)$ zwei metrische Räume. Eine 
{\it abstandserhaltende Abbildung} von $X$ nach $Y$  ist eine Abbildung
$f:X\longrightarrow Y,$ für die gilt:
$$\forall x_1,x_2 \in X: d(x_1,x_2) = e(f(x_1),f(x_2)).$$
Solche Abbildungen sind immer injektiv. Eine surjektive abstandserhaltende
Abbildung heißt eine {\it Isometrie}. 

Die Menge der Isometrien von $X$ nach $X$ ist eine Untergruppe der { 
symmetrischen Gruppe} 
${\rm Sym}(M)$ aller Bijektionen von $X$ nach $X$.
}\end{defini}

Aber eigentlich ist das momentan kein Begriff, der unsere Aufmerksamkeit zu 
stark in Anspruch nehmen sollte. 

\begin{defini} {\bf Kugeln}

{\rm Es seien $(X,d)$ ein metrischer Raum und $x\in X$ sowie $r>0$ eine reelle 
Zahl. Dann heißt
$$B_r(x) := \{y\in X \mid d(x,y)<r\}$$
die {\it offene Kugel} vom Radius $r$ um den Mittelpunkt $x.$ 

{\bf Vorsicht:} Weder $x$ noch $r$ müssen durch die Menge $B_r(x)$ eindeutig 
bestimmt sein.
Wenn zum Beispiel $X$ mit der diskreten Metrik ausgestattet ist, so ist 
$X=B_2(x) = B_3(y)$ für alle $x,y\in X.$
}
\end{defini}

\begin{defini} \label{stetig1}{\bf Stetigkeit}

{\rm 
Es seien $(X,d)$ und $(Y,e)$ zwei metrische Räume. Dann heißt eine
Abbildung $f:X\longrightarrow Y$ {\rm stetig}, falls für jedes 
$x\in X$ und jedes $\varepsilon>0$ ein $\delta>0$ existiert, sodass
$$f(B_\delta(x)) \subseteq B_\varepsilon(f(x)).$$
In Worten: Jede offene Kugel um $f(x)$ enthält das Bild einer offenen Kugel 
um $x$.
}
\end{defini}

So ist zum Beispiel jede Abbildung von $X$ nach $Y$ stetig, wenn auf $X$ die 
diskrete Metrik vorliegt. Denn dann ist ja $\{x\} = B_{\frac12}(x)$ im Urbild
jeder offenen Kugel um $f(x)$ enthalten.

\begin{bsp}{\bf Noch einmal die Analysis}

{\rm Es seien $X,Y$ metrische Räume. Dann bezeichnen wir mit 
${\cal C}(X,Y)$ die Menge aller stetigen Abbildungen von $X$ nach $Y,$ und 
mit ${\cal C}_0(X,Y)$ die Menge aller beschränkten stetigen Abbildungen
von $X$ nach $Y.$ 

Wenn $Y$ vollständig ist, dann ist auch ${\cal C}_0(X,Y)$ (als Teilraum von
${\cal B}(X,Y)$) vollständig. 

Im Fall $Y=\mathbb R$ lässt man das $Y$ auch häufig weg und schreibt nur 
${\cal C}(X)$ bzw.\ ${\cal C}_0(X).$
}\end{bsp}

\begin{hilfs}\label{Normen}{\bf Normen}

Es sei $V=\mathbb R^n$ mit dem Standardskalarprodukt versehen und $N$ eine
Norm auf $V,$ d.h. $N:V\longrightarrow \mathbb R_{\geq 0}$ erfüllt 
\begin{itemize}
\item $\forall v\in V:N(v) = 0 \iff v=0$ (Positivität),
\item $\forall a\in \mathbb R,v\in V: N(av) = |a|N(v)$ (Homogenität),
\item $\forall v,w\in V: N(v+w) \leq N(v) + N(w)$ (Dreicksungleichung).
\end{itemize}
Dann ist $N$ stetig bezüglich der Standardmetrik.
\end{hilfs}
{\it Beweis.} Das Urbild von $(-\varepsilon , \varepsilon)$ unter $N$ ist 
konvex, d.h. 
$$\forall v,w\in V: [N(v),N(w)\leq \varepsilon\Rightarrow 
\forall \lambda\in [0,1]: N(\lambda v + (1-\lambda) w) \leq \varepsilon].$$
Das folgt sofort aus den drei aufgelisteten Eigenschaften der Normabbildung. 

Wegen der Positivität und der Homogenität gibt es eine Konstante $c>0$ 
(abhän\-gig von $\varepsilon$), sodass die Vektoren 
$\pm c e_i,\ 1\leq i\leq n,$
in $N^{-1}(-\varepsilon , \varepsilon)$ liegen. Dabei ist $\{e_1,\dots ,e_n\}$ 
die Standardbasis von $\mathbb R^n.$

Es sei $v\in B_{c/\sqrt n}(0)\subseteq V,$ d.h. $v= \sum_{i=1}^n a_i c e_i, 
\sum_i a_i^2 < 1/n.$ Dann ist aber die Summe $\sum_i|a_i| < 1.$ Für
$\alpha = \sum |a_i|$ ist also $v=\alpha\cdot \sum_i \frac{a_i}{\alpha}
ce_i$ das $\alpha$-fache einer Konvexkombination 
von $\pm ce_1,\dots \pm ce_n.$ Wegen  $|\alpha|<1$ liegt wegen der 
Homogenität von $N$ auch $v$ im Urbild von $(-\varepsilon, \varepsilon)$, und
damit liegt die offene Kugel $B_{c\sqrt n}(0)$ im Urbild: $N$ ist stetig im 
Ursprung. 

Nun seien $x\in V$ beliebig und $\delta>0$ vorgegeben. Dann gibt es nach dem 
eben gesehenen ein $\varepsilon>0$ mit 
$$\forall y\in B_\varepsilon (0) : |N(y| < \delta.$$
Für $y\in B_\varepsilon (0)$ gilt demnach wegen
$N(x) = N(x+y-y)\geq N(x+y) + N(y):$
$$-N(y) \leq  N(x+y) - N(x) \leq N(y),$$
und daher $N(B_\varepsilon(x)) \subseteq B_\delta(N(x)).$ 

Das zeigt die Stetigkeit von $N.$ \hfill{$\bigcirc$}

\begin{defini} {\bf Die Topologie eines metrischen Raums}

{\rm Es sei $(X,d)$ ein metrischer Raum. Eine Teilmenge $A\subseteq X$ heißt
{\it offen} \index{offen}, falls für jedes $x\in A$ eine reelle Zahl $r>0$
existiert, sodass $B_r(x)\subseteq A$ gilt. 

Die Gesamtheit aller offenen Mengen in $X$ heißt die {\it Topologie} von 
$(X,d).$
}

\end{defini}

\begin{bem} {\bf Eigenschaften}

{\rm Die offenen Mengen eines metrischen Raums haben die folgenden beiden
Eigenschaften: beliebige Vereinigungen und endliche Durchschnitte von offenen
Mengen sind wieder offen.

Eine Abbildung $f:X\longrightarrow Y$ zwischen metrischen Räumen ist genau 
dann stetig, wenn für jede offene Teilmenge $U\subseteq Y$ das Urbild
$f^{-1}(U)$ offen in $X$ ist.

Es kann sehr viele verschiedene Metriken auf $X$ geben, die zur selben 
Topologie führen. So stimmen zum Beispiel für zwei Normen auf $\mathbb R^n$
die zugehörigen Topologien überein, was im Wesentlichen aus Hilfssatz 
\ref{Normen} folgt. Dieser Hilfssatz sagt nämlich, dass die Identität auf
$\mathbb R^n$ stetig ist, wenn wir auf Seiten des Definitionsbereichs die
euklidische Standardlänge als Norm benutzen, und auf Bildseite die Norm $N.$
Auch in der anderen Richtung ist die Identität stetig (das muss man noch 
beweisen!), und das impliziert die Gleichheit der zugehörigen Topologien.   
}
\end{bem}




\chapter{Topologische Grundbegriffe}

\section{Topologische Räume und ein paar Konstruktionen}

\begin{defini} {\bf Topologischer Raum}

{\rm Ein {\it topologischer Raum} \index{topologischer Raum} ist eine Menge 
$X$, für die eine Teilmenge 
${\cal T} \subseteq {\cal P} (X)$ mit folgenden Eigenschaften ausgewählt 
wurde:
\begin{itemize}
\item $\emptyset, X\in {\cal T}$
\item $\forall A,B\in {\cal T}: A\cap B\in {\cal T}$
\item $\forall {\cal S}\subseteq{\cal T}: \bigcup_{A\in {\cal S}}A \in {\cal T}.$
\end{itemize}
Hierbei heißt $\cal T$ die Topologie auf $X$, und die Elemente von 
$\cal T$ sind die \index{offen} {\it offenen Mengen} des 
topologischen Raums $(X,{\cal T}).$ 

Die Mengen $X\smallsetminus A, A\in {\cal T},$ heißen {\it abgeschlossene
Teilmengen}\index{abgeschlossen} von $X$.


}\end{defini}
\begin{bsp}{\bf Alte Bekannte}

{\rm Die offenen Mengen eines metrischen Raums $X$ bilden eine Topologie auf 
$X$.


Ist $X$ eine beliebige Menge, so ist die Potenzmenge ${\cal P}(X)$  eine
Topologie auf $X.$ Sie ist sogar die Topologie zu einer Metrik auf $X$, zur 
diskreten Metrik nämlich. Sie heißt die {\it diskrete Topologie}).

Auch $\{\emptyset, X\}$ ist eine Topologie auf $X.$

Auf $\{0,1\}$ gibt es die Topologie $\{\emptyset, \{0\}, \{0,1\}\}.$
Diese kommt nicht von einer Metrik her.
}

\end{bsp}
\begin{defini}{\bf Inneres, Abschluss und der zu schmale Rand}

{\rm Es seien $X$ ein topologischer Raum und $A\subseteq X$ eine offene 
Teilmenge. Das {\it Innere} ${\stackrel\circ A}$ von $A$ ist definiert als die
Vereinigung
$$\stackrel\circ A := \bigcup_{U\subseteq_o A} U,$$
wobei das Symbol $\subseteq_o$ bedeutet, dass $U$ eine (in $X$) offene 
Teilmenge von $A$ ist.

$\stackrel\circ A$ ist offen.

Der {\it Abschluss} $\bar A$ von $A$ ist definiert als der Durchschnitt aller 
abgeschlossenen Teilmengen von $X$, die $A$ enthalten.

$\bar B$ ist abgeschlossen.

Der {\it Rand} von $A$ ist die Menge 
$\partial A:=\bar A\smallsetminus \stackrel\circ A.$

}
\end{defini}
\begin{defini} \label{dicht}{\bf Dichtheit, Diskretheit}

{\rm Es sei $X$ ein topologischer Raum. Eine Teilmenge $D\subseteq X$ heißt 
{\it dicht}\index{dicht}, wenn ihr Abschluss ganz $X$ ist. 

Jede Teilmenge ist also dicht in ihrem Abschluss, wenn man diesen wie in 
\ref{Spurtopologie} als topologischen Raum betrachtet.

Eine Teilmenge $D\subseteq X$ heißt \index{diskret}{\it diskret}, wenn
jeder Punkt $x\in X$ eine Umgebung besitzt, die mit $D$ endlichen Durchschnitt 
hat.

Für metrische Räume heißt das gerade, dass $D$ keinen Häufungspunkt 
besitzt. 
}
\end{defini}


\begin{defini}{\bf Umgebungen, Basis einer Topologie}

{\rm Es sei $(X,{\cal T})$ ein topologischer Raum.
\begin{itemize}
\item[a)] Für $x\in X$ heißt eine Teilmenge $A\subset X$ eine 
{\it Umgebung}\index{Umgebung} von $x,$ falls eine offene Teilmenge 
$U\subseteq X$ existiert mit $x\in U\subseteq A.$ Ist $A$ selbst schon offen, 
so heißt es eine {\it offene Umgebung} von $x$ (falls $x\in A$).
\item[b)] Eine Teilmenge ${\cal B}\subseteq {\cal T}$ heißt eine {\it Basis}
von ${\cal T},$ falls jedes Element von $\cal T$ sich schreiben lässt als
Vereinigung von Elementen aus ${\cal B}.$

(So sind zum Beispiel die offenen Kugeln $B_r(x)$ eine Basis der Topologie auf 
einem metrischen Raum.)

$\cal B$ heißt eine {\it Subbasis} von $\cal T$, falls sich jedes 
$U\in {\cal T}$ als Vereinigung von endlichen Durchschnitten von Elementen aus
${\cal B}$ schreiben lässt.
\item[c)] Für $x\in X$ heißt eine Menge $\cal U$ von Umgebungen von $x$ 
eine {\it Umgebungsbasis} von $x,$ wenn jede Umgebung von $x$ ein Element von 
$\cal U$ als Teilmenge enthält.
\end{itemize}}
\end{defini}
\begin{bem} {\bf Einsichtig}

{\rm Eine Teilmenge ${\cal B }\subseteq {\cal T}$ ist genau dann eine Basis 
der Topologie $\cal T$, wenn sie für jedes $x\in X$ eine Umgebungsbasis 
enthält.

Für jede Teilmenge $\cal B$ von ${\cal P}(X)$ gibt es genau eine Topologie,
die $\cal B$ als Subbasis besitzt. Sie ist die {\it von ${\cal B}$ erzeugte } 
Topologie, und besitzt 
$$\{U_1\cap \dots \cap U_n \mid n\in\mathbb N, U_i\in {\cal B}\}$$
als Basis.
}\end{bem}

\begin{defini} {\bf Feinheiten}

{\rm Wenn ${\cal T}_1,{\cal T}_2$ zwei Topologien auf einer Menge $X$ sind, so
heißt ${\cal T}_1$ {\it feiner} als ${\cal T}_2,$ wenn ${\cal T}_2\subseteq
{\cal T}_1,$ also wenn ${\cal T}_1$ mehr offene Mengen besitzt als ${\cal T}_2.$

Die feinste Topologie auf $X$ ist also die diskrete, während 
$\{\emptyset, X\}$ die gröbste Topologie auf $X$ ist.

Zu je zwei Topologien gibt es eine gemeinsame Verfeinerung. Die gröbste 
gemeinsame Verfeinerung ist die Topologie, die die Vereinigung der beiden 
gegebenen als Subbasis besitzt.}
\end{defini}

\begin{defini} \label{Spurtopologie}{\bf Teilräume und Produkte}

{\rm
\begin{itemize}
\item[a)] Es seien $X$ eine Menge und $(Y,{\cal S})$ ein Topologischer Raum. 
Weiter sei $f:X\longrightarrow Y$ eine Abbildung. Für zwei Teilmegen
$A,B\subseteq Y$ gilt
$$f^{-1}(A\cup B) = f^{-1}(A)\cup f^{-1}(B), \ \ 
f^{-1}(A\cap B) = f^{-1}(A)\cap f^{-1}(B).$$
Das zeigt im wesentlichen bereits, dass 
$${\cal T}:= \{ f^{-1}(U) \mid U\in {\cal S}\}$$
eine Topologie auf $X$ ist. Man nennt sie die \index{Spurtopologie}
{\it Spurtopologie} auf $X$ 
(bezüglich $f$).

Damit können wir unheimlich viele neue topologische Räume konstruieren.
(Tun Sie das!) 
\item[b)] Ist speziell $X\subseteq Y$ und $f$ die Einbettung dieser Teilmenge,
so nennt man $X$ (mit der Spurtopologie) einen {\it Teilraum} von $Y.$

Eine Teilmenge $A$ von $X$ ist genau dann offen bezüglich der Spurtopologie, 
wenn es eine offene Teilmenge $U$ von $Y$ gibt mit $A= U\cap X.$
\item[c)] Sind $X,Y$ zwei topologische Räume, so definieren wir auf 
$X\times Y$ die {\it Produkttopologie} \index{Produkttopologie}, indem wir 
als Basis die Produkte $U\times V$ für offene $U\subseteq X$ und
$V\subseteq Y$ verwenden. 
\end{itemize}
}
\end{defini}

\begin{defini} \label{Quotienten}{\bf Quotiententopologie}

{\rm Es sei $X$ ein topologischer Raum und $\equiv$ eine \"Aquivalenzrelation
auf $X.$ Dann wird auf dem Raum $X/\equiv$ der \"Aquivalenzklassen von $X$ 
eine Topologie eingeführt, indem man für offenes $U\subseteq X$ die
Menge 
$$\{[u]\mid u\in U\}$$
aller \"Aquivalenzklassen von Elementen aus $U$ zur offenen Menge erklärt
und die davon erzeugte Topologie verwendet. Diese Topologie heißt die
{\it Quotiententopologie}\index{Quotiententopologie} auf $X/\equiv.$

Damit bekommen wir zum Beispiel eine Topologie auf dem projektiven Raum
$\mathbb P^n(\mathbb R)$ oder $\mathbb P^n(\mathbb C).$ Die Topologie auf 
$\mathbb P^1(\mathbb C)$ verdient hier historisch und didaktisch besondere
Aufmerksamkeit. Eine Teilmenge $A\subseteq\mathbb P^1(\mathbb C) = \mathbb C
\cup\{\infty\}$ ist genau dann offen, wenn $A\cap \mathbb C$ offen ist und wenn
zusätzlich im Fall $\infty\in A$ ein $R>0$ existiert mit 
$$\{z\in \mathbb C \mid |z|>R \} \subseteq A.$$

}

\end{defini}
\section{Wichtige Eigenschaften topologischer Räume}
\begin{defini} {\bf Kompaktheit}

{\rm Ein topologischer Raum $X$ heißt \index{kompakt}{\it kompakt}, wenn
jede Überdeckung $X=\bigcup_{i\in I}U_i$ von $X$ durch offene Mengen eine
endliche Teilüberdeckung enthält: 
$$\exists n\in \mathbb N, i_1,\dots i_n\in I: X=\bigcup_{k=1}^n U_{i_k}.$$
Genauso heißt eine Teilmenge von $X$ kompakt, wenn sie bezüglich der 
Spurtopologie (der Inklusion) kompakt ist.

Anstelle des Begriffs „kompakt“ wird auch gelegentlich 
„überdeckungsendlich“ verwendet. Es ist nicht ganz einheitlich, ob 
zur Kompaktheit auch die Eigenschaft, hausdorff'sch zu sein (siehe später), 
gehört oder 
nicht. Wir wollen hier Kompaktheit so verstehen wie gesagt.
}
\end{defini}
\begin{bem} {\bf Kompakta in metrischen Räumen}

{\rm Ein kompakter metrischer Raum $X$ ist sicher beschränkt, denn 
$$X=\bigcup_{n\in \mathbb N} B_n(x)$$
gilt für jedes $x\in X,$ und das ist eine offene Überdeckung von $X.$

Eine kompakte Teilmenge $A$ eines metrischen Raums $X$ ist 
auch abgeschlossen. Ist nämlich $x\in X\smallsetminus A$ im Komplement von 
$A,$ so ist 
$$A\subseteq \bigcup_{n\in \mathbb N} \{y\in X\mid d(y,x)> 1/n \}$$
eine offene Überdeckung von $A,$ und damit langen endlich viele dieser
Mengen, um $A$ zu überdecken. Es ist also 
$$A\subseteq \{y\in X\mid d(y,x)> 1/n \}$$
für ein festes $n\in \mathbb N,$ und daher ist $B_{1/n}(x)$ in 
$X\smallsetminus A$ enthalten.

Ein etwas feineres Argument zeigt, dass ein kompakter metrischer Raum sogar
vollständig ist.

Eine abgeschlossene Teilmenge $A$ eines kompakten Raums $X$ ist kompakt, denn
für jede Überdeckung $\ddot U$ von $A$ durch offene Teilmengen von $X$ 
ist $\ddot U\cup\{X\smallsetminus A\}$ eine offene Überdeckung von $X$, also
langen endlich viele davon, um $X$ zu überdecken, und von diesen endlich
vielen kann man notfalls $X\smallsetminus A$ weglassen, um eine endliche 
Teilüberdeckung von $A$ zu erhalten.


}

\end{bem}

\begin{satz} {\bf \`a la Heine\footnote{Heinrich-Eduard Heine, 1821-1881}-Borel\footnote{Emile Borel, 1871-1956}}

Es sei $X$ ein vollständiger metrischer Raum, in dem sich jede beschränkte
Menge $A\subseteq X$ für jedes $\varepsilon>0$ durch endlich viele Mengen
von Durchmesser $\leq \varepsilon$ überdecken lässt. 

Dann gilt:
 
Eine Teilmenge $A\subseteq X$ ist genau dann kompakt, wenn sie
abgeschlossen und beschränkt ist.

\end{satz}

{\it Beweis:} In der einen Richtung haben wir es schon gesehen: ein Kompaktum 
in einem metrischen Raum ist abgeschlossen und beschränkt.

Sei umgekehrt $A\subseteq X$ abgeschlossen und beschränkt. Weiter sei 
$\ddot U$ eine offene Überdeckung von $A.$ Nehmen wir an, es gebe in 
$\ddot U$ keine endliche Teilüberdeckung von $A.$

In $A$ gibt es eine abgeschlossene Teilmenge $A_1\subseteq A$ von Durchmesser
$\leq 1,$ die sich nicht durch endlich viele $U\in \ddot U$ überdecken 
lässt, da $A$ ja nach Voraussetzung eine endliche Vereinigung von Mengen vom
Durchmesser $\leq 1$ ist.  

Wir wählen -- wieder unter Ausnutzung der Eigenschaft von $X$ -- sukzessive 
Teilmengen 
$$A\supseteq A_1\supseteq A_2\dots \supseteq A_k\supseteq \dots$$
derart, dass $A_k$ Durchmesser $\leq 1/2^k$ hat und sich nicht durch endlich 
viele $U\in \ddot U$ überdecken lässt. 

Für jedes $i\in \mathbb N$ wählen wir nun ein Element $x_i\in A_i$
(so etwas gibt es, nicht wahr?).

Dann ist $(x_i)_{i\in \mathbb N}$ eine Cauchy-Folge, denn 
$$d(x_i,x_k) \leq 1/2^{\max (i,k)}.$$

Also konvergiert die Folge gegen ein $x\in A,$ da $X$ vollständig und $A$
abgeschlossen ist.

Dieses $x$ liegt also in einem $U\in\ddot U,$ und da $U$ offen ist, gibt es 
ein $\varepsilon>0,$ sodass $B_\varepsilon(x)\subseteq U$ gilt. Daher liegt für
großes $k$ auch $A_k$ ganz in $U$, was der Konstruktion der Teilmengen
$A_k$ widerspricht. 

Diese ist also nicht möglich, un damit ist $A$ eben doch kompakt. 
\hfill{$\bigcirc$}


\begin{bem} \label{einige Kompakta} {\bf Beispielmaterial}

{\rm 

\begin{itemize}
\item[a)] Als Spezialfall erhalten wir den klassischen Satz von Heine-Borel,
der sagt, dass eine Teilmenge von $\mathbb R^n$ genau dann überdeckungsendlich
ist, wenn sie abgeschlossen und beschränkt ist. 

Heine hat diesen Satz 1872 
für Intervalle in $\mathbb R$ benutzt, um zu zeigen, dass eine stetige 
Funktion auf einem beschränkten und abgeschlossenen Intervall gleichmäß
ig stetig ist.

\item[b)]
Ein metrischer Raum, in dem diese \"Aquivalenz nicht gilt, ist zum Beispiel
der folgende:

Es sei $X:={\cal C}_0(\mathbb N)$ der Raum der beschränkten Funktionen 
auf $\mathbb N$ (siehe \ref{L_unendlich}). 

Für $n\in\mathbb N$ sei $\delta_n$ die Funktion auf $\mathbb N,$ die
auf $n$ den Wert 1 annimmt, und sonst den Wert 0. Die Menge
$$D:=\{ \delta_n \mid n\in \mathbb N\}$$
ist eine beschränkte, abgeschlossene Teilmenge von $X.$ Aber kompakt ist sie 
nicht, denn in $B_{1/2}(\delta_n)$ liegt kein weiteres 
$\delta_k, k\in \mathbb N,$ und so ist
$$D = \bigcup_{n\in \mathbb N} B_{1/2}(\delta_n)$$
eine offene Überdeckung von $D$ ohne endliche Teilüberdeckung.

In der Funktionalanalysis spielen ähnliche Räume eine wichtige Rolle, und
insbesondere die Frage, wann die abgeschlossene Einheitskugel in einem 
normierten Vektorraum kompakt ist.

\item[c)] Es gibt auch topologische Räume, in denen {\it jede Teilmenge} 
kompakt ist, egal ob offen, abgeschlossen, keins von beiden\dots

Als Beispiel hierfür nehme ich eine (beliebige!) Menge $X$ und versehe sie
mit der {\it koendlichen} Topologie. Dies heißt, dass neben der leeren Menge
genau die Mengen offen sind, deren Komplement in $X$ endlich ist.

Klar: hier ist alles kompakt. Denn für $A\subseteq X$ und offenes 
$U\neq\emptyset$ 
überdeckt $U$ bereits alles bis auf endlich viele Elemente von $A.$

\item[d)] Eine wichtige Beispielklasse für kompakte Räume sind die
projektiven Räume $\mathbb P^n(\mathbb R)$ und $\mathbb P^n(\mathbb C)$.

Im Fall $n=1$ sieht man die Kompaktheit sehr schön wie folgt: Ist $\dots U$ 
eine offene Überdeckung von $X=\mathbb P^1(K)$ (mit $K=\mathbb R$ oder 
$\mathbb C$), so gibt es darin eine Menge $U_\infty\in \dots U,$ sodass
$\infty\in U_\infty.$ Das Komplement von $U_\infty$ ist nach Konstruktion der
opologie auf $X=K\cup\{\infty\}$ offen und beschränkt (siehe 
\ref{Quotienten} d)) und daher kompakt wegen Heine-Borel. Also reichen endlich viele
weitere Elemente aus $\dots U$, um $K\smallsetminus U_\infty$ zu überdecken.

\end{itemize}
}
\end{bem}

\begin{defini}\label{Zusammenhang} {\bf zusammenhängend}

{\rm Es sei $X$ ein topologischer Raum. Dann heißt $X$ {\it 
zusammenhängend}\index{zusammenhängend}, wenn $\emptyset$ und $X$ die 
einzigen Teilmengen von $X$ sind, die sowohl offen als auch abgeschlossen sind.
Das ist äquivalent dazu, dass es keine Zerlegung von $X$ in zwei nichtleere,
disjunkte und offenen Teilmengen gibt.

Eine Teilmenge $A\subseteq X$ heißt zusammenhängend, wenn sie bezüglich 
der
Teilraumtopologie zusammenhängend ist, also genau dann, wenn sie nicht 
in der Vereinigung zweier offener Teilmengen von $X$ liegt, deren Schnitte 
mit $A$ nichtleer und disjunkt sind.

Die Vereinigung zweier zusammenhängender Teilmengen mit nichtleerem
Durchschnitt ist wieder zusammenhängend.
}
\end{defini}

\begin{bsp} {\bf Intervalle}

{\rm Die zusammenhängenden Teilmengen von $\mathbb R$ sind gerade die 
Intervalle, egal ob offen oder abgeschlossen oder \dots. Dabei werden auch 
die leere Menge und einelementige Mengen als Intervalle gesehen.

Ist nämlich $A\subseteq \mathbb R$ zusammenhängend und sind 
$x<y$ beide in $A$, so liegt auch jeder Punkt $z$ zwischen $x$ und $y$ in $A$,
da sonst 
$$A = (A\cap(-\infty,z)) \bigcup (A\cap (z,\infty))$$
eine disjunkte, nichttriviale, offene Zerlegung von $A$ wäre.

Ist umgekehrt $A$ ein Intervall, so sei $A=B\cup C$ eine nichttriviale
disjunkte Zerlegung. Ohne Einschränkung gebe es ein $b_0\in B$ und ein 
$c_0\in C$ mit $b_0<c_0.$ 

Es sei $z:= \sup\{b\in B\mid b<c_0\}.$ Dies liegt in $A,$ und damit auch in 
$B$ oder $C.$ Wäre $z\in B$ und $B$ offen in $A$, so müsste es ein $r>0$ 
geben mit
$$\forall a\in A: |z-a| < r \Rightarrow z\in B.$$
Also kann $z$ nicht zu $B$ gehören, wenn dies offen in $A$ ist, denn es gibt 
Elemente $c\in C, c>z,$ die beliebig nahe an $z$ dran liegen. 

Wäre $z\in C$ und $C$ offen in $A$, so gäbe es ein $r>0$, sodass
$$\forall a\in A: |z-a| < r \Rightarrow z\in C.$$
Das wiederum geht nicht, denn $z$ ist das Supremum einer Teilmenge von $B.$

Also sind weder $B$ noch $C$ offen in $A$, und das zeigt, dass $A$ 
zusammenhängend ist.
}
\end{bsp}

\begin{defini} {\bf Zusammenhangskomponenten}

{\rm Es sei $X$ ein topologischer Raum. Wir nennen zwei Punkte $x,y$ in $X$
{\it äquivalent}, falls es eine zusammenhängende Teilmenge von $X$ gibt, 
die beide enthält. Dies ist tatsächlich eine \"Aquivalenzrelation:
\begin{itemize}
\item $x\simeq x$ ist klar für alle $x\in X,$ denn $\{x\}$ ist 
zusammenhängend. 
\item Symmetrie ist auch klar, nicht wahr?
\item Transitivität: Es seien $x\simeq y$ und $y\simeq z$, dann gibt es 
zusammenhängende $A,B\subseteq X,$ sodass $x,y\in A$ und $y,x\in B$. 
Aber $A\cup B$ ist auch zusammenhängend, denn aus $A\cup B = U\cup V$ (offene
Zerlegung) folgt
$A=(A\cap U) \cup (A\cap V)$ und analog für $B$, wir hätten also 
disjunkte offene Überdeckungen von $A$ und $B$, und damit folgt OBdA 
$A\subseteq U, A\cap V = \emptyset.$ Genauso ist auch $B$ in einer der beiden 
Mengen enthalten und hat mit der anderen leeren Schnitt. Aus 
$B\subseteq U$ folgt $V=\emptyset,$ während aus $B\subseteq V$ folgt, dass
$U$ und $V$ nicht disjunkt sind: beide enthalten $y.$
\end{itemize}
Die \"Aquivalenzklasse ovn $x$ heißt die {\it Zusammenhangskomponente} von 
$x$. Diese ist weder zwangsläufig offen noch zwangsläufig abgeschlossen.
}
\end{defini}

\begin{defini}{\bf Hausdorff'sch}

{\rm Ein topologischer Raum $X$ heißt \index{hausdorff'sch}
{\it hausdorff'sch}\footnote{Felix Hausdorff, 1868-1942}, wenn je zwei
Punkte $x\neq y$ in $X$ disjunkte Umgebungen haben.

Man sagt dann auch, $X$ erfülle das Trennungsaxiom $T2:$ verschiedene Punkte
lassen sich durch Umgebungen trennen.}

\end{defini}
 
\begin{bem} {\bf Vererbung}

{\rm Wenn $X$ hausdorff'sch ist, so auch jeder Teilraum von $X.$

Jeder metrische Raum ist hausdorff'sch.

Das Produkt zweier Hausdorffräume ist wieder hausdorff'sch.

Nicht hausdorff'sch ist beispielsweise ein Raum $X$ mit mindestens zwei 
Elementen und der Topologie $\{\emptyset, X\}.$
}
\end{bem}

\begin{bem}\label{fuer Liouville} {\bf Kompakta in Hausdorffräumen}

{\rm Jedes Kompaktum $K$ in einem Hausdorffraum $X$ ist abgeschlossen. Denn:
Ist $x\in X\smallsetminus K$, so gibt es für jedes $k\in K$ disjunkte
offene Umgebungen $U_k$ von $k$ und $V_k$ von $x$. Es ist $\dots U := \{ U_k\mid k\in K\}$ eine offene Überdeckung von $K,$ und wegen der Kompaktheit 
gibt es endlich viele $k_1,\dots ,k_n$ in $K,$ sodass 
$$K \subseteq \bigcup_{i=1}^n U_{k_i}.$$
Dazu disjunkt ist $\bigcap_{i=1}^n V_{k_i},$ aber das ist eine offene Umgebung
von $x$. Also liegt $x$ nicht im Abschluss von $K.$
}
\end{bem}


\section{Stetigkeit}

\begin{defini} {\bf Stetige Abbildungen}

{\rm Eine Abbildung $f:X\longrightarrow Y$ zwischen zwei topologischen Räumen
heißt {\it stetig}\index{stetig}, falls für jede offene Teilmenge $U$
von $Y$ das Urbild $f^{-1}(U)$ in $X$ offen ist.

Wir hatten dies bei metrischen Räumen als äquivalent zur klassischen 
$\delta-\varepsilon$-Definition gesehen.

Wie bei metrischen Räumen werden wir mit ${\cal C}(X,Y)$ die Menge aller 
stetigen Abbildung zwischen den topologischen Räumen $X$ und $Y$ bezeichnen.

Eine stetige Abbildung, die bijektiv ist, und deren Umkehrabbildung auch
stetig ist, heißt ein {\it Homöomorphismus}. Zwei topologische Räume,
zwischen denen es einen Homöomorphismus gibt, heißen kreativer Weise
{\it homöomorph}.
}

\begin{bem} {\bf Sysiphos\footnote{Ignatz Sysiphos, -683- -651}}

{\rm In der Topologie betrachtet man zwei homöomorphe topologische Räume 
als
im Wesentlichen gleich. Eine Eigenschaft eines topologischen Raums $X$ 
heißt eine {\it topologische Eigenschaft}, wenn jeder zu $X$ homöomorphe
Raum diese Eigenschaft auch hat. Kompaktheit, Zusammenhang, Hausdorffizität
sind solche Eigenschaften. Beschränktheit oder Vollständigkeit eines 
metrischen Raums ist keine topologische Eigenschaft.

Natürlich möchte man eine Übersicht gewinnen,
wann zwei topologische Räume homöomorph sind, oder welche 
Homöomorphieklassen es insgesamt gibt. Das ist in dieser Allgemeinheit ein
aussichtsloses Unterfangen. Es gibt (mindestens) zwei Möglichkeiten, die
Wünsche etwas abzuschwächen: man kann sich entweder auf etwas speziellere
topologische Räume einschränken oder den Begriff des Homöomorphismus
ersetzen.

Das erstere passiert zum Beispiel bei der Klassifikation der topologischen
Flächen.

Für das zweitere bietet sich der Begriff der Homotopie an.

Auf beides kommen wir später noch zu sprechen.

Oft genug ist es sehr schwer nachzuweisen, dass zwei gegebene Räume nicht 
zueinander homöomorph sind. Wenn ich keine bistetige Bijektion finde, sagt 
das vielleicht mehr über mich aus als über die Räume. Hier ist es 
manchmal hilfreich, topologischen Räumen besser greifbare Objekte aus anderen
Bereichen der Mathematik zuordnen zu können, die für homöomorphe Räume 
isomorph sind, und wo dies besser entschieden werden kann. Das ist eine 
Motivation dafür, algebraische Topologie zu betreiben oder allgemeiner
eben Funktoren von der Kategorie der topologischen Räume in andere
Kategorien zu untersuchen. 
}
\end{bem}
\end{defini}

\begin{bem} {\bf Ringkampf}

{\rm 

\begin{itemize}

\item[a)] Die Identität auf $X$ ist stets ein Homöomorphismus (wenn man 
nicht zwei verschiedene Topologien benutzt\dots). Eine konstante Abbildung ist 
immer stetig. 

\item[b)]
Die Verknüpfung zweier stetiger Abbildungen $f:X\longrightarrow  Y, 
g:Y\longrightarrow Z$ ist wieder stetig. 

Insbesondere zeigt das, dass homöomorph zu sein eine \"Aquivalenzrelation
auf jeder Menge von topologischen Räumen ist.

\item[c)]
Sind $f:X\longrightarrow Y$ und $g:X\longrightarrow Z$ stetig, so ist auch 
$f\times g:X\longrightarrow Y\times Z$ stetig bezüglich der Produkttopologie.
Diese ist die feinste Topologie auf $Y\times Z$ mit dieser Eigenschaft.

\item[d)]

${\cal C}(X)$ ist wieder der Raum  der stetigen reellwertigen Funktionen auf 
$X$ (wobei $\mathbb R$ bei so etwas immer mit der Standardtopologie versehen 
ist!). Dies ist wieder ein Ring (bezüglich der üblichen Verknüpfungen), 
denn die Addition und Multiplikation sind stetige Abbildungen von 
$\mathbb R^2$ nach $\mathbb R,$ und wir können  b) und c) 
anwenden.
\end{itemize}
}

\end{bem}



\begin{hilfs}\label{Erhaltungssatz} {\bf Ein Erhaltungssatz}

Es sei $f:X\longrightarrow Y$ stetig. Dann gelten:
\begin{itemize}
\item[a)] Wenn $X$ kompakt ist, dann auch $f(X).$
\item[b)] Wenn $X$ zusammenhängend ist, dann auch $f(X).$
\item[c)] Wenn $Y$ hausdorff'sch ist und $f$ injektiv, dann ist $X$ 
hausdorff'sch.
\end{itemize}
\end{hilfs}

{\it Beweis.} 
\begin{itemize}
\item[a)] Es sei $\ddot V$ eine offene Überdeckung von $f(X)$ in $V.$ Dann
ist $\ddot U:= \{f^{-1}(U)\mid U\in \ddot U\}$ eine offene Überdeckung 
von $X.$ Da $X$ kompakt ist, gibt es endlich viele $V_1,\dots,V_n\in \ddot V,$
sodass bereits $\{f^{-1}(V_i)\mid 1\leq i\leq n\}$ eine Überdeckung von $X$ 
ist. 

Aus $f(f^{-1}(V_i)) = f(X)\cap V_i$ folgt, dass $\{V_1,\dots ,V_n\}$ das Bild
von $f$ überdecken.
\item[b)] Es sei $f(X) = A\cup B$ eine disjunkte Zerlegung von $f(X)$ in nicht 
leere Teilmengen. Wenn $A,B$ in der Spurtopologie offen wären, dann gäbe 
es offene Teilmengen $V,W$ von $Y$ mit $A=V\cap f(X), B=W\cap f(X).$

Mithin wäre $f^{-1}(V),f^{-1}(W)$ eine offene Überdeckung von $X$, die 
noch dazu disjunkt ist, da sich $V$ und $W$ nicht in $f(X)$ schneiden.

Andererseits wäre diese Teilmengen von $X$ nicht leer (weil $A$ und $B$ nicht
leer sind), und das widerspricht der Definition von Zusammenhang.

\item[c)] Es seien $x_1\neq x_2$ Punkte in $X.$ Dann sind ihre Bilder in $Y$ 
verschieden, denn $f$ soll injektiv sein. Daher haben $f(x_1),f(x_2)$ in $Y$ 
disjunkte Umgebungen, und deren Urbilder sind disjunkte Umgebungen von $x_1$ 
und $x_2.$ \hfill{$\bigcirc$}
\end{itemize}

Die Umkehrungen gelten jeweils natürlich nicht, wie einfache Gegenbeispiele 
lehren. Aber wir treffen bei näherem Hinsehen 

\begin{Folgerung} {\bf alte Bekannte}

Es sei $f:X\longrightarrow \mathbb R$ stetig.

\begin{itemize}
\item[a)]
Wenn $X$ kompakt ist, dann nimmt $f$ ein Maximum und ein Minimum an.

Als Spezialfall hiervon erinnnern wir an \ref{Normen}: eine Norm $N$ auf dem 
$\mathbb R^n$ ist immer stetig bezüglich der Standardmetrik. Daher nimmt sie 
auf der (kompakten) Einheitssphäre ein positives Minimum $m$ und ein Maximum
$M$ an, und das führt wegen der Homogenität der Norm zu
$$\forall x\in \mathbb R^n : m|x| \leq N(x) \leq M|x|.$$
Dies zeigt, dass $N$ und die Standardmetrik dieselbe Topologie liefern.
\item[b)]
Wenn $X\subseteq \mathbb R$ ein Intervall ist, dann ist auch $f(X)$ ein 
Intervall -- das ist der Zwischenwertsatz.\index{Zwischenwertsatz}
\phantom{.}\hfill{$\bigcirc$}
\end{itemize}
\end{Folgerung}

Insbesondere ist also ${\cal C}(X) = {\cal C}_0(X),$ wenn $X$ kompakt ist, und 
dies ist als Teilraum von ${\cal B}(X)$ ein metrischer Raum. Hier gibt es nun 
den wichtigen Satz 

\begin{satz} {\bf von Stone\footnote{Marshall Harvey Stone, 1903-1989}-Weierstra\ss\footnote{Karl Theodor Wilhelm Weierstra\ss, 1815-1897}}

Es sei $K$ ein kompakter topologischer Raum und ${\cal A}\subseteq {\cal C}(K)$
ein Teilring, der die konstanten Funktionen enthält und folgende Bedingung 
erfüllt:

$$\forall x\neq y\in K:\exists f\in {\cal A}:f(x)=0, f(y)=1.$$

Dann ist $\cal A$ dicht in ${\cal C}(K).$

\end{satz}

{\bf Das heißt:} Jede stetige Funktion auf $X$ lässt sich gleichmäßig
durch eine Folge in ${\cal A}$ approximieren. 

Die Bedingung an ${\cal A}$ hat einen Namen: man sagt, ${\cal A} $ 
{\it trenne die Punkte} von $X.$
Insbesondere impliziert dies, dass $K$ hausdorff'sch ist.

Ein beliebter Spezialfall des
Satzes ist der eines Kompaktums $X\subseteq \mathbb R^n$, wobei man dann für
${\cal A}$ gerne den Ring der Polynomfunktionen (in $n$ Variablen) auf $X$ 
wählt. Klar: Schon die linearen Abbildungen langen, um Punkte zu trennen.

{\it Beweis} des Satzes. Hier folge ich den Grundzügen der modernen Analysis
von Dieudonn\'e\footnote{Jean Alexandre Eug\`ene Dieudonn\'e, 1906-1992}.

Wir bezeichnen mit $\overline{\cal A}$ den Abschluss von $\cal A$ in 
${\cal C}(K).$

Wir führen den Beweis des Satzes in mehreren Schritten.

\begin{itemize}
\item[1.] Es gibt eine Folge von reellen Polynomen $u_n\in \mathbb R[X],$
die auf dem Intervall $[0,1]$ gleichmäßig gegen die Wurzelfunktion 
konvergiert.

Um dies einzusehen setzen wir $u_1\equiv 0$ und definieren rekursiv
$$u_{n+1} (t) := u_n(t) + \frac12(t-u_n(t)^2), n\geq 1.$$
Dann ist $(u_n)$ punktweise monoton steigend (auf $[0,1]$ wohlgemerkt, nur dort
betrachten wir das) und beschränkt. Punktweise gilt also (wegen des 
Monotoniekriteriums) $\lim_{n\to\infty} u_n(t) = \sqrt(t).$ Dann impliziert der 
Satz von Dini\footnote{Ulisse Dini, 1845-1918}, was in 1.\ behauptet wird.

\item[2.] Für jedes $f\in {\cal A}$ gehört $|f|$ zu $\overline{\cal A}.$

Denn für $a:= \max_{x\in K}|f(x)|$ ist $(u_n(f^2/a^2))$ (mit $u_n$ aus Punkt 
1.) eine Folge in $\cal A,$ die gegen $|f|$ konvergiert.

\item[3.] Für $x\neq y\in K$ und $a,b\in \mathbb R$ gibt es $f\in {\cal A}$ 
mit $f(x)=a, f(y) = b.$

Denn: Es gibt ja nach Voraussetzung in ${\cal A}$ eine Funktion $g$ mit
$g(x)\neq g(y).$ Setze nun 
$$f:= \frac a{g(x)-g(y)}(g-g(y)) + \frac b{g(y)-g(x)}(g-g(x)).$$
NB: $x,y$ sind fest, die Variable versteckt sich hinter dem nackten $g.$

\item[4.] Für jedes $f\in {\cal C}(K),$ jedes $x\in K$ und jedes 
$\varepsilon >0$ gibt es eine Funktion $g\in \overline{\cal A}$ derart, dass
$$g(x) = f(x),\ \ \forall y\in K: g(y)\leq f(y)+\varepsilon.$$

Zunächst gibt es wegen 3.\ für jedes $z\in K$ eine Hilfsfunktion 
$h_z\in {\cal A},$ sodass $h_z(x) = f(x),\ h_z(z)\leq f(z)+\frac\varepsilon 2.$
Da $h_z$ stetig ist, gibt es eine Umgebung $U_z$ von $z,$ in der 
$$h_z(y) \leq f(y)+\varepsilon, \ \ y\in U_z$$
gilt. Da $K$ kompakt ist, wird es von endlich vielen der $U_z$ überdeckt, 
es gibt also $z_1,\dots , z_n\in K: K=\cup_i U_{z_i}.$

Setze nun $g(y):= inf_i h_{z_i}(y).$ 
Diese Funktion liegt wegen 
$\inf(p,q) = \frac12(p+q-|p-q|)$ in $\overline{\cal A}$ und hat die 
gewünschte Eigenschaft.

\item[5.] $\overline{\cal A} = {\cal C}(K).$

Es sei $f\in {\cal C}(K)$ und $\varepsilon >0.$ Für jedes $x\in K$ gibt es
eine Funktion $h_x\in {\cal A}$ mit 
$$h_x(x) = f(x),\ \ \forall y\in K: h_x(y)\leq f(y)+\varepsilon.$$
Die Stetigkeit von $h_x$ zeigt, dass jedes $x\in K$ eine offene Umgebung 
$V_x$ hat mit
$$\forall y\in V_x: h_x(y)\geq f(y)-\varepsilon.$$
\"Ahnlich wie in 4.\ gibt es $x_1,\dots ,x_r\in K,$ sodass $K$ von den 
$V_x$ überdeckt wird. Auch ähnlich wie eben liegt das Supremum $g$ der 
Funktionen $h_{x_i},1\leq i\leq r,$ in $\overline{\cal A},$ und es gilt
$$||f-g||_\infty \leq\varepsilon.$$
Das ist die Behauptung.\phantom{.}\hfill{$\bigcirc$}

\end{itemize}

\begin{defini} {\bf Wo ein Weg ist\dots}


{\rm Es sei $X$ ein topologischer Raum. Ein {\it Weg} ist eine stetige 
Abbildung eines kompakten reellen Intervalls $[a,b]$ mit $a<b$ nach $X.$

Sind $f:[a,b]\longrightarrow X$ und $[b,c]\longrightarrow X$ zwei Wege mit
$f(b) = g(b),$ so ist $g\ast f: [a,c]\longrightarrow X$ ein Weg, wenn wir
$$g\ast f(t) = \left\{ \begin{array}{ll}f(t), & t\in [a,b]\\
                                   g(t), & t\in [b,c]\\
\end{array}\right. $$
definieren. 

$X$ heißt {\it wegzusammenhängend}, wenn es für alle $x,y\in X$
einen Weg $f$ mit $f(a) = x, f(b)=y$ gibt.

Ein wegzusammenhängender Raum ist zusammenhängend. Wegen 
\ref{Erhaltungssatz} ist ja das Bild eines Weges zusammenhängend, und 
wegen des letzten Satzes in \ref{Zusammenhang} ist die Verenigung der Bilder
der Wege, die bei einem festen $x\in X$ anfangen, auch zusammenhängend. Naja,
das zweite muss man sich vielleicht noch
einmal überlegen, wir haben ja jetzt unendlich viele beteiligte Teilmengen. 
Das ist eine nette Übung.}

\end{defini}



\begin{defini} {\bf Offenheit}

{\rm Eine Abbildung $f:X\longrightarrow Y$ zwischen zwei topologischen 
Räumen heißt {\it offen}, wenn für jede offene Teilmenge $A\subseteq X$
das Bild $f(A)$ in $Y$ offen ist.

$f$ heißt {\it offen in } $x\in X,$ falls jede Umgebung von $x$ unter $f$ 
auf eine Umgebung von $f(x)$ abgebildet wird.

Insbesondere ist ein Homöomorphismus also eine stetige und offene Bijektion.
}
\end{defini}

\begin{bsp} \label{kte Wurzel}{\bf Vorbereitung}
 
{\rm Es sei $k>0$ ein natürliche Zahl.

Auf der Menge der komplexen Zahlen ist die Abbildung $z\mapsto z^k$ eine offene 
und surjektive Abbildung, wie in der Beschreibung durch Polarkoordinaten 
ersichtlich ist.

Außerdem zeigen die Polarkoordinaten, dass diese Abbildung auf 
$\mathbb C\smallsetminus\{0\}$ lokal injektiv ist. Das heißt: jedes 
$z_0\neq 0$ hat eine offene Umgebung, auf der die Abbildung $z\mapsto z^k$ 
injektiv ist.

Das wiederum impliziert, dass es für jedes $z_0\in \mathbb 
C\smallsetminus\{0\}$ eine Umgebung gibt, auf der sich eine stetige und offene
$k$-te Wurzel $z\mapsto z^{1/k}$ definieren lässt. In Wirklichkeit ist
diese sogar reell differenzierbar.

Dies wird nun implizieren, dass komplexe nichtkonstante Polynome offen sind.
Das werden wir später benutzen, um den Fundamentalsatz der Algebra zu 
beweisen.
}
\end{bsp}
 
\begin{hilfs} \label{Polynom-offen}{\bf komplexe Polynome}

Es sei $f:\mathbb C\longrightarrow \mathbb C$ eine polynomiale Abbildung,
die nicht konstant ist. Dann ist $f$ offen.
\end{hilfs}

{\it Beweis.}

Wir zeigen, dass das Bild einer Umgebung der $0$ unter $f$ eine Umgebung von 
$f(0)$ ist. Da Translationen in $\mathbb C$ Homöomorphismen sind und aus 
Polynomen wieder Polynome machen, zeigt das, dass für jedes $z\in \mathbb C$
und jede Umgebung $U$ von $z$ die Menge $f(U)$ eine Umgebung von $f(z)$ ist,
und das ist gerade die Behauptung.

Hierbei dürfen wir uns auf den Fall zurückziehen, dass $f(0)=0$ gilt.

Es sei also $f(z) = \sum_{i=1}^d a_iz^i.$

Wir bemerken zunächst, dass $f$ im Nullpunkt reell differenzierbar ist. 
Die Ableitung im Nullpunkt ist die $\mathbb R$-lineare Abbildung, die durch 
Multiplikation mit $a_1$ zustande kommt. Wegen der 
binomischen Formeln gilt hier ja
$$\lim_{|h|\to 0} \frac{f(z_0+h) - f(z_0) - a_1h }{|h|} = 0.$$

Wenn $a_1\neq 0$ gilt, dann ist die Ableitung ein Isomorphismus, und der 
Satz von der impliziten Funktion sagt, dass
es eine Umgebung $U$ von $0$ und eine Umgebung $V$ von $f(0)=0$ gibt, sodass
$f$ auf $U$ injektiv ist, $f(U) = V,$ und die lokale Umkehrabbildung zu $f$
auf $V$ differenzierbar. Das heißt, dass auch $f^{-1}$ in $0$ stetig ist, 
$f$ also offen.

Es bleibt der Fall $a_1=0.$ Es sei $k=\min\{n\in \mathbb N \mid a_k\neq 0\}.$
Dann ist $k>1,$ da $a_0=a_1=0.$ Wir wollen zeigen, dass $f$ in einer Umgebung
der $0$ eine $k$-te Wurzel hat: $f(z) = g(z)^k,$ und dass $g$ offen gewählt 
werden kann. Dann sagt uns die Offenheit von $z\mapsto z^k,$ dass auch $f$ im
Nullpunkt offen ist, und wir sind fertig.

Dazu schreiben wir $f(z) = z^k\cdot h(z)$ mit 
$h(z) = \sum_{i=k}^d a_i z^{i-k}.$
Das Polynom $\tilde h$ hat also im Nullpunkt den Wert $a_k\neq 0.$
In einer Umgebung von $a_k$ gibt es wegen \ref{kte Wurzel} eine stetige, offene 
$k$-te Wurzel. Die $k$-te Wurzel $h(z)^{1/k}$ ist also in einer Umgebung der $0$
definiert, und bei näherem Hinsehen sieht man, dass die Ableitung im 
Nullpunkt regulär ist.

Daher gilt in einer Umgebung der $0:$ 
$$f(z) = z^k (h(z)^{1/k})^k = (z h(z)^{1/k})^k.$$ 
Das ist die $k$-te Potenz einer bei $0$ offenen Abbildung, und damit ist $f$ 
selbst im Ursprung offen.
\hfill{$\bigcirc$}

\begin{satz} \index{Liouville}\label{Liouville}{\bf \`a la Liouville}

Es seien $f:X\longrightarrow Y$ eine stetige und offene Abbildung, $X$ sei 
nichtleer und kompakt, $Y$ sei zusammenhängend und hausdorff'sch.

Dann ist $f$ surjektiv und insbesondere ist $Y$ auch kompakt.

\end{satz}

{\it Beweis.} Das Bild von $f$ ist offen nach Definition der Offenheit und
kompakt wegen \ref{Erhaltungssatz}. Als Kompaktum in $Y$ ist $f(X)$ 
abgeschlossen, siehe \ref{fuer Liouville}. Es ist mithin 
$Y= f(X) \cup (Y\smallsetminus f(X))$ eine Zerlegung von $Y$ als Vereinigung
zweier offener disjunkter Teilmengen. Da $f(X)$ nicht leer ist und $Y$
zusammenhängend ist, muss $Y\smallsetminus f(X)$ leer sein: $f$ ist 
surjektiv. \hfill{$\bigcirc$}

\begin{satz} \index{Fundamentalsatz der Algebra}\label{Fundamentalsatz} 
{\bf Fundamentalsatz der Algebra}

Es sei $f:\mathbb C\longrightarrow \mathbb C$ ein nichtkonstantes Polynom. 
Dann besitzt $f$ eine komplexe Nullstelle.

\end{satz}

{\it Beweis.} Nach Hilfssatz \ref{Polynom-offen} ist $f$ offen. Außerdem 
gilt (siehe \ref{Lasso}), dass $|f(z)|$ mit $|z|$ gegen unendlich geht. 

Wir können demnach $f$ zu einer stetigen Abbildung von 
$\mathbb P^1(\mathbb C)$ auf sich selbst fortsetzen, und man verifiziert, dass
auch die Fortsetzung offen ist. Also ist die Fortsetzung von $f$ surjektiv nach
Liouville, und es gibt ein $z\in\mathbb P^1(\mathbb C)$ mit $f(z) = 0.$
Da $z$ nicht $\infty$ sein kann (hier wird $f$ ja unendlich) ist $z\in 
\mathbb C$ wie behauptet. 
\phantom{Anfang}\hfill{$\bigcirc$}

\section{Topologische Mannigfaltigkeiten}

\begin{defini} {\bf Atlas}

{\rm Es sei $X$ ein topologischer Raum. Ein $n$-dimensionaler 
\index{Atlas}{\it Atlas} auf $X$ 
besteht aus einer offenen Überdeckung $\ddot U$ von $X,$ sodass für jedes 
$U\in \ddot U$ ein Homöomorphismus 
$$\varphi_U: U \longrightarrow Z(U)\subseteq \mathbb R^n$$
existiert, wobei $Z(U)$ in $\mathbb R^n$ offen ist.}

\end{defini}

Zum Beispiel besitzt jede offene Teilmenge $U$ des $\mathbb R^n$ einen Atlas; 
wir nehmen einfach $U$ selbst als Überdeckung und die Identität als
Kartenabbildung. 

{\bf Vorsicht:} Wir halten im Vorübergehen fest, dass es nicht {\it a
priori} klar ist, dass die Dimension eines Atlas durch die Topologie auf $X$
festliegt. Das ist so, aber der Beweis ist nicht so offensichtlich.
Schließlich muss man so etwas zeigen, wie dass es für $m\neq n$ keine 
offene stetige Abbildung einer $m$-dimensionalen Kugel in eine 
$n$-dimensionale gibt.

\begin{defini} {\bf topologische Mannigfaltigkeit}

{\rm Ein topologischer Raum $X$ ist eine $n$-dimensionale {\it topologische
Mannigfaltigkeit}\index{topologische Mannigfaltigkeit}, wenn er hausdorff'sch
ist, mit einem Atlas
ausgerüstet werden kann und eine abzählbare Basis der Topologie besitzt.
}

\end{defini}

\begin{bem} {\bf Abzählbarkeitsaxiome}

{\rm Die letzte Bedingung erm\"glicht einige Konstruktionen mit topologischen 
Mannigfaltigkeiten, die sich als sehr hilfreich erweisen. 
Sie impliziert zum Beispiel, dass jede offene Überdeckung von $X$ eine
abzählbare Teilüberdeckung hat. 

Man nennt sie auch das {\it zweite Abzählbarkeitsaxiom}.


Der Name schreit nach einem Vorgänger: ein topologischer Raum erfüllt das 
{\it erste Abzählbarkeitsaxiom}, wenn jeder Punkt eine abzählbare 
Umgebungsbasis besitzt. Metrische Räume haben diese Eigenschaft zum Beispiel,
sie ist eine {\bf lokale} Bedingung, sagt sie doch nur etwas über Umgebungen 
von einem jeden Punkt aus. Das werden wir im nächsten Hilfssatz einmal
austesten.

Wenn es eine Abzählbare Umgebungsbasis von $x$ gibt, so gibt es auch eine 
der Gestalt
$$U_1\supseteq U_2\supseteq U_3\dots$$
Das sieht man durch sukzessive Schnittbildung einer gegebenen abgezählten 
Umgebungsbasis.

Das zweite Abzählbarkeitsaxiom impliziert offensichtlich das erste.
}
\end{bem}

\begin{defini} {\bf schon wieder Folgen}

{\rm Eine Folge $(x_n)$ in einem topologischen Raum $X$ {\it konvergiert gegen}
$x\in X,$ falls in jeder Umgebung von $x$ alle bis auf endlich viele 
Folgenglieder liegen.

{\bf Vorsicht:} \underline{Der} Grenzwert ist im Allgemeinen nicht mehr 
eindeutig, also eigentlich der bestimmter Singular verboten. Für die
Eindeutigkeit des Grenzwerts braucht man ein Trenungsaxiom, zum Beispiel ist 
hausdorff'sch hinreichend.

$X$ heißt {\it folgenkompakt}, wenn
jede Folge in $X$ eine konvergente Teilfolge besitzt.
}
\end{defini}

\begin{hilfs} {\bf Folgen für die Folgenkompaktheit}

Es sei $X$ ein topologischer Raum.
\begin{itemize}
\item[a)] Ist $X$ kompakt und erfüllt das erste Abzählbarkeitsaxiom, so
ist $X$ folgenkompakt.
\item[b)] Ist $X$ folgenkompakt und metrisch, so ist $X$ auch kompakt.
\end{itemize}
\end{hilfs}

{\it Beweis.} 
\begin{itemize}
\item[a)] Es sei $(x_n)$ eine Folge in $X.$ Dann gibt es ein $x\in X$, sodass
in jeder Umgebung $U$ von $x$ für unendlich viele $n\in \mathbb N$ der Punkt
$x_n$ liegt. 

Anderenfalls ließe sich für alle $x\in X$ eine Umgebung $U_x$ finden, die
nur endlich viele Folgenglieder enthält, und weil 
$$X=\bigcup_{x\in X} U_x$$
eine endliche Teilüberdeckung hat, hätte man einen Widerspruch.

Nun haben wir so ein $x.$ Dieses besitzt eine abzählbare Umgebungsbasis
$$U_1\supseteq U_2 \supseteq U_3,\dots$$
und wir können bequem eine Teilfolge $x_{n_k}$ wählen mit 
$$\forall k\in \mathbb N:n_{k+1} > n_{k} \ \ {\rm und }\ \ x_{n_k}\in U_k.$$
\item[b)] Es sei $\ddot U$ eine offene Überdeckung des folgenkompakten 
metrischen Raums $X.$ 

Für jedes $x\in X$ wählen wir ein $U_x\in \ddot U$ derart, dass eine der
beiden folgenden Bedingungen erfüllt ist:
$$B_1(x)\subseteq U_x\ \ {\rm  oder }\ \ 
\exists r(x)>0 : B_{r(x)}(x)\subseteq U_x, \forall U\in \ddot U: 
B_{2r(x)}(x)\not\subseteq U.$$
Jetzt nehmen wir an, dass $\ddot U$ keine endliche Teilüberdeckung besitze.
Wir starten mit einem beliebigen $x_1\in X$ und wählen 
$$x_2\in X\smallsetminus U_{x_1},\ 
x_3\in X\smallsetminus(U_{x_1}\cup U_{x_2}),\dots$$
Da $X$ folgenkompakt ist, gibt es eine Teilfolge $(x_{n_k})$, die gegen ein
$a\in X$ konvergiert. Wir wählen ein $r\in (0,1)$ derart, dass
$B_r(a)\subseteq U_a.$ Dann liegt $x_{n_k}$  für 
großes $k$ in $B_{r/5}(a),$ und es gilt 
$$d(x_{n_k},x_{n_{k+1}}) < 2r/5.$$
Andererseits zeigt
$$x_{n_k}\in B_{4r/5}(x_{n_k})\subseteq B_{r}(a)\subseteq U_a\in \ddot U,$$
dass $r(x_{n_k})\geq 2r/5,$ und damit auch 
$$d(x_{n_k},x_{n_{k+1}}) \geq 2r/5.$$
Dieser Widerspruch besiegelt das Schicksal unserer irrigen Annahme, $\ddot U$
habe keine endliche Teilüberdeckung. 

Also ist $X$ kompakt, da $\ddot U$ beliebig war.\hfill{$\bigcirc$}
\end{itemize}

\begin{bsp} {\bf Schönheiten des Abendlandes}

{\rm Nach diesem Grundlagenexkurs kehren wir nun zu den topologischen 
Mannigfaltigkeiten zurück. Wir kennen noch keine Beispiele. Oder doch?

\begin{itemize}
\item[a)] Jede offene, nichtleere Teilmenge von $\mathbb R^n$ ist eine
$n$-dimensionale topologische Mannigfaltigkeit. Hier muss man vor allem das
zweite Abzählbarkeitsaxiom testen. {\bf Tun Sie das!}

Jeder Hausdorffraum mit einem endlichen Atlas ist dann auch eine topologische 
Mannigfaltigkeit.

\item[b)] Es sei $K=\mathbb R$ oder $\mathbb C.$ Dann ist der projektive Raum
$\mathbb P^n(K)$ mit der früher eingeführten Quotiententopologie eine 
topologische Mannigfaltigkeit.

Denn er lässt sich überdecken durch die offenen Mengen 
$$U_k:=\{(x_i)_{1\leq i\leq n+1} \mid x_k = 1\},\ 1\leq k\leq n+1,$$
und diese werden beim Quotientenbilden mit ihrem Bild auch topologisch 
identifiziert, liefern also einen endlichen Atlas von $\mathbb P^n(K).$

\item[c)] Keine topologische Mannigfaltigkeit ist zum Beispiel der folgende
Raum, obwohl er einen endlichen Atlas hat: Wir nehmen die Einheitssphäre 
$S^1\subseteq \mathbb C$ und definieren $X=S^1/\simeq,$
wobei die \"Aquivalenzrelation $\simeq$ durch $x\simeq -x$ für $x\neq 
\pm1$ definiert ist. 

Ein offener Halbkreis wird hierbei injektiv nach $X$ abgebildet, und wir 
erhalten einen schönen Atlas, von dem sogar zwei Karten genügen. Aber $X$ 
ist nicht hausdorff'sch, weil die Klassen von $\pm 1$ sich nicht durch offene
Umgebungen trennen lassen.
\end{itemize}
}
\end{bsp}





%\input Stichworte.tex

\end{document}


