\documentclass{article}
\newcounter{chapter}
\setcounter{chapter}{15}
\usepackage{ana}

\title{Existenz- und Eindeutigkeitss�tze f�r Dgl.Systeme 1. Ordnung}
\author{Christian Schulz}
% Wer nennenswerte �nderungen macht, schreibt sich bei \author dazu

\begin{document}
\maketitle

Stets in diesem Paragraphen: $D \subseteq \MdR^{m+1}, (x_0,y_0) \in D$ und $x_0 \in \MdR, 
y_0 \in \MdR^m$ und $f=(f_1,...,f_m): D \to \MdR^m$ eine Funktion.\\
Ein \begriff{System von Dgl. 1. Ordnung} hat die Form: \\
$
\begin{cases} 
y_1' = f_1(x,y_1,...,y_m) \\
y_2' = f_2(x,y_1,...,y_m) \\
\vdots \\
y_m' = f_m(x,y_1,...,y_m)\\ 
\end{cases}$

Setzt man $y=(y_1,...,y_m)$, so schreibt sich das System in der Form $y' = f(x,y)$. Wir 
betrachten auch noch das AWP (A) $\begin{cases} y' = f(x,y) \\ y(x_0) = y_0 \end{cases}$ \\
Wir �bertragen die S�tze aus den Paragraphen 12 und 13 auf Systeme. Die Beweise dort lassen sich fast w�rtlich f�r Systeme wiederholen. (beachte 14.2) ($\|\cdot\|$ anstatt $|\cdot|$).

\begin{satz} [Peano]


\begin{itemize}
\item[(1)] 
Sei $D=I \times \MdR^m$ und $I=[a,b] \subseteq \MdR$, $x_0 \in I, y_0 \in \MdR^m$ und $ f \in C(D,\MdR^m)$ sei beschr�nkt. Dann hat das AWP (A) eine L�sung auf I.

\item[(2)] 
Es sei $I = [a,b] \subseteq \MdR, x_0 \in I, y_0 \in \MdR^m, s > 0$ und 
$D:=\{(x,y) \in \MdR^{m+1} | ||y-y_0|| < s\}$.
Es sei $f \in C(D,\MdR^m), M := \max \{ ||f(x,y)|| : (x,y) \in D \}$ und 
$J := I \cap [x_0-\frac{s}{M},x_0+\frac{s}{M}].$ Dann hat das AWP (A) eine L�sung auf J.

\item[(3)] Sei $D$ offen, $(x_0,y_0) \in D$ und $f \in C(D, \MdR^m)$. Dann ex. eine L�sung 
$y: K \to \MdR^m$ von (A) mit $x_0 \in K$ und $ K \subseteq \MdR$ ein Intervall.
\end{itemize}
\end{satz}

\begin{definition}
\begin{itemize}
\item[(1)]
\begriff{$f$ gen�gt auf $D$ einer Lipschitzbedingung (LB) bzgl. $y$ :} \equizu \\
$\exists \; \gamma \geq 0 : ||f(x,y) - f(x,\overline{y})|| \leq \gamma ||y - \overline{y}|| \; \forall (x,y), (x,\overline{y}) \in D$ \; (*)
\item[(2)]
Sei $D$ offen. \textbf{$f$ gen"ugt auf $D$ einer lokalen LB bzgl. $y$} $:\equizu \forall (x_0,y_0)\in D\ \exists $ Umgebung $U$ von $(x_0,y_0)$ mit: $U\subseteq D$ und $f$ gen"ugt auf $U$ einer LB bzgl. $y$.
\end{itemize}
\end{definition}

\begin{satz} [Picard-Lindel�f]
\begin{itemize}
\item[(1)]
$I, x_0, y_0, D$ seien wie in 15.1(1) und $f \in C(D,\MdR^m)$ gen�ge auf D einer LB bzgl. $y$.
Dann hat das AWP (A) auf I genau eine L�sung. Ist  $y^{[0]} \in C(I,\MdR^m)$ beliebig und setzt man $y^{[n+1]}(x) := y_0  + \int_{x_0}^{x} f(t,y^{[n]}(t)) dt \; (x \in I, n \in \MdN)$. Dann konvergiert $(y^{[n]})$ auf $I$ glm. gegen die L�sung von (A).

\item[(2)] 
$I, x_0, y_0, D, s, M$ und $J$ seien wie in 15.1(2) und $f \in C(D,\MdR^m)$ gen�ge auf $D$ eine LB bzgl. y. Dann hat (A) auf J genau eine Lsg.

\item[(3)] 
Es sei $D$ offen, $f$ gen"uge auf $D$ einer lokalen LB bzgl. $y$. Dann ist das AWP (A) eindeutig l�sbar.
\end{itemize}

\end{satz}


\end{document}
 
