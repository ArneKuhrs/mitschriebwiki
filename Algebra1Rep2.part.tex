\chapter{Ringe}

\section{Euklidische Ringe}

\begin{Def}
    \begin{enum}
\item Ein Integrit�tsbereich $R$ hei�t \emp{euklidisch}, wenn es
eine Abbildung: $\delta: R\setminus\{0\} \ra \mathbb{N}$ mit
folgender Eigenschaft gibt: zu $f,g \in R, g\neq 0$ gibt es $q,r \in
R$ mit $f = qg + r$ mit $r=0$ oder $\delta(r) < \delta(q)$.

\item Sei $R$ euklidisch, $a,b \in R\setminus\{0\}$. Dann gilt:
    \begin{enum}
        \item[(i)] in $R$ gibt es einen ggT von $a$ und $b$.
        \item[(ii)] $d \in (a,b)$ (dh $\exists x,y \in R$ mit $d=xa
        +yb$)
        \item[(iii)] $(d) = (a,b)$
    \end{enum}
\item Jeder euklidische Ring ist ein Hauptidealring.
\end{enum}
\sbsp{1.0}{$\mathbb{Z}$ mit $\delta(a) = |a|, K[X]$ mit $\delta(f)
=$ Grad($f$)}
\end{Def}

\section{Hauptidealringe}

\begin{Def}
Ein komutativer Ring mit Eins hei�t \emp{Hauptidealring}, wenn
jedes Ideal in $R$ ein Hauptideal ist.
\end{Def}

\begin{Satz}
Jeder nullteilerfreie Hauptidealring ist faktoriell.
\end{Satz}

%Bosch S.46
\begin{Satz}
Es sei $R$ ein Hauptidealring $p \in R$ eine von $0$ verschiedene Nichteinheit.
Dann ist �quivalent:
\begin{enumerate} 
  \item[(i)] $p$ ist irreduzibel
  \item[(ii)] $p$ ist Primelement
  \item[(iii)] $(p)$ ist maximales Ideal in $R$
\end{enumerate}
\end{Satz}

\section{Faktorielle Ringe}
\begin{PropDef}
\label{2.21}
Sei $R$ ein Integrit�tsbereich.
\begin{enum}
\item Folgende Eigenschaften sind �quivalent:
    \begin{enumerate}
        \item[(i)] Jedes $x \in R\setminus\{0\}$ l��t sich eindeutig
        als Produkt von Primelementen schreiben.
        \item[(ii)] Jedes $x \in R\setminus\{0\}$ l��t sich
        ''irgendwie'' als Produkt von Primelementen schreiben.
        \item[(iii)] Jedes $x \in R\setminus\{0\}$ l��t sich eindeutig als
        Produkt von irreduziblen Elementen schreiben.
    \end{enumerate}

\item Sind diese drei Eigenschaften f�r $R$ erf�llt, so hei�t $R$
\emp{faktorieller} Ring. (Oder \emp{ZPE-Ring} (engl.: UFD)). Dabei
ist in (a) ''eindeutig'' gemeint, bis auf Reihenfolge und
Multiplikation mit Einheiten. Pr�ziser: Sei $\mathcal{P}$ ein
Vertretersystem der Primelemnte ($\neq 0$) bez�glich
''assoziiert''.
\newline Dann hei�t (i) $\forall x \in R \setminus\{0\}\; \exists!\;e \in
R^x$ und f�r jedes $p \in \mathcal{P}$ ein $\ds\nu_p(x) \geq
0:x=e\prod_{p \in \mathcal{P}} p^{\nu_p}$. (beachte $\nu_p \neq 0$
nur f�r endlich viele $p$).
\end{enum}
\end{PropDef}

\begin{Bem}
Ist $R$ faktorieller Ring, so gibt es zu
allen $a,b \in R\setminus\{0\}$ einen ggT($a$,$b$).
\end{Bem}

\begin{Bem}
Sei $R$ ein faktoriellen Ring, $a \in R$.
\[a \mbox{ irreduzibel} \lra a \mbox{ prim} \]
\end{Bem}

\section{Vererbung auf den Polynomring}

\begin{Bem}
Sei $R$ ein Ring und $R[X]$ der zugeh�rige Polynomring, dann vererben sich
folgende Eigenschaften von $R$ auf $R[X]$:
\begin{enumerate}
  \item hat Eins
  \item kommutativ
  % TODO vererbt sich die Nullteilerfreiheit auch ohne obige Eigenschaften?
  % [ ] Ja
  % [ ] Nein
  % [ ] Vielleicht
  \item Integrit�tsbereich
  \item faktoriell
\end{enumerate}
\end{Bem}