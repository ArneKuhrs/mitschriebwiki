\documentclass{article}
\newcounter{chapter}
\setcounter{chapter}{15}
\usepackage{ana}
\def\gdw{\equizu}
\def\Arg{\text{Arg}}
\def\MdD{\mathbb{D}}
\def\Log{\text{Log}}
\def\Tr{\text{Tr}}
\def\abnC{\ensuremath{[a,b]\to\MdC}}
\def\wegint{\ensuremath{\int\limits_\gamma}}
\def\iint{\ensuremath{\int\limits}}
\def\ie{\rm i}

\title{meromorphe Funktionen, Moebiustransformationen}
% Da war noch , $\hat{\MdC}$ im titel, aber das kann latexki nicht
\author{Christian Schulz, Bernhard Konrad} % Wer nennenswerte Änderungen macht, schreibt euch bei \author dazu

\begin{document}
\maketitle
\begin{definition}
Es sei $\infty$ irgendein Element $\not \in \MdC$. $\hat{\MdC} := \MdC \cup \{ \infty
\}$ heißt die \begriff{Vollebene}. $\infty$ heißt ``\begriff{der Punkt
$\infty$}``. Wir definieren: \\ \\
\centerline{$z + \infty := \infty + z := \infty -z := z - \infty := \infty$
$\forall z \in \MdC$;} \centerline{$\infty z := z \infty := \infty$ $\forall z \in \MdC
\{0\}$; $\frac{z}{\infty} := 0 (z \in \MdC)$, $\frac{z}{0}:= \infty$ ($ z \in
\hat{\MdC} \backslash \{0\} $)}\\ \\
$S := \{ (x_1, x_2, x_3) \in \MdR^3: x_1^2+ x_2^2 + (x_3 - \frac{1}{2})^2 =
\frac{1}{4}\}$ heißt \begriff{Riemannsche Zahlenkugel}. $N := (0,0,1)$ wird als
\begriff{Nordpol} bezeichnet .
\end{definition}
\begin{definition}
$\sigma: S \to \hat{\MdC}$ durch \\ \\
\centerline{$\sigma(N) := \infty$} \\
\centerline{$\sigma(x_1, x_2, x_3) := \frac{x_1}{1-x_3} + i \frac{x_2}{1-x_3}$ für
$(x_1,x_2,x_3) \in S \backslash N$}. \\ \\
$\sigma$ heißt \begriff{stereographische Projektion}. Anschaulich
(nachrechnen!): Ist $P \in S \backslash N$, so trifft die Gerade durch $N$ und $P$
die komplexe Ebene im Punkt $\sigma(P)$.
\end{definition}

%Satz 15.1
\begin{satz}
$\sigma$ ist injektiv auf $S$ und $\sigma(S) = \hat{\MdC}.\ \sigma^{-1}:
\hat{\MdC} \to S$ ist gegeben durch \\ $\sigma^{-1}(\infty) = N$, $\sigma^{-1}(z) =
\frac{1}{1+ |z|^2}(\text{Re} z, \text{Im} z, |z|^2)$, falls $z \in \MdC$.
\end{satz}
%
%
%Was ist mit dem Beweis von 15.1?
%den hat er nicht gebracht
%
%Ab hier: Bernhard

%Satz 15.2
\begin{satz}[Der chordale Abstand]
Seien $z,w \in \hat{\MdC}.\ d(z,w):= || \sigma^{-1}(z) - \sigma^{-1}(w) ||$ hei"st der \begriff{chordale Abstand} von $z$ und $w$ (wobei $||\cdot| |=$ eukl. Norm im $\MdR^3)$.\\
F"ur $z,w,u \in \hat{\MdC}: d(z,w) \geq 0;\ d(z,w)=0 \Leftrightarrow z=w;\ d(z,w)=d(w,z);\ d(z,w) \leq d(z,u) + d(u,w)\ (\triangle-$Ungl.)\\
$(\hat{\MdC},d)$ ist also ein metrischer Raum.\\
F"ur $z,w \in \MdC: d(z,\infty) = (1+|z|^2)^{-\frac12};\ d(z,w) = |z-w|(1+|z|^2)^{-\frac12}(1+|w|^2)^{-\frac12}$
\end{satz}

\begin{beweis}
"Ubung!
\end{beweis}

\begin{definition}
Sei $(z_n)$ eine Folge in $\hat{\MdC}$ und $z_0 \in \hat{\MdC}. (z_n)$ \begriff{konvergiert in C} gegen $z_0 :\Leftrightarrow d(z_n,z_0) \rightarrow 0 (n \rightarrow \infty)$
\end{definition}

Aus 15.2 folgt:
%Satz 15.3
\begin{satz}
Sei $(z_n)$ eine Folge in $\hat{\MdC}, z_0 \in \hat{\MdC}$
\begin{liste}
\item $d(z_n,z_0) \rightarrow 0 \Leftrightarrow |z_n-z_0| \rightarrow 0$
\item $d(z_n,\infty) \rightarrow 0 \Leftrightarrow |z_n| \rightarrow \infty$
\end{liste}
\end{satz}

Ersetzt man $|z-w| \ (z,w \in \MdC)$ durch $d(z,w) \ (z,w \in \hat{\MdC})$, so lassen sich die topologischen Begriffe der §en 2,3 auch in $\hat{\MdC}$ definieren.

\begin{beispiele}
\item Sei $A \subseteq \hat{\MdC}$. Eine Funktion $f: A \rightarrow \hat{\MdC}$ hei"st stetig in $z_0 \in A :\Leftrightarrow$ f"ur jede Folge $(z_n)$ in $A$ mit $d(z_n,z_0) \rightarrow 0$ gilt: $d(f(z_n),f(z_0)) \rightarrow 0$.
\item $A \subseteq \hat{\MdC}$ hei"st offen $: \Leftrightarrow \ \forall a \in A \ \exists \delta = \delta(a) > 0: \{ z \in \hat{\MdC}: d(z,a) < \delta \} \subseteq A$.
\end{beispiele}

{\bf Konvention}\\
Sei $D \subseteq \MdC$ offen. $z_0 \in D. f \in H(D \backslash\{z_0\})$ und $z_0$ sei ein  Pol von $f$. Wegen 13.5 und 15.3 setzt man $f(z_0):=\infty$. Dann ist $f$ auf ganz $D$ definiert, also $f: D \rightarrow \hat{\MdC}$ und in jedem $z \in D$ stetig.

\begin{definition}
Sei $D \subseteq \MdC$ und $f: D \rightarrow \hat{\MdC}$ und $P(f) := \{ z \in D: f(z) = \infty\}. f$ hei"st auf $D$ \begriff{meromorph} $:\Leftrightarrow$
\begin{itemize}
\item[(i)] $P(f)$ ist in $D$ diskret
\item[(ii)] $f_{|D \backslash P(f)} \in H( D \backslash P(f))$
\item[(iii)] jedes $z_0 \in P(f)$ ist ein Pol von $f$.
\end{itemize}
\end{definition}
\[
M(D) := \{ f: D \rightarrow \hat{\MdC}: f \mbox{ ist auf } D \mbox{ meromorph } \}
\]

\begin{beispiele}
\item $P(f)= \emptyset$ zugelassen. Dann: $H(D) \subseteq M(D)$.
\item Seien $f,g \in H(D), g \not= 0$ auf $D$. Dann $\frac{f}{g} \in M(D). P(\frac{f}{g}) \subseteq Z(g)$.
\item $f(z) = \frac{1}{\sin(\frac{1}{z})}, P(f)= \{\frac{1}{k\pi}, k \in \MdZ \backslash \{0\}\}. 0$ ist kein Pol von $f$, $0$ ist HP der Pole $\frac1{k\pi}$. Also: $f \notin M(\MdC)$, aber $f \in M(\MdC \backslash \{0\}).$
\end{beispiele}

{\bf Moebiustransformationen:}\\
Seien $a,b,c,d \in \MdC$ und es gelte $ad-bc \not= 0$. Eine Abbildung der Form $T(z) := \frac{az+b}{cz+d}$ hei"st eine \begriff{Moebiustransformation} (MB) $(z \in \hat{\MdC})$. Also: $T: \hat{\MdC} \rightarrow \hat{\MdC}$. Die Matrix
$\left( \begin{array}{cc}
a & b \\
c & d
\end{array}
\right) := \Phi_T$ hei"st die zu $T$ geh"orende \begriff{Koeffizientenmatrix}.

\begin{bemerkungen}
\item Die Bedingung $ad-bc \not= 0$ sichert, dass T nicht konstant ist.
\item Sei $c = 0 \Rightarrow d \not= 0 \Rightarrow T(z) = \frac{a}{d}z + \frac{b}{d}.\ T(\infty) = \infty, T_{|\MdC} \in H(\MdC).$
\item $c \not= 0.\ T(\infty) = \frac{a}{c}; T(-\frac{d}{c})=\infty. -\frac{d}{c}$ ist ein Pol der Ordnung $1$ von $T;\ T \in M(\MdC)$.
\end{bemerkungen}
$\mathcal{M} := $Menge aller Moebiustransformationen.

%Satz 15.4
\begin{satz}
Seien $T,S \in \mathcal{M}.$
\begin{liste}
\item $T(\hat{\MdC}) = \hat{\MdC}; T$ ist stetig und injektiv auf $\hat{\MdC};\ T^{-1} \in \mathcal{M};\ T^{-1}(w) = \frac{-dw+b}{cw-a}$
\item $T \circ S \in \mathcal{M}.\ \Phi_{T \circ S} = \Phi_T \cdot \Phi_S$
\end{liste}
$\mathcal{M}$ ist also eine Gruppe.
\end{satz}

\begin{beweis}
"Ubung!
\end{beweis}

{\bf spezielle Moebiustransformationen:}\\
$T(z) := az$ \ (Drehstreckung)\\
$T(z) := z+a$ \ (Translation)\\
$T(z) := \frac1z$ \ (Inversion)

%Satz 15.5
\begin{satz}
$T \in \mathcal{M}$ l"asst sich darstellen als Hintereinandersausf"uhrung von Drehstreckung, Translation und Inversion.
\end{satz}

\begin{beweis}
Sei $T(z) = \frac{az+b}{cz+d}.$
\begin{liste}
\item[Fall 1:] $c=0$. Dann $d \not= 0$ und $T(z) = \frac{a}{d}z + \frac{b}{d}$. Setze $T_1=\frac{a}{d}z$ und $T_2=z+ \frac{b}{d} \Rightarrow T = T_2 \circ T_1.$
\item[Fall 2:] $c \not= 0.\ T(z) = \frac{a}{c}+\frac{\frac{b}{c}-\frac{ad}{c^2}}{z+\frac{d}{c}} = \alpha + \frac{\beta}{z+ \gamma}.\ T_1(z) := z+ \gamma;\ T_2(z) := \frac1z;\ T_3(z) := \beta z;\ T_4(z):=z + \alpha \Rightarrow T=T_4 \circ T_3 \circ T_2 \circ T_1.$
\end{liste}
\end{beweis}

%Satz 15.6
\begin{satz}
Sei $T \in \mathcal{M}$. Dann hat $T$ einen oder zwei Fixpunkte oder es ist $T(z) = z$.
\end{satz}

\begin{beweis}
Sei $T(z) = \frac{az+b}{cz+d}$
\begin{liste}
\item[Fall 1:] $T(\infty) = \infty.$ Dann ist $c = 0, d \not= 0. \Rightarrow T(z) = \frac{a}{d}z + \frac{b}{d} = \alpha z + \beta.$ Sei $z_1$ ein Fixpunkt von $T, z_1 \not= \infty$. Also $z_1 = \alpha z_1 + \beta \Leftrightarrow (1-\alpha)z_1=\beta.$
\begin{liste}
\item[Fall 1.1:] $\alpha = 1 \Rightarrow \beta = 0 \Rightarrow T(z) = z$.
\item[Fall 1.2:] $\alpha \not=1 \Rightarrow z_1 = \frac{\beta}{1-\alpha}.$
\end{liste}
\item[Fall 2:] $T(\infty) \not= \infty$. Sei $z_0 \in \MdC.\ T(z_0) = z_0 \Leftrightarrow az_0+b = z_0(cz_0+d)$ quadratische Gleichung $\Rightarrow$ ein oder zwei L"osungen.
\end{liste}
\end{beweis}

\begin{definition}
Seien $z_1,z_2,z_3 \in \hat{\MdC}$ paarweise verschieden. F"ur $z \in \hat{\MdC}$ hei"st
\[
DV(z,z_1,z_2,z_3) := \left\{
\begin{array}{cl}
\frac{z-z_1}{z-z_3}:\frac{z_2-z_1}{z_2-z_3} & \mbox{, falls } z_1,z_2,z_3 \in \MdC \\
\frac{z_2-z_3}{z-z_3} & \mbox{, falls } z_1 = \infty \\
\frac{z-z_1}{z-z_3} & \mbox{, falls } z_2 = \infty \\
\frac{z-z_1}{z_2-z_1} & \mbox{, falls } z_3 = \infty
\end{array} \right.
\]
das \begriff{Doppelverh"altnis} von $z,z_1,z_2,z_3$.
\end{definition}

%Satz 15.7
\begin{satz}
Seien $z_1,z_2,z_3 \in \hat{\MdC}$ wie oben.
\begin{liste}
\item Sind $T_1,T_2 \in \mathcal{M}$ und gilt $T_1(z_j) = T_2(z_j) \ (j=1,2,3) \Rightarrow T_1 = T_2.$
\item Es ist $T(z) := DV(z,z_1,z_2,z_3) \ (z \in \hat{\MdC})$ eine Moebiustransformation. $T$ ist die einzige Moebiustransformation mit $T(z_1)=0; \ T(z_2) = 1 \ ; T(z_3) = \infty$.
\item Sind $w_1,w_2,w_3 \in \hat{\MdC}$ paarweise verschieden, so existiert genau ein $S \in \mathcal{M}: S(z_j) = w_j \ (j = 1,2,3)$
\item $DV(z,z_1,z_2,z_3) = DV(S(z),S(z_1),S(z_2),S(z_3)) \ \forall z \in \hat{\MdC} \ \forall S \in \mathcal{M}$ \ (Invarianz des Doppelverh"altnisses)
\end{liste}
\end{satz}

\begin{beweis}
\begin{liste}
\item $T := T_2^{-1} \circ T_1.$ 15.4 $\Rightarrow T \in \mathcal{M}.\ T(z_j) = T_2^{-1}(T_1(z_j)) = T_2^{-1}(T_2(z_j)) = z_j \ (j=1,2,3).$ 15.6 $\Rightarrow T(z) = z \ \forall z \in \hat{\MdC} \Rightarrow T_1 = T_2$.
\item Klar: $T \in \mathcal{M}$. Nachrechnen: $T(z_1) = 0; \ T(z_2) = 1; \ T(z_3) = \infty$. Eindeutigkeit folgt aus (1).
\item Eindeutigkeit: (1). Existenz: $T_1(z):= DV(z,z_1,z_2,z_3); \ T_2(z) := DV(z,w_1,w_2,w_3).\ S:= T_2^{-1}\circ T_1.\ S(z_1) = T_2^{-1}(T_1(z_1)) \stackrel{(2)}{=} T_2^{-1}(0) \stackrel{(1)}{=} w_1.$ Analog: $S(z_2) = w_2; \ S(z_3) = w_3.$
\item "Ubung.
\end{liste}
\end{beweis}

{\bf Kreisgleichung:}\\
Sei $z_0 \in \MdC, r>0. |z-z_0|=r \Leftrightarrow (z-z_0)(\bar{z}-\bar{z_0}) = r^2 \Leftrightarrow |z|^2 - \bar{z_0}z-z_0\bar{z}+|z_0|^2-r^2=0 \Leftrightarrow |z|^2 + \bar{\alpha}z + \alpha\bar{z}+\beta=0,$ wobei $\alpha = -z_0 \in \MdC. \beta = |z_0|^2-r^2 \in \MdR$ und $|\alpha|^2-\beta = |z_0|^2-|z_0|^2+r^2 > 0$, also $ \beta < |\alpha|$.\\
\\
{\bf Geradengleichung:}\\
$mx+ny+d=0 \ (m,n,d,x,y \in \MdR). \ x = $Re$ z, \ y= $Im$ z; \alpha = \frac{m}{2}+i\frac{n}{2} \in \MdC, \ \beta := d \in \MdR. mx+ny+d = 0 \Leftrightarrow \bar{\alpha}z+\alpha\bar{z}+\beta=0$.\\
\\
{\bf Fazit:}\\
Sind $\alpha \in \MdC, \beta \in \MdR$, so ist $\varepsilon |z|^2 + \bar{\alpha}z + \alpha\bar{z}+\beta = 0$
\begin{liste}
\item[-] Die Gleichung eines Kreises, falls $\varepsilon = 1$ und $\beta < |\alpha|^2$
\item[-] Die Gleichung einer Geraden, falls $\varepsilon = 0.$
\end{liste}

%Satz 15.8
\begin{satz}
Sei $T \in \mathcal{M}.\ T$ bildet eine Gerade (einen Kreis) auf eine Gerade oder einen Kreis ab.
\end{satz}

\begin{beweis}
Die Behauptung ist klar f"ur Drehstreckungen und Translationen. Wegen 15.5 gen"ugt es die Behauptung f"ur Inversionen $(T(z) = \frac1z)$ zu zeigen. Sei $\varepsilon|z|^2 + \bar{\alpha}z + \alpha\bar{z}+\beta =0$. die Gleichung einer Geraden oder eines Kreises und $w = \frac1z$. Dann: $\varepsilon \frac1{|w|^2}+ \bar{\alpha}\frac1w + \alpha \frac1{\bar{w}}+ \beta = 0 \Rightarrow \varepsilon + \bar{\alpha}\bar{w}+\alpha w + \beta |w|^2 = 0.$
\begin{liste}
\item[Fall 1:] $\beta = 0 \rightarrow$ Gerade.
\item[Fall 2:] $\beta \not=0$. Dann: $\frac{\varepsilon}{\beta} + \overline{\left(\frac{\alpha}{\beta}\right)} \bar{w} + \frac{\alpha}{beta}w + |w|^2 = 0 \rightarrow$ Kreis.
\end{liste}
\end{beweis}

\begin{beispiel}
Bestimme ein $T \in \mathcal{M}$ mit: $T(\partial \mathbb{D}) = \MdR \cup \{ \infty \}. z_1 = 1; \ z_2=i; \ z_3 = -1. T(z) := DV(z,1,i,-1) = -i\frac{z-1}{z+1}$. 15.7 $\Rightarrow T(1) = 0; \ T(i) = 1; \ T(-1) = \infty.$ 15.8 $\Rightarrow T(\partial \mathbb{D}) = \MdR \cup \{ \infty \}.$
\end{beispiel}

\end{document}
