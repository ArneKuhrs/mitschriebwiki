\documentclass[a4paper,twoside,DIV15,BCOR12mm]{scrbook}

\usepackage[utf8]{inputenc}
\usepackage{ngerman, url}
\usepackage{hyperref}
\usepackage{microtype}
\usepackage{tikz}
\usepackage{schnaubelt}
\usepackage{analysis}

\pdfinfo{
	/Author (Die Mitarbeiter von http://mitschriebwiki.nomeata.de/)
	/Title   (Funktionentheorie I)
	/Subject (Funktionentheorie I)
	/Keywords (Analysis)
}

\author{Die Mitarbeiter von \url{http://mitschriebwiki.nomeata.de/}}
\title{Funktionentheorie I}
\makeindex

\begin{document}

\maketitle

\chapter*{Vorwort}

\section*{Über dieses Skriptum}
Dies ist ein Mitschrieb der Vorlesung \glqq Funktionentheorie I\grqq\
von Herrn Prof. Dr. Schnaubelt im Sommersemester 2009 an der Universität Karlsruhe (TH).
%Die Mitschriebe der Vorlesung werden mit ausdrücklicher Genehmigung von Herrn Schnaubelt hier veröffentlicht,
Herr Schnaubelt ist für den Inhalt nicht verantwortlich.

\section*{Wer}
Beteiligt am Mitschrieb sind Michael Fütterer, Jonathan Zachhuber und
Alexander Koch.

\section*{Wo}
Alle Kapitel inklusive \LaTeX-Quellen können unter \url{http://mitschriebwiki.nomeata.de} abgerufen werden.
Dort ist ein \emph{Wiki} eingerichtet und von Joachim Breitner um die \LaTeX-Funktionen erweitert.
Das heißt, jeder kann Fehler nachbessern und sich an der Entwicklung
beteiligen. Auf Wunsch ist auch ein Zugang über \emph{Subversion} möglich.


\tableofcontents




\chapter{Komplexe Differenzierbarkeit}

\section{Vorbemerkungen zu \C}


%TODO Anfang

\begin{dfn} \label{dfn1.1}
  Eine Funk tion $f\colon D \to \C$ heißt \emph{(komplex) differenzierbar} in $z_0 \in D$, wenn
  \[\lim_{z\to z_0}{\frac{f(z)-f(z_0)}{z-z_0}} \ad f^\prime(z_0)\]
  existiert. Dann heißt $f^\prime(z_0)$ die \emph{Ableitung} von $f$ bei $z_0$. Wenn $f$ bei allen $z_0\in D$ komplex
  differenzierbar ist, dann heißt $f$ \emph{holomorph}. Wir schreiben dann $f\in H(D)$. Iterativ definiert man höhere
  Ableitungen.
\end{dfn}

\begin{bem} \label{bem1.2}
\begin{enumerate}
\item Offenbar sind die Funktionen $f(x)=1$, $g(z)=z$ auf $\C$ holomorph mit $f^\prime(z)=0$, $g^\prime(z)=1$.
\item Genau wie in Analysis 1 zeigt man: Seien $f, g\colon D \to \C$ in $z\in D$ differenzierbar und $\alpha, \beta \in
  \C$. Dann sind auch $\alpha f + \beta g$, $f\cdot g$ und $\frac1f$ (wenn $f(z)\neq0$) in $z$ differenzierbar und es gelten die
  bekannten Regeln. Ebenso gilt die Kettenregel.
\item Polynome $p$ sind auf $\C$ und rationale Funktionen $f = \frac{p}{q}$ sind auf $\{z\in\C: q(t) \neq 0\}$ holomorph mit den
  reellen Formeln für $p^\prime$, $f^\prime$ (wobei $q \neq 0$ ein Polynom ist).
\end{enumerate}
\end{bem}

\paragraph{Erinnerung an Analysis 1:} Gegeben seien $\kmplx{a_n}$ ($\nat{n}_0$) und ein $\kmplx{c}$. Die Potenzreihe
\[f(z) = \sum_{n=0}^\infty a_n(z-c)^n\]
konvergiert absolut für alle $z$ mit
\[|z-c| < \rho \da (\ls_{n\ra\infty}\sqrt[n]{|a_n|})^{-1}\in[0,+\infty]\] 
und divergiert, falls $z\notin \bar{B}(c,\rho)$.

Reduktion auf $c=0$: Betrachte
\[h(w) \da f(c+w) = \sum_{n=0}^\infty a_n w^n\]
wobei $w=z-c\in B(0,\rho)$, $z=c+w$.

\begin{satz}[vgl. Analysis 1, Theorem 4.12] \label{satz1.3} Sei
\[f(z) = \sum_{n=0}^\infty a_n(z-c)^n,\ z\in B(c,\rho),\]
eine Potenzreihe mit Konvergenzradius $\rho > 0$. Dann ist $f\in H(B(c,\rho))$ und
\[f^\prime(z) = \sum_{n=1}^\infty n a_n(z-c)^{n-1} =: g(z)\ (\forall z\in B(c,\rho)),\]
wobei die Potenzreihe $g$ den gleichen Konvergenzradius $\rho$ hat.

Sei $\nat{m}$. Iterativ folgt:
\[\exists f^{(m)}(z) = \sum_{n=m}^\infty n\dotsc (n-m+1) a_n(z-c)^{n-m}, \forall z\in B(c,\rho).\]
\end{satz}
\begin{proof} Wie in Analysis 1: $g$ hat Konvergenzradius $\rho$. Sei oBdA $c = 0$.

Seien $z\in B(0,\rho)$, $\ep>0$, $r>0$ mit $|z|<r<\rho$. Sei $w\in \bar{B}(0,r)$ mit $w\neq z$.

Da $\sum_{n=1}^\infty na_nr^{n-1}$ absolut konvergiert, existiert ein $N = \nat{N_\ep}$ mit
\[0 \leq \sum_{n=N+1}^\infty n |a_n| r^{n-1} \leq \ep.\tag{$\ast$}\]

Ferner:
\[0\leq d(w) \da \left| \frac{f(w)-f(z)}{w-z} - g(z)\right| = \left| \sum_{n=1}^\infty a_n\underbrace{(\frac{w^n-z^n}{w-z} - z^{n-1})}_{\ad p_n(w)}\right|\]

% sieht irgendwie scheiße aus, vielleicht sollte man einfach "mit p(w) = .... in den nächsten Absatz schreiben oder so ....
% hab ich mal gemacht...

mit $p_n(w) = w^{n-1}+zw^{n-2}+\dotsc+z^{n-1}-nz^{n-1}$.
Dabei gelten:
\begin{itemize}
\item $p_n(w) \ra 0$, $w\ra z$ (für jedes feste $\nat{n}$).
\item $|p_n(w)| \leq r^{n-1} + rr^{n-2} + \dotsc + r^{n-1} + nr^{n-1} = 2nr^{n-1}$ \hfill($\ast\ast$)
\end{itemize}

Damit folgt:
\begin{eqnarray*}
0&\leq d(w) \leq \sum_{n=1}^\infty |a_n||p_n(w)| \stackrel{(\ast\ast)}{\leq} \sum_{n=1}^N |a_n||p_n(w)| + \sum_{n=N+1}^\infty 2n|a_n|r^{n-1}\\
&\stackrel{(\ast)}{\leq} N \max\{|a_1||p_1(w)|,\dotsb,|a_n||p_n(w)|\} + 2\ep \ra 0,\ w\ra z\ (N = N_\ep\ \text{fest!})
\end{eqnarray*}

$\displaystyle\folgt \lim_{w\ra z} d(w) \leq 2\ep$. Da $\ep>0$ beliebig war, folgt $\displaystyle\lim_{w\ra z} d(w) = 0$.
\end{proof}

\paragraph{Beispiele} mit $\rho = \infty$.\begin{enumerate}
\item $\displaystyle \exp(z) \da \sum_{n=0}^\infty \frac{z^n}{n!},\ \exp^\prime(z) = \sum_{n=1}^\infty \frac{z^{n-1}}{(n-1)!} = \exp(z)$.
\item $\displaystyle \sin(z) = \sum_{n=0}^\infty \frac{(-1)^n}{(2n+1)!}z^{2n+1},\ \sin^\prime(z) = \sum_{n=0}^\infty \frac{(-1)^n}{(2n)!}z^{2n} = \cos(z).$
\item $\displaystyle \cos(z) = \sum_{n=0}^\infty \frac{(-1)^n}{(2n)!}z^{2n},\newline \cos^\prime(z) = \sum_{n=1}^\infty \frac{(-1)^n}{(2n-1)!}z^{2n-1} \stackrel{l=n-1}{=} \sum_{l=0}^\infty \frac{(-1)^{l+1}}{(2l+1)!}z^{2l+1} = -\sin(z).$
\end{enumerate}

%eine Trennlinie oder so?

\noindent Seien $f:D\ra\C$, $z_0\in D$, $z\in D$. Setze wieder $u = \Re f$, $v = \Im f$, $x_0 = \Re z_0$, $y_0 = \Im z_0$, also
\[f(x,y) = u(x,y) + \i v(x,y) = \cmplx{u(x,y)}{v(x,y)}.\]
Sei $z\neq z_0$ und $f$ bei $z_0$ komplex differenzierbar. Dann gilt:
\[\frac{1}{(z-z_0)}|f(z)-f(z_0)-f^\prime(z_0)(z-z_0)| = \left|\frac{f(z)-f(z_0)}{z-z_0} - f^\prime(z_0)\right| \ra 0,\ z\ra z_0.\]

Die Zahl $\kmplx{f^\prime(z_0)}$ kann als $\C$-lineare Abbildung $w\mapsto f^\prime(z_0)w$ aufgefasst werden. Diese ist dann auch $\R$-linear auf $\R^2$, kann also durch eine reelle $2\times 2$-Matrix dargestellt werden. Nach Analysis 2 ist nun $f$ in $z_0 = \cmplx{x}{y}$ reell differenzierbar und somit existieren die partiellen Ableitungen von $u$ und $v$ und es gilt:
\[f^\prime(x_0,y_0) = \left(\!\!\!\begin{array}{c c}\frac{\partial u}{\partial x} (x_0,y_0) & \frac{\partial u}{\partial y} (x_0,y_0)\\
\frac{\partial v}{\partial x} (x_0,y_0) & \frac{\partial v}{\partial y} (x_0,y_0)\end{array}\!\!\!\right).\tag{+}\]

\begin{satz} \label{satz1.4}
  Sei $f\colon D\ra\C$, $z_0 = x_0 + \i y_0\in D$. Dann sind äquivalent:
\begin{enumerate}
\item $f$ ist in $z_0$ komplex differenzierbar.
\item $f$ ist in $z_0$ reell differenzierbar und es gelten die \emph{Cauchy-Riemannschen Differenzialgleichungen}
\begin{equation} \label{CR}
\frac{\partial u}{\partial x} (x_0,y_0) = \frac{\partial v}{\partial y} (x_0,y_0),\ \frac{\partial u}{\partial y} (x_0,y_0) =
-\frac{\partial v}{\partial x} (x_0,y_0). \tag{CR}
\end{equation}
Insbesondere ist $f^\prime(z_0)$ schiefsymmetrisch.
\end{enumerate}
\end{satz}

\begin{proof} Die letzte Behauptung folgt aus (+) und (\ref{CR})$_2$.

\noindent $(a) \folgt (b):$ Sei $r > 0$ mit $B(z_0,r) \subseteq D$, $\rat{t}$ mit $0 < |t| < r$. Dann:
\begin{align}
f^\prime(z_0) &= \lim_{t\ra 0} \frac{1}{t} (f(z_0+t) - f(z_0))\nonumber\\
&=  \lim_{t\ra 0} \left[\frac{1}{t} (u(x_0+t,y_0)-u(x_0,y_0)) + \frac{\i}{t} (v(x_0+t,y_0)-v(x_0,y_0))\right]\nonumber\\
&=\frac{\partial u}{\partial x} (x_0,y_0) + \i\frac{\partial v}{\partial x} (x_0,y_0)\\
f^\prime(z_0) &= \lim_{t\ra 0} \underbrace{\frac{1}{\i t}}_{=-\frac{\i}{t}} (f(z_0+\i t) - f(z_0))\nonumber\\
&=  \lim_{t\ra 0} \left[-\i\frac{1}{t} (u(x_0,y_0+t)-u(x_0,y_0)) + \frac{\i}{\i t} (v(x_0,y_0+t)-v(x_0,y_0))\right]\nonumber\\
&=-\i\frac{\partial u}{\partial y} (x_0,y_0) + \frac{\partial v}{\partial y} (x_0,y_0)
\end{align}
Vergleichen von Real- und Imaginärteil liefert (\ref{CR}).

\noindent $(b) \folgt (a):$ Setze
\[w=\frac{\partial u}{\partial x} (x_0,y_0) + \i\frac{\partial v}{\partial x}\stackrel{\text{(\ref{CR})}}{=} \frac{\partial v}{\partial y} (x_0,y_0) - \i\frac{\partial u}{\partial y}(x_0,y_0) \in\C.\]
Dann:
\[w(z-z_0) = (\Re w)(x-x_0) - (\Im w)(y-y_0) + \i((\Re w)(y-y_0) + (\Im w)(x-x_0))\]
\[\stackrel{\text{Def. $w$}}{=} \cmplx{\frac{\partial u}{\partial x} (x_0,y_0)(x-x_0) + \frac{\partial u}{\partial y}(x_0,y_0)(y-y_0)}{\frac{\partial v}{\partial y} (x_0,y_0)(y-y_0) + \frac{\partial v}{\partial x}(x_0,y_0)(x-x_0)}\ \text{(in $\R^2$)}.\]
\begin{align*}
\displaystyle\folgt &|f(z)-f(z_0)-w(z-z_0)|\frac{1}{|z-z_0|}\\
&\qquad= \left|\cmplx{x-x_0}{y-y_0}\right|_2^{-1} \left|\cmplx{u(x,y)-u(x_0,y_0)-(\nabla u(x_0,y_0)|\cmplx{x-x_0}{y-y_0})}{v(x,y)-v(x_0,y_0)-\left(\nabla v(x_0,y_0)|\cmplx{x-x_0}{y-y_0}\right)}\right|_2 \ra 0,
\end{align*}% große oder kleine Klammern besser?

$\displaystyle\underbrace{(x,y)}_{=z} \ra \underbrace{(x_0,y_0)}_{=z_0}$, da $u$, $v$ differenzierbar.
\end{proof}

\begin{bsp} \label{bsp1.5}
\begin{enumerate}
\item $f(z) = \bar{z}$, $\kmplx{z}$ ist nirgends komplex differenzierbar, obwohl $f(x,y) = \cmplx{x}{-y}$ reell $C^\infty$ ist. Denn $u(x,y) = x$, $v(x,y) = -y$; also 
\[\frac{\partial u}{\partial x}(x,y) = 1 \neq -1 = \frac{\partial v}{\partial y}(x,y),\]
was (\ref{CR})$_1$ widerspricht.
\item $f(z) = |z|^2 = x^2 + y^2$, $\kmplx{z}$, ist nur in $z=0$ komplex differenzierbar, denn: $u(x,y) = x^2+y^2$, $v(x,y) = 0$ und somit:
\[\frac{\partial u}{\partial x}(x,y) = 2x \stackrel{!}{=} \frac{\partial v}{\partial y}(x,y) = 0 \gdw x=0,\]
\[\frac{\partial u}{\partial y}(x,y) = 2y \stackrel{!}{=} -\frac{\partial v}{\partial x}(x,y) = 0 \gdw y=0.\]
\item $\displaystyle f(z) = \frac{1}{z} = \frac{\bar{z}}{|z|^2} = \underbrace{\frac{x}{x^2+y^2}}_{=u} + \i\underbrace{\frac{-y}{x^2+y^2}}_{=v}$ ist holomorph für $z\neq 0$ (Bem.~\ref{bem1.2}).
\end{enumerate}
\end{bsp}

\begin{bem} \label{bem1.6}
Sei $f$ in $z = x+\i y$ komplex differenzierbar. Nach (+) und (\ref{CR}) gilt:
\begin{equation}
A \da f^\prime(z) = \left(\!\!\!\begin{array}{c c}\frac{\partial u}{\partial x}(x,y)&\frac{\partial u}{\partial y}(x,y)\\-\frac{\partial u}{\partial x}(x,y)&\frac{\partial u}{\partial x}(x,y)\end{array}\!\!\!\right) = \left(\!\!\!\begin{array}{c c}\frac{\partial v}{\partial y}(x,y)&-\frac{\partial v}{\partial x}(x,y)\\\frac{\partial v}{\partial x}(x,y)&\frac{\partial v}{\partial y}(x,y)\end{array}\!\!\!\right)
\end{equation}
Also: $\rho \da \displaystyle \det A = \frac{\partial u}{\partial x}(x,y)^2 + \frac{\partial u}{\partial y}(x,y)^2 = \frac{\partial v}{\partial x}(x,y)^2 + \frac{\partial v}{\partial y}(x,y)^2 \geq 0$,
\[f^\prime(z) \neq 0 \gdw \det A > 0.\]
Ferner: $A^TA = (\det A)\I$. Sei $f^\prime(z) \neq 0$. Dann gilt
\[\frac{1}{\sqrt{\rho}}A \text{ orthogonal} \tag{$*$}\]
\[\folgt \abs{Av}_2 = \sqrt{\rho} \abs{v}_2 \quad (\forall v \in \R^2). \tag{$**$}\]
Sei $\gamma_j \in C^1((-1,1), \R^2)$ eine Kurve in $D$ mit $\gamma_j(0) = (x, y)$, $\gamma_j^\prime(0) \ad v_j \in \R^2 \setminus \{(0,0)\}$ ($j=1,2$). Dann ist $Av_j = (f \circ \gamma_j)^\prime(0) = f^\prime(x, y) \gamma_j^\prime(0)$ ein Tangentenvektor der Bildkurve $f \circ \gamma_j$ bei $f(x, y)$ ($j=1,2$). Weiter:
\[\frac{(v_1|v_2)}{\abs{v_1}_2 \abs{v_2}_2} \nach{=}{($*$),($**$)} \frac{\frac1\rho (Av_1|Av_2)}{\frac1\rho \abs{Av_1}_2 \abs{Av_2}_2} \folgtwegen{\arccos} \sphericalangle(v_1,v_2) = \sphericalangle(Av_1,Av_2)\]
Hierbei sind die Winkel ohne Orientierung.\\
Also ist der Winkel der Urbildtangenten gleich dem Winkel der Bildtangenten unter $f$. Falls also $f^\prime(z)\neq0$, dann ist $f$ bei $z=x+\i y$ \emph{winkeltreu} (``konform''). Ferner ist $f$ orientierungstreu, da $\det A > 0$.
\end{bem}
%TODO Bild!

\begin{dfn} \label{dfn1.7}
Seien $U, V \subseteq \C$ offen und nichtleer. Sei $f\colon U \to V$ bijektiv und $f\in H(U)$, $f^{-1}\in H(V)$. Dann heißt $f$ \emph{biholomorph}. (Dann heißen $U$ und $V$ auch ``konform äquivalent''.)
\end{dfn}

\begin{satz} \label{satz1.8}
\begin{enumerate}
\item Seien $U, V \subseteq \C$ offen und nichtleer, $f\colon U \to V$ biholomorph. Dann ist $f^\prime(z)\neq 0$ für alle $z\in U$ und es gilt
\[(f^{-1})^\prime(f(z)) = (f^{-1})^\prime(w) = \frac{1}{f^\prime(f^{-1}(w))} = \frac{1}{f^\prime(z)} \quad (\forall z\in U,\ \forall w=f(z) \in V)\]
\item Seien $f\in H(D) \cap C^1(D, \R^2)$, $z_0\in D$, $f^\prime(z_0)\neq0$. Dann exisieren offene $U, V \subseteq \C$ mit $z_0\in U$, $f(z_0)\in V$, sodass $f\colon U \to V$ biholomorph ist. Somit ist a) auf $f$ für alle $z\in U$ und $w=f(z)\in V$ anwendbar.
\end{enumerate}
\end{satz}
\begin{proof}
\begin{enumerate}
\item Nach Bem.~\ref{bem1.2} und $z=f^{-1}(f(z))$ ($\forall z\in U$) folgt $1 = (f^{-1})^\prime(f(z))f^\prime(z)$. Durchdividieren ergibt Behauptung a).
\item Nach Bem.~\ref{bem1.6} ist $f^\prime(z_0)$ als $2\times2$-Matrix invertierbar. Der Umkehrsatz aus Analysis 2 liefert Behauptung b).
\end{enumerate}
\end{proof}

\begin{dfn*} Eine Funktion $u\in C^2(D, \R)$ heißt \emph{harmonisch} auf $D$, wenn für alle $(x, y) \in D$ gilt:
\[\triangle u(x, y) \da \frac{\partial^2u}{\partial x^2}(x, y) + \frac{\partial^2u}{\partial y^2}(x, y) = 0.\]
\end{dfn*}

\begin{satz} \label{satz1.9}
\begin{enumerate}
\item Sei $f\in H(D) \cap C^2(D, \R^2)$. Dann sind $u=\Re f$, $v = \Im f$ harmonisch auf $D$.
\item Sei $u \in C^2(D, \R)$ auf $D$ harmonisch, $B_0 \da B((x_0,y_0),r) \subseteq D$ für ein $r>0$, $(x_0,y_0)\in D$. Dann existiert ein $f\in H(B_0)$ mit $u = \Re f$.
\end{enumerate}
\end{satz}
\begin{proof}
\begin{enumerate}
\item Der Satz von Schwarz aus Analysis 2 (Satz 2.21) liefert
\[\frac{\partial^2u}{\partial x^2} + \frac{\partial^2u}{\partial y^2} \nach{=}{(\ref{CR})}
\frac{\partial}{\partial x} \frac{\partial v}{\partial y} - \frac{\partial}{\partial y} \frac{\partial v}{\partial x} = 0.\]
\item Setze für $(x, y) \in B_0$
\[v(x, y) = - \Integral{x_0}{x}{\frac{\partial u}{\partial y}(s, y_0)}{s} +
\Integral{y_0}{y}{\frac{\partial u}{\partial x}(x, s)}{s}.\]
Beachte: die Strecken von $(x_0,y_0)$ nach $(x,y_0)$ und von $(x,y_0)$ nach $(x,y)$ liegen in $B_0$. Analysis 1 und zwei liefern: $v\in C^1(B_0)$ und
\begin{multline*}
\frac{\partial v}{\partial x}(x, y) = - \frac{\partial u}{\partial y}(x, y_0) + \Integral{y_0}{y}{\underbrace{\frac{\partial^2 u}{\partial x^2}(x, s)}_{\nach{=}{n.V.} - \frac{\partial^2 u}{\partial y^2}(x,s)}}{s}\\
= - \frac{\partial u}{\partial y}(x, y_0) - \frac{\partial u}{\partial y}(x, y) + \frac{\partial u}{\partial y}(x, y_0)
= - \frac{\partial u}{\partial y}(x, y)
\end{multline*}
\folgt (\ref{CR})$_2$.
Ferner $\frac{\partial v}{\partial y}(x, y) = \frac{\partial u}{\partial x}(x, y)$ \folgt (\ref{CR}) gilt. Mit Satz~\ref{satz1.4} folgt: $f=u+\i v$ ist auf $B_0$ holomorph.
\end{enumerate}
\end{proof}

\begin{bsp*}
Sei $f(z) = z^3 = (x+\i y)^3 = x^3 + 3\i x^2 y - 3xy^2 - \i y^3$. Satz~\ref{satz1.9} liefert: $u(x,y) = \Re f(x, y) = x^3 - 3xy^2$ ist harmonisch auf $\C$.
\end{bsp*}


\section{Elementare Funktionen}

\subsection{Möbiustransformationen}

\noindent Sei $A = \mat{a}{b}{c}{d} \in \mathrm{M}_{22}(\C)$ mit $\det A = ad-bc \neq 0$ (dann ist $c\neq0$ oder
$d\neq0$). Setze
\[m_A(z) = \frac{az+b}{cz+d} \quad \  \text{für} \quad z \in D_A \da
\begin{cases} \C \setminus \left\{-\frac{d}{c}\right\} ,& c \neq 0, \\ \C ,& c = 0. \end{cases}\]
$m_A$ heißt \emph{Möbius-Transformation}. Offensichtlich ist $m_A \in H(D_A)$.

\paragraph{Eigenschaften.} Sei $A$ wie oben und $\tilde{A} = \mat{\tilde{a}}{\tilde{b}}{\tilde{c}}{\tilde{d}} \in
\mathrm{M}_{22}(\C)$ mit $\det \tilde{A} \neq 0$.
\begin{enumerate}
\item Es gilt
  \[m_A(z) = z \; (\forall z \in D_A) \gdw A = \mat{a}{0}{0}{a} \text { für ein } a \in \C \setminus \{0\}.\]
  \begin{proof}
    "`$\Leftarrow$"': einsetzen! "`$\Rightarrow$"': Für alle $z\in D$ gilt:
    \[m_A(z)=z \folgt cz^2 + (d-a)z - b = 0 \text{ (}\forall z\in D_A\text{) }\]
    \[\folgt c=0, d=a, b=0 \folgt A = \mat{a}{0}{0}{a}.\]
  \end{proof}
\item $m_{\alpha A} = m_A$ ($\forall \alpha \in \C$)
\item $m_A(m_{\tilde{A}}(z)) = m_{A\tilde{A}}(z)$ (soweit alle definiert).
\begin{proof} Für passende $z$:
\[m_A(m_{\tilde{A}}(z)) = \frac{a\frac{\tilde{a}z+\tilde{b}}{\tilde{c}z+\tilde{d}}+b}{c\frac{\tilde{a}z+\tilde{b}}{\tilde{c}z+\tilde{d}}+d} = \frac{(a\tilde{a}+b\tilde{c})z+(a\tilde{b}+b\tilde{d})}{(c\tilde{a}+d\tilde{c})z+(c\tilde{b}+d\tilde{d})} = m_{A\tilde{A}}(z).\]
\end{proof}
\item Es gilt: $\displaystyle A^{-1} = \frac{1}{ad-bc}\mat{d}{-b}{-c}{a}$. Damit folgt: $\displaystyle D_{A^{-1}}=
  \begin{cases} \C\setminus\left\{\frac{a}{c}\right\},&c\neq 0,\\\C,&c=0.\end{cases}$

Zeige:
\[m_A(D_A) \subseteq D_{A^{-1}},\tag{$\ast$}\]
\[m_{A^{-1}}(D_{A^{-1}}) \subseteq D_A.\tag{$\ast\ast$}\]
Also folgt aus (c): $m_A(D_A) = D_{A^{-1}}$ (wende auf ($**$) $m_A$ an), $m_{A^{-1}}(D_{A^{-1}}) = D_A$ (wende auf ($*$) $m_{A^{-1}}$ an) und $(m_A)^{-1} = m_{A^{-1}}$.

Insbesondere sind $m_A\colon D_A\ra D_{A^{-1}}$, $m_{A^{-1}}\colon D_{A^{-1}}\ra D_A$ biholomorph.
\item Sei $c=0$ (also $d\neq0$). Dann ist $m_A(z) = \frac{a}{d}z + \frac{b}{d}$ eine affine Abbildung, also $m_A = T \circ S$
  mit $Tw=w+\frac{d}{d}$ (Translation) und $Sw = \frac{a}{d}w$ (Drehstreckung). Sei nun $c\neq0$. Dann gilt $m_A = A_2 \circ J
  \circ A_1$, wobei $A_1w = cw+d$, $A_2w = \frac{a}{c} - \frac{ad-bc}{c}w$ (affin) und $Jw = \frac1w$ ($w\neq0$) (Inversion).
  \begin{proof}
    Es gilt $\displaystyle\frac{az+b}{cz+d} = \frac{a}{c} - \frac{ad-bc}{c}\frac{1}{cz+d} = A_2(J(A_1(z)))$.
  \end{proof}
\end{enumerate}


\begin{dfn} \label{dfn1.10}
  Setze $\C_\infty = \C \cup \{\infty\}$, $\R_\infty = \R \cup \{\infty\}$, wobei gelten soll:
\begin{eqnarray*}
&\forall\kmplx{z}: z+\infty = \infty + z = \infty,\ \frac{z}{\infty} = 0,\\
&\forall\kmplx{z}\setminus \{0\}: z\cdot\infty = \infty\cdot z = \infty,\ \frac{z}{0} = \infty.
\end{eqnarray*}
\textbf{Verboten:} $0\cdot\infty$, $\infty\cdot 0$, $\frac{0}{0}$, $\frac{\infty}{\infty}$.

Schreibe $z_n \ra \infty$ wenn $|z_n| \ra +\infty\ (n\ra\infty)$.

Beachte: Dieses $\infty \neq \pm\infty$ in $\R$ aus Analysis 1.
\end{dfn}

Setze bezüglich dieser ``Konvergenz'' $m_A$ ``stetig'' fort durch
\[m_A (\infty) \da \left\{\!\!\!\begin{array}{r @{,\ } l}\frac{a}{c}&c\neq 0\\\infty&c=0\end{array}\right..\]
Beachte:
\[m_A(z)=\frac{a+\frac{b}{z}}{c+\frac{d}{z}}\ (z\neq 0),\ m_A\left(-\frac{d}{c}\right) = \infty\]
(beachte: $-\frac{ad}{c} + b\neq 0$, da $\det A\neq 0$).

Mit etwas Rechnung folgt: $m_A\colon\C_\infty\ra\C_\infty$ ist bijektiv mit $(m_A)^{-1} = m_{A^{-1}}\colon\C_\infty\ra\C_\infty$.

Sei $\mathcal{M}\da\{m_A\colon\C_\infty\ra\C_\infty\mid\det A\neq 0\}$. Mit obigen Eigenschaften (und etwas Rechnung bezüglich $\infty$) folgt:
\begin{itemize}
\item $\mathcal{M}$ ist eine Gruppe bezüglich der Komposition.
\item $\Phi\colon \mathrm{GL}(2,\C)\ra\mathcal{M}$, $A\mapsto m_A$ ist ein surjektiver Gruppenhomomorphismus mit
  $\mathrm{Kern }\Phi = \{\alpha \I \mid \kmplx{\alpha}\setminus\{0\}\}.$
\end{itemize}


\end{document}
