\documentclass[a4paper,twoside,DIV15,BCOR12mm]{scrbook}

\usepackage[utf8]{inputenc}
\usepackage{ngerman, url}
\usepackage{hyperref}
\usepackage{microtype}
\usepackage{tikz}
\usepackage{schnaubelt}
\usepackage{analysis}

\pdfinfo{
	/Author (Die Mitarbeiter von http://mitschriebwiki.nomeata.de/)
	/Title   (Funktionentheorie I)
	/Subject (Funktionentheorie I)
	/Keywords (Analysis)
}

\author{Prof. Dr. Roland Schnaubelt}
\publishers{Die Mitarbeiter von \url{http://mitschriebwiki.nomeata.de/}}
\title{Funktionentheorie I}
\date{Sommersemester 2009}
\makeindex

\begin{document}

\maketitle

\chapter*{Vorwort}

\section*{Über dieses Skriptum}
Dies ist ein Mitschrieb der Vorlesung \glqq Funktionentheorie I\grqq\
von Herrn Prof. Dr. Schnaubelt im Sommersemester 2009 an der Universität Karlsruhe (TH).
%Die Mitschriebe der Vorlesung werden mit ausdrücklicher Genehmigung von Herrn Schnaubelt hier veröffentlicht,
Herr Schnaubelt ist für den Inhalt nicht verantwortlich.

\section*{Wer}
Beteiligt am Mitschrieb sind Michael Fütterer, Jonathan Zachhuber und
Alexander Koch.

\section*{Wo}
Alle Kapitel inklusive \LaTeX-Quellen können unter \url{http://mitschriebwiki.nomeata.de} abgerufen werden.
Dort ist ein \emph{Wiki} eingerichtet und von Joachim Breitner um die \LaTeX-Funktionen erweitert.
Das heißt, jeder kann Fehler nachbessern und sich an der Entwicklung
beteiligen. Auf Wunsch ist auch ein Zugang über \emph{Subversion} möglich.


\tableofcontents




\chapter{Komplexe Differenzierbarkeit}

\section{Vorbemerkungen zu \C}


%TODO Anfang

\begin{dfn} \label{dfn1.1}
  Eine Funktion $f\colon D \to \C$ heißt \emph{(komplex) differenzierbar} in $z_0 \in D$, wenn
  \[\lim_{z\to z_0}{\frac{f(z)-f(z_0)}{z-z_0}} \ad f^\prime(z_0)\]
  existiert. Dann heißt $f^\prime(z_0)$ die \emph{Ableitung} von $f$ bei $z_0$. Wenn $f$ bei allen $z_0\in D$ komplex
  differenzierbar ist, dann heißt $f$ \emph{holomorph}. Wir schreiben dann $f\in H(D)$. Iterativ definiert man höhere
  Ableitungen.
\end{dfn}

\begin{bem} \label{bem1.2}
\begin{enumerate}
\item Offenbar sind die Funktionen $f(x)=1$, $g(z)=z$ auf $\C$ holomorph mit $f^\prime(z)=0$, $g^\prime(z)=1$.
\item Genau wie in Analysis 1 zeigt man: Seien $f, g\colon D \to \C$ in $z\in D$ differenzierbar und $\alpha, \beta \in
  \C$. Dann sind auch $\alpha f + \beta g$, $f\cdot g$ und $\frac1f$ (wenn $f(z)\neq0$) in $z$ differenzierbar und es gelten die
  bekannten Regeln. Ebenso gilt die Kettenregel.
\item Polynome $p$ sind auf $\C$ und rationale Funktionen $f = \frac{p}{q}$ mit einem Polynom $q \neq 0$ sind auf $\{z\in\C:
  q(t) \neq 0\}$ holomorph mit den reellen Formeln für $p^\prime$, $f^\prime$.
\end{enumerate}
\end{bem}

\paragraph{Erinnerung an Analysis 1:} Gegeben seien $\kmplx{a_n}$ ($\nat{n}_0$) und ein $\kmplx{c}$. Die Potenzreihe
\[f(z) = \sum_{n=0}^\infty a_n(z-c)^n\]
konvergiert absolut für alle $z$ mit
\[|z-c| < \rho \da \left(\ls_{n\ra\infty}\sqrt[n]{|a_n|}\right)^{-1}\in[0,+\infty]\] 
und divergiert, falls $z\notin \bar{B}(c,\rho)$.

Reduktion auf $c=0$: Betrachte
\[h(w) \da f(c+w) = \sum_{n=0}^\infty a_n w^n,\]
wobei $w=z-c\in B(0,\rho)$, $z=c+w$.

\begin{satz}[vgl. Analysis 1, Theorem 4.12] \label{satz1.3} Sei
\[f(z) = \sum_{n=0}^\infty a_n(z-c)^n,\ z\in B(c,\rho),\]
eine Potenzreihe mit Konvergenzradius $\rho > 0$. Dann ist $f\in H(B(c,\rho))$ und
\[f^\prime(z) = \sum_{n=1}^\infty n a_n(z-c)^{n-1} =: g(z)\ (\forall z\in B(c,\rho)),\]
wobei die Potenzreihe $g$ den gleichen Konvergenzradius $\rho$ hat.

Sei $\nat{m}$. Iterativ folgt:
\[\exists f^{(m)}(z) = \sum_{n=m}^\infty n\cdot\dotsc\cdot (n-m+1) a_n(z-c)^{n-m}, \forall z\in B(c,\rho).\]
\end{satz}
\begin{proof} Wie in Analysis 1 zeigt man: $g$ hat Konvergenzradius $\rho$. Sei oBdA $c = 0$.

Seien $z\in B(0,\rho)$, $\ep>0$, $r>0$ mit $|z|<r<\rho$. Sei $w\in \bar{B}(0,r)$ mit $w\neq z$.

Da $\sum_{n=1}^\infty na_nr^{n-1}$ absolut konvergiert, existiert ein $N = \nat{N_\ep}$ mit
\[0 \leq \sum_{n=N+1}^\infty n |a_n| r^{n-1} \leq \ep.\tag{$\ast$}\]

Ferner:
\[0\leq d(w) \da \left| \frac{f(w)-f(z)}{w-z} - g(z)\right| = \left| \sum_{n=1}^\infty a_n\underbrace{(\frac{w^n-z^n}{w-z} - z^{n-1})}_{\ad p_n(w)}\right|\]
mit $p_n(w) = w^{n-1}+zw^{n-2}+\dotsc+z^{n-1}-nz^{n-1}$.
Dabei gelten:
\begin{itemize}
\item $p_n(w) \ra 0$, $w\ra z$ (für jedes feste $\nat{n}$)
\item $|p_n(w)| \leq r^{n-1} + rr^{n-2} + \dotsc + r^{n-1} + nr^{n-1} = 2nr^{n-1}$ \hfill($\ast\ast$)
\end{itemize}

Damit folgt:
\begin{align*}
0&\leq d(w) \leq \sum_{n=1}^\infty |a_n||p_n(w)| \stackrel{(\ast\ast)}{\leq} \sum_{n=1}^N |a_n||p_n(w)|\ + \sum_{n=N+1}^\infty 2n|a_n|r^{n-1}\\
&\stackrel{(\ast)}{\leq} N \max\{|a_1||p_1(w)|,\dotsb,|a_n||p_n(w)|\} + 2\ep \ra 2\ep\ (w\ra z)\ (N = N_\ep\ \text{fest!})
\end{align*}

$\displaystyle\folgt \lim_{w\ra z} d(w) \leq 2\ep$. Da $\ep>0$ beliebig war, folgt $\displaystyle\lim_{w\ra z} d(w) = 0$.
\end{proof}

\paragraph{Beispiele} mit $\rho = \infty$.\begin{enumerate}
\item $\displaystyle \exp(z) \da \sum_{n=0}^\infty \frac{z^n}{n!},\ \exp^\prime(z) = \sum_{n=1}^\infty \frac{z^{n-1}}{(n-1)!} = \exp(z)$.
\item $\displaystyle \sin(z) = \sum_{n=0}^\infty \frac{(-1)^n}{(2n+1)!}z^{2n+1},\ \sin^\prime(z) = \sum_{n=0}^\infty \frac{(-1)^n}{(2n)!}z^{2n} = \cos(z).$
\item $\displaystyle \cos(z) = \sum_{n=0}^\infty \frac{(-1)^n}{(2n)!}z^{2n},\newline \cos^\prime(z) = \sum_{n=1}^\infty \frac{(-1)^n}{(2n-1)!}z^{2n-1} \stackrel{l=n-1}{=} \sum_{l=0}^\infty \frac{(-1)^{l+1}}{(2l+1)!}z^{2l+1} = -\sin(z).$
\end{enumerate}

%eine Trennlinie oder so?

\noindent Seien $f\colon D\ra\C$, $z_0\in D$, $z\in D$. Setze wieder $u = \Re f$, $v = \Im f$, $x_0 = \Re z_0$, $y_0 = \Im z_0$, also
\[f(x,y) = u(x,y) + \i v(x,y) = \cmplx{u(x,y)}{v(x,y)}.\]
Sei $z\neq z_0$ und $f$ bei $z_0$ komplex differenzierbar. Dann gilt:
\[\frac{1}{(z-z_0)}|f(z)-f(z_0)-f^\prime(z_0)(z-z_0)| = \left|\frac{f(z)-f(z_0)}{z-z_0} - f^\prime(z_0)\right| \ra 0,\ z\ra z_0.\]

Die Zahl $\kmplx{f^\prime(z_0)}$ kann als $\C$-lineare Abbildung $w\mapsto f^\prime(z_0)w$ aufgefasst werden. Diese ist dann auch $\R$-linear auf $\R^2$, kann also durch eine reelle $2\times 2$-Matrix dargestellt werden. Nach Analysis 2 ist nun $f$ in $z_0 = (x,y)$ reell differenzierbar und somit existieren die partiellen Ableitungen von $u$ und $v$ und es gilt:
\[f^\prime(x_0,y_0) = \left(\!\!\!\begin{array}{c c}\frac{\partial u}{\partial x} (x_0,y_0) & \frac{\partial u}{\partial y} (x_0,y_0)\\
\frac{\partial v}{\partial x} (x_0,y_0) & \frac{\partial v}{\partial y} (x_0,y_0)\end{array}\!\!\!\right).\tag{+}\]

\begin{satz} \label{satz1.4}
  Sei $f\colon D\ra\C$, $z_0 = x_0 + \i y_0\in D$. Dann sind äquivalent:
\begin{enumerate}
\item $f$ ist in $z_0$ komplex differenzierbar.
\item $f$ ist in $z_0$ reell differenzierbar und es gelten die \emph{Cauchy-Riemannschen Differentialgleichungen}
\begin{equation} \label{CR}
\frac{\partial u}{\partial x} (x_0,y_0) = \frac{\partial v}{\partial y} (x_0,y_0),\quad \frac{\partial u}{\partial y} (x_0,y_0) =
-\frac{\partial v}{\partial x} (x_0,y_0). \tag{CR}
\end{equation}
Insbesondere ist $f^\prime(z_0)$ schiefsymmetrisch.
\end{enumerate}
\end{satz}

\begin{proof} Die letzte Behauptung folgt aus (+) und (\ref{CR})$_2$.

\noindent $(a) \folgt (b)$: Sei $r > 0$ mit $B(z_0,r) \subseteq D$, $\rat{t}$ mit $0 < |t| < r$. Dann gelten
\begin{align}
f^\prime(z_0) &= \lim_{t\ra 0} \frac{1}{t} (f(z_0+t) - f(z_0))\nonumber\\
&=  \lim_{t\ra 0} \left(\frac{1}{t} (u(x_0+t,y_0)-u(x_0,y_0)) + \frac{\i}{t} (v(x_0+t,y_0)-v(x_0,y_0))\right)\nonumber\\
&=\frac{\partial u}{\partial x} (x_0,y_0) + \i\frac{\partial v}{\partial x} (x_0,y_0)
\end{align} und \begin{align}
f^\prime(z_0) &= \lim_{t\ra 0} \underbrace{\frac{1}{\i t}}_{=-\frac{\i}{t}} (f(z_0+\i t) - f(z_0))\nonumber\\
&=  \lim_{t\ra 0} \left(-\i\frac{1}{t} (u(x_0,y_0+t)-u(x_0,y_0)) + \frac{\i}{\i t} (v(x_0,y_0+t)-v(x_0,y_0))\right)\nonumber\\
&=-\i\frac{\partial u}{\partial y} (x_0,y_0) + \frac{\partial v}{\partial y} (x_0,y_0).
\end{align}
Vergleichen von Real- und Imaginärteil liefert (\ref{CR}).

\noindent $(b) \folgt (a)$: Setze
\[w=\frac{\partial u}{\partial x} (x_0,y_0) + \i\frac{\partial v}{\partial x}\stackrel{\text{(\ref{CR})}}{=} \frac{\partial v}{\partial y} (x_0,y_0) - \i\frac{\partial u}{\partial y}(x_0,y_0) \in\C.\]
Dann gilt:
\[w(z-z_0) = (\Re w)(x-x_0) - (\Im w)(y-y_0) + \i((\Re w)(y-y_0) + (\Im w)(x-x_0))\]
\[\stackrel{\text{Def. $w$}}{=} \cmplx{\frac{\partial u}{\partial x} (x_0,y_0)(x-x_0) + \frac{\partial u}{\partial y}(x_0,y_0)(y-y_0)}{\frac{\partial v}{\partial y} (x_0,y_0)(y-y_0) + \frac{\partial v}{\partial x}(x_0,y_0)(x-x_0)}\ \text{(in $\R^2$)}.\]
\begin{align*}
\displaystyle\folgt &|f(z)-f(z_0)-w(z-z_0)|\frac{1}{|z-z_0|}\\
&\qquad= \left|\cmplx{x-x_0}{y-y_0}\right|_2^{-1} \left|\cmplx{u(x,y)-u(x_0,y_0)-\left(\nabla u(x_0,y_0)\mid\cmplx{x-x_0}{y-y_0}\right)}{v(x,y)-v(x_0,y_0)-\left(\nabla v(x_0,y_0)\mid\cmplx{x-x_0}{y-y_0}\right)}\right|_2 \ra 0,
\end{align*}

%$\displaystyle\underbrace{(x,y)}_{=z} \ra \underbrace{(x_0,y_0)}_{=z_0}$,
für $(x_0,y_0)=z_0 \to z=(x,y)$,
da $u$, $v$ differenzierbar.
\end{proof}

\begin{bsp} \label{bsp1.5}
\begin{enumerate}
\item $f(z) = \bar{z}$, $\kmplx{z}$, ist nirgends komplex differenzierbar, obwohl $f(x,y) = \cmplx{x}{-y}$ reell $C^\infty$ ist. Denn $u(x,y) = x$, $v(x,y) = -y$; also 
\[\frac{\partial u}{\partial x}(x,y) = 1 \neq -1 = \frac{\partial v}{\partial y}(x,y),\]
was (\ref{CR})$_1$ widerspricht.
\item $f(z) = |z|^2 = x^2 + y^2$, $\kmplx{z}$, ist nur in $z=0$ komplex differenzierbar, denn hier ist $u(x,y) = x^2+y^2$, $v(x,y) = 0$ und somit:
\[\frac{\partial u}{\partial x}(x,y) = 2x \stackrel{!}{=} \frac{\partial v}{\partial y}(x,y) = 0 \gdw x=0,\]
\[\frac{\partial u}{\partial y}(x,y) = 2y \stackrel{!}{=} -\frac{\partial v}{\partial x}(x,y) = 0 \gdw y=0.\]
\item $\displaystyle f(z) = \frac{1}{z} = \frac{\bar{z}}{|z|^2} = \underbrace{\frac{x}{x^2+y^2}}_{=u} + \i\underbrace{\frac{-y}{x^2+y^2}}_{=v}$ ist holomorph für $z\neq 0$ (Bem.~\ref{bem1.2}).
\end{enumerate}
\end{bsp}

\begin{bem} \label{bem1.6}
Sei $f$ in $z = x+\i y$ komplex differenzierbar. Nach (+) und (\ref{CR}) gilt:
\begin{equation}
A \da f^\prime(z) = \left(\!\!\!\begin{array}{c c}\frac{\partial u}{\partial x}(x,y)&\frac{\partial u}{\partial y}(x,y)\\-\frac{\partial u}{\partial x}(x,y)&\frac{\partial u}{\partial x}(x,y)\end{array}\!\!\!\right) = \left(\!\!\!\begin{array}{c c}\frac{\partial v}{\partial y}(x,y)&-\frac{\partial v}{\partial x}(x,y)\\\frac{\partial v}{\partial x}(x,y)&\frac{\partial v}{\partial y}(x,y)\end{array}\!\!\!\right).
\end{equation}
Also gilt $\rho \da \displaystyle \det A = \frac{\partial u}{\partial x}(x,y)^2 + \frac{\partial u}{\partial y}(x,y)^2 =
\frac{\partial v}{\partial x}(x,y)^2 + \frac{\partial v}{\partial y}(x,y)^2 \geq 0$ und \\ $f^\prime(z) \neq 0 \gdw \det A > 0$.
Ferner ist $A^TA = (\det A)\I$. Sei $f^\prime(z) \neq 0$. Dann gilt
\[\frac{1}{\sqrt{\rho}}A \text{ orthogonal} \tag{$*$}\]
\[\folgt \abs{Av}_2 = \sqrt{\rho} \abs{v}_2 \quad (\forall v \in \R^2). \tag{$**$}\]
Sei $\gamma_j \in C^1((-1,1), \R^2)$ eine Kurve in $D$ mit $\gamma_j(0) = (x, y)$, $\gamma_j^\prime(0) \ad v_j \in \R^2
\setminus \{(0,0)\}$ ($j=1,2$). Dann ist $Av_j = (f \circ \gamma_j)^\prime(0) = f^\prime(x, y) \gamma_j^\prime(0)$ ein
Tangentenvektor der Bildkurve $f \circ \gamma_j$ bei $f(x, y)$ ($j=1,2$). Weiter gilt: \[\frac{(v_1\mid v_2)}{\abs{v_1}_2
  \abs{v_2}_2} \nach{=}{($*$),($**$)} \frac{\frac1\rho (Av_1\mid Av_2)}{\frac1\rho \abs{Av_1}_2 \abs{Av_2}_2},\]
woraus durch Anwenden des Arcuscosinus folgt: $\sphericalangle(v_1,v_2) = \sphericalangle(Av_1,Av_2)$ (Winkel ohne Orientierung).\\
Also ist der Winkel der Urbildtangenten gleich dem Winkel der Bildtangenten unter $f$. Falls also $f^\prime(z)\neq0$, dann ist
$f$ bei $z=x+\i y$ \emph{winkeltreu} (``konform''). Ferner ist $f$ orientierungstreu, da $\det A > 0$.
\end{bem}
%TODO Bild!

\begin{dfn} \label{dfn1.7}
Seien $U, V \subseteq \C$ offen und nichtleer. Sei $f\colon U \to V$ bijektiv und $f\in H(U)$, $f^{-1}\in H(V)$. Dann heißt $f$ \emph{biholomorph}. (Dann heißen $U$ und $V$ auch ``konform äquivalent''.)
\end{dfn}

\begin{satz} \label{satz1.8}
\begin{enumerate}
\item Seien $U, V \subseteq \C$ offen und nichtleer, $f\colon U \to V$ biholomorph. Dann ist $f^\prime(z)\neq 0$ für alle $z\in U$ und es gilt
\[(f^{-1})^\prime(f(z)) = (f^{-1})^\prime(w) = \frac{1}{f^\prime(f^{-1}(w))} = \frac{1}{f^\prime(z)} \quad (\forall z\in U,\ \forall w=f(z) \in V)\]
\item Seien $f\in H(D) \cap C^1(D, \R^2)$, $z_0\in D$, $f^\prime(z_0)\neq0$. Dann exisieren offene $U, V \subseteq \C$ mit $z_0\in U$, $f(z_0)\in V$, sodass $f\colon U \to V$ biholomorph ist. Somit ist a) auf $f$ für alle $z\in U$ und $w=f(z)\in V$ anwendbar.
\end{enumerate}
\end{satz}
\begin{proof}
\begin{enumerate}
\item Nach Bem.~\ref{bem1.2} und $z=f^{-1}(f(z))$ ($\forall z\in U$) folgt $1 = (f^{-1})^\prime(f(z))f^\prime(z)$. Durchdividieren ergibt Behauptung a).
\item Nach Bem.~\ref{bem1.6} ist $f^\prime(z_0)$ als $2\times2$-Matrix invertierbar. Der Umkehrsatz aus Analysis 2 liefert Behauptung b).\qedhere
\end{enumerate}
\end{proof}

\begin{dfn*} Eine Funktion $u\in C^2(D, \R)$ heißt \emph{harmonisch} auf $D$, wenn für alle $(x, y) \in D$ gilt:
\[\triangle u(x, y) \da \frac{\partial^2u}{\partial x^2}(x, y) + \frac{\partial^2u}{\partial y^2}(x, y) = 0.\]
\end{dfn*}

\begin{satz} \label{satz1.9}
\begin{enumerate}
\item Sei $f\in H(D) \cap C^2(D, \R^2)$. Dann sind $u=\Re f$, $v = \Im f$ harmonisch auf $D$.
\item Sei $u \in C^2(D, \R)$ auf $D$ harmonisch, $B_0 \da B((x_0,y_0),r) \subseteq D$ für ein $r>0$, $(x_0,y_0)\in D$. Dann existiert ein $f\in H(B_0)$ mit $u = \Re f$.
\end{enumerate}
\end{satz}
\begin{proof}
\begin{enumerate}
\item Der Satz von Schwarz aus Analysis 2 (Satz 2.21) liefert
\[\frac{\partial^2u}{\partial x^2} + \frac{\partial^2u}{\partial y^2} \nach{=}{(\ref{CR})}
\frac{\partial}{\partial x} \frac{\partial v}{\partial y} - \frac{\partial}{\partial y} \frac{\partial v}{\partial x} = 0.\]
\item Setze für $(x, y) \in B_0$
\[v(x, y) = - \Integral{x_0}{x}{\frac{\partial u}{\partial y}(s, y_0)}{s} +
\Integral{y_0}{y}{\frac{\partial u}{\partial x}(x, s)}{s}.\]
Beachte: die Strecken von $(x_0,y_0)$ nach $(x,y_0)$ und von $(x,y_0)$ nach $(x,y)$ liegen in $B_0$. Analysis 1 und 2 liefern: $v\in C^1(B_0)$ und
\begin{multline*}
\frac{\partial v}{\partial x}(x, y) = - \frac{\partial u}{\partial y}(x, y_0) + \Integral{y_0}{y}{\underbrace{\frac{\partial^2 u}{\partial x^2}(x, s)}_{\nach{=}{n.V.} - \frac{\partial^2 u}{\partial y^2}(x,s)}}{s}\\
= - \frac{\partial u}{\partial y}(x, y_0) - \frac{\partial u}{\partial y}(x, y) + \frac{\partial u}{\partial y}(x, y_0)
= - \frac{\partial u}{\partial y}(x, y)
\end{multline*}
\folgt (\ref{CR})$_2$.
Ferner $\frac{\partial v}{\partial y}(x, y) = \frac{\partial u}{\partial x}(x, y)$ \folgt (\ref{CR}) gilt. Mit
Satz~\ref{satz1.4} folgt: $f=u+\i v$ ist auf $B_0$ holomorph. \qedhere
\end{enumerate}
\end{proof}

\begin{bsp*}
Sei $f(z) = z^3 = (x+\i y)^3 = x^3 + 3\i x^2 y - 3xy^2 - \i y^3$. Satz~\ref{satz1.9} liefert: $u(x,y) = \Re f(x, y) = x^3 - 3xy^2$ ist harmonisch auf $\C$.
\end{bsp*}


\section{Elementare Funktionen}

\subsection{Möbiustransformationen}

\noindent Sei $A = \mat{a}{b}{c}{d} \in \mathrm{M}_{22}(\C)$ mit $\det A = ad-bc \neq 0$ (dann ist $c\neq0$ oder
$d\neq0$). Setze
\[m_A(z) = \frac{az+b}{cz+d} \quad \  \text{für} \quad z \in D_A \da
\begin{cases} \C \setminus \left\{-\frac{d}{c}\right\} ,& c \neq 0, \\ \C ,& c = 0. \end{cases}\]
$m_A$ heißt \emph{Möbius-Transformation}. Offensichtlich ist $m_A \in H(D_A)$.
\paragraph{Eigenschaften.} Sei $A$ wie oben und $\tilde{A} = \mat{\tilde{a}}{\tilde{b}}{\tilde{c}}{\tilde{d}} \in
\mathrm{M}_{22}(\C)$ mit $\det \tilde{A} \neq 0$.
\begin{enumerate}
\item Es gilt
  \[m_A(z) = z \; (\forall z \in D_A) \gdw A = \mat{a}{0}{0}{a} \text { für ein } a \in \C \setminus \{0\}.\]
  \begin{proof}
    "`$\Leftarrow$"': einsetzen! "`$\Rightarrow$"': Für alle $z\in D$ gilt:
    \[m_A(z)=z \folgt cz^2 + (d-a)z - b = 0 \text{ (}\forall z\in D_A\text{) }\]
    \[\folgt c=0, d=a, b=0 \folgt A = \mat{a}{0}{0}{a}. \qedhere\]
  \end{proof}
\item Für alle $\alpha\in\C$ gilt $m_{\alpha A} = m_A$.
\item Es gilt $m_A(m_{\tilde{A}}(z)) = m_{A\tilde{A}}(z)$ (soweit alles definiert).
\begin{proof} Für passende $z$:
  \[m_A(m_{\tilde{A}}(z)) =
  \frac{a\frac{\tilde{a}z+\tilde{b}}{\tilde{c}z+\tilde{d}}+b}{c\frac{\tilde{a}z+\tilde{b}}{\tilde{c}z+\tilde{d}}+d} =
  \frac{(a\tilde{a}+b\tilde{c})z+(a\tilde{b}+b\tilde{d})}{(c\tilde{a}+d\tilde{c})z+(c\tilde{b}+d\tilde{d})} =
  m_{A\tilde{A}}(z). \qedhere\]
\end{proof}
\item Es gilt: $\displaystyle A^{-1} = \frac{1}{ad-bc}\mat{d}{-b}{-c}{a}$. Damit folgt: $\displaystyle D_{A^{-1}}=
  \begin{cases} \C\setminus\left\{\frac{a}{c}\right\},&c\neq 0,\\\C,&c=0.\end{cases}$

Min zeigt leicht, dass gilt:
\[m_A(D_A) \subseteq D_{A^{-1}},\tag{$\ast$}\]
\[m_{A^{-1}}(D_{A^{-1}}) \subseteq D_A.\tag{$\ast\ast$}\]
Also folgt aus c): $m_A(D_A) = D_{A^{-1}}$ (wende auf ($**$) $m_A$ an), $m_{A^{-1}}(D_{A^{-1}}) = D_A$ (wende auf ($*$) $m_{A^{-1}}$ an) und $(m_A)^{-1} = m_{A^{-1}}$.

Insbesondere sind $m_A\colon D_A\ra D_{A^{-1}}$, $m_{A^{-1}}\colon D_{A^{-1}}\ra D_A$ biholomorph.
\item\label{eigenschaften:e} Sei $c=0$ (also $d\neq0$). Dann ist $m_A(z) = \frac{a}{d}z + \frac{b}{d}$ eine affine Abbildung, also $m_A = T \circ S$
  mit $Tw=w+\frac{d}{d}$ (Translation) und $Sw = \frac{a}{d}w$ (Drehstreckung). Sei nun $c\neq0$. Dann gilt $m_A = A_2 \circ J
  \circ A_1$, wobei $A_1w = cw+d$, $A_2w = \frac{a}{c} - \frac{ad-bc}{c}w$ (affin) und $Jw = \frac1w$ ($w\neq0$) (Inversion).
  \begin{proof}
    Es gilt $\displaystyle\frac{az+b}{cz+d} = \frac{a}{c} - \frac{ad-bc}{c}\frac{1}{cz+d} = A_2(J(A_1(z)))$.
  \end{proof}
%%%%%%%%%%%%%%%%%%%%%%%%%%%%
% Eigentlich aus der nächsten Vorlesung, muss aber glaub
% ich hier hin ....
%%%%%%%%%%%%%%%%%%%%%%%%%%%%
Fasse jede Gerade in $\C$ als verallgemeinerten Kreis über $\infty$ auf. Also ist ein verallgemeinerter Kreis $K$ entweder eine Gerade oder eine echte Kreislinie. Beachte: $K$ wird durch die Angabe dreier verschiedener Punkte in $\C_\infty$ eindeutig bestimmt.
\item\label{eigenschaften:f} Jede Möbiustransformation bildet einen verallgemeinerten Kreis bijektiv auf einen verallgemeinerten Kreis ab.
\begin{proof} Nach \ref{eigenschaften:e}
%%%%%%%%%%%%%%%%%%%%%%%
% Da steht im Moment eine 5, obwohl da ein e sein sollte ...
% das Problem hatten wir glaub ich schonmal, weiß aber nicht
% mehr die Lösung ;-)
%%%%%%%%%%%%%%%%%%%%%%%
 ist die Behauptung nur für Translationen $T$, Drehstreckungen $S$ und die Inversion $J$ zu zeigen. Klar: $T$, $S$ sind ``verallgemeinert kreistreu.''

Zu $J$: Sei $r>0$, $\kmplx{z_0}$. Dann:
\[z\in K\da \partial B(z_0,r)\gdw r^2 = |z-z_0|^2 = (z-z_0)(\bar{z}-\bar{z_0}) = |z|^2-z_0\bar{z}-\bar{z_0}z+|z_0|^2.\]
Damit: $K=\{\kmplx{z}: \alpha|z|^2+c\bar{z}+\bar{c}z+\delta = 0\}$\hfill($*$)
für feste $\alpha$, $\delta\in\R$, $\kmplx{c}$ mit $|c|^2>\alpha\delta$, wobei $z_0=-\frac{c}{\alpha}$, $r^2=\frac{|c|^2}{\alpha^2}-\frac{\delta}{\alpha}$, falls $\alpha\neq 0$.

$\leadsto$ Für $\alpha\neq 0$ beschreibt ($*$) die echte Kreislinie $\partial B(z_0,r)$. Für $\alpha = 0$ beschreibt ($*$) die Gerade $\Re(\bar{c}z) = -\frac{\delta}{2}$ (Beachte: $c\neq 0$). Multipliziere ($*$) mit $\frac{1}{|z|^2}$. Dann erfüllt $w = Jz = \frac{1}{z} = \frac{\bar{z}}{|z|^2}$ die Gleichung:
\[ 0 = \underbrace{\alpha}_{\ad\delta'} + \underbrace{cw}_{\ad -\bar{c}'} + \underbrace{\bar{c}\bar{w}}_{\ad -c'} + \underbrace{\delta|w|^2}_{\ad\alpha'}.\]
Weiter gilt: $|c'|^2-\alpha'\delta' = |c|^2-\alpha\delta > 0$. Wenn $K$ durch ($*$) beschrieben wird, dann folgt also $J(K)\subseteq K'$, wobei $K'$ ein verallgemeinerter Kreis ist. Genauso: $J(K')\subseteq K$. Da $J^2 = \id$ folgt $K'=J^2(K')\subseteq J(K)\folgt J(K)=K'$.
\end{proof}
\end{enumerate}

%%%%%%%%%%%%%%%%%%%%%%%
% Aus der vorherigen Vorlesung, gehört aber glaub
% ich dahinter .....
%%%%%%%%%%%%%%%%%%%%%%%
\begin{dfn} \label{dfn1.10}
  Setze $\C_\infty = \C \cup \{\infty\}$, $\R_\infty = \R \cup \{\infty\}$, wobei gelten soll:
\begin{eqnarray*}
&\forall\kmplx{z}: z+\infty = \infty + z = \infty,\ \frac{z}{\infty} = 0,\\
&\forall\kmplx{z}\setminus \{0\}: z\cdot\infty = \infty\cdot z = \infty,\ \frac{z}{0} = \infty.
\end{eqnarray*}
\textbf{Verboten:} $0\cdot\infty$, $\infty\cdot 0$, $\frac{0}{0}$, $\frac{\infty}{\infty}$.

\noindent Wir schreiben $z_n \ra \infty$, wenn $\abs{z_n} \ra +\infty\ (n\ra\infty)$.

\noindent Beachte: Dieses $\infty$ ist ein anderes als $\pm\infty$ in $\R$ aus Analysis 1.
\end{dfn}

Setze bezüglich dieser ``Konvergenz'' $m_A$ ``stetig'' fort durch
\[m_A (\infty) \da \left\{\!\!\!\begin{array}{r @{,\ } l}\frac{a}{c}&c\neq 0\\\infty&c=0\end{array}\right..\]
Beachte:
\[m_A(z)=\frac{a+\frac{b}{z}}{c+\frac{d}{z}}\ (z\neq 0),\ m_A\left(-\frac{d}{c}\right) = \infty\]
(beachte: $-\frac{ad}{c} + b\neq 0$, da $\det A\neq 0$).

Mit etwas Rechnung folgt: $m_A\colon\C_\infty\ra\C_\infty$ ist bijektiv mit $(m_A)^{-1} = m_{A^{-1}}\colon\C_\infty\ra\C_\infty$.

Sei $\mathcal{M}\da\{m_A\colon\C_\infty\ra\C_\infty\mid\det A\neq 0\}$. Mit obigen Eigenschaften (und etwas Rechnung bezüglich $\infty$) folgt:
\begin{itemize}
\item $\mathcal{M}$ ist eine Gruppe bezüglich der Komposition.
\item $\Phi\colon \mathrm{GL}(2,\C)\ra\mathcal{M}$, $A\mapsto m_A$ ist ein surjektiver Gruppenhomomorphismus mit
  $\mathrm{Kern }\Phi = \{\alpha \I \mid \kmplx{\alpha}\setminus\{0\}\}.$
\end{itemize}
%%%%%%%%%%%%%%%%%%%%%%%%%%%%%

\begin{bsp}[Cayley-Transformation]
Sei $C=\mat{1}{-\i}{1}{\i}$ ($\leadsto\det C = 2\i\neq 0$) $\displaystyle\leadsto M_C(z) = \frac{z-\i}{z+\i}$.

Dabei: $\displaystyle M_C(-\i) = \frac{-2}{0} = \infty$, $\displaystyle M_C(\infty) = \frac{1-\frac{\i}{\infty}}{1+\frac{\i}{\infty}} = 1$. Weiter: $\displaystyle M_C(0) = -1$, $\displaystyle M_C(1) = \frac{1-\i}{1+\i} = -\i$. Durch $\{0,1,\infty\}$ läuft der verallgemeinerte Kreis $\R_\infty = \R\cup\{\infty\}$. Nach \label{eigenschaften:f} %%s.o.
verläuft der Bildkreis $M_C(\R_\infty)$ durch die Bilder $-1,-\i,1$. Dies ist $\set{S} = \partial \set{D}$ mit $\set{D} = B(0,1)$. Also $M_C(\R_\infty) = \set{S}$.

Ferner: Für $b\in\R\setminus\{-1\}$ gilt $\displaystyle M_C(\i b) = \frac{b-1}{b+1}$, insbesondere $M_C(\i) = 0$. $\i\R_\infty$ verläuft durch $0,\i,\infty$. $\R_\infty$ verläuft durch die Bildpunkte $-1,0,1 \folgt M_C(\i\R_\infty) = \R_\infty$. Mit Ana 1: $M_C(\i\R_+) = [-1,1)$.

%% Bild
\begin{tikzpicture}[scale=2]

\draw[->] (0,-1.5) -- (0,1.5);
\draw[->] (-1.5,0) -- (1.5,0);
\filldraw [gray] (0,0) circle (1pt);
\draw (0,0) node [anchor=north west] {$0$};
\filldraw [gray] (1pt,0.95) -- (0,1.07) -- (-1pt,0.95) -- cycle;
\draw (-1pt,1) -- (1pt,1);
\draw (0,1) node [anchor=west] {$\i$};
\draw [blue] (-1.5,0.6) -- (1.5,0.6);
\filldraw [blue] (0,0.6) circle (1pt);
\draw [blue] (0,0.73) arc (90:180:1.3mm);
\draw [blue] (-0.5mm,0.65) node {.};
\draw (1,1pt) -- (1,-1pt);
\draw (0.97,1pt) -- (1.03,-1pt);
\draw (0.97,-1pt) -- (1.03,1pt);
\draw (1,0) node [anchor=north] {$1$};
\draw [gray] (-1.8,0) node [shape=rectangle,fill] {};
\draw (-1.8,0) node [anchor=north] {$\infty$};
\draw [red] (-1.8,0) -- (1.5,0);
\draw [green] (0,-1.5) -- (0,1.5);

\begin{scope}[xshift=4cm]
\draw[->] (0,-1.5) -- (0,1.5);
\draw[->] (-1.5,0) -- (1.5,0);
\draw[red] (0,0) circle (1cm);
\filldraw[gray] (-1,0) circle (1pt);
\draw (-1,0) node [anchor=north] {$-1$};
\draw (-1pt,-1) -- (1pt,-1);
\draw (-1pt,-0.97) -- (1pt,-1.03);
\draw (-1pt,-1.03) -- (1pt,-0.97);
\draw (0,-1) node [anchor=east] {$-\i$};
\filldraw [gray] (1pt,-1pt) -- (-1pt,-1pt) -- (0,1pt) -- cycle;
\draw (0,0) node [anchor=north west] {$0$};
\draw [gray] (1,0) node [shape=rectangle,fill] {};
\draw (1,0) node [anchor=north] {$1$};
\filldraw [blue] (-0.6,0) circle (1pt);
\draw [blue] (0.2,0) circle (0.8);
\draw [blue] (-0.6,0) -- (-0.6,0.13) arc (90:180:1.3mm);
\draw [blue] (-0.65,0.5mm) node {.};
\draw [blue] (-1,-1) node[anchor=north] {$M_C(\i b)$};
\draw [blue,->] (-1,-1) -- (-0.65,-0.05);
\draw [green] (-1.5,0) -- (1.5,0);
\end{scope}

\draw[->] (1.5,0.8) .. controls (1.8,1.2) and (2.2,1.2) .. (2.5,0.8);
\draw (2,1.3) node {$M_C$};

\end{tikzpicture}

Die Gerade $K\colon \i b + x$, $\reell{x}$, ($b>0$ fest) wird auf den verallgemeinerten Kreis $K'$ durch $M_C(\i b)\in (-1,1)$ und $1 = M_C(\infty)$ abgebildet. Nach Bem. \ref{bem1.6} und da $K'$ die imaginäre Achse im Winkel $\frac{\pi}{2}$ schneidet, schneidet $M_C(K)$ die reelle Achse ($=M_C(\i\R_\infty)$) auch senkrecht in $M_C(\i b)$. $\folgt M_C(K)$ ist symmetrisch zur $x$-Achse und liegt in $\set{D}\folgt M_C(\set{H}_+)\subseteq\set{D}$, mit $\set{H}_+ = \{\kmplx{z}:\Im{z}>0\}$. Sei weiter $w\in\set{D}$. Sei $K$ der Kreis durch $w$ und $1$, der symmetrisch zur $x$-Achse ist. Sei $a\in (-1,1)$ der zweite Schnittpunkt von $K$ mit der $x$-Achse $\folgt$ Es gibt genau ein $b\in (0,+\infty)$ mit $\displaystyle a = \frac{b-1}{b+1}$. Also: Ist $w\in M_C(\i b + \R_\infty)\folgt M_C(\set{H}_+) = \set{D}\folgt M_C\colon\set{H}_+\ra\set{D}$ ist biholomorph.
\end{bsp}

\subsection{Potenzen und Wurzeln}

Sei $\nat{n}$, $n\geq 2$. Mit $\sqrt[n]{x}$ wird stets die reelle Wurzel bezeichnet. Für $\theta\in (0,\pi]$ definiere den (offenen) Sektor:
\[\Sigma_\theta \da \{z\in\C\setminus\{0\}: |\arg z| < \theta\}\]
speziell: $\Sigma_\pi = \C\setminus\R_-$ (geschlitzte Ebene), $\Sigma_\frac{\pi}{2} = \{\kmplx{z}: \Re{z} > 0\} = \C_+$ (rechte Halbebene)

\mbox{
\begin{minipage}[c]{.5\textwidth}
\begin{center}
\begin{tikzpicture}[scale=1.5]
\draw [->] (0,-1) -- (0,1);
\draw [->] (-1,0) -- (1,0);
\draw (0,0) -- (50:1.2);
\draw (0,0) -- (-50:1.2);
\draw (50:0.6) arc (50:-50:0.6);
\draw (30:1) node [anchor=west] {$\Sigma_\theta$, $\theta < \frac{\pi}{2}$};
\end{tikzpicture}
\end{center}
\end{minipage}
\begin{minipage}[c]{.5\textwidth}
$z = r\e^{\i\phi}\in\Sigma_\theta \gdw r > 0,\ |\phi| < \theta$

(wobei $|\phi| \leq \pi$)
\end{minipage}
}

Betrachte: $P_n(z) = z^n$, $\kmplx{z}$. Dann: $P_n(r\e^{\i\phi})=r^n\e^{\i\phi n}$. Also bildet $P_n$ den Halbstrahl $\{r\e^{\i\phi},\ r>0\}$ mit Winkel $\phi\in(-\pi,\pi]$ bijektiv auf den Halbstrahl $\{s\e^{\i\phi n},\ s>0\}$ mit $n$-fachem Winkel (modulo $2\pi$) ab.

Setze $p_n\da {P_n}|_{\Sigma_\frac{\pi}{n}}$. Dann ist $p_n\colon\Sigma_\frac{\pi}{n} \ra \Sigma_\pi$ bijektiv.

Beachte: $P_n$ ist schon auf $\bar{\Sigma_\frac{\pi}{n}}$ nicht mehr injektiv. Beispiel für $n=2$:
\[+\i,-\i\in\bar{\Sigma_\frac{\pi}{2}} = \bar{\C_+},\ P_2(\i) = -1 = P_2 (-\i).\]

\begin{dfn}\label{dfn1.12} Der Hauptzweig der $n$-ten Wurzel ist die Umkehrabbildung $r_n = p_n^{-1}\colon \Sigma_\pi\ra \Sigma_\frac{\pi}{n}$. Man schreibt $r_n(w) \ad w^\frac{1}{n}$ für $w\in\Sigma_\pi$.
\end{dfn}

%% Strich?

Per Definition haben wir $r_n(z^n) = z\ (\forall z\in\Sigma_\frac{\pi}{n})$, $r_n(w)^n = w\ (\forall w\in\Sigma_\pi)$. Es gilt:
\begin{equation}\label{1.5}
r(s\e^{\i\psi}) = \sqrt[n]{s}\e^{\i\frac{\psi}{n}}\ (\forall s > 0,\ |\psi| < \pi)
\end{equation}
(denn: $z=\sqrt[n]{s}\e^{\i\frac{\psi}{n}}\in\Sigma_\frac{\pi}{n}$ und $z^n = s\e^{\i\psi}$)

Insbesondere: $r_n(x) = \sqrt[n]{x}$ für $x>0$.

Weiter: ${p_n}^\prime(z) = nz^{n-1}\neq 0\ (\forall z\in\Sigma_\frac{\pi}{n})$.

Mit Satz \ref{satz1.8} sind $r_n\in H(\Sigma_\pi)$ und
\[\left.\begin{array}{r l}p_n\colon&\Sigma_\frac{\pi}{n}\ra\Sigma_\pi\\r_n\colon&\Sigma_\pi\ra\Sigma_\frac{\pi}{n}\end{array}\right\}\text{biholomorph}\]

\paragraph{Abbildungsverhalten der Quadratfunktion.}
\begin{itemize}
\item {\bf Vertikale Gerade} $\Re z = a$ mit einem festen $a>0$. Also gilt für $z=x+\i y$, dass $w \da z^2=a^2-y^2 + \i\cdot2ay$
  (mit $a=x$) (\reell{y} ist Parameter). Also ist $\Im w = 2ay$, also $y = \frac{\Im w}{2a}$. Damit folgt:
  \[\Re w = a^2-y^2 = a^2 - \frac{(\Im w)^2}{4a^2} \le a^2.\]
  Also ist $\Im w = \pm 2a\sqrt{a^2-\Re w}$ eine nach links offene Parabel mit Scheitel $(a^2,0)$.
\item {\bf Horizontale Gerade} $\Im z = b$ mit einem festen $b>0$. Also gilt für $z=x+\i y$, dass $w \da z^2 = x^2-b^2 + \i\cdot
  2bx$ (mit $y=b$) (\reell{x} ist Parameter). Wie oben erhält man $\Im w = 2bx$, also $x=\frac{\Im w}{2b}$ und $\Re w =
  x^2-b^2$. Also ist $\Im w = \pm 2b\sqrt{b^2+\Re w}$ eine nach rechts offene Parabel mit Scheitel $(-b^2,0)$.
\end{itemize}

%TODO Bild

\paragraph{Weitere Zweige der Wurzel.} Sei $\beta\in(-\pi,\pi]$. Setze \[E_\beta=\{t\e^{\i\psi} \mid t>0,
\psi\in(\beta,\beta+2\pi)\} = \C \setminus \{r\e^{\i\beta} \mid r\ge0\}.\] Sei nun $n=2$, dann ist \[W_{\alpha,2} =
\{s\e^{\i\phi} \mid s>0, \phi\in(\alpha,\alpha+\pi)\}\] eine gedrehte Halbebene.


%Vorlesung vom 7. Mai (Donnerstag)
\paragraph{Exponentialfunktion und Logarithmus:}
\smallskip
\begin{center}
\begin{tikzpicture}[scale=2]

\draw [->] (0,-1.3) -- (0,1.3);
\draw [->] (-1.3,0) -- (1.3,0);
\draw [blue] (-1.3,1) -- (1.3,1);
\filldraw [blue] (0.8,1) circle (1pt);
\draw [blue] (-1.3,-1) -- (1.3,-1);
\filldraw [blue] (0.8,-1) circle (1pt);
\draw [red] (-1.3,0.3) -- (1.3,0.3);
\filldraw [red] (-0.6,0.3) circle (1pt);
\draw (-2pt,1) -- (2pt,1);
\draw (0,1) node [anchor=north west] {$\i\pi$};
\draw (-2pt,0.5) -- (2pt,0.5);
\draw (0,0.5) node [anchor=west] {$\i\frac{\pi}{2}$};
\draw (-2pt,-1) -- (2pt,-1);
\draw (0,-1) node [anchor=south west] {$-\i\pi$};
\draw (-2pt,0.3) -- (2pt,0.3);
\draw (0,0.3) node [anchor=north west] {$\i b$};
\draw (1,0.6) node {$S_i$};

\begin{scope}[xshift=4cm]
\draw [->] (0,-1.3) -- (0,1.3);
\draw [->] (-1.3,0) -- (1.3,0);
\draw (1,-2pt) -- (1,2pt);
\draw (1,0) node [anchor=north] {1};
\draw (1.1,0.7) node {$\Sigma_\pi = \C\setminus\R$};
\draw [red] (55:1pt) -- (55:1.5);
\draw [red] (0,0) circle (1pt);
\filldraw [red] (55:0.2) circle (1pt);
\draw [red] (0.3,0) arc (0:55:0.3);
\draw [red] (35:0.3) node [anchor=west] {$b$};
\draw [blue] (-1pt,0) -- (-1.3,0);
\filldraw [blue] (-1.2,0) circle (1pt);
\end{scope}

\draw[->] (1.5,0.3) .. controls (1.8,0.7) and (2.2,0.7) .. (2.5,0.3);
\draw (2,0.7) node {$\exp$};
\draw[<-] (1.5,-0.3) .. controls (1.8,-0.7) and (2.2,-0.7) .. (2.5,-0.3);
\draw (2,-0.7) node {$\log$};

\end{tikzpicture}
\end{center}
\paragraph{wobei:} $\e^{x+\i\pi} = \e^x\e^{\i\pi} = -\e^x$, $\e^{x-\i\pi} = \e^x\e{-\i\pi} = -\e^x$, $\e^{x+\i b} = \e^x\e^{\i b}$ für $\reell{x}$, $b\in(-\pi,\pi)$ und es gilt:
\begin{equation}\label{1.10}
\log r\e^{\i b} = \ln (r) + \i b
\end{equation}
wobei $r>0$ und $b\in(-\pi,\pi)$, also $r\e^{\i b}\in\Sigma_\pi$.

\paragraph{Vorsicht:} Logarithmusgesetz gilt in $\C$ nur eingeschränkt. Beispiel: Seien $\phi,\psi\in(-\pi,\pi)$, $\phi+\psi > r\folgt \phi+\psi-2\pi\in(-\pi,\pi)$. Damit:
\[\log(\e^{\i\phi}\e^{\i\psi}) = \log\e^{\i(\phi+\psi-2\pi)}\stackrel{\eqref{1.10}}{=}\i(\phi+\psi-2\pi)\neq\i\phi+\i\psi\stackrel{\eqref{1.10}}{=}\log(\e^{\i\phi})+\log(\e^{\i\psi}).\]

\begin{dfn}\label{dfn:1.14}
Seien $z=r\e^{\i\phi}\in\Sigma_\pi$, $r>0$, $\phi\in(-\pi,\pi)$, $w = x+\i y$, $\reell{x,y}$. Setze
\[z^w \da \exp(w\log z)\stackrel{\eqref{1.10}}{=}\exp((x+\i y)(\ln(r)+\i\phi)) = \underbrace{\e^{x\ln r}}_{=|z|^x} \e{-y\phi}\e^{\i(x\phi + y\ln(r))}.\]
\end{dfn}

\begin{bsp*}
$\i^\i \leadsto r = 1$, $\phi = \frac{\pi}{2}$, $x =0$, $y=1 \folgt \i^\i = \e^{-\frac{\pi}{2}}$.
\end{bsp*}

\begin{bem}\label{bem:1.15}
Seien $z,w$ wie in Definition \ref{dfn:1.14}, $\nat{n}$, $w_k=x_k+\i y_k$ ($k=1,2$), $\reell{x_k,y_k}$. Dann:
\begin{enumerate}
\item $z^n \stackrel{\ref{dfn:1.14}}{=} \exp (n \log z) \stackrel{\eqref{1.6}}{=} \exp(\log z)\dotsm\exp(\log z) \stackrel{\ref{dfn:1.14}}{=} \underbrace{z\dotsm z}_{n\text{-fach}} \stackrel{\text{alte Def.}}{=} z^n$.
\begin{itemize}
\item $z^0 = \exp(0\log z) = 1$.
\item $\displaystyle z^{-1} \stackrel{\ref{dfn:1.14}}{=} \exp(-\log z) \stackrel{\eqref{1.6}}{=}\frac{1}{\exp(\log z)} = \frac{1}{z}.$
\end{itemize}
$\folgt$ Def. \ref{dfn:1.14} passt zu ganzen Exponenten.

\item $z^{w_1+w_2} \stackrel{\ref{dfn:1.14}}{=} \exp((w_1+w_2)\log z) \stackrel{\eqref{1.6}}{=}\exp(w_1\log z)\cdot\exp(w_2\log z) \stackrel{\ref{dfn:1.14}}{=}z^{w_1}z^{w_2}$

$\folgt z = z^{\frac{1}{n} + \dotsm + \frac{1}{n}} = \underbrace{z^{\frac{1}{n}}\dotsm z^{\frac{1}{n}}}_{n\text{-fach}} = (z^{\frac{1}{n}})^n$.

$\folgt$ Für $w=\frac{1}{n}$ stimmen Definition \ref{dfn:1.14} und \ref{dfn:1.12} überein.

\item $\displaystyle \frac{\partial}{\partial z}z^w = \frac{\partial}{\partial z}\exp(w \log z) = \exp(w\log z)\frac{w}{z} \stackrel{\text{(b)}}{=}wz^{w-1}$,

$\displaystyle\frac{\partial}{\partial w} z^w = \frac{\partial}{\partial w} \exp (w\log z) = \log (z) z^w$.

\item $\displaystyle |z^w| \stackrel{\ref{dfn:1.14}}{=} |z|^{\Re{w}}\e^{-\Im{(w)}\arg(z)} \leq |z|^{\Re{w}}\e^{\pi|\Im{w}|}$.
\end{enumerate}
\end{bem}

\subsection{Kosinus und Sinus}

Seien $\kmplx{z,w}$, $z = x +\i y$ ($\reell{x,y}$) Aus den Reihendarstellungen folgen (Ana 1):
\begin{align}
\label{1.11}&\sin(-z) = -\sin(z),\quad \cos(-z) = \cos(z)\\
\label{1.12}&\exp(\i z) = \cos(z)+\i\sin(z),\quad\cos^2(z)+\sin^2(z)=1\\
\label{1.13}\text{mit }\eqref{1.11}:\quad&\cos(z) = \frac{1}{2}(\e^{\i z} + \e^{-\i z}),\quad\sin(z) = \frac{1}{2\i}(\e^{\i z} - \e^{-\i z})\\
\folgtwegen{\eqref{1.7}}\quad&\cos z = \frac{1}{2}(\e^{\i x}\e^{-y}+\e^{\i x}\e^y) = \frac{1}{2}\e^{-y}(\cos x + \i\sin x) + \frac{1}{2}\e^y(\cos x-\i\sin x)\nonumber\\
&\qquad\qquad=\cos(x)\cosh(y)+\i\sin(x)\sinh(y)\label{1.14}\\
&\text{wobei } \cosh(y) = \frac{1}{2}(\e^y+e^{-y}),\ \sinh(y) = \frac{1}{2}(\e^y-\e^{-y})\nonumber\\
\label{1.15}\text{Genauso:}\quad&\sin z = \sin(x)\cosh(y) + \i\cos(x)\sinh(y)
\end{align}

$\leadsto\sin$, $\cos$ sind in imaginärer Richtung unbeschränkt.

Aus \eqref{1.13} folgt ferner (Ana 1):
\begin{equation}\label{1.16}
\cos(z)-\cos(w) = -2\sin\left(\frac{z+w}{2}\right)\sin\left(\frac{z-w}{2}\right)
\end{equation}

Weiter gilt mir \eqref{1.13} und \eqref{1.6}: $\cos\left(z+\frac{\pi}{2}\right) = \frac{1}{2}\left(\e^{\i z}\underbrace{\e^{\i\frac{\pi}{2}}}_{=\i} + \e^{-\i z}\underbrace{\e^{-\i\frac{\pi}{2}}}_{=-\i}\right) = -\sin(z)$.

Genauso: $\sin(z+\frac{\pi}{2}) = \cos z$.

$\folgt \cos (z+2\pi) = \cos (z),\quad\sin(z+2\pi)= \sin(z)$.

\paragraph{Nullstellen:} $\sin(z) = 0 \stackrel{\eqref{1.13}}{\gdw} \e^{\i z} = \e^{-\i z} \stackrel{\eqref{1.7}}{\gdw} \e^{-y} (\cos x + \i\sin x) = \e^y (\cos x -\i\sin x)$
\[\gdw \left\{\begin{array}{r @{:\ } l}\text{Imaginärteil}&\sin x = 0\  (\text{da } \e^y\neq\e^{-y})\\\text{Realteil}&\cos x = 0 \text{ oder } \e^y = \e^{-y} (\gdw y=0)\end{array}\right\} \gdw z = k\pi \text{ für ein } k\in\set{Z}.\]
Damit: $\cos(z) = 0\gdw z = k\pi+\frac{\pi}{2}$ für ein $k\in\set{Z}$.

\vspace*{-10pt}

\begin{equation}\label{1.17}
\text{\it Zusammengefasst: }\sin,\cos \text{\it haben auf }\C\text{ \it nur die reellen Nullstellen und Perioden.}
\end{equation}

\vspace*{-20pt}

\begin{equation}\label{1.18}
\begin{split}
\text{\it Wenn der reelle $\sin$ bzw. $\cos$ auf $(a,b)\subseteq\R$  injektiv ist,}\\\text{\it dann ist der komplexe $\sin$ bzw. $\cos$ auf $S_r(a,b)$ injektiv.}
\end{split}\end{equation}
\begin{proof}
(für $\cos$): Nach Vorraussetzung muss $\cos$ auf $(a,b)$ strikt monoton sein.\\ $\folgt (a,b)\subseteq(k\pi,(k+1)\pi)$ für ein $k\in\set{Z}$.

\noindent Seien also $z,w\in S_r(a,b)$ (wobei $z\neq w$) $\folgt z+w, z-w \neq 2j\pi\ (\forall j\in\set{Z})\\ \folgtwegen{\eqref{1.10}}\cos z - \cos w\neq 0$.
\end{proof}

$\folgt\cos$ ist auf $S_r(0,\pi)$ injektiv, $\sin$ ist auf $S_r(-\frac{\pi}{2},\frac{\pi}{2})$ injektiv.

\paragraph{Bild von $\cos$ auf $S_r(0,\pi) \ad S_r$:} Horizontale Gerade: $z=x+\i b$ mit $\reell{x}$ und festem $\reell{b}$. Mit \eqref{1.14}:
\[\cos(t) = \cosh(b)\cos(x)+\i\sinh(b)\sin(x)\]
$\folgt$ Bild der Geraden ist eine Ellipse $E(b)$ mit Scheiteln $(\pm\cosh(b),0)$ und $(0,\pm\sinh(b))$.

Für $x\in(0,\pi)$ erhalten wir für $b>0$ den oberen Bogen von $E(b)$, den unteren Bogen für $b<0$. Da $x\neq 0,\pi$ sind diese Bogen ohne die Endpunkte:
\[(\pm\underbrace{\cosh(b)}_{\geq 1},0)\folgt \cos(S_r)\subseteq D \da \C\setminus((-\infty,-1]\cup[1,\infty)).\]

Klar: $\cos((0,\pi)) = (-1,1)$. Sei $w\in D\setminus(-1,1)$, also $\C\setminus\R$, da hier $w = u+\i v$, $\reell{u}$, $\reell{v}\setminus\{0\}$. Suche $\reell{b}\setminus\{0\}$ mit 
\[w\in E(b) \gdw f(b) \da \frac{u^2}{\cosh^2(b)} + \frac{v^2}{\sinh^2(b)} \stackrel{!}{=} 1.\]
So ein $b$ existiert, da $f$ stetig ist (ZWS): $f(b)\ra 0$ für $b\ra\pm\infty$, $f(b)\ra\pm\infty$ für $b\ra 0$, da $v\neq 0$.

$\folgt\cos\colon S_r\ra D$ bijektiv $\folgt$ haben Hauptzweig des Arcuscosinus:
\[\arccos\colon D\ra S_r,\quad\arccos = (\cos|_{S_r})^{-1}.\]

Da $\cos^\prime(z) = -\sin(z) \neq 0\ \forall z\in S_r$ sind
\[\cos\colon S_r\ra D,\quad\arccos\colon D\ra S_r\]
biholomorph. Mit Verschiebung: $\sin\colon S_r(-\frac{\pi}{2},\frac{\pi}{2})\ra D$ ist biholomorph mit Inversem $\arcsin$.

\chapter{Der Integralsatz von Cauchy}
\section{Komplexe Kurvenintegrale}

Sei $f\colon [a,b]\ra \C$ stückweise stetig (d.h. für alle $t\in[a,b]$ existieren die rechts- und linksseitigen Grenzwerte und diese sind bis auf endlich viele $t_k\in[a,b]$ $(k=1,\dotsc,m)$ gleich.

$\leadsto f$ hat endlich viele (oder keine) Sprungstellen.

$\folgt\Re{f}$, $v=\Im{f}$ sind beschränkt und messbar.

$\displaystyle\folgt\exists\integral{a}{b}{f(t)}{t} = \integral{}{}{\!\!f}{t} \da \integral{a}{b}{\Re{f(t)}}{t} + \i\integral{a}{b}{\Im{f(t)}}{t}$.

Setze $\displaystyle \|f\|_1 = \integral{a}{b}{|f(t)|}{t}$.

\paragraph{Eigenschaften.} Es seien $f,g\colon[a,b]\to\C$ stückweise stetig, \reell{c}, \kmplx{\alpha,\beta}. Dann gelten die folgenden Aussagen (vgl. Analysis 3):
\begin{enumerate}
\item $\displaystyle \Re{\integral{a}{b}{f(t)}{t}} = \integral{a}{b}{\Re{f(t)}}{t}$, $\displaystyle \Im{\integral{a}{b}{f(t)}{t}} = \integral{a}{b}{\Im{f(t)}}{t}$, 
$\displaystyle \overline{\integral{a}{b}{f(t)}{t}} = \integral{a}{b}{\overline{f(t)}}{t}$ (folgt direkt aus der Definition)
\item $\displaystyle \abs{\integral{a}{b}{f(t)}{t}} \le \integral{a}{b}{\abs{f(t)}}{t} \le (b-a) \|f\|_\infty$ (zum Beweis: setze $h=\e^{-\i\varphi}f$)
\item $\displaystyle \integral{a}{b}{(\alpha f + \beta g)(t)}{t} = \alpha \integral{a}{b}{f(t)}{t} + \beta \integral{a}{b}{g(t)}{t}$,
$\displaystyle \integral{a}{c}{f(t)}{t} + \integral{c}{b}{f(t)}{t} = \integral{a}{b}{f(t)}{t}$
\begin{proof}
Für $\alpha = \gamma + \i\delta$, $g=0$, $u=\Re{f}$, $v=\Im{f}$ folgt
\begin{align*}
\integral{a}{b}{\alpha f(t)}{t} &= \integral{a}{b}{(\gamma u(t) - \delta v(t))}{t} + \i \integral{a}{b}{(\delta u(t) + \gamma v(t))}{t}\\
&= \gamma \integral{a}{b}{u(t)}{t} - \delta \integral{a}{b}{v(t)}{t} + \i\delta \integral{a}{b}{u(t)}{t} + \i\gamma \integral{a}{b}{v(t)}{t}\\
&= (\gamma+\i\delta) \integral{a}{b}{(u(t)+\i v(t))}{t}\\
&= \alpha \integral{a}{b}{f(t)}{t}.
\end{align*}
\end{proof}
\item Seien $f,f_n\colon[a,b]\to\C$ stückweise stetig, $f_n\to f$ gleichmäßig (\ninf). Dann gilt
\[\abs{\integral{a}{b}{f(t)}{t}-\integral{a}{b}{f_n(t)}{t}} \le (b-a) \|f-f_n\|_\infty \to 0 \quad (\ninf).\]
\end{enumerate}

\noindent Eine Funktion $f\colon[a,b]\to\C$ heißt \emph{differenzierbar} in $t_0\in[a,b]$, wenn der Grenzwert \[\lim_{t\to t_0}{\frac{f(t)-f(t_0)}{t-t_0}} \ad f^\prime(t_0)\] existiert. Dies ist genau dann der Fall, wenn $u=\Re{f}$ und $v=\Im{f}$ bei $t_0$ differenzierbar sind. Dann gilt $f^\prime(t_0) = u^\prime(t_0) + \i v^\prime(t_0)$. Wenn $f$ bei allen $t\in[a,b]$ differenzierbar ist und die Ableitung $f^\prime\colon[a,b]\to\C$ stetig ist, so schreiben wir $f\in C^1([a,b],\C)$. In diesem Fall gilt
\begin{equation}
\integral{a}{b}{f^\prime(t)}{t} = \integral{a}{b}{u^\prime(t)}{t} + \i\integral{a}{b}{v^\prime(t)}{t} = u(t)\vert_a^b + \i v(t)\vert_a^b = f(b) - f(a).
\end{equation}
Genauso zeigt man: Falls $g\colon[a,b]\to\C$ stetig ist, so existiert
\begin{equation}
\frac{\mathrm{d}}{\mathrm{d}t} \integral{a}{t}{g(s)}{s} = g(t)
\end{equation}
für jedes $t\in[a,b]$.\\
Seien $f\in C^1([a,b],\C)$ und $\phi\in C^1([\alpha,\beta])$ mit $\phi([\alpha,\beta])\subseteq[a,b]$. Dann gilt die Substitutionsregel
\[\integral{\phi(a)}{\phi(b)}{f(t)}{t} = \integral{a}{b}{f(\phi(s))\phi^\prime(s)}{s}\]
(Beweis wie im reellen Fall).

\begin{bsp} \label{bsp2.1}
Es gilt $\displaystyle\integral{a}{b}{\e^{zt}}{t} = \integral{a}{b}{\left(\frac1z \e^{zt}\right)^\prime}{t} = \frac1z \e^{zt} \vert_a^b = \frac1z\left(\e^{zb}-\e^{za}\right)$.
\end{bsp}

\begin{dfn} \label{dfn2.2}
Sei $\gamma\in C([a,b],\C)$. Dann heißt $\Gamma=\gamma([a,b])$ \emph{stetige Kurve} oder \emph{Weg} von $\gamma(a)$ nach $\gamma(b)$ mit Parametrisierung $\gamma$. Die Kurve heißt \emph{geschlossen}, wenn $\gamma(a)=\gamma(b)$ und \emph{einfach}, wenn $\gamma$ auf $[a,b)$ injektiv ist. Die Kurve $\Gamma\subseteq\C$ heißt \emph{stückweise $C^1$}, wenn $\gamma\in C([a,b],\C)$ und es Zahlen $a=t_0<t_1<\dotso<t_m=b$ gibt, sodass die Einschränkung $\gamma_k$ von $\gamma$ auf $[t_{k-1},t_k]$ stetig differenzierbar für jedes $k=1,\dotsc,m$ ist. Wir setzen dann $\gamma^\prime(t)=\gamma_k^\prime(t)$ für $t\in[t_{k-1},t_k)$ und $k\in\{1,\dotsc,m\}$, $\gamma^\prime(b)=\gamma_m^\prime(b)$. Wenn zusätzlich jedes $\gamma_k$ eine affine Funktion ist, so heißt $\Gamma$ \emph{Streckenzug}.
\end{dfn}

\noindent Im Folgenden bezeichnet $\Gamma$ stets eine Kurve, die stückweise $C^1$ ist, und $\gamma$ ist eine Parametrisierung (soweit nichts anderes gesagt wird).

\begin{bsp} \label{bsp2.3}
\begin{enumerate}
\item \label{bsp2.3a} Die einfach im Gegenuhrzeigersinn durchlaufene Kreislinie $\Gamma=\partial B(c,r)$ hat z.B. die Parametrisierungen $\gamma=c+r\e^{\i t}$ mit $t\in[0,2\pi]$ und $\gamma_1=c+r\e^{2\i t}$ mit $t\in[0,\pi]$. Sie ist einfach und geschlossen.
\item \label{bsp2.3b} Sei \nat{m}, $m\ge2$. Die $m$-fach durchlaufene Kreislinie $\Gamma=\partial B(c,r)$ hat die Parametrisierung $\gamma=c+r\e^{\i t}$ mit $t\in[0,2\pi m]$. Sie ist geschlossen, aber nicht einfach.
\item \label{bsp2.3c} Die Strecke $\overrightarrow{wz}$ von $w$ nach $z \neq w$ hat die Parametrisierung $\gamma(t)=w+t(z-w)$ mit $t\in[0,1]$. Sie ist einfach, aber nicht geschlossen.
\item \label{bsp2.3d} Seien $\Gamma_1,\Gamma_2$ stückweise $C^1$-Kurven mit Parametrisierungen $\gamma_j\in C([a_j,b_j],\C)$ ($j=1,2$) und $\gamma_1(b_1)=\gamma_2(a_2)$. Dann hat der Summenweg $\Gamma=\Gamma_1+\Gamma_2$ die Parametrisierung \[\gamma(t)=\begin{cases} \gamma_1(t),&a_1\le t\le b_1,\\ \gamma_2(t-b_1+a_2),&b_1\le t\le b_1+b_2-a_2.\end{cases}\] Er ist auch stückweise $C^1$ und verläuft von $\gamma_1(a_1)$ nach $\gamma_2(b_2)$.\\
Beispiel: Dreiecksweg $\overrightarrow{z_1z_2}+\overrightarrow{z_2z_3}+\overrightarrow{z_3z_1}$ oder Kreis mit Griff $\partial B(c,r)\cup[1,2]$.
\item \label{bsp2.3e} Der Rückwärtsweg $-\Gamma$ wird durch $\hat{\gamma}(t)=\gamma(b-t+a)$ mit $t\in[a,b]$ parametrisiert. Dabei sind $\hat{\gamma}(a)=\gamma(b)$ und $\hat{\gamma}(b)=\gamma(a)$. Er ist auch stückweise $C^1$ und verläuft von $\gamma(b)$ nach $\gamma(a)$.
\end{enumerate}
\end{bsp}

\begin{dfn}
Seien $\Gamma$ eine stückweise $C^1$-Kurve mit Parametrisierung $\gamma\in C([a,b],\C)$ und $f\in C(\Gamma,\C)$. Dann heißt
\[\integral{\Gamma}{}{f}{z} = \integral{\Gamma}{}{f(z)}{z} = \integral{a}{b}{f(\gamma(t))\gamma^\prime(t)}{t}\]
\emph{komplexes Kurvenintegral}.
\end{dfn}

\begin{bem}
Aus den Eigenschaften des komplexen Integrals folgt sofort für $f,g\in C(\Gamma,\C)$ und \kmplx{\alpha,\beta}:
\begin{enumerate}
\item $\displaystyle \integral{\Gamma}{}{(\alpha f + \beta g)(z)}{z} = \alpha \integral{\Gamma}{}{f(z)}{z} + \beta \integral{\Gamma}{}{g(z)}{z}$
\item $\displaystyle \abs{\integral{\Gamma}{}{f(z)}{z}} \le l(\gamma) \max_{z\in\Gamma}{\abs{f(z)}$, wobei $l(\gamma) = \integral{a}{b}{\gamma^\prime(t))}{t}$ die Kurvenlänge von $\Gamma$ ist (vgl. Analysis 2/3).
\item Sei $\Gamma=\Gamma_1+\Gamma_2$ wie in Bsp.~\ref{bsp2.3}~\ref{bsp2.3d}. Dann gilt \[\integral{\Gamma}{}{f(z)}{z} = \integral{\Gamma_1}{}{f(z)}{z} + \integral{\Gamma_2}{}{f(z)}{z}.\] Speziell folgt mit den Bezeichnungen aus Def.~\ref{dfn2.2} \[\integral{\Gamma}{}{f(z)}{z} = \sum_{k=1}^m{\integral{t_{k-1}}{t_k}{f(\gamma(t))\gamma^\prime(t)}{t}}.\]
\item Für den Rückwärtsweg $-\Gamma$ aus Bsp.~\ref{bsp2.3}~\ref{bsp2.3e} gilt
\begin{align*}
\integral{-\Gamma}{}{f(z)}{z} &= \integral{a}{b}{f(\hat{\gamma}(t))\hat{\gamma}^\prime(t)}{t} \\
&= -\integral{a}{b}{f(\gamma(b-t+a))\gamma^\prime(b-t+a)}{t}\\
&= \integral{b}{a}{f(\gamma(s))\gamma^\prime(s)}{s} \quad (\text{Subst. } s=b-t+a) \\
&= -\integral{a}{b}{f(\gamma(s))\gamma^\prime(s)}{s}\\
&= - \integral{\Gamma}{}{f(z)}{z}.
\end{align*}
\end{enumerate}
\end{bem}


\end{document}
