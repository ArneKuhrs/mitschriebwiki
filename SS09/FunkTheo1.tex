\documentclass[a4paper,twoside,DIV15,BCOR12mm]{scrbook}
\usepackage{mathe}
\usepackage{saetze-schnaubelt}

\usepackage[utf8]{inputenc}
\usepackage{ngerman, url}
\usepackage{hyperref}
\usepackage{microtype}
\usepackage{tikz}
%\usepackage{schnaubelt}
\usepackage{analysis}

\pdfinfo{
	/Author (Die Mitarbeiter von http://mitschriebwiki.nomeata.de/)
	/Title   (Funktionentheorie I)
	/Subject (Funktionentheorie I)
	/Keywords (Analysis)
}

\author{Die Mitarbeiter von \url{http://mitschriebwiki.nomeata.de/}}
\title{Funktionentheorie I}
\makeindex

%sollten am besten noch in eine class, hab ich aber der Übersicht halber erstmal %hier gemacht ...

% vorläufig, bis wir schnaubelt.sty hochladen ....
% Theorem style
%\theoremstyle{plain}
%  \newtheorem{satz}{Satz}[chapter]
%  \newtheorem{lem}[satz]{Lemma}
%  \newtheorem{kor}[satz]{Korollar}
%  \newtheorem{thm}[satz]{Theorem}
%\theoremstyle{definition}
%  \newtheorem{dfn}[satz]{Definition}
%  \newtheorem*{dfn*}{Definition}
%  \newtheorem{bsp}[satz]{Beispiel}
%  \newtheorem*{bsp*}{Beispiel}
%\theoremstyle{remark}
%  \newtheorem{bem}[satz]{Bemerkung}
%  \newtheorem*{bem*}{Bemerkung}
%\renewcommand{\proofname}{Beweis}

\renewcommand{\labelenumi}{\alph{enumi})}
\setkomafont{paragraph}{\normalfont\bfseries}

% Lim sup und inf
\newcommand{\ls}{\varlimsup}
\newcommand{\li}{\varliminf}


% FIXME: Darstellung der charakteristischen Funktion
\newcommand{\charfnkt}[1]{\ensuremath{\mathbf{1}_{#1}}}

\renewcommand{\Re}[1]{\ensuremath\operatorname{Re}{#1}}
\renewcommand{\Im}[1]{\ensuremath\operatorname{Im}{#1}}
\newcommand{\C}{\set{C}} % verdient in dieser VL find ich schon seinen eigenen Befehl ;-)
\newcommand{\R}{\set{R}}
\newcommand{\cmplx}[2]{\left(\!\!\!\begin{array}{c}#1\\#2\end{array}\!\!\!\right)}

\begin{document}
\maketitle

\renewcommand{\thechapter}{\Roman{chapter}}
%\chapter{Inhaltsverzeichnis}
\addcontentsline{toc}{chapter}{Inhaltsverzeichnis}
\tableofcontents

\chapter{Vorwort}

\section{Über dieses Skriptum}
Dies ist ein Mitschrieb der Vorlesung \glqq Funktionentheorie I\grqq\ von Herrn Schnaubelt im
Sommersemester 2009 an der Universität Karlsruhe (TH).
%Die Mitschriebe der Vorlesung werden mit ausdrücklicher Genehmigung von Herrn Schnaubelt hier veröffentlicht,
Herr Schnaubelt ist für den Inhalt nicht verantwortlich.

\section{Wer}
Beteiligt am Mitschrieb sind Michael Fütterer und andere.

\section{Wo}
Alle Kapitel inklusive \LaTeX-Quellen können unter \url{http://mitschriebwiki.nomeata.de} abgerufen werden.
Dort ist ein \emph{Wiki} eingerichtet und von Joachim Breitner um die \LaTeX-Funktionen erweitert.
Das heißt, jeder kann Fehler nachbessern und sich an der Entwicklung
beteiligen. Auf Wunsch ist auch ein Zugang über \emph{Subversion} möglich.

\setcounter{chapter}{0}

\chapter{Erstes Kapitel}

\begin{bsp}
\begin{enumerate}
\item Polynome $p$ sind auf $\C$ und rationale Funktionen $f = \frac{p}{q}$ sind auf $\{z\in\C: q(t) \neq 0\}$ holomorph mit den reellen Formeln für $p'$, $f'$ (wobei $q \neq 0$ ein Polynom ist).
\end{enumerate}
\end{bsp}

\paragraph{Erinnerung Ana 1:} Gegeben seien $\kmplx{a_n}$ ($\nat{n}_0$) und ein $\kmplx{c}$. Die Potenzreihe
\[f(z) = \sum_{n=0}^\infty a_n(z-c)^n\]
konvergiert absolut für alle $z$ mit
\[|z-c| < \rho \da (\ls_{n\ra\infty}\sqrt[n]{|a_n|})^{-1}\in[0,+\infty]\] 
und divergiert, falls $z\notin \bar{B}(c,\rho)$.

Reduktion auf $c=0$: Betrachte
\[h(w) \da f(c+w) = \sum_{n=0}^\infty a_n w^n\]
wobei $w=z-c\in B(0,\rho)$, $z=c+w$.

\begin{satz}[Vgl. Ana I, Theorem 4.12] Sei
\[f(z) = \sum_{n=0}^\infty a_n(z-c)^n,\ z\in B(c,\rho),\]
eine Potenzreihe mit Konvergenzradius $\rho > 0$. Dann ist $f\in H(B(c,\rho))$ und
\[f'(z) = \sum_{n=1}^\infty n a_n(z-c)^{n-1} =: g(z)\ (\forall z\in B(c,\rho)),\]
wobei die Potenzreihe $g$ den gleichen Konvergenzradius $\rho$ hat.

Sei $\nat{m}$. Iterativ folgt:
\[\exists f^{(m)}(z) = \sum_{n=m}^\infty n\dotsc (n-m+1) a_n(z-c)^{n-m}, \forall z\in B(c,\rho).\]
\end{satz}
\begin{proof} Wie in Ana 1: $g$ hat Konvergenzradius $\rho$. Sei oBdA $c = 0$.

Seien $z\in B(0,\rho)$, $\ep>0$, $r>0$ mit $|z|<r<\rho$. Sei $w\in \bar{B}(0,r)$ mit $w\neq z$.

Da $\sum_{n=1}^\infty na_nr^{n-1}$ absolut konvergiert, existiert $N = \nat{N_\ep}$ mit
\[0 \leq \sum_{n=N+1}^\infty n |a_n| r^{n-1} \leq \ep.\tag{$\ast$}\]

Ferner:
\[0\leq d(w) \da \left| \frac{f(w)-f(z)}{w-z} - g(z)\right| = \left| \sum_{n=1}^\infty a_n\underbrace{(\frac{w^n-z^n}{w-z} - z^{n-1})}_{\ad p_n(w) = w^{n-1}+zw^{n-2}+\dotsc+z^{n-1}-nz^{n-1}}\right|\]
%sieht irgendwie scheiße aus, vielleicht sollte man einfach "mit p(w) = .... in den nächsten Absatz schreiben oder so ....

Dabei: %kann man hier irgendwie den Zeilenumbruch verhindern?!
\begin{itemize}
\item $p_n(w) \ra 0$, $w\ra z$ (für jedes feste $\nat{n}$).
\item $|p_n(w)| \leq r^{n-1} + rr^{n-2} + \dotsc + r^{n-1} + nr^{n-1} = 2nr^{n-1}$ \hfill($\ast\ast$)
\end{itemize}

Damit:
\begin{eqnarray*}
0&\leq d(w) \leq \sum_{n=1}^\infty |a_n||p_n(w)| \stackrel{(\ast\ast)}{\leq} \sum_{n=1}^N |a_n||p_n(w)| + \sum_{n=N+1}^\infty 2n|a_n|r^{n-1}\\
&\stackrel{(\ast)}{\leq} N \max\{|a_1||p_1(w)|,\dotsb,|a_n||p_n(w)|\} + 2\ep \ra 0,\ w\ra z\ (N = N_\ep\ \text{fest!})
\end{eqnarray*}

$\displaystyle\folgt \lim_{w\ra z} d(w) \leq 2\ep \folgtnach{$\ep>0$ bel.} \lim_{w\ra z} d(w) = 0.$
\end{proof}

\paragraph{Beispiele} mit $\rho = \infty$.\begin{enumerate}
\item $\displaystyle \exp(z) \da \sum_{n=0}^\infty \frac{z^n}{n!},\ \exp'(z) = \sum_{n=1}^\infty \frac{z^{n-1}}{(n-1)!} = \exp(z)$.
\item $\displaystyle \sin(z) = \sum_{n=0}^\infty \frac{(-1)^n}{(2n+1)!}z^{2n+1},\ \sin'(z) = \sum_{n=0}^\infty \frac{(-1)^n}{(2n)!}z^{2n} = \cos(z).$
\item $\displaystyle \cos(z) = \sum_{n=0}^\infty \frac{(-1)^n}{(2n)!}z^{2n},\newline \cos'(z) = \sum_{n=1}^\infty \frac{(-1)^n}{(2n-1)!}z^{2n-1} \stackrel{l=n-1}{=} \sum_{l=0}^\infty \frac{(-1)^{l+1}}{(2l+1)!}z^{2l+1} = -\sin(z).$
\end{enumerate}

%eine Trennlinie oder so?

Seien $f:D\ra\C$, $z_0\in D$, $z\in D$. Setze wieder $u = \Re f$, $v = \Im f$, $x_0 = \Re z_0$, $y_0 = \Im z_0$, also
\[f(x,y) = u(x,y) + iv(x,y) = \cmplx{u(x,y)}{v(x,y)}.\]
Sei $z\neq z_0$ und $f$ bei $z_0$ komplex differenzierbar. Dann gilt:
\[\frac{1}{(z-z_0)}|f(z)-f(z_0)-f'(z_0)(z-z_0)| = \left|\frac{f(z)-f(z_0)}{z-z_0} - f'(z_0)\right| \ra 0,\ z\ra z_0.\]

Die Zahl $\kmplx{f'(z_0)}$ kann als $\C$-lineare Abbildung $w\mapsto f'(z_0)w$ aufgefasst werden. Diese ist dann auch $\R$-linear auf $\R^2$, kann also durch eine reelle $2\times 2$-Matrix dargestellt werden. Nacha Ana 2 ist nun $f$ in $z_0 = \cmplx{x}{y}$ reell differenzierbar und somit existieren die partiellen Ableitungen von $u$ und $v$ und es gilt:
\[f'(x_0,y_0) = \left(\!\!\!\begin{array}{c c}\frac{\partial u}{\partial x} (x_0,y_0) & \frac{\partial u}{\partial y} (x_0,y_0)\\
\frac{\partial v}{\partial x} (x_0,y_0) & \frac{\partial v}{\partial y} (x_0,y_0)\end{array}\!\!\!\right).\tag{+}\]

\begin{satz} Sei $f:D\ra\C$, $z_0 = x_0 + iy_0\in D$. Dann sind äquivalent:
\begin{enumerate}
\item $f$ ist in $z_0$ komplex differenzierbar.
\item $f$ ist in $z_0$ reell differenzierbar und es gelten die Cauchy-Riemanschen-DGL (CR):
\[\frac{\partial u}{\partial x} (x_0,y_0) = \frac{\partial v}{\partial y} (x_0,y_0),\ \frac{\partial u}{\partial y} (x_0,y_0) = -\frac{\partial v}{\partial x} (x_0,y_0)\]
Insbesondere ist $f'(z_0)$ schiefsymmetrisch.
\end{enumerate}
\end{satz}

\begin{proof} Die letzte Behauptung folgt aus (+) und (CR)$_2$.

\noindent $(a) \folgt (b):$ Sei $r > 0$ mit $B(z_0,r) \subseteq D$, $\rat{t}$ mit $0 < |t| < r$. Dann:
\begin{align}
f'(z_0) &= \lim_{t\ra 0} \frac{1}{t} (f(z_0+t) - f(z_0))\nonumber\\
&=  \lim_{t\ra 0} \left[\frac{1}{t} (u(x_0+t,y_0)-u(x_0,y_0)) + \frac{i}{t} (v(x_0+t,y_0)-v(x_0,y_0))\right]\nonumber\\
&=\frac{\partial u}{\partial x} (x_0,y_0) + i\frac{\partial v}{\partial x} (x_0,y_0)\\
f'(z_0) &= \lim_{t\ra 0} \underbrace{\frac{1}{it}}_{=-\frac{i}{t}} (f(z_0+it) - f(z_0))\nonumber\\
&=  \lim_{t\ra 0} \left[-i\frac{1}{t} (u(x_0,y_0+t)-u(x_0,y_0)) + \frac{i}{it} (v(x_0,y_0+t)-v(x_0,y_0))\right]\nonumber\\
&=-i\frac{\partial u}{\partial y} (x_0,y_0) + \frac{\partial v}{\partial y} (x_0,y_0)
\end{align}
Vergleiche Real-/Imaginärteil $\folgt$ (CR).

\noindent $(b) \folgt (a):$ Setze
\[w=\frac{\partial u}{\partial x} (x_0,y_0) + i\frac{\partial v}{\partial x}\stackrel{\text{(CR)}}{=} \frac{\partial v}{\partial y} (x_0,y_0) - i\frac{\partial u}{\partial y}(x_0,y_0) \in\C.\]
Dann:
\[w(z-z_0) = (\Re w)(x-x_0) - (\Im w)(y-y_0) + i((\Re w)(y-y_0) + (\Im w)(x-x_0))\]
\[\stackrel{\text{Def. $w$}}{=} \cmplx{\frac{\partial u}{\partial x} (x_0,y_0)(x-x_0) + \frac{\partial u}{\partial y}(x_0,y_0)(y-y_0)}{\frac{\partial v}{\partial y} (x_0,y_0)(y-y_0) + \frac{\partial v}{\partial x}(x_0,y_0)(x-x_0)}\ \text{(in $\R^2$)}.\]
\begin{align*}
\displaystyle\folgt &|f(z)-f(z_0)-w(z-z_0)|\frac{1}{|z-z_0|}\\
&\qquad= \left|\cmplx{x-x_0}{y-y_0}\right|_2^{-1} \left|\cmplx{u(x,y)-u(x_0,y_0)-(\nabla u(x_0,y_0)|\cmplx{x-x_0}{y-y_0})}{v(x,y)-v(x_0,y_0)-\left(\nabla v(x_0,y_0)|\cmplx{x-x_0}{y-y_0}\right)}\right|_2 \ra 0,
\end{align*}% große oder kleine Klammern besser?

$\displaystyle\underbrace{(x,y)}_{=z} \ra \underbrace{(x_0,y_0)}_{=z_0}$, da $u$, $v$ differenzierbar.
\end{proof}

\begin{bsp}
\begin{enumerate}
\item $f(z) = \bar{z}$, $\kmplx{z}$ ist nirgdends komplex differenzierbar, obwohl $f(x,y) = \cmplx{x}{-y}$ reell $C^\infty$ ist. Denn $u(x,y) = x$, $v(x,y) = -y$; also 
\[\frac{\partial u}{\partial x}(x,y) = 1 \neq -1 = \frac{\partial v}{\partial y}(x,y) \blitz\text{ (CR)}_1.\]
\item $f(z) = |z|^2 = x^2 + y^2$, $\kmplx{z}$, ist nur in $z=0$ komplex differenzierbar, denn: $u(x,y) = x^2+y^2$, $v(x,y) = 0$ und somit:
\[\frac{\partial u}{\partial x}(x,y) = 2x \stackrel{!}{=} \frac{\partial v}{\partial y}(x,y) = 0 \gdw x=0,\]
\[\frac{\partial u}{\partial y}(x,y) = 2y \stackrel{!}{=} -\frac{\partial v}{\partial x}(x,y) = 0 \gdw y=0.\]
\item $\displaystyle f(z) = \frac{1}{z} = \frac{\bar{z}}{|z|^2} = \underbrace{\frac{x}{x^2+y^2}}_{=u} + i\underbrace{\frac{-y}{x^2+y^2}}_{=v}$ ist holomorph für $z\neq 0$ (Bem. 1.2). %Link?!
\end{enumerate}
\end{bsp}

\begin{bem} Sei $f$ in $z = x+iy$ komplex differenzierbar. Nach (+) und (CR) gilt:
\begin{equation}
A \da f'(z) = \left(\!\!\!\begin{array}{c c}\frac{\partial u}{\partial x}(x,y)&\frac{\partial u}{\partial y}(x,y)\\-\frac{\partial u}{\partial x}(x,y)&\frac{\partial u}{\partial x}(x,y)\end{array}\!\!\!\right) = \left(\!\!\!\begin{array}{c c}\frac{\partial v}{\partial y}(x,y)&-\frac{\partial v}{\partial x}(x,y)\\\frac{\partial v}{\partial x}(x,y)&\frac{\partial v}{\partial y}(x,y)\end{array}\!\!\!\right)
\end{equation}
Also: $\displaystyle \det A = \frac{\partial u}{\partial x}(x,y)^2 + \frac{\partial u}{\partial y}(x,y)^2 = \frac{\partial v}{\partial x}(x,y)^2 + \frac{\partial v}{\partial y}(x,y)^2 \geq 0$,
\[f'(z) \neq 0 \gdw \det A > 0.\]
Ferner: $A^TA = (\det A)I$\hfill($\ast$)

\noindent$\folgt$ Falls $f'(z) \neq 0$, dann ist $\frac{1}{\sqrt{\det A}}A$ orthogonal.
\end{bem}
\chapter{Satz um Satz (hüpft der Has)}
\listtheorems{satz,wichtigedefinition}

\renewcommand{\indexname}{Stichwortverzeichnis}
\addcontentsline{toc}{chapter}{Stichwortverzeichnis}
\printindex

\end{document}
