\documentclass[a4paper,twoside,DIV15,BCOR12mm]{scrbook}
\usepackage{mathe}
\usepackage{saetze-schnaubelt}

\pdfinfo{
	/Author (Die Mitarbeiter von http://mitschriebwiki.nomeata.de/)
	/Title   (Funktionentheorie I)
	/Subject (Funktionentheorie I)
	/Keywords (Analysis)
}

\author{Die Mitarbeiter von \url{http://mitschriebwiki.nomeata.de/}}
\title{Funktionentheorie I}
\makeindex

\begin{document}
\maketitle

\renewcommand{\thechapter}{\Roman{chapter}}
%\chapter{Inhaltsverzeichnis}
\addcontentsline{toc}{chapter}{Inhaltsverzeichnis}
\tableofcontents

\chapter{Vorwort}

\section{Über dieses Skriptum}
Dies ist ein Mitschrieb der Vorlesung \glqq Funktionentheorie I\grqq\ von Herrn Schnaubelt im
Sommersemester 2009 an der Universität Karlsruhe (TH).
%Die Mitschriebe der Vorlesung werden mit ausdrücklicher Genehmigung von Herrn Schnaubelt hier veröffentlicht,
Herr Schnaubelt ist für den Inhalt nicht verantwortlich.

\section{Wer}
Beteiligt am Mitschrieb sind Michael Fütterer und andere.

\section{Wo}
Alle Kapitel inklusive \LaTeX-Quellen können unter \url{http://mitschriebwiki.nomeata.de} abgerufen werden.
Dort ist ein \emph{Wiki} eingerichtet und von Joachim Breitner um die \LaTeX-Funktionen erweitert.
Das heißt, jeder kann Fehler nachbessern und sich an der Entwicklung
beteiligen. Auf Wunsch ist auch ein Zugang über \emph{Subversion} möglich.

\setcounter{chapter}{0}

\chapter{Erstes Kapitel}

\chapter{Satz um Satz (hüpft der Has)}
\listtheorems{satz,wichtigedefinition}

\renewcommand{\indexname}{Stichwortverzeichnis}
\addcontentsline{toc}{chapter}{Stichwortverzeichnis}
\printindex

\end{document}
