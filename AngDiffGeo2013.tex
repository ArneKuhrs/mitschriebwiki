%%
%% Skript Angewandte Differentialgeometrie im Sommersemester 2013
%% Zur Vorlesung von Dr. Grensing am KIT in Karlsruhe
%%

\documentclass[paper=A4, twoside, chapterprefix=true, bibliography=totoc, headsepline]{scrbook}

%========================================================================================================================
%	P R AE A M B E L
%========================================================================================================================

% Meta-Daten fuer Latexki
\usepackage{latexki}
\lecturer{Dr. Sebastian Grensing}
\semester{Sommersemester 13}
\scriptstate{current}

% Abstand zwischen zwei Textbloecken
\setlength\parskip{\smallskipamount}

% Nummerierung der Paragraphen anpassen (sonst kommt etwas wie "Definition 2.9.1" heraus)
\renewcommand{\thesection}{\arabic{section}}

% Aendert die Kapitelbeschriftung in der Kopfzeile der linken Seiten
\renewcommand*{\chaptermarkformat}{\chapappifchapterprefix{\ }\thechapter:\enskip}
\renewcommand*{\sectionmarkformat}{\thesection\autodot\enskip}

% Einheitliche Schriftart (KOMA Script verwendet fuer einige Ueberschriften eine serifenlose Schrift, mischt also
% Schriftarten. Ich habe mir die Argumente dafuer durchgelsen und war nicht ueberzeugt. Wenn jemand, der mehr als
% ich von der Materie versteht, anderer Meinung ist kann er diese Zeilen hier einfach auskommentieren)
\setkomafont{chapter}{\Huge\bfseries\rmfamily}
\setkomafont{chapterentry}{\bfseries\rmfamily}
\setkomafont{disposition}{\bfseries\rmfamily}
\setkomafont{descriptionlabel}{\bfseries\rmfamily}

\usepackage[utf8x]{inputenc}
\usepackage[T1]{fontenc}
\usepackage{lmodern}

\usepackage[ngerman]{babel}
\usepackage[top=2.5cm, bottom=3cm, left=2.5cm, right=4.5cm]{geometry}

\usepackage{fancyhdr} % erlaubt mehr Optionen in Kopf- und Fusszeile
\usepackage{xcolor} % Farben
\usepackage{marginnote} % Randnotizen
\usepackage{enumitem} % Fuer mehr Einstellungmoeglichkeiten bei Aufzaehlungen
\usepackage{xifthen} % Erlaubt die Verwendung von if-then-else Befehlen im Code
\usepackage{index} % Index erzeugen
\newindex{default}{idx}{ind}{Stichwortverzeichnis}
\usepackage{xspace} % intelligende Leerzeichen bei Macros
\usepackage[normalem]{ulem} % unterstreichen von Text
\usepackage{cancel} % schraeg durchstreichen von Text
\usepackage{units} % schoenere Schreibweise fuer Einheiten mit Bruechen, laedt auch das nicefrac Paket

\renewcommand{\CancelColor}{\color{gray}} % Farbe zum schraegen Druchstreichen in grau

\definecolor{rltred}{rgb}{0.75,0,0}
\definecolor{rltgreen}{rgb}{0,0.5,0}
\definecolor{rltblue}{rgb}{0,0,0.75}

%sichere Fraben, die sich auch bei einem SW-Druck unterscheiden lassen (Platzhalter momentan)
\definecolor{color1}{cmyk}{1,0,0,0} %cyan
\definecolor{color2}{rgb}{0,1,0} %green

\usepackage[hyperindex=true]{hyperref} % Verweise als Hyperlinks
\hypersetup{
	pdftitle={Angewandte Differentialgeometrie Dr. Grensing},
	pdfsubject={Angewandte Differentialgeometrie Geometrie},
	pdfkeywords={Angewandte Differentialgeometrie Grensing},
	pdfproducer={pdfLaTeX},
	pdfpagemode={UseOutlines},
	colorlinks=true,
	bookmarksopen=true,
	bookmarksnumbered=true,
	urlcolor=rltblue,
	filecolor=rltgreen,
	linkcolor=rltblue,
	backref=true,
	pagebackref=true,
	pdfpagemode=None,
	citecolor=rltblue
}

% vertausche die Theta, Phi, Rho und Epsilon mit ihren "var" Versionen
%\newcommand{\swapcmd}[2]{
%	\let\temp\#1
%	\left\#1\#2
%	\let\#2\temp
%}
\let\temp\phi
\let\phi\varphi
\let\varphi\temp

\let\temp\theta
\let\theta\vartheta
\let\vartheta\temp

\let\temp\epsilon
\let\epsilon\varepsilon
\let\varepsilon\temp

\let\temp\rho
\let\rho\varrho
\let\varrho\temp


%%
%% Fuer Zeichnungen in TikZ
%%

\usepackage{tikz}
\usepackage{float} % Floater Elemente
%\usepackage{transparent}
%\usepackage{wrapfig}
\usetikzlibrary{matrix,arrows,calc,intersections, through, positioning, patterns, decorations.text, decorations.pathmorphing, decorations.markings, decorations.pathreplacing}

% neue Befehle fuer haeufig benutzte TikZ Formen; erstes Argument steht fuer die Position, Zweites fuer die Groesse
\newcommand{\tikzrichtung}[3][1]{ % zeichnet eine rote Linie von einem Punkt in eine Richtung mit rotem Knoten am Ende
	\draw[red] #2 -- ($#2 + #1*#3$) circle(0.05);
}

\newcommand{\tikzgitter}[3][0.25]{ %Hilfsgitter, das optionale Argument steht fuer die kleine Maschenweite, die Grosse ist doppelt so gross
	\draw[step=#1,gray!15] #2 grid #3;
	\draw[step=2*#1,gray!30] #2 grid #3;
	\fill (0,0) circle(0.1); 
}

\newcommand{\tikzschnuller}[2][1]{
	% definiere die Knoten relativ zum ersten Knoten skaliert mit dem Faktor
	\coordinate (schnuller1) at #2; \coordinate (schnuller2) at ($(schnuller1)+#1*(-1.75,-0.75)$); \coordinate (schnuller3) at ($(schnuller1)+#1*(-2.5,-2.25)$); \coordinate (schnuller4) at ($(schnuller1)+#1*(0,-2)$); \coordinate (schnuller5) at ($(schnuller1)+#1*(1.75,-0.25)$);
    %\fill (schnuller1) circle (0.05) (schnuller2) circle (0.05) (schnuller3) circle (0.05) (schnuller4) circle (0.05) (schnuller5) circle (0.05);
    
    % die Richtungsvektoren der Bezier Tangenten fuer die einzelnen Knoten (der Erste und der letzte haben keine Tangente)
    \coordinate (ctrls1) at ($#1*(1.25,0.25)$); \coordinate (ctrls2) at ($-0.5*(ctrls1)$); \coordinate (ctrls4) at ($#1*(1,-1)$); \coordinate (ctrls3) at ($-0.5*(ctrls4)$); \coordinate (ctrls6) at ($#1*(1,1.5)$); \coordinate (ctrls5) at ($-0		.33*(ctrls6)$);
	% die eigentlichen Tangenten
    \coordinate (tang1) at ($(schnuller2)+(ctrls1)$); \coordinate (tang2) at ($(schnuller2)+(ctrls2)$); \coordinate (tang3) at ($(schnuller3)+(ctrls3)$); \coordinate (tang4) at ($(schnuller3)+(ctrls4)$); \coordinate (tang5) at ($(schnuller4)+(ctrls5)$); \coordinate (tang6) at ($(schnuller4)+(ctrls6)$);
    %\fill[red] (tang1) circle (0.05); \fill[red] (tang2) circle (0.05); \fill[red] (tang3) circle (0.05); \fill[red] (tang4) circle (0.05); \fill[red] (tang5) circle (0.05); \fill[red] (tang6) circle (0.05);
    %\draw[red] (tang1) -- (tang2); \draw[red] (tang3) -- (tang4); \draw[red] (tang5) -- (tang6);
	
	\draw (schnuller1) ..controls(schnuller1) and (tang1).. (schnuller2) ..controls(tang2) and (tang3).. (schnuller3) ..controls(tang4) and (tang5).. (schnuller4) ..controls(tang6) and (schnuller5).. (schnuller5);
	
	% zeichne nun das Loch in der Mitte
	\def\angle{20} % Rotationswinkel
	\coordinate (c) at ($#2+#1*(-1.25,-1.25)$); % Mittelpunkt der Ellipse die den unteren Bogen bildet
	\begin{scope}
		\clip[rotate=\angle] ($(c)-#1*(1,0.6)$) rectangle ($(c)+#1*(1,-0.1)$);
		\path[draw,rotate=\angle,name path=l] (c) ellipse(#1*1 and #1*0.5);
	\end{scope}
	\path[name path=u,rotate=\angle] ($(c)-#1*(0,0.5)$) ellipse(#1*0.75 and #1*0.5);
	\path[name intersections={of=u and l}];
	\begin{scope}
		\clip[rotate=\angle] (intersection-1) rectangle ($(intersection-2)+#1*(0,0.5)$);
		\draw[rotate=\angle] ($(c)-#1*(0,0.5)$) ellipse(#1*0.75 and #1*0.5);
	\end{scope}		
}

\newcommand{\tikzsegel}[2][1]{
	% definiere die Knoten relativ zum ersten Knoten skaliert mit dem Faktor
	\coordinate (segel1) at #2; \coordinate (segel2) at ($(segel1)+#1*(4,1.5)$); \coordinate (segel3) at ($(segel1)+#1*(2,-0.5)$);
	%\fill (segel1) circle (0.05) (segel2) circle (0.05) (segel3) circle (0.05);
	
	% die Richtungsvektoren der Bezier Tangenten fuer die einzelnen Knoten (der Erste und der letzte haben keine Tangente)
	\coordinate (ctrls1) at ($#1*(0.75,1.5)$); \coordinate (ctrls2) at ($#1*(-0.75,0.25)$); \coordinate (ctrls3) at ($#1*(-0.5,-0.25)$); \coordinate (ctrls4) at ($#1*(0.25,1)$); \coordinate (ctrls5) at ($#1*(-0.375,0.375)$); \coordinate (ctrls6) at ($#1*(0.75,0.125)$);
	% die eigentlichen Tangenten
	\coordinate (tang1) at ($(segel1)+(ctrls1)$); \coordinate (tang2) at ($(segel2)+(ctrls2)$); \coordinate (tang3) at ($(segel2)+(ctrls3)$); \coordinate (tang4) at ($(segel3)+(ctrls4)$); \coordinate (tang5) at ($(segel3)+(ctrls5)$); \coordinate (tang6) at ($(segel1)+(ctrls6)$);
%	\fill[red] (tang1) circle (0.05); \fill[red] (tang2) circle (0.05); \fill[red] (tang3) circle (0.05); \fill[red] (tang4) circle (0.05); \fill[red] (tang5) circle (0.05); \fill[red] (tang6) circle (0.05);
 %   \draw[red] (tang1) -- (segel1) -- (tang6); \draw[red] (tang2) -- (segel2) -- (tang3); \draw[red] (tang4) -- (segel3) -- (tang5);
	
	\draw (segel1) ..controls(tang1) and (tang2).. (segel2) ..controls(tang3) and (tang4).. (segel3) ..controls(tang5) and (tang6).. (segel1) --cycle;
}

\newcommand{\tikztorus}[2][1]{
%	\tikzgitter{(-6,-1)}{(6,5)}
	% zuerst die aeussere Ellipse
	\draw[] #2  ellipse (#1*2 and #1*1);
	
	% dann das Loch
	\begin{scope}
		\clip ($#2 - #1*(1, 0.5)$) rectangle ($#2 + #1*(1, 1)$);
		\path[draw,name path=gkreis] ($#2 + #1*(0,0.75)$) ellipse (#1*1.25 and #1*1);
	\end{scope}
	\path[name path=kkreis] ($#2 - #1*(0,0.5)$) ellipse (#1*1 and #1*0.75);
	\path[name intersections={of=gkreis and kkreis}];
	\begin{scope}
		\clip (intersection-1) rectangle ($(intersection-2)+(0,0.5)$);
		\draw ($#2 - #1*(0,0.5)$) ellipse (#1*1 and #1*0.75);
	\end{scope}
	
	% definiere Werte auf die wir in der restlichen Zeichnung zurueckgreifen koennen
	\def\torusbreite{#1*2}
	\def\torushoehe{#1*1}
	\def\torusdicke{#1*0.75}
	\coordinate (torusUntenLoch) at ($#2 - #1*(0,0.25)$);
	\coordinate (torusUnten) at ($#2 - #1*(0,1)$);
}

\usepackage[toc]{glossaries} % Symbolverzeichnis
\glossarystyle{treehypergroup}
\makeglossaries

% Mathe Pakete
\usepackage{amsmath}
\usepackage{amssymb}
\usepackage{wasysym} % noch mehr symbole
\usepackage{stmaryrd}
\usepackage{bm} % fette Mathe Zeichen
\usepackage{mathtools} % zusaetzliche Mathe Befehle (zusaetzlich zu AMS Math)
\usepackage[hyperref,amsmath,thmmarks,thref]{ntheorem}

% --- Mathe Symbole ---

% canonic sets
\DeclareMathOperator{\C}{\mathbb{C}}
\DeclareMathOperator{\F}{\mathbb{F}}
\DeclareMathOperator{\K}{\mathbb{K}}
\DeclareMathOperator{\N}{\mathbb{N}}
\DeclareMathOperator{\Q}{\mathbb{Q}}
\DeclareMathOperator{\R}{\mathbb{R}}
\DeclareMathOperator{\RP}{\mathbb{RP}} % real projection plane
\DeclareMathOperator{\Tor}{\mathbb{T}} % torus
\DeclareMathOperator{\Z}{\mathbb{Z}}
\DeclareMathOperator{\B}{\mathbb{B}} % unit ball

% Redeclare \P (Prim or Propability) and put the old, reversed "breakline P" in \BreakLineP
\let\BreakLineP\P
\renewcommand{\P}{\ensuremath{\mathbb{P}}}
% das ungarische H umdefinieren
\let\umlautsH\H % long Hungarian umlaut (double acute)
\renewcommand{\H}{\ensuremath{\mathbb{H}}}
% das Paragraphenzeichen umdefinieren
\let\ParagraphS\S
\renewcommand{\S}{\mathbb{S}}


%  geschwungene Buchstaben
\DeclareMathOperator{\calD}{\mathcal{D}}
\DeclareMathOperator{\calI}{\mathcal{I}}
\DeclareMathOperator{\calJ}{\mathcal{J}}
\DeclareMathOperator{\calL}{\mathcal{L}}
\DeclareMathOperator{\calT}{\mathcal{T}}
\DeclareMathOperator{\calV}{\mathcal{V}}

% common mathematical operators and sets
\DeclareMathOperator{\aff}{aff} % affine Huelle
\DeclareMathOperator{\cs}{cs} % allgemeiner Cosinus
\DeclareMathOperator{\ct}{ct} % allgemeiner Cotangens
\DeclareMathOperator{\diam}{diam} % diameter
\DeclareMathOperator{\dist}{dist} % distance
\DeclareMathOperator{\dR}{dR} % deRahm
\DeclareMathOperator{\eukl}{eukl} % euklidisch
\DeclareMathOperator{\ggT}{ggT} % goesster gemeinsamer Teiler
\DeclareMathOperator{\id}{id} % identity
\DeclareMathOperator{\inh}{inh} % Inhalt
\DeclareMathOperator{\grad}{grad} % Gradient
\DeclareMathOperator{\kgV}{kgV} % kleinstes gemeinsames Vielfaches
\DeclareMathOperator{\mmod}{mod} % modulo
\DeclareMathOperator{\mspan}{span} % Lineare Huelle
\DeclareMathOperator{\n}{n} % Umlaufzahl
\DeclareMathOperator{\offen}{offen}
\DeclareMathOperator{\pr}{pr}
\DeclareMathOperator{\res}{res} % Residuum
\DeclareMathOperator{\rg}{rg} % rank (i)
\DeclareMathOperator{\ric}{ric} % Ricci Tensor
\DeclareMathOperator{\scal}{scal} % Skalarkruemmung
\DeclareMathOperator{\sgn}{sgn} % Signum
\DeclareMathOperator{\sn}{sn} % allgemeiner Sinus
\DeclareMathOperator{\spur}{spur} % Spur
\DeclareMathOperator{\supp}{supp} % support
\DeclareMathOperator{\sternf}{sternf}
\DeclareMathOperator{\tr}{tr} % Spur

\DeclareMathOperator{\Abb}{Abb} % maps
\DeclareMathOperator{\Aut}{Aut} % automorphisms
\DeclareMathOperator{\Bild}{Bild}
\DeclareMathOperator{\CAT}{CAT}
\DeclareMathOperator{\Charakteristik}{char}
\DeclareMathOperator{\Charakt}{char}
\DeclareMathOperator{\D}{D} % Jacobi matrix or derivative
\DeclareMathOperator{\Diff}{Diff}
\DeclareMathOperator{\End}{End} % endomorphisms
\DeclareMathOperator{\Gl}{GL} % general linear group
\DeclareMathOperator{\GL}{GL} % general linear group
\DeclareMathOperator{\Gr}{Gr}
\DeclareMathOperator{\Graph}{Graph}
\DeclareMathOperator{\Hh}{H} % Hessesche
\DeclareMathOperator{\Hom}{Hom} % homomorphisms
\DeclareMathOperator{\Id}{id} % identity
\DeclareMathOperator{\Ind}{Ind} % Index
\DeclareMathOperator{\Inn}{Inn} % Untergruppe der inneren Automorphismen
\DeclareMathOperator{\Iso}{Iso}
\DeclareMathOperator{\Kern}{Kern}
\DeclareMathOperator{\Oo}{O} % Matrizen sie mit ihrer Transponierten multipiziert die Einheitsmatrix ergeben
\DeclareMathOperator{\Relation}{\scriptstyle\mathrm{R}} % custom Relation
\DeclareMathOperator{\Ric}{Ric} % Ricci Tensor field
\DeclareMathOperator{\Rang}{Rang} % rank (ii)
\DeclareMathOperator{\SL}{SL} % Matrizen mit Deteminante 1
\DeclareMathOperator{\Stab}{Stab} % Stabilisator
\DeclareMathOperator{\Sym}{Sym} % symmetric group
\DeclareMathOperator{\T}{T} % tangent bundle

\newcommand{\Zentrum}[1]{\ensuremath{\mathrm Z(#1)}} % Zentrum einer Gruppe
\newcommand{\Ordnung}[1][]{ % Ordnung einer Gruppe
  \ifthenelse{\isempty{#1}}{
    \#
  }{
    \left|#1\right|
  }
}

% X als Malzeichen und andere Operatorzeichen
\newcommand{\dop}{\mathrm{d}}
\newcommand{\X}{\times}
\newcommand{\tensor}{\otimes}
\newcommand{\ii}{{\mathrm i}}

%Realteil und Imaginaerteil
\renewcommand{\Re}{\ensuremath{\operatorname{Re}}}
\renewcommand{\Im}{\ensuremath{\operatorname{Im}}}

% stellt einen gro"sen vertikalen Strich an einen Term, nuetzlich in Bruechen
\newcommand{\bigvert}[1]{\left. #1 \right|}

% Differentialoperatoren als Br"uche
\newcommand{\difffrac}[3][]{\ifthenelse{\isempty{#1}}{\frac{\dop #2}{\dop #3}}{\left. \frac{\dop #2}{\dop #3} \right|_{#1}}}
\newcommand{\pdifffrac}[3][]{\ifthenelse{\isempty{#1}}{\frac{\partial #2}{\partial #3}}{\left. \frac{\partial #2}{\partial #3} \right|_{#1}}}

% ein schoener aussehender Faktorraum anstatt einfach nur A/B
\newcommand{\FakRaum}[2]{
	\raisebox{0.7ex}{\ensuremath{#1}}
	\ensuremath{\mkern-3mu}\big/\ensuremath{\mkern-3mu}
	\raisebox{-0.6ex}{\ensuremath{#2}}}
\newcommand{\smallFakRaum}[2]{
	\scriptsize{\raisebox{0.7ex}{\ensuremath{#1}}
	\ensuremath{\mkern-3mu}\ / \ensuremath{\mkern-3mu}
	\raisebox{-0.6ex}{\ensuremath{#2}}}}

% quotient space or group
\newcommand{\modulo}[1]{\ensuremath{/_{\displaystyle #1}}}

% declaring Index for group theory
\newcommand{\Index}[2]{\ensuremath{(#1 \SlimDdot #2)}}


% Theoremartige Umgebungen
\theoremstyle{break}
\newtheorem{Dfn}{Definition}[chapter]
\newtheorem{Satz}[Dfn]{Satz}
\newtheorem{Lemma}[Dfn]{Lemma}
\newtheorem{Kor}[Dfn]{Korollar}
\newtheorem{Prop}[Dfn]{Proposition}
\theorembodyfont{\normalfont}
\newtheorem{Bsp}[Dfn]{Beispiel}
\newtheorem{Bspe}[Dfn]{Beispiele}
\newtheorem{Bem}[Dfn]{Bemerkung}

\theoremstyle{nonumberbreak}
\newtheorem{dfn}{Definition}
\newtheorem{satz}{Satz}
\newtheorem{lemma}{Lemma}
\newtheorem{kor}{Korollar}
\newtheorem{prop}{Proposition}

\newtheorem{bsp}{Beispiel}
\newtheorem{bspe}{Beispiele}
\newtheorem{bem}{Bemerkung}

\theorembodyfont{\normalfont}
\theoremsymbol{\ensuremath{\Box}}
\newtheorem{bew}{Beweis}
\theoremsymbol{\ensuremath{(\Box)}}
\newtheorem{bewSkiz}{Beweisskizze}

\theoremsymbol{}
\theoremstyle{emptybreak}
\newtheorem{emptythm}{}% druckt nur den optionalen Namen aus

\theoremstyle{break}
\newtheorem{Aufg}{Aufgabe}
\newtheorem{Loes}{L\"osung}

% changing enumerations
\setlist[enumerate]{label=(\arabic*), itemsep=0cm, leftmargin=1cm}
\setlist[itemize]{itemsep=0cm}

% set line distances
\linespread{1.1}

% Add a ':' for mathmode with tiny whitespaces around
\newcommand{\SlimDdot}{\ensuremath{\mathrm{:}}}

% indexing support

% Print and index given text
% usage: \CmIndex{[(optionally put another text for the index in here)]}{(text to print and add to index)}
\newcommand{\CmIndex}[2][]{\ifthenelse{\isempty{#1}}{\index{#2}}{\index{#1}}#2}

% Highlight(bold) and index the given text
% usage: \CmMark[(optionally put another text for the index in here)]{(text to highlight and add to index)}
\newcommand{\CmMark}[2][]{\textbf{\CmIndex[#1]{#2}}}

\parindent0pt % der Einschub bei neuem Absatz

% Befehl fuer Anfuerungszeichen unten und oben
\newcommand{\quot}[1]{\textrm{\glqq}{#1}\textrm{\grqq}}

%========================================================================================================================
%	G L O S S A R
%========================================================================================================================


%========================================================================================================================
%	I N F O R M A T I O N E N	Z U M	S K R I P T
%========================================================================================================================

\subject{inoffizielles Skript}
\title{Angewandte Differentialgeometrie}
\subtitle{Gehalten von Dr. S. Grensing im Sommersemester 2013}
\author{getippt von Aleksandar Sandic\thanks{\href{mailto:aleksandar.sandic@student.kit.edu}{Aleksandar.Sandic@student.kit.edu}}}

%========================================================================================================================
%	I N H A L T
%========================================================================================================================
\begin{document}
\maketitle

% Inhaltsverzeichnis
\pdfbookmark[1]{Inhaltsverzeichnis}{contents}
\setlength\parskip{0.6pt}
\tableofcontents

\chapter*{Vorwort}
\addcontentsline{toc}{chapter}{Vorwort}

\section*{\"Uber dieses Skript}
Dies ist eine Mitschrieb der Vorlesung \quot{Angewandte Differentialgeometrie} von Dr. S. Grensing, gehalten im Sommersemester 2013 am Karlsruher Institut f"ur Technologie.
Dr. Grensing ist nicht f"ur den Inhalt verantwortlich und es besteht weder eine Garantie f"ur Vollst"andigkeit, noch Korrektheit der enthaltenen Aussagen.

\section*{Wer}
Das Skript wurde in von Aleksandar Sandic getippt.
Bei Anmerkungen bzw. beim Auffinden von Fehlern schickt bitte eine E-Mail an
\begin{center}
  \href{mailto:aleksandar.sandic@student.kit.edu}{aleksandar.sandic@student.kit.edu}
\end{center}

\section*{Wo}
Link zur Vorlesung: \url{http://www.math.kit.edu/iag5/lehre/angdifgeo2013s/de}\\
Link zum Skript: \url{http://mitschriebwiki.nomeata.de/AngDiffGeo2013.pdf}\\
Link zum Mitschiebwiki: \url{http://mitschriebwiki.nomeata.de/}

\section*{To Do}\begin{itemize}
\item die Vorlesung l"auft noch
\end{itemize}

%========================================================================================================================
%	V O R L E S U N G
%========================================================================================================================

%-------------------------- Kapitel 1 --------------------------
% Vorlesung vom 16. 4.
\chapter{R\"auber und Gendarm}

\section{Metrische R\"aume mit oberen Kr\"ummungsschranken}

Es sei $(X, d)$ ein metrischer Raum. Ein Weg $c: \calI \to X$ hei"st \CmMark[Geod\"atische!minimierende]{(minimierende) Geod"atische}, wenn $d(c(t), c(t')) = (t - t')$ f"ur alle $t, t' \in \calI$ gilt.
Der Raum $X$ hei"st \CmMark[Raum!geod\"atischer]{geod"atischer Raum}, wenn f"ur alle $x, y \in X$ eine Geod"atische von $x$ nach $y$ existiert, beziehungsweise \CmMark[Raum!$R$-geod\"atischer]{$\bm{R}$-geod"atisch}, wenn dies f"ur $d(x, y) \le R$ gilt.
Ist $c: [a, b] \to X$ ein Weg so hei"st er \CmMark[Weg!rektifizierbarer]{rektifizierbar}, falls seine L"ange \marginnote{\begin{tikzpicture}[font=\scriptsize,scale=0.5,baseline=0]
%\tikzgitter{(-4,-1)}{(4,3)}
	\coordinate (1) at (-3,-0.5); \coordinate (2) at (3,2.5);
	\coordinate (ctrl1) at (1,2.5); \coordinate (ctrl2) at (-0.5,-2);
	\draw (1) ..controls($(1) + (ctrl1)$) and ($(2) + (ctrl2)$).. coordinate[pos=0.1](a) coordinate[pos=0.5](b) coordinate[pos=0.75](c) coordinate[pos=0.95](d) node[pos=0.65,below]{$c$} (2);
	\draw[dashed] (a) node[left]{$c(t_i)$} -- node[below]{$d_i$} (b) node[above]{$c(t_{i+1})$};
	\draw[decorate,decoration={brace}] (c) -- node[above left]{$d_n$} (d);
\end{tikzpicture}}
\begin{align*}
	\calL(c) = \sup \left\{ \sum d(c(t_i), c(t_{i+1})) \mid t_1 < \ldots t_n, t_i \in \calI \right\}
\end{align*}
endlich ist.
Es gilt $d(c(a), c(b)) \le \calL(c)$.
Die Kurvenl"ange ist invariant unter monotonen Reparametrisierungen.
Die L"angenfunktion $t \mapsto \calL(c|_{[a,t]})$ ist monoton wachsend; insbesondere besitzt jede Kurve eine Bogenl"angenparametrisierung.
Der Raum $(X, d)$ hei"st \CmMark[Raum!L\"angen-]{L"angenraum} (oder $d$ \CmMark[Metrik!innere]{innere Metrik}), falls die Metrik
\begin{align*}
	\overline{d}(x, y) = \inf \{ \calL(c) \mid c: \calI \to X \text{ rektif'barer Weg}, c(0) = x, c(1) = y\}
\end{align*}
mit $d$ "ubereinstimmt.

\begin{bspe}\begin{enumerate}[label=\arabic*), leftmargin=*]
\item
	$\R^n$ mit der euklidischen Metrik ist ein geod"atischer Raum und ein L"angenraum.
\item
	Jede Riemannsche Mannigfaltigkeit ist ein L"angenraum (Beweis zur "Ubung).
\item
	Ist $(X, d)$ ein metrischer Raum, so folgt dass $(X, \overline d)$ eine L"angenmetrik ist, das heißt $\overline{\overline{d}} = \overline{d}$ (das iterieren der Konstruktion liefert keine neue Metrik).
\item
	Jeder geod"atische Raum ist ein L"angenraum.
	Es gilt stets $d \le \overline{d}$.
	Sind $x, y \in X$ und ist $\gamma : [a, b] \to X$ eine Geod"atische von $x$ nach $y$, so gilt:
	\begin{align*}
		\calL(\gamma) &= \sup \left\{ \sum d(\gamma(t_i), \gamma(t_{i+1})) \right\} \\
		&= \sup \left\{ \sum (t_{i+1} - t_i) \right\} \\
		&= t - a = d(x, y). \shortintertext{Also}
		\overline{d}(x, y) &= \inf_c \calL(c) \le \calL(\gamma) = d(x, y)
	\end{align*}
\item
	$\R^2 \setminus \{0\}$ ist ein L"angenraum (bez"uglich der euklidischen Metrik), aber kein geod"atischer Raum.
	\begin{center}
		\begin{tikzpicture}[font=\scriptsize]
			%\tikzgitter{(-4,-1)}{(4,3)}
			\draw[->] (-1.5,0) -- (1.5,0);
			\draw[->] (0,-0.5) -- (0,1.5);
			\fill (-1,0) circle(0.05) node[below]{$-1$} (1,0) circle(0.05) node[below]{$1$};
			\draw (-1,0) to[out=20,in=160] (1,0);
		\filldraw[fill=white] (0,0) circle(0.05);
		\end{tikzpicture}\\
		Es gibt keine kürzeste Verbindung
	\end{center}
\end{enumerate}\end{bspe}

\begin{satz}[von Hopf-Rinow; Chom-Vossen 1935]
Es sei $X$ ein lokalkompakter L"angenraum. Dann sind die folgenden Aussagen "aquivalent:
\begin{enumerate}[label=(\roman*), widest=iii]
\item
	$X$ ist vollst"andig
\item
	$X$ ist \CmMark[vollst\"andig!geod\"atisch]{geod"atisch vollst"andig}, das hei"st jede Geod"atische $c: [0,1) \to X$ kann in $1$ fortgesetzt werden.
\item
	Beschr"ankte abgeschlossene Mengen sind kompakt.
\end{enumerate}
\end{satz}

Jede der obigen Aussagen impliziert, dass $X$ ein geod"atischer Raum ist.
\begin{center}\begin{tikzpicture}[font=\scriptsize]
	%\tikzgitter{(-4,-1)}{(4,3)}
	\coordinate (x) at (-2.5,-0.5); \coordinate (y) at (2.5,1.5);
	\foreach \i/\j in {40/180, 55/170, 70/160, 85/150} {
		\draw (x) ..controls($(x)+(\i:2.5)$) and ($(y)+(\j:2.5)$).. (y);
	}
	\fill (x) circle(0.05)node[below]{$x$} (y) circle(0.05)node[right]{$y$};
\end{tikzpicture}\\
Kanten $e_n$ der L"ange $1+\frac{1}{n}$ $\leadsto$ Es gibt keine Kurve der L"ange $1$\end{center}
Es bezeichne $M_\kappa^2$ die (eindeutigen) 2-dimensionalen, einfachzusammenh"angenden Riemannschen Mannigfaltigkeiten mit der Schnittkr"ummung $\sec \equiv \kappa$, und $D_\kappa$ ihren Durchmesser:
\begin{align*}
	M_\kappa^2 &= \begin{cases}
		\S_{\frac{1}{\sqrt{\kappa}}}^2 & ,\kappa > 0 \\
		\R^2 & ,\kappa = 0 \\
		\H_{\frac{1}{\sqrt{\kappa}}}^2 & ,\kappa < 0
	\end{cases} \\
	D_\kappa &= \begin{cases}
		\frac{\pi}{\sqrt{\kappa}} & ,\kappa > 0 \\
		\infty & ,\kappa \le 0
	\end{cases} \shortintertext{wobei}
	\H_{\frac{1}{\sqrt{\kappa}}}^2 &= \left\{ x \in \R^3 \mid x_1^2 + x_2^2 - x_3^2 = -\frac{1}{\kappa} \right\}.
\end{align*}
Es sei $X$ ein metrischer Raum. Ein \CmMark[Dreieck!geod"atisches]{geod"atisches Dreieck} $\Delta(x, y, z)$ besteht aus Geod"atischen $c_{xy} = \overline{xy}$, $c_{yz} = \overline{yz}$ und $c_{zx} = \overline{zx}$ zwischen den Punkten $x$, $y$ und $z$.
Ein Vergleichsdreieck in $M_\kappa^2$ ist ein geod"atisches Dreieck $\overline\Delta(\overline{x}, \overline{y}, \overline{z})$ mit gleichen Kantenl"angen wie das urspr"ungliche Dreieck.
Ein solches Vergleichsdrieck existiert eindeutig, falls der Umfang $d(x, y) + d(y, z) + d(z, x) \le 2 D_\kappa$ ist.
Ein Punkt $\overline p \in \overline\Delta(\overline{x}, \overline{y}, \overline{z})$ hei"st \CmMark[Punkt!Vergleichs-]{Vergleichspunkt} zu $p \in \Delta(x, y, z)$, $p \in \overline{xy}$, falls $\overline{d}(\overline{x}, \overline{p}) = d(x, p)$.
Das Dreieck $\Delta(x, y, z)$ erf"ullt die \CmMark[CAT(k)-Ungleichung@$\CAT(\kappa)$-Ungleichung]{$\bm{\CAT(\kappa)}$-Ungleichung}, wenn f"ur alle $p, q \in \Delta(x, y, z)$ mit vergleichpunkten $\overline{p}, \overline{q}$ gilt:
\begin{align*}
	d(p, q) \le \overline{d}(\overline{p}, \overline{q})
\end{align*}
\begin{center}\begin{tikzpicture}[font=\scriptsize]
	%\tikzgitter{(-4,-1)}{(4,3)}
	\coordinate (yShift) at (3,0.25); \coordinate (zShift) at (2,2);
	\coordinate (x) at (-3.5,0); \coordinate (y) at ($(x)+(yShift)$); \coordinate (z) at ($(x)+(zShift)$);
	\coordinate (x') at ($(x)+(4.5,0)$); \coordinate (y') at ($(x')+(yShift)$); \coordinate (z') at ($(x')+(zShift)$);
	\draw (x) to[out=20,in=170] coordinate[pos=0.5] (p) (y) to[out=135,in=270] (z) to[out=240,in=35] coordinate[pos=0.5] (q) (x) -- cycle;
	\fill (x) circle(0.05)node[below]{$x$} (y) circle(0.05)node[right]{$y$} (z) circle(0.05)node[above]{$z$};
	\fill (p) circle(0.05) (q) circle(0.05);
	\draw[dashed] (p) node[below]{$p$} -- (q) node[above]{$q$};
	\node at ($(x)+(3,1.5)$){$\subseteq X$};
	
	\draw (x') --coordinate[pos=0.5] (p') (y') -- (z') --coordinate[pos=0.5] (q') (x') -- cycle;
	\fill (x') circle(0.05)node[below]{$\overline x$} (y') circle(0.05)node[right]{$\overline y$} (z') circle(0.05)node[above]{$\overline z$};
	\fill (p') circle(0.05) (q') circle(0.05);
	\draw[dashed] (p') node[below]{$\overline p$} -- (q') node[above]{$\overline q$};
	\node at ($(x')+(3,1.5)$){$\subseteq \R^2$};
\end{tikzpicture}\end{center}
Der Raum $X$ besitzt die obere Kr"ummungsschranke $\kappa$, wenn f"ur alle $x \in X$ eine geod"atische Umgebung existiert, in der alle Dreiecke die $\CAT(\kappa)$-Ungleichung erfüllen. F"ur $\kappa = 0$ hei"st $X$ \CmMark[ger\"ummt!nicht-positiv]{nicht-positiv gekr"ummt}. Erf"ullt $X$ global die $\CAT(\kappa)$-Ungleichung, so hei"st $X$ \CmMark[Raum!$\CAT(\kappa)$-]{$\bm{\CAT(\kappa)}$-Raum}.

\begin{bspe}\begin{enumerate}[label=\arabic*)]
\item
	$\R^2$ ist $\CAT(0)$, $M^2 = \FakRaum{\R^2}{\Z^2}$ ist nichtpositiv gekr"ummt.
\item
	$\S^1$ ist $\CAT(1)$.
\item
	$\R^2 \setminus Q_1$, $Q_1 = \{ x_1 > 0, x_2 > 0 \}$, mit der induzierten L"angenmetrik ist $\CAT(0)$.
	\begin{center}\begin{tikzpicture}[font=\scriptsize]
		%\tikzgitter{(-4,-1)}{(4,3)}
		\coordinate (x) at (-2,-1); \coordinate (y) at (1.5,-0.5); \coordinate (z) at (-0.5,1);
		
		\draw (x) -- (y) -- (0,0) -- (z) -- (x) -- cycle;
		\fill (x) circle(0.05)node[below]{$x$} (y) circle(0.05)node[below]{$y$} (z) circle(0.05)node[above]{$z$};z
		
		\fill[fill=gray!30,fill opacity=0.5] (0,0) rectangle(2.4,1.4);
		\node at (1,0.75) {$Q_1$};
		\draw[->] (-2.5,0) -- (2.5,0);\draw[->] (0,-1.5) -- (0,1.5);
	\end{tikzpicture}\end{center}
\item
	$(M, g)$ Riemannsche Manngifaltigkeit, Schnittkr"ummung $\sec_g \le \kappa$, vollst"andig. nach dem Satz von Topogonov hat $M$ die obere Kr"ummungsschranke $\kappa$.
\end{enumerate}\end{bspe}

\begin{emptythm}[Eigenschaften von $\CAT(\kappa)$-R"aumen]\begin{itemize}
\item
	Geod"atische der L"ange $< D_\kappa$ sind eindeutig.
\item
	B"alle vom Radius $< \frac{1}{2} D_\kappa$ sind konvex und zusammenziehbar.
\item
	$\CAT(0)$-R"aume sind zusammenziehbar.
\end{itemize}\end{emptythm}

\begin{emptythm}[Charakterisierung von $\CAT(\kappa)$-R"aumen]
F"ur jedes geod"atische Dreieck $\Delta(x, y, z)$ vom Umfang $< 2 D_\kappa$ und Punkte $p \in \overline{xy}$ und $q \in \overline{xz}$ gilt f"ur Vergleichsdreiecke $\overline\Delta(\overline{x}, \overline{y}, \overline{z})$ und $\overline\Delta(\overline{x}, \overline{p}, \overline{q})$:
\begin{align*}
	\varangle_{\overline{x}} (\overline{p}, \overline{q}) \le \varangle_{\overline{x}} (\overline{y}, \overline{z}) \tag{in $M_\kappa^2$}
\end{align*}
\begin{center}\begin{tikzpicture}[font=\scriptsize,scale=1.5]
	%\tikzgitter{(-4,-1)}{(4,3)}
	\coordinate (yShift) at (3,0.25); \coordinate (zShift) at (2,2);
	\coordinate (x) at (-3.5,0); \coordinate (y) at ($(x)+(yShift)$); \coordinate (z) at ($(x)+(zShift)$);
	\coordinate (x') at ($(x)+(4.5,0)$); \coordinate (y') at ($(x')+(yShift)$); \coordinate (z') at ($(x')+(zShift)$);
	\draw (x) to[out=20,in=170] coordinate[pos=0.5] (p) (y) to[out=135,in=270] (z) to[out=240,in=35] coordinate[pos=0.5] (q) (x) -- cycle;
	\fill (x) circle(0.05)node[below]{$x$} (y) circle(0.05)node[right]{$y$} (z) circle(0.05)node[above]{$z$};
	\fill (p) circle(0.05) (q) circle(0.05);
	\draw (p) node[below]{$p$} -- (q) node[above]{$q$};
	
	\fill (x') circle(0.05)node[below]{$\overline x$} (y') circle(0.05)node[right]{$\overline y$} (z') circle(0.05)node[above]{$\overline z$};
	\coordinate (p') at ($(x')+(15:2)$); \coordinate (q') at ($(x')+(35:1.75)$);
	\fill (p') circle(0.05)node[right]{$\overline p$} (q') circle(0.05)node[above]{$\overline q$};
	\node at ($(x')+(3,1.5)$){$\Delta(\overline{x},\overline{y},\overline{z})$};
	\node at ($(x')+(22:2.3)$){$\overline{\Delta}(\overline{x},\overline{p},\overline{q},)$};

	\draw[clip] (x') -- (y') -- (z') -- (x') -- cycle;
	\draw (x') circle(1.25);
	\draw[clip] (x') -- (p') -- (q') -- (x') -- cycle;
	\draw (x') circle(1);
\end{tikzpicture}\end{center}
Ist $X$ ein $\CAT(\kappa)$-Raum und sind $c_1, c_2$ Geod"atische mit gleichem Startpunkt $c_1(0) = c_2(0) = x$ so ist der Winkel $\varangle_{\overline{x}} (\overline{c_1(s)}, \overline{c_2(t)})$ monoton wachsend in $s$ und $t$.
Damit ist der Winkel
\begin{align*}
	\varangle(c_1, c_2) = \lim_{\mathclap{s,t \to 0}} \varangle_{\overline{x}} (\overline{c_1(s)}, \overline{c_2(t)}) 
	= \lim_{\mathclap{t \to 0}} \varangle_{\overline{x}} (\overline{c_1(t)}, \overline{c_2(t)})
\end{align*}
wohldefiniert.
\end{emptythm}

\begin{satz}[von Hadamart-Cartan; Alexander-Bishop 1990]
Es sei $X$ ein vollst"andiger metrischer Raum mit der oberen Kr"ummungsschranke $\kappa \le 0$. Dann ist seine universelle "Uberlagerung $\tilde X$ (bez"uglich der induzierten Metrik) (global) $\CAT(\kappa)$.
\end{satz}

% Vorlesung vom 29. 4.

\section{Konvexit\"at}
Eine Teilmenge $C$ eines $\CAT(0)$-Raumes (geod"atischer Raum) hei"st \CmMark{konvex}, falls f"ur alle $x, y \in C$ das geod"atische Segment von $x$ nach $y$ existiert und in ganz $C$ verl"auft.
Eine Abbildung $f: X \to \R$ auf einem $\CAT(0)$-Raum hei"st \CmMark{konvex}, falls f"ur jede Geod"atische $c: [0,1] \to X$ die Funktion $f \circ c: [0,1] \to \R$ konvex im gew"ohnlichen Sinne ist.

Ist $X$ ein $\CAT(0)$-Raum, so ist die Metrik konvex, das hei"st f"ur Geod"atische $c_1,c_2: [0,1] \to X$ ist die Funktion
\begin{align*}
	t \mapsto d(c_1(t), c_2(t))
\end{align*}
konvex, wie man in der folgenden Beweisskizze erkennt:

\begin{bewSkiz}
Wir betrachten ohne Einschr"ankung den Fall dass beide Geod"atischen den gleichen Startpunkt $x = c_1(0) = c_2(0)$ haben:
\begin{center}\textcolor{red}{[BILD]}\end{center}
Bei Betrachtung des Vergleichsdreiecks zu $\Delta(c_1(0), c_1(1), c_2(1))$ erhalten wir aus der $\CAT(0)$-Ungleichung:
\begin{align*}
	d(c_1(t), c_2(t)) \le d_{\R^2} (\overline{c_1(t)}, \overline{c_2(t)}) 
	= t \cdot d_{\R^2} (\overline y, \overline z) 
	= t \cdot d_{\R^2} (c_1(1), c_2(1))
\end{align*}
Betrachte nun $c$ von $c_2(0)$ nach $c_1(1)$.
\begin{center}\textcolor{red}{[BILD]}\end{center}
Nach der Dreiecksungleichung gilt:
\begin{align*}
	d(c_1(t), c_2(t)) &\le d(c_1(t), c(t)) + d(c_2(t), c(t)) \\
	&\le (1-t) \cdot d(c_1(0), c_2(0)) + t \cdot d(c_1(1), c_2(1))
\end{align*}
\end{bewSkiz}

Geod"atische Segmente in X sind immer eindeutig, wie man mit der folgenden Skizze erkennen kann:
\begin{center}\textcolor{red}{[BILD]}\end{center}
Die Abstandsfunktion $c(c_1(\cdot), c_2(\cdot))$ ist konvex mit Nullstellen in $0$ und $1$, also konstant Null und damit $c_1 = c_2$.

Fixiere $x_0 \in X$, und sei $c_x$ das geod"atische Segment von $x_0$ nach $x$.
\begin{center}\textcolor{red}{[BILD]}\end{center}
Damit wird $H(x,t) = c_x(t)$ eine Retraktion von $X$ auf $\{x_0\}$.

Ist $C$ eine vollst"andige konvexe Teilmenge  von $X$, so existiert eine \quot{orthogonale Projektion} $\pi: X \to C$ mit folgenden Eigenschaften:
\begin{enumerate}[label=(\roman*),widest=iii]
\item
	F"ur jedes $x \in X$ ist $\pi(x) \in X$ der eindeutig bestimmte Punkt mit
	\begin{align*}
		d(x, \pi(x)) = d(x, C) = \inf \{d(x, c) \mid c \in C\}
	\end{align*}
\item
	F"ur Punkte $y \in C$ und $x \notin C$, mit $\pi(x) \ne y$, gilt $\varangle_{\pi(x)}(x, y) \ge \frac{\pi}{2}$.
\item
	Es gilt $\pi(\overline{x \pi(x)}) = \{\pi(x)\}$.
\item
	$\pi$ ist eine Retraktion von $X$ auf $C$, welche Abst"ande nicht vergr"o"sert.
\end{enumerate}
\begin{center}\textcolor{red}{[BILD]}\end{center}

\begin{emptythm}[Konstruktion]
Es seien $X_1$ und $X_2$ zwei $\CAT(0)$-R"aume. Dann ist $X_1 \X X_2$ ebenfalls ein $\CAT(0)$-Raum bez"uglich der Produktmetrik. Die Geod"atischen von $X_1 \X X_2$ sind genau die Produkt $c_1 \X c_2$ von Geod"atischen $c_i$ in $X_i$.
\begin{center}\textcolor{red}{[BILD]}\end{center}
Weitere $\CAT(0)$-R"aume erh"alt man durch \quot{Verkleben} entlang konvexer Teilmengen.
Es sei $A$ ein vollst"andiger metrischer Raum und seien $\iota_i: A \hookrightarrow A_i \subseteq X_i$ Isometrien auf konvexen Teilmengen in $X_1$ beziehungsweise $X_2$.
Dann ist $X_1 \cup_A X_2 = \FakRaum{X_1 \dot{\cup} X_2}{\{\iota_1(a)=\iota_2(a) \in A\}}$ mit der Metrik
\begin{align*}
	d(x,y) = \begin{cases}
		d_{X_i}(x,y) & x,y \in X_i \\
		\inf_{a \in A} \{ d(x, \iota_i(a)) + d(\iota_j(a), y) \} & x \in X_i, y \in X_j, i \ne j
	\end{cases}
\end{align*}
ein $\CAT(0)$-Raum.
\begin{center}\textcolor{red}{[BILD]}\\
Betrachte die Vergleichsdreiecke $\Delta(\overline x, \overline{x'}, \overline{z'})$ und $\Delta(\overline{z'}, \overline{y}, \overline{x'})$
\end{center}
\end{emptythm}

\section{R\"auber und Gendarm}

\begin{emptythm}[Regeln]
Das Spielfeld $D$ sei eine zusammenh"angende Teilmenge des $\R^n$.
Eine Startkonfiguration sei gegeben durch eine endliche Anzahl von Verfolgern $P_1, \ldots, P_N \in D$ und einen Flüchtigen $E \in D$.
Wir betrachten zun"achst (zeitlich) diskrete Modelle.
Es bezeichnen $P_k^t$ beziehungsweise $E^t$ die Positionen der Verfolger beziehungsweise des Fl"uchtigen zum Zeitpunkt $t \in \N$.
\end{emptythm}

\begin{emptythm}[Spielverlauf]
Zum Zeitpunkt $t$ w"ahlt zun"achst $E^t$ eine neue Position $E^{t+1}$ mit dem Abstand $d(E^t, E^{t+1}) \le 1$, also h"ochstens eine Einheit entfernt von der vorherigen Position, danach (simultan) die Verfolger $P_k^t$, entsprechend mit $d(P_k^t, P_k^{t+1}) \le 1$.
Die Verfolger gewinnen, wenn f"ur jedes $C > 1$ ein $t \in N$ existiert, so dass $d(E^t, P_k^t) < C$ f"ur ein $k \in \N$ gilt.
\end{emptythm}

\begin{satz}
Auf jedem Kompaktum $D \subseteq \R^n$ ist \quot{\CmMark{Greedy}} stets erfolgreich (mit mindestens einem Verfolger).
\end{satz}

\begin{emptythm}[Greedy]
Der Verfolger bewegt sich um die Distanz $1$ auf der Strecke $\overline{P^t E^t}$:
\begin{center}\textcolor{red}{[BILD]}\end{center}
\end{emptythm}

\begin{bewSkiz}[allgemein sp"ater]
Skizze zum Greedy Algorithmus:
\begin{center}\textcolor{red}{[BILD]}\end{center}
Angenommen der Fl"uchtige w"urde entkommen. Dann w"are $d^\infty > 1$ und damit der Winkel $\alpha^t \xrightarrow{t \to \infty} 0$, das bedeutet $D$ w"urde beliebig lange Geradensegmente enthalten, w"are also nicht kompakt.
Das ist ein Widerspruch zur Voraussetzung.
\end{bewSkiz}

%========================================================================================================================
%	A N H A N G
%========================================================================================================================

\appendix

%========================================================================================================================
%	UE B U N G
%========================================================================================================================

% Die Benennung der "section" so aendern, dass "\"Ubung 123 vom " am Anfang steht
% Der Code ist fast genau der vom Anfang der Praeambel, dort steht die Erklaerung
\renewcommand*{\othersectionlevelsformat}[3]{\ifstr{#1}{section}{\"Ubung\ #3\ vom\ }{#3\autodot\enskip}}

% Das Format der "section" in Kopfzeile der rechten Seiten
\renewcommand*{\sectionmarkformat}{\"Ubung \thesection\autodot\ vom\enskip}

%========================================================================================================================
%	S T I C H W O R T V E R Z E I C H N I S
%========================================================================================================================

\addcontentsline{toc}{chapter}{Stichwortverzeichnis} % <- damit es auch im Inhaltsverzeichnis erscheint
\printindex

%========================================================================================================================
%	G L O S S A R
%========================================================================================================================



%========================================================================================================================
%	L I T E R A T U R V E R Z E I C H N I S
%========================================================================================================================
\bibliographystyle{plain}

\end{document}
