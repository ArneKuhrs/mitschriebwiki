% !TEX TS-program = pdflatexmk
% das oben sorgt dafuer dass die pdflatexmk Engine fuer das Setzten verwendet wird (bei Verwendung des Editors TeXShop)

%%
%% Skript Angewandte Differentialgeometrie im Sommersemester 2013
%% Zur Vorlesung von Dr. Grensing am KIT in Karlsruhe
%%

\documentclass[paper=A4, twoside, chapterprefix=true, bibliography=totoc, headsepline]{scrbook}

%========================================================================================================================
%	P R AE A M B E L
%========================================================================================================================

% Meta-Daten fuer Latexki
\usepackage{latexki}
\lecturer{Dr. Sebastian Grensing}
\semester{Sommersemester 13}
\scriptstate{current}

% Abstand zwischen zwei Textbloecken
\setlength\parskip{\smallskipamount}

% Nummerierung der Paragraphen anpassen (sonst kommt etwas wie "Definition 2.9.1" heraus)
\renewcommand{\thesection}{\arabic{section}}

% Aendert die Kapitelbeschriftung in der Kopfzeile der linken Seiten
\renewcommand*{\chaptermarkformat}{\chapappifchapterprefix{\ }\thechapter:\enskip}
\renewcommand*{\sectionmarkformat}{\thesection\autodot\enskip}

% Einheitliche Schriftart (KOMA Script verwendet fuer einige Ueberschriften eine serifenlose Schrift, mischt also
% Schriftarten. Ich habe mir die Argumente dafuer durchgelsen und war nicht ueberzeugt. Wenn jemand, der mehr als
% ich von der Materie versteht, anderer Meinung ist kann er diese Zeilen hier einfach auskommentieren)
\setkomafont{chapter}{\Huge\bfseries\rmfamily}
\setkomafont{chapterentry}{\bfseries\rmfamily}
\setkomafont{disposition}{\bfseries\rmfamily}
\setkomafont{descriptionlabel}{\bfseries\rmfamily}

\usepackage[utf8x]{inputenc}
\usepackage[T1]{fontenc}
\usepackage{lmodern}

\usepackage[ngerman]{babel}
\usepackage[top=2.5cm, bottom=3cm, left=2.5cm, right=4.5cm]{geometry}

\usepackage{fancyhdr} % erlaubt mehr Optionen in Kopf- und Fusszeile
\usepackage{xcolor} % Farben
\usepackage{marginnote} % Randnotizen
\usepackage{enumitem} % Fuer mehr Einstellungmoeglichkeiten bei Aufzaehlungen
\usepackage{xifthen} % Erlaubt die Verwendung von if-then-else Befehlen im Code
\usepackage{index} % Index erzeugen
\newindex{default}{idx}{ind}{Stichwortverzeichnis}
\usepackage{xspace} % intelligende Leerzeichen bei Macros
\usepackage[normalem]{ulem} % unterstreichen von Text
\usepackage{cancel} % schraeg durchstreichen von Text
\usepackage{units} % schoenere Schreibweise fuer Einheiten mit Bruechen, laedt auch das nicefrac Paket

\renewcommand{\CancelColor}{\color{gray}} % Farbe zum schraegen Druchstreichen in grau

\definecolor{rltred}{rgb}{0.75,0,0}
\definecolor{rltgreen}{rgb}{0,0.5,0}
\definecolor{rltblue}{rgb}{0,0,0.75}

%sichere Fraben, die sich auch bei einem SW-Druck unterscheiden lassen (Platzhalter momentan)
\definecolor{color1}{cmyk}{1,0,0,0} %cyan
\definecolor{color2}{rgb}{0,1,0} %green

\usepackage[hyperindex=true]{hyperref} % Verweise als Hyperlinks
\hypersetup{
	pdftitle={Angewandte Differentialgeometrie Dr. Grensing},
	pdfsubject={Angewandte Differentialgeometrie Geometrie},
	pdfkeywords={Angewandte Differentialgeometrie Grensing},
	pdfproducer={pdfLaTeX},
	pdfpagemode={UseOutlines},
	colorlinks=true,
	bookmarksopen=true,
	bookmarksnumbered=true,
	urlcolor=rltblue,
	filecolor=rltgreen,
	linkcolor=rltblue,
	backref=true,
	pagebackref=true,
	pdfpagemode=None,
	citecolor=rltblue
}

% vertausche die Theta, Phi, Rho und Epsilon mit ihren "var" Versionen
%\newcommand{\swapcmd}[2]{
%	\let\temp\#1
%	\left\#1\#2
%	\let\#2\temp
%}
\let\temp\phi
\let\phi\varphi
\let\varphi\temp

\let\temp\theta
\let\theta\vartheta
\let\vartheta\temp

\let\temp\epsilon
\let\epsilon\varepsilon
\let\varepsilon\temp

\let\temp\rho
\let\rho\varrho
\let\varrho\temp


%%
%% Fuer Zeichnungen in TikZ
%%

\usepackage{tikz}
\usepackage{float} % Floater Elemente
%\usepackage{transparent}
%\usepackage{wrapfig}
\usetikzlibrary{matrix,arrows,calc,intersections, through, positioning, patterns, decorations.text, decorations.pathmorphing, decorations.markings, decorations.pathreplacing}

% neue Befehle fuer haeufig benutzte TikZ Formen; erstes Argument steht fuer die Position, Zweites fuer die Groesse
\newcommand{\tikzrichtung}[3][1]{ % zeichnet eine rote Linie von einem Punkt in eine Richtung mit rotem Knoten am Ende
	\draw[red] #2 -- ($#2 + #1*#3$) circle(0.05);
}

\newcommand{\tikzgitter}[3][0.25]{ %Hilfsgitter, das optionale Argument steht fuer die kleine Maschenweite, die Grosse ist doppelt so gross
	\draw[step=#1,gray!15] #2 grid #3;
	\draw[step=2*#1,gray!30] #2 grid #3;
	\fill (0,0) circle(0.1); 
}

\newcommand{\tikzschnuller}[2][1]{
	% definiere die Knoten relativ zum ersten Knoten skaliert mit dem Faktor
	\coordinate (schnuller1) at #2; \coordinate (schnuller2) at ($(schnuller1)+#1*(-1.75,-0.75)$); \coordinate (schnuller3) at ($(schnuller1)+#1*(-2.5,-2.25)$); \coordinate (schnuller4) at ($(schnuller1)+#1*(0,-2)$); \coordinate (schnuller5) at ($(schnuller1)+#1*(1.75,-0.25)$);
    %\fill (schnuller1) circle (0.05) (schnuller2) circle (0.05) (schnuller3) circle (0.05) (schnuller4) circle (0.05) (schnuller5) circle (0.05);
    
    % die Richtungsvektoren der Bezier Tangenten fuer die einzelnen Knoten (der Erste und der letzte haben keine Tangente)
    \coordinate (ctrls1) at ($#1*(1.25,0.25)$); \coordinate (ctrls2) at ($-0.5*(ctrls1)$); \coordinate (ctrls4) at ($#1*(1,-1)$); \coordinate (ctrls3) at ($-0.5*(ctrls4)$); \coordinate (ctrls6) at ($#1*(1,1.5)$); \coordinate (ctrls5) at ($-0		.33*(ctrls6)$);
	% die eigentlichen Tangenten
    \coordinate (tang1) at ($(schnuller2)+(ctrls1)$); \coordinate (tang2) at ($(schnuller2)+(ctrls2)$); \coordinate (tang3) at ($(schnuller3)+(ctrls3)$); \coordinate (tang4) at ($(schnuller3)+(ctrls4)$); \coordinate (tang5) at ($(schnuller4)+(ctrls5)$); \coordinate (tang6) at ($(schnuller4)+(ctrls6)$);
    %\fill[red] (tang1) circle (0.05); \fill[red] (tang2) circle (0.05); \fill[red] (tang3) circle (0.05); \fill[red] (tang4) circle (0.05); \fill[red] (tang5) circle (0.05); \fill[red] (tang6) circle (0.05);
    %\draw[red] (tang1) -- (tang2); \draw[red] (tang3) -- (tang4); \draw[red] (tang5) -- (tang6);
	
	\draw (schnuller1) ..controls(schnuller1) and (tang1).. (schnuller2) ..controls(tang2) and (tang3).. (schnuller3) ..controls(tang4) and (tang5).. (schnuller4) ..controls(tang6) and (schnuller5).. (schnuller5);
	
	% zeichne nun das Loch in der Mitte
	\def\angle{20} % Rotationswinkel
	\coordinate (c) at ($#2+#1*(-1.25,-1.25)$); % Mittelpunkt der Ellipse die den unteren Bogen bildet
	\begin{scope}
		\clip[rotate=\angle] ($(c)-#1*(1,0.6)$) rectangle ($(c)+#1*(1,-0.1)$);
		\path[draw,rotate=\angle,name path=l] (c) ellipse(#1*1 and #1*0.5);
	\end{scope}
	\path[name path=u,rotate=\angle] ($(c)-#1*(0,0.5)$) ellipse(#1*0.75 and #1*0.5);
	\path[name intersections={of=u and l}];
	\begin{scope}
		\clip[rotate=\angle] (intersection-1) rectangle ($(intersection-2)+#1*(0,0.5)$);
		\draw[rotate=\angle] ($(c)-#1*(0,0.5)$) ellipse(#1*0.75 and #1*0.5);
	\end{scope}		
}

\newcommand{\tikzsegel}[2][1]{
	% definiere die Knoten relativ zum ersten Knoten skaliert mit dem Faktor
	\coordinate (segel1) at #2; \coordinate (segel2) at ($(segel1)+#1*(4,1.5)$); \coordinate (segel3) at ($(segel1)+#1*(2,-0.5)$);
	%\fill (segel1) circle (0.05) (segel2) circle (0.05) (segel3) circle (0.05);
	
	% die Richtungsvektoren der Bezier Tangenten fuer die einzelnen Knoten (der Erste und der letzte haben keine Tangente)
	\coordinate (ctrls1) at ($#1*(0.75,1.5)$); \coordinate (ctrls2) at ($#1*(-0.75,0.25)$); \coordinate (ctrls3) at ($#1*(-0.5,-0.25)$); \coordinate (ctrls4) at ($#1*(0.25,1)$); \coordinate (ctrls5) at ($#1*(-0.375,0.375)$); \coordinate (ctrls6) at ($#1*(0.75,0.125)$);
	% die eigentlichen Tangenten
	\coordinate (tang1) at ($(segel1)+(ctrls1)$); \coordinate (tang2) at ($(segel2)+(ctrls2)$); \coordinate (tang3) at ($(segel2)+(ctrls3)$); \coordinate (tang4) at ($(segel3)+(ctrls4)$); \coordinate (tang5) at ($(segel3)+(ctrls5)$); \coordinate (tang6) at ($(segel1)+(ctrls6)$);
%	\fill[red] (tang1) circle (0.05); \fill[red] (tang2) circle (0.05); \fill[red] (tang3) circle (0.05); \fill[red] (tang4) circle (0.05); \fill[red] (tang5) circle (0.05); \fill[red] (tang6) circle (0.05);
 %   \draw[red] (tang1) -- (segel1) -- (tang6); \draw[red] (tang2) -- (segel2) -- (tang3); \draw[red] (tang4) -- (segel3) -- (tang5);
	
	\draw (segel1) ..controls(tang1) and (tang2).. (segel2) ..controls(tang3) and (tang4).. (segel3) ..controls(tang5) and (tang6).. (segel1) --cycle;
}

\newcommand{\tikztorus}[2][1]{
%	\tikzgitter{(-6,-1)}{(6,5)}
	% zuerst die aeussere Ellipse
	\draw[] #2  ellipse (#1*2 and #1*1);
	
	% dann das Loch
	\begin{scope}
		\clip ($#2 - #1*(1, 0.5)$) rectangle ($#2 + #1*(1, 1)$);
		\path[draw,name path=gkreis] ($#2 + #1*(0,0.75)$) ellipse (#1*1.25 and #1*1);
	\end{scope}
	\path[name path=kkreis] ($#2 - #1*(0,0.5)$) ellipse (#1*1 and #1*0.75);
	\path[name intersections={of=gkreis and kkreis}];
	\begin{scope}
		\clip (intersection-1) rectangle ($(intersection-2)+(0,0.5)$);
		\draw ($#2 - #1*(0,0.5)$) ellipse (#1*1 and #1*0.75);
	\end{scope}
	
	% definiere Werte auf die wir in der restlichen Zeichnung zurueckgreifen koennen
	\def\torusbreite{#1*2}
	\def\torushoehe{#1*1}
	\def\torusdicke{#1*0.75}
	\coordinate (torusUntenLoch) at ($#2 - #1*(0,0.25)$);
	\coordinate (torusUnten) at ($#2 - #1*(0,1)$);
}

\usepackage[toc]{glossaries} % Symbolverzeichnis
\glossarystyle{treehypergroup}
\makeglossaries

% Mathe Pakete
\usepackage{amsmath}
\usepackage{amssymb}
\usepackage{wasysym} % noch mehr symbole
\usepackage{stmaryrd}
\usepackage{bm} % fette Mathe Zeichen
\usepackage{mathtools} % zusaetzliche Mathe Befehle (zusaetzlich zu AMS Math)
\usepackage[hyperref,amsmath,thmmarks,thref]{ntheorem}

% --- Mathe Symbole ---

% canonic sets
\DeclareMathOperator{\B}{\mathbb{B}} % unit ball
\DeclareMathOperator{\C}{\mathbb{C}}
\DeclareMathOperator{\F}{\mathbb{F}}
\DeclareMathOperator{\K}{\mathbb{K}}
\DeclareMathOperator{\N}{\mathbb{N}}
\DeclareMathOperator{\Q}{\mathbb{Q}}
\DeclareMathOperator{\R}{\mathbb{R}}
\DeclareMathOperator{\RP}{\mathbb{RP}} % real projection plane
\DeclareMathOperator{\Tor}{\mathbb{T}} % torus
\DeclareMathOperator{\Z}{\mathbb{Z}}

% Redeclare \P (Prim or Propability) and put the old, reversed "breakline P" in \BreakLineP
\let\BreakLineP\P
\renewcommand{\P}{\ensuremath{\mathbb{P}}}
% das ungarische H umdefinieren
\let\umlautsH\H % long Hungarian umlaut (double acute)
\renewcommand{\H}{\ensuremath{\mathbb{H}}}
% das Paragraphenzeichen umdefinieren
\let\ParagraphS\S
\renewcommand{\S}{\mathbb{S}}


%  geschwungene Buchstaben
\DeclareMathOperator{\calB}{\mathcal{B}}
\DeclareMathOperator{\calD}{\mathcal{D}}
\DeclareMathOperator{\calE}{\mathcal{E}}
\DeclareMathOperator{\calG}{\mathcal{G}}
\DeclareMathOperator{\calI}{\mathcal{I}}
\DeclareMathOperator{\calJ}{\mathcal{J}}
\DeclareMathOperator{\calL}{\mathcal{L}}
\DeclareMathOperator{\calT}{\mathcal{T}}
\DeclareMathOperator{\calV}{\mathcal{V}}

\newcommand{\E}{\calE}
\newcommand{\G}{\calG}
\newcommand{\V}{\calV}

% common mathematical operators and sets
\DeclareMathOperator{\aff}{aff} % affine Huelle
\DeclareMathOperator{\const}{const} % allgemeiner Cosinus
\DeclareMathOperator{\conv}{conv} % konvexe Huelle
\DeclareMathOperator{\convo}{\stackrel{\circ}{\conv}} % konvexe Huelle
\DeclareMathOperator{\cs}{cs} % allgemeiner Cosinus
\DeclareMathOperator{\ct}{ct} % allgemeiner Cotangens
\DeclareMathOperator{\diam}{diam} % diameter
\DeclareMathOperator{\dist}{dist} % distance
\DeclareMathOperator{\dR}{dR} % deRahm
\DeclareMathOperator{\eukl}{eukl} % euklidisch
\DeclareMathOperator{\ggT}{ggT} % goesster gemeinsamer Teiler
\DeclareMathOperator{\id}{id} % identity
\DeclareMathOperator{\inh}{inh} % Inhalt
\DeclareMathOperator{\grad}{grad} % Gradient
\DeclareMathOperator{\kgV}{kgV} % kleinstes gemeinsames Vielfaches
\DeclareMathOperator{\lk}{lk} % Link
\DeclareMathOperator{\mmod}{mod} % modulo
\DeclareMathOperator{\mspan}{span} % Lineare Huelle
\DeclareMathOperator{\n}{n} % Umlaufzahl
\DeclareMathOperator{\offen}{offen}
\DeclareMathOperator{\pr}{pr}
\DeclareMathOperator{\proj}{proj}
\DeclareMathOperator{\res}{res} % Residuum
\DeclareMathOperator{\rg}{rg} % rank (i)
\DeclareMathOperator{\ric}{ric} % Ricci Tensor
\DeclareMathOperator{\scal}{scal} % Skalarkruemmung
\DeclareMathOperator{\sgn}{sgn} % Signum
\DeclareMathOperator{\sn}{sn} % allgemeiner Sinus
\DeclareMathOperator{\spur}{spur} % Spur
\DeclareMathOperator{\supp}{supp} % support
\DeclareMathOperator{\sternf}{sternf}
\DeclareMathOperator{\tr}{tr} % Spur

\DeclareMathOperator{\Abb}{Abb} % maps
\DeclareMathOperator{\Aut}{Aut} % automorphisms
\DeclareMathOperator{\Bild}{Bild}
\DeclareMathOperator{\CAT}{CAT}
\DeclareMathOperator{\Charakteristik}{char}
\DeclareMathOperator{\Charakt}{char}
\DeclareMathOperator{\D}{D} % Jacobi matrix or derivative
\DeclareMathOperator{\Diff}{Diff}
\DeclareMathOperator{\End}{End} % endomorphisms
\DeclareMathOperator{\Gl}{GL} % general linear group
\DeclareMathOperator{\GL}{GL} % general linear group
\DeclareMathOperator{\Gr}{Gr}
\DeclareMathOperator{\Graph}{Graph}
\DeclareMathOperator{\Hh}{H} % Hessesche
\DeclareMathOperator{\Hom}{Hom} % homomorphisms
\DeclareMathOperator{\Id}{id} % identity
\DeclareMathOperator{\Ind}{Ind} % Index
\DeclareMathOperator{\Inn}{Inn} % Untergruppe der inneren Automorphismen
\DeclareMathOperator{\Iso}{Iso}
\DeclareMathOperator{\Kern}{Kern}
\DeclareMathOperator{\Oo}{O} % Matrizen sie mit ihrer Transponierten multipiziert die Einheitsmatrix ergeben
\DeclareMathOperator{\Relation}{\scriptstyle\mathrm{R}} % custom Relation
\DeclareMathOperator{\Ric}{Ric} % Ricci Tensor field
\DeclareMathOperator{\Rang}{Rang} % rank (ii)
\DeclareMathOperator{\SL}{SL} % Matrizen mit Deteminante 1
\DeclareMathOperator{\Stab}{Stab} % Stabilisator
\DeclareMathOperator{\Sym}{Sym} % symmetric group
\DeclareMathOperator{\T}{T} % tangent bundle

\newcommand{\Zentrum}[1]{\ensuremath{\mathrm Z(#1)}} % Zentrum einer Gruppe
\newcommand{\Ordnung}[1][]{ % Ordnung einer Gruppe
  \ifthenelse{\isempty{#1}}{
    \#
  }{
    \left|#1\right|
  }
}

% X als Malzeichen und andere Operatorzeichen
\newcommand{\dop}{\mathrm{d}}
\newcommand{\X}{\times}
\newcommand{\tensor}{\otimes}
\newcommand{\ii}{{\mathrm i}}
\newcommand{\normVek}[1]{\frac{#1}{\|#1\|}} % Bruch mit Eintrag im Zahler und dessen Norm im Nenner, gedacht fuer normierte Vektoren

%Realteil und Imaginaerteil
\renewcommand{\Re}{\ensuremath{\operatorname{Re}}}
\renewcommand{\Im}{\ensuremath{\operatorname{Im}}}

% stellt einen gro"sen vertikalen Strich an einen Term, nuetzlich in Bruechen
\newcommand{\bigvert}[1]{\left. #1 \right|}

% Differentialoperatoren als Br"uche
\newcommand{\difffrac}[3][]{\ifthenelse{\isempty{#1}}{\frac{\dop #2}{\dop #3}}{\left. \frac{\dop #2}{\dop #3} \right|_{#1}}}
\newcommand{\pdifffrac}[3][]{\ifthenelse{\isempty{#1}}{\frac{\partial #2}{\partial #3}}{\left. \frac{\partial #2}{\partial #3} \right|_{#1}}}

% ein schoener aussehender Faktorraum anstatt einfach nur A/B
\newcommand{\FakRaum}[2]{
	\raisebox{0.7ex}{\ensuremath{#1}}
	\ensuremath{\mkern-3mu}\big/\ensuremath{\mkern-3mu}
	\raisebox{-0.6ex}{\ensuremath{#2}}}
\newcommand{\smallFakRaum}[2]{
	\scriptsize{\raisebox{0.7ex}{\ensuremath{#1}}
	\ensuremath{\mkern-3mu}\ / \ensuremath{\mkern-3mu}
	\raisebox{-0.6ex}{\ensuremath{#2}}}}

% quotient space or group
\newcommand{\modulo}[1]{\ensuremath{/_{\displaystyle #1}}}

% declaring Index for group theory
\newcommand{\Index}[2]{\ensuremath{(#1 \SlimDdot #2)}}

% Woerter in Kapitaelchenschrift
\newcommand{\kapit}[1]{\textsc{#1}\xspace} % \textsc fueg keinen Abstand nach dem Wort ein, also baue ich einen neuen Befehl mit \xspace am Ende
\newcommand{\Greedy}{\kapit{Greedy}}
\newcommand{\Planes}{\kapit{Planes}}
\newcommand{\Spheres}{\kapit{Spheres}}
\newcommand{\RotSpheres}{\kapit{Rotating Spheres}}


% Theoremartige Umgebungen
\theoremstyle{break}
\newtheorem{Dfn}{Definition}[chapter]
\newtheorem{Satz}[Dfn]{Satz}
\newtheorem{Lemma}[Dfn]{Lemma}
\newtheorem{Kor}[Dfn]{Korollar}
\newtheorem{Prop}[Dfn]{Proposition}
\theorembodyfont{\normalfont}
\newtheorem{Bsp}[Dfn]{Beispiel}
\newtheorem{Bspe}[Dfn]{Beispiele}
\newtheorem{Bem}[Dfn]{Bemerkung}

\theoremstyle{nonumberbreak}
\newtheorem{dfn}{Definition}
\newtheorem{satz}{Satz}
\newtheorem{lemma}{Lemma}
\newtheorem{kor}{Korollar}
\newtheorem{prop}{Proposition}

\newtheorem{bsp}{Beispiel}
\newtheorem{bspe}{Beispiele}
\newtheorem{bem}{Bemerkung}

\theorembodyfont{\normalfont}
\theoremsymbol{\ensuremath{\Box}}
\newtheorem{bew}{Beweis}
\theoremsymbol{\ensuremath{(\Box)}}
\newtheorem{bewSkiz}{Beweisskizze}

\theoremsymbol{}
\theoremstyle{emptybreak}
\newtheorem{emptythm}{}% druckt nur den optionalen Namen aus

\theoremstyle{break}
\newtheorem{Aufg}{Aufgabe}
\newtheorem{Loes}{L\"osung}

% changing enumerations
\setlist[enumerate]{label=(\arabic*), itemsep=0cm, leftmargin=1cm}
\setlist[itemize]{itemsep=0cm}

% set line distances
\linespread{1.1}

% Add a ':' for mathmode with tiny whitespaces around
\newcommand{\SlimDdot}{\ensuremath{\mathrm{:}}}

% indexing support

% Print and index given text
% usage: \CmIndex{[(optionally put another text for the index in here)]}{(text to print and add to index)}
\newcommand{\CmIndex}[2][]{\ifthenelse{\isempty{#1}}{\index{#2}}{\index{#1}}#2}

% Highlight(bold) and index the given text
% usage: \CmMark[(optionally put another text for the index in here)]{(text to highlight and add to index)}
\newcommand{\CmMark}[2][]{\textbf{\CmIndex[#1]{#2}}}

\parindent0pt % der Einschub bei neuem Absatz

% Befehl fuer Anfuerungszeichen unten und oben
\newcommand{\quot}[1]{\textrm{\glqq}{#1}\textrm{\grqq}}

%========================================================================================================================
%	G L O S S A R
%========================================================================================================================


%========================================================================================================================
%	I N F O R M A T I O N E N	Z U M	S K R I P T
%========================================================================================================================

\subject{inoffizielles Skript}
\title{Angewandte Differentialgeometrie}
\subtitle{Gehalten von Dr. S. Grensing im Sommersemester 2013}
\author{getippt von Aleksandar Sandic\thanks{\href{mailto:aleksandar.sandic@student.kit.edu}{Aleksandar.Sandic@student.kit.edu}}}

%========================================================================================================================
%	I N H A L T
%========================================================================================================================
\begin{document}
\maketitle

% Inhaltsverzeichnis
\pdfbookmark[1]{Inhaltsverzeichnis}{contents}
\setlength\parskip{0.6pt}
\tableofcontents

\chapter*{Vorwort}
\addcontentsline{toc}{chapter}{Vorwort}

\section*{\"Uber dieses Skript}
Dies ist eine Mitschrieb der Vorlesung \quot{Angewandte Differentialgeometrie} von Dr. S. Grensing, gehalten im Sommersemester 2013 am Karlsruher Institut f"ur Technologie.
Dr. Grensing ist nicht f"ur den Inhalt verantwortlich und es besteht weder eine Garantie f"ur Vollst"andigkeit, noch Korrektheit der enthaltenen Aussagen.

\section*{Wer}
Das Skript wurde in von Aleksandar Sandic getippt.
Bei Anmerkungen bzw. beim Auffinden von Fehlern korrigierte bitte die entsprechende Stelle in Wiki oder schickt eine E-Mail an
\begin{center}
	\href{mailto:aleksandar.sandic@student.kit.edu}{aleksandar.sandic@student.kit.edu}
\end{center}

\section*{Wo}
Link zur Vorlesung: \url{http://www.math.kit.edu/iag5/lehre/angdifgeo2013s/de}\\
Link zum Skript: \url{http://mitschriebwiki.nomeata.de/AngDiffGeo2013.pdf}\\
Link zum Mitschiebwiki: \url{http://mitschriebwiki.nomeata.de/}

\section*{To Do}\begin{itemize}
\item die Vorlesung l"auft noch
\end{itemize}

%========================================================================================================================
%	V O R L E S U N G
%========================================================================================================================

%-------------------------- Kapitel 1 --------------------------
% Vorlesung vom 16. 4.
\chapter{R\"auber und Gendarm}

\section{Metrische R\"aume mit oberen Kr\"ummungsschranken}

Es sei $(X, d)$ ein metrischer Raum. Ein Weg $c: \calI \to X$ hei"st \CmMark[Geod\"atische!minimierende]{(minimierende) Geod"atische}, wenn $d(c(t), c(t')) = (t - t')$ f"ur alle $t, t' \in \calI$ gilt.
Der Raum $X$ hei"st \CmMark[Raum!geod\"atischer]{geod"atischer Raum}, wenn f"ur alle $x, y \in X$ eine Geod"atische von $x$ nach $y$ existiert, beziehungsweise \CmMark[Raum!$R$-geod\"atischer]{$\bm{R}$-geod"atisch}, wenn dies f"ur $d(x, y) \le R$ gilt.
Ist $c: [a, b] \to X$ ein Weg so hei"st er \CmMark[Weg!rektifizierbarer]{rektifizierbar}, falls seine L"ange \marginnote{\begin{tikzpicture}[font=\scriptsize,scale=0.5,baseline=0]
%\tikzgitter{(-4,-1)}{(4,3)}
	\coordinate (1) at (-3,-0.5); \coordinate (2) at (3,2.5);
	\coordinate (ctrl1) at (1,2.5); \coordinate (ctrl2) at (-0.5,-2);
	\draw (1) ..controls($(1) + (ctrl1)$) and ($(2) + (ctrl2)$).. coordinate[pos=0.1](a) coordinate[pos=0.5](b) coordinate[pos=0.75](c) coordinate[pos=0.95](d) node[pos=0.65,below]{$c$} (2);
	\draw[dashed] (a) node[left]{$c(t_i)$} -- node[below]{$d_i$} (b) node[above]{$c(t_{i+1})$};
	\draw[decorate,decoration={brace}] (c) -- node[above left]{$d_n$} (d);
\end{tikzpicture}}
\begin{align*}
	\calL(c) = \sup \left\{ \sum d(c(t_i), c(t_{i+1})) \mid t_1 < \ldots t_n, t_i \in \calI \right\}
\end{align*}
endlich ist.
Es gilt $d(c(a), c(b)) \le \calL(c)$.
Die Kurvenl"ange ist invariant unter monotonen Reparametrisierungen.
Die L"angenfunktion $t \mapsto \calL(c|_{[a,t]})$ ist monoton wachsend; insbesondere besitzt jede Kurve eine Bogenl"angenparametrisierung.
Der Raum $(X, d)$ hei"st \CmMark[Raum!L\"angen-]{L"angenraum} (oder $d$ \CmMark[Metrik!innere]{innere Metrik}), falls die Metrik
\begin{align*}
	\overline{d}(x, y) = \inf \{ \calL(c) \mid c: \calI \to X \text{ rektif'barer Weg}, c(0) = x, c(1) = y\}
\end{align*}
mit $d$ "ubereinstimmt.

\begin{bspe}\begin{enumerate}[label=\arabic*), leftmargin=*]
\item
	$\R^n$ mit der euklidischen Metrik ist ein geod"atischer Raum und ein L"angenraum.
\item
	Jede Riemannsche Mannigfaltigkeit ist ein L"angenraum (Beweis zur "Ubung).
\item
	Ist $(X, d)$ ein metrischer Raum, so folgt dass $(X, \overline d)$ eine L"angenmetrik ist, das heißt $\overline{\overline{d}} = \overline{d}$ (das iterieren der Konstruktion liefert keine neue Metrik).
\item
	Jeder geod"atische Raum ist ein L"angenraum.
	Es gilt stets $d \le \overline{d}$.
	Sind $x, y \in X$ und ist $\gamma : [a, b] \to X$ eine Geod"atische von $x$ nach $y$, so gilt:
	\begin{align*}
		\calL(\gamma) &= \sup \left\{ \sum d(\gamma(t_i), \gamma(t_{i+1})) \right\} \\
		&= \sup \left\{ \sum (t_{i+1} - t_i) \right\} \\
		&= t - a = d(x, y). \shortintertext{Also}
		\overline{d}(x, y) &= \inf_c \calL(c) \le \calL(\gamma) = d(x, y)
	\end{align*}
\item
	$\R^2 \setminus \{0\}$ ist ein L"angenraum (bez"uglich der euklidischen Metrik), aber kein geod"atischer Raum.
	\begin{center}
		\begin{tikzpicture}[font=\scriptsize]
			%\tikzgitter{(-4,-1)}{(4,3)}
			\draw[->] (-1.5,0) -- (1.5,0);
			\draw[->] (0,-0.5) -- (0,1.5);
			\fill (-1,0) circle(0.05) node[below]{$-1$} (1,0) circle(0.05) node[below]{$1$};
			\draw (-1,0) to[out=20,in=160] (1,0);
		\filldraw[fill=white] (0,0) circle(0.05);
		\end{tikzpicture}\\
		Es gibt keine kürzeste Verbindung
	\end{center}
\end{enumerate}\end{bspe}

\begin{satz}[von Hopf-Rinow; Chom-Vossen 1935]
Es sei $X$ ein lokalkompakter L"angenraum. Dann sind die folgenden Aussagen "aquivalent:
\begin{enumerate}[label=(\roman*), widest=iii]
\item
	$X$ ist vollst"andig
\item
	$X$ ist \CmMark[vollst\"andig!geod\"atisch]{geod"atisch vollst"andig}, das hei"st jede Geod"atische $c: [0,1) \to X$ kann in $1$ fortgesetzt werden.
\item
	Beschr"ankte abgeschlossene Mengen sind kompakt.
\end{enumerate}
\end{satz}

Jede der obigen Aussagen impliziert, dass $X$ ein geod"atischer Raum ist.
\begin{center}\begin{tikzpicture}[font=\scriptsize]
	%\tikzgitter{(-4,-1)}{(4,3)}
	\coordinate (x) at (-2.5,-0.5); \coordinate (y) at (2.5,1.5);
	\foreach \i in {20,30,40,50} { \draw (x) to[relative,out=\i,in=180-\i] (y); }
	\fill (x) circle(0.05)node[below]{$x$} (y) circle(0.05)node[right]{$y$};
\end{tikzpicture}\\
Kanten $e_n$ der L"ange $1+\frac{1}{n}$ $\leadsto$ Es gibt keine Kurve der L"ange $1$\end{center}
Es bezeichne $M_\kappa^2$ die (eindeutigen) 2-dimensionalen, einfachzusammenh"angenden Riemannschen Mannigfaltigkeiten mit der Schnittkr"ummung $\sec \equiv \kappa$, und $D_\kappa$ ihren Durchmesser:
\begin{align*}
	M_\kappa^2 &= \begin{cases}
		\S_{\frac{1}{\sqrt{\kappa}}}^2 & ,\kappa > 0 \\
		\R^2 & ,\kappa = 0 \\
		\H_{\frac{1}{\sqrt{\kappa}}}^2 & ,\kappa < 0
	\end{cases} \\
	D_\kappa &= \begin{cases}
		\frac{\pi}{\sqrt{\kappa}} & ,\kappa > 0 \\
		\infty & ,\kappa \le 0
	\end{cases} \shortintertext{wobei}
	\H_{\frac{1}{\sqrt{\kappa}}}^2 &= \left\{ x \in \R^3 \mid x_1^2 + x_2^2 - x_3^2 = -\frac{1}{\kappa} \right\}.
\end{align*}
Es sei $X$ ein metrischer Raum. Ein \CmMark[Dreieck!geod"atisches]{geod"atisches Dreieck} $\Delta(x, y, z)$ besteht aus Geod"atischen $c_{xy} = \overline{xy}$, $c_{yz} = \overline{yz}$ und $c_{zx} = \overline{zx}$ zwischen den Punkten $x$, $y$ und $z$.
Ein Vergleichsdreieck in $M_\kappa^2$ ist ein geod"atisches Dreieck $\overline\Delta(\overline{x}, \overline{y}, \overline{z})$ mit gleichen Kantenl"angen wie das urspr"ungliche Dreieck.
Ein solches Vergleichsdrieck existiert eindeutig, falls der Umfang $d(x, y) + d(y, z) + d(z, x) \le 2 D_\kappa$ ist.
Ein Punkt $\overline p \in \overline\Delta(\overline{x}, \overline{y}, \overline{z})$ hei"st \CmMark[Punkt!Vergleichs-]{Vergleichspunkt} zu $p \in \Delta(x, y, z)$, $p \in \overline{xy}$, falls $\overline{d}(\overline{x}, \overline{p}) = d(x, p)$.
Das Dreieck $\Delta(x, y, z)$ erf"ullt die \CmMark[CAT(k)-Ungleichung@$\CAT(\kappa)$-Ungleichung]{$\bm{\CAT(\kappa)}$-Ungleichung}, wenn f"ur alle $p, q \in \Delta(x, y, z)$ mit vergleichpunkten $\overline{p}, \overline{q}$ gilt:
\begin{align*}
	d(p, q) \le \overline{d}(\overline{p}, \overline{q})
\end{align*}
\begin{center}\begin{tikzpicture}[font=\scriptsize]
	%\tikzgitter{(-4,-1)}{(4,3)}
	\coordinate (yShift) at (3,0.25); \coordinate (zShift) at (2,2);
	\coordinate (x) at (-3.5,0); \coordinate (y) at ($(x)+(yShift)$); \coordinate (z) at ($(x)+(zShift)$);
	\coordinate (x') at ($(x)+(4.5,0)$); \coordinate (y') at ($(x')+(yShift)$); \coordinate (z') at ($(x')+(zShift)$);
	\draw (x) to[out=20,in=170] coordinate[pos=0.5] (p) (y) to[out=135,in=270] (z) to[out=240,in=35] coordinate[pos=0.5] (q) (x) -- cycle;
	\fill (x) circle(0.05)node[below]{$x$} (y) circle(0.05)node[right]{$y$} (z) circle(0.05)node[above]{$z$};
	\fill (p) circle(0.05) (q) circle(0.05);
	\draw[dashed] (p) node[below]{$p$} -- (q) node[above]{$q$};
	\node at ($(x)+(3,1.5)$){$\subseteq X$};
	
	\draw (x') --coordinate[pos=0.5] (p') (y') -- (z') --coordinate[pos=0.5] (q') (x') -- cycle;
	\fill (x') circle(0.05)node[below]{$\overline x$} (y') circle(0.05)node[right]{$\overline y$} (z') circle(0.05)node[above]{$\overline z$};
	\fill (p') circle(0.05) (q') circle(0.05);
	\draw[dashed] (p') node[below]{$\overline p$} -- (q') node[above]{$\overline q$};
	\node at ($(x')+(3,1.5)$){$\subseteq \R^2$};
\end{tikzpicture}\end{center}
Der Raum $X$ besitzt die obere Kr"ummungsschranke $\kappa$, wenn f"ur alle $x \in X$ eine geod"atische Umgebung existiert, in der alle Dreiecke die $\CAT(\kappa)$-Ungleichung erfüllen. F"ur $\kappa = 0$ hei"st $X$ \CmMark[ger\"ummt!nicht-positiv]{nicht-positiv gekr"ummt}. Erf"ullt $X$ global die $\CAT(\kappa)$-Ungleichung, so hei"st $X$ \CmMark[Raum!$\CAT(\kappa)$-]{$\bm{\CAT(\kappa)}$-Raum}.

\begin{bspe}\begin{enumerate}[label=\arabic*)]
\item
	$\R^2$ ist $\CAT(0)$, $M^2 = \FakRaum{\R^2}{\Z^2}$ ist nichtpositiv gekr"ummt.
\item
	$\S^1$ ist $\CAT(1)$.
\item
	$\R^2 \setminus Q_1$, $Q_1 = \{ x_1 > 0, x_2 > 0 \}$, mit der induzierten L"angenmetrik ist $\CAT(0)$.
	\begin{center}\begin{tikzpicture}[font=\scriptsize]
		%\tikzgitter{(-4,-1)}{(4,3)}
		\coordinate (x) at (-2,-1); \coordinate (y) at (1.5,-0.5); \coordinate (z) at (-0.5,1);
		
		\draw (x) -- (y) -- (0,0) -- (z) -- (x) -- cycle;
		\fill (x) circle(0.05)node[below]{$x$} (y) circle(0.05)node[below]{$y$} (z) circle(0.05)node[above]{$z$};z
		
		\fill[fill=gray!30,fill opacity=0.5] (0,0) rectangle(2.4,1.4);
		\node at (1,0.75) {$Q_1$};
		\draw[->] (-2.5,0) -- (2.5,0);\draw[->] (0,-1.5) -- (0,1.5);
	\end{tikzpicture}\end{center}
\item
	$(M, g)$ Riemannsche Manngifaltigkeit, Schnittkr"ummung $\sec_g \le \kappa$, vollst"andig. nach dem Satz von Topogonov hat $M$ die obere Kr"ummungsschranke $\kappa$.
\end{enumerate}\end{bspe}

\begin{emptythm}[Eigenschaften von $\CAT(\kappa)$-R"aumen]\begin{itemize}
\item
	Geod"atische der L"ange $< D_\kappa$ sind eindeutig.
\item
	B"alle vom Radius $< \frac{1}{2} D_\kappa$ sind konvex und zusammenziehbar.
\item
	$\CAT(0)$-R"aume sind zusammenziehbar.
\end{itemize}\end{emptythm}

\begin{emptythm}[Charakterisierung von $\CAT(\kappa)$-R"aumen]
F"ur jedes geod"atische Dreieck $\Delta(x, y, z)$ vom Umfang $< 2 D_\kappa$ und Punkte $p \in \overline{xy}$ und $q \in \overline{xz}$ gilt f"ur Vergleichsdreiecke $\overline\Delta(\overline{x}, \overline{y}, \overline{z})$ und $\overline\Delta(\overline{x}, \overline{p}, \overline{q})$:
\begin{align*}
	\varangle_{\overline{x}} (\overline{p}, \overline{q}) \le \varangle_{\overline{x}} (\overline{y}, \overline{z}) \tag{in $M_\kappa^2$}
\end{align*}
\begin{center}\begin{tikzpicture}[font=\scriptsize,scale=1.5]
	%\tikzgitter{(-4,-1)}{(4,3)}
	\coordinate (yShift) at (3,0.25); \coordinate (zShift) at (2,2);
	\coordinate (x) at (-3.5,0); \coordinate (y) at ($(x)+(yShift)$); \coordinate (z) at ($(x)+(zShift)$);
	\coordinate (x') at ($(x)+(4.5,0)$); \coordinate (y') at ($(x')+(yShift)$); \coordinate (z') at ($(x')+(zShift)$);
	\draw (x) to[out=20,in=170] coordinate[pos=0.5] (p) (y) to[out=135,in=270] (z) to[out=240,in=35] coordinate[pos=0.5] (q) (x) -- cycle;
	\fill (x) circle(0.05)node[below]{$x$} (y) circle(0.05)node[right]{$y$} (z) circle(0.05)node[above]{$z$};
	\fill (p) circle(0.05) (q) circle(0.05);
	\draw (p) node[below]{$p$} -- (q) node[above]{$q$};
	
	\fill (x') circle(0.05)node[below]{$\overline x$} (y') circle(0.05)node[right]{$\overline y$} (z') circle(0.05)node[above]{$\overline z$};
	\coordinate (p') at ($(x')+(15:2)$); \coordinate (q') at ($(x')+(35:1.75)$);
	\fill (p') circle(0.05)node[right]{$\overline p$} (q') circle(0.05)node[above]{$\overline q$};
	\node at ($(x')+(3,1.5)$){$\Delta(\overline{x},\overline{y},\overline{z})$};
	\node at ($(x')+(22:2.3)$){$\overline{\Delta}(\overline{x},\overline{p},\overline{q},)$};

	\draw[clip] (x') -- (y') -- (z') -- (x') -- cycle;
	\draw (x') circle(1.25);
	\draw[clip] (x') -- (p') -- (q') -- (x') -- cycle;
	\draw (x') circle(1);
\end{tikzpicture}\end{center}
Ist $X$ ein $\CAT(\kappa)$-Raum und sind $c_1, c_2$ Geod"atische mit gleichem Startpunkt $c_1(0) = c_2(0) = x$ so ist der Winkel $\varangle_{\overline{x}} (\overline{c_1(s)}, \overline{c_2(t)})$ monoton wachsend in $s$ und $t$.
Damit ist der Winkel
\begin{align*}
	\varangle(c_1, c_2) = \lim_{\mathclap{s,t \to 0}} \varangle_{\overline{x}} (\overline{c_1(s)}, \overline{c_2(t)}) 
	= \lim_{\mathclap{t \to 0}} \varangle_{\overline{x}} (\overline{c_1(t)}, \overline{c_2(t)})
\end{align*}
wohldefiniert.
\end{emptythm}

\begin{satz}[von Hadamart-Cartan; Alexander-Bishop 1990]
Es sei $X$ ein vollst"andiger metrischer Raum mit der oberen Kr"ummungsschranke $\kappa \le 0$. Dann ist seine universelle "Uberlagerung $\tilde X$ (bez"uglich der induzierten Metrik) (global) $\CAT(\kappa)$.
\end{satz}

% Vorlesung vom 29. 4.

\section{Konvexit\"at}
Eine Teilmenge $C$ eines $\CAT(0)$-Raumes (geod"atischer Raum) hei"st \CmMark{konvex}, falls f"ur alle $x, y \in C$ das geod"atische Segment von $x$ nach $y$ existiert und in ganz $C$ verl"auft.
Eine Abbildung $f: X \to \R$ auf einem $\CAT(0)$-Raum hei"st \CmMark{konvex}, falls f"ur jede Geod"atische $c: [0,1] \to X$ die Funktion $f \circ c: [0,1] \to \R$ konvex im gew"ohnlichen Sinne ist.

Ist $X$ ein $\CAT(0)$-Raum, so ist die Metrik konvex, das hei"st f"ur Geod"atische $c_1,c_2: [0,1] \to X$ ist die Funktion
\begin{align*}
	t \mapsto d(c_1(t), c_2(t))
\end{align*}
konvex, wie man in der folgenden Beweisskizze erkennt:

\begin{bewSkiz}
Wir betrachten ohne Einschr"ankung den Fall dass beide Geod"atischen den gleichen Startpunkt $x = c_1(0) = c_2(0)$ haben:
\begin{center}\begin{tikzpicture}[font=\scriptsize,scale=1.5]
	%\tikzgitter{(-4,-1)}{(4,3)}
	\coordinate (yShift) at (3,0.25); \coordinate (zShift) at (2,2);
	\coordinate (x) at (-3.5,0); \coordinate (y) at ($(x)+(yShift)$); \coordinate (z) at ($(x)+(zShift)$);
	\coordinate (x') at ($(x)+(4.5,0)$); \coordinate (y') at ($(x')+(yShift)$); \coordinate (z') at ($(x')+(zShift)$);
	\draw (x) to[out=20,in=170] coordinate[pos=0.5] (p) (y) to[out=135,in=270] (z) to[out=240,in=35] coordinate[pos=0.5] (q) (x) -- cycle;
	\fill (x) circle(0.05)node[below]{$x = c_1(0) = c_2(0)$} (y) circle(0.05)node[below]{$c_1(1)=y$} (z) circle(0.05)node[above]{$c_2(1)=z$};
	\fill (p) circle(0.05) node[below]{$c_1(t)$} (q) circle(0.05) node[above left]{$c_2(t)$};
	\draw[dashed] (p) -- (q);
	\node at ($(y)+(0,1)$) {$\subseteq X$};
	
	\draw (x') --coordinate[pos=0.5] (p') (y') -- (z') --coordinate[pos=0.5] (q') (x') -- cycle;
	\fill (x') circle(0.05)node[below]{$\overline x$} (y') circle(0.05)node[below]{$\overline y$} (z') circle(0.05)node[above]{$\overline z$};
	\fill (p') circle(0.05)node[below]{$\overline{c_1(t)}$} (q') circle(0.05)node[above left]{$\overline{c_2(t)}$};
	\draw[dashed] (p') -- (q');
	\node at ($(y')+(0,1)$) {$\subseteq \R^2$};
\end{tikzpicture}\end{center}
Bei Betrachtung des Vergleichsdreiecks zu $\Delta(c_1(0), c_1(1), c_2(1))$ erhalten wir aus der $\CAT(0)$-Ungleichung:
\begin{align*}
	d(c_1(t), c_2(t)) \le d_{\R^2} (\overline{c_1(t)}, \overline{c_2(t)}) 
	= t \cdot d_{\R^2} (\overline y, \overline z) 
	= t \cdot d_{\R^2} (c_1(1), c_2(1))
\end{align*}
Betrachte nun $c$ von $c_2(0)$ nach $c_1(1)$.
\begin{center}\begin{tikzpicture}[font=\scriptsize,scale=1]
	%\tikzgitter{(-4,-4)}{(4,4)}
	\coordinate (1) at (-2,1); \coordinate (2) at (2,1.25);
	\coordinate (3) at (-2,-0.5); \coordinate (4) at (2,-1.5);
	
	\draw (1) to[out=0,in=190] coordinate[pos=0.3] (5) (2);
	\draw (3) to[out=0,in=150] coordinate[pos=0.5] (6) (4);
	\draw (1) to[out=340,in=140] coordinate[pos=0.2] (7) node[pos=0.6][above]{$c$} (4);
	
	\fill (1) circle(0.05)node[above]{$c_2(0)$} (2) circle(0.05)node[above]{$c_2(1)$} (3) circle(0.05)node[below]{$c_1(0)$} (4) circle(0.05)node[below]{$c_1(1)$};
	\fill (5) circle(0.05)node[above]{$c_2(t)$} (6) circle(0.05)node[below]{$c_1(t)$} (7) circle(0.05)node[below left]{$c(t)$};
	
	\draw (5) -- (6);
	\draw[dashed] (5) --(7) -- (6);	
\end{tikzpicture}\end{center}
Nach der Dreiecksungleichung gilt:
\begin{align*}
	d(c_1(t), c_2(t)) &\le d(c_1(t), c(t)) + d(c_2(t), c(t)) \\
	&\le (1-t) \cdot d(c_1(0), c_2(0)) + t \cdot d(c_1(1), c_2(1))
\end{align*}
\end{bewSkiz}

Geod"atische Segmente in X sind immer eindeutig, wie man mit der folgenden Skizze erkennen kann:
\begin{center}\begin{tikzpicture}
[font=\scriptsize,scale=1]
	%\tikzgitter{(-4,-4)}{(4,4)}
	\coordinate (x) at (-1.5,-0.5); \coordinate (y) at (2,0.5);
	\draw (x) to[out=45,in=160] node[above]{$c_1$} (y);
	\draw (x) to[out=350,in=230] node[below]{$c_2$} (y);
	
	\fill (x) circle(0.05) node[left]{$x$} (y) circle(0.05) node[right]{$y$};
\end{tikzpicture}\end{center}
Die Abstandsfunktion $d(c_1(\cdot), c_2(\cdot))$ ist konvex mit Nullstellen in $0$ und $1$, also konstant Null und damit $c_1 = c_2$.
Nun fixiere $x_0 \in X$, und sei $c_x$ das geod"atische Segment von $x_0$ nach $x$.
\begin{center}\begin{tikzpicture}[font=\scriptsize,scale=1]
	%\tikzgitter{(-4,-4)}{(4,4)}
	\coordinate (x0) at (-1.5,-0.5); \coordinate (x) at (2.5,1); \coordinate (y) at (1.5,-1);
	\draw (x) to[out=190,in=30] node[above]{$c_x$} coordinate[pos=0.3] (cxt) (x0) to[out=5,in=150] node[below]{$c_y$} (y);
	\fill (x0) circle(0.05) node[left]{$x_0$} (x) circle(0.05) node[right]{$x$} (y) circle(0.05) node[right]{$y$} (cxt) circle(0.05) node[below]{$c_x(t)$};
\end{tikzpicture}\end{center}
Damit wird $H(x,t) = c_x(t)$ eine Retraktion von $X$ auf $\{x_0\}$.

Ist $C$ eine vollst"andige konvexe Teilmenge  von $X$, so existiert eine \quot{orthogonale Projektion} $\pi: X \to C$ mit folgenden Eigenschaften:
\begin{enumerate}[label=(\roman*),widest=iii]
\item
	F"ur jedes $x \in X$ ist $\pi(x) \in X$ der eindeutig bestimmte Punkt mit
	\begin{align*}
		d(x, \pi(x)) = d(x, C) = \inf \{d(x, c) \mid c \in C\}
	\end{align*}
\item
	F"ur Punkte $y \in C$ und $x \notin C$, mit $\pi(x) \ne y$, gilt $\varangle_{\pi(x)}(x, y) \ge \frac{\pi}{2}$.
\item
	Es gilt $\pi(\overline{x \pi(x)}) = \{\pi(x)\}$.
\item
	$\pi$ ist eine Retraktion von $X$ auf $C$, welche Abst"ande nicht vergr"o"sert.
\end{enumerate}
\begin{center}\begin{tikzpicture}[font=\scriptsize,scale=1.25]
	%\tikzgitter{(-1,-4)}{(5,4)}
	\coordinate (x) at (2.5,0.5); \coordinate (y) at (-0.25,0.75); \coordinate (z) at (3,-1.25);
	
	\draw (0,-1.5) to[out=60,in=300] coordinate[pos=0.25](pz) coordinate[pos=0.6](px) (0,1.5);
	\node at (-0.25, 1.25) {$C$};
	\foreach \i in {0.05, 0.1, ...,1}{
		\path (0,-1.5) to[out=60,in=300] coordinate[pos=\i] (pkt) (0,1.5);
		\draw[opacity=0.5] (pkt) -- +(225:0.2);
	}
	
	\draw (z) -- (x) --coordinate[pos=0.3](x') (px) -- (pz) (px) -- (y);
	\draw[dashed] (pz) -- (z) -- (px);
	
	\fill (x) circle(0.05)node[right]{$x$} (y) circle(0.05)node[below]{$y$} (z) circle(0.05)node[right]{$z$};
	\fill (x') circle(0.05)node[below]{$x'$}node[above]{$\pi(x')=\pi(x)$};
	\fill (px) circle(0.05)node[below left]{$\pi(x)$} (pz) circle(0.05)node[left]{$\pi(z)$};
	
	\begin{scope}
		\clip (x) -- (px) -- (y) -- cycle;
		\draw (px) circle(0.2);
	\end{scope}
	\begin{scope}
		\clip (x) -- (px) -- (pz) -- (z) -- cycle;
		\draw (px) circle(0.25);
		\draw (pz) circle(0.25);
	\end{scope}
	\node at (0.8,0) {$\ge\frac{\pi}{2}$};
	\node at (0.8,-0.6) {$\ge\frac{\pi}{2}$};
\end{tikzpicture}\end{center}

\begin{emptythm}[Konstruktion]
Es seien $X_1$ und $X_2$ zwei $\CAT(0)$-R"aume. Dann ist $X_1 \X X_2$ ebenfalls ein $\CAT(0)$-Raum bez"uglich der Produktmetrik. Die Geod"atischen von $X_1 \X X_2$ sind genau die Produkt $c_1 \X c_2$ von Geod"atischen $c_i$ in $X_i$.
\begin{center}\begin{tikzpicture}[font=\scriptsize,scale=0.8]
	%\tikzgitter{(-4,-4)}{(5,4)}
	
	\draw (-1.5,0) circle(2) (1.5,0) circle(2);
	\node at (-3.25,1.5) {$X_1$}; \node at (3.25,1.5) {$X_2$};
	\begin{scope}
		\clip (-1.5,0) circle(2);
		\fill[gray,opacity=0.3] (1.5,0) circle(2);
	\end{scope}
	
	\coordinate (x1) at (-2,1.5); \coordinate (x2) at (-1.5,-1); \coordinate (y) at (2.5,0.25);
	\fill (x1) circle(0.05)node[left]{$x_1$} (x2) circle(0.05)node[left]{$x_2$} (y) circle(0.05)node[right]{$y$};
	
	\draw (x1) -- (y);
	\draw (x1) ..controls(-1,2) and (-1,1.5).. (-0.25,1.25) ..controls(0.5,1) and (1.75,1.25).. (y);
	\draw (x1) ..controls(-1,1.25) and (-0.75,0.5).. (-0,0.25) ..controls(0.75,0) and (1,0.75).. (y);
	\draw (x1) ..controls(-1.5,0.5) and (0,1).. (-0.5,1.25) ..controls($0.9*(-1,1.5)$) and (-0.5,-0.25).. (y);
\end{tikzpicture}\end{center}
Weitere $\CAT(0)$-R"aume erh"alt man durch \quot{Verkleben} entlang konvexer Teilmengen.
Es sei $A$ ein vollst"andiger metrischer Raum und seien $\iota_i: A \hookrightarrow A_i \subseteq X_i$ Isometrien auf konvexen Teilmengen in $X_1$ beziehungsweise $X_2$.
Dann ist $X_1 \cup_A X_2 = \FakRaum{X_1 \dot{\cup} X_2}{\{\iota_1(a)=\iota_2(a) \in A\}}$ mit der Metrik
\begin{align*}
	d(x,y) = \begin{cases}
		d_{X_i}(x,y) & x,y \in X_i \\
		\inf_{a \in A} \{ d(x, \iota_i(a)) + d(\iota_j(a), y) \} & x \in X_i, y \in X_j, i \ne j
	\end{cases}
\end{align*}
ein $\CAT(0)$-Raum.
\begin{center}\begin{tikzpicture}[font=\scriptsize,scale=1]
	%\tikzgitter{(-4,-4)}{(5,4)}
	
	\coordinate (1) at (-1,1.5); \coordinate (2) at (1,-1.5);
	\draw[name path=mitte] (1) -- (2);
	\draw (1) arc(40:100:2) (1) arc(200:150:2);
	\draw (2) arc(380:320:2) (2) arc(210:280:2);
	
	\node at (-3,1.5) {$X_1$}; \node at (-0.25,3) {$X_2$};
	\node at (-1,1) {$A_1$}; \node at (-0.5,1.5) {$A_2$};
	
	\coordinate (x) at (-0.5,-2.5); \coordinate (y) at (3,-1.5); \coordinate (z) at (-1.5,-0.5);
	
	\draw (z) -- (x) -- (2) -- (y);
	\draw[name path=diagonale] (z) -- (y);
	\path[name intersections={of=mitte and diagonale}];
	
	\draw[dashed] (z) -- (2) (x) -- (intersection-1);
	
	\fill (x)circle(0.05)node[below]{$x$} (y)circle(0.05)node[right]{$y$} (z)circle(0.05)node[above]{$z$} (2)circle(0.05)node[above right]{$x'$} (intersection-1)circle(0.05)node[above right]{$z'$};
\end{tikzpicture}\\
Betrachte die Vergleichsdreiecke $\Delta(\overline x, \overline{x'}, \overline{z'})$ und $\Delta(\overline{z'}, \overline{y}, \overline{x'})$
\end{center}
\end{emptythm}

\section{R\"auber und Gendarm}

\begin{emptythm}[Regeln]
Das Spielfeld $D$ sei eine zusammenh"angende Teilmenge des $\R^n$.
Eine Startkonfiguration sei gegeben durch eine endliche Anzahl von Verfolgern $P_1, \ldots, P_N \in D$ und einen Flüchtigen $E \in D$.
Wir betrachten zun"achst (zeitlich) diskrete Modelle.
Es bezeichnen $P_k^t$ beziehungsweise $E^t$ die Positionen der Verfolger beziehungsweise des Fl"uchtigen zum Zeitpunkt $t \in \N$.
\end{emptythm}

\begin{emptythm}[Spielverlauf]
Zum Zeitpunkt $t$ w"ahlt zun"achst $E^t$ eine neue Position $E^{t+1}$ mit dem Abstand $d(E^t, E^{t+1}) \le 1$, also h"ochstens eine Einheit entfernt von der vorherigen Position, danach (simultan) die Verfolger $P_k^t$, entsprechend mit $d(P_k^t, P_k^{t+1}) \le 1$.
Die Verfolger gewinnen, wenn f"ur jedes $C > 1$ ein $t \in N$ existiert, so dass $d(E^t, P_k^t) < C$ f"ur ein $k \in \N$ gilt.
\end{emptythm}

\begin{satz}
Auf jedem Kompaktum $D \subseteq \R^n$ ist \quot{\index{Greedy@\Greedy}\Greedy} stets erfolgreich (mit mindestens einem Verfolger).
\end{satz}

\begin{emptythm}[\Greedy]
Der Verfolger bewegt sich um die Distanz $1$ auf der Strecke $\overline{P^t E^t}$:
\begin{center}\begin{tikzpicture}[font=\scriptsize,scale=1]
	%\tikzgitter{(-4,-4)}{(4,4)}
	\coordinate (1) at (-2.5,-0.5); \coordinate (2) at (2,1.5);
	\path (1) --coordinate[pos=0.35](3) (2);
	\fill (1) circle(0.05) node[left]{$P^t$} (2) circle(0.05) node[above right]{$E^t$} (3) circle(0.05)node[above]{$P^{t+1}$};
	\draw[->] (1) -- (3);
	\draw[dashed] (3) -- (2);
\end{tikzpicture}\end{center}
\end{emptythm}

\begin{bewSkiz}[allgemein sp"ater]
Skizze zum \Greedy Algorithmus:
\begin{center}\begin{tikzpicture}[font=\scriptsize,scale=1]
	%\tikzgitter{(-4,-4)}{(4,4)}
	\coordinate (1) at (-2.5,-0.5); \coordinate (2) at (2,1.5);
	\path (1) --coordinate[pos=0.35](3) (2);
	\coordinate (4) at (3.5, 0.5);
	\fill (1) circle(0.05) node[left]{$P^t$} (2) circle(0.05) node[above right]{$E^t$} (3) circle(0.05)node[above]{$P^{t+1}$} (4) circle(0.05) node[right]{$E^{t+1}$};
	\draw[->] (1) --node[above]{$1$} (3);
	\draw (3) -- (2) --node[above right]{$1$} (4) --node[below]{$d^{t+1} > 1 + \epsilon$} (3);
	\begin{scope}
		\clip (3) -- (2) -- (4) -- cycle;
		\draw (3) circle(1.5);
		\node at ($(3) + (15:1.75)$) {$\alpha^t$};
	\end{scope}
	\draw[decorate,decoration={brace}] ([yshift=12,xshift=-5]1) --node[above]{$d^t$} ([yshift=12,xshift=-5]2);
\end{tikzpicture}\\
$d^{t+1} \le 1 + (d^t - 1) = d^t \Rightarrow \lim_{t \to \infty} d^t =: d^\infty$
\end{center}
Angenommen der Fl"uchtige w"urde entkommen. Dann w"are $d^\infty > 1$ und damit der Winkel $\alpha^t \xrightarrow{t \to \infty} 0$, das bedeutet $D$ w"urde beliebig lange Geradensegmente enthalten, w"are also nicht kompakt.
Das ist ein Widerspruch zur Voraussetzung.
\end{bewSkiz}

\begin{emptythm}[Mehrere Verfolger in $\R^n$]
Wann ist ein Entkommen bei mehreren Verfolgern "uberhaupt m"oglich? Wir schauen uns die folgende Zeichnung an:
\begin{center}\begin{tikzpicture}[font=\scriptsize,scale=1]
	%\tikzgitter{(-4,-4)}{(5,4)}
	
	\coordinate (1) at (-1,0.5); \coordinate (2) at (-1,-1); \coordinate (3) at (-2,0);
	\coordinate (4) at (-2,-1.25); \coordinate (5) at (-2,1); \coordinate (6) at (-3.5,-0.25);
	\coordinate (e) at (0.1,0);
	
	\foreach \i in {1,2,...,6}{
		\fill (\i)circle(0.05)node[below]{$P_{\i}$};
		\draw[->] (\i) -- ($(\i)!1cm!(e)$);
	}
	\fill (e)circle(0.05)node[below]{$E$};
	\foreach \i in {0,1,2}{
		\draw[->] ($(e)+(\i,0)$) -- ++(1,0);
	}
	
	\foreach \i in {-0.5,0.25,1,1.75}{
		\draw[dashed,opacity=0.5] (\i,-2) -- (\i,2);
	}
	
	\node at (0.75,1) {Hyperebenen};
\end{tikzpicture}\end{center}
Sei $C^t = \conv(P_1^t, \ldots, P_N^t)$ die konvexe H"ulle zum Zeitpunkt $t$.
Ein Entkommen ist m"oglich, wenn es eine trennende Hyperebene gibt, also wenn $E \notin \convo(P_1^t, \ldots, P_N^t)$, wobei $\convo$ das Innere der konvexen H"ulle bezeichnet.
Es stellt sich die Frage, ob die Umkehrung auch zutrifft, also ob nie ein Entkommen m"oglich ist, wenn $E \in \convo(P_1, \ldots, P_N)$.
Bei \Greedy ist das nicht der Fall, der Beweis ist dem Leser zur "Ubung "uberlassen.
Wir weden uns nun also einen neuen Algorithmus anschauen.
\end{emptythm}

\begin{emptythm}[\Planes]
Die idee hinter dem \Planes\index{Planes@\Planes} Algorithmus ist den Bereich um die Verfolger durch Hyperebenen einzugrenzen und systematisch zu verkleinern.
\begin{center}\begin{tikzpicture}[font=\scriptsize,scale=1]
	%\tikzgitter{(-4,-4)}{(5,4)}
	
	\coordinate (e) at (0,0);
	\def\clipRadius{2}
	\fill (e)circle(0.05) node[below left]{$E^0$};
	\coordinate (1) at (1.25,-0.5); \coordinate (2) at (1.5,1.5); \coordinate (3) at (-0.5,1.25);
	\coordinate (4) at (-1.5,-0.25); \coordinate (5) at (-0.25,-1);
	
	\begin{scope}
		\clip (0,0.25) ellipse(2.3 and 2.2);
		\foreach \i in {1,2,3,4,5}{
			\fill (\i) circle(0.05);
			\draw[dashed] (\i) -- (e);
			\draw ($(\i)!\clipRadius!90:(e)$) -- ($(\i)!-\clipRadius!90:(e)$);
			\begin{scope}
				\clip (e) -- (\i) -- ($(\i)!\clipRadius!90:(e)$) -- (e);
				\draw (\i) circle(0.35);
				\fill ($(\i)!0.2cm!45:(e)$)circle(0.03);
			\end{scope}
		}
	\end{scope}
	\draw[dashed] (1)node[right]{$P_1$} -- (2)node[above right]{$P_2$} -- (3)node[above]{$P_3$} -- (4)node[left]{$P_4$} -- (5)node[below]{$P_5$} -- cycle;
\end{tikzpicture}\end{center}
\begin{itemize}
\item
	$P_k^{t+1} = P_k^{t+1}(P_k^t, E^t, E^{t+1})$
\item
	Gerade $\calL_k^{t+1}$ parallel zu $\overline{E^tP_k^t}$ durch $E^{t+1}$
	\begin{center}\begin{tikzpicture}[font=\scriptsize,scale=1]
		%\tikzgitter{(-4,-4)}{(5,4)}
		
		\def\angle{25}
		\draw[dashed] (-1.5,0.5) --coordinate[pos=0.2](p) coordinate[pos=0.8](e) +(\angle:3.5);
		\draw[dashed,name path=gerade] (-2,-0.75) --node[pos=0.2,below]{$\calL_k^{t+1}$} +(\angle:6);
		
		\draw ($(p)!-2cm!90:(e)$) -- ($(p)!0.75cm!90:(e)$);
		
		\fill (p) circle(0.05)node[left]{$P_k^t$} (e) circle(0.05)node[above]{$E^t$};
		\draw[->] ($(p)!0.3cm!90:(e)$) -- ($($(p)!0.3cm!90:(e)$)!0.5cm!90:(p)$);
		
		\path[name path=flucht] (e) -- ($(e)!2cm!90:(p)$);
		\path[name intersections={of=flucht and gerade}];
		\coordinate (e') at (intersection-1);
		
		\draw[->] (e) --node[right]{$r$} (e');
		
		\path[name path=kreis] (p) circle(1.25);
		\path[name intersections={of=kreis and gerade}];
		\coordinate (p') at (intersection-2);
		\fill (p') circle(0.05) node[right]{$P_k^{t+1}$};
		\draw[->] (p) --node[above]{$1$} (p');
		\draw ($(p')!-1cm!90:(e')$) -- ($(p')!1.75cm!90:(e')$);
	\end{tikzpicture}\end{center}
\item
	$P_k^{t+1} \in \calL_k^{t+1}$ mit $d(P_k^t, P_k^{t+1}) = 1$ und $d(_k^{t+1}, E^{t+1})$ minimal
\end{itemize}
\end{emptythm}

\begin{satz}[Kopperty-Ravishankar '05]
Es sei $D = \R^n$. Falls $E \in \convo(P_1, \ldots, P_N)$ gilt, so ist \Planes erfolgreich.
\end{satz}

\begin{bew}
Zun"achst stellen wir mit der folgenden Zeichnung einige Vor"uberlegungen an:
\begin{center}\begin{tikzpicture}[font=\scriptsize,scale=2]
	%\tikzgitter{(-4,-4)}{(5,4)}
	
	\draw (-2.25,0.5) --node[above]{$d^t$} (2,0.5);
	\draw (-2.25,-0.5) --node[below]{$d^{t+1}$} (2,-0.5);
	
	\coordinate (p) at (-1.75,0.5); \coordinate (e) at (1,0.5); \coordinate (e') at (1.5,-0.5);
	\coordinate (1) at (-1.75,-0.5); \coordinate (2) at (-1.25,-0.5); \coordinate (3) at (-0.5,-0.5);
	
	\fill (p) circle(0.025)node[above]{$P^t$} (e) circle(0.025)node[above]{$E^t$} (e') circle(0.025)node[below]{$E^{t+1}$};
	\fill (1) circle(0.025) (2) circle(0.025) (3) circle(0.025);
	
	\draw (e) --node[right]{$r$} (e') (p) --node[right]{$1$} (3) (p) --node[pos=0.7,right]{$r$} (2);
	\draw[dashed] ($(p)+(0,0.2)$) --node[left,pos=0.6]{$a = r \sin \theta$} ($(1)-(0,0.2)$);
	
	\draw[decorate,decoration={brace,mirror}] (1) --node[below]{$-r \cos \theta$} (2);
	\draw[decorate,decoration={brace,mirror}] (2) --node[below]{$d^t-d^{t+1}$} (3);
	\draw[decorate,decoration={brace,mirror}] ([yshift=-7]1) --node[below]{$b$} ([yshift=-7]3);
	
	\begin{scope}
		\clip(p) -- (2) -- (-2.25,-0.5) -- (-2.25,0.5) -- cycle;
		\draw (p) circle(0.25);
		\node at ($(p)+(225:0.3)$) {$\theta$};
	\end{scope}
	\begin{scope}
		\clip(e) -- (e') -- (3) -- (p) -- cycle;
		\draw (e) circle(0.25);
		\node at ($(e)+(225:0.3)$) {$\theta$};
	\end{scope}
\end{tikzpicture}\end{center}
Falls $\theta \le \frac{\pi}{2}$ gilt $r \cos \theta \ge 0$ und damit folgt $d^t - d^{t+1} \ge 0$. Im Fall $\theta > \frac{\pi}{2}$ gilt $r \cos \theta = -r \sqrt{1 - \sin^2 \theta}$ und damit folgt dann
\begin{align*}
	d^t - d^{t+1} = \sqrt{1 - r \sin^2 \theta} - r \sqrt{1 - \sin^2 \theta} \overset{r \le 1}{\ge} (1 - r) \sqrt{1 - \sin^2 \theta} \ge 0.
\end{align*}
Es bezeichne $v_k := \normVek{P_K^0 - E^0}$ den Einheitsvektor in Richtung $\overline{E^0P_k^0}$.
\marginnote{\begin{tikzpicture}[font=\scriptsize,scale=0.4]
	%\tikzgitter{(-4,-4)}{(4,4)}
	\coordinate (e) at (0,0);
	\fill (e) circle (0.15) node[below left]{$E$};
	\coordinate (z1) at (250:3); \coordinate (z2) at (310:2.5); \coordinate (z3) at (20:3.25); \coordinate (z4) at (110:2.5); \coordinate (z5) at (175:3);
	\foreach \i in {1,2,3,4,5}{
		\draw[->] (e) -- ($(e)!1.5cm!(z\i)$);
		\draw[dotted] ($(e)!1.5cm!(z\i)$) -- (z\i);
		\fill (z\i)circle(0.05);
	}
	\path (e) --node[above]{$v_k$} ($(e)!1.5cm!(z3)$) (z3)node[right]{$P_k$};
\end{tikzpicture}}
Da $E \in \convo(P_1, \ldots, P_N)$ gilt, existiert ein $\epsilon > 0$ mit der Sph"are $\B_\epsilon \subseteq \convo(v_1, \ldots, v_N)$.
Ohne Einschr"ankung kann man annehmen, dass die $v_1, \ldots, v_{n+1}$ in allgemeiner Lage sind und dass $\B_\epsilon(0) \subseteq \convo(v_1, \ldots, v_{n+1})$, sowie $E = 0$ ist.
F"ur ein $v \in \S^{n-1}$ existieren $\lambda_1, \ldots, \lambda_{n+1} \in (0,1)$ mit $\epsilon v = \sum_{k=1}^{n+1} \lambda_k v_k$ und es folgt
\begin{align*}
	\epsilon = \epsilon \|v\|^2 = \langle v, \epsilon v \rangle = \sum \lambda_k \langle v, v_k \rangle.
\end{align*}
man findet also ein $k$, so dass $\langle v, v_k \rangle \ge \frac{\epsilon}{n+1} = d_{\min} > 0$ gilt.
Es seien $v = \normVek{E^{t+1} - E^t}$ und $k = k(v)$ wie oben. Dann gilt $\cos \theta_k^t = \langle v, v_k \rangle \ge d_{\min}$.
\begin{center}\begin{tikzpicture}[font=\scriptsize,scale=1]
	%\tikzgitter{(-4,-4)}{(5,4)}
	
	\draw[->] (0,0) --coordinate[pos=0.7](1) node[above,pos=0.3]{$r$} +(20:3) node[below,pos=0.9]{$v$};
	\draw[->] (0,0) --coordinate[pos=1.5] (2) +(350:3) node[below,pos=0.9]{$v_k$};
	\draw[dashed] (350:3) -- (2);
	
	\fill (0,0) circle(0.05)node[left]{$E^t$} (1) circle(0.05)node[above]{$E^{t+1}$} (2) circle(0.05)node[right]{$P^t$};
\end{tikzpicture}\end{center}
Es gilt:
\begin{align*}
	d^t - d^{t+1} &= \sqrt{1 - r^2 \sin^2 \theta_k^t} + r \cos \theta_k^t \\
		&\ge \sqrt{1 - r^2} + r \cos \theta_k^t \\
		&= \sqrt{1 - r^2} - \cos \theta_k^t (1 - r) + \cos \theta_k^t \\
		&\ge \sqrt{1 - r} (\sqrt{1 + r} - \sqrt{1 - r}) + \cos \theta_k^t \\
		&\ge \langle v, v_k \rangle \ge d_{\min} > 0
	\shortintertext{und}
	d^{t+1} &= \sum_k d_k^{t+1} \le \left( \sum d_k^t \right) - d_{\min} \le \ldots \le \sum_k d_k^0 - t d_{\min}
\end{align*}
\end{bew}

\begin{emptythm}[Modifikation: \Spheres]\index{Spheres@\Spheres}
Wir betrachten das \textsc{Lion and Man}\index{Lion and Man@\textsc{Lion and Man}} Problem: ein Mensch versucht einem L"owen zu entkommen.
\begin{center}\begin{tikzpicture}[font=\scriptsize,scale=1]
	%\tikzgitter{(-4,-4)}{(5,4)}
	
	\draw (-0.5,0) -- (4,0);
	\draw (0,-0.5) -- (0,3);
	
	\coordinate (m) at (1,0.5); \coordinate (m') at ($(m)+(130:0.5)$); \coordinate (l) at (1.5,1);
	
	\fill (m)circle(0.05)node[below]{Man} (m')circle(0.05) (l)circle(0.05)node[right]{Lion};
	\draw[->] (m) -- (m');
	\path (m) --coordinate[pos=5](z) (l);
	\draw[dashed] (m) -- (z);
	\draw[dashed] (m') --coordinate[pos=0.1] (l') node[above]{$\calL^{t+1}$} (z);
	\draw[->] (l) -- (l');
	
	\begin{scope}
		\clip (0.5,2.5) rectangle (3,0);
		\node[draw] at (z) [circle through={(l)}] {};
	\end{scope}
\end{tikzpicture}\end{center}
Der L"owe w"ahlt einen Punkt, so dass er zwischen den Menschen und dem Punkt steht.
Als Abgrenzung dient im Gegensatz zu \Planes keine Hyperebene, sondern eine Sph"are durch den Punkt und die Position des L"owen.
\end{emptythm}

\begin{emptythm}[\Spheres]\begin{itemize}
\item
	\emph{Initialisierung:} W"ahle Punkte $p_k$ auf Geraden durch $E$ und $P_k$ so, dass $P_k \in \overline{p_k E}$ gilt und die Komponente von $\R^n \setminus \B_{d(p_k, P_k)}(p_k)$, welche $E$ enth"alt, beschrieben ist.
	\begin{center}\begin{tikzpicture}[font=\scriptsize,scale=1]
		%\tikzgitter{(-4,-4)}{(4,4)}
		
		\coordinate (e) at (0,0);
		\fill (e) circle(0.05) node[below]{$E$};
		\coordinate (z1) at (-1,-3); \coordinate (z2) at (2.5,-1.75); \coordinate (z3) at (3,1.5); \coordinate (z4) at (-0.25,3); \coordinate (z5) at (-3,0.25);
		
		\def\radius{2}
		\begin{scope}
		\clip (e) circle(2.25);
			\foreach \i in {1,2,3,4,5}{
				\draw[name path=kreis] (z\i) circle(\radius) (z\i) circle(0.1);
				\path[name path=direkt] (z\i) -- (e);
				\path[name intersections={of=kreis and direkt}];
				\coordinate (p\i) at (intersection-1);
				\draw[dashed] ($(p\i)!2cm!90:(e)$) -- ($(p\i)!-2cm!90:(e)$);
			}
		\end{scope}
		\fill (p1)circle(0.05)node[below]{$P_1$} (p2)circle(0.05)node[below right]{$P_2$} (p3)circle(0.05)node[right]{$P_3$} (p4)circle(0.05)node[above]{$P_4$} (p5)circle(0.05)node[left]{$P_5$};
		\draw[dashed] (e) -- (z3)node[right]{$p_3$};
	\end{tikzpicture}\end{center}
\item
	$P_k^{t+1} = P_k^{t+1}(p_k, P_k^t, E^{t+1})$
\item
	$P_k^{t+1} \in \overline{p_k E^{t+1}}$ mit $d(P_k^t, P_k^{t+1}) = 1$ und $d(P_k^{t+1}, E^{t+1})$ minimal.
	\begin{center}\begin{tikzpicture}[font=\scriptsize,scale=1]
		%\tikzgitter{(-4,-4)}{(4,4)}
		
		\draw (0,0) --coordinate[pos=0.5](p) coordinate[pos=0.9](e) (3,0.25);
		\draw[dashed] (0,0) --coordinate[pos=0.65](p') coordinate[pos=0.85](e') (3,1);
		
		\draw[->] (p)node[below]{$P_k^t$} --node[right]{$1$} (p')node[above left]{$P_k^{t+1}$};
		\draw[->] (e)node[below]{$E^t$} --node[right]{$r$} (e')node[above]{$E^{t+1}$};
		
		\fill (p)circle(0.05) (e)circle(0.05);
	\end{tikzpicture}\end{center}
\end{itemize}\end{emptythm}

\begin{satz}
Es sei $D= \R^n$. Falls $E \in \convo(P_1, \ldots, P_N)$ gilt, so ist \Spheres erfolgreich.
\end{satz}

\begin{bewSkiz}
Wir betrachten in dem folgenden Bild den Unterschied zwischen \Planes und \Spheres
\begin{center}\begin{tikzpicture}[font=\scriptsize,scale=1.3]
	%\tikzgitter{(-4,-4)}{(5,4)}
	
	\coordinate (p) at (-3,0); \coordinate (e) at ($(p)+(25:3.5)$); \coordinate (e') at ($(e)+(305:0.75)$);
	
	\draw (p)node[left]{$p_k$} --coordinate[pos=0.5](P) (e)node[above]{$E^{t}$};
	\draw[->] (e) --node[right]{$r$} (e')node[below right]{$E^{t+1}$};
	\draw (p) --coordinate[pos=0.5](P') ($(e')!-1cm!(p)$);
	\draw[dashed] ($(e')+(25:1)$) --coordinate[pos=0.5](Q) ($(e')+(180+25:4)$);
	\draw (P) -- (P') (P) -- (Q);
	
	\fill (p)circle(0.05) (e)circle(0.05) (P)circle(0.05)node[above]{$P_k^t$} (P')circle(0.05)node[above]{$P_k^t$} (Q)circle(0.05)node[below right]{$Q$ (Wahl nach \Planes)};
	
	\coordinate (p) at (3.5,0); \coordinate (e) at ($(p)+(40:3)$); \coordinate (p') at ($(p)+(5:1.5)$); \coordinate (Q) at ($(p)+(355:1.75)$);
	
	\draw (p)node[left]{$P_k^t$} -- (e)node[above]{$E^{t+1}$} -- (p')node[above]{$P_k^{t+1}$} -- cycle (p) -- (Q)node[below]{$Q$} -- (e) -- cycle;
	\fill (p) circle(0.05) (p') circle(0.05) (e) circle(0.05) (Q) circle(0.05);
	\begin{scope}
		\clip (p) -- (e) -- (Q) -- cycle;
		\draw (p) circle(0.7);
		\node at ($(p)+(25:0.8)$) {$\alpha$};
	\end{scope}
	\begin{scope}
		\clip (p) -- (e) -- (p') -- cycle;
		\draw (p) circle(0.5);
	\end{scope}
	\node[right] (beta) at ($(p)+(0,-0.5)$) {$\beta = \varangle_{P_k^t(Q, E)}$};
	\draw[->] (beta) -- ($(p)+(25:0.3)$);
\end{tikzpicture} \\
$\Rightarrow d(D^{t+1}, E^{t+1}) < d(Q, E^{t+1})$\end{center}
Sei $R^t = \conv(p_1, \ldots, p_N) \setminus \bigcup_{k=1}^N \B_{d(p_k, P_k^t)}(p_k)$ der zum Zeitpunkt $t$ von den Sph"aren eingeschlossene Bereich.
\begin{center}\begin{tikzpicture}[font=\scriptsize,scale=1]
	%\tikzgitter{(-4,-4)}{(4,4)}
	
	\node (e) at (0,0) {$R^t$};
	\coordinate (z1) at (-1,-3); \coordinate (z2) at (2.5,-1.75); \coordinate (z3) at (3,1.5); \coordinate (z4) at (-0.25,3); \coordinate (z5) at (-3,0.25);
	
	\def\radius{2}
	\begin{scope}
	\clip (e) circle(2.25);
		\foreach \i in {1,2,3,4,5}{
			\draw[name path=kreis] (z\i) circle(\radius) (z\i) circle(0.1);
			\path[name path=direkt] (z\i) -- (e);
			\path[name intersections={of=kreis and direkt}];
			\coordinate (p\i) at (intersection-1);
		}
	\end{scope}
	\fill (p1)circle(0.05)node[below]{$P_1^t$} (p2)circle(0.05)node[below right]{$P_2^t$} (p3)circle(0.05)node[right]{$P_3^t$} (p4)circle(0.05)node[above]{$P_4^t$} (p5)circle(0.05)node[left]{$P_5^t$};
\end{tikzpicture}\end{center}
Wie f"ur \Planes zeigt man: Es existiert ein $\epsilon > 0$, so dass f"ur alle Punkte $x \in R^t$ und Vektoren $v \in \S^{n-1}$ ein $k \le N$ und ein $v_k = \normVek{P_k - x}$ mit $\langle v_k(x), v \rangle \ge d_{\min} > 0$ existieren.
Es sei $v = \normVek{E^{t+1} - E^t}$ und $v_k = v_k(E^t)$.
Dann folgt wie f"ur \Planes:
\begin{align*}
	\underbrace{d(P^t, E_k^t)}_{=d^t} - d(Q, E^{t+1}) \ge \langle v_k, v \rangle \ge d_{\min} > 0
\end{align*}
Daraus folgt dann $d_k^t - d_k^{t+1} \ge d_{\min}$.
\end{bewSkiz}

% Vorlesung vom 6. 5.

Unser Ziel ist es nun eine Verallgemeinerung auf beliebige konvexe Gebiete zu finden.
Kopparty \& Ravishankar diskutieren den folgenden Fall:
Das Spielfeld  $D$ besteht aus dem Schnitt von endlich vielen Halbr"aumen $H_l$.
\begin{center}\begin{tikzpicture}[font=\scriptsize,scale=1]
	%\tikzgitter{(-4,-4)}{(5,4)}
	
	\coordinate (e) at (0,0);
	\coordinate (1) at (130:1); \coordinate (2) at (35:1.5); \coordinate (3) at (330:1.5);
	\coordinate (4) at (270:1); \coordinate (5) at (200:1.25);
	
	\draw[dashed] (e) -- (1) (e) -- (2) (e) -- (3) (e) -- (4) (e) -- (5);
	\draw[dashed,gray] (1) -- (2) -- (3) -- (4) -- (5) -- cycle;
	
	\path[name path=h3] ($(5)!1cm!90:(e)$) -- ($(5)!-1cm!90:(e)$);
	\path[name path=h2] ($(4)!1cm!90:(e)$) -- ($(4)!-1.25cm!90:(e)$);
	\path[name path=h1] ($(3)!0.5cm!90:(e)$) -- ($(3)!-1cm!90:(e)$);
	\path[name intersections={of=h3 and h2}];
	\coordinate (a) at (intersection-1);
	\path[name intersections={of=h2 and h1}];
	\coordinate (b) at (intersection-1);
	\draw ($(5)!1.25cm!90:(e)$) --node[pos=0.3,left]{$H_3$} (a) --node[pos=0.8,below]{$H_3$} (b) --node[pos=0.7,right]{$H_3$} ($(3)!-1.5cm!90:(e)$);
	
	\fill (5)circle(0.05)node[below left]{$F_3$} (4)circle(0.05)node[below]{$F_2$} (3)circle(0.05)node[below right]{$F_1$};
	\fill (e)circle(0.05)node[above]{$E$}; \node at (-1.25,0.75) {$D$};
\end{tikzpicture}\end{center}
Die orthogonale Projektion $F_l$ von $E$ auf den Rand von $H_l$ stell zus"atzliche (virtuelle) Verfolger dar, deren Bewegungen auf den Rand $\partial H_l$ von $H_l$ beschr"ankt sind.
F"ur \Spheres l"asst sich zeigen, dass die den Bewegungsspielraum von $E$ einschr"ankenden B"alle \quot{monoton wachsen}. Genauer:
\begin{center}\begin{tikzpicture}[font=\scriptsize,scale=1]
	%\tikzgitter{(-4,-4)}{(5,4)}
	
	\coordinate (1) at (3,0); \coordinate (2) at (3,1);
	\draw[dashed] (1) --coordinate[pos=0.5](3) (0,0) --coordinate[pos=0.7](4) (2);
	\draw (3) -- (4);
	\fill (0,0) circle(0.05)node[left]{$p_k$} (1) circle(0.05)node[right]{$E^t$} (2) circle(0.05)node[right]{$E^{t+1}$};
	\fill (3) circle(0.05)node[below]{$P_k^t$} (4) circle(0.05)node[above]{$P_k^{t+1}$};
\end{tikzpicture}\end{center}
Falls also $d(p_k, E_k^{t+1}) \le$ konst. f"ur alle $t \in \N$ gilt, so ist \Spheres erfolgreich.
Dies ist "aquivalent dazu, dass $E$ im Inneren der konvexen H"ulle von $P_1, \ldots, P_k, F_1, \ldots, F_k$ liegt.

Kopparty \& Ravishankar behaupten, dies sei notwendig und hinreichend f"ur den Erfolg von \Spheres.
Bei unserer bisherigen Betrachtung unseres Algorithmus fallen zwei Probleme ins Auge:
\begin{itemize}
\item
	$F_l$ liegt unter Umst"anden nicht in $D$
	\begin{center}\begin{tikzpicture}[font=\scriptsize,scale=1]
		%\tikzgitter{(-4,-4)}{(5,4)}
		
		\coordinate (1) at (2,0.75);
		\draw (-2,0.5) -- (0,0);
		\draw (1) --node[below]{$H_1$} (0,0);
		\draw[dashed] (0,0) --coordinate[pos=0.45](2) ($(1)!2!(0,0)$);
	
		\draw[dashed] (2) node[below]{$F_1$} -- node[left,pos=0.3]{$H_2$} ($(2)!1.1cm!90:(0,0)$) coordinate (e);
		\fill (e) circle(0.05) node[above]{$E$};
		\clip (0,0) -- (2) -- (e) -- cycle;
		\draw (2) circle(0.25);
		\fill ($(2)+(62:0.15)$) circle(0.04);
	\end{tikzpicture}\end{center}
\item
	Der Algorithmus ist unter Umst"anden nicht wohldefiniert
	\begin{center}\begin{tikzpicture}[font=\scriptsize,scale=1]
		%\tikzgitter{(-4,-4)}{(5,4)}
		
		\draw (-2,0) --node[below,pos=0.1]{$H_l$} (2,0);
		\coordinate (1) at (2,-0.75);
		\draw[->] (1) --coordinate[pos=0.5](3) ++(150:3.5)node[above]{$E^t$} -- ++(230:0.9) coordinate(2) node[left]{$E^{t+1}$};
		\draw[dashed] (1) --coordinate[pos=0.6](4) (2);
		\draw (3)node[above]{$P^t$} -- (4)node[below]{$P^{t+1}$};
		\fill (3)circle(0.05) (4)circle(0.05) ($(1)+(150:3.5)$)circle(0.05);
	\end{tikzpicture}\end{center}
	Es werden m"oglicherweise Verfolgerpositionen au"serhalb von $D$ errechnet.
	Das Verwerfen aller Verfolger $P_k$ mit $p_k \notin D$ f"uhrt unter Umst"anden dazu, dass alle $P_k$ verworfen werden
	\begin{center}\begin{tikzpicture}[font=\scriptsize,scale=1]
		%\tikzgitter{(-4,-4)}{(5,4)}
		
		\coordinate (e) at (2.85,0);
		\draw (0,0) --coordinate[pos=0.8](1) +(10:4) (0,0) --coordinate[pos=0.8](2) +(-10:4);
		\coordinate (3) at ($(e)!1.25cm!(1)$); \coordinate (4) at ($(e)!1.25cm!(2)$);
		\draw[dashed] (e) -- (3) (e) -- (4);
		\fill (e)circle(0.05)node[left]{$E$} (1)circle(0.05)node[below right]{$P_1$} (2)circle(0.05)node[above right]{$P_2$} (3)circle(0.05)node[right]{$p_1$} (4)circle(0.05)node[right]{$p_2$};
	\end{tikzpicture}\end{center}
\end{itemize}
Die folgende Bedingung ist "aquivalent zum ersten Problem:

\begin{dfn}[Alexander-Bishop-Ghrist '09]
Es sei $D \subseteq \R^n$ abgeschlossen und konvex.
F"ur $k \le N$ bezeichne $H_k$ die abgeschlossene Halbebene, welche $E$ enth"alt und deren Rand durch $P_k$ und orthogonal zu $\overline{E P_k}$ verl"auft.
\begin{center}\begin{tikzpicture}[font=\scriptsize,scale=0.8]
	%\tikzgitter{(-4,-4)}{(5,4)}
	
	\draw (4.5,2) ..controls(4.5,2) and (-2.5,2).. (-2.5,0) ..controls(-2.5,-2) and (4.5,-2).. (4.5,-2);
	\coordinate (e) at (-1,0); \coordinate (p1) at ($(e)+(340:3)$); \coordinate (p2) at ($(e)+(20:2.5)$);
	
	\draw ($(p1)!2cm!90:(e)$) --node[right,pos=0.1]{$H_1$} ($(p1)!4cm!270:(e)$);
	\foreach \i in {0.05, 0.1, ...,1}{
		\path ($(p1)!2cm!90:(e)$) --coordinate[pos=\i] (pkt) ($(p1)!4cm!270:(e)$);
		\draw[gray,opacity=0.7] (pkt) -- +(200:0.25);
	}
	\begin{scope}
		\clip (e) -- (p1) -- ($(p1)!4cm!270:(e)$) -- cycle;
		\draw (p1) circle(0.25); \fill ($(p1)+(110:0.13)$) circle(0.04);
	\end{scope}
	
	\draw ($(p2)!4cm!90:(e)$) --node[right,pos=0.9]{$H_2$} ($(p2)!2cm!270:(e)$);
	\foreach \i in {0.05, 0.1, ...,1}{
		\path ($(p2)!4cm!90:(e)$) --coordinate[pos=\i] (pkt) ($(p2)!2cm!270:(e)$);
		\draw[gray,opacity=0.7] (pkt) -- +(230:0.25);
	}
	\begin{scope}
		\clip (e) -- (p2) -- ($(p2)!4cm!90:(e)$) -- cycle;
		\draw (p2) circle(0.25); \fill ($(p2)+(250:0.13)$) circle(0.04);
	\end{scope}
	
	\fill (e)circle(0.05)node[left]{$E$} (p1)circle(0.05)node[right]{$P_1$} (p2)circle(0.05)node[right]{$P_2$};
	\draw[dashed] (e) -- (p1) (e) -- (p2);
\end{tikzpicture}\end{center}
Die Konfiguration erf"ullt die Bedingung \CmMark[BC@(BC)]{(BC)}, falls $D \cap \bigcap_{k \le N} H_k$ beschr"ankt ist.
\end{dfn}

Die L"osung des zweiten Problems bedarf einer Modifikation von \Spheres.
Im zweidimensionalen Fall "andern Randkollisionen der Verfolger (heuristisch) nichts.
\begin{center}\begin{tikzpicture}[font=\scriptsize,scale=1]
	%\tikzgitter{(-4,-4)}{(5,4)}
	
	\coordinate (l) at (-3,1); \coordinate (r) at (3,1);
	\coordinate (L) at (-0.5,2); \coordinate (R) at (0.5,2);
	\coordinate (e) at (98:1);
	
	\draw (l) --coordinate[pos=0.2](1) (0,0) --coordinate[pos=0.8](2) (r) (L) -- (0,0); \draw[dashed] (0,0) -- (R);
	\fill (1)circle(0.05)node[below]{$P_2$} (2)circle(0.05)node[below]{$P_2$} (e)circle(0.05)node[left]{$E$};
	\draw[->] (e) -- +(20:0.3) node[right]{$E'$}; \draw[->] ($(e)+(20:0.4)$) -- +(160:0.4);
	\draw[->] ($(e)+(20:0.4)+(160:0.5)$) -- +(20:0.6);
	
	\begin{scope}
		\clip (l) -- (0,0) -- (r) -- cycle;
		\draw (0,0) circle(0.25);
	\end{scope}
	\node at (0.5,-0.1) {$\gg \frac{\pi}{2}$};
	\begin{scope}
		\clip (L) -- (0,0) -- (R) -- cycle;
		\draw (0,0) circle(0.75);
	\end{scope}
	\node at (0.5,0.6) {$\ll \frac{\pi}{2}$};
\end{tikzpicture}\end{center}
Im dreidimensionalen Fall sollte es m"oglich sein eine Konstellation in einem \quot{breiten und flachen} Gebiet die Verfolger in jedem Schritt zu Kollisionen zu zwingen, so dass eine Fluchtstrategie existiert.
Zun"achst l"asst sich zeigen, dass (BC) notwendig ist:
Falls $D \cap \bigcap_{k \le N} H_k$ beschr"ankt ist, so existiert ein Strahl $c$ mit dem Startpunkt $E$ und $c \cap H_k = \emptyset$ f"ur alle $k \in \N$.
\begin{center}\begin{tikzpicture}[font=\scriptsize,scale=1]
	%\tikzgitter{(-4,-4)}{(5,4)}
	
	\coordinate (vec) at (10:1); \coordinate (vec') at (70:1);
	\coordinate (1) at (-0.5,1); \coordinate (2) at ($(1)+0.5*(vec)+0.5*(vec')$);
	
	\draw (1) -- ++($2.25*(vec)$) -- ++(vec') -- ++($-2.25*(vec)$) --node[left]{$H_k$} ++($-1*(vec')$);
	\draw ($(2)+3*(vec)$) -- (2) -- ++(280:1.25)coordinate (e) --node[below]{$c$} ++(10:5);
	\fill (e)circle(0.05)node[below]{$E$};
	\begin{scope}
		\clip ($(2)+3*(vec)$) -- (2) -- ++(280:1.25) -- cycle;
		\draw (2) circle(0.25); \fill ($(2)+(320:0.15)$)circle(0.03);
	\end{scope}
	\begin{scope}
		\clip (2) -- ++(280:1.25) -- ++(10:5) -- cycle;
		\draw ($(2)+(280:1.25)$) circle(0.25);
	\end{scope}
	\node at (0.75,0.75) {$> \frac{\pi}{2}$};
\end{tikzpicture}\end{center}
Dann gilt $\varangle_E (c, \overline{E P_k}) > \frac{\pi}{2}$ f"ur alle $k \le N$. Somit enth"alt der Halbraum $H = \{ x \mid \langle x, \dot c \rangle \le \langle E, c \rangle \}$ alle Verfolger $P_k$ und $c$ ist eine Fluchtstrategie.

\begin{emptythm}[Zur L"osung des \quot{Kollisionsproblems}]
Modifikation von \Spheres nach (A-B-G '09) zu \RotSpheres\index{Rotating Spheres@\RotSpheres}. Die Idee besteht darin die Mittelpunkte $p_k$ in jedem Schritt so zu bewegen, dass
\begin{itemize}
\item
	$P_k^{t+1} \in D$
\item
	(BC) erhalten bleibt
\item
	Die B"alle um $P_k$ hinreichend wachsen, so dass der Bewegungsspielraum von $E$ eingeschr"ankt wird.
\end{itemize}
\end{emptythm}

\begin{emptythm}[\RotSpheres]
\begin{itemize}
\item
	\emph{Initialisierung:} W"ahle Punkte $p_k^0$ auf dem Strahl von $E$ durch $P_k$ so, dass $P_k \in \overline{E p_k}$ gilt und die Komponente von $D \setminus \bigcup_{k \le N} \B_{d(p_k, P_k)}(p_k)$ welche $E$ enth"alt, beschr"ankt ist. (folgt aus (BC))
\item
	$P_k^{t+1} = P_k^{t+1}(D, p_k^t, P_k^t, E^{t+1})$
\item
	$P_k{''} \in \overline{p_k^t E^{t+1}}$ mit $d(P_k^t, P'') = 1$ und $d(P'', E^{t+1})$ minimal
	\begin{center}\begin{tikzpicture}[font=\scriptsize,scale=1]
		%\tikzgitter{(-4,-4)}{(5,4)}
		
		\coordinate (1) at (30:2.5); \coordinate (2) at (10:4);
		
		\draw[dashed] (0,0)node[left]{$p_k$} -- (1)node[above]{$P^t$} (0,0) --coordinate[pos=0.8] (3) (2)node[right]{$E^{t+1}$};
		\draw (1) -- (3)node[below]{$P{''}$};
		\fill (0,0)circle(0.05) (1)circle(0.05) (2)circle(0.05) (3)circle(0.05);
	\end{tikzpicture}\end{center}
	\begin{description}
	\item[Falls $\bm{P'' \in D}$:]
		$P_k^{t+1} = P{''}$, $p_k^{t+1} = p_k^t$
	\item[Falls $\bm{P'' \notin D}$:]
		Betrachte die folgende Zeichnung
		\begin{center}\begin{tikzpicture}[font=\scriptsize,scale=1]
			%\tikzgitter{(-4,-4)}{(5,4)}
			
			\def\R{6}
			\def\r{1}
			\coordinate (e) at (-3,0.5); \coordinate (p') at ($(e)+(350:\R)$); \coordinate (p) at ($(e)+(335:\R)$);
			
			\draw[name path=gerade] (-4,0) -- (4,0);
			\draw[dashed,name path=strich] (e)node[above]{$E^{t+1}$} -- (p)node[right]{$p^t$} (e) -- (p')node[right]{$p^{t+1}$};
			
			\path[name intersections={of=strich and gerade}];
			\coordinate (P*) at (intersection-2);
			\path[name path=senkr] (P*) -- +(0,-2);
			\path[name intersections={of=senkr and strich}];
			\coordinate (P'') at (intersection-1);
			
			\draw (P*)node[above]{$P*$} -- (P'')node[below]{$P{''}$};
			
			\draw let \p1=($(p)-(e)$), \n0={veclen(\x1,\y1)}, \n1={atan2(\x1,\y1)}, \p2=($(p')-(e)$), \n2={atan2(\x2,\y2)} in (p) arc[radius=\n0,start angle=\n1, end angle=\n2];
			
			\draw[dashed] (p) --coordinate[pos=0.7](P) ++(125:4);
			\draw (P) node[above right]{$P^t$} -- (P'');
			\path[name path=kreis] let \p1=($(P'')-(P)$), \n0={veclen(\x1,\y1)} in (P) circle(\n0);
			\path[name intersections={of=strich and kreis}];
			\coordinate (P') at (intersection-3);
			
			\draw let \p1=($(P'')-(P)$), \n0={veclen(\x1,\y1)}, \p2=($(P')-(P)$), \n2={atan2(\x2,\y2)} in (P')node[above]{$P'$} arc[radius=\n0,start angle=\n2,end angle=-110];
			
			\foreach \a in{e, p, p', P, P*, P', P''}{
				\fill (\a) circle(0.05);
			}
			
			\draw ($(P*)-(0,0.25)$) arc[radius=0.25,start angle=270, end angle=360];
			\fill ($(P*)+(315:0.15)$) circle(0.03);
		\end{tikzpicture}\end{center}
		E bezeichne $P^*$ die orthogonale Projektion von $P{''}$ auf $D$ und $p_k^{t+1}$ den Punkt auf dem Strahl von $E^{t+1}$ durch $P^*$ mit $d(E^{t+1}, p_k^t) = d(E^{t+1}, p_k^{t+1})$.
		Dann sei $P_k^{t+1} = P'$ der Punkt auf $\overline{p_k^{t+1} E^{t+1}}$ mit $d(P_k^t, P_k^{t+1}) = 1$ und $d(P_k^{t+1}, E^{t+1})$ minimal.
	\end{description}
\end{itemize}
\end{emptythm}

\begin{satz}[A-B-G '09]
F"ur jedes konvexe Gebiet $D$ ist \RotSpheres erfolgreich, wenn die Konfiguration $D, P_1, \ldots, P_N$ die Bedingung (BC) erf"ullt.
\end{satz}

% Vorlesung vom 6. 5.

\begin{bewSkiz}\begin{enumerate}[label=\arabic*),leftmargin=*,widest=2]
\item
	Es gilt $d(p_k^{t+1}, P_k^{t+1})^2 > d(p_k^t, P_k^t)^2 + 1$.
	Setze $p=p_k^t$, $p'=p_k^{t+1}$, $P=P_k^t$, $P'=P_k^{t+1}$.
	Wir unterscheiden nun zwei F"alle.
	\begin{description}[leftmargin=*]
	\item[Falls \bm{$P'' \in D$}:]
		Es gilt
		\begin{center}\begin{tikzpicture}[font=\scriptsize,scale=1]
			%\tikzgitter{(-4,-4)}{(5,4)}
			
			\draw (0,0) --coordinate[pos=0.2](P'') (3.5,0)coordinate(p)node[right]{$p$} --coordinate[pos=0.6](P) (0,1);
			\draw (P)node[above]{$P$} --node[left]{$1$} (P'')node[below]{$P''$};
			\draw[decorate,decoration={brace},xshift=-0.1cm] (0,0) --node[left]{$<1$} (0,1);
			
			\foreach \pkt in {p,P,P''}{
				\fill (\pkt) circle(0.05);
			}
			
			\clip (P'') -- (p) -- (P) -- cycle;
			\draw (P)circle(0.25) node[shift={(0.03,-0.13)}] {$\alpha$};
		\end{tikzpicture}\end{center}
		$d(p,P'')^2 = d(p, P)^2 + d(P, P'')^2 - 2 d(p, P) d(P, P'') \cdot \underbrace{\cos \alpha}_{<0} > d(p, P)^2 + 1$
	\item[Falls \bm{$P'' \notin D$}:]
		Es gilt
		\begin{center}\begin{tikzpicture}[font=\scriptsize,scale=1]
		%\tikzgitter{(-4,-4)}{(5,4)}
		
		\draw (3,0)coordinate(p')node[right]{$p'$} --coordinate[pos=0.6](P') coordinate[pos=0.3](P*) (0,0)coordinate(E')node[above]{$E'$} --coordinate[pos=0.75](P'') (3,-1) coordinate (p)node[right]{$p$};
		\draw (P')node[above]{$P'$} -- (P'')node[below]{$P''$} -- (P*)node[above,xshift=0.3cm]{$P^* \in D$};
		\draw[dashed] (p) -- (p');
			
		\foreach \pkt in {p,p',P'',P',P*,E'}{
			\fill (\pkt) circle(0.05);
		}
		
		\clip (P') -- (P'') -- (P*) -- cycle;
		\draw (P*)circle(0.25) node[shift={(-0.1,-0.15)}] {$\beta$};
		\end{tikzpicture}\end{center}
		$P^* = \proj_D(P'') \in D$, $P'' \notin D$, $E \in D$.
		Damit folgt f"ur den Winkel $\beta = \varangle_{P*}(E', P'') \ge \frac{\pi}{2}$ und $d(p, P'') \le d(p', P^*) \le d(p', P')$, der Rest geht weiter wie oben.
	\end{description}
\item
	Setze $C^t = D \setminus \bigcup_{k \le N} \B_k^t$, $\B_k^t = \B_{d(p_k^t, P_k^t)}(p_k^t)$.
	\emph{Behauptung:} $\overline{C^{t+1}} \subseteq C^t$.
	\begin{center}\begin{tikzpicture}[font=\scriptsize,scale=0.9]
		%\tikzgitter{(-4,-4)}{(5,4)}
		
		\draw[dashed,opacity=0.5,name path=hstern] (-4,0) --node[below,pos=0.95]{$H^*$} (4,0);
		\foreach \i in {0,0.1,0.2,...,1}{
			\path (-4,0) --coordinate[pos=\i](pkt) (4,0);
			\draw[opacity=0.5] (pkt) -- +(45:0.25);
		}
		\draw[dashed,opacity=0.5,name path=hstrich] (-4,0.5)coordinate(0) --node[below,pos=0.95]{$H'$} coordinate[pos=0.85](1) (4,-1.75);
		\foreach \i in {0,0.1,0.2,...,1}{
			\path (-4,0.5) --coordinate[pos=\i](pkt) (4,-1.75);
			\draw[opacity=0.5] (pkt) -- +(30:0.25);
		}
		\path[name intersections={of=hstern and hstrich}];
		\coordinate (sp) at (intersection-1);
		\node at (3.75,1) {$D$};
		
		\draw[name path=kasten] ($(0)!-0.2cm!90:(1)$) -- ($(1)!0.5cm!90:(0)$)coordinate(p)node[right]{$p$} -- ($(1)!-0.5cm!90:(0)$) coordinate(p')node[right]{$p'$} -- ($(0)!0.2cm!90:(1)$);
		
		\begin{scope}
			\clip (0) -- (1) -- (p') -- cycle;
			\draw (1) circle(0.25);
			\fill ($(1)+(115:0.13)$) circle(0.03);
		\end{scope}
		
		\def\rad{2.5};% um p herum 
		\coordinate (Bt) at ($(p)+(90:\rad)$);
		\coordinate (P) at ($(p)+(110:\rad)$);
		\draw[gray] let \p1=($(Bt)-(p)$), \n0={atan2(\x1,\y1)} in (Bt)node[right]{$\B^t$} arc[radius=\rad,start angle=\n0,end angle=180];
		
		\def\rad{3.8};% um p herum 
		\coordinate (B'') at ($(p)+(90:\rad)$);
		\draw[name path=kreis2,gray] let \p1=($(B'')-(p)$), \n0={atan2(\x1,\y1)} in (B'')node[right]{$\B''$} arc[radius=\rad,start angle=\n0,end angle=180];
		
		\def\rad{3.8};% um p' herum 
		\coordinate (B) at ($(p')+(150:\rad)$);
		\draw[name path=kreis3,gray] let \p1=($(B)-(p')$), \n0={atan2(\x1,\y1)} in (B) arc[radius=\rad,start angle=\n0,end angle=210]node[below]{$\B''$};
		
		\def\rad{4.5};% um p' herum 
		\coordinate (Bt+1) at ($(p')+(155:\rad)$);
		\draw[name path=kreis4,gray] let \p1=($(Bt+1)-(p')$), \n0={atan2(\x1,\y1)} in (Bt+1) arc[radius=\rad,start angle=\n0,end angle=205]node[below]{$\B^{t+1}$};
		
		\path[name intersections={of=kreis2 and kreis3}];
		\coordinate (2) at (intersection-1);
		
		\path[name intersections={of=kasten and kreis2}];
		\coordinate (P'') at (intersection-1);
		
		\path[name intersections={of=kasten and kreis4}];
		\coordinate (P') at (intersection-2);
		
		\path[name intersections={of=kasten and hstern}];
		\coordinate (P*) at (intersection-2);
		
		\draw (P)node[above]{$P$} -- (P'')node[below]{$P''$} -- (P*)node[above]{$P^*$};
		\node[above] at (P') {$P'$};
		\draw[->] (-3,-0.5)node[left]{$\partial H' \cap \partial H^*$} to[out=0,in=270] ($(sp)-(0,0.07)$);
		
		\foreach \pkt in {2,sp,p,p',P,P',P'',P*}{
			\fill (\pkt) circle(0.05);
		}
		
		\clip (P'') -- (P*) -- (-3,0) -- cycle;
		\draw (P*) circle(0.25);
		\fill ($(P*)+(225:0.125)$) circle(0.03);
	\end{tikzpicture}\end{center}
	\begin{description}[leftmargin=*]
	\item[Falls \bm{$p^{t+1}=p'=p=p^t$}:]
		Aus 1) folgt $\B^{t+1} \supset \B^t$.
	\item[Falls \bm{$p^{t+1}=p'\ne p=p^t$}:]
		\begin{enumerate}[label=\roman*),leftmargin=*,widest=iii]
			\item
				$\B^t \subseteq \B''$ und $\B \subseteq \B^{t+1}$
			\item
				$\B$ und $\B''$ haben den gleichen Radius und $\partial \B \cap \partial \B'' \subseteq \partial H'$.
				Daraus folgt
				\begin{align*}
					\B'' \cap H' \subseteq \B \cap H' = \{ x \mid d(x, p') \le d(x,p) \}
				\end{align*}
			\item
				$\partial H^* \cap \partial H'$ und $\partial \B'' \cap \partial H'$ werden durch $\overline{P'' P^*} \cap \partial H'$ \quot{getrennt}, damit folgt $\B'' \cap H^* \subset H'$.
		\end{enumerate}
		Insgesamt folgt aus diesen drei Punkten:
		\begin{align*}
			\B^t \cap D & \overset{\text{(i)}}{\subset} \B'' \cap D \subseteq \B'' \cap H^* \cap D \overset{\text{(iii)}}{\subseteq} \B'' \cap H' \cap D \overset{\text{(ii)}}{\subset} \B \cap H' \cap D \\
			& \overset{\text{(i)}}{\subseteq} \B^{t+1} \cap H' \cap D \subseteq \B^{t+1} \cap D
		\end{align*}
		Daraus folgt $\overline{C^{t+1}} \subset C^t$.
	\end{description}
\item
	Es existiert $C_1$ so, dass f"ur alle $x_k \in \B_k^0 \cap D$ gilt $d(x_k, p_k^t) \subseteq C_1$.
	Wie oben gilt $x_k \in \B_k^0 \cap D \subseteq \ldots \subseteq \B_k^t \cap D$.
	Mit den Bezeichnungen wie oben sei $H_k^{t+1} = H'$ der Halbraum durch den Mittelpunkt von $\overline{p_k^t, p_k^{t+1}}$ mit $H' \perp \overline{p_k^t p_k^{t+1}}$.
	Es gilt
	\begin{align*}
		x_k \in \B_k^t \cap D &\subseteq \B_k{''} \cap H_k^* \\
		&\subseteq H' = \{ x \mid d(x, p^{t+1}) \le d(x, p) \}.
	\end{align*}
	Also gilt $d(x_k, p_k^{t+1}) \le d(x_k, p_k^t)$ und damit ist $d(x_k, p_k^t)$ uniform beschr"ankt.
	\begin{center}\begin{tikzpicture}[font=\scriptsize,scale=0.9]
		%\tikzgitter{(-4,-4)}{(5,4)}
		
		\draw (-2,1)coordinate(1) --coordinate[pos=0.7](2) (2,-0.25)coordinate(3);
		\foreach \i in {0.05,0.15,...,1} {
			\path (1) --coordinate[pos=\i](0) (3);
			\draw (0) -- +(20:0.25);
		}
		\node at (-1.5,1.5) {$H'=H_k^{t+1}$};
		
		\draw[dashed] ($(2)!1.25cm!90:(1)$) coordinate (pt) node[below]{$p^t$} -- ($(2)!-1.25cm!90:(1)$) coordinate (x) node[above]{$x$} -- +(0.25,-1)coordinate (pt+1) node[right]{$p^{t+1}$};
		
		\foreach \pkt in {pt,x,pt+1} {
			\fill (\pkt) circle(0.05);
		}
	\end{tikzpicture}\end{center}
\item
	$E^t \in C^t \overset{(2)}{\subset} C^0$ ist beschr"ankt.
	Damit folgt dass $d(x_k, E^t)$ uniform beschr"ankt ist durch $C_2$.
	\begin{align*}
		(C_1 + C_2)^2 & \overset{\mathclap{(3)}}{\underset{(4)}{\ge}} (d(x_k, p_k^t) + d(x_k, E^t))^2 \\
		& \ge d(p_k, E^t)^2 \\
		& \ge d(p_k, P_k^t)^2 \\
		& \overset{(1)}{\ge} d(p_k^0, P_k^0) + t
	\end{align*}
\end{enumerate}\end{bewSkiz}

\begin{emptythm}[Anwendung/Verallgemeinerung]
Betrachte Gebiete $D \subseteq \R^n$, welche sich als endliche Vereinigung konvexer abgeschlossener $D_\alpha$ schreiben lassen.
\begin{center}\begin{tikzpicture}[font=\scriptsize,scale=1]
	%\tikzgitter{(-4,-4)}{(5,4)}
	
	\draw (-2,0.5) -- ++(3,0.5) -- ++(-3,0.5)node[above]{$D_3$};
	\draw (-2,-1.5)node[right]{$D_4$} ..controls(-2,-1.5) and (-0.5,0.5)..coordinate[pos=0.75](e4) (1,0.5) ..controls(2.5,0.5) and (3,-1.5).. (3,-1.5);
	\draw (3.5,2) ..controls(3.5,2) and (0,2.5).. (0,1)coordinate (e1) ..controls(0,-2) and (3.5,0.5).. (3.5,0.5)node[right]{$D_1$};
	\draw (0,0.5) circle(1.5); \node[above] at (0,2) {$D_2$};
	
	\coordinate (p) at (3,1); \node[right] at (p) {$P$};
	\coordinate (e) at (-0.8,1.1);
	\node[right] at (e1) {$E_1$}; \node[left] at (e) {$E=E_2=E_3$}; \node[below] at (e4) {$E_4$};
	
	\draw[dashed] (e1) -- (e) -- (e4);
	
	\begin{scope}
		\clip (-2,-1.5) ..controls(-2,-1.5) and (-0.5,0.5).. (1,0.5) -- (e4) -- (e) -- (-2,-1.5);
		\clip ($(e4)-(1,1)$) rectangle ($(e4)+(0,1)$);
		\draw (e4) circle(0.25); \fill ($(e4)+(170:0.125)$) circle(0.03);
	\end{scope}
	\begin{scope}
		\clip (3.5,2) ..controls(3.5,2) and (0,2.5).. (0,1) -- (e) -- cycle;
		\draw (e1) circle(0.25); \fill ($(e1)+(120:0.125)$) circle(0.03);
	\end{scope}
	
	\foreach \pkt in {e,e4,e1,p}{
		\fill (\pkt) circle (0.05);
	}
\end{tikzpicture}\end{center}
Es seien $D = \bigcup_\alpha D_\alpha$ eine endliche Vereinigung abgeschlossener konvexer Gebiete $D_\alpha$.
F"ur $E \in D$ bezeichne $E_\alpha = \proj_{D_\alpha}(E)$.
Eine Konfiguration $D, E, \{P_k\}$ erf"ullt \CmMark[EBC@(EBC)]{(EBC)}, falls eine Partition von $\{P_k\}$ in Mengen $\{P_{\alpha i} \mid i \le N_\alpha \} \subset D_\alpha$ existiert, so dass f"ur $D_\alpha, E_\alpha, \{P_{\alpha,i}\}$ (BC) gilt.
In diesem Fall existiert eine erfolgreiche Verfolgerstrategie.
Die Schrittweite $\le 1$ f"ur $E$ (bez"uglich der L"angenmetrik von $D = \bigcup D_\alpha$) impliziert $d(E^t, E^{t+1}) \le 1$ in der Metrik (jedes) $\R^{n_\alpha}$.

Da jedes $\proj_{D_\alpha}$ Abst"ande nicht vergr"o"sert, gilt auch f"ur alle $E_\alpha$ Schrittweite $\le 1$.
In jedem Gebiet $D_\alpha$ gilt nach endlicher Zeit $P_{\alpha, i}^t = E_\alpha = \proj_{D_\alpha}(E)$.
Falls $E_\alpha \ne E$, setze $P_{\alpha,i}^T = E_\alpha^T$ f"ur alle $T \ge t$.
Nach endlicher Zeit gilt dann $P_{\alpha,i_\alpha}^t = E_\alpha^t$ f"ur alle $\alpha$ und damit ist $E^t = E_{\overline\alpha}^t$ f"ur ein $\overline\alpha$, das hei"st $P_{\overline\alpha, i_\alpha}^t = E^t$.
\end{emptythm}

\textbf{Bhadanin-Ister '12:} In jedem einfachen kompakten Polygon mit Hinternissen existiert eine erfolgreiche Verfolgerstrategie.
\begin{center}\begin{tikzpicture}[font=\scriptsize,scale=0.8]
	%\tikzgitter{(-4,-4)}{(5,4)}
	
	\coordinate(1) at (-1.5,1.5); \coordinate(2) at (-3,0.5); \coordinate(3) at (-2.5,-2);
	\coordinate(4) at (1,-2.5); \coordinate(5) at (3,-1);
	\coordinate (a) at (-2,0); \coordinate (b) at (0,-0.25); \coordinate (c) at (-1.5,-1);
	\coordinate (d) at (1.5,-0.75); \coordinate (e) at ($(d)+(100:1.5)$);
	
	\draw (1) -- (2) -- (3) -- (4) -- (5) -- +(-0.5,3) (1) -- +(0,0.5);
	\filldraw[fill=gray,fill opacity=0.4] (a) -- (b) -- (c) -- cycle (d) -- (e) -- ($(e)!0.5cm!90:(d)$) -- ($(d)!-0.5cm!90:(e)$) -- cycle;
	\draw[dashed] (1) -- ($(2)!1.5!(1)$) (3) -- (c) -- (4) (b) -- (d) -- (4) -- cycle (b) -- (e) (a) -- (0,2);
	\fill (-2.25,0.5) circle(0.05) node[above right]{$E$};
\end{tikzpicture}\end{center}
Bei Polygonen kann man f"ur Hindernisse eine Triangulierung w"ahlen.
Der Nachteil dabei ist dass man einen Verfolger pro Gebiet braucht.
Dies funktioniert auch, wenn das Gebiet in eine Richtung unbeschr"ankt ist.

\begin{enumerate}[label=\Alph*),leftmargin=*,widest=B]
\item
	\textbf{(Aigner-Fromme '84)} Jedes $P$ kann ein geod"atisches Segment $c$ \quot{bewachen}, das hei"st $E$ wird gefangen, falls es $c$ "uberquert.
\item
	\textbf{(Ister et al. '05)} in jedem einfach zusammenh"angenden Polygon existiert eine erfolgreiche Strategie mit einem Verfolger.
	\begin{center}\begin{tikzpicture}[font=\scriptsize,scale=0.75]
		%\tikzgitter{(-4,-4)}{(5,4)}
		
		\coordinate (1) at (-3,0.5); \coordinate (2) at (-1,1.5); \coordinate (3) at (3.5,0.75);
		\coordinate (4) at (3,-1); \coordinate (5) at (0,-1.5); \coordinate (6) at (-2.5,-1);
		
		\draw (1) -- (2) -- (3) (4) -- (5) -- (6);
		\draw[very thick] (6) --node[left]{$c_1$} (1) (3) --node[right]{$c_2$} (4) (2) --node[left,pos=0.3]{$c_3$}coordinate[pos=0.3](a)coordinate[pos=0.7](b) (6) --node[below,pos=0.6]{$c_1'$} (0,0) -- cycle;
		\draw[->] ($(6)!0.3!(1)$) to[out=40,in=120] ($(6)!0.3!(0,0)$);
		\filldraw[fill=gray,fill opacity=0.5] (0,0) -- (a) -- (b) -- cycle;
		\fill (1.5,-0.5) circle(0.05) node[above]{$E$};
	\end{tikzpicture}\end{center}
	W"ahle Kanten(z"uge) $c_1 \ni P_1$ und $c_2 \ni P_2$.
	Setzt $c_3$ so, dass $D$ in Komponenten zerf"allt.
	Dann ist entweder $P_1$ oder $P_2$ frei.
	Nach endlicher Zeit liegt $E$ in einem einfach zusammenh"angenden Polygon.
\end{enumerate}

% Vorlesung vom 13. 5.

\section{Geometrische Charakterisierung von Flucht- und Verfolgerkurven}

Im Folgenden sei das Soielfeld $D$ ein $\CAT(0)$- (beziehungsweise $\CAT(1)$-) Raum.
Alle Resultate beziehen sich auf \Greedy mit der Schrittweise $d > 0$, das hei"st $P^{t+1} \in \overline{P^t E^t}$ mit $d(P^t, P^{t+1}) = d$.
\begin{center}\begin{tikzpicture}[font=\scriptsize,scale=1]
	%\tikzgitter{(-4,-4)}{(5,4)}
	
	\def\d{1.5}
	\draw (-2.5,-0.5)coordinate(p)node[below]{$P^t$} --node[above]{$d$} ++(10:\d)coordinate(p')node[below]{$P^{t+1}$} -- ++(10:2*\d)coordinate(e)node[right]{$E^{t}$} -- ++(80:\d)coordinate(e')node[right]{$E^{t+1}$};
	\foreach \pkt in {p, p', e, e'}{
		\fill (\pkt)circle(0.05);
	}
\end{tikzpicture}\end{center}
Wie spiegelt sich der Ausgang eines Spiels in der Geometrie der Kurven $P^t$, beziehungsweise $E^t$, wider?
Dabei verstehen wir die Wege $P^t$, beziehungsweise $E^t$, als st"uckweise geod"atische Kurven $P = \bigcup \overline{P^t P^{t+1}}$.
F"ur $d = 1$ ist $P$ beziehungsweise $E$ nach der Bogenl"ange parametrisiert.

\begin{satz}[Alexander-Bishop-Ghrist 2010]
Ist $D$ ein $\CAT(0)$-Raum, so ist $D$ genau dann kompakt, wenn $P$ stets gewinnt (mit einem Verfolger).
\end{satz}

\begin{bew}
Ist $D$ nicht kompakt, so existiert ein geod"atischer Strahl als Fluchtstrategie.

Angenommen es existiert eine Fluchtstrategie. F"ur den Verfolgerabstand $d^t = d(P^t, E^t)$ gilt $d^{t+1} \le (d^t - d) + d = d^t$.
\begin{center}\begin{tikzpicture}[font=\scriptsize,scale=1]
	%\tikzgitter{(-4,-4)}{(5,4)}
	
	\def\d{1.5}
	\draw (-2.5,-0.5)coordinate(p)node[below]{$P^t$} --node[above]{$d$} ++(10:\d)coordinate(p')node[below]{$P^{t+1}$} -- ++(10:2*\d)coordinate(e)node[right]{$E^{t}$} --node[right]{$d$} ++(80:\d)coordinate(e')node[right]{$E^{t+1}$} --node[above]{$d^{t+1}$} (p');
	\fill[gray,opacity=0.3] (p') -- (e) -- (e') -- cycle;
	\begin{scope}
		\clip (p') -- (e) -- (e') -- cycle;
		\draw (p') circle(1.5);
		\node at ($(p')+(20:1.9)$) {$\alpha^{t+1}$};
	\end{scope}
	\draw[decoration={brace,mirror},decorate] ([yshift=-0.4cm]p') --node[below]{$d^t - d$} ([yshift=-0.4cm]e);
	\foreach \pkt in {p, p', e, e'}{
		\fill (\pkt)circle(0.05);
	}
\end{tikzpicture}\end{center}
Also konvergiert $d^t$ monoton fallend gegen ein $d^\infty > d$.
Betrachtet man den euklidischen Vergleichswinkel $\overline{\alpha}^t$ zu $\alpha^t = \varangle_{P^t}(E^{t+1}, E^t)$, so folgt
\begin{align*}
	\cos \overline{\alpha}^{t+1} &= \frac{(d^{t+1})^2 + (d^t - d)^2 - d^2}{2d^{t+1}(d^t - d)} \\
	&= \frac{(d^{t+1})^2 + (d^t)^2 - 2d^t}{2d^{t+1}(d^t - d)} \\
	&\rightarrow \frac{2d^\infty(d^\infty - d)}{2d^\infty(d^\infty - d)} = 1.
\end{align*}
Also konvergiert $\overline{\alpha}^t$ gegen $0$.
Da f"ur Vergleichswinkel stets $\alpha^t \le \overline{\alpha}^t$ gilt, folgt $\alpha^t \xrightarrow{t \to \infty} 0$.
\end{bew}

\begin{emptythm}[Erinnerung]
Es sei $c$ eine regul"are ebene Kurve $c: [0, l] \to \R^2$ und $\kappa$ die Kr"ummung von $c$.
Dann ist $\tau_c = \int_0^l \kappa$ die \CmMark{Totalkr\"ummung} von $c$.
Ist $c$ nach Bogenl"ange parametrisiert, so gilt $\kappa = \| \ddot{c} \|$.
Bezeichnet $\theta$ die Drehwinkelfunktion von $\dot{c}$, das hei"st $\dot{c}(t) = (\cos \theta(t), \sin \theta(t))$, so gilt $\|\ddot{c}\| = \| (-\theta' \sin \theta, \theta' \cos \theta) \| = | \theta' |$.
Damit ist die Totalkr"ummung die akkumulierte Winkel"anderung von $c$.
\begin{center}\begin{tikzpicture}[font=\scriptsize,scale=1]
	%\tikzgitter{(-4,-4)}{(5,4)}
	
	\draw (-2.5,-2)coordinate(0)node[below]{$c(0)$} --node[right]{$c_0$} ++(60:1.75)coordinate(1) --node[above]{$c_1$} ++(10:2.5)coordinate(2) --node[left]{$c_2$} ++(60:1.75)coordinate(3);
	\draw[dashed] (1) -- +(60:1.25) (2) -- +(10:1.25);
	\draw (0) ..controls($(0)+(100:0.25)$) and ($(1)+(180:0.5)$).. (1) ..controls($(1)+(0:0.5)$) and ($(2)+(225:0.75)$).. (2) ..controls($(2)+(225-180:0.25)$) and ($(1.75,1.25)$).. (3) node[right]{$c$};
	%\fill (0) circle(0.05);
	\begin{scope}
		\clip (1) -- (2) -- ($(1)+(60:1.25)$) -- cycle;
		\draw (1) circle(0.5);
		\node at ($(1)+(30:0.75)$) {$\alpha_1$};
	\end{scope}
	\begin{scope}
		\clip (2) -- (3) -- ($(2)+(10:1.25)$) -- cycle;
		\draw (2) circle(0.5);
		\node at ($(2)+(30:0.75)$) {$\alpha_2$};
	\end{scope}
\end{tikzpicture}\end{center}
ist $c_1, \ldots, c_n$ eine Polynomapproximation von $c$, so ist die Summe der Komplement"arwinkel $\alpha_i = \pi - \beta_i = \pi - \varangle(c_{i-1}, c_i)$ eine Approximation der Totalkr"ummung von $c$.
\end{emptythm}

\begin{dfn}[Totalk"ummung]
Es sei $c$ eine st"uckweise geod"atische Kurve in einem $\CAT(\kappa)$-Raum.
Sind $0 = t_0 < t_1 < \ldots < t_n = l$ so, dass $c_i = c|_{[t_{i-1}, t_i]}$ geod"atisch Segmente sind, so bezeichne $\beta_i = \varangle_{c(t_i)}(c_{i-1}, c_i)$ die Innenwinkel.
Die \CmMark{Totalkr\"ummung}  von $c$ sei $\tau_c = \sum_i \pi - \beta_i$. F"ur eine beliebige Kurve sei
\begin{align*}
&\tau_\gamma = \lim\sup \{\tau_c \mid c \text{ Polygonapproximation von } \gamma \} \\
&\tau_\gamma(t) = \tau_\gamma|_{[0, t]}
\end{align*}
Im Fall von $\CAT(0)$ gilt Monotonit"at.
\end{dfn}

Ist $c$ eine Verfeinerung einer Polygonapproximation $c'$ von $\gamma$, so gilt $\tau_{c'} \le \tau_c$, wie man an der Zeichnung erkennen kann:
\begin{center}\begin{tikzpicture}[font=\scriptsize,scale=1.4]
	%\tikzgitter{(-4,-4)}{(5,4)}
	
	\draw (-2.5,-2)coordinate(0) -- ++(60:1.75)coordinate(1) -- ++(10:2.5)coordinate(2) --node[right]{$c$} ++(60:1.75)coordinate(3) (1) --node[left]{$c$} ++(40:1.5)coordinate(4) -- (2);
	
	\begin{scope}
		\clip (0) -- (1) -- (2) -- cycle;
		\draw (1) circle(0.2);
		\node at ($(1)+(310:0.3)$) {$\beta_1'$};
	\end{scope}
	\begin{scope}
		\clip (4) -- (2) -- (3) -- cycle;
		\draw (2) circle(0.3);
		\node at ($(2)+(100:0.4)$) {$\beta_2'$};
	\end{scope}	
	\begin{scope}
		\clip (1) -- (4) -- (2) -- cycle;
		\draw (1) circle(0.4);
		\node at ($(1)+(25:0.5)$) {$\alpha_1$};
		\draw (2) circle(0.4);
		\node at ($(2)+(175:0.5)$) {$\alpha_2$};
		\draw (4) circle(0.3);
		\node at ($(4)+(270:0.4)$) {$\beta$};
	\end{scope}
	\begin{scope}
		\clip (0) -- (1) -- (4) -- (3) -- ++(0,-1) -- cycle;
		\draw (1) circle(0.6);
		\node at ($(1)+(350:0.75)$) {$\beta_1$};
	\end{scope}
	\begin{scope}
		\clip (1) -- (2) -- (3) -- cycle;
		\draw (2) circle(0.65);
		\node at ($(2)+(100:0.75)$) {$\beta_2$};
	\end{scope}
	
	\node[right,align=left] at (1.5,0) {$\tau_{c'} = \pi - \beta_1' + \pi - \beta_2'$ \\ $\tau_c = \pi - \beta_1 + \pi - \beta_2
$};
\end{tikzpicture}\end{center}
Es gilt
\begin{align*}
\tau_c &= \pi - \beta_1 + \pi - \beta + \pi - \beta_2 \\
&\ge \pi - (\beta_1' + \alpha_1) + \pi - \beta + \pi - (\beta_2' + \alpha_2) \\
&= \underbrace{\pi - \beta_1' + \pi - \beta_2'}_{\tau_{c'}} + \pi - (\underbrace{\beta + \alpha_1 + \alpha_2}_{\le \overline\beta + \overline{\alpha_1} + \overline{\alpha_2} = \pi}) \\
& \ge \tau_{c'}
\end{align*}

\begin{dfn}[Umfang einer Kurve]
Es sei $c$ eine Kurve.
Der \CmMark{Umfang} $R_c$ von $c$ ist der Radius des kleinsten Balles, der das Bild von $c$ enth"alt:
\begin{align*}
	R_c(t) = \inf \{ R \mid \Bild c|_{[0,t]} \subset \B_R(c(0)) \} = \sup_t \{ d(c(0), c(t)) \}
\end{align*}
\end{dfn}

\begin{satz}[Alexander-Bishop-Ghrist 2010]
Es sei $c$ eine nach Bogenl"ange parametrisierte Kurve in einem $\CAT(0)$-Raum. Dann gilt
\begin{enumerate}[label=(\roman*),leftmargin=*,widest=ii]
\item
	Gilt $\lim\inf_{t\to\infty} \frac{\tau_c(t)}{t} = 0$, so folgt dass $c$ unbeschr"ankt ist.
\item
	Falls $\tau_c(t) \le \text{const} \cdot t^\lambda$ f"ur ein $\lambda  \in (0,1)$ gilt (das hei"st $\tau_c \in O(t^\lambda)$), so folgt $R_c(t) \ge \text{const} \cdot t^{1-\lambda}$ f"ur hinreichend gro"se $t$ (das hei"st $R_c \in \Omega(t^{1-\lambda})$).
\end{enumerate}
\end{satz}

Man kann $c|_{[0,t]}$ durch eine hinreichend feine Polynomapproximation ersetzen:
\begin{center}\begin{tikzpicture}[font=\scriptsize,scale=1.0]
	%\tikzgitter{(-4,-4)}{(5,4)}
	
	\def\len{1.25}
	\draw (-3,-1.5)coordinate(0)node[below]{$c(0)$} --node[left]{$c_1$} ++(60:1.75)coordinate(1) -- ++(10:2.25)coordinate(2) -- ++(350:2)coordinate(3)node[above]{$c(t_1)$} -- ++(0:1.75)coordinate(4) -- ++(45:1)coordinate(5)node[right]{$c(t_2)$} -- ++(90:1)coordinate(6) (0) --node[below]{$\sigma_1$} (3);
	\draw[dashed] (1) -- +(60:\len) (2) -- +(10:\len) (3) -- +(350:\len) (4) -- +(0:\len) (5) -- +(45:\len);
	\fill[gray,opacity=0.3] (0) -- (1) -- (2) -- (3) -- cycle;
	
	\begin{scope}
		\clip (1) -- ($(1)+(60:\len)$) -- (2) -- cycle;
		\draw (1) circle(0.5);
	\end{scope}\begin{scope}
		\clip (2) -- ($(2)+(10:\len)$) -- (3) -- cycle;
		\draw (2) circle(0.5);
	\end{scope}\begin{scope}
		\clip (3) -- ($(3)+(350:\len)$) -- (4) -- cycle;
		\draw (3) circle(0.5);
	\end{scope}\begin{scope}
		\clip (4) -- ($(4)+(0:\len)$) -- (5) -- cycle;
		\draw (4) circle(0.5);
	\end{scope}\begin{scope}
		\clip (5) -- ($(5)+(45:\len)$) -- (6) -- cycle;
		\draw (5) circle(0.5);
	\end{scope}
	
	\node at (-1.5,-0.5) {$\rho$};
\end{tikzpicture}\end{center}
W"ahle eine Partitionierung $0 = t_0 < t_1 < \ldots < t_n = t$ so, dass $c_i = c|_{[t_{i-1}, t_i]}$ eine Totalkr"ummung $\le \frac{\pi}{2}$ hat mit $n \le \frac{\tau_c(t)}{\frac{\pi}{2}} + 1$.
Es bezeichne $\sigma_i$ die Sekante $\overline{c(t_{i-1}) c(t_i)}$ und $\rho_i$ das durch $c_i$ und $\sigma_i$ definierte Polygon.
Jede Kurve (und insbesondere jedes Polygon) in einem $\CAT(\kappa)$-Raum l"asst sich \quot{ausf"ullen}.

\begin{satz}[Rechetnyak 1968]
Es sei $c$ eine geschlossene Kurve in einem $\CAT(\kappa)$-Raum $X$ mit $l(c) < \frac{2\pi}{\sqrt\kappa}$.
Dann existiert ein konvexes Gebiet $C \subseteq M_\kappa^2$ und eine Bogenl"angenparametrisierung $\overline c$ von $\partial C$, sowie eine nicht-expandierende Abbildung $\phi: C \to X$ mit $\phi \circ \overline c = c$.
\end{satz}

\begin{bewSkiz}
Man sieht leicht ein, dass falls $c|_{[a,b]}$ geod"atisch ist, so auch $\overline c|_{[a,b]}$ und dass f"ur Winkel $\beta$, beziehungsweise $\overline\beta$ solcher aufeinanderfolgender Segmente $\beta \le \overline\beta$ gilt.
\end{bewSkiz}

\begin{bew}[zu Alexander-Bishop-Ghrist 2010]\begin{enumerate}[label=(\roman*),leftmargin=*,widest=ii]
\item
	Es existiert zu $\rho_i$ ein konvexes Polygon $\overline\rho_i$ mit den gleichen Seitenl"angen wie $\rho_i$.
	Wie auch $\rho_i$ besteht $\overline\rho_i$ aus einem konvexen Polygon $\overline c_i$ mit der Totalkr"ummung $\tau_{\overline c_i} = \sum \pi - \overline\beta_i \le \sum \pi - \beta_i \le \frac{\pi}{2}$ und eine Sekante $\sigma_i$.
	\begin{center}\begin{tikzpicture}[font=\scriptsize,scale=1.0]
		%\tikzgitter{(-4,-4)}{(5,4)}
		
		\draw (-2,-1)coordinate(0) -- ++(45:1.5)coordinate(1) --node[above]{$\overline c_i$} ++(10:3)coordinate(2) -- ++(330:1.5)coordinate(3);
		\draw[dashed] (0) --node[below]{$\overline\sigma_i$} (3) (1) -- +(45:1)coordinate(1') (2) -- +(10:1)coordinate(2');
		\begin{scope}
			\clip (0) -- (1) -- (2) -- (3) -- cycle;
			\draw (1) circle(0.25)node[xshift=0.3cm,yshift=-0.4cm]{$\overline\beta_1$} (2) circle(0.25)node[xshift=-0.3cm,yshift=-0.4cm]{$\overline\beta_2$};
		\end{scope}
		\begin{scope}
			\clip(1) -- (1') -- (2) -- cycle;
			\draw (1) circle(0.4);
		\end{scope}
		\begin{scope}
			\clip(2) -- (2') -- (3) -- cycle;
			\draw (2) circle(0.4);
		\end{scope}
		\node at (4,0) {$\subseteq \R^2$};
	\end{tikzpicture}\end{center}
	Es gilt $\frac{l(\overline c_i)}{l(\overline\sigma_i)} \le \sqrt 2$, also $\sqrt 2 \ge \frac{l(\overline c_i)}{l(\overline\sigma_i)} = \frac{l(c_i)}{l(\sigma_i)} = \frac{t_i - t_{i-1}}{l(\sigma_i)}$. Damit folgt
	\begin{align*}
		&t = \sum_i t_i - t_{i-1} \le \sqrt 2 \sum_{i=0}^n l(\sigma_i) \le \sqrt 2 \left( \frac{\tau_c(t)}{\frac{\pi}{2}} + 1 \right) \sup(\sigma_i) \shortintertext{also}
		&\frac{\tau_c(t)}{t} \ge \frac{\pi}{2} \left( \frac{1}{\sqrt 2 \sup l(\sigma_i)} - \frac{1}{t} \right).
	\end{align*}
	Ist $c$ beschr"ankt, so folgt $\frac{\tau_c(t)}{t} \ge \text{const} > 0$.
	Sei nun ohne Einschr"ankung $c$ eine Polynomapproximation.
	Zerlege $c$ in $n \le \frac{\tau_c(t)}{\frac{\pi}{2}} + 1$ Teilkurven $c_i$ mit $\tau_{c_i} \le \frac{\pi}{2}$.
	Bezeichnet $\sigma_i$ die Sekante der Teilkurve $c_i$, so gilt
	\begin{align*}
		\frac{t_i - t_{i-1}}{l(\sigma_i)} = \frac{l(c_i)}{l(\sigma_i)} \le \sqrt 2
	\end{align*}
	und es folgt
	\begin{align*}
		t = \sum t_i - t_{i-1} \le \sqrt{2} \sum l(\sigma_i) \le \sqrt{2} \left(\frac{\tau_c(t)}{\frac{\pi}{2}} + 1\right) \sup l(\sigma_i) \tag{*}
	\end{align*}
	und
	\begin{align*}
		\frac{\tau_c(t)}{t} \ge \frac{\pi}{2} \left( \frac{1}{\sqrt{2} \sup l(\sigma_i)} - \frac{1}{t} \right)
	\end{align*}
	Wenn $c$ unbeschr"ankt ist, so ist $l(\sigma_i)$ beschr"ankt.
	Daraus folgt $\frac{\tau_c(t)}{t} \ge \text{const} > 0$ f"ur $t \gg 0$ und damit folgt die Aussage von (i).
\item
	Gilt $\tau_c(t) \le \text{const} \cdot t^\lambda$, so folgt aus (*)
	\begin{align*}
		t \le \sqrt{2} \Big( \underbrace{\frac{\tau_c(t)}{\frac{\pi}{2}}}_{\mathclap{\le \text{const} \cdot t^\lambda}} + 1 \Big) \underbrace{\sup  l(\sigma_i)}_{\le 2 R_c(t)} \le \text{const} \cdot t^\lambda \cdot R_c(t)
	\end{align*}
	Daraus folgt die Behauptung von (ii)
\end{enumerate}\end{bew}

\begin{bew}[zum Beweis des ersten Satzes]
Es wurde bereits gezeigt, dass eine eine erfolgreiche Flucht gilt:
\begin{center}\begin{tikzpicture}[font=\scriptsize,scale=1]
	%\tikzgitter{(-4,-4)}{(5,4)}
	
	\def\d{1.5}
	\draw (-2.5,-0.5)coordinate(p)node[below]{$P_{t-1}$} -- ++(10:\d)coordinate(p')node[below]{$P_{t}$} --node[below]{$d_{t-1}-d$} ++(10:2*\d)coordinate(e)node[right]{$E_{t-1}$} --node[right]{$d$} ++(80:\d)coordinate(e')node[right]{$E_{t}$} --node[above]{$d_{t}$} (p');
	
	\begin{scope}
		\clip (p') -- (e) -- (e') -- cycle;
		\draw (p') circle(1.3);
		\node at ($(p')+(20:1.5)$) {$\alpha_{t}$};
	\end{scope}
	\begin{scope}
		\clip (p) -- (p') -- (e') -- ++(180:3) -- cycle;
		\draw (p') circle(0.4);
		\node at ($(p')+(90:0.6)$) {$\beta_{t}$};
	\end{scope}
	
	\foreach \pkt in {p, p', e, e'}{
		\fill (\pkt)circle(0.05);
	}
\end{tikzpicture}\end{center}
Der Winkel $\alpha_t = \varangle_{P^t}(E^{t-1}, E^t)$ konvergiert f"ur $t \to \infty$ gegen $0$.
Das liefert $\frac{\tau(t)}{t} = \frac{1}{t} \sum \pi - \beta_i = \frac{1}{t} \sum \alpha_i \xrightarrow{t \to \infty} 0$.
Damit folgt aus Satz (i), dass die Verfolgerkurve $P$, und damit $D$ unbeschr"ankt ist.
\end{bew}

\begin{satz}[Alexander-Bishop-Ghrist 2010]
Es sei D ein $\CAT(0)$-Raum.
Dann gilt $\tau_P \le \tau_E + \pi$.
\end{satz}

\begin{bew}
Betrachte die folgende Zeichnung:
\begin{center}\begin{tikzpicture}[font=\scriptsize,scale=1]
	%\tikzgitter{(-4,-4)}{(5,4)}
	
	\def\d{1.5}
	\draw (-2.5,-0.5)coordinate(p)node[below]{$P_{i-1}$} -- ++(10:\d)coordinate(p')node[below]{$P_{i}$} -- ++(10:2*\d)coordinate(e)node[right]{$E_{i-1}$} -- ++(80:\d)coordinate(e')node[right]{$E_{i}$} -- ++(60:\d)coordinate(e'')node[right]{$E_{i+1}$} -- ($(p')!\d cm!(e')$)coordinate(p'')node[above left]{$P_{i+1}$} (p') -- (e');
	
	\begin{scope}
		\clip (p') -- (e) -- (e') -- cycle;
		\draw (p') circle(1.3);
		\node at ($(p')+(20:1.5)$) {$\alpha_{i}$};
		\draw (e) circle(0.3) node[xshift=-0.4cm,yshift=0.2cm]{$\gamma_i$};
		\draw (e') circle(0.3) node[xshift=-0.25cm,yshift=-0.35cm]{$\delta_i$};
	\end{scope}
	\begin{scope}
		\clip (p) -- (p') -- (e') -- ++(180:3) -- cycle;
		\draw (p') circle(0.4);
		\node at ($(p')+(90:0.6)$) {$\beta_{t}$};
	\end{scope}
	\begin{scope}
		\clip (e) -- (e') -- (e'') -- ++(330:2) -- cycle;
		\draw (e') circle(0.5)node[xshift=0.65cm,yshift=0.2cm]{$\theta_i$};
	\end{scope}
	
	\foreach \pkt in {p, p',p'', e, e',e''}{
		\fill (\pkt)circle(0.05);
	}
	
	\node[align=left,font=\normalsize] at (4.5,0.5) {$\alpha_i + \gamma_i + \delta_i \le \pi$ \\ $\theta_i \le \gamma_{i+1} + \delta_i$};
\end{tikzpicture}\end{center}
Es gilt
\begin{align*}
	\tau_P(nd) &= \sum_{i=1}^{n-1} \pi - \beta_i \le \sum_{i=1}^{n-1} \alpha_i \le \sum_{i=1}^{n-1} \pi - \gamma_i - \delta_i \\
	\tau_E(nd) &= \sum_{i=1}^{n-1} \pi - \theta_i \ge \sum_{i=1}^{n-1} \pi - \gamma_{i+1} - \delta_i \\
	\tau_P - \tau_E(nd) &= \sum_{i=1}^{n-1} (\pi - \gamma_i - \delta_i) - (\pi - \gamma_{i+1} - \delta_i) \\
	&= \sum_{i=1}^{n-1} \gamma_{i+1} - \gamma_i = \gamma_n - \gamma_1 \le \pi
\end{align*}
\end{bew}

\section{Positive Kr\"ummungsschranken}

Im Folgenden Sei $D$ stets ein $\CAT(1)$-Raum.
Beispiele f"ur solche R"aume sind die Einheitssph"are $\S^n$, die euklidische Ebene mit ausgeschnittenem offenen Ball $\R^2 \setminus \B_1^o$ oder verallgemeinert $\R^n \setminus \dot\bigcup \B_1^o(p_i)$

\begin{satz}[Alexander-Bishop-Ghrist 2010]
Es sei $D$ ein $\CAT(1)$-Raum und $E_t$ eine erfolgreiche Fluchtstrategie mit $d_0 = d(E_0, P_0) < \pi$.
Dann gilt
\begin{align*}
	\tau_p(t) \le \const \cdot \sqrt{t}.
\end{align*}
\end{satz}

\begin{bewSkiz}
Betrachte die Zeichnung
\begin{center}\begin{tikzpicture}[font=\scriptsize,scale=1]%\tikzgitter{(-4,-4)}{(5,4)}
	
	\def\d{1.5}
	\draw (-2.5,-0.5)coordinate(p)node[below]{$P_{i-1}$} to[out=5,in=200] node[below,pos=0.8]{$d_{i-1}-d$} coordinate[pos=0.35](p') ++(10:2.5*\d)coordinate(e)node[right]{$E_{i-1}$} to[out=65,in=270]node[right]{$d$} ++(80:\d)coordinate(e')node[right]{$E_{i}$} to[out=200,in=60]node[above]{$d_{i}$} (p');
	
	\begin{scope}
		\clip (-2.5,-0.5) to[out=5,in=200] ++(10:2.5*\d) to[out=65,in=270] ++(80:\d) to[out=200,in=60] (p');
		\draw (p') circle(0.8);
		\node at ($(p')+(25:1.0)$) {$\alpha_{i}$};
	\end{scope}
	\begin{scope}
		\clip (e') to[out=200,in=60] (p') -- (p) -- ++(90:3) -- cycle;
		\draw (p') circle(0.3);
		\node at ($(p')+(110:0.5)$) {$\beta_{i}$};
	\end{scope}
	
	\foreach \pkt in {p,p',e, e'}{
		\fill (\pkt)circle(0.05);
	}
\end{tikzpicture}\end{center}
Aus der Dreiecksungleichung folgt $d_i \le (d_{i-1} - d) + d = d_{i-1}$, also ist $d_i$ monoton fallend.
Es gilt
\begin{align*}
	d_i \le d_0 < \pi
\end{align*}
Damit hat jedes Dreieck $\Delta P_iE_{i-1}E_i$ den Umfang $<2\pi$ und besitzt damit ein eindeutiges Vergleichsdreieck in $\S^2$.
Damit gilt f"ur den Vergleichswinkel $\overline\alpha_i$ dass $\alpha_i \le \overline\alpha_i$ und es gilt
\begin{align*}
	\tau_p(t) = \sum \pi - \beta_i \le \sum \alpha_i \le \sum \overline\alpha_i.
\end{align*}
Falls $E_i$ eine erfolgreiche Fluchtstrategie ist, gilt $d_\infty = \lim_{i \to \infty} d_i > d$.
Es sei $\delta_i = d_i - d_{i-1} \ge 0$.
Wendet man den sph"arischen Kosinussatz auf das Vergleichsdreieck an, so folgt
\begin{align*}
	\cos d \le \cos d + \delta_i \sin d - \const \cdot \overline\alpha_i^2
\end{align*}
und daraus folgt $\overline\alpha_i^2 \le \const \cdot \delta_i \cdot d$.
Aus der Cauchy-Schwarz-Ungleichung folgt
\begin{align*}
	\tau_p(nd) &\le \sum \overline\alpha_i \le \sqrt{(n-1) \sum \overline\alpha_i^2} \\
	&\le \sqrt{(n-1) \cdot \const \cdot d \cdot (d_0 - d_\infty)} \\
	&\le \const \sqrt{nd}
\end{align*}
\end{bewSkiz}

\begin{bem}[zu $d_0 < \pi$]
Betrachte das folgende Szenario in $\R^2 \setminus \B_{\frac{3}{2}}(0)$
\begin{center}\begin{tikzpicture}[font=\scriptsize,scale=1]%\tikzgitter{(-4,-4)}{(5,4)}
	
	\def\r{1.25}
	\def\angle{60}
	\draw (0,0) circle(\r);
	\draw[very thick] (90:\r)node[above]{$E_0=E_2=\ldots$} arc[radius=\r,start angle=90,end angle=90+\angle]node[left]{$\ldots=E_3=E_1$} (270:\r)node[below]{$P_0=P_2=\ldots$} arc[radius=\r,start angle=270,end angle=270+\angle]node[right]{$P_1=P_3=\ldots$};
	\foreach \pkt in {(90:\r),(90+\angle:\r),(270:\r),(270+\angle:\r)}{
		\fill \pkt circle(0.05);
	}
\end{tikzpicture}\end{center}
$E$ und $P$ starten an gegen"uberliegenden Polen.
Im ersten Schritt l"auft $E$ den Kreis entlang nach links, w"ahrend P nach rechts l"auft.
Da $P$ jedes mal eine Geod"atische w"ahlt und es nun zwei kürzeste Wege gibt kann $P$ diese ungeschickte Wahl treffen.
In zweiten Schritt l"auft $E$ wieder zurück, w"ahrend $P$ erneut die ungl"uckliche Wahl trifft und ebenfalls zur"uckl"auft.
$E$ und $P$ wiederholen diese Prozedur immer wieder und treffen sich niemals, weshalb $\tau_P(t)$ in diesem Szenario linear w"achst.
\end{bem}

\begin{kor}
Ist $D$ ein $\CAT(0)$-Raum und $E_t$ erfolgreich, so gilt $\tau_P(t) \le \const \sqrt{t}$.
\end{kor}

\begin{bew}
Winkel sind invariant unter Skalierungen.
Damit  sind die $\CAT(0)$ Bedingung und die Totalkr"ummung invariant.
Sei $\tilde{d}(\cdot, \cdot) = \frac{\pi}{2 d_0} d(\cdot, \cdot)$ die skalierte Metrik, dann ist $\tilde{d}_0 = \frac{\pi}{2} < \pi$.
Der Rest folgt aus dem Satz.
\end{bew}

\begin{kor}
Unter denselben Voraussetzungen wie in obigem Korollar gilt $R_E(t) \ge \const$.
\end{kor}

\begin{bew}
Es gilt $R_P(t) \ge \const \sqrt{t}$ nach dem obigen Korollar und Satz (ii).
Ferner gilt
\begin{align*}
	d(E_0, E_t) &\ge d(P_0, P_t) - d(P_t, E_t) - d(E_0,P_0) \\
	&\ge d(P_0, P_t) - \const
\end{align*}
Daraus folgt
\begin{align*}
	R_E(t) = \sup \{ d(E_0, E_t) \} \ge \ldots \ge \const \cdot R_P(t) \ge \const \sqrt t
\end{align*}
\end{bew}

%-------------------------- Kapitel 2 --------------------------
% Vorlesung vom 21. 5.

\chapter{Bewegungsplanung \& Konfigurationsr\"aume}

Als Motivation f"ur das folgende Kapitel k"onnen wir uns beispielsweise die Bewegungssteuerung gleichartiger Automaten, beziehungsweise Fahrzeuge, in einer Umgebung, etwa in der euklidischen Ebene $\R^2$ oder der Ebene $\R^2 \setminus \bigcup_{i \le k} I_i$ mit Hindernissen, anschauen.

Unser Ziel ist es, bestimmte Positionen oder Zyklen von Bewegungen, zum Beispiel zur Steuerung von Fahrzeugen in einer Lagerhalle, zu finden.
Als Einschr"ankung w"ahlen wir dabei das Umfahren von Hindernissen.

Wir modellieren das Problem, so dass $x_i$ die Position des $i$-ten Automaten mit $x_i \notin O_i$ beschreibt, und Kollisionsfreiheit, das hei"st $x_i \ne x_j$ f"ur $i \ne j$.
\marginnote{\begin{tikzpicture}[font=\scriptsize,scale=0.8]
	%\tikzgitter{(-4,-4)}{(5,4)}
	\draw (-1.75,0.5) -- (-0.5,0.5) -- (-0.5,1.75) (-1.75,-0.5) -- (-0.5,-0.5) -- (-0.5,-1.75) (1.75,0.5) -- (0.5,0.5) -- (0.5,1.75) (1.75,-0.5) -- (0.5,-0.5) -- (0.5,-1.75);
	\fill[gray,opacity=0.5] (-1.75,0.5) -- (-0.5,0.5) -- (-0.5,1.75) -- (-1.75,1.75) (1.75,0.5) -- (0.5,0.5) -- (0.5,1.75) -- (1.75,1.75) (-1.75,-0.5) -- (-0.5,-0.5) -- (-0.5,-1.75) -- (-1.75,-1.75) (1.75,-0.5) -- (0.5,-0.5) -- (0.5,-1.75) -- (1.75,-1.75);
	\def\rad{0.2}
	\def\x{0.15}
	\def\y{1.2}
	\draw[->] (-\x,\y) -- (-\x,\rad) arc[radius=\rad,start angle=90,end angle=270] -- (-\x,-\y);
	\draw[->] (\x,-\y) -- (\x,-\rad) arc[radius=\rad,start angle=270,end angle=450] -- (\x,\y);
	\fill[fill=gray] (-\x,\y) circle[radius=\rad] (\x,-\y) circle[radius=\rad];
\end{tikzpicture}\\
\scriptsize{lokale Kollisionsvermeidung}}
Wir setzen
\begin{align*}
	X = \times_{i=1}^{n} (\R^2 \setminus \bigcup_k O_k ) \setminus \Delta && \Delta = \{(x_1, \ldots, x_n) \in \R^{2n} \mid x_i = x_j, i \ne j \}
\end{align*}

Wir werden uns auf Graphen beschr"anken, das ist zum Einen motiviert durch Anwendung (Schienensysteme), und zum Anderen durch Methodik (lokale Kollisionsvermeidung).

Es sei $X$ ein topologischer Raum. Die \CmMark[Diagonale!verallgemeinerte]{verallgemeinerte Diagonale} von $X^n$ ist $\Delta = \Delta_n(X) = \{(x_1, \ldots, x_n) \in X^n \mid \exists i \ne j : x_i = x_j \}$.
Der Raum
\begin{align*}
	F_NX = X^n \setminus \Delta_n(X)
\end{align*}
hei"st der \CmMark{Konfigurationsraum} von $X$.

\begin{bsp}
$F_2\R^2 = \R^3 \X \S^1$ (Beweis zur "Ubung)
\end{bsp}

Die symmetrische Fruppe $S_n$ wirkt frei auf $F_nX$.
Dann hei"st der Quotient $C_nX = \FakRaum{F_nX}{S_n}$ der \CmMark[Konfigurationsraum!ungeordneter]{(ungeordnete) Konfigurationsraum} von $X$.
Die Fundamentalgruppe $\pi_1(C_nX)$ hei"st die $n$-te \CmMark{Zopfgruppe} von $X$.
Die klassischen Zopfgruppen sind $B_n = \pi_1(C_n\R^2)$:
\begin{center}\begin{tikzpicture}[font=\scriptsize,scale=0.8]
	%\tikzgitter{(-4,-4)}{(5,4)}
	\def\angle{0}
	\def\Dist{2}
	\def\dist{2}
	\def\hor{0.75}
	
	\coordinate (1) at (0,1.5); \coordinate (2) at ($(1)-(0,\Dist)$);
	\coordinate (x1) at ($(1)+(0.25,-0.75)$); \coordinate (x2) at ($(x1)+(\hor,0)$); \coordinate (x3) at ($(x2)+(\hor,0)$);
	\coordinate (x4) at ($(x3)+(\hor,0)$); \coordinate (x5) at ($(x4)+(\hor,0)$); \coordinate (x6) at ($(x5)+(\hor,0)$);
	\foreach \i in {1,2,3,4,5,6}{
		\coordinate (x\i') at ($(x\i)-(0,\dist)$);
	}
	
	\draw (x2) to[out=270+\angle,in=90+\angle] (x1');
	\draw[preaction={draw=white,line width=0.1cm}] (x1) to[out=270+\angle,in=90+\angle] (x2');
	\draw (x3) -- (x3');
	\draw (x6) to[out=270+\angle,in=90+\angle] (x4');
	\draw[preaction={draw=white,line width=0.1cm}] (x4) to[out=270+\angle,in=90+\angle] (x5') (x5) to[out=270+\angle,in=90+\angle] (x6');
	
	\draw (1) -- ++(-1,-1.5) -- ++(6,0) (2) -- ++(-1,-1.5) -- ++(6,0);
	
	\coordinate (temp) at (x1'); \coordinate (x1') at (x2'); \coordinate (x2') at (temp);
	\foreach \i in {1,2,3}{
		\fill (x\i) circle(0.05) node[above]{$x_{\i}$};
		\fill (x\i') circle(0.05) node[below]{$x_{\i}$};
	}
	\foreach \i in {4,5,6}{
		\fill (x\i) circle(0.05);
		\fill (x\i') circle(0.05);
	}
	\fill (x6) circle(0.05) node[above]{$x_{n}$};
\end{tikzpicture}\end{center}
\begin{align*}
	[x_1, x_2, \ldots, x_n] = [x_2, x_1, \ldots, x_n]
\end{align*}
Darstellung von $B_1$:
\begin{center}\begin{tikzpicture}[font=\scriptsize,scale=1,baseline=0,scale=0.8]
	%\tikzgitter{(-4,-4)}{(5,4)}
	\def\angle{0}
	\def\Dist{2}
	\def\dist{2}
	\def\hor{0.75}
	
	\coordinate (1) at (0,1.5); \coordinate (2) at ($(1)-(0,\Dist)$);
	\coordinate (x1) at ($(1)+(0.25,-0.75)$); \coordinate (x2) at ($(x1)+(2*\hor,0)$); \coordinate (x3) at ($(x2)+(\hor,0)$);
	\node at ($(x1)+(\hor,0)$) {$\cdots$}; \node at ($(x3)+(\hor,0)$) {$\cdots$};
	\foreach \i in {1,2,3}{
		\coordinate (x\i') at ($(x\i)-(0,\dist)$);
	}
	
	\draw (x1)node[above]{$x_{1}$} -- (x1')node[below]{$x_{1}$};
	\draw (x3)node[above]{$x_{i+1}$} to[out=270+\angle,in=90+\angle] (x2');
	\draw[preaction={draw=white,line width=0.1cm}] (x2)node[above]{$x_{i}$} to[out=270+\angle,in=90+\angle] (x3');

	\draw (1) -- ++(-1,-1.5) -- ++(6,0) (2) -- ++(-1,-1.5) -- ++(6,0);
	
	\foreach \i in {1,2,3}{
		\fill (x\i) circle(0.05);
		\fill (x\i') circle(0.05);
	}
\end{tikzpicture}\end{center}
Erzeuger: $g_i$, $i \le n-1$, Relationen:
\begin{itemize}
\item
	$g_i g_j = g_j g_i$, $|i-j| \ge 2$
	\begin{center}\begin{tikzpicture}[font=\scriptsize,scale=1,baseline=0]
			%\tikzgitter{(-4,-4)}{(5,4)}
			\def\angle{0}
			\def\x{0.25}
			\def\y{0.75}
			\coordinate (1) at (-\x,\y); \coordinate (2) at (\x,0); \coordinate (3) at (\x,\y); \coordinate (4) at (-\x,0);
			\draw (3) to[out=270+\angle,in=90+\angle] (4) -- ++(0,-\y);
			\draw[preaction={draw=white,line width=0.1cm}] (1)node[left]{$i$} to[out=270+\angle,in=90+\angle] (2) -- ++(0,-\y);
		\end{tikzpicture}
		$\cdots$
		\begin{tikzpicture}[font=\scriptsize,scale=1,baseline=0]
			%\tikzgitter{(-4,-4)}{(5,4)}
			\def\angle{0}
			\def\x{0.25}
			\def\y{0.75}
			\coordinate (1) at (-\x,-\y); \coordinate (2) at (\x,0); \coordinate (3) at (\x,-\y); \coordinate (4) at (-\x,0);
			\draw[preaction={draw=white,line width=0.1cm}] (1) to[out=90+\angle,in=270+\angle] (2) -- ++(0,\y);
			\draw[preaction={draw=white,line width=0.1cm}] (3) to[out=90+\angle,in=270+\angle] (4) -- ++(0,\y)node[left]{$j$};
		\end{tikzpicture}
		=
		\begin{tikzpicture}[font=\scriptsize,scale=1,baseline=0]
			%\tikzgitter{(-4,-4)}{(5,4)}
			\def\angle{0}
			\def\x{0.25}
			\def\y{0.75}
			\coordinate (1) at (-\x,-\y); \coordinate (2) at (\x,0); \coordinate (3) at (\x,-\y); \coordinate (4) at (-\x,0);
			\draw[preaction={draw=white,line width=0.1cm}] (1) to[out=90+\angle,in=270+\angle] (2) -- ++(0,\y);
			\draw[preaction={draw=white,line width=0.1cm}] (3) to[out=90+\angle,in=270+\angle] (4) -- ++(0,\y)node[left]{$i$};
		\end{tikzpicture}
		$\cdots$
		\begin{tikzpicture}[font=\scriptsize,scale=1,baseline=0]
			%\tikzgitter{(-4,-4)}{(5,4)}
			\def\angle{0}
			\def\x{0.25}
			\def\y{0.75}
			\coordinate (1) at (-\x,\y); \coordinate (2) at (\x,0); \coordinate (3) at (\x,\y); \coordinate (4) at (-\x,0);
			\draw (3) to[out=270+\angle,in=90+\angle] (4) -- ++(0,-\y);
			\draw[preaction={draw=white,line width=0.1cm}] (1)node[left]{$j$} to[out=270+\angle,in=90+\angle] (2) -- ++(0,-\y);
		\end{tikzpicture}
	\end{center}
\item
	$g_i g_{i+1} g_i = g_{i+1} g_i g_{i+1}$
	\begin{center}
	\begin{tikzpicture}[font=\scriptsize,scale=1,baseline=-0.5cm]
		%\tikzgitter{(-4,-4)}{(5,4)}
		\def\angle{0}
		\def\x{0.75}
		\def\y{1}
			
		\coordinate (1) at (-\x,\y); \coordinate (2) at (-\x,0); \coordinate (3) at (-\x,-\y); \coordinate (4) at (-\x,-2*\y);
		\coordinate (6) at (0,\y); \coordinate (7) at (0,0); \coordinate (8) at (0,-\y); \coordinate (9) at (0,-2*\y);
		\coordinate (11) at (\x,\y); \coordinate (12) at (\x,0); \coordinate (13) at (\x,-\y); \coordinate (14) at (\x,-2*\y);
		
		\draw[preaction={draw=white,line width=0.1cm}] (11) to[out=270+\angle,in=90+\angle] (12) to[out=270+\angle,in=90+\angle] (8) to[out=270+\angle,in=90+\angle] (4);
		\draw[preaction={draw=white,line width=0.1cm}] (6)node[above]{$i+1$} to[out=270+\angle,in=90+\angle] (2) to[out=270+\angle,in=90+\angle] (3) to[out=270+\angle,in=90+\angle] (9);
		\draw[preaction={draw=white,line width=0.1cm}] (1)node[above]{$i$} to[out=270+\angle,in=90+\angle] (7) to[out=270+\angle,in=90+\angle] (13) to[out=270+\angle,in=90+\angle] (14);
	\end{tikzpicture}
	=
	\begin{tikzpicture}[font=\scriptsize,scale=1,baseline=-0.5cm]
		%\tikzgitter{(-4,-4)}{(5,4)}
		\def\angle{0}
		\def\x{0.75}
		\def\y{1}
			
		\coordinate (1) at (-\x,\y); \coordinate (2) at (-\x,0); \coordinate (3) at (-\x,-\y); \coordinate (4) at (-\x,-2*\y);
		\coordinate (6) at (0,\y); \coordinate (7) at (0,0); \coordinate (8) at (0,-\y); \coordinate (9) at (0,-2*\y);
		\coordinate (11) at (\x,\y); \coordinate (12) at (\x,0); \coordinate (13) at (\x,-\y); \coordinate (14) at (\x,-2*\y);
		
		\draw[preaction={draw=white,line width=0.1cm}] (11) to[out=270+\angle,in=90+\angle] (7) to[out=270+\angle,in=90+\angle] (3) to[out=270+\angle,in=90+\angle] (4);
		\draw[preaction={draw=white,line width=0.1cm}] (6)node[above]{$i+1$} to[out=270+\angle,in=90+\angle] (12) to[out=270+\angle,in=90+\angle] (13) to[out=270+\angle,in=90+\angle] (9);
		\draw[preaction={draw=white,line width=0.1cm}] (1) to[out=270+\angle,in=90+\angle] (2) to[out=270+\angle,in=90+\angle] (8) to[out=270+\angle,in=90+\angle] (14);
	\end{tikzpicture}
	\end{center}
\end{itemize}

Ein \CmMark{Graph} $\calG$ besteht aus einer nichtleeren \CmMark[Ecke]{Eckenmenge} $\calV = \calV\calG$, einer \CmMark[Kante]{Kantenmenge} $\calE = \calG\calE$ und (surjektiven) \CmMark[Randabbildung]{Randabbildungen} $\partial^{\pm}: \calE \to \calV$.
\marginnote{\begin{tikzpicture}[font=\scriptsize,scale=0.8]
	\draw[decoration={markings,mark=at position 0.6 with{\arrow{>}}},postaction={decorate}] (0,0)coordinate(v)node[below]{$v=\partial^-e$} --node[below,pos=0.6]{$e$} (30:2)coordinate(w)node[above]{$w=\partial^+e$};
	\fill (v)circle[radius=0.05] (w)circle[radius=0.05];
\end{tikzpicture}}
Zur Notation der \CmMark[Kante!Orientierung]{Kantenorientierung} sei $\tilde{\E} = \E \X \{+1, -1\}$ mit den Randabbildungen
\begin{align*}
	e^+ = (e,1) \mapsto \partial^\pm e && \text{und} && e^- = (e,-1) \mapsto \partial^\mp e.
\end{align*}
Dann gilt f"ur die Kanteninversion $\overline{e}^{\pm} \to e^{\mp}$, $\partial^+ \overline{e} = \partial^- e$ und $\partial^- \overline e = \partial^+ e$.
Es bezeichne $|e|$ stets die unorientierte Kante $e$.

\begin{bsp}
$Y = $
\begin{tikzpicture}[font=\scriptsize,scale=0.75,baseline=0.5cm]
	%\tikzgitter{(-4,-4)}{(5,4)}
	\def\rad{1.5}
	
	\coordinate (v) at (0,0); \coordinate (w2) at ($(v)+(30:\rad)$); \coordinate (w3) at ($(v)+(150:\rad)$); \coordinate (w1) at ($(v)+(270:\rad)$);
	
	\draw (v)node[below right]{$v$} --node[right]{$e_1$} (w1)node[below]{$w_1$} (v) --node[right]{$e_2$} (w2)node[above right]{$w_2$} (v) --node[left]{$e_3$} (w3)node[above left]{$w_3$};
	\foreach \pkt in {v,w1,w2,w3}{
		\fill (\pkt) circle(0.05);
	}
\end{tikzpicture}
,
\begin{tikzpicture}[font=\scriptsize,scale=1,baseline=1.15cm]
	%\tikzgitter{(-4,-4)}{(5,4)}
	
	\coordinate (v) at (0,0); \coordinate (w) at (2,1);
	\draw[decoration={markings,mark=at position 0.5 with{\arrow{<}}},postaction={decorate}] (v) to[out=50,in=190] (w) ;
	\draw[decoration={markings,mark=at position 0.6 with{\arrow{<}}},postaction={decorate}] (v) to[out=70,in=170] (w);
	\draw[decoration={markings,mark=at position 0.6 with{\arrow{>}}},postaction={decorate}] (v) to[out=100,in=140]node[above,pos=0.6]{$e_n$} (w);
	\draw[decoration={markings,mark=at position 0.6 with{\arrow{>}}},postaction={decorate}] (v) to[out=20,in=210]node[below]{$e_2$} (w);
	\draw[decoration={markings,mark=at position 0.6 with{\arrow{>}}},postaction={decorate}] (v) to[out=355,in=250]node[below,pos=0.6]{$e_1$} (w);
	\fill (v)circle(0.05)node[below]{$v$} (w)circle(0.05)node[right]{$w$};
\end{tikzpicture}
,
$Q =$
\begin{tikzpicture}[font=\scriptsize,scale=1,baseline=0.5cm]
	%\tikzgitter{(-4,-4)}{(5,4)}
	
	\coordinate (v) at (0,0); \coordinate (w1) at (0,-0.75);, \coordinate (w2) at (0,1);
	\draw (0,0.5) circle[radius=0.5]node[right,xshift=0.5cm]{$e_2$} node[left,xshift=-0.5cm]{$e_3$} (v) -- (w1);
	
	\fill (v)circle(0.05)node[below right]{$v$} (w1)circle(0.05)node[below]{$w_1$} (w2)circle(0.05)node[above]{$w_2$};
\end{tikzpicture}
,
\begin{tikzpicture}[font=\scriptsize,scale=1,baseline=0.25cm]
	%\tikzgitter{(-4,-4)}{(5,4)}
	
	\coordinate (v) at (0,0);
	
	\draw (v) ..controls(0:0.5) and (0.5,-1).. (0.25,-1)node[right]{$e_1$} ..controls(-0.25,-1) and (-0.5,0).. (v);
	\draw (v) ..controls(135:0.3) and (0.5,1).. (0.75,0.75)node[right]{$e_2$} ..controls(1,0.5) and (315:0.2).. (v);
	\draw (v) ..controls(-0.5,-0.25) and (-1,0.5).. (-0.75,0.75)node[above]{$e_3$} ..controls(-0.5,1) and (0,0.25).. (v);
	\fill (v)circle(0.05)node[above right]{$v$};
\end{tikzpicture}
\end{bsp}

Der \CmMark[Grad!einer Ecke]{Grad} $\deg v$ einer Ecke ist die Zahl, wie oft $v$ als Ecke einer Kante vorkommt.
Eine Ecke mit Grad $\ge 3$ hei"st \CmMark[Ecke!essentielle]{essentiell}, mit Grad 1 hei"st sie \CmMark[Ecke!freie]{frei}.

Die geometrische Realisierung $|\G|$ eines Graphen $\G$ erh"alt man durch das Verkleben von (Einheits-) Intervallen zu jeder Kante entsprechender Randabbildungen.
Man kann sich, aus topologischer Sicht, auf Graphen beschr"anken, deren Ecke entweder frei oder essentiell ist.
Jede essentielle Ecke $v$ besitzt eine Umgebung $U$, so dass $U \setminus \{v\}$ in $\G$ in mindestens drei Zusammenhangskomponenten zerf"allt.

\begin{bsp}
Wir betrachten nun speziell $F_2Y$.
Die Position zweier Punkte $x$ und $y$ auf benachbarten Kanten ist durch ein Einheitsquadrat parametrisiert.
Die Achsen sind f"ur ihre jeweiligen Punkte zugelassene Positionen, solange nicht beide Punkte null sind, da es dann zu einer Kollision k"ame.
\begin{center}\begin{tikzpicture}[font=\scriptsize,scale=1]
	%\tikzgitter{(-7,-2)}{(2,2)}
	\def\rad{1.5}
	
	\coordinate (v) at (-6,1); \coordinate (w2) at ($(v)+(30:\rad)$); \coordinate (w3) at ($(v)+(150:\rad)$); \coordinate (w1) at ($(v)+(270:\rad)$);
	\draw (v)node[below right]{$v$} --coordinate[pos=0.5](y)node[pos=0.5,right]{$y$} (w1) (v) --coordinate[pos=0.5](x)node[pos=0.5,right]{$x$} (w2) (v) -- (w3);
	\foreach \pkt in {v,x,y}{
		\fill (\pkt) circle(0.05);
	}
	
	\coordinate (v) at (-3,1); \coordinate (w2) at ($(v)+(30:\rad)$); \coordinate (w3) at ($(v)+(150:\rad)$); \coordinate (w1) at ($(v)+(270:\rad)$);
	\draw (v)node[below right]{$v = x$} --coordinate[pos=0.5](y)node[pos=0.5,right]{$y$} (w1) (v) -- (w2) (v) -- (w3);
	\foreach \pkt in {v,y}{
		\fill (\pkt) circle(0.05);
	}
	
	\draw[->] (-0.5,0) -- (2,0); \draw[->] (0,-0.5) -- (0,2);
	\draw[very thick] (0,0) -- (1.5,0) (0,0) --node[left,pos=0.4]{$x \equiv 0 = v$} (0,1.5);
	\filldraw[fill=gray,fill opacity=0.3] (0,0)node[below left,opacity=1]{$0$} -- (0,1.5)node[left,opacity=1]{$1$} -- (1.5,1.5) -- (1.5,0)node[below,opacity=1]{$1$};
	\def\x{1}
	\def\y{1.2}
	\draw[dashed] (\x,0)node[below]{$x$} -- (\x,1.5) (1.5,\y) -- (0,\y)node[left]{$y$};
	\fill (\x,\y) circle(0.05);
	\filldraw[fill=white] (0,0)circle(0.05);
	\draw[->] (1.5,-1)node[right]{$y \equiv 0 = v$} to[out=180,in=270] (0.5,-0.1);
\end{tikzpicture}\end{center}
Sechs dieser Parameterbereiche werden entlang der Kanten $x \equiv 0 = v$ beziehungsweise $y \equiv 0 = v$ miteinander verklebt
\begin{center}\begin{tikzpicture}[font=\scriptsize,scale=1]
	%\tikzgitter{(-4,-4)}{(4,4)}
	\def\angle{55}
	\def\r{1.25}
	\def\R{2.5}
	\coordinate (o) at (0,0);
	
	\draw[very thick] (o) -- (0:\R)coordinate(1) (o) -- (\angle:\r)coordinate(2) (o) -- (180-\angle:\r)coordinate(3) (o) -- (180:\R)coordinate(4) (o) -- (180+\angle:\r)coordinate(5) (o) -- (360-\angle:\r)coordinate(6);
	\draw (0:\R) -- ++(\angle:\r) -- (\angle:\r) -- ++(180-\angle:\r) -- (180-\angle:\r) -- ++(180:\R) -- (180:\R) -- ++(180+\angle:\r) -- ++ (0:\R) -- ++(360-\angle:\r) -- ++(\angle:\r) -- ++ (0:\R) -- cycle;
	
	\foreach \i in {1,2,...,6} {
		\draw (\i) -- ++(0,1.25) coordinate(pkt);
		\draw[dashed] (pkt) -- (o);
	}
	
	\filldraw[fill=white] (o) circle(0.05);
\end{tikzpicture}\end{center}
Liegen zwei Punkte $x$ und $y$ auf einer Kante, so sind ihre Koordinaten durch ein \quot{halbes} Quadrat parametrisiert.
\begin{center}\begin{tikzpicture}[font=\scriptsize,scale=1]
	%\tikzgitter{(-5,-2)}{(4,2)}
	\def\rad{1.5}
	
	\coordinate (v) at (-3,1); \coordinate (w2) at ($(v)+(30:\rad)$); \coordinate (w3) at ($(v)+(150:\rad)$); \coordinate (w1) at ($(v)+(270:\rad)$);
	\draw (v)node[below left]{$v$} -- (w1) (v) --coordinate[pos=0.2](y)node[pos=0.2,below right]{$y$} coordinate[pos=0.7](x)node[pos=0.7,below right]{$x$} (w2) (v) -- (w3);
	\foreach \pkt in {v,x,y}{
		\fill (\pkt) circle(0.05);
	}
	
	\draw[->] (-0.5,0) -- (2,0); \draw[->] (0,-0.5) -- (0,2);
	\draw[very thick] (0,0) -- (1.5,0);
	\draw (1.5,1.5) -- (1.5,0)node[below,opacity=1]{$1$};
	\def\x{1.25}
	\def\y{0.3}
	\draw[dashed] (0,0) -- (1.5,1.5) (\x,0)node[below]{$x$} -- (\x,\x) -- (0,\x)node[left]{$x$} (0,\y)node[left]{$y$} -- (\x,\y);
	\fill (\x,\y) circle(0.05);
	\filldraw[fill=white] (0,0)circle(0.05) node[below left]{$0$};
\end{tikzpicture}\end{center}
F"ur $C_2Y$ ergibt sich dann
\begin{center}\begin{tikzpicture}[font=\scriptsize,scale=1]
	%\tikzgitter{(-4,-4)}{(4,4)}
	\def\angle{55}
	\def\x{2}
	\def\y{1}
	\coordinate (o) at (0,0);
	\coordinate (w1) at (\x,0); \coordinate (w2) at (-\y,\y); \coordinate (w3) at (-\y,-\y);
	
	\foreach \i in {1,2,3} {
		\draw (o) -- (w\i);
	}
	\draw (w1) -- ++(-\y,\y) -- (w2) -- ++(-\y,-\y) -- (w3) -- ++(\x,0) -- cycle;
	\foreach \i in {1,2,3} {
		\draw (w\i) -- ++(0,1.25) coordinate(pkt);
		\draw[dashed] (pkt) -- (o);
	}
	
	\filldraw[fill=white] (o) circle(0.05);
\end{tikzpicture}\end{center}
\end{bsp}

\begin{satz}[Swiatowski 2001]
Es sei $\G$ ein endlicher Graph.
Dann existiert ein nichtpositiv gekr"ummter-Kuben"-kom"-plex $K_n\G \hookrightarrow C_n\G = \FakRaum{F_n\calG}{S_n}$, sodass $K_n\calG$ ein Deformationsretrakt von $C_n\G$ ist.
Es gilt $\dim K_n\calG = \min \{n, b\}$, wobei $b$ die Anzahl der essentiellen Ecken von $\G$ ist.
\end{satz}

% Vorlesung vom 21. 5.

Die Fundamentalgruppe $\pi_1(C_n\G)$ enth"alt eine frei abelsche Untergruppe vom Rang $\min\{b, \lfloor \frac{n}{2} \rfloor\}$.
Die Konstruktion basiert darauf, die Verteilung von Punkten auf Ecken, beziehungsweise Kanten, beziehungsweise die "Uberg"ange zwischen solchen Zust"anden zu beschreiben.

\begin{emptythm}[Ansatz (Abrams 2000)]
Wie erh"alt man die Zellenstruktur beziehungsweise die Kombinatorik von $|\G|^n = \{ \sigma = \sigma_1 \X \ldots \X \sigma_n \mid \sigma \text{ Zelle von } \G \}$ in $C_n\G$?
Man erh"alt $F_n\G$ durch das \quot{L"oschen} der verallgemeinerten Diagonale.
\begin{center}\begin{tikzpicture}[font=\scriptsize,scale=1,baseline=1.5cm]
	%\tikzgitter{(-4,-4)}{(5,4)}
	\def\len{1}
	
	\foreach \i in {0, 1, 2, 3}{
		\draw (0,\i*\len) -- (3*\len,\i*\len) (\i*\len,0) -- (\i*\len,3*\len);
	}
	\fill[gray,opacity=0.3] (2*\len,0) rectangle (3*\len,\len) (0,3*\len) rectangle (\len,2*\len);
	\draw[dashed] (0,0) -- (3*\len,3*\len);
	\foreach \x / \y in {1/0,2/0,3/0,2/1,3/1,3/2}{
		\fill (\x*\len,\y*\len) circle(0.05) (\y*\len,\x*\len) circle(0.05);
	}
	\foreach \x / \y / \z / \w in {1/0/3/0,2/1/3/1,2/0/2/1,3/0/3/2} {
		\draw[thick] (\x*\len,\y*\len) -- (\z*\len,\w*\len) (\y*\len,\x*\len) -- (\w*\len,\z*\len);
	}
	\foreach \x / \y in {0/1,1/2,2/3} {
		\draw[pattern=north west lines,opacity=0.3] (\x*\len,\x*\len) rectangle (\y*\len,\y*\len);
	}
\end{tikzpicture} $|\G|^2$\end{center}
\end{emptythm}

\begin{emptythm}[Analogon]
$|\G|^n$ ist ein Zellkomplex mit Zellen $\sigma = \sigma_1 \X \ldots \X \sigma_n$.
Die kombinatorische Diagonale von $|\G|^n$ besteht aus Zellen der Form $\sigma = \sigma_1 \X \ldots \X \sigma_n$ mit $\partial\sigma_i \cap \partial\sigma_j \ne \emptyset$ f"ur ein $i \ne j$.
Betrachten den kombinatorischen Konfigurationsraum mit Zellen $\sigma = \sigma_1 \X \ldots \X \sigma_n$ mit $\partial\sigma_i \cap \partial\sigma_j = \emptyset$ f"ur alle $i \ne j$.
Dies sind genau die $n$-Tupel von Punkten, welche paarweise durch eine (offene) Kante getrennt sind.
\end{emptythm}

\begin{bsp}
$\G = Y$: 1-Zellen der Form $v \X e$, wobei $e$ eine der Ecke $v$ gegen"uberliegende Kante ist.
\marginnote{\begin{tikzpicture}[font=\scriptsize,scale=1]
	%\tikzgitter{(-4,-4)}{(5,4)}
	\def\angle{120}
	\def\len{1}
	\draw (0,0)node[left]{$v$} --node[right]{$e_2$}node[pos=0.4,sloped]{$\X$} (30:\len)node[above]{$v_2$} (0,0) --node[above]{$e_3$} (30+\angle:\len)node[above]{$v_3$} (0,0) --node[right]{$e_1$} (30+2*\angle:\len)node[below]{$v_1$};
	\foreach \pkt in {(0,0),(30:\len),(30+\angle:\len),(30+2*\angle:\len)} {
		\fill \pkt circle(0.05);
	}
\end{tikzpicture}}
\begin{center}\begin{tikzpicture}[font=\scriptsize,scale=1]
	%\tikzgitter{(-4,-4)}{(5,4)}
	
	\coordinate (0) at (-3,1.5);
	\coordinate (1) at ($(0)+(0.5,-1)$); \coordinate (2) at ($(0)+(1,0)$); \coordinate (3) at ($(0)+(-0.5,1)$); \coordinate (4) at ($(0)+(-2.5,0.5)$);
	\draw ($(1)!0.6!(2)$) -- (2)node[right,sloped]{$v_1 \X v_2$} --node[right]{$v_1 \X e_2$} (3)node[right]{$v_1 \X v$} -- ($(3)!0.4!(4)$);
	\draw[dashed] (1) -- (2) (3) -- (4);
	
	\foreach \pkt in {(2),(3)} {
		\fill \pkt circle(0.05);
	}
	
	\def\len{1}
	
	\foreach \i in {0, 1, 2, 3}{
		\draw (0,\i*\len) -- (3*\len,\i*\len) (\i*\len,0) -- (\i*\len,3*\len);
	}
	\fill[gray,opacity=0.3] (2*\len,0) rectangle (3*\len,\len);
	\foreach \i in {0,1,2} {
		\path (\i*\len,\i*\len) --coordinate[pos=0.25](a)coordinate[pos=0.5](b)coordinate[pos=0.75](c) (\i*\len+\len,\i*\len+\len);
		\draw[->] (a) -- +(315:0.25);
		\draw[->] (b) -- +(315:0.5);
		\draw[->] (c) -- +(315:0.25);
	}
	\foreach \pkt in {([xshift=0.05cm,yshift=-0.05cm]\len,\len),([xshift=0.05cm,yshift=-0.05cm]2*\len,2*\len)} {
		\foreach \angl / \dist in {285/0.9,305/1.1,325/1.1,345/0.9} {
			\draw[->] \pkt -- +(\angl:\dist);
		}
	}
	\draw[dashed] (0,0) -- (3*\len,3*\len);
	\foreach \x / \y in {1/0,2/0,3/0,2/1,3/1,3/2}{
		\fill (\x*\len,\y*\len) circle(0.05);
	}
	\foreach \x / \y / \z / \w in {1/0/3/0,2/1/3/1,2/0/2/1,3/0/3/2} {
		\draw[thick] (\x*\len,\y*\len) -- (\z*\len,\w*\len);
	}
\end{tikzpicture}\end{center}
\end{bsp}

Anschlie"send l"asst sich der kombinatorische Konfigurationsraum auf $C_n\G$ retraktieren.
Leider erh"alt man im Allgemeinen \emph{nicht} die Topologie von $C_n\G$: Beispiel \begin{tikzpicture}[font=\scriptsize,scale=1,baseline=0]
	%\tikzgitter{(-4,-4)}{(5,4)}
	\def\angle{15}
	\draw (-0.5,0)node[left]{$v$} to[out=\angle,in=180-\angle]node[above]{$e_2$} (0.5,0)node[right]{$v$} (-0.5,0) to[out=360-\angle,in=180+\angle]node[below]{$e_1$} (0.5,0);
	\fill (-0.5,0) circle(0.05) (0.5,0) circle(0.05);
\end{tikzpicture} $\leadsto v \X w$.
Es gilt zumindest

\begin{satz}[Abtrams 2000]
Ist $\G$ ein einfacher Graph, das hei"st enth"alt keine Schleifen der L"ange $\le 2$, so ist sein kombinatorischer Konfigurationsraum auf zwei Punkten ein Deformationsretrakt von $C_2\G$.
\end{satz}

\begin{bsp}
$\pi_1(C_2, Y)$
\begin{center}\begin{tikzpicture}[font=\scriptsize,scale=1,baseline=0]
	%\tikzgitter{(-4,-4)}{(5,4)}
	
	\def\angle{55}
	\def\x{2}
	\def\y{1}
	\coordinate (o) at (0,0);
	\coordinate (w1) at (\x,0); \coordinate (w2) at (-\y,\y); \coordinate (w3) at (-\y,-\y);
	
	\foreach \i in {1,2,3} {
		\draw (o) -- (w\i);
	}
	\draw (w1) -- ++(-\y,\y) -- (w2) -- ++(-\y,-\y) -- (w3) -- ++(\x,0) -- cycle;
	
	\foreach \i in {1,2,3} {
		\draw (w\i) -- ++(0,1.25) coordinate(pkt);
		\draw[dashed] (pkt) -- (o);
	}
	
	\draw[decoration={markings,mark=at position 0.0 with{\arrow{>}}},postaction={decorate}] ($(o)-(0,0.5)$) ..controls +(0.25,0) and +(0,-0.25)..coordinate[pos=0.5](a)coordinate[pos=0.7](b) ($(o)+(0.5,0)$) ..controls+(0,0.25) and +(0.25,0).. ($(o)+(0,0.5)$) ..controls+(-0.25,0) and +(0,0.25).. ($(o)-(0.5,0)$) ..controls+(0,-0.25) and +(-0.25,0).. ($(o)+(0,-0.5)$);
	
	\fill (a) circle(0.05);
	
	\filldraw[fill=white] (o) circle(0.05);
	
	\node at (-2*\y,\y) {$C_2Y$};
	\node (a) at ($(o)-(-1,1.5)$) {Erzeuger von $\pi_1(C_2Y) = \Z$};
	\draw[->] (a) to[out=80,in=0] ([xshift=0.1cm]b); 
\end{tikzpicture}
\begin{tikzpicture}[font=\scriptsize,scale=1,baseline=0]
	%\tikzgitter{(-4,-4)}{(5,4)}
	\def\frac{0.7}
	
	\coordinate (o) at (-2,1.25);
	\coordinate (1) at ($(o) + (30:1)$); \coordinate (2) at ($(o) + (150:1)$); \coordinate (3) at ($(o) + (270:1)$);
	\draw (o) --coordinate[pos=\frac](a) (1) (o) -- (2) (o) --coordinate[pos=\frac](b) (3);
	\draw[->] (a) -- +(210:0.4);
	\foreach \pkt in {(a),(b)} { \fill \pkt circle(0.05);}
	
	\coordinate (o) at (0,1.25);
	\coordinate (1) at ($(o) + (30:1)$); \coordinate (2) at ($(o) + (150:1)$); \coordinate (3) at ($(o) + (270:1)$);
	\draw (o) --coordinate[pos=0](a) (1) (o) -- (2) (o) --coordinate[pos=\frac](b) (3);
	\draw[->] (a) -- +(150:0.4);
	\foreach \pkt in {(a),(b)} { \fill \pkt circle(0.05);}
	
	\coordinate (o) at (2,1.25);
	\coordinate (1) at ($(o) + (30:1)$); \coordinate (2) at ($(o) + (150:1)$); \coordinate (3) at ($(o) + (270:1)$);
	\draw (o) -- (1) (o) --coordinate[pos=\frac](a) (2) (o) --coordinate[pos=\frac](b) (3);
	\draw[->] (b) -- +(90:0.4);
	\foreach \pkt in {(a),(b)} { \fill \pkt circle(0.05);}
	
	\coordinate (o) at (-3,-0.75);
	\coordinate (1) at ($(o) + (30:1)$); \coordinate (2) at ($(o) + (150:1)$); \coordinate (3) at ($(o) + (270:1)$);
	\draw (o) -- (1) (o) --coordinate[pos=\frac](a) (2) (o) --coordinate[pos=0](b) (3);
	\draw[->] (b) -- +(30:0.4);
	\foreach \pkt in {(a),(b)} { \fill \pkt circle(0.05);}
	
	\coordinate (o) at (-1,-0.75);
	\coordinate (1) at ($(o) + (30:1)$); \coordinate (2) at ($(o) + (150:1)$); \coordinate (3) at ($(o) + (270:1)$);
	\draw (o) --coordinate[pos=\frac](b) (1) (o) --coordinate[pos=\frac](a) (2) (o) -- (3);
	\draw[->] (a) -- +(330:0.4);
	\foreach \pkt in {(a),(b)} { \fill \pkt circle(0.05);}
	
	\coordinate (o) at (1,-0.75);
	\coordinate (1) at ($(o) + (30:1)$); \coordinate (2) at ($(o) + (150:1)$); \coordinate (3) at ($(o) + (270:1)$);
	\draw (o) --coordinate[pos=\frac](b) (1) (o) --coordinate[pos=0](a) (2) (o) -- (3);
	\draw[->] (a) -- +(270:0.4);
	\foreach \pkt in {(a),(b)} { \fill \pkt circle(0.05);}
	
	\coordinate (o) at (3,-0.75);
	\coordinate (1) at ($(o) + (30:1)$); \coordinate (2) at ($(o) + (150:1)$); \coordinate (3) at ($(o) + (270:1)$);
	\draw (o) --coordinate[pos=\frac](b) (1) (o) -- (2) (o) --coordinate[pos=\frac](a) (3);
	%\draw[->] (a) -- +(270:0.4);
	\foreach \pkt in {(a),(b)} { \fill \pkt circle(0.05);}
\end{tikzpicture}\end{center}
\end{bsp}

\begin{dfn}[Swiatowski]
Es sei $\G$ ein endlicher Graph, dessen nichtessentialle Ecken frei sind, mit Kanten $\E$ und Ecken $\V$.
Es bezeichne $\calB$ die Menge der essentiellen Ecken.
Es sei $P_n^{(k)}\G$ die Menge der Paare $(f, S)$ mit
\begin{enumerate}[label=(\roman*)]
\item
	$f: \E \cup \calB \to \N_0$ eine Abbildung
\item
	$S= \{ e_1, \ldots, e_k \}$ paarweise verschiedene orientierte Kanten
\item
	$v_{e_i} = \partial^+e_i \in \calB$ und $v_{e_i} \ne v_{e_j}$ f"ur $i \ne j$
\item
	$f(v) \in \{0, 1\}$ f"ur alle $v \in \calB$ und $f(v_{e_i}) = 0$ f"ur alle $i \le k$
\item
	$\sum_{|a| \in \E \cup \calB} f(|a|) = n - k$.
\end{enumerate}
Es gelte dabe $(f,S) \prec (g, S \dot\cup \{e\})$ mit $e \notin S$, falls entweder
\begin{enumerate}[label=(\alph*)]
\item $f(v_e) = g(v_e) + 1 (=1)$ und sonst $f(a) = g(a)$
\end{enumerate}
oder
\begin{enumerate}[label=(\alph*)]
\item[(b)] $f(|e|) = g(|e|) + 1$ und sonst $f(a) = g(a)$
\end{enumerate}
gilt.
Es bezeichne $\prec$ die davon erzeugte partielle Ordnung auf $P_n\G = \dot\cup P_n^{(k)}\calG$ und $K_n\G$ seine geometrische Realisierung.
\end{dfn}

0-Skelett: Paare $(f, \emptyset)$ mit $f: \E \cup \calB \to \N_0$, $\sum_{|a| \in \E \cup \calB} f(|a|) = n$, $f$ \quot{z"ahlt} wieviele Punkte auf einer Kante beziehungsweise essentiellen Ecke sitzen.
\begin{center}\begin{tikzpicture}[font=\scriptsize,scale=1,baseline=0]
	%\tikzgitter{(-4,-4)}{(5,4)}
	
	\coordinate (v) at (0,1); \coordinate (w) at (0,-1);
	
	\draw[decoration={markings,mark=at position 0.5 with{\arrow{>}}},postaction={decorate}] (v)node[right]{$v$} --node[right]{$e$}node[pos=0.25]{$\X$} node[pos=0.75]{$\X$} (w)node[right]{$w$};
	\draw (v) -- +(45:0.5) (v) -- +(135:0.5) (w) -- +(225:0.5) (w) -- +(270:0.5) (w) -- +(315:0.5);
	\foreach \pkt in {(v),(w)} { \fill \pkt circle(0.05);}
\end{tikzpicture}
$\sim (f,\emptyset)$ mit $f(e) = 2$, $f(v) = f(w) = 0$\hspace{1cm} $\tau_0 = (f, \emptyset)$\end{center}
Zwei solche 0-Zellen sind durch eine Kante (1-Zelle) verbunden, wenn einer der Punkte das Innere einer Kante durch eine essentielle Ecke betritt oder verl"asst.
\begin{center}\begin{tikzpicture}[font=\scriptsize,scale=1,baseline=0]
	%\tikzgitter{(-4,-4)}{(5,4)}
	
	\coordinate (v) at (0,1); \coordinate (w) at (0,-1);
	
	\draw[decoration={markings,mark=at position 0.5 with{\arrow{>}}},postaction={decorate}] (v)node[right]{$v$} --node[right]{$e$}node[pos=0.25]{$\X$} node[pos=0.75]{$\X$} (w)node[right]{$w$};
	
	\foreach \pkt in {(v),(w)} { \fill \pkt circle(0.05);}
\end{tikzpicture}
$\leftrightsquigarrow$
\begin{tikzpicture}[font=\scriptsize,scale=1,baseline=0]
	%\tikzgitter{(-4,-4)}{(5,4)}
	
	\coordinate (v) at (0,1); \coordinate (w) at (0,-1);
	
	\draw[decoration={markings,mark=at position 0.5 with{\arrow{>}}},postaction={decorate}] (v)node[right]{$v$} --node[right]{$e$}node[pos=0.25]{$\X$} node[pos=1]{$\X$} (w)node[right]{$w$};
	
	\foreach \pkt in {(v),(w)} { \fill \pkt circle(0.05);}
	
	\node[align=left,font=\normalsize] at (1.5,0) {$\tau_1 = (g, \emptyset)$ \\ $g([e|) = 1$ \\ $g(w) = 1$};
\end{tikzpicture}
\begin{tikzpicture}[font=\scriptsize,scale=1,baseline=0]
	\draw (-1.5,0)node[above]{$\tau_0$}[fill]circle(0.05) -- (1.5,0)node[above]{$\tau_1$}[fill]circle(0.05);
\end{tikzpicture}
\hspace{1cm}
$\sigma = (h, \{e\})$
\end{center}
$h(|e|) = 1 \xRightarrow{\text{(b)}} \tau_0 \prec \sigma$,
$h(w) = 0 \xRightarrow{\text{(a)}} \tau_1 \prec \sigma$
$(h(v) = 0)$

Die h"oherdimensionalen Zellen bestehen aus $k$-Tupeln von unabh"angigen Z"ugen wie oben.
Ist $\sigma = (g, \{e_1, \ldots, e_k\})$ eine $k$-Zelle in $K_n\G$, so besitzt $\sigma$ genau $2k$ Facetten der Kodimension 1:
\begin{align*}
	\partial_{e_i}^\pm \sigma = ( g_i^\pm, \{e_1, \ldots, \hat{e_i}, \ldots, e_k \})
\end{align*}
mit $g_i^+(v_{e_i}) = 1$, beziehungsweise $g_i^-(|e_i|) = g(|e_i|) + 1$.
Damit ist $K_n\G$ ein Kubenkomplex: Jede Menge $\{ \tau \mid \tau \prec \sigma \}$ ist ein W"urfel.

% Vorlesung vom 28. 5.

Wir definieren die Dimension als $\dim K_n\G = \min \{ |\calB|, n\}$.
Es gilt dann $\dim K_n\G = \max \{k \mid P_n^{(k)}\G \ne \emptyset\}$.
Wir beweisen die Aussage:

Aus (ii) und (iii) folgt $\dim K_n\G \le |\calB|$.
Ist $k = \min \{|\calB|, n \}$, so existieren paarweise verschiedene essentielle Ecken $v_1, \ldots, v_n$.
Da $v_i$ essentiell ist, existieren paarweise verschiedene Kanten $e_1, \ldots, e_k$ mit $v_i = v_{e_i}$.
Es sei $S = \{e_1, \ldots, e_k\}$ und
\begin{align*}
	f: \E \cup \calB \to \N_0 && |a| \mapsto \begin{cases} n-k & |a| = e_1 \\ 0 & \text{sonst} \end{cases}.
\end{align*}
Dann gilt $(f, S) \in P_n^{(k)}\G$.

\begin{bsp}
Betrachte wieder den Graphen $\G = Y$, wir suchen $K_2Y$ mit $\calB = \{v\}$, $\E = \{e_1, e_2, e_3\}$ und $\dim K_2Y = \min \{|\calB|, n\} = \min \{2,1\} = 1$.
\begin{center}$\G = $ \begin{tikzpicture}[font=\scriptsize,scale=1,baseline=0]
	%\tikzgitter{(-4,-4)}{(5,4)}
	\def\r{1}
	
	\draw[decoration={markings,mark=at position 0.5 with{\arrow{<}}},postaction={decorate}] (0,0) --node[above]{$e_2$} (30:\r);
	\draw[decoration={markings,mark=at position 0.5 with{\arrow{<}}},postaction={decorate}] (0,0) --node[above]{$e_3$} (150:\r);
	\draw[decoration={markings,mark=at position 0.5 with{\arrow{<}}},postaction={decorate}] (0,0) --node[right]{$e_1$} (270:\r);
	
	\fill (0,0) circle(0.05)node[right]{$v$};
\end{tikzpicture}\end{center}
F"ur die 1-Zellen setze $\sigma = (g, S)$ mit $|S| = 1$, also $S = \{e_i\}$ für ein $i \in \{1, 2, 3\}$.
Dann folgt $g(v_{e_i}) = g(v) = 0$ nach (iv).
Daraus ergibt sich f"ur die Summe
\begin{align*}
	\sum_{|a| \in \calE \cup \calB} g(|a|) = g(|e_1|) + g(|e_2|) + g(|e_3|) + \underset{\mathclap{=g(v)}}{0} = n - k = 2 - 1 = 1
\end{align*}
Da alle Summanden positiv sind muss es genau ein $j$ mit $g(|e_j|) = 1$ geben, und damit existieren insgesamt genau neun 1-Zellen $\sigma_{ij} = (|e_j| \mapsto 1, \{e_i\})$ als R"ander.
F"ur die 0-Zellen ergibt sich dann:
\begin{enumerate}[label=(\alph*)]
\item
	$\partial^+ \sigma_{ij} = (\begin{smallmatrix} v &\mapsto& 1 \\ |e_j| &\mapsto& 1 \end{smallmatrix}, \emptyset) = \tau_j$ ($\deg\tau_j = 3$)
\item
	$i \ne j$: $\partial^- \sigma_{ij} = (\begin{smallmatrix} |e_i| &\mapsto& 1 \\ |e_j| &\mapsto& 1 \end{smallmatrix}, \emptyset) = \tau_{ij} = \tau_{ji}$ ($\deg\tau_{ij} = 2$) \\
	$i = j$: $\partial^- \sigma_{ii} = (|e_i| \mapsto 2, \emptyset) = \tau_{ii}$ ($\deg\tau_{ii} = 1$)
\end{enumerate}
Anschaulich ergibt sich schlie"slich folgendes Bild:
\begin{center}\begin{tikzpicture}[font=\scriptsize,scale=1,baseline=0] 
	%\tikzgitter{(-4,-4)}{(5,4)}
	
	\def\r{1.25}
	\coordinate (t1) at (\r,0); \coordinate (t2) at (240:\r); \coordinate (t3) at (120:\r);
	\coordinate (t13) at (60:\r); \coordinate (t12) at (300:\r); \coordinate (t23) at (180:\r);
	
	\draw[decoration={markings,mark=at position 0.5 with{\arrow{<}}},postaction={decorate}] (t1)node[left]{$\tau_{1}$} --node[right]{$\sigma_{21}$} (t12);
	\draw[decoration={markings,mark=at position 0.5 with{\arrow{>}}},postaction={decorate}] (t12)node[below right]{$\tau_{21}=\tau_{12}$} --node[below]{$\sigma_{12}$} (t2);
	\draw[decoration={markings,mark=at position 0.5 with{\arrow{<}}},postaction={decorate}] (t2)node[above right]{$\tau_{2}$} --node[left]{$\sigma_{32}$} (t23);
	\draw[decoration={markings,mark=at position 0.5 with{\arrow{>}}},postaction={decorate}] (t23)node[left]{$\tau_{23}=\tau_{32}$} --node[left]{$\sigma_{23}$} (t3);
	\draw[decoration={markings,mark=at position 0.5 with{\arrow{<}}},postaction={decorate}] (t3)node[below right]{$\tau_{3}$} --node[above]{$\sigma_{13}$} (t13);
	\draw[decoration={markings,mark=at position 0.5 with{\arrow{>}}},postaction={decorate}] (t13)node[above right]{$\tau_{13}=\tau_{31}$} --node[right]{$\sigma_{31}$} (t1);
	
	\draw[decoration={markings,mark=at position 0.5 with{\arrow{<}}},postaction={decorate}] (t1) --node[above]{$\sigma_{11}$} +(0:\r)coordinate(t11)node[right]{$\tau_{11}$};
	\draw[decoration={markings,mark=at position 0.5 with{\arrow{<}}},postaction={decorate}] (t2) --node[right]{$\sigma_{22}$} +(240:\r)coordinate(t22)node[below]{$\tau_{22}$};
	\draw[decoration={markings,mark=at position 0.5 with{\arrow{<}}},postaction={decorate}] (t3) --node[right]{$\sigma_{33}$} +(120:\r)coordinate(t33)node[above]{$\tau_{33}$};
	
	\foreach \i in {1,2,3,13,12,23,11,22,33} {
		\fill (t\i) circle(0.05);
	}
\end{tikzpicture}\end{center}
\end{bsp}

\begin{emptythm}[Einbettung $\iota: K_n\G \hookrightarrow C_n\G$]
Sei die Verteilung von $k$ Punkten auf einer Kante $e$ durch $D_e(k, (s, t)) \subset C_n\G \cap e$, mit $(s,t) \in [0,1]$, gegeben mit
\begin{center}\begin{tikzpicture}[font=\scriptsize,scale=1]
	%\tikzgitter{(-4,-4)}{(5,4)}
	\def\angle{15}
	\def\frac{0.11}
	\def\len{0.125}
	
	\draw (0,0)coordinate(1)node[left]{$v_{\overline e}$} --node[below,pos=0.6]{$e$} (\angle:3.5)node[right]{$v_{e}$}coordinate(2);
	
	\foreach \i in {2,3,4} {
		\draw ($(1)!0.8*\frac+(\i-1)*\frac!(2) -(90+\angle:\len)$) -- +(90+\angle:2*\len)node[above]{$x_{\i}$}; 
	}
	\draw ($(1)!1*0.8*\frac!(2) -(90+\angle:\len)$) -- +(90+\angle:2*\len)node[above]{$x_{1}$};
	\draw ($(1)!1-1*0.7*\frac!(2) -(90+\angle:\len)$) -- +(90+\angle:2*\len)node[above]{$x_{k}$};
	
	\fill (0,0) circle(0.05) (15:3.5)circle(0.05);
	
	\node[align=left,font=\normalsize] at (7.5,0.5) {$\bullet x_1 < \ldots < x_k$ \\ $\bullet |x_i - x_{i+1}| = d$ \\ $\bullet |v_{\overline e} - x_1| = s \cdot d$ und $|v_e - v_k| = t \cdot d$};
\end{tikzpicture}\end{center}
als koordinaten auf einem W"urfel $\sigma = (f,S) \in P_n^{(k)}$.
Definiere $t: \sigma \to [0,1]^s$ durch lineare Fortsetzung der Abbildung auf seinen 0-Zellen $p= (f, \emptyset)$:
\begin{align*}
	t(p): S \to [0,1] && e \mapsto 1 - f(e).
\end{align*}
F"ur jedes $x \in \sigma = (g,S) \subset K_n\G$ ist  $t(x)$ eine Abbildung von $S$ in $[0,1]$.
Setze fort
\begin{align*}
	t(x): \E \to [0,1] && e \mapsto \begin{cases} t(x)(e) & e \in S \\ 1 & \text{sonst} \end{cases}.
\end{align*}
Daraus erh"alt man eine Einbettung $\iota_\sigma: \sigma = (g, S) \to C_n\G$, $x \mapsto \{v \in \calB \mid g(v) = 1 \} \bigcup_{e \in \E} D_e(\tilde{g}(|e|), (t(x)(\overline e), t(x)(e)))$, wobei $\tilde{g}(|e|) = g(|e|) + \# \{s \in S \mid |s| = e \}$.

Es sei $x \in \tau = (f, S) \prec \sigma = (g, S \dot\cup \{e\})$.
Wir beschr"anken und im Folgenden zun"achst nur auf den Fall (a), der Fall (b) folgt dann analog.
Sei $f(v_e) = g(v_e) + 1 = 1$, also $t_\sigma(x)(e) = 0$ und $v_e \in v_\tau(x)$ f"ur $e \notin S$.
Dann folgt $t_\tau(x)(e) = 1$.
Aus $f(|e|) = g(|e|)$ folgt $\tilde{g}(|e|) = \tilde{f}(|e|) + 1$.

Es folgt
\marginnote{\begin{tikzpicture}[font=\scriptsize,scale=1]
	%\tikzgitter{(-4,-4)}{(5,4)}
	\def\angle{15}
	\draw (0,0) -- (\angle:2.5)node[above right]{$v$}node[below right]{$t \cdot d = 0$} [fill]circle(0.05);
\end{tikzpicture}}
\begin{align*}
	&D_e(\underbrace{\tilde g(|e|)}_{=\tilde f(|e|) + 1}, (t_\sigma(x)(\overline e), \underbrace{t_\sigma(x)(e)}_{=0})) \\
	&= D_e(\tilde f(|e|) + 1, (s, 0)) \\
	&= \{v_e\} \cup D_e(\tilde f(|e|), (s,1)) & \Rightarrow \iota_\sigma(x) = \iota_\tau(x)
\end{align*}
Damit definiert $\iota = \bigcup_\sigma \iota_\sigma$ eine Einbettung $K_n\G \hookrightarrow C_n\G$.
\end{emptythm}

\begin{emptythm}[Retraktion $r: C_n\G \to K_n\G$]
Sei $C \in C_n\G$ und bezeichne $n_e^C = \#(C \cap e) \setminus \calB$ die Anzahl der Punkt auf $e$.
Setze
\marginnote{\begin{tikzpicture}[font=\scriptsize,scale=1]
	%\tikzgitter{(-4,-4)}{(5,4)}
	\def\angle{15}
	\draw (0,0) --coordinate[pos=0.65](1)node[pos=0.65,sloped]{$\X$} (\angle:2.5)node[above right]{$v_e$}coordinate(2) [fill]circle(0.05);
	\draw[decoration={brace,mirror},decorate] ([yshift=-0.1cm]1) --node[below]{$=d_e^C$} ([yshift=-0.1cm]2);
\end{tikzpicture}}
\begin{align*}
d_e^0 = \begin{cases} 1 & n_e^C = 0 \\ \min\{ |v_e - x| \mid x \in C \cap e \setminus \calB\} & \text{sonst}\end{cases} && \delta_e^C = d_e^C(n_e^C + 1)
\end{align*}
und die mittlere Segmentl"ange $\frac{1}{n_e^C+1}$. Setzte weiterhin
\begin{align*}
	t_e^C = \begin{cases}1 & \text{falls } v_e \text{ frei oder } v_e \in C \\ \min\{1, \frac{\delta_e^C}{\min\{ \delta_e^C \mid e' \ne e \wedge v_{e'} = v_e\}} & \text{sonst}\end{cases}
\end{align*}
Definiere
\begin{align*}
	r: C_n\G \to \iota(K_n\G) && C \mapsto (C \cap \calB) \cup \bigcup_e D_e(n_e^C, (t_{\overline e}^C, t_e^C))
\end{align*}
\end{emptythm}

\begin{emptythm}[Homotopie $\id \cong r$]
F"ur beliebiges $C \in C_n\G$ gilt
\begin{itemize}
\item
	$C \cap \calB = r(C) \cap \calB$
\item
	$\#(C \cap e) \setminus \calB = \# (r(C) \cap e) \setminus \calB$ f"ur alle $e \in \E$
\end{itemize}
Definiere die Homotopie kantenweise, so dass Punkte in die vorgegebene Standardposition bewegt werden.
\end{emptythm}

% Vorlesung vom 3. 6.

\section{Geometrie des Kubenkomplexes $K_n\G$}

\begin{dfn}[$M_\kappa$-Komplex]
Es sei $\sigma_i$, f"ur $i \in I$, eine disjunkte Familie konvexer Polyeder, das hei"st $\sigma_i$ ist eine konvexe H"ulle endlich vieler Punkte in $M_\kappa^{n_i}$.
Es sei $\sim$ eine "Aquivalenzrelation auf $\sqcup_{i \in I} \sigma_i$, $X = \FakRaum{\sqcup_i \sigma_i}{\sim}$ und $\pi: \sqcup_i \sigma_i \to X$ die kanonische Projektion.
$X$ hei"st \CmMark[MkPolyederkomplex@$M_\kappa$-Polyederkomplex]{$\bm{M_\kappa}$-Polyederkomplex}, falls gilt:
\begin{enumerate}[label=(\roman*)]
\item
	Ist $\tau \le \sigma_i$ eine Seite, so ist $\pi|_{\mathring\tau}$ injektiv.
	\marginnote{\tiny{$\mathring\tau$ offenes Inneres von $\tau$}}
\item
	Sind $x_1 \in \sigma_1$ und $x_2 \in \sigma_2$ mit $\pi|_{\sigma_1}(x_1) = \pi|_{\sigma_2}(x_2)$, so existiert eine Isometrie $\phi: \supp(x_1) \to \supp(x_2)$ mit $\pi|_{\sigma_1}(y) = \pi|_{\sigma_2}(\phi(y))$ f"ur alle $y \in \supp(x_1)$; wobei $\supp(x_i)$ die eindeutige Seite $\tau < \sigma_i$ mit $\mathring\tau \ni x_i$ ist.
	\begin{center}\begin{tikzpicture}[font=\scriptsize,scale=1]
		%\tikzgitter{(-4,-4)}{(5,4)}
		
		\coordinate (o) at (-2.5,0);
		\coordinate (1) at ($(o) + (2,0.25)$);
		\coordinate (2) at ($(o) + (1.5,1.25)$);
	
		\draw (o) -- (1) -- (2) --coordinate[pos=0.6](x)node[left]{$x$} (o) -- cycle;
		\node at ($($(1)!0.5!(2)$)!0.25!(o)$) {$\sigma_i$};
		\foreach \a in {o, 2, x} { \fill (\a) circle(0.05);}
		
		\coordinate (o) at (1,0);
		\coordinate (1) at ($(o) + (2,0.25)$);
		\coordinate (2) at ($(o) + (1.5,1.25)$);
	
		\draw (2) --node[left]{$\supp(x) =$} (o);
		\draw[dashed] (o) -- (1) -- (2);
		\foreach \a in {o, 2} { \fill (\a) circle(0.05);}
	\end{tikzpicture}\end{center}
\end{enumerate}
\end{dfn}

Ein $M_\kappa$-Polyederkomplex ist im Allgemeinen \emph{kein} simplizialer Komplex. Betrachte die folgenden beiden Beispiele:
\begin{enumerate}[label=(\arabic*)]
\item
	2-Torus:
	\begin{center}\begin{tikzpicture}[font=\scriptsize,scale=1]
		%\tikzgitter{(-4,-4)}{(5,4)}
		
		\draw[->] (-0.5,0) --node[above]{$\pi$}node[below]{$\sim$} (0.5,0);
		\tikztorus[0.5]{(2,0)}
		
		\def\len{0.75}
		\coordinate (o) at (-1 - \len,0);
		\draw[decoration={markings,mark=at position 0.6 with{\arrow{>}},mark=at position 0.4 with{\arrow{>}}},postaction={decorate}] ($(o)+(\len,-\len)$) -- ($(o)+(\len,\len)$);
		\draw[decoration={markings,mark=at position 0.6 with{\arrow{>}},mark=at position 0.4 with{\arrow{>}}},postaction={decorate}] ($(o)+(-\len,-\len)$) -- ($(o)+(-\len,\len)$);
		\draw[decoration={markings,mark=at position 0.5 with{\arrow{>}}},postaction={decorate}] ($(o)+(-\len,-\len)$) -- ($(o)+(\len,-\len)$);
		\draw[decoration={markings,mark=at position 0.5 with{\arrow{>}}},postaction={decorate}] ($(o)+(-\len,\len)$) -- ($(o)+(\len,\len)$);
	\end{tikzpicture}\end{center}
	$\sigma = [0,1]^2$ einziger Polyeder, $\sim$ Kantenidentifikation (wie "ublich)
	
	\emph{Kein} simplizialer Komplex, denn $\pi|_\sigma$ ist nicht injektiv
\item
	Digon: zwei (maximale) Polyeder $e_1 \cong e_2 \cong [0,1]$, $\sim =$ Identifikation der Randpunkte
	\begin{center}\begin{tikzpicture}[font=\scriptsize,scale=1]
		%\tikzgitter{(-4,-4)}{(5,4)}
		\path (0,0)coordinate(v)node[left]{$v$} -- +(15:2)coordinate(w)node[right]{$w$};
		\draw (v) to[relative,out=30,in=150] node[above]{$e_1$} (w);
		\draw (v) to[relative,out=330,in=210] node[below]{$e_2$} (w);
		\foreach \x in {v,w} {\fill (\x) circle(0.05);}
	\end{tikzpicture}\end{center}
	\begin{enumerate}[label=(\roman*)]
	\item
		$\pi|_{e_i}$ ist injektiv.
	\item
		$\pi(e_1) \cap \pi(e_2) = \{v, w \}$ ist \emph{nicht} gemeinsame Seite des Polyeders.
	\end{enumerate}
\end{enumerate}

Der Kubenkomplex $K_n\G$ ist ein $M$-Polyederkomplex im obigen Sinne.
$K_n\G$ enh"alt keine Schleifen.
Ist $\sigma \subset K_n\G$ ein 1-W"urfel, das hei"s $\sigma = (f, \{e\})$, so gilt $\partial^+(f, \{e\}) = (\begin{smallmatrix}v_e &\mapsto& 0 \\ |e| &\mapsto& f(|e|)+1\end{smallmatrix}, \emptyset) \ne \partial^-(f, \{e\}) = (\begin{smallmatrix}v_e &\mapsto& 1 \\ |e| &\mapsto& f(|e|)\end{smallmatrix}, \emptyset)$.

Es gibt Digone in $K_n\G$:
Es gelte $n \ge 1$ und $\G$ enthalte eine Schleife.
betrachte die folgenden Konfigurationen:
\begin{center}\begin{tikzpicture}[font=\scriptsize,scale=1,baseline=0]
	%\tikzgitter{(-4,-4)}{(5,4)}
	\def\rad{0.5}
	\draw (0,0) circle(\rad) (0,-\rad) -- +(0,-0.75);
	\fill (0,-\rad) circle(0.05) node[below right]{$v$};
	\node[right] at (\rad,0) {$e$};
	\node at (0,\rad) {$\X$};
	\node at (0,-\rad) {$\X$};
\end{tikzpicture}
und
\begin{tikzpicture}[font=\scriptsize,scale=1,baseline=0]
	%\tikzgitter{(-4,-4)}{(5,4)}
	\def\rad{0.5}
	\draw (0,0) circle(\rad) (0,-\rad) -- +(0,-0.75);
	\fill (0,-\rad) circle(0.05);
	\node at (0,\rad) {$\X$};
	\node at (250:\rad) {$\X$};
\end{tikzpicture}
bzw.
\begin{tikzpicture}[font=\scriptsize,scale=1,baseline=0]
	%\tikzgitter{(-4,-4)}{(5,4)}
	\def\rad{0.5}
	\draw (0,0) circle(\rad) (0,-\rad) -- +(0,-0.75);
	\fill (0,-\rad) circle(0.05);
	\node at (0,\rad) {$\X$};
	\node at (290:\rad) {$\X$};
\end{tikzpicture}\end{center}
Es gelte $v_e = v = v_{\overline e}$, damit existieren zwei Intervalle in $K_n\G$, welche das Verlassen der orientierten Kante $e$ parametrisieren:
\begin{center}\begin{tikzpicture}[font=\scriptsize,scale=1]
	%\tikzgitter{(-4,-4)}{(5,4)}

	\path (0,0)coordinate(v)node[left]{$(h,\emptyset)$} -- +(20:4)coordinate(w)node[right]{$(g,\emptyset)$};
	\draw (v) to[relative,out=30,in=150] node[above left]{$(f,\{e\})$} (w);
	\draw (v) to[relative,out=330,in=210] node[below right]{$(f,\{\overline{e}\})$} (w);
	\foreach \x in {v,w} {\fill (\x) circle(0.05);}
	
	\node[align=left] at ($(v)-(3,0)$) {$h(v) = h(v_e) = h(v_{\overline{e}}) = 1$ \\ $h(|e|) = h(|\overline{e}|) = 1$};
	\node[align=left] at ($(w)+(3,0)$) {$g(v) = g(v_e) = g(v_{\overline{e}}) = 1$ \\ $g(|e|) = g(|\overline{e}|) = 1$};
	\node[align=left] at ($(w)+(2.5,-1.5)$) {$f(|e|) = 1 = h(|\overline e|)$ \textcolor{gray}{(def. Digon in $K_n\G$)} \\ $f(v_e) = 0$};
\end{tikzpicture}\end{center}
Jeder h"oherdimensionale W"urfel in $K_n\G$ ist eindeutig durch seine Kodimension-1-Seiten bestimmt:
Es seien $(g, S)$ und $(h, T)$ W"urfel der Dimension $\ge 2$ in $K_n\G$ mit $\{\partial_i^{\pm}(g,S)\} = \{\partial_i^{\pm}(h,T)\}$.
Es gilt
\begin{align*}
	2 |S| = \# \{\partial_i^{\pm}(g,S)\} = \# \{\partial_i^{\pm}(h,T)\} = 2 |T|.
\end{align*}
Daraus folgt $|S| = |T| \ge 2$, also $S = T$.
Man sieht leicht ein, dass dann auch $g = h$ gilt, also $(g, S) = (h, T)$.

\subsection*{Metrik auf $K_n\G$ (bzw. $\bm{M_\kappa}$-Polyederkomplexen)}

Es seien $x, y \in X$. Ein $s \in \{x_0, \ldots, x_k\}$ mit $x_0 = x$, $x_k = y$ und $x_i, x_{i+1} \in \sigma_i$ f"ur einen W"urfel (bzw. Polyeder) hei"st \CmMark[k-Kette@$k$-Kette]{$\bm{k}$-Kette}.
Mit $l(s) = \sum_i d_{\sigma_i}(x_i, x_{i+1})$ sei ihre L"ange bezeichnet, wobei $d_{\sigma_i}$ die Metrik auf $\sigma_i$ ist.
Es sei
\begin{align*}
	d(x, y) = \inf \{l(s) \mid s \text{ Kette von } x \text{ nach } y \}.
\end{align*}
Dies definiert im Allgemeinen \emph{keine} Metrik.
Es sei $X$ der Polyederkomplex aus abz"ahlbar vielen Kanten $e_k$ der L"ange $\frac{1}{k}$.
Dann gilt
\begin{align*}
	d(v, w) \le \inf \{ l(e_k) \mid k \in \N \} = \inf_k \frac{1}{k} = 0.
\end{align*}
Auf dem Kubenkomplex $K_n\G$ definiert $d$ eine Metrik:
$K_n\G$ ist zusammenh"angend, da $\G$ zusammenh"angend ist.
Es gen"ugt zu zeigen, dass $d(x,y) \ne 0$ f"ur $x \ne y$ gilt:
Liegen $x$ und $y$ in einem W"urfel $\sigma$, so ist dies offensichtlich ($d|_\sigma$ ist eine euklidische Metrik auf diesem W"urfel).
Andernfalls betrachte f"ur $x \in \sigma$
\begin{align*}
	\epsilon_\sigma(x) = \inf \{ d_\sigma(x, \tau) \mid \tau < \sigma, \tau \not\ni x \}
\end{align*}
Dies ist in $K_n\G$ stets positiv.
Man kann zeigen, dass $l(s) \ge \epsilon_\sigma(x)$ f"ur jede Kette $s$ von $x$ nach $y$ gilt, also $d(x, y) > 0$.
Man sieht leicht ein, dass $d$ eine L"angenmetrik ist.

\subsection*{Geod\"azit\"at und Vollst\"andigkeit}
\begin{satz}[Bridson 1991]
Es sei $X$ ein zusammenh"angender $M_\kappa$-Polyederkomplex.
Falls in $X$ nur endlich viele Isometrietypen von Polyedern existieren, so ist $X$ ein geod"atisch vollst"andiger metrischer Raum.
\end{satz}

Ohne Einschr"ankung kann man annehmen, dass $X$ simplizial ist.
Betrachte straffe Ketten $s = \{ x_0, \ldots, x_k \}$.
\begin{enumerate}[label=(\roman*)]
\item
	Kein Tripel $X_{i-1}, x_i, x_{i+1}$ liege in einem gemeinsamen Simplex.
\item
	Gilt $x_{i-1}x_i \in \sigma_{i-1}$ und $x_i x_{i+1} \in \sigma_i$, so sei $\overline{x_{i-1}x_{i}} \cup \overline{x_{i}x_{i+1}}$ geod"atisches Segment in $\sigma_{i-1} \cup_{\sim} \sigma_i$.

	\emph{Beachte:} Das hei"st nicht, dass $\overline{x_{i-1}x_{i}} \cup \overline{x_{i}x_{i+1}}$ geod"atisch in $X$ ist, wie man an dieser Zeichnung erkennt:
	\begin{center}\begin{tikzpicture}[font=\scriptsize,scale=1]
		%\tikzgitter{(-4,-4)}{(5,4)}
		
		\coordinate (o) at (-1.5,0);
		\coordinate (1) at ($(o) + (-1,-0.75)$);
		\coordinate (2) at ($(o) + (1,-0.75)$);
		\coordinate (3) at ($(o) + (0,1.25)$);
		
		\draw (o) -- (1) --coordinate[pos=0.25](a)node[left]{$\sigma_1$} (3) --coordinate[pos=0.75](b)node[right]{$\sigma_2$} (2) -- (o) (o) -- (3);
		\draw[gray] (a)node[left]{$x_1$} -- (o)node[above right]{$x_2$} -- (b)node[right]{$x_3$};
		\foreach \x in {a, b, o} { \fill[gray] (\x) circle(0.05); }
		
		\coordinate (o) at (1.5,0);
		\coordinate (1) at ($(o) + (-1,-0.75)$);
		\coordinate (2) at ($(o) + (1,-0.75)$);
		\coordinate (3) at ($(o) + (0,1.25)$);
		
		\draw (o) -- (1) --coordinate[pos=0.25](a)node[left]{$\sigma_1$} (3) --coordinate[pos=0.75](b)node[right]{$\sigma_2$} (2) -- (o) (o) -- (3) (1) --node[below]{$\sigma_3$} (2);
		\draw[gray] (a)node[left]{$x_1$} -- (o)node[above right]{$x_2$} -- (b)node[right]{$x_3$};
		\draw[gray,dashed] (a) -- (b);
		\foreach \x in {a, b, o} { \fill[gray] (\x) circle(0.05); }
	\end{tikzpicture}\end{center}
\end{enumerate}

Es gilt $d(x, y) = \inf \{ l(s) \mid s \text{ straffe Kette von } x \text{ nach } y \}$.
F"ur jede L"ange $l$ existiert eine Schranke $N$, abh"angig von $l$ und den Isometrietypen von Simplices, so dass jede straffe Kette der L"ange $l$ h"ochstens $N$ Simplices durchl"auft.

F"ur $x, y \in X$ durchlaufen straffe Ketten hinreichend kleiner L"ange nur endlich viele \quot{Modellr"aume}, welche durch die endlichen Kombinationen der endlich vielen Isometrietypen gegeben sind.
Jeder solche \quot{Modellraum} enh"alt eine k"urzeste Geod"atische (Moussong 1988).


% Vorlesung vom 3. 6.

\section{Nichtpositive Kr\"ummung von Kubenkomplexen}
Das Innere der Kuben ist $\CAT(0)$, da jeder Kubus die euklidische Metrik tr"agt.
Die K"ummung des Komplexes \quot{konzentriert} sich in den Ecken, das hei"st 0-dimensionalen W"urfeln.

Zur Erinnerung: Das Komplement eines Quadranten in $\R^2$ ist $\CAT(0)$ und gegeben durch $\R^2 \setminus Q = \{ (x_1, x_2) \in \R^2 \mid x_1 \le 0, x_2 \le 0 \}$.
\begin{center}\begin{tikzpicture}[font=\scriptsize]
	%\tikzgitter{(-4,-1)}{(4,3)}
	\coordinate (x) at (-2,-1); \coordinate (y) at (1.5,-0.5); \coordinate (z) at (-0.5,1);
	
	\draw (x) -- (y) -- (0,0) -- (z) -- (x) -- cycle;
	\fill (x) circle(0.05)node[below]{$x$} (y) circle(0.05)node[below]{$y$} (z) circle(0.05)node[above]{$z$};z
	
	\fill[fill=gray!30,fill opacity=0.5] (0,0) rectangle(2.4,1.4);
	\node at (1,0.75) {$Q$};
	\draw[->] (-2.5,0) -- (2.5,0);\draw[->] (0,-1.5) -- (0,1.5);
\end{tikzpicture}\end{center}
Das Komplement eines Oktanten in $\R^3$ ist \emph{nicht} $\CAT(0)$.
\begin{center}\textcolor{red}{[BILD]}\end{center}
Die folgenden W"urfelkomplexe sind euklidisch und $\CAT(0)$:
\begin{center}\textcolor{red}{[BILD]}\end{center}
Allerdings ist dies hier nicht mehr $\CAT(0)$, vergleichbar mit dem Oktanten:
\begin{center}\textcolor{red}{[BILD]}\end{center}
Man kann den Knoten $v$ als Spitze eines Kegels betrachten und den Link $\lk(v)$ als den Grundkreis. Je gr"o"ser Link, umso flacher der Kegel.
\begin{center}\textcolor{red}{[BILD]}\end{center}
Es sei $v$ eine Ecke.
Der \CmMark{Link} $\lk(v)$ ist ein abstrakter simplizialer Komplex mit einem $k$-Simplex f"ur jeden $(k+1)$-W"urfel, der $v$ enth"alt.
F"ur $K$ sieht der Link $\lk_K(v)$ folenderma"sen aus:
\begin{center}\begin{tikzpicture}[font=\scriptsize,scale=1]
	%\tikzgitter{(-4,-4)}{(5,4)}
	\def\rad{0.8}
	
	\draw[rotate=22.5] (0:\rad) -- (45:\rad) -- (90:\rad) -- (135:\rad) -- (180:\rad) --node[below left]{$\sigma_1$} (225:\rad) --node[below]{$\sigma_2$} (270:\rad) -- (315:\rad) -- cycle;
	\fill[rotate=22.5] (0:\rad)circle(0.05) (45:\rad)circle(0.05) (90:\rad)circle(0.05) (135:\rad)circle(0.05) (180:\rad)circle(0.05) (225:\rad)circle(0.05) (270:\rad)circle(0.05) (315:\rad)circle(0.05);
\end{tikzpicture}\end{center}

\begin{satz}[Gromovs Link-Bedingung]
Es sei $K$ ein endlichdimensionaler Kubenkomplex.
$K$ ist genau dann nichtpositiv gekr"ummt, wenn der Link jeder Ecke ein Fahnenkomplex ist.
\end{satz}

Ein \CmMark{Fahnenkomplex} ist ein simplizialer Komplex, in welchem jede Menge von Ecken, die paarweise durch Kanten verbunden sind, einen Simplex aufspannen.
Ein Komplex, welcher maximal unter allen Komplexen mit dem gleichen 1-Skelett ist, ist ein Fahnenkomplex.
\begin{center}\textcolor{red}{[BILD]}\end{center}

\subsection*{Link-Bedingung in $K_n\G$}

Es sei $x \in K_n\G$ eine Ecke, das hei"st $x = (f, \emptyset)$ mit $f: \E \cup \calB \to \N_0$, $\sum_{a \in \E \cup \calB} f(|a|) = n$ und $f(v) \in \{0, 1\}$ f"ur alle $v \in \calB$.
Es gilt $\lk(x) = \{ \sigma \in K_n \mid (f, \emptyset) \prec \sigma \}$.
Ein $\sigma = (g, S)$ enth"alt genau dann $(f, \emptyset)$, wenn gilt
\begin{align*}
	g(a) = \begin{cases}
		f(a) & a \in \calB \wedge a \ne v_e, e \in S \\
		0 & a = v_e, e \in S \\
		f(a) + \sum_{\substack{s \in S \\ |s| = a}} (f(v_s) - 1) & a \in \E
	\end{cases}
\end{align*}
Solche $\sigma = (g, S)$ sind eindeutig bestimmt duch Teilmengen $S \subset \E$, welche die folgenden Bedingungen erf"ullen:
\begin{enumerate}[label=(\roman*)]
\item
	$v_e \in \calB$ f"ur alle $e \in S$
\item
	$v_e \ne v_{e'}$ f"ur $e, e' \in S$
\item
	$f(|e|) + f(v_e) \ge 1$ f"ur alle $e \in S$
\item
	$f(|e|) + f(v_e) + f(v_{\overline{e}}) \ge 2$
\end{enumerate}
Der Link von $x$ besteht aus den Teilmengen $S$, die (i) bis (iv) erf"ullen.
Jede solche Teilmenge $S$ definiert einen $(|S| - 1)$-Simplex von $\lk(x)$.
Die Bedingungen (i) bis (iv) gelten genau dann f"ur eine Teilmenge $S$, wenn wie f"ur alle ein- und zweielementigen Teilmengen gelte.

Ist dann $(g_i, \{e_i\})_{i \le k}$ eine Menge von Ecken in $\lk(x)$, das hei"st Kanten in $K_n\G$, die $x$ enthalten, und sind diese alle durch Kanten $(h_{ij}, \{e_i, e_j\})$ verbunden, so sind (i) - (iv) f"ur $\{e_i\}$ und $\{e_i, e_j\}$ erf"ullt.
Damit gelten die Bedingungen auch f"ur $S = \{e_1, \ldots, e_k\}$ und es existiert ein $g$ mit $(g, S) \succ x = (f, \emptyset)$; Die Kanten $(h_{ij}, \{e_i, e_j\})$ in $\lk(x)$ spannen also einen Simplex auf.
Damit ist $\lk(x)$ ein Fahnenkomplex.

Somit ist nach Gromovs Link-Bedingung $K_n\G$ nichtpositiv gekr"ummt. Das liefert uns
\begin{itemize}
\item
	$\widetilde{K_n\G}$ ist (global) $\CAT(0)$
\item
	$\pi_k(K_n\G) = 0$ f"ur alle $k \ge 2$
\item
	$K_n\G \hookrightarrow C_n\G$ Deformationsretrakt $\Rightarrow \pi_*(K_n\G) = \pi_*(C_n\G)$
\item
	$\pi_1(C_n\G) = \pi_1(K_n\calG)$, $\pi_k(C_n\G) = 0$, $k \ge 2$
\end{itemize}
Sei $k = \min \{ |\calB|, \lfloor \frac{n}{2} \rfloor \}$, dann gilt $\pi_1(C_n\G) \ge \Z^k = \pi_1(T)$, ein Torus durch die Wirkung $\pi_1(C_n\G) \curvearrowright K_n\G$.
Finde $T^k$ in $K_n\G$, das hei"st \quot{finde $\S^1$en in $K_n\G$}.
\begin{center}\textcolor{red}{[BILD]}
$\leftrightsquigarrow$
\textcolor{red}{[BILD]}\end{center}
Jede essentielle Ecke $v \in \calB$ in $\G$ hat $\deg(v) \ge 3$.
F"ur jeden Erzeuger einer $\S^1$ in $K_n\G$ ben"otigt man \textcolor{red}{zwei Kanten} und eine essentielle Ecke von $\G$.



%========================================================================================================================
%	A N H A N G
%========================================================================================================================

\appendix

%========================================================================================================================
%	UE B U N G
%========================================================================================================================

% Die Benennung der "section" so aendern, dass "\"Ubung 123 vom " am Anfang steht
% Der Code ist fast genau der vom Anfang der Praeambel, dort steht die Erklaerung
\renewcommand*{\othersectionlevelsformat}[3]{\ifstr{#1}{section}{\"Ubung\ #3\ vom\ }{#3\autodot\enskip}}

% Das Format der "section" in Kopfzeile der rechten Seiten
\renewcommand*{\sectionmarkformat}{\"Ubung \thesection\autodot\ vom\enskip}

%========================================================================================================================
%	S T I C H W O R T V E R Z E I C H N I S
%========================================================================================================================

\addcontentsline{toc}{chapter}{Stichwortverzeichnis} % <- damit es auch im Inhaltsverzeichnis erscheint
\printindex

%========================================================================================================================
%	G L O S S A R
%========================================================================================================================



%========================================================================================================================
%	L I T E R A T U R V E R Z E I C H N I S
%========================================================================================================================
\bibliographystyle{plain}

\end{document}