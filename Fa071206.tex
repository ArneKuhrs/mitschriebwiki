\documentclass[a4paper,11pt]{book}

\usepackage{amssymb}
\usepackage{amsmath}
\usepackage{amsfonts}
\usepackage{ngerman}
%\usepackage{graphicx}
\usepackage{fancyhdr}
\usepackage{euscript}
\usepackage{makeidx}
\usepackage{hyperref}
\usepackage[amsmath,thmmarks,hyperref]{ntheorem}
\usepackage{enumerate}
\usepackage{url}
\usepackage{mathtools}
\usepackage[arrow, matrix, curve]{xy}
%\usepackage{pst-all}
%\usepackage{pst-add}
%\usepackage{multicol}

\usepackage[latin1]{inputenc}

%%Zahlenmengen
%Neue Kommando-Makros
\newcommand{\R}{{\mathbb R}}
\newcommand{\C}{{\mathbb C}}
\newcommand{\N}{{\mathbb N}}
\newcommand{\Q}{{\mathbb Q}}
\newcommand{\Z}{{\mathbb Z}}
\newcommand{\K}{{\mathbb K}}
\newcommand{\ssL}{{\mathcal L}}
\newcommand{\sn}[1]{||#1||_{\infty}}
\newcommand{\eps}{{\varepsilon}}
\newcommand{\begriff}[1]{\textbf{#1}} %das sollte man noch ändern!
\newcommand{\eb}{\begin{flushright} \rule{1ex}{1ex} \end{flushright}}
\newcommand{\ind}{1\hspace{-0,9ex}\raisebox{-0,2ex}{1}}
\newcommand{\re}{\ensuremath{\text Re}\,}

% Seitenraender
\textheight22cm
\textwidth14cm
\topmargin-0.5cm
\evensidemargin0,5cm
\oddsidemargin0,5cm
\headheight14pt

%%Seitenformat
% Keine Einrückung am Absatzbeginn
\parindent0pt

\DeclareMathOperator{\unif}{Unif}
\DeclareMathOperator{\var}{Var}
\DeclareMathOperator{\cov}{Cov}


\def\AA{ \mathcal{A} }
\def\PM{ \EuScript{P} } 
\def\EE{ \mathcal{E} }
\def\BB{ \mathfrak{B} } 
\def\DD{ \mathcal{D} } 
\def\NN{ \mathcal{N} } 

% Komische Symbole
\def\folgt{\ensuremath{\implies}}
\newcommand{\folgtnach}[1]{\ensuremath{\DOTSB\;\xRightarrow{\text{#1}}\;}}
\def\equizu{\ensuremath{\iff}}
\def\d{\mbox{d}}
\def\fs{\stackrel{f.s.}{\rightarrow }}

%Nummerierungen
\newtheorem{Def}{Definition}[chapter]
\newtheorem*{DefNO}{Definition}
\newtheorem{Sa}[Def]{Satz}
\newtheorem{Lem}[Def]{Lemma}
\newtheorem{Kor}[Def]{Korollar}
\newtheorem*{TheoNO}{Theorem}
\newtheorem{Theo}[Def]{Theorem}
\theorembodyfont{\normalfont}
\newtheorem*{BspNO}{Beispiel}
\newtheorem{Bsp}[Def]{Beispiel}
\newtheorem*{BemNO}{Bemerkung}
\newtheorem{Bem}[Def]{Bemerkung}
\theoremsymbol{\ensuremath{_\blacksquare}}
\theoremstyle{nonumberplain}
\newtheorem{Bew}{Beweis}
\setcounter{chapter}{1}
\setcounter{section}{5}
\setcounter{Def}{77}

% Kopf- und Fusszeilen
\pagestyle{fancy}
\fancyhead[LE,RO]{\thepage}
\fancyfoot[C]{}
\fancyhead[LO]{\rightmark}

\title{07.12.06}
\author{Das \texttt{latexki}-Team\\[8 cm]}

\date{Stand: \today}
\begin{document}
Counter wird sp"ater gesetzt!

Hier entsteht Kapitel 2. Das Datum ist also irrelevant. Gru"s - Bernhard.\\

\begin{BspNO}[s. 1.68]
$R' = L, L' = R$\\
Alternativ: $(Lx|y) = \sum_{k=1}^{\infty} x_{k+1} \overline{y_k} \stackrel{j=k+1}{=} \sum_{j=2}^{\infty} x_j \overline{y_{j-1}} = (x,Ry)$\\
Insbesondere: $T$ sa. $\Leftrightarrow a_{kl} = \overline{a_{lk}} \ \forall\, k,l \in \N$.\\
Schreibe $z \in \R^{n \times n}$ als $z = (x,y)$ mit $x \in \R^n, y \in \R^n$. Seien $A \in \ssL_m, B \in \ssL_n \Rightarrow A \times B \in L_{m+n}$.\\
F"ur $f: A \times B \rightarrow \C$ bzw $[0,\infty)$ schreiben wir $f^y(x) = f(x,y) \ (y \in B$ fest) und $f^x(y) = f(x,y) \ (x \in A$ fest) sowie (soweit existent):
\[
F(x) = \int_B f(x,y)dy \ , x \in A; \quad G(y) = \int_A f(x,y)dx \ , y \in B
\]
(setze $F(x),G(y) = 0$, falls die Integrale nicht existieren.)
\end{BspNO}



\begin{TheoNO}[Fubini]
\begin{enumerate}
\item[a)] Sei $f: A \times B \rightarrow [0,\infty)$ messbar. Dann sind $f^y$ f"ur f.a. $y \in B, f^x$ f"ur f.a. $x \in A F,G$ messbar und es gilt:
\[
(2.5) \ \int_{A \times B} f(x,y)d(x,y) = \int_A ( \int_B f(x,y)dy)dx = \int_B ( \int_A f(x,y)dx)dy
\]

\item[b)] Sei $f \in L^1(A \times B)$. Dann sind $f^y$ f"ur f.a. $y \in B, f^x$ f"ur f.a. $x \in A F,G$ integrierbar und es gilt (2.5)
\end{enumerate}
\end{TheoNO}


\begin{BemNO}
Analog: $n$-fache Integrale
\end{BemNO}


%Beispiel 2.26
\begin{Bsp}
\textbf{Integraloperatoren}\\
Sei $k \in L^2(A \times A), A \in \ssL_d, f \in L^2(A)$. Nach Bem. 1.34 ist
\[
(x,y) \mapsto \varphi(x,y) := k(x,y)f(y) \text{ messbar}
\]
(Beachte, dass auch $(x,y) \mapsto f(y)$ messbar ist.) Ferner ist $(x,y) \mapsto |\varphi(x,y)|$ messbar. Fubini a) und H"older liefern:
\[
\int_{A \cap B(0,n)} ( \int_A |\varphi(x,y)|dy)^2 dx \leq \int_A ( \int_A |k(x,y)|^2 dy)^{frac12}2 dx \|f\|_2^2 = \|k\|_2^2 \|f\|_2^2 \ (\ast)
\]
\[
\Rightarrow ( \int_{A \cap B(0,n)} ( \int_A |\varphi(x,y)|dy)dx)^2 \stackrel{1.39}{\leq} c(n) \int_{A \cap B(0,n)} ( \int_A |\varphi(x,y)|dy^2 dx \leq c(n) \|k\|_2^2 \|f\|_2^2 \Rightarrow \varphi \in L^1((A \cap B(0,n) \times A) \ \forall\, n \in \N
\]
$\stackrel{\text{Fubini b)}}{\Rightarrow} Tf(x) = \int_A k(x,y)f(y)dy$ existiert f"ur f.a. $x$ und ist messbar. Da $|Tf(x)|^2 \leq (\int_A |\varphi(x,y)|dy)^2$ liefert $\sup_n$ in $(\ast)$, dass $Tf \in L^2(A)$ und $\|Tf\|_2 \leq \|k\|_2 \|f\|_2 \Rightarrow T \in B(L^2(A)), \|T\| \leq \underbrace{\|k\|_2}_{\text{MS-Norm von } T}$\\
Sei $g \in L^2(A)$. Dann gilt:
\begin{eqnarray*}
(Tf|g) & = & \int_A (\int_A k(x,y)f(y)dy)\overline{g(x)}dx \stackrel{\text{H"older und Fubini}}{=} \int_A (\int_A k(x,y)f(y)\overline{g(x)} dx)dy \\
& = & \int_A f(y) \overline{(\int_A \overline{k(x,y)} g(x) dx)}dy = (f|T'g)
\end{eqnarray*}
$\Rightarrow T'g(t) = \int_A \overline{k(s,t)}g(s) ds \ (t \in A) \Rightarrow T$ sa $\Leftrightarrow k(x,y) = k(y,x)$ f"ur f.a. $(x,y) \in A \times A$.
\end{Bsp}


%Satz 2.27
\begin{Sa}
Seien $X,Y$ HRe, $T \in B(X,Y)$
\[
T \text{ ist Isometrie } \Leftrightarrow (T'Tx|z)_X = (Tx|Tz)_Y = (x|z)_X \ \forall x,z \in X
\]
Insbesondere:\\
$T$ unit"ar $\Leftrightarrow T$ bijektiv und $T$ Isometrie $\Leftrightarrow T$ bijektiv und erh"alt Skalarprodukt
\end{Sa}

\begin{Bew}
\begin{enumerate}
\item["` $\Leftarrow$ "'] Setze $x = z$.

\item["` $\Rightarrow$ "'] Sei $\alpha \in \K, x,z \in X$. Dann:\\
$(T(x+\alpha z)|T(x+\alpha z)) \stackrel{2.1}{=} \|Tx\|^2 + \|\alpha Tz\|^2 + 2 \re (Tx|\alpha Tz) = \|x\|^2 + |\alpha| \|z\|^2 + 2 \re (\overline{\alpha}(Tx|Tz))$\\
Andererseits gilt:\\
$|(T(x+\alpha z)|T(x+\alpha z))| = \|T(x+\alpha z)\|^2 = \|x+\alpha z\|^2 \stackrel{(2.1)}{=} \|x\|^2 + 2 \re (\overline{\alpha}(x|z)) + |\alpha| \|z\|^2$\\
$\Rightarrow \re(\overline{\alpha}(Tx|Tz)) = \re(\overline{\alpha}(x|z)) \stackrel{\alpha=1,\alpha=i}{\Rightarrow} (Tx|Tz) = (x|z).$
\end{enumerate}
\end{Bew}


%Satz 2.28
\begin{Sa}
Sei $\K = \C, X$ HR, $T \in B(X)$.
\[
T \text{ sa } \Leftrightarrow (Tx|x) \in \R \text{ f"ur alle } x \in X
\]
\end{Sa}

\begin{Bew}
\begin{enumerate}
\item["` $\Rightarrow$ "´] $(Tx|x) = (x|Tx) = \overline{(Tx|x)} \Rightarrow$ Beh.

\item["` $\Leftarrow$ "´] Sei $\alpha \in \K, x,y \in X$\\
$(T(x+\alpha y)|x+\alpha y) = (Tx|x) + \overline{\alpha} (Tx|y) + \alpha (Ty|x) + |\alpha|^2(Ty|y) =: a \stackrel{\text{Vor.}}{=} \overline{a} \stackrel{\text{Vor.}}{=} (Tx|x) + \alpha(y|Tx) + \overline{\alpha}(x|Ty) + |\alpha|^2(Ty|y)$
\begin{eqnarray}
\stackrel{\alpha = 1}{\Rightarrow} (Tx|y) + (Ty|x) = (y|Tx) + (x|Ty) \\
\stackrel{\alpha = i}{\Rightarrow} i(Tx|y) - i(Ty|x) = -i(y|Tx) + i(x|Ty) \\
\end{eqnarray}
$\Rightarrow (Ty|x) = (y|Tx) \stackrel{x,y \text{ bel.}}{\Rightarrow} T$ sa.
\end{enumerate}
\end{Bew}


\begin{BspNO}
$X = \R^2, T =
\[
\left(\begin{array}{cc}
0 & 1 \\
-1 & 0 \\
\end{array} \right)
\]
$\Rightarrow T$ nicht sa, $(Tx|x) = 0 \ \forall\, x \in \R^2$
\end{BspNO}


%Satz 2.29
\begin{Sa}
Sei $X$ HR, $T \in B(X)$ sei sa. Dann gilt:
\[
\| T \| = \sup_{\|x\| \leq 1} \left| (Tx|x) \right| =: M
\]
Insbesondere:
\[
(Tx|x) = 0 \ \forall\, x \in X \Rightarrow T = 0
\]
\end{Sa}

\begin{Bew}
\begin{enumerate}
\item["` $\geq$ "'] Klar.

\item["` $\leq$ "'] Seien $x,y \in X$ mit $\|x\|,\|y\| \leq 1.$\\
$(T(x+y)|x+y) - (T(x-y)|x-y) \stackrel{2.1}{=} 2(Tx|y) + 2(Ty|x) = 2(Tx|y) + 2 \overline{(Tx|y)} = 4 \re (Tx|y)$\\
$\Rightarrow 4 \re (Tx|y) \leq M( \|x+y\|^2 - \|x-y\|^2) \stackrel{(2.2)}{=} 2M(\|x\|^2 + \|y\|^2) \leq 4M$\\
Sei $(Tx|y) \not =0$ ersetze oben $x$ durch $|(Tx|y)|(Tx|y)^{-1}x$. Dann:
\[
|(Tx|y)| = |(x|Ty)| \leq M \stackrel{2.12; \sup \|x\| \leq 1}{\Longrightarrow} \|Ty\| \leq M \Longrightarrow \|T\| \leq M.
\]
\end{enumerate}
\end{Bew}


%Lemma 2.30
\begin{Le}
Sei $X$ HR, $T \in B(X)$ sei normal. Dann gilt:
\[
\| Tx \| = \| T'x \| \quad \forall\, x \in X
\]
Insbesondere gilt:
\[
N(T) = N(T') \stackrel{2.23}{=} R(T)^{\bot}
\end{Le}

\begin{Bew}
\[
0 = ((T'T-TT)x|x) = |\ Tx \|^2 - \| T'x \|^2 \quad \forall\, x \in X
\]
\end{Bew}


%Satz 2.31
\begin{Sa}
Sei $X$ HR, $P \in B(X)$ eine Projektion mit $P \not= 0$. Dann sind "aquivalent:
\begin{enumerate}
\item[a)] $P$ ist orthogonal

\item[b)] $\|P\| = 1$

\item[c)] $P = P'$ (d.h. $P$ sa.)

\item[d)] $P$ ist normal

\item[e)] $(Px|x) = 0 \ \forall\, x \in X$.
\end{enumerate}
\end{Sa}


\begin{Bew}
\begin{enumerate}
\item[a) $\Rightarrow$ c)] F"ur $x,y \in X$ gilt: $(Px|y) = (Px|Py+\underbrace{(I-P)y}_{\in N(P)}) \stackrel{a)}{=} (Px|Py)$.\  Genauso: $(x|Py) = (Px|Py) \Longrightarrow P = P'$.

\item[c) $\Rightarrow$ d)] Klar.

\item[d) $\Rightarrow$ a)] Lemma 2.30

\item[c) $\Rightarrow$ e)] $(Px|x) = (PPx|x) \stackrel{c)}{=} (Px|Px) \geq 0 \ \forall\, x \in X.$

\item[e) $\Rightarrow$ c)] $\K = \C$: Satz 2.29; $\K = \R$: Werner V 5.9

\item[a) $\Rightarrow$ b)] Theorem 2.11

\item[b) $\Rightarrow$ a)] Sei $\alpha \in \K, x \in N(P), y \in R(P)$. Dann:
\[
\|\alpha y\|^2 = \|P(x+\alpha y)\|^2 \stackrel{b)}{\leq} \|x + \alpha y\|^2 \stackrel{(2.1)}{=} \|x\|^2 + 2 \re \overline{\alpha} (x|y) + |\alpha| \|y\|^2
\]
W"ahle $\alpha = \frac{(x|y)}{|(x|y)|} \Longrightarrow (x|y) = 0$.
\end{enumerate}
\end{Bew}

\end{document}