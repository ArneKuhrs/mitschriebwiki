\documentclass{article}
\newcounter{chapter}
\setcounter{chapter}{19}
\usepackage{ana}

\title{Messbare Mengen und messbare Funktionen}
\author{Wenzel Jakob}

\begin{document}
\maketitle

\def\L{\mathfrak{L}}

\indexlabel{messbar}
\indexlabel{Lebesguemaß}
\begin{definition}
$A\subseteq\MdR^n$ hei"st \textbf{(Lebesgue-)messbar} (mb) :$\equizu\exists$ Folge quadrierbarer Mengen $(A_k)$ mit
\[
	A=\bigcup_{k=1}^\infty A_k
\]
$\L_n:=\{A\subseteq\MdR^n:\ A \text{ ist messbar}\}$. Ist $A$ quadrierbar $\folgt A\in\L_n$. Die Abbildung $\lambda_n\to\tilde{\MdR}$ definiert durch
\[
	\lambda_n(A):=\begin{cases}
		v_n(A) &\text{, falls } A \text{ quadrierbar}\\
		\infty &\text{, falls } A \text{ nicht quadrierbar}
	\end{cases}
\]
hei"st das \textbf{n-dimensional Lebesguema"s}.
\end{definition}

\begin{beispiel}
$\MdR^n\in\L_n, \lambda_n(\MdR^n)=\infty$
\end{beispiel}

\begin{satz}
Es seien $A, B, A_1, A_2, \ldots\ \in\L_n$
\begin{liste}
\item $A\ \backslash \ B,\ \ds\bigcup_{j=1}^\infty A_j,\ \ds\bigcap_{j=1}^\infty A_j\in\L_n$.
\item Sei $B\subseteq A$
\begin{liste}
\item $\lambda_n(B)\le\lambda_n(A)$.
\item Ist $B$ quadrierbar $\folgt\lambda_n(A\ \backslash\ B)=\lambda_n(A)-\lambda_n(B)$
\end{liste}
\item $\lambda_n(\ds\bigcup_{j=1}^\infty A_j)\le\ds\sum_{j=1}^\infty\lambda_n(A_j)$.
\item Aus $A_1\subseteq A_2\subseteq A_3\subseteq\ldots$ folgt
\[
\lambda_n(\bigcup_{j=1}^\infty A_j)=\lim_{j=1} \lambda_n(A_j)
\]
\item Ist $A_1$ quadrierbar und $A_1\supseteq A_2\supseteq A_3\supseteq\ldots$ folgt
\[
\lambda_n(\bigcap_{j=1}^\infty A_j)=\lim_{j=1} \lambda_n(A_j)
\]
\item Ist $A_j\cap A_k=\emptyset\ (j\ne k)$ folgt
\[
\lambda_n(\bigcup_{j=1}^\infty A_j)=\sum_{j=1}^\infty \lambda_n(A_j)
\]
\end{liste}
Ohne Beweis!
\end{satz}

\begin{folgerung}
\begin{liste}
\item Ist $A\subseteq\MdR^n$ offen $\folgt A\in\L_n$
\item Ist $A\subseteq\MdR^n$ abgeschlossen $\folgt A\in\L_n$
\end{liste}
\end{folgerung}

\begin{beweise}
\item folgt aus 17.10
\item $\MdR^n\ \backslash\ A$ ist offen \folgtnach{(1)} $\MdR^n\ \backslash\ A\in\L_n\folgtnach{19.1(1)}A\in\L_n$.
\end{beweise}

\indexlabel{messbar}
\begin{definition}
Sei $A\in\L_n$ und $F:A\to\tilde{\MdR}$ eine Funktion. $f$ hei"st \textbf{messbar}$:\equizu\exists$ Folge $(\varphi_k)$ in $\T_n$: $(\varphi_k)$ konvergiert fast "uberall auf $\MdR^n$ punktweise gegen $f_A$.
\end{definition}

\begin{satz}
$A\in\L_n, f,g:A\to\tilde{\MdR}$ seien Funktionen.
\begin{liste}
\item Ist $f\in L(A)\folgt f$ ist messbar.
\item Sind $f$,$g$ messbar \folgt $f+g, f^{+}, f^{-}, cf\ (c\in\MdR), |f|^p\ (p>0), \max(f,g), \min(f,g)$ sind messbar $(\infty^p:=\infty)$
\end{liste}
Ohne Beweis!
\end{satz}

\end{document}
