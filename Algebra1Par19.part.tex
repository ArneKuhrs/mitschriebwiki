\section{Kompositionsreihen}

\textbf{Vorüberlegung}: $G$ Gruppe, $N
\trianglelefteq G$ Normalteiler und $G/N$ die Faktorgruppe.
\newline Läßt sich nun $G$ aus $N$ und $G/N$
rekonstruieren? \smallskip
\newline \textbf{Schreibweise}: \[\begin{array}{lr}1 \ra N \ra G \ra G/N \ra 1 & (\ast) \end{array}\] ist
\emp{exakt}.

\begin{Def}
Sei $(\ast) \dots \ra G_{i-1}
\overset{\alpha_{i-1}}{\ra} G_i \overset{\alpha_i}{\ra} \dots$ eine
Sequenz (Folge) von Gruppen und Gruppenhomomorhpismen.
\newline $(\ast)$ heißt \emp{exakt} an der Stelle $i$, wenn
Kern$(\alpha_i) =$ Bild$(\alpha_{i-1})$. \smallskip\newline \bsp{
\[0 \ra \mathbb{Z}/2\mathbb{Z} \ra \mathbb{Z}/4\mathbb{Z} \ra
\mathbb{Z}/2\mathbb{Z} \ra 0\] und \[0 \ra \mathbb{Z}/2\mathbb{Z}
\ra \mathbb{Z}/2\mathbb{Z} \oplus \mathbb{Z}/2\mathbb{Z} \ra
\mathbb{Z}/2\mathbb{Z} \ra 0\] sind exakt.}

Die Aufgabe, Gruppen zu klassifizieren zerfällt in zwei
Teilaufgaben:
\begin{enumerate}
\renewcommand{\labelenumi}{(\theenumi)}
\item Geg.: $N$ und $G/N$. Welche Möglichkeiten gibt es für $G$?
\item Welche ''unzerlegbaren'' Gruppen gibt es?
\end{enumerate}
\end{Def}

\begin{Def}
Sei $G$ eine Gruppe.
\begin{enum}
\item $G$ heißt \emp{einfach}, wenn $G$ nur die trivialen
Normalteiler $G$ und $\{e\}$ besitzt.
\item Eine Reihe der Form
\[\begin{array}{lr} G = G_0 \trianglerighteq G_1 \trianglerighteq G_2
\dots \trianglerighteq G_n = \{e\} & (\ast \ast) \end{array}\] (für ein
$n \in \mathbb{N}$) heißt \emp{Normalreihe}, wenn $G_{i+1}$
Normalteiler in $G_i$ ist ($i=0,\dots,n-1$) und $G_{i+1} \neq G_i$.
\item Eine Normalreihe heißt \emp{Kompositionsreihe}, wenn sie sich
nicht verfeinern läßt, dh. wenn $G_i/G_{i+1}$ einfach ist für
$i=0,\dots,n-1$
\end{enum}
\end{Def}

\begin{Bem}
\mbox{}
\begin{enum}
\item $\mathbb{Z}/n\mathbb{Z}$ ist einfach $\lra$ $n$ ist Primzahl.
\item $\mathbb{Z}$ besitzt keine Kompositionsreihe.
\item Eine abelsche Gruppe ist einfach $\lra G \cong
\mathbb{Z}/p\mathbb{Z}$ für eine Primzahl $p$.
\item Jede endliche Gruppe besitzt eine Kompositionsreihe.
\item Ist $G$ endlich, $(\ast \ast)$ eine Normalreihe, so gilt:
\[ |G| = \prod_{i=0}^{n-1} | G_i/G_{i+1} |\]
\end{enum}
\end{Bem}

\begin{Prop}
Für $n \neq 4$ ist $A_n$ einfach. \[|A_4|
= 12\] $A_4$ enthält $4 \cd 2$ Dreizyklen und $3$ ''Doppelzweier''.
\newline $A_4$ ist auch die Symmetriegruppe des Tetraeders.
\newline\bew{}{
\item[(1)] Jedes $\sigma \in A_n$ ist als Produkt von
$3$-Zyklen darstellbar.
\newline \textbf{denn:} \[(12)(23) = (123)\] \[(12)(34) =
(123)(234)\]
$A_n$ lässt sich als Produkt mit gerader Anzahl an Transpositionen schreiben.
\item[(2)] Je zwei $3$-Zyklen in $A_n$ sind konjugiert ($n \geq 5$)
\newline \textbf{denn:} z.z.: $(i j k)$ ist zu $(123)$ konjugiert.
\smallskip
\newline\textbf{1. Fall:} (ijk) = (132)
\newline Sei $p = (23) \Ra p^{-1} (132) p = (123)$, aber $p \not \in
A_n$.
\newline \textbf{Rettung:} $p = (23)(45) \Ra
p^{-1}(132)p=(123)$
\item[(3)] Enthält $N$ einen ''Doppelzweier'', so ist $N=A_n$ \newline(da $N$
Normalteiler in $A_n$).
\newline \textbf{denn:} Sei $\sigma = (12)(34) \in N$, sei $\tau =
(12)(35)$.
\newline Dann ist $\underset{\in N}{\underbrace{\sigma}}
\underset{\in N}{\underbrace{(\tau \sigma
\tau^{-1})}} = (345) \in N$
\item[(4)] $N$ enthält einen $3$-Zyklus oder einen Doppelzweier.
\newline
\sbew{0.8}{Es genügt zu zeigen: $N$ enthält ein $\sigma \neq id$ mit
$\sigma(i) \neq i$ für höchstens $4$ verschiedene $i \in \{1,\dots,
n\}$ Für jedes $\sigma \in A_n$ sei $k_{\sigma} \defeqr |\{ i \in
\{1,\dots,n\} : \sigma(i) \neq i\}| $.
\newline Sei $\sigma \in N\setminus\{id\}$ mit minimalem $k_{\sigma}$
Annahme: $k_{\sigma} \geq 5$
\newline \textbf{1. Fall} $\sigma$ enthält einen Zyklus der Länge
$\geq 3$
\newline \OE:  $\sigma(1) = 2$, $\sigma(2) = 3$, $\sigma(4)
\neq 4$, $\sigma(5) \neq 5$.
\newline Sei $\alpha \defeqr \sigma^{-1} (345) \sigma (354)$.
\newline Für alle $i$ mit $\sigma(i) = i$ ist $\alpha(i) = i \Ra
k_\alpha < k_\sigma$. Außerdem ist $\alpha(1) = 1 \Ra k_\alpha <
k_\sigma \Ra \blitzc$!
\newline \textbf{2.Fall:} $\sigma$ ist Produkt von disjunkten
Transpositionen (mind. 4).
\newline \OE: $\sigma = (12)(34)(56)(78)\widetilde{\sigma}$ mit
$\widetilde{\sigma} \in A_n$, $\widetilde{\sigma}(i) = i$ für
$i=1,\dots,8$ $\alpha = \sigma^{-1}(345)\sigma(354)$ erfüllt
$\alpha(i)=i$, falls $\sigma(i) = i$, und $\alpha(1) = 1 \Ra
k_{\alpha} < k_{\sigma} \Ra \blitzc$!  }\newline -}
\end{Prop}

\begin{Satz}[Jordan-Hölder]
Sei $G$ eine Gruppe, \[G = G_0
\triangleright \dots \triangleright G_m = \{1\}\]\[G = H_0 \triangleright
H_1 \dots \triangleright H_l = \{1\}\] Kompositionsreihe für $G$.
\smallskip\newline Dann ist $m =l$ und es gibt eine Permutation
$\sigma \in S_m$ mit $G_i / G_{i+1} \cong
H_{\sigma(i)}/H_{\sigma(i)+1}$, $i=0,\dots,m-1$.\newline

\sbew{1.0}{Induktion über $m$:
\begin{description}
\item[$\mathbf{m=1}$:] Dann ist $G$ einfach, also auch $l=1$
\item[$\mathbf{m>1}$:] Sei $\bar G \defeqr G/G_1, \pi: G \ra \bar G$ die
Restklassenabbildung.
\newline$\Ra \bar{H_i} = \pi(H_i)$ ist
Normalteiler in $\bar{H_{i-1}}$
\newline (Sei $\overset{=\pi(h)}{\bar{h_i}} \in \bar H_i$, $\overset{=\pi(g)}{\bar
g} \in \bar{H_{i-1}} \Ra \bar g \bar{h}_i \bar{g}^{-1} = \pi(\underset{\in
H_i}{\underbrace{gh_ig^{-1}}}) \in \bar{H_i}$ )
\newline Nach Voraussetzung ist $\bar G$ einfach $\Ra \exists j \in
\{0,\dots,l-1\}$ mit $ \bar{H_0} = \dots = \bar{H_j} = \bar{G},
\bar{H_{j+1}} = \dots = \bar{H_l} = \{1\}$.
\newline Sei $C_i \defeqr H_i \cap G_1$.
\smallskip\newline \textbf{Beh.1}: $G_1 = C_0 \triangleright C_1
\triangleright \dots \triangleright C_j \triangleright C_{j+2} \triangleright \dots
\triangleright C_l = \{1\}$ ist Kompositionsreihe für $G$.
\newline Dann: $G_1 \triangleright G_2 \triangleright G_2
\triangleright \dots \triangleright G_m = \{1\}$ ist ebenfalls
Kompositionsreihe. $\overset{\mbox{IV}}{\Ra} m-1 = l-1$ und es gilt
$\sigma: \{1, \dots, m\} \to \{0,\dots,j,j+2,\dots,l\} $
bijektiv mit \[C_{i-1}/C_i \cong G_{\sigma(i)-1}/G_{\sigma(i)}
\mbox{ für } i\neq j + 1\] und $\ds C_j/C_{j+1} \cong
G_{\sigma(j)}/G_{\sigma(j)+2}$. \medskip\newline\textbf{Beh.2}
\begin{enum}
\item $C_j = C_{j+1}$
\item $C_{i-1}/C_i \cong H_{i-1}/H_i$ für $i \neq j+1$
\item $H_j/H_{j+1} \cong G/G_1 = \bar G$
\end{enum}
\smallskip\textbf{Beh.1 folgt aus Beh.2:}
\newline $C_i$ ist Normalteiler in $C_{i-1}$ ($i=1,\dots,n$)
% TODO was ist den bitte n? ich hab leider kein Bereich für i dastehen
\newline$x \in C_i = H_i \cap G_1$, $y = H_{i-1} \cap G_1 \Ra yxy^{-1} \in H_i \cap
G_1$.
\newline $C_{j+1}$ ist Normalteiler in $C_j$ wegen Beh.2(a).
\newline $C_{i-1}/C_i$ sind wegen Beh.2(b) einfach und $\neq \{1\}$
($i\neq j+1$) \end{description}} \bew{\textbf{Bew. von Beh.2:}}{
\item $\bar{H}_{j+1} = \{1\}$, dh. $H_{j+1} \subseteq G_1 \Ra C_{j+1} =
H_{j+1}$. $C_j = H_j \cap G_1$ ist Normalteiler in $H_j$. (weil $G_1$
Normalteiler in $G$ ist)
\newline Da $\bar{H_j} \neq \{1\}$, ist $C_j \neq H_j \Ra H_{j+1}
\trianglelefteq C_j \triangleleft H_j \overset{H_j/H_{j+1}
\mbox{einfach}}{\Ra} C_j = H_{j+1} = C_{j+1}$
\item Für $i \geq j+1$ ist $\bar{H_i} = \{1\}$, also $H_i \subseteq
G_1$ und damit $C_i = H_i$.
\newline Für $i \leq j$ ist $\bar{H_i} = \bar G = G/G_1 \Ra H_i G_1
= G_1 H_i = G$
\[ C_{i-1}/C_i = C_{i-1}/H_i \cap C_{i-1}
\overset{\mbox{Ü4 A1}}{\cong} 
C_{i-1}H_i/H_i\] zu zeigen also: $C_{i-1} H_i = H_{i-1}$
\smallskip\newline \textbf{denn:}
$''\subseteq'' \chk$
\newline $''\supseteq'':$ Da $G_1 H_i = G$ ist, gibt es zu $x \in
H_{i-1}$ ein $h \in H_i$ und $g \in G_1$ mit $x=gh \Ra g=xh^{-1} \in H_{i-1}
\cap G_1 = C_{i-1}$
\item $\ds H_{j+1} \subseteq G_1 \Ra H_j/H_{j+1} = H_j/C_{j+1}
\overset{(a)}{=} H_j/C_j = H_j/H_j \cap G_1 \cong H_j G_1/G_1 = G/G_1$ }
\end{Satz}

\begin{DefBem}
\mbox{}
\begin{enum}
\item Eine Gruppe heißt \emp{auflösbar}, wenn sie eine Normalreihe
mit abelschen Faktorgruppen besitzt.
\item Eine endliche Gruppe ist genau dann auflösbar, wenn die
Faktoren in ihrer Kompositionsreihe zyklisch von Primzahlordung
sind.
\item Sei $1 \ra G' \ra G \ra G'' \ra 1$ kurze exakte Sequenz von Gruppen.
Dann gilt: \newline$G$ auflösbar $\lra G'$ und $G''$ auflösbar.
\end{enum}
\end{DefBem}
