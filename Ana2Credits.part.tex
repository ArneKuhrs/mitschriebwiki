\chapter{Credits f�r Analysis II} Abgetippt haben die folgenden Paragraphen:\\% no data in Ana2Vorwort.tex
\textbf{� 1: Der Raum $\MdR^n$}: Wenzel Jakob, Joachim Breitner\\
\textbf{� 2: Konvergenz im $\MdR^n$}: Joachim Breitner und Wenzel Jakob\\
\textbf{� 3: Grenzwerte bei Funktionen, Stetigkeit}: Wenzel Jakob, Pascal Maillard\\
\textbf{� 4: Partielle Ableitungen}: Joachim Breitner und Wenzel Jakob\\
\textbf{� 5: Differentiation}: Wenzel Jakob, Pascal Maillard, Jonathan Picht\\
\textbf{� 6: Differenzierbarkeitseigenschaften reellwertiger Funktionen}: Jonathan Picht, Pascal Maillard, Wenzel Jakob\\
\textbf{� 7: Quadratische Formen}: Wenzel Jakob\\
\textbf{� 8: Extremwerte}: Wenzel Jakob\\
\textbf{� 9: Der Umkehrsatz}: Wenzel Jakob und Joachim Breitner\\
\textbf{� 10: Implizit definierte Funktionen}: Wenzel Jakob\\
\textbf{� 11: Extremwerte unter Nebenbedingungen}: Pascal Maillard\\
\textbf{� 12: Wege im $\MdR^n$}: Joachim Breitner, Wenzel Jakob und Pascal Maillard\\
\textbf{� 13: Wegintegrale}: Pascal Maillard und Joachim Breitner\\
\textbf{� 14: Stammfunktionen}: Joachim Breitner und Ines T�rk\\
\textbf{� 15: Integration von Treppenfunktionen}: Ines T�rk\\
\textbf{� 16: Das Lebesguesche Integral}: Joachim Breitner, Pascal Maillard und Jonathan Picht\\
\textbf{� 17: Quadrierbare Mengen}: Joachim Breitner, Pascal Maillard und Wenzel Jakob\\
\textbf{� 18: Konvergenzs�tze}: Wenzel Jakob und Pascal Maillard\\
\textbf{� 19: Messbare Mengen und messbare Funktionen}: Wenzel Jakob\\
\textbf{� 20: Satz von Fubini / Substitutionsregel}: Wenzel Jakob\\
\textbf{� 21: Parameterabh�ngige Integrale}: Wenzel Jakob\\
