\documentclass[a4paper,12pt]{report}

% Deutsche Sprache
\usepackage{ngerman}

% Verschiebt \sections auf die naechste seite falls sie zu tief sind. Muss vor
% hyperref kommen.
\usepackage[nobottomtitles]{titlesec}

% Schicke Schrift
\usepackage[utf8]{inputenc}
\usepackage[T1]{fontenc}
\usepackage{lmodern}

% schmaler rand
\usepackage{geometry}
\geometry{a4paper,tmargin=2cm,lmargin=2cm,rmargin=2cm}

\setlength\parskip{\smallskipamount}
\setlength\parindent{0pt}
\tolerance=900

% Vokabelliste erzeugen
\usepackage{index}
\newindex{default}{idx}{ind}{Vokabeln}

% links
\usepackage{color}
\usepackage{hyperref}

\definecolor{rltred}{rgb}{0.75,0,0}
\definecolor{rltgreen}{rgb}{0,0.5,0}
\definecolor{rltblue}{rgb}{0,0,0.75}

\hypersetup{
  pdftitle={Algebraische Geometrie Prof. Herrlich},
  pdfsubject={Algebraische Geometrie},
  pdfkeywords={Algebraische Geometrie Herrlich},
  pdfproducer={pdfLaTeX},
  pdfpagemode={UseOutlines},
  colorlinks=true,
  bookmarksopen=true,
  bookmarksnumbered=true,
  urlcolor=rltblue,
  filecolor=rltgreen,
  linkcolor=rltblue,
  backref=true,
  pagebackref=true,
  pdfpagemode=None
}

% Mathe-Pakete
\usepackage{amssymb}
\usepackage{amsmath}
\usepackage{amsfonts}
\usepackage{stmaryrd}

% Für Diagramme
\usepackage[arrow,matrix,curve]{xy}

\usepackage{etex}
\usepackage{pictex}
\usepackage{graphicx}

% Theorem-Umgebung
\usepackage[hyperref,amsmath,thmmarks,thref]{ntheorem}

% keine kursiv schrift in theorems
\theorembodyfont{}


% Theoreme definieren
\theoremstyle{break}
    \newtheorem{Satz}{Satz}
    \newtheorem{SatzDef}[Satz]{Satz + Definition}
    \newtheorem{Def}{Definition}[section]
    \newtheorem{DefBem}[Def]{Definition + Bemerkung}
    \newtheorem{Bem}[Def]{Bemerkung}
    \newtheorem{BemDef}[Def]{Bemerkung + Definition}
    \newtheorem{Prop}[Def]{Proposition}
    \newtheorem{PropDef}[Def]{Proposition + Definition}
    \newtheorem{Folg}[Def]{Folgerung}
    \newtheorem{Bsp}[Def]{Beispiele}
    \newtheorem{DefProp}[Def]{Definition + Proposition}
    \newtheorem{anBew}[Def]{Beweis}
    \newtheorem{Kor}[Def]{Korollar}
\theoremstyle{nonumberbreak}
    \newtheorem{Bew}{Beweis}
    \newtheorem{nnBem}{Bemerkung}
    \newtheorem{nnBsp}{Beispiele}
    \newtheorem{nnSatz}{Satz}
    \newtheorem{nnFolg}{Folgerung}
    \newtheorem{Beo}{Beobachtung}
    \newtheorem{Eri}{Erinnerung}
    \newtheorem{Beh}{Behauptung}
\theoremstyle{nonumberplain}
\theoremsymbol{\ensuremath{\Box}}
    \newtheorem{proof}{Beweis}

\newcommand{\emp}[1]{\textbf{\emph{#1}}}
\newcommand{\begriff}[1]{{\index{#1}}\emp{#1}}
\newcommand{\defeqr}[0]{\mathrel{\mathop:}=}
\newcommand{\defeql}[0]{=\mathrel{\mathop:}}

\newcommand{\myref}[2]{%
\hyperref[#2]{#1~\ref*{#2}}%
}

\DeclareMathOperator{\Aut}{Aut}
\DeclareMathOperator{\Hom}{Hom}
\DeclareMathOperator{\Quot}{Quot}

\DeclareMathOperator{\Kern}{Kern}
\DeclareMathOperator{\Bild}{Bild}
\DeclareMathOperator{\Rat}{Rat}
\DeclareMathOperator{\id}{id}

\newcommand{\FakRaum}[2]{
  \raisebox{0.7ex}{\ensuremath{#1}}
  \ensuremath{\mkern-3mu}\big/\ensuremath{\mkern-3mu}
  \raisebox{-0.6ex}{\ensuremath{#2}}} 

\renewcommand{\labelenumi}{\theenumi}    
\renewcommand{\theenumi}{(\alph{enumi})}

\renewcommand{\OE}{O\!\!E}

% Weniger Abstand nach der Ueberschrift des Inhaltsverzeichnises
\makeatletter
\let\@my@starttoc\@starttoc
\renewcommand*{\@starttoc}[1]{%
  \addvspace{-1.5em}%
  \@my@starttoc{#1}%
}
\makeatother

% Einige Anstrengungen, um den § vor die Section-Nummer zu stellen
% \renewcommand{\thesection} allein führt zu einem Konflikt mit ntheorem-hyper
\makeatletter
\def\@seccntformat#1{\@ifundefined{#1@cntformat}%
{\csname the#1\endcsname\quad}% default
{\csname #1@cntformat\endcsname}% individual control
}
\def\section@cntformat{§\@arabic\c@section\quad}
\makeatother
\setcounter{secnumdepth}{-1}

\title{Algebraische Geometrie - Wintersemester 2008/2009\\ Prof. Dr. F. Herrlich}
\author{Die Mitarbeiter von \url{http://mitschriebwiki.nomeata.de/}}

\begin{document}
\maketitle

% Inhaltsverzeichnis
\pdfbookmark[1]{Inhaltsverzeichnis}{contents}
\setlength\parskip{0.6pt}
\tableofcontents

% Liste der benannten Saetze
\section*{Benannte Sätze}
\pdfbookmark[1]{Benannte Sätze}{contents}

\theoremlisttype{optname}
\listtheorems{Satz,SatzDef,Def,DefBem,BemDef,Prop,PropDef,Bsp,DefProp}

\setlength\parskip{\smallskipamount}

\chapter{Vorwort}
\setcounter{secnumdepth}{2}
\section*{Über dieses Skriptum}
Dies ist ein Mitschrieb der Vorlesung \glqq Algebraische Geometrie\grqq\ von Prof. Dr. F. Herrlich im
Wintersemester 08/09 an der Universität Karlsruhe.
Die Mitschriebe der Vorlesung werden mit ausdrücklicher Genehmigung von Prof. Dr. F. Herrlich hier veröffentlicht,
Prof. Dr. F. Herrlich ist für  den Inhalt nicht verantwortlich.
\section*{Wer}
Getippt wurde das Skriptum soweit von Diego de Filippi und Felix Wellen, die Wiki-Technik ist von Joachim Breitner.

\section*{Wo}
Alle Kapitel inklusive \LaTeX-Quellen können unter \url{http://mitschriebwiki.nomeata.de} abgerufen werden.
Dort ist ein von Joachim Breitner programmiertes \emph{Wiki}, basierend auf \url{http://latexki.nomeata.de} installiert. 
Das heißt, jeder kann Fehler nachbessern und sich an der Entwicklung
beteiligen. Auf Wunsch ist auch ein Zugang über \emph{Subversion} möglich.


\chapter{Affine Varietäten}
\section{Der Polynomring}
Sei $k$ ein Körper, $k[X_1,...,X_n]$, $n\geq 0$ der Polynomring über $k$ in $n$ Variablen.
\subsection*{Universelle Abbildungseigenschaft (UAE) des Polynomrings}
Ist $A$ eine $k$-Algebra und sind $a_1,...,a_n\in A$, so gibt es genau einen $k$-Algebra-Homomorphismus $f: k[X_1,...,X_n] \rightarrow A$ mit $f(X_i)=a_i$ für $i= 1,...,n$\\
\underline{Folgerung}: Jede endlich erzeugte $k$-Algebra ist Faktorring eines Polynomrings.
\subsection*{$n=1$, also $k[X]$}
Euklidischer Algorithmus: Zu $f,g\in k[X], g\neq 0$ gibt es $q,r \in k[X]$ mit $f=qg+r$ und $deg(r)<deg(g)$ oder $r=0$.\\
\underline{Folgerung}: $k[X]$ ist Hauptidealring.
\subsection*{Eindeutige Primfaktorzerlegung}
$k[X_1,...,X_n]$ ist faktorieller Ring.\\
\underline{Folgerung}: Jedes irreduzible Polynom erzeugt ein Primideal.
\subsection*{Hilbertscher Basissatz}
$k[X_1,...,X_n]$ ist noethersch, d.h.\\
$\bullet$ Jedes Ideal ist endlich erzeugbar.\\
$\bullet$ Jede aufsteigende Kette von Idealen wird stationär.
\section{Die Zariski-Topologie}
Sei $k$ ein algebraisch abgeschlossener Körper.
\begin{Def}
Eine Teilmenge $V\subseteq k^n$ heißt \begriff{affine Varietät}, wenn es eine Menge von Polynomen $F \subseteq k[X_1,...,X_n]$ gibt, so dass 
$V(F)=V= \{x= (x_1,...,x_n)\in k^n: f(x)=0 \textrm{~für~alle~} f \in F\}$
\end{Def}
\begin{Bsp}
1) $n=1:~~V\subseteq k$ affine Varietät $\Leftrightarrow V$ endlich oder $V=k$ \\
2) $f\in k[X_1,...,X_n]$ linear (d.h. deg($f$)=1)$\Rightarrow V(f)$ ist affine Hyperebene.\\
$f_1,...,f_r$ linear $\Rightarrow V(f_1,...,f_r)$ ist affiner Unterraum. (Jeder affine Unterraum lässt sich so beschreiben).\\
3) Quadriken sind affine Varietäten.\\
4) Lemniskate
$$C=\{P(x,y) \in \mathbb{R}^2: d(P,P_1)=d(P,P_2)=c\}$$
für Punkte $P_1P_2\in k^2, c>0$.\\
Für $P_1(-1,0)$ und $P_2(1,0)$ ist $C=V(f)$ mit $f=((x+1)^2+y^2)((x-1)^2+y^2)-1$.
Dies ist aber keine affine Varietät, da das in $\mathbb{C}^2$ nicht klappt.
\end{Bsp}
\begin{Bem}
(i) Für $F_1 \subseteq F_2 \subseteq k[X_1,...,X_n]$ ist $V(F_1)\supseteq V(F_2)$\\
(ii) $V(f_1\cdot f_2)= V(f_1)\cup V(f_2)$\\
(iii) $V(F)=V((F))$ für das von $F$ erzeugte Ideal $(F)\subset k[X_1,...,X_n]$\\
(iv) $V(F)=V(\sqrt{(F)})$ für das von $F$ erzeugte Radikalideal
$$\sqrt{(F)}=\{g\in k[X_1,...,X_n]: \exists d>0 ~ mit~ g^d \in (F) \}$$
(v) Zu jeder affinen Varietät $V\subseteq k^n$ gibt es endlich viele Polynome $f_1,...,f_r$, so dass \\ $V=V(f_1,...,f_r)$, da jedes Ideal in $k[X_1,...,X_n]$ endlich erzeugbar ist.
\end{Bem}
\begin{proof}
"' $\subseteq$"' Sei $x\in V(F), g\in (F)$. Schreibe $g=a_1f_1+...+a_rf_r$ mit $f_i \in F, a_i \in k[X_1,...,X_n]$, dann ist $g(x)=a_1(x)f_1(x)+...+a_r(x)f_r(x)=0$
\end{proof}
\begin{Def}
(i) Für eine Teilmenge $V\subseteq k^n$ heißt
$I(V):=\{f\in k[X_1,...,X_n]: f(x)=0 \text{ für alle } x \in V\}$
das \begriff{Verschwindungsideal}.\\
(ii) $A(V):= k[X_1,...,X_n]/I(V)$ heißt \begriff{affiner Koordinatenring} von $V$\\
Für $f,g\in k[X_1,...,X_n]$ gilt: $f\mid_V = g\mid_V\Leftrightarrow f-g \in I(V)$
\end{Def}
\begin{Bem}
Für jede Teilmenge $V\subseteq k^n$ gilt:\\
(i) $I(V)$ ist Radikalideal,\\
(ii) $V\subseteq V(I(V))$,\\
(iii) $V(I(V))$ ist die kleinste Varietät, die $V$ umfasst. Schreibweise: $V(I(V))\defeql\overline{V}$.\\
(iv) Sind $V_1, V_2$ affine Varietäten, so gilt:
$$V_1\subseteq V_2 \Leftrightarrow I(V_1)\supseteq I(V_2)$$
\end{Bem}
\begin{proof}
(iii) Sei $V'$ eine affine Varietät mit $V\subseteq V'$ und sei $I'\subseteq k[X_1,...,X_n]$ ein (das) Ideal mit $V'=V(I')$. Dann ist $I'\subseteq I(V)\Rightarrow V(I')\supseteq V(I(V))$.\\
(iv) "'$\Leftarrow$"' $I(V_1)\supseteq I(V_2)\Rightarrow V(I(V_1))\subseteq V(I(V_2))$. Mit $V_1 = V(I(V_1))$ und $V_2= V(I(V_2))$ folgt die Behauptung.
\end{proof}
\begin{Bem}
Für jede Teilmenge $V\subseteq k^n$ gilt:\\
(i) $A(V)$ ist reduzierte $k$-Algebra, d.h. es gibt in $A(V)$ keine nilpotenten Elemente (also $f^d\neq 0$ für alle $f \neq 0, d>0$).\\
(ii) Ist $V\subseteq V'$, so gibt es einen surjektiven $k$-Algebra-Hohomorphismus $A(V')\longrightarrow A(V)$.
\end{Bem}
\begin{proof}
(i) Sei $g\in A(V), f\in k[X_1,...,X_n]$ mit $\overline{f}=g$. Dann ist ($g^d=0$ (in $A(V)$) $\Leftrightarrow f^d\in I(V)$) und da $I(V)$ Radikalideal ist, folgt $f\in I(V)$ und somit $g=0$.\\
(ii) Es ist $I(V')\subseteq I(V)$, also 
\[
\begin{xy}
\xymatrix{
k[X_1,...,X_n]\ar[r]\ar[d]&A(V)=k[X_1,...,X_n]/I(V) \\
A(V')=k[X_1,...,X_n]/I(V')\ar@{.>}[ur]_{\exists !}&
}
\end{xy}
\]
\end{proof}
\begin{Def} {\textbf{und Satz}}\\
Die affinen Varietäten in $k^n$ bilden die abgeschlossenen Mengen einer Topologie, der \begriff{Zariski-Topologie}.
\end{Def}
\begin{proof}
$\bullet~~ k^n=V(0)$ und $\emptyset = V(1)$ sind affine Varietäten.\\
$\bullet$ Seien $V_1=V(I_1)$ und $V_2=V(I_2)$ affinen Varietäten. Dann ist $V_1\cup V_2=V(I_1\cdot I_2)=V(I_1\cap I_2)$.\\
\underline{Denn:} "'$\subseteq$"' klar "'$\supseteq$"': Sei $x\in V(I_1\cdot I_2), x\notin V_1$. (Zu zeigen: $x\in V_2$).\\
Dann gibt es ein $f\in I_1$ mit $f(x) \neq 0$.\\
Da $x\in V(I_1\cdot I_2)$ ist $f(x)\cdot g(x)=0$ für alle $g\in I_2 \Rightarrow x\in V(I_2) = V_2$.\\
$\bullet$ Seien $V_i = V(I_i), i\in J,$ affine Varietäten $\Rightarrow \bigcap_{i\in J}V_i = V(\sum_{i\in J}I_i)$.\\
\underline{Denn:} "'$\supseteq$"' klar "'$\subseteq$"': Sei $x\in\bigcap V_i, f\in \sum I_i$. Schreibe $f=a_1f_1+...+a_rf_r$ mit $f_k\in I_{i_k}, a_k \in k[X_1,...,X_n] \Rightarrow f(x)=a_1(x) \cdot 0 +...+ a_r(x)\cdot 0 = 0$
\end{proof}
\begin{Bem}
(i) Für $f\in k[X_1,...,X_n]\backslash \{0\}$ ist $D(f):= k^n-V(f)$ nichtleere offene Teilmenge von $k^n$\\ 
(ii) Die $D(f)$ bilden eine Basis der Zariski-Topologie. 
\end{Bem}
\begin{proof}
(ii) Zu zeigen: Jede offene Menge $U$ ist Vereinigung von Mengen der Form $D(f)$.\\
Zeige dazu: Zu jedem  $x\in U$ gibt es ein $f$ mit $x\in D(f)\subseteq U$.\\
Sei $V=k^n-U$, also $V=V(I)$ für ein Ideal $I$. Da $x\notin V$, gibt es $f\in I$
mit $f(x)\neq 0 \Rightarrow x\in D(f)$. Weil $f\in I$, ist $V\cap D(f) = \emptyset \Rightarrow D(f) \subseteq U$
\end{proof}
\begin{Bem}
Die Zariski-Topologie auf $k^n$ ist nicht hausdorffsch.
\end{Bem}

\begin{proof} Wegen 2.7 genügt zu zeigen, dass $D(f)\cap D(g)\neq \emptyset$ für alle $f,g \in k[X_1,...,X_n]\setminus\{0\}$\\
\underline{Inuktion über $n$}:~\\
\underline{$n=1$}: $V(f)$ und $V(g)$ sind endlich $\Rightarrow D(f)\cap D(g)=k-V(f\cdot g)$ ist unendlich.\\
\underline{$n>1$}:  Zerlege $f$ und $g$ in Primfaktoren (vgl. 1.3) und wähle $a\in k$, so dass $(X_n-a)$ nicht Teiler von $f$ oder $g$ ist. Identifiziere $V(X_n-a)= \{(x_1,...,x_n)\in k^n: x_n=a\}$ mit $k^{n-1}$.\\
Nach der Wahl von $a$ sind $f\mid_{V(X_n-a)}$ und $g\mid_{V(X_n-a)}$ nicht identisch $0$, also $f'=f(X_1,...,X_{n-1},a) \neq 0 \neq g(X_1,...,X_{n-1},a)=:g'$ in $k[X_1,...,X_n]$. Nach Indunktionsvoraussetzung gibt es $x'\in k^{n-1}$ mit $f'(x')\neq 0\neq g'(x')\Rightarrow$ für $x=(x',a)\in k^n$ gilt $f(x)=f'(x')\neq 0 \neq g'(x')=g(x)$
\end{proof}
\section{Irreduzible Komponenten}
\begin{Def}
a) Ein topologischer Raum $X$ heißt \underline{irreduzibel}, wenn es nicht Vereinigung von zwei echten abgeschlossenen Teilmengen ist.\\
b) Eine abgeschlossene Teilmenge von $X$ heißt \underline{irreduzible Komponente}, wenn sie irreduzibel ist (bzgl. der induzierten Topologie) und maximal (bzgl. Inklusion)
\end{Def}
\begin{Prop}
Eine affine Varietät $V\in k^n$ ist genau dann irreduzibel, wenn $I(V)$ Primidel in $k[X_1,...,X_n]$ ist. Das ist genau dann der Fall, wenn der affine Koordinatenring $A(V):=k(V)$ nullteilerfrei ist.
\end{Prop}
\begin{proof}
\underline{"'$\Rightarrow$"'} Seien $f_1, f_2\in k[X_1,...,X_n]$ mit $f_1\cdot f_2\in I(V)$. Sei $f_1\notin I(V)$.\\
Dann ist $V\nsubseteq V(f_1)$. \\
Nach Voraussetzung ist $V\subseteq V(f_1\cdot f_2)= V(f_1)\cup V(f_2)$.\\
$V$ irreduzibel $\Rightarrow V\subseteq V(f_2)\\
\Rightarrow f_2(x)=0$ für alle $x\in V\\
\Rightarrow f_2\in I(V)$.\\
\underline{"'$\Leftarrow$"'} Sei $V=V_1\cup V_2$ mit $V_i=V(I_i)$, $i=1,2$. Sei $V_1 \neq V.\\
\Rightarrow V \nsubseteq V(I_1)\\
\Rightarrow \exists x \in V$ und $f\in I_1$ mit $f(x)\neq 0$\\
Also $f\notin I(V)\subseteq I(V_1)$\\
Andererseits ist $V=V(I_1)\cup V(I_2)= V(I_1\cdot I_2)\Rightarrow I_1\cdot I_2\subseteq I(V)\\
\Rightarrow f\cdot g(x)=0$ für alle $g\in I_2\\
I(V)$ prim und $f\notin I(V)\Rightarrow g\in I(V)$ für alle $g\in I_2\\
\Rightarrow V_2 = V(I_2)\supseteq V(I(V))=V$
\end{proof}
\begin{Satz}
Jede affine Varietät $V\in k^n$ hat eine eindeutige Zerlegung in endlich viele irreduzible Komponenten. Diese Zerlegung ist eindeutig. 
\end{Satz}

\begin{proof}
\underline{1. Schritt} $V$ ist endliche Vereinigung von irreduziblen Untervarietäten.\\
Sei dazu $\mathcal{B}$ die Menge der Varietäten in $k^n$, die nicht Vereinigung von endlichen Untervarietäten sind. Sei weiter $\mathcal{J}:=\{ I(V)\mid V\in \mathcal{B}\}$.\\
Zu zeigen: $\mathcal{B}=\emptyset$\\
Annahme: $\mathcal{J}\neq \emptyset$. Dann enthält $\mathcal{J}$ ein maximales Element $I_0=I(V_0)$ für ein $V_0\in \mathcal{B}\\
\Rightarrow V_0$ ist minimales Element in $\mathcal{B}\\
V_0 \in \mathcal{B} \Rightarrow V_0$ reduzibel\\
$\Rightarrow V_0 = V_1 \cup V_2$ mit $V_1\neq V_0\neq V_2, V_i$ abgeschlossen\\
$\Rightarrow V_i \notin\mathcal{B}, i=1,2$ (da $V_0$ minimales Element in $\mathcal{B})\\
\Rightarrow V_i$ ist endliche Vereinigung von irreduziblen Untervarietäten\\
$\Rightarrow V_0$ auch. Widerspruch!\\
\underline{2. Schritt} "'Irreduzible Komponenten"'\\
Sei $V=V_1\cup ... \cup V_n$ mit irreduziblen Varietäten $V_1,... , V_n$.
\\Ohne Einschränkung sei $V_i\nsubseteq V_j$ für $i\neq j$.\\
Sei $W\subseteq V$ irreduzibel und $V_i\subseteq W$ für ein $i$.\\
Es ist $W=V\cap W= (V_1 \cup .... \cup V_n)\cap W= (V_1 \cap W)\cup ... \cup (V_n \cap W)$\\
$W$ irreduzibel $\Rightarrow\exists j$ mit 
\[
V_j\cap W = W\\
\Rightarrow V_i \subseteq W= W \cap V_j\subseteq V_j\Rightarrow i=j \Rightarrow W=V_i\\
\Rightarrow V_1,..., V_n
\]
 sind irreduzible Komponenten von $V$.\\
Genauso: $W\subseteq V$ irreduzible Komponente $\Rightarrow \exists j : W \subseteq V_j$,\\
da $W$ maximal $\Rightarrow$ Zerlegung eindeutig.
\end{proof}
\begin{Bsp}
$f=y^2-x(x-1)(x+1) \in\mathbb{R}[x,y]~~~~~~~~E:=V(f)$
\end{Bsp}
\section{Der Hilbertsche Nullstellensatz}
\begin{Satz}[Hilbertscher Nullstellensatz]
Sei  $k$ ein Körper, $n\geq 1, m\subseteq k[X_1,...,X_n]$ maximales Ideal. Dann ist $L=k[X_1,...,X_n]/m$ eine endlich erzeugte Körpererweiterung von $k$.
\end{Satz}
\begin{proof} Siehe Algebra II, Theorem 4. \end{proof}
\begin{nnFolg}[4.2]
Ist $k$ algebraisch abgeschlossen, so entsprechen die maximalen Ideale in $k[X_1,...,X_n]$ bijektiv den Punkten in $k^n$.
\end{nnFolg}
\begin{proof}~\\
$x=(x_1,...,x_n)\mapsto (X_1-x_1,...,X_n-x_n)$ (maximal, da Faktorring Körper) ist eine injektive Zuordnung $\varphi: k^n \rightarrow\operatorname{Spec}m(k[X_1,...,X_n])\\
\varphi$ surjektiv:\\
Sei $m\in\operatorname{Spec} m (k[X_1,...,X_n]), \alpha: k[X_1,...,X_n]/m \rightarrow k$ der Isomorphismus, den es nach Satz 4.1 gibt.\\
$\Rightarrow X_i - \alpha(X_i)\in m, i= 1,...,n$ (da $\alpha\in \operatorname{Hom}_k\Rightarrow \alpha(X_i-\alpha(X_i))=0$)\\
$\Rightarrow (X_1-\alpha(X_1),...,X_n-\alpha(X_n)) \subseteq m$
\end{proof}
\begin{nnFolg}[4.3 (Schwacher Nullstellensatz)]
Für jedes echte Ideal $I\subsetneqq k[X_1,...,X_n]$ ist $V(I)\neq \emptyset$.
\end{nnFolg}
\begin{proof}
$I\subseteq m$ für ein maximales Ideal $m \Rightarrow V(I) \supseteq V(m) \neq \emptyset$
\end{proof}
Sei jetzt  $k$ algebraisch abgeschlossen, $n\geq 1$, und
$$\mathcal{V}_n:= \{V\subseteq k^n\mid V \textrm{affine~Varietät}\}$$
$$\mathcal{I}_n:=\{I\subseteq k[X_1,...,X_n]\mid I ~ \textrm{Radikalideal}\}$$

\begin{Satz}[Hilbertscher Nullstellensatz]
Die Zuordnungen
$$V: \mathcal{I}_n \rightarrow \mathcal{V}_n,~~~I\mapsto V(I)$$
$$I: \mathcal{V}_n \rightarrow \mathcal{I}_n,~~~V\mapsto I(V)$$
sind bijektiv und zueinander invers.
\end{Satz}
\begin{proof}
Zu zeigen: (1) $V(I(V))=V$ für jedes $V\in \mathcal{V}_n$\\
(2) $I(V(I))=I$ für jedes $I\in \mathcal{I}_n$\\
(1): Ist Bemerkung 2.4 (iii).\\
(2): Zeige: $I(V(I))= \sqrt{I}$ für jedes Ideal $I\subseteq k[X_1,...,X_n]$\\
\underline{"'$\supseteq$"'}: $\surd$\\
\underline{"'$\subseteq$"'}: Sei $g\in I(V(I))$, seien $f_1,...,f_m$ Erzeuger von $I$.\\
Zu zeigen: $\exists d:g^d=\sum_{i=1}^m a_if_i$ für gewisse $a_i\in k[X_1,...,X_n]$.\\
Betrachte in $k[X_1,...,X_n,Y]$ das von $f_1,...,f_m$ und $gY-1$ erzeugte Ideal $J$.\\
Es ist $V(J)=\emptyset$\\
Schwacher Nullstellensatz $\Rightarrow J=k[X_1,...,X_n,Y]\\
\Rightarrow \exists b_i,b \in k[X_1,...,X_n,Y]$ sodass $1=\sum_{i=1}^m b_if_i + b(gY-1)$\\
In $R:= k[X_1,...,X_n,Y]/(gY-1)$ gilt also\\
$1=\sum_{i=1}^m\tilde{b_i}f_i~~ (\tilde{b_i}\in k[X_1,...,X_n,\frac{1}{g}]$ die Restklasse von $b_i$). Multipliziere mit Hauptnenner $g^d$
\end{proof}
\begin{Bem}
Sei $k$ algebraisch abgeschlossen, $V\subseteq k^n$ eine affine Varietät. Dann entsprechen die Punkte in $V$ bijektiv den maximalen Idealen in $k[V]$ $(=k[X_1,...,X_n]/ I(V))$
\end{Bem}
\begin{proof}
Die maximalen Ideale in $k[V]$ entsprechen bijektiv den maximalen Idealen in $k[X_1,...,X_n]$, die $I(V)$ umfassen, also nach 4.2 den Punkten $(x_1,...,x_n)$, für die $(X_1-x_1,...,X_n-x_n)\supseteq I(V)$ ist\\
$\Leftrightarrow \underbrace{V(X_1-x_1,...,X_n-x_n)}_{\{(x_1,...,x_n)\}}\subseteq V(I(V))= V$
\end{proof}
\section{Morphismen}
\begin{DefBem}
(a) Sei $k$ algebraisch abgeschlossener Körper, $V\subseteq k^n$ und $W\subseteq k^m$ affine Varietäten. Eine Abbildung $f:V\rightarrow W$ heißt \underline{Morphismus}, wenn es Polynome $f_1,...,f_m\in k[X_1,...,X_n]$ gibt, so dass $f(x)=(f_1(x),...,f_m(x))$ für jedes $x\in V$.\\
(b) Jeder Morphismus $V\rightarrow W$ ist Einschränkung eines Morphismus $k^n\rightarrow k^m$.\\
(c) Die affinen Varietäten über $k$ bilden zusammen mit den Morphismen aus (a) eine Kategorie $\operatorname{Aff}(k)$. Als Objekte von $\operatorname{Aff}(k)$ bezeichnen wir $k^n$ mit $\mathbb{A}^n(k)$ 
\end{DefBem}
\begin{Bsp}
(1) Projektionen und Einbettungen $\mathbb{A}^n(k) \rightarrow\mathbb{A}^m(k)$\\\\
(2) Jedes $f\in k[X_1,...,X_n]$ ist Morphismus $\mathbb{A}^n(k)\rightarrow\mathbb{A}^1(k)$\\\\
(3) $V= \mathbb{A}^1(k), W=V(Y^2-X^3)\subseteq \mathbb{A}^2(k)$ ("'Neilsche Parabel"')\\
$f: V\rightarrow W, x\mapsto (x^2,x^3)$ ist Morphismus.\\
$f$ ist bijektiv: injektiv $\surd$\\
surjektiv: Sei $(x,y)\in W\backslash\{(0,0)\}$, d.h. $y^2=x^3$\\
Dann ist $(x,y)= f(\frac{y}{x})= ((\frac{y}{x})^2,(\frac{y}{x})^3)=(\frac{x^3}{x^2},\frac{y^3}{y^2})$, $f(0) = (0,0)$\\
Umkehrabbildung:\\
$g(x,y)=\begin{cases}
0&(x,y)=(0,0)\\
\frac{y}{x}&sonst
\end{cases}$
ist kein Morphismus.\\\\
(4) Sei char($k$)$=p>0, f:\mathbb{A}^n(k)\rightarrow\mathbb{A}^n(k), (x_1,...,x_n)\mapsto (x_1^p,...,x_n^p)$, heißt Frobenius-Morphimus. $f$ ist bijektiv, aber kein Isomorphimus. Die Fixpunkte von $f$ sind die Elemente von $\mathbb{A}^n(\mathbb{F}_p)$.
\end{Bsp}
\begin{Bem}
\label{bem:5.3}
Morphismen sind stetig bezüglich der Zariski-Topologie.
\end{Bem}
\begin{proof}
Ohne Einschränkung sei $\mathbb{A}^n(k)\rightarrow\mathbb{A}^m(k)$. Sei $V\subseteq \mathbb{A}^m(k)$ abgeschlossen, $V=V(I)$ für ein Radikalideal $I\subseteq k[X_1,...,X_n]$. Zu zeigen: $f^{-1}(V)$ abgeschlossen in $\mathbb{A}^n(k)$.\\
Genauer gilt: $f^{-1}(V)=V(J)$ mit $J=\{g\circ f \mid g\in I\}$\\
denn: $x\in f^{-1}(V) \Leftrightarrow f(x)\in V \Leftrightarrow g(f(x))=0$ für alle $g\in I\Leftrightarrow x\in V(J)$
\end{proof}
\begin{Bem}
Jeder Morphismus $f:V\rightarrow W$ induziert einen $k$-Algebra-Homomorphismus $f^{\sharp}: k[W]\rightarrow k[V]$ (durch Hintereinanderschalten).\\
Genauer: Sei $V\subseteq \mathbb{A}^n(k), W\subseteq \mathbb{A}^m(k)$
\[
\begin{xy}
\xymatrix{
k[X_1,...,X_m]\ar[d]\ar[r]^{g\mapsto g\circ f}&k[X_1,...,X_n]\ar[d]\\
k[W]=k[X_1,...,X_m]/I(W)\ar@{.>}[r]^{f^{\sharp}}&k[X_1,...,X_n]/I(V)=k[V]
}
\end{xy}
\]
$f^{\sharp}$ existiert, weil für alle $g\in I(W)$ gilt: $g\circ f(x) = g(f(x))=0$ für alle $x\in V$
\end{Bem}
\begin{Prop}
Sei $f: V\rightarrow W$ ein Morphismus von affinen Varietäten, $\alpha:= f^{\sharp}: k[W]\rightarrow k[V]$ der induzierte $k$-Algebra-Homomorphismus. Seien $x\in V$, $y\in W$ und $m_x\subset k[V]$, $m_y\subset k[W]$ die Verschwindungsidale zum jeweiligen Punkt. Dann gilt:
$$f(x)=y\Leftrightarrow \alpha^{-1}(m_x)= m_y$$
\end{Prop}
\begin{proof}
\underline{"'$\Rightarrow$"'} $g\in m_y \Leftrightarrow g(y)=0\Rightarrow g\circ f(x)=0 \Leftrightarrow \underbrace{g\circ f}_{= \alpha(g)} \in m_x \Leftrightarrow g\in \alpha^{-1}(m_x) \Leftrightarrow m_y \subseteq \alpha^{-1}(m_x)$. Gleichheit folgt daraus, dass $m_y$ maximales Ideal ist.\\
\underline{"'$\Leftarrow$"'} Wäre $f(x)\neq y$, dann gäbe es  ein $g\in k[W]$ mit $g(f(x))=0$ und $g(y)=1$.\\
Andererseits:\\
$\alpha(g)(x)=(g\circ f)(x)=g(f(x))=0\Leftrightarrow \alpha(g)\in m_x \Leftrightarrow g\in \alpha^{-1}(m_x)=m_y \Leftrightarrow g(y)=0$
\end{proof}
\begin{Satz}
\label{satz:3}
Sei $k$ ein algebraisch abgeschlossener Körper. Dann ist 
$$\Phi:\underline{\operatorname{Aff}} \longrightarrow \underline{k-\operatorname{Alg}}^{\circ}$$
$$V \longmapsto k[V]$$
$$f\longmapsto f^{\sharp}$$
eine kontravariante äquivalenz von Kategorien.
Hierbei bezeichnet $\underline{k-\operatorname{Alg}}^{\circ}$ die Kategorie der endlich erzeugten, reduzierten k-Algebren.
\end{Satz}
\begin{proof}
$\Phi$ ist ein Funktor: $\surd$\\
Definiere Umkehrfunktor $\Psi$:~\\
(i) Sei $A\in \underline{k-\operatorname{Alg}}^{\circ}$, $a_1,...,a_n$ Erzeuger von $A$\\
$\Rightarrow p_A: k[X_1,...,X_n] \rightarrow A, X_i \mapsto a_i$ ist surjektiver $k$-Algebra-Homomophismus.\\
Sei $I_A:=$ Kern($p_A$) (Radikalideal)\\
$\Psi(A):= V(I_A)\subseteq k^n$ affine Varietät mit $k[V(I_A)]\cong A$.\\
(ii) Sei $\alpha: A\rightarrow B$ $k$-Algebra-Homomorphismus in $\underline{k-\operatorname{Alg}}^{\circ}$.\\
Definiere die Abbildung $f_{\alpha} := V(I_B) \rightarrow V(I_A)$ durch $f_{\alpha}(y)=x$, falls $m_x=\alpha^{-1}(m_y)$. Diese ist wohldefiniert aufgrund der folgenden 
\begin{Prop}
Sei $A\rightarrow B$ ein Homomorphismus von endlich erzeugten $k$-Algebren, $m\subset B$ ein maximales Ideal. Dann ist $\alpha^{-1}(m)\subset A$ ein maximales Ideal.\\
(\underline{Warnendes Bsp.:} $\alpha:\mathbb{Z}\rightarrow \mathbb{Q},~~~ \alpha^{-1}(0)$ ist kein maximales Ideal.) 
\end{Prop}
\begin{proof}
\[
\begin{xy}
\xymatrix{
A\ar[r]^{\alpha}\ar[d]&B\ar[d] \\
A/\alpha^{-1}(m)\ar@{.>}[r]^{\overline{\alpha}}&B/m
}
\end{xy}
\]
$\alpha$ induziert einen injektiven $k$-Algebra-Homomorphismus $\overline{\alpha}$. Nach den HNS ist $B/m=k$.\\
$k$ hat keine echte $k$-Unteralgebra $\Rightarrow A/\alpha^{-1}(m)=k$.(Prop.)\\
\end{proof}
\paragraph{Ende des Beweises des Satzes}
Noch zu zeigen: $f_{\alpha}: V(I_B) \rightarrow V(I_A)$ ist ein Morphismus.\\
Schreibe dazu $A\cong k[X_1,...,X_n]/I_A$, $B=k[X_1,...,X_n]/I_B$
\[
\begin{xy}
\xymatrix{
k[X_1,...,X_n]\ar@{.>}[r]^{\tilde{\alpha}}\ar[d]&k[Y_1,...,Y_n]\ar[d] \\
A\ar[r]^{\overline{\alpha}}&B
}
\end{xy}
\]
Bastle Lift $\tilde{\alpha}$ von $\alpha$:\\
$\tilde{\alpha}(X_i)=f_i$ mit $\overline{f_i}=\alpha(\overline{X_i})$\\
\underline{Beh.}: Für $y\in V(I_B)$ ist $f_{\alpha}(y)=(f_1(y),...,f_n(y))$\\
\underline{Denn}: Sei $y=(y_1,...,y_m)$, dann ist $m_y$ das Bild in $B$ von $M_y=(Y_1-y_1,...,Y_m-y_m)\Rightarrow \alpha^{-1}(m_y)$ ist das Bild in $A$ von $\tilde{\alpha}^{-1}(M_y)= (X_1-f_1(y),...,X_n-f_n(y))$.\\
Nachrechnen: $\Phi\circ\Psi\cong\operatorname{id}_{\underline{k-\operatorname{Alg}^{\circ}}}$, ~~~~$\Psi\circ\Phi\cong\operatorname{id}_{\underline{\operatorname{Aff}}(k)}$
\end{proof}

\section{Reguläre Funktionen}
\begin{Bem}
Sei $V\subseteq\mathbb{A}^n(k)$ eine affine Varietät. Dann gilt für $h\in k[X_1,...,X_n]$:\\
$\overline{h}$ ist Einheit in $k[V]\Leftrightarrow V(h)\cap V = \emptyset$
\end{Bem}
\begin{proof} $V(h)\cap V = \emptyset \Leftrightarrow (h)+I(V)=k[X_1,...,X_n]\\
\Leftrightarrow1=g\cdot h+f$ für $g\in k[X_1,...,X_n]$ und $f\in I(V)\\
\Leftrightarrow 1= \overline{g}\cdot\overline{h}$ in $k[V]$.
\end{proof}
\begin{Def}
Sei $V\subseteq\mathbb{A}^n(k)$ eine affine Varietät, $U\subseteq V$ offen.\\
a) Eine Abbildung $f: U\rightarrow \mathbb{A}^1(k)$ heißt \underline{reguläre Funktion} auf $U$, wenn es zu jedem $x\in U$ eine Umgebung $U(x)\subseteq U$
$g_x,h_x\in k[V]$ gibt mit $h_x(y)\neq0$ für alle $y\in U(x)$ und $f(x)=\frac{g_x(y)}{h_x(y)}$ für alle $y\in U(x)$\\
b) Eine Abbildung $f: U\rightarrow U'\subseteq \mathbb{A}^m(k)$ offen heißt \underline{Morphismus}, wenn es reguläre Funktionen $f_1,...,f_m$ auf U gibt mit $f(x)=(f_1(x),...,f_m(x))$.
\end{Def}
\begin{Bsp} $\frac{1}{x}$ ist eine reguläre Funtion auf $k\backslash\{0\}$.\\
Dann ist $U\rightarrow \mathbb{A}^2(k), ~~x\mapsto(x,\frac{1}{x})$ ein Isomorphismus mit Bild $V(XY-1)$
\end{Bsp}
\begin{Def}
a) Eine \underline{Prägarbe} besteht aus einer k-Algebra $\mathcal{O}(U)$ für jede offene Menge $U\subseteq V$ zusammen mit k-Algebra-Homomorphismen
$$\rho_{UU'}: \mathcal{O}(U)\rightarrow \mathcal{O}(U') ~~~\forall U'\subseteq U~~offen$$
so dass $\rho_{UU''}=\rho_{U'U''}\circ\rho_{UU'}$ für $U''\subseteq U'\subseteq U$ gilt.\\
b) Eine Prägarbe heißt \underline{Garbe}, falls zusätzlich noch folgende Bedingungen gelten:\\
Sei $U\subseteq V$ offen und $(U_i)_{i\in I}$ eine offene überdeckung von U.\\
(i) Ist $f\in \mathcal{O}(U)$ und $\rho_{UU'}(f):= f\mid_{U_i} = 0$ für alle $i\in I$, so ist $f=0$.\\
(ii) Ist für jedes $i\in I$ ein $f_i\in \mathcal{O}(U_i)$ gegeben, so dass für alle $i,j\in I$ gilt $f_i\mid_{U_i\cap U_j}=f_j\mid_{U_i\cap U_j}$, so gibt es $f\in \mathcal{O}(U)$ mit $f\mid_{U_i}=f_i$ für jedes $i\in I$.
\end{Def}

\begin{Bem}
Sei $V\subseteq \mathbb{A}^n(k)$ eine affine Varietät.\\
(a) Für jedes offene $U\subseteq V$ ist
$$\mathcal{O}(U):= \{f: U \rightarrow k \mid f \textrm{regulär}\}$$
eine $k$-Algebra.\\
(b) $f\mapsto \frac{f}{1}$ ist eine $k$-Algebra-Homomorphismus $k[V]\rightarrow \mathcal{O}(U)$ für jedes offene $U\subseteq V$. Dieses ist injektiv, falls U dicht in V ist. Dies ist für alle $\emptyset\neq U$ der Fall, wenn V irreduzibel ist. (Gegenbsp.: $V(X\cdot Y), U=D(x), f=y$)\\
(c) Die Zuordnung $U\mapsto \mathcal{O}(U)$ ist eine Garbe $\mathcal{O}=\mathcal{O}_V$ von k-Algebren auf V.
\end{Bem}
\begin{proof}
Seien $f_1,f_2\in\mathcal{O}(U)$. Ohne Einschränkung sei $U_1(x)=U_2(x):=U(x)$ für alle $x\in U$.
Sei $f_i=\frac{g_{i,x}}{h_{i,x}}$ auf $U(x)$.\\
Dann ist auch $h_{1,x}(y)\cdot h_{2,x}(y)\neq 0$ für alle $y\in U(x) \Rightarrow f_1\pm f_2$ und $f_1\cdot f_2$ sind reguläre Funktionen.\\
Mit $h_x=1$ und $f_x=f$ für alle $x$ ist jedes $f\in k[V]$ reguläre Funktion auf jedem offenen $U$.
\end{proof}
\begin{Prop}
Für jede affine Varietät $V\subseteq\mathbb{A}^n(k)$ gilt $\mathcal{O}(U)=k[V]$.
\end{Prop}
\begin{proof} Nach Bem. 6.3.b ist $k[V] \rightarrow \mathcal{O}(U)$ injektiv, also gilt ohne Einschränkung $k[V]\subseteq \mathcal{O}(U)$.\\
Sei zunächst $V$ irreduzibel: Sei $f\in \mathcal{O}(U), x_i\in V, i=1,2, U_i\subseteq V$ offene Umgebungen von $x_i$, auf denen $f(y)=\frac{g_i(y)}{h_i(y)}$ gilt für geeignete $g_i,h_i\in k[V], h_i(y)\neq 0 \forall y\in U_i$.\\
Dann ist $U:=U_1\cap U_2$ offen \underline{und dicht} in $V \Rightarrow g_1h_2-g_2h_1\in I(U)$ (weil $\frac{g_1(y)}{h_1(y)}=f(y)=\frac{g_2(y)}{h_2(y)}$ für alle $y\in U$)\\
Mit $V(I(U))=V$ folgt $g_1h_2=g_2h_1$ in $k[V]\Rightarrow \frac{g_1}{h_1}=\frac{g_2}{h_2}$ auf $U_1\cap U_2$, d.h. $\exists g,h\in k[V]$ mit $\frac{g_i}{h_i}=\frac{g}{h}, i=1,2$.\\
Ist $V$ zusammenhängend, so sei $V=V_1\cup ... \cup V_r$ eine Zerlegung in irreduzible Komponenten. die Agumentation ist die gleiche, allerdings für $x\in V_1\cap V_i$ ($V_i$ geeignet).\\
Ist  $V=V_1\stackrel{\cdot}{\cup} V_2$ disjunkte Vereinigung von affinen Varietäten $V_1,V_2$, so ist\\
$\mathcal{O}(U)=\mathcal{O}(V_1)\oplus\mathcal{O}(V_2)$ (folgt aus der Definition von regulären Funktionen) und\\
$k[V]=k[V_1]\oplus k[V_2]$ (Übung).
\end{proof}
\begin{Prop}
Sei $V\subseteq\mathbb{A}^n(k)$ eine affine Varietät $f\in k[V]$. dann ist $\mathcal{O}(D(f))\cong k[V]_f$ (Lokalisierung von $k[V]$ nach dem multiplikativen System $S=\{f^d: d\geq 0\}$)
\end{Prop}
\begin{proof}
1) $V=\mathbb{A}^1(k),~~f=x,~~ D(f)=k\backslash \{0\}\\
\mathcal{O}(D(f))=\{\frac{g}{h}: g,h\in k[X]$ mit $h(x)\neq 0$ für alle $x\neq 0\}\\
=\{\frac{g}{x^d}: g\in k[X], d\geq 0\}$\\
2) $V=V(x\cdot y)\subseteq \mathbb{A}^2(k), f=x\in k[V]=k[X,Y]/(X\cdot Y)$\\
$D(f)=V-V(x)= x$-Achse ohne die $0$\\
$k[V]_x=\{\frac{g}{x^d}: g\in k[V], d\geq 0\}/\sim$ mit der äquivalenzrelation $\frac{g}{x^d} \sim 0\Leftrightarrow \exists d'\geq 0$ mit $x^{d'}\cdot g=0\Leftrightarrow g=y\cdot g'$ für ein $g'\in k[V]\Rightarrow \operatorname{Kern}(k[V]\rightarrow k[V]_x)=(y)\Rightarrow k[V]_x\cong k[X]_x$.
\end{proof}
\begin{proof}
Sei $I=I(V)$, also $k[V]\cong k[X_1,...,X_n]/I$. Sei weiter $\tilde{f}\in k[X_1,...,X_n]$ Repräsentant von $f$.\\
\underline{Beh.:} $D(f)$ ist isomorph zu einer affinen Varietät $W:=V(\underbrace{I+(\tilde{f}X_{n+1}-1)}_{:=\tilde{I}})\subseteq \mathbb{A}^{n+1}(k)$\\
\underline{Beweis:} übung.\\
Nach Prop. 6.4: $\mathcal{O}(D(f))\cong\mathcal{O}(W)=k[W]=k[X_1,...,X_{n+1}]/\tilde{I}$\\
Sei $\alpha: k[X_1,...,X_{n+1}] \rightarrow k[V]_f$ der durch $x_i\mapsto\begin{cases}
x_i: i=1,...,n\\
\frac{1}{f}: i=n+1
\end{cases}$ erzeugte Homomorphismus.\\
\underline{Beh.:} Kern($\alpha=\tilde{I}\\
"'\supseteq"' \surd\\
"'\subseteq"' \alpha$ induziert einen Homomorphismus: $\tilde{\alpha}:\underbrace{k[X_1,...,X_{n+1}]}_{k[V][X_{n+1}]}/I\rightarrow k[V]_f$\\
zu zeigen ist also: $A$ $k$-Algebra, $f\in A$\\
$\alpha: A[X]\rightarrow A_f$, so ist $\Kern(\alpha)=(Xf-1)$.
\end{proof}

\subsection*{Nachtrag}
\[
\begin{xy}
\xymatrix{
k[Y_1,...,Y_m]\ar[r]^{\tilde{\alpha}}\ar[d]&k[X_1,...,X_n]\ar[d] \\
k[W]\ar[r]\ar@{}[d]|{\cup}&k[V]\ar@{}[d]|{\cup}&\text{ $k$-Algebrenhomomorhismus } \\
m_y\defeqr\alpha^{-1}(m_x)\ar[r]\ar@{<->}[d]&m_x\ar@{<->}[d] \\
y\ar@{}[d]|{\in}& x\ar@{|->}[l]\ar@{}[d]|{\in}\\
W&V\ar[l]_{f_\alpha}
}
\end{xy}
\]
\begin{Beh}
Für $x \in V$ ist $f_\alpha(x)=(f_1(x),\dots,t_n(x))\defeql y$. Noch zu zeigen: $\alpha^{-1}(m_x)=m_y$. Es ist $m_y=\overline{(Y_1-f_1(x),\dots,Y_m-f_m(x))}$. Dann ist $\alpha(m_y)$ das von $\overline{\tilde{\alpha}(Y_i)-f_i(x)}$, $i=1,\dots,n$ erzeugte Ideal. Also:
\begin{align*}
&\Rightarrow \alpha(m_y)\subseteq m_x \\
&\Rightarrow m_y\subseteq\alpha^{-1}(m_x) \\
&\Rightarrow m_y=\alpha^{-1}(m_x)
\end{align*}
\end{Beh}
\begin{Prop}
\label{prop:6.6}
  Seien $V\subseteq\mathbb A^n(k), W\subseteq\mathbb A^m(k)$ affine Varietäten und $U_1\subseteq V, U_2\subseteq W$ offen. Dann gilt: Eine Abbildung $f:U_1\longrightarrow U_2$ ist genau dann ein Morphismus, wenn $f$ stetig ist und für jedes offene $U\subseteq U_2$ gilt:
  \begin{align*}
    g\circ f \in \mathcal O(f^{-1}(U)) \text{ für jedes } g\in\mathcal O(U)
  \end{align*}

\end{Prop}

\begin{proof}
  ''$\Rightarrow$'' f ist stetig nach \ref{bem:5.3}. 
  Seien $g\in\mathcal O(U), x\in f^{-1}(U), U'$ Umgebung von $f(x)$, sodass $g(y)=\frac{h_1(y)}{h_2(y)}$ für alle $y\in U'$, wobei $h_1,h_2\in k[W], h_2(y)\neq 0$ für alle $y\in U'$. Daraus folgt für $z\in f^{-1}(U')$:
  \begin{align*}
    g\circ f(z)=\frac{h_1(f(z))}{h_2(f(z))}=(*)
  \end{align*}
  weil $f$ ein Morphismus ist gilt $f(z)=\left(\frac{a_1(z)}{b_1(z)},\dots,\frac{a_m(z)}{b_m(z)}\right)$ für geeignete $a_i,b_i\in k[V]$ und \OE~  alle $z\in f^{-1}(U')$ und damit
  \begin{align*}
    (*)=\frac{h_1\left(\frac{a_1(z)}{b_1(z)},\dots,\frac{a_m(z)}{b_m(z)}\right)}{h_2\left(\frac{a_1(z)}{b_1(z)},\dots,\frac{a_m(z)}{b_m(z)}\right)}\defeql\frac{\tilde{ h_1}}{\tilde{h_2}}(z)\text{, mit } \tilde{h_i}\in k[V]
  \end{align*}
  ''$\Rightarrow$'' Seien $x\in U_1$ und $U\subseteq W$ eine offene Umgebung von $f(x)\Rightarrow f^{-1}(U)\subseteq V$ ist offen. \\
  Sei $p_i:U\longrightarrow k$ die $i$-te Koordinatenfunktion, also $p_i(y_1,\dots,y_n)=y_i, i=1,\dots,m$. Nach Voraussetzung ist $p_i\circ f\in\mathcal O(f^{-1}(U)), i=1,\dots,m$. Also gibt es $g_i, h_i \in k[V]$ mit $p_i\circ f(y)=\frac{g_i(y)}{h_i(y)}$ für alle $y$ in einer geeigneten Umgebung von $x$.
  \begin{align*}
    \Rightarrow f(z)=\left(\frac{g_1(z)}{h_1(z)},\dots,\frac{g_m(z)}{h_m(z)}\right) \Rightarrow f \text{ ist ein Morphismus.}
  \end{align*}
\end{proof}
\begin{Def}
  \label{def:6.7}
  Sei $V\subseteq\mathbb A^n(k)$ eine affine Varietät und irreduzibel. Dann heißt $k(V)\defeqr\Quot(k[V])$ \begriff{Funktionenkörper} von $V$.
\end{Def}
\begin{Bsp}
  \begin{enumerate}
  \item $V=\mathbb A^n(k)\Rightarrow k(V)=k(X_1,\dots,X_n)$ \\
  \item $V=V(Y^2-X^2)\subseteq\mathbb A^2(k)$ \\
    $k[V]=\FakRaum{k[X,Y]}{(Y^2-X^2)}\cong k[T^2,T^3]\subseteq k[T]$ \\
    $\Rightarrow k(V)\cong k(T)$
  \end{enumerate}
\end{Bsp}
\begin{Prop}
  \label{prop:6.8}
  Sei $f:V\longrightarrow W$ ein Morphismus von irreduziblen affine Varietäten.
  \begin{enumerate}
  \item $f$ induziert genau dann einen Körperhomomorphismus $\varphi_f:k(W)\longrightarrow k(V)$ der den $k$-Algebrenhomomorhismus $f^\sharp:k[W]\longrightarrow k[V]$ fortsetzt, wenn $f^\sharp$ injektiv ist.
  \item $f^\sharp$ ist genau dann injektiv, wenn $f(V)$ dicht in $W$ ist (in diesem Fall heißt $f$ \begriff{dominant}).
  \end{enumerate}
\end{Prop}
\begin{proof}
  \begin{enumerate}
  \item $k(W)=\Quot(k[W])$. Für $x=\frac{a}{b}\in k(W)$ mit $a,b\in k[W],b\neq 0$ muss gelten $\varphi_f(x)=\frac{f^\sharp(a)}{f^\sharp(b)}$. Das ist wohldefiniert $\Leftrightarrow f^\sharp(b)\neq 0$ für alle $b\neq 0$.
  \item Sei $\alpha\defeqr f^\sharp:k[W]\longrightarrow k[V]$, dann gilt $\alpha^{-1}(I(Z))=f(I(Z))$ für jede Teilmenge $Z\subseteq V$, denn: \\
    \begin{align*}
      &g\in\alpha^{-1}(I(Z)) \\
      \Leftrightarrow&\forall z\in Z:\alpha(g)(z)=0 \\
      \Leftrightarrow&\forall z\in Z:(g\circ f)(z)=0 \\
      \Leftrightarrow&g\in I(f(Z))
    \end{align*}
    Für $Z=V$ heißt das: $\alpha^{-1}(I(V))=I(f(V))=\alpha^{-1}(0)=\Kern(\alpha)$. Also: $\Kern(\alpha)=0 \Leftrightarrow I(f(V))=0 \Leftrightarrow V(I(f(V)))=\overline{f(V)}=W$
  \end{enumerate}
\end{proof}
\section{Rationale Abbildungen}
\begin{DefBem}
  \label{defbem:7.1}
  Sei $V\subseteq\mathbb A^n(k)$ eine affine Varietät.
  \begin{enumerate}
  \item Eine rationale Funktion auf $V$ ist eine Äquivalenzklasse von Paaren $(U,f)$, 
    wobei $U\subseteq V$ offen und dicht und $f\in\mathcal O(U)$ ist. 
    Dabei sei $(U,f)\sim(U',f'):\Leftrightarrow f_{\vert U\cap U'}={f'}_{\vert U\cap U'}$
  \item In jeder Äquivalenzklasse $[(U',f')]$ gibt es ein (bezüglich ''$\subseteq$'')
    maximales Element $(U,f)$, dessen $U$ \begriff{Definitionsbereich} der rationalen Funktion heißt.
    $V-U$ heißt \begriff{Pol(stellen)menge}.
  \item Die rationalen Funktionen auf $V$ bilden eine $k$-Algebra $\Rat(V)$.
  \item Ist $V$ irreduzibel, so ist $\Rat(V)\cong k(V)$.
  \end{enumerate}
\end{DefBem}
\begin{proof}
  \begin{enumerate}
  \item $\sim$ ist transitiv: Seien $(U,f)\sim(U',f'),(U',f')\sim(U'',f'')$ dann folgt: 
    $f_{\vert U\cap U'\cap U''}={f''}_{\vert U\cap U' \cap U''}$. Da $U\cap U' \cap U''$ dicht in $V$ ist, 
    ist dann auch $f_{\vert U\cap U''}={f''}_{\vert U\cap U''}$.
  \item Sei $(U,f)\sim(U',f')$ so definiere auf $U\cup U'$ eine Funktion $\tilde{f}$ durch 
    $\tilde{f}(x)=
    \begin{cases}
      f(x)&x\in U \\
      f'(x)&x\in U'
    \end{cases}$
    Dann ist $f\in\mathcal O(U\cup U')$
  \item $f\pm g, f\cdot g$ sind reguläre Funktionen auf $U\cap U'$, wobei $(U,f)$ und $(U',g)$
    Repräsentanten sind.
  \item $\frac{g}{h}\in k(V)$ ist eine reguläre Funktion auf $D(h)$. $D(h)$ liegt dicht in $V$,
    weil $V$ irreduzibel ist. 
    Es folgt: $\frac{g}{h}\mapsto (D(h),\frac{g}{h})$ ist ein wohldefinierter $k$-Algebrenhomomorhismus
    $\alpha:k\longrightarrow\Rat(V)$. \\
    $\alpha$ ist surjektiv, denn: \\
    Sei $(U,f)$ ein Repräsentant einer rationalen Funktion auf $V$. Dann gibt es
    offenes $U'\subseteq U$ und $g,h\in k[V]$ mit $f(x)=\frac{g(x)}{h(x)}$ 
    für alle $x\in U'$. Da $V$ irreduzibel ist, ist $U'$ dicht in $V$. Also ist $\alpha(\frac{g}{h})$
    gleich der Klasse $(U',\frac{g}{h})$, was gleich der Klasse von $(U,f)$ ist.
  \end{enumerate}
\end{proof}
\begin{DefBem}
  \label{defbem:7.2}
  Seien $V\subseteq\mathbb A^n(k),W\subseteq\mathbb A^m(k)$ affine Varietäten.
  \begin{enumerate}
  \item Eine \begriff{rationale Abbildung} $f:V\dashrightarrow W$ ist eine Äquivalenzklasse von Paaren $(U,f_U)$,
    wobei $U\subseteq V$ offen und dicht ist und $f_U:\longrightarrow U$ ein Morphismus ist;
    dabei sei $(U,f_U)\sim (U',f_U):\Leftrightarrow {f_U}_{\vert U\cap U'}={f_{U'}}_{\vert U\cap U'}$
  \item Rationale Funktion sind rationale Abbildungen $V\dashrightarrow \mathbb A^1(k)$.
  \item jede rationale Abbildung hat einen maximalen Definitionsbereich.
  \item Die Komposition von dominanten rationale Abbildungen ist wieder eine dominante rationale Abbildung,
    wegen $\overline{f(U)}=\overline{f(\overline{U})}$.
  \item Jede dominante rationale Abbildung $f:V\dashrightarrow W$ induziert einen $k$-Algebrenhomomorhismus
    $\Rat(W)\longrightarrow \Rat(V)$.
  \item Eine dominante rationale Abbildung $f:V\dashrightarrow W$ heißt \begriff{birational}, 
    wenn es eine rationale Abbildung $g:W\dashrightarrow V$ gibt
    mit $f\circ g\sim\id_W$ und $g\circ f\sim\id_V$.
  \end{enumerate}
\end{DefBem}
\begin{nnBsp}
  1) $f:\mathbb A^1(k)\dashrightarrow \mathbb A^2(k), x\mapsto(x,\frac{1}{x})$ ist eine rationale Abbildung.
  2) $\sigma:\mathbb A^2(k)\dashrightarrow \mathbb A^2(k), (x,y)\mapsto(\frac{1}{x},\frac{1}{y})$ ist eine 
  birationale Abbildung. Es gilt $\sigma\circ\sigma=\id$ auf $\mathbb A^2(k)-V(XY)$.
\end{nnBsp}
\begin{Prop}
  \label{prop:7.3}
  Seien $V,W$ irreduzible affine Varietäten. Dann gibt es zu jedem Körperhomomorphismus $\alpha :k(W)
  \longrightarrow k(V)$ eine rationale Abbildung $f:V\dashrightarrow W$ mit $\alpha=\alpha_f$.
\end{Prop}
\begin{proof}
  Wähle Erzeuger $g_1,\dots,g_m$ von $k(W)$ als $k$-Algebra. Für $\alpha(g_i)\in k(V)=\Rat(V)$ sei
  $U_i\subseteq V$ der Definitionsbereich. Sei $\tilde{U}\defeqr\bigcap_{i=1}^m U_i$, $\tilde{U}$ ist offen und dicht in $V$.
Sei $U\subseteq\tilde{U}$ affin und dicht (sowas gibt es!)
\begin{align*}
\Rightarrow&\alpha(g_i)\in\mathcal O(U)=k[U], i=1,\dots,m \\
  \Rightarrow&\alpha(k[W]:k[W]\longrightarrow k[U] \text{ ist $k$-Algebrenhomomorhismus.} \\
\Rightarrow&\text{es gibt einen Morphismus $f:U\longrightarrow W$ mit $f^\sharp=\alpha$.}
\end{align*}
$\alpha_f$ ist der von $f^\sharp$ induzierte Homomorphismus auf $\Quot(k[W])$.
\end{proof}
\begin{Prop}
  \label{prop:7.4}
Zu jeder endlich erzeugten Körpererweiterung $K/k$ gibt es eine irreduzible affine $k$-Varietät $V$ mit $k\cong k(V)$.
\end{Prop}
\begin{proof}
  Seien $x_1,\dots,x_n\in K$ Erzeuger der Körpererweiterung $K/k$. Sei weiter $A\defeqr k[x_1,\dots,x_n]$ die von
den $x_i$ erzeugte $k$-Algebra. $A$ ist nullteilerfrei, da $A\subseteq K$. Nach \ref{satz:3} gibt es eine 
affine Varietät $V$ mit $A\cong k[V]$. $V$ ist irreduzibel, da $A$ nullteilerfrei. $k(V)=\Quot(k[V])\cong\Quot(A)=K$.
\end{proof}
\begin{Kor}
  \label{kor:7.5}
Die Kategorie der endlich erzeugten Körpererweiterungen $K/k$ (mit $k$-Algebrenhomomorhismen)
ist äquivalent zur Kategorie der irreduziblen affine Varietäten über $k$ mit dominanten rationalen Abbildungen.
\end{Kor}
\chapter{Projektive Varietäten}
\begin{Eri}
  \begin{align*}
  \mathbb P^n(k)=& \{ \text{ Geraden in $k^{n+1}$ durch $0$} \}     \\
 =&\FakRaum{(k^{n+1}-{0})}{\sim}\text{ mit }(x_0,\dots,x_n)\sim(y_0,\dots,y_n):\Leftrightarrow\exists\lambda\in k^\times:\lambda x_i=y_i\text{ für } i=1,\dots,n
  \end{align*}
Schreibweise $(x_0:\dots:x_n);=[(x_0,\dots,x_n)]$ (``homogene Koordinate'')
\begin{Bsp}
  \begin{enumerate}
  \item[$n=0$:] $\mathbb P^0(k)$ ist ein Punkt.
  \item[$n=1$:]
 $\mathbb P^1(k)\longrightarrow  k \cup \{ \infty \} \text{ ist bijektiv } \\
(x_0:x_1) \mapsto  
\begin{cases}
  \frac{x_1}{x_0}:&x_0\neq 0 \\
\infty:&x_0=0
\end{cases} $
Also: $\mathbb P^1(\mathbb R)=\FakRaum{S^1}{\{ \pm 1\} }$
  \end{enumerate}
\end{Bsp}
\end{Eri}



\appendix

\def\indexspace{\par\medskip}
\printindex[default][\phantomsection\addcontentsline{toc}{chapter}{Vokabeln}\vspace{-1.2em}]

\end{document}
		
