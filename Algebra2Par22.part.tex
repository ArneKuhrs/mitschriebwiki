\section{Ganze Ringerweiterungen}

\begin{Def} 
  Sei $S/R$ eine Ringerweiterung (d.h. $R \subseteq S$).
  \begin{enumerate} 
    \item $b \in S$ heißt \emp{ganz} über $R$, wenn es ein \emp{normiertes}
          Polynom $f \in R[X]$ gibt mit $f(b) = 0$.
    \item $S$ heißt \emp{ganz}\index{Ringerweiterung!ganze} über $R$, wenn jedes
    $b \in S$ ganz über $R$ ist.
\end{enumerate}
\end{Def}

\begin{nnBsp} 
  $\sqrt{2} \in \RR$ ist ganz über $\ZZ$.\\
  $\frac{1}{2} \in \QQ$ ist nicht ganz über $\ZZ$ (Nullstelle von
  $2X -1$).
\end{nnBsp}

\begin{Prop}
\label{2.7}
  Sei $S/R$ Ringerweiterung. Für $b \in S$ sind äquivalent:
  \begin{enumerate} 
    \item[(i)] $b$ ist ganz über $R$
    \item[(ii)] $R[b]$ ist endlich erzeugbarer $R$-Modul
    \item[(iii)] $R[b]$ ist enthalten in einem Unterring $S' \subseteq S$, der
                 als $R$-Modul endlich erzeugt ist.
\end{enumerate}
\end{Prop}

\begin{Bew} 
  \textbf{(i) $\Rightarrow$ (ii):}
  Nach Voraussetzung gibt es $a_0, \dots , a_{n-1} \in R$, sodass 
  \[
  b^n = a_{n-1} b^{n-1} + \dots + a_0
  \] 
  $\Rightarrow b^n$ ist in dem von $1, b, \dots , b^{n-1}$
  erzeugtem $R$-Untermodul von $S$ enthalten $\Rightarrow b^{n+1} = a_{n-1} b^n
  + \dots + a_0 b = a_{n-1} (\sum_{i=0}^{n-1} a_i b^i) + \dots + a_0 b \in M
  \overset{\text{\scriptsize Induktion}}{\Rightarrow} b^k \in M$ für alle $k \ge
  0 \Rightarrow M = R[b]$\\
  \textbf{(iii) $\Rightarrow$ (i):}
  $S'$ werde als $R$-Modul von $s_1, \dots , s_n$ erzeugt $\Rightarrow b \cdot
  s_i \in S'$, d.h. $b \cdot s_i = \sum_{k = 1}^n a_{i k} s_k, i=1,\dots,n
  \Rightarrow \sum_{k = 1}^n (a_{i k} -\delta_{i k} \cdot b) \cdot s_k = 0$ für
  $i=1, \dots, n$.\\
  Für die Matrix $A = (a_{i k} - \delta_{i k} \cdot b)_{i,k =1, \dots, n} \in S
  ^{n \times n}$ gibt es also $A \cdot \begin{pmatrix}s_1 \\ \vdots \\ s_n
  \end{pmatrix} = 0$.
  $\det(A)$ ist normiertes Polynom in b vom Grad $n$ mit Koeffizienten in $R$\\
  \textbf{Beh.:} $\det(A)$ = 0.\\
  \textbf{Bew.:} Cramersche Regel:\\
  $A^{\#} \defeqr (b_{i j}) \text{ mit } b_{i j} = (-1)^{i+j} \det(A'_{j i}),
  i,j= 1, \dots, n$ wobei $A'_{j i}$ durch Streichen der $j$-ten Zeile und
  der $i$-ten Spalte aus $A$ hervor geht. Es folgt 
  \[
  A^{\#} \cdot A = (c_{i k})
  \]
  mit
\[ c_{i k} = \sum_{j = 1}^n a_{i j} b_{j
  k} =
  \sum_{j = 1}^n a_{i j} (-1)^{i+j} det(A'_{k j}) = \begin{cases}
  i=k: & \det(A) \text{ (Laplace)}\\
  i \not= k: & \det(A_k^i) = 0
  \end{cases}
\]
  $\det(A_k^i) = 0:$ in der $k$-ten Zeile steht $a_{i 1}, \dots , a_{i n}
  \Rightarrow i$-te und $k$-te Zeile sind gleich.\\
  $\Rightarrow A \cdot A^{\#} = \det(A) \cdot E_n = A^{\#} \cdot A
  \Rightarrow 0 = A^{\#} \cdot A \cdot \begin{pmatrix}s_1 \\ \vdots \\ s_n 
  \end{pmatrix} = \det(A) \cdot \begin{pmatrix}s_1 \\ \vdots \\ s_n 
  \end{pmatrix} \Rightarrow \det(A) \cdot s_i = 0$ für $i = 1, \dots,
  n$.
  Da $1 \in S'$, gibt es $\lambda_i \in R$ mit $1 = \sum_{i=1}^n \lambda_i s_i
  \Rightarrow \det(A) \cdot 1 = \sum_{i=1}^n \lambda_i \cdot \det(A)
  \cdot s_i = 0 \Rightarrow \det(A) = 0$.
\end{Bew}

\begin{Prop}
\label{2.8}
  Ist $S/R$ Ringerweiterung, so ist $\bar{R} \defeqr \{ b \in S: b \text{ ganz 
  über } R \}$ ein Unterring von $S$.
\end{Prop}

\begin{Bew} 
  Seien $b_1, b_2 \in \bar{R}$.\\
  Zu zeigen: $b_1 \pm b_2, b_1 \cdot b_2 \in \bar{R}$\\
  Nach \ref{2.7} genügt es zu zeigen: $R[b_1, b_2]$ ist endlich erzeugt als
  $R$-Modul.
  Dazu: $R[b_1]$ ist endlich erzeugt als $R$-Modul nach \ref{2.7}.
  $R[b_1, b_2] = (R[b_1])[b_2]$ ist endlich erzeugt als $R[b_1]$-Modul (von
  $y_1, \dots, y_m$).
  Dann erzeugen die $x_i y_j \; R[b_1, b_2]$ als $R$-Modul.
\end{Bew}

\begin{Def} 
  Sei $S/R$ Ringerweiterung.
  \begin{enumerate} 
    \item $\bar{R}$ (wie in \ref{2.8}) heißt der \emp{ganze
          Abschluss}\index{Abschluss!ganzer} von $R$ in $S$.
    \item Ist $R = \bar{R}$, so heißt $R$ \emp{ganz
          abgeschlossen}\index{Ring!ganz abgeschlossener} in $S$.
    \item Ein nullteilerfreier Ring $R$ heißt \emp{normal}\index{Ring!normaler}, wenn er ganz
          abgeschlossen in Quot$(R)$ ist.
    \item Ist $R$ nullteilerfrei, so heißt der ganze Abschluss $\bar{R}$ von $R$
          in Quot$(R)$ die \emp{Normalisierung}\index{Normalisierung} von $R$.
  \end{enumerate}
\end{Def}

\begin{Bem}
\label{2.10}
  Jeder faktorielle Ring ist normal.
\end{Bem}

\begin{Bew} 
  Sei $K=\Quot{R}, x= \frac{a}{b} \in K^\times, a,b \in R$ teilerfremd.
  Sei $x$ ganz über $R$. Dann gibt es $\alpha_0, \dots, \alpha_{n-1} \in R$ mit
  $x^n + \alpha_{n-1} x^{n-1} + \dots + \alpha_0 = 0 \overset{\cdot b^n}{\Rightarrow}
  a^n + \alpha_{n-1} b a^{n-1} + \dots + \alpha_1 b^{n-1} a + \alpha_0 b^n = 0
  \Rightarrow b \mid a^n$ Widerspruch zu teilerfremd.
\end{Bew}
