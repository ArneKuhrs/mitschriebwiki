\documentclass{scrartcl}

\usepackage{latexki}
\title{Informatik 4 -- Formelsammlung}
%\lecturer{}
\semester{Sommersemester 2006}
\scriptstate{complete}

\usepackage[german]{babel}
\usepackage[utf8]{inputenc}

\usepackage{amsfonts}
\usepackage{amssymb}
\usepackage{amsmath}

\usepackage{multicol}

\begin{document}
\pagestyle{empty}
\begin{center}
\sectfont \LARGE Info IV Formelsammlung\\
\normalfont \normalsize Joachim Breitner, \today
\end{center}
\begin{flushleft}
\begin{multicols}{2}
\section{Rabin-Karp}
\paragraph{Hashfunktion}
\[ h(x) = x \bmod q \]
\paragraph{Hashwert}
\begin{align*}
x &= a[i]d^{m-1} + a[i+1]d^{m-2} \\
    &\quad + \cdots + a[i+m-1]
\end{align*}
\paragraph{Bei Verschiebung:}
\[ x' = d(x-a[i]d^{m-1}) + a[i+m] \]

\section{Bildanalyse}
\paragraph{Mittlerer Grauwert}
\[ \overline{g} = \frac{1}{mn} \sum_x \sum_y g(x,y)\]
\paragraph{Grauwertabweichung}
\[ \Delta g = \frac{1}{mn} \sum_x \sum_y \big(g(x,y) - \overline g\big)^2 \]
\paragraph{Affine Abbildung}
\[ f(g) = a\cdot g + b \]
($a$ Kontraständerung, $b$ Helligkeitsänderung)
\paragraph{Hough-Transformation}
\[ h:(x,y) \mapsto p = x\cdot \cos \theta + y \cdot \sin \theta \]

\section{Texturanalyse}
\paragraph{Grauwertübergangs-Matrix}
\[ c_{d,ij} = \#\{x \mid g(x) = i, g(x+d) = j\} \]
\[ C_d = \big[c_{d,ij}\big]\]
Im folgenden wird $C_d$ normiert angenommen.
\paragraph{Energie}
\[ 1\cdot\big[c_{d,ij}^2\big]\cdot1\]
\paragraph{Entropie}
\[ -\sum_i \sum_j c_{d,ij} \log c_{d,ij} \]
\paragraph{Kontrast}
\[ \sum_i \sum_j |i-j|^a\cdot c_{d,ij}^b \]
(typisch: $a=2$, $b=1$)

\section{Informationstheorie}
\paragraph{Entopie}
\[H(x) = -\sum_x p(x) \cdot \log p(x) \]
\paragraph{Verbundentropie}
\[H(X,Y) = -\sum_x\sum_y p(x,y) \cdot \log p(x,y) \]
\paragraph{Bedingte Entropie}
\begin{align*}
H(X|Y) &= -\sum_x\sum_y p(x,y) \cdot \log p(x|y) \\
       &= H(X,Y) - H(Y)
\end{align*}
\paragraph{Relative Entropie}
\[ D(p\|q) = \sum_x p(x) \log \frac{p(x)}{q(x)} \]
\paragraph{Transinformation Entropie}
\begin{align*}
I(X;Y) &= \sum_x\sum_y p(x,y) \cdot \log \frac{p(x,y)}{p(x)p(y)} \\
       &= H(X) - H(X|Y)
\end{align*}


\end{multicols}
\end{flushleft}
\end{document}
