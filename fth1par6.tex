\documentclass{article}
\newcounter{chapter}
\setcounter{chapter}{6}
\usepackage{ana}
\def\gdw{\equizu}
\def\Arg{\text{arg}}
\def\Log{\text{Log}}
\title{Exponentialfunktion und trigonometrische Funktionen}
\author{Ferdinand Szekeresch}
% Wer nennenswerte Änderungen macht, schreibt euch bei \author dazu

\begin{document}
\maketitle

Bekannt aus §5: $\sum_{n=0}^{\infty}\frac{z^n}{n!}$ konvergiert absolut in jedem $z \in \MdC$
$$ e^z := \exp(z):=\sum_{n=0}^{\infty}\frac{z^n}{n!} (z \in \MdC)$$

klar: $e^0 = 1, e^1 = e$

\begin{satz} %6.1
\begin{liste}
\item $\sum_{n=0}^{\infty}\frac{z^n}{n!}$ konvergiert auf $\MdC$ lokal gleichmäßig.
\item $\exp \in H(\MdC )$ und $\exp '(z) = \exp (z) \forall z \in \MdC$
\item \begriff{Additionstheorem}: $ e^{z+w} = e^ze^w \forall z,w \in \MdC$
\item $e^z\cdot e^{-z} = 1$, insbesondere $e^z \neq 0$
\item Für $z = x + iy (x,y \in \MdR): e^z = e^xe^{iy}, |e^{iy}| = 1, |e^z| = e^x$
\end{liste}
\end{satz} 

\begin{beweis}
\begin{liste}
\item folgt aus 5.2
\item $5.4 \folgt \exp \in H(\MdC )$ und $\exp '(z) = \sum_{n=1}^{\infty}\frac{z^{n-1}}{(n-1)!} = \exp (z) (z \in \MdC )$
\item Sei $c \in \MdC$ zunächst fest. \\
$f(z) := e^ze^{c-z} (z \in \MdC )$, \\
$f\in H(\MdC )$ und $f'(z) = e^ze^{c-z} + e^ze^{c-z}(-1) = 0 \quad \forall z \in \MdC$ \\
$\MdC$ ist ein Gebiet $\folgtnach{4.2} f$ ist auf $\MdC$ konstant. \\
$f(0) = e^c$. Also: $e^ze^{c-z} = e^c \quad \forall z\in \MdC \forall c\in \MdC$ \\
Setze $c := z + w$
\item folgt aus (3)
\item Nur zu zeigen: $|e^{iy}| = 1 (y \in \MdR)$ \\
$$ \overline{e^{iy}} = \overline{\sum_{n=0}^{\infty}\frac{(iy)^n}{n!}} = \sum_{n=0}^{\infty}\frac{(\overline{{iy}})^n}{n!} = \sum_{n=0}^{\infty}\frac{(-iy)^n}{n!} = e^{-iy}$$ \\
$\folgt |e^{iy}|^2 = e{iy}\overline{e^{iy}} = e^{iy}e^{-iy} \gleichnach{(4)} 1$ 
\end{liste}
\end{beweis}

\begin{definition}
Für $z \in \MdC$ 
\begin{eqnarray}\notag \cos z &:=& \frac{1}{2}(e^{iz} + e^{-iz}) \quad \begriff{Cosinus}\\
\notag\sin z &:=& \frac{1}{2i}(e^{iz} + e^{-iz}) \quad \begriff{Sinus} \end{eqnarray}
\end{definition}

\begin{satz} %6.2
\begin{liste}
\item $$\cos z = \sum_{n=0}^{\infty}(-1)^n\frac{z^{2n}}{(2n)!}$$\\
$$\sin z = \sum_{n=0}^{\infty}(-1)^n\frac{z^{2n+1}}{(2n+1)!} \quad \forall z \in \MdC$$
\item $\cos , \sin \in H(\MdC )$\\
$\cos 'z = -\sin z,\; \sin 'z = \cos z \;\forall z \in \MdC$
\item $e^{iz} = \cos z + i\sin z \; \forall z \in \MdC.$ \\
Insbesondere: $e^{i\varphi} = \cos\varphi + i\sin\varphi \;\forall\varphi\in\MdR$. Damit lautet für $z \in \MdC\backslash\{0\}$ die Darstellung in Polarkoordinaten: $z = |z|e^{i\arg z}$.
\item Additionstheoreme:\begin{eqnarray}\notag\cos (z+w) &=& \cos z\cos w - \sin z\sin w \\
\notag \sin (z+w) &=& \sin z\cos w + \sin w\cos z \quad\forall z,w \in \MdC \end{eqnarray}
\item $\cos ^2z + \sin ^2z = 1 \;\forall z\in\MdC$
\end{liste}
\end{satz}



%\pagebreak




\begin{beweis}
\begin{liste}
\item nur für $\cos$: \\
$\forall z \in \MdC :$ 
$$\cos z = \frac{1}{2}\sum_{n=0}^{\infty}\underbrace{\frac{i^n + (-i)^n}{n!}}_{\begin{cases}0,\; n \text{ ungerade} \\ 2(-1)^n,\;n = 2k\end{cases}}z^n$$ \\
$\folgt \cos z = \sum_{k=0}^{\infty}(-1)^k\frac{z^{2k}}{2k!}$
\item Aus der Definition folgt: $\cos \in H(\MdC)$ und \\
$\cos 'z = \frac{1}{2}(ie^{iz} - ie^{-iz}) = \frac{i}{2}(e^{iz} - e^{-iz}) = \frac{-1}{2i}(e^{iz} - e^{-iz}) = -\sin z$ \\
Analog für den Sinus.
\item , (4) , (5) folgen aus der Definition.
\end{liste}
\end{beweis}

\begin{folgerung}
\begin{liste}
\item $e^{2k\pi i} = 1 \; \forall k \in \MdZ$; insbesondere: $e^{2\pi i} = 1$
\item $e^{i\pi} + 1 = 0$
\item Für $z \in \MdC : e^z = 1 \gdw \exists k\in \MdZ : z = 2k\pi i$
\item $e^{z + 2\pi i} = e^z \;\forall z\in \MdC$ (Die Exponentialfkt. hat die Periode $2\pi$)
\item Für $z \in \MdC :$\\ 
$\sin z = 0 \folgt\exists k \in \MdZ : z = k\pi$\\
$\cos z = 0 \folgt\exists k \in \MdZ : z = \frac{2k+1}{2}\pi$
\end{liste}
\end{folgerung}

\begin{beweis}
\begin{liste}
\item 6.2 (3) $\folgt e^{2k\pi i} = \cos (2k\pi) + i\sin (2k\pi) = 1 (k \in \MdZ)$
\item $e^{i\pi} \gleichnach{6.2(3)} \cos\pi + i\sin\pi = -1$
\item $"\folgt "$ Sei $z = x + iy \in \MdC (x,y \in \MdR )$ und $e^z = 1$ \\
$\folgt e^xe^{iy} = e^x(\cos y +i\sin y) = 1$ \\
$\folgt e^x\cos y = 1, e^x\sin y = 0 \folgt \sin y = 0 \folgt \exists k \in \MdZ : y = k\pi$\\
$ 1 = |e^z| = e^x \folgt x=0 \folgt \cos y = 1 \folgt k=2j (j\in \MdZ) \folgt z = i2j\pi$
\item $e^{z+2\pi i} = e^ze^{2\pi i} = e^z$
\item Nur für $\sin$. Sei $z \in \MdC :$ \\
$\sin z = 0 \gdw e^iz = e^{-iz} \gdw e^{2iz} = 1 \stackrel{(3)}{\gdw} \exists k \in \MdZ : 2iz = 2k\pi i$\\
$\gdw \exists k \in \MdZ : z = k\pi$.
\end{liste}
\end{beweis}

%Kein Plan ob das hier rein gehört, naja... (Bernhard)

\begin{definition}
F"ur $z \in \MdC$:
\begin{eqnarray} 
\notag \tan z &:=& \frac{\sin z}{\cos z}, \quad z \in \MdC \backslash \{ \frac{2k+1}{2}\pi: k\in \MdZ \} \quad \begriff{Tangens}\\
\notag \\
\notag \cot z &:=& \frac{\cos z}{\sin z}, \quad z \in \MdC \backslash \{ k\pi: k\in \MdZ \}\quad \begriff{Cotangens}
\end{eqnarray}
\end{definition}
tan und cot sind auf ihrem Definitionsbereichen holomorph.


\end{document}
