%\part{Der Bruhat-Tits-Baum für $\GL_2$}

%===========
\section{$p$-adische Zahlen}\label{sec_padisch}

Es sei $p$ eine Primzahl. Eine Zahl $n\in\NN$ lässt sich eindeutig
schreiben als
\[
n = \SUM{k}{i=0} a_i p^i
\]
mit $a_i\in\{0,\ldots,p-1\}$ und $k\geq 0$.

\BEM\label{def_ubertrag}
Wir rechnen mit Übertrag: Für $n=\sum a_i p^i$ und
$m=\sum b_i p^i$ ist
\begin{gather*}
m+n = \sum c_i p^i, \\
mn = \sum d_i p^i
\end{gather*}
mit
$c_i = \tilde{c}_i - \rho_i p$,
$d_i = \tilde{d}_i - \sigma_i p$,
wobei 
\[
\tilde{c}_i=a_i+b_i+\rho_{i-1},\quad
\tilde{d}_i=(\sum_{l=0}^i a_l b_{i-l})+\sigma_{i-1}
\]
und
$\rho_i = \left\{\begin{matrix}
1, & \tilde{c}_i \geq p \\
0, & \tilde{c}_i < p
\end{matrix}\right.,$
$\sigma_i=\max\{s\in\NN:sp\leq \tilde{d}_i\}$
ist.

\DB Die Menge
\[
\ZZ_p = \left\{
\SUM{\infty}{i=0} a_i p^i : a_i \in \{0,\ldots,p-1\}
\right\}
\]
wird mit $+$ und $\cdot$ wie in Definition \ref{def_ubertrag}
zu einem kommutativen Ring mit $1$. Er heißt
\emph{Ring der ganzen $p$-adischen Zahlen}.\index{p-adische Zahlen!Ring}\index{$\ZZ_p$ ($p$-adische Zahlen)}

\bew Wir zeigen, dass $(\ZZ_p,+)$ eine Gruppe ist:
\begin{align*}
-(\sum a_i p^i) &= (p-a_0) + (p-a_1-1) p + (p-a_2-1) p^2 + \ldots \\
&= (p-a_0) + \SUM{\infty}{i=1} (p-a_i-1) p^i.
\end{align*}
Es gibt also zu jedem Element ein additiv Inverses.

Die übrigen Ringeigenschaften sind klar.
\qed

\PROP\ \begin{enumerate}
\item Die Inklusion $\NN\subset \ZZ_p$ induziert eine Einbettung
$\iota:\ZZ\hookrightarrow \ZZ_p$.
\item $\ZZ_p$ ist nullteilfrei.
\end{enumerate}
\bew \begin{enumerate}
\item Für $n\in\ZZ$, $n<0$, setze $\iota(n):=-\iota(-n)$.
\item Es seien $a=\sum a_i p^i, b=\sum b_i p^i \in \ZZ_p\backslash\{0\}$ und
\[
i_a := \min\{i:a_i \neq 0\},\quad
i_b := \min\{i:b_i \neq 0\}.
\]
Für $ab=\sum d_i p^i$ ist dann $\tilde{d}_{i_a+i_b}=a_{i_a}b_{i_b}$
nicht durch $p$ teilbar, da $p$ prim ist.
Folglich ist auch $d_{i_a+i_b}=\tilde{d}_{i_a+i_b}-\sigma_i p$
nicht durch $p$ teilbar und insbesondere $\neq 0$.
Also ist $ab\neq 0$.
\qed
\end{enumerate}

\BEM Es sei
\[
\mm_p = \left\{
a=\SUM{\infty}{i=0} a_i p^i \in \ZZ_p : a_0\neq 0
\right\}.
\]
Dann gilt:
\begin{enumerate}
\item $\mm_p$ ist ein maximales Ideal.
\item $\ZZ_p/\mm_p \cong \ZZ/p\ZZ \cong \FF_p$.
\item $a\in \ZZ_p^\times$ genau dann, wenn $a\not\in\mm_p$.
\end{enumerate}
Insbesondere ist $\mm_p$ das einzige maximale Ideal in $\ZZ_p$,
d.h. $\ZZ_p$ ist ein lokaler Ring (vgl. Kapitel II.4 in 
Lang \cite{lang}).

\bew \begin{enumerate}
\item Offensichtlich ist $\mm_p$ ein Ideal. Die Maximalität
folgt aus Teil 2 oder 3.
\item Die Abbildung $a=\sum a_i p^i \mapsto \bar{a}_0\in \ZZ/p\ZZ$
ist ein surjektiver Ringhomomorphismus mit Kern $\mm_p$.
Der Homomorphiesatz liefert die Behauptung.
\item Es sei $a=\sum a_i p^i \in \ZZ_p\backslash \mm_p$, also
$a_0\neq 0$. Gesucht ist $b=\sum b_i p^i$ mit $ab=1$.
Definiere die $b_i$ induktiv: Da $p$ prim ist, kann man $b_0$ so 
wählen, dass $a_0 b_0 \equiv 1 \mod p$ gilt.
Hat man für $i\geq 1$ schon $b_0,\ldots,b_{i-1}$ gefunden, wähle
$b_i$ so, dass
\[
a_0 b_i + \SUM{i}{l=1} a_l b_{i-l} \equiv 0 \mod p
\]
gilt.
\qed
\end{enumerate}

\DB\
\begin{enumerate}
\item Der Quotientenkörper
\[
\QQ_p = \mathrm{Quot}(\ZZ_p)
\]
heißt \emph{Körper der $p$-adischen Zahlen}.\index{p-adische Zahlen!Körper}\index{$\QQ_p$ ($p$-adische Zahlen)}
\item Die Inklusion $\ZZ\hookrightarrow \ZZ_p$ induziert eine
Inklusion $\QQ\hookrightarrow \QQ_p$ (d.h. $\Char(\QQ_p)=0$).
\item Jedes $a\in \QQ_p^\times=\QQ_p\backslash\{0\}$ hat eine
eindeutige Darstellung
\[
a=\SUM{\infty}{i=i_a} a_i p^i
\]
mit $a_i\in\{0,\ldots,p-1\}$ und $i_a\in\ZZ$ minimal, so dass
$a_{i_a}\neq 0$.
\end{enumerate}
\textsc{Beweis von 3.:} Ist $a=\sum a_i p^i\in\ZZ_p\backslash\{0\}$,
so sei $i_a=\min\{i:a_i\neq 0\}$.
Dann ist $a=\SUM{\infty}{i=i_a} a_i p^i$ die gewünschte Darstellung.

Es ist $a=p^{i_a} u$ mit $u=\SUM{\infty}{i=0}a_{i+i_a} p^i\in\ZZ_p^\times$.
Nun sei $\frac{a}{b}\in\QQ_p$ mit $a,b\in\ZZ_p$, $b\neq 0$.
Schreibe $a=p^{i_a} u$ und $b=p^{i_b}v$ mit $i_a,i_b\in\NN$ und
$u,v\in\ZZ_p^\times$.
Es folgt
$\frac{a}{b}=p^{i_a-i_b} \ub{uv^{-1}}{\in\ZZ_p^\times}$,
wie gewünscht.
\qed

\DB\
\begin{enumerate}
\item Für $a=\SUM{\infty}{i=i_a}a_i p^i \in \QQ_p^\times$ sei
\begin{gather*}
\vv(a) := i_a,\\
|a| := p^{-i_a}.
\end{gather*}
\item Ist $a\in \ZZ$, so ist $\vv(a)=\max\{n:p^n|a\}$.
\item $a\in\ZZ_p\backslash\{0\}\ \lra\ \vv(a)\geq 0\ \lra\ |a|\leq 1$.
\item Setze $\vv(0):=\infty$, $|0|:=0$.
\item Für alle $a,b\in\QQ_p^\times$ gilt:
\begin{enumerate}
\item $\vv(ab)=\vv(a)+\vv(b)$ bzw. $|ab|=|a|\cdot |b|$.
\item $\vv(a+b)\geq\min\{\vv(a),\vv(b)\}$ bzw.
$|a+b|\leq\max\{|a|,|b|\}$.
\item Ist $\vv(a)\neq\vv(b)$, so ist $\vv(a+b)=\min\{\vv(a),\vv(b)\}$
bzw. $|a+b|=\max\{|a|,|b|\}$.
\item $\vv(a)=\vv(-a)$.
\end{enumerate}
\item $\vv:\QQ_p^\times \Ra \ZZ$ heißt \emph{$p$-adische Bewertung},
$|\cdot|:\QQ_p^\times \Ra \RR$ heißt \emph{$p$-adischer Betrag}.
\index{p-adische Bewertung}\index{p-adischer Betrag}\index{$\vv(a)$, $|a|$ ($p$-adische Bewertung)}
\end{enumerate}

\PROP\
\begin{enumerate}
\item $d(x,y):=|x-y|$ ist eine Metrik auf $\QQ_p$.
\item Jedes Dreieck in $\QQ_p$ ist gleichschenklig und die dritte
Seite ist höchstens so lang wie einer der gleichen Schenkel.
\end{enumerate}
\bew
\begin{enumerate}
\item Es gilt
\begin{gather*}
d(x,y)=0\ \lra\ x-y=0\ \lra\ x=y,\\
d(y,x)=|y-x|=|-(x-y)|=|-1|\cdot|x-y|=d(x,y)\\
\text{und}\\
d(x,z)=|x-z|=|(x-y)+(y-z)|\leq\max\{|x-y|,|y-z|\}\leq d(x,y)+d(x,z).
\end{gather*}
\item ...
\end{enumerate}