\section{Symmetrische und äußere Algebra}

 \begin{Def} Seien $M,N$ $R$-Moduln, $n \geq 1$, $\Phi: M^n \rightarrow N$ $R$-multilinear.
  \begin{enumerate}
   \item[ a) ] $\Phi$ hei\ss t \emp{symmetrisch}\index{Abbildung!symmetrische}, wenn für alle $(x_1, \dots, x_n ) \in M^n$ und alle $\sigma \in S_n$ gilt:
    \[
    \Phi(x_1, \dots , x_n) = \Phi( x_{\sigma(1)},  \dots, x_{\sigma(n)}).
    \]
   \item[ b) ] $\Phi$ heißt \emp{alternierend}\index{Abbildung!alternierende}, wenn für alle $(x_1, \dots, x_n ) \in M^n$ gilt:\\
    Ist $x_i = x_j$ für ein Paar $(x_i,x_j)$ mit $i \neq j$, so ist $\Phi(x_1, \dots , x_n) = 0$
   \item[ c) ] $\Sym{M}{n}{N} := \{ \Phi: M^n \rightarrow N: \Phi$ multilinear, symmetrisch $\}$\\
    $\Alt{M}{n}{N} := \{ \Phi: M^n \rightarrow N: \Phi$ multilinear, alternierend $\}$\\
	 $\Sym{M}{n}{N}$ und  $\Alt{M}{n}{N}$ sind $R$-Moduln.
  \end{enumerate}
 \end{Def}
 \begin{Satz} Zu jedem $R$-Modul $M$ und jedem $n \geq 1$ gibt es $R$-Moduln $S^n(M)$ und $\Lambda^n(M)$ ( genannt die n-te \emp{symmetrische}\index{Potenz!symmetrische} bzw. \emp{äu\ss ere Potenz}\index{Potenz!\"au\ss ere} von $M$)
  und eine symmetrische bzw. alternierende multilineare Abb. $M^n \rightarrow S^n(M)$ bzw.  $M^n \rightarrow \Lambda ^n(M)$ mit folgender UAE:
\[
\begin{xy}
  \xymatrix{
      M^n \ar[rr] \ar[rd]_{\Phi\in \Sym{M}{n}{N}}  &     &  S^n(M) \ar[dl]^{\exists!\varphi\text{\ linear}}  \\
                             &  N  &
  }
\end{xy}
\]
beziehungsweise:
\[
\begin{xy}
  \xymatrix{
      M^n \ar[rr] \ar[rd]_{\Psi\in \Alt{M}{n}{N}}  &     &  \Lambda^n(M) \ar[dl]^{\exists!\psi\text{\ linear}}  \\
                             &  N  &
  }
\end{xy}
\]
  Mit $S^0(M) := R =: \Lambda^0(M)$ heißt $S(M):= \bigoplus_{n\geq 0} S^n(M)$ die \emp{symmetrische Algebra}\index{Algebra!symmetrische} über $M$\\
  $\Lambda (M) := \bigoplus_{n\geq 0} \Lambda^n(M)$ die \emp{äu\ss ere
  Algebra}\index{Algebra!\"au\ss ere} (oder \emp{Graßmann-Algebra}\index{Gra\ss mann-Algebra}) über $M$.
 \end{Satz}
 
 \begin{Bew}
  Sei $\mathbb{J}^n(M)$ der Untermodul von $T^n(M)$, der erzeugt wird von allen\\
  $x_1 \ten \dots \ten x_n -x_{\sigma(1)} \ten \dots \ten x_{\sigma(n)}$,
  $x_i \in M$, $ \sigma \in S_n$ und \\
  $\mathbb{I}^n(M)$ der Untermodul von $T^n(M)$, der erzeugt wird von allen
  $x_1 \ten \dots \ten x_n$ für die $x_i = x_j$ für ein Paar $(i,j)$ mit $i \neq j$.\\
  Setze $S^n(M) := T^n(M)/\mathbb{J}^n(M)$\\
  $\Lambda^n(M) := T^n(M)/\mathbb{I}^n(M)$ \\
  Sei $\Phi: M^n \rightarrow N$ multilinear und symmetrisch. $\Phi$ induziert 
  $\tilde{\varphi}:T^n(M) \rightarrow N$ $R$-linear (weil $\Phi$ multiliniear), da $\Phi$ symmetrisch ist, ist
  $\mathbb{J}^n(M) \subseteq \K{\varphi}$. $\tilde{\varphi}$ induziert also $\varphi:S^n(M) \rightarrow N$
  $R$-linear; genauso falls $\Psi : M^n \rightarrow N$ alternierend.
 \end{Bew}

 \begin{Prop}
  Sei $M$ freier $R$-Modul mit Basis $e_1, \dots, e_r$. Dann gilt für jedes $n \geq 1$:
  \begin{enumerate}
   \item[ a) ] $S^n(M)$ ist freier Modul mit Basis $\{e_1^{\nu_1} \cdot \dots
   \cdot e_r^{\nu_r}: \sum_{i=1}^{r}{\nu_i} =  n \}$
   \item[ b) ] $S(M) \cong R[X_1, \dots, X_r]$
   \item[ c) ] $\Lambda^n(M)$ ist freier $R$-Modul mit Basis \\
    $\{e_{i_1} \wedge \dots \wedge e_{i_n}: 1 \leq i_1 < i_2 < \dots < i_n \leq r \}$
   \item[ d) ] $\Lambda^n(M) = 0$ für $n > r$ 
  \end{enumerate}
 \end{Prop}
 
 \begin{Bew}
  \begin{enumerate}
   \item[ b) ] folgt aus a)
   \item[ d) ] folgt aus c)
   \item[ c) ] $\Lambda^rM$ wird erzeugt von $e_1 \wedge \dots \wedge e_r$: klar.\\
 $\Lambda^rM$ ist frei (vom Rang 1), denn aus $a \cdot e_1 \wedge \dots \wedge e_r = 0$ folgt $a=0$. \\
    Für $n<r$ bilden  $\{e_{i_1} \wedge \dots \wedge e_{i_n}: 1 \leq i_1 < i_2 < \dots < i_n \leq r \}$
    ein Erzeugendensystem. \\
    Zu zeigen: $\{e_{i_1} \wedge \dots \wedge e_{i_n}: 1 \leq i_1 < i_2 < \dots < i_n \leq r \}$ ist linear unabh\"angig.\\
 Sei dazu $\sum_{1 \leq i_1 < \dots < i_n \leq r} a_{\underline{i}}e_{i_1} \wedge \dots \wedge e_{i_n} = 0$\\
    F\"ur $j = (j_1, \dots, j_n)$ mit $1 \leq j_1 < \dots < j_n \leq r$ sei $\sigma_j \in S_r$ mit 
    $\sigma_j(\nu) = j_{\nu}$ für $\nu = 1, \dots n$. Dann ist $ 0= (\sum a_i e_{i_1} \wedge \dots \wedge e_{i_n}) \wedge 
    e_{\sigma_j(n+1)}  \wedge \dots \wedge e_{\sigma_j(r) } = a_j e_1 \wedge \dots \wedge e_r$ $\Rightarrow a_j = 0 \Rightarrow $l.u. 
  \end{enumerate}
 \end{Bew}
