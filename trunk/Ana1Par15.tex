\documentclass{article}
\newcounter{chapter}
\setcounter{chapter}{15}
\usepackage{ana}

\title{$g$-adische Entwicklungen}
\author{Joachim Breitner}

\begin{document}
\maketitle

\begin{vereinbarung}
Stets in diesem Paragraphen: $g\in\MdN$, $g\ge 2$, $G := \{0,1,\ldots,g-1\}$.
\end{vereinbarung}

\begin{satz}[Konvergenz $g$-adischer Entwicklungen]
\begin{liste}
\item Sei $(z_n)_{n\ge1}$ eine Folge in $G \folgt \reihe{\frac{z_n}{g^n}}$ ist konvergent.
\item Ist $m\in\MdN \folgt \sum_{n=m}^\infty{\frac{g-1}{g^n}} = \frac{1}{g^{m-1}}$
\end{liste}
\end{satz}

\begin{beweise}
\item $\frac{|z_n|}{g^n} = \frac{z_n}{g^n} \le \frac{g-1}{g^n} \ \forall n\in\MdN$. $\reihe{\frac{g-1}{g^n}}$ ist konvergent $\folgtnach{12.2}$ Behauptung.
\item $\sum_{n=m}^\infty\frac{g-1}{g^n} = \frac{g-1}{g^m} + \frac{g-1}{g^{m+1}} + \ldots = \frac{g-1}{g^m} \cdot ( 1+ \frac{1}{g} + \frac{1}{g^2} + \ldots ) = \frac{g-1}{g^m} \cdot \frac{1}{1-\frac{1}{g}} = \frac{1}{g^{m-1}}$.
\end{beweise}

\begin{definition}
Sei $(z_n)_{n\ge 1}$ eine Folge in  $G$ und es gelte $(*) z_n \ne g-1$ f�r unendlich viele $n\in\MdN$. Dann hei�t $0,z_1z_2z_3\ldots := \reihe{\frac{z_n}{g^n}}$ ein \begriff{$g$-adischer Bruch} oder eine \begriff{$g$-adische Entwicklung}.
\end{definition}

\begin{beispiele}
\item $g=10$ (Dezimalentwicklung); $0,333\ldots = \reihe{\frac{3}{10^n}} = \frac{1}{3}$.
\item $g=2$ (Dualentwicklung); $0,111000\ldots = \frac{1}{2} + \frac{1}{4} + \frac{1}{8} = \frac{7}{8}$.
\end{beispiele}

\begin{bemerkung}
\begin{liste}
\item Die Negation von $(*)$ lautet: $z_n = g-1 \ffa n\in\MdN$.
\item Ist $0,z_1z_2z_3\ldots$ ein $g$-adischer Bruch und existiert ein $\natn: z_n = 0$ f�r $n>m$, so schreibt man: $0,z_1z_2z_3\ldots z_m$
\item $\reihe{a_n}$ und $\reihe{b_n}$ seien konvergent und es gelte $a_n \le b_n \ \forall n\in\MdN \folgt \reihe{a_n} \le \reihe{b_n}$. Gilt zus�tzlich $a_n<b_n$ f�r ein \natn, so gilt $\reihe{a_n} < \reihe{b_n}$ (Beweis in �bung).
\end{liste}
\end{bemerkung}

\begin{satz}[Eindeutigkeit der $g$-adischen Entwicklung]
Sei $a=0,z_1z_2z_3\ldots$ ein $g$-adischer Bruch.
\begin{liste}
\item $a\in[0,1)$
\item Ist $0,w_1w_2w_3\ldots$ eine weitere $g$-adische Entwicklung von $a$, so gilt $z_n = w_n \ \forall n\in\MdN$.
\end{liste}
\end{satz}

\begin{beweise}
\item $0 \le a = \reihe{\frac{z_n}{g^n}} \stackrel{\text{(*)}, Bem. (2)}{<} \reihe{\frac{g-1}{g^n}} \gleichnach{15.1} 1$.
\item \textbf{Annahme:} $\exists n\in\MdN: z_n \ne w_n$. Sei $m$ der kleinste solche Index, also $z_m \ne w_m$ und $z_j = w_j$ f�r $j=1,\ldots ,m-1$. Etwa $z_m < w_m \folgt z_m - w_m < 0 \overset{z_m - w_m \in\MdZ}{\folgt} z_m - w_m \le -1$. $\forall n\in\MdN: z_n - w_n \le z_n \le g-1$. $\exists \nu \in\MdN$ mit $\nu \ge m+1$ und $z_\nu - w_\nu < g-1$. (andererenfalls $z_\nu - w_\nu = g-1 \ \forall \nu \ge m+1 \folgt z_\nu = w_\nu + g-1 \ \forall \nu \ge m+1 \folgt w_\nu = 0 \ \forall \nu \ge m+1 \folgt z_\nu = g-1 \ \forall \nu \ge m+1$. Widerspruch zu $(*)$). Dann: 
$\displaystyle{0 = a-a = \reihe{\frac{z_n}{g^n}} - \reihe{\frac{w_n}{g^n}} = \reihe{\frac{z_n - w_n}{g^n}} = \sum_{n=m}^\infty{\frac{z_n - w_n}{g^n}}}$\\
$\displaystyle{= \underbrace{\frac{z_m - w_m}{g^m}}_{\le -\frac{1}{g^m}} + \underbrace{\sum_{n=m+1}^\infty{\frac{z_n-w_n}{g^n}}}_{< \sum_{n=m+1}^\infty{\frac{g-1}{g^n}}} < - \frac{1}{g^m} + \underbrace{\sum_{n=m+1}^\infty{\frac{g-1}{g^n}}}_{\gleichnach{15.1} \frac{1}{g^n}} = 0}$\\
$\folgt 0<0 \text{  Widerspruch.}$
\end{beweise}

\begin{satz}[Existenz der $g$-adischen Entwicklung]
Ist $a\in[0,1)$, so l�sst sich $a$ eindeutig als $g$-adischer Bruch darstellen.
\end{satz}

\begin{beweis}
Eindeutigkeit siehe 15.2.\\
Existenz: Definiere $(z_n)_{n\ge1}$ wie folgt: $z_1:=[a\cdot g], z_{n+1} := [ ( a - \frac{z_1}{g} - \frac{z_2}{g} - \cdots -\frac{z_n}{g}) \cdot g^{n+1} ] \ (n\ge1)$. \\
In der �bung: $z_n \in G \ \forall n\in\MdN$

Es gilt: $(**) \underbrace{\frac{z_1}{g} + \frac{z_2}{g^2} + \cdots \frac{z_n}{g^n}}_{=: s_n} \le a < \underbrace{\frac{z_1}{g} + \frac{z_2}{g^2} + \cdots \frac{z_n}{g^n}}_{=: s_n} + \frac{1}{g^n} \ \forall n\in\MdN \folgt s_n \le a < s_n + \frac{1}{g^n} \ \forall n\in\MdN \folgtwegen{n\to\infty} a = \reihe{\frac{z_n}{g^n}}$.

Noch zu zeigen ist: $z_n \ne g-1$ f�r unendlich viele $n$.
\textbf{Annahme}: $\exists{m\in\MdN}: z_n = g-1 \ \forall n\ge m$.  Dann: $a = \reihe{z_n}{g^n} = \underbrace{\sum_{n=1}^{m-1}{\frac{z_n}{g^n}}}_{= s_{n-1}} + \underbrace{\sum_{n=m}^\infty{\frac{g-1}{g^n}}}_{= \frac{1}{g^{m-1}}} \folgt a = s_{n-1} + \frac{1}{g^{n-1}}$ Widerspruch zu $(**)$.
\end{beweis}

\begin{bemerkung}
Ist $a\in\MdR$, $a\ge0$, so l�sst sich $a$ eindeutig in der Form $a = [a]+0,a_1a_2a_3\ldots$ darstellen. Ist $g=10$, so schreibt man daf�r $a=[a],z_1z_2z_3\ldots$. Beispiel: $1,333\ldots$
\end{bemerkung}

\begin{satz}[$\MdR$ ist �berabz�hlbar]
Die Menge der reellen Zahlen ist �berabz�hlbar.
\end{satz}

\begin{beweis}
Es gen�gt zu zeigen: $[0,1)$ ist �berabz�hlbar.

\textbf{Annahme}: $[0,1)$ ist abz�hlbar, also $[0,1) = \{a_1,a_2,\ldots\}, a_j \ne a_k$ f�r $j\ne k$. 
F\"ur $j \in\MdN$ sei $a_j = 0,z_1^{(j)} z_2^{(j)} z_3^{(j)}\ldots$ die 3-adische Entwicklung von $a_j$. ($z_k^{(j)} = \{0,1,2\}$).
$$ z_k := \begin{cases} 1 & \text{falls } z_k^{(j)} \in \{0,2\} \\0 & \text{falls } z_k^{(j)}= 1 \end{cases}$$
Dann: $z_k \ne z_k^{(k)} \ \forall k \in\MdN$, $z_k \ne g-1 \ \forall k\in\MdN$. $a := 0,z_1z_2z_3\ldots = \reihe{\frac{z_n}{g^n}}.$ 15.2 $\folgt a \in [0,1) \folgt \exists m\in\MdN: a= a_m \folgt 0,z_1z_2z_3\ldots = 0,z_1^{(m)}z_2^{(m)}z_3^{(m)}\ldots$. 15.2 $\folgt z_j = z_j^{(m)} \ \forall j\in\MdN \folgt z_m = z_m^{(m)}$. Widerspruch!
\end{beweis}

\end{document}
