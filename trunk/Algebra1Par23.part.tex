\section{Quotienten}

\begin{DefBem}
\begin{enum}
\item Sei $I$ Ideal in $R$. Durch die Verkn�pfung $\bar x \cd \overline
y \defeqr \overline{xy}$ wird die Faktorgruppe $(R,+)/(I,+)$ ein
kommutativer Ring mit Eins. $R/I$ hei�t \emp{Faktorring} oder
\emp{Quotientenring} von $R$ und $I$

\item Die Restklassenabbildung $\pi: R \ra R/I, \; x \mapsto
\bar x$ ist surjektiver Ringhomomorhpismus mit Kern($\pi$)$=I$.

\item (UAE des Faktorrings:) Sei $\varphi: R \ra R'$ ein
Ringhomomorphismus. Dann gibt es zu jedem Ideal $I \subseteq R$ mit
$I \subseteq$ Kern($f$) einen eindeutig bestimmten
Ringhomomorphismus $\bar \varphi: R/I \ra R'$ mit $\varphi = \bar
\varphi \circ \pi$

%\[\begindc{\commdiag}
%\obj(1,3){$R$}
%\obj(3,3){$R'$}
%\obj(2,1){$R/I$}
%\mor{$R$}{$R'$}{$\varphi$}[-1,0]
%\mor{$R$}{$R/I$}{$\pi$}[-1, 0]
%\mor{$R/I$}{$R'$}{$\exists!\; \bar{\varphi}$}[-1,1]
%\enddc\]

\item (Homomorhpiesatz f�r Ringe:) Ist $\varphi: R \ra R'$
surjektiver Ringhomomorhpismus, dann ist $R' \cong
R/$Kern($\varphi$).
\end{enum}
\bew{}{\item \emp{Wohldef. des Produkts:} Seien $x', y' \in R$ mit
$\overline{x'} = \overline x, \; \overline{y'} = \overline{y}$. Dann gibt es $a,b \in I$
mit $x' = x+a,\; y' = y+b$. $\Ra x' y' = (x+a) (y+b) = xy + \underset{\in
I}{\underbrace{ay + bx + ab}} \Ra \bar{x'} \bar{y'} = \bar x
\bar y$. \newline Die restlichen Eigenschaften vererben sich dann
von $R$.
\item $\pi$ ist surjektiver Gruppenhomomorphismus mit
Kern($\varphi$)$=I$ nach Satz \ref{Satz 1}(a). $\pi(xy) = \pi(x) \cd \pi(y)$
nach Definition der Verkn�pfung.
\item Nach Satz \ref{Satz 1}(b) gibt es einen eindeutig bestimmten
Gruppenhomomorphismus $\bar \varphi: R/I \ra R'$ mit $\varphi = \bar
\varphi \circ \pi$.
\newline Zeige also: $\bar \varphi$ ist Ringhomomorphismus: F�r
$x,y \in R$ ist $\bar \varphi(\overline{x}\overline{y}) = \varphi(xy) =
\varphi(x) \varphi(y) = \bar\varphi(\bar x) \bar\varphi(\bar y)$
\item Folgt aus (c) und Satz \ref{Satz 1}(a)
}
\end{DefBem}

\begin{DefBem}
\begin{enum}
\item Ein Ideal $I \subsetneq R$ hei�t \emp{maximal}, wenn es kein
Ideal $I'$ in $R$ gibt mit $I \subsetneq I' \subsetneq R$

\item Ein Ideal $I \subsetneq R$ hei�t \emp{Primideal}, wenn f�r $x,y
\in R$ mit $xy \in I$ gilt: $x \in I$ oder $y \in I$
\newline\sbsp{0.9}{\begin{enum} \item[(1)]$p$ prim $\lra p \mathbb{Z}$
ist Primideal in $\mathbb{Z}$ (sogar maximal)
\item[(2)] $(x)$ ist Primideal in $R \llbracket X \rrbracket \lra R$
ist K�rper.\end{enum}}

\item $R$ ist nullteilerfrei $\lra (0)$ ist Primideal.

\item Jedes maximale Ideal ist Primideal.
\end{enum}
\bew{}{\item[(c)] $R$ ist nicht nullteilerfrei $\lra \exists a,b \in
R \setminus\{0\}$: $ab = 0 \lra (0)$ kein Primideal.
\item[(d)] Seien $x,y \in R$ mit $xy \in I$ und $x \not \in I$. Dann
ist $(x) + I \supset I \overset{I \scriptsize \mbox{ max.}}{\Ra} (x)
+ I = R \Ra 1 \in (x) + I,$ dh. es gibt $x \in R, a \in I$ mit $1 = rx
+a \Ra y = \underset{\in I}{\underbrace{rxy}} + \underset{\in
I}{\underbrace{ay}} \in I \Ra I$ Primideal. }
\end{DefBem}

\begin{Bem}
Sei $I \subset R$ ein Ideal. Dann gilt:
\begin{enum}
\item $I$ ist Primideal $\lra R/I$ ist nullteilerfrei.
\item $I$ ist maximales Ideal $\lra R/I$ ist K�rper.
\end{enum}
\bew{}{\item $R/I$ ist nicht nullteilerfrei $\lra \exists \overline{x}
\neq \overline{0} \neq \overline{y} \in R/I$ mit $\overline{x} \cd \overline{y}
= \overline{0} =
\overline{xy} \lra xy \in I,\;x,y \not \in I \Ra I$ kein Primideal.
\item Nach 2.6(d) ist $R/I$ genau dann K�rper, wenn $(0)$ und $R/I$
die einzigen Ideal in $R/I$ sind. Nach �B7/A3 entsprechen die Ideale
in $R/I$ bijektiv den Idealen in $R$, die $I$ enthalten.}

\bsp{ Sei $C = \{(a_n)_{n \in \mathbb{N}}: (a_n)$ Cauchy-Folge, $a_n
\in \mathbb{Q}\}$ (dh. f�r $k \in \mathbb{N}\; \exists n \in
\mathbb{N}: |a_i - a_j| < \frac{1}{k}$ f�r $i,j \geq n$) \newline
$C$ ist Ring mit komponentenweiser $+$ und $\cd$ (vornehm: $C
\subset \prod_{n \in \mathbb{N}} \mathbb{Q})$.
\newline $N = \{(a_n) \in C : (a_n)$ Nullfolge $\}$ (dh. f�r $k \in
\mathbb{N}\; \exists n \in \mathbb{N} : |a_i| < \frac{1}{k}\; \forall i
> n$)\newline
 $N$ ist Ideal in $C:\chk$ \newline
\textbf{Beh.}: $C/N$ ist K�rper (bzw. $N$ ist maximal)
\newline \sbew{0.9}{Sei $a = (a_n)_{n \in \mathbb{N}} \in C
\setminus N$. zu zeigen: $1 \in N + (a) = \langle N \cup
\{a\}\rangle$. \newline $(a_n) \not \in N \Ra a_n = 0$ nur f�r
endlich viele $n$, dh. $a_i\neq 0$ f�r $i > n_0$. \[b_n \defeqr \left\{
\begin{array}{crl} 0 &,& a_i = 0 | i \leq n_0
\\ \frac{1}{a_i} &,& a_i \neq 0 | i > n_0 \end{array} \right.\]
$b=(b_n) \in C$.
\[ ab = (c_n),\; c_n = \left\{ \begin{array}{crl} 0
&:& n < n_0 \\ 1 &:& n \geq n_0 \end{array} \right.\] \[\Ra 1 -ab=(d_n),\;
d_n = \left\{
\begin{array}{crl} 1 &:& n < n_0 \\ 0 &:& n \geq n_0 \end{array}
\right.\] $\Ra (d_n) \in N \Ra 1 = (d_n) + ba \in N + (a) \Ra N$
maximal.
\[\Ra C/N = \mathbb{R} \mbox{!}\] }} 
\end{Bem}

\begin{Satz}[Chinesischer Restsatz]
\label{Satz 8}
Sei $R$ kommutativer Ring mit Eins, $I_1,\dots,I_n$
Ideale in $R$ mit $I_\nu + I_\mu = R$ f�r alle $\nu \not= \mu$ (dann
hei�en $I_\nu, I_\mu$ \emp{relativ prim} oder \emp{koprim}) F�r $\nu
= 1,\dots,n$ sei $p_\nu : R \ra R/I_\nu$ die Restklassenabbildung.
Dann gilt:
\begin{enum}
\item $\begin{array}{ccc}\varphi: R &\ra& R/I_1 \times \dots \times R/I_n \\
x &\mapsto& (p_1(x),\dots,p_n(x)) \end{array}$ ist surjektiv.

\item $R/I_1 \times \dots \times R/I_n \cong R/\bigcap_{\nu=1}^n
I_\nu$ (klar nach Homomorphiesatz: Kern($\varphi$)$= \bigcap_{\nu =
1}^n I_\nu$
\item (Simultane Kongruenzen:) \newline F�r paarweise teilerfremde ganze
Zahlen $m_1,\dots,m_n$ und beliebige $r_1,$ $\dots,r_n \in
\mathbb{Z}$ gibt es $x \in \mathbb{Z}$ mit $x \equiv r_\nu \mod
m_\nu$ f�r $\nu = 1,\dots,n$ (Spezialfall von (a) f�r
$R=\mathbb{Z}$) \end{enum} \noindent \bew{}{\item Es gen�gt z.z.:
$\begin{array}{lcl}(0,\dots,0,&\underbrace{1},&0,\dots,0) \\ &
\nu\mbox{-te Stelle} &
\end{array} \in$ Bild($\varphi$) f�r jedes $\nu$, dh. es gibt $e_\nu
\in R\; (\nu=1,\dots,n)$ mit $e_\nu \in I_\mu$ f�r $\nu \neq \mu$ und
$1-e_\nu \defeql a_\nu \in I_\nu$ (denn f�r $x = (\bar
x_1,\dots, \bar x_n) \in R/I_1 \times \dots \times R/I_n$ sei $e
\defeqr \sum_{\nu=1}^n r_\nu e_\nu$ mit $r_\nu \in p_\nu^{-1}(\bar
r_\nu) \Ra \varphi(e) = \sum p_\nu (r_\nu e_\nu) = x$)
\newline Nach Voraussetzung gibt es f�r jedes $\ds \mu = \nu\; a_\mu \in
I_\nu,b_\mu \in I_\mu$ mit \[a_\mu + b_\mu = 1 \Ra 1 =
\prod_{\substack{\mu = 1 \\ \mu \neq \nu}}^n (a_\mu + b_\mu) =
\underset{\ds \defeql e_\nu \in \bigcap_{\substack{\mu=1 \\ \mu \neq
\nu}}^n I_\mu}{\underbrace{\prod_{\substack{\mu = 1 \\ \mu \neq
\nu}}^n b_\mu}} + \underset{\in I_\nu}{\underbrace{a_\nu}}\] $\Ra 1
% TODO n�chste Zeile sollte eigentlich hei�en e_\nu - a_\nu ???
= e_\nu - a_\nu$ wie gew�nscht.}
\end{Satz}
