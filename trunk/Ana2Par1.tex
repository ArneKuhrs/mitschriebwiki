\documentclass{article}
\newcounter{chapter}
\setcounter{chapter}{1}
\usepackage{ana}

\title{Der Raum $\MdR^n$}
\author{Wenzel Jakob, Joachim Breitner}
% Wer nennenswerte �nderungen macht, schreibt euch bei \author dazu

\begin{document}
\maketitle

Sei $n\in\MdN$. $\MdR^n=\{(x_1, \ldots, x_n) : x_1,\ldots, x_n \in \MdR\}$ ist mit der "ublichen Addition und Skalarmultiplikation ein reeller Vektorraum.\\
$e_1:=(1,0,\ldots,0),\ e_2:=(0,1,0,\ldots, 0),\ \ldots,\ e_n:=(0,\ldots,0,1) \in \MdR^n$.

\begin{definition}
Seien $x=(x_1, \ldots, x_n), y=(y_1, \ldots, y_n) \in \MdR^n$
\begin{liste}
\item $x\cdot y := xy := x_1y_1+\cdots+x_ny_n$ hei"st das \textbf{Skalar}\indexlabel{Skalarprodukt}- oder \begriff{Innenprodukt} von $x$ und $y$.
\item $\|x\|=(x\cdot x)^\frac{1}{2} = (x_1^2 + \cdots + x_n^2)^\frac{1}{2}$ hei"st die \begriff{Norm} oder \begriff{L"ange} von $x$.
\item \indexlabel{Abstand!zwischen zwei Vektoren}$\|x-y\|$ hei"st der \textbf{Abstand} von $x$ und $y$.
\end{liste}
\end{definition}

\begin{beispiele}
\item $\|e_j\|=1\ (j=1,\dots,n)$
\item $n=3: \|(1,2,3)\|=(1+4+9)^{\frac{1}{2}}=\sqrt{14}$
\end{beispiele}

\textbf{Beachte: }
\begin{liste}
\item $x \cdot y \in \MdR$
\item $\|x\|^2=x \cdot x$
\end{liste}

\begin{satz}[Rechnenregeln zur Norm]
Seien $x,y,z \in \MdR^n,\ \alpha, \beta \in \MdR,\ x=(x_1, \ldots, x_n),\ y=(y_1, \ldots, y_n)$
\begin{liste}
\item $(\alpha x + \beta y)\cdot z=\alpha(x\cdot z)+\beta(y \cdot z),\ x(\alpha y + \beta z)=\alpha(xy)+\beta(xz)$
\item $\|x\|\ge 0; \|x\|=0\equizu x=0$
\item $\|\alpha x\|=|\alpha|\|x\|$
\item $|x \cdot y|\le\|x\| \|y\|$ \begriff{Cauchy-Schwarzsche Ungleichung} (\begriff{CSU})
\item $\|x+y\|\le\|x\|+\|y\|$
\item ${\left|\|x\|-\|y\|\right|}\le \|x-y\|$
\item $|x_j|\le\|x\|\le |x_1|+|x_2|+\ldots+|x_n|\ (j=1,\ldots,n)$
\end{liste}
\end{satz}

\begin{beweise}
\item[(1)], (2), (3)\ nachrechnen.
\item[(6)] "Ubung.
\item[(4)] O.B.d.A: $y\ne0$ also $\|y\|>0$. $a:=x\cdot x=\|x\|^2,\ b:=xy,\ c:=\|y\|^2=y\cdot y,\ \alpha:=\frac{b}{c}.\ 0\le\sum_{j=1}^n(x_j-\alpha y_j)^2=\sum_{j=1}^n(x_j^2-2\alpha x_jy_j+a^2y^2)=a-2\alpha b + \alpha^2 c=a-2\frac{b}{c}b+\frac{b^2}{c^2}c=a-\frac{b^2}{c}\folgt0\le ac-b^2\folgt b^2\le ac\folgt(xy)^2\le\|x\|^2\|y\|^2$.
\item[(5)] $\|x+y\|^2=(x+y)(x+y)\gleichnach{(1)}x\cdot x + 2xy+y \cdot y=\|x\|^2+2xy+\|y\|^2\le\|x\|^2+2|xy|+\|y\|^2\overset{\text{(4)}}{\le}\|x\|^2+2\|x\|\|y\|+\|y\|^2=(\|x\|+\|y\|)^2$.
\item[(7)] $|x_j|^2=x_j^2\le x_1^2 + \ldots + x_n^2 = \|x\|^2\folgt$ 1. Ungleichung; $x=x_1e_1+\ldots+x_ne_n\folgt\|x\|=\|x_1e_1+\ldots+x_ne_n\|\overset{(5)}{\le}\|x_1e_1\|+\ldots+\|x_ne_n\|=|x_1|+\ldots+|x_n|$
\end{beweise}

Seien $p,q,l \in \MdN$. Es sei $A$ eine reelle $p${\tiny x}$q$-Matrix.

$$A = \begin{pmatrix}
\alpha_{11} & \cdots & \alpha_{1q}\\
\vdots & & \vdots\\
\alpha_{p1} & \cdots & \alpha_{pq}
\end{pmatrix}\qquad \|A\|:=\left(\sum_{j=1}^p\sum_{k=1}^q\alpha^2_{jk}\right)^\frac{1}{2} \text{\textbf{Norm} von A}$$
Sei $B$ eine reelle $q${\tiny x}$l$-Matrix ($\folgt AB$ existiert). \textbf{"Ubung}: $\|AB\|\le\|A\|\|B\|$\\
Sei $x=(x_1,\ldots,x_q) \in \MdR^q$. $Ax:=A\begin{pmatrix}x_1\\ \vdots \\ x_q\end{pmatrix}$ (\begriff{Matrix-Vektorprodukt}). \\
Es folgt: $$\|Ax\|\le\|A\|\|x\|$$

\begin{definition}
Sei $x_0 \in \MdR^n$, $\delta > 0$, $A, U\subseteq \MdR^n$.
\begin{liste}
 \item $U_\delta(x_0) := \{ x \in \MdR^n: \|x-x_0\|<\delta\}$ hei�t $\delta$-Umgebung von $x_0$ oder \begriff{offene Kugel} um $x_0$ mit Radius $\delta$.
 \item $U$ ist eine \begriff{Umgebung} von $x_0$ $:\equizu$ $\exists \delta > 0 : U_\delta(x_0) \subseteq U$.
 \item \indexlabel{Beschr�nktheit!einer Menge}$A$ hei�t \textbf{beschr�nkt} $:\equizu$ $\exists c \ge 0: \|a\|\le c \forall a\in A$.
 \item $x_0\in A$ hei�t ein \begriff{innerer Punkt} von A $:\equizu$ $\exists \delta>0: U_\delta(x_0) \subseteq A$. \\
   $A^\circ:=\{ x\in A: x \text{ x ist innerer Punkt von }A\}$ hei�t das \indexlabel{Inneres einer Menge}\textbf{Innere} von A. Klar: $A^\circ\subseteq A.$
 \item $A$ hei�t offen $:\equizu$ $A=A^\circ$. Zur �bung: $A^\circ$ ist offen.
\end{liste}
\end{definition}

\begin{beispiele}
 \item offene Kugeln sind offen, $\MdR^n$ ist offen, $\emptyset$ ist offen.
 \item $A=\{x\in\MdR^n: \|x-x_0\|\le \delta\}$, $A^\circ = U_\delta(x_0)$
 \item $n=2$: $A=\{(x_1,x_2)\in\MdR^n: x_2 = x_1^2\}$, $A^\circ=\emptyset$
\end{beispiele}

\begin{definition}
 $A\subseteq \MdR^n$
 \begin{liste}
 \item $x_0\in \MdR^n$ hei�t ein \begriff{H�ufungspunkt} (HP) von $A$ $:\equizu$ $\forall \delta > 0: (U_\delta(x_0) \backslash \{x_0\}) \cap A \ne \emptyset$. $\H(A) := \{ x\in\MdR^n: x \text{ ist H�ufungspunkt von } A\}$.
 \item  $x_0\in\MdR^n$ hei�t ein \begriff{Ber�hrungspunkt} (BP) von $A$ $:\equizu$ $\forall\delta>0: U_\delta(x_0) \cap A \ne \emptyset$. $\bar{A}:=\{x\in\MdR^n: x \text{ ist ein Ber�hrungspunkt von } A\}$ hei�t die \begriff{Abschlie�ung} von $A$. Klar: $A\subseteq\bar{A}$. Zur �bung: $\bar{A} = A \cup \H(A)$.
 \item \indexlabel{abgeschlossene Menge}$A$ hei�t \textbf{abgeschlossen} $:\equizu$ $A=\bar{A}$. Zur �bung: $\bar{A}$ ist abgeschlossen.
 \item $x_0\in\MdR^n$ hei�t ein \begriff{Randpunkt} von $A$ $:\equizu$ $\forall\delta>0: U_\delta(x_0) \cap A \ne \emptyset$ und $U_\delta(x_0) \cap (\MdR^n\backslash A) \ne \emptyset$. $\partial A := \{x\in\MdR^n: x \text{ ist ein Randpunkt von } A \}$ hei�t der \begriff{Rand} von $A$. Zur �bung: $\partial A = \bar{A}\backslash A^\circ$.
 \end{liste}
\end{definition}

\begin{beispiele}
\item $\MdR^n$ ist abgeschlossen, $\emptyset$ ist abgeschlossen; \\ 
  $\bar{A}=\bar{U_\delta(x_0)} = \{x\in\MdR^n: \|x-x_0\| \le \delta\}$ (\begriff{abgeschlossene Kugel} um $x_0$ mit Radius $\delta$)
\item $\partial U_\delta(x_0) = \{x\in\MdR^n: \|x-x_0\|=\delta\} = \partial \bar{U_\delta(x_0)}$
\item $A = \{(x_1,x_2)\in\MdR^2; x_2 = x_1^2\}$. $A=\bar{A}=\partial A$
\end{beispiele}

\begin{satz}[Offene und abgeschlossene Mengen]
 \begin{liste}
 \item Sei $A\subseteq\MdR^n$. $A$ ist abgeschlossen $:\equizu$  $\MdR^n\backslash A$ ist offen.
 \item Die Vereinigung offener Mengen ist offen.
 \item Der Durschschnitt abgeschlossener Mengen ist abgeschlossen.
 \item Sind $A_1,\ldots,A_n\subseteq\MdR^n$ offen $\folgt$ $\bigcap_{j=1}^nA_j$ ist offen
 \item Sind $A_1,\ldots,A_n\subseteq\MdR^n$ abgeschlossen $\folgt$ $\bigcap_{j=1}^nA_j$ ist abgeschlossen
 \end{liste}
\end{satz}

\begin{beispiel}
  $(n=1)$. $A_t := (0,1+t)\ (t>0)$. Jedes $A_t$ ist offen. $\bigcap_{t>0}A_t = (0,1]$ ist nicht offen.
\end{beispiel}

\begin{beweise}
 \item \glqq\folgt\grqq: Sei $x_0\in\MdR^n\backslash A$. Annahme: $\forall \delta>0: U_\delta(x_0) \nsubseteq \MdR^n\backslash A\folgt\forall\delta>0: U_\delta(x_0)\cap A\ne\emptyset \folgt x_0\in\bar{A} \gleichnach{Vor.} A$, Widerspruch \\
 \glqq$\Leftarrow$\grqq: Annahme: $\subset\bar{A} \folgt \ \exists x_0\in\bar{A}: x_0\notin A$; also $x_0\in\MdR^n\backslash A$. Voraussetzung $\folgt \ \exists \delta > 0: U_\delta(x_0) \subseteq \MdR^n\backslash A \folgt U_\delta(x_0) \cap A = \emptyset \folgt x_0 \notin \bar{A}$, Widerspruch!
 \item Sei $(A_\lambda)_{\lambda\in M}$ eine Familie offener Mengen und $V := \bigcup_{\lambda\in M} A_\lambda$. Sei $x_0\in V \folgt \exists\lambda_0\in M: x_0 \in A_{\lambda_0}$. $A_{\lambda_0}$ offen $\folgt \ \exists \delta > 0: U_\delta(x_0) \subseteq A_{\lambda_0} \subseteq V$
 \item folgt aus (1) und (2) (Komplemente!)
 \item $D:=\bigcap_{j=1}^mA_j$. Sei $x_0\in D$. $\forall j\in\{1,\ldots,m\}: x_0\in A_j$, also eixistiert $\delta_j>0: U_\delta(x_0)\subseteq A_j$. $\delta := \min\{\delta_j,\ldots,\delta_m\} \folgt U_\delta(x_0) \subseteq D$
 \item folgt aus (1) und (4)
\end{beweise}

\end{document}
