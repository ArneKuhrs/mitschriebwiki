\documentclass{article}
\newcounter{chapter}
\usepackage{ana}


\author{Wenzel Jakob}
\title{Gleichm��ige Stetigkeit}
\setcounter{chapter}{20}

\setlength{\parindent}{0pt}
\setlength{\parskip}{2ex}

\begin{document}
\maketitle

\begin{vereinbarung}
In diesem Paragraphen seien stets: $\emptyset \ne D \subseteq \MdR, f: D \to \MdR$ eine Funktion.
\end{vereinbarung}
\begin{erinnerung}
Sei $f \in C(D), x_0 \in D$ und $\ep>0$. 17.1 $\folgt \exists \delta=\delta(\ep,x_0)$ mit: $(*)\ |f(x)-f(x_0)|<\ep\ \forall x \in D$ mit $|x-x_0|<\delta$ Im allgemeinen h"angt $\delta$ von $\ep$ und $x_0$ ab.
\end{erinnerung}

\begin{definition}
$f$ hei�t auf $D$ \begriff{gleichm"a�ig stetig} $:\equizu\ \forall \ep>0\ \exists \delta=\delta(\ep)>0: (**)\ |f(x)-f(z)| < \ep\ \forall x,z \in D$ mit $|x-z|<\delta.$
\end{definition}

\textbf{Beachte:} Ist $f$ gleichm��ig stetig auf $D \folgt f \in C(D)$; Die Umkehrung ist im allgemeinen falsch.

\begin{beispiel}
$D=[0, \infty), f(x):=x^2$. Klar: f $\in C(D)$. Annahme: $f$ ist auf $D$ gleichm"a�ig stetig. Dann existiert zu $\ep=1$ ein $\delta>0: |x^2-z^2|<1\ \forall x,z \in D$ mit $|x-z|<\delta$. Sei $x \in D$. $z:=x+\frac{\delta}{2} \folgt |x-z|=\frac{\delta}{2} \folgt |x^2-z^2|=|x+z||x-z|=(2x + \frac{\delta}{2})\frac{\delta}{2}=x\delta + \frac{\delta^2}{4}<1 \folgt x\delta<1 \folgt \delta<\frac{1}{x}$. Also: $\delta<\frac{1}{x}\ \forall x>0 \folgtwegen{x\to\infty} \delta \le 0$, Widerspruch!
\end{beispiel}

\begin{definition}
$f$ hei�t auf $D$ \begriff{Lipschitz stetig} $:\equizu \exists L\ge 0: \underbrace{|f(x)-f(z)|\le L|x-z|}_{(***)}\ \forall x,z \in D$
\end{definition}

\begin{satz}[Stetigkeitsst�tze]
\begin{liste}
\item Ist $f$ auf $D$ Lipschitz stetig $\folgt f$ ist auf $D$ gleichm"a�ig stetig
\item Ist $D$ beschr"ankt und abgeschlossen und $f\in C(D) \folgt f$ ist auf D gleichm"a�ig stetig (\begriff*{Satz von Heine}\indexlabel{Heine, Satz von}).
\end{liste}
\end{satz}

\begin{beweise}
\item Sei $L\ge 0$ und es gelte $(***)$. O.B.d.A.: $L\> 0$. Sei $\ep>0$. $\delta:=\frac{\ep}{L}$. Seien $x, z \in D$ und $|x-z|<\delta$ \folgt $|f(x)-f(z)| \le L|x-z|<L\delta=\ep$
\item Annahme: $f$ ist auf $D$ nicht gleichm"a�ig stetig $\folgt \ \exists \ep>0: (**)$ ist f"ur kein $\delta>0$ richtig. $\folgt \ \forall\delta>0\ \exists x=x(\delta),z=z(\delta) \in D: |x-z|<\delta$ aber $|f(x)-f(z)|\ge \ep$. $\folgt \forall n \in \MdN\ \exists x_n, z_n: |x_n-z_n|<\frac{1}{n}$, aber $|f(x_n)-f(z_n)| \ge \ep$. D beschr"ankt $\folgt (x_n)$ beschr"ankt $\folgtnach{8.2} (x_n) $ enth"alt eine konvergente Teilfolge $(x_{n_k}), x_0:=\lim x_{n_k}$. D abgeschlossen $\folgt x_0 \in D$. $|x_{n_k}-z_{n_k}| \le \frac{1}{n_k}\ \forall k \in \MdN \folgt z_{n_k} - x_{n_k} \to 0 (k \to \infty) \folgt z_{n_k} = z_{n_k} - x_{n_k} + x_{n_k} \to x_0$. f stetig $\folgt |f(x_{n_k})-f(z_{n_k})| \to |f(x_0)-f(z_0)|=0$. Widerspruch zu $|f(x_{n_k})-f(z_{n_k})|\ge \ep \ \forall k \in \MdN$
\end{beweise}

\begin{beispiel}
$D=[0,1], f(x):=\sqrt{x}$. Satz $\folgt f$ ist auf D gleichm"a�ig stetig. Annahme: $\ \exists L>0: |\sqrt{x}-\sqrt{z}|\le L|x-z| \ \forall x,z \in [0,1] \folgt \sqrt{x} \le Lx\ \forall x \in [0,1]\ \folgt 1\le L\sqrt{x}\ \forall x \in (0,1] \folgtwegen{x\to0}1\le0$, Widerspruch!
\end{beispiel}

\end{document}
