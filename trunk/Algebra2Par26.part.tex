\section{Nakayama, Krull und Artin-Rees}

\begin{DefBem}

Sei $R$ ein Ring.
\begin{enumerate}
	\item $$\mathcal{J}(R) \defeqr \bigcap_{\mathfrak{m}\mbox{\scriptsize maximales Ideal in }R}\mathfrak{m}$$ hei�t \emp{Jacobson-Radikal}\index{Jacobson-Radikal} von $R$.
	\item $\mathcal{J}(R)$ ist Radikalideal.
	\item F�r jedes $a \in \mathcal{J}$ ist $1-a$ eine Einheit in $R$.
\end{enumerate}

\end{DefBem}

\begin{Bew}

\begin{enumerate}
	\stepcounter{enumi}
	\item Sei $x \in R, \; x^n \in \mathcal{J}(R)$; zu zeigen: $x \in \mathcal{J}(R)$.\\
	Sei $\mathfrak{m}$ maximales Ideal von $R$, dann ist $x^n \in \mathfrak{m} \overset{\mathfrak{m} \; \mbox{\scriptsize prim}}{\Rightarrow} x \in \mathfrak{m} \Rightarrow x \in \mathcal{J}(R)$
	\item Ist $1-a \notin R^{\times}$, so gibt es ein maximales Ideal $\mathfrak{m}$ mit $1-a \in \mathfrak{m}$, aber: $a$ ist auch $\in \mathfrak{m}$, also auch $1 = 1-a+a \in \mathfrak{m} \Rightarrow$ Widerspruch
\end{enumerate}
  
\end{Bew}

\begin{nnBsp}
  $\mathcal{J}(\mathbb{Z}) = 0, \quad \mathcal{J}(k[X]) = 0$\\
  $R$ lokaler Ring $\Rightarrow \mathcal{J}(R) = \mathfrak{m}$ (es gibt nur ein maximales Ideal in $R$)
\end{nnBsp}

\begin{Satz}[Lemma von Nakayama]
\label{Satz8}
  Sei $R$ ein Ring, $I \subseteq \mathcal{J}(R)$ ein Ideal, $M$ ein endlich erzeugbarer $R$-Modul, $N \subseteq M$ ein Untermodul.\\
  Dann gilt: Ist $M = I \cdot M + N$, so ist $N = M$.
\end{Satz}

\begin{Bew}
  Sei $M=I \cdot M + N \Rightarrow M/N = (I \cdot M)/N = I \cdot (M/N)$, also O\!\!E N=0\\
  Annahme: $M \not= 0$\\
  Dann sei $x_1, \dots, x_n$ ein minimales Erzeugendensystem von $M$, also $M' \defeqr \langle x_2, \dots, x_n\rangle \subsetneq M$.
  Nach Voraussetzung ist $M = I \cdot M$, also $x_1 \in I \cdot M \Rightarrow \exists a_1, \dots, a_n \in I$ mit $x_1 = \sum_{i=1}^n a_i x_i = a_1 x_1 + \underset{\in M}{\underbrace{a_2 x_2 + \dots + a_n x_n}} \Rightarrow x_1\underset{\in R^{\times}}{\underbrace{(1-a_1)}} \in M' \Rightarrow x_1 \in M' \Rightarrow$ Widerspruch
\end{Bew}

\begin{Folg}
\label{2.21}
  $R, \; I, \; M$ wie in Satz  \ref{Satz8}.\\
  Dann gilt f�r $x_1, \dots, x_n \in M$:
  \[ x_1, \dots, x_n \mbox{ erzeugt } M \Leftrightarrow \bar{x}_1, \dots, \bar{x}_n \mbox{ erzeugen } \overline{M} = M/IM\]
\end{Folg}

\begin{Bew}
  \textbf{``$\Rightarrow$``:} klar\\
  \textbf{``$\Leftarrow$``:} Sei $N$ der von $x_1, \dots, x_n$ erzeugte Untermodul von $M$. Dann ist $M = N + I \cdot M \overset{\mbox{\scriptsize Satz \ref{Satz8}}}{\Rightarrow} M = N$.
\end{Bew}

\begin{nnBsp}
  $R$ lokaler Ring mit maximalem Ideal $\mathfrak{m}$.\\
  $M = N \overset{\mbox{\scriptsize falls }\mathfrak{m} \; \mbox{\scriptsize endlich erzeugt}}{\Rightarrow} \mathfrak{m}^2 = \mathfrak{m} \Rightarrow \mathfrak{m} = 0$, also $R$ K�rper, falls $\mathfrak{m}$ endlich erzeugt, d.h. falls $R$ noethersch.
\end{nnBsp}

\begin{Satz}[Durchschnittssatz von Krull]
\label{Satz9}
  Sei $R$ noethersch, $M$ endlicher erzeugbarer $R$-Modul, $I \subseteq R$ Ideal.\\
  Dann gilt f�r $N \defeqr \bigcap_{n \geq 0} I^n M: \; I \cdot N = N$.
\end{Satz}

\begin{Folg}
\label{2.22}
  \begin{enumerate}
    \item \label{2.22a} Ist in Satz \ref{Satz9} $I \subseteq \mathcal{J}(R)$, so ist $N = 0$.
    \item Ist $R$ nullteilerfrei, so ist $\bigcap_{n \geq 0} I^n = 0$, falls $I \neq R$.
  \end{enumerate}
\end{Folg}

\begin{Bew}
  \begin{enumerate}
    \item klar
    \item Sei $\mathfrak{m}$ ein maximales Ideal mit $I \subseteq \mathfrak{m}$.
    $R_{\mathfrak{m}}$ die Lokalisierung von $R$ nach $\mathfrak{m}$.
    $R_{\mathfrak{m}}$ ist noethersch, lokal, also $\mathcal{J}(R) = \mathfrak{m} R_{\mathfrak{m}}$.\\
    $i: R \to R_{\mathfrak{m}}, \; a \mapsto \frac{a}{1}$ ist injektiv, da $R$ nullteilerfrei.\\
    Dann ist $i(\bigcap_{n \geq 0} I^n) \subseteq \bigcap_{n \geq 0} i(I^n) \subseteq \bigcap_{n \geq 0}(\mathfrak{m} R_{\mathfrak{m}})^n \overset{\scriptsize\ref{2.22a}}{=} 0$. Da $i$ injektiv ist, folgt $\bigcap_{n \geq 0} I^n = 0$.
  \end{enumerate}
\end{Bew}

\begin{Prop}[Artin-Rees]
\label{2.23}
  Sei $R$ noethersch, $I \subseteq R$ Ideal, $M$ endlich erzeugbarer $R$-Modul, $N \subseteq M$ Untermodul.
  Dann gibt es ein $n_0 \in \mathbb{N}$, sodass f�r alle $n \geq n_0$ gilt:
  \[I^n M \cap N = I ^{n-n_0} (I^{n_0}M \cap N)\]
\end{Prop}

\begin{Bew}[Satz \ref{Satz9}]
  Setze in Prop. \ref{2.23} $N = \bigcap_{n > 0} I^n M$.\\
  Dann ist $N = I^{n_0 +1} M \cap N = I (I^n M \cap N) = I \cdot N$.
\end{Bew}

\begin{Bew}[Prop \ref{2.23}]
  F�hre Hilfsgr��en ein:\\
  $R' \defeqr \bigoplus_{n \geq 0} I^n$ ist graduierter Ring, $R_0' = R$ ist noethersch, $I$ ist endlich erzeugt $\Rightarrow R'$ ist noethersch (als endlich erzeugte $R$-Algebra), $M' \defeqr \bigoplus_{n \geq 0} I^n M$ ist graduierter, endlich erzeugter $R'$-Modul, $N' \defeqr \bigoplus_{n \geq 0} \underset{\defeql N'_n}{\underbrace{I^nM \cap N}}$ ist graduierter $R'$-Modul, Untermodul von $M'$, also auch endlich erzeugbar.\\
  $N'$ werde erzeugt von den homogenen Elementen $x_1, \dots, x_n$ mit $x_i \in N'_{n_i}$.
  F�r $n \geq n_0 \defeqr \max \{n_1, \dots, n_r\}$ ist dann $N'_{n+1} = \{\sum_{i=1}^r a_i x_i: \; a_i \in R'_{n+1-n_i} = I^{n+1-n_i}\}$. $I \cdot N'_n = I \cdot \{\sum_{i=1}^r a_i x_i: \; a_i \in R'_{n-n_i} = I^{n-n_i}\} = \{\sum_{i=1}^r \tilde{a_i} x_i: \; \tilde{a_i} \in I \cdot I^{n-n_i} = I^{n+1-n_i}\} = N'_{n+1}$.\\
  Mit Induktion folgt die Behauptung.
\end{Bew}

\begin{nnBsp}
  \begin{enumerate}
    \item[1.)] $R = \mathbb{Z}^2 = \mathbb{Z} \oplus \mathbb{Z}$ ist noethersch, aber nicht nullteilerfrei.\\
    Sei $I$ das von $e_1 = (1,0)$ erzeugte Ideal, $I^2 = (e_1^2)= (e_1) = I$ ($e_1$ ist "`idempotent"').
    $e \in R$ hei�t idempotent, wenn $e^2 = e$ ist. Dann ist $(e-1)e = 0$.\\
    Frage: was ist $\mathbb{Z}^2$ lokalisiert nach $I$?\\
    Antwort: $(\mathbb{Z} \oplus \mathbb{Z})_I = \mathbb{Q}$.
    \item[2.)] $R = \mathcal{C}^{\infty}(-1,1), \; I = \{f \in R: \; f(0)=0\}$. $R/I = \mathbb{C}$ (oder $\mathbb{R}$).
    $I$ ist Hauptideal, erzeugt von $f(x) = x$. $\bigcap I^n = ?$ z.B. $f(x) = e^{-\frac{1}{x^2}} \in \bigcap I^n$.\\
    $R$ ist nicht noethersch!
    \item[3.)] $R = k[X,Y], \; I = (X,Y), \; k$ algebraisch abgeschlosssen.\\
    $R' = R \oplus I \oplus I^2 \oplus \dots = \bigoplus_{n \geq 0} I^n = R[u,v]/(X v - Y u)$.\\
    Was sind die maximalen homogenen Ideale in $R'$, die nicht ganz $R'_+$ enthalten?\\
    Typ 1: maximale Ideale in $R, \not= (X,Y): (X-a, Y-b)$ mit $(a,b) \not=(0,0)$\\
    Typ 2: $(X,Y, \alpha u + \beta v), \; (\alpha, \beta) \not= (0,0)$
  \end{enumerate}
\end{nnBsp}
