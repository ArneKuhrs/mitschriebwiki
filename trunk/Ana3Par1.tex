\documentclass{article}
\newcounter{chapter}
\setcounter{chapter}{1}
\usepackage{ana}

\title{Satz von Arzel�-Ascoli}
\author{Joachim Breitner}
% Wer nennenswerte �nderungen macht, schreibt euch bei \author dazu

\begin{document}
\maketitle

In diesem Paragraphen sei $\emptyset \ne A \subseteq \MdR$ und $\F$ sei eine Familie (Menge) von Funktionen $f:A\to\MdR$.

\begin{definition}
$\F$ hei�t auf $A$ 
\begin{liste}
\item \begriff{punktweise beschr�nkt} $:\equizu$ $\forall x\in A \ \exists c = c(x) \ge 0:$
\[|f(x)| \le c \ \forall f\in \F \]
\item \begriff{gleichm��ig beschr�nkt} $:\equizu$ $\exists \gamma \ge 0: $
\[ |f(x)| \le \gamma  \ \forall x \in A \ \forall f\in \F \]
\item \begriff{gleichstetig} $:\equizu$ $\forall \ep > 0 \ \exists \delta = \delta(\ep) > 0$:
\[ |f(x) - f(y)|<\ep \ \forall x,y \in A \text{ mit } |x-y| < \delta \text{ und } \forall f\in \F \]
\end{liste}
\end{definition}

\begin{satz*}[Satz von Arzel�-Ascoli]
$A$ sei beschr�nkt und abgeschlossen, $\F$ sei punktweise beschr�nkt und gleichstetig auf $A$ und $(f_n)$ sei eine Folge in $\F$.

Dann enth�lt $(f_n)$ eine Teilfolge, welche auf $A$ gleichm��ig konvergiert.
\end{satz*}

\begin{beweis}
Analysis II, 2.3 $\folgt$ es existiert eine abz�hlbare Teilmenge $B=\{x_1,x_2,\ldots\} \subseteq A$ mit $\overline{B} = A$.

$(f_n(x_1))$ ist beschr�nkt \folgtnach{Analysis I} $(f_n)$ enth�lt eine Teilfolge $(f_{1,n})$ mit $(f_{1,n}(x_1))$ konvergent.\\
$(f_{1,n}(x_2))$ ist beschr�nkt \folgtnach{Analysis I} $(f_{1,n})$ enth�lt eine Teilfolge $(f_{2,n})$ mit $(f_{2,n}(x_2))$ konvergent.

Wir erhalten Funktionenfolgen
\begin{eqnarray*}
(f_{1,n}) &=& (f_{1,1},f_{1,2},f_{1,3},\ldots)  \\
(f_{2,n}) &=& (f_{2,1},f_{2,2},f_{2,3},\ldots)  \\
(f_{3,n}) &=& (f_{3,1},f_{3,2},f_{3,3},\ldots)   \\
& \vdots &
\end{eqnarray*}

$(f_{k+1,n})$ ist eine Teilfolge von $(f_{k,n})$ und $(f_{k,n}(x_k))_{n=1}^\infty$ konvergiert ($k\in\MdN$).

$g_j := f_{j,j}\ (j\in\MdN)$; $(g_j)$ ist eine Teilfolge von $(f_n)$.

$(g_k, g_{k+1}, g_{k+2}, \ldots)$ ist eine Teilfolge von $(f_{k,n})$ $\folgt$ $(g_j(x_k))_{j=1}^\infty$ ist konvergent $(k=1,2,\ldots)$.

Sei $\ep > 0$. Wir zeigen: 
\[(*)\quad \exists j_0\in\MdN:\ |g_j(x) - g_\nu(x)| < 3\ep \ \forall j,\nu \ge j_0\ \forall x\in A\] 
(woraus die gleichm��ige Konvergenz von $(g_j)$ folgt)

$\F$ gleichstetig \folgt 
\[ (i) \quad \exists \delta >0: |g_j(x)-g_j(y)| < \ep \ \forall x,y\in A\text{ und }|x-y| <\delta\ \forall j\in\MdN\]
$A\subseteq \bigcup_{x\in A} U_{\frac\delta2}(x)$. Analysis II, 2.3 \folgt $\exists y_1,\ldots,y_p \in A4$:
\[ (ii) \quad A \subseteq \bigcup_{j=1}^p U_{\frac\delta2}(y_j) \]
$\overline{B} = A \folgt \ \forall q\in\{1,\ldots,p\} \ \exists z_q \in B: z_q \in U_{\frac\delta2}(y_q)$
$(g_j)(z_q))_{j=1}^\infty$ ist konvergent f�r alle $q\in\{1,\ldots,p\}$ \folgt $\exists j_0\in\MdN$:
\[ (iii)\quad | g_j(z_q) - g_\nu(z_q) | < \ep \ \forall j,\nu \ge j_0 \ (q=1,\ldots,p)\]
Seien $j,\nu \ge j_0$ und $x\in A \folgtwegen{(ii)} \ \exists q\in\{1,\ldots,p\}: x\in U_{\frac\delta2}(y_q) \folgt |x-z_q| = |x-y_q+y_q-z_q| \le |x-y_q| + |y_q-z_q| < \frac{\delta}2 + \frac\delta2 = \delta \folgtwegen{(i)} |g_j(x) - g_j(z_q)| < \ep$, $|g_\nu (x) - g_\nu (z_q)|<\ep$ $(iv)$

\begin{eqnarray*}
\folgt |g_j(x)-g_\nu(x)| &=& |g_j(x)-g_j(z_q) + g_j(z_q) - g_\nu(z_q) + g_\nu(z_q)-g_\nu(x)| \\
&\le& \underbrace{|g_j(x) - g_j(z_q)|}_{< \ep\ (iv)} + \underbrace{|g_j(z_q) - g_\nu(z_q)|}_{< \ep\ (iii)} + \underbrace{|g_\nu(z_q) - g_\nu(x)|}_{< \ep\ (iv)}\\
&<& 3 \ep \folgt (*)
\end{eqnarray*}
\
 
\end{beweis}


\end{document}
