\documentclass{article}
\newcounter{chapter}
\setcounter{chapter}{14}
\usepackage{ana}
\def\gdw{\equizu}
\def\Arg{\text{Arg}}
\def\MdD{\mathbb{D}}
\def\Log{\text{Log}}
\def\Tr{\text{Tr}}
\def\abnC{\ensuremath{[a,b]\to\MdC}}
\def\wegint{\ensuremath{\int\limits_\gamma}}
\def\iint{\ensuremath{\int\limits}}
\def\ie{\rm i}

\title{Laurententwicklung}
\author{Franziranzi, Christian Schulz} % Wer nennenswerte �nderungen macht, schreibt euch bei \author dazu

\begin{document}
\maketitle

F�r $z_0 \in \MdK$: $U_\infty(z_0) := \MdC$, $\dot{U}_\infty(z_0) = \MdC \backslash \{z_0\}$, 
$\frac{1}{0} := \infty$. Erinnerung: Satz 9.5: Sei $\gamma$ ein st�ckweise glatter Weg in $\MdC$, 
$\varphi \in \MdC \backslash \text{Tr}(\gamma))$ und $g(z) = \frac{1}{2 \pi i}\int_{\gamma}\frac{\varphi(w)}{w-z}dw$ 
$(z \in  \MdC \backslash \text{Tr}(\gamma) )$. Dann: \\
$g \in H( \MdC \backslash \text{Tr}(\gamma))$.

\begin{satz}
  Seien $ 0 \le r < R \le \infty$; $A := \{ z \in \MdC : r < |z| < R \}$ und $f \in H(A)$. F�r $s \in (r,R)$ sei 
  $\gamma_s(t) := se^{it}$, $t \in [0,2 \pi ]$ und $J(s) := \int_{\gamma_s} f(z) dz$. \\
  Dann ist $J$ konstant auf $(r, R)$.
\end{satz}

\begin{beweis} 
  $g(z) := z f(z)$  $(z \in A)$. Dann: 
  $f(z) = \frac{g(z)}{z}$ und $g \in H(A)$. \\
  $J(s) = \int_{\gamma_s} \frac{g(z)}{z}(z) dz = \int_0^{2 \pi} \frac{g(se^{it})}{se^{it}}sie^{it} dt =  
  \int_0^{2 \pi} ig(se^{it}) dt$\\
  $J$ ist auf $(r,R)$ db und $J'(s) = \int_0^{2 \pi} i \frac{d}{ds}g(se^{it}) dt = \int_0^{2 \pi} i g'(se^{it})e^{it}  dt = 
  \frac{1}{s} \int_0^{2 \pi} g'(se^{it})sie^{it}  dt =  \frac{1}{s} \int_0^{2 \pi} g'(\gamma_s(t))\gamma'_s(t) dt =
  \frac{1}{s} \int_{\gamma_s} g'(z) dz \stackrel{8.5}{=} 0 \Rightarrow J(s)$ konstant.
\end{beweis}

\begin{satz} [Laurententwicklung]
  Sei $A$ wie in 14.1 und $f \in H(A)$. Dann existieren eindeutig bestimmte Funktionen $g \in H(U_R(0))$ und $h \in H(U_{\frac{1}{r}}(0))$ mit: \\
  (*) $f(z) = g(z) + h(\frac{1}{z})\ \forall z \in A$ und $h(0) = 0$ \\
  (*) hei�t die Laurententzerlegung von $f$, $g$ hei�t Nebenteil von $f$ und die Funktion $ z \rightarrow h(\frac{1}{z})$ ist der Hauptteil von $f$. 
\end{satz}

\begin{beispiel}
  $f(z) = e^{(\frac{1}{z}}$ $A = \MdC \backslash \{0\}$ $(r = 0, R = \infty)$. Es gilt: \\
  $f(z) = 1 + (  e^{(\frac{1}{z}} - 1 )$, also $g(z) = 1$, $h(z) = e^z -1 $
\end{beispiel}

\begin{beweis}
  \begin{itemize}
    \item[ 1. ] Eindeutigkeit: es sei $g, g_1 \in H(U_R(0))$, $h, h_1 \in H(U_{\frac{1}{r}}(0))$;
	$h_1(0) = 0 = h(0)$ und $g(z) + h(\frac{1}{z}) = f(z) = g_1(z) + H_1(\frac{1}{z})\ \forall z \in A$.\\
	$G := g - g_1 \in H(U_R(0))$, $H:= h_1 - h \in  H(U_{\frac{1}{r}}(0))$ \\
	$\Rightarrow G(z) = H(\frac{1}{z}) \ \forall z \in A$
	Dann ist $F:\MdC \rightarrow \MdC$, definiert durch \\
	$F(z) = $ uf $\MdC$ wohldefiniert. $F \in H(\MdC)$. \\
	Sei $(z_n)$ eine Folge in $\MdC$ und $|z_n| \rightarrow \infty$. Dann $(\frac{1}{z}) \rightarrow 0$ und
	$z_n > r \ \forall n \ge n_0$. $F(z_n) = H(\frac{1}{z_n}) = h_1(\frac{1}{z_n}) - h(\frac{1}{z_n}) \rightarrow h_1(0) - h(0) = 0$\\
	Also $F(z) \rightarrow 0 \ (|z| \rightarrow \infty)$ Somit: \\
	$\exists \varrho > 0: |F(z)| \le 1 \ \forall z \in \MdC \backslash U_{\varrho}(0)$. $F$ stetig auf $ \overline{U_{\varrho}(0)} \rightarrow$
	$F$ ist auf $\MdC$ beschr�nkt. 10.2 $\Rightarrow$ $F$ ist auf $\MdC$ konstant; wegen 
	$F(z) \rightarrow 0 \ (z \rightarrow \infty)$ folgt: $F$ %%% hier gehts noch weiter! 
    \item[ 2. ] Existenz: fehlt hier nicht was ? Doch! 
  \end{itemize}
\end{beweis}

\begin{definition}
Sei $(a_n)_{n \in \MdZ}$ eine Folge in $\MdC, z_0 \in \MdC$ und $A \subseteq
\MdC$. Eine Reihe der Form $\sum\limits_{n=-\infty}^{\infty} a_n(z-z_0)^n$ hei�t eine
\begriff{Laurentreihe}. \\
Diese Reihe hei�t in $z \in \MdC$ (absolut) konvergent $: \gdw
\sum\limits_{n=0}^{\infty} a_n(z-z_0)^n$  und $\sum\limits_{n=1}^{\infty}
a_{-n}(z-z_0)^{-n}$ konvergieren absolut. \\ In diesem Fall: $\sum\limits_{n=-\infty}^{\infty}
a_n(z-z_0)^n := \sum\limits_{n=0}^{\infty} a_n(z-z_0)^n + \sum\limits_{n=1}^{\infty}
a_{-n}(z-z_0)^{-n}$. Die Laurentreihe hei�t auf $A$ (lokal) gleichm��ig
konvergent $:\gdw$ $\sum\limits_{n=0}^{\infty} a_n(z-z_0)^n$ und $\sum\limits_{n=1}^{\infty}
a_{-n}(z-z_0)^{-n}$ konvergieren auf A (lokal) gleichm��ig.
\end{definition}
\begin{satz}
Sei $0 \leq r < R \leq \infty$, $A := \{z \in \MdC: r < |z-z_0| < R\}$ und $f
\in H(D).$ \\Dann hat $f$ auf $A$ die \begriff{Laurententwicklung} 
$f(z) = \sum\limits_{n=-\infty}^{\infty} a_n(z-z_0)^n$. Die Laurentreihe
konvergiert auf $A$ absolut und lokal gleichm��ig. Die Koeffizienten $a_n$ $(n
\in \MdZ )$ sind eindeutig bestimmt. Ist $r < \rho < R$ und $\gamma(t) := z_0 +
\rho e^{it}$ $t \in [0,2\pi]$, so gilt: \\ \\
\centerline{$a_n = \frac{1}{2\pi i} \int\limits_{\gamma} 
\frac{f(w)}{(w-z_0)^{n+1}} dw$ $\forall n \in \MdZ$}. \\
$\sum\limits_{n=0}^{\infty} a_n(z-z_0)^n$ hei�t \begriff{Nebenteil} von $f$, \\
$\sum\limits_{n=1}^{\infty} a_{-n}(z-z_0)^{-n} = \frac{a_{-1}}{z-z_0}+
\frac{a_{-2}}{(z-z_0)^2} + \ldots$ hei�t der \begriff{Hauptteil} von f.
\end{satz}
\begin{beweis}
O.B.d.A: $z_0 = 0$. 14.2 $\Rightarrow$ $\exists g \in H(U_R(0)), \exists h \in
H(U_{\frac{1}{r}}(0)): f(z) = g(z) + h(\frac{1}{z})$ $\forall z \in A$ und $h(0)
= 0.$ 10.4 $\Rightarrow$ $g(z) = \sum\limits_{n=0}^{\infty} a_n z^n$
$\forall z \in U_R(0)$ und $h(z) = \sum\limits_{n=0}^{\infty} b_n z^n$
$\forall z \in U_{\frac{1}{r}}$. Setze $a_{-n} := b_n$ f�r $n \geq 1$. Dann: 
$f(z) = \sum\limits_{n=-\infty}^{\infty} a_n z^n$. 5.4 $\Rightarrow$ die
Laurentreihe konvergiert auf $A$ absolut und lokal gleichm��ig. \\
14.2 $\Rightarrow$ $g$ und $h$ sind eindeutig bestimmt \\ 5.4 $\Rightarrow$
$a_n$ eindeutig bestimmt f�r $n \in \MdZ$. Sei $n \in \MdZ$; $\gamma(t) := \rho
e^{it}$ $(t \in [0, 2 \pi])$ $r<\rho <R$. Sei $w \in Tr(\gamma)$: \\
\centerline{$\frac{f(w)}{w^{n+1}} = \sum\limits_{\nu=-\infty}^{\infty} a_\nu w^{\nu - n -1}$}.
Die letzte Reihe konvergiert auf Tr$(\gamma)$ gleichm��ig. \\
8.4 $\Rightarrow$ $\int\limits_{\gamma} \frac{f(w)}{w^{n+1}}= \sum\limits_{\nu=-\infty}^{\infty} a_\nu
\underbrace{\int\limits_{\gamma} w^{\nu -n -1}}_{
= \begin{cases} 0 &, \nu \neq n \\
  				2 \pi i & , \nu = n
  \end{cases}}$
\end{beweis}
\begin{satz}
$D \subseteq \MdC$ sei offen, $z_0 \in D$, $\dot{D} := D \backslash \{ z_0 \}$
und $f \in H(\dot{D})$ ($z_0$ ist also eine isolierte Singularit�t). Sei $R > 0$
so, da� $U_R(z_0) \subseteq D$. $f$ hat also, nach 14.3, auf $\dot{U}_R(z_0)$ die
Laurententwicklung \\
\centerline{$f(z) = \sum\limits_{n=-\infty}^{\infty} a_n(z-z_0)^n$ $(z \in
\dot{U}_R(z_0))$} 
\begin{liste}
\item $f$ hat in $z_0$ eine hebbare Singularit�t $\gdw$ $a_{-n} = 0$ $\forall n
\in \MdN$
\item $f$ hat $z_0$ einen Pol der Ordnung $m \in \MdN$ $\gdw$ $a_{-m} \neq 0$, 
$a_{-n} = 0$ $\forall n > m$
\item $f$ hat in $z_0$ eine wesentliche Singularit�t $\gdw$ $a_{-n} \neq 0$ f�r
unendlich viele $n \in \MdN$
\end{liste}
\end{satz}

\begin{definition}
Vorraussetzung wie in 14.4. Res$(f, z_0) := a_{-1}$ hei�t das \begriff{Residuum}
von $f$ in $z_0$. \\
Ist $0 < \rho < R$ und $\gamma(t) = z_0 + \rho e ^{it}$ $( t \in [0, 2 \pi])$,
so folgt aus 14.3: \\ \centerline{Res$(f, z_0) = \frac{1}{2\pi i} \int\limits_{\gamma} f(z)
dz$}
\end{definition}

\begin{beweis}
\begin{liste}
\item Klar.
\item O.B.d.A: $z_0 = 0$. \\
`` $\Rightarrow$ ``: 13.2 $\Rightarrow$ $\exists g \in H(D): f(z) =
\frac{g(z)}{z^m}$ $\forall z \in \dot{D}$ und $g(w) \neq 0$  \\
10.4 $\Rightarrow$ $g(z) = c_0 + c_1z + c_2 z^2 + \ldots $ $\forall z \in
U_R(0)$ $\Rightarrow$ $f(z) = \frac{c_0}{z^m}  + \frac{c_1}{z^{m-1}} + \ldots +
\frac{c_{m-1}}{z} + \sum\limits_{n=m}^{\infty} c_n z^{n-m}$ $\forall z \in
\dot{U}_R(0)$. Eindeutigkeit der Laurententwicklung $\Rightarrow$ $c_0 =
a_{-m}$, also $a_{-m} = g(0) \neq 0$; weiter: $a_{-n} = 0$ $\forall n > m$
``$\Leftarrow$'': $f(z) = \sum\limits_{n=0}^{\infty} a_n z^n + \frac{a_{-1}}{z}
+ \ldots + \frac{a_{-m}}{z^m}$ $\forall z \in \dot{U}_R(0)$ \\
$\Rightarrow$ $z^m f(z) = \underbrace{a_{-m} + \ldots + a_{-1} z ^{m-1} +
\sum\limits_{n=0}^{\infty} a_n z^{n+m}}_{ =: g(z)}$ $\forall z \in \dot{U}_R(0)$
\\
Es ist $g \in H(U_R(0))$, $g(0) = a_{-m} \neq 0$ und $f(z) = \frac{g(z)}{z^m}$
$\forall z \in \dot{U}_R(0)$. 13.2 $\Rightarrow$ $f$ hat einen Pol der Ordnung $m$.
\item folgt aus (1) und (2).
\end{liste}
\end{beweis}

\begin{beispiele}
\begin{liste}
\item $f(z) = \frac{1}{z-1}$ Laurententwicklung in $\MdC \backslash \{1\}: f(z)
= \frac{1}{z-1}$, Res$(f,1) = 1$
\item  $f(z) = \frac{1}{z-1}$ Laurententwicklung in $\{ z \in \MdC : 1 < |z| <
\infty \}$.\\ F�r $|z| >
1: $ $\frac{1}{z-1} = \frac{1}{z} \frac{1}{1- \frac{1}{z}} = 
\frac{1}{z} \sum\limits_{n=0}^{\infty} \frac{1}{z^n}$
\item $f(z) = \frac{cos(z)}{z^3}.$ Laurententwicklung in $\MdC \backslash
\{0\}$. $f(z) = \frac{1}{z^3}(1 - \frac{z^2}{2!} + \frac{z^4}{4!} -+ \ldots)
= \underbrace{\frac{1}{z^3} - \frac{1}{2z}}_{\text{Hauptteil}} + \underbrace{\frac{z}{4!}-
\frac{z^3}{6!} +- \ldots}_{\text{Nebenteil}}$, Res$(f,0) = -\frac{1}{2}$
\end{liste}
\end{beispiele}
\end{document}
