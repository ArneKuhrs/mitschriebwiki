\documentclass{article}
\newcounter{chapter}
\setcounter{chapter}{9}
\usepackage{ana}
\usepackage{mathrsfs}

\title{Einige Typen von Differentialgleichungen 1. Ordnung}
\author{Wenzel Jakob}
% Wer nennenswerte �nderungen macht, schreibt sich bei \author dazu

\begin{document}
\maketitle

\paragraph{(I): }$y'=f(\frac{y}{x})$. Setze $u:=\frac{y}{x}$. Dies f"uhrt auf eine Differentialgleichung mit getrennten Ver"anderlichen f"ur u.

\begin{beispiel}
$$\text{AWP: }\begin{cases}y'=\frac{y}{x}-\frac{x^2}{y^2}\\ y(1)=1\end{cases}$$
\begin{eqnarray*}
u:=\frac{y}{x}&\folgt&y=xu\\
y'=u+xu'&\folgt&u+xu'=u-\frac{1}{u^2}\\
&\folgt& u'=-\frac{1}{xu^2}\\
&\folgt& \frac{du}{\ud x}=-\frac{1}{xu^2}\\
&\folgt& u^2\ud u=-\frac{1}{x}\ud x\\
&\folgt& \frac{1}{3}u^3=-\log x+c\\
&\folgt& u^3=-3\log x+3c\ (c\in\MdR)\\
u(1)=\frac{y(1)}{1}=1 & \folgt & 1=u^3(1)=3c\\
&\folgt &c=\frac{1}{3}\\
u^3=1-3\log x&\folgt& y(x)=x\sqrt[3]{1-3\log x}\text{ auf }(0, \sqrt[3]{e})\text{ (L"osung des AWPs)}
\end{eqnarray*}
\end{beispiel}

\paragraph{(II) Bernoullische Differentialgleichung: }
$y'+p(x)y+q(x)y^\alpha=0$, wobei $p$ und $q$ stetig sind und $0\ne\alpha\ne 1$.
Dividiere durch $y^\alpha$ und setze $u:=y^{1-\alpha}$. Dies f"uhrt auf eine
lineare Differentialgleichung f"ur $u$.

\begin{beispiel}
$(*)\ y'-xy+3xy^2=0\ (\alpha=2)$. Dann: $\frac{y'}{y^2}-\frac{x}{y}+3x=0;\ 
u:=\frac{1}{y}\folgt u'=-\frac{y'}{y^2}\folgt -u'-xu+3x=0\folgt u'=-xu+3x$. Allgemeine L"osung hiervon:
$u(x)=ce^{-\frac{1}{2}x^2}+3\ (c\in\MdR)$. Allgemeine L"osung von $(*)$:
$y(x)=\frac{1}{ce^{-\frac{1}{2}x^3}+3}\ (c\in\MdR)$
\end{beispiel}

\paragraph{(III) Riccatische Differentialgleichung: }
$(*)\ y'+g(x)y+h(x)y^2=k(x)$, wobei g, h, k stetig sind.
Sei $y_1$ eine bekannte L"osung von $(*)$; setze $z:=\frac{1}{y-y_1}$.
Nachrechnen: $(**)\ z'=(g(x)+2y_1(x)h(x))z+h(x)$ (lin. Dgl f"ur $z$). Die
%                                     ^ hier muss ein x hin, hab es nachgerechnet
allgemeine L"osung von $(*)$ lautet: $y(x)=y_1(x)+\frac{1}{z(x)}$ wobei
$z$ die allgemeinen L"osungen von $(**)$ durchl"auft.

\end{document}
