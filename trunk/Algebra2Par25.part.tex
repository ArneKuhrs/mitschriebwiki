\section{Invarianten endlicher Gruppen}

\begin{DefBem}
\label{2.19}
  Sei $k$ ein K�rper, $n \ge 1, \; k[X] \defeqr k[X_1, \dots, X_n]$.\\
  Sei $G \subseteq \mbox{Aut}(k[X])$ eine Untergruppe der
  $k$-Algebra-Automorphismen.
  \begin{enumerate}
    \item $k[X]^G \defeqr \{f \in k[X]: \sigma (f) = f \mbox{ f�r alle } \sigma
          \in G\}$ hei�t \emp{Invariantenring}\index{Invariantenring} von $k[X]$
          bzgl. $G$.
    \item $k[X]^G$ ist $k$-Algebra.
    \item $G$ hei�t \emp{linear}, wenn jedes $\sigma \in G$ graderhaltend ist. Dann
          ist $\sigma|_{k[X]_1}$ ein $k$-VR-Automorphismus und $\sigma \mapsto
          \sigma|_{k[X]_1}$ ist ein Gruppenhomomorphismus $G \to GL_n(k)$.
  \end{enumerate}
\end{DefBem}

\begin{nnBsp}
  \begin{enumerate}
    \item[1.)] $n=2, G = \{id, \sigma\}$ mit $\sigma(X) = Y, \; \sigma(Y) = X
               \Rightarrow k[X,Y]^G$ wird erzeugt von $X+Y$ und $X \cdot Y$.\\
               $X^k+Y^k - (X + Y)^k = -X^{k-1} Y - ... - X Y^{k-1} = -X Y
               (X^{k-2} + Y^{k-2}) - ...$
    \item[2.)] $n=2, G = \{ id, \varphi\}$ mit $\varphi(X) = -X, \; \varphi(Y) =
               -Y$ (char $k \not= 2$).\\
               $k[X,Y]^G$ wird erzeugt von $X^2, Y^2, XY$.
  \end{enumerate}
\end{nnBsp}

\begin{Satz}
  Seien $k, G, k[X]$ wie in Def. \ref{2.19}, $G$ linear und endlich.
  \begin{enumerate}
    \item (Hilbert) $k[X]^G$ ist endlich erzeugbare $k$-Algebra
    \item (E. Noether) Ist $m = |G|$, so wird $k[X]^G$ von Elementen vom Grad
          $\le m$ erzeugt.
  \end{enumerate}
\end{Satz}

\begin{Bew}
  \begin{enumerate}
    \item 
      Sei $S \defeqr k[X]^G$ (graduierte Unteralgebra von $k[X]$).\\
      $S_+ = \bigoplus_{i > 0} S_i, \; I \defeqr S_+ k[X]$ (Ideal in $k[x]$) 
      $\Rightarrow I$ ist endlich erzeugt.\\
      Seien $f_1, \dots, f_r \in S_+$ homogene Erzeuger von $I$, $S' \defeqr k[f_1,
      \dots, f_r] \subseteq S$\\
      \textbf{Beh.:} $S=S'$\\
      \textbf{Bew.:} Sei also $f = \sum_{i=0}^n \tilde{f}_i \in S, \; \tilde{f}_i
      \in S_i$. Zeige mit Induktion: $S_d \subset S'$ f�r jedes $d \ge 0$.\\
      $d = 0$: $S_0 = k = S'_0$\\
      $d = 1$: Sei $f \in S_d \Rightarrow f \in S_+ \subseteq I \Rightarrow f =
      \sum_{i=1}^r g_i f_i$ mit $g_i \in k[X]_{d- d_i}, \; d_i = \mbox{deg }f_i
      \Rightarrow \mbox{deg }g_i < d$\\
      ''Mittelung'': Die Abbildung $\varphi: k[X] \to S, \; f \mapsto
      \frac{1}{|G|} \sum_{\sigma \in G} \sigma(f)$ ist linear, graderhaltend,
      Projektion.\\
      $\Rightarrow f = \varphi(f) = \sum_{i=1}^r \varphi(g_i) f_i$ mit
      $\varphi(g_i) \in S,\mbox{ deg }\varphi(g_i)<d$\\
      Also nach Induktionsvoraussetzung $\varphi(g_i) \in S' \Rightarrow f \in
      S'$
  \end{enumerate}
\end{Bew}

\begin{nnBsp}
  $S_n$ operiert auf $k[X_1, \dots, X_n]$ durch $\varphi(X_i) \defeqr X_{\varphi(i)}$. $k[X_1, \dots, X_n]^{S_n}$ sind die symmetrischen Polynome.\\
  \textbf{Beh.1:} $k[X_1, \dots, X_n]^{S_n}$ wird (als $k$-Algebra) erzeugt von den ``elementarsymmetrischen`` Polynomen:\\
  $S_1 \defeqr X_1 + \dots + X_n$\\
  $S_2 \defeqr X_1X_2 + X_1X_3 + \dots + X_{n-1}X_n$\\
  $\vdots$\\
  $S_n \defeqr X_1 \cdot \ldots \cdot X_n$\\
  \textbf{Beh.2:} $k[X_1, \dots, X_n]^{S_n}$ wird erzeugt von den Potenzsummen\\
  $f_1 = s_1 = \sum X_i$\\
  $f_k = \sum_{i=1}^n X_i^k, \; k = 1, \dots, n$
\end{nnBsp}

\begin{nnBem}
  $\varphi: k[X] \to k[X]^G, \; f \mapsto \frac{1}{|G|}\sum_{\sigma \in G} \sigma(f)$ ist $k$-lineare graderhaltende Projektion.
\end{nnBem}

\begin{Bew}
  \begin{enumerate}
    \item[(b)] Sei $\tilde{S}$ die von den $\varphi(X^{\nu}), |\nu| \le |G|$
    erzeugte Unteralgebra von $k[X]^G$.
    Dabei sei f�r $\nu = ( \nu_1, \dots, \nu_n) \in \mathbb{N}^n: \; X^{\nu}
    \defeqr X_1^{\nu_1} \cdot ... \cdot X_n^{\nu_n}$ und $|v| \defeqr \sum
    \nu_i$.\\
    Zu zeigen: $\varphi(x^{\nu}) \in \tilde{S}$ f�r alle $\nu \in
    \mathbb{N}^n$.\\
    Hilfsgr��e: F�r $d > 0$ sei $F_d \defeqr \sum_{\sigma \in
    G}(\underset{\defeql Z_{\sigma}}{\underbrace{\sum_{i=1}^n
    \sigma(X_i)Y_i}})^d \in k[X_1, \dots, X_n, Y_1, \dots, Y_n] = \sum_{\sigma
    \in G} Z_{\sigma}^d \overset{G=\{\sigma_1, \dots, \sigma_n \},\\ |G| = m}{=}
    \sum_{i=1}^m Z_j$ mit $Z_j \defeqr Z_{\sigma_j}$.\\
    Umformungen:
    \begin{enumerate}
      \item[(1)] $F_d = \sum_{\sigma \in G} \sum_{|\nu|=d} \gamma_{\nu}
      \sigma(X^{\nu})>^{\nu}$ (mit $\gamma_{\nu} = \frac{d!}{\nu_1! \cdot ...
      \cdot \nu_n! }$) $= \sum_{|\nu| = d} \gamma_{\nu}(\sum_{\sigma \in G}
      \sigma(X^{\nu})Y^{\nu}) = \sum_{|\nu| = d} \gamma_{\nu} m \varphi(X^{\nu})
      Y^{\nu}$.\\
      Nach Beh.2 gibt es $a_\mu \in k, \; \mu \in \mathbb{N}^n$ mit $\sum_{i=1}^m
      i \mu_i = d$
      \item[(2)] $F_d = \sum_{\mu \in \mathbb{N}^m} a_{\mu} F_1^{\mu_1} \cdot \ldots \cdot F_m^{\mu_m} \overset{(1)}{=} \sum_{\mu \in \mathbb{N}} a_{\mu} \prod_{j=1}^m (\sum_{|\nu| = j} \gamma_{\nu} m \varphi (X^{\nu}) Y ^{\nu}{\nu_j} \overset{\mbox{sortieren nach Y}}{=} \sum_{\lambda \in \mathbb{N}^m} P_{\lambda}(X)Y^{\lambda}$ mit $P_{\lambda} \in \tilde{S}$.\\
      Koeffizientenvergleich zwischen (1) und (2) ergibt:\\
      $P_{\lambda}= \begin{cases} 0 &, |\lambda| \not= d\\ \gamma_{\lambda} m \varphi(X^{\lambda})&, |\lambda|=d \end{cases}$\\
      $\Rightarrow \varphi(X^{\lambda}) \in \tilde{S}$ f�r alle $\lambda \in \mathbb{N}^m$
    \end{enumerate}
  \end{enumerate}
\end{Bew}

\begin{nnBsp}
  $n=2, \; G=\langle \sigma \rangle, \; \sigma(X) = Y, \; \sigma(Y) = -X \Rightarrow G \cong \mathbb{Z}/4\mathbb{Z}$\\
% Hier sollte dann besser eine Tabelle stehen
  $\begin{array}{ccccc}
    id & \sigma & \sigma^2 & \sigma^3 & \sum_{\sigma \in G} \sigma\\
    X & Y & -X & -Y & 0\\
    Y & -X & -Y & X & 0\\
    X^2 & Y^2 & X^2 & Y^2 & 2(X^2+Y^2)\\
    Y^2 & X^2 & Y^2 & X^2 & 2(X^2+Y^2)\\
    XY & -YX & XY & -YX & 0\\
    X^3 & Y^3 & -X^3 & -Y^3 & 0\\
    Y^3 & -X^3 & -Y^3 & X^3 & 0\\
    X^2Y & -XY^2 & -X^2Y & X Y^2 & 0\\
    X Y^2 & X^2Y & -XY^2 & -X^2Y & 0\\
    X^4 & Y^4 & X^4 & Y^4 & 2(X^4+Y^4)\\
    XY^3 & -X^3Y & XY^3 & -X^3Y & 2XY(Y^2-X^2)\\
    X^2Y^2 & X^2Y^2& X^2Y^2& X^2Y^2 & 4(X^2Y^2)\\
  \end{array}$\\
  $\Rightarrow k[X,Y]^G$ wird erzeugt von $I_1 = X^2+Y^2, \; I_2 = X^2Y^2, \; I_3 = XY(X^2-Y^2)$ (und $I_4 = X^4 + Y^4 = I_1^2-2I_2$). Zwischen $I_1,I_2,I_3$ besteht die Gleichung $I_3^2 = I_2(X^4+Y^4-2X^2Y^2)=I_1(I_1^2-4I_2)$
\end{nnBsp}
