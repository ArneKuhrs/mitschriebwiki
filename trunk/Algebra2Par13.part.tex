\section{Flache Moduln}

\begin{Bem}
  F�r jeden $R$-Modul $M$ ist die Zuordnung $M \mapsto M \otimes_R N$ ein Funktor
  \[\otimes_R N: R\mbox{-Mod} \to R\mbox{-Mod}\]
\end{Bem}

\begin{Bew} 
  Ist $\varphi: M \to M'$ $R$-linear, so sezte $\varphi_N: M \otimes_R N \to M'
  \otimes N, x \otimes y \mapsto \varphi(x) \otimes y, \displaystyle
  \sum_{i=0}^na_i(x_i \otimes y_i) \mapsto \sum a_i(\varphi(x_i) \otimes y_i)$
\end{Bew}

\begin{Prop}
\label{1.12}
  Der Funktor $\otimes_R N$ ist rechtsexakt, d.h. ist $0 \to M'
  \overset{\varphi}{\to} M \overset{\Psi}{\to} M'' \to 0$ exakt, so ist $ M'
  \otimes N \overset{\varphi_N}{\to} M \otimes N \overset{\Psi_N}{\to} M'' \otimes N
  \to 0$ exakt.
\end{Prop}

\begin{nnBsp} 
  $R = \mathbb{Z}, N = \mathbb{Z}/2 \mathbb{Z}$\\
  $0 \to \mathbb{Z} \overset{ \cdot 2}{\to} \mathbb{Z} \to \mathbb{Z}/2
  \mathbb{Z} \to 0$\\
  $\varphi_N: \mathbb{Z}/2 \mathbb{Z} \to \mathbb{Z} / 2 \mathbb{Z}$ ($\cong
  \mathbb{Z} \otimes_{\mathbb{Z}} \mathbb{Z} / 2 \mathbb{Z}$ nach 1.9a)
  $\Rightarrow \varphi_N$ ist nicht surjektiv
\end{nnBsp}

\begin{Bew} 
  \textbf{1. Schritt:} $\mbox{Bild}(\varphi_N) \subseteq \mbox{Kern}(\Psi_N)$,
  denn: $\Psi_N(\varphi_N(x \otimes y)) = \Psi(\varphi(x) \otimes y) =
  \underset{=0}{\underbrace{\Psi(\varphi}}(x)) \otimes y = 0$. Homomorphiesatz
  liefert ein $\bar{\Psi}: M \otimes N/\mbox{Bild}(\varphi_N) \to M'' \otimes
  N$\\
  \textbf{2. Schritt:} $\bar{\Psi}$ ist Isomorphismus.\\
  Dann ist $\bar{\Psi}$ und damit $\Psi_N$ surjektiv und $\mbox{Kern}(\Psi_N) =
  \mbox{Bild}(\varphi_N)$.\\
  Konstruiere Umkehrabbildung $\sigma: M'' \otimes N \to \bar{M} \defeqr M
  \otimes N/\mbox{Bild}(\varphi_N)$. W�hle zu jedem $x'' \in M''$ ein Urbild
  $\chi(x'') \in \Psi^{-1}(x'') \subset M$.
  Definere $\tilde{\sigma}: M'' \times N \to \bar{M}$ durch $(x'', y) \mapsto
  \chi(x'') \otimes y$\\
  $\tilde{\sigma}$ wohldefiniert:
  Sind $x_1,x_2 \in M$ mit $\Psi(x_1) = \Psi(x_2) = x''$, so ist $\underset{= \varphi(x')
  }{\underbrace{x_1 - x_2}} \in \mbox{Bild}(\varphi) \Rightarrow \overline{x_1
  \otimes y} - \overline{x_2 \otimes y} = \underset{\in \mbox{Bild}(\varphi_N)
  }{\underbrace{\overline{\varphi(x') \otimes y}}} = 0$\\
  Rest klar!!
\end{Bew}

% ---

\begin{DefProp}
\label{1.13}
  Sei $N$ ein $R$-Modul.
  \begin{enumerate}
    \item $N$ hei\ss t \emp{flach}\index{R-Modul!flacher}, wenn, wenn der Funktor $\otimes_R N$ exakt ist,
    d.h. f\"ur jede kurze exakte Sequenz von $R$-Moduln 
    $0\to M'\to M\to M''\to 0$
    auch $0\to M'\otimes_R N\to M\otimes_R N\to M''\otimes_R N\to 0$ exakt ist.
    \item $N$ ist genau dann flach, wenn f\"ur jeden $R$-Modul $M$ und jeden Untermodul $M'$ von $M$
    die Abbildung $i:M'\otimes_R N\to M\otimes_R N$ injektiv ist.
    \item Jeder projektive $R$-Modul ist flach.
    \item Ist $R=K$ ein K\"orper, so ist jeder $R$-Modul flach.
    \item F\"ur jedes multiplikative Modoid $S$ ist $R_S$ flacher $R$-Modul.
  \end{enumerate}
\end{DefProp}

\begin{Bew}
\begin{enumerate}
\item[(b)] folgt aus \ref{1.12}
\item[(e)] Sei $M$ ein $R$-Modul, $M'\subseteq R$ Untermodul.
Nach \"U2A4 ist $M\otimes_R R_S \cong M_S$.\\
Zu zeigen: Die Abbildung $M'_S\to M_S, \frac{a}{s}\to \frac{a}{s}$ ist injektiv. \\
Sei also $a\in M'$ und $\frac{a}{s}=0$ in $M_S$, d.h. in $M$ gilt: $t\cdot a=0$ f\"ur ein $t\in S$.
$\Rightarrow t\cdot a = 0$ in $M'\Rightarrow \frac{a}{s}=0$ in $M'_S$.
\item[(d)] folgt aus (c), weil jeder $K$-Modul frei ist, also projektiv.
\item[(c)] Sei $N$ projektiv. Nach Prop. \ref{1.6} gibt es einen $R$-Modul
$N'$, sodass $N \oplus N'\defeql F$ frei ist.\\ \textbf{Beh. 1}: $F$ ist flach.\\
Dann sei $M$ $R$-Modul, $M'\subseteq M$ Untermodul; dann ist $F\otimes_R M'\to F\otimes_R M$ injektiv.\\
\textbf{Beh. 2}: Tensorprodukt vertauscht mit direkter Summe.\\
$\begin{array}{lccccc}
\textrm{Dann ist }  &   M'\otimes_R F & \cong M'\otimes_R(N\oplus N')   &\cong (M'\otimes_R N)&\oplus &(M'\otimes_R N') \\
                    &   \downarrow &                                 &\downarrow               &   &\downarrow\\
  \textrm{und}       &   M\otimes_R R &                               &\cong (M\otimes_R N)&\oplus &(M\otimes_R N')\\
\end{array}$
Die Abbildung $M'\to M\otimes F$ bildet $M'\otimes N$ auf $M\otimes N$ ab, 
$M'\otimes N\to M\otimes N$ ist also als Einschr\"ankung einer injektiven Abbildung selbst injektiv.\\
\textbf{Bew. 1}: Sei $\{e_i:i\in I\}$ Basis  von $F$, also $F=\bigoplus_{i\in I} R e_i\cong \bigoplus_{i\in I} R$.
Wegen Beh. 2 ist $M\otimes_R F\cong M\otimes \bigoplus_{i\in I}R\cong \bigoplus_{i\in I}(M\otimes_R R)=\bigoplus_{i\in I}M$.
Genauso: $M' \otimes F\cong \bigoplus_{i\in I}M'$.\\
Die Abbildung $M'\otimes F\to M\otimes F$ ist in jeder Komponente die Einbettung $M'\hookrightarrow M$, also injektiv.\\
\textbf{Bew. 2}: Sei $M=\bigoplus_{i\in I} M_i$, zu zeigen: $M\otimes_R N \cong \bigoplus_{i\in I}(M_i\otimes_R N)$.\\
Die Abbildung $M\times N \to \bigoplus_{i\in I} (M_i\otimes_R N), \left((x_i)_{i \in I}, y\right)\mapsto (x_i\otimes y)_{i\in I}$ ist bilinear, induziert
also eine $R$-linare Abbildung $\varphi: M\otimes N\to \bigoplus_{i\in I}(M_i\otimes N)$.\\
Umgekehrt: F\"ur jedes $i\in I$ induziert $M_i\hookrightarrow M$ $\psi_i : M_i\otimes N\to M\otimes N$;
die $\psi_i$ induzieren $\psi:\bigoplus_{i\in I}(M_i\otimes N)\to M\otimes N$ (UAE der direkten Summe).
``Nachrechnen'': $\phi$ und $\psi$ sind zueinander invers. 
\end{enumerate}
\end{Bew}
