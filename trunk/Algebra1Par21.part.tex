\chapter{Ringe}

\section{Grundlegende Definitionen und Eigenschaften}

\begin{DefBem}
\begin{enum}
\item Ein \emp{Ring} ist eine Menge $R$ mit Verkn�pfungen $+$ und
$\cd$, so da� gilt:
\begin{enumerate}
\renewcommand{\labelenumii}{(\roman{enumii})}
\item $(R,+)$ ist abelsche Gruppe
\item $(R,\cd)$ ist Halbgruppe
\item Die Distributivgesetze gelten:
\[ \left. \begin{array}{lcl}
    x\cd(y+z) &=& xy + xz \\
    (x+y)\cd z &=& xz + yz
    \end{array}
    \right\} \mbox{f�r alle }x,y,z \in R\]
\end{enumerate}

\item $R$ hei�t \emp{Ring mit Eins}, wenn $(R,\cd)$ Monoid ist.
\item $R$ hei�t \emp{kommutativer Ring}, wenn $(R,\cd)$ kommutativ
ist.
\item $R$ hei�t \emp{Schiefk�rper}, wenn $R^x = R \setminus \{0\}$,
dh. wenn jedes $x \in R \setminus \{0\}$ invertierbar bzgl. $\cd\;$
ist.
\item Ein kommutativer Schiefk�rper hei�t \emp{K�rper}.
\newline\newline\sbsp{0.9}{ \[H = \{a+bi + cj +dk: a,b,c,d \in R\}\] mit
komponentenweiser Addition und folgender Multiplikation: \[ i^2 =
j^2 = k^2 = -1;\; ij = k = -ji \] (z.B. ist dann $ik = iij = -j,\;
kj = ijj = i(-1) = -i$ etc.)\newline

\textbf{Es gilt}: $H$ ist Schiefk�rper (\emp{Hamilton-Quaternionen})

\[\ds (a+bi+cj+dk)(a-bi-cj-dk) =\] \[a^2 - abi - acj -adk + abi + b^2
+ \dots + c^2 + d^2 = a^2 + b^2 + c^2 + d^2\] \[\Ra
\frac{1}{a+bi+cj+dk} = \frac{a}{a^2+b^2 +c^2+d^2} + \frac{b}{\sum}
+\frac{c}{\sum} + \frac{d}{\sum}\;\hfill((a,b,c,d) \neq (0,0,0,0)) \] }

\item In jedem Ring gilt:
\[\left. \begin{array}{l}
x\cd 0 = 0 = 0\cd x \\
x(-y) = - (xy) = (-x)y \\
(-x)(-y) = xy \end{array} \right\} \mbox{f�r alle } x,y \in R\]
\sbew{0.9}{$x\cd 0 = x\cd(0+0) = x\cd0 +x\cd0$ (genauso f�r $0\cd x$)
\newline $x(-y) + xy = x(-y +y) =x\cd 0 = 0$
\newline $(-x)(-y) = -((-x)y) = -(-(xy)) = xy$}

\item Ist $R$ ein Ring mit Eins und $R \neq \{0\}$, so ist $0 \neq
1$ in $R$ \newline \sbew{0.9}{W�re $0 = 1$, so g�lte f�r jedes $x
\in R :x =x\cd 1 = x\cd 0 = 0$, also doch $R = \{0\}$}
\end{enum}
\end{DefBem}

\begin{Def}
Sei $(R,+,\cd)$ ein Ring.
\begin{enum}
\item $R' \subseteq R$ hei�t \emp{Unterring}, wenn $(R',+,\cdot)$ Ring
ist. umgekehrt hei�t $R$ dann \emp{Ring\-erweiterung} von $R'$.
\item $I \subseteq R$ hei�t (zweiseitiges) \emp{Ideal}, wenn $(I,+)$
Untergruppe von $(R,+)$ ist und $rx \in I, xr \in I$ f�r alle $x \in
I, r \in R$.
\item $x\in R$ hei�t (Links-/Rechts-)\emp{Nullteiler}, wenn es ein
$y \in R \setminus \{0\}$ gibt mit $xy = 0$. (bzw. $yx = 0$)
\item $R$ hei�t \emp{nullteilerfrei}, wenn $0$ der einzige
Nullteiler in $R$ ist. (dh. wenn aus $xy = 0$ folgt, da� $x=0$ oder
$y=0$)
\item $R$ hei�t \emp{Integrit�tsbereich} (integral[domain]), wenn er
nullteilerfrei und kommutativ ist und eine Eins besitzt.
\end{enum}
\end{Def}

\begin{DefBem}
\begin{enum}
\item Eine Abbildung $\varphi: R \ra R'$ ($R,R'$ Ringe) hei�t
\emp{Ringhomomorphismus}, wenn $\varphi: (R,+) \ra (R',+)$
Gruppenhomomorphismus und $\varphi(R,\cd) \ra (R',\cd)$
Halbgruppenhomomorphismus ist.
\item Sind $R,R'$ Ringe mit Eins, so hei�t ein Ringhomomorphismus
$\varphi: R \ra R'$ ein \emp{Ringhomomorphismus mit Eins}, wenn
$\varphi(1_R) = 1_{R'}$.
\item Die Ringe bilden mit Ringhomomorphismus eine Kategorie
\item Die Ringe mit Eins bilden mit Homomorphismen von Ringen mit
Eins eine Kategorie (echte Unterkategorie der Ringe)
\item $\underset{\stackrel{\downarrow \varphi} (R',+,\cd)}{(R,+,\cd)}
\hookrightarrow \underset{\stackrel{\downarrow} (R',+)}{(R,+)}$ ist
kovarianter Funktor: Ringe $\ra$ abelsche Gruppen.
\medskip\newline$(R,+, \cd) \mapsto (R^x, \cd)$ ist kovarianter
Funktor: Ringe mit Eins $\ra$ Gruppen.
\end{enum}
\end{DefBem}

\begin{Bem}
Sei $\varphi: R \ra R'$ Ringhomomorphismus.
Dann gilt:
\begin{enum}
\item Bild ($\varphi$) ist Unterring von $R'$
\item Kern ($\varphi$) ist Ideal in $R$, Kern($\varphi$) =
$\varphi^{-1}(0)$ \newline
\sbew{0.9}{ Sei $x \in$ Kern($\varphi$),
$r \in R \Ra \varphi(rx) = \varphi(r)\varphi(x) = \varphi(r)0 = 0
\Ra rx \in$ Kern($\varphi$)}
\item Ist $R$ Schiefk�rper, $R'$ Ring mit Eins, $\varphi$ Homomorphismus von Ringen mit
Eins, so ist $\varphi$ injektiv (oder $R' = \{0\}$)
\newline \sbew{0.9}{Sei $x \in R \setminus \{0\} \Ra \varphi(x)
\varphi(x^{^-1}) = \varphi(1) = 1_{R'} \neq 0$, sofern $R' \neq
\{0\}$ $\Ra \varphi(x) \neq 0 \Ra$ Kern($\varphi$) = $\{0\} \Ra
\varphi$ injektiv.}
\end{enum}
\end{Bem}

\begin{DefBem}
Sei $R$ Ring mit Eins.
\begin{enum}
\item \[\varphi_R: \mathbb{Z} \ra R,\; n \mapsto \left\{ \begin{array}{lc}
n \cd 1 = \underset{n}{\underbrace{1+\dots+1}} & n \geq 0 \\
-n \cd 1 & n \leq 0 \end{array} \right.\] ist Homomorphismus von
Ringen mit Eins. %\end{enum}
\item Ist Kern($\varphi_R$)$=n\mathbb{Z}\;(n\geq 0)$, so hei�t $n$
die \emp{Charakteristik} von $R:\;n =$char($R$)
\item Ist $R$ nullteilerfrei, so ist char($R$)$= 0$, oder
char($R$)$=p$ f�r eine Primzahl $p$.
\item Ist $K$ (Schief-)K�rper der Charakteristik $p>0$, so ist
Bild($\varphi_K$)$\cong \mathbb{Z}/n\mathbb{Z} = \mathbb{F}_p$ der
kleinste Teilk�rper von $K$. Er hei�t \emp{Primk�rper}. Ist
char($K$)$=0$, so ist der kleinste Teilk�rper von $K$ isomorph zu
$Q$
\end{enum}
\bsp{ $R$ Ring, $R^{n \times n}$ Ring der $n \times n$-Matrizen mit
Eintr�gen in $R$ \newline F�r $n \geq 2$ ist $R^{n \times n}$ nicht
kommutativ und nicht nullteilerfrei. \medskip\newline Die Eins in
$R^{n \times n}$ ist die Einheitsmatrix:
$\begin{pmatrix} 1_R & & 0 \\
                 & \ddots \\
                 0 & &  1_R
\end{pmatrix}$
\medskip \newline (vorausgesetzt, $R$ hat eine Eins).
Die Einheiten in $R^{n \times n}$ sind die invertierbaren Matrizen
\medskip\newline$(R^{n\times n})^x = GL_n(R) = \{A \in R^{n \times
n} : \det A \in R^x \}$

Zur Definition von $\det A$ mu� $R$ kommutativ sein!
\newline $SL_n(R) = \{A \in GL_n(R): det A = 1\}$ ist Normalteiler
von $GL_n(R)$
\[\begin{array}{cccc} GL_n(R)&/&SL_n(R) &\cong R^x \\
                     A&\cd& SL_n(R) & \mapsto det A \end{array}\] }
\end{DefBem}

\begin{DefBem}
\begin{enum}
\item Sei $R$ ein Ring, $a \in R$. Dann ist:
\newline $(a) \defeqr a R = \{ ar : r \in R\}$ ein \emp{Rechtsideal} in
$R$. Es ist $a \in (a)$, falls $R$ eine Eins hat.
\item Ein (Rechts-)Ideal $I$ in $R$ hei�t \emp{Hauptideal}, wenn es
ein $a \in R$ gibt mit $I = (a)$
\item Ein kommutativer Ring mit Eins hei�t \emp{Hauptidealring}, wenn
jedes Ideal in $R$ ein Hauptideal ist.
\newline \sbsp{0.9}{Sei $I \subseteq \mathbb{Z}$ Ideal, $a \in I$ mit $|a| \leq |b| \forall b
\in I \setminus \{0\}$.
\newline Beh.: $I=(a)$ \textbf{denn}:
\newline ''$\supseteq$'' $\chk\;$ ''$\subseteq$'': Sei $b \in I$.
Teile $b$ durch $a:\;b=qa+r$ mit $r \leq |a| \Ra r = b -qa \in I \Ra
r = 0$.
\newline Analog sind alle Polynomringe mit Koeffizienten in einem K\"orper
auch Hauptidealringe.}
\item Sei $R$ kommutativer Ring mit Eins, $R \neq \{0\}$. Dann gilt:
\newline $R$ ist K�rper $\lra (0)$ und $R$ sind die einzigen Ideale
in $R$
\newline\sbew{0.9}{''$\Ra$'' Sei $ I \subset R$ Ideal, $a \in I
\setminus \{0\} \Ra$ es gibt $a^{-1} \in R \Ra 1 = a a^{-1} \in I
\Ra I = R\;(x \in R \Ra x = 1x)$
\newline ''$\Leftarrow$'' Sei $a \in R \setminus\{0\} \Ra (a) = R \Ra
\exists b \in R : ab =1$ }
\end{enum}
\end{DefBem}

\begin{DefBem}
Sei $R$ Ring, $I_1, I_2$
Ideale in $R$. Dann gilt:
\begin{enum}
\item
\begin{enum}
\item[] $I_1 \cap I_2$ ist Ideal.
\item[] $I_1 + I_2 = \{a+b:a \in I_1, b\in I_2\}$ ist Ideal.
\item[] $I_1 \cd I_2 = \{\displaystyle \sum_{i=1}^{<\infty} a_i b_i: a_i \in I_1,
b_i \in I_2 \}$ ist Ideal.
\end{enum}
\item $I_1 \cd I_2 \subseteq I_1 \cap I_2$ (aber im allgemeinen
$\neq$!)
\item Ein beliebiger Durchschnitt von Idealen ist Ideal.
\item Sei $R$ kommutativ mit Eins, $X \subseteq R$
\[(X) = \bigcap_{\substack{I \subseteq R \mbox{ \scriptsize Ideal} \\
X \subseteq I}} I = \{ \sum_{\mbox{\scriptsize endl.}} r_i x_i:\;
r_i \in R, x_i \in X\}\] hei�t das von $X$ erzeugte Ideal.
\item $\ds I_1 + I_2 = (I_1 \cup I_2),\; I_1 \cd I_2 = (\{ab: a \in
I_1, b \in I_2\})$
\end{enum}
\end{DefBem}