\documentclass{article}
\newcounter{chapter}
\setcounter{chapter}{20}
\usepackage{ana}
\usepackage{mathrsfs}

\title{Lineare Differentialgleichungen $m$-ter Ordnung mit konstanten Koeffizienten}
\author{Ferdinand Szekeresch}
% Wer nennenswerte �nderungen macht, schreibt sich bei \author dazu

\begin{document}
\maketitle

Wir gehen wie in �17 den Weg �ber das Komplexe: \\
$I \subseteq \MdR$ sei ein Intervall, $a_0, a_1, \ldots, a_{m-1} \in \MdC, b: I \rightarrow \MdC$ sei stetig;$ x_0 \in I, y_0, \ldots, y_{m-1} \in \MdC$ \\
Wir betrachten die DGL:
$$Ly := y^{(m)} + a_{m-1}y^{(m-1)} + \ldots + a_1y' + a_0y = b(x)$$
�18/19 obiger Gleichung entspricht das folgende System
$$z' = \begin{pmatrix}
0 	& 1 	 & 0 	  & \cdots & 0 \\ 
\vdots 	& \ddots & \ddots & \ddots & \vdots \\
\vdots 	& 	 & \ddots & \ddots & 0 \\
0	& \cdots & \cdots & 0	   & 1 \\
-a_0	& -a_1	 & \cdots & \cdots & -a_{m-1} \end{pmatrix}z + \begin{pmatrix}0 \\ \vdots \\ \vdots \\ 0 \\ b(x) \end{pmatrix}$$

Aus �17 folgt:

\begin{satz} %20.1
\begin{liste}
\item $$\text{das AWP} \begin{cases} Ly = b(x) \\ y(x_0) = y_0, y'(x_0) = y_1, \ldots, y^{(m-1)}(x_0) = y_{m-1}\end{cases}$$ hat auf $I$ genau eine L�sung.
\item Die Definitionen und S�tze des �en 19 gelten auch im Komplexen. $\MdL$ ist ein komplexer VR, $\dim\MdL = m$. \\
Wir betrachten zun�chst die homogene Gleichung (H) $Ly=0$ \\
$p(\lambda) := \lambda ^m + a_{m-1}\lambda ^{m-1} + \ldots + a_1\lambda + a_0$ hei�t das charakteristische Polynom von (H). Beachte: $p(\lambda) = \det (\lambda E - A)$.
\end{liste}
\end{satz}

\begin{satz}[ohne Beweis] %20.2
Sie $p$ das char. Polynom von (H) \\
\begin{liste}
\item $\lambda_0$ sei eine $q$-fache Nullstelle von $p$. Dann sind $e^{\lambda_0x}, xe^{\lambda_0x}, \ldots, x^{q-1}e^{\lambda_0x}$ linear unabh�ngige L�sungen von (H).
\item F�hrt man (1) f�r jede Nullstelle von $p$ durch, so erh�lt man ein (komplexes) FS von (H).
\item Es seien $a_0, a_1, \ldots, a_{m-1} \in \MdR$. Dann erh�lt man ein reelles FS von (H) wie folgt: \\
Sei $\lambda$ eine Nullstelle von $p$.
\begin{enumerate}
\item Ist $\lambda_0 \in \MdR$, so �bernehme die L�sungen aus (1).
\item Ist $\lambda_0 \notin \MdR$, und $y$ eine L�sung aus (1), so bilde die reellen L�sungen $\Re y$ und $\Im y$ und streiche die zu $\overline{\lambda_0}$ geh�renden L�sungen.
\end{enumerate}
\end{liste}
\end{satz}

\begin{beispiele}
\item $y^{(6)} - 6y^{(5)} + 9y^{(4)} = 0 \\
p(\lambda) = \lambda ^6 - 6\lambda ^5 + 9 \lambda ^4= \lambda ^4(\lambda ^2 - 6\lambda + 9) = \lambda ^4(\lambda - 3)^2$ \\\\
$$\begin{array}{ll} \lambda_1 = 0: 1, x, x^2, x^3 \\ \lambda_2 = 3: e^{3x}, xe^{3x} \end{array}\Bigg\}\text{FS obiger Gleichung}$$\\
Allgemeine L�sung: $y(x) = c_1 + c_2x + c_3x^2 + c_4x^3 + c_5e^{3x} + c_6xe^{3x}$

\item $y''' - 2y'' + y' - 2y = 0 \\
p(\lambda) = \lambda ^3 - 2\lambda ^2 + \lambda - 2 = (\lambda - 2)(\lambda ^2 + 1) = (\lambda - 2)(\lambda - i)(\lambda + i) \\
\lambda_1 = i : $ komplexe L�sung $e^{ix} = \cos x + i \sin x \\
\lambda_2 = 2 : e^{2x} $ \\
FS : $e^{2x} , \cos x, \sin x$ \\
Allgemeine L�sung : $y(x) = c_1e^{2x} + c_2\cos x + c_3 \sin x \; (c_1, c_2, c_3 \in \MdR)$

\item L�se das AWP: 
$$\begin{cases}y''' - 2y'' + y' - 2y = 0 \\ y(0) = 0, y'(0) = 1, y''(0) = 0\end{cases}$$
Allgemeine L�sung der DGL: $y(x) = c_1e^{2x} + c_2\cos x + c_3 \sin x \\
0 = y(0) = c_1 + c_2 \quad c_1 = -c_2 \\
1 = y'(0) = 2c_1e^{2\cdot 0} - c_2\sin 0 + c_3\cos 0 = 2c_1 + c_3 \\
y''(x) = 4c_1e^{2x} - c_2\cos x - c_3\sin x \folgt 0 = 4c_1 - c_2 \\
\folgt c_1 = c_2 = 0, c_3 = 1 $ \quad L�sung des AWPs: $y(x) = \sin x$

\item $y'' - 2y' + 5y = 0 \\
p(\lambda) = \lambda ^2 - 2\lambda + 5 = (\lambda - (1+2i))(\lambda - (1-2i)) \\
\lambda = 1+2i : $ komplexe L�sung $e^{1+2i}x = e^xe^{2ix} = e^x(\cos 2x + i\sin 2x)$ \\
FS: $e^x\cos (2x), e^x\sin (2x)$

\item L�se das \begriff{Randwertproblem} (RWP): \\
$y'' + y = 0, y(0) = 1, y(\frac{\pi}{2}) = 1 \\
p(\lambda) = \lambda ^2 + 1 = (\lambda - i)(\lambda + i).$ FS: $\cos x, \sin x$ \\
Allgemeine L�sung der DGL: $y(x) = c_1\cos x + c_2\sin x \\
1 = y(0) = c_1, 1 = y(\frac{\pi}{2}) = c_2$ L�sung des RWPs: $y(x) = \cos x + \sin x$

\item L�se das RWP:
$y'' + \pi ^2y = 0, y(0) = y(1) = 0 \\
p(\lambda) = \lambda ^2 + \pi ^2 = (\lambda - i\pi)(\lambda + i\pi)$ \\
Allgemeine L�sung der DGL $y(x) = c_1\cos (\pi x) + c_2\sin (\pi x) \\
0 = y(0) = c_1, 0 = y(1) = c_2 \sin \pi$ \\
L�sungen des RWPs: $y(x) = c\cdot\sin (\pi x) \quad c \in \MdR$


\end{beispiele}

% Michael Knoll, 21.1.2006

Wir betrachten nun den inhomogenen Fall:
\begin{displaymath}
(IH) \, Ly = b(x)
\end{displaymath}

Um eine spezielle L�sung des inhomogenen Problems zu finden, kann man 19.6 anwenden (L�sung eines inhomogenen Systems).

Sei dazu $p$ das charakteristische Polynom von $(H)$. 
\begin{definition}[0-fache Nullstelle]
$\mu \in \mathbb{C}$ ist eine \begriff{0-fache Nullstelle} von $p: \Leftrightarrow p(\mu) \ne 0$
\end{definition}


\begin{satz}[Regel - ohne Beweis]
Seien $\alpha, \beta \in \MdR$, $n, q \in \MdN_0$ und $b$ sei von der Form:

$b(x)=(b_0+b_1x+ \dots +b_n x^n) \cdot e^{\alpha x} \cdot \cos \beta x$ bzw.

$b(x)=(b_0+b_1x+ \dots +b_n x^n) \cdot e^{\alpha x} \cdot \sin \beta x$

Ist $\alpha + i \beta$ eine $q$-fache Nullstelle von $p$, so gibt es eine spezielle L�sung $y_s$ von $(IH)$ der Form

$y_s(x)= x^q \cdot e^{\alpha x}((A_0+A_1 x+ \dots + A_n x^n)\ cos \beta x + (B_0 + B_1 x + \dots + B_n x^n) \sin \beta x)$
\end{satz}

\begin{beispiel}
\begin{itemize}
	\item[(1)] $y''' - y' = x-1$
	
	Erster Schritt: L�sung der homogenen Gleichung $y''' - y' = 0$. Charakteristisches Polynom: $p(\lambda) = \lambda^3-\lambda = \lambda(\lambda^2 - 1) = \lambda( \lambda + 1)(\lambda - 1)$ \\	
	Fundamentalsystem: $1, e^x, e^{-x}$
	
	Zweiter Schritt: Spezielle L�sung der inhomogenen Gleichung. System ist von obiger Form mit $\alpha= \beta = 0$; $\alpha + i \beta = 0$ ist 1-fache Nullstelle von $p$. Ansatz: F�r eine spezielle L�sung der inhomogenen Gleichung:
	
	$y_s(x) = x ( A_0 + A_1 x) = A_0x + A_1 x^2$
	
	$y_s'(x) = A_0 + 2x A_1$
	
	$y_s'''(x) = 0$
	
	$x-1 \stackrel{!}{=}y_s''' - y_s' = -A_0 - 2xA_1 \Rightarrow A_0 = 1; A_1 = -\frac{1}{2}$
	
	Allgemeine L�sung der Differentialgleichung:
	
	$y(x) = c_1 + c_2e^x + c_3e^{-x} + x - \frac{1}{2}x^2$ $(c_1, c_2, c_3 \in \MdR)$
	
	\item[(2)] $y'' - y = x e^x$
	
	1. Schritt: L�sung der homogenen Gleichung $y'' - y = 0$. Charakteristisches Polynom $p(\lambda) = \lambda^2 - 1 = (\lambda - 1)(\lambda + 1)$
	
	Fundamentalsystem: $e^x, e^{-x}$
	
	2. Schritt: Spezielle L�sung der inhomogenen Gleichung. System ist von obiger Form mit $\alpha = 1, \beta = 0$; $\alpha + i \beta = 1$ ist einfach Nullstelle von $p$. Ansatz f�r eine spezielle L�sung:
	
	$y_s(x) = x(A_0 + A_1 x) e^x$
	
	Nachrechnen: $y_s''(x) - y_s(x) = (2 A_0 + 2 A_1 + 4 A_1 x) e^x \stackrel{!}{=} xe^x \Leftrightarrow 2 A_0 + 2 A_1 + 4 A_1 x = x \Rightarrow A_1 = \frac{1}{4}, A_0 = - \frac{1}{4}$
	
	$y_s(x) = \frac{1}{4} x ( x-1) e^x$
	
	Allgemeine L�sung der Differentialgleichung:
	
	$y(x) = c_1 e^x + c_2 e^{-x} + \frac{1}{4}x(x-1)e^x$ $(c_1,c_2 \in \MdR)$
	
\end{itemize}
\end{beispiel}
	


\end{document}
