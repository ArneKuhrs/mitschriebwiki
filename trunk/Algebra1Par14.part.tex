\section{Zyklische Gruppen}

\begin{DefBem}
    Sei $G$ Gruppe, $A \subseteq G$ Teilmenge.
    
    \begin{enum}
        \item $\displaystyle \langle A \rangle \defeqr \bigcap_{\substack{H 
        \subseteq G\; Ugr. \\ A \subseteq H}} H\;$ hei�t die \emp{von 
        $\mathbf{A}$ erzeugte Untergruppe von $\mathbf{G}$}. \newline
        \sbew{0.9}{
        z.z.: $\displaystyle \langle A \rangle = \bigcap_{\substack{H
        \subseteq G\; Ugr.\\ A \subseteq H}} H$ ist Untergruppe in G.

        \begin{enum}
            \item[(i)] $\forall H \subseteq G$, $H$ Untergruppe$: e \in H \Ra e
            \in \langle A \rangle \Ra \langle A \rangle \neq \emptyset$

            \item[(ii)] Seien $x, y \in \langle A \rangle$, $H$ Untergruppe von
            $G$
            mit $A \subseteq H \Ra x,y \in H \overset{H \; Ugr.}{\Ra} xy^{-1}
            \in H \Ra
            xy^{-1} \in \langle A \rangle. \Ra \langle A \rangle$ Untergruppe
            von $G$.
        \end{enum}
        }
        
        \item $G$ hei�t \emp{zyklisch}, wenn es ein $g \in G$ mit $G=\langle g
        \rangle$ gibt.

        \item F�r $g \in G$ ist $\langle g \rangle = \{g^n: n \in \mathbb{Z} \}$
        \newline
        \sbew{0.9}{
             '' $\supseteq$ '': $\chk$ ''$\subseteq$'': Nach \ref{1.9} ist 
             \{$g^n: n \in \mathbb{Z}\} = \mbox{Bild}(\varphi_g)$ Untergruppe
             von $G$. $\chk$
        }
      
        \item Jede zykische Gruppe ist isomorph zu $\mathbb{Z}$ oder zu 
        $\mathbb{Z}/n\mathbb{Z}$ f�r genau ein $n \in \mathbb{N} \setminus
        \{0\}$ \newline
        \sbew{0.9}{
            Sei $G=\langle g \rangle$, $\varphi_g: \mathbb{Z} \ra G,\; n \mapsto
            g^n$ (siehe \ref{1.9}) \newline $\varphi_g$ ist surjektiver 
            Gruppenhomomorphismus. \newline
            Nach Satz \ref{Satz 1} ist $G \cong
            \mathbb{Z}/\mbox{Kern}(\varphi_g)$ \newline
            Da jede Untergruppe von $\mathbb{Z}$ von der Form $H=n\mathbb{Z}$
            f�r ein $n \in \mathbb{N} \Ra$ Behauptung.
        }

        \item Jede Untergruppe einer zyklischen Gruppe ist zyklisch.\newline
        \sbew{0.9}{
            Sei $G = \langle g \rangle$ zyklisch, $H \subseteq G$ Untergruppe, 
            $n \defeqr \min \{k \in \mathbb{N} \setminus \{0\} : g^k \in
            H\}$.\newline Dann ist $\langle g^n \rangle \subseteq H$. W�re $h
            \in H \setminus \langle g^n \rangle$, aber $h = g^m$ mit $m \not \in
            n\mathbb{Z} \Ra d\defeqr \mbox{ggT}(m,n) < n$.\newline Nach Euklid
            gibt es $a, b \in \mathbb{Z}$ mit $am + bn = d \Ra g^d = (g^m)^a
            (g^n)^b \in H $!\newline
            Widerspruch zur Minimalit�t von $n$ mit $g^n \in H$
        }
        
        \item F�r $g \in G$ hei�t ord$(g) \defeqr |\langle g \rangle|$ die 
        \emp{Ordnung von $\mathbf g$ in $\mathbf G$}. \newline Es ist ord$(g) =
        \left \{ \begin{array}{cc} \min\{n \in \mathbb{N} \setminus \{0\} : g^n =
        e\} & g^n = e\\
        \infty & g^n \neq e, \forall n \in \mathbb{N} \setminus \{0\} 
        \end{array} \right.$

        \item Ist $G$ endlich, so ist $\forall g \in G$ ord$(g)$ ein Teiler von
        $|G|$.\newline
        \sbew{0.9}{
            Folgt aus dem Satz von Langrange(\ref{1.12})
        }
    \end{enum}
\end{DefBem}

\begin{DefBem}
    \mbox{}
   
    \begin{enum}
        \item Die Abbildung $\varphi: \mathbb{N} \setminus \{0\} \ra 
        \mathbb{N},\; n \mapsto \varphi(n) \defeqr |\{k \in \mathbb{N}: 1 \leq k
        \leq n : \mbox{ggT}(k,n) = 1\}|$ hei�t \emp{Eulersche 
        $\varphi$ -Funktion}.

        \item Ist $G$ zyklische Gruppe der Ordnung $n$, so gibt es f�r jeden 
        Teiler $d$ von $n$ $|\{x \in G : \mbox{ord}(x) = d\}| = \varphi(d)$
        \newline
        \sbew{0.9}{
            Sei $G=\langle g \rangle$. F�r $x = g^k \in G$ ist ord$(x) = 
            \frac{n}{\mbox{\small ggT}(k,n)}$. Also ist ord$\ds (x) = d \lra 
            \mbox{ggT}(k,n) = \frac{n}{d}\\$ $G \mbox{ ist } |\{l \in
            \mathbb{N} : 1 \leq l \leq d,\; \mbox{ggT}(l,d) = 1 \}| \defeqr
            \varphi(d)$, $k \mapsto \frac{k}{\frac{n}{d}}$
        }

        \item F�r jedes $n \in \mathbb{N} \setminus \{0\}$ gilt: $\ds n =
        \sum_{d \mid n} \varphi(d)$
        \newline
        \sbew{0.9}{
            $n = |G| = \displaystyle \sum_{d \mid n} | \{x \in G, \mbox{ord}(x)
            = d\}| 
            \overset{(b)}{=} \sum_{d/n} \varphi(d)$
        }
    \end{enum}

    \bsp{
    
    \begin{enumerate}
        \item[(1)] \[\{ e^{\frac{2\pi \imath k}{n}} : n\in \mathbb{N} \setminus
        \{0\}, 0 \leq k < n \}\] ist zyklische Untergruppe von $\mathbb{C}^*$ der
        Ordnung $n$. ($n$-te Einheitswurzel)

        \item[(2)] Sei $V = \{ id, \tau, \sigma_1, \sigma_2 \}$ mit $\tau=$ 
        Drehung im $\mathbb{R}^2$ $\begin{pmatrix} -1 & 0 \\ 0 & -1 
        \end{pmatrix}$, \newline
        $\sigma_1=$ Spiegelung an der $x$-Achse $\begin{pmatrix} 1 & 0 \\ 0 & -1
        \end{pmatrix}$, \newline
        $\sigma_2=$ Spiegelung an der $y$-Achse $\begin{pmatrix} -1 & 0 \\ 0 & 1
        \end{pmatrix}$.
        $V$ ist abelsche Gruppe, aber \textbf{nicht} zyklisch. $V$ hei�t 
        \emp{Kleinsche Vierergruppe} $V \cong \mathbb{Z}/2\mathbb{Z} \oplus 
        \mathbb{Z}/2\mathbb{Z}$

        \item[(3)] $\begin{array}{ccccc} \mathbb{Z}/6\mathbb{Z} & \cong &
        \mathbb{Z}/2\mathbb{Z} & \oplus & \mathbb{Z}/3\mathbb{Z} \\ 
        \{1,a,a^2,a^3,a^4,a^5\} & & \{1,\sigma\} & &\{1, \tau, \tau^2\}  \\    
        a & \mapsto & (\sigma, \tau) \end{array}$
    \end{enumerate}
    }
\end{DefBem}