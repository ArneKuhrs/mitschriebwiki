\section{Abelsche Gruppen}

\begin{DefBem}
\label{1.18}
    Sei $(A,+)$ eine abelsche Gruppe, $X \subseteq A$.
    \begin{enum}
        \item $A$ hei�t \emp{freie abelsche Gruppe} mit Basis $X$, wenn jedes $a
        \in A$ eine eindeutige Darstellung $\ds a = \sum_{x\in X} n_x x$ hat mit
        $n_x \in \mathbb{Z}\;, n_x \neq 0$ nur f�r endlich viele $x \in X$. Ist
        in dieser Situation $|X| = n$, so hei�t $n$ der \emp{Rang} von $A$. $A$
        ist isomorph zu $\ds \mathbb{Z}^X \defeqr \bigoplus_{x \in X}
        \mathbb{Z}$ \newline
        \sbew{0.9}{
            $A \ra \mathbb{Z}^X : \sum n_x x \mapsto (n_x)_{x \in X}$ ist 
            Isomorphismus.
        }

        \item (UAE der freien abelschen Gruppe) \newline
        Zu jeder abelschen Gruppe $A$ und jeder Abbildung $f:X \ra A$ gibt es
        genau einen Homomorphismus $\varphi: \mathbb{Z}^X \ra A$ mit $\forall x
        \in X: \varphi(x) = f(x)$ \newline
        \sbew{0.9}{
            Setze $\ds \varphi(\sum_{x\in X} n_x x) \defeqr \sum_{x\in X} n_x 
            f(x)$
        } 
    \end{enum}

    \bsp{
        (wichtig!) $X$ endlich, $X=\{x_1, \dots, x_n\}$. Dann ist $\mathbb{Z}^X
        \cong \mathbb{Z}^n\\$ $\mathbb{Z}^n$ ist ''so etwas �hnliches'' wie ein
        Vektorraum \textit{(''freier Modul'')}. Insbesondere lassen sich die
        Gruppenhomomorphismen $\mathbb{Z}^n \ra \mathbb{Z}^m$ durch eine $m
        \times n$-Matrix mit Eintr�gen in $\mathbb{Z}$ beschreiben.
    }
\end{DefBem} 


\begin{Satz}[Elementarteilersatz]
\label{Satz 2}
    Sei $H$ eine Untergruppe von $\mathbb{Z}^n$ $(n \in \mathbb{N} \setminus
    \{0\})$. Dann gibt es eine Basis $\{x_1, \dots, x_n\}$ von $\mathbb{Z}^n$,
    ein $r \in \mathbb{N}$ mit $0 \leq r \leq n$ und $a_1, \dots, a_r \in 
    \mathbb{N} \setminus \{0\}$ mit $a_i$ teilt $a_{i+1}$ f�r $i = 1,\dots,r-1$,
    so da� $a_1 x_1, \dots, a_r x_r$ eine Basis von $H$ ist. Insbesondere ist
    $H$ ebenfalls eine freie abelsche Gruppe.\smallskip \newline
    \sbew{1.0}{\newline
        \textbf{1. Schritt}: $H$ ist endlich erzeugt: Induktion �ber $n$:
        \newline
        \textbf{$n=1$ }:
            $\chk$\newline
        \textbf{$n > 1$}:
            Sei $e_1, \dots, e_n$ Basis von $\mathbb{Z}^n$,
            $\pi: \mathbb{Z}^n \ra \mathbb{Z}$, $\displaystyle \sum_{i=1}^n a_i
            e_i \mapsto a_n$ \newline
            (Projektion auf letze Komponente).
        \bigskip \newline
        \textbf{1. Fall}:
            $\pi(H) = \{0\} \Ra H \subseteq \mathbb{Z}^{n-1}$, also endlich erzeugt
            nach \textbf{IV}.
            \smallskip\newline
        \textbf{2. Fall}:
            $\pi(H) = l\mathbb{Z}$ f�r ein $l \in \mathbb{N} \setminus \{0\}$
            Sei $y \in H$ mit $\pi(y) = l$ \newline
            \textbf{Beh}.:
                $H \cong \langle y \rangle \oplus (H \cap \mbox{Kern}(\pi))$
            Dann folgt die Behauptung von Schritt 1, da Kern$(\pi) \cong
            \mathbb{Z}^{n-1}$, $H \cap$ Kern$(\pi)$ Untergruppe von
            $\mathbb{Z}^{n-1}$, existiert also nach \textbf{IV} $\Ra$
            \smallskip\newline
            \textbf{Bew. der Beh.}.:
                $\langle y \rangle \cap (H \cap \mbox{Kern}(\pi)) = \{0\}$ nach
                Definition von $y \Ra$ Summe direkt. \newline Sei $z \in H$ mit
                $\pi(z) = k \cd l$ f�r ein $k \in \mathbb{Z} \Ra z - ky \in H
                \cap \mbox{Kern}(\pi) \Ra$ Beh. \newline $\longrightarrow$
    }

    \noindent \sbew{1.0}{
        \smallskip \newline
        \textbf{2. Schritt}:
            Sei $y_1, \dots, y_r$ Erzeugendensystem von $H$. Nach Schritt 1 kann
            $r \leq n$ erreicht werden. Schreibe $y_j = \displaystyle \sum_{i=1}^n a_{ij}
            e_i$. Dann ist $A\defeqr(a_{ij}) \in \mathbb{Z}^{n \times r}$ eine 
            Darstellungsmatrix der Einbettung $H \hookrightarrow \mathbb{Z}^n$ 
            bzgl. der Basen $\{y_1, \dots, y_r\}$ von $H$ und 
            $\{e_1,\dots,e_n\}$ von $\mathbb{Z}^n$. Zeilen- und
            Spaltenumformungen entsprechen Basiswechseln in $H$ bzw. $\mathbb{Z}^n$.
            \newline
            \textbf{Vorsicht}:
                Dabei d�rfen nur \textbf{ganzzahlige} Basiswechselmatrizen 
                benutzt werden, deren inverse Matrix ebenfalls ganzzahlige Eintr�ge hat!
            \newline \textbf{Ziel}: Bringe $A$ durch elementare Zeilen- und 
            Spaltenumformungen auf Diagonalgestalt: \newline $\widetilde{A} = 
            \begin{pmatrix} a_1 & & 0 \\ &  \ddots \\ 0  & & a_r \end{pmatrix}$
            mit $a_i \in \mathbb{Z}$ und $a_i$ teilt $a_{i+1}\; \forall\;
            i=1,\dots,r-1 \\$ $\longrightarrow$
    }
    \noindent\sbew{1.0}{\newline
        \textbf{3. Schritt}:
            Das geht! Ganzzahliger Gau�-Algorithmus.
            \begin{enum}
                \item[(i)] Suche den betragsm��ig kleinsten Matrixeintrag $\neq
                0$ und bringe diesen nach $a_{11}$. Dazu braucht man h�chstens
                eine Zeilen- und eine Spaltenumformung.
                
                \item[(ii)] Stelle fest, ob alle $a_{i1}\; (i=2,\dots,n)$ durch
                $a_{11}$ teilbar sind. Falls nicht, teile $a_{i1}$ mit Rest durch
                $a_{11} : a_{i1} = q a_{11} + r$ mit $0 < r < |a_{11}|$. Ziehe
                dann von der $i$-ten Zeile das $q$-fache der ersten ab. Die
                neue $i$-te Zeile beginnt nun mit $\widetilde{a_{i1}} = r \Ra$
                Zur�ck zu (i)
                
                \item[(iii)] Sind schlie�lich alle $a_{i1}$ durch $a_{11}$
                teilbar, so wird die erste Spalte zu \[ \left( \begin{array}{c}
                a_{11} \\ 0 \\ \vdots \\ 0 \end{array} \right) \] gemacht,
                indem man von der $i$-ten Zeile das $\frac{a_{i1}}{a_{11}}$-fache
                der ersten Zeile abzieht.
            \end{enum}
    }
    
    \sbew{1.0}{
        \begin{enum}
            \item[(iv)] Genauso wird die erste Zeile zu $(a_{11}, 0, \dots, 0)$
            
            \item[(v)] Gibt es jetzt noch einen Matrixeintrag, der nicht durch 
            $a_{11}$ teilbar ist, schreibe $a_{ij} = q a_{11} + r$ mit $0 < r <
            |a_{11}|$ Ziehe von der $i$-ten Zeile das $q$-fache der ersten ab. 
            Die neue $i$-te Zeile lautet dann: \[(-q a_{11}, a_{i2}, \dots, 
            a_{ij}, \dots, a_{ir})\] (da $a_{i1} = 0, a_{1k} = 0$ f�r $1<k\leq
            r$) \newline
            Addiert man zur $j$-ten Spalte die erste, so ist das neue Element
            $\widetilde{a_{ij}} = a_{ij} - q a_{11} = r \Ra$ Zur�ck zu (i)
            
            \item[(vi)] Nach endlich vielen Schritten erhalte Matrix 
            \[\begin{pmatrix} a_{11} & 0 & \hdots & 0 \\ 0 \\ \vdots &  & 
            \mbox{\LARGE $A'$}\\ 0 \end{pmatrix},\] in der alle Eintr�ge von
            $A'$ durch $a_{11}$ teilbar sind. Wende nun den Algorithmus auf $A'$ an.
        \end{enum}
    }

    \textbf{Erg�nzung}:
        \begin{enumerate}
            \item[(1)]In der Situation von Satz 2 hei�en die $a_{ii}\; 
            i=1,\dots,r$ die \emp{Elementarteiler} von $H$.
            
            \item[(2)] Ist $A = (h_1,\dots,h_r) \in \mathbb{Z}^{n\times r}$, so
            erzeugen die Spalten $h_1,\dots,h_r$ eine Untergruppe von 
            $\mathbb{Z}^n$. $A$ ist Darstellungsmatrix der Einbettung $H \hookrightarrow 
            \mathbb{Z}^n$.\newline Die Elementarteiler von $H$ hei�en auch 
            Elementarteiler von $A$.
        \end{enumerate}
\end{Satz}

\begin{Satz}[Struktursatz f�r endlich erzeugte abelsche Gruppen]
\label{Satz 3}
    Sei $A$ endlich erzeugte abelsche Gruppe.
    \[ \Ra A \cong \mathbb{Z}^r \oplus \bigoplus_{i=1}^m
    \mathbb{Z}/a_i\mathbb{Z}\]
    mit $a_1,\dots,a_m \in \mathbb{N}$, $\forall i: a_i \geq 2$, $a_i \mbox{
    teilt }
    a_{i+1}$ f�r $i=1,\dots,m-1$. Dabei sind $r,m$ und die $a_i$ eindeutig
    bestimmt.

    \sbew{1.0}{
        Sei $x_1,\dots,x_n$ ein Erzeugendensystem von A.\newline
        Nach \ref{1.18} gibt es einen surjektiven Gruppenhomomorphismus
        $\varphi: \mathbb{Z}^n \to A$ mit $\varphi(e_i) = x_i$, f�r
        $i=1,\dots,n$.\newline
        Nach Homomorphiesatz (Satz \ref{Satz 1}) ist dann $A \cong \mathbb{Z}^n /
        $Kern$(f)$. \newline
        Nach Satz \ref{Satz 2} gibt es $m \in \mathbb{N}, m \leq
        n$, eine Basis $\{z_1, \dots , z_n\}$ von $\mathbb{Z}^n$ und
        Elementarteiler $a_1, \dots, a_m$ mit $a_i$ teilt $a_{i+1}$ f�r $i = 1,
        \dots, m-1$, so dass $\{a_1z_1, \dots, a_mz_m\}$ Basis von Kern$(\varphi)$
        ist. Dann ist $A \cong \mathbb{Z}^n /$Kern$(\varphi) \cong \left(
        \displaystyle \bigoplus_{i=1}^m z_i \mathbb{Z} \right) / \left(
        \displaystyle \bigoplus_{i=1}^m a_iz_i \mathbb{Z} \right) \cong 
        \displaystyle \bigoplus_{i=1}^m \left( z_i \mathbb{Z} / a_i z_i
        \mathbb{Z} \right) \oplus \displaystyle \bigoplus_{i=m+1}^n z_i
        \mathbb{Z} \cong \displaystyle \bigoplus_{i=1}^m \mathbb{Z} / a_i
        \mathbb{Z} \oplus \mathbb{Z}^{n-m}$
        
        \textbf{Eindeutigkeit}:
            $r$ ist die maximale Anzahl linear unabh�ngiger Elemente in $A$. Sei
            dann $T \defeqr \displaystyle \bigoplus_{i=1}^m \mathbb{Z}/a_i\mathbb{Z} \cong
            \bigoplus_{j=1}^{m'}
            \mathbb{Z}/b_j\mathbb{Z} \defeqr T'$ mit $b_j$ teilt $b_{j+1}$ f�r $j
            =1,\dots,m-1$ . z.z.: $m' = m$ und $a_i =
            b_i$ f�r $i=1,\dots,m$. \smallskip \newline
            \textbf{Beh}.:
                F�r jedes $x \in T$ ist ord$(x)$ Teiler von $a_m$. Ebenso ist
                f�r jedes $y \in T'$ ord$(y)$ Teiler von $b_{m'}$. T enth�lt ein
                Element von Ordnung $a_m$, n�mlich $(\bar 0, \dots, \bar 0, \bar
                1) \Ra \mbox{auch} T'$ enth�lt ein Element der Ordnung $a_m \Ra
                a_m$ teilt $b_m'$. Umgekehrt: $b_{m'}$ teilt $a_m \Ra a_m =
                b_{m'}$. \newline
                Sei \[\begin{array}{cc} \bar T \defeqr T/(\mathbb{Z}/a_m
                \mathbb{Z}), & \tilde{T} \defeqr T/(\mathbb{Z}/b_m \mathbb{Z})
                \\ \cong   & \cong \\ \displaystyle \bigoplus_{i=1}^{m-1}
                \mathbb{Z}/a_i\mathbb{Z}   & \displaystyle 
                \bigoplus_{j=1}^{m'-1} 
                \mathbb{Z}/b_j\mathbb{Z} \end{array} \]
                Induktion �ber $m$: Eindeutigkeit gilt f�r $\bar{T} \Ra$ Satz.
            \newline
            \textbf{Bew. der Beh.:}
                Sei $x=(x_1,\dots,x_m) \in T$ mit $x_i \in \mathbb{Z}/a_i
                \mathbb{Z} \Ra a_m x = (a_m x_1, \dots, a_m x_m) = (0,\dots,0)$,
                da $a_i$ Teiler von $a_m$ ist, $i=1,\dots,m \Ra$ ord$(x)$ ist
                Teiler von $a_m \Ra r$ eindeutig
    }
\end{Satz}