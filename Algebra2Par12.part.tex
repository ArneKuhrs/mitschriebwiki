\section{Tensorprodukt}

\begin{Def}
  Seien $M, N, P$ $R$-Moduln.

  \begin{enumerate}
    \item Eine Abbildung $\Phi: M \times N \to P$ heißt
          \emp{$R$-bilinear}\index{R-bilinear}, wenn für jedes $x_0 \in M$ und
          jedes $y_0 \in N$ die Abbildungen 
          \[\Phi_{x_0}: N \to P, y \mapsto \Phi(x_0,y)\]
          \[\Phi_{y_0}: M \to P, x \mapsto \Phi(x,y_0)\] $R$-linear sind.
    \item Ein \emp{Tensorprodukt}\index{Tensorprodukt} von $M$ und $N$ (über $R$)
          ist ein $R$-Modul $T$ zusammen mit einer bilinearen Abbildung $\tau: M
          \times N \to T$, sodass\\
          (UAE) Für jede bilineare Abbildung $\Phi: M \times N \to P$ gibt es
          genau eine lineare Abbildung $\varphi: T \to P$ mit $\Phi = \varphi \circ
          \tau$
          \[
            \begin{xy}
              \xymatrix{
                M \times N \ar[rr]^{\tau} \ar[rd]_{\Phi}  &     &  T \ar@{-->}[dl]^{\exists!\varphi}  \\
                                                          &  P  &
              }
            \end{xy}
          \]
          ($\tau$ ist die ''universelle'' bilineare Abbildung)
  \end{enumerate}
\end{Def}

\begin{nnBsp}
  \begin{enumerate}
    \item[1.)] $M, N$ freie $R$-Moduln mit Basis $\{e_i, i \in I\} bzw. \{f_j, j
               \in J\}$. Dann ist $M \ten[R] N$ freier $R$-Modul mit Basis
               $\{e_i\ten f_j, i \in I, j \in J\}$ ein Tensorprodukt mit
               $\tau(e_i,f_j) = e_i\ten f_j$.\\
               Denn: Sei $\Phi: M \times N \to P$ bilinear. Setze
               $\varphi(e_i \ten f_j) \defeqr \Phi(e_i,f_j$), das bestimmt eindeutig
               $\varphi: M \ten[R] N \to P$ ($R$-linear) mit $\Phi(e_i,f_j) =
               \varphi(\tau(e_i,f_j))$ für alle $i,j$.\\
               Sind $I, J$ endlich, so ist rg$(M \ten[R] N) = \mbox{rg}(M) \cdot
               \mbox{rg}(N)$, dagegen ist rg$(M \times N) = \mbox{rg}(M) +
               \mbox{rg}(N)$. $\tau$ ist also höchstens in Trivialfällen
               surjektiv. $\tau$ ist nicht injektiv: $\tau(x,0) = \tau(x,0 \cdot
               y) = 0 \cdot \tau(x,y) = 0$ (da linear im 2. Argument), genauso
               $\tau(0,y) = 0$. Bild$(\tau)$ ist kein Untermodul, aber $\langle
               \mbox{Bild}(\tau) \rangle = M \ten[R] N$.
    \item[2.)] $0$ ist ein Tensorprodukt der $\mathbb{Z}$-Moduln
               $\mathbb{Z}/2\mathbb{Z}$ und $\mathbb{Z}/3\mathbb{Z}$.\\
               Denn: jede bilineare Abbildung $\Phi: \mathbb{Z}/2\mathbb{Z}
               \times \mathbb{Z}/3\mathbb{Z} \to P$ ist die Nullabbildung.
               $\Phi(\bar{1},\bar{1}) = \Phi(3 \cdot \bar{1},\bar{1}) = 3 \cdot
               \Phi(\bar{1},\bar{1}) = \Phi(\bar{1},3 \cdot \bar{1}) =
               \Phi(\bar{1},\bar{0})= 0$, genauso $\Phi(\bar{1},-\bar{1}) = 0$.
  \end{enumerate}
\end{nnBsp}

\begin{Satz}
  Zu je zwei $R$-Moduln $M,N$ gibt es ein Tensorprodukt. Dieses ist eindeutig
  bestimmt bis auf eindeutigen Isomorphismus.
\end{Satz}

\begin{anBew}
  Sei F der freie $R$-Modul mit Basis $M \times N$.
  Sei $Q$ Untermodul, der erzeugt wird von den \[(x + x',y) - (x,y) - (x',y),
  (\alpha x,y) - \alpha (x,y)\] \[(x,y + y') - (x,y) - (x,y'),
  (x,\alpha y) - \alpha (x,y)\] für alle $x,x' \in M,\; y,y' \in N,\;\alpha \in
  R$\\
  Setze $T \defeqr F/Q$, $\tau: M \times N \to T, (x,y) \mapsto [(x,y)]
  \mbox{ mod } Q$. $\tau$ ist bilinear nach Konstruktion.
  Ist $\Phi: M \times N \to P$ bilinear, so setze $\tilde{\varphi}((x,y))
  \defeqr \Phi(x,y)$, $\tilde{\varphi}: F \to P$ ist linear. $Q \subseteq
  \mbox{Kern}(\tilde{\varphi})$, weil $\Phi$ bilinear $\overset{Hom.-Satz}{\Rightarrow}
  \tilde{\varphi}$ induziert $\varphi: T \to P$ mit $\Phi = \varphi \circ \tau$.\\
  Noch zu zeigen: Eindeutigkeit\\
  Seien $(T, \tau)$, $(T', \tau')$ Tensorprodukte von $M$ und $N$. Dann gibt es eine $R$-lineare 
  Abbildung $\varphi: T \rightarrow T'$ mit $\tau'= \varphi \circ \tau$
  \[\begin{xy}
      \xymatrix{
                                                  & T \ar@/_/[dd]_{\varphi} \\
        M \times N \ar[ur]^{\tau} \ar[dr]_{\tau'} &                     \\
                                                  & T' \ar@/_/[uu]_{\psi}
      }
    \end{xy}\]
  und $\psi: T' \rightarrow T$ mit $\tau = \psi \circ\tau'$\\
  Behauptung: $\psi \circ \varphi = id_T$ und $\varphi \circ \psi = id_{T'}$. Dazu:
  \[\begin{xy}
      \xymatrix{
                                                 & T \ar@/_/[dd]_{id}
                                                 \ar@/^/[dd]^{\psi \circ \varphi} \\
        M \times N \ar[ur]^{\tau} \ar[dr]_{\tau} &                     \\
                                                 & T
      }
    \end{xy}\]
  ist kommutativ, d. h. 
  $(\psi \circ \varphi ) \circ \tau = \psi \circ ( \varphi \circ \tau) = \psi \circ \tau' = \tau$
  mit $id: T \rightarrow T$ ist das Diagramm auch kommutativ. Wegen der Eindeutigkeit in der 
  Definition des Tensorprodukts muss gelten: $\psi \circ \varphi = id_T$
  ($\varphi \circ \psi = id_{T'}$ analog )
\end{anBew}

\begin{Bem}
  Für alle $R$-Moduln $M, N, M_1, M_2, M_3$ gilt:
  \begin{enumerate}
  	\item $M \ten[R] R \cong M$
  	\item $M \ten[R] N \cong N \ten[R] M$
  	\item $\underbrace{(M_1 \ten[R] M_2) \ten[R] M_3}_{=: \mbox{rechts}} \cong 
	\underbrace{M_1 \ten[R] ( M_2 \ten[R] M_3)}_{=: \mbox{links}}$
  \end{enumerate}
\end{Bem}
\begin{Bew}
  \begin{enumerate}
    \item[a)] Zeige: $M$ ist Tensorprodukt der $R$-Moduln $M$ und $R$.\\
	  $\tau: M \times R \rightarrow M$, $(x,a) \rightarrow a \cdot x$ ist bilinear ( wegen 
	  Moduleigenschaften ). Sei $\Phi: M \times R \rightarrow P$ bilinear\\
	  Gesucht: $\varphi: M \rightarrow P$ linear mit $\Phi = \varphi \circ \tau$, d. h. 
	  $\Phi(x,a) = \varphi(a \cdot x)$\\
	  Setze $\varphi(x) := \Phi(x,1)$ $\varphi$ ist $R$-linear, da $\Phi( \cdot, 1)$ linear
	  ist, $\Phi(x,a) = a\Phi(x,1) = a\varphi(x) = \varphi(a \cdot x ) = \varphi(\tau(x,a))$\\
	  $\varphi$ ist eindeutig: es muss gelten: $\varphi(\tau(x,1)) = \Phi(x,1) =: \varphi(x)$, 
	  damit ist $\varphi$ eindeutig bestimmt ( wegen $\varphi \circ \tau = \Phi$ ).
    \item[b)] $M \times N \cong N \times M$
    \item[c)] Finde lineare Abbildung: rechts $\rightarrow$ links
	  \begin{enumerate}
	    \item[ 1. ] Für festes $x \in M_3$ ist $\Phi_z: M_1 \times M_2 \rightarrow \mbox{links}$, 
		  $(x,y) \rightarrow x \ten (y \ten z) := \tau ( y,z )$\\
		  $\Phi_z$ bilinear: klar\\
		  $\Phi_z$ induziert eine lineare Abbildung: $\varphi_z: M_1 \ten[R] M_2 \rightarrow \mbox{links}$
		  Weiter ist $\Psi: ( M_1 \ten[R] M_2 ) \times M_3 \rightarrow \mbox{links}$,
		  $(w,z) \rightarrow \varphi_z (w)$\\
		  bilinear: linear in $w$, weil $\varphi_z$ linear; linear in $z$ weil $\Phi_z$ bilinear.\\
		  Induziert also lineare Abbildung $\Psi: \mbox{rechts} \rightarrow \mbox{links}$
	    \item[ 2. ] Umkehrabbildung genauso!
	  \end{enumerate}
  \end{enumerate}
\end{Bew}

\begin{Prop}
  Sei $M$ ein $R$-Modul, $I \subseteq R$ ein Ideal. Dann ist $I \cdot M = 
  \{ a \cdot x \in M: x \in M, a \in I \}$ Untermodul von $M$ und es gilt: \\
  $M/IM \cong M \ten[R] R/I$
\end{Prop}

\begin{Bew}
  Sei $\tilde{\varphi}: M \rightarrow M \ten[R] R/I$, 
  $ x \mapsto x \ten \overline 1$\\
  $\tilde{\varphi}$  ist $R$-linear.\\
  $I \cdot M \subseteq \mbox{Kern}(\tilde{\varphi}): \forall a \in I, x \in M$ ist
  $\tilde{\varphi}(ax) = ax \ten \overline 1 = x \ten \ub{ a \cdot \overline 1
  }{\overline a} = 0$\\
  $\tilde{\varphi}$ induziert also lineare Abbildung\\
  $\varphi: M/_{I \cdot M} \rightarrow M \ten[R] R/I\\
  Umgekehrt: \Psi: M \times R/I \rightarrow M /_{I \cdot M}$, $(x, \overline a) \rightarrow \overline{ax}$ \\
  $\Psi$ ist wohldefiniert: ist $\overline b = \overline a$, so ist $\overline{b \cdot x} - \overline{a \cdot x} =
  \overline{ \underbrace {( \underbrace{b-a}_{\in I}) \cdot x}_{ \in I \cdot M}} =0$\\
  $\Psi$ ist bilinear, induziert also $\psi: M \ten[R] R/I \to M/IM$ (linear). Es ist 
  $(\psi \circ \varphi)(\overline x )= \psi(x \ten \overline 1 ) =
  \overline{1x} = \overline x$ und
  $(\varphi \circ \psi)(x \ten \overline a ) = \varphi(\overline{a \cdot x }) =
  ax \ten \overline 1 = x \ten a\cdot \overline 1 = x \ten \overline a$.
\end{Bew}
