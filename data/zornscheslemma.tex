\documentclass{article}
\newcounter{chapter}
%\setcounter{chapter}{20}
\usepackage{ana}
\usepackage{mathrsfs}

\title{Einschub: Das Zornsche Lemma}
\author{Ferdinand Szekeresch}
% Wer nennenswerte Änderungen macht, schreibt sich bei \author dazu

\begin{document}

\maketitle

Es sei $\emptyset \neq \mathcal{L}$ eine Menge und $\triangleleft$ eine \begriff{Ordnungsrelation} auf $\mathcal{L}$, d.h. für $a, b, c \in \mathcal{L}$ gilt:
\begin{liste}
\item $a \triangleleft a$ \\
\item aus $a \triangleleft b$ und $b \triangleleft a \folgt a = b$ \\
\item aus $a \triangleleft b$ und $b \triangleleft c \folgt a \triangleleft c$ \\
\end{liste}

Es sei $\emptyset \neq \mathcal{K} \subseteq \mathcal{L}$. $\mathcal{K}$ heißt eine \begriff{Kette} $:\equizu$ aus $a, b \in \mathcal{K}$ folgt stets: $a \triangleleft b$ oder $b \triangleleft a$. Sei $\mathcal{M} \subseteq \mathcal{L}$ und $a \in \mathcal{L}$. $a$ heißt eine \begriff{obere Schranke} von $\mathcal{M} :\equizu x \triangleleft a \; \forall x \in \mathcal{M}$. $v \in \mathcal{L}$ heißt ein \begriff{maximales Element} von $\mathcal{L} :\equizu$ aus $a \in \mathcal{L}$ und $v \triangleleft a$ folgt: $v = a$

\begin{lemma}[\textbf{Das Zornsche Lemma}]
$\mathcal{L}$ und $\triangleleft$ seien wie oben. Besitzt \textbf{jede} Kette in $\mathcal{L}$ eine obere Schranke in $\mathcal{L}$, so enthält $\mathcal{L}$ ein maximales Element.
\end{lemma}


\end{document}
