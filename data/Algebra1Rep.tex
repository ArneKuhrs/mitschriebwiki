\documentclass[a4paper,10pt,german]{scrbook}

\usepackage[T1]{fontenc}
\usepackage{babel}
\usepackage{cmbright}

\usepackage[ansinew]{inputenc}

\usepackage{stmaryrd}
%\usepackage{ulsy}
%blitztest
\newcommand{\blitza}[0]{\lightning}
\newcommand{\blitzb}[0]{\lightning}
\newcommand{\blitzc}[0]{\lightning}
\newcommand{\blitzd}[0]{\lightning}

%erstmal auskommentiert, leider kann das amscd-Paket nicht so viel :/
%\usepackage{pictexwd,dcpic}

\usepackage{colortbl}
\usepackage{xcolor}

\usepackage{hyperref}

% Mathe-Pakete
\usepackage{amssymb}
\usepackage{amsmath}
\usepackage{amsfonts}
\usepackage{amsthm}

%\usepackage{algebra}



\newtheoremstyle{s�tze}% Name des Stils
{3pt}% vertikaler Abstand zum vorangehenden Text
{3pt}% vertikaler Abstand zum folgenden Text
{}% Schriftart des Textk�rpers
{}% Abstand des Erstzeileneinzugs der Kopfzeile
{\Large \bfseries}% Schriftart des Kopfes
{}% Punktierung nach dem Kopf
{\newline}% Abstand nach dem Kopf (z.B. \newline)
{}% Kopfspezifikation (leer bedeutet 'normal')

\newtheoremstyle{definitionen}% Name des Stils
{3pt}% vertikaler Abstand zum vorangehenden Text
{3pt}% vertikaler Abstand zum folgenden Text
{}% Schriftart des Textk�rpers
{}% Abstand des Erstzeileneinzugs der Kopfzeile
{\large \bfseries}% Schriftart des Kopfes
{}% Punktierung nach dem Kopf
{\newline}% Abstand nach dem Kopf (z.B. \newline)
{}% Kopfspezifikation (leer bedeutet 'normal')

\theoremstyle{s�tze}
    \newtheorem{Satz}{Satz}
    
\theoremstyle{definitionen}
    \newtheorem{Def}{Definition}[chapter]
    \newtheorem{DefBem}[Def]{Definition + Bemerkung}
    \newtheorem{Bem}[Def]{Bemerkung}
    \newtheorem{BemDef}[Def]{Bemerkung + Definition}
    \newtheorem{Prop}[Def]{Proposition}
    \newtheorem{PropDef}[Def]{Proposition + Definition}
    \newtheorem{Folg}[Def]{Folgerung}
    \newtheorem{Bsp}[Def]{Beispiele}
    \newtheorem{DefProp}[Def]{Definition + Proposition}

\title{Algebra I - Wintersemester 05/06 - Zusammenfassung}
\author{Die Autoren}

\begin{document}


\setkomafont{sectioning}{\normalfont\normalcolor\bfseries}
\setkomafont{descriptionlabel}{\normalfont\normalcolor\bfseries}

%\renewcommand*{\othersectionlevelsformat}[1]{\llap{\csname the#1 \endcsname\autodot\enskip}}

%\newcommand{\ssubsection}[1]{\subsection{#1 \thesubsection} \label{\thesubsection}}

\newcommand{\emp}[1]{\textbf{\emph{#1}}}

\newenvironment{define}
    { \begin{flushleft} \begin{description} }
    { \end{description} \end{flushleft} }

\newenvironment{enum} {
      \begin{enumerate}
      \renewcommand{\labelenumi}{(\alph{enumi})}

      }
      { \end{enumerate} }

\newcommand{\bla}[0]
{\begin{tiny} $\left\{
\begin{array}{l}
             . \\
             . \\
             . \end{array}
        \right\}$
\end{tiny}}

\newcommand{\blab}[0]
{\begin{tiny} $\left\{
\begin{array}{l}
             Halbgruppen \\
             Monoiden \\
             Gruppen \end{array}
        \right\}$
\end{tiny}}

\newcommand{\bew}[2]
{\noindent\fcolorbox{white}{blue!7!white}{
 \begin{minipage}{1.0\textwidth}
 \textit{\textbf{Beweis: }}
 #1
 \begin{enum} #2
 \hfill \rule{2,1mm}{2,1mm} \end{enum}  \end{minipage} }
}

\newcommand{\sbew}[2]
{\noindent \fcolorbox{white}{blue!7!white}{
 \begin{minipage}{#1\textwidth}
 \textit{\textbf{Beweis: }} #2
 \hfill \rule{2,1mm}{2,1mm} \end{minipage} }
}

\newcommand{\bsp}[1]
{\noindent \fcolorbox{white}{green!12!white}{
 \begin{minipage}{1.0\textwidth}
 \textit{\textbf{Beispiel: }}
 #1
 \hfill \end{minipage} }
}

\newcommand{\sbsp}[2]
{\noindent \fcolorbox{white}{green!12!white}{
 \begin{minipage}{#1\textwidth}
 \textit{\textbf{Beispiel: }} #2
 \hfill \end{minipage} }
}

\newcommand{\ra}[0]{\rightarrow}
\newcommand{\ds}[0]{\displaystyle}
\newcommand{\cd}[0]{\cdot}
\newcommand{\lra}[0]{\Leftrightarrow}
\newcommand{\Ra}[0]{\Rightarrow}

\newcommand{\para}[1]{\noindent \Large\textbf{#1} \normalsize}

\newcommand{\defeqr}[0]{\mathrel{\mathop:}=}
\newcommand{\defeql}[0]{=\mathrel{\mathop:}}

\newcommand{\chk}[0]{\checkmark}
\newcommand{\wt}[0]{\widetilde}

\maketitle

\chapter{Gruppen}

\section{Grundlagen}

\section{Homomorphie- und Isomorphies\"atze}

%!!! Homomorphies\"atze

Sind $G$ und $G'$ Gruppen und $\varphi:G\longrightarrow G'$ ein Gruppenhomomorphismus.
Dann gilt:
$$G/\textrm{Kern}(\varphi)\cong \textrm{Bild}(\varphi)$$

\begin{Bsp}
\begin{enum}
\item $G/Z(G)\cong \textrm{Aut}_i(G)$
\end{enum}
\end{Bsp}

\begin{Satz}
Sei $G$ eine Gruppe, $N\subseteq G$ ein Normalteiler und $H\subseteq G$ eine Untergruppe.
\begin{enum}
\item Es gilt:
$$ H/(H\cap N) \cong HN/N$$
Dabei sei $HN:=\{h\cd n:h\in H, n\in N\}$
\item Ist $N\subseteq H$ und $H$ ein Normalteiler in $G$, so gilt:
$$(G/N)/(H/N)\cong G/H$$
\end{enum}
\end{Satz}

\subsection{Exakte Sequenzen}

\begin{DefBem}
\begin{enumerate}
\item Eine Sequenz
$$1\longrightarrow G_0\longrightarrow \cdots \longrightarrow G_{i-1} \overset{\varphi_i}{\longrightarrow} G_{i} 
\overset{\varphi_{i+1}}{\longrightarrow} G_{i+1}\longrightarrow \cdots \longrightarrow G_{n}\longrightarrow 1$$ von Gruppenhomomorphismen hei\ss t
\emph{exakt an der Stelle $G_i$}, wenn 
$$\textrm{Bild}(\varphi_{i})=\textrm{Kern}(\varphi_{i+1})$$ gilt. Die Sequenz hei\ss t \emph{exakt}, 
wenn sie an jeder Stelle exakt ist.

\item Eine Sequenz
$$1\longrightarrow G' \overset{\alpha}{\longrightarrow} G \overset{\beta}{\longrightarrow} G''\longrightarrow 1$$
von Gruppenhomorphismen hei\ss t \emph{kurze Sequenz}. 

\item Eine kurze Sequenz ist genau dann exakt, wenn
die folgenden Bedingungen erf\"ullt sind:
\begin{enumerate}
\item $\alpha$ ist injektiv (sprich: man kann $G'$ als Untergruppe von $G$ auffassen)
\item $\beta$ ist surjektiv
\item $\textrm{Bild}(\alpha)=\textrm{Kern}(\beta)$
\end{enumerate}
\end{enumerate}
\end{DefBem}

\begin{Bsp}
Ist $G$ eine Gruppe und $N\trianglelefteq G$, so ist die folgende kurze Sequenz exakt:
$$ 1\longrightarrow N \longrightarrow G \longrightarrow G/N\longrightarrow 1$$ %???
\end{Bsp}

Eine kurze exakte Sequenz \emph{spaltet}, wenn es einen Gruppenhomomorphismus 
$\gamma: G''\longrightarrow G$ gibt mit $\beta\circ\gamma=\textrm{id}_{G''}$.
$$\begin{array}{ccc}
1\longrightarrow G' \overset{\alpha}{\longrightarrow} G &\overset{\beta}{\longrightarrow} &G''\longrightarrow 1\\
&\overset{\gamma}{\longleftarrow}&\\
\end{array}$$

In diesem Fall ist $\gamma$ injektiv, man kann also auch $G''$ als Untergruppe von $G$ auffassen.

\section{Kommutatoren}

\begin{Bem} Grundlegende Eigenschaften von Kommutatoren.
\begin{enumerate}
\item Genau dann ist eine Gruppe $G$ abelsch, wenn $K(G)=\{e\}$.
\end{enumerate}
\end{Bem}

%% Wie sehen die Elemente aus $K(G)$ aus? !!!

\begin{Bem} Kommutatoren und Homomorphismen.

Seien $G$ und $G'$ eine Gruppen und $\varphi:G\rightarrow G'$ ein Gruppenhomomorphismus. 
\begin{enum}
\item $\varphi([a,b]) = [\varphi(a), \varphi(b)]$.
\item $\varphi(K(G))\subseteq K(G')$
\item Ist $\varphi$ zudem noch surjektiv, so gilt: $\varphi(K(G))=K(G')=K(\varphi(G))$.
\end{enum}
\bew{}{
%\begin{enum}
\item $\varphi([a,b])=\varphi(aba^{-1}b^{-1})=\varphi(a)\varphi(b)\varphi(a)^{-1}\varphi(b)^{-1}=[\varphi(a), \varphi(b)]$
\item Sei $[a,b]\in K(G)$. Dann ist $\varphi([a,b])=[\varphi(a), \varphi(b)]\in K(G')$
\item Sei $[a', b']\in K(G')$. Da $\varphi$ surjektiv ist, gibt es $a, b\in G$ mit $\varphi(a)=a'$ und
$\varphi(b)=b'$. Dann gilt $[a', b']=[\varphi(a), \varphi(b)]=\varphi([a,b])\in \varphi(K(G))$.
%\end{enum}
}
\end{Bem}

\begin{Bem} Kommutatoren und Normalteiler.

Es sei $G$ eine Gruppe.
\begin{enum} 
\item Es sei $N\trianglelefteq G$.\\
Genau dann ist $G/N$ abelsch, wenn $K(G)\subseteq N$.
\item $G^{ab}:=G/K(G)$ ist eine abelsche Gruppe.
\end{enum}

\bew{}{
\item
\begin{description}
\item[$\Rightarrow$] Es sei $G$ abelsch. Also: $K(G)=\{e\}\subseteq N$.
\item[$\Leftarrow$] Es sei $K(G)\subseteq N$ und $\pi: G\rightarrow G/N$ die kanonische Projektion.
Da $\pi$ surjektiv ist, gilt: $\pi(K(G)) = K(\pi(G)) = K(G/N)$. Da $K(G)\subseteq N$, ist $K(G/N)=\{N\}$.
Also ist $G/N$ abelsch.
\end{description}
\item Blatt 3, Aufgabe 1, a).
\item Blatt 3, Aufgabe 1, b).
%\end{enum}
}

\end{Bem}

\begin{Bsp}
\begin{enumerate}
\item Symmetrische Gruppe:
    \begin{enumerate}
    \item $K(S_1)= $ %!!!
    \item $K(S_n)=A_n$ (f\"ur $n\geq 2$)
    \end{enumerate}
\item Alternierende Gruppe:
    \begin{enumerate}
    \item $K(A_2)=K(A_3)=\{\textrm{id}\}$
    \item $K(A_4)=V_4=\mathbb{Z}/2\mathbb{Z}\times\mathbb{Z}/2\mathbb{Z}$ (kleinsche Vierergruppe)
    \item $K(A_n)=A_n$ (f\"ur $n\geq 5$)
    \end{enumerate}
\item Diedergruppe.
\end{enumerate}
\end{Bsp}

\section{Konstruktion von Gruppen}

\subsection{Direktes Produkt}

\subsection{Semidirektes Produkt}

Seien $H, N$ Gruppen und $\phi:H\rightarrow \textrm{Aut}(N)$ ein Gruppenhomomorphismus.
Auf der Menge $G:=N\times H$ definiert man eine Verkn\"upfung $\star$ wie folgt:

$$ (n_1, h_1)\star(n_2, h_2) := (n_1\phi(h_1)(n_2), h_1h_2),$$

wobei jeweils die Verkn\"upfungen in $N$ bzw. $H$ verwendet werden.

$(G, \star)$ hei\ss t \emph{semidirektes Produkt} von $H$ mit $N$.

$G$ ist eine Gruppe, die $N\times \{e_H\}$ als Normalteiler und $\{e_N\}\times H$ als Untergruppe enth\"ahlt.

\begin{Bem} (Splitting Lemma)\\
Sei

$$ 1\longrightarrow G' \overset{\alpha}{\longrightarrow} G \overset{\beta}{\longrightarrow} G'' \longrightarrow 1$$

eine kurze exakte Sequenz, die spaltet. Das bedeutet, dass es einen Gruppenhomomorphismus $\gamma:G''\rightarrow G$
gibt mit $\beta\circ\gamma=\textrm{id}_{G''}$

$G$ ist dann bez\"uglich einer geeigneten Abbildung $\varphi:G''\longrightarrow \textrm{Aut}(G')$ ein semidirektes
Produkt von $G'$ und $G''$.

\end{Bem}

Setze
$$\varphi(h)(n):=\alpha^{-1}(\gamma(h)\alpha(n)\gamma(h^{-1}))$$

Da $\alpha$ und $\gamma$ injektiv sind, kann man sich $G'$ ung $G''$ als Untergruppe von $G$ vorstellen.
In diesem Fall ergibt sich
$$\varphi(h)(n):=hnh^{-1}$$


\section{Eigenschaften von Gruppen }

\subsection{Zyklische Gruppen}

\begin{DefBem}
    \begin{enum}
    \item $G$ hei\ss t zyklisch, wenn es ein $g\in G$ gibt mit $G=\langle g\rangle$.
    \end{enum}
\end{DefBem}

\subsection{Abelsche Gruppen}

\begin{DefBem}
    \begin{enum}
        \item $A$ hei\ss t \emp{freie abelsche Gruppe} mit Basis $X$, wenn jedes $a
        \in A$ eine eindeutige Darstellung $\ds a = \sum_{x\in X} n_x x$ hat mit
        $n_x \in \mathbb{Z}\;, n_x \neq 0$ nur f\"ur endlich viele $x \in X$. Ist
        in dieser Situation $|X| = n$, so hei\ss t $n$ der \emp{Rang} von $A$. $A$
        ist isomorph zu $\ds \mathbb{Z}^X \defeqr \bigoplus_{x \in X}
        \mathbb{Z}$
        \item (UAE der freien abelschen Gruppe) \newline
        Zu jeder abelschen Gruppe $A$ und jeder Abbildung $f:X \ra A$ gibt es
        genau einen Homomorphismus $\varphi: \mathbb{Z}^X \ra A$ mit $\forall x
        \in X: \varphi(x) = f(x)$
    \end{enum}

\end{DefBem}

\begin{Bsp}
        $X$ endlich, $X=\{x_1, \dots, x_n\}$. Dann ist $\mathbb{Z}^X
        \cong \mathbb{Z}^n\\$ $\mathbb{Z}^n$ ist ''so etwas \"ahnliches'' wie ein
        Vektorraum \textit{(''freier Modul'')}. Insbesondere lassen sich die
        Gruppenhomomorphismen $\mathbb{Z}^n \ra \mathbb{Z}^m$ durch eine $m
        \times n$-Matrix mit Eintr\"agen in $\mathbb{Z}$ beschreiben.
\end{Bsp}

\begin{Satz}[Elementarteilersatz]
    Sei $H$ eine Untergruppe von $\mathbb{Z}^n$ $(n \in \mathbb{N} \setminus
    \{0\})$. Dann gibt es eine Basis $\{x_1, \dots, x_n\}$ von $\mathbb{Z}^n$,
    ein $r \in \mathbb{N}$ mit $0 \leq r \leq n$ und $a_1, \dots, a_r \in 
    \mathbb{N} \setminus \{0\}$ mit $a_i$ teilt $a_{i+1}$ f\"r $i = 1,\dots,r-1$,
    so da\ss\ $a_1 x_1, \dots, a_r x_r$ eine Basis von $H$ ist. Insbesondere ist
    $H$ ebenfalls eine freie abelsche Gruppe.
\end{Satz}

Klassifizierung:

\begin{Satz}[Struktursatz f\"ur endlich erzeugte abelsche Gruppen]
    Sei $A$ endlich erzeugte abelsche Gruppe.
    \[ \Ra A \cong \mathbb{Z}^r \oplus \bigoplus_{i=1}^m
    \mathbb{Z}/a_i\mathbb{Z}\]
    mit $a_1,\dots,a_m \in \mathbb{N}$, $\forall i: a_i \geq 2$, $a_i \mbox{
    teilt }
    a_{i+1}$ f�r $i=1,\dots,m-1$. Dabei sind $r,m$ und die $a_i$ eindeutig
    bestimmt.
\end{Satz}


Abgeschlossenheit:
\begin{enumerate}
\item Untergruppen abelscher Gruppen sind abelsch.
\item Faktorgruppen abelscher Gruppen sind abelsch.
\item Produkte abelscher Gruppen sind abelsch.
\item Direkte Summen abelscher Gruppen sind abelsch.
\item Seien $G, G'$ Gruppen, $\varphi:G\longrightarrow G'$ ein Gruppenhomomorphismus.\\
Ist $G$ abelsch, so ist $\varphi(G)$  auch abelsch.
\end{enumerate}

Beispiele f\"ur abelsche Gruppen:
\begin{itemize}
\item zyklische Gruppen
\item Gruppen der Ordnung $p$ oder $p^2$
\item $\textrm{Aut}(G)$ ist zyklisch
\item $G=H/[H,H]$
\item F\"ur alle $x\in G$ gilt $x^2=e$.
\end{itemize}

Beispiele f\"ur \emph{nicht} abelsche Gruppen:
\begin{itemize}
\item $D_n$
\item $S_n$ (f\"ur $n\geq 3$)
\item $A_n$ (f\"ur $n\geq 4$)
\end{itemize}

\subsection{Einfache Gruppen}

\begin{Bsp}
\begin{enum}
\item Es gibt keine einfachen Gruppen der Ordnung $21$.

\sbew{0.9}{
  Die S\"atze von Sylow liefern, dass es nur eine 7-Sylowgruppe gibt.
  Diese muss also auch Normalteiler sein.
}
\item Es gibt keine einfachen Gruppen der Ordnung $30$.

\sbew{0.9}{
  Es sei $G$ eine Gruppe der Ordnung $30$.
  Die S\"atze von Sylow liefern $s_3\in\{1, 10\}$ und $s_5\in\{1, 6\}$.
  Falls $s_3=1$ oder $s_5=1$ gilt, so gibt es nach dem vorigen Argument
  einen Normalteiler in $G$. Es gelte also im folgenden $s_3=10$ und $s_5=6$.
  Die $5$-Sylowgruppen sind zyklisch und bis auf das Neutralelement disjunkt.
  In den $5$-Sylowgruppen liegen also $6\cdot 4=25$ Elemente ($\ne e_G$).
  Die $3$-Sylowgruppen sind zyklisch und bis auf das Neutralelement diskunkt.
  In den $3$-Sylowgruppen liegen also $10\cdot 2=20$ Elemente ($\ne e_G$).
  Je eine $3$-Sylowgruppe und eine $5$-Sylowgruppe schneiden sich trivial.
  Es gibt also mindestens $25+20+1=46$ Elemente in $G$. Widerspruch.
}
\item Es gibt keine einfachen Gruppen der Ordnung $36$.

\sbew{0.9}{
  Es sei $G$ eine Gruppe der Ordnung $36$.
  Die S\"atze von Sylow liefern $s_2\in\{1,3,9\}$ und $s_3\in\{1,4\}$.
  Ohne Einschr\"ankung gelte $s_3=4$. Je 2 $3$-Sylowgruppen sind
  konjugiert, deshalb operiert $G$ auf der Menge $M$ der $3$-Sylowgruppen
  durch Konjugation (nichttrivial). Nenne diese $3$-Sylowgruppen
  $M=\{1,2,3,4\}$. Man erh\"ahlt durch diese Operation einen Gruppenhomomorphismus
  $\varphi: G\ra \textrm{Perm}(M)=S_4$. $\varphi$ ist nicht injektiv, da $|G|=36$,
  $|S_4|=24$. $\varphi$ ist nicht der triviale Homomorphismus, da $G$ nichttrivial auf $M$
  operiert. $\textrm{Kern}(\varphi)$ ist also ein echter, nichttrivialer Normalteiler in $G$.
}
\item Es gibt keine einfachen Gruppen der Ordnung $300$.

\sbew{0.9}{
  Es sei $G$ eine Gruppe der Ordnung $300$.
  Die S\"atze von Sylow liefern $s_2\in\{1, 3, 5, 15, 25, 75\}$,
  $s_3\in\{1, 4, 10, 25, 100\}$ und $s_5\in\{1, 6\}$. Ohne Einschr\"ankung gelte
  $s_5=6$. Je 2 $5$-Sylowgruppen sind konjugiert, deshalb operiert $G$ auf der Menge $M$
  der $5$-Sylowgruppen durch Konjugation (nichttrivial). Nenne diese $5$-Sylowgruppen
  $M=\{1,2,3,4,5,6\}$. Man erh\"ahlt durch diese Operation einen Gruppenhomomorphismus
  $\varphi: G\ra\textrm{Perm}(M)=S_6$. $|G|=300$ ist kein Teiler von $|S_6|=720$, also
  ist $\varphi$ nicht injektiv. 
  $\varphi$ ist nicht der triviale Homomorphismus, da
  $G$ nichttrivial auf $M$ operiert. $\textrm{Kern}(\varphi)$ ist also ein echter,
  nichttrivialer Normalteiler in $G$.
}
\item Gruppen der Ordnung $2m$ ($m$ ungerade) enthalten einen Normalteiler der Ordnung $m$.
Hinweis: Satz von Cayley. Zeige, dass eine Untergruppe der $S_n$, die eine ungerade Permutation enth\"alt, einen Normalteiler
von Index $2$ besitzt (Isomorphies\"atze).

\sbew{0.9}{
  Es sei $U$ eine Untergruppe der $S_n$, $\sigma\in U\setminus A_n$ (d.h. $\sigma$ ungerade).
  $A_n$ ist ein Normalteiler in $S_n$, $U$ ist eine Untergruppe in $S_n$, also ist nach den
  Isomorphies\"atzen $UA_n$ eine Untergruppe von $S_n$ und $U\cap A_n$ ein Normalteiler in $U$.
  Weiter gilt: $U/(U\cap A_n)\cong UA_n/A_n$.
  Andererseits ist $UA_n\lneq A_n\leq S_n$. Da $(S_n:A_n)=2$ muss also $UA_n=S_n$ gelten.
  Einsetzen: $U/(U\cap A_n)\cong S_n/A_n$. Insbesondere: $(U:(U\cap A_n))=(S_n:A_n)=2$

  \vspace{0.3cm}

  Zu der eigentlichen Aussage: Sei $G$ eine Gruppe der Ordnung $2m$, $m$ ungerade.

  Nach dem Satz von Cayley ist $\tau: G\ra \textrm{Perm}(G),\; g\mapsto \tau_g$ ein injektiver
  Homomorphismus ($\tau_g$: Konjugation mit $g$). Nummeriert man die Elemente von $G$ durch,
  so kann man den Homomorphismus auch als $\tau:G\ra S_{2m}$ auffassen. Da $\tau$ injektiv ist,
  ist $U:=\tau(G)$ eine Untergruppe von $S_n$ und $U\cong G$. In $G$ gibt es ein Element
  der Ordnung $2$ (Sylow), in $U$ also auch. Es sei also $\sigma\in U$ mit $\textrm{ord}(\sigma)=2$.

  to be continued
  %% http://www-m11.ma.tum.de/lehre/ws0506/alg1/lsg/lsgblatt05.pdf
}
\end{enum}
\end{Bsp}


\subsection{Aufl\"osbare Gruppen}

\begin{DefBem}

\begin{enum}

\item Eine Gruppe hei\ss t \emph{aufl\"osbar}, wenn sie eine Normalreihe mit abelschen
Faktorgruppen besitzt.
\item Eine endliche Gruppe ist genau dann aufl\"osbar, wenn die Faktoren in ihrer 
Kompositionsreihe zyklisch von Primzahlordnung sind.
\item Sei $$1\longrightarrow G'\longrightarrow G \longrightarrow G''\longrightarrow 1$$
eine kurze exakte Sequenz von Gruppen. Dann gilt:\\
$G$ ist aufl\"osbar $\Leftrightarrow$ $G'$ und $G''$ sind aufl\"osbar.\\
Ist $N$ ein Normalteiler in $G$, so gilt also:\\
$G$ ist aufl\"osbar $\Leftrightarrow$ $N$ und $G/N$ sind aufl\"osbar.
\end{enum}
\end{DefBem}

\subsection{Freie Gruppen}

\section{Monographien von Gruppen}

\subsection{Symmetrische Gruppe}

Eigenschaften:
\begin{itemize}
\item Anzahl der Elemente: $|S_n| = n!$
\item Im allgemeinen \emph{nicht} abelsch.
\end{itemize}

\subsection{Alternierende Gruppe}

Eigenschaften:
\begin{itemize}
\item Anzahl der Elemente: $|A_n| = n!/2$
\item Im allgemeinen \emph{nicht} abelsch.
\end{itemize}

\subsection{Diedergruppe}

\begin{itemize}
\item Definition: $D_n := <D, S>$, $\textrm{ord}(D)=n$, $\textrm{ord}(S)=n$ %!!! gen\"gt das?
\item Anzahl der Elemente: $|D_n|=2n$
\item Im allgemeinen \emph{nicht} abelsch.
\end{itemize}

Charakterisierende Eigenschaft:
\begin{itemize}
\item Es gibt ein Element $S$ der Ordnung $2$.
\item Es gibt ein Element $D$ der Ordnung $n$.
\item $SD=D^{-1}S$
\end{itemize}

Rechenregeln in der Diedergruppe
\begin{enumerate}
\item $D^n=e$
\item $S^2=e$
\item $(D^iS)^2=e$
\item $SD=D^{-1}S$
\item $SD^i = D^{n-i}S$
\end{enumerate}

Weitere Eigenschaften:
\begin{enumerate}
\item Zentralisator: $<D>$
\end{enumerate}

%% mit Drehk�stchen
%% wie sehen die 2n Elemente der Diedergruppe aus?
%% Diedergruppe als Permutationen?

\begin{Bsp}
\begin{itemize}
\item $D_6$, $N:=<D^3>\trianglelefteq D_6$
\item $D_{12}$, $N:=<D^3>\trianglelefteq D_{12}$\\$D_{12}/N$ ist nicht abelsch.
\end{itemize}
\end{Bsp}

\section{Bestimmung aller Isomorphieklassen}

Einige Kandidaten f\"ur Untergruppen:
\begin{itemize}
\item Zyklische Gruppen
\item Abelsche Gruppen
\item Diedergruppe $D_n$
\item Alternierende Gruppe $A_n$
\item Kleinsche Vierergruppe $V_4$
\item Quaternionengruppe
\end{itemize}

Bestimmen Sie alle Isomorphieklassen von Gruppen der Ordnung $n$.
\begin{itemize}
\item Satz von Lagrange
\item S\"atze von Sylow
\item abelsch oder nicht abelsch? (Klassifizierung endlicher abelscher Gruppen)
\end{itemize}

Spezialf\"alle
\begin{itemize}
\item $n=p$ Primzahl (nur die zyklische Gruppe)
\item $n=p^2$, $p$ Primzahl ($\mathbb{Z}_{p^2}$ oder $\mathbb{Z}_{p}\times \mathbb{Z}_p$)
\item $n=2p$, $p\geq 3$ Primzahl (nur Diedergruppe und zyklische Gruppe)
\item $n=pq$, $p, q$ Primzahlen, $p>q$, $q$ teilt nicht $p-1$: $\mathbb{Z}_{pq}$
\end{itemize}

% Gibt es ein Element der Ordnung $n$, so ist die Gruppe zyklisch.
% Ordnung von Produkten von Elementen

Seien $U_1, \ldots, U_k$ $k$ paarweise (bis auf das Neutralelement) disjunkte Untergruppen von $G$.
Dann gilt: $xy=yx$, f\"ur $\in G_i, y\in G_j$

Wenn alle Sylowgruppen normal in einer Gruppe G sind, 
so ist G isomorph zum direkten Produkt dieser Sylowgruppen.

%Primzahlkardinalit\"t -> zyklisch
\chapter{Ringe}

\section{Euklidische Ringe}

\begin{Def}
    \begin{enum}
\item Ein Integrit�tsbereich $R$ hei�t \emp{euklidisch}, wenn es
eine Abbildung: $\delta: R\setminus\{0\} \ra \mathbb{N}$ mit
folgender Eigenschaft gibt: zu $f,g \in R, g\neq 0$ gibt es $q,r \in
R$ mit $f = qg + r$ mit $r=0$ oder $\delta(r) < \delta(q)$.

\item Sei $R$ euklidisch, $a,b \in R\setminus\{0\}$. Dann gilt:
    \begin{enum}
        \item[(i)] in $R$ gibt es einen ggT von $a$ und $b$.
        \item[(ii)] $d \in (a,b)$ (dh $\exists x,y \in R$ mit $d=xa
        +yb$)
        \item[(iii)] $(d) = (a,b)$
    \end{enum}
\item Jeder euklidische Ring ist ein Hauptidealring.
\end{enum}
\sbsp{1.0}{$\mathbb{Z}$ mit $\delta(a) = |a|, K[X]$ mit $\delta(f)
=$ Grad($f$)}
\end{Def}

\section{Hauptidealringe}

\begin{Def}
Ein komutativer Ring mit Eins hei�t \emp{Hauptidealring}, wenn
jedes Ideal in $R$ ein Hauptideal ist.
\end{Def}

\begin{Satz}
Jeder nullteilerfreie Hauptidealring ist faktoriell.
\end{Satz}

%Bosch S.46
\begin{Satz}
Es sei $R$ ein Hauptidealring $p \in R$ eine von $0$ verschiedene Nichteinheit.
Dann ist �quivalent:
\begin{enumerate} 
  \item[(i)] $p$ ist irreduzibel
  \item[(ii)] $p$ ist Primelement
  \item[(iii)] $(p)$ ist maximales Ideal in $R$
\end{enumerate}
\end{Satz}

\section{Faktorielle Ringe}
\begin{PropDef}
\label{2.21}
Sei $R$ ein Integrit�tsbereich.
\begin{enum}
\item Folgende Eigenschaften sind �quivalent:
    \begin{enumerate}
        \item[(i)] Jedes $x \in R\setminus\{0\}$ l��t sich eindeutig
        als Produkt von Primelementen schreiben.
        \item[(ii)] Jedes $x \in R\setminus\{0\}$ l��t sich
        ''irgendwie'' als Produkt von Primelementen schreiben.
        \item[(iii)] Jedes $x \in R\setminus\{0\}$ l��t sich eindeutig als
        Produkt von irreduziblen Elementen schreiben.
    \end{enumerate}

\item Sind diese drei Eigenschaften f�r $R$ erf�llt, so hei�t $R$
\emp{faktorieller} Ring. (Oder \emp{ZPE-Ring} (engl.: UFD)). Dabei
ist in (a) ''eindeutig'' gemeint, bis auf Reihenfolge und
Multiplikation mit Einheiten. Pr�ziser: Sei $\mathcal{P}$ ein
Vertretersystem der Primelemnte ($\neq 0$) bez�glich
''assoziiert''.
\newline Dann hei�t (i) $\forall x \in R \setminus\{0\}\; \exists!\;e \in
R^x$ und f�r jedes $p \in \mathcal{P}$ ein $\ds\nu_p(x) \geq
0:x=e\prod_{p \in \mathcal{P}} p^{\nu_p}$. (beachte $\nu_p \neq 0$
nur f�r endlich viele $p$).
\end{enum}
\end{PropDef}

\begin{Bem}
Ist $R$ faktorieller Ring, so gibt es zu
allen $a,b \in R\setminus\{0\}$ einen ggT($a$,$b$).
\end{Bem}

\begin{Bem}
Sei $R$ ein faktoriellen Ring, $a \in R$.
\[a \mbox{ irreduzibel} \lra a \mbox{ prim} \]
\end{Bem}

\section{Vererbung auf den Polynomring}

\begin{Bem}
Sei $R$ ein Ring und $R[X]$ der zugeh�rige Polynomring, dann vererben sich
folgende Eigenschaften von $R$ auf $R[X]$:
\begin{enumerate}
  \item hat Eins
  \item kommutativ
  % TODO vererbt sich die Nullteilerfreiheit auch ohne obige Eigenschaften?
  % [ ] Ja
  % [ ] Nein
  % [ ] Vielleicht
  \item Integrit�tsbereich
  \item faktoriell
\end{enumerate}
\end{Bem}

\end{document}
