\documentclass[11pt]{book}
\usepackage{etex}
\usepackage{amssymb}
\usepackage[utf8]{inputenc}
\usepackage[T1]{fontenc}
\usepackage{german}
\usepackage[german]{babel}
\usepackage{geometry}
\usepackage{paralist}
\usepackage[explicit]{titlesec}
\usepackage{amsmath}
\usepackage{setspace}
\usepackage{polynom}
\usepackage{stmaryrd}
\usepackage[arrow, matrix, curve]{xy}
\usepackage{float}
\usepackage{titletoc}
\usepackage{pgfplots}
\usepackage{graphicx} 
\usepackage{wrapfig}
\usepackage{mathabx}
\usepackage{caption}
\usepackage{subcaption}
\usepackage{ifthen}

% Meta-Daten fuer Latexki
\usepackage{latexki}
\lecturer{Prof. Dr. Frank Herrlich}
\semester{Sommersemester 2015}
\scriptstate{complete}


\usepackage[amsmath,thmmarks]{ntheorem}


\newtheorem{theorem}{Satz}[section]
\newtheorem{lemma}[theorem]{Lemma}
\newtheorem{prop}[theorem]{Proposition}
\newtheorem{cor}[theorem]{Korollar}
\newtheorem{motivation}[theorem]{Motivation}
\newtheorem{definition}[theorem]{Definition}
\newtheorem{remark}[theorem]{Bemerkung}
\newtheorem{example}[theorem]{Beispiel}
\newtheorem*{exampleo}[theorem]{Beispiel}
\newtheorem{bemdefini}[theorem]{Bemerkung + Definition}
\newtheorem{definibem}[theorem]{Definition + Bemerkung}
\newtheorem{erinndefini}[theorem]{Erinnerung + Definiton}
\newtheorem{erinnerbem}[theorem]{Erinnerung + Bemerkung}
\newtheorem{erinnerprop}[theorem]{Erinnerung + Proposition}
\newtheorem{erinner}[theorem]{Erinnerung}
\newtheorem{propdefini}[theorem]{Proposition + Definiton}
\newtheorem{definiprop}[theorem]{Definiton + Proposition}
\newtheorem{folger}[theorem]{Folgerung}
\newtheorem{theoremdefini}[theorem]{Satz + Definition}

\newenvironment{proof}[1][\it Beweis.]{\begin{trivlist}
\item[\hskip \labelsep {#1}]}{\end{trivlist}}





\theoremstyle{nonumberbreak} 
\theoremseparator{:} 
\theoremindent0.5cm 
\theoremheaderfont{\scshape} 
\theorembodyfont{\normalfont} 
\theoremsymbol{\ensuremath{_\bigboxvoid}} 
\RequirePackage{amssymb} 
\qedsymbol{\ensuremath{_\bigboxvoid}}

\newenvironment{defin}[1][]{\ifthenelse{\equal{#1}{}}{\definition}{\definition[#1]}\rm}{\enddefinition}
\newenvironment{motiv}[1][]{\ifthenelse{\equal{#1}{}}{\motivation}{\motivation[#1]}\rm}{\endmotivation}
\newenvironment{pr}[1][]{\ifthenelse{\equal{#1}{}}{\proof}{\proof[#1]}\rm}{\endproof}
\newenvironment{ex}[1][]{\ifthenelse{\equal{#1}{}}{\example}{\example[#1]}\rm}{\endexample}
\newenvironment{exo}[1][]{\ifthenelse{\equal{#1}{}}{\exampleo}{\exampleo[#1]}\rm}{\endexampleo}
\newenvironment{bemdefin}[1][]{\ifthenelse{\equal{#1}{}}{\bemdefini}{\bemdefini[#1]}\rm}{\endbemdefini}
\newenvironment{erinndefin}[1][]{\ifthenelse{\equal{#1}{}}{\erinndefini}{\erinndefini[#1]}\rm}{\enderinndefini}
\newenvironment{erinnbem}[1][]{\ifthenelse{\equal{#1}{}}{\erinnerbem}{\erinnerbem[#1]}\rm}{\enderinnerbem}
\newenvironment{erinnprop}[1][]{\ifthenelse{\equal{#1}{}}{\erinnerprop}{\erinnerprop[#1]}\rm}{\enderinnerprop}
\newenvironment{propdefin}[1][]{\ifthenelse{\equal{#1}{}}{\propdefini}{\propdefini[#1]}\rm}{\endpropdefini}
\newenvironment{definprop}[1][]{\ifthenelse{\equal{#1}{}}{\definiprop}{\definiprop[#1]}\rm}{\enddefiniprop}
\newenvironment{theoremdefin}[1][]{\ifthenelse{\equal{#1}{}}{\theoremdefini}{\theoremdefini[#1]}\rm}{\endtheoremdefini}
\newenvironment{er}[1][]{\ifthenelse{\equal{#1}{}}{\erinner}{\erinner[#1]}\rm}{\enderinner}
\newenvironment{definbem}[1][]{\ifthenelse{\equal{#1}{}}{\definibem}{\definibem[#1]}\rm}{\enddefinibem}
\newenvironment{folg}[1][]{\ifthenelse{\equal{#1}{}}{\folger}{\folger[#1]}\rm}{\endfolger}
\newenvironment{thm}[1][]{\ifthenelse{\equal{#1}{}}{\theorem}{\theorem[#1]}}{\endtheorem}

\newcommand{\QED}{\hskip\hsize\hskip-\marginparwidth\hskip-\marginparsep\qedsymbol} 

\newcommand{\spec}{\rm{spec}}
\newcommand{\p}{\mathfrak{p}}
\newcommand{\q}{\mathfrak{q}}
\newcommand{\K}{\mathbb{K}}
\newcommand{\ideal}{\trianglelefteqslant}




\titlecontents{chapter}[2em]{\addvspace{2pc}\bfseries}{\contentslabel{1.7em}}{}{\titlerule*[1.5pc]{}\contentspage}
\titlecontents{section}[4.5em]{}{\contentslabel{2.5em}}{}{\titlerule*[0.5pc]{.}\contentspage}

\titleformat{\subsection}{\normalfont\normalsize\bfseries}{}{0em}{#1 \thesubsection}
\titleformat{\section}{\normalfont\Large\bfseries}{}{0em}{\thesection  #1}
\renewcommand*\thechapter{\Roman{chapter}\quad}
\titlespacing{\chapter}{0pt}{*5}{*1.5}
\titlespacing{\section}{0pt}{*5}{20pt}
\titlespacing{\subsection}{0pt}{16pt}{0pt}
\geometry{a4paper, top=20mm, left=20mm, right=20mm, bottom=20mm, headsep=10mm, footskip=10mm}
\renewcommand{\labelenumi}{(\roman{enumi})}
\renewcommand{\labelenumii}{(\arabic{enumii})}
\setlength{\parindent}{0pt}
\newcommand{\slant}[2]{{\raisebox{.1em}{$#1$}\left/\raisebox{-.1em}{$#2$}\right.}}
\newcommand{\bigslant}[2]{{\raisebox{.2em}{$#1$}\left/\raisebox{-.2em}{$#2$}\right.}}
\usepackage{graphicx}
\newcommand\tabrotate[1]{\rotatebox{90}{#1\hspace{\tabcolsep}}}
\newcommand\verschiebung[1][-.75\normalbaselineskip]{\hspace{#1}}
\makeatletter
\renewcommand*{\env@matrix}[1][*\c@MaxMatrixCols c]{
  \hskip -\arraycolsep
  \let\@ifnextchar\new@ifnextchar
  \array{#1}}
\makeatother
\renewcommand{\chaptermark}[1]{ 
  \markboth{ 
     \MakeUppercase{\thechapter #1} 
  }{} 
} 
\renewcommand{\sectionmark}[1]{ 
  \markright{ 
     \MakeUppercase{\thesection#1} 
  } 
}

\usepackage{hyperref}

\title{Algebraische Geometrie}

\begin{document}



\begin{titlepage}

\textrm{ }\\[64pt]

\begin{center}
{\fontsize{40}{40} \selectfont \textbf{Algebraische Geometrie}}
\end{center}
\textrm{ } \\[36pt]
\begin{center} \large{\textrm{gelesen von Prof. Dr. Frank Herrlich im Sommersemester 2015 am KIT}} \end{center}
\textrm{ } \\[320pt]
\begin{center} \large{\textit{Geschrieben in } \LaTeX \textit{ von Arthur Martirosian, arthur.martirosian.93@gmail.com}}\end{center}
\textrm{ }\\[24pt]
\begin{center} \large{\today} \end{center}

\end{titlepage}
\thispagestyle{empty}



\begin{spacing}{1.62}
\setcounter{tocdepth}{1}
\tableofcontents
\thispagestyle{empty}
\end{spacing}
\newpage



\begin{spacing}{1.3}
\thispagestyle{empty}


\chapter{Affine Varietäten} %KAPITEL I
\setlength\abovedisplayshortskip{0pt}
\setlength\belowdisplayshortskip{10pt}
\setlength\abovedisplayskip{10pt}
\setlength\belowdisplayskip{10pt}


\renewcommand*\thesection{§ \arabic{section}\quad}
\section{Nullstellenmengen und Verschwindungsideale} %PARAGRAPH 1
\renewcommand*\thesection{\arabic{section}}
\thispagestyle{empty}

Sei $\mathbb{K}$ ein Körper, $n \in \mathbb{N}$. 

\begin{defin} % Definition 1.1

Eine Menge $V \subseteq \mathbb{K}^n$ heißt  \textit{affine Varietät}, falls es eine Teilmenge $F\subseteq \mathbb{K}[X_1, \ldots, X_n]$ gibt, sodass gilt
$$V=V(F)=\{x \in \mathbb{K}^n \ \vert \ f(x)=0 \ \textrm{ für alle } f \in F \}$$
\end{defin}

\begin{ex}  %%Beispiel 1.2
\begin{compactenum}
\item Wir definieren $\mathbb{K}^n:=V(\{0\})=V(\emptyset)$. Damit wird $\mathbb{K}^n$ zur affinen Varietät.
\item Mit $V(\{1\})=V(1)=\emptyset$ wird $\emptyset$ zur affinen Varietät.
\item Für jedes $a=(a_1, \ldots, a_n) \in \mathbb{K}^n$ ist $\{x\}$ affine Varietät via
$$\{a\}=V(X_1-a_1, \ldots, X_n-a_n)$$
\item Allgemeiner gilt: Jeder affine Untervektorraum des $\mathbb{K}^n$ ist affine Varietät.
\end{compactenum}
\end{ex}

\begin{remark}%Bemerkung 1.3

\begin{compactenum}
\item Aus $F_1 \subseteq F_2$ folgt $V(F_1) \supseteq V(F_2)$.
\item Für $f_1, f_2 \in \mathbb{K}[X_1, \ldots, X_n]$ gilt $V(f_1f_2)=V(f_1) \cup V(f_2)$.
\item Für $F \subseteq \mathbb{K}[X_1, \ldots, X_n]$ gilt $V(F)=V(\langle F\rangle )$, wobei $\langle F \rangle$ das von $F$ erzeugte Ideal bezeichnet.
\end{compactenum}
\begin{pr}
\begin{compactenum}
\item Ist $x \in V(F_2)$, so gilt $f(x) = 0$ für alle $f \in F_2$. Wegen $F_1 \subseteq F_2$ folgt $f(x)=0$ für alle $f \in F_1$, also $x \in V(F_1)$.
\item Es gilt $x \in V(f_1f_2)$ genau dann, wenn $(f_1f_2)(x)=0$, also $f_1(x)=0$ oder $f_2(x)=0$ und damit $x \in V(f_1) \cup V(f_2)$. 
\item $"\supseteq"$ folgt aus (i) mit $F \subseteq \langle F \rangle$. Für die andere Inklusion wähle $x \in V(F)$ und $f \in \langle F \rangle$. Schreibe
$$f=\sum_{i=0}^r a_i f_i, \qquad a_i \in \mathbb{K}[X_1, \ldots, X_n], f_i \in F$$
Dann ist
$$f(x)=\sum_{i=0}^r a_i(x) f_i(x)=0$$
und damit $x \in V(\langle F \rangle$. $\hfill \Box$
\end{compactenum}
\end{pr}
\end{remark}


\begin{folg} %Folgerung 1.4

Sei $V \subseteq \mathbb{K}^n$ affine Varietät.
\begin{compactenum}
\item Dann gibt es ein Ideal $I \trianglelefteqslant \mathbb{K}[X_1, \ldots, X_n]$ mit $V=V(I)$.
\item Dann gibt es es $f_1, \ldots, f_r \in \mathbb{K}[X_1,\ldots, X_n]$ mit $V=V(f_1, \ldots, f_r)$.
\end{compactenum}
\textit{Beweis.}
\begin{compactenum}
\item Für $V=V(F)$ wähle $I=\langle F \rangle$.
\item Folgt aus dem Hilbertschen Basissatz. $\hfill \Box$
\end{compactenum}
\end{folg}

\begin{bemdefin} %Definition + Bemerkung 1.5

\begin{compactenum}
\item Sei $R$ (kommutativer) Ring (mit $1$) und $I \trianglelefteqslant R$ ein Ideal. Dann heißt
$$\sqrt{I}:= \{f \in R \ \vert \ \textrm{ es existiert } n \in \mathbb{N} \ \textrm{mit } f^n \in I \}$$
das \textit{Radikal} von $I$.
\item $\sqrt{I}$ ist Ideal.
\item $I$ heißt \textit{Radikalideal}, falls $I=\sqrt{I}$.
\item Jedes Primadeal ist Radikalideal.
\item $n\mathbb{Z}$ ist Radikalideal genau dann, wenn $n$ quadratfrei ist, d.h. $\nu_p(n)\in \{0,1\}$ für alle $p \in \mathbb{P}$.
\item Für jedes Ideal $I \trianglelefteqslant \mathbb{K}[X_1, \ldots, X_n]$ gilt
$V(I)=V(\sqrt{I})$.
\end{compactenum}

\end{bemdefin}

\begin{definbem} %Definition + Bemerkung 1.6

Sei $V \subseteq \mathbb{K}^n$.
\begin{compactenum}
\item Das \textit{Verschwindungsideal von} $V$ ist 
$$I(V):=\{f \in \mathbb{K}[X_1, \ldots, X_n] \ \vert \ f(x)=0 \textrm{ für alle } x \in V \}$$
\item $I(V) \trianglelefteqslant \mathbb{K}[X_1, \ldots, X_n]$ ist Radikalideal.
\item $V(I(V))\supseteq V$.
\end{compactenum}
\begin{pr}
\begin{compactenum}
\item Folgt aus $(f+g)(x)=f(x)+g(x)$ und $(h\cdot f)(x)=h(x)f(x)$.
\item Folgt aus $f^d(x)=f(f(\ldots f(x) \ldots ))=f(x)^d$.
\item Klar. $\hfill \Box$
\end{compactenum}
\end{pr}
\end{definbem}

\begin{ex} %Beispiel 1.7
\begin{compactenum}
\item $I(\emptyset)=\mathbb{K}[X_1, \ldots, X_n]$.
\item $I(\mathbb{K}^n)=\{0\}$ genau dann wenn (!) $\mathbb{K}$ unendlich ist.
\item Für $n=2$ betrachte $V=\{(0,0)\} \subseteq \mathbb{K}^2$. Dann ist
$$I(V)=\left\{f=\sum_{i=1}^n \sum_{j=1}^m a_{i,j} X^{i} Y^{j} \ \bigg \vert \ a_{0,0}=0 \right\}$$
\end{compactenum}
\end{ex}

\begin{prop} %Proposition 1.8
Seien $V, V_1, V_2$ affine Varietäten in $\mathbb{K}^n$. Dann gilt
\begin{compactenum}
\item $V(I(V))=V$.
\item $V_1 \subseteq V_2$ genau dann, wenn $I(V_1) \supseteq I(V_2)$.
\end{compactenum}
\begin{pr}
\begin{compactenum}
\item \begin{compactitem}
\item["$\supseteq$"] Klar.
\item["$\subseteq$"] Sei $V=V(I')$ für ein Ideal $I' \trianglelefteqslant \mathbb{K}[X_1, \ldots, X_n]$.  Dann ist $I' \subseteq I(V)$, denn für $f \in I'$ ist $f(x)=0$ für alle $x \in V=V(I')$, also $V(I') \supseteq V(I(V))$.
\end{compactitem}
\item Folgt aus (i) und 1.2. $\hfill \Box$
\end{compactenum}
\end{pr}
\end{prop}

\begin{remark} %Beispiel 1.9

Frage: Gilt auch $I(V(I))=I$ für ein Radikalideal? Antwort: Nicht uneingeschränkt!\\
Betrachte $$I= \langle X^2+1 \rangle \in \mathbb{K}[X]$$Dann ist für $\mathbb{K}=\mathbb{R}$ $V(I)= \emptyset$, also $I(V(I))=I(\emptyset)=\mathbb{R}[X]$.\\
Gehen wir dagegen in einen algebraisch abgeschlossenen Körper, z.B. $\mathbb{C}$ über:\\
Dann ist $V(I)=\{i,-i\}$, also $$I(V(I))=\langle X-i\rangle \cap \langle X+i \rangle = \langle X^2+1\rangle.$$
Unser Ziel soll es also sein, zu zeigen, dass dies allgemein in algebraisch abgeschlossenen Körpern gilt.
\end{remark}

\begin{definbem} %Definition + Bemerkung 1.10

Sei $V \subseteq \mathbb{K}^n$ affine Varietät, $I(V)$ das Verschwindungsideal.
\begin{compactenum}
\item Wir definieren die \textit{affine Algebra} bzw. den \textit{affinen Koordinatenring} zu $V$ als
$$A(V):=\slant{\mathbb{K}[X_1, \ldots, X_n]}{I(V)}$$
\item $A(V)$ ist eine endlich erzeugte, reduzierte $\mathbb{K}$-Algebra, d.h. $A(V)$ enthält keine nilpotenten Elemente, d.h. für $a\neq 0$ gilt $a^d \neq 0$ für alle $d \in \mathbb{N}$.
\item Ist $V'\subseteq V$ affine Varietät, so erhalten wir einen surjektiven Homomorphismus von $\mathbb{K}$-Algebren $p: A(V) \longrightarrow A(V')$.
\end{compactenum}
\begin{pr}
\begin{compactenum}
\item[(ii)] Sei $a \in A(V)$ mit $a \neq 0$ und $a^d=0$ für ein $d\geqslant 1$. Wähle $f \in \mathbb{K}[X_1, \ldots, X_n]$ mit $\overline{f}=a$ in $A(V)$. Dann ist $f^d \in I(V)$, denn $\overline{f^d}=\overline{f}^d=a^d=0$, und damit auch $f \in I(V)$, da $I(V)$ Radikalideal ist. Dann gilt $a=0$, also ein Widerspruch.
\item[(iii)] Wegen 1.6 ist $I(V') \supseteq I(V)$. Mit dem Homomorphiesatz erhalten wir eine Faktorisierung\\
$$
\begin{xy}
\xymatrix{
\mathbb{K}[X_1, \ldots, X_n] \ar[rr] \ar[rd] && A(V') \\ & A(V) \ar[ru]_{p} & 
}
\end{xy}
$$
welche den gewünschten Homomorphismus liefert. $\hfill \Box$
\end{compactenum}
\end{pr}
\end{definbem}


\renewcommand*\thesection{§ \arabic{section}\quad}
\section{Die Zariski-Topologie} %PARAGRAPH 2
\renewcommand*\thesection{\arabic{section}}

Es sei $\mathbb{K}$ ein Körper, $n \in \mathbb{N}$.

\begin{definbem} %Definition + Bemerkung 2.1
\begin{compactenum}
\item Die affinen Varietäten in $\mathbb{K}^n$ bilden die abgeschlossenen Mengen einer Topologie auf $\mathbb{K}^n$.
\item Diese Topologie heißt \textit{Zariski-Topologie}.
\item Es bezeichne $\mathbb{A}^n(\mathbb{K})$ den topologischen Raum $\mathbb{K}^n$ mit der Zariski-Topologie.
\end{compactenum}
\begin{pr}
Wir rechnen die Axiome einer Topologie nach.
\begin{compactenum}
\item[(1)] Per Definition sind $\mathbb{K}^n=V(0)$ und $\emptyset=V(1)$ affine Varietäten.
\item[(2)] Seien $V_1=V(I_1), V_2=V(I_2)$ affine Varietäten.
\begin{compactenum}
\item[\textbf{Beh. (a)}] Es gilt $V(I_1 I_2) \overset{(i)}{\subseteq} V_1 \cup V_2 \overset{(ii)}{\subseteq} V(I_1 \cup I_2) \overset{(iii)}{\subseteq}V(I_1 I_2)$.
\end{compactenum}
Dann gilt an jeder Stelle Gleichheit und damit ist auch $V_1 \cup V_2$ affine Varietät.
\begin{compactenum}
\item[\textbf{Bew. (a)}] Es gilt
\begin{compactenum}
\item[(iii)] $I_1 I_2 \subseteq I_1 \cap I_2$, also $V(I_1 I_2) \supseteq V(I_1 \cap I_2)$.
\item[(ii)] Es ist $I_k \cap I_2 \subseteq I_k$, also $V_k \subseteq V(I_1 \cap I_2)$ für $k \in \{1,2\}$, also auch $V_1 \cup V_2 \subseteq V(I_1 \cap I_2)$.
\item[(i)] Sei $x \in V(I_1 I_2)$, ohne Einschränkung $x \notin V_1$. Zu zeigen: $x \in V_2$.\\
Da $x \notin V_1$, gibt es ein $f \in I_1$, sodass $f(x) \neq 0$. Sei nun $g \in I_2$. Nach Voraussetzung ist dann
$$0=(f\cdot g)(x)=f(x)\cdot g(x)$$
und damit $g(x)=0$. Dies impliziert $x \in V_(I_2)=V_2$.
\end{compactenum}
\end{compactenum}
\item[(3)] Seien für eine beliebige Indexmenge $J$ $V_i$, $i \in J$ affine Varietäten, es gelte $V_i=V(I_i)$. Dann ist
$$\bigcap_{i\in J} V_i = V\left(\bigcup_{i\in J} I_i \right) = V \left( \bigg\langle \bigcup_{i \in J} I_i \bigg\rangle \right) := V\left(\sum_{i\in J} I_i \right)$$
ebenfalls affine Varietät, was zu zeigen war. $\hfill\Box$
\end{compactenum}
\end{pr}
\end{definbem}

\begin{ex}%Beispiel 2.2
Betrachte $n=1$. Dann ist $V \subseteq \mathbb{A}^1(\mathbb{K})$ abgeschlossen genau dann, wenn $V= \mathbb{A}^1(\mathbb{K})$ oder $V$ endlich ist. Insbesondere ist $\mathbb{A}^1(\mathbb{K})$ damit nicht hausdorffsch.
\end{ex}

\begin{ex} %Beispiel 2.3
Ist $\mathbb{K}$ endlich, so ist die Zariski-Topologie auf $\mathbb{K}^n$ die diskrete Topologie.
\end{ex}

\begin{prop} %Proposition 2.4
Sei $\mathbb{K}$ unendlich. Dann ist $\mathbb{A}^n(\mathbb{K})$ nicht hausdorffsch.\\
\begin{pr}
Siehe Übung.
\end{pr}
\end{prop}

\begin{prop} % Proposition 2.5
Für $f \in \mathbb{K}[X_1, \ldots, X_n]$ sei
$$D(f):=\{x \in \mathbb{K}^n \ \vert \ f(x) \neq 0 \} = \mathbb{K}^n \setminus V(f)$$
Dann bildet die Familie $\{D(f)\}_{f \in \mathbb{K}[X_1, \ldots, X_n]}$ eine Basis der Zariski-Topologie auf $\mathbb{A}^n(\mathbb{K})$.\\
\begin{pr}
Sei $U \subseteq \mathbb{A}^n(\mathbb{K})$ offen. Es ist zu zeigen, dass U eine Menge $D(f)$ für ein geeignetes $f$ enthält. Offenbar ist $V:=\mathbb{K}^n \setminus U$ abgeschlossen, also eine affine Varietät. Dann schreibe $V=V(I)$ für ein Ideal $I \trianglelefteqslant \mathbb{K}[X_1, \ldots, X_n]$.\\
Sei nun $x \in U$. Da $x\notin V$ existiert ein $f \in I(V)$, sodass gilt $f(x) \neq 0$, also $x \in D(f)$.
Da $f \in I$, gilt $\langle f\rangle \subseteq I$, also $V(f) \supseteq V(I)=V$ und damit $D(f) \subseteq U$, was zu zeigen war. $\hfill\Box$
\end{pr}
\end{prop}

\begin{definbem} %Definition + Bemerkung 2.6
Für jede affine Varietät $V \subseteq \mathbb{A}^n(\mathbb{K})$ heißt die Spurtopologie ebenfalls \textit{Zariski-Topologie}.\\
Für $f \in A(V)\setminus \mathbb{K}$ sei
$$D(f):=\{x \in V \ \vert \ f(x)\neq 0\}$$
Dann ist die Familie $\{D(F)\}_{f \in A(V)\setminus \mathbb{K}}$ offen und eine Basis der Zariski-Topologie.
\end{definbem}



\renewcommand*\thesection{§ \arabic{section}\quad}
\section{Irreduzible Varietäten} %PARAGRAPH 3
\renewcommand*\thesection{\arabic{section}}

\begin{defin} %Definition 3.1
 Sei $X$ ein topologischer Raum.
 \begin{compactenum}
 \item $X$ heißt \textit{reduzibel}, falls es echte abgeschlossene Teilmengen $V,W \subset X$ gibt mit $V \cup W = X$.
 \item Ist $X$ nicht reduzibel, so heißt $X$ \textit{irreduzibel}.
 \item Eine maximale irreduzible Teilmenge von $X$ heißt \textit{irreduzible Komponente}.
 \end{compactenum}
 \end{defin}
 
\begin{ex} %Beispiel 3.2
 Sei $X$ ein Hausdorffraum. Dann ist $X$ irreduzibel genau dann wenn $\vert X \vert \leqslant 1$, also $X \in \{\{\textrm{pt}\}, \emptyset\}$.
\begin{pr}
 Seien $x,y \in X$, $x \neq y$ und $U_x, U_y$ offene Umgebungen von $x,y$ mit $U_x \cap U_y = \emptyset$. Dann sind $V_x :=X\setminus U_x$, $V_y:=X \setminus U_y$ abgeschlossene Mengen mit
 $$V_x \cup V_y = (X \setminus U_x) \cup (X \setminus U_y)= X\setminus (U_x \cap U_y) =X$$
\end{pr}
\end{ex}

\begin{remark} %Bemerkung 3.3
\begin{compactenum}
\item Sei $X$ topologischer Raum, $V \subseteq X$ irreduzibel. Dann ist auch $\overline{V}$ irreduzibel.
\item Irreduzible Komponenten sind abgeschlossen.
\end{compactenum}
\end{remark}

\begin{ex} %Beispiel 3.4
\begin{compactenum}
\item Sei $\mathbb{K}$ unendlicher Körper. Dann ist $\mathbb{A}^n(\mathbb{K})$ irreduzibel für alle $n \in \mathbb{N}$.
\begin{pr}
Sei $\mathbb{A}^n(\mathbb{K})=V(I_1) \cup V(I_2)$ mit $I_1 \neq \langle 0 \rangle \neq I_2$. Dann gilt nach Bemerkung 2.1
$$\mathbb{A}^n(\mathbb{K})=V(I_1) \cup V(I_2) = V(I_2 I_2)$$
Wähle also $f \in I_2 \setminus \{0\}, g \in I_2 \setminus \{0\}$. Dann gilt für alle $x \in \mathbb{K}^n$ $(f\cdot g)(x)=0$, also $f\cdot g =0$, ein Widerspruch zur Nullteilerfreiheit.
\item $V(X\cdot Y)\subseteq \mathbb{A}^2(\mathbb{K})$ ist reduzibel mit $V(X\cdot Y)=V(X)\cup V(Y)$.
\end{pr}
\end{compactenum}
\end{ex}

\begin{theorem} %Satz 3.5

Sei $V \subseteq \mathbb{A}^n(\mathbb{K})$ affine Varietät, $V \neq \emptyset$. Dann gilt
$$V \textrm{ ist irreduzibel } \quad \Longleftrightarrow \quad I(V)\trianglelefteqslant \mathbb{K}[X_1, \ldots, X_n] \textrm{ ist Primideal } $$
\begin{pr}
\begin{compactenum}
\item["$\Rightarrow$"] Seien $f,g \in \mathbb{K}[X_1, \ldots, X_n]$ mit $fg \in I(V)$, ohne Einschränkung $f \notin I(V)$. Dann gibt es $x \in V$ mit $f(x) \neq 0$, das heißt es gilt
$V \nsubseteq V(f)$, nach Voraussetzung aber $V\subseteq V(fg)=V(f)\cup V(g)$. Damit ist
$$(V\cap V(f)) \cup (V\cap V(g))=V$$
Da $V$ aber irreduzibel ist und $V \cap V(f) \neq V$, muss gelten $V \cap V(g)=V$, also $V(g)\subseteq V$ und damit $g \in I(V)$.
\item["$\Leftarrow$"] Sei $V=V_1 \cup V_2$ ein Zerlegung von $V$ in zwei abgeschlossene Teilmengen $V_1$ und $V_2$. Dann ist $V_1=V(I_1), V_2=V(I_2)$ für Ideale $I_1, I_2 \trianglelefteqslant \mathbb{K}[X_1, \ldots, X_n]$. Sei ohne Einschränkung $V\neq V_1$, also $V\nsubseteq V(I_1)$. Dann gibt es $x \in V, f \in I_1$ mit $f(x)\neq 0$, also $f \notin I(V)$. Zeige also $V\subseteq V(I_2)$. Hierfür genügt zu zeigen $I_2 \subseteq I(V)$.\\
Sei nun $g \in I_2$. Dann ist $fg\in I_1 I_2$. Wegen $V=V_1\cup V_2$, also $V=V(I_1)\cup V(I_2)=V(I_1 I_2)$, gilt $I_1 I_2 \subseteq I(V)$, also $fg \in I(V)$. Da $I(V)$ prim ist und $f \notin I(V)$, gilt $g \in I(V)$ und damit $I_2 \subseteq I(V)$.
\end{compactenum}
\end{pr}
\end{theorem}

\begin{theorem}%Satz 3.6
Sei $V\subseteq \mathbb{A}^n(\mathbb{K})$ affine Varietät. Dann gilt
\begin{compactenum}
\item $V$ ist endliche Vereinigung irreduzibler Varietäten.
\item Gilt $V=V_1 \cup \ldots \cup V_r$ mit irreduziblen Varietäten $V_1, \ldots, V_r$ und $V_i \nsubseteq V_j$ für $i \neq j$ (das heißt, kein $V_i$ ist überflüssig), so sind die $V_i$ die irreduziblen Komponenten von $V$, also insbesondere eindeutig.
\end{compactenum}
\begin{pr}
\begin{compactenum}
\item Definiere
$$\mathcal{B}:=\{V \vert \ \textrm{V ist nicht endliche Vereinigung irreduzibler Varietäten } \} $$
$$\mathcal{I}:= \{I(V) \ \vert \ V\in \mathcal{B} \}$$
Zu zeigen: $\mathcal{B}, \mathcal{I}$ sind leer.\\
Angenommen $\mathcal{I}\neq \emptyset$. Dann enthält $\mathcal{I}$ ein maximales Element $I_0$, denn $\mathbb{K}[X_1, \ldots, X_n]$ ist noethersch, also wird jede aufsteigende Kette von Idealen stationär. Schreibe $I_0=I(V_0)$ mit $V_0 \in \mathcal{B}$. $V_0$ ist reduzibel, schreibe also $V_0=V_1 \cup V_2$ mit abgeschlossenen Mengen $V_1 \subsetneq V_0 \supsetneq V_2$, also gilt dann $I(V_1) \supsetneq I_0 \subsetneq I(V_2)$. Da $I_0$ maximal ist, ist $I(V_1), I(V_2) \notin \mathcal{I}$ und damit $V_1, V_2 \notin \mathcal|{B}$. Per Definition sind dann also $V_1, V_2$ darstellbar als endliche Vereinigungen irreduzibler Varietäten:
$$V_1=\bigcup_{i=1}^n U_{i}, \qquad V_2=\bigcup_{i=n+1}^m U_{i}$$
Damit ist aber $V$ endliche Vereinigung von irreduziblen Komponenten, also $V \in \mathcal{B}$, ein Widerspruch zur Voraussetzung.
\item Zeige zunächst die Eindeutigkeit. Sei hierfür $W\subseteq V$ irreduzible Komponente. Zu zeigen: $W=V_i$ für ein $1 \leqslant i \leqslant r$. Schreibe
$$W=W\cap V=\bigcup_{i=1}^r\underbrace{(W\cap V_i)}_{\textrm{abgeschlossen}}$$
Da $W$ irreduzibel ist, gilt bereits $W \cap V_i=W$, also $W\subseteq V_i$ für ein $1 \leqslant i \leqslant r$. Da aber auch $V_i$ irreduzibel ist, gilt $W=V_i$.\\
Zeige nun, dass $V_1, \ldots, V_r$ irreduzible Komponenten sind. Sei $1 \leqslant i \leqslant r$. Dann existiert eine irreduzible Komponente $W$ von $V$ mit $V_i \subseteq W$, also $W=V_j$ für ein $j \in \{1,\ldots, r\}$. Also erhalten wir $V_i \subseteq V_j$ und wegen $V_i \nsubseteq V_j$ dann $i=j$ und schließlich $W=V_i$. Damit ist $V_i$ irreduzible Komponente.
\end{compactenum}
\end{pr}
\end{theorem}

\begin{folg} %Folgerung 3.7
Sei $V\subseteq \mathbb{A}^n(\mathbb{K})$ affine Varietät, $I=I(V)$ ihr Verschwindungsideal und $A(V):=\slant{\mathbb{K}[X_1, \ldots, X_N]}{I(V)}:=\mathbb{K}[V]$ ihr affiner Koordinatenring. Dann gilt
\begin{compactenum}
\item $\mathbb{K}[V]$ hat nur endlich viele minimale Primideale.
\item In $\mathbb{K}[X_1, \ldots, X_n]$ gibt es nur endlich viele Primideale, die $I$ umfassen und minimal mit dieser Eigenschaft sind.
\end{compactenum}
\begin{pr}
\begin{compactenum}
\item Folgt aus (ii), denn (surjektive) (Ur-)Bilder von Primidealen sind wieder Primideale.
\item Ist $p \trianglelefteqslant \mathbb{K}[X_1, \ldots, X_n]$ prim sodass $\mathfrak{p} \supseteq I$ und minimal mit dieser Eigenschaft. Dann ist $V(\mathfrak{p}) \subseteq V(I)$ irreduzible Komponente und nach 3.5 ist die Anzahl dieser endlich.
\end{compactenum}
\end{pr}
\end{folg}


\renewcommand*\thesection{§ \arabic{section}\quad}
\section{Der Hilbertsche Nullstellensatz} %PARAGRAPH 4
\renewcommand*\thesection{\arabic{section}}

\begin{thm}[\rm \it Hilbertscher Nullstellensatz]   %Satz 4.1

Sei $\mathbb{K}$ algebraisch abgeschlossener Körper, $n \in \mathbb{N}$.
\begin{compactenum}
\item Für jedes Ideal $\{0\} \neq I \trianglelefteqslant \mathbb{K}[X_1, \ldots, X_n]$ ist $V(I) \neq \emptyset$.
\item Für jedes Ideal $I\trianglelefteqslant \mathbb{K}[X_1, \ldots, X_n]$ ist $I(V(I))=\sqrt{I}$.
\end{compactenum}
\end{thm}

\begin{theorem}[\rm \it Algebraische Version des Hilbertschen Nullstellensatzes]   %Satz 4.2
Sei $\mathbb{K}$ Körper, $n \in \mathbb{N}$, $\mathfrak{m} \triangleleft \mathbb{K}[X_1, \ldots, X_n]$ ein maximales Ideal. Dann ist $\mathbb{L}:=\slant{\mathbb{K}[X_1, \ldots, X_n]}{\mathfrak{m}}$ eine algebraische Körpererweiterung von $\mathbb{K}$.
\end{theorem}

\begin{folg} %%Folgerung 4.3
Sei $\mathbb{K}$ algebraisch abgeschlossener Körper, $n \in \mathbb{N}$. Dann gibt es Bijektionen zwischen folgenden Mengen:
\begin{compactenum}
\item $\{x=(x_1, \ldots, x_n) \in \mathbb{K}^n \}$
\item Ideale $\{I_x=\langle X_1-x_1, \ldots, X_n - x_n \rangle \trianglelefteqslant \mathbb{K}[X_1 \ldots, X_n]\}$
\item Maximale Ideale $\{\mathfrak{m}\triangleleft \mathbb{K}[X_1, \ldots, X_n]\}$.
\end{compactenum}
\begin{pr}
\begin{compactenum}
\item["(i)$\Rightarrow$(ii)"] Klar.
\item["(ii)$\Rightarrow$(iii)"] Sei für $x \in \mathbb{K}^n$ $\phi: \mathbb{K}[X_1, \ldots, X_n] \longrightarrow \mathbb{K}, \ f \mapsto f(x_1, \ldots, x_n)$. Dann ist offenbar $\ker(\phi)=I_x$, und da $\phi$ surjektiv ist damit
$$\slant{\mathbb{K}[X_1, \ldots, X_n]}{I_x} \cong \mathbb{K}$$
\item["(iii)$\Rightarrow$(ii)"] Sei $\mathfrak{m}\triangleleft \mathbb{K}[X_1, \ldots, X_n]$ maximales Ideal. Mit Satz 4.2 gilt also
$$\slant{\mathbb{K}[X_1, \ldots, X_n]}{\mathfrak{m}} \cong \mathbb{K}$$
Sei nun $$\phi: \mathbb{K}[X_1, \ldots, X_n] \longrightarrow \slant{\mathbb{K}[X_1, \ldots, X_n]}{\mathfrak{m}} \cong \mathbb{K}$$
die Restklassenabbildung, $x_i:=\phi(X_i)$. Dann gilt $\mathfrak{m}=\ker(\phi)$ und $\mathfrak{m}=I_x$.$\hfill \Box$
\end{compactenum}
\end{pr}
\end{folg}


\begin{pr}[\it Beweis von Satz 4.1.] %Beweis 4.1
\begin{compactenum}
\item Sei $I\triangleleft \mathbb{K}[X_1, \ldots, X_n]$ ein echtes Ideal. Dann existiert nach Zorn's Lemma ein maximales Ideal $I \subseteq \mathfrak{m} \triangleleft \mathbb{K}[X_1, \ldots, X_n]$. Damit gilt $V(I) \supseteq V(\mathfrak{m})$. Nach Folgerung 4.3 gibt es damit $x=(x_1, \ldots, x_n) \in \mathbb{K}^n$ mit
$$\mathfrak{m}=\langle X_1-x_1, \ldots, X_1-x_1 \rangle$$
und damit $x \in V(\mathfrak{m})\subseteq V(I)$, also gerade $V(I) \neq \emptyset$.
\item Sei $I \trianglelefteqslant \mathbb{K}[X_1, \ldots, X_n]$ Ideal. Dann gilt offenbar $\sqrt{I} \subseteq I(V(I))$. Zu zeigen ist somit $I(V(I)) \subseteq \sqrt{I}$.\\
Sei also $g \in I(V(I))$. Zeige: Es existiert $d \in \mathbb{N}$ mit $g^d \in \sqrt{I}$. Seien dazu $f_1, \ldots, f_m$ Erzeuger von $I$ und
$$J \ \trianglelefteqslant \ \mathbb{K}[X_1, \ldots, X_n, Y]$$
das von $f_1 \ldots, f_m$ und dem Polynom $gY-1$ erzeugte Ideal.
\begin{compactenum}
\item[\textbf{Beh. (a)}] Es gilt $V(J) = \emptyset$.
\item[\textbf{Bew. (a)}] Sei das Tupel $(x_1, \ldots, x_n,y):=(x',y):=x \in V(J)$. Dann gilt für alle $i \in \{1, \ldots, m\}$
$$0=f_i(x)=f_i(x') \quad \Longrightarrow \quad x' \in V(I)$$
Da $g \in I(V(I))$, gilt $g(x')=0$, denn $x'\in V(I)$. Dann folgt wegen $g(x')=0$
$$0=(gY-1)(x)=\left(g(x)Y(x)-1\right)=g(x')\cdot Y -1 = -1,$$
ein Widerspruch. Also gilt $V(J)=\emptyset$.
\end{compactenum}
Damit folgt $J=\mathbb{K}[X_1, \ldots, X_n]$, also insbesondere $1 \in J$. Schreibe
$$1=\sum_{i=1}^m b_i f_i + c \cdot (gY-1), \qquad b_i, c \in \mathbb{K}[X_1, \ldots, X_n,Y]$$
Betrachte nun den Ring
$$R:=\slant{\mathbb{K}[X_1, \ldots, X_n]}{\langle gY-1 \rangle} \cong \mathbb{K}[X_1, \ldots, X_n] \left[\frac{1}{g} \right]$$
Für die Isomorphie betrachte den surjektiven Ringhomomorphismus
$$\phi: \mathbb{K}[X_1, \ldots, X_n,Y] \longrightarrow \mathbb{K}[X_1, \ldots, X_n] \left[\frac{1}{g} \right],\qquad \begin{cases} X_i \mapsto X_i \\ Y \mapsto \frac{1}{g} \end{cases}$$
Für diesen gilt 
$$\ker(\phi)=\bigg\langle \left\{Y=\frac{1}{g} \right\} \bigg\rangle = \langle gY-1 \rangle$$
In $R$ gilt
$$1\ =\ \sum_{i=1}^m \overline{b_i f_i} + \overline{c (gY-1)} \ = \ \sum_{i=1}^m \overline{b_i} f_i$$
Schreibe
$$\overline{b_i}=\sum_{j=1}^{d_i} c_j \frac{1}{g^j} = \sum_{j=1}^{d_i} \frac{c_0g^{d_i}+c_1g^{d_i-1}+ \ldots + c_{d_i}}{g^{d_i}} := \frac{c_i}{g^{d_i}}$$
Sei $d:= \max_{1 \leqslant i \leqslant m} \{d_i \}$. Dann gilt $g^d \ \overline{b_i} \in \mathbb{K}[X_1, \ldots, X_n]$. Schließlich ist 
$$g^d=g^d \cdot 1= g^d \cdot \sum_{j=1}^m \overline{b_i} f_i = \sum_{j=1}^m \underbrace{g^d \ \overline{b_i}}_{ \in \mathbb{K}[X_1, \ldots, X_n]} f_i \in I,$$
was die Behauptung liefert. $\hfill \Box$
\end{compactenum}
\end{pr}

\begin{pr}[\it Beweis von Satz 4.2.] %% %Beweis 4.2
Durch Induktion über $n$.
\begin{compactenum}
\item[\textbf{n=1}] $\mathfrak{m}=\langle f \rangle$ für ein irreduzibles Polynom $f \in \mathbb{K}[X]$ (Algebra).
\item[\textbf{n>1}] Angenommen $\mathbb{L}/\mathbb{K}$ ist nicht algebraisch. Dann ist ohne Einschränkung $X_1$ transzendent über $\mathbb{K}$, $x_i:=\overline{X_i}$. Dann ist
$$\mathbb{K}(X) =\textrm{Quot}(\mathbb{K}[X]) \cong \mathbb{K}(x_1) \subseteq \mathbb{L}$$
Weiter wird $\mathbb{L}$ über $\mathbb{K}(x_1)$ von $x_2, \ldots, x_n$ erzeugt. Damit ist
$$\mathbb{L} \cong \slant{\mathbb{K}(x_1)[X_2, \ldots, X_n]}{\mathfrak{m'}}$$
für ein maximales Ideal $\mathfrak{m'} \triangleleft \mathbb{K}(x_1)[X_2, \ldots, X_n]$. Per Induktionsvoraussetzung sind $x_2, \ldots, x_n$ damit algebraisch über $\mathbb{K}(x_1)$, es gibt also normierte Minimalpolynome $f_2, \ldots, f_m \in \mathbb{K}(x_1)[X]$ aus denen Gleichungen
$$f_i(x_i)\ =\ x_i^{m_i} \ + \ \sum_{j=0}^{m_i-1} a_{ij} x_i^j \ = \ 0$$
mit geeigneten $a_{ij} \in \mathbb{K}(x_1)$. Sei nun $R$ der kleinste Teilring vom $\mathbb{K}(x_1)$, der $\mathbb{K}[x_1]$ und alle $a_{ij}$ enthält. Dann sind $x_2, \ldots, x_n$ ganz über $R$, also $\mathbb{L}/R$ eine ganze Ringerweiterung. Wir erhalten:
\begin{compactenum}
\item[(1)] $R$ ist kein Körper, da $\mathbb{K}[x_1]$ unendlich viele Primelemente enthält, $R$ aber nur endlich viele Primfaktoren als Nenner enthält.
\item[(2)] Jedes $a \in R\setminus \{0\}$ besitzt ein Inverses in $R$. Offenbar ist $\frac{1}{a}$ in $\mathbb{L}$ enthalten. Andererseits ist $\frac{1}{a}$ ganz über $R$, das heißt es existiert eine Darstellung
$$\left(\frac{1}{a}\right)^m \ + \ \sum_{j=0}^{m-1} b_j \left(\frac{1}{a}\right)^j \ = \ 0$$
für geeignete $b_j \in R$. dann gilt aber
$$1\ = \ - \sum_{j=0}^{m-1} b_j \ a^{m-j} \ =\ -a \cdot \sum_{j=0}^{m-1} b_j \ a^{m-j-1}$$
und damit $a \in R^{\times}$, womit $R$ zum Körper wird, also ein Widerspruch zu (1).
\end{compactenum}
Damit war die Annahme zu Beginn falsch und $x_1$, und damit auch $\mathbb{L}$ ist algebraisch über $\mathbb{K}$. $\hfill \Box$
\end{compactenum}
\end{pr}
\begin{folg} %Folgerung 4.4
Für jeden Körper $\mathbb{K}$ und jedes $n \in \mathbb{N}$ sei
$$\mathcal{V}_n(\mathbb{K}):=\{ V\subseteq \mathbb{K}^n \ \vert \ V \textrm{ is affine Varietät } \}$$
$$\mathcal{I}_n(\mathbb{K}):= \{ I \trianglelefteqslant \mathbb{K}[X_1, \ldots, X_n] \ \vert \ I=\sqrt{I} \}$$
$$V:=V_{n, \mathbb{K}}: \mathcal{I}_n(\mathbb{K}) \longrightarrow \mathcal{V}_n(\mathbb{K}), \quad I \mapsto V(I)$$
$$I:=I_{n, \mathbb{K}}: \mathcal{V}_n(\mathbb{K}) \longrightarrow \mathcal{I}_n(\mathbb{K}), \quad V \mapsto I(V)$$
Dann gilt
$$\mathbb{K} \textrm{ ist algebraisch abgeschlossen } \quad \Longleftrightarrow \quad I \textrm{ und } V \textrm{ sind zueinander invers }$$
\end{folg}

\begin{remark} %Bemerkung 4.5
Ist $\mathbb{K}$ algebraisch abgeschlossen und ist $V \subseteq \mathbb{A}^n(\mathbb{K})$ affine Varietät, so entsprechen die Punkte in $V$ bijektiv den maximalen Idealen in $A(V):=\slant{\mathbb{K}[X_1, \ldots, X_n]}{I(V)}$.
\end{remark}
\pagebreak




\renewcommand*\thesection{§ \arabic{section}\quad}
\section{Morphismen zwischen affinen Varietäten} %PARAGRAPH 5
\renewcommand*\thesection{\arabic{section}}

\begin{definbem} %Definition + Bemerkung 5.1
Sei $\mathbb{K}$ Körper, $V \subseteq \mathbb{A}^n(\mathbb{K}), W \subseteq \mathbb{A}^m(\mathbb{K})$ affine Varietäten.
\begin{compactenum}
\item Eine Abbildung $f:V \longrightarrow W$ heißt \textit{Morphismus}, falls es Polynome $f_1, \ldots, f_m \in \mathbb{K}[X_1, \ldots, X_n]$ gibt mit $f(x)=\left(f_1(x), \ldots, f_m(x)\right)$ für alle $x=(x_1, \ldots, x_n)\in V \subseteq \mathbb{K}^n$.
\item Jeder Morphiums $f:V\longrightarrow W$ ist die Einschränkung eines Morphismus $\tilde{f}: \mathbb{A}^n(\mathbb{K}) \longrightarrow \mathbb{A}^m(\mathbb{K})$.
\item Ein Morphismus $f: V \longrightarrow W$ heißt \textit{Isomorphismus}, falls es einen Morphismus $g: W \longrightarrow V$ gibt mit $f \circ g = \textrm{id}_W$ und $g \circ f = \textrm{id}_V$.
\item Die affinen Varietäten bilden zusammen mit den Morphismen eine Kategorie $\textrm{\underline{Aff}}(\mathbb{K})$

\end{compactenum}
\end{definbem}

\begin{ex} %Beispiel 5.2
\begin{compactenum}
\item[(0)] Die Identität $$\textrm{id}_{\mathbb{A}^n(\mathbb{K})}: \mathbb{A}^n(\mathbb{K}) \longrightarrow \mathbb{A}^n(\mathbb{K}),\qquad (x_1, \ldots, x_n) \mapsto (x_1, \ldots, x_n)$$ ist ein Morphismus mit $f_i=X_i$.
\item Weitere Morphismen sind
$$\textit{Einbettungen:} \ \ \mathbb{A}^n(\mathbb{K}) \longrightarrow \mathbb{A}^{n+m}(\mathbb{K}),\ (x_1, \ldots, x_n) \mapsto (x_1, \ldots, x_n, 0 \ldots, 0)$$
$$\textit{Projektionen:} \ \ \mathbb{A}^{n+m}(\mathbb{K}) \longrightarrow \mathbb{A}^n(\mathbb{K}),\ (x_1, \ldots x_{n+m}) \mapsto (x_1,\ldots, x_n)$$
$$\textit{Vertauschungen:} \ \ \mathbb{A}^n(\mathbb{K}) \longrightarrow \mathbb{A}^n(\mathbb{K}),\ (x_1, \ldots, x_n) \mapsto (x_{\sigma(1)}, \ldots, x_{\sigma(n)}), \quad \sigma \in S_n$$
\item Jedes $f \in \mathbb{K}[X_1, \ldots, X_n]$ ist ein Morphimsus $f:\mathbb{A}^n(\mathbb{K}) \longrightarrow \mathbb{A}(\mathbb{K})$.
\item Sei $V=\mathbb{A}^1(\mathbb{K})$, $W=V(Y^2-X^3)\subseteq \mathbb{A}^2(\mathbb{K})$. Definiere
$$f:V\longrightarrow W, \ x \mapsto (x^2, x^3)$$
Dann ist $f$ ein Morphismus. Außerdem ist $f$ bijektiv mit Umkehrabbildung
$$g(x,y)=\begin{cases} \frac{y}{x} & x\neq 0 \\ 0 & x=0 \end{cases}$$
denn es gilt
$$f(g(x,y))=f\left(\frac{y}{x}\right)=\left(\frac{y^2}{x^2}, \frac{y^3}{x^3}\right)=\left(\frac{x^3}{x^2}, \frac{y^3}{y^2}\right)=(x,y)$$
$$g(f(x))=g\left(x^2, x^3\right)=\left(\frac{x^3}{x^2}\right)=x$$
Aber: $g$ ist kein Morphismus (und $f$ damit kein Isomorphismus), falls $\mathbb{K}$ algebraisch abgeschlossen ist, denn andernfalls gäbe es ein Polynom $h \in \mathbb{K}[X,Y]$ mit $h(X,Y)=\frac{Y}{X}$, also
$$X\cdot h - Y \in I(W)=I(V(\langle Y^2-X^3\rangle))=\langle Y^2-X^3 \rangle$$
und damit $X \cdot h - Y = H \cdot (Y^2-X^3)$ für ein $H \in \mathbb{K}[X,Y]$ $\lightning$.
\item Sei $\textrm{char}(\mathbb{K})=p>0$. Dann heißt
$$f: \mathbb{A}^n(\mathbb{K}) \longrightarrow \mathbb{A}^n(\mathbb{K}), \quad (x_1, \ldots, x_n) \mapsto (x_1^p, \ldots, x_n^p)$$
\textit{Frobenius-Homomorphismus}. Es gilt:
$f$ ist injektiv, denn für $x^p=y^p$ gilt
$$0=x^p-y^p=(x-y)^p \ \Longrightarrow \ x-y=0 \ \Longrightarrow \ x=y$$
$f$ ist surjektiv, falls $\mathbb{K}$ enthalten ist in $\overline{\mathbb{F}}_p$ (im Allgemeinen jedoch nicht!).
Damit sind die Fixpunkte unter $f$ gerade jene, deren Koordinaten alle in $\mathbb{F}_p$ liegen, also
$$f(x)=x \quad \Longleftrightarrow \quad x \in \mathbb{F}_p^n \ \textrm{ für } x=(x_1, \ldots, x_n)$$
\end{compactenum}
\end{ex}

\begin{remark} %Bemerkung 5.3
Morphismen affiner Varietäten sind stetig bezüglich der Zariski-Toipologie.\\[-22pt]
\begin{pr} Seien $V \subseteq \mathbb{A}^n(\mathbb{K})$, $W\subseteq \mathbb{A}^m(\mathbb{K})$ affine Varietäten und $f:V\longrightarrow W$ ein Morphismus. Zeige, dass das Urbild abgeschlossener Mengen wieder abgeschlossen ist.\\
Sei $Z \subseteq W$ abgeschlossen. Dann ist $Z$ auch abgeschlossen in $\mathbb{A}^m(\mathbb{K})$, also existiert ein Ideal $$J\ \trianglelefteqslant \ \mathbb{K}[X_1, \ldots, X_n]$$ mit $Z=V(J)$. Zeige: $Z$ ist abgeschlossen, also affine Varietät.
\begin{compactenum}
\item[\textbf{Beh. (a)}] Es gilt $f^{-1}(Z)=V(I)$ mit $I= \{g \circ f \ \vert \ g \in J \} \trianglelefteqslant \mathbb{K}[X_1, \ldots, X_n]$. Dazu:
\item[\textbf{Bew. (a)}] Zunächst sehen wir ein
$$\mathbb{A}^n(\mathbb{K}) \overset{f}{\longrightarrow} \mathbb{A}^m(\mathbb{K}) \overset{g}{\longrightarrow} \mathbb{A}^1(\mathbb{K})$$
Damit ist $g\circ f: \mathbb{A}^n(\mathbb{K}) \longrightarrow \mathbb{A}(\mathbb{K})$ Morphismus, also gerade $g \circ f \in \mathbb{K}[X_1, \ldots, X_n]$. Nun gilt
\begin{alignat*}{5}
x \in f^{-1}(Z) \ \Longleftrightarrow \ f(x) \in Z \ &\Longleftrightarrow \ g(f(x))=0=(g\circ f)(x) \textrm{ für alle } g \in J\\
&\Longleftrightarrow \ h(x)=0 \textrm{ für alle } h \in I \\
&\Longleftrightarrow \ x \in V(I)
\end{alignat*}
also gerade die Behauptung. $\hfill \Box$
\end{compactenum}
\end{pr}
\end{remark}

\begin{remark} %Bemerkung 5.4
Für jede affine Varietät $V \subseteq \mathbb{A}^n(\mathbb{K})$ bilden die Morphismen $V \longrightarrow \mathbb{A}^1(\mathbb{K})$ eine $\mathbb{K}$-Algebra $\mathbb{K}[V]$. Es gilt
$$\mathbb{K}[V] \ \cong\ \slant{ \mathbb{K}[X_1, \ldots, X_n]}{I(V)} \ =\ A(V)$$
\begin{pr}
Offenbar ist $\textrm{Mor}(V, \mathbb{A}^1(\mathbb{K}))$ eine $\mathbb{K}$-Unteralgebra von $\textrm{Abb}(V, \mathbb{K})$. Weiter ist die Abbildung
$$\phi: \mathbb{K}[X_1, \ldots, X_n]  \longrightarrow \mathbb{K}[V], \quad f \mapsto f \vert_{V}$$
surjektiver Homomorphimus mit $\ker(\phi)=I(V)$, also
$$\mathbb{K}[V] \ \cong \ \slant{\mathbb{K}[X_1, \ldots, X_n]}{I(V)},$$
was zu zeigen war. $\hfill \Box$
\end{pr}
\end{remark}

\begin{prop} %%Proposition 5.5
Seien $V \subseteq \mathbb{A}^n(\mathbb{K})$, $W \subseteq \mathbb{A}^m(\mathbb{K})$ affine Varietäten.
\begin{compactenum}
\item Für jeden Morphismus $f:V \longrightarrow W$ ist die Abbildung
$$f^\#: \mathbb{K}[W] \longrightarrow \mathbb{K}[V], \quad g \mapsto g \circ f$$
ein Homomorphismus von $\mathbb{K}$-Algebren.
\item Die Abbildung
$$\alpha: \rm{Mor}\it(V,W) \longrightarrow \rm{Hom}\it_{\mathbb{K}}(\mathbb{K}[W], \mathbb{K}[V]), \quad f \mapsto f^\#$$
ist bijektiv.
\end{compactenum}
\begin{pr}
\begin{compactenum}
\item $g \circ f$ ist als Komposition von Morphismen ein Morphismus $g\circ f: V \longrightarrow \mathbb{A}^n(\mathbb{K})$ und es gilt
$$(g_1+g_2)\circ f = g_1 \circ f  + g_2 \circ f $$
usw. (diese Eigenschaften kennen wir bereits lange). Damit ist $f^\#$ Homomorphismus.
\item Offenbar ist die Abbildung mit (i) wohldefiniert. Für die Bijektivität zeige
\begin{compactenum}
\item[\textit{injektiv.}] Seien $f_1, f_2 \in \textrm{Mor}(V,W)$ mit $f_1^\#=f_2^\#$, also $g\circ f_1= g \circ f_2$ für alle $g \in \mathbb{K}[W]$. Insbesondere erhalten für die Projektionen $p_i$ anstelle von $g$ für alle $1 \leqslant i \leqslant m$
$$p_i \circ f_1 = p_i \circ f_2 \quad \Longrightarrow \quad f_{1i}=f_{2i} \quad \Longrightarrow \quad f_1=f_2$$
\item[\textit{surjektiv.}] Sei $\phi: \mathbb{K}[W] \longrightarrow \mathbb{K}[V]$ Homomorphismus von $\mathbb{K}$-Algebren, also $\phi \in \textrm{Hom}_{\mathbb{K}}(\mathbb{K}[W], \mathbb{K}[V])$. Definiere
$$f: V \longrightarrow \mathbb{A}^m(\mathbb{K}), \quad x \mapsto \left(\phi(p_1)(x), \ldots, \phi(p_m)(x) \right)$$
Zeige nun
\begin{compactenum}
\item[\textbf{Beh. (1)}] $f(V) \subseteq W$.
\item[\textbf{Beh. (2)}] $f^\# = \phi$.
\end{compactenum}
Dann ist $f$ ein Urbild von $\phi$ und die Behauptung folgt.
\begin{compactenum}
\item[\textbf{Bew. (2)}] Für $i \in \{1, \ldots, m \}$ gilt $f^\#(p_i)=p_i \circ f \overset{Def.}{=} \phi(p_i)$. Da die $p_i$ die $\mathbb{K}$-Algebra $\mathbb{K}[W]$ erzeugen, gilt $f^\#=\phi$.
\item[\textbf{Bew. (1)}] Zu zeigen ist $f(V) \subseteq V(I(W)) \subseteq W$. Sei also $x \in V$ und $g \in I(W)$ und zeige $g(f(x)) = (g \circ f)(x)=0$. Sei dazu
$$\tilde{\phi}: \mathbb{K}[X_1, \ldots, X_m] \longrightarrow \mathbb{K}[X_1, \ldots, X_n]$$
ein Homomorphismus, der die $X_i$ abbildet auf eine Reptäsentanten von $\phi(p_i)$ für $1 \leqslant i \leqslant m$. Genauer, betrachte
$$
\begin{xy}
\xymatrix{
\mathbb{K}[X_1, \ldots, X_m] \ar[rr]^{\tilde{\phi}} \ar[d]^{\pi_W} && \mathbb{K}[X_1, \ldots, X_n] \ar[d]^{\pi_V} \\ W \ar[rr]^{\phi} && \mathbb{K}[V] 
}
\end{xy}
$$
Es gilt $\tilde{\phi}(I(W)) \subseteq I(V)$, also $\tilde{\phi}(g) \in I(V)$. Damit erhalten wir
$$0=\tilde{\phi}(g)(x)\ =\ g\left( \phi(p_1)(x), \ldots, \phi(p_m)(x) \right)\ =\ (g \circ f)(x)$$
und damit die Behauptung. $\hfill \Box$
\end{compactenum}
\end{compactenum}
\end{compactenum}
\end{pr}
\end{prop}


\begin{folg} %Folgerung 5.6
Die Zuordnung $V \longrightarrow \mathbb{K}[V]$ ist ein \textit{kontravarianter} (=richtungsumkehrender) und \textit{volltreuer} (=bijektiver) \textit{Funktor} via
$$\Phi: \textrm{\underline{Aff}}(\mathbb{K}) \longrightarrow \underline{ \mathbb{K}\textrm{-Alg}}^{\textrm{red}}$$
$\Phi$ ist ein Morphismus auf Objekte:
$$\Phi(f)=f^\#, \quad \Phi(V)=\mathbb{K}[V]$$
Für $f \in \textrm{Mor}(V,W) $ ist
$$\Phi(f): \Phi(W)=\mathbb{K}[W] \longrightarrow \mathbb{K}[V]=\Phi(V), \quad g \mapsto g \circ f=f^\#$$
$$\Phi(\textrm{id})=\textrm{id}=\textrm{id}^\#$$
$$\Phi(f_2 \circ f_1)=\left(f_2 \circ f_1\right)^\#=f_1^\# \circ f_2^\#=\Phi(f_1) \circ \Phi(f_2)$$
Das heißt, wir haben kommutative Diagramme\\[-10pt]
$$
\begin{xy}
\xymatrix{
V \ar[rr]^{f_1} \ar[rrdd]_{f_2 \circ f_1} && W \ar[dd]^{f_2} &&&& \mathbb{K}[Z] \ar[rrdd]_{(f_2\circ f_1)^\#} \ar[rr]^{f_2^\#} && \mathbb{K}[W] \ar[dd]^{f_1^\#} \\ &&&&&&&& \\ && Z &&&&&& \mathbb{K}[V]
}
\end{xy}
$$
\end{folg}

\begin{remark} %Bemerkung 5.7

Seien $V,W$ affine Varietäten über $\mathbb{K}$ und
$$\phi: \mathbb{K}[W] \longrightarrow \mathbb{K}[V]$$
ein Homomorphismus von $\mathbb{K}$-Algebren. Ist $f:V \longrightarrow W$ der zugehörige Morphismus (also $f^\#=\phi$), so gilt für jedes $x \in V$:
$$\mathfrak{m}_{f(x)}=\phi^{-1}(\mathfrak{m}_x)$$
\begin{pr}
Es gilt 
$$\mathfrak{m}_x= \{ g \in \mathbb{K}[V] \ \vert \ g(x)=0 \},$$
also
$$\phi^{-1}(\mathfrak{m}_x) = \{ h \in \mathbb{K}[W] \ \vert \ \phi \circ h \in \mathfrak{m}_x \} = \{ h \in \mathbb{K}[W] \ \vert \ h(f(x))=0 \} = \mathfrak{m}_{f(x)},$$
was zu zeigen war. $\hfill \Box$
\end{pr}
\end{remark}

\begin{ex} %%Beispiel

Betrachte die Abbildung
$$f: \mathbb{A}^1(\mathbb{K}) \longrightarrow V(Y^2-X^3) \subseteq \mathbb{A}^2(\mathbb{K}), \quad x \mapsto (x^2, x^3)$$
Dann ist 
$$f^\#: \slant{\mathbb{K}[X,Y]}{\langle Y^2-X^3\rangle} \longrightarrow \mathbb{K}[T], \quad X\mapsto T^2, Y \mapsto T^3$$
Offensichtlich ist $f^\#$ Homomorphismus. Aber: Kein Isomorphismus, denn $f^\#$ ist zwar injektiv (nach Konstruktion), aber nicht surjektiv, da $T \notin \textrm{Bild}(f^\#)$.\\
Bemerkung: Bei Übergang in den Quotientenkörper existiert die Fortsetzung $\tilde{f}^\#$, da $\langle X^2-Y^3\rangle$ prim und damit $\slant{\mathbb{K}[X,Y]}{\langle Y^2-X^3\rangle}$ nullteilerfrei ist. Hier gilt $T=\tilde{f}^\#\left(\frac{y}{x}\right)$, $\tilde{f}^\#$ ist also Isomorphismus.
\end{ex}

\begin{theorem} %Satz 5.8
Sei $\mathbb{K}$ algebraisch abgeschlossener Körper. Dann ist 
$$\Phi: \textrm{\underline{Aff}}(\mathbb{K}) \longrightarrow \underline{ \mathbb{K}\textrm{-Alg}}^{\textrm{red}}, \quad V \mapsto \mathbb{K}[V]$$
eine Äquivalenz von Kategorien, das heißt, es existiert ein Funktor
$$\Psi: \underline{\mathbb{K}\textrm{-Alg}}^{\textrm{red}} \longrightarrow \underline{\textrm{Aff}}_{\mathbb{K}}$$
sodass $\Phi \circ \Psi$ und $\Psi \circ \Phi$ (als Funktoren) isomorph zur Identität sind.\\
\begin{pr}
Sei $A$ endlich erzeugte, reduzierte $\mathbb{K}$-Algebra und $a_1, \ldots, a_n$ Erzeuger von $A$ als $\mathbb{K}$-Algebra. Dann gibt es einen surjektiven Homomorphismus von $\mathbb{K}$-Algebren
$$\pi: \mathbb{K}[X_1, \ldots, X_n] \longrightarrow A, \quad X_i \mapsto a_i$$
Setze $V:=V(\ker(\pi))$. Dann ist 
$$\mathbb{K}[V] \cong \slant{\mathbb{K}[X_1, \ldots, X_n]}{I(V(\ker(\pi))} \overset{HNS}{=} \slant{\mathbb{K}[X_1, \ldots, X_n]}{\ker(\pi)} = A$$
\end{pr}
\end{theorem}



\renewcommand*\thesection{§ \arabic{section}\quad}
\section{Reguläre Funktionen} %PARAGRAPH 6
\renewcommand*\thesection{\arabic{section}}

In diesem Paragraph sei $\mathbb{K}$ stets ein algebraisch abgeschlossener Körper.

\begin{remark} %Bemerkung 6.1
Sei $V \subseteq \mathbb{A}^n(\mathbb{K})$ affine Varietät, $h \in \mathbb{K}[X_1, \ldots, X_n]$. Dann gilt
$$ \overline{h} \in \left(\mathbb{K}[V]\right)^{\times} \quad \Longleftrightarrow \quad V(h) \cap V = \emptyset$$
\begin{pr}
Wir erhalten folgende Kette von Äquivalenzen:\\
$V \cap V(h)=V(I(V)+\langle h \rangle ) \overset{!}{=} \emptyset$\\
$\Longleftrightarrow \ \ \textrm{HNS: } I(V) + \langle h \rangle = \mathbb{K}[X_1, \ldots, X_n]$ \\
$\Longleftrightarrow \ \ 1 = f + gh$ für ein $f \in I(V)$ und $g \in \mathbb{K}[X_1, \ldots, X_n]$\\
$\Longleftrightarrow \ \ 1 = \overline{g} \overline{h} \mod \mathbb{K}[V]$\\
$\Longleftrightarrow \ \ \overline{h} \in \left(\mathbb{K}[V]\right)^{\times}$.
\end{pr}
\end{remark}

\begin{defin} %Definition 6.2
Sei $V \subseteq \mathbb{A}^n (\mathbb{K})$ affine Varietät, $U \subseteq V$ offen, $x \in U$. 
\begin{compactenum}
\item Eine Abbildung $f:U \longrightarrow \mathbb{K}$ heißt \textit{regulär in} $x$, falls es eine offene Umgebung $U_x \subseteq U$ von $x$ und $g,h \in \mathbb{K}[V]$ gibt, sodass für alle $y \in U_x$ gilt
$$h(y) \neq 0 \quad \textrm{und} \quad f(y)=\frac{g(y)}{h(y)}$$
\item $f$ heißt \textit{regulär auf} $U$, falls $f$ regulär in $x$ ist für alle $x \in U$.
\item Die Menge der regulären Funktionen auf $U$
$$\mathcal{O}_V(U):= \{ f: U \longrightarrow \mathbb{K} \ \vert \ f \textrm{ ist reguläre Funktion auf } U \}$$
ist eine $\mathbb{K}$-Algebra.
\item Die Einschränkung $$\rho_U: \mathbb{K}[V] \longrightarrow \mathcal{O}_V(U), \quad f \mapsto f\vert _{U}$$ ist ein Homomorphismus von $\mathbb{K}$-Algebren.
\item $\rho_U$ ist injektiv genau dann, wenn $U$ dicht in $V$ ist.
\end{compactenum}
\begin{pr}[\it Beweis von (v)]
\begin{compactenum}
\item["$\Leftarrow$"] Sei $f \in \ker(\rho_U)$, also $f\vert_{U}=0$. Dann gilt $U \subseteq V(f)$, und da $V(f)$ abgeschlossen ist also auch $\overline{U} \subseteq V(f)$. Da $U$ dicht in $V$ ist erhalten wir $V=\overline{U} \subseteq V(f)$ und damit $f=0$ in $\mathbb{K}[V]$.
\item["$\Rightarrow$"] Angenommen es gelte $\overline{U} \neq V$. Wähle $x \in V \setminus \overline{U}$. Da $V(I(U))=\overline{U}$, existiert $f \in I(U)$ mit $f(x)\neq0$. Damit ist $f \neq 0$ in $\mathbb{K}[V]$ mit $f\vert_{U}=\rho_U(f)=0$, also ist $\rho_U$ nicht injektiv.
\end{compactenum}
\end{pr}
\end{defin}

\begin{ex}
\begin{compactenum}
\item $V=\mathbb{A}^1(\mathbb{K})$, $U=\mathbb{A}^1(\mathbb{K}) \setminus \{0 \}$. Dann ist $\frac{1}{X} \in \mathcal{O}_V(U)$. Setze dafür $g(y)=1$ und $h(y)=X$.
\item $V=V(Y^2-X^3)$, $U=V \setminus\{(0,0)\}$. Dann ist $\frac{Y}{X} \in \mathcal{O}_V(U)$.
\item $V= \mathbb{A}^n(\mathbb{K})$, $f \in \mathbb{K}[X_1, \ldots, X_n]$. Dann ist $\frac{1}{f} \in \mathcal{O}_V(D(f))$. 
\end{compactenum}
\end{ex}

\begin{propdefin} %Proposition + Definition 6.3

Für jede affine Varietät $V \subseteq \mathbb{A}^n(\mathbb{K})$ ist die Zuordnung $U \mapsto \mathcal{O}_V(U)$ für alle $U \subseteq V$ offen eine \textit{Garbe von Ringen} auf $V$. Das bedeutet im Einzelnen:
\begin{compactenum}
\item  Für offene Teilmengen $U' \subseteq U \subseteq V$ ist 
$$\rho^{U}_{U'}: \mathcal{O}_V(U) \longrightarrow \mathcal{O}_V(U'), \quad f \mapsto f\vert _{U'}$$
ein Homomorphismus von $\mathbb{K}$-Algebren und es gilt für $U''\subseteq U'\subseteq U \subseteq V$ offen:
$$\rho^{U}_{U''} \ =\ \rho^{U'}_{U''} \ \circ\ \rho^{U}_{U'}$$
\item Sei $U \subseteq V$ offen, $(U_i)_{i \in I}$ offene Überdeckung von $U$. Dann gilt
\begin{compactenum}
\item Für $f \in \mathcal{O}_V(U)$ ist $f=0$ \ $\Longleftrightarrow$ \ $\rho^{U}_{U_i}(f)=0$ für alle $i \in I$.
\item Ist für jedes $i \in I$ ein $f_i \in \mathcal{O}_V(U_i)$ gegeben, sodass
$$\rho^{U_i}_{U_i \cap U_j} (f_i) \ = \ \rho^{U_j}_{U_i \cap U_j} (f_j) \quad \textrm{ für alle } i,j \in I$$
so gibt es $f \in \mathcal{O}_V(U)$ mit $f_i=\rho^{U}_{U_i}(f)$ für alle $i \in I$.  
\end{compactenum}

\end{compactenum}
\end{propdefin}


\begin{prop} %Proposition 6.4
Sei $V\subseteq \mathbb{A}^n(\mathbb{K})$ affine Varietät. Dann gelten folgende Endlichkeitsaussagen:
\begin{compactenum}
\item Jede absteigende Kette von abgeschlossenen Teilmengen von $V$ wird stationär, d.h. $V$ ist \textit{noetherscher} topologische Raum.
\item Jede offene Überdeckung von $V$ besitzt eine endliche Teilüberdeckung, d.h. $V$ ist kompakt.
\item Jede offene Teilmenge von $V$ ist kompakt.
\end{compactenum}
\begin{pr}
\begin{compactenum}
\item Sei $V_1 \supseteq V_2 \supseteq V_3 \supseteq \ldots$ abgeschlossen in $V$, d.h. $V_j=V(I_j)$ mit Idealen $I_j \trianglelefteqslant \mathbb{K}[V]$ für all $j \in I$. Aus $V_j \supseteq V_{j+1}$ folgt $I_j \subseteq I_{j+1}$, also ist $I_1 \subseteq I_2 \subseteq I_3 \subseteq \ldots$ absteigende Kette von Idealen. Da $\mathbb{K}[V]$ noethersch ist, wird wird die Kette der Ideale stationär, so auch die $V_j$.
\item Folgt unmittelbar aus (iii).
\item Sei $(U_i)_{i\in I}$ offene Überdeckung von $U \subseteq V$ offen. Angenommen, es gibt eine Folge $(U_k)_{k\in \mathbb{N}} \subseteq (U_i)_{i\in I}$ mit
$$\bigcup_{n=1}^k U_n \neq U \quad \textrm{und}\quad U_{k+1} \nsubseteq \bigcup_{n=1}^k U_n \quad \textrm{ fÃŒr alle } k \in \mathbb{N}.$$
FÃŒr $$V_k := V \setminus \bigcup_{n=1}^k U_n$$
wäre $V_1 \supsetneq V_2 \supsetneq \ldots$ eine nicht stationär werdende, absteigende Kette von abgeschlossenen Teilmengen, ein Widerspruch zu (i). $\hfill \Box$
\end{compactenum}
\end{pr}
\end{prop}

\begin{theorem} %Satz 6.5
Sei $V\subseteq \mathbb{A}^n(\mathbb{K})$ affine Varietät, $f \in \mathbb{K}[V] \setminus\{0\}$. Dann ist 
$$\mathcal{O}_V\left(D(f)\right) \ \cong \ \mathbb{K}[V]_f$$
wobei $\mathbb{K}[V]_f$ die Lokalisierung von $\mathbb{K}[V]$ nach den Potenzen von $f$ bezeichne, also gerade
$$\mathbb{K}[V]_f =\bigg\{ \frac{g}{f^d} \ \big\vert \ g \in \mathbb{K}[V], \ d \geqslant 0 \bigg\}$$
Dabei gilt, da $\mathbb{K}[V]$ nicht notwendigerweise nullteilerfrei ist
$$\frac{g_1}{f^{d_1}} = \frac{g_2}{f^{d_2}} \ \Longleftrightarrow \ f^d \left( g_1 f^{d_2} - g_2 f^{d_1}\right)=0 \quad \textrm{für ein } d \geqslant 0$$
Insbesondere erhalten wir für $f=1$
$$O_V(V) \ \cong \ \mathbb{K}[V]$$
\begin{pr}
Definiere
$$\alpha: \mathbb{K}[V]_f \longrightarrow \mathcal{O}_V\left(D(f)\right), \quad \frac{g}{f^d}(y) \mapsto \frac{g(y)}{f(y)^d}$$
Zeige nun, dass $\alpha$ der gewünschte Isomorphismus ist.\\
\textit{wohldefiniert.} Seien dafür für $d_1, d_2 \geqslant 0, g_1, g_2 \in \mathbb{K}[V]$
$$\frac{g_1}{f^{d_1}} = \frac{g_2}{f^{d_2}} \textrm{ in } \mathbb{K}[V] \ \Longleftrightarrow \ f^d \left(g_1 f^{d_2} - g_2 f^{d_1}\right)=0 \textrm{ für ein } d\geqslant 0$$
Damit gilt für alle $y\in V$
$$f(y)^d \big( g_1(y) f(y)^{d_2} - g_2(y) f(y)^{d_1}\big) = 0,$$
wegen der Nullteilerfreiheit von $\mathbb{K}$ also 
$$g_1(y)f(y)^{d_2} - g_2(y)f(y)^{d_1} = 0 \ \Longleftrightarrow \ \frac{g_1(y)}{f(y)^{d_1}} = \frac{g_2(y)}{f(y)^{d_2}}$$
\textit{injektiv.} Sei
$$\frac{g}{f^d} \in \ker(\alpha) \ \Longleftrightarrow \ \alpha\bigg(\frac{g}{f^d}(y)\bigg) = \frac{g(y)}{f(y)^d} = 0 \ \textrm{ für alle } y \in D(f)$$
Dann ist $g(y)=0$ auf ganz $D(f)$, also $g \in I(D(f))$ und somit $f \cdot g=0$ auf $V$. Dann gilt
$$f \big(g \cdot 1 - 1 \cdot 0 \big)=0 \ \Longleftrightarrow \ \frac{g}{1} = \frac{0}{1}$$
und somit $g =0$. Folglich ist $\alpha$ injektiv.\\
\textit{surjektiv.} Sei $g \in \mathcal{O}_V(V)$. Finde $\tilde{g} \in \mathbb{K}[V]_f$ mit $\alpha(\tilde{g})=g$.\\
Für jedes $x \in D(f)$ gibt es offene Umgebungen $U_x \subseteq D(f), g_x, h_x \in \mathbb{K}[V]$, sodass gilt
$$g(y)= \frac{g_x(y)}{h_x(y)} \quad \textrm{ für alle } y \in U_x$$
Wegen 6.4 gibt es endlich viele $x_1, \dots x_m \in D(f)$ mit 
$$\bigcup_{i=1}^m U_{x_i} = D(f)$$
Setze $g_i:=g_{x_i}, h_i := h_{x_i}, U_i:=U_{x_i}$ für alle $1 \leqslant i \leqslant m$. Wegen $U_i \subseteq D(h_i)$ ist mit Komplementbildung
$$D(f)=\bigcup_{i=1}^m U_i \subseteq \bigcup_{i=1}^m D(h_i)$$
und damit
$$V(f) \supseteq \bigcap_{i=1}^m V(h_i) \ \Longleftrightarrow \ f \in I\bigg(\bigcap_{i=1}^m V(h_i) \bigg) = \sqrt{\langle h_1, \ldots, h_n \rangle}$$
Folglich finden wir $d \in \mathbb{N}$ mit 
$$f^d = \sum_{i=1}^m b_i h_i \quad \textrm{ für geeignete } b_i \in \mathbb{K}[V]$$
Sei $$\tilde{g} := \sum_{i=1}^m b_i g_i \in \mathbb{K}[V]$$
Dann gilt für $1 \leqslant j \leqslant m$ und $y \in U_j$:
$$g(y)=\frac{g_j(y)}{h_j(y)} = \frac{(g_jf^d)(y)}{(h_jf^d)(y)} = \frac{g_j \sum_{i=1}^m b_i h_i }{h_j f^d}(y) \overset{Beh.(i)}{=} \frac{\big(\sum_{i=1}^m b_i g_i\big) h_j}{h_j f^d} (y) = \frac{\tilde{g}}{f^d}(y) = \frac{\tilde{g}(y)}{f^d(y)}$$
Also $\alpha(\tilde{g})=g$.\\
Es bleibt zu zeigen:
$$g_j \bigg( \sum_{i=1}^m b_i h_i \bigg) = \bigg( \sum_{i=1}^m b_i g_i \bigg) h_j$$
also $g_i h_j = g_j h_i$ auf ganz $U_j$, nicht nur auf $U_j \cap U_i$. Dafür
\begin{compactenum}
\item[\textbf{Beh. (1)}] Ohne Einschränkung ist $g_ih_j = g_j h_i$ in $\mathbb{K}[V]$.
\item[\textbf{Beh. (2)}] Ohne Einschränkung ist $U_i=D(h_i)$.
\end{compactenum}
Nun folgt die Behauptung des Satz.
\begin{compactenum}
\item[\textbf{Bew.(1)}] Aus Beh. (2) folgt $g_i h_j = g_jh_i$ auf $U_i \cap U_j$, also gerade $D(h_i) \cap D(h_j) = D(h_i h_j)$.Weiter gilt
$$h_i h_j (g_ih_j - g_j h_i) = 0 \ \textrm{ auf } \mathbb{K}[V]\quad (*)$$
Setze
$$\tilde{g}_i:=g_i h_i, \quad \tilde{h}_i := h_i^2, \quad \tilde{g}_j:= g_j h_j. \quad \tilde{h}_j:= h_j^2$$
Dann wird $(*)$ zu
$$\tilde{g}_i \tilde{h}_j - \tilde{g}_j \tilde{h}_i = 0 \quad \textrm{ in } \mathbb{K}[V]$$
und es gilt
$$\frac{\tilde{g}_i}{\tilde{h}_i} = \frac{g_i}{h_i}, \qquad \frac{\tilde{g}_j}{\tilde{h}_j} = \frac{g_j}{h_j} \quad \textrm{ auf } U_i \cap U_j$$
wobei $U_i=D(h_i)$ und $D(h_j)=U_j$, also folgt die Behauptung.
\item [\textbf{Bew. (2)}] Es gilt $U_i \subseteq D(h_i)$. Es bilden die $\{D(h) \ \vert \ h \in \mathbb{K}[V]\}$ eine Basis der Zariski-Topologie, d.h. es existiert $h \in \mathbb{K}[V]$ mit $x_i \in D(h)$ und $D(h) \subseteq U_i$.\\
$\Longrightarrow \ D(h) \subseteq D(h_i) \ \Longrightarrow \ V(h) \supseteq V(h_i)$\\
$\Longrightarrow \ h \in I(V(h_i)) = \sqrt{h_i}$.\\
Damit finden wir $d \in \mathbb{N}$, sodass gilt
$$h^d = a \cdot h_i \quad \textrm{ für ein } a \in \mathbb{K}[V]$$
Ersetze nun $g_i$ durch $g_i \cdot a$, $h_i$ durch $h^d=a \cdot h_i$, $U_i$ durch $D(h_i)$ und setze $\tilde{g}_i, \tilde{h}_i$ wie oben. Dann gilt für $y \in D(h)$
$$g(y)=\frac{g_i}{h_i}(y)=\frac{g_i\cdot a}{h_i \cdot a}(y)=\frac{\tilde{g}_i}{\tilde{h}_i}(y),$$
es folgt die Behauptung. $\hfill \Box$

\end{compactenum}
\end{pr}

\end{theorem}

\begin{prop} %Proposition 6.6
Seien $V \subseteq \mathbb{A}^n(\mathbb{K})$, $W \subseteq \mathbb{A}^m(\mathbb{K})$ affine Varietäten, $f:V \longrightarrow W$ ein Morphismus, $U \subseteq W$ offen. Dann ist
$$f_U^\#: \mathcal{O}_W(U) \longrightarrow \mathcal{O}_V(f^{-1}(U)), \quad g \mapsto g \circ f$$
ein Homomorphismus von $\mathbb{K}$-Algebren.\\
\begin{pr}
Zu zeigen ist: $g \circ f \in \mathcal{O}_V(f^{-1}(U))$.
Seien dazu $g \in \mathcal{O}_W(U)$, $x \in f^{-1}(U)$, $y=f(x) \in U$. Nach Voraussetzung gibt es eine Umgebung $U_y\subseteq U$ von $y$, sodass
$$g = \frac{g_y}{h_y} \quad \textrm{ für geeignete } g_y, h_y \in \mathbb{K}[V]$$
Für $z \in f^{-1}(U_y) \subseteq f^{-1}(U)$ ist dann
$$(g \circ f )(z) = \frac{g_y(f(z))}{h_y(f(z))} = \frac{g_y \circ f }{h_y \circ f }(z) = \frac{f^\#(g_y) }{f ^\# (h_y)}(z)$$
mit $f^\#: \mathbb{K}[V] \longrightarrow \mathbb{K}[V], g \mapsto g \circ f$ wie gewöhnlich.
Damit ist $g \circ f $ regulär auf $f^{-1}(U_y)$ und damit insbesondere in $x$. $\hfill \Box$
\end{pr}
\end{prop}

\begin{bemdefin} %Bemerkung + Definition 6.7
\begin{compactenum}
\item Sind $\mathcal{F}, \mathcal{G}$ Garben, so ist ein \textit{Garbenmorphismus} $\Phi: \mathcal{F} \longrightarrow \mathcal{G}$ eine Kollektion von Morphismen $\phi_U: \mathcal{F}(U) \longrightarrow \mathcal{G}(U)$, welche mit der Einschränkungsabbildung verträglich sind.
\item Die Homomorphismen $f_U^\#$ für $U \subseteq W$ offen bilden einen Garbenmorphismus $$f^\#: \mathcal{O}_W \longrightarrow f_{*}\mathcal{O}_V, \quad U \mapsto (f_{*} \mathcal{O}_V)(U) = \mathcal{O}_v(f^{-1}(U))$$
d.h. für offene Mengen $U' \subseteq U \subseteq W$ ist das folgende Diagramm kommutativ:
$$
\begin{xy}
\xymatrix{
\mathcal{O}_W(U) \ar[rrr]^{\rho^{U}_{U'}} \ar[dd]^{f_U^\#} &&& \mathcal{O}_W(U') \ar[dd]^{f_{U'}^\#} \\
&&& \\
\mathcal{O}_V(f^{-1}(U)) \ar[rrr]^{\rho^{f^{-1}(U)}_{f^{-1}(U')}} &&& \mathcal{O}_V(f^{-1}(U'))
}
\end{xy}
$$
\end{compactenum}
\end{bemdefin}

\begin{lemma} %Lemma 6.8
Seien $V \subseteq \mathbb{A}^n(\mathbb{K}), W \subseteq \mathbb{A}^m(\mathbb{K})$ affine Varietäten. Dann ist eine Abbildung $f: V \longrightarrow W$ Morphismus genau dann, wenn für jedes offene $U \subseteq W$ und jedes $g \in \mathcal{O}_W(U)$ gilt: $g \circ f \in \mathcal{O}_V(f^{-1}(U))$.
\begin{pr}
\begin{compactenum}
\item["$\Rightarrow$"] Siehe 6.6
\item["$\Leftarrow$"] Zu zeigen: $p_i \circ f $ ist Polynom für jedes $i \in \{1, \ldots, n\}$, wobei $$p_i: \mathbb{A}^n(\mathbb{K}) \longrightarrow \mathbb{A}^1(\mathbb{K}), \quad (x_1, \ldots, x_n) \mapsto x_i$$
die Projektionen auf die $i$-te Komponente ist.\\
Es gilt $p_i \in \mathcal{O}_W(U)$ für jedes offene $U \subseteq W$, nach Voraussetzung also auch $p_i \circ f \in \mathcal{O}_V(f^{-1}(U))$.\\
Dann gilt $p_i \circ f \in \mathcal{O}_V(V) \overset{6.5}{=} \mathbb{K}[V]$. $\hfill \Box$
\end{compactenum}
\end{pr}
\end{lemma}

\begin{ex}   %%beispiel
Sei $U= \mathbb{A}^1(\mathbb{K}) \setminus \{0\}$. Dann ist $g:= \frac{1}{x} \in \mathcal{O}_{\mathbb{A}^1(\mathbb{K})}(U)$. Sei
$$f: U \longrightarrow \mathbb{A}^2(\mathbb{K}), \quad x \mapsto (x, g(x)) = \left(x, \frac{1}{x}\right)$$
Dann ist $$f(U) = V(XY-1) \subseteq \mathbb{A}^2(\mathbb{K})$$
Die Projektion
$$p_1: \mathbb{A}^2(\mathbb{K}) \longrightarrow \mathbb{A}^1(\mathbb{K}), \quad (x,y) \mapsto x $$
ist die Umkehrabbildung zu $f$.
\end{ex}

\begin{definbem} %Defintion + Bemerkung 6.9
\begin{compactenum}
\item Ein offene Teilmenge $U \subseteq \mathbb{A}^n(\mathbb{K})$ heißt \textit{quasiaffine Varietät}, wenn $U$ Zariski-offen in einer affinen Varietät $V \subseteq \mathbb{A}^n(\mathbb{K})$ ist.
\item Eine Abbildung $f: U_1 \longrightarrow U_2$ von quasiaffinen Varietäten heißt \textit{Morphismus} oder auch \textit{regulär}, falls $f$ stetig bezüglich der Zariski-Topologie ist und für jede offene Teilmenge $U \subseteq U_2$ und $g \in \mathcal{O}_{U_2}(U)$ gilt: $g \circ f \in \mathcal{O}_{U_1}(f^{-1}(U))$.
\item Seien $U_1 \subseteq \mathbb{A}^n(\mathbb{K}), U_2 \subseteq \mathbb{A}^m(\mathbb{K})$ quasiaffine Varietäten. Dann ist die Abbildung $f:U_1 \longrightarrow U_2$ genau dann regulär, wenn es reguläre Funktionen $f_1, \ldots, f_m \in \mathcal{O}_{U_1}(U_1)$ gibt, sodass
$$f(x)= \left(f_1(x), \ldots, f_m(x) \right) \quad \textrm{ für alle } x \in U_1$$
\item Die quasiaffinen Varietäten bilden zusammen mit den regulären Abbildungen eine Kategorie, von der $\textrm{\underline{Aff}}(\mathbb{K})$ eine volle Unterkategorie ist.
\item  Eine quasioffene Varietät heißt \textit{affin} (als abstrakte Varietät), falls sie isomorph zu einer affinen Varietät, also einer Zariski-abgeschlossenen Teilmenge des $\mathbb{A}^n(\mathbb{K})$ für ein $n \geqslant 1$ ist.
\end{compactenum}
\begin{pr}
\begin{compactenum}
\item[(iii)] Folgt aus 6.8.
\item[(iv)] Zeige dass für affine Varietäten die regulären Abbildungen bereits Morphismen sind (also dass $\textrm{\underline{Aff}}(\mathbb{K})$ eine volle Unterkategorie bildet). \\
Sei $f: V \longrightarrow W$ regulär zwischen affinen Varietäten $V$ und $W$. Mit (iii) folgt
$$f(x)= \left(f_1(x), \ldots, f_m(x) \right) \quad \textrm{ für alle } x \in V$$
mit $f_i \in \mathcal{O}_V(V)=\mathbb{K}[V]$. Dann folgt bereits, dass $f$ Morphismus ist. $\hfill \Box$
\end{compactenum}
\end{pr}
\end{definbem}


\begin{remark} %Bemerkung 6.10
Für $f \in \mathbb{K}[X_1, \ldots X_n]$ ist $D(f)$ affin als abstrakte Varietät.\\
\begin{pr}
Definiere
$$G:= f \cdot X_{n+1} -1 \in \mathbb{K}[X_1, \ldots, X_n , X_{n+1}]$$
und $V:=V(G) \subseteq \mathbb{A}^{n+1}(\mathbb{K})$.\\
Die Projektion $$\pi_{n+1}: \mathbb{A}^{n+1}(\mathbb{K}) \longrightarrow  \mathbb{A}^n(\mathbb{K}), \quad (x_1, \ldots, x_{n+1}) \mapsto (x_1, \ldots, x_n)$$
ist Morphismus mit $\pi_{n+1}(V) \subseteq D(f)$. Weiter ist 
$$\phi: D(f) \longrightarrow \mathbb{A}^{n+1}(\mathbb{K}), \quad x \mapsto \left(x, \frac{1}{f(x)}\right)$$
regulär mit $\phi(D(f)) \subseteq V$. $\pi_{n+1}, \phi$ sind invers zueinander, also gilt $D(f) \cong V$. $\hfill \Box$
\end{pr}
\end{remark}





\renewcommand*\thesection{§ \arabic{section}\quad}
\section{Rationale Abbildungen und Funktionenkörper} %PARAGRAPH 7
\renewcommand*\thesection{\arabic{section}}

Sei $\mathbb{K}$ weiterhin algebraisch abgeschlossen.

\begin{definbem} %Definition + Bemerkung 7.1
Sei $V \subseteq \mathbb{A}^n(\mathbb{K})$ quasiaffine Varietät.
\begin{compactenum}
\item Eine \textit{rationale Funktion} auf $V$ ist eine Äquivalenzklasse von Paaren $(U,f)$ mit $U \subseteq V$ offen und dicht sowie $f \in \mathcal{O}_V(U)$. Dabei sei 
$$(U_1, f_1) \sim (U_2, f_2) \quad \Longleftrightarrow \quad f_1\vert _{U_1 \cap U_2} = f_2 \vert _{U_1 \cap U_2}$$
\item In jeder Äquivalenzklasse gibt es eine bezüglich der Inklusion maximalen Vertreter $(U_{max}, f_{max})$. $U_{max}$ heißt \textit{Definitionsbereich} von $(U,f)_{\sim}$. $V\setminus U_{max}$ heißt \textit{Polstellenmenge} von $(U,f)_{\sim}$.
\item Die rationalen Funktionen auf $V$ bilden eine $\mathbb{K}$-Algebra.
\item Ist $V$ irreduzibel, so ist
$$\textrm{Rat}(V) \cong \mathbb{K}(V) = \textrm{Quot}(\mathbb{K}[V])$$
der \textit{Funktionenkörper} von $V$.
\end{compactenum}
\begin{pr}
\begin{compactenum}
\item Zu zeigen ist lediglich die Transitivität: Gelte $(U_1, f_1) \sim (U_2, f_2), (U_2, f_2) \sim (U_3, f_3)$.\\
Dann gilt per Definition $f_1 \vert _{U_1 \cap U_2 \cap U_3}= f_2 \vert _{U_1 \cap U_2 \cap U_3}$ und damit, da $U_1 \cap U_2 \cap U_3$ dicht in $U_1 \cap U_3$ ist: $f_1 \vert _{U_1 \cap U_3} = f_3 \vert _{U_1 \cap U_3}$.
\item Setze 
$U_{max}= \bigcup_{(U',f') \in (U,f)_{\sim}} U'.$
\item klar.
\item Sei
$$\alpha: \mathbb{K}(V) \longrightarrow \textrm{Rat}(V), \quad \frac{f}{g} \mapsto \left(D(g), \frac{f}{g} \right)_{\sim} $$
Dann ist $\alpha$ offensichtlich Homomorphismus von $\mathbb{K}$-Algebren.\\
\textit{injektiv:} klar.\\
\textit{surjektiv:} Sei $(U,f)_{\sim} \in \textrm{Rat}(V)$. Dann ist $f \in \mathcal{O}_V(U)$ und es existiert eine offene dichte Teilmenge $U' \subseteq U$ und $g,h \in \mathcal{O}_V(U)$, sodass gilt
$$f= \frac{g}{h} \quad \textrm{ auf  }\ U' \quad \Longleftrightarrow \quad \alpha\left(\frac{g}{h}\right)=f,$$
was zu zeigen war. $\hfill \Box$
\end{compactenum}
\end{pr}
\end{definbem}

\begin{ex}

Sei $U \subseteq V$ von der Form $U=D(h)$ für ein $h \in \mathbb{K}[V]$. Dann ist
$$\mathcal{O}_V(D(h)) = \mathbb{K}[V]_h = \left\{\frac{f}{g} \in \textrm{Quot}(\mathbb{K}[V]) \ \bigg\vert \ f \in \mathbb{K}[V], g = h^d \textrm{ für ein } d \in \mathbb{N}_0 \right\}$$
Dann ist 
$$\textrm{Quot}\left(\mathcal{O}_V(D(h))\right) = \textrm{Quot}(\mathbb{K}[V]_h) = \textrm{Quot}(\mathbb{K}[V])= \mathbb{K}(V),$$
denn es gilt 
$$\mathbb{K}[V] \subseteq \mathbb{K}[V]_h \subseteq \mathbb{K}[V]_{\mathbb{K}[V] \setminus \{0\}} = \textrm{Quot}(\mathbb{K}[V]).$$
\end{ex}

\begin{definbem} % Definition + Bemerkung 7.2

Seien $V,W$ quasi-affine Varietäten.
\begin{compactenum}
\item Eine \textit{rationale Abbildung} $f: V \dashrightarrow W$ ist eine Äquivalenzklasse von Paaren $(U,f)$ mit $U \subseteq V$ offen und dicht sowie $f: U \longrightarrow W$ reguläre Abbildung. Dabei gelte wieder
$$(U_1, f_1) \sim (U_2, f_2) \quad \Longleftrightarrow \quad f_1\vert_{U_1 \cap U_2} = f_2 \vert _{U_1 \cap U_2}$$
\item In jeder Äquivalenzklasse $(U,f)_{\sim}$ gibt es ein maximales $U:=\textrm{Def}(f)$. $U$ heißt \textit{Definitionsbereich}.
\item Rationale Abbildungen $f: V \dashrightarrow \mathbb{A}^1(\mathbb{K})$ entsprechen den rationalen Funktionen auf $V$.

\end{compactenum}
\end{definbem}

\begin{defin} %Definiton 7.3
Ein Morphismus $f:V \longrightarrow W$ von quasiaffinen Varietäten heißt \textit{dominant}, falls $f(V)\subseteq W$ dicht in $W$ ist.
\end{defin}

\begin{bemdefin} %Bemerkung + Definition 7.4

\begin{compactenum}
\item Die irreduziblen quasiaffinen Varietäten über $\mathbb{K}$ bilden mit den dominanten rationalen Abbildungen eine Kategorie.
\item Isomorphismen in dieser Kategorie heißen \textit{birationale Abbildungen}.
\end{compactenum}
\begin{pr}[ \it Beweis von (i).]
Sind $f: V \dashrightarrow W, g : W \dashrightarrow Z$ dominante rationale Abbildungen irreduzibler Varietäten $V,W,Z$, so ist $f^{-1}(\textrm{Def}(g))$ offen und nichtleer, da $f$ dominant ist.\\
Damit ist $U:= f^{-1}(\textrm{Def}(g))$ dicht in $V$.\\
$\Longrightarrow U \subseteq \textrm{Def}(g \circ f)$, also ist $g \circ f$ rationale Abbildung. $\hfill \Box$
\end{pr}
\end{bemdefin}

\begin{ex} %Beispiel 7.5

\begin{compactenum}
\item Sei $V=V(XY) \subseteq \mathbb{A}^2(\mathbb{K})$,
$$f: V \longrightarrow \mathbb{A}^1(\mathbb{K}), \quad (x,y) \mapsto x$$
$$g: \mathbb{A}^1(\mathbb{K}) \dashrightarrow \mathbb{A}^1(\mathbb{K}), \quad x \mapsto \frac{1}{x}$$
Dann ist $f$ surjektiv, $g$ ist dominante rationale Abbildung. Aber es gilt $\textrm{Def}(g \circ f)=V \setminus V(X)$ ist nicht dicht in $V$, also ist $g \circ f$ keine rationale Abbildung.
\item Betrachte 
$$ \sigma: \mathbb{A}^2(\mathbb{K}) \dashrightarrow \mathbb{A}^2(\mathbb{K}), \quad (X,Y) \mapsto \left(\frac{1}{X}, \frac{1}{Y}\right)$$
$\sigma$ ist rationale Abbildung mit $\textrm{Def}(\sigma)= D(XY), \sigma^2=\textrm{id}$ als birationale Abbildung. Damit ist $\sigma$ eine birationale Abbildung.  
\end{compactenum}
\end{ex}

\begin{prop} %Proposition 7.6

Sei $f: V \longrightarrow W$ Morphismus von affinen Varietäten und $$f^\#: \mathbb{K}[V] \longrightarrow \mathbb{K}[V], \quad g \mapsto g \circ f$$
der induzierte $\mathbb{K}$-Algebren Homomorphismus der zwischen den Koordinatenringen. Dann gilt
$$f^\# \textrm{ ist injektiv} \quad \Longleftrightarrow \quad f \textrm{ ist dominant }$$
\begin{pr}
\begin{compactenum}
\item[\textbf{Beh. (1)}] Für $Z \subseteq W$ abgeschlossen gilt
$$(f^\#)^{-1} (I(Z)) = I(\overline{f(Z)}) = I(f(Z))$$
\item[\textbf{Bew. (1)}] Es gilt: $g \in (f^\#)^{-1}(I(Z))$ \\
$\Longleftrightarrow \ f^\#(g) \in I(Z)$\\
$\Longleftrightarrow \ g \circ f \in I(Z) $\\
$\Longleftrightarrow \ g(f(z)) = 0 \textrm{ für alle } z \in Z$ \\
$\Longleftrightarrow \ g(y) = 0 \textrm{ für alle } y \in f(Z) $\\
$\Longleftrightarrow \ g \in I(\overline{f(Z)}) = I(f(Z))$
\end{compactenum}
Damit gilt für $Z=V$ wegen $I(V)=0$ in $\mathbb{K}[V]$ mit Beh. (1):
$$ (f^\#)^{-1}(0)= (f^\#)^{-1}(I(V)) = I(\overline{f(V)}) \overset{dom.}{=} I(W) = 0$$
Also gerade $\ker(f^\#)=\{0\}$, also ist $f^\#$ injektiv. $\hfill \Box$
\end{pr}
\end{prop}

\begin{folg} %Folgerung 7.7

Jede dominante Abbildung $f: V \dashrightarrow W$ zwischen irreduziblen quasiaffinen Varietäten induziert einen Körperhomomorphismus $f^\#: \mathbb{K}(W) \longrightarrow \mathbb{K}(V)$.\\
\begin{pr}
Seien $V,W$ affin. Ist $f$ Morphismus, so ist 
$$f^\#: \mathbb{K}[W] \longrightarrow \mathbb{K}[V], \quad g \mapsto g \circ f$$
injektiv und lässt sich damit fortsetzen zu 
$$f^\#: \mathbb{K}(W) \longrightarrow \mathbb{K}(V), \quad \frac{g}{h} \mapsto \frac{f^\#(g)}{f^\#(h)}$$
Ist $\textrm{Def}(f) \neq V$, so sei $g \in \mathbb{K}[V]$ mit $D(g) \supseteq \textrm{Def}(f)$. Für $h \in \mathbb{K}[W]$ ist dann
$$f^\#(h) = h \circ f \in \mathcal{O}_V(D(g)),$$
also induziert $f$ einen Homomorphismus
$$f^\#: \mathbb{K}[W] \longrightarrow \mathcal{O}_V(D(g)) = \mathbb{K}[V]_g$$
$D(g)$ ist nach 6.10 affin, mit 7.6 folgt also die Injektivität von $f^\#$. Damit existiert die Fortsetzung
$$f^\#: \mathbb{K}(W) \longrightarrow \textrm{Quot}(\mathbb{K}[V]_g)=\textrm{Quot}(\mathbb{K}[V])=\mathbb{K}(V)$$
\end{pr}
\end{folg}

\begin{theorem} %Satz 7.8

Ist $\mathbb{K}$ algebraisch abgeschlossen, so ist die Zuordnung
\begin{alignat*}{5}
\Phi: \begin{Bmatrix} \textrm{irreduzible, quasi-affine Varietäten} \\ \textrm{dominante rationale Abbildungen} \end{Bmatrix} \quad &\longrightarrow \quad &&\begin{Bmatrix} \mathbb{L}/\mathbb{K} \textrm{ endlich erzeugt } \\ \mathbb{K}\textrm{-Algebra Homomorphismen} \end{Bmatrix} \\[12pt]
\begin{Bmatrix} V \\ f: V \dashrightarrow W \end{Bmatrix} \quad &\mapsto \quad && \begin{Bmatrix} \mathbb{K}(V) \\ f^\#: \mathbb{K}(W) \longrightarrow \mathbb{K}(V) \end{Bmatrix}
\end{alignat*}
eine Äquivalenz von Kategorien.
\begin{pr}
Offensichtlich ist $\Phi$ ein Funktor. Zu zeigen bleibt also noch 
\begin{compactenum}
\item Zu jeder endlich erzeugten Körpererweiterung $\mathbb{L}/\mathbb{K}$ gibt es $V$ mit $\mathbb{K}(V)\cong \mathbb{L}$.
\item Die Zuordnung 
$$\textrm{Rat}^{\textrm{Dom}}(V,W) \longrightarrow \textrm{Hom}_{\mathbb{K}}(\mathbb{K}(W), \mathbb{K}(V)), \quad f \mapsto f^\#$$
ist eine Bijektion.
\item[\textit{zu} (i)] Seien $x_1, \ldots, x_n$ Erzeuger von $\mathbb{L}$ über $\mathbb{K}$ und $A:= \mathbb{K}[x_1, \ldots, x_n]$ die von den $x_i$ erzeugte $\mathbb{K}$-Algebra. $A$ ist als solche offenbar endlich erzeugt und reduziert, da $A$ Teilmenge eines Körpers ist. Damit existiert eine affine Varietät $V$ mit $A \cong \mathbb{K}[V]$. Da $A$ nullteilerfrei ist, ist $V$ sogar irreduzibel und damit
$$\mathbb{K}(V) = \textrm{Quot}(\mathbb{K}[V]) \cong \textrm{Quot}(A) = \mathbb{L}$$
\item[\textit{zu} (ii)] Es gilt\\
\textit{injektiv.} Seien $f,g:V \dashrightarrow W$ mit $f^\# = g^\#$. Wähle $U=D(h) \subseteq \textrm{Def}(f) \cap \textrm{Def}(g)$ offen und affin. $f\vert _U$ und $g \vert_U$ sind Morphismen von $U$ nach $W$.\\
Diese induzieren $\mathbb{K}$-Algebren Homomorphismen
$$g_U^\#, f_U^\#: \mathbb{K}[W] \longrightarrow \mathbb{K}[U] \subseteq \mathbb{K}(V)$$
\textit{surjektiv.} Sei 
$$\alpha: \mathbb{K}(W) \longrightarrow \mathbb{K}(V)$$
ein $\mathbb{K}$-Algebren Homomorphismus. Wähle Erzeuger $g_1, \ldots, g_n$ von $\mathbb{K}[W]$ also $\mathbb{K}$-Algebra. Für jedes $1 \leqslant i \leqslant n$ ist $\alpha(g_i)$ rationale Funktion auf $V$.\\
Da $V$ irreduzibel ist, ist 
$$\bigcap_{i=1}^n \textrm{Def}(\alpha(g_i))$$
offen und affin für geeignete $g \in \mathbb{K}[V]$. Nach Konstruktion induziert $\alpha$ einen $\mathbb{K}$-Algebren Homomorphismus
$$\alpha: \mathbb{K} \longrightarrow \mathcal{O}_U(U) = \mathbb{K}[U]$$
Damit gilt nach 5.8 gilt dann $\alpha= f^\#$ für einen Morphismus $f: U \longrightarrow W$.\\
Da außerdem $U$ dicht in $V$ ist, ist $(U,f)$ rationale Abbildung, denn $f$ ist dominant, da $f^\#$ als Körperhomomorphismus injektiv ist. $\hfill \Box$
\end{compactenum}
\end{pr}
\end{theorem}






% KAPITEL II


\chapter{Projektive Varietäten}


\setcounter{section}{7}

\renewcommand*\thesection{§ \arabic{section}\quad}
\section{Varietäten im projektiven Raum} %PARAGRAPH 8
\renewcommand*\thesection{\arabic{section}}

\begin{erinner} %Erinnerung 8.1
 Sei $\mathbb{K}$ ein Körper, $n \in \mathbb{N}_0$.
 \begin{compactenum}
 \item Der \textit{projektiven Raum} ist
 $$\mathbb{P}^n(\mathbb{K}):= \slant{\mathbb{K}^{n+1}}{_\sim}$$
 mit
 $$(x_0, \ldots, x_n) \sim (y_0, \ldots, y_n) \quad \Longleftrightarrow \quad \textrm{ es ex. } \lambda \in \mathbb{K}^{\times} \textrm{ mit } \lambda x_i = y_i \textrm{ für alle } 0\leqslant i \leqslant n$$
 Anschaulich sind die Elemente des projektiven Raums gerade die Ursprungsgeraden des $\mathbb{K}^{n+1}$.\\
 Schreibeweise: Es sei $(x_0: \ldots :x_n) \in \mathbb{P}^n(\mathbb{K})$ die Äquivalenzklasse von $(x_0, \ldots, x_n) \in \mathbb{K}^{n+1}$.
 \item Für $i \in \{0,\ldots, n \}$ sei 
 $$U_i:= \{(x_0: \ldots x_n) \in \mathbb{P}^n(\mathbb{K}) \ \vert \ x_i \neq 0\}$$
 Es gilt $\mathbb{P}^n(\mathbb{K}) = U_0 \cup \ldots \cup U_n$. Für ein festes $i \in \{0, \ldots, n\}$ betrachte
 $$\psi_i: \mathbb{K}^{n} \longrightarrow \mathbb{P}^n(\mathbb{K}), \quad (y_1, \ldots, y_n) \mapsto (y_1: \ldots y_i : 1 : y_{i+1} : \ldots y_n)$$
 Offenbar ist $\psi_i$ injektiv mit $\textrm{Bild}(\psi_i)=U_i$. Die Umkehrabbildung ist
 $$\phi_i: U_i \longrightarrow \mathbb{K}^n, \quad (x_0: \ldots : x_n) \mapsto \left( \frac{x_0}{x_i}, \ldots, \frac{x_{i-1}}{x_i}, \frac{x_{i+1}}{x_i}, \ldots, \frac{x_n}{x_i}\right)$$
 \item Die Abbildung
 $$\rho_i: \mathbb{P}^n(\mathbb{K}) \setminus U_i \longrightarrow \mathbb{P}^{n-1}(\mathbb{K}), \quad (x_0: \ldots: x_n) \mapsto (x_0:\ldots x_{i-1} : x_{i+1}: \ldots x_n)$$
 ist bijektiv. Induktiv erhalten wir
 $$\mathbb{P}^n(\mathbb{K}) = \mathbb{K}^n \cup \mathbb{K}^{n-1} \cup \dots \cup \mathbb{K}^2 \cup \mathbb{K} \cup \{\infty \}$$
 wobei die Wahl von $\mathbb{\infty}$ willkürlich ist. Insbesondere gilt also 
 $$\mathbb{P}^1(\mathbb{K}) = \mathbb{K} \cup \{\infty \}$$
 \item $\mathbb{P}^n(\mathbb{R})$ und $\mathbb{P}(\mathbb{C})$ sind $n$-dmensionale Mannigfaltigkeiten.
 
 \end{compactenum}
 \end{erinner}
 
\begin{definbem} %Definition + Bemerkung 8.2
 
 Sei $\mathbb{K}$ ein Körper, $n \in \mathbb{N}_0$.
 \begin{compactenum}
 \item Ein Polynom
 $$f=\sum_{(i_0 \ldots i_n) \in \mathbb{N}_0^{n+1}} ^{< \infty} a_{i_0 \ldots i_n} X_0^{i_0} \ldots X_n^{i_n} \ \ \in \mathbb{K}[X_0, \ldots, X_n]$$
 heißt \textit{homogen von Grad} $d\geqslant 0$, falls für alle nichtverschwindenden Koeffizienten der Gesamtgrad konstant ist, also 
 $$a_{i_0 \ldots i_n} \neq 0 \quad \Longrightarrow \quad i_0 + \ldots + i_n =d \quad \textrm{ für alle } i$$
 \item Ist $f \in \mathbb{K}[X_0, \ldots, X_n]$ homogen von Grad $d$, so gilt für alle $x=(x_0, \ldots x_n) \in \mathbb{K}^{n+1}$ und $\lambda \in \mathbb{K}^{\times}$:
 $$f(\lambda x_0, \ldots, \lambda x_n ) = \lambda^d f(x_0, \ldots, x_n)$$
 \item Ist $f \in \mathbb{K}[X_1, \ldots, X_n]$ homogen, so ist die Nullstellenmenge $V(f)\subseteq \mathbb{P}^n(\mathbb{K})$ wohldefiniert.
 \end{compactenum}
 \end{definbem}
 
\begin{defin} %Definition 8.3
 
 Ein Teilmenge $V \subset \mathbb{P}^n(\mathbb{K})$ heißt \textit{projektive Varietät}, wenn es eine Menge $\mathcal{F} \subseteq \mathbb{K}[X_0, \ldots, X_n]$ von homogenen Polynomen gibt, sodass
 $$V=\{x=(x_0, \ldots, x_n) \in \mathbb{P}^n(\mathbb{K}) \ \vert \ f(x) = 0 \textrm{ für alle } f \in \mathcal{F}\}$$
 \end{defin}
 
\begin{ex}
\begin{compactenum}
\item Für $i \in \{0, \ldots, n\}$ ist
$$V(X_i) = \mathbb{P}^n(\mathbb{K}) \setminus U_i \cong \mathbb{P}^{n-1}(\mathbb{K})$$
eine projektive Varietät. 
\item Es gilt $V(X_0, \ldots, X_n) = \emptyset$.
\end{compactenum}
\end{ex}

\begin{remark} %Bemerkung 8.4

Ist $V \subseteq \mathbb{P}^n(\mathbb{K})$ projektive Varietät, so ist 
$$\phi_i(V \cap U_i) \subseteq \mathbb{A}^n(\mathbb{K})$$
affine Varietät für alle $i \in \{0, \ldots, n \}$.
\begin{pr}
Es genügt, die Aussage für $V(f)$, $f \in \mathbb{K}[X_0, \ldots, X_n]$ homogen zu zeigen, denn:
$$V(\mathcal{F})= \bigcap_{f\in \mathcal{F}}V(f) \quad\Longrightarrow \quad \phi_i(V \cap U_i) = \bigcap_{f \in \mathcal{F}} \phi_i(V(f) \cap U_i)$$
Sei nun $$\tilde{f}:=f(X_0, \ldots, X_{i-1}, 1, X_{i+1}, \ldots, X_n) \in \mathbb{K}[X_0, \ldots, X_{i-1}, X_{i+1}, \ldots, X_n] =\mathbb{K}[Y_1, \ldots, Y_n]$$
\begin{compactenum}
\item[\textbf{Beh. (1)}] Es gilt $V(\tilde{f})=\phi_i(V(f) \cap U_i)$.
\item[\textbf{Bew. (1)}]
Wir haben
\begin{compactenum}
\item["$\supseteq$"] Sei $x \in V(f)\cap U_i$, $x=(x_0 \vert \ldots \vert x_n)$. Dann gilt
$$x_i \neq 0, \quad \phi_i(x) = \left(\frac{x_0}{x_i}, \ldots, \frac{x_{i-1}}{x_i}, \frac{x_{i+1}}{x_i}, \ldots, \frac{x_n}{x_i} \right)$$
Also
$$\tilde{f}(\phi_i(x)) = f \left(\frac{x_0}{x_i}, \ldots, \frac{x_{i-1}}{x_i}, 1 , \frac{x_{i+1}}{x_i}, \ldots, \frac{x_n}{x_i} \right) = \frac{1}{x_i^d} f(x_0, \ldots, x_{i-1}, x_i, x_{i+1}, \ldots, x_n) =0$$
\item["$\subseteq$"] Sei nun $y=(y_1, \ldots, y_n) \in V(\tilde{f})$. Dann gilt
$$\tilde{f}(y_1, \ldots, y_n) = f(y_1, \ldots, y_i, 1, y_{i+1}, \ldots, y_n) = 0$$
Also gilt $x:=(y_1: \ldots :y_i : 1 :y_{i+1}: \ldots :y_n) \in U_i \cap V(f)$ und $\phi_i(x)=y$, also gerade die Behauptung. $\hfill \Box$
\end{compactenum}
\end{compactenum}
\end{pr}
\end{remark}

\begin{ex}

Betrachte $V=V(X_0X_2 - X_1^2) \subseteq \mathbb{P}^2(\mathbb{K})$.
Es gilt
$$\phi_0(V \cap U_0) = V(X_2 - X1_1^2) \qquad \textit{Parabel}$$
$$\phi_1(V \cap U_1) = V(X_0X_2-1) \qquad \textit{Hyperbel}$$
$$\phi_2(V \cap U_2) = V(X_0 - X_1^2) \qquad \textit{Parabel}$$
\end{ex}

\begin{remark}  %Bemerkung 8.5

Zu jeder affine Varietät $V \subseteq \mathbb{A}^n(\mathbb{K})$ gibt es eine projektive Varietät $\tilde{V}_i \subseteq \mathbb{P}^n(\mathbb{K})$ mit $\phi_i(\tilde{V}_i \cap U_i)=V$.
\begin{pr}
Sei ohne Einschränkung $V=V(f)$ für ein $f \in \mathbb{K}[Y_1, \ldots, Y_n]$. Schreibe
$$f = \sum_{k=0}^d f_k$$
mit homogenen Polynomen $f_k$ von Grad $k$ für $1 \leqslant k \leqslant d$, $d=\deg(f)$. Sei
$$F:= \sum_{k=0}^d X_i^{d-k} f_k \in \mathbb{K}[Y_1, \ldots, Y_i, X_i, Y_{i+1}, \ldots, Y_n]$$
Dann ist $F$ homogen von Grad $d$ und es gilt:
\begin{compactenum}
\item[\textbf{Beh. (1)}] Es gilt $\phi_i(V(F) \cap U_i) = V(f)$.
\item[\textbf{Bew. (1)}] Wir haben
\begin{compactenum}
\item["$\supseteq$"] Sei $y:=(y_1, \ldots, y_n) \in V(f)$, d.h es gilt $f(y)=0$. Setze
$$x:=\psi_i(y)=\left(y_1: \ldots : y_i : 1 : y_{i+1}: \ldots y_n \right) \in U_i, \quad \phi_i(x)=y.$$
Dann gilt
$$F(x)= \sum_{k=0}^d X_i^{d-k} f_k(y_1, \ldots, y_n) = f(y) = 0.$$
\item["$\subseteq$"] Sei nun $y \in \phi_i ( V(F) \cap U_i)$, d.h. es gilt $y= \phi(x)$ mit $x \in V(F) \cap U_i$.\\
Damit gilt $x= (x_1: \ldots : x_{i}:1:x_{i+1}: \ldots : x_n )$ und
$$0\ = \ F(x)=  \ \sum_{k=0}^d f_k(x_1, \ldots, x_n) \ = \ f(x_1, \ldots, x_n)= f(\phi_i(x)) = f(y),$$
also $y \in V(f)$. $\hfill \Box$
\end{compactenum}
\end{compactenum}
\end{pr}
\end{remark}

\begin{definbem} %Definition + Bemerkung 8.6

Sei $\mathbb{K}$ ein Körper, $n \geqslant 1$.
\begin{compactenum}
\item Für $i \in \{1, \ldots, n\}$ heißt die Abbildung
$$
\begin{matrix}[rcl]
 D_i: \mathbb{K}[X_0, \ldots, X_n] \quad & \longrightarrow \quad & \mathbb{K}[X_0, \ldots, X_{i-1}, X_{i+1}, \ldots, X_n] \cong \mathbb{K}[Y_1, \ldots, Y_n] , \\
f(x_0, \ldots, x_n) \quad &\mapsto \quad  &f(x_0, \ldots, x_{i-1}, 1, x_{i+1}, \ldots, x_n)
\end{matrix}
$$
\textit{Dehomogenisierung} nach der $i$-ten Variable. $D_i$ ist als Auswertung ein $\mathbb{K}$-Algebren Homomorphismus.
\item Für $i \in \{1, \ldots, n\}$ heißt die Abbildung
$$
\begin{matrix}[rcl]
H_i: \mathbb{K}[Y_1, \ldots, Y_n] & \longrightarrow & \mathbb{K}[Y_1, \ldots, Y_n, X_i] \cong \mathbb{K}[X_0, \ldots, X_n] \\
f = \sum_{k=0}^d f_k & \mapsto & \sum_{k=0}^d X_i^{d-k} f_k 
\end{matrix}
$$
($i$-te)\textit{Homogenisierung}, wobei $f_k$ homogene Polynoms von Grad $k$ sind. Es gilt
$$H_i(fg) = H_i(f) H_i(g)$$
$$H_i(f+g) \neq H_i(f) + H_i(g), \qquad \textrm{ falls } \deg(f) \neq \deg(g)$$
\item Es gilt
$$D_i \circ H_i = \textrm{id}_{\mathbb{K}[Y_1, \ldots, Y_n]}$$
$$(H_i \circ D_i)(f) = \frac{1}{X_i^{e}} f, \quad e= \max_{e \in \mathbb{N}_0} \{ X_i^{e}\ \vert \ X_i^{e} \mid f, X_i^{e+1} \nmid f \}, \ \ \textrm{ falls } f \textrm{ homogen. }$$
\end{compactenum}
\end{definbem}



\renewcommand*\thesection{§ \arabic{section}\quad}
\section{Die Zariski Topologie auf $\mathbb{P}^n(\mathbb{K})$} %PARAGRAPH 9
\renewcommand*\thesection{\arabic{section}}

\begin{defin} %Definition 9.1

Für $V \subseteq \mathbb{P}^n(\mathbb{K})$ sei $I(V) \trianglelefteqslant \mathbb{K}[X_0, \ldots, X_n]$ das von allen homogenen Polynomen $f \in \mathbb{K}[X_0, \ldots, X_n]$ mit $f(x)=0$ für alle $x \in V$ erzeugte Ideal. $I(V)$ heißt \textit{Verschwindungsideal} von $V$.
\end{defin}

\begin{definbem} %Definiton 9.2
\begin{compactenum}
\item Ein (kommutativer) Ring (mit 1) $R$ heißt \textit{graduiert}, falls es eine Zerlegung
$$R= \bigoplus_{d=0}^{\infty} R_d$$
in abelsche Gruppen $R_d$ gibt, sodass für alle $f \in R_d, g \in R_e$ gilt: $f\cdot g \in R_{d+e}$.
\item eine $\mathbb{K}$-Algebra $S$ heißt \textit{graduiert}, wenn
$$S=\bigoplus_{d=0}^{\infty} S_d$$
graduierter Ring ist und $S_0 = \mathbb{K}$. Dies impliziert, dass die $S_d$ sogar zu $\mathbb{K}$-Vektorräumen werden.
\item Die Elemente in $R_d$ bzw. $S_d$ heißen \textit{homogen vom Grad $d$}.
\item Ein Ideal in $R$ heißt \textit{homogen}, wenn es von homogenen Elementen erzeugt werden kann.
\item Für ein Ideal $I \trianglelefteqslant R$ sind äquivalent:
\begin{compactenum}
\item[(a)] $I$ ist homogen.
\item[(b)] $I$ besitzt eine Darstellung$$I= \bigoplus_{d=0}^{\infty} (I \cap R_d)$$
\item[(c)] Für jedes $f \in I$ mit  $$f= \sum_{d=0}^{\infty} f_d, \quad f_d \in R_d$$ gilt bereits $f_d \in I$ für alle $d \in \mathbb{N}_0$.
\end{compactenum}
\item Ist $I \trianglelefteqslant R$ homogenes Ideal, so ist $\slant{R}{I}$ graduiert mit
$$\slant{R}{I} = \bigoplus_{d=0}^{\infty} \slant{R_d}{(R_d \cap I)}$$
\item Summe, Produkt, Durchschnitt und Radikal von homogenen Idealen sind wieder homogen.
\end{compactenum}
\begin{pr}
\begin{compactenum}
\item[(v)] \begin{compactenum}
\item["(a)$\Rightarrow$(b)"] \begin{compactenum}
\item["$\supseteq$"] Klar.
\item["$\subseteq$"] Seien $a_i, i \in J$ homogene Erzeuger von $I$. Es genügt zu zeigen:
$$r \cdot a_i \in \bigoplus_{d=0}^{\infty} I \cap R_d \qquad \textrm{ für alle } r \in R$$
Schreibe
$$r= \sum_{d=1}^n r_d, \qquad r_d \in R_d$$
Dann gilt mit $d_i :=\deg(a_i)$
$$r\cdot a_i = \sum_{d=1}^n r_d a_i, \quad r_da_i \in R_{d+d_i} \cap I$$
also gerade die Behauptung.
\end{compactenum}
\item["(b)$\Rightarrow$(c)"] Klar.
\item["(c)$\Rightarrow$(a)"] Klar.
\end{compactenum}
\item[(vi)] Für jedes Ideal $I \trianglelefteqslant R$ ist 
$$\pi: \bigoplus_{d=0}^{\infty} \slant{R_d}{(R_d \cap I)} \longrightarrow \slant{R}{I}$$
surjektiv, denn für $d \in \mathbb{N}_0$ ist $R_d \longrightarrow R$ surjektiv. Für den Kern betrachte
$$\sum_{d=0}^n r_d \mod (R_d \cap I) \in \ker(\pi) \ \ \Longleftrightarrow \ \ \sum_{d=0}^n r_d \in I \ \ \Longleftrightarrow \ \ r_d \in I \ \ \Longleftrightarrow \ \ \sum_{d=0}^n r_d \equiv 0 \mod (R_d \cap I)$$
Damit folgt die Behauptung.
\item[(vii)] Seien $I_1, I_2$ homogene Ideale, also mit homogenen Erzeugern $\{f_i\}, \{g_j\}$.\\
Dann wird $I_1 + I_2$ von $\{f_i + g_j\}$ erzeugt und $I_1 I_2$ von $\{f_i g_j\}$.\\
\textit{Durchschnitt.} Für $I_1 \cap I_2$ verwende (v)(b):
$$\bigoplus_{d=0}^{\infty} ((I_1 \cap I_2) \cap R_d) = \bigoplus_{d=0}^{\infty} \left(( I_1 \cap R_d) \cap (I_2 \cap R_d)\right) = \bigoplus_{d=0}^{\infty} (I_1 \cap R_d)\ \cap\ \bigoplus_{d=0}^{\infty} I_2 \cap R_d \ = \ I_1 \cap I_2$$
\textit{Radikal.} Sei nun $I$ homogen, $x \in \sqrt{I}$. Schreibe
$$x= \sum_{d=0}^n x_d, \quad x_d \in R_d$$
Nach Voraussetzung existiert $m \geqslant 1$, sodass $x^m \in I$, also
$$I \ni \left(\sum_{d=0}^n d_d \right)^m = x_n^m + \mathcal{O}(x_n^{m-1})$$
Damit gilt $x_n^m \in I$ und somit $x_n \in \sqrt{I}$ und $(x-x_n) \in \sqrt{I}$.\\
Per Induktion über $\deg(x)$ folgt nun die Behauptung. $\hfill \Box$
\end{compactenum}
\end{pr}
\end{definbem}

\begin{prop} %Proposition 9.3
\begin{compactenum}
\item Für jede Teilmenge $V \subseteq \mathbb{P}^n(\mathbb{K})$ ist $I(V)$ ein Radikalideal.
\item Die projektiven Varietäten bilden die abgeschlossenen Mengen der \textit{Zariski-Topologie} auf $\mathbb{P}^n(\mathbb{K})$.
\item Eine projektive Varietät $V$ ist irreduzibel genau dann, wenn $I(V)$ ein Primadeal ist.
\item Jede projektive Varietät ist endliche Vereinigung ihrer irreduziblen Komponenten.
\end{compactenum}
\begin{pr}
\begin{compactenum}
\item Zu zeigen ist: $ \sqrt{I(V)} \subseteq I(V)$.\\
Nach 9.2 (vii) ist $\sqrt{I(V)}$ ein homogenes Ideal. Sei also $f \in \sqrt{I(V)}$ homogen und $m \in \mathbb{N}$, sodass
$$f^m \in I(V) \quad \Longrightarrow \quad f(x)^m = 0 \textrm{ für alle } x \in V$$
Damit gilt $f \in I(V)$, also die Behauptung.
\item Folgt wie im affine Fall aus 9.2 (vii).
\item Wörtlich wie in 3.5 mit gelöster Übungsaufgabe.
\item Wie in 3.6 $\hfill \Box$
\end{compactenum}
\end{pr}
\end{prop}

\begin{folg} %Folgerung 9.4

Bezüglich der Einschränkung der Zariskitopologie von $\mathbb{P}^n(\mathbb{K})$ auf $U_i$ ist die Bijektion $\phi_i: U_i \longrightarrow \mathbb{A}^n(\mathbb{K})$ ein Homoömorphismus.
\begin{pr}
Folgt aus Bemerkung 8.4 und 8.5.
\end{pr}
\end{folg}

\begin{remark} %Bemerkung 9.5

Sei $V \subseteq \mathbb{A}^n(\mathbb{K})$ affine Vareität, $I= I(V) \trianglelefteqslant \mathbb{K}[X_1, \ldots, X_n]$ ihr Verschwindungsideal und $I^{*} \trianglelefteqslant \mathbb{K}[X_0, \ldots, X_n]$ das von den Homogenisierungen $H_0(f)$ aller $f \in I$ erzeugte Ideal.\\
Dann ist $V(I^{*})=\overline{V}$ der Zariski-Abschluss von $V$ in $\mathbb{P}^n(\mathbb{K})$.
\begin{pr}
Aus dem Beweis von Bemerkung 8.5 folgt $V(I^{*}) \cap U_0=V$.\\
Sei $\tilde{V}\subseteq \mathbb{P}^n(\mathbb{K})$ abgeschlossen mit $V \subseteq \tilde{V}$. Zeige $V(I^{*}) \subseteq \tilde{V}$. Sei dazu $\tilde{V}=V(J)$ für ein homogenes Ideal $J$. Dann genügt es zu zeigen: $J \subseteq I^{*}$.\\
Sei dazu $f \in J$ homogen. Für $y\in \tilde{V}$ ist dann $D_0(f)(y)=0$, also $D_0(f) \in I$. Per Definition ist dann $H_0(D_0(f)) \in I^{*}$. Es gilt aber 
$H_0(D_0(f))= f \cdot X_0^{-e}$ für ein $e \geqslant 0,$
es folgt also die Behauptung. $\hfill \Box$
\end{pr}
\end{remark}

\begin{definbem} %Definition + Bemerkung 9.6

\begin{compactenum}
\item Eine Teilmenge $W \subseteq \mathbb{P}^n(\mathbb{K})$ heißt \textit{quasiprojektive Varietät}, wenn $W$ offene Teilmenge einer projektiven Varietät $V \subseteq \mathbb{P}^n(\mathbb{K})$ ist.
\item $W \subseteq \mathbb{P}^n(\mathbb{K})$ ist quasiprojektiv genau dann, wenn es eine offene Teilmenge $U \subseteq \mathbb{P}^n(\mathbb{K})$ und eine abgeschlossene Menge $V \subseteq \mathbb{P}^n(\mathbb{K})$ gibt, sodass gilt $W=U \cap V$.
\item Die Zariski-Topologie auf einer quasiprojektiven Varietät hat eine Basis aus (abstrakt) affine Varietäten.
\item Jede quasi-projektive Varietät ist kompakt.
\end{compactenum}

\begin{pr}
\begin{compactenum}
\item[(iii)] Sei $W\subseteq \mathbb{P}^n(\mathbb{K})$ und $U \subseteq W$ offen. Dann ist $U \cap U_i$ offen für alle $i \in \{0, \ldots, n \}$ und der Zariskiabschluss $\overline{U \cap U_i}$ von $U \cap U_i$ in $U_i$ eine affine Varietät.\\
Nach Proposition 2.5 bilden die $D(f)$ für $f \in \mathbb{K}[X_1, \ldots, X_n]$ eine Basis der Zariski-Topologie auf $\overline{U \cap U_i}$, d.h. es existiert $f$ mit $D(f) \subseteq U \cap U_i$. Nach 6.11 Ist $D(f)$ isomorph zu einer affine Varietät, es folgt die Behauptung.
\item[(iv)] Nach Proposition 6.5(iii) ist $W \cap U_i$ kompakt für alle $i \in \{0, \ldots, n \}$. Also ist 
$$W= \bigcup_{i=0}^n W \cap U_i$$
ebenfalls kompakt.
\end{compactenum}
\end{pr}
\end{definbem}


\begin{definbem} %Defiition + Bemerkung 9.7

Sei $V \subseteq \mathbb{P}^n(\mathbb{K})$ projektive Varietät, $V \neq \emptyset$.
\begin{compactenum}
\item Der \textit{affine Kegel} von $V$ ist definiert als
$$\tilde{V} := \{ (x_0 ,\ldots ,x_n) \in \mathbb{K}^{n+1} \ \vert \ (x_0: \ldots : x_n) \in V \} \cup \{(0, \ldots, 0 )\}$$
\item $\tilde{V}$ ist affine Varietät. Genauer gilt: Ist $V=V(I)$ für ein homogenes Ideal $I \trianglelefteqslant \mathbb{K}[X_0, \ldots, X_n]$, so ist $\tilde{V}=V_{\textrm{aff}}(I)$ die Nullstellenmenge vom $I$ in $\mathbb{A}^{n+1}(\mathbb{K})$.
\item Falls $\mathbb{K}$ unendlich ist, gilt $I(V)=I(\tilde{V})$.
\end{compactenum}
\begin{pr}
\begin{compactenum}
\item[(ii)] Nach Definition ist $(x_0, \ldots, x_n) \in \tilde{V} \setminus\{(0, \ldots, 0 )\}$ genau dann, wenn $(x_0: \ldots : x_n) \in V$. Es bleibt also noch zu zeigen: $(0, \ldots, 0) \in V_{\textrm{aff}}(I)$.\\
Ist $f \in I$ homogen, so ist $\deg(f) >0$, also $f(0, \ldots, 0 ) = 0$.
\item[(iii)] Für jedes homogene $f \in \mathbb{K}[X_0, \ldots, X_n]$ gilt:
$$f \in I(V) \ \ \Longleftrightarrow \ \ f \in I(\tilde{V})$$
Zu zeigen ist also: $I(\tilde{V})$ ist homogen. Sei dazu 
$$f= \sum_{i=0}^d f_i \in I(\tilde{V}), \qquad f_i \textrm{ homogen von Grad } i$$
Zu zeigen ist: $f_i \in I(\tilde{V})$ für alle $0 \leqslant i \leqslant d $.\\
Für jedes $x=(x_0, \ldots, x_n) \in \tilde{V} \setminus \{(0, \ldots, 0 ) \}$ und jedes $ \lambda \in \mathbb{K}$ ist $(\lambda x_0, \ldots, \lambda x_n) \in \tilde{V}$, also
$$0= f(\lambda x_0, \ldots, \lambda x_n) = \sum_{i=0}^d \lambda^{i} f_i(x_0, \ldots, x_n)$$
Sind $\lambda_0, \ldots, \lambda_d$ verschiedene Elemente in $\mathbb{K}$, so hat das LGS
$$\begin{pmatrix} 1 & \lambda_0 & \ldots & \lambda_0^d \\ 1 & \lambda_1 & \ldots & \lambda_1^d \\ \vdots & \vdots & \ddots & \vdots \\ 1 & \lambda_d & \ldots & \lambda_d^d \end{pmatrix} \cdot \begin{pmatrix} f_0(x_0, \ldots, x_n) \\ f_1(x_0, \ldots, x_n) \\ \vdots \\ f_d(x_0, \ldots, x_n) \end{pmatrix} = \begin{pmatrix} 0 \\ 0 \\ \vdots \\ 0 \end{pmatrix}$$
nur die triviale Läösung (Vandermonde-Matrix)
$$f_0(x_0, \ldots, x_n) = \ldots = f_d(x_0, \ldots, x_n) = 0,$$
woraus die Behauptung folgt. $\hfill \Box$
\end{compactenum} 
\end{pr}
\end{definbem}

\begin{theorem}[\rm \it Projektiver Nullstellensatz]

Sei $\mathbb{K}$ algebraisch abgeschlossen, $n \geqslant 0$. Dann gilt für jedes homogene Radikalideal $I \trianglelefteqslant \mathbb{K}[X_0, \ldots, X_n], I \neq \langle X_0, \ldots, X_n \rangle:$\\
$$I(V(I))=\sqrt{I}=I$$
Das Ideal $\langle X_0, \ldots, X_n\rangle$ heißt auch \textit{irrelevantes Ideal}.
\begin{pr}
Offenbar stimmt die Aussage für $I= \mathbb{K}[X_0, \ldots, X_n]$. Sei nun also $I$ ein echtes Ideal, also 
$$I \subset \langle X_0, \ldots, X_n \rangle$$
Seien $V_{\textrm{aff}}(I) \subseteq \mathbb{A}^{n+1}(\mathbb{K})$ die affine und $V=V_{\textrm{proj}}(I) \subseteq \mathbb{P}^n(\mathbb{K})$ die projektive Nullstellenmenge von $I$.\\
Dann ist $\tilde{V}:=V_{\textrm{aff}}(I)$ der affine Kegel von $V$.\\
Da $I \neq \langle X_0, \ldots, X_n \rangle$, ist nach HNS $V_{\textrm{aff}}(I) \neq \{0 \}$, also $V \neq \emptyset$. Nach 9.7(iii) gilt dann
$$I(V(I)) = I(V)=I (\tilde{V})= I ( V_{\textrm{aff}}(I)) \overset{HNS}{=} I,$$
was zu zeigen war. $\hfill \Box$
\end{pr}
\end{theorem}

\begin{ex} %%Beispiel

Es sei $E_0:=V(Y^2-X^3+X)$ und $E:=\overline{E_0}$ der projektive Abschluss von $E_0$ in $\mathbb{P}^2(\mathbb{K})$, also
$$E=V(Y^2Z-X^3+XZ^2)$$
Dann gilt 
$$E \setminus E_0 = E \cap V(Z) = \{(0:1:0) \}$$
Es sei nun $\mathbb{L}\subseteq \mathbb{P}^2(\mathbb{K})$ eine Gerade also $L=V(aX+bY+cZ)$, wobei $(a,b,c) \neq (0,0,0)$.\\
Dann kann man zeigen: Unter der Bedingung, dass $\mathbb{K}$ algebraisch abgeschlossen ist, Tangenten doppelt  und Wendetangenten dreifach zählen, gilt
$$\# (L \cap E) = 3$$

Genauer folgt dies aus dem Satz von Bézout. Im folgenden möchten wir eine Gruppenstruktur auf $E$ definieren. Sei hierzu

$$\tilde{\mu}: E \times E \longrightarrow E, \quad (P,Q) \mapsto \textrm{dritter Schnitterpunkt der Gerade durch P und Q}$$

\begin{center}
\begin{tikzpicture}[
x=1cm, y=1cm,  scale=1.4, 
font=\footnotesize,
 punkt/.style={circle,inner sep=1pt,fill=white,draw},
 gerade/.style={thick,blue,samples=2,domain=#1},
>=latex   %Voreinstellung fÃŒr Pfeilspitzen
]


% Gitternetzlinien
\draw[ help lines] (-2.5,-2.5) grid (3.5,2.5);

% x-Achse
\draw[->] (-2,0) -- (3.5,0) node[below] {$x$}; 
%Zahlen auf x-Achse
\foreach \x in {-2,...,3}
\draw[shift={(\x,0)},color=black] (0pt,2pt) -- (0pt,-2pt) node[below] {\footnotesize $\x$};

% y-Achse 
\draw[->] (0,-2.5) -- (0,2.5) node[left] {$y$};%node[above left]
%Zahlen auf y-Achse
\foreach \y in {-2,...,2}
\draw[shift={(0,\y)},color=black] (2pt,0pt) -- (-2pt,0pt) node[left] {\footnotesize $\y$};

%Ursprung
\draw[color=black] (0pt,-5pt) node[right] {\footnotesize $0$};

%%%%%%%%%%%%%%%%
%% FUNKTIONEN %%
%%%%%%%%%%%%%%%%
% Kubisches Polynom

%Rechter Teil
% Wurzel aus Kubischem Polynom - positiver Ast
\draw[thick,color=red] plot[samples=200, domain=1:2.3] 
(\x,{sqrt(\x^3-\x)}) node[above, xshift=2em]{};

% Wurzel aus Kubischem Polynom - negativer Ast
\draw[thick,color=red] plot[samples=200, domain=1:2.3] 
(\x,{-sqrt(\x^3-\x)}) node[below, xshift=2em]{};

%Linker Teil
% Wurzel aus Kubischem Polynom - positiver Ast
\draw[thick,color=red] plot[samples=200, domain=0:-1] 
(\x,{sqrt(\x^3-\x)}) node[] {};

% Wurzel aus Kubischem Polynom - negativer Ast
\draw[thick,color=red] plot[samples=200, domain=0:-1] 
(\x,{-sqrt(\x^3-\x)}) node[] {};

\node[label=below:P, name=p, punkt] at (-0.9,-sqrt{0.171}){};
\node[label={Q}, name=q, punkt] at (-0.1, sqrt{0.099}){};

\draw[gerade=-2:2.5,name=g] plot(\x,{\x*(sqrt{0.171}+sqrt{0.099})/0.8+sqrt{0.099}+0.1*(sqrt{0.171}+sqrt{0.099})/0.8});

\end{tikzpicture}
\end{center}



Zunächst einmal ist diese innere Verknüpfung wohldefiniert und kommutativ. Allerdings finden wir kein neutrales Element: \\
Denn gäbe es $P_0 \in E$ mit $\tilde{\mu}(P,P_0)=P$ für alle $P \in E$, so müssten alle Tangenten an $E$ durch $P_0$ gehen. Das ist offenbar falsch, weshalb $\tilde{\mu}$ nicht der richtige Weg ist.\\
Wir nehmen nun folgende Modifikation vor: Für ein festes $P_0 \in E$ definieren wir eine Abbildung
$$\otimes_{P_0}: E \times E \longrightarrow E, \quad (P,Q) \mapsto P \otimes_{P_0} Q := \tilde{\mu}({P_0}, \tilde{\mu}(P,Q))$$
Dann gilt:
\begin{compactenum}
\item Die Verknüpfung ist wohldefiniert
\item $P_{0}$ ist das neutrale Element der Verknüpfung, d.h. es gilt
$$P \oplus_{P_{0}}P_{0} = P \qquad \textrm{ f\"ur alle } P \in E$$
\item Die Verknüpfung $\oplus_{P_{0}}$ ist assoziativ 
\item und kommutativ

\end{compactenum}
Damit haben wir eine Gruppenstruktur auf unserer Varietät definiert.
Nun stellt sich die Frage nach Elementen endlicher Ordnung? Gibt es sie? Ja!
\begin{compactenum}
\item Die drei Punkte mit senkrechter Tangente haben Ordnung $2$, bilden mit $P_{0}$ also eine Klein'sche Vierergruppe.
\item Die $8$ Punkte mit Wendetangente (nur $2$ sichtbar!) haben Ordnung $3$.
\end{compactenum}
\end{ex}




\renewcommand*\thesection{§ \arabic{section}\quad}
\section{Reguläre Funktionen} %PARAGRAPH 10
\renewcommand*\thesection{\arabic{section}}

\begin{defin} %Definition 10.1
Sei $V \subseteq \mathbb{P}^{n}(\mathbb{K})$ projektive Varietät und $I(V)$ das zugehörige Verschwindungsideal von $V$. Dann heißt
$$\mathbb{K}[V]:= \slant{\mathbb{K}[X_{0}, \ldots, X_{n}]}{I(V)}$$
\textit{homogener Koordinatenring} von $V$. Nach 9.2 (vi) ist $\mathbb{K}[V]$ ein graduierter Ring.
\end{defin}

\begin{remark} %Bemerkung 10.2
Sind $F,G \in \mathbb{K}[X_0, \ldots, X_n]$ homogen von gleichem Grad, so ist $\frac{F}{G}$ eine wohlbestimmte Funktion aus $D(G)$.
\end{remark}

\begin{defin} %Definition 10.3

Sei $W \subseteq \mathbb{P}^n(\mathbb{K})$ projektive Varietät, $f: W \longrightarrow \mathbb{K}$ eine Abbildung.
\begin{compactenum}
\item $f$ heißt \textit{regulär in} $x\in W$, wenn es eine Umgebung $U_x \ni x, U_x \subseteq W$ und homogene Polynome $F,G \in \mathbb{K}[X_0, \ldots, X_n]$ vom selben Grad gibt, sodass $U_x \subseteq D(G)$ und 
$$f(y)= \frac{F}{G}(y) \qquad \textrm{ für alle } y \in U_x$$
\item $f$ heißt \textit{reguläre Funktion auf} $W$, wenn $f$ in jedem $x \in W$ regulär ist.
\end{compactenum}
\end{defin}

\begin{remark} %Bemerkung 10.4
Sei $W \subseteq \mathbb{P}^n(\mathbb{K})$ projektive Varietät, $f: W \longrightarrow \mathbb{K}$ Abbildung: Dann gilt

$$f \textrm{ ist regulär } \Longleftrightarrow f \vert _{U_i \cap W} = f \circ \psi_i \textrm{ ist regulär im Sinne von 6.2 für alle } i \in \{0, \ldots, n \}$$
\begin{pr}
\begin{compactenum}
\item["$\Rightarrow$"] Sei $x \in W \cap U_i$ für ein $i \in \{0, \ldots, n \}$ sowie $f= \frac{F}{G}$ in einer Umgebung $U_x$ von $x$ und homogenen Polynomen gleichen Grades $F,G \in \mathbb{K}[X_0, \ldots, X_n]$. Ohne Einschränkung sei $U_x \subseteq U_i$, ansonsten verkleinere $U_x$. Auf $U_x$ gilt dann
$$(f \circ \psi_i)(x_1, \ldots, x_n) = \frac{F}{G} (x_1: \ldots, x_i : 1 : x_{i+1}: \ldots x_n) = \frac{D_i(F)}{D_i(G)},$$
also ist $f \circ \psi_i$ regulär im Sinne von 6.2.
\item["$\Leftarrow$"] Sei $x \in W \cap U_i$ sowie $f = \frac{g}{h}$ in einer Umgebung $x \ni U_x \subseteq U_i$, $f,g \in \mathbb{K}[X_0, \ldots, X_{i-1}, X_{i+1}, \ldots, X_n]$. Sei $G:= H_i(g), H:=H_i(h)$. Ohne Einschränkung sei $\deg G \leqslant \deg H$. Dann ist $$\frac{\tilde{G}}{H}, \qquad \tilde{G} := G \cdot X_i^{\deg H - \deg G}$$
reguläre Funktion im Sinne von Definition 12.2 auf $U_x$ mit $f= \frac{\tilde{G}}{H}$ auf $U_x$. $\hfill \Box$
\end{compactenum}
\end{pr}
\end{remark}

\begin{definbem}  %Definition+ Bemerkung 10.5

Sei $W \subseteq \mathbb{P}^n(\mathbb{K})$ projektive Varietät.
\begin{compactenum}
\item Für $U \subseteq W$ offen sei 
$$\mathcal{O}_W(U) := \{f: U \longrightarrow \mathbb{K} \ \vert \ f \textrm{ ist regulär } \}$$
\item $\mathcal{O}_W(U)$ ist $\mathbb{K}$-Algebra.
\item Die Zuordnung
$U \mapsto \mathcal{O}_W(U)$
ist eine Garbe von $\mathbb{K}$-Algebren auf $W$.
\end{compactenum}
\end{definbem}

\begin{ex}
Es gilt
$$\mathcal{O}_{\mathbb{P}^n(\mathbb{K})}(U_i) = \mathcal{O}_{U_i}(U_i) = \mathbb{K}[Y_1, \ldots, Y_n]_{X_i}$$
via der Zuordnung

$$f\left( \frac{x_0}{x_i}, \ldots, \frac{x_{i-1}}{x_i}, \frac{x_{i+1}}{x_i}, \ldots, \frac{x_n}{x_i} \right)\quad  \longleftarrow\joinrel\mapstochar \quad f$$
Ist zum Beispiel $i=0, n=3$, so haben wir

$$Y_1 Y_3^2 - 2 Y_2^2 \quad \longmapsto \quad \frac{X_1}{X_0} \left( \frac{X_3}{X_0}\right)^2 - 2 \left( \frac{X_2}{X_0}\right)^2 = \frac{X_1X_3^2 - 2 X_2^2 X_0 }{X_0^3}$$
Bemerke:
$$f\left( \frac{x_0}{x_i}, \ldots, \frac{x_{i-1}}{x_i}, \frac{x_{i+1}}{x_i}, \ldots, \frac{x_n}{x_i} \right) = \frac{H_i(f)}{X_i^d}$$
mit $d = \deg(f)$. Damit erhalten wir
$$\mathcal{O}_{\mathbb{P}^n(\mathbb{K})}(U_i) = \left\{ \frac{H}{X_i^d} \ \bigg \vert  \ H \in \mathbb{K}[X_0, \ldots, X_n] \textrm{ homogen von Grad }d \right\}$$
\end{ex}

\begin{theorem}[ \rm \it Homogene Lokalisierung]

Sei $\mathbb{K}$ algebraisch abgeschlossen, $V \subseteq \mathbb{P}^n(\mathbb{K})$ projektive Varietät.
\begin{compactenum}
\item Für $F \in \mathbb{K}[V]$ homogen von Grad $\deg F \geqslant 1$ gilt
$$\mathcal{O}_V(D(F)) = \left( \mathbb{K}[V]_F\right)_0 := \left\{ \frac{H}{F^d} \ \bigg \vert \ H \in \mathbb{K}[V] \textrm{ homogen von Grad } \deg H=d \cdot \deg F \right\}$$
\item Falls $V$ zusammenhägend ist, gilt
$$\mathcal{O}_V(V)= \mathbb{K}$$
\end{compactenum}
\begin{pr}
\begin{compactenum}
\item Definiere
$$\psi: \mathbb{K}[V]_F \longrightarrow \mathcal{O}_V(D(F)), \quad \frac{G}{F^d} \mapsto \left( x \mapsto \frac{G}{F^d}(x) \right)$$
Dann ist $\psi$ wohldefinierter Homomorphismus von $\mathbb{K}$-Algebren. \\
\textit{injektiv.} Ist 
$$\frac{G}{F^d}(x) = 0$$
für alle $x \in D(F)$, so gilt $D(F) \subseteq V(G)$, also $F \cdot G = 0$ auf $V$. Dann ist aber
$$\frac{G}{F^d}=0 \quad \textrm{ in } \mathbb{K}[V]_F,$$
also $\psi$ injektiv.\\
\textit{surjektiv.} Sei $h \in \mathcal{O}_V(D(F))$.\\
Für $i \in \{0, \ldots, n\}$ sei $f_i:=D_i(F)$ die $i$-te Dehomogenisierung von $F$. Dann ist
$$D(F) \cap U_i = D(f_i)$$
Nach Satz 6.5 gibt es dann $G_i \in \mathbb{K}[Y_1,\ldots, Y_n]$ und $d_i \geqslant 0$, sodass $h(D(F))$ regulär ist, also
$$h(D(F) \cap U_i) = \frac{g_i}{f_i^{d_i}}$$
Mit $G_i := H_i(g_i)$ ist dann
$$h \vert_{D(F) \cap U_i} = \frac{G_i}{F^{d_i} X_i^{e_i}}, \quad e_i \in \mathbb{Z}$$
Auf $D(F) \cap U_i \cap U_j$ ist weiter
$$\frac{G_i}{F^{d_i}X_i^{e_i}} = \frac{G_j}{F^{d_j} X_j^{e_j}}$$
also 
$$G_j F^{d_j} X_j^{e_j} - G_j F^{d_i} X_i^{e_i} =0$$
und schließlich
$$G_j F^{d_j+1} X_j^{e_j}X_i X_j - G_j F^{d_i +1 } X_i^{e_i}X_i X_j =0 \quad \textrm{ auf } V \ \ (*)$$
Sei nun ohne Einschränkung $d_i=1$ für $i \in \{0, \ldots, n \}$, da $V(F^{d_i}) = V(F)$ für alle $d_i \geqslant 1$.\\
Da $\deg(F) \geqslant 1$ ist $F \in \langle X_0, \ldots, X_n \rangle$, also 
$$F^m \in \langle X_0^{e_0+1}, \ldots, X_n ^{e_n+1} \rangle$$
für hinreichend großes $m$ und wegen
$$F^{m+1} = F \cdot F^m \in \langle F X_0 ^{e_0+1}, \ldots, F X_n^{e_n+1} \rangle$$
damit
$$F^{m+1} = \sum_{i=0}^n H_i F X_i^{e_i +1 }, \quad H_i \in \mathbb{K}[X_0, \ldots, X_n] \textrm{ homogen }$$
Beobachtung: Sei $I:= \langle a_1, \ldots, a_m \rangle$ mit homogenen $a_i$.Ist $b$ homogen, so können wir $b$ schreiben als
$$b = \sum_{i=1}^n r_i a_i \quad \textrm{ mit  geeigneten homogenen }r_i \in R$$
(Man kann dies leicht durch Ausmultiplizieren und Koeffizientenvergleich einsehen).\\
Schreibe nun also 
$$G= \sum_{i=0}^n H_i G_i$$
Dann ist
$$X_j F^{m+1} G_j = X_j \sum_{i=0}^n H_i F X_i^{e_i+1} G_j = \sum_{i=0}^n H_i G_i F X_j^{e_j+1} X_i = G F X_j^{e_j+1}$$
also
$$h(D(F) \cap U_j) = \frac{G_j}{F X_j^{e_j}} = \frac{G}{F^{m+1}}$$
Daraus folgt
$$\psi\left( \frac{G}{F^{m+1}}\right) = h,$$
also die Behauptung.

\item Sei $V$ ohne Einschränkung irreduzibel. Denn dann ist $h \in \mathcal{O}_V(V)$ aus jeder irreduziblen Komponente konstant, und da $V$ zusammenhängend ist, stimmen diese Konstanten überein.\\
Damit ist $I(V)$ prim, der homogene Koordinatenring $\mathbb{K}[V]$ also nullteilerfrei.\\
Sei $\mathbb{L}:= \textrm{Quot}(\mathbb{K}[V])$, $f \in \mathcal{O}_V(V)$ und ohne Einschränkung $U_i \cap V \neq \emptyset$ für $i \in \{0, \ldots, n \}$. Sei weiter $f_i := f \vert_{V \cap U_i }$.
Nach (i) ist
$$f_i = \frac{G}{X_i^{d_i}} \quad \textrm{ für ein homogenes } G_i \in \mathbb{K}[V], \deg(G_i)=d_i$$
\begin{compactenum}
\item[\textbf{Beh. (1)}] $f_i$ ist ganz über $\mathbb{K}[V]$.
\end{compactenum}
Dann gibt es $m \geqslant 1$, $a_0, \ldots, a_{m-1} \in \mathbb{K}[V]$ mit
$$f_i^m + \sum_{j=0}^{m-1} a_j f_i^j =0 \qquad (I)$$
und durch Multiplikation mit $X_i^{d_i m }$
$$G_i^m + \sum_{j=0}^{m-1} a_j G_i^j X_i^{d_i(m-j)} \qquad (II)$$
Ohne Einschränkung gelte $a_j \in \mathbb{K}$, denn $(II)$ muss im Grad $d_i m$ erfüllt sein.\\
Dann ist $(I)$ mit $a_j \in \mathbb{K}$ erfüllt, $f_i$ also ganz über $\mathbb{K}$. Da $\mathbb{K}$ algebraisch abgeschlossen ist, ist $f_i$ sogar konstant, es folgt also die Behauptung.
\begin{compactenum}
\item[\textbf{Bew. (1)}] Es gilt 
$$f \vert_{U_i \cap V} = \frac{G}{X_i^{d_i}} \in \mathbb{L}$$
Setze 
$$d:= \sum_{i=0}^n d_i$$
und 
$$\mathbb{K}[V]_d := \left\{H \in \mathbb{K}[V]  \vert \ H \textrm{homogen von Grad } d \right\}$$
\item[\textbf{Beh. (2)}] Es gilt $\mathbb{K}[V]_d f_i^j \subseteq \mathbb{K}[V]_d$ für alle $j \geqslant 0$.
\item[\textbf{Bew. (1)}] Dann ist $X_i^d f_i^j \in \mathbb{K}[V]$, also
$$f_i^j \in \frac{1}{X_i^d} \mathbb{K}[V] \quad \Longrightarrow \quad  \mathbb{K}[V] [f_i ] \subseteq \frac{1}{X_i^d} \mathbb{K}[V]$$
Da $\mathbb{K}[V]$ noethersch und endlich erzeugt ist, ist auch $\mathbb{K}[V][f_i]$ endliche erzeugter $\mathbb{K}[V]$-Modul. Dann existiert $m \geqslant 1 $, sodass $f_i^m$ in dem von $1,f_i, \ldots, f_i^{m-1}$ erzeugten $\mathbb{K}[V]$-Modul liegt. Damit folgt die Behauptung.
\item[\textbf{Bew. (2)}] $\mathbb{K}[V]_d$ wird als $\mathbb{K}$-Vektorraum von den Restklassen der Monome $X_o^{j_0}, \ldots, X_n^{j_n}$ mit
$$\sum_{i=0}^n j_i = d = \sum_{i=0}^n d_i$$
erzeugt. Für jedes solcher Monome gibt es einen Index $i$ mit $j_i \geqslant d_i$, also
$$X_0^{j_0} \dots X_n^{j_n} \cdot f_i = X_0^{j_0} \cdots X_i^{j_i-d_i} \cdots X_n^{j_n} \cdot G_i \in \mathbb{K}[V]_d,$$
was zu zeigen war. $\hfill \Box$
\end{compactenum}

\end{compactenum}
\end{pr}
\end{theorem}





\renewcommand*\thesection{§ \arabic{section}\quad}
\section{Morphismen} %PARAGRAPH 11
\renewcommand*\thesection{\arabic{section}}

\begin{propdefin} %Proposition + Definiton 11.1
Seien $V \subseteq \mathbb{P}^n(\mathbb{K}), W \subseteq \mathbb{P}^m(\mathbb{K})$ quasiprojektive Vareitäten, $f: V \longrightarrow W$ eine Abbildung. Dann sind die folgenden Eigenschaften äquivalent:
\begin{compactenum}
\item Für jedes $x \in V$ gibt es eine offene Umgebung $U_x$ von $x$ und homogene Polynoms $F_0, \ldots, F_m \in \mathbb{K}[X_0, \ldots, X_n]$ von gleichem Grad, sodass für alle $y \in U_x$ gilt:
$$f(y)= \left(F_0(y), \ldots, F_m(y) \right)$$
\item Für alle $i \in \{0, \ldots, n \}$ und $j \in \{0, \ldots, m \}$ mit $U_{ij}:= U_i \cap f^{-1}(W \cap U_j) \neq \emptyset$ ist
$$f\left(U_i \cap f^{-1}(W \cap U_j)\right): U_{ij} \longrightarrow W \cap U_j$$
Morphismus von quasiaffinen Varietäten.
\item $f$ ist stetig und für jedes offene $U \subseteq W$ und jede reguläre Funktion $g \in \mathcal{O}_W(U)$ ist
$$ g \circ f \in \mathcal{O}_V(f^{-1}(U))$$
\end{compactenum}
Ist eine und damit alle jede der Bedingungen erfüllt, so heißt $f$ \textit{Morphismus}.
\begin{pr}
\begin{compactenum}
\item["(ii) $\Leftrightarrow$ (iii) "] Folgt aus 10.4 und 6.9
\item["(i) $\Rightarrow$ (iii)"] Die Stetigkeit von $f$ folgt wie im affinen Fall.\\
Ist $g \in \mathcal{O}_W(U)$ regulär, so gilt lokal $g= \frac{G}{H}$ mit homogenen Polynomen $G,H$ von gleichem Grad. Damit ist
$$g \circ f = \frac{G(F_0(y), \ldots, F_m(y))}{H(F_0(y), \ldots, F_m(y))}$$
regulär auf einer geeigneten offenen Menge.
\item["(ii)$\Rightarrow$(i)"] Sei $j=0$ und $x \in V \cap U_i$ und $f$ in einer offenen Umgebung von $x$ gegeben durch 
$$f(y)=(f_1(y), \ldots, f_m(y))$$
mit 
$$f_k= \frac{g_k}{h_k}, \qquad g_k, h_k \in \mathbb{K}[Y_1, \ldots, Y_n]$$
Durch Homogenisieren erhalten wir
$$f(y)=(1:f_1(y): \ldots: f_m(y))$$
Multiplizieren mit dem Hauptnenner und bei Bedarf mit einer Potenz von $X_0$ ergibt die gewünschten Polynome von gleichem Grad. $\hfill \Box$
\end{compactenum}
\end{pr}
\end{propdefin}

\begin{ex} %Beispiel 11.2

Sei $$f: \mathbb{P}^2(\mathbb{K}) \setminus \{(0:1:0)\} \longrightarrow \mathbb{P}^1(\mathbb{K}), \quad (x:y:z) \mapsto (x:z)$$
Dann ist $f$ Morphismus. Aber: $f$ lässt sich nicht zum Morphismus $\mathbb{P}^2(\mathbb{K}) \longrightarrow \mathbb{P}^1(\mathbb{K})$ fortsetzen. Denn:\\
Betrachte $f(\lambda: \mu : \lambda) = (1:1)$ für ein $\lambda \in \mathbb{K}^{\times}, \mu \in \mathbb{K}$. Es gilt
$$\left\{ ( \lambda: \mu : \lambda) \in \mathbb{P}^2(\mathbb{K}) \ \vert \ \lambda \neq 0 \right\} = V(X-Z) \setminus \{(0:1:0)\}$$
das heißt, $f$ ist konstant auf $V(X-Z) \setminus \{(0:1:0)\}$, also auch auf $\overline{V(X-Z) \setminus \{(0:1:0) \} } = V(X-Z)$, falls $\mathbb{K}$ unendlich ist.\\
Betrachte nun  $f(\lambda: \mu : - \lambda) = (1:-1)$ für ein $\lambda \in \mathbb{K}^{\times}, \mu \in \mathbb{K}$. Analog erhält man hier, dass $f$ konstant auf $V(X+Z)$ ist, also 
$$f(V(X-Z)) = (1:1), \qquad f(V(X+Z)) = (1:-1)$$
Damit kann es eine solche Fortsetzung nicht geben.
\end{ex}

\begin{ex} % Beispiel 11.3

Sei $E:= V(Y^2Z-X^3+XZ^2)$, Siehe Beispiel 9.9, und
$$f: E \setminus \{(0:1:0)\} \longrightarrow \mathbb{P}^1(\mathbb{K}), \quad (x:y:z) \mapsto (x:z)$$
Dann lässt sich $f$ zum Morphismus $E \longrightarrow \mathbb{P}^1(\mathbb{K})$ fortsetzen.\\
Betrachte hierzu die Tangente an $E$ in $P_{\infty}:=(0:1:0)$ Diese ist die Gerade $Z=0$, denn die Tangente ist gerade der lineare Term. Dann gilt $f\vert_{V(Z)} = (1:0)$. Setze nun $P_0:=(0:0:1)$ und
$$g(x:y:z) = \begin{cases} (x:z) & \textrm{ für } (x:y:z) \in E \setminus \{ P_{\infty} \} \\ (y^2+xz:x^2) & \textrm{ für } (x:y:z) \in E \setminus \{ P_0 \} \end{cases}$$
$g$ ist Morphismus. Es bleibt zu zeigen: Für $(x:y:z) \in E \setminus \{ P_0, P_{\infty} \}$ ist 
$(x:z) = ( y^2+xz : x^2).$
Es gilt aber
$$y^2z + xz^2 = x^3 \quad \Longleftrightarrow \quad \frac{y^2+xz}{x^2}=\frac{x}{z}$$
und damit
$$(x:z)=(x(y^2+xz):z(y^2+x^2)) \overset{(x:y:z) \in E}{=} (xy^2+x^2z : x^3) \overset{x \neq 0}{=} (y^2+xz:x^2)$$
Außerdem ist $y^2+xz\neq 0$, da sonst $0=y^2z - x^3+xz^2 = (y^2+xz)z - x^3 = -x^3$, also $x=0$, also $(x:y:z) \in \{P_0, P_{\infty} \}$.
\end{ex}

\begin{prop} %Proposition 11.4

Ist $f: \mathbb{P}^n(\mathbb{K}) \longrightarrow \mathbb{P}^m(\mathbb{K})$ Morphismus, so gibt es homogene Polynome $F_0, \ldots, F_m$ von gleichem Grad, sodass gilt
$$f(x)= \left(F_0(x): \ldots : F_m(x) \right) \qquad \textrm{ für alle } x \in \mathbb{P}^n(\mathbb{K})$$
\begin{pr}
Übung. Hauptgrund: $\mathbb{K}[X_0, \ldots, X_n]$ ist faktoriell.
\end{pr}
\end{prop}

\begin{remark} %Bemerkung 11.5
Für jede quasiprojektive Varietät $V$ ist
$$\textrm{Aut}(V):= \{f: V \longrightarrow V \ \vert \ f \textrm{ ist Isomorphismus } \}$$
eine Gruppe.
\end{remark}

\begin{ex} %Beispiel 11.6

Es gilt
$\textrm{Aut}\left(\mathbb{P}^1(\mathbb{K})\right) \cong \slant{\textrm{GL}_2(\mathbb{K})}{\mathbb{K}^{\times} I_2} \cong \mathbb{P}\textrm{GL}_2(\mathbb{K})$
mit Isomorphismus
$$\phi:  \textrm{PGL}_2(\mathbb{K}) \longrightarrow \textrm{Aut}\left(\mathbb{P}^1(\mathbb{K})\right), \qquad \begin{pmatrix} a & b \\ c & d \end{pmatrix} \mapsto \left( (X_0:X_1) \mapsto \left(aX_0+bX_1 : cX_0+dX_1\right) \right)$$
Analog ist 
$$\textrm{Aut}\left(\mathbb{P}^n(\mathbb{K})\right) \cong \textrm{PGL}_{n+1}(\mathbb{K})$$
\end{ex}

\begin{ex} % Beispiel 11.7

Sei wieder $E:= V (Y^2Z-X^3+XZ^2)$ wie in Beispiel 9.9. Wir haben bereits eine Gruppenstruktur auf $E$ via
$$\oplus:=\oplus_{P_0}: E \times E \longrightarrow E, \qquad (P,Q) \mapsto P \oplus_{P_0} Q$$
Mit den Formeln für $\oplus$, die man sich analytisch herleiten kann, sieht man: $\oplus$ ist Morphismus.\\
Für jedes $P \in E$ ist also 
$$\mu_P: E \longrightarrow E, \qquad Q \mapsto P \oplus Q$$
ein Automorphismus. Damit enthält $\textrm{Aut}(E)$ eine zu $E$ isomorphe Untergruppe. Einen weiteren Automorphismus finden wir zum Beispiel via
$$X \mapsto -X, \quad Y \mapsto i \cdot Y, \quad Z \mapsto Z$$
\end{ex}












% KAPITEL III


\chapter{Lokale Eigenschaften von Varietäten}


\setcounter{section}{11}

\renewcommand*\thesection{§ \arabic{section}\quad}
\section{Lokale Ringe} %PARAGRAPH 12
\renewcommand*\thesection{\arabic{section}}


\begin{definbem} %Definition und Bemerkung 12.1

Sei $\mathbb{K}$ Körper, $V$ eine quasiprojektive Varietät über $\mathbb{K}$, $x \in V$.
\begin{compactenum}
\item Der \textit{lokale Ring von $V$ in $x$} ist definiert als
$$\mathcal{O}_{V,x}:= \left\{ (U,f)_{\sim} \ \big\vert \ U \subseteq V \textrm{ offen }, x \in U, f \in \mathcal{O}_V(U) \right\}$$
wobei
$$(U_1, f_1) \sim (U_2, f_2) \ \ \Longleftrightarrow \ \ \textrm{Es existiert } U \subseteq U_1 \cap U_2 \textrm{ offen mit } f_1 \vert_U = f_2 \vert_U$$
\item Die Elemente von $\mathcal{O}_{V,x}$ heißen \textit{Keime von regulären Funktionen}. Notation: $(U,f)_{\sim} =: f_x$.
\item $\mathcal{O}_{V,x}$ ist $\mathbb{K}$-Algebra und die Abbildung
$$\phi_x: \mathcal{O}_{V,x} \longrightarrow \mathbb{K}, \quad (U,f)_{\sim} \mapsto f(x)$$
ist surjektiver Homomorphismus von $\mathbb{K}$-Algebren.
\item $\mathcal{O}_{V,x}$ ist lokaler Ring mit maximalem Ideal
$$\mathfrak{m}_x = \{ (U,f)_{\sim} \ \vert \ f(x)=0 \} = \ker\phi_x$$
\end{compactenum}
\begin{pr}
\begin{compactenum}
\item[(iii)] Klar.
\item[(iv)] Nach dem Homomorphiesatz und (iii) gilt
$$\slant{\mathcal{O}_{V,x}}{\mathfrak{m}_x} \cong \mathbb{K}$$
also ist $\mathfrak{m}_x$ maximales Ideal. Zeige nun, dass $\mathfrak{m}_x$ das einzige ist. Sei hierfür $f \in \mathcal{O}_V(U)$ für ein $U \subseteq V$ mit $x \in U$ und es gelte $f(x)\neq0$. Zeige: $f_x$ ist Einheit in $\mathcal{O}_{V,x}$.\\
Es gilt $x \in D(f) \subseteq V$ offen, d.h. $(U,f) \sim (D(f), f)$. Damit haben wir 
$$\frac{1}{f} \in \mathcal{O}_V(D(f))$$
also schließlich
$$\left(D(f), \frac{1}{f} \right) \cdot \left( D(f), f \right) = 1_x,$$
was behauptet wurde. $\hfill \Box$
\end{compactenum}
\end{pr}
\end{definbem}

\begin{remark} %Bemerkung 12.2

Für jedes offene $U \subseteq V$ mit $x \in U$ ist 
$$\psi_x^{U}: \mathcal{O}_V(U) \longrightarrow \mathcal{O}_{V,x}, \quad f \mapsto f_x = (U,f)_{\sim} $$
ein Homomorphismus von $\mathbb{K}$-Algebren.\\
Dabei sind die $\psi_x^{U}$ verträglich mit Restriktionsabbildungen und es gilt
$$\mathcal{O}_{V,x} = \varinjlim_{U \subseteq V, x \in U} \mathcal{O}_V(U)$$
\end{remark}

\begin{prop} %Proposition 12.3

Sei $V$ quasiprojektive Varietät über $\mathbb{K}$, $V_0 \subseteq V$ affin, offen und $x \in V_0$. Dann ist 
$$ \mathcal{O}_{V,x} \cong \mathbb{K}[V_0]_{\mathfrak{m}_x^{V_0}},$$
wobei 
$$ \mathfrak{m}_x^{V_0} = \{f \in \mathbb{K}[V_0] \ \vert \ f(x) = 0 \}$$
das zu $x$ zugehörige maximale Ideal des affinen Koordinatenrings $\mathbb{K}[V_0]$ ist.
\begin{pr}
Sei
$$ \alpha: \mathbb{K}[V_0]_{\mathfrak{m}_x^{V_0}} \longrightarrow \mathcal{O}_{V,x}, \quad \frac{f}{g} \mapsto \left(\frac{f}{g} \right)_x$$
wobei $f,g \in \mathbb{K}[V_0]$ und $g \notin \mathfrak{m}_x^{V_0}$, d.h. $g(x) \neq 0$. Dann ist $\alpha$ wohldefinierter Homomorphismus. Zeige, dass dieser die gewünschte Isomorphie der $\mathbb{K}$-Algebren liefert.\\
\textit{injektiv.} Sei $$\frac{f}{g} \in \ker \alpha, \  \textrm{ also } \alpha\left(\frac{f}{g}\right)=0.$$
Dann gibt es eine Umgebung $U$ von $x$, $U \subseteq D(g)$ mit $f(y)=0$ für alle $y \in U$.\\
Sei $W = V_0 \setminus U$. Dann ist $W$ abgeschlossen in $V_0$ und es gilt $x \notin W$.\\
Damit existiert $h \in I(W)$ mit $h(x) \neq0$, also $ h \notin \mathfrak{m}_x^{V_0}$ und $(h \circ f) (y) =0$ für alle $y \in V_0$.
Dann ist $h \circ f =0$ in $ \mathbb{K}[V_0]$, also $$\frac{f}{g} =0 \textrm{ in } \mathbb{K}[V_0]_{\mathfrak{m}_x^{V_0}}$$
\textit{surjektiv.} Sei nun $(U,f)_{\sim} \in \mathcal{O}_{V,x}$, ohne Einschränkung sei $U \subseteq V_0$ und $ U = D(h)$ für ein $h \in \mathbb{K}[V_0]$ mit $h(x) \neq 0$.Dann gilt
$$f \in \mathcal{O}_V(U) = \mathcal{O}_{V_0}(U) = \mathcal{O}_{V_0}(D(h)) = \mathbb{K}[V_0]_h$$
d.h. es ist
$$f = \frac{g}{h^k}, \quad k \geqslant 0, g \in \mathbb{K}[V_0] \ \ \Longrightarrow \ \ \frac{g}{h^k} \in \mathbb{K}[V_0] _{\mathfrak{m}_x^{V_0}}$$
Damit gilt
$$(U,f)_{\sim} = \left(\frac{g}{h^k}\right)_x = \alpha\left(\frac{g}{h^k}\right),$$
wie behauptet. $\hfill \Box$
\end{pr}
\end{prop}

\begin{remark} %Bemerkung 12.4

Sei $\phi:V \longrightarrow W$ Morphismus quasiprojektiver Varietäten. Für jedes $x \in V$ induziert $\phi$ einen Homomorphismus von $\mathbb{K}$-Algebren
$$\phi_x^\# : \mathcal{O}_{W, \phi(x)} \longrightarrow \mathcal{O}_{V,x}$$
Weiter gilt
$$\phi_x^\#\left(\mathfrak{m}_{\phi(x)}\right) \subseteq \mathfrak{m}_x$$
\begin{pr}
Ohne Einschränkung seien $V,W$ affin, denn $x$ und $\phi(x)$ sind in affine Teilmengen enthalten. $\phi$ induziert also 
$$\phi^\#: \mathbb{K}[W] \longrightarrow \mathbb{K}[V] \hookrightarrow \mathbb{K}[V]_{\mathfrak{m}_x^{V}}, \quad f \mapsto f \circ \phi = \phi^\#(f)$$
Dabei ist
$$f \in \mathfrak{m}_{\phi(x)}^W \ \Longleftrightarrow \ f(\phi(x))=0 \ \Longleftrightarrow \ (f \circ \phi)(x)=0 \ \Longleftrightarrow \ f \circ \phi = \phi^{\#} (f) \in \mathfrak{m}_x^V$$
und es gilt also $$\phi^{\#} \left(\mathbb{K}[W] \setminus \mathfrak{m}_{\phi(x)}^{W}\right) \subseteq \left(\mathbb{K}[V]_{\mathfrak{m}_x^{V}} \right)^{\times}.$$
Mit der universellen Eigenschaft der Lokalisierung lässt sich $\phi^{\#}$ also fortsetzen zu 
$$\phi^\#_x: \mathcal{O}_{W, \phi(x)} = \mathbb{K}[W]_{\mathfrak{m}_{\phi(x)}^W} \longrightarrow \mathbb{K}[V]_{\mathfrak{m}_x^V} = \mathcal{O}_{V,x}$$
Weiter gilt
$$\phi_x^\#(\mathfrak{m}_{\phi(x)}) = \phi_x^\# \left( \mathfrak{m}_{\phi(x)}^W \cdot \mathbb{K}[W]_{\mathfrak{m}_{\phi(x)}^W}\right) \subseteq \mathfrak{m}_x^V \cdot \mathbb{K}[V]_{\mathfrak{m}_x^V} = \mathfrak{m}_x,$$
was zu zeigen war. $\hfill \Box$
\end{pr}
\end{remark}

\begin{prop} % Proposition 12.5

Seien $V,W$ quasiprojektive Varietäten $x \in V, y \in W$.Gilt
$$\mathcal{O}_{V,x} \cong \mathcal{O}_{W, y}$$
als $\mathbb{K}$-Algebren, so gibt es offene Umgebungen $U \subseteq V$ von $x$ und $U' \subseteq W$ von $y$ und einen Isomorphismus
$$f: U \longrightarrow U', \quad x \mapsto y$$
\begin{pr}
Ohne Einschränkung seien $V,W$ affin. Sei 
$$\phi: \mathcal{O}_{V,x} = \mathbb{K}[V]_{\mathfrak{m}_x^V} \longrightarrow \mathbb{K}[W]_{\mathfrak{m}_y^W} = \mathcal{O}_{W,y}$$
ein Isomorphismus. Seien $f_1, \ldots, f_r$ die Erzeuger von $\mathbb{K}[V]$ als $\mathbb{K}$-Algebra. Für die Keime $(f_i)_x$ gilt also
$$\phi \left( (f_i)_x\right) = \left(\frac{g_i}{h_i}\right)_y, \quad g_i, h_i \in \mathbb{K}[W], h_i(y) \neq 0$$
Sei $U_2 \subseteq W$ offen, affin mit $y \in U_2$ und es gelte
$$\frac{g_i}{h_i} \in \mathcal{O}_W(U_2) \ \Longleftrightarrow \ \frac{g_i}{h_i} \textrm{ regulär für alle } i \in \{1, \ldots, r \}$$
\begin{compactenum}
\item[\textbf{Beh. (1)}] Falls $x$ auf jeder irreduziblen Komponente von $V$ liegt, ist $\psi_x^V$ injektiv.\end{compactenum}
Dann folgt daraus:
$$\phi \circ \psi_x^V: \mathbb{K}[V] \longrightarrow \mathbb{K}[W]$$
ist injektiv. Damit induziert $\phi \circ \psi_x^V$ einen dominanten Morphismus $g: W \longrightarrow V$. Selbiges Vorgehen mit $\phi^{-1}$ liefert einen dominanten Morphismus $f: V \longrightarrow W$ mit $g \circ f = \textrm{id}_V$und $ f \circ g = \textrm{id}_W$
\begin{compactenum}
\item[\textbf{Bew. (1)}] 
Es gilt:
$$\psi_x^V: \mathbb{K}[V]\longrightarrow \mathbb{K}[V]_{\mathfrak{m}_x^V}$$
ist injekitv genau dann, wenn $\mathbb{K}[V] \setminus \mathfrak{m}_x^V$ keine Nullteiler enthält. Sei also $ h \in \mathbb{K}[V] \setminus \mathfrak{m}_x^V$ Nullteiler in $\mathbb{K}[V]$, d.h. es gibt $g \in \mathbb{K}[V] \setminus \{0\} $ mit $h \cdot g = 0$, also $h(x) \neq 0$.\\
Sei $Z$ eine irreduzible Komponente mit $g \vert_Z \neq 0$, d.h. $V(g) \cap Z \neq Z$. Da $ x \in Z$, gilt auch $V(h) \cap Z \neq Z$. 
Damit ist $( V(h) \cap V(g)) \cap Z \neq Z$, da $Z$ irreduzibel ist und $V(h), V(g)$ echt abgeschlossen sind. Damit folgt $g \cdot h \neq 0$, ein Widerspruch zur Annahme. $\hfill \Box$
\end{compactenum}
\end{pr}
\end{prop}




\renewcommand*\thesection{§ \arabic{section}\quad}
\section{Dimension} %PARAGRAPH 13
\renewcommand*\thesection{\arabic{section}}

\begin{defin} % Definition 13.1

Für einen topologischen Raum $X$ heißt
$$ \dim(X) := \sup \{ n \in \mathbb{N}_0 \ \vert \ \textrm{ es existiert eine Kette } V_0 \subsetneq V_1 \subsetneq \ldots \subsetneq V_n, V_i \textrm{ abgeschlossen und irreduzibel } \}$$
die \textit{Krull-Dimension} von $X$.
\end{defin}


\begin{ex}

\begin{compactenum}
\item Für einen Hausdorffraum $H$ gilt $\dim(H) = 0$.
\item Es gilt $\dim(\mathbb{A}^1(\mathbb{K})) = 1$, falls $\mathbb{K}$ unendlich ist.
\end{compactenum}
\end{ex}

\begin{er} %Erinnerung 13.2

Sei $R$ ein Ring, $\mathfrak{p} \trianglelefteqslant R$ ein Primadeal.
\begin{compactenum}
\item Die \textit{Höhe} von $\mathfrak{p}$ in $R$ ist
$$\textrm{ht}(\mathfrak{p}) := \sup \{ n \in \mathbb{N}_0 \ \vert \ \textrm{ es existiert eine Kette von Primidealen } \mathfrak{p_0} \subsetneq \mathfrak{p}_1 \subsetneq \ldots \subsetneq \mathfrak{p}_n = \mathfrak{p} \}$$
\item Die \textit{Krull-Dimension} von $R$ ist
$$\dim(R) := \sup \{ \textrm{ht}(\mathfrak{p}) \ \vert \ \mathfrak{p} \trianglelefteqslant R \textrm{ prim } \}$$
\end{compactenum}
\end{er}

\begin{prop} %Proposition 13.3

Ist $\mathbb{K}$ algebraisch abgeschlossen, $V \subseteq \mathbb{A}^n(\mathbb{K})$ affine Varietät, so gilt
$$\dim(V) = \dim( \mathbb{K}[V] )$$
\begin{pr}
Irreduzible Teilmengen von $V$ entsprechen gerade bijektiv den Primaidealen in $\mathbb{K}[V]$. $\hfill \Box$
\end{pr}
\end{prop}

\begin{erinnbem} %Erinnerung un d Bemerkung 13.4

Für eine Körpererweiterung $\mathbb{L}/\mathbb{K}$ ist 
$\textrm{trdeg}_{\mathbb{K}} \mathbb{L}$
die Maximalzahl an algebraisch unabhängigen Elementen in $\mathbb{L}$ über $\mathbb{K}$. Beispielsweise ist $\textrm{trdeg}_{\mathbb{K}} \mathbb{K}(X)=1$. Wir halten fest:
\begin{compactenum}
\item Es gilt $\textrm{trdeg}_{\mathbb{K}} \mathbb{K}(X_1, \ldots, X_n) = n$.
\item Es gilt $\textrm{trdeg}_{\mathbb{K}} \mathbb{L}=0$, falls $\mathbb{L}/\mathbb{K}$ algebraisch ist.
\item \textit{Noether-Normalisierung light}: Sei $A$ endlich erzeugte $\mathbb{K}$-Algebra. Dann ist $A$ ganze Ringerweiterung eines Polynomrings $\mathbb{K}[X_1 \ldots, X_n]$.
\item Ist $S/R$ ganze Ringerweiterung, so gilt $\dim R = \dim S$.
\item Es gilt $\dim \mathbb{K}[X_1, \ldots, X_n]=n$.
\item \textit{Noether-Normalisierung deluxe}:
Sei $I \trianglelefteqslant A$ ein Ideal. Dann gibt es einen Polynomring, sodass $A / \mathbb{K}[X_1, \ldots X_n]$ ganze Ringerweiterung ist und 
$$I \cap \mathbb{K}[X_1, \ldots X_n] = \langle X_{\delta + 1}, \ldots, X_{n} \rangle$$
für ein $0 \leqslant \delta \leqslant n$.
\end{compactenum}
\end{erinnbem}

\begin{ex}

Es sei $A:= \mathbb{K}[X,Y]$ und $I$ das vom Polynom $f:=Y^2-X^3+X \in A$ erzeugte Ideal.\\
Es wird $f= Y^2-X^3+X$ als Variable in einem neuen Polynomring betrachtet, setze also $B:= \mathbb{K}[X,f] \subseteq A$. Dann wird $A$ als Ringerweiterung von $B$ offenbar durch das Elemente $Y$ erzeugt. Weiter ist $Y$ ganz über $B$, denn für das normierte Polynom $g:= Z^2-X^3+X-f \in B[Z]$ gilt
$$g(Y)= Y^2-X^3+X-f = f-f = 0$$
und damit ist $A/B$ ganze Ringerweiterung. Weiter gilt $I \cap B = \langle f \rangle$.\\Beachte: $f$ ist nun eine Variable, das heißt, wir haben für $\delta = 1$ ein Beispiel für eine Noether-Normalisierung gefunden.
\end{ex}

\begin{lemma} % Lemma 13.5
Für eine irreduzible Varietät $V$ gilt
$$\dim V = \rm{trdeg}\it_{\mathbb{K}} \mathbb{K}(V)$$
\begin{pr}
Nach 13.3 gilt
$\dim V = \dim \mathbb{K}[V].$
Mit Bemerkung 13.4 (iii) folgt, dass $\mathbb{K}[V]$ als endlich erzeugte $\mathbb{K}$-Algebra eine ganze Ringerweiterung von $\mathbb{K}[X_1, \ldots, X_n]$ für ein $n \in \mathbb{N}$ ist. Mit (iv) gilt
$$\dim \mathbb{K}[V] = \dim \mathbb{K}[X_1, \ldots, X_n] = n.$$
Damit ist $\mathbb{K}(V) / \mathbb{K} = \textrm{Quot}(\mathbb{K}[V]) / \mathbb{K}$
algebraische Erweiterung von $\mathbb{K}(X_1, \ldots, X_n)$ und es folgt
$$\textrm{trdeg}_{\mathbb{K}} \mathbb{K}(V) = \textrm{trdeg}_{\mathbb{K}} \mathbb{K}(X_1, \ldots, X_n) = n,$$
die Behauptung. $\hfill \Box$
\end{pr}
\end{lemma}

\begin{prop} % Proposition 13.6

Sei $V \subseteq \mathbb{P}^n(\mathbb{K})$ quasiprojektive Varietät.
\begin{compactenum}
\item Dann gilt für jede affine Varietät $V_0 \subseteq V$, die in $V$ offen und dicht ist:
$$\dim(V) = \dim(V_0)$$
\item Seien $Z_1, \ldots, Z_r$ die irreduziblen Komponenten von $V$. Dann ist 
$$\dim(V)= \max_{i \in \{1, \ldots, r \}} \dim(Z_i)$$
\end{compactenum}
\begin{pr}
\begin{compactenum}
\item Es gilt:
\begin{compactenum}
\item["$\geqslant$"] Diese Aussage gilt allgemein für einen topologischen Raum und einer Teilmenge $Y \subseteq X$, denn:\\
Ist $\emptyset \subsetneq Y_0 \subset \ldots \subsetneq Y_d$ eine Kette von abgeschlossenen, irreduziblen Teilmengen von $Y$, so gilt für die Abschlüsse $X_i := \overline{Y}_i$: $X_i$ ist irreduzibel in $Y$ und $X_i \cap Y=Y_i$ für alle $i \in \{1, \ldots, d \}$ und damit $X_{i+1} \neq X_i$. Da die $Y_i$ abgeschlossen sind, folgt die Inklusion.
\item["$\leqslant$"]Wegen (ii) dürfen wir $V$ und damit auch $V_0$ irreduzibel voraussetzen. Sei
$$\emptyset \neq Z_0 \subsetneq Z_1 \subset \ldots \subsetneq Z_d$$
eine Kette von abgeschlossenen, irreduziblen Teilmengen von $V$ und $d = \dim V$. Dann ist $Z_0$ offenbar ein Punkt (andernfalls verlängern wir die Kette).\\
Sei nun $V_0 \subseteq V$ eine affine, offene, dichte Untervarietät mit $Z_0 \in V_0$. Dann ist $X_i = Z_i \cap V_0$ nichtleer und abgeschlossen in $V_0$ und damit $\overline{X}_i = Z_i$, da sonst
$$Z_i = \overline{X}_i \cup \left( Z_i \setminus V_0 \right)$$
eine unerlaubte Zerlegung von $Z_i$ wäre. Damit ist $X_i$ irreduzibel mit $X_{i+1} \neq X_i$, es folgt also die Behauptung.
\end{compactenum}
\item Es gilt allgemeiner: Ist $X$ toplogischer Raum mit 
$$X= \bigcup_{i=1}^r\ Z_i, \qquad Z_i \subseteq X \textrm{ abgeschlossen, }$$
so gilt
$$\dim X= \max_{i \in \{1, \ldots, r \}} \dim Z_i,$$
denn:
\begin{compactenum}
\item["$\geqslant$"] Klar.
\item["$\leqslant$"] Sei
$\emptyset \subsetneq X_0 \subsetneq X_1 \subsetneq \ldots \subsetneq X_d$
eine Kette von abgeschlossenen, irreduziblen Teilmengen von $X$. Dann ist
$$X_d = \bigcup_{i=1}^r\ X_d \cap Z_i$$
und da $X_d \cap Z_i$ abgeschlossen in $X_d$ ist und $X_d$ irreduzibel ist, existiert ein $i \in \{1, \ldots r \}$ mit $X_d \subseteq Z_i$. Damit ist bereits die gesamte Kette in $Z_i$ enthalten und es folgt $d \leqslant \dim Z_i$. $\hfill\Box$
\end{compactenum}
\end{compactenum}
\end{pr}
\end{prop}

\begin{prop} %Proposition 13.7

Ist $A$ endlich erzeugbare, nullteilerfreie $\mathbb{K}$-Algebra, so haben alle maximalen Primidealketten in $A$ dieselbe Länge. Dabei heißt eine Kette $\langle 0 \rangle \subsetneq \mathfrak{p}_0 \subsetneq \ldots \subset \mathfrak{p}_d$ maximal, falls es kein Primadeal $\mathfrak{p} \trianglelefteqslant A$ gibt mit $\mathfrak{p}_{i-1} \subsetneq \mathfrak{p} \subsetneq \mathfrak{p}_{i}$ für alle $i \in \{1, \ldots d\}$
\end{prop}

\begin{definprop} % Definition + Proposition 13.8

Sei $V \subseteq \mathbb{P}^n(\mathbb{K})$ quasiprojektive Varietät, $x \in V$.
\begin{compactenum}
\item $\dim_x V := \dim \mathcal{O}_{V,x}$ heißt \textit{lokale Dimension} von $V$ in $x$.
\item Es gilt $$\dim_x V = \textrm{ht} (\mathfrak{m}_x) = \textrm{ht}(\mathfrak{m}_x^{V_0})$$
für jede offene, affine Umgebung $V_0 \subseteq V$ von $x$.
\item Es gilt $\dim_x V = \dim V$, falls $V$ irreduzibel ist.
\item Allgemeiner gilt
$$\dim_x V = \max \{ \dim Z \ \vert \ Z \subseteq V \textrm{ ist irreduzible Komponente von $V$ mit } x \in Z \}$$
\end{compactenum}
\begin{pr}
\begin{compactenum}
\item[(ii)] Es gilt $\mathcal{O}_{V,x} = \mathbb{K}[V_0]_{\mathfrak{m}_x^{V_0}}$ und damit
$\dim \mathcal{O}_{V,x} = \textrm{ht}( \mathfrak{m}_x^{V_0}).$
\item[(iii)] Ohne Einschränkung sei $V$ affin (vgl. 13.4). Dann gilt nach (ii)
$$\dim_x V = \textrm{ht}(\mathfrak{m}_x^V)$$
Wegen 13.7 haben alle maximalen Ideale in $\mathbb{K}[V]$ dieselbe Höhe. Damit folgt bereits
$$\dim V = \dim \mathbb{K}[V] = \textrm{ht} ( \mathfrak{m}_x^V) = \dim_x V. $$

\item[(iv)] Ohne Einschränkung sei $V$ wieder affin. Es gilt
$$\dim_xV = \dim \mathcal{O}_{V,x} = \textrm{ht} \left( \mathfrak{m}_x^V \right) = \sup \{ k \in \mathbb{N} \ \vert \ \textrm{es gibt eine Primidealkette } \langle 0 \rangle \neq \mathfrak{p_0} \subsetneq \ldots, \subsetneq \mathfrak{p}_k = \mathfrak{m}_x^V \}$$
Damit entspricht $\mathfrak{p}_0$ einer irreduziblen Komponente $Z$ mit $x \in Z$ Mit Proposition 13.7 hat diese Kette die Länge $\dim Z$ und damit folgt die Behauptung. $\hfill \Box$

\end{compactenum}
\end{pr}
\end{definprop}


\begin{cor} %Korollar 13.9

Ist $\mathbb{K}$ algebraisch abgeschlossen, so gilt für jede irreduzible Varietät $V \subseteq \mathbb{A}^n(\mathbb{K})$:
$$\dim V + \rm{ht}\it(I(V)) = n.$$
\begin{pr}
Sei
$0 \subsetneq \mathfrak{p}_1 \subset \ldots \subsetneq \mathfrak{p}_d$
eine maximale Primidealkette in $\mathbb{K}[X_1, \ldots X_n]$, die $I(V)$ enthält. Dann gilt $I(V) = \mathfrak{p}_i$ für ein $i \in \{1, \ldots d \}$. Es folgt $i = \textrm{ht}(I(V))$ und wegen 13.9 auch $d=n$. Außerdem ist
$$0 = \slant{\mathfrak{p}_i}{I(V)} \subset \ldots \subset \slant{\mathfrak{p}_n}{I(V)}$$
eine maximale Primidealkette für
$\slant{\mathbb{K}[X_1, \ldots, X_n]}{I(V)} = \mathbb{K}[V],$
und erneut mit 13.9 folgt
$$n-i = \dim \mathbb{K}[V] = \dim V,$$
was zu zeigen war. $\hfill \Box$
\end{pr}
\end{cor}


\begin{cor} %Korollar 13.10

Sei $\mathbb{K}$ algebraisch abgeschlossen, $V \subseteq \mathbb{A}^n(\mathbb{K})$ eine Hyperfläche, d.h. $V=V(f)$ für ein $f \in \mathbb{K}[X_1, \ldots X_n]$ mit $\deg f \geqslant 1 $. Dann ist 
$$\dim V = n-1.$$
\begin{pr}
Aus 13.9 folgt
$$\dim V = n - \textrm{ht}(\langle f \rangle ).$$
Zeige also: $\textrm{ht}(\langle f \rangle ) =1$. 
\begin{compactenum}
\item["$\geqslant$"] Klar.
\item["$\leqslant$"] Sei $\mathfrak{p} \trianglelefteqslant \mathbb{K}[X_1, \ldots X_n]$ ein Primideal mit $\langle 0 \rangle \subsetneq \mathfrak{p} \subseteq \langle f \rangle$. Sei $h \in \mathfrak{p} \setminus \{0\}$ mit minimalem Grad. Da $\mathfrak{p} \subseteq \langle f \rangle$, gilt $h = f \cdot g $ für ein $g \in \mathbb{K}[X_1, \ldots X_n]$. Wir erhalten
$$\deg h = \deg f + \deg g > \deg g$$
und damit ist $g \notin \mathfrak{p}$. Da $\mathfrak{p}$ prim ist, folgt $f \in \mathfrak{p}$ und damit $\mathfrak{p}= \langle f \rangle$. $\hfill \Box$
\end{compactenum}
\end{pr}
\end{cor}

\begin{theorem}[\rm \it "Going down", Cohen-Seidenberg]

Sei $A$ endlich erzeugte, nullteilerfreie $\mathbb{K}$-Algebra, $A/B$ mit $B:= \mathbb{K}[X_1, \ldots X_n]$ via Noether-Normalisierung eine ganze Ringerweiterung. Sei weiter $\mathfrak{P}_1 \subset A$ ein Primadeal, $\mathfrak{p}_0 \subset B$ mit $\mathfrak{p}_0 \subset \mathfrak{p}_1:= \mathfrak{P}_1 \cap B$. Dann gibt es ein Primadeal $\mathfrak{P}_0 \subset A$ mit $\mathfrak{P}_0 \subset \mathfrak{P}_1$ und $\mathfrak{P}_0 \cap B = \mathfrak{p}_0$.
\begin{pr}
Nach dem "Going up"-Theorem in der Algebra (Prop. 13.7) gibt es ein Primadeal $\mathfrak{P}_0' \subset A$ mit $\mathfrak{P}_0' \cap B = \mathfrak{p}_0$ und ein Primadeal $\mathfrak{P}_1' \subset A$ mit $\mathfrak{P}_0' \subset \mathfrak{P}_1'$ und $\mathfrak{P}_1' \cap B = \mathfrak{p}_1$. Setze
$$\mathbb{M} := \textrm{Quot}(B), \qquad \mathbb{L} := \textrm{Quot}(A).$$
Dann ist $\mathbb{L}/\mathbb{M}$ eine endliche, algebraische Körpererweiterung.
\begin{compactenum}
\item[\textbf{Fall (a)}] Es ist $\mathbb{L}/\mathbb{M}$ Galoiserweiterung. Dann ist 
$$\textrm{Gal}(\mathbb{L}/\mathbb{M}) = \{ \sigma_1 = \textrm{id}, \sigma_2, \ldots \sigma_n \}, \quad n := [\mathbb{L}: \mathbb{M}].$$
Sei nun $\mathfrak{P}_i := \sigma_i( \mathfrak{P}_1)$ für $i \in \{1, \ldots n\}$. Dann ist $\mathfrak{P}_i$ ein Primadeal in $A$ für ein $i \in \{1, \ldots n \}$ (nichttrivial! Warum gilt $\sigma_i(A) \subseteq A$?).\\
Angenommen, $\mathfrak{P}_i' \neq \mathfrak{P}_i$ für alle $i \in \{1, \ldots n \}$. Dann ist auch $\mathfrak{P}_1' \nsubseteq \mathfrak{P}_i$, da
$$\mathfrak{P}_i' \cap B = \mathfrak{P}_1 \cap B = \mathfrak{p} = \mathfrak{P}_i \cap B.$$
Dann folgt
$$\mathfrak{P}_i' \nsubseteq \bigcup_{i=1}^n \mathfrak{P}_i$$
(diese Aussage gilt nicht nur für Primideale). Also existiert $a \in \mathfrak{P}_1'$ mit $a \notin \mathfrak{P}_i$ für alle $i \in \{1, \ldots n \}$ und es gilt $\sigma_j(a) \in \mathfrak{P}_i$ für alle $i,j \in \{1, \ldots n \}$. Schließlich ist
$$\mathbb{M} \ni N_{\mathbb{L}/\mathbb{M}} \overset{(*)}{=} \prod_{j=1}^n \sigma_j(a) \in \mathfrak{P}_i \qquad \textrm{ für alle } i \in \{1, \ldots n \},$$
andererseits aber $$N_{\mathbb{L}/\mathbb{M}} \in \mathbb{M} \cap \mathfrak{P}_1' = B \cap \mathfrak{P}_1' = \mathfrak{p}_1$$
und $\mathfrak{p}_i \subseteq \mathfrak{P}_i$ für alle $i \in \{1, \ldots, n \}$, ein Widerspruch!\\
Damit war die Annahme falsch und es gibt einen Index $i \in \{1, \ldots n \}$, sodass
$$\mathfrak{P}_i'= \sigma_i(\mathfrak{P}_i).$$
Das Ideal 
$\mathfrak{P}_0 = \sigma_i^{-1} ( \mathfrak{P}_0')$ erfüllt damit 
$$\mathfrak{P}_0 \subset \mathfrak{P}_1 \qquad \textrm{ und } \mathfrak{P}_0 \cap B = \mathfrak{P}_0' \cap B = \mathfrak{p}_0.$$

\item[\textbf{Fall (b)}] $\mathbb{L}/\mathbb{M}$ ist nicht Galois. Ist $\mathbb{L}/\mathbb{M}$ nicht separabel, so ändert dies nichts an dem Beweis, bis auf die Tatsache, dass der Ausdruck in (*) nicht der Norm entspricht, sondern nur eine gewissen Wurzel von ihr.\\
Ist andererseits $\mathbb{L}/\mathbb{M}$ nicht normal, so betrachten wir die die normale Hülle $\tilde{\mathbb{M}} \supset \mathbb{M}$. Hier wird der Beweis ein wenig technischer, im Wesentlichen ändert sich jedoch trotzdem nicht viel. $\hfill \Box$

\end{compactenum}
\end{pr}
\end{theorem}


\begin{pr}[\it (Beweis von 13.9)]

Es sei
$$\langle 0 \rangle = \mathfrak{P}_1 \subsetneq \mathfrak{P}_1 \subsetneq \ldots \subsetneq \mathfrak{P}_m$$
eine maximale Kette von Primidealen in $A$. Sei weiter $A/B$ mit $B:= \mathbb{K}[X_1, \ldots X_d]$ eine via Noether-Normalisierung erhaltene ganze Ringerweiterung . Setze
$$\mathfrak{p}_i := \mathfrak{P}_i \cap B \qquad \textrm{ für } i \in \{1, \ldots m \}.$$
\begin{compactenum}
\item[\textbf{Beh. (a)}] Wir haben eine maximale Kette von Primideale in $B$:
$$\langle 0 \rangle \subsetneq \mathfrak{p}_1 \subsetneq  \ldots \subsetneq \mathfrak{p}_m$$
\end{compactenum}
Da $\dim A = \dim B$, genügt es nun zu zeigen: $m=d$. Zeige dies über Induktion nach $d$:
\begin{compactenum}
\item[\textbf{d=1}] Klar.
\item[\textbf{d $\geqslant$ 1}] Sei $C/B$ mit $C:= \mathbb{K}[Y_1, \ldots Y_d]$ eine via Noether-Normalisierung erhaltene ganze Ringerweiterung, sodass gilt $\mathfrak{p}_1 \cap C = \langle Y_{\delta +1}, \ldots, Y_d \rangle$ für ein $\1 \leqslant \delta \leqslant d$. Für
$$\mathfrak{q}_i := \mathfrak{p}_i \cap C, \qquad i \in \{1, \ldots m \}$$
ist wegen der Behauptung
$$\langle 0 \rangle \subsetneq \mathfrak{q}_1 \subsetneq \ldots \subsetneq \mathfrak{q}_m$$
eine maximale Kette in $C$. damit folgt $\textrm{ht}(\mathfrak{q}_1)=1$, also $\delta=d-1$.\\
Sei nun $C':= \slant{C}{\mathfrak{q}_1} \cong \mathbb{K}[Y_1, \ldots, Y_{d-1}]$. Dann ist
$$\langle 0 \rangle = \slant{\mathfrak{q}_1}{\mathfrak{q}_1} \subsetneq \ldots \subsetneq \slant{\mathfrak{q}_m}{\mathfrak{q}_1}$$
eine maximale Kette in $C'$, d.h. es gilt $m-1=d-1$, also $m=d$.
\end{compactenum}
Es bleibt nun also, die Behauptung (a) zu zeigen.
\begin{compactenum}
\item[\textbf{Bew. (a)}] Nach Definition ist $\mathfrak{p}_i \subseteq \mathfrak{p}_{i+1}$. Es ist also zu zeigen: $\mathfrak{p} \neq \mathfrak{p}_{i+1}$. Sei dazu ohne Einschränkung $i=0$ - andernfalls ersetze $A$ durch $\slant{A}{\mathfrak{P}_i}$ und $B$ durch $\slant{B}{\mathfrak{p}_i}$.\\
Sei $b \in \mathfrak{P}_1 \setminus \{0\} = \mathfrak{P}_1 \setminus \mathfrak{P}_0$. Da $b$ ganz ist über $B$, gibt es eine Gleichung
$$b^n + a_{n-1}N^{n-1} + \ldots + a_1b + a_0 = 0, \quad a_i \in B \textrm{ für alle } i \in \{1, \ldots n-1\}.$$
Wir wählen $n$ minimal, sodass gilt $a_0=0$.Dann ist 
$$a_0 = -b \cdot \left(b^{n-1} + a_{n-1}b^{n-1} + \ldots + a_1 \right) \in B \cap \mathfrak{P}_1= \mathfrak{p}_1,$$
also $\mathfrak{p}_1 \neq \langle 0 \rangle 0$.\\
Schließlich muss noch gezeigt werden, dass die Kette tatsächlich maximal ist, d.h. es gibt für kein $i \in \{1, \ldots m  \}$ ein Primideal $\mathfrak{q}$ mit $\mathfrak{p}_{i-1} \subsetneq \mathfrak{q} \subsetneq \mathfrak{p}_i$. Proposition 13.11 liefert und jedoch genau dies. Damit ist die Behauptung gezeigt.
\end{compactenum}
\end{pr}







\renewcommand*\thesection{§ \arabic{section}\quad}
\section{Tangentialraum und Singularitäten} %PARAGRAPH 14
\renewcommand*\thesection{\arabic{section}}


\begin{er} %Erinnerung 14.1
Sei $f \in \mathbb{K}[X_1, \ldots, X_n]$, $a=(a_1, \ldots, a_n) \in \mathbb{K}^n$.
\begin{compactenum}
\item Es gilt
$$f= \sum_{(\nu_1, \ldots, \nu_n) \in \mathbb{N}_0^n} \frac{1}{(\nu_1+ \ldots \nu_n)!} \left( \left(\frac{ \partial}{\partial X_1} \right)^{\nu_1} \cdots \left( \frac{\partial}{\partial X_n} \right)^{\nu_n} f \right) (a) \prod_{i=1}^n \left(X_i-a_i\right)^{\nu_i} $$
\item Es ist
$$f = f(a) + \sum_{i=1}^n \frac{\partial f}{\partial X_i}(a) (X_i-a_i) + \textrm{ höhere Terme }$$
\end{compactenum}
\end{er}

\begin{definbem} %Definition + Bemerkung 14.2

Sei $f \in \mathbb{K}[X_1, \ldots, X_n], a= (a_1, \ldots, a_n) \in \mathbb{K}^n$.
\begin{compactenum}
\item Die \textit{Linearisierung von } $f$ in $a$ ist
$$f_a^{(1)} = \sum_{i=1}^n \frac{\partial f}{\partial X_i} (a) X_i =: D_a(f)$$
\item Sei $V \subseteq \mathbb{A}^n(\mathbb{K})$ affine Varietät, $a \in V$, $I=I(V) \trianglelefteqslant \mathbb{K}[X_1, \ldots, X_n]$. Sei weiter $I_a$ das von den Linearisierungen $f_a^{(1)}$ für alle $f \in I$ erzeugte Ideal in $\mathbb{K}[X_1, \ldots, X_n]$. Dann heißt
$$T_a = T_{V,a} := V(I_a)$$
\textit{Tangentialraum} an $V$ in $a$.
\item Ist $I(V)=\langle f_1, \ldots, f_r \rangle$, so ist $I_a = \langle (f_1)_a^{(1)}, \ldots, (f_r)_a^{(1)} \rangle$.
\item $T_{V,a}$ ist ein Untervektorraum des $\mathbb{K}^n$. Genauer ist 
$$T_{V,a} = \ker \mathcal{J}_{f_1, \ldots, f_r} (a), \quad \mathcal{J}:= \mathcal{J}_{f_1, \ldots, f_r} = \left( \frac{\partial f}{\partial X_i} \right)_{i,j}$$
\end{compactenum}
\begin{pr}
\begin{compactenum}
\item[(iii)] Es gilt\\[-48pt]

\begin{alignat*}{5}  D_a(f+g)\ & = &&\quad D_a(f) + D_a(g) \\
D_a(fg)\ & = &&\quad (f \cdot g)_a^{(1)} \ =\ \sum_{i=1}^n \frac{\partial}{\partial X_i} (fg) (a) X_i 
\ \ = \ \ \sum_{i=1}^n \left( f(a) \frac{\partial g}{\partial X_i} (a) + g(a) \frac{\partial f}{\partial X_i}(a) \right) X_i\\
&=&& \quad f(a) \sum_{i=1}^n \frac{\partial g}{\partial X_i} (a) X_i + g(a) \sum_{i=1}^n \frac{\partial f}{\partial X_i} (a) X_i\\
&=&& \quad f(a) D_a(g) + g(a) D_a(f)
\end{alignat*}

Ist nun also 
$$f = \sum_{k=1}^r g_k f_k \in I(V), \qquad g_k \in \mathbb{K}[X_1, \ldots, X_n],$$
so ist
$$D_a(f) \ = \ \sum_{k=1}^r \left(f_k(a) D_a(g_k) + g_k(a) D_a(f_k) \right) \ = \ \sum_{k=1}^r g_k(a) (f_k)_a^{(1)} \ \in \ \langle (f_a)_a^{(1)}, \ldots, (f_r)_a^{(1)} \rangle $$

\item[(iv)] Folgt aus (iii).
\end{compactenum}
\end{pr}
\end{definbem}

\begin{ex}

\begin{compactenum}
\item Sei $f=Y^2-X^3-X^2 \in \mathbb{K}[X,Y]$, $V=V(f)$. Ist $(a,b) \in V$, so gilt
$$f_{(a,b)}^{(1)} = -a (3a+2)X+2bY$$
Trivial wird dieses Gleichungssystem für $(a,b)=0$ und $(a,b)=\left(-\frac{2}{3}, 0 \right)$. Da aber der zweite Punkt nicht auf $V$ liegt, erhalten wir als Tangentialraum eine Gerade außerhalb von $(0,0)$ und $T_{V,(0,0)} = \mathbb{K}^2$. 
\item Sei $f= Y^2-X^3 \in \mathbb{K}[X,Y]$, $V=V(f)$. Dann ist 
$$f_{(a,b)}^{(1)} = -3a^2 X + 2bY$$
und mit selbiger Argumentation ist $T_{V,(0,0)}= \mathbb{K}^2$ und außerhalb von $(0,0)$ eine Gerade.
\item Sei $f=X^2+Y^2-Z^2 \in \mathbb{K}[X,Y,Z]$, $V=V(f)$. Es ist 
$$f_{(a,b)}^{(1)} = 2aX + abY - 2cZ,$$
also ist $T_{V,(0,0,0)}= \mathbb{K}^3$ und eine Ebene außerhalb von $(0,0)$. 
\end{compactenum}
\end{ex}


\begin{remark}   %Bemerkung 14.3
 Seien $V_0 \subseteq V \subseteq \mathbb{A}^n(\mathbb{K})$ affine Varietäten, $V_0$ dicht in $V$, $a \in V_0$. Dann ist 

$$T_{V_0,a} \cong T_{V,a}.$$
\begin{pr}
Ohne Einschränkung sei $V_0= D(g)$ für ein $g \in \mathbb{K}[V]$. Sei $I(V)= \langle f_1, \ldots, f_r \rangle$. Dann ist $V_0 \cong V_0':= V(f_1, \ldots, f_r, g X_{n+1}-1) \subseteq \mathbb{A}^{n+1}(\mathbb{K})$. Dabei entspricht der Punkt $a=(a_1,\ldots, a_n) \in V_0$ dem Punkt $a'=\left(a_1, \ldots, a_n, \frac{1}{g(a)}\right)$. Weiter ist 
$$T_{V_0', a'} = V\left((f_1)_{a'}^{(1)}, \ldots, (f_r)_{a'}^{(1)}, \frac{1}{g(a)} g_{a'}^{(1)} + g(a) X_{n+1} \right) \subseteq \mathbb{K}^{n+1}.$$
Da der Term $\frac{1}{g(a)} g_{a'}^{(1)} + g(a) X_{n+1}$ als einziger $X_{n+1}$ enthält, gilt
$$
\begin{array}{rcl}
 \dim T_{V',a'} & = &  n+1 - \textrm{Rang} \left( (f_1)_{a'}^{(1)}, \ldots, (f_r)_{a'}^{(1)}, \frac{1}{g(a)} g_{a'}^{(1)} + g(a) X_{n+1} \right) \\
 & = & n- \textrm{Rang} \left( (f_1)_{a'}^{(1)}, \ldots, (f_r)_{a'}^{(1)} \right)\\
&=&  \dim T_{V,a},
\end{array}
$$
was zu zeigen war. $\hfill \Box$
\end{pr}
\end{remark}


\begin{defin}   %Definition 14.4

Sei $V \subseteq \mathbb{P}^n(\mathbb{K})$ quasiprojektive Varietät, $a \in V$. Dann ist der \textit{Tangentialraum in $a$ an} $V$ definiert als
$$T_{V,a} := T_{V_0, a},$$
wobei $V_0 \subseteq V$ ein offene, affine Umgebung von $a$ ist.
\end{defin}

\begin{defin}   %definition 14.5

Sei $V \subseteq \mathbb{P}^n(\mathbb{K})$ quasiprojektive Varietät.
\begin{compactenum}
\item $a \in V$ heißt \textit{nichtsingulärer} oder \textit{regulärer} Punkt, falls $\dim T_{V,a}= \dim_a V$. Andernfalls heißt $a$ singulär.
\item $V$ heißt \textit{nichtsingulär}, wenn jedes $a \in V$ nichtsingulär ist.	
\end{compactenum}
\end{defin}


\begin{prop}[Jacobi-Kriterium]

Sei $V \subseteq \mathbb{A}^n(\mathbb{K})$ affine Variteät, $a \in V$, $I=I(V)= \langle f_1, \ldots, f_r \rangle $. Dann gilt

$$a \textrm{ ist nichtsingulär } \quad \Longleftrightarrow \quad \textrm{Rang}\left(\mathcal{J}_{f_1, \ldots, f_r}(a) \right) = n - \dim _a V.$$
\begin{pr}
Nach Bemerkung 14.2 ist 
$$T_{V,a} = \ker \left( \frac{\partial f_i}{\partial X_i} (a) \right)_{i,j}$$
Mit $$\textrm{Rang}(\mathcal{J}_{f_1, \ldots, f_r}(a))= n - \dim \ker \mathcal{J}(a) = n - \dim T_{V,a}$$
folgt die Behauptung. $\hfill \Box$
\end{pr}
\end{prop}

\begin{ex}   %Beispiel

\begin{compactenum}
\item Sei $V=V(f) \subseteq \mathbb{A}^n(\mathbb{K})$ Hyperfläche. Dann ist 
$$\mathcal{J}_f(a)= \left( \frac{\partial f}{\partial X_1}(a), \ldots, \frac{\partial f}{\partial X_n}(a) \right)$$
also 
$$ a \textrm{ ist singulär } \quad \Longleftrightarrow \frac{\partial f }{\partial X_1}(a)= \ldots = \frac{\partial f}{\partial X_n}(a) = f(a)=0.$$
\item Sei $f= Y^2-X^3+X \in \mathbb{K}[X,Y]$, $V=V(f) \subseteq \mathbb{A}^2(\mathbb{K})$. Dann ist
    $$\mathcal{J}_f(x,y) = \left(-3x^2+1, 2y\right).$$
    Dann gilt: 
    $$a=(x_0, y_0) \textrm{ ist singulär } \quad \Longleftrightarrow \quad y_0=0,\ \  3x_0^2=1 \quad \Longleftrightarrow \quad a=\left(\frac{1}{\sqrt{3}}, 0 \right).$$
    Aber es gilt: $f(a) \neq 0 \ \Longleftrightarrow \ a \notin V$. Damit ist $V(f)$ nichtsingulär.\\
    Wir betrachten nun den projektiven Abschluss $\overline{V}=V(Y^2Z-X^3+XZ^2)\subseteq \mathbb{P}^2(\mathbb{K})$. Der einzige neu auftretende Punkt ist $P_{\infty}=(0:1:0)$. Wir betrachten eine affine Umgebung
    $$U:= U_Y \cap \overline{V} = V(Z-X^3+XZ^2).$$
    Dann ist für $G=Z-X^3+XZ^2$:
    $$\mathcal{J}_g(x,z) = (-3x^2+z^2, 2xz +1)\quad \Longrightarrow \quad  \mathcal{J}_g(P_{\infty}) = (0,1)$$
    womit $P_{\infty}$ ein regulärer Punkt ist. Also ist sogar $\overline{V}$ nichtsingulär.
\item Wir variieren nun die Varietät aus Beispiel (ii). Setze hierfür
$$f_{a,b}:= Y^2-X^3-aX-b.$$
 Dann ist
 $$\mathcal{J}_{f_{a,b}}(x,y) = (-3x^2-a, 2y)$$
 Sei nun $x_0, y_0) \in E_{a,b}=V(f_{a,b})$ singulär. Dann ist $y_0=0$ und $-a= 3x_0^2$. Weiter muss der Punkt auf $E_{a,b}$ liegen, wir erhalten also die Bedingung
 $$x_0^3-3x_0^3+b = 0 \quad \Longleftrightarrow \quad b = 2x_0^3 \quad \Longleftrightarrow \quad b^2= 4x_0^6 = 4 \frac{-a^3}{27} \Longleftrightarrow \quad 27b^2+4a^3=0.$$
 Andererseits gilt
 $$f_{a,b9=0 \quad \Longleftrightarrow \quad Y^2 = X^3+ aX + b =:g_{a,b}(X}$$
 und damit 
 $$\Delta(a,b) = 0 \quad \Longleftrightarrow \quad g_{a,b} \textrm{ hat eine doppelte Nullstelle.}$$
 Wobei mit $\Delta(a,b)$ die Diskriminante von $a$ und $b$ bezeichnet wird. Damit erhalten wir 
 $$\overline{E}_{a,b} \textrm{ ist nichtsingulär} \quad \Longleftrightarrow \quad \Delta(a,b) \neq 0,$$
 was zu zeigen war. $\hfill \Box$
\end{compactenum}
\end{ex}


\begin{theorem} %Satz 14.8

Sei $\mathbb{K}$ algebraisch abgeschlossen, $V \subseteq \mathbb{A}^n(\mathbb{K})$ affine Varietät. Sei $\mathfrak{m}_x=\{ f_x \in \mathcal{O}_{V,x} \ \vert \ f_x(x)=0 \}$ das zum Punkt $x \in V$ zugehörige maximale Ideal. Bezeichne weiterhin 
$\left( \slant{\mathfrak{m}_x}{\mathfrak{m}_x^2}\right)^{*}$
den Dualraum des $\mathbb{K}$-Vektorraums $\slant{\mathfrak{m}_x}{\mathfrak{m}_x^2}$. Dann gibt es einen natürlichen Isomorphismus von $\mathbb{K}$-Vektorräumen 
$$\alpha: T_{V,x} \longrightarrow \left( \slant{\mathfrak{m}_x}{\mathfrak{m}_x^2} \right)^{*}$$
\begin{pr}
Zur Wohldefiniertheit der Behauptung: Es ist $\slant{\mathfrak{m}_x}{\mathfrak{m}_x^2}$ ein Modul über $\mathcal{O}_{V,x}$, das heißt, Multiplikation mit Ringelementen aus $\mathcal{O}_{V,x}$ ist definiert. Multiplikation mit einem Elemente aus $\mathfrak{m}_x$ ist die Nullabbildung. Damit ist $\slant{\mathfrak{m}_x}{\mathfrak{m}_x^2}$ ein $\slant{\mathcal{O}_{V,x}}{\mathfrak{m}_x}$-Modul, also ein $\mathbb{K}$-Vektorraum und der Dualraum dazu ist wohldefiniert. Dieser wird auch als \textit{Zariski-Tangentialraum} bezeichnet. \\
Nun zur Behauptung. Definiere 
$$\alpha: T_{V,x} \longrightarrow \left( \slant{\mathfrak{m}_x}{\mathfrak{m}_x^2}\right)^{*}, \quad v=(v_1, \ldots, v_n) \mapsto \alpha(v)(\overline{f}):= \sum_{i=1}^n \frac{\partial f}{\partial X_i} (x) v_i$$
Dann ist $\alpha$ wohldefiniert, denn für $g,h \in \mathfrak{m}_x$ gilt
$$\alpha(v)(gh) = \sum_{i=1}^n \frac{\partial (gh)}{\partial X_i} (x) v_i = \sum_{i=1}^n \left( g(x) \frac{\partial h}{\partial X_i} (x) + h(x) \frac{\partial g}{\partial X_i}(x) \right) v_i =0$$
Damit ist dann auch für alle $f \in \mathfrak{m}_x^2$ bereits $\alpha(v)(f)=0$.  Definiere nun umgekehrt
$$\beta: \left( \slant{\mathfrak{m}_x}{\mathfrak{m}_x^2}\right) ^{*} \longrightarrow T_{V,x}, \quad l \mapsto \left(l (\overline{X_1-x_1}), \ldots, l (\overline{X_n-x_n}) \right)$$
Zeige zunächst: $\beta(l) \in T_{V,x}$ für alle $l \in \left(\slant{\mathfrak{m}_x}{\mathfrak{m}_x^2}\right)^{*}$. Sei dazu $f \in I(V)$ und
$$f_x^{(1)} = \sum_{i=1}^n \frac{\partial f }{\partial X_i} (a) X_i \ \in I$$
seine Linearisierung. Dann ist 

$$f_x^{(1)} \left(\beta(l)\right) = \sum_{i=1}^n \frac{\partial f}{\partial X_i} (x) l\left(\overline{X_i-x_i} \right) = l\left( \sum_{i=1}^n \overline{\frac{\partial f}{\partial X_i} (x) (X_i-x_i)} \right) = l \left( \overline{f_x^{(1)}- f_x^{(1)}(x)} \right) = 0$$
Die letzte Gleichheit gilt, da $f_x^{(1)} - f_x^{(1)} (x) \in \mathfrak{m}_x^2$, denn es gilt

$$\mathbb{K}[V] \ni f = \underbrace{ f(x)}_{=0} + f_x^{(1)} - f_x^{(1)} (x) + \textrm{ Terme in } \mathfrak{m}_x^2$$
Wir rechnen nach:
\begin{compactenum}
\item Es gilt \\[-36pt]
\begin{alignat*}{5}
(\beta \circ \alpha) (v) \ = \ \beta \left(\alpha(v) \right) \quad &=&& \quad \beta \left( f \mapsto \sum_{i=1}^n \frac{ \partial f}{\partial X_i} (x) v_i \right)\\
&=&& \quad \left( \sum_{i=1}^n \frac{\partial (\overline{X_i-x_i})}{\partial X_i} (x) v_1, \ldots, \sum_{i=1}^n \frac{ \partial (\overline{X_n-x_n})}{\partial X_i} (x) v_n \right)\\
&=&& \quad (v_1, \ldots, v_n)
\end{alignat*}
\item sowie für $l \in \slant{\mathfrak{m}_x}{\mathfrak{m}_x^2}$ und $f \in \mathfrak{m}_x$ \\[-36pt]

\begin{alignat*}{7}
(\alpha \circ \beta)(l)(f) \quad &=&& \quad \alpha \left( l ( \overline{X_1-x_1}), \ldots, l ( \overline{X_n-x_n}) \right)\\
&=&& \quad \sum_{i=1}^n \frac{\partial f}{\partial X_i} (x) l(\overline{X_i-x_i}) \\
&=&& \quad l \left(\overline{f_x^{(1)}- f_x^{(1)}(x)} \right)\\
&=&& \quad l(\overline{f}),
\end{alignat*}  
es folgt also die Behauptung. $\hfill \Box$
\end{compactenum}
\end{pr}
\end{theorem}


\begin{folg} %Folgerung 14.9
Sei $V$ quasiprojektive Varietät, $x \in V$. Dann gilt
$$x \textrm{ ist nichtsingulär }\quad  \Longleftrightarrow \quad \dim \slant{\mathfrak{m}_x}{\mathfrak{m}_x^2} = \dim \mathcal{O}_{V,x}$$
\end{folg}

\begin{defin} %Definition 14.10

Ein noetherscher lokaler Ring $R$ heißt \textit{regulär}, falls
$$\dim_{\mathbb{K}} \slant{\mathfrak{m}}{\mathfrak{m}^2} = \dim R,$$
wobei $\mathfrak{m}$ das maximale Ideal in $R$ sowie $\mathbb{K}$ den zugehörigen Restklassenkörper bezeichne.
\end{defin}

\begin{ex}

 Betrachte $R= \mathbb{Z}_{\langle p \rangle}$ für eine Primzahl $p \in \mathbb{P}$. Dann ist $\mathfrak{m}= p \mathbb{Z}_{\langle p \rangle}$ sowie $\mathbb{K}= \slant{\mathbb{Z}_{\langle p \rangle}}{p \mathbb{Z}_{\langle p \rangle}} \cong \mathbb{F}_p$. Weiter ist
 $$ \dim \mathbb{Z}_{\langle p \rangle} = 1 = \dim_{\mathbb{F}_p} \mathbb{F}_p = \dim_{\mathbb{F}_p} \slant{p \mathbb{Z}_{\langle p \rangle}}{p^2 \mathbb{Z}_{\langle p \rangle}} = \dim_{\mathbb{F}_p} \slant{\mathfrak{m}}{\mathfrak{m}^2},$$
 folglich ist $\mathbb{Z}_{\langle p \rangle }$ regulär.
 \end{ex}
 
 
\begin{lemma}[\rm \it Nakayama-Lemma]

Sei $R$ lokaler Ring mit maximalem Ideal $\mathfrak{m}$ und $M$ endlich erzeugter $R$-Modul, $N \subseteq M$ Untermodul. Dann gilt
$$ M = N+ \mathfrak{m} M \quad \Longrightarrow \quad M=N.$$
\begin{pr}
Ohne Einschränkung gelte $N=0$, denn aus $M= \mathfrak{m} M + N$ folgt 
$$\slant{M}{N} = \slant{(N+ \mathfrak{m}M)}{N} \cong \slant{\mathfrak{m} M}{N \cap \mathfrak{m}M} \cong \slant{\mathfrak{m}M }{N}$$
Sei nun also $M = \mathfrak{m}M$ und nehme an, es gelte $M \neq 0$. Dann sei $x_1,\ldots x_n$ ein minimales Erzeugendensystem von $M$. Dann gilt
$$x_1 = \sum_{i=1}^n a_i x_i \qquad \textrm{ für geeignete } a_i \in \mathfrak{m},$$
also wegen $R^{\times} = R \setminus \mathfrak{m}$ 
$$x_1( \underbrace{ 1- a_1}_{ \notin \mathfrak{m}}) = \sum_{i=2}^n a_i x_i \in \langle x_2, \ldots x_n \rangle,$$
ein Widerspruch zur Minimalität. $\hfill \Box$
\end{pr}
\end{lemma}

\begin{lemma} %Lemma 14.11a

Sei $(R, \mathfrak{m})$ noetehrscher lokaler Ring. Dann bilden $x_1, \ldots, x_n \in \mathfrak{m}$ ein minimales Erzeugendensystem von $\mathfrak{m}$ genau dann, wenn die Restklassen $\overline{x}_1, \ldots, \overline{x}_n \in \slant{\mathfrak{m}}{\mathfrak{m}^2}$ eine $\mathbb{K}$-Vektorraumbasis von $\slant{\mathfrak{m}}{\mathfrak{m}^2}$ bilden.

\begin{pr}
\begin{compactenum}
\item["$\Rightarrow$"] Sei also $x_1, \ldots x_n$ ein minimales Erzeugendensystem von $\mathfrak{m}$. Sicherlich bildet $S:=\{\overline{x}_1, \ldots, \overline{x}_n\}$ ein Erzeugendensystem für $\slant{\mathfrak{m}}{\mathfrak{m}^2}$. Angenommen, $S$ ist linear abhängig, d.h. ohne Einschränkung finden wir eine Darstellung
$$\overline{x}_1 = \sum_{i=2}^n \lambda_i \overline{x}_i, \qquad \lambda_i \in \mathbb{K}.$$
Für $\tilde{\lambda}_i \in R$ mit $\overline{\tilde{\lambda}}_i = \lambda_i$ gilt dann
$$x_1 - \sum_{i=2}^n \tilde{\lambda}_i x_i \in \mathfrak{m}^2.$$
Andererseits wird $\mathfrak{m}^2$ erzeugt von den $x_i x_j$. Schreibe also
$$x_1 - \sum_{i=2}^n \tilde{\lambda}_i x_i = \sum_{j=1}^n \mu_{1j} x_1 x_j + \underbrace{\sum_{i,j=2}^n \mu_{ij} x_i x_j}_{=: y} = y+ x_1 \sum_{j=1}^n \mu_1j x_j ,$$
wobei $\mu_i \in R$ geeignete Konstanten sind. Dann folgt
$$x_1 \left( \underbrace{ 1 - \sum_{i=1}^n \mu_i x_i}_{\notin \mathfrak{m}} \right) \in \langle x_2, \ldots, x_n \rangle,$$
also ein Widerspruch zur Minimalität von $S$.

\item["$\Leftarrow$"] Sei nun umgekehrt $\overline{x}_1, \ldots, \overline{x}_n$ eine $\mathbb{K}$-Basis von $\slant{\mathfrak{m}}{\mathfrak{m}^2}$. Zeige nun, dass $x_1, \ldots, x_n$ $\mathfrak{m}$ erzeugen. Die Minimalität ist klar. Sei dazu $N:= \langle x_1, \ldots x_n \rangle \subseteq \mathfrak{m}$. Dann gilt 
 $$\mathfrak{m} = N + \mathfrak{m}^2$$
 und mit Lemma 14.11 folgt $N= \mathfrak{m}$.  $\hfill \Box$


\end{compactenum}
\end{pr}
\end{lemma}

\begin{prop} %Proposition 14.11

Ein noetherscher lokaler Ring $(R, \mathfrak{m})$ ist genau dann regulär, wenn $\mathfrak{m}$ von $\dim R = \textrm{ht} ( \mathfrak{m})$ Elementen erzeugt werden kann.
\begin{pr}
\begin{compactenum}
\item["$\Rightarrow$"] Sei $R$ regulär. Dann gilt $\dim R = \dim \slant{\mathfrak{m}}{\mathfrak{m}^2}=:n$. Dan kann $\mathfrak{m}$ also von $n$ Elementen erzeugt werden.
\item["$\Leftarrow$"] Kann nun umgekehrt $\mathfrak{m}$ von $n:= \dim R$ Elementen erzeugt werden, so auch $\slant{\mathfrak{m}}{\mathfrak{m}^2}$, das heißt, mit Lemma 14.14 gilt bereits $\dim \slant{\mathfrak{m}}{\mathfrak{m}^2} \leqslant \dim R$. Krulls Hauptidealsatz (ohne Beweis) liefert die umgekehrte Ungleichung und damit $\dim R = \dim \slant{\mathfrak{m}}{\mathfrak{m}^2}$. $\hfill \Box$
\end{compactenum}
\end{pr}
\end{prop}

\begin{folg} %Folgerung 14.4

Sei $V \subseteq \mathbb{P}^n(\mathbb{K})$ quasiprojektive Varietät, $x \in V$. Dann gilt
$$x \textrm{ ist singulär } \quad \Longleftrightarrow \quad \dim \slant{\mathfrak{m}_x}{\mathfrak{m}_x^2} > \dim_x V$$
\end{folg}


\begin{prop} %Proposition 14.15
Jede irreduzible $d$-dimensionale Varietät ist birational äquivalent zu einer Hyperfläche in $\mathbb{A}^{d+1}(\mathbb{K})$. \\
\begin{pr}
Zuz zeigen: $\mathbb{K}(V)$ ist isomorph zum Funktionenkörper einer Hyperfläche, also
$$\mathbb{K}(V) \cong \textrm{Quot}\left(\slant{\mathbb{K}[X_1, \ldots, X_n]}{\langle f \rangle } \right)$$
für ein geeignetes $f \in \mathbb{K}[X_1, \ldots, X_n]$. Sei hierfür $\mathbb{K}[V] / \mathbb{K}[X_1, \ldots, X_d]$ eine durch Noethernormalisierung erhaltene, ganze Ringerweiterung. Dann ist $\mathbb{K}(V) / \mathbb{K}(X_1, \ldots, X_d)$ eine endliche Körperweiterung. Ohne Einschränkung sei diese separabel. Dann liefert der Satz vom primitiven Element ein $y \in \mathbb{K}(X_1, \ldots, X_d)$, sodass gilt
$$\mathbb{K}(V) = \mathbb{K}(X_1, \ldots, X_d) [y].$$
Sei $h \in \mathbb{K}(X_1, \ldots, X_d)[Y]$ das Minimalpolynom von $y$ und $g$ der Hauptnenner von $h$. Dann ist 
$$f=g \cdot h \in \mathbb{K}[X_1, \ldots, X_d, Y] \cong \mathbb{K}[X_1, \ldots, X_{d+1}]$$
und
$$\textrm{Quot}\left( \slant{\mathbb{K}[X_1, \ldots, X_d, Y]}{\langle f \rangle} \right) = \mathbb{K}(V),$$
was die Behauptung liefert. $\hfill \Box$
\end{pr}
\end{prop}

\begin{theorem} % Satz 14.16

Sei $\mathbb{K}$ algebraisch abgeschlossen, $V \subseteq \mathbb{P}^n(\mathbb{K})$ nichtleere, quasiprojektive Varietät. Dann ist 
$$\rm{Sing}\it(V):= \{ x \in V \ \vert \ x \textrm{ ist singulär } \}$$
eine echte abgeschlossene Teilmenge.
\begin{pr}
Zeige zunächst, dass $\textrm{Sing}(V)$ abgeschlossen ist. Ohne Einschränkung sei hierfür $V$ irreduzibel. Denn sind $V_1, \ldots, V_r$ die irreduziblen Komponenten von $V$, so gilt
$$\textrm{Sing}(V) = \bigcup_{i=1}^r \textrm{Sing}(V_i) \cup \bigcup_{i\neq j}^r V_i \cap V_j.$$
Weiter sei $V$ ohne Einschränkung affin, denn Abgeschlossenheit ist eine lokale Eigenschaft. Wähle nun Erzeuger $f_1, \ldots, f_r$ von $I(V) \trianglelefteqslant \mathbb{K}[X_1, \ldots, X_n]$ und betrachte die Jacobimatrix $\mathcal{J}:= \left( \frac{\partial f_i}{\partial X_j} \right)_{i,j}$. Dann gilt
\begin{alignat*}{5}
\textrm{Sing}(V) \ &=&& \quad   \{x \in V \ \vert \ \textrm{Rang}( \mathcal{J}(x)) < n - \dim V =:s\}\\
 &=&& \quad  \{ x \in V \ \vert \ \det M(x) = 0 \textrm{ für alle } s \times s \textrm{ Untermatrizen } M \textrm{ von } \mathcal{J} \}
 \end{alignat*}
Da die Determinante ein Polynom in $n$ Variablen ist, ist $\textrm{Sing}(V)= V(\det)$ und $\textrm{Sing}(V)$ als affine Varietät abgeschlossen. 
Zeige nun, dass $\textrm{Sing}(V)$ eine echte Teilmenge von $V$ ist. Ohne Einschränkung sei hierfür $V$ irreduzibel, denn: Sei $Z$ eine irreduzible Komponente von $V$ mit $\textrm{Sing}(Z) \neq Z$, so ist $Z \setminus \textrm{Sing}(Z)$ offen, nichtleer, also dicht in $Z$. Damit enthält $Z \setminus \textrm{Sing}(Z)$ einen Punkt $z$, die auf keiner anderen irreduziblen Komponente liegt. Wegen $\mathcal{O}_{Z,z} = \mathcal{O}_{V,z}$ folgt $z \in V \setminus \textrm{Sing}(V)$, also $\textrm{Sing}(V) \neq V$. \\
Wegen Proposition 14.17 genügt es, denn Spezialfall $V = V(f) \subseteq \mathbb{A}^n(\mathbb{K})$ zu betrachten, wobei $f \in \mathbb{K}[X_1, \ldots, X_n]$ ein irreduzibles Polynom von Grad $\deg f >0$ ist. Es ist
$$\textrm{Sing}(V) = \left\{ x \in V \ \bigg\vert \ \frac{\partial f}{\partial X_1}(x), \ldots, \frac{\partial f}{\partial X_n}(x) = 0 \right\}.$$
Angenommen es gelte $\textrm{Sing}(V)=V$. Dann wäre $\frac{\partial f}{\partial X_i} \in I(V)= \langle f \rangle$ für alle $i \in \{1, \ldots, n \}$. Ist $\textrm{char}(\mathbb{K}) = 0$, so folgt daraus, dass $f$ konstant ist, ist $\textrm{char}(\mathbb{K})=p >0$, so gilt $f \in \mathbb{K}[X_1^p, \ldots, X_n^p]$, also $f=g^p$ für ein $g \in \mathbb{K}[X_1, \ldots, X_n]$. In beide Fällen erhalten wir einen Widerspruch zur Wahl von $f$, es folgt die Behauptung. $\hfill \Box$
\end{pr}
\end{theorem}












% KAPITEL IV


\chapter{Nichtsinguläre Kurven}


\setcounter{section}{14}

\renewcommand*\thesection{§ \arabic{section}\quad}
\section{Diskrete Bewertungsringe} %PARAGRAPH 15
\renewcommand*\thesection{\arabic{section}}

\begin{defin}
Eine zusammenhängende, quasiprojektive Varietät $C$ mit $\dim C=1$ über einem algebraisch abgeschlossenen Körper $\mathbb{K}$ heißt \textit{Kurve}.
\end{defin}

\begin{lemma} %Lemma 15.1

Sei $(R, \mathfrak{m})$ lokaler, noetherscher, nullteilerfreier Ring und es gelte $\dim R=1$. 
Falls $\mathfrak{m}$ ein Hauptideal ist, so ist bereits $R$ ein Hauptidealring.
\begin{pr}
Es sei $I \trianglelefteqslant R$ ein Ideal sowie $t \in \mathfrak{m}$ ein Erzeuger von $\mathfrak{m}$. Ohne Einschränkung gelte $0 \neq I \neq R$, das heißt, es gilt $I \subseteq \mathfrak{m}$. Wähle $n$ maximal, sodass $I \subseteq \mathfrak{m}^n$. Sei $x \in I \cap \left(\mathfrak{m}^n \setminus \mathfrak{m}^{n+1}\right)$. Wegen $\mathfrak{m}^n \supseteq \langle t^n \rangle$ können wir $x$ schreiben als
$$x= u \cdot t^n, \quad u \in R.$$
Wäre $u \notin R^{\times}$, so wäre $u \in \mathfrak{m}$ und damit $x= u \cdot t^n\in \mathfrak{m}^{n+1}$, Widerspruch zur Annahme. Damit ist $t^n = u^{-1} x \in \langle x \rangle \subseteq I \cap \left( \mathfrak{m}^n \setminus \mathfrak{m}^{n+1} \right)$. Dies ergibt $\langle t^n \rangle \subseteq \mathfrak{m}^n$, also $\langle t ^n\rangle = \mathfrak{m}^n$ und
$$ \langle t^n \rangle = \mathfrak{m}^n \subseteq I,$$
also insgesamt $\mathfrak{m}^n=I$, also ist $I$ Hauptideal. Es bleibt zu zeigen. dass man ein solches $n$ wählen kann. Angenommen, es gäbe keines. Dann gilt
$$I \subseteq \bigcap _{n=1}^{\infty} \mathfrak{m}^n =: N.$$
$N$ ist lokal in (einem noetherschen Ring) $R$, also endlich erzeugt. Damit ist
$$\mathfrak{m} \cdot N = \bigcap _{n=1}^{\infty} \mathfrak{m}^{n+1} = N$$
und das Nakayama-Lemma liefert $N=0$ - also $I=0$, ein Widerspruch zur Annahme. $\hfill \Box$
\end{pr}
\end{lemma}




\begin{prop} %Proposition  15.2

Es sei $C$ eine Kurve, $x \in C$. Dann gilt
$$x \textrm{ ist nichtsingulär } \quad \Longleftrightarrow \quad \mathcal{O}_{C,x} \textrm{ ist diskreter Bewertungsring. }$$
\begin{pr}
Da $\mathcal{O}_{C,x}$ noetherscher lokaler Ring von Dimension 1 ist, genügt die Eigenschaft Hauptidealring, um die Behauptung zu zeigen. Nach Lemma 15.2 genügt es hierfür wiederum zu zeigen, dass $\mathfrak{m}$ ein Hauptideal ist. Nach Folgerung 14.10 gilt
$$x \textrm{ ist regulär } \quad \Longleftrightarrow \quad \dim \slant{\mathfrak{m}_x}{\mathfrak{m}_x^2} = \dim \mathcal{O}_{C,x} = 1, $$
nach Krulls Hauptidealsatz kann $\mathfrak{m}_x$ also von einem Element erzeugt werden, ist also ein Hauptideal. Damit folgt die Behauptung. $\hfill \Box$
\end{pr}
\end{prop}


\begin{bemdefin} %Bemerkung + Definition 15.3

Sei $C$ eine Kurve, $C$ irreduzibel, $x \in C$ regulär, $t \in \mathcal{O}_{C,x}$ ein Erzeuger von $\mathfrak{m}_x$. Dann lässt sich $f \in \mathbb{K}(C)^{\times}= \textrm{Quot}(\mathcal{O}_{C,x})^{\times}$ schreiben als
$$f = u \cdot t^n, \qquad n \in \mathbb{Z}, u \in \mathcal{O}_{C,x}^{\times}.$$
Dann heißt $n:= \textrm{ord}_xf$ die \textit{Ordnung} von $f$ in $x$. Weiter ist die Zuordnung $f \mapsto \textrm{ord}_xf$ eine diskrete Bewertung.
\end{bemdefin}


\begin{ex}

Sei $C=V(Y^2-X^3+X)$, $P=(0,0)$ sowie $x,y \in \mathbb{K}(C)$. Es gilt
$Y^2=X(X^2-1)$ auf $C$. Wegen
$$X= \underbrace{\frac{1}{X^2-1}}_{\in \mathcal{O}_{C,P}^{\times}} \cdot Y^2 \in \mathcal{O}_{C,P} \qquad (*)$$
erhalten wir
$$\textrm{ord}_P (x) = 2 \textrm{ord}_P (y).$$
Weiter wird $\mathfrak{m}_P$ erzeugt von $\left( \overline{X-0}, \overline{Y-0} \right)$, mit $(*)$ gilt also 
$$\textrm{ord}_P( y) =1, \qquad \textrm{ord}_P (x )= 2$$
\end{ex}



\begin{prop} %Proposition 15.4

Sei $C$ nichtsinguläre irreduzible Kurve, $f \in \mathbb{K}(C)^{\times}$. Dann gibt es nur endlich viele Punkte $x \in C$ mit $\textrm{ord}_x f \neq 0$.
\begin{pr}
Es gilt
$$ \textrm{ord}_x f >0 \quad \Longleftrightarrow \quad f \in \mathfrak{m}_x \quad \Longleftrightarrow \quad f(x)=0$$
sowie
$$\textrm{ord}_x f <0 \quad \Longleftrightarrow \quad \textrm{ord}_x \frac{1}{f} >0 \quad \Longleftrightarrow \quad \frac{1}{f} \in \mathfrak{m}_x \quad \Longleftrightarrow \quad \frac{1}{f}(x) = 0$$
damit ist 
$$\{x \in C \ \vert \ \textrm{ord}_x f \neq 0 \} = V(f) \cup V \left(\frac{1}{f}\right).$$
Da $f \neq 0 \neq \pm \frac{1}{f}$, sind $V(f), V\left(\frac{1}{f}\right)$ abgeschlossene, echte Teilmengen von $C$. Da $\dim C=1$ und $C$ irrreduzibel ist, haben $V(f)$ und $V\left(\frac{1}{f}\right)$ Dimension $0$, das heißt, die irreduziblen Komponenten der beiden Verschwindungsmengen sind Punkte. Da beide aus endlich vielen irreduziblen Komponenten bestehen, ist auch die Vereinigung endlich und somit folgt die Behauptung.
\end{pr}
\end{prop}


\begin{prop} %Proposition 15.5

Sei $C$ nichtsinguläre, irreduzible Kurve, $U \subseteq C$ offen und nichtleer, $V$ projektive Varietät sowie $f: U \longrightarrow V$ ein Morphismus. Dann gibt es genau einen Morphismus $ \overline{f}: C \longrightarrow V$ mit $\overline{f}\vert_{U} = f$, das heißt, $f$ lässt sich in regulären Punkten fortsetzen.
\begin{pr}
\textit{Eindeutigkeit.} Seien $g,h: C \longrightarrow V, g\vert_{U}=f= h \vert_{U}$. Dann ist 
$$U= \{x \in C \ \vert \ g(x)=h(x) \}$$
abgeschlossen und wegen $\overline{U} = C$ folgt $g=h$. \\
\textit{Existenz.} Ohne Einschränkung sei $C \setminus U = \{p\}$ sowie $V = \mathbb{P}^n(\mathbb{K})$. Außerdem gelte $f(U) \nsubseteq V(X_i)$ (denn sonst wäre $f(U) \subseteq \mathbb{P}^{n-1}(\mathbb{K})$). Sei weiter
$$W:= f^{-1} \left( \bigcap_{i=0}^n U_i \right).$$
$W_i$ ist nichtleer, offen und damit dichte Teilmenge. Definiere
$$h_{ij} = \left( \frac{X_i}{X_j} \circ f \right) = \textrm{"} \frac{f_i}{f_j} \textrm{"}.$$
$h_{ij}$ ist eine wohldefinierte, reguläre Funktion auf $W$ für alle $i,j \in \{0, \ldots, n \}$, also $h_{ij} \in \mathbb{K}(C)^{\times}$. \\
Sei $$r_i := \textrm{ord}_p h_{i0}, \quad i \in \{0, \ldots, n \}$$
und wähle $k$, sodass
$$r_k = \min \{ r_i \ \vert \ i \in \{0, \ldots, n \} \}.$$
Es gilt
$$\textrm{ord}_p h_{ik} = \textrm{ord}_p \frac{h_{i0}}{h_{k0}} = \textrm{ord}_p h_{i0} - \textrm{ord}_p h_{k0} = r_i - r_k \geqslant 0, $$
also $h_{ik} \in \mathcal{O}_{C,p}$. Damit existiert eine Umgebung $\tilde{U}$ von $p$ mit $h_{ik} \in \mathcal{O}(\tilde{U})$. Setze 
$$\overline{f}(x) := \begin{cases} f(x), & \ x \neq p \\ (h_{0k}(x): \ldots : h_{nk}(x) ), & \ x=p \end{cases}$$
Beachte: $\overline{f}$ ist wohldefiniert, da $h_{kk}=1$. In einer Umgebung $\tilde{U}$ von $p$ gilt $x \in \tilde{U} \setminus \{p\}$, also
\begin{alignat*}{5}
\overline{f}(x) = f(x) \ \ &=&& \quad  \left( (X_0 \circ f )(x) : \ldots : \l(X_n \circ f)(x) \right)\\
&=&& \quad \left( \left( \frac{X_0}{X_k} \circ f \right) (x) : \ldots : \left( \frac{X_n}{X_k} \circ f \right) (x) \right)\\
&=&& \quad \left( h_{0k}(x): \ldots : h_{nk}(x) \right),
\end{alignat*}
also ist $\overline{f}$ Morphismus. $\hfill \Box$
\end{pr}
\end{prop}

\begin{folg} %Folgerung 15.6

\begin{compactenum}
\item Eine Funktion $f \in \mathbb{K}(C)$ induziert einen Morphismus $f: C \longrightarrow \mathbb{P}^1(\mathbb{K})$.
\item Ist $C$ nichtsinguläre, zusammenhängende Kurve, so ist $C$ bereits irreduzibel, denn gäbe es zwei irreduzible Komponenten mit nichtleerem Schnitt, so wäre $x \in Z_1 \cap Z_2$ singulär (Übung 12.2).
\end{compactenum}
\end{folg}


\newcommand{\ordp}{\textrm{ord}_P}
\newcommand{\Pn}{\mathbb{P}^n(\mathbb{K})}
\newcommand{\Pone}{\mathbb{P}^1(\mathbb{K})}
\newcommand{\Aone}{\mathbb{A}^1(\mathbb{K})}
\newcommand{\An}{\mathbb{A}^n(\mathbb{K})}




\renewcommand*\thesection{§ \arabic{section}\quad}
\section{Divisoren} %PARAGRAPH 16
\renewcommand*\thesection{\arabic{section}}

In diesem Abschnitt sei $C$ nichtsinguläre Kurve über einem algebraisch abgeschlossenen Körper $\mathbb{K}$.


\begin{defin} %Defintion 16.1

\begin{compactenum}
\item Ein\textit{(Weil-) Divisor} $D$ auf $C$ ist eine formale Summe
$$D= \sum_{i=1}^n n_i P_i, \qquad n_i \in \mathbb{Z}, n \in \mathbb{N}, P_i \in C$$
Schreibweise: $(P)$ für $1 \cdot P$.
\item Die \textit{Divisorengruppe} auf $C$ ist 
$$\textrm{Div}(C):= \{ D \ \vert \ D \textrm{ ist Divisor auf } C \}$$
\item $\textrm{Div}(C)$ ist freie abelsche Gruppe über der Menge $C$.
\item Für eine Divisor $D$ wie in (i) heißt
$$\deg(D) := \sum_{i=1}^n n_i$$
der \textit{Grad} von $D$.
\item Wir haben einen surjektiven Gruppenhomomorphismus
$$\deg: \textrm{Div}(C) \longrightarrow \mathbb{Z}, \quad D \mapsto \deg(D)$$
\item Ein Divisor $D= \sum_{i=1}^n n_i P_i \in \textrm{Div}(C)$ heißt \textit{effektiv}, falls $n_i \geqslant 0$ für alle $i \in \{1, \ldots, n \}$. Schreibweise: $D \geqslant 0$. 
	
\end{compactenum}
\end{defin}

\newcommand{\Div}{\textrm{Div}}

\begin{definbem} %Definition + Bemerkung 16.2

\begin{compactenum}

\item Für $f \in \mathbb{K}(C)^{\times}$ heißt
$$\textrm{div}(f) := \sum_{p \in C} \textrm{ord}_p(f) \cdot P$$
der \textit{Divisor von} $f$.
\item $\textrm{div}(f)$ ist Divisor.
\item Ein Divisor $D \in \textrm{Div}(C)$ heißt \textit{Hauptdivisor}, falls es $f \in \mathbb{K}(C)^{\times}$ gibt mit $D= \textrm{div}(f)$.
\item Haben einen Gruppenhomomorphismus
$$\textrm{div}:\mathbb{K}(C)^{\times} \longrightarrow \textrm{Div}(C), f \mapsto \textrm{div}(f),$$
d.h. es gilt für alle $f, g \in \Div(C)$:
$$\textrm{div}(f \cdot g) = \textrm{div}(f) + \textrm{div}(g)$$
\item Die Hauptdivisoren bilden eine Untergruppe 
$$ \textrm{Div}_H(C) := \textrm{Im} \ \textrm{div} $$
\item $D,D'$ heißen \textit{linear äquivalent}, wenn ihre Differenz $D- D'$ ein Hauptdivisor ist, schreibe $D \equiv D'$. 
\item Der Quotient 
$$\textrm{Cl}(C) := \slant{\Div(C)}{ \Div _H}(C)$$
heißt \textit{Divisorenklassengruppe} von $C$.
\end{compactenum}
\end{definbem}

\begin{ex} Sei $C:= \mathbb{P}^1(\mathbb{K})$. Da $\mathbb{K}$ algebraisch abgeschlossen ist, lässt sich jedes $f \in \mathbb{K}(C)^{\times} = \mathbb{K}(X)^{\times}$ eindeutig schreiben als
$$f = \frac{ \prod_{i=1}^n (X-a_i)}{\prod_{j=1}^m X-b_j}, \qquad a_i \neq b_j \in \mathbb{K} \textrm{ für alle } i,j.$$
Schreibe $\mathbb{P}^1(\mathbb{K}) = \mathbb{A}^1 (\mathbb{K}) \cup \{ \infty \}$. Für $P \in \mathbb{A}^1(\mathbb{K})$ ist
$$\ordp f = \big\vert \{ i\in \{1, \ldots, n \} \ \vert \ a_i =P \} \big\vert - \big\vert \{ j\in \{1, \ldots, m \} \ \vert \ b_j = P \} \big\vert, $$
denn 
$$\mathcal{O}_{\Pone, P} = \mathcal{O}_{\Aone, P}  = \mathbb{K}[X]_{\langle X-p \rangle}$$
wird von $X-p$ erzeugt. Für $P= \infty$ ist 
$$\mathcal{O}_{\Pone, \infty} = \mathbb{K}\left[\frac{1}{X}\right]_{\langle \frac{1}{X} \rangle}$$
Schreibe 
$$f = \frac{X^n}{X^m} \cdot \frac{ \prod_{i=1}^n 1- \frac{a_i}{X}}{\prod_{j=1}^m 1- \frac{b_j}{X}}\ =\ \left( \frac{1}{X}\right)^{m-n} \cdot \frac{\prod_{i=1}^n 1- \frac{a_i}{X}}{\prod_{j=1}^m 1- \frac{b_j}{X}}.$$
Dann folgt
$\textrm{ord}_{\infty} f = m-n.$
Damit ist 
$$\textrm{div}(f) = \sum_{i=1}^n 1 \cdot a_i - \sum_{j=1}^m 1 \cdot b_j + (m-n) \cdot \infty, $$
also $\deg \textrm{div} (f) = 0$.\\
Sei umgekehrt $D \in \Div (C)$ mit $\deg D =0$. Schreibe
$$D= \sum_{i=1}^m 1 \cdot a_i - \sum_{j=1}^m 1 \cdot b_j, \qquad a_i \neq b_j \textrm{ für alle } i,j.$$
Setze
$$f:= \frac{\prod_{a_i \neq \infty} X - a_i }{\prod_{b_j \neq \infty} X- b_j} \ \in \mathbb{K}(C)^{\times}.$$
Dann gilt $\textrm{div}(f) = D$ und damit
$$\Div _H( \Pone) = \{ D \in \Div (\Pone) \ \vert \ \deg D=0 \} = \ker \deg$$
und mit dem Homomorphiesatz
$$\textrm{Cl}(\Pone) = \slant{\Div(\Pone)}{\Div _H ( \Pone)} \cong \mathbb{Z}.$$
\end{ex}

Weiters Vorgehen: Zeige $\deg \textrm{div}(f) =0$ für alle Kurven $C$ und $f \in \mathbb{K}(C)^{\times}$. Fasse hierfür $f$ als Morphismus $f: C \longrightarrow \Pone$ auf. Wollen haben:
\begin{compactenum}
\item $\textrm{div}(f) = f^{*} \left( (0) - (\infty)\right)$ = "Nulstellen minus Polstellen".
\item $\deg f^{*}(D) = \deg f \deg D$.	
\end{compactenum}

\begin{bemdefin} %Bemerkung + Definition 16.4

Sei $f: C_1 \longrightarrow C_2$ surjektiver, nichtkonstanter Morphismus zwischen zwei nichtsingulären Kurven.
\begin{compactenum}
\item Sei $Q \in C_2$, $P \in f^{-1}(Q) \subseteq C_1$ sowie $t \in \mathfrak{m}_Q$ eine Uniformisierende, d.h. es gilt $\langle t \rangle = \mathfrak{m}_Q$. Dann heißt
$$e_P:= e_P (f) = \ordp(t \circ f)$$
der \textit{Verzweigungsindex} von $f$ in $P$. 
\item Definiere den Gruppenhomomorphismus
$$f^{*}: \Div (C_2) \longrightarrow \Div(C_1), \quad Q \mapsto \sum_{P \in f^{-1}(Q)} e_P(f) \cdot P$$
\item Für $g \in \mathbb{K}(C_2)^{\times}$ gilt:
$$f^{*}\left( \textrm{div}(g)\right) = \textrm{div}( g \circ f).$$
Insbesondere ist $f^{*}( \Div _H (C_2)) \subseteq \Div _H (C_1)$.
\item $f$ induziert einen Homomorphismus
$$f^{*}: \textrm{Cl}(C_2) \longrightarrow \textrm{Cl}(C_1), \quad [D] \mapsto [f^{*}(D)]$$
\end{compactenum}
\begin{pr}
\begin{compactenum}
\item Zu zeigen: $e_P(f)$ ist unabhängig von der Wahl von $t$. Sei $t' \in \mathfrak{m}_Q$ eine weitere Uniformisierende. Dann gibt es $u \in \mathcal{O}_{C_2,x}^{\times}$ mit $t'= ut$. Damit ist 
$$\ordp (t' \circ f) = \ordp (ut \circ f) = \ordp ( (u \circ f) \cdot (t \circ f)) = \underbrace{\ordp (u \circ f)}_{=0} +\ordp(t \circ f) = \ordp (t \circ f),$$
wobei letzte Gleichung gilt, da $u \circ f$ Einheit in $\mathcal{O}_{C_1, P}$ mit Inverser $\frac{1}{u} \circ f$ ist.
\item Zu zeigen: $f^{-1}(Q)$ ist endlich, denn dann ist $f^{*}(Q)$ Divisor. Da $f$ stetig ist, ist $f^{-1}(Q)$ abgeschlossen und echte Teilmenge von $C_1$, denn $f^{-1}(Q) \neq C_1$ (da sonst $f$ konstant wäre). Da $\dim C_1 =1$, folgt damit $\dim f^{-1}(Q) =0$, also ist $f^{-1}(Q)$ nach 2.2 endlich.
\item Es gilt
$$f^{*}\left(\textrm{div}(g)\right) = f^{*} \left( \sum_{Q \in C_2} \textrm{ord}_Q (g) \cdot Q \right) \ = \ \sum_{Q \in C_2} \textrm{ord}_Q (g) \cdot f^{*}(Q)\ = \ \sum_{Q \in C_2} \sum_{P \in f^{-1}(Q)} \textrm{ord}_Q (g) e_P (f) \cdot P$$
sowie
$$\textrm{div}(g \circ f) = \sum_{P \in C_1} \ordp (g \circ f) \cdot P = \sum_{Q \in C_2} \sum_{P \in f^{-1}(Q)} \ordp ( g \circ f) \cdot P$$
das heißt, es ist zu zeigen:
$$s:= \ordp( g \circ f) = \textrm{ord}_Q (g) e_P(f) =: r \cdot e_P(f)$$
für alle $Q= f(P)$. \\
Seien dazu $t_Q, t_P$ Uniformisierende von $\mathfrak{m}_Q$ bzw. $\mathfrak{m}_P$, d.h. es gilt $\langle t_Q \rangle = \mathfrak{m}_Q, \langle t_P \rangle = \mathfrak{m}_P$.\\ Dann gibt es $u.u' \in \mathcal{O}_{C_1, P}^{\times}$ sowie $v \in \mathcal{O}_{C_2, Q}^{\times}$ sodass gilt:
$$g \circ f = u \cdot t_P^s, \quad g = v \cdot t_Q^r, \quad t_Q \circ f = u' \cdot t_P^{r \cdot e_P(f)}.$$
Wir rechnen
$$u t_P^s = g \circ f  = ( v \cdot t_Q^r) \circ f = (v \circ f) \cdot ( t_Q \circ f)^r = (v \circ f) \left( u't_P^{e_P(f)}\right)^r = (v \circ f) \cdot u'^r \cdot t_P^{e_P(f) \cdot r}$$
und wegen der Eindeutigkeit der Darstellungen links und rechts folgt
$$s= e_P(f) \cdot r,$$
also die Behauptung.
\item Folgt aus (ii) und (iii). $\hfill \Box$

\end{compactenum}
\end{pr}
\end{bemdefin}

\begin{folg} %Folgerung 15.5

Sei $C$ nichtsingulär, $f \in \mathbb{K}(C)^{\times}$. Dann definiert $f$ einen Morphismus $f: C \longrightarrow \mathbb{P}^1(\mathbb{K})$ und es gilt
$$\textrm{div}(f)= f^{*}( (0) - (\infty) ).$$
\begin{pr}
Die erste Aussage folgt aus Proposition 15.7.\\
Sei $P \in C$ mit $f(P)=0$. Dann ist $X$ eine Uniformisierende von $\mathfrak{m}_P$ und wir erhalten
$$e_P(f)= \textrm{ord}_P(X \circ f ) = \textrm{ord}_P(f)$$
Ist $P = \infty$, so ist $\frac{1}{X}$ Uniformisierende von $\mathfrak{m}_P$ und wir erhalten
$$e_P(f) = \textrm{ord}_P\left(\frac{1}{X} \circ f \right) = \textrm{ord}_P\left( \frac{1}{f} \right) = - \textrm{ord}_P(f).$$
Damit gilt
$$f^{*} ( (0) - (\infty)) = \sum_{P\in f^{1-}(0)} e_P(f) \cdot P -\sum_{P \in f^{-1}(\infty)} e_P(f) \cdot P = \sum_{P \in C} \textrm{ord}_P(f) \cdot P = \textrm{div}(f), $$
was zu zeigen war. $\hfill \Box$
\end{pr}
\end{folg}

\begin{bemdefin} %Bemerkung + Definition 16.6

Sei $f: C_1 \longrightarrow C_2$ surjektiver Morphismus nichtsingulärer, projektiver Kurven. Dann induziert $f$ einen Körperhomomorphismus
$$f^{\#}: \mathbb{K}(C_2) \longrightarrow \mathbb{K}(C_1)$$
$\mathbb{K}(C_2)$ kann damit via $f^{\#}$ als Teilkörper von $\mathbb{K}(C_1)$ aufgefasst werden. Die Erweiterung $\mathbb{K}(C_1)/ \mathbb{K}(C_2)$ ist endlich. $\deg f := [ \mathbb{K}(C_1) : \mathbb{K}(C_2) ]$ heißt \textit{Grad} von $f$.\\
\begin{pr}
Sicherlich sind $\mathbb{K}(C_1), \mathbb{K}(C_2)$ endlich erzeugt über $\mathbb{K}$. Weiter gilt $\textrm{trdeg}_{\mathbb{K}}\mathbb{K}(C_1) = 1 = \textrm{trdeg}_{\mathbb{K}} \mathbb{K}(C_2)$, d.h. die Erweiterung ist algebraisch. Insgesamt folgt also $[\mathbb{K}(C_1): \mathbb{K}(C_2) ] < \infty$. $\hfill \Box$
\end{pr}
\end{bemdefin}

\begin{theorem} %Satz 16.7

\begin{compactenum}
\item Jeder Hauptdivisor auf einer nichtsingulären, projektiven Kurve hat Grad $0$.
\item Sei $f: C_1 \longrightarrow C_2$ surjektiver Morphismus nichtsingulärer, projektiver Kurven. Dann gilt für jeden Punkt $Q \in C_2$
$$\deg f^{*}(Q) = \sum_{P \in f^{-1}(Q)} e_P(f) = \deg f.$$
Weiter gilt damit für jeden Divisor $D \in \textrm{Div}(C_2)$
$$\deg f^{*}(D) = \deg f \cdot \deg D.$$
\end{compactenum}
\begin{pr}
\begin{compactenum}
\item Es sei $f \in \mathbb{K}(C)^{\times}$. Dann lässt sich $f$ fortsetzen zu $f: C \longrightarrow \mathbb{P}^1(\mathbb{K})$. Damit ist
$$\deg (\textrm{div}f) = \sum_{P \in f^{-1}(0)} e_P(f) - \sum_{P \in f^{-1}(\infty)} e_P(f) = \deg f^{*}( (0)- (\infty)) = \deg f \cdot \deg\left( (0)-(\infty)\right) = 0.$$
\item Wird noch hinzugefügt. $\hfill \Box$
\end{compactenum}
\end{pr}
\end{theorem}







\renewcommand*\thesection{§ \arabic{section}\quad}
\section{Der Satz von Riemann-Roch} %PARAGRAPH 17
\renewcommand*\thesection{\arabic{section}}

In diesem Paragraphen sei $C$ stets nichtsinguläre, projektive Kurve über einem algebraisch abgeschlossenen Körper $\mathbb{K}$.



\begin{definbem} %Definition + Bemerkung 17.1

Es sei $D= \sum_{P \in C} n_P \cdot P$ ein Divisor auf $C$.
\begin{compactenum}
\item Der \textit{Riemann-Roch-Raum} zu $D$
$$\mathcal{L}(D) := \{f \in \mathbb{K}(C)^{\times} \ \vert \ \textrm{div}(f) + D \geqslant 0 \} \cup \{0 \}$$
ist ein $\mathbb{K}$-Vektorraum.
\item $l(D):= \dim_{\mathbb{K}} \mathcal{L}(D)$.
\item Es gilt $\mathcal{L}(0)= \mathbb{K}$.
\item Ist $\deg D <0$, so ist $\mathcal{L}(D)= \{0\}$.
\item Für linear äquivalente Divisoren gilt $\mathcal{L}(D) \cong \mathcal{L}(D')$.
\item Für $D' \geqslant D$ gilt $\mathcal{L}(D) \leqslant \mathcal{L}(D')$.
\end{compactenum}
\begin{pr}
\begin{compactenum}
\item[(i)] Es gilt $f \in \mathcal{L}(D) \Longleftrightarrow $ für jeden Punkt $P \in C$ ist $\textrm{ord}_P(f) + n_P \geqslant 0$. Für $f,g \in \mathcal{L}(D)$ ist 
$$\textrm{ord}_P(f+g) \geqslant \min \{ \textrm{ord}_P(f), \textrm{ord}_P(g) \} \geqslant -n_P,$$
also $f+g \in \mathcal{L}(D)$.
\item[(iii)] Es gilt $f \in \mathcal{L}(0)$ genau dann, wenn $\textrm{ord}_P(f) \geqslant 0$ für alle $P \in C$. Damit gilt $f \in \mathcal{O}_C(C)= \mathbb{K}$.
\item[(iv)] Es gilt $\deg ( \textrm{div} f ) =0$, also $\deg ( \textrm{div} f + D) = \deg D < 0$ für alle $f \in \mathbb{K}(C)^{\times}$.
\item[(v)] Es sei $D'  = D + \textrm{div} f$ für ein $f \in \mathbb{K}(C)^{\times}$. Dann ist
$$\alpha: \mathcal{L}(D') \longrightarrow \mathcal{L}(D), \quad g \mapsto f\cdot g$$
ein $\mathbb{K}$-Vektorraumisomorphismus, denn es gilt
$$g \in \mathcal{L}(D') \ \Longleftrightarrow \ \textrm{div} g + D' \geqslant 0 \ \Longleftrightarrow \ \textrm{div}g + \textrm{div}f + D \geqslant 0 \ \Longleftrightarrow \ \textrm{div}f\cdot g + D \geqslant 0.
 \ \Longleftrightarrow \ f\cdot g \in \mathcal{L}(D).$$
 Damit folgt insgesamt die Behauptung. $\hfill \Box$
\end{compactenum}
\end{pr}
\end{definbem}

\begin{prop} %Proposition 17.2

Für jeden Divisor $D \in \textrm{Div}(C)$ und jeden Punkt $P \in C$ gilt
\begin{compactenum}
\item $l(D+P) \leqslant l(D) + 1$.
\item $l(D) \leqslant \deg D + 1$, falls $\deg D \geqslant -1$.
\end{compactenum}
Insbesondere ist $\mathcal{L}(D)$ endlichdimensional.
\begin{pr}
\begin{compactenum}
\item Es gilt $\mathcal{L}(D) \subseteq \mathcal{L}(D+P)$ nach 17.1. Für $f \in \mathcal{L}(D+P) \setminus \mathcal{L}(D)$ gilt $\textrm{ord}_P(f) = -n_P-1$. Für $f,g \in \mathcal{L}(D+P) \setminus \mathcal{L}(D)$ ist also
$$\textrm{ord}_P(f) = \textrm{ord}_P(g)=-n_P-1.$$
Sei nun $t \in \mathfrak{m}_P$ Uniformisierende, d.h. es gilt $\langle t \rangle = \mathfrak{m}_P.$ Schreibe 
$$f = u \cdot t^{-n_P-1}, \quad g = v \cdot t^{-n_P-1}, \quad u,v \in \mathcal{O}_{C,P}^{\times}.$$
Für
$$h = u(P) g- v(p) f \ \ \in \mathcal{L}(D+P)$$
gilt
$$\textrm{ord}_P(h) = \textrm{ord}_P\left( (u(P) v- v(P)  u ) t^{-n_P-1}\right) \geqslant -n_P,$$
also $h \in \mathcal{L}(D)$. Damit ist $g \in \mathcal{L}(D) + \langle f \rangle$, also 
$$\dim \mathcal{L}(D+P) \leqslant \dim \mathcal{L}(D) + 1.$$
\item per Induktion über $d= \deg D$:\\
\textit{d=-1.} Klar, denn es ist $\mathcal{L}(0)=0$.\\
\textit{d$\geqslant 0$.} Sei $P \in C$, $D'= D-P$. Mit der Induktionsvoraussetzung folgt $l(D') \leqslant \deg D' +1 = d$, also mit (i)
$l(D) = l (D'+P) \leqslant d+1.$ $\hfill \Box$
\end{compactenum}
\end{pr}
\end{prop}


\begin{theoremdefin}[\rm \it Satz von Riemann]

Es gibt eine Konstante $\gamma \in \mathbb{N}_0$, sodass für jeden Divisor $D \in \textrm{Div}(C)$ gilt:
$$l(D) \geqslant \deg D +1 - \gamma.$$
Das kleinste $\gamma$ mit dieser Eigenschaft nennen wir das\textit{Geschlecht von} $C$. Schreibe
$$g:= g(C) = \min \{ \gamma \in \mathbb{N}_0 \ \vert \ l(D) \leqslant \deg D +1 - \gamma \}$$

%\begin{pr}
%Wird hinzugefügt.
%\end{pr}
\end{theoremdefin} 

\begin{theorem}[\rm \it Satz von Riemann-Roch]

Es gibt einen (bis auf lineare Äquivalenz eindeutigen) Divisor $K$ auf $C$, der sogenannte \rm{kanonische Divisor}\it, sodass für alle Divisoren $D \in \rm{Div}\it (C)$ gilt:
$$l(D) - l (K-D) = \deg D +1 - g(C).$$
\end{theorem}




\end{spacing}



\end{document}
