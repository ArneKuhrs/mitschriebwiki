\documentclass{article}
\usepackage[utf8]{inputenc}
\usepackage{mathrsfs}
\usepackage{stmaryrd}

\usepackage{mathe}

\title{3. Topologie-Übung}
\author{Joachim Breitner}
\date{7. November 2007}

\begin{document}
\maketitle
\section*{Aufgabe 1}

Sei $(X, d)$ ein beschränkter metrischer Raum, $\mathcal C_b(x) (X) \da \{\text{beschränkte reellwertige Fnktionen auf $X$}\}$, $|f|_\infty \da \sup \{ |f(x)| \mid x \in X\}$, Metrik $d_\infty(f,g) = |f-g|_\infty$ auf $\mathcal C_b(X)$.

\paragraph{Behauptung:} Für jedes $a\in X$ ist die Funktion $f_a: X\to \MdR$, $x\mapsto d(a,x)$ stetig und beschränkt.

Seien $x\in x$, $f_a(x) \da r \in \MdR$, $\ep > 0$. Dann gilt für alle $y\in B_{\frac\ep2}(x)$ dass $d(a,y) \le d(a,x) + d(x,y) \le d(a,x) + \frac\ep2$ sowie dass $d(a,x) \le d(a,y) + d(y,x) \le d(a,y) + \frac\ep2$. Also ist $d(x,a)-\frac\ep2 \le d(y,a) \le d(x,a) + \frac\ep2$ und $f_a(y)= d(a,y) \in B_\ep(f_a(x))$.

Mit $\delta = \frac\ep2$ gilt also: $f_a(B_\delta(x)) \subseteq B_\ep(f_a(x)) = B_\ep(r)$, also ist $f_a$ stetig.

$f_a$ ist beschränkt, da $X$ beschränkt ist (klar nach Definition von $f_a$).\hfill$\blacksquare$

\paragraph{Behauptung:} $\varphi : X\to \mathcal C_b(X)$, $x\mapsto f_X$ ist abstandserhaltend bezüglich $d$ und $d_\infty$.

Zu zeigen ist: $\forall x,y\in X: d_\infty(f_x,f_y) = d(x,y)$. Einerseits gilt: $d_\infty(f_x,f_y) = \sup_{x\in X} | f_x(t) - f_y(t) | = \sup _{x\in X} |d(x,t) - d(y,t)| \le d(x,y)$ wegen der Dreiecksungleichung für $d$. Andererseits gilt: $d_\infty(f_x, f_y) = \sup_{x\in X} |f_x(t) - f_y(t)| \ge |f_x(y) - f_y(y)| = |d(x,y) - 0| = d(x,y)$. Insgesamt gilt also: $d_\infty(f_x,f_y) = d(x,y)$ \hfill$\blacksquare$

\section*{Aufgabe 2}

$(X,d)$ metrischer Raum, $f:\mathbb R_{\ge0} \to \mathbb R_{\ge0}$ monoton wachsend, nicht konstant, konkav mit $f(0)=0$.

\paragraph{Behauptung:} $f\circ d$ ist Metrik auf $X$.

\begin{itemize}
\item Symmetrie: \checkmark
\item $f(d(x,x)) = f(0) = 0$ für alle $x\in X$.
\item $f(d(x,y)) = 0 \iff x=y$;

Da $f$ monoton wachsend, nicht konstant und konkav ist, ist $0$ die einzige Nullstelle von $f$: Es gibt ein $\tilde x \in X : f(\tilde x) > 0$, da $f$ nicht konstant ist. Wäre $x\ne 0$ eine weitere Nullstelle von $f$, so gelte $x<\tilde x$, da $f$ monoton ist. Dann: $0 = f(x) = f(\frac x {\tilde x}\cdot \tilde x)\ge \frac x {\tilde x} f(\tilde x) > 0$, $\lightning$.

\item $\Delta$-Ungleichung. Zu zeigen: $f(a) + f(b) \ge f(a + b)$, denn dann gilt $\forall x,y,z\in X: f(d(x,y)) + f(d(y,z)) \ge f(d(x,y) + d(y,z)) \ge f(d(x,z))$

Es ist: $f(a) = f(\frac a {a+b}(a+b)) = f(\frac a {a+b} (a+b) + 0 ) \ge \frac a{a+b} f(a+b) + (1- \frac a{a+b}) f(0) = \frac a{a+b} f(a+b)$. Ebenso ist: $f(b) \ge \frac b{a+b} f(a+b)$. Addiert man diese Ungleichungen, erhält man $f(a) + f(b) \ge \frac a{a+b}f(a+b) + \frac b{a+b}f(a+b) = f(a+b)$.\hfill$\blacksquare$
\end{itemize}

\paragraph{Behauptung:} Ist $f$ streng monoton, dann definieren $f$ und $f\circ d$ die selbe Topologie auf $X$.

$f$ streng Monoton wachsend, also gilt $\forall \ep>0: d(x,y) <\ep \iff f(d(x,y)) < f(\ep)$. Also ist: $B_\ep^d(x) = \{y\in X\mid d(x,y) <\ep \} = \{y\in X\mid f(d(x,y))< f(\ep)\} = B_{f(x)}^{f\circ d}(x)$. Also ist jeder offene Ball bezüglich $d$ ist ein offener Ball bezüglich $f\circ g$ und umgekehrt.

\emph{Das ist falsch!} Gegenbeispiel: $X\in R$, $d$ der euklidische Abstand, $f:\mathbb R_{\ge0} \to \mathbb R_{\ge0}$, $0 \mapsto 0$, $x\mapsto 1+x$ für $x>0$. Dann ist $B_{\frac 12}^{f\circ d}(x) = \{x\}$

Der Beweis funktioniert mit der zusätzlichen Annahme $\lim_{x\to 0} f(x) = 0$.

\paragraph{Behauptung} Auch wenn $f$ streng monoton ist könnte $X$ bezüglch $f\circ d$ vollständig sein und bezüglich $d$ nicht.

Beispiel: Sei $(X,d)$ ein nicht vollständiger Raum und $f$ wie im letzten Gegenbeispiel. Dann sind Cauchyfolgen gerade die, die konstant wird, also konvergieren sie.

\section*{Aufgabe 3}

Sei $X=\MdR^2$ versehen mit der SNCF-Metrik:
\[
d(x,y) =
\begin{cases}
|x-y|, &x = \lambda y,\, y\in \MdR \\
|x|+|y|, &\text{sonst.}
\end{cases}
\]

Einfach zu zeigen: $d$ ist eine Metrik. Skizzen für $B_1( (\begin{smallmatrix} 2 \\0 \end{smallmatrix}) )$ und $B_3( (\begin{smallmatrix} 2 \\0 \end{smallmatrix}) )$ hier ausgelassen.

\paragraph{Behauptung:} $K = \{x\in \MdR^2 \mid 1\le |x|\le 2\}$ ist beschränkt und abgeschlossen.

$K$ ist beschränkt, da $\forall x,y\in K, x\ne \lambda y: d(x,y) = |x|+|y| \le 2+2 \le 4$ und $\forall x,y\in K, x=\lambda y : d(x,y) \le 4$.

$K$ ist abgeschlossen: Wir zeigen, dass $\MdR^2\setminus K$ offen ist. Sei $x\in \MdR^2\setminus K$. Ist $\ep < |x|$, so ist $B_\ep(x) \subseteq \{y | y = tx, t\in \MdR\}$. Wählt man $\ep$ klein genug, so ist $B_\ep (x) \subseteq \MdR^1\setminus K)$, also ist $\MdR^2\setminus K$ offen.

\paragraph{Behauptung:} $K$ ist nicht kompakt.

$K$ hat eine offene Überdeckung $U = \{B_{1}(x)\mid x\in K_{\frac 32}\}$, wobei $K_{\frac 32} = \{ x\in \MdR^2 \mid |x| = \frac 32 \}$, aus der man keine endliche Teilüberdeckung auswählen kann.

\section*{Aufgabe 4}

Sei $p\ge 3$ ein Primzahl und $d$ der $p$-Adische Abstand auf $\MdQ$: $d(x,y) = |x-y|_p$, wobei für $x\in \MdQ$ gilt: $x = p^{k} \frac ab$, mit $p\nmid a,b$ und $|x|_p = p^{-k}$

\paragraph{Behauptung:} Die Abbildung $x\mapsto x^2$ ist stetig auf $\MdQ$.

Zu zeigen: $\forall x\in, \ep >0$ $\exists \delta > 0 : f(B_\delta(x))\subset B_\ep(f(x))$. Seien also $x\in \MdQ$, $\ep >0$, und sei $y\in B_{\sqrt x}(x)$, dann ist $|x-y|_p < \sqrt x \implies y-x = p^k \frac ab - p^l \frac cd$ mit $p^{-k} <\sqrt \ep$. O.B.d.A: $k<l$, $p\nmid a,b,c,d$. Dann ist $(y-x)^2 = p^{2k} \frac {a^2}{b^2} - p^kp^l \frac{ac}{bd} + p^{2l} \frac{c^2}{d^2} = p^{2k} ( \frac {a^2}{b^2} - p^{l-k} \frac{ac}{bd} + p^{2(l-k)} \frac{c^2}{d^2})$, also ist $|(y-x)^2|_p = p^{-2k} = (p^{-k})^2 < (\sqrt{\ep})^2 = \ep$. Hier wurde ein Denkfehler in der Beweisführung entdeckt, und eine korrekte Version für später, im Internet, angekündigt.

\paragraph{Behauptung:} Sei $a\in \MdZ$ mit $d(a^2,-1) \le \frac 1{p^k}$ für $k\ge 1$. Dann gibt es ein $c\in \MdZ$ mit $d( (a+cp^k)^2, -1) \le \frac 1{p^{k+1}}$.

$|a^2 - (-1)|_p = |a^2 +1|_p \le \frac1{p^k}$, also $a^2 +1 = p^k b$, $b\in \MdZ$.Gesucht ist ein $c\in \MdZ$ mit $|(a + cp^k)^2+1| \le \frac1{p^{k+1}}$, also $(a + cp^k)^2 + 1 = a^2 + 2p^kac + p^{2k}c^2 + 1 = p^k b + 2p^kac + p^{2k}c^2 = p^k(2ac + b) + p^{2k}c^2 \stackrel!= p^{k+1} \tilde a$. Es muss gelten: $2ac + b \equiv 0 \pmod p$. So ein $c$ existiert, da $\MdZ/p\MdZ$ Körper ist.

\paragraph{Behauptung:} Es gibt eine Cauchyfolge in $\MdQ$, die bezüglich $d$ für $p=5$ nicht konvergiert.

Setze $x_0\da 2$, denn $d(x_0^2,-1) = |4+1|_p = \frac 15$. Nach der letzten Teilaufgabe gibt es ein $c_1\in \MdZ$, so dass $d( (x_0 + c_1p)^2, -1)\le \frac1{5^2}$. Setzte $x_1 \da x_0 + c_1p$, und analog $x_2,\ldots$.

Das ist eine Cauchy-Folge: $|x_{n+1} - x_n|_p \le \frac 1{p^n}$ nach Konstruktion, und wegen $|x+y|_p \le \max\{ |x|_p, |y|_p\}$ ist $(x_n)$ eine Cauchy-Folge. Nach Konstruktion konvergiert $(x_n^2)$ gegen $-1$. Wir wissen, dass $x\mapsto x^2$ stetig ist. Konvergierte also die Folge $(x_n)$, so müsste für den Grenzwert $x$ gelten: $x^2 = -1$, im Widerspruch zu $x\in \MdQ$.
\end{document}
