\documentclass{article}
\usepackage[utf8]{inputenc}
\usepackage{mathrsfs}
\usepackage{stmaryrd}

\usepackage{mathe}
\usepackage{enumerate}

\title{9. Topologie-Übung}
\author{Joachim Breitner}
\date{19. Dezember 2007}

\begin{document}
\maketitle
\section*{Aufgabe 1}

Sei $K\da \overline{B_1(0)} \subseteq \MdR^2$.

\paragraph{Behauptung:} Jede stetige Abbildung $G:K\to K$ hat mindestens einen Fixpunkt.

Wir nehmen an, $G$ habe keinen Fixpunkt, also $\forall x\in K: G(x)\ne x$.

Für $x\in K$ definiere $\lambda_x\in \MdR_{>0}$ als die eindeutig bestimmte Zahl, für die gilt: $G(x)+\lambda_x(x-G(x))\in S^1$. Behauptung: $\lambda: K\to \MdR_{>0}$, $x\mapsto \lambda_x$, stetig. Dann ist auch $F: K\to S^1,$ $x\mapsto G(x)+\lambda_x(x-G(x))$ stetig und $F|_{S^1}=\operatorname{id}_{S^1}$, was laut Vorlesung nicht geht.

$\lambda$ ist stetig: Schreibe $G(x) = 
(\begin{smallmatrix}
G_1\\G_2
\end{smallmatrix}), x=
(\begin{smallmatrix}
x_1\\x_2
\end{smallmatrix})$. Es ist
\begin{multline*}
\|G(x)+\lambda_x(x-G(x))\|=1 \iff \|
\begin{pmatrix}
G_1\\G_2
\end{pmatrix} +
\lambda_x\begin{pmatrix}
x_1 - G_1 \\
x_2 - G_2
\end{pmatrix}\|
\\= (G_1+\lambda_x(x_1-G_1))^2 + (G_2 + \lambda_x(x_2-G_2))^2 = 1
\end{multline*}
eine quadratische Gleichung mit Lösung $\lambda_x$, also hängt $\lambda_x$ stetig von $x$ und $G(x)$ ab.


\paragraph{Behauptung:} Das gilt auch für jeden zu $K$ homöomorphen Raum $X$.

Sei $H: K\to X$ ein Homöomorphismus und $G:X\to X$ stetig. Zu zeigen ist: $\exists x\in X: G(x)=x$. Sei $f\da H\circ G \circ H^{-1} : K\to K$. $f$ ist stetig, also gibt es ein $a\in K$: mit $f(a)=a \iff H\circ G \circ H^{-1}(a) = a \iff G(H^{-1}(a)) = H^{-1}(a)$. Also ist $x\da H^{-1}(a)$ ein Fixpunkt von $G$.

\section*{Aufgabe 2}

Sei $\gamma:[0,1]\to \MdR^2$ eine stetige geschlossene Kurve, $x\in \MdR^2$.

\paragraph{Behauptung:} $\chi(\gamma,x)$ hängt stetig von $x\in \MdR^2\setminus\gamma([0,1])$ ab.

Zur Erinnerung: Sei $\sigma: [0,1]\to S^1$, dann gibt es genau ein $\lambda:[0,1]\to \MdR$, so dass $\sigma = \pi \circ \lambda$, wobei $\pi:t\mapsto (\cos(2\pi t), \sin(2\pi t))$ gilt. Die Umlaufzahl von $\sigma$ umd $0$ ist dann definiert als $\lambda(1)-\lambda(0)$.

Sei $\gamma:[0,1]\to \MdR^1\setminus\{0\}$ eine stetige geschlossene Kurve, dann ist 
\[
\gamma(t) = \|\gamma(t)\| \cdot \underbrace{\frac{\gamma(t)}{\|\gamma(t)\|}}_{\eqqcolon \sigma}
\]
und $\chi(\gamma,0) \da \lambda(1)-\lambda(0)$.

Für $\lambda:[0,1]\to \MdR^2$, $x\in \MdR^2\setminus\gamma([0,1])$, definiere die Umlaufzahl $\chi(\gamma,x)\da \chi(\tilde\gamma,0)$, wobei $\tilde\gamma(t) \da \gamma(t) -x$.

Sei $(x_n)$ eine Folge in $\MdR^2\setminus\gamma([0,1])$ mit $x_n \to x$ für $n\to \infty$. Zu zeigen: $\chi(\gamma,x_n) \to \chi(\gamma,x)$. Definiere
$ \Gamma: [0,1]\times[0,1]\to \MdR^2\setminus\gamma([0,1])$ stetig mit $\Gamma(0,t) = \gamma(t)-x\coloneqq \tilde\gamma_0(t)$ und $\Gamma(1,t) = \gamma(t)-x_n\coloneqq \tilde\gamma_1(t)$. Laut Vorlesung gilt in diesem Fall: $\chi(\tilde\gamma_1,0)=\chi(\tilde\gamma_0,0)=\chi(\gamma,x)=\chi(\gamma,x_n)$.

Definiere also $\Gamma(r,t)\da \gamma(t) - ( (1-r)\cdot x + r x_n) \in \MdR^2\setminus\{0\}$. Für $n$ groß genug ist das die gesuchte Abbildung. Für alle $n\ge N_0$ gilt dann: $\chi(\tilde\gamma_1,0)=\chi(\tilde\gamma_0,0) \implies \forall n\ge N_0: \chi(\gamma,x_n)=\chi(\gamma,x) \implies \chi(\gamma,x_n)\to \chi(\gamma,x) \implies$ Behauptung.

\paragraph{Behauptung:} Es gibt eine Zusammenhangskomponente, auf der die Umlaufzahl von $\gamma$ Null ist.

$\gamma([0,1])$ ist kompakt, also gibt es ein $r\in \MdR$, so dass $\gamma([0,1])\subseteq B_r(0)$). Sei $x\in \MdR^2$ mit $\|x\|\ge 2r$. Sei
\[
\tilde\gamma(t) = \|\tilde\gamma(t)\| \cdot \underbrace{\frac{\tilde\gamma(t)}{\|\tilde\gamma(t)\|}}_{\eqqcolon \sigma(t)}
\]
Es ist $\chi(\gamma,x)=\chi(\tilde\gamma,0)=0$, denn:

Angenommen $\gamma(1) \ne \gamma(0) \implies \operatorname{Bild}(\pi\circ \gamma)=S^1$, im Widerspruch zur Skizze an der Tafel.

\section*{Aufgabe 4}

Sei $X$ ein topologischer Raum, $x\in X$ und $A\subseteq X$.

\paragraph{Behauptung:} $x\in\bar A$ genau dann, wenn es einen Filter $\mathcal F$ gibt mit $A\in \mathcal F$ und $\mathcal F \to x$.

„$\implies$“: Sei $x\in \bar A$. Die Obermengen der Mengen $\{U\cap A \mid U$ Umgebung von $x\}$ bilden einen Filter mit $A\in \mathcal F$, der gegen $X$ konvergiert. $\emptyset\notin\mathcal F$, da jede Umgebung von $x\in \bar A$ nichtleeren Schnitt mit $A$ hat.

„$\impliedby$“: Sei $\mathcal F$ ein Filter mit $A\in \mathcal F$, der gegen $x$ konvergiert. Also liegen alle Umgebungen  $U$ von $x$ in $\mathcal F$. $U\cap A\ne \emptyset$ (sonst wäre $\emptyset\in \mathcal F$). Ist $x\in A$, so ist $x\in \bar A$ sowieso. Ist $x\notin A$, so gilt für jede Umgebung $U$ von $x$: $U\cap A\ne \emptyset$ und $U\cap (X\setminus A)\ne \emptyset$, also ist $x\in \partial A\subseteq \bar A$.

\paragraph{Behauptung:} Es gibt einen toplogischen Raum $X$, $A\subseteq X$ und $x\in \bar A$, so dass keine Folge $(x_n)$ in $A$ gegen $x$ konvergiert.

Setzte $X\da \MdN_0\times \MdN_0$, definiere Topologie $J$ durch $A\in J \iff (0,0) \ne A$, oder $\{n\in\MdN_0\mid (n,m)\notin A\}$ ist endlich für fast alle $M$. $(X,J)$ ist ein topologischer Raum. $A\da X\setminus \{(0,0)\}$. Es gibt keine Folge in $A$, die gegen $(0,0)$ konvergiert, aber $(0,0)\in X = \bar A$:

Sei $(x_i)_{i\in \MdN} \ad (n_i,m_i)_{i\in\MdN}$ eine Folge in $A$.

1. Fall: Es gibt  ein $m\in \MdN_0$, so dass $m_i=m$ für unendlich viele $i\in\MdN$. Dann ist $U\da X\setminus \{(n,m)\mid n\in \MdN_0\}\cup\{(0,0)\}$ ist eine Umgebung von $(0,0)$, in der mehr als endlich viele Elemente der Folge nicht liegen, also konverigiert die Folge nicht.

2. Fall: Für alle $m\in \MdN_0$ gilt: $m_i=m$ für endlich viele $i$. Dann ist $U\da X\setminus\{x_i|i\in \MdN\}$ ist Umgebung von $(0,0)$, in der keine Folgenglieder liegen, also konvergiert auch hier die Folge nicht.
\end{document}
