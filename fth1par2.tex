\documentclass{article}
\newcounter{chapter}
\setcounter{chapter}{2}
\usepackage{ana}
\def\gdw{\equizu}
\title{Topologische Begriffe}
\author{Christian Schulz}
% Wer nennenswerte Änderungen macht, schreibt euch bei \author dazu

\begin{document}
\maketitle

\begin{definition}
$(a_n)$ sei eine Folge in $\MdC$.
\begin{liste}
	\item $(a_n)$ heißt \begriff{beschränkt} $\gdw \exists c \geq 0 : |a_n| \leq c$ $\forall n \in \MdN$
	\item $(a_n)$ heißt eine \begriff{Cauchy-Folge} (CF) $:\gdw$ $\forall \epsilon > 0 \exists n_0 \in \MdN : |a_n-a_m|< \epsilon $ $\forall n,m \geq n_0$
	\item $(a_n)$ heißt \begriff{konvergent} $:\gdw$ $\exists a \in \MdC : |a_n -a | \to 0$ $ ( \gdw \forall \epsilon > 0 \exists n_0 \in \MdN : |a_n - a| < \epsilon$ $\forall n \geq n_0)$ \\
	      In diesem Fall ist a eindeutig bestimmt (Übung) und heißt der \begriff{Grenzwert} (GW) oder \begriff{Limes} von $(a_n)$. Man schreibt : $\lim_{n \to \infty} a_n = a$ oder $a_n \to a (n \to \infty)$ 
	\item $(a_n)$ heißt \begriff{divergent} $:\gdw (a_n)$ konvergiert nicht.
\end{liste}
\end{definition}

\begin{beispiel}
$a_n = \frac{1}{n} + i(1+\frac{1}{n})$; $|a_n -i| = |\frac{1}{n}-i\frac{1}{n}| = \frac{|1-i|}{n} \to 0 (n \to \infty)$
$\Rightarrow a_n \to i$
\end{beispiel}
Wie in $\MdR$ bzw. mit 1.3, zeigt man:

%Satz 2.1 
\begin{satz}
$(a_n),(b_n)$ seien Folgen in $\MdC$; $a,b \in \MdC$
\begin{liste}
	\item $(a_n)$ konvergent $\Rightarrow$ $a_n$ ist beschränkt.
	\item $(a_n)$ konvergent $:\gdw$ $($Re $ a_n), ($Im $a_n)$ sind konvergent. In diesem Fall gilt $\lim a_n = \lim$ Re $a_n + i \lim$ Im $a_n$
	\item Es gelte $(a_n) \to a, (b_n) \to b$. Dann: \\ $a_n+b_n \to a+b$, $a_nb_n \to ab$, $\bar{a_n} \to \bar{a}$, $|a_n| \to |a|$ \\
		  Ist $a \neq 0$ $\Rightarrow \exists m \in \MdN : a_n \neq 0$ $\forall n \geq m$ und $\frac{1}{a_n} \to \frac{1}{a}$ \\
		  Ist $a_{n_k}$ eine Teilfolge (TF) von $(a_n)$ $\Rightarrow$ $a_{n_k} \to a (k \to \infty)$
	\item Ist $(a_n)$ beschränkt $\Rightarrow$ $(a_n)$ enthält eine konvergente TF (\begriff{Bolzano-Weierstraß})
	\item $(a_n)$ ist eine CF $\gdw$ $(a_n)$ ist konvergent (\begriff{Cauchykriterium})
\end{liste}
\end{satz}

\begin{definition}
$(a_n)$ sei eine Folge in $\MdC$ und $s_n := \sum_{i=1}^{n} a_i$ $n\in \MdN$. $(s_n)$ heißt eine \begriff{unendliche Reihe} und wird mit 
$\sum_{n=1}^{\infty} a_n$ bezeichnet. $\sum_{n=1}^{\infty} a_n$ heißt konvergent/divergent $\gdw$ $(s_n)$ konvergent/divergent. \\
Ist $\sum_{n=1}^{\infty} a_n$ konvergent, so schreibt man $\sum_{n=1}^{\infty} a_n$ $:= \lim_{n \to \infty} s_n$ 
\end{definition}

\begin{beispiel}[Geometrische Reihe]
$\sum_{n=0}^{\infty} z^n = 1+z+\cdots$ $(z \in \MdC)$. Wie in $\MdR$ zeigt man:
\begin{liste}
\item $1+z+\cdots+z^n =\begin{cases}
		\frac{1-z^{n+1}}{1-z} &, \text{falls } z \neq 1 \\
		n+1 &, \text{falls } z = 1
		\end{cases}
		$
\item $\sum_{n=0}^{\infty} z^n$ konvergent $\gdw$ $|z| < 1$. In diesem Fall $\sum_{n=0}^{\infty} z^n = \frac{1}{1-z}$ 
\end{liste}
\end{beispiel}

\begin{definition}
$\sum_{n=1}^{\infty} a_n $ heißt \begriff{absolut konvergent} $:\gdw$ $\sum_{n=1}^{\infty} |a_n|$ konvergent.
\end{definition}

Wörtlich wie in $\MdR$ beweist, bzw. formuliert man:
%Satz 2.2 
\begin{satz}
$(a_n)$ sei eine Folge in $\MdC$
\begin{liste}
\item Ist $\sum_{n=1}^{\infty} a_n$ konvergent $\Rightarrow$ $a_n \to 0$
\item Ist $\sum_{n=1}^{\infty} a_n$ absolut konvergent $\Rightarrow$ $\sum_{n=1}^{\infty} a_n$ konvergent und $|\sum_{n=1}^{\infty} a_n| \leq \sum_{n=1}^{\infty} |a_n|$
\item Es gelten Cauchykriterium, Majorantenkriterium, Minorantenkriterium, Wurzelkriterium, Quotientenkriterium und der Satz über das Cauchyprodukt.
\end{liste}
\end{satz}

\begin{definition}
Sei $A \subseteq \MdC, z_0 \in \MdC$ und $\epsilon > 0$
\begin{liste}
\item $U_{\epsilon}(z_0) := \{ z \in \MdC : |z-z_0| < \epsilon \}$ \begriff{$\epsilon$-Umgebung von $z_0$} oder \begriff{offene Kreisscheibe} von $z_0$ mit Radius $\epsilon$ \\
      $\overline{U_{\epsilon}(z_0)} := \{ z \in \MdC |$ $|z-z_0| \leq \epsilon \}$ (\begriff{abgeschlossene Kreisscheibe} von $z_0$ mit Radius $\epsilon$) \\
      $\dot{U}_{\epsilon}(z_0) := U_{\epsilon}(z_0) \backslash \{z_0\}$ (\begriff{punktierte Kreisschreibe})
\item $z_0 \in A$ heißt \begriff{innerer Punkt von A} $:\gdw$ $\exists \epsilon > 0 : U_{\epsilon}(z_0) \subseteq A$ \\
      $A^o := \{ z\in A | z $ innerer Punkt von A$ \}$ heißt das \begriff{Innere von A}. Klar ist: $A^o \subseteq A$\\
      $A$ heißt offen $:\gdw$ $A=A^o$
\item $A$ heißt \begriff{abgeschlossen} $:\gdw$ $\MdC \backslash A$ ist offen.
\item $A$ heißt \begriff{beschränkt} $:\gdw$ $\exists c \geq 0 : |a| \leq c$ $\forall a \in A$
\item $A$ heißt \begriff{kompakt} $:\gdw$ $A$ ist beschränkt und abgeschlossen.
\item $z_0$ heißt ein \begriff{Häufungspunkt} von $A$ $:\gdw$ $\forall \epsilon > 0:  \dot{U}_{\epsilon}(z_0) \cap A \neq \emptyset$.\\
      $\bar{A} := \{ z \in \MdC | z $ ist HP von $A$$ \} \cup A$ heißt die \begriff{Abschließung} von A
\item $z_0$ heißt ein \begriff{Randpunkt} von $A$ \\ $:\gdw$ $\forall \epsilon > 0: U_{\epsilon}(z_0) \cap A \neq \emptyset$ und $U_{\epsilon}(z_0) \cap (\MdC \backslash A) \neq \emptyset$ \\
	  $\partial A := \{ z \in \MdC | z $ ist Randpunkt von $A$ $\}$ wird als \begriff{Rand von A} bezeichnet
\end{liste}
\end{definition}

Wie in $\MdR$ zeigt man:
%Satz 2.3 
\begin{satz}
\begin{liste}
\item $A$ heißt abgeschlossen $\gdw$ $A = \bar{A}$ $\gdw$ der Grenzwert jeder konvergenten Folge aus $A$ gehört zu $A$.
\item $z_0$ ist HP von $A$ $\gdw$ $\exists$ Folge $(z_n)$ in $A \backslash \{z_0\}: z_n \to z_0$
\item $A$ ist kompakt $:\gdw$ jede Folge in $A$ enthält eine konvergente Teilfolge deren Limes zu $A$ gehört \\
	  $\gdw$ jede offene Überdeckung von $A$ enthält eine endliche Überdeckung von $A$.
\end{liste}
\end{satz}
\end{document}
