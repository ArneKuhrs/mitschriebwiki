\section{Tensoralgebra}

\begin{Def}
\label{1.14}
Eine \emp{$R$-Algebra}\index{R-Algebra} ist ein (kommutativer) Ring (mit Eins) $R'$
zusammen mit einem Ringhomomorphismus $\alpha: R\to R'$.
Ist $\alpha$ injektiv, so hei\ss t $R'/R$ auch Ringerweiterung.
\end{Def}

\begin{Bem}
\label{1.15}
Sei $R'$ eine $R$-Algebra.
\begin{enumerate}
\item Die Zuordnung $M\to M\ten[R] R'$ ist ein kovarianter rechtsexakter Funktor
\[
\ten[R] R':\KatRMod\to \KatRMod[R'].
\]
Dabei wird $M\ten[R] R'$ zum $R'$-Modul durch
\[
b\cdot (x\ten a)\defeqr x\ten b\cdot a.
\]
\item Sei $V: \KatRMod[R']\to \KatRMod$ der \glqq Vergiss-Funktor\grqq, der
jeden $R'$-Modul als 
$R$-Modul auffasst, mit der Skalarmultiplikation $a\cdot x\defeqr\alpha(a)\cdot x$
f\"ur $a\in R, x\in M$.\\
Dann ist $\ten[R] R'$ \glqq links adjungiert\grqq\ zu $V$, d.h. f\"ur alle 
$R$-Moduln $M$ und $R'$-Moduln $M'$ sind $\Hom[R]{M}{V(M')}$ und 
$\Hom[{R'}]{M\ten[R] R'}{M'}$ isomorph (als $R$-Moduln).
\end{enumerate}
\end{Bem}

\begin{Bew}
\item[(b)] Die Zuordnungen $$\begin{array}{rcl}
\Hom[R]{M}{V(M')} & \to & \Hom[R']{M\ten[R] R'}{M'}\\
\varphi & \mapsto & (x\ten a\mapsto a\cdot \varphi(x))\\
(x\mapsto \psi(x\ten 1))&\mapsfrom & \psi \\
\end{array}$$
sind zueinander invers.
\end{Bew}

\begin{nnBsp}
Sei $R'$ eine $R$-Algebra, $F$ freier $R$-Modul mit Basis $\{e_i:i\in I\}$. Dann ist $F\ten[R] R'$ ein freier
$R'$-Modul mit Basis $\{e_i\ten 1:i\in I\}$.\\
\textbf{denn}: Sei $f:\{e_i\ten 1: i\in I\} \to M$ Abb. ($M$ bel. $R'$-Modul).
Dann gibt es genau eine $R$-lineare Abbildung $\varphi: F\to V(M)$ mit $\varphi(e_i)=f(e_i\ten 1)$ (UAE f\"ur $F$).
Mit \ref{1.15} (b) folgt: dazu geh\"ort eine eindeutige $R'$-lineare Abbildung
$\tilde\varphi: F\ten[R] R'\to M$ mit $\tilde\varphi(e_i\ten 1)=\varphi(e_i)$.
\end{nnBsp}

\begin{Prop}
\label{1.16}
Seien $R', R''$ $R$-Algebren.
\begin{enumerate}
\item $R'\ten[R] R''$ wird zur $R$-Algebra durch $(a_1\ten b_1)\cdot (a_2 \ten b_2)\defeqr a_1a_2\ten b_1 b_2$
\item $\sigma': R'\to R'\ten[R] R'', a\mapsto a\ten 1$ und \\
$\sigma'': R''\to R'\ten[R] R'', b\mapsto 1\ten b$
sind \RAlgHoms.
\item UAE: in der Kategorie der $R$-Algebren gilt:
\[
\begin{xy}
\xymatrix{
R' \ar[d]_{\sigma'} \ar[drr]^{\varphi'}    & & \\
R' \ten[R] R'' \ar[rr]^{\exists!\varphi} & & A \\
R'' \ar[u]^{\sigma''} \ar[urr]_{\varphi''} & &
}
\end{xy}
\]
\end{enumerate}
\end{Prop}

\begin{nnBsp} $R'$ sei eine $R$-Algebra. Dass ist $R'[X] \cong R[X] \ten[R] R'$ (als $R'$-Algebren), denn: Zeige, dass $R[X] \ten[R] R'$ 
die UAE des Polynomrings $R'[X]$ erfüllt. 
Sei $A$ eine $R'$-Algebra und $a \in A$. Zu zeigen: $\exists ! R'$-Algebrahom. $\varphi: R[X] \ten[R] R' \rightarrow A$ mit
$\varphi(X \ten 1 ) = a$. Ein solcher wird als \RAlgHom\ induziert von $\varphi': R[X] \to A,\ X \mapsto a$
und $\varphi'': R' \rightarrow A$ (der Strukturhom. $\alpha$ aus der Def.).\\
Noch zu zeigen: $\varphi$ ist $R'$-linear (richtig, weil $\varphi''$ Ringhomom.).
\end{nnBsp}

\begin{DefBem}
Sei $M$ ein $R$-Modul
\begin{enumerate}
\item[ a)] $T^0(M) := R$, $ T^n(M) = M \ten[R] T^{n-1}(M), n \geq 1$
\item[ b)] $T(M) := \bigoplus^{\infty}_{n = 0 } T^n(M)$ wird durch
\[
(x_1 \ten \dots \ten x_n) \cdot (y_1 \ten \dots \ten y_m) :=
x_1 \ten \dots \ten x_n \ten y_1 \ten \dots \ten y_m \in T^{n + m}(M)
\]
zur $R$-Algebra, genannt die
\emp{Tensoralgebra}\index{Tensoralgebra}\index{Algebra!Tensor-}\index{$T(M)$
(Tensoralgebra von $M$)} von $M$.
\item[ c)] $T(M)$ ist nicht kommutativ (außer im Trivialfall), denn $ x \ten y \neq y \ten x$
\item[ d)] $T(M)$ erfüllt UAE: Ist $R'$ $R$-Algebra (nicht notwendig
kommutativ) $\varphi: M \rightarrow R'$ $R$-linear, so $\exists!$ \RAlgHom\ 
$\tilde{\varphi}:T(M) \rightarrow R'$ mit $\tilde{\varphi}|\ub{_{T^1(M)}}{=M}=\varphi$.
\end{enumerate}
\end{DefBem}
