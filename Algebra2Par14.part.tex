\section{Tensoralgebra}

\begin{Def}
\label{1.14}
Eine \emp{$R$-Algebra}\index{R-Algebra} ist ein (kommutativer) Ring (mit Eins) $R'$
zusammen mit einem Ringhomomorphismus $\alpha: R\to R'$.
Ist $\alpha$ injektiv, so hei\ss t $R'/R$ auch Ringerweiterung.
\end{Def}

\begin{Bem}
\label{1.15}
Sei $R'$ eine $R$-Algebra.
\begin{enumerate}
\item Die Zuordnung $M\to M\otimes_R R'$ ist ein kovarianter rechtsexakter Funktor
$\otimes_R R':R-Mod\to R'-Mod$; dabei wird $M\otimes_R R'$ zum $R'$-Modul durch
$b\cdot (x\otimes a)\defeqr x\otimes b\cdot a$.
\item Sei $V: R'\mbox{-Mod}\to R\mbox{-Mod}$ der ``Vergiss-Funktor``, der jeden $R'$-Modul als 
$R$-Modul auffasst, mit der Skalarmultiplikation $a\cdot x\defeqr\alpha(a)\cdot x$
f\"ur $a\in R, x\in M$.\\
Dann ist $\otimes R'$ ``links adjungiert'' zu $V$, d.h. f\"ur alle 
$R$-Moduln $M$ und $R'$-Moduln $M'$ sind $\textrm{Hom}_R(M, V(M'))$ und 
$\textrm{Hom}_{R'}(M\otimes_R R', M')$ isomorph (als $R$-Moduln).
\end{enumerate}
\end{Bem}

\begin{Bew}
\item[(b)] Die Zuordnungen $$\begin{array}{rcl}
\textrm{Hom}_R(M, V(M')) & \to & \textrm{Hom}_{R'}(M\otimes_R R', M')\\
\varphi & \mapsto & (x\otimes a\mapsto a\cdot \varphi(x))\\
(x\mapsto \psi(x\otimes 1))&\mapsfrom & \psi \\
\end{array}$$
sind zueinander invers.
\end{Bew}

\begin{nnBsp}
Sei $R'$ eine $R$-Algebra, $F$ freier Modul mit Basis $\{e_i:i\in I\}$. Dann ist $F\otimes_R R'$ ein freier
$R'$-Modul mit Basis $\{e_i\otimes 1:i\in I\}$.\\
\textbf{denn}: Sei $f:\{e_i\otimes 1: i\in I\} \to M$ Abb. ($M$ bel. $R'$-Modul).
Dann gibt es genau eine $R$-lineare Abbildung $\varphi: F\to V(M)$ mit $\varphi(e_i)=f(e_i\otimes 1)$ (UAE f\"ur $F$).
Mit \ref{1.15} (b) folgt: dazu geh\"ort eine eindeutige $R'$-lineare Abbildung
$\tilde\varphi: F\otimes_R R'\to M$ mit $\tilde\varphi(e_i\otimes 1)=\varphi(e_i)$.
\end{nnBsp}

\begin{Prop}
\label{1.16}
Seien $R', R''$ $R$ Algebren.
\begin{enumerate}
\item $R'\otimes_R R''$ wird zur $R$-Algebra durch $(a_1\otimes b_1)\cdot (a_2 \otimes b_2)\defeqr a_1a_2\otimes b_1 b_2$
\item $\sigma': R'\to R'\otimes_R R'', a\mapsto a\otimes 1$ und \\
$\sigma'': R''\to R'\otimes_R R'', b\mapsto 1\otimes b$
sind $R$-Algebrenhomomorphismen.
\item UAE: in der Kategorie der $R$-Algebren gilt:
\[
\begin{xy}
\xymatrix{
R' \ar[d]_{\sigma'} \ar[drr]^{\varphi'}    & & \\
R' \otimes_R R'' \ar[rr]^{\exists!\varphi} & & A \\
R'' \ar[u]^{\sigma''} \ar[urr]_{\varphi''} & &
}
\end{xy}
\]
\end{enumerate}
\end{Prop}

\begin{nnBsp} $R'$ sei eine $R$-Algebra. Dass ist $R'[X] \cong R[X] \otimes_R R'$ ( als $R'$-Algebren ), denn: Zeige, dass $R[X] \otimes_R R'$ 
die UAE des Polynomrings $R'[X]$ erfüllt. 
Sei $A$ eine $R'$-Algebra und $a \in A$. Zu zeigen: $\exists ! R'$-Algebrahom. $\varphi: R[X] \otimes_R R' \rightarrow A$ mit
$\varphi(X \otimes 1 ) = a$. Ein solcher wird als $R$-Algebra-Homomorphismus induziert von $\varphi': R[X] \rightarrow A, x \rightarrow a$
und $\varphi'': R' \rightarrow A$ ( der Strukturhom. $\alpha$ aus der Def. )\\
Noch zu zeigen: $\varphi$ ist $R'$-linear ( richtig, weil $\varphi''$ Ringhomom. ).
\end{nnBsp}
