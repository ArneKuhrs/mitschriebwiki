\documentclass[a4paper,DIV15,BCOR12mm]{article}
\newcounter{chapter}
\setcounter{chapter}{8}
\usepackage{ztheo}
%\usepackage{tikz}

\title{Ganzzahlige quadratische Formen}
\author{Joachim Breitner}

\begin{document}
\maketitle

\section{Grundbegriffe und Bezeichnungen}

\paragraph{Problem:} Man diskutiert die diophantische Gleichung
\[ k = ax^2 + bxy + cy^2 \quad (*)\]
Gegeben sind $a,b,z,k \in\MdZ$, gesucht ist ein $\underline{x} = (x,y)\in\MdZ^2$, für die $(*)$ gilt.

Gegeben $Q=aX^2+bXY+cY^2 \in \MdZ[X,Y]$, $a,b,c \ne 0$, mit Kurzbezeichnung $Q=[a,b,c]$. Dieses $Q$ heißt ganzzahlige binäre (wegen den 2 Variablen) quadratische ($\grad q=2$) Form.

Nun betrachtet man $Q$ als Abbildung $\MdZ^2 \to \MdZ^2$, $\underline x = (x,y) \mapsto Q(x,y)$.

\begin{definition}
\begin{enumerate}
\item $\underline x$ primitiv $\iff \ggt(x,y)=1$
\item $Q$ primitiv $\iff \ggt(a,b,c) = 1$
\item $Q$ stellt $k\in\MdZ$, $k\ne 0$ (primitiv) da $\iff \exists \underline x \in \MdZ^3$ ($\underline x$ primitiv), mit $Q(\underline x) = k$
\end{enumerate}
\end{definition}

\paragraph{Problem:} Welche Formen stellen welche Zahlen dar? $Q(\MdZ^2) =$ ?

Falls $k\in Q(\underline x)$, welche weiteren $\underline x'$ erzeugen $k=Q(\underline x')$? $Q^{-1}(\{k\})=$ ?

\begin{bemerkung}
\begin{enumerate}
\item $z\in\MdZ$, so $Q(z\cdot \underline x)= z^2 \cdot Q(\underline x)$
\item Mit $Q$ ist auch $mQ$ eine Quadratische Form ($m\in\MdZ$, $m\ne 0$)
\end{enumerate}
Wegen (1) genügt es meist, primitive Darstellungen zu betrachten.
\end{bemerkung}

Aus der Linearen Algebra ist über reelle Quadriken bekannt: Es gibt Darsellungsmatrixen $A_Q = \MdR^{2\times 2}$ mit $Q(x) = xA_Q x^\top$, wobei 
\[ A_Q = \begin{pmatrix} a & \frac b 2 \\ \frac b 2 & c \end{pmatrix}\]

\paragraph{Idee} (Gauß?) Wegen $\MdZ^2 U = \MdZ^2$ für $U\in GL_2(\MdZ)$ gilt $Q(\MdZ^2) = Q\cdot (\MdZ^2 U)$. $Q(\underline xU) = \underline xU \cdot A_Q \cdot (x U)^\top = \underline x (U A_Q U^\top) x^\top$

\begin{definition}
\begin{enumerate}
\item Zu $Q$ sei $U.Q$ die Quadratische Form mit Darstellungsmatrix $UA_QU^\top$
\item $Q$ und $Q^\top$ heißen (eigentlich) äquivalent ($Q \sim Q'$ bzw $Q \approx Q'$) $\iff \exists U\in GL_2(\MdZ)$ (bzw. $\exists I \in SL_2(\MdZ)$, wobei $SL_2(\MdZ) = \{ U\in\MdZ^{2\times 2}\mid \det U = 1\}$) mit $Q' = U.Q$.
\end{enumerate}
\end{definition}

$\sim$, $\approx$ unterscheiden sich wenig, sozusagen höchstens um eine Matrix $\begin{pmatrix} 0 & 1 \\ 1 & 0\end{pmatrix}$.

\begin{bemerkung}
\begin{enumerate}
\item $1_2.Q = Q$, $U,V \in GL_2(\MdZ)$. $(UV).Q = U.(V.Q)$.\\
"`$GL_2(\MdZ)$ bzw. $SL_2(\MdZ)$ operiert auf der Menge der Quadratischen Formen"'
\item $\sim$, $\approx$ sind Äquivalenzrelationen
\item Äquivalente Formen stellen die selben Zahlen dar.
\end{enumerate}
\end{bemerkung}

\begin{beweis}
\begin{enumerate}
\item $UV.Q$: $UVA_Q(UV)^\top = U(VA_QV^\top)U^\top: U.(V.Q)$.

Folgt $Q'=U.Q$, so $U^{-1}.Q' = U^{-1}.(U.Q)=(U^{-1}U).Q = 1_2. Q =Q$.

Also ist $\sim$ symetisch: $Q\sim Q$.

Transitivität: $Q\sim Q'$, $Q'=U.Q$ und $Q'\sim Q''$, $Q'' = V.Q$, mit $U,V\in GL_2(\MdZ)$, so ist $Q''=V.(U.Q)=(VU).Q \folgt Q''\sim Q$
\end{enumerate}
\end{beweis}

\section{Die Diskriminante}

Sei $Q=[a,b,c]$ eine Quadratische Form.

\begin{definition}
$\Delta =-4 \cdot \det A_Q = b^2 - 4ac = \dis(Q) \in \MdZ$ heißt Diskriminante von $Q$.
\end{definition}

Bemerkung aus der Linearen Algebra: $\mathcal{V} = \mathcal{V}_{Q-k}(\MdR) = \{\underline x \in \MdR^2 \mid Q(\underline x) = k\}$ ist reelle Quadrik, abgesehen von ausgearteten Fällen gilt: $\Delta <0$: $\mathcal{V}$ Ellipse, $\Delta >0$, $\mathcal{V}$ Hyperbel.

\begin{beispiel}
$X^2 + 5Y^2$ Ellipse: $\Delta = 0-4\cdot 5 = -20 < 0$\\
$X^2 + -2Y^2$ Hyperbel: $\Delta = 0-4\cdot (-2) = 8 > 0$
\end{beispiel}

\paragraph{Problem:} Welche $(x,y)\in\MdZ^2$ (Gitterpunkte) liegen auf $\mathcal{V}$.

\begin{satz}[Diskriminantensatz]
Sei $Q$ eine Quadratische Form.
\begin{enumerate}
\item Ist $Q\sim Q'$, so gilt $\dis(Q) = \dis(Q')$.
\item Ist $\Delta = \dis Q$ ein Quadrat in $\MdZ \iff$ "`$Q$ zerfällt über $\MdZ$"', also $\exists u,v,w,z\in\MdZ$ mit $Q=(uX+vY)(wX+zY)$
\item Ist $\dis Q \ne 0$, so gilt 
\begin{align*}
 Q \text{ definit } &\iff \dis Q <0 \\
 Q \text{ indefinit } &\iff \dis Q > 0
 \end{align*}
\item $0\ne d\in \MdZ$ ist Diskriminante $\iff d\equiv 0,1 \mod 4$
\end{enumerate}
\end{satz}

Anwendung: $\Delta = \dis Q$ sei ein Quadrat
$Q(\underline x) = k \ne 0 \iff \exists d\in\MdZ, d\mit k$: $ux+vy=d$, $wx+zy=\frac kd$. Die Frage nach den darstellbaren $k$ läuft zurück auf a) Bestimmung aller Teiler von $k$, b) Diskussion eines ganzzahligen LSG.

Ab jetzt interessieren nur noch nichtquadratische Diskriminanten.

\begin{beweis}
\begin{enumerate}
\item[(4)] $\delta = \dis Q = b^2 - 4ac \equiv b^2 \equiv 0,1 \mod 4$. \\
$d\equiv 0 \mod 4$: $Q=[1,0,-\frac d 4]$ \\
$d\equiv 1 \mod 4$: $Q=[1,1,-\frac {1-d} 4]$ \\
Für diese Formen gilt $\dis Q = d \equiv \Delta$. Diese Form heißt "`Hauptform"' der Diskriminante.
\item[(1)] $\det UA_QA^\top = \det U \cdot \det U^\top \cdot \det A_Q = (\det U)^2 \cdot \det A_Q = \det A_Q \folgt$ Behauptung.
\item[(2)] (Skizze)
\begin{enumerate}
\item["`$\Leftarrow$"'] Nachrechnen
\item["`$\Rightarrow$"'] $\Delta = \dis Q = q^2$. Sei $t=\ggt(a, \frac {b-a} 2)$, dann (Übung):
\[ Q=(\frac a t X + \frac {b-q} {2t} Y)(t X + \frac{b+q}{2\frac at} Y) \]
\end{enumerate}
\item[(3)] $a=0\folgt \Delta > 0$, $Q=bXY + cY^2 = (bX + cY) Y$ indefinit\\
$a\ne 0$: $aQ = (aX+bY)^2 - \frac 14 \Delta Y^2$. Offensichtlich: $\Delta < 0$: definit, $\Delta > 0$: indefinit
\end{enumerate}
\end{beweis}<++>

\section{Darstellung von Zahlen durch QFen}
Vor. $Q$ QF, $\dis Q = \Delta$ sei kein Quadrat.\\
$U.Q$ QF mit Matrix $UA_qU^T, U \in GL_2(\MdZ)$\\
$U = \begin{pmatrix}r & s \\ u & v\end{pmatrix} \Rightarrow U.Q = [Q(r,s), 2rU\cdot a + (rv + su)b + 2sv\cdot c, Q(u,v)]$

Spezialfälle:\\
$Q' = \begin{pmatrix}1 & 0\\t & 1\end{pmatrix}.Q = [a, t \cdot 2a + b, at^2 + bt + c]$\\
$Q' = \begin{pmatrix}\cdot & 1\\ -1 & t\end{pmatrix}.Q = [c, -b + 2ct, ct^2 - bt + a]$\\
$Q' = \begin{pmatrix}\cdot & 1\\ -1 & \cdot\end{pmatrix}.Q = [c, -b, a]$\\
$Q' = \begin{pmatrix}1 & \cdot\\1 & 1\end{pmatrix}.Q = [a, 2a + b, a + b + c]$

Wunsch:\\
Algorithmus der feststellt, ob $Q$ $k$ darstellt oder nicht.

\begin{satz}[1. Darstellungssatz]
$Q$ stellt $0 \not= k \in \MdZ$ genau dann primitiv dar, wenn: $\exists Q' = [k,l,m]$ mit $Q' \approx Q \wedge -|k| < l \le |k|$.
\end{satz}
Hat man also einen Algorithmus, der feststellt, ob $Q \approx Q' \vee Q \not\approx Q'$, so hat man einfach $2k$ Formen zu testen (auf Äquivalenz zu $Q$). ($m = \frac{l^2 - \Delta}{4k}$)

Spezialfall:\\
$k = 1, Q$ stellt $1$ dar $\Leftrightarrow Q \approx [1, 0, \frac{-\Delta}{4}]$ (für $\Delta \equiv 0 \mod 4$)\\
--HIER FEHLT NOCH EINE ZEILE, WELCHE NICHT RICHTIG KOPIERT WURDE --

$Q \approx [1, 1, \frac{1 - \Delta}{4}]$ (für $\Delta \equiv 1 \mod 4$).\\
Ergebnis: Genau die zur Hauptform äquivalenten Formen stellen $1$ dar.

\begin{beweis}
\begin{itemize}
\item[\underline{"`$\Leftarrow$"':}] $Q'(1,0) = k$. Hat man $Q' \approx Q \Rightarrow Q$ stellt $k$ dar
\item[\underline{"`$\Rightarrow$"':}] $k = Q(x,y), \ggt (x,y) = 1$. LinKomSatz liefert $u,v \in \MdZ$ mit $xv-yu = 1 \Rightarrow U := \begin{pmatrix}x & y\\u & v\end{pmatrix} \in Sl_2(\MdZ)$\\
$Q_1 := U.Q = [\underbrace{Q(x,y)}_{=k}, l', \text{irgendwas}]$, $l := l' \text{ mods } 2|k|, \exists t: l = l' + 2tk \Rightarrow Q' = \begin{pmatrix}1 & \cdot \\ t & 1\end{pmatrix}.Q_1$ wie verlangt.
\end{itemize}
\end{beweis}

\begin{satz}[2. Darstellungssatz]
Sei $k \in \MdZ, k \not= 0$. Genau dann gibt es eine Form $Q$ mit $\dis Q = \Delta$, die $k$ primitiv darstellt, wenn die Kongruenz $l^2 \equiv \Delta \mod 4k$ so lösbar ist, dass $\ggt (k, l, \frac{l^2 - \Delta}{4k}) = 1$.
\end{satz}

\begin{beweis}
\begin{itemize}
\item[\underline{"`$\Leftarrow$"':}] Einfach, die Form $[k, l, \frac{l^2 - \Delta}{4k}]$ tut es
\item[\underline{"`$\Rightarrow$"':}] $k$ so darstellbar $Q \approx Q' = [k, l, \frac{l^2 - \Delta}{4k}]$ nach \emph{1. Darstellungssatz} (für (mindestens) ein $l$) $\Rightarrow \frac{l^2 - \Delta}{4k} \in \MdZ \Rightarrow l^2 \equiv \Delta \mod 4k$ [ggT stimmt auch]
\end{itemize}
\end{beweis}

Spezialfälle:\\
Sei $k = p \in \MdP$
\begin{itemize}
\item $p \nmid \Delta, p \not= 2: p$ so darstellbar $\Leftrightarrow (\frac{\Delta}{p}) = 1$
\item $p \mid \Delta, p \not= 2: p$ so darstellbar $\Leftrightarrow v_p(\Delta) = 1$
\item $p = 2 \mid \Delta$: $2$ so darstellbar $\Leftrightarrow \Delta \equiv 8, 12 \mod 16$
\end{itemize}

Zu den Spezialfällen
\begin{itemize}
\item $p \nmid \Delta: (\frac{\Delta}{p}) = 1$ lösbar, $l_1^2 \equiv \Delta \mod p \Leftrightarrow l_1^2 \equiv \Delta \mod 4p \leadsto ChRs$
\item $2 \not= p \mid \Delta$: Löse $l \equiv 0 \equiv \Delta \mod p (\ast)$, $l^2 \equiv \mod 4 \Rightarrow l^2 \equiv \Delta \mod 4p$\\
$\ggt (\underbrace{p, l}_{\ggt = p}, \frac{l^2 - \Delta}{4p}) = 1 \Leftrightarrow p \nmid \frac{l^2 - \Delta}{4p} \Leftrightarrow p^2 \nmid l^2 - \Delta \Leftrightarrow p^2 \nmid \Delta$, da $p^2 \mid l^2$ nach ($\ast$). ($\Rightarrow v_p(\Delta) = 1$)
\item $p = 2 \mid \Delta$: Ü.
\end{itemize}

\begin{definition}
Die \underline{Klassenzahl} $h(\Delta)$ ist die Anzahl der Klassen eigentlich äquivalenter Formen mit Diskriminante $\Delta$. "`Schöne Resultate"', falls $h(\Delta) = 1$.\\
$\Rightarrow$ Alle Formen der Diskriminante $\Delta$ stellen $k$ dar $\Leftrightarrow$ Bed. 2. DarstSatz.
\end{definition}

Später. $h(-4) = 1, Q = [1,0,1]$
Ergebnis: $2 \not= p \in \MdP$ wird durch $Q = x^2 + y^2$ dargestellt $\Leftrightarrow 1 = (\frac{-4}{p}) = \frac{-1}{p} = (-1)^{\frac{p-1}{2}} \Leftrightarrow p \equiv 1 \mod 4$
Andere Beispiele, etwa $\Delta = -164$ (Klassenzahl 1, betragsmäßig größte negative Zahl. Im positiven unbekannt)

\section{Reduktion der definiten Formen}

Sei $\Delta < 0$ [und damit "`Nicht-Quadrat"'], $\Delta = b^2 - 4ac \Rightarrow ac > 0$. Ohne Einschränkung positiv definit, d.h. $a > 0, c > 0$.
\begin{definition}[Gauß]
$Q$ (mit Diskr $\Delta$) heißt \underline{reduziert} $\Leftrightarrow |b| \le a \le c$
\end{definition}

In dieser Vorlesung:\\
$Q$ heißt \underline{vollreduziert} $\Leftrightarrow Q$ ist reduziert und falls $(c = 0 \wedge b \not= 0) \vee (|b| = a)$ auch noch $b > 0$ ist.

Idee (Gauß):\\
Setzte $|Q| := a + |b|$. Versuche $Q' \approx Q$ zu finden mit $|Q'| < |Q|$. Das geht, solange $Q$ nicht reduziert ist.
\begin{itemize}
\item[Fall I:] $a > c, Q' := \begin{pmatrix}\cdot & 1 \\ -1 & \cdot\end{pmatrix}, Q = [\underbrace{c}_{-a'}, \underbrace{-b}_{b'}, \underbrace{a}_{c'}]$. $|Q'| = a' + |b'| = |b| + c < |b| + a = |Q|$
\item[Fall II:] $a \le c, |b| > a$ (da $Q$ nicht-reduziert) Division von $b$ mit Rest durch $2a$: $\exists t \in \MdZ: b = b' - 2ta, -a < b' \le a$. $Q' = \begin{pmatrix}1 & \cdot \\ t & 1\end{pmatrix}.Q = [a, \underbrace{b + 2ta}_{b'}, c']$. $|Q'| = |b'| + a \le a + \underbrace{|a|}_{= a (\text{ da }-a \le a)}$
\end{itemize}

Dies ergibt Vollreduktionsalgorithmus $red(Q)$, der $\tilde Q$ berechnet mit $\tilde Q \approx Q \wedge \tilde Q$ vollreduziert. Wiederholte Anwendung von $Q := Q'$ aus Fall I,II endet nach endlich vielen Schritten mit reduziertem $Q_1 \approx Q$. Falls $Q_1$ vollreduziert, so $\tilde Q := Q_1$.\\
Falls $Q_1$ nicht vollreduziert, so $2$ Fälle für $Q_1 = [a,b,c]$
\begin{itemize}
\item $c = a$, aber $b < 0: \tilde Q := \begin{pmatrix}\cdot & 1 \\ -1 & \cdot\end{pmatrix}.Q_1 = [a,-b,a]$, jetzt $-b > 0$
\item $|b| = a$, also $b = -a < 0$. $\tilde Q = \begin{pmatrix}1 & \cdot\\1 & 1\end{pmatrix}.[a,-a,c] = [a,a,c], c' = a+b+c = c$ ist vollreduziert ($b' = a > 0$).
\end{itemize}
Ziel: $2$ vollreduzierte Formen der Disk $\Delta$ sind äquivalent $\Leftrightarrow$ sie sind gleich. Es folgt:\\
$Q \approx Q' \Leftrightarrow \text{ red }Q = \text{ red }Q'$. Daher gibt es einen Algorithmus, der entscheidet, ob $Q \approx Q' \vee Q \not\approx Q'$

Hilfsatz:\\
$Q = [a, b, c]$ sei reduziert. Dann:
\begin{itemize}
\item[(i)] $a = \min Q(\MdZ^2 \backslash 0)$
\item[(ii)] Für $a < c$ ist $Q^{-1}(\{a\}) = \{\pm(1,0)\}$ (klar: $Q(\underline x) = Q(-\underline x)$)\\
Für $0 \le b < a = c$ ist $Q^{-1}(\{a\}) = \{\pm (1,0), \pm (0,1)\}$. (Für $|b| = a = c$ (=$1$, da $Q$ primitiv) $Q[1, \pm 1, 1] = x^2 \pm yx + y^2 \Rightarrow \# Q^{-1}\{a\} = 6)$\\
\end{itemize}
$|b| \le a \le c$\\
$(\ast)$ $Q(x,y) = ax^2 + bxy + cy^2 \stackrel{(1)}{\ge} ax^2 - |bxy| - ay^2 \ge a(|x|-|y|)^2 + (2a - |b|)|xy| \ge a(\underbrace{(|x|-|y|)^2 + |xy|}_{\in \MdZ, \not= 0, \text{ wenn } (x,y) \not= 0, \text{ also } \ge 1} \stackrel{(4)}{\ge} a$.



Erinnerung:\\
$Q = [a,b,c]$ reduziert $\Leftrightarrow |b| \le a \le c$\\
Vollreduziert: Falls $a = c \wedge b \not= 0 \vee a = c = |b|$, so $b > 0 \leadsto$ Vollreduktionsalgorithmus red.

Sei $Q(x,y) = a \Rightarrow$ in ($\ast$) überall "`c"'\\
$a < c \Rightarrow y = 0$ (sonst bei (1) >)\\
"`="' bei (4) $\Rightarrow (|x|-|y|)^2 + |xy| = 1 \Rightarrow (x,y) \in M = \{\pm (1,0), \pm (0,1), (\pm 1, \pm 1)\}$
\begin{itemize}
\item[Fall I:] $Q^{-1}(a) = \{\pm (1,0)\}, \# Q^{-1}(a) = 2$
\item[Fall II:] $a = c$, aber $|b| < a \Rightarrow 2a-|b| > a \Rightarrow$ "`="' nur für $|xy| = 0$. $Q^{-1}(a) = \{\pm (1,0), \pm (0,1)\}$
\item[Fall III:] $a = c = |b|$, etwa $b > 0$, so $x^2 + xy + y^2 = 1$ von $(\pm 1, \pm 1)$ in $M$ nur $\pm (1, -1)$ [dazu noch $\pm (1,0), \pm(0,1)$] $\Rightarrow \#Q^{-1}(a) = 6$
\end{itemize}

Folgerung: Sei $Q, Q'$ vollständig reduziert und $Q \approx Q'$, so ist $Q = Q'$.
\begin{beweis}
$a = \min (Q(\MdZ^2 \backslash 0)) = \min (Q'(\MdZ^2 \backslash 0)) = a'$.
\begin{itemize}
\item[Fall I:] $a < c \wedge U = \begin{pmatrix}r & s \\ u & v\end{pmatrix}$ mit $U.Q = Q'$. $a = Q(1,0) = Q'(1,0) = Q((1,0)U) = Q(r,s) \Rightarrow (r,s) = \pm (1,0) \Rightarrow s = 0, \pm U = \begin{pmatrix}1 & 0\\ 0(?) & 1\end{pmatrix} = U$.\\
$Q' = (a,b + 2au, \ast (?)), |b| \le a, Q' \text{ red}$. $|b'| = |b + 2au| < a$. Wegen $|b| < a \Rightarrow U = 0, \pm U = \begin{pmatrix}1 & \cdot\\ \cdot & 1\end{pmatrix} \Rightarrow Q = Q'$
\item[Fall II:] $a = c, |b| \not= a$. $\# Q^{-1}(a) = 4 \Rightarrow$ II liegt auch für $Q'$ vor $\Rightarrow a = a' = c' \Rightarrow b^2 = b^{'2} \Rightarrow b' = \pm b$, aber nur $b$ möglich, da $Q'$ vollständig reduziert $\Rightarrow Q' = Q$.
\item[Fall III:] $a = c = |b| = b \Rightarrow$ Fall II auch für $Q'$ $\Rightarrow a = a' = c' = b'$
\end{itemize}
\end{beweis}

\begin{satz}[Hauptsatz über definite QFen]
Sei $\Delta \in \MdZ, \Delta \equiv 0,1 \mod 4, \Delta < 0$.
\begin{itemize}
\item[(i)] Zwei Formen $Q, Q'$ mit Diskriminante $\Delta$ sind nau dann eigentlich äquivalent, wenn $\text{red }(Q) = \text{ red }(Q')$ (mit VollredAlgo $\text{red}$)
\item[(ii)] Die vollreden Formen der Diskriminanten $\Delta$ bilden ein volles Vertretersystem aller eigentlichen Formenklassen, insbesondere ist die Klasse zu $U$ $h(\Delta)$ endlich.
\end{itemize}
\end{satz}

\begin{beweis}
\begin{itemize}
\item [(i)] $\exists U, U'$ mit $\text{red }Q = U.Q, \text{ red }Q' = U'.Q' (U, U' \in Sl_2(\MdZ))$ können in $\text{ red }$ berechnet werden. Multipliziere die Matrizen bei den Reduktionsschritten, $Q \approx \text{ red }Q, Q' \approx \text{ red }Q'$. $Q \approx Q' \Leftrightarrow \text{ red }Q \approx \text{ red }Q' \stackrel{\text{Folgerung}}{\Leftrightarrow} \text{ red}(Q) = \text{ red}(Q')$.
\item[(ii)] $Q$ reduziert $\Leftrightarrow |b| \le a \le c \Rightarrow b^2 \le ac \Rightarrow |\Delta| = -\Delta = -b^2 + 4ac \ge -b^2 + 4b^2 = 3b^2$. Abschätung: $|b| \le \sqrt{\frac{|\Delta|}{3}} \Rightarrow$ Nur endlich viele reduzierte $Q$s.\\
Dies ergibt Algorithmus zur Bestimmung von $h(\Delta)$: $h(\Delta) = \#$ vollreduzierten Formen zu $\Delta$. Reduzierte Form $Q = [a,b,c] \Leftrightarrow |b| \le \sqrt{\frac{|\Delta|}{3}}$, $\equiv \Delta \mod 2$, da $b^2 \equiv \Delta \mod 4$. $|b| \le a \le c \le ac = \frac{b^2 - \Delta}{4}$. Stelle alle diese $(a,b,c)$ auf, streiche die nicht vollreduzierten.
\end{itemize}
\end{beweis}

\begin{satz}[Heegner/Stark (1969)]
Für $\Delta < 0$ gilt: $h(\Delta) = 1 \Leftrightarrow \Delta \in \{-3,-4,-7,-8,-11,-12,-16,-19,-27,-28,-43,-67,-163\}$
\end{satz}
Beweis im Netz!

\begin{satz}[Siegel]
Für negative Diskriminanten $\Delta$ gilt $\lim_{|\Delta| \to \infty} h(\Delta) = \infty$
\end{satz}
($\Rightarrow$ Für jedes feste $\hat h \in \MdN$ gibt es $\infty$ viele $\Delta$ mit $h(\Delta) = \hat h$.)


Gauß definiert eine Verknüpfung (Komposition) zweier Formen $Q_1, Q_2 \Rightarrow Cl(\Delta) =$ Menge aller Formenklassen wird (endliche abelsche Gruppe \underline{"`Klassengruppe"'} genannt.\\
$\leadsto$ viele Vermutungen, wenige Sätze bis heute Gaußsche Geschlechtertheorie ersetzt $h(\Delta) = 1$ durch etwas schwächere Bedingung.

\section{Reduktion indefiniter Formen}

Vor: $Q = [a,b,c], \Delta = b^2 - 4ac > 0, \sqrt{\Delta} \not\in \MdQ$ ($\Delta$ kein Quadrat in $\MdZ$) [aber $a,c \not= 0$]\\
Ärger: Theorie viel komplizierter als bei $\Delta < 0$
\begin{definition}
\begin{itemize}
\item[(i)] $Q$ heißt \underline{halbreduziert} $\Leftrightarrow \sqrt{\Delta} - |2a| < b < \sqrt{\Delta}$
\item[(ii)] $Q$ heißt \underline{reduziert} $\Leftrightarrow 0 < b < \sqrt{\Delta} \wedge \sqrt{\Delta} - b < |2a| < \sqrt{\Delta} + b$
\end{itemize}
\end{definition}

\begin{satz}[Reduktionsungleichungen]
Für eine reduzierte Form $Q = [a,b,c]$ gilt:
\begin{itemize}
\item[] $ac < 0$
\item[] $0 \stackrel{(1)}{<} b \stackrel{(2)}{<} \sqrt{\Delta}$
\item[] $\sqrt{\Delta} - b \stackrel{(3)}{<} |2a| \stackrel{(5)}{<} \sqrt{\Delta} + b$
\item[] $\sqrt{\Delta} - b \stackrel{(4)}{<} |2c| \stackrel{(6)}{<} \sqrt{\Delta} + b$
\end{itemize}
$Q$ ist genau dann reduziert, wenn $(2), (3), (4)$ gelten.
\end{satz}

\begin{beweis}
Abschätzen $\leadsto$ Netz
\end{beweis}

\begin{folgerung}[Reduktionskriterium]
Sei $Q$ halbreduziert. Dann ist $Q$ reduziert, wenn eine der folgenden Ungleichungen gilt:
\begin{itemize}
\item[(i)] $|a| \le |c|$
\item[(ii)] $\sqrt{\Delta} - b < |2c|$
\end{itemize}
\end{folgerung}

\begin{beweis}
$(2), (3)$ ok bei halbreduzierten Formen
\begin{itemize}
\item[(ii)] fordert $(4)$
\item[(i)] Bei $|a| \le |c|: (3) \Rightarrow (4)$
\end{itemize}
\end{beweis}

\begin{bemerkung}
Zu $Q = [a,b,c] \exists ! t \in \MdZ$ mit $Q' = \begin{pmatrix}\cdot & 1 \\ -1 & t\end{pmatrix}.Q$ halbreduziert, denn $Q' = [\underbrace{c}_{=a'}, \underbrace{-b+2ct}_{=b'}, ct^2 - bt + c]$.\\
Zu erreichen. $\sqrt{\Delta} - \underbrace{|2a'|}_{|2c|} < b' < \sqrt{\Delta} \exists ! t$, so dass das stimmt.
\end{bemerkung}

Benennungen:
\begin{itemize}
\item[(i)] $Q' = [a',b',c']$ heißt \underline{rechter} (linker) \underline{Nachbar} von $Q = [a,b,c]$, wenn gilt: $b + b' \equiv 0 \mod 2c$ und $a' = c$ $(a = c')$ und $Q'$ halbreduziert.
\item[(ii)] $T =: T_Q$ aus Bew (oder Bem?) heiße \underline{Nachbarmatrix} (also $Q' = T_Q.Q$)
\end{itemize}

Leicht zu sehen: Jede QF hat je genau einen reuzierten rechten bzw. linken Nachbarn.

Reduktionsalgorithmus:\\
Wiederhole das Bilden des rechten Nachbars so lange, bis reduzierte Form erreicht ist.\\
Wieso terminiert? Ist $Q' = [c, -b+2ct, c']$ nicht-reduziert, so muss $(i)$ im Reduktionskriteriumg nicht vorliegen, d.h. $|a'| = |c| > |c'|$ (für $Q'$). Der Koeffizient $|c|$ kann nicht unendlich oft verkleinert werden.

\begin{satz}[Nachbarreduktionssatz]
\begin{itemize}
\item[(i)] Ist $Q = [a,b,c]$ reduziert, so ist auch der rechte Nachbar $Q'$ von $Q$ reduziert und es ist $\text{sign}(a) = -\text{sign}(a')$
\item[(ii)] Es gibt nur endlich viele reduzierte Formen.
\end{itemize}
\end{satz}

\begin{beweis}
\begin{itemize}
\item[(i)] Abschätzen $\leadsto$ mühsam
\item[(ii)] Klar. Nur endlich viele $b$ zu $\Delta$. Nur endlich viele $a,c$ laut Ungleichungen zu $B \Rightarrow$ Algorithmus zur Aufstellung aller reduzierten Formen.
\end{itemize}
\end{beweis}

$\Delta = -1$ bzw $\Delta = -4m, m \in \MdN, qf, 2 \nmid m$. Dann: Formen zu $\Delta$ stellen $p \in \MdP$ dar mit $p \mid m$ kann zur Faktorisierung von $m$ ausgenutzt werden. Hierzu schneller, hochgezüchteter Algorithmus von Shanks:
\begin{itemize}
\item[WH:] $Q$ indefinit, $\Delta > 0, \sqrt{\Delta} \not\in \MdQ$
\item[1.] $Q = [a,b,c]$ halbreduziert $\Leftrightarrow 0 < b < \sqrt{\Delta}, \sqrt{\Delta} - b < |2a| < \sqrt{\Delta} + b$. Rechter (halbreduzierter) Nachbar von $Q$ ist $Q' = [a',b',c'], Q' = \begin{pmatrix}\cdot & 1\\-1 & t\end{pmatrix}.Q, t$ mit $\sqrt{\Delta} - |2c| < -bt2ct < \sqrt{\Delta}$. Also $t = \text{sign}(c)\cdot \lfloor \frac{\sqrt{\Delta} + b}{|2c|} \rfloor$.
\end{itemize}
Algorithmus: Wiederholtes Nachbarbilden ergibt (irgendwann) reduzierte Form. 

Sei $Q = Q_0$ reduziert.$Q_{j+1} = Q'_j (j \ge 0)$. Da es nur endlich viele reduzierte Formen gibt, muss vorkommen: $\exists k,l \in \MdN, l > 0$ mit $Q_k = Q_{k+l}$. \\
Der reduzierte linke Nachbar ist $Q_{k-1} = Q_{kl-1}$ (da eindeutig bestimmt, usw gibt $Q_0 = Q_l$ (mit $l > 0$)). Ist hier $l$ minimal, so $2 \mid l$ (wegen $\text{sign}(a') = -\text{sign}(a)$)), und $Q_0, ..., Q_{l-1}$ sind alle verschieden.

Benennung:\\
$\zeta(Q) = [Q_0, Q_1,...,Q_{l-1}]$ heißt \underline{Zyklus von $Q$} ($Q$ reduziert)

Klar: Die Menge der reduzierten Formen zerfällt disjunkt in Zyklen.

\begin{satz}[Satz von Mertens]
Sei $U \in Sl_2(\MdZ), U \not= \pm 1_2$. Die Formen $Q$ und $\tilde Q := U.Q$ seien reduziert. Dann ist eine der Matrizen $\pm U, \pm U^{-1}$ ein Produkt von Nachbarmatrizen aufeinanderfolgender rechter Nachbarn. Insbesondere sind $Q$ und $\tilde Q$ im selben Zyklus.
\end{satz}

\begin{folgerung}
Für $2$ definite QFen $Q_1, Q_2$ sei $\Delta > 0$ usw (<- kein Quadrat) und es gilt:\\
$Q_1 \approx Q_2 \Leftrightarrow \text{red}(Q_2)$ ist im Zyklus $\zeta(\text{red}(Q_1)) \Leftrightarrow \zeta(\text{red}(Q_2)) = \zeta(\text{red}(Q_1))$.
\end{folgerung}

Klar:
\begin{itemize}
\item[1.] Es gibt einen Algorithmus, der entscheidet, ob $Q_1 \approx Q_2$ oder nicht
\item[2.] Die Zyklen entsprechen den Formklassen zu $\Delta \Rightarrow$ ist Algorithmus, der $h(\Delta)$ berechnet (stelle alle reduzierten Formen auf, berechne Zyklen!).
\end{itemize}

Zum Beweis des Satzes von Merteus: Viele mühsame Abschätzungen.\\

$U.Q = (-U).Q$, da $U = \begin{pmatrix}r & s\\u & v\end{pmatrix}, -U = \begin{pmatrix}-r & -s\\-u & -v\end{pmatrix}, 1 = \text{det}U =rv - us$. $U^{-1} = \begin{pmatrix}v & -s\\-u & r\end{pmatrix}, -U^{-1} = \begin{pmatrix}-v & s\\u & -r\end{pmatrix}$.

Die richtige Wahl entscheidet sich für passende positive Vorzeichen.\\
Ohne Einschränkung $r > 0, v > 0$, setzte $U' = UT_Q^{-1} = \begin{pmatrix}r' & s'\\u' & v'\end{pmatrix}$. Man zeigt: $IU, IU^{-1}$ keine Nachbarmatrix $\not= \pm 1 \Rightarrow 0 < r' < r$\\
Induktionshypothese für $U', Q' \Rightarrow$ Behauptung.

Über $h(\Delta)$ und Struktur der Klassengruppe bei $\Delta > 0$ "`fast"' keine allgemeine Sätze bekannt. Unbekannt z.B: existieren unendlich viele $\Delta$ mit $h(\Delta) = 1$?

\section{Automorphismengruppen}

\begin{definition}
\begin{itemize}
\item[(i)] $U \in Sl_2(\MdZ)$ heißt \underline{eigentlicher Automorphismus} der QF $Q = [a,b,c] :\Leftrightarrow U.Q = Q$.
\item[(ii)] $Aut_+(Q) = \{U \in Sl_2(\MdZ): U.Q = Q\}$ (ist UGR von $Sl_2(\MdZ) \leadsto$ Untergruppenkriterium) heißt \underline{eigentliche Automorphismengruppe} von $Q$.
\end{itemize}
\end{definition}

\begin{beweis}
\begin{itemize}
\item[(i)] $\Delta > 0 \Rightarrow \text{Aut}_+(Q)$ abelsch und $\#\text{Aut}(Q) = \infty. Q(\Delta) = k, U \in \text{Aut}_+(Q) \Rightarrow k = U.Q(\underline x) = Q(\underline xU)$. Mit $\underline x$ stellt auch $\underline xU$ die Zahl $k$ dar $\Rightarrow$ existieren unendlich viele $\underline y \in \MdZ^2: Q(\underline y) = k$.\\
Man kann zeigen: Es gibt $\underline x_1, ... \underline x_l, l \in \MdN_+$, so dass $\{\underline x \big| Q(\underline x) = k\} = \underline x_1G \dot \cup .. \dot \cup \underline x_lG$ mit $G = \text{Aut}_+(Q)$ (falls $k$ überhaupt darstellbar)
\end{itemize}
\end{beweis}

\begin{definition}
$[Q_0, ..., Q_{2l-1}] = \zeta(Q), Q = Q_0$ reduziert. Die Matrix $-T_Q, T_Q =: R$ heißt \underline{Doppelnachbarmatrix} zu $Q$ ($Q'$ rechter Nachbar). $B: R_{2l-2} \cdot ... \cdot R_2 \dot R_0$ heißt \underline{Grundmatrix} zu $Q$.
\end{definition}

Klar nach Definition: $B.Q = Q$, d.h. $B \in \text{Aut}_+(Q)$. Betrachte $V \in \text{Aut}_+(Q)$, so $\pm V, \pm V^{-1}$ (eines davon) nach Satz von Mertes ein Produkt von Nachbarmatrizen.

$\Rightarrow$ Eine dieser Matrizen ist Potenz von $B$! [würde sonst irgendwo mitten im Zyklus stehenbleiben]

\begin{satz}
$\text{Aut}_+(Q) = \{\pm B^m \big| m \in \MdZ\}$ ist sogar abelsch.
\end{satz}

Wieso unendlich? Man zeigt leichct: $R$ hat alle Koeffizienten $> 0 \Rightarrow B$ auch $\Rightarrow$ Alle Matrizen $\pm B^m$ sind verschieden.

Es gibt auch Aussagen für nicht-reduziertes $Q$. Ist $Q' = V.Q, V \in Sl_2(\MdZ)$, so ist die Abbildung $\phi: \text{Aut}_+(Q) \to \text{Aut}_+(Q'), U \mapsto VUV^{-1} =: \phi(U)$ ein Isomorphismus von Gruppen.

Moderne Theorie: Theorie der QFen zu $\Delta$ weitgehend äquivalent zur algZT in quadratischem "`Zahlkörper"' $K = Q(\sqrt{\Delta})$. Norm $n(a + b\sqrt{\Delta}) = (a + b\sqrt{\Delta})(a-b\sqrt{\Delta}) = a^2 - b^2 \Delta$ ist QF für a,b.


\end{document}
