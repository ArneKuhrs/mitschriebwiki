\documentclass{article}
\newcounter{chapter}
\setcounter{chapter}{12}
\usepackage{ana}

\title{Der Existenzsatz von Peano}
\author{Christian Schulz, Ferdinand Szekeresch}
% Wer nennenswerte �nderungen macht, schreibt sich bei \author dazu

\begin{document}
\maketitle

\begin{definition}
Sei $D \subseteq \MdR^2, f:D \to \MdR$ eine Funktion und $(x_0, y_0) \in D$ und $I \subseteq \MdR$ ein Intervall. Die Gleichung:
\begin{liste}
\item[$(i)$] $\ds y(x) = y_0 + \int_{x_0}^{x} f(t,y(t)) dt \quad (x \in I)$
\end{liste} 
\end{definition} 
hei�t eine \begriff{Integralgleichung}. $y \in C(I)$ hei�t eine \begriff{L�sung von $(i)$ auf $I$} $:\equizu$ $(t,y(t)) \in D$ $\forall t \in I$ 
und es gilt $(i)$ $\forall x \in I$. \\
Wir betrachten auch noch das AWP 
\begin{liste}
\item[$(ii)$] $\begin{cases} y' = f(x,y) \\ y(x_0) = y_0 \end{cases}$
\end{liste} 

\begin{satz}[Zusammenhang Integral- und Differenzialgleichung]
$D,f,(x_0,y_0)$ und $I$ seien wie oben und $y \in C(I)$. Es sei $f \in C(D,\MdR) $.
\begin{liste}
\item[(1)] y ist eine L�sung von $(i)$ auf $I$ $\equizu$ y ist eine L�sung von $(ii)$ auf $I$
\item[(2)] Sei $I=[a,b]$ und $D= I \times R$. Ist $T: C(I) \to C(I)$ def. durch $(T_y)(x) := y_0+\int_{x_0}^{x} f(t,y(t)) dt$, $x \in I$, so gilt:
y ist eine L�sung von $(ii)$ auf $I$ $\equizu$ $T_y = y$
\end{liste} 
\end{satz}

\begin{beweis}
\begin{liste}
\item[(1)] "$\folgt$": $y(x_0) = y_0$; Durch Differentation: $y'(x) = f(x,y(x))$ $\forall x \in I$ \\
"$\Leftarrow$": $y'(x) = f(t,y(t))$ $\forall t \in I$ und $y(x_0) = y_0$ $\folgt \int_{x_0}^{x} f(t,y(t)) dt = \int_{x_0}^{x} y'(t) dt = y(x)-y(x_0) = y(x) - y_0$ $\forall x \in I$ 
\item[(2)] $T_y = y \equizu y$ l�st $(i)$ auf $I$ $\equizu$ y l�st $(ii)$ auf $I$.
\end{liste} 
\end{beweis}

\begin{satz}[L�sungen auf Teilintervallen]
Sei $D \subseteq \MdR^2, f: D \to \MdR$ eine Funktion und $\Gamma \neq \emptyset$ ($\Gamma$ ist Indexmenge). 
F�r jedes $\gamma \in \Gamma$ sei $y_{\gamma}: I_{\gamma} \to \MdR$ (wobei $I_{\gamma} \subseteq \MdR$ ein Intervall) eine L�sung der Dgl.: $$(+)\text{ } y'(x) = f(x,y)$$ 
auf $I_{\gamma}$.\\ 
Weiter sei $\bigcap_{\gamma \in \Gamma} I_{\gamma} \neq \emptyset$ und f�r je zwei L�sungen $y_{\gamma_1}: I_{\gamma_1} \to \MdR$, $y_{\gamma_2}: I_{\gamma_2} \to \MdR$ von $(+)$ gelte 
$y_{\gamma_1} = y_{\gamma_2}$ auf $I_{\gamma_1} \cap I_{\gamma_2}$.\\
Setzt man $I := \bigcup_{\gamma \in \Gamma} I_{\gamma}$ und $y(x) := y_{\gamma}(x)$, falls $x \in I_{\gamma}$, so ist $I$ ein Intervall und y eine L�sung von $(+)$
auf $I$. 
\end{satz}

\begin{beweis}
�bung.
\end{beweis}

\begin{folgerung}
Sei $I = [a,b], S:= I \times \MdR, f:S \to \MdR$ eine Funktion, $x_0 \in (a,b), y_0 \in \MdR, I_1 := [a,x_0], I_2 := [x_0,b]$ und 
$y_1: I_1 \to \MdR, y_2: I_2 \to \MdR$ seien L�sungen des AWPs \\
$$\begin{cases} y' = f(x,y) \\ y(x_0) = y_0 \end{cases}$$ \\ auf $I_1$ bzw $I_2$. Definiert man $y:I \to \MdR$ durch \\
$$y(x):=
\begin{cases}
y_1(x), & \text{falls } x \in I_1  \\
y_2(x), & \text{falls } x \in I_2
\end{cases}$$\\
so ist $y$ eine L�sung des AWPs auf $I$.
\end{folgerung}

\begin{satz}[Der Existenzsatz von Peano (Version I)]
$I$ und $S$ seien wie in 12.3, $x_0 \in I, y_0 \in \MdR$ und $ f \in C(S,\MdR)$ sei beschr�nkt.
Dann hat das AWP: \\\ $$\begin{cases} y' = f(x,y) \\ y(x_0) = y_0 \end{cases}$$ \\\ eine L�sung auf $I$. \\
\end{satz}

Wir f�hren zwei Beweise. In beiden sei $M := \sup \{|f(x,y)| : (x,y) \in S\}$ und $T: C(I) \to C(I)$ sei definiert durch
$(T_y)(x) := y_0 + \int_{x_0}^{x}f(t,y(t))$ ($x \in I$)

\begin{beweis}[mit 11.3]
Sei $A \subseteq C(I)$ sei wie in 11.5 (mit obigen M). 11.5 $\folgt A \neq \emptyset$, $A$ ist konvex und kompakt. $T:A \to C(I)$ ist stetig. 
Wegen 11.3 und 12.1(2) ist nur noch zu zeigen: $T(A) \subseteq A$. Sei $y \in A$. Dann $(T_y)(x_0) = y_0$. Weiter gilt \\ 
$\forall x,\overline{x} \in I : | (T_y)(x) - (T_y)(\overline{x}) | = | \int_{x}^{\overline{x}} \underbrace{f(t,y(t))}_{\leq M} dt | \leq M \cdot |x-\overline{x}|$. 
Also: $T_y \in A.$ Somit: $T(A) \subseteq A$
\end{beweis}

\begin{beweis}[Nr.2]
Wir unterscheiden 3. F�lle: $x_0 = a, x_0 = b$ und $x_0 \in (a,b)$. Wir f�hren den Beweis nur f�r den Fall $x_0 = a$ (den Fall $x_0 = b$ zeigt man analog; 
der Fall $x_0 \in (a,b) $ folgt aus 12.3 und den ersten beiden F�llen). \\ Sei also $x_0 = a$. o.B.d.A. $x_0+\frac{1}{n}=a+\frac{1}{n} \in I$ $\forall n \in I$. \\
F�r $n \in \MdN$ definieren wir $z_n : (-\infty,b] \to \MdR$ durch 
$$z_n(x):=
\begin{cases}
y_0, & \text{falls } x \leq x_0 = a  \\
y_0 +\int_{x_0}^{x}f(t,z_n(t-\frac{1}{n}) dt, & \text{falls } x \in I
\end{cases}$$\\\
Beh.: $z_n$ ist auf $I$ wohldefiniert.\\
Sei $x \in [x_0, x_0+\frac{1}{n}]$ und $ t \in [x_0, x]$ $\folgt t-\frac{1}{n} \leq x-\frac{1}{n} \leq x_0$
$\folgt z_n(t-\frac{1}{n}) = y_0 \folgt z_n(x) = y_0 + \int_{x_0}^{x}f(t,y_0) dt$, also $z_n(x)$ ist wohldef.\\
Sei $x \in [x_0+\frac{1}{n},x_0+\frac{2}{n}]$ und $t \in [x_0,x]$ $\folgt t-\frac{1}{n} \leq x-\frac{1}{n} \in [x_0, x_0+\frac{1}{n}]$
$\folgt z_n(t-\frac{1}{n})$ wohldef. $\folgt z_n(x)$ ist wohldefiniert, etc... \\\\
�bung: $z_n \in C(-\infty, b]$. \\\\
Insbesondere: $z_n \in C(I)$. Es ist $z_n(x_0) = y_0$. F�r $x,\overline{x} \in I: 
|z_n(x)-z_n(\overline{x})| = | \int_{x}^{\overline{x}}f(t,z_n(t-\frac{1}{n})) dt | \leq M \cdot |x-\overline{x}|.$ 
11.4 $\folgt (z_n)$ enth�lt eine auf $I$ gleichm��ige konvergente Teilfolge. o.B.d.A.: $(z_n)$ konvergiert auf $I$ glm. \\
$y(x):= \lim_{n \to \infty} z_n(x)$ $(x \in I)$. AI $\folgt y \in C(I)$. Also $z_n \to y$ bzgl. $\left\|\cdot\right\|_{\infty}$. 
($\left\|z_n-y\right\|_{\infty} \to 0$ $( n \to \infty ) $); \\ $g_n(t) := z_n(t-\frac{1}{n})$ $(t \in I)$. 
$\forall t \in I : |g_n(t)-y(t)| = | g_n(t)-z_n(t)+z_n(t)-y(t) | \leq |\underbrace{z_n(t-\frac{1}{n})-z_n(t)}_{\leq \frac{M}{n}}|+|\underbrace{z_n(t)-y(t)}_{\leq \left\|z_n-y \right\|_{\infty}} |$\\
$\folgt \left\| g_n(t) - y(t) \right\|_{\infty} \leq \frac{M}{n} +\left\|z_n-y \right\|_{\infty} \forall n \in \MdN$
$\folgt g_n \to y $ bzgl. $\left\|\cdot\right\|_{\infty}$ (glm. konv.) \\
$T: C(I) \to C(I)$ ist stetig $\folgt T_{g_n} \to T_y $ bzgl. $\left\|\cdot\right\|_{\infty}$ \\
$(T_{g_n})(x) = y_0 + \int_{x_0}^{x}f(t,z_n(t-\frac{1}{n})) dt = z_n(x) \forall x \in I$ 
$\folgt T_{g_n}=z_n$ auf $I$. \\ Also $T_y = y$ und damit folgt, $y$ l�st das AWP auf $I$.
\end{beweis}

\begin{satz}[Der Existenzsatz von Peano (Version II)]
Es sei $I = [a,b] \subseteq \MdR, x_0 \in I, y_0 \in \MdR, s > 0$ und $R:=I \times [y_0-s, y_0+s]$ \\
Es sei $f \in C(R,\MdR), M := \max \{ |f(x,y)| : (x,y) \in R \}$ und \\
$J := I \cap [x_0-\frac{s}{M},x_0+\frac{s}{M}].$ Dann hat das AWP:
\\\ $$\begin{cases} y' = f(x,y) \\ y(x_0) = y_0 \end{cases}$$ \\\ eine L�sung auf $J$.
\end{satz}

\begin{beweis}
$S := I \times \MdR.$ Def. $g: S \rightarrow \MdR$ durch: \\
\ $$ g(x,y) = \begin{cases} f(x,y), & (x,y) \in \MdR \\ f\left(x, y_0 + s \frac {y - y_0}{|y - y_0|}\right), & x \in I, |y - y_0| \geq s\end{cases}$$ \\
Dann: $g = f$ auf $R$ , $|g| \leq M$ auf $S$ und $g \in C(S,\MdR)$ \\
Betrachte das AWP \\ $$(+)\begin{cases} y'= g(x,y) \\ y(x_0) = y_0 \end{cases}$$ \\
12.4 $\folgt (+)$ hat eine L�sung $\overline{y}$ auf $I$. 12.1 $\folgt$ \\
$$ (*) \; \overline{y}(x) = y_0 + \int_{x_0}^{x}g(t,\overline{y}(t)) dt \; \forall x \in I $$
Sei $x \in J.$ Sei $y := \overline{y}|_J.$ Dann: $|y(x) - y_0| = |\overline{y}(x) - y_0|$ \\
$\stackrel{(*)}{=} | \int_{x_0}^{x} g(t,\overline{y}(t)) dt | \leq M |x - x_0| \leq M \cdot \frac{s}{M} = s 
\folgt (x,y(x)) \in R$ \\
$\folgt (t,y(t)) \in R$ f�r $t$ zwischen $x$ und $x_0$. \\
$\folgt y(x) = y_0 + \int_{x_0}^{x} f(t,g(t)) dt \; \forall x \in J$ \\
$\folgtnach{12.1} y$ l�st das AWP auf $J$
\end{beweis}

\begin{satz}[Der Existenzsatz von Peano (Version III)]
Sei $D \in \MdR^{2}$ offen, $(x_0,y_0) \in D$ und $f \in C(D, \MdR)$. Dann ex. $\delta > 0$ : das AWP
$$\begin{cases} y' = f(x,y) \\ y(x_0) = y_0 \end{cases}$$
hat eine L�sung $y: K \rightarrow \MdR$, wobei $K = [x_0 - \delta, x_0 + \delta]$ (also $x_0 \in K^{0}$)
\end{satz} 

\begin{beweis}
$D$ offen $\folgt \exists \; r, s > 0 : R := [x_0 - r, x_0 + r] \times [y_0 - s, y_0 + s] \subseteq D \\
M := \max \{|f(x,y)| : (x,y) \in \MdR \} \\
\delta := \max \{r, \frac {s}{M}\}, K := [x_0 - \delta, x_0 + \delta]$ 12.5 $\folgt$ Beh.
\end{beweis}


\end{document}
