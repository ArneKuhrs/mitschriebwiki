\documentclass{scrartcl}
\usepackage{german}
\usepackage[utf8]{inputenc} 

\usepackage[arrow, matrix, curve]{xy}

\title{Insel der Zahlen}
\author{Ali und Rachid}
\date{die Epoche}

\begin{document}
%% Titel und Inhaltsverzeichnis des Artikels hier
%% ausgeben

\maketitle

Klappt das? Auch \LaTeX{}? Anscheined ja. :-)

Scheiße, alle Formatierungsbefehle vergessen :-(
\newline
Wie ging das nochmal mit 'ner Gleichung?

Warum darf ich denn in einer displaymath-Umgebung keine Leerzeilen haben? ein Bug?

Aus l2kurz.tex (z.B. vom Dante-Server):

\begin{displaymath}
\sum_{i=1}^{n} \qquad
\int_{0}^{\frac{\pi}{2}} \qquad
\int \limits_{-\infty}^{+\infty}\\
\end{displaymath}

\begin{displaymath}
\textrm{Ok, meine Stunde}\ldots \texttt{:-)} \qquad \  \qquad
\int \limits_{a}^{b} f(x) d x = F(b) - F(a)
\end{displaymath}

Gibt's denn im WS ein gleiches Projekt? Kollaborative Skipterstellung?? Eine sch"one Idee. Hat denn "uberhaupt jemand mitgearbeitet am Skript?

$ \varnothing $
BTW: Koma-Script rulez \texttt{:-)}

\begin{xy}
  \xymatrix{
      A \ar[r]^f \ar[d]_i    &   B \ar[d]^j  \\
      C \ar[r]_g             &   D   
  }
\end{xy}

SVN-Testtest

\end{document}
