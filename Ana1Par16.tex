\documentclass{article} 
\newcounter{chapter}
\setcounter{chapter}{16}
\usepackage{ana}


\author{Joachim Breitner}
\title{Grenzwerte bei Funktionen}

\begin{document}
\maketitle

\begin{definition}[H�ufungspunkt]
Sei $D\subseteq\MdR$ und $x_0 \in\MdR$. $x_0$ hei�t ein \begriff{H�ufungspunkt} (HP) von $D :\equizu \exists$ Folge $x_n$ in $D\backslash \{x_0\}$ mit $x_n \to x_0$.
\end{definition}

\begin{beispiele}
\item Ist $D$ endlich, so hat $D$ keine H�ufungspunkte.
\item $D = (0,1]$. $x_0$ ist H�ufungspunkt von $D \equizu x_0 \in [0,1]$.
\item $D = \{\frac{1}{n}: n\in\MdN\}$. $D$ hat genau einen H�ufungspunkt: $x_0 = 0$
\item $D = \MdQ$. 8.1(2) $\folgt$ jedes $x_0 \in\MdR$ ist ein H�ufungspunkt von $\MdQ$.
\end{beispiele}

\begin{bemerkung}
Unterscheide zwischen \glqq $x_0$ ist H�ufungswert von $(a_n)$\grqq und \glqq $x_0$ ist H�ufungspunkt von $\{a_1,a_2,\ldots\}$\grqq. Beispiel: $a_n=(-1)^n$. $\H(a_n) = \{1,-1\}, \{a_1,a_2,\ldots\} = \{-1,1\}$ hat keine H�ufungspunkte.
\end{bemerkung}

\paragraph{Zur �bung:} Sei $D\subseteq\MdR$, $x_0 \in\MdR$. $x_0$ ist H�ufungspunkt von $D \equizu \forall\ep>0$ gilt: $D\cap(U_\ep(x_0) \backslash\{x_0\}) \ne \emptyset$

\begin{vereinbarung}
Ab jetzt sei in dem Paragraphen gegeben: $\emptyset \ne D \subseteq \MdR$. $x_0$ ist H�ufungspunkt von $D$ und $f: D \to \MdR$ eine Funktion.
\end{vereinbarung}

\begin{definition}
$\displaystyle\lim_{x\to x_0}f(x)$ exisitiert $:\equizu \exists a\in\MdR$ mit: f�r jede Folge $(x_n)$ in $D\backslash\{x_o\}$ mit $x_n \to x_0$ gilt: $f(x_n) \to a$. In diesem Fall schreibt man: $\displaystyle\lim_{x\to x_0}f(x) = a$ oder $f(x) \to a \ (x\to x_0)$
\end{definition}

\begin{bemerkung}
\begin{liste}
\item Existiert $\displaystyle\lim_{x\to x_0}f(x)$, so ist obiges $a$ eindeutig bestimmt. (�bung)
\item Falls $x_0 \in D$, so ist der Wert $f(x_0)$ in obiger Definition nicht relevant.
\end{liste}
\end{bemerkung}

\begin{beispiele}
\item $D= (0,1]$. $$f(x) = \begin{cases} x^2 & \text{falls }x\in(0,\frac{1}{2}) \\ \frac{1}{2} & \text{falls }x=\frac{1}{2} \\ 1 & \text{falls } x \in(\frac{1}{2},1) \\ 0 & \text{falls } x= 1\end{cases}$$

$x_0=0$: Sei $(x_n)$ eine Folge in $D$ mit $x_n \to 0$. Dann $x_n < \frac{1}{2} \ffa n\in\MdN \folgt f(x_n) = x^2 \ffa n\in\MdN \folgt f(x_n) \to 0$, d.h. $\displaystyle\lim_{x\to0}f(x) = 0$.

$x_0 = 1$: Analog: $\displaystyle\lim_{x\to0}f(x) = 1$.

$x_0= \frac{1}{2}$: Sei $(x_n)$ eine Folge in $D\backslash\{\frac{1}{2}\}$ und $x_n < \frac{1}{2} \ \forall n\in\MdN \folgt f(x_n) = x_n^2 \to \frac{1}{4}.$
Sei $(z_n)$ eine Folge in $D\backslash\{\frac{1}{2}\}$ und $z_n > \frac{1}{2} \ \forall n\in\MdN \folgt f(z_n) = z_n^2 \to 1$
d.h.: $\displaystyle\lim_{x\to\frac{1}{2}}f(x)$ existiert nicht. Aber: $\displaystyle\mathop{\lim_{x\to\frac{1}{2}}}_{x\in(0,\frac{1}{2})} f(x)$ existiert und ist $\frac{1}{4}$ und $\displaystyle\mathop{\lim_{x\to\frac{1}{2}}}_{x\in(\frac{1}{2},1)} f(x)$ existiert und ist $1$. Daf�r schreibt man: $\displaystyle\lim_{x\to\frac{1}{2}-} f(x)=\frac{1}{4}$ und $\displaystyle\lim_{x\to\frac{1}{2}+} f(x)=1$.

\item $D=[0,\infty)$, $p\in\MdN$, $f(x) = \sqrt[p]{x}$. Sei $x_0 \in D$. Sei $(x_n)$ Folge in $D\backslash\{x_0\}$ mit $x_n \to x_0$. 7.1 $\folgt \sqrt{p}{x_n} = \sqrt[p]{x_0}$. Das hei�t: $\displaystyle\lim_{x\to x_0 } f(x) = f(x_0)$.
\end{beispiele}

\begin{vereinbarung}
F�r $\delta >0$: $D_\delta(x_n) = D \cap U_\delta(x_0)$. $\dot D_\delta(x_0) = D_\delta(x_0) \backslash \{x_0\}$.
\end{vereinbarung}

\begin{satz}[Grenzwerts�tze bei Funktionen]
\begin{liste}
\item $\displaystyle\lim_{x\to x_0}f(x)$ existiert $\equizu$ f�r jede Folge $(x_n)$ in $D\backslash\{x_n\}$ mit $x_n \to x_0$ ist $f(x_n)$ konvergent.
\item F�r $a\in\MdR$ gilt: $\displaystyle\lim_{x\to x_0}f(x)$ existiert und ist gleich $a$ $\equizu$ $\forall \ep>0 \ \exists \delta(\ep) > 0$ mit $(*)$ $|f(x) - a|< \ep \ \forall x \in \dot D_\delta(x_0)$. 
\item \textit{Cauchykriterium}\indexlabel{Cauchykriterium!bei Funktionsgrenzwerten}: $\displaystyle\lim_{x\to x_0} f(x)$ existiert $\equizu$ $\forall \ep>0 \ \exists \delta = f(x)>0$: $(f(x) - f(x'))<\ep \forall x,x' \in \dot D_\delta(x_0)$
\end{liste}
\end{satz}

\begin{beweise}
\item \glqq $\folgt$ \grqq: aus Definition. \\
\glqq $\Leftarrow$ \grqq: Seien $(x_n), (z_n)$ Folgen in $D\backslash\{x_0\}$ mit $x_n \to x_0$, $z_n \to x_0$. Voraussetzung $\folgt$ es existiert $a := \lim f(x_n)$ und $b := \lim f(z_n)$. Zu zeigen ist: $a=b$. Sei $t_n$ definiert durch $(t_n) := (x_1,z_1,x_2,z_2,\ldots)$. $(t_n)$ ist Folge in $D \backslash\{x_0\}$ mit $t_n\to x_0$, Voraussetzung $\folgt  \exists c := \lim f(t_n)$. $(f(x_n))$ ist Teilfolge von $(f(t_n)) \folgt a=c$, analog: $b=c \folgt a = b$.
\item \glqq $\folgt$ \grqq: Sei $\ep > 0$. \textbf{Annahme}: Es gibt kein $\delta > 0$, so dass $(*)$ gilt. Das hei�t: $\forall \delta > 0$ exisistert ein $x_j \in \dot D_\delta(x_j)$: $|f(x_j) - a| \ge \ep$, also $\forall n\in\MdN \ \exists x_n \in \dot D_{\frac{1}{n}}(x_0): |f(x_n) - a|\ge\ep$. Das hei�t: $(x_n)$ ist eine Folge in $D \backslash\{x_0\}$ mit $x_n \to x_0$ und $f(x_n) \nrightarrow a$, Widerspruch. \\
\glqq $\Leftarrow$ \grqq: Sei $x_n$ eine Folge in $D \backslash\{x_n\}$ mit $x_n \to x_0$. Zu zeigen ist: $f(x_n) \to a$. Sei $\ep>0$. $\exists \delta > 0$ so dass $(*)$ gilt. Dann: $x_n \in \dot D_\delta(x_0) \ffa n\in\MdN \folgt |f(x_n) - a| < \ep \ffa n\in\MdN$.
\item In �bung.
\end{beweise}

\begin{satz}[Rechnen mit Funktionsgrenzwerten]
Sei $g, h: D \to \MdR$ zwei weitere Funktionen und es gelte $f(x) \to a$, $g(x)\to b$ $(x\to x_0)$.
\begin{liste}
\item $f(x)+g(x) \to a+b$, $f(x)\cdot g(f) \to ab$, $|f(x)| \to |a|$ $(x\to x_0)$
\item Ist $a \ne 0 \folgt \exists \delta>0: f(x)\ne 0 \ \forall x\in\dot D_\delta(x_0)$. F�r $\frac{1}{f}: \dot D_\delta(x_0) \to \MdR$ gilt: $\frac{1}{f(x)} \to \frac{1}{a}$.
\item Existiert ein $\delta > 0 $ mit $f \le g$ auf $\dot D_\delta(x_0) \folgt a \le b$
\item Existiert ein $\delta > 0$ mit $f \le h \le g$ auf $\dot D_\delta(x_0)$ und $a = b \folgt \displaystyle\lim_{x\to x_0} h(x) = a$.
\end{liste}
\end{satz}

\begin{beweis}
folgt aus 6.2

Zum Beispiel: (3) Sei $(x_n)$ Folge in $D\backslash\{x_0\}$ und $x_n \to x_0$. Dann: $x_n \in \dot D_\delta(x_0) \ffa n\in\MdN \folgt f(x_n) \le g(x_n) \ffa n\in\MdN \folgt a = \lim f(x_n) \stackrel{\text{5.2}}{\le} \lim g(x_n) = b$.
\end{beweis}

\begin{definition}
\begin{liste}
\item Sei $(a_n)$ eine Folge in $\MdR$. $\lim a_n = \infty$ (oder $a_n \to \infty$) $:\equizu \forall c>0 \ \exists n_0 = n_0(c)\in\MdN: a_n > c \forall n\ge n_0$. $\lim a_n = -\infty$ (oder $a_n \to -\infty$) $:\equizu \forall c>0 \ \exists n_0 = n_0(c)\in\MdN: a_n < c \forall n\ge n_0$. 
\item $\displaystyle\lim_{x\to x_0} f(x) = \infty$ (oder $f(x) \to \infty\ (x\to x_0)$) $:\equizu$ f�r jede Folge $(x_n)$ in $D\backslash\{x_0\}$  und $x_n \to x_0$ gilt: $f(x_n) \to \infty$. \\
$\displaystyle\lim_{x\to x_0} f(x) = -\infty$ (oder $f(x) \to -\infty\ (x\to x_0)$) $:\equizu$ f�r jede Folge $(x_n)$ in $D\backslash\{x_0\}$  und $x_n \to x_0$ gilt: $f(x_n) \to -\infty$.
\item Sei $D$ nicht nach oben beschr�nkt. $\displaystyle\lim_{x\to \infty} f(x) = a$ (oder $f(x) \to \infty$) $:\equizu$ f�r jede Folge $(x_n)$ in $D$ mit $x_n\to \infty$ gilt: $f(x_n) \to a$ ($a = \pm\infty$ zugelassen). \\
Sei $D$ nicht nach unten beschr�nkt. $\displaystyle\lim_{x\to -\infty} f(x) = a$ (oder $f(x) \to -\infty$) $:\equizu$ f�r jede Folge $(x_n)$ in $D$ mit $x_n\to -\infty$ gilt: $f(x_n) \to a$ ($a = \pm\infty$ zugelassen). \\
\end{liste}
\end{definition}

\begin{beispiele}
\item $a_n := x^n\ (x > 1)$. Behauptung: $x^n\to \infty \ (n\to\infty)$. Sei $c>0$. $c<\frac{1}{x^n}<1 \folgt \frac{1}{x^n} \to 0 \folgt \frac{1}{x^n}<\frac{1}{c} \ffa n\in\MdN \folgt x^n > c \ffa n\in\MdN$.
\item Sei $p\in\MdN$. Dann $x^p \to \infty \ (x\to\infty)$. Siehe �bung.
\item $\frac{1}{x} \to \infty \ (x \to 0+)$, $\frac{1}{x} \to -\infty \ (x\to 0-)$.
\end{beispiele}

\begin{satz}[Grenzwerte der Exponentialfunktion]
$$e^x = \sum_{n=0}^{\infty} \frac{x^n}{n!} \ (x\in\MdR)$$
\begin{liste}
\item F�r $p\in\MdN_0: \frac{e^x}{x^p} \to \infty\ (x\to\infty)$
\item $e^x \to \infty \ (x\to\infty)$
\item $e^x \to 0 \ (x\to-\infty)$
\end{liste}
\end{satz}

\begin{beweis}
\begin{liste}
\item $e^x = 1 + x + \frac{x^2}{2!} + \cdots + \frac{x^p+1}{(p+1)!} + \cdots \ge \frac{p+1}{(p+1)!} \ \forall x \ge 0 \folgt \frac{e^x}{x^p} \ge \frac{x}{(p+1)!} \ \forall x>0 \folgt$ Behauptung.
\item Folgt aus 1 mit $p = 0$.
\item $e^{-x} = \frac{1}{e^x} \tonach{(2)} 0 \ (x \to -\infty) \folgt e^x \to 0 \ (x \to -\infty)$.
\end{liste}
\end{beweis}

\end{document}
