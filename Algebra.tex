\documentclass[12pt]{book}
\usepackage{amssymb}
\usepackage[latin9]{inputenc}
\usepackage[T1]{fontenc}
\usepackage[english]{babel}
\usepackage{geometry}
\usepackage{paralist}
\usepackage[explicit]{titlesec}
\usepackage[fleqn]{amsmath}
\usepackage{setspace}
\usepackage{polynom}
\usepackage{stmaryrd}
\usepackage[arrow, matrix, curve]{xy}
\usepackage{float}
\usepackage{titletoc}
\usepackage{latexki}
\lecturer{Prof. Dr. F. Herrlich}
\semester{Wintersemester 14/15}
\scriptstate{complete}

\titlecontents{chapter}[2em]{\addvspace{2pc}\bfseries}{\contentslabel{1.7em}}{}{\titlerule*[1.5pc]{}\contentspage}
\titlecontents{section}[4.5em]{}{\contentslabel{2.5em}}{}{\titlerule*[0.5pc]{.}\contentspage}

\setlength{\mathindent}{0pt}

\titleformat{\subsection}{\normalfont\normalsize\bfseries}{}{0em}{#1 \thesubsection}
\titleformat{\section}{\normalfont\large\bfseries}{}{0em}{\thesection  #1}
\renewcommand*\thechapter{\Roman{chapter}\quad}
\titlespacing{\chapter}{0pt}{*5}{*1.5}
\titlespacing{\section}{0pt}{*5}{20pt}
\titlespacing{\subsection}{0pt}{18pt}{0pt}
\geometry{a4paper, top=20mm, left=10mm, right=10mm, bottom=20mm, headsep=10mm, footskip=10mm}
\renewcommand{\labelenumi}{(\roman{enumi})}
\renewcommand{\labelenumii}{(\arabic{enumii})}
\setlength{\parindent}{0pt}
\newcommand{\slant}[2]{{\raisebox{.1em}{$#1$}\left/\raisebox{-.1em}{$#2$}\right.}}
\newcommand{\bigslant}[2]{{\raisebox{.2em}{$#1$}\left/\raisebox{-.2em}{$#2$}\right.}}
\usepackage{graphicx}
\newcommand\tabrotate[1]{\rotatebox{90}{#1\hspace{\tabcolsep}}}
\newcommand\verschiebung[1][-.75\normalbaselineskip]{\hspace{#1}}
\makeatletter
\renewcommand*{\env@matrix}[1][*\c@MaxMatrixCols c]{
  \hskip -\arraycolsep
  \let\@ifnextchar\new@ifnextchar
  \array{#1}}
\makeatother
\renewcommand{\chaptermark}[1]{ 
  \markboth{ 
     \MakeUppercase{\thechapter #1} 
  }{} 
} 
\renewcommand{\sectionmark}[1]{ 
  \markright{ 
     \MakeUppercase{\thesection#1} 
  } 
}

\usepackage{hyperref}

\usepackage{makeidx}
\makeindex

\begin{document}



\begin{titlepage}

\textrm{ }\\[64pt]

\begin{center}
{\fontsize{40}{40} \selectfont \textbf{Algebra}}
\end{center}
\textrm{ } \\[36pt]
\begin{center} \large{\textrm{lectured by Prof. Dr. Frank Herrlich during fall 2014/2015 at the KIT}} \end{center}
\textrm{ } \\[320pt]
\begin{center} \large{\textit{Written in } \LaTeX \textit{ by Arthur Martirosian, arthur.martirosian@student.kit.edu}}\end{center}
\textrm{ }\\[24pt]
\begin{center} \large{\today} \end{center}

\end{titlepage}
\thispagestyle{empty}



\begin{spacing}{1.7}
\setcounter{tocdepth}{1}
\tableofcontents
\thispagestyle{empty}
\end{spacing}
\newpage



\begin{spacing}{1.3}
\thispagestyle{empty}
\chapter{Galois theory} %CHAPTER I
\setlength\abovedisplayshortskip{0pt}
\setlength\belowdisplayshortskip{10pt}
\setlength\abovedisplayskip{10pt}
\setlength\belowdisplayskip{10pt}


\renewcommand*\thesection{\S\ \arabic{section}\quad}
\section{Algebraic field extensions}\index{field extension!algebraic} %SECTION 1
\renewcommand*\thesection{\arabic{section}}
\thispagestyle{empty}

\subsection{Notations} % Notations 1.1
If $\mathbb{K},\mathbb{L}$ are fields and $\mathbb{K}\subseteq\mathbb{L}$, $\mathbb{L}/\mathbb{K}$ is called a \textit{field extension}.\\
The \textit{dimension} $[\mathbb{L}:\mathbb{K}]:=\dim_{\mathbb{K}}\mathbb{L}$ of $\mathbb{L}$ considered as a $\mathbb{K}$-vector space, is called the \textit{degree} of the field extension of $\mathbb{L}$ over $\mathbb{K}$.\\
A field extension $\mathbb{L}/\mathbb{K}$ is called \textit{finite}, if $[\mathbb{L}:\mathbb{K}]<\infty$.\\
The \textit{polynomial ring} over $\mathbb{K}$ is defined as
$$\mathbb{K}[X]:=\left\{f=\sum_{i=0}^n a_i X^{i} \ \bigg \vert \ n \geqslant 0, a_i \in \mathbb{K} \textit{ } \forall i \in \{0,...,n\}, a_n \neq 0 \right\} \cup \{0\}$$

\subsection{Reminder} % Reminder 1.2
Let $\mathbb{L}/\mathbb{K}$ a field extension, $\alpha \in \mathbb{L}$, $f \in \mathbb{K}[X]$.
\begin{compactenum}
\item $f(\alpha)$ is well defined.
\item $\phi_{\alpha}: \mathbb{K}[X] \rightarrow \mathbb{L}$, $f \mapsto f(\alpha)$ is a homomorphism.
\item $\textrm{im}(\phi_{\alpha}):=\mathbb{K}[\alpha]$ is the smallest subring of $\mathbb{L}$ containing $\mathbb{K}$ and $\alpha$.
\item $\ker(\phi_{\alpha})=\left\{f \in \mathbb{K}[\alpha] \ \big \vert \ f(\alpha)=0\right\} \triangleleft \mathbb{K}[X]$ is a prime ideal.
\item $\ker(\phi_{\alpha})$ is a principle ideal.
\item If $f_{\alpha}\neq 0$ and the leading coefficient of $f_{\alpha}$ is $1$, $f_{\alpha}$ is called the \textit{minimal polynomial} of $\alpha$, i.e. $f_{\alpha}(\alpha)=0$ and $f_{\alpha}$ is the polynomial of smallest degree with this property. In this case, $f_{\alpha}$ is irreducible and $\ker(\phi_{\alpha})=\langle f_{\alpha} \rangle$ is a maximal ideal.
\item Then $L_{\alpha}:=\slant{\mathbb{K}[X]}{\ker(\phi_{\alpha})}=\slant{\mathbb{K}[X]}{\langle f_{\alpha}\rangle}$ is a field.
\item We have $\mathbb{K}[\alpha]=\textrm{im}(\phi_{\alpha}) \cong \slant{\mathbb{K}[X]}{\ker(\phi_{\alpha})}=\mathbb{L}_{\alpha}$, if $f_{\alpha}\neq 0$.
Moreover $\mathbb{K}[\alpha]=\mathbb{K}(\alpha)$, where $\mathbb{K}(\alpha)$ is the smallest field containing $\mathbb{K}$ and $\alpha$. In particular, $\frac{1}{\alpha} \in \mathbb{K}[\alpha]$.
\item The degree of the field extension $\mathbb{K}[\alpha]/\mathbb{K}$ is $[\mathbb{K}[\alpha]:\mathbb{K}]=\deg(f_{\alpha})$. 
\end{compactenum}
\pagebreak
\textit{proof.}
\begin{compactenum}
\item[(ii)]  For $f,f_1,f_2 \in \mathbb{K}[X]$, $\lambda \in \mathbb{K}$ we have
$$(f_1+f_2)(\alpha)=f_1(\alpha)+f_2(\alpha)\rm{ and }(\lambda f)(\alpha)=\lambda f(\alpha)$$
\item[(iii)] Clear.
\item[(iv)] Let $f,g \in \mathbb{K}[X]$ such that $f\cdot g$, $\in \ker(\phi_{\alpha})$: Then
$$0=(f\cdot g)(\alpha)=f(\alpha) \cdot g(\alpha)$$
and since $\mathbb{L}$ has no zero divisors, $f(\alpha)=0$ or $g(\alpha)=0$ and hence $f \in \ker(\phi_{\alpha})$ or $g \in \ker(\phi_{\alpha})$
\item[(v)] Remember that the polynomial ring is euclidean. Take $f_{\alpha} \in \ker(\phi_{\alpha})$ of minimal degree. We will show, that $\ker(\phi_{\alpha})$ is generated by $f_{\alpha}$. Let $g \in \ker(\phi_{\alpha})$ arbitrary and write $$g=q \cdot f_{\alpha} + r\textrm{ with }q,r \in \mathbb{K}[X], \textrm{ }\deg(r)<\deg(f_{\alpha})\textrm{ or }r=0.$$Since $r=q\cdot f_{\alpha} \in \ker(\phi_{\alpha})$ and the choice of $f_{\alpha}$, $\deg(r) \nless \deg(f_{\alpha})$, hence $r=0$ $\Rightarrow g \in \langle f_{\alpha} \rangle$.
\item[(vi)] If $f_{\alpha}=g \cdot h$, either $g(\alpha)=0$ or $h(\alpha)=0$. As above, this implies $g \in \mathbb{K}$ or $h \in \mathbb{K}^{\times}$, i.e. $f$ or $g$ is irreducible.\\
Now assume, there is and ideal $I \trianglelefteqslant \mathbb{K}[X]$ satisfying $\langle f_{\alpha}\rangle \subsetneq I \subsetneq \mathbb{K}[K]$.\\
Let $g \in  I \setminus \langle f_{\alpha}\rangle$, such that $\langle g \rangle = I $. Such a $g$ exists by proof of (v). Then $f_{\alpha}=g \cdot h$, $h \in \mathbb{K}[X]$. This implies, that either $g$ or $h$ is a constant polynomial, hence a unit. In the first case, $I=\mathbb{K}[X]$ and in the second one $I=\langle f_{\alpha}\rangle$, which implies the claim.
\item[(vii)] We show the more general argument: If $R$ is a ring, $\mathfrak{m} \triangleleft R$ a maximal ideal, then $\slant{R}{\mathfrak{m}}$ is a field. Let $\overline{a} \in \slant{R}{\mathfrak{m}}$ for some $a \in R$, $\overline{a}\neq 0$. Let $I:=\langle \mathfrak{m}, a\rangle$ the smallest ideal in $R$ containing $\mathfrak{m}$ and $a$. Since $\overline{a} \neq 0$, hence $a \notin \mathfrak{m}$ we have $\mathfrak{m} \subsetneq I$ and since $\mathfrak{m}$ is a maximal ideal, $I=R$. Hence $1 \in I$, so we can write $1=x+ab$ for some $x \in \mathfrak{m}$ and $b \in R$. Then we get\\
$\overline{1}=\overline{x+ab}=\overline{x}+\overline{a}\overline{b}=\overline{a}\overline{b}$, hence $\overline{a}$ is invertible in $\slant{R}{\mathfrak{m}}$.
\item[(viii)] Let $$f_{\alpha}=\sum_{i=0}^n a_i X^{i}$$
Note, that $a_n=1$ and $a_0 \neq 0$, since $f_{\alpha}$ is irreducible. We get\\[-27pt]
\begin{alignat*}{5}
&\Longrightarrow \textrm{   } &0&\ =\ f_{\alpha}(\alpha)=\sum_{i=0}^n a_i \alpha^{i}=a_0 + a_1 \alpha + \dots + a_n \alpha^n\\
&\Longrightarrow \textrm{   } &a_0&\ =\ - \alpha \cdot \left(a_1+a_2 \alpha+ \dots + a_{n-2}\alpha^{n-2}+\alpha^{n-1}\right)\\
&\Longrightarrow \textrm{   } &1&\ =\ -\alpha \cdot \left(\frac{a_1}{a_0}+\frac{a_2}{a_0}\alpha+ \dots + \frac{a_{n-2}}{a_0} \alpha^{n-2}+\frac{1}{a_0} \alpha{n-1}\right)\\
&\Longrightarrow \textrm{   } &\frac{1}{\alpha}&\ =\ -\frac{a_1}{a_0}-\frac{a_2}{a_0}\alpha- \dots - \frac{a_{n-2}}{a_0} \alpha^{n-2}-\frac{1}{a_0} \alpha^{n-1}
\end{alignat*}\\[-24pt]
Hence $\frac{1}{\alpha} \in \mathbb{K}[X]$ and $\mathbb{K}[X]$ is a field.
\item[(ix)] The family $\{1,\alpha, \dots , \alpha^{n-1}\}$ forms a basis of $\mathbb{K}[\alpha]$ as a $\mathbb{K}$-vector space.
\end{compactenum}

\titleformat{\subsection}{\normalfont\normalsize\bfseries}{}{0em}{#1}
\subsection*{Example} %Example
\titleformat{\subsection}{\normalfont\normalsize\bfseries}{}{0em}{#1 \thesubsection}
Let $\mathbb{K}=\mathbb{Q}$, $\mathbb{L}=\mathbb{C}$, $\alpha=1+i$, $\beta = \sqrt{2}$. Then the minimal polynomials of $\alpha$ and $\beta$ are
\begin{center}$f_{\alpha}=\left(X-1\right)^2+1$, \textrm{  } $f_{\beta}=X^2-2$.\end{center}

\titleformat{\subsection}{\normalfont\normalsize\bfseries}{}{0em}{#1 \thesubsection \quad \textnormal{\textit{(Kronecker)}}}
\subsection{Proposition}\index{Kronecker} %Proposition 1.3
\titleformat{\subsection}{\normalfont\normalsize\bfseries}{}{0em}{#1 \thesubsection}
Let $\mathbb{K}$ be a field, $f \in \mathbb{K}[X]$, $\deg(f)\geqslant 1$.\\
Then there exists a finite field extension $\mathbb{L}/\mathbb{K}$ and $\alpha \in \mathbb{L}$, such that $f(\alpha)=0$.\\
\textit{proof.}\\
W.l.o.g. we may assume, that $f$ is irreducible, since $f=g \cdot h =0$ $\Rightarrow$ $g=0$ or $h=0$. Then by 1.2 $\langle f \rangle= \left\{f \cdot g \ \big \vert \ g \in \mathbb{K}[X]\right\}$ is a maximal ideal and $\mathbb{L}:=\slant{\mathbb{K}}{\langle f \rangle}$ is a field.\\
Clearly $\mathbb{K}$ is a subfield of $\mathbb{L}$, since $\langle f\rangle$ does not contain any constant polynomial, i.e., if 
\begin{center}$ \pi: \mathbb{K}[X] \longrightarrow \slant{\mathbb{K}[X]}{\langle f \rangle}$\end{center}
denotes the residue map, we have $\ker(\pi) \cap \mathbb{K}=\{0\}$, hence $\pi|_{\mathbb{K}}$ is injective.\\
Write $$f=\sum_{i=0}^n a_i X^{i}$$ Then we have
$$f\left(\pi(X)\right)=\sum_{i=0}^n a_i \pi(X)^{i}= \sum_{i=0}^{n}\pi(a_i) \pi(X)^{i}=\pi \left(\sum_{i=0}^n a_i X^{i}\right) = \pi(f)=0$$
Hence $\alpha:=\pi(X)$ is a zero of $f$ in $\mathbb{L}$.\\
Moreover $\mathbb{L}/\mathbb{K}$ is finite with degree $[\mathbb{L}:\mathbb{K}]=\deg(f)=n$, since $\{1,\alpha, \dots ,\alpha^{n-1}\}$ is basis of $\mathbb{L}$ as a $\mathbb{K}$-vector space:\\
For the independence write $$\sum_{i=0}^{n-1} \lambda_i \alpha^{i}=0$$Assume, there is $0\leqslant j \leqslant n-1$ with $\lambda_j \neq 0$. Then the polynomial $$g=\sum_{i=0}^{n-1} \lambda_i X^{i}$$ satisfies $g(\alpha)=0$ with $\deg(g)<\deg(f)$, which is not possible by irreducibility of $f$.\\
It remains to show, that $\mathbb{L}$ is generated by the powers of $\alpha$. We have $\alpha^n+a_{n-1}\alpha^{n-1}+ \dots +a_1 \alpha+a_0 = 0$, hence we write 
$$\alpha^n=-\left(a_{n-1}\alpha^{n-1}+ \dots + a_1 \alpha + a_0\right) \in \langle 1, \dots , \alpha^{n-1}\rangle$$
By induction on $n$, we get $\alpha^k \in \langle 1, \dots , \alpha^{n-1}\rangle$ for all $k \geqslant n$.

\titleformat{\subsection}{\normalfont\normalsize\bfseries}{}{0em}{#1}
\subsection*{Example} %Example
\titleformat{\subsection}{\normalfont\normalsize\bfseries}{}{0em}{#1 \thesubsection}
Let $\mathbb{K}=\mathbb{Q}$, $f=X^n-a$ for some $a \in \mathbb{Q}$. For now we assume that $f$ is irreducible (we may be able to prove this later). Then
$$\mathbb{L}:=\slant{\mathbb{Q}[X]}{\langle f \rangle}=\slant{\mathbb{Q}[X]}{\langle X^n-a\rangle} \cong \mathbb{Q}[\sqrt[n]{a}]=\mathbb{Q}(\sqrt[n]{a})$$

\subsection{Definition} %Definition 1.4
Let $\mathbb{L}/\mathbb{K}$ a field extension, $\alpha \in \mathbb{L}$.
\begin{compactenum}
\item $\alpha$ is called \textit{algebraic over} $\mathbb{K}$, if there exists $f \in \mathbb{X}[X] \setminus \{0\}$, such that $f(\alpha)=0$.
\item Otherwise $\alpha$ is called \textit{transcendental}.
\item $\mathbb{L}/\mathbb{K}$ is called an \textit{algebraic field extension}, if every $\alpha \in \mathbb{L}$ is algebraic over $\mathbb{K}$.
\end{compactenum}

\subsection{Proposition} %Proposition 1.5
Every finite field extension $\mathbb{L}/\mathbb{K}$ is algebraic.\\
\textit{proof.}\\
Let $\alpha \in \mathbb{L}$, $n:=[\mathbb{L}:\mathbb{K}]$ the degree of $\mathbb{L}/\mathbb{K}$. Then $1,\alpha, \dots \alpha^{n}$ are linearly dependant over $\mathbb{K}$, i.e. there exist $\lambda_0,...,\lambda_n \in \mathbb{K}$, $\lambda_j \neq 0$ for at least one $0 \leqslant j \leqslant n$, such that $$\sum_{i=0}^{n} \lambda_i \alpha^{i}=0$$
Hence the polynomial $$f=\sum_{i=0}^n \lambda_i X^{i} \neq 0$$ satifies $f(\alpha)=0$, thus $\alpha$ is algebraic over $\mathbb{K}$. Since $\alpha$ was arbitrary, $\mathbb{L}/\mathbb{K}$ is algebraic.

\subsection{Proposition} % Proposition 1.6
Let $\mathbb{L}/\mathbb{K}$ a field extension, $\alpha, \beta \in \mathbb{L}$.
\begin{compactenum}
\item If $\alpha, \beta$ are algebraic over $\mathbb{K}$, then $\alpha + \beta$, $\alpha - \beta$, $\alpha \cdot \beta$ are also algebraic over $\mathbb{K}$.
\item If $\alpha \neq0$ is algebraic over $\mathbb{K}$, then $\frac{1}{\alpha}$ is also algebraic over $\mathbb{K}$.
\item $\mathbb{K}_{\mathbb{L}}:=\left\{\alpha \in \mathbb{L} \big \vert \alpha \textrm{ is algebraic over } \mathbb{K}\right\} \subseteq \mathbb{L}$ is a subfield of $\mathbb{L}$.
\end{compactenum}
\textit{proof.}
\begin{compactenum}
\item Since $\alpha \in \mathbb{L}$ is algebraic over $\mathbb{K}$ $\Rightarrow$ $\mathbb{K}[\alpha]=\mathbb{K}(\alpha)$ is a finite field extension of $\mathbb{K}$.\\
Since $\beta$ is algebraic over $\mathbb{K} \Rightarrow$ $\beta$ is algebraic over $\mathbb{K}[\alpha]$, hence $\left(\mathbb{K}[\alpha]\right)[\beta]/\mathbb{K}[\alpha]$ is a finite field extension.\\
Further, we have $$\mathbb{K} \subseteq \mathbb{K}[a] \subseteq \left(\mathbb{K}[\alpha]\right)[\beta]=\mathbb{K}[\alpha, \beta]$$ $\Rightarrow \mathbb{K}[\alpha, \beta]/\mathbb{K}$ is algebraic with Proposition 1.5.
This implies the claim, as $\alpha + \beta$, $\alpha-\beta$, $\alpha \cdot \beta \in \mathbb{K}[\alpha, \beta]$.
\item If $\alpha \neq 0$, $\frac{1}{\alpha}$ is algebraic over $\mathbb{K}$ with part (i).
\item Follows from (i) and (ii).
\end{compactenum}

\subsection{Definition + Proposition} % Definition + Proposition 1.7
Let $\mathbb{K}$ be a field, $f \in \mathbb{K}[X]$, $\deg(f)=n$.
\begin{compactenum}
\item A field extension $\mathbb{L}/\mathbb{K}$ is called a \textit{splitting field of} $f$, if $\mathbb{L}$ is the smallest field in which $f$ decomposes into linear factors.
\item A splitting field $\mathbb{L}(f)$ exists.
\item The field extension $\mathbb{L}(f)/\mathbb{K}$ is algebraic over $\mathbb{K}$.
\item For the degree we have $[\mathbb{L}(f):\mathbb{K}] \leqslant n!$.
\end{compactenum}
\textit{proof.}
\begin{compactenum}
\item[(ii)] Do this by induction on $n$.
\begin{compactenum}
\item[\textbf{n=1}] Clear.
\item[\textbf{n>1}] Write $f= f_1 \cdot \cdot \cdot f_r$ with irreducible polynomials $f_i \in \mathbb{K}[X]$. Then $f$ splits if and only every $f_i$ splits. Hence we may assume that $f$ is irreducible\\
Consider $\mathbb{L}_1:=\slant{\mathbb{K}}{\langle f \rangle}$. Then $f$ has a zero in $\mathbb{L}_1$; say $\alpha$. Then we have $\mathbb{L}_1=\mathbb{K}[\alpha]$. Now we can write $f = (X-\alpha) \cdot g$ for some $g \in \mathbb{K}[X]$ with $\deg(g)=n-1$. By induction hypothesis, there exists a splitting field $\mathbb{L}(g)$ for $g$. Then $f$ splits over $\mathbb{L}(g)[\alpha]$.
\end{compactenum}
\item[(iii)] Follows by part (iv) and Proposition 1.5
\item[(iv)] Do this again by induction.
\begin{compactenum}
\item[\textbf{n=1}] Clear.
\item[\textbf{n>1}] In the notation of part (ii) we have $[\mathbb{K}[\alpha]:\mathbb{K}]=\deg(f)=n$. By the multiplication formula for the degree and induction hypothesis we have
$$[\mathbb{L}(f):\mathbb{K}]=[\mathbb{L}(g)[\alpha]:\mathbb{K}]=[\mathbb{L}(g)[\alpha]:\mathbb{L}(g)] \cdot [\mathbb{L}(g): \mathbb{K}] \leqslant n \cdot (n-1)!=n!$$
\end{compactenum}

\end{compactenum}

\subsection{Definition + Proposition} % Definition + Proposition 1.8
Let $\mathbb{K}$ be a field.
\begin{compactenum}
\item $\mathbb{K}$ is called \textit{algebraically closed}, if every $f \in \mathbb{K}[X]$ splits over $\mathbb{K}$.
\item The following statements are equivalent:
\begin{compactenum}
\item $\mathbb{K}$ is algebraically closed
\item Every nonconstant polynomial $f \in \mathbb{K}[X]$ has a zero in $\mathbb{K}$.
\item There is no proper algebraic field extension of $\mathbb{K}$.
\item If $f \in \mathbb{K}[X]$ is irreducible, then $\deg(f)=1$.
\end{compactenum}
\end{compactenum}
\textit{proof.}
\begin{compactitem}
\item['(1) $\Rightarrow$ (2)'] Let $f \in \mathbb{K}[X]$ be a non-constant polynomial of degree $n$. Then $f$ splits over $\mathbb{K}$, i.e. $$f=\prod_{i=0}^n (X-\lambda_i)$$ with $\lambda_i \in \mathbb{K}$ for $1\leqslant i \leqslant n$. Every $\lambda_i$ is a zero. Since $n\geqslant 1$, we find a zero for any nonconstant polynomial.\pagebreak
\item['(2) $\Rightarrow$ (3)'] Assume $\mathbb{L}/\mathbb{K}$ is algebraic, $\alpha \in \mathbb{L}$. Let $f_{\alpha}$ be the minimal polynomial of $\alpha$. By assumption, $f_{\alpha}$ has a zero in $\mathbb{K}$. Since $f_{\alpha}$ is irreducible, we must have $f_{\alpha}=X-\alpha$, hence $\alpha \in \mathbb{K}$, since $f \in \mathbb{K}[X]$.

\item['(3) $\Rightarrow$ (4)'] Let $f \in \mathbb{K}[X]$ irreducible. Then $\mathbb{L}:=\slant{\mathbb{K}[X]}{\langle f\rangle}$ is an algebraic field extension. By (3), $\mathbb{L}=\mathbb{K}$, hence $1=[\mathbb{L}:\mathbb{K}]=\deg(f)$.
\item['(4) $\Rightarrow$ (1)'] For $f \in \mathbb{K}[X]$ write $f=f_1 \cdot \cdot \cdot f_r$ with irreducible polynomials $f_i$ for $1\leqslant i \leqslant r$.\newline With (4), $\deg(f_i)=1$ for any $i$, hence $f$ splits.
\end{compactitem}

\subsection{Lemma} % Lemma 1.9
Let $\mathbb{K}$ be a field.
Then there exists an algebraic field extension $\mathbb{K}'/\mathbb{K}$, such that every $f \in \mathbb{K}[X]$ has a zero in $\mathbb{K}'$.\\
\textit{proof.}\\
For every irreducible polynomial $f \in \mathbb{K}[X]$ introduce a symbol $X_f$ and consider
$$R:=\mathbb{K}[ \left\{X_f \big \vert  f \in \mathbb{K}[X] \textrm{ irreducible}\right\}] \supseteq \mathbb{K}$$
Monomials in $R$ look like $$g=\lambda \cdot X_{f_1}^{n_1}X_{f_2}^{n_2}\cdots X_{f_k}^{n_k}$$ with $\lambda \in \mathbb{K}$, $n_i \in \mathbb{N}$.
Let $I \trianglelefteqslant R$ be the ideal generated by the $f(X_f)$, $f \in \mathbb{K}[X]$ irreducible.\\
The following claims prove the lemma:\\
\textbf{Claim (a)} $I \neq R$\\
\textbf{Claim (b)} There exists a maximal ideal $\mathfrak{m} \trianglelefteqslant R$ containing $I$.\\
\textbf{Claim (c)} $\mathbb{K}?=\slant{R}{\mathfrak{m}}$\\
To finish the proof, it remains to show the claims.
\begin{compactenum}
\item[\textbf{(a)}] Assume $I=R$. Then $1 \in  I$, i.e. $$1= \sum_{i=1}^k g_{f_i} f_i\left(X_{f_i}\right)$$ for suitable $g_{f_i} \in R$.\\
Let $\mathbb{L}/\mathbb{K}$ be a field extension in which all $f_i$ have a zero $\alpha_i$. Define a ring homomorphism
\begin{center} $\pi: R \longrightarrow \mathbb{L}$, $X_f \mapsto \begin{cases} \alpha_i, & f=f_i \\ 0, & \textrm{otherwise} \end{cases}$\end{center}
Then we obtain
$$1=\pi(1)= \pi \left(\sum_{i=1}^k g_{f_i} f_i\left(X_{f_i}\right)\right)=\sum_{i=1}^k \pi(g_{f_i}) f_i \left(\pi(X_{f_i})\right)=\sum_{i=1}^k \pi(g_{f_i}) f_i \left(\alpha_i\right)=0$$
Hence our assumption was false and we have $I \neq R$.
\item[\textbf{(b)}] Let $\mathcal{S}$ be the set of all proper ideals of $R$ containing $I$. By claim 2, $I \in \mathcal{S}$.
Let now $$S_1 \subseteq S_2 \subseteq S_3 \subseteq \dots$$ be elements of $\mathcal{S}$.
More generally let $N$ be a totally ordered subset of $\mathcal{S}$ and $$S:=\bigcap_{J \in N} J$$Then $S \in \mathcal{S}$, hence $\mathcal{S}$ is nonempty. By Zorn's Lemma we know that $\mathcal{S}$ contains a maximal element $\mathfrak{m}\neq R$. Then $\mathfrak{m}$ is maximal ideal of $R$, since an ideal $J \trianglelefteqslant R$ satisfying $\mathfrak{m} \subsetneq J \subsetneq R$ is contained in $\mathcal{S}$, which is a contradiction considering the choice of $\mathfrak{m}$.
\item[\textbf{(c)}] Clearly $\mathbb{K}'$ is a field extension of $\mathbb{K}$.
Let $f \in \mathbb{K}[X]$ be irreducible and\\ $\pi: R \longrightarrow \slant{\mathbb{K}}{\mathfrak{m}}$ denote the residue map. Then $$f(X_f) \in I \subseteq \mathfrak{m}$$i.e. we have $$\pi(X_f)=0$$ and thus $f\left(\pi(X_f)\right)=0$. Hence $\pi(X_f)$ is algebraic over $\mathbb{K}$.\\
Since $\mathbb{K}?$ is generated by the $\pi(X_f)$, $\mathbb{K}?/\mathbb{K}$ is algebraic, which finishes the proof.
\end{compactenum}

\subsection{Theorem} %Theorem 1.10
Let $\mathbb{K}$ be a field. Then there exists an algebraic field extension $\overline{\mathbb{K}}/\mathbb{K}$ such that $\overline{\mathbb{K}}$ is algebraically closed. $\overline{\mathbb{K}}$ is called the \textit{algebraic closure} of $\mathbb{K}$.\\
\textit{proof.}\\
By Lemma 1.9 there is an algebraic field extension $\mathbb{K}'/\mathbb{K}$, such that every $f \in \mathbb{K}[X]$ has a zero in $\mathbb{K}'$.
Then let $$\mathbb{K}_0:=\mathbb{K}, \rm{ }\mathbb{K}_1=\mathbb{K}_0', \rm{ }\mathbb{K}_2=\mathbb{K}_1', \mathbb{K}_{i+1}=\mathbb{K}_i' \textrm{ }\quad \textrm{ for } \textit{i} \geqslant 1$$
Clearly $\mathbb{K}_i$ is algebraic over $\mathbb{K}$ for all $i \in \mathbb{N}_0$ and $\mathbb{K}_i \subseteq \mathbb{K}_{i+1}$.
Define $$\overline{\mathbb{K}}:= \bigcup_{i \in \mathbb{N}_0} \mathbb{K}_i$$Then $\overline{\mathbb{K}}/\mathbb{K}$ is an algebraic field extension. For $f \in \overline{\mathbb{K}}[X]$ we find $i \in \mathbb{N}_0$ with $f \in \mathbb{K}_i[X]$, hence $f$ has a zero in $\mathbb{K}_i$. With proposition 1.8, $\overline{\mathbb{K}}$ is algebraically closed.

 % SECTION 2

\renewcommand*\thesection{\S\ \arabic{section}\quad}
\section{Simple field extensions}
\renewcommand*\thesection{\arabic{section}}

\subsection{Definition}\index{field extension!simple} % Definition 2.1
A field extension $\mathbb{L}/\mathbb{K}$ is called \textit{simple}, if there exists some $\alpha \in \mathbb{L}$ such that $\mathbb{L}=\mathbb{K}[\alpha]$

\titleformat{\subsection}{\normalfont\normalsize\bfseries}{}{0em}{#1}
\subsection*{Example} %Example
\titleformat{\subsection}{\normalfont\normalsize\bfseries}{}{0em}{#1 \thesubsection}
Let $f \in \mathbb{K}[X]$ be irreducible, $\mathbb{L}:=\slant{\mathbb{K}[X]}{\langle f\rangle}$.\\
Then $\mathbb{L}=\mathbb{K}[\alpha]$ where $\alpha= \pi(X)=\overline{X}$ and $\pi: \mathbb{K}[X] \longrightarrow \mathbb{L}$ denotes the residue map.\\
Conversely, if $\mathbb{L}/\mathbb{K}$ is simple and algebraic, then $\mathbb{L}=\mathbb{K}[\alpha]$ for some algebraic $\alpha \in \mathbb{L}$. Let $f \in  \mathbb{K}[X]$ be the minimal polynomial of $\alpha$ over $\mathbb{K}$, then $$\mathbb{L}=\mathbb{K}[\alpha]=\mathbb{K}(\alpha)=\slant{\mathbb{K}[X]}{\langle f \rangle}$$

\subsection{Proposition} % Proposition 2.2
Let $\mathbb{L}$ be a field. Then any finite subgroup $G$ of the multiplicative group $\mathbb{L}^{\times}$ is cyclic.\\
\textit{proof.}\\
Let $\alpha \in G$ be an element of maximal order, $n :=\textrm{ord}(\alpha)$. Define $$G':=\{\beta \in G: \textrm{ord}(\beta) \big \vert n \}$$
We first show $G'=G$ and then $G'=\langle \alpha \rangle$.\\
Let $\beta \in G$, $m:= \textrm{ord}(\beta)$. Then $$\rm{ord}(\alpha \beta)=lcm(m,n)\leqslant n$$by the property of $n$. Thus $m \big \vert n$ and $\beta \in G'$ and hence $G \subseteq G'$. Since $G' \subseteq G$ by definition, we have $G'=G$.\\
Let now $\gamma \in G'$. We have $\gamma^n=1$, hence $\gamma$ is zero of $$f= X^n-1$$$f$ has at most $n$ zeros, but since $\vert\langle \alpha \rangle\vert=n$, we have $\langle \alpha \rangle = G'$ which finishes the proof.

\subsection{Corollary} % Corollary 2.3
Let $\mathbb{K}$ be a finite field. Then every finite field extension $\mathbb{L}/\mathbb{K}$ is simple.\\
\textit{proof.}\\
We have $|\mathbb{L}|=|\mathbb{K}|^{[\mathbb{L}:\mathbb{K}]}$ and thus $\mathbb{L}$ is also finite. With proposition 2.2 there exists some $\alpha \in \mathbb{L}$ such that\linebreak
$\mathbb{L}^{\times}=\mathbb{L} \setminus \{0\}=\langle \alpha \rangle$, hence $$\mathbb{L}=\mathbb{K}[\alpha]$$

\subsection{Remark} %Remark 2.4
Let $\mathbb{L}/\mathbb{K}$ be a finite field extension, $f \in \mathbb{K}[X]$ and $\alpha \in \mathbb{L}$ a zero of $f$. Let $\overline{\mathbb{K}}$ be an algebraic closure of $\mathbb{K}$ and $\sigma: \mathbb{L} \longrightarrow \overline{\mathbb{K}}$ a homomorphism of field such that $\sigma|_{\mathbb{K}}=\textrm{id}_{\mathbb{K}}$.\\
Then $\sigma(\alpha)$ is a zero of $f$.
\pagebreak \\
\textit{proof.}\\
Write $$f=\sum_{i=0}^n a_i X^{i}$$ with coefficients $a_i \in \mathbb{K}$, hence we have $\sigma(a_i)=a_i$ for $0\leqslant i \leqslant n$. We obtain
$$f\left(\sigma(\alpha)\right)=\sum_{i=0}^n a_i \left(\sigma(\alpha)\right)^{i}=\sum_{i=0}^n \sigma(a_i) \left(\sigma(\alpha)\right)^{i}=\sigma\left(\sum_{i=0}^n a_i \alpha^{i}\right)=\sigma\left(f(\alpha)\right)=\sigma(0)=0$$

\subsection{Theorem} % Theorem 2.5
Let $\mathbb{L}/\mathbb{K}$ be a finite field extension of degree $n:=[\mathbb{L}:\mathbb{K}]$ and $\overline{\mathbb{K}}$ an algebraic closure of $\mathbb{K}$.\\
If there exist $n$ different field homomorphisms $\sigma_1, \dots \sigma_n: \mathbb{K} \longrightarrow \mathbb{L}$ such that\\ $\sigma_i |_{\mathbb{K}}=\textrm{id}_{\mathbb{K}}$, then $\mathbb{L}/\mathbb{K}$ is simple.\\
\textit{proof.}\\
Let $\mathbb{L}=\mathbb{K}[\alpha_1,...,\alpha_r]$ for some $r\geqslant 1 $ and $\alpha_i \in \mathbb{L}$. Prove the statement by induction on $r$.
\begin{compactitem}
\item[\textbf{r=1}] $\mathbb{L}=\mathbb{K}[\alpha_1]$, hence $\mathbb{L}$ is simple.
\item[\textbf{r>1}] Let now $\mathbb{L}'=\mathbb{K}[\alpha_1, \dots \alpha_{r-1}]$. By hypothesis, $\mathbb{L}'/\mathbb{K}$ is simple, say $\mathbb{L}=\mathbb{K}[\beta]$. Then we have $$\mathbb{L}=\mathbb{K}[\alpha_1, \dots \alpha_r]=\mathbb{L}'[\alpha_r]=\mathbb{K}[\alpha, \beta]$$ with $\alpha:=\alpha_r$.\\
For $\lambda \in \mathbb{K}$ consider $$\gamma:=\gamma_{\lambda}=\alpha + \lambda \beta$$By remark 2.4 it suffices to show $$\sigma_i(\gamma) \neq \sigma_j(\gamma) \textrm{ for }i\neq j$$Assume there are $i\neq j$ such that $\sigma_i(\gamma)=\sigma_j(\gamma)$.\\
Then $$\sigma_i(\alpha)+ \lambda \sigma_i(\beta)=\sigma_j(\alpha)+\lambda \sigma_j(\beta),$$so we get $$\sigma_i(\alpha)-\sigma_j(\alpha)+\lambda\left(\sigma_i(\beta)-\sigma_j(\beta)\right)=0$$Consider the polynomial $$g:=\prod_{1\leqslant i \neq j \leqslant n}\sigma_i(\alpha)-\sigma_j(\alpha)+X\cdot\left(\sigma_i(\beta)-\sigma_j(\beta)\right)$$
By proposition 2.2 we may assume, that $\mathbb{K}$ is infinite. Note that $g$ is not the zero polynomial: If $g=0$, we find $i\neq j$ such that $\sigma_i(\alpha)=\sigma_j(\alpha)$ and $\sigma_i(\beta)=\sigma_j(\beta)$. Since $\alpha, \beta$ generate $\mathbb{L}$, $\sigma_i$ and $\sigma_j$ must be equal on $\mathbb{L}$, which is a contradiction.\\
Therefore we find $\lambda \in \mathbb{K}$, such that $g(\lambda)\neq 0$. Hence the minimal polynomial $m_{\gamma_{\lambda}}$ of $\gamma_{\lambda}=\alpha + \lambda \beta$ has at least $n$ zeroes, i.e. $$\deg(m_{\gamma_{\lambda}}) \geqslant n \Rightarrow [\mathbb{K}[\gamma_{\lambda}]:\mathbb{K}]\geqslant n$$ and hence $\mathbb{K}[\gamma_{\lambda}]=\mathbb{L}$.
\end{compactitem}

\subsection{Proposition} % Proposition 2.6
Let $\mathbb{L}=\mathbb{K}[\alpha]$ be a simple, finite field extension, $\overline{\mathbb{K}}$ an algebraic closure of $\mathbb{K}$. Let $f \in \mathbb{K}[X]$ the minimal polynomial of $\alpha$. Then for every zero $\beta$ of $f$ in $\overline{\mathbb{K}}$ there exists a unique homomorphism of fields $$\sigma: \mathbb{L} \longrightarrow \overline{\mathbb{K}}$$ such that $\sigma(\alpha)=\beta$\\
\textit{proof.}\\
The uniqueness is clear. It remains to show the existence.\\
Define $$\phi_{\beta}:\mathbb{K}[X] \longrightarrow \overline{\mathbb{K}}, \qquad g \mapsto g(\beta)$$We have $$f(\beta)=0 \ \Longrightarrow \ \langle f \rangle \subseteq ker(\phi_{\beta})$$hence $\phi_{\beta}$ factors to a homomorphism $$\overline{\phi_{\beta}}: \mathbb{L} \cong \slant{\mathbb{K}[X]}{\langle f \rangle} \longrightarrow \overline{\mathbb{K}}$$such that $\phi_{\beta}=\overline{\phi_{\beta}}\circ \pi$ where $\pi: \mathbb{K}[X] \longrightarrow \slant{\mathbb{K}[X]}{\langle f \rangle}$ denotes the residue map. Let $$\tau: \mathbb{L} \longrightarrow \slant{\mathbb{K}[X]}{\langle f \rangle}$$ be an isomorphism. Then
$$\sigma:= \overline{\phi_{\beta}} \circ \tau: \mathbb{L} \longrightarrow \overline{\mathbb{K}}$$ satisfies $$\sigma(\alpha)=\left(\overline{\phi_{\beta}} \circ \tau\right)(\alpha)=\overline{\phi_{\beta}}\left(\tau(\alpha)\right)=\overline{\phi_{\beta}}(\overline{X})=\overline{\phi_{\beta}}\left(\pi(X)\right)=\phi_{\beta}(X)=\beta$$

\subsection{Corollary} % Corollary 2.7
Let $f \in \mathbb{K}[X]$ be a nonconstant polynomial. Then the splitting field of $f$ over $\mathbb{K}$ is unique, i.e. any two splitting fields $\mathbb{L}, \mathbb{L}'$ of $f$ over $\mathbb{K}$ are isomorphic.\\
\textit{proof.}\\
Let $\mathbb{L}=\mathbb{K}[\alpha_1, \dots \alpha_n]$, $\mathbb{L}'=\mathbb{K}[\beta_1, \dots \beta_m]$.\\
Assume that $f$ is irreducible. W.l.o.g. we have $f(\alpha_1)=f(\beta_1)=0$. By Proposition 2.6 we find field homomorphisms\\
$$\sigma_1: \mathbb{K}[\alpha_1] \longrightarrow \mathbb{K}[\beta_2] \textrm{ such that }\sigma_1 |_{\mathbb{K}}=\rm{id}_{\mathbb{K}}\textrm{ and }\alpha_1 \mapsto \beta_1$$
$$\tau_1: \mathbb{K}[\beta_1] \longrightarrow \mathbb{K}[\alpha_1] \textrm{ such that }\tau_1 |_{\mathbb{K}}=\rm{id}_{\mathbb{K}}\textrm{ and }\beta_1 \mapsto \alpha_1$$
Hence, since $\sigma_1 \circ \tau_1 = \rm{id}_{\mathbb{K}[\beta_1]}$ and $\tau_1 \circ \sigma_1 = \rm{id}_{\mathbb{K}[\alpha_1]}$, $\sigma_1$ and $\tau_1$ are isomorphisms, i.e $\mathbb{K}[\alpha_1] \cong \mathbb{K}[\beta_1]$.\\
By induction on $n$ the corollary follows.

\subsection{Definition + Proposition} % Definition + Proposition 2.8
Let $\mathbb{L}/\mathbb{K}$, $\mathbb{L}'/\mathbb{K}$ be field extension.
\begin{compactenum}
\item We define $$\rm{Hom}_{\mathbb{K}}(\mathbb{L},\mathbb{L}'):=\left\{\sigma: \mathbb{L}\longrightarrow \mathbb{L}' \textrm{ field homomorphism s.t. } \sigma |_{\mathbb{K}}=\rm{id}_{\mathbb{K}}\right\}$$
$$\rm{Aut}_{\mathbb{K}}(\mathbb{L}):=\left\{\sigma: \mathbb{L} \longrightarrow \mathbb{L} \textrm{ field automorphism s.t. } \sigma |_{\mathbb{K}}=\rm{id}_{\mathbb{K}} \right\}$$
\item If $\mathbb{L}/\mathbb{K}$ is finite, $\overline{\mathbb{K}}$ an algebraic closure of $\mathbb{K}$, then 
$$ \vert \rm{Hom}_{\mathbb{K}}(\mathbb{L},\mathbb{L}') \vert \leqslant [\mathbb{L}:\mathbb{K}]$$
\end{compactenum}
\textit{proof.}\\
Assume first $\mathbb{L}=\mathbb{K}[\alpha]$ for some algebraic $\alpha \in \mathbb{L}$.\\
Let $f$ be the minimal polynomial of $\alpha$ over $\mathbb{K}$, i.e. $f \in \mathbb{K}[X]$, $\deg(f)=[\mathbb{L}:\mathbb{K}]$.\\
By 2.4 and 2.6, the elements oh $\rm{Hom}_{\mathbb{K}}(\mathbb{L}, \overline{\mathbb{K}})$ correspond bijectively to the zeroes of $f$. Then we get
$$ \vert \rm{Hom}_{\mathbb{K}}(\mathbb{L}, \overline{\mathbb{K}}) \vert = \vert \{\textrm{zeroes of f in }\overline{\mathbb{K}} \}\vert \leqslant deg(f) = [\mathbb{L}:\mathbb{K}]$$
Now consider the general case. Let $\mathbb{L}=\mathbb{K}[\alpha_1, \dots \alpha_n]$ and $\mathbb{L}'=\mathbb{K}[\alpha_1, \dots \alpha_{n-1}] \subseteq \mathbb{L}= \mathbb{L}'[\alpha_n]$.\\
By induction on $n$ we have $\vert \rm{Hom}_{\mathbb{K}}(\mathbb{L}', \overline{\mathbb{K}}) \leqslant [\mathbb{L}':\mathbb{K}]$.
Let now $$f=\sum_{i=0}^d a_i X^{i} \in \mathbb{L}'[X]$$ with coefficients $a_i \in \mathbb{L}'$ be the minimal polynomial of $\alpha_n$ over $\mathbb{L}'$.  Let $\sigma \in \rm{Hom}_{\mathbb{K}}(\mathbb{L}, \overline{\mathbb{K}})$ and $\sigma'=\sigma |_{\mathbb{L}'} \in \rm{Hom}_{\mathbb{K}}(\mathbb{L}', \overline{\mathbb{K}})$, $f^{\sigma'}:= \sum_{i=0}^d \sigma'(a_i) X^{i}$. Then
$$f^{\sigma'}\left(\sigma(\alpha_n)\right)=\sum_{i=0}^d \sigma'(a_i) \left(\sigma(\alpha_n)\right)^{i}=\sum_{i=0}^d \sigma(a_i) \left(\sigma(\alpha_n)\right)^{i} = \sigma \left(\sum_{i=0}^d a_i \alpha_n^{i}\right)=0$$
Thus
$$ \vert \{\textrm{Hom}_{\mathbb{L}'}(\mathbb{L}, \overline{\mathbb{K}}) \}\vert=\vert \{\sigma \in \rm{Hom}_{\mathbb{K}}(\mathbb{L}, \overline{\mathbb{K}}) \big \vert \sigma |_{\mathbb{L}'}=\textrm{id}_{\mathbb{L}'}\} \vert \leqslant \deg(f^{\sigma'}) = deg (f) = [\mathbb{L}':\mathbb{L}]$$
So all in all we have
$$\vert \rm{Hom}_{\mathbb{K}}(\mathbb{L}, \overline{\mathbb{K}}) \vert \leqslant \vert \rm{Hom}_{\mathbb{K}}(\mathbb{L}', \overline{\mathbb{K}}) \vert \cdot [\mathbb{L}:\mathbb{L}'] \leqslant [\mathbb{L}:\mathbb{L}'] \cdot [\mathbb{L}':\mathbb{K}]=[\mathbb{L}:\mathbb{K}]$$

\subsection{Definition} %Definition 2.9
Let $\mathbb{K}$ be a field, $f=\sum_{i=0}^d a_i X^{i} \in \mathbb{K}[X]$, $\overline{\mathbb{K}}$ an algebraic closure of $\mathbb{K}$, $\mathbb{L}/\mathbb{K}$ an algebraic field extension.
\begin{compactenum}
\item $f$ is called \textit{separable} over $\mathbb{K}$, if $f$ has $\deg(f)$ different roots in $\overline{\mathbb{K}}$, i.e. there are no multiple roots.
\item $\alpha \in \mathbb{L}$ is called \textit{separable} over $\mathbb{K}$, if the minimal polynomial of $\alpha$ over $\mathbb{K}$ is separable.
\item $\mathbb{L}/\mathbb{K}$ is called \textit{separable}, if any $\alpha \in \mathbb{L}$ is separable over $\mathbb{K}$. 
\item We define the \textit{formal derivative} of $f$ by $$f':=\sum_{i=1}^d i \cdot a_iX^{i-1}$$We have well known properties of the derivative:
$$(f+g)'=f'+g', \qquad1'=0, \qquad (f\cdot g)'=f\cdot g'+f'\cdot g$$
\end{compactenum}

\subsection{Proposition} %Proposition 2.10
Let $$f=\prod_{i=1}^n (X-\alpha_i) \in \mathbb{K}[X], \qquad a_i \in \overline{\mathbb{K}}\textrm{ for }1 \leqslant i\leqslant n$$
Then the following statements are equivalent:
\begin{compactenum}
\item $f$ is separable.
\item $(X-\alpha_i) \nmid f'$ for $1 \leqslant i \leqslant n$.
\item $\gcd(f,f')=1$ in $\mathbb{K}[X]$.
\end{compactenum}
\textit{proof.}
\begin{compactitem}
\item['(i) $\Leftrightarrow$ (ii)'] We have $$f'=\sum_{i=1}^n \prod_{j\neq i} (X-\alpha_j)$$Then we get
$$(X-\alpha_i) \mid f' \Leftrightarrow (X-\alpha_i) \mid \prod_{j\neq i} (X-\alpha_j) \Leftrightarrow \alpha_i = \alpha_j\textrm{ for some }i\neq j$$
\item['(ii) $\Rightarrow$ (iii)'] Assume $(X-\alpha_i) \nmid f'$ for all $1\leqslant i \leqslant n$. Then $$\gcd(f,f')=1 \textrm{ in }\overline{\mathbb{K}}[X] \Longrightarrow \gcd(f,f')=1 \textrm{ in }\mathbb{K}[X]$$ 
\item['(iii) $\Rightarrow$ (ii)'] Let now $\gcd(f,f')=1$ in $\mathbb{K}[X]$. Then we can write $$1=af+bf', \textrm{  }a,b \in \mathbb{K}[X]$$Since again $\mathbb{K}[X] \subseteq \overline{\mathbb{K}}[X]$, we can write $1=af+bf'$ for $a,b \in \overline{\mathbb{K}}[X]$ an hence we obtain $\gcd(f,f')=1$ in $\overline{\mathbb{K}}[X]$. This implies $$(X-\alpha_i) \nmid f' \textrm{ for all }1 \leqslant i \leqslant n$$
\end{compactitem}

\subsection{Corollary} % Corollary 2.11
\begin{compactenum}
\item An irreducible polynomial $f \in \mathbb{K}[X]$ is separable if and only if $f' \neq 0$. 
\item Any algebraic field extension in characteristic $0$ is separable.
\end{compactenum}

\titleformat{\subsection}{\normalfont\normalsize\bfseries}{}{0em}{#1}
\subsection*{Example} %Example 
\titleformat{\subsection}{\normalfont\normalsize\bfseries}{}{0em}{#1 \thesubsection}
Let $\rm{char}(\mathbb{K})=p>0$. Then $$X^p-1=(X-1)^p$$Let $\mathbb{K}=\mathbb{F}_p(t)$ and $f=X^p-t \in \mathbb{F}_p(t)[X]$.\\
Then $f'=0$, hence $f$ is not separable, but $f$ is irreducible in $\mathbb{F}_p(t)[X]$.

\subsection{Definition + Proposition} % Definition 2.12
Let $\mathbb{L}/\mathbb{K}$ be a finite field extension, $\overline{\mathbb{K}}$ an algebraic closure of $\mathbb{K}$ and $\mathbb{L}$. 
\begin{compactenum}
\item $[\mathbb{L}:\mathbb{K}]_s:= \vert \rm{Hom}_{\mathbb{K}}(\mathbb{L}, \overline{\mathbb{K}}) \vert$ is called the \textit{degree of separability} of $\mathbb{L}/\mathbb{K}$.
\item If $\mathbb{L}=\mathbb{K}[\alpha]$ for some separable $\alpha \in \mathbb{L}$ with minimal polynomial $m_{\alpha}$ over $\mathbb{K}$, then
$$[\mathbb{L}:\mathbb{K}]_s=\deg(m_{\alpha})=[\mathbb{L}:\mathbb{K}]$$
\item If $\mathbb{L}=\mathbb{K}[\alpha]$ for some $\alpha \in \mathbb{L}$, $\rm{char}(\mathbb{K})=p>0$, then there exists $n \geqslant 0$, such that
$$[\mathbb{L}:\mathbb{K}] = p^n \cdot [\mathbb{L}:\mathbb{K}]_s$$
\item  If $\mathbb{K} \subseteq \mathbb{F} \subseteq \mathbb{L}$ is an intermediate field extension, then
$$[\mathbb{L}:\mathbb{K}]_s=[\mathbb{L}:\mathbb{F}]_s \cdot [\mathbb{F}:\mathbb{K}]_s$$
\end{compactenum}
\textit{proof.}
\begin{compactenum}
\item[(i)] This follows from Propoition 2.6:
$$[\mathbb{L}:\mathbb{K}]_s=\vert \textrm{Hom}_{\mathbb{K}}(\mathbb{L}, \overline{\mathbb{K}}) \vert = \vert \{ \textrm{ different zeroes of } f \} \vert =n = [\mathbb{L}:\mathbb{K}]$$
\item[(iii)] Write 
$$f=\sum_{i=0}^n a_iX{i}$$
If $\alpha$ is separable over $\mathbb{K}$, we are done with part (ii). Otherwise by Corollary 2.11 we have
$$f?=\sum_{i=1}^n i \cdot a_i \cdot X^{i-1}\overset{!}{=} 0 \ \Longleftrightarrow \ i \cdot a_i \equiv 0 \mod p \ \textrm{ for all } 0 \leqslant i \leqslant n $$
Thus we can write $f=g(X^p)$ for some $g \in \mathbb{K}[X]$.\\
Continue this until we can write $f=g(X^{p^n})$ for some $n \in \mathbb{N}_0$ and separable $g$. Then 
$$[\mathbb{K}[\alpha]:\mathbb{K}]_s=\vert \{ \textrm{ zeroes of }g \textrm{ in } \overline{\mathbb{K}} \} \vert =\deg(g)$$
and thus we obtain
$$[\mathbb{K}[\alpha]:\mathbb{K}]=\deg(f)=\deg(g) \cdot p^n =p^n \cdot [\mathbb{K}[\alpha]:\mathbb{K}]_s$$
\item[(iv)] Consider first the simple case $\mathbb{L}=\mathbb{K}(\alpha)$. Let
$$f=\sum_{i=0}^n a_i X^{i} \in\mathbb{F}[X]$$
be the minimal polynomial of $\alpha$ over $\mathbb{F}$. Let $\tau \in \textrm{Hom}_{\mathbb{K}}(\mathbb{F}, \overline{\mathbb{K}})$ and let
$$f^{\tau}=\sum_{i=0}^n \tau(a_i) X^{i}$$
Given $\sigma \in \textrm{Hom}_{\mathbb{K}}(\mathbb{L}, \overline{\mathbb{K}})$ with $\sigma |_{\mathbb{F}}=\tau$, notice that $\sigma(\alpha)$ is a zero of $f^{\tau}$. Moreover by Proposition 2.6, every zero $\beta$ of $f^{\tau}$ determines a unique $\sigma$ such that $\sigma(\alpha)=\beta$. \\
Thus we have
\begin{alignat*}{5}
\big\vert \{ \sigma \in \textrm{Hom}_{\mathbb{K}}(\mathbb{L}, \overline{\mathbb{K}}) \ \mid\ \sigma|_{\mathbb{F}}=\tau \} \big \vert \ &=&& \ \big \vert \{ \beta \in \overline{\mathbb{K}} \ \mid\ f^{\tau}(\beta)=0 \} \big\vert \\
&=&& \ \big\vert \{\beta \in \overline{\mathbb{K}} \ \mid \ f(\beta)=0 \} \big\vert \overset{2.6}{=} [\mathbb{L}:\mathbb{F}]_s
\end{alignat*}
We conclude
\begin{alignat*}{5}
[\mathbb{L}:\mathbb{K}]_s \ &=&& \ \big\vert \textrm{Hom}_{\mathbb{K}}(\mathbb{L}, \overline{\mathbb{K}}) \big\vert \ = \ \Big\vert \bigcup_{\tau \in \textrm{Hom}_{\mathbb{K}}(\mathbb{F}, \overline{\mathbb{K}})} \left\{ \sigma \in \textrm{Hom}_{\mathbb{K}}(\mathbb{L}, \overline{\mathbb{K}}) \ \mid \ \sigma|_{\mathbb{F}}=\tau \right\} \Big\vert \\
&=&& \ \big\vert \left\{ \sigma \in \textrm{Hom}_{\mathbb{K}}(\mathbb{L}, \overline{\mathbb{K}}) \ \mid \ \sigma|_{\mathbb{F}}=\tau \right\} \big\vert \cdot \big\vert \textrm{Hom}_{\mathbb{K}}(\mathbb{F}, \overline{\mathbb{K}}) \big\vert\\ 
&=&& \ [\mathbb{L}:\mathbb{F}]_s \cdot [\mathbb{F}:\mathbb{K}]_s
\end{alignat*}
For the general case we can write $\mathbb{L}=\mathbb{F}(\alpha_1, \ldots, \alpha_n)$. Define $\mathbb{L}_i:=\mathbb{F}(\alpha_1, \ldots, \alpha_i)$, $\mathbb{L}_0:=\mathbb{F}$ and $\mathbb{L}_n=\mathbb{L}$. Then $\mathbb{L}_i / \mathbb{L}_{i-1}$ is simple and by the special case above we get
\begin{alignat*}{5}
[\mathbb{L}:\mathbb{K}]_s \ &=&& \ [\mathbb{L}_n:\mathbb{L}_{n-1}]_s \cdot [\mathbb{L}_{n-1}:\mathbb{K}]_s \\
&\ \ \vdots&& \\
&=&& \ [\mathbb{L}_n:\mathbb{L}_{n-1}]_s \cdot \cdot \cdot [\mathbb{L}_2:\mathbb{L}_1]_s \cdot [\mathbb{L}_1:\mathbb{L}_0]_s \cdot [\mathbb{L}_0:\mathbb{K}]_s \\
&=&& \ [\mathbb{L}_n:\mathbb{L}_{n-1}]_s \cdot \cdot \cdot [\mathbb{L}_2:\mathbb{L}_1]_s \cdot [\mathbb{L}_1:\mathbb{F}]_s \cdot [\mathbb{F}:\mathbb{K}]_s \\
&=&& \ [\mathbb{L}_n:\mathbb{L}_{n-1}]_s \cdot \cdot \cdot [\mathbb{L}_2:\mathbb{F}]_s \cdot [\mathbb{F}:\mathbb{K}]_s \\
&\ \ \vdots&&\\
&=&& \ [\mathbb{L}_n:\mathbb{F}]_s \cdot [\mathbb{F}:\mathbb{K}]_s \\
&=&& \ [\mathbb{L}:\mathbb{F}]_s \cdot [\mathbb{F}:\mathbb{K}]_s
\end{alignat*}

\end{compactenum}

\subsection{Proposition}% Proposition 2.13
A finite field extension $\mathbb{L}/\mathbb{K}$ is separable if and only if $[\mathbb{L}:\mathbb{K}]=[\mathbb{L}:\mathbb{K}]_s$.
\pagebreak \\
\textit{proof.}
\begin{compactitem}
\item['$\Rightarrow$'] Let $\mathbb{L}=\mathbb{K}[\alpha_1, \dots \alpha_n]$. Prove this by induction on $n$.
\begin{compactitem}
\item[\textbf{n=1}] This is proposition 12.2(ii)
\item[\textbf{n>1}] Let $\mathbb{L}'=\mathbb{K}[\alpha_1, \dots \alpha_{n-1}]$. Then by induction hypothesis $[\mathbb{L}':\mathbb{K}]_s=[\mathbb{L}':\mathbb{K}]$. Moreover $[\mathbb{L}:\mathbb{L}']_s=[\mathbb{L}:\mathbb{L}']$, since $\mathbb{L}/\mathbb{L}'$ is simple by $\mathbb{L}=\mathbb{L}'[\alpha_n]$. By proposition 12.2 (iv) we get
$$[\mathbb{L}:\mathbb{K}]_s=[\mathbb{L}:\mathbb{L}']_s \cdot [\mathbb{L}':\mathbb{K}]_s=[\mathbb{L}:\mathbb{L}'] \cdot [\mathbb{L}'.\mathbb{K}]=[\mathbb{L}:\mathbb{K}]$$
\end{compactitem}
\item['$\Leftarrow$'] Let $\alpha \in \mathbb{L}$ and $f= m_{\alpha} \in \mathbb{K}[X]$ its minimal polynomial. If $\rm{char}(\mathbb{K})=0$, $f$ is separable, so $\alpha$ is separable by corollary 2.11. Let now $\textrm{char}(\mathbb{K})=p>0$. \\
By proposition 12.2 there exists $n \geqslant 0$ such that $$[\mathbb{K}[\alpha]:\mathbb{K}]=p^n \cdot [\mathbb{K}[\alpha]:\mathbb{K}]_s$$We find
$$
[\mathbb{L}:\mathbb{K}]\ =\ [\mathbb{L}:\mathbb{K}[\alpha]] \ \cdot\ [\mathbb{K}[\alpha]:\mathbb{K}] \ \geqslant \ [\mathbb{L}:\mathbb{K}[\alpha]]_s\ \cdot\ p^n \ [\mathbb{K}[\alpha]:\mathbb{K}]_s \ = \ p^n\ [\mathbb{L}:\mathbb{K}]_s = \ p^n  \ [\mathbb{L}:\mathbb{K}]$$
Hence we must have $n=0$, i.e. $[\mathbb{K}[\alpha]:\mathbb{K}]=[\mathbb{K}[\alpha]:\mathbb{K}]_s$. Thus $\alpha$ is separable over $\mathbb{K}$.
\end{compactitem}

%SECTION 3

\renewcommand*\thesection{\S\ \arabic{section}\quad}
\section{Galois extensions}
\renewcommand*\thesection{\arabic{section}}

\subsection{Definition} %Definition 3.1
A field extension $\mathbb{L}/\mathbb{K}$ is called \textit{normal}, if there is a subset $\mathcal{F} \subseteq \mathbb{K}[X]$ such that $\mathbb{L}$ is the smallest field which any $f \in \mathcal{F}$ splits over.

\subsection{Remark} % Remark 3.2
Let $\mathbb{L}/\mathbb{K}$ be a normal field extension, $\overline{\mathbb{K}}$ an algebraic closure of $\mathbb{K}$. Then 
$$\rm{Hom}_{\mathbb{K}}(\mathbb{L},\overline{\mathbb{K}})=\rm{Aut}_{\mathbb{K}}(\mathbb{L})$$
\textit{proof.}
\begin{compactitem}
\item['$\supseteq$'] Clear.
\item['$\subseteq$'] Let $\mathbb{L}$ be the splitting field of $\mathcal{F}$. Let $$f=\sum_{i=0}^d a_i X^{i} \in \mathcal{F}$$ and $\alpha \in  \mathbb{L}$ such that $f(\alpha)=0$. Let $\sigma \in \rm{Hom}_{\mathbb{K}}(\mathbb{L}, \overline{\mathbb{K}})$. Then
$$f\left(\sigma(\alpha)\right)=\sum_{i=0}^d a_i \sigma(\alpha)^{i} = \sum_{i=0}^d \sigma(a_i) \sigma(\alpha)^{i} = \sigma\left(\sum_{i=0}^d a_i \alpha^{i}\right)= \sigma \left(f(\alpha)\right)=0$$hence $\sigma(\alpha)$ is zero of $f$. Since $f$ splits over $\mathbb{L}$, i.e. all zeroes of $f$ are in $\mathbb{L}$, we have $\sigma(\alpha) \in \mathbb{L}$. Moreover $\mathbb{L}$ is generated over $\mathbb{K}$ by the zeroes of $f \in \mathcal{F}$, thus $\sigma(\mathbb{L})\subseteq \mathbb{L}$ and hence we get $\sigma \in \rm{Hom}_{\mathbb{K}}(\mathbb{L},\mathbb{L})$.\\
It remains to show bijectivity. $\sigma$ is clearly injective. For the surjectivity consider that $\sigma$ permutes all the zeroes of any $f \in \mathcal{F}$. Finally $\sigma \in \rm{Aut}_{\mathbb{K}}(\mathbb{L})$.
\end{compactitem}

\subsection{Definition} % Definition 3.3
An algebraic field extension $\mathbb{L}/\mathbb{K}$ is called \textit{Galois extension} or \textit{Galois}, if it is normal and separable. In this case, the \textit{Galois group} of $\mathbb{L}/\mathbb{K}$ is defined as
$$\rm{Gal}(\mathbb{L},\mathbb{K}):=\rm{Aut}_{\mathbb{K}}(\mathbb{L})$$

\subsection{Proposition} %Proposition 3.4
A finite field extension $\mathbb{L}/\mathbb{K}$ is Galois if and only if $\vert \rm{Aut}_{\mathbb{K}}(\mathbb{L}) \vert=[\mathbb{L}:\mathbb{K}]$.\\
\textit{proof.}
\begin{compactitem}
\item['$\Rightarrow$'] We have
$$ \vert \rm{Aut}_{\mathbb{K}}(\mathbb{L}) \vert = \vert \rm{Hom}_{\mathbb{K}}(\mathbb{L}, \overline{\mathbb{K}}) \vert = [\mathbb{L}:\mathbb{K}]_s=[\mathbb{L}:\mathbb{K}]$$
\item['$\Leftarrow$'] We have to show that $\mathbb{L}/\mathbb{K}$ is separable and normal. First we see
$$[\mathbb{L}:\mathbb{K}]= \vert \rm{Aut}_{\mathbb{K}}(\mathbb{L}) \vert \leqslant \vert \rm{Hom}_{\mathbb{K}}(\mathbb{L}, \overline{\mathbb{K}}) \vert = [\mathbb{L}:\mathbb{K}]_s \leqslant [\mathbb{L}:\mathbb{K}]$$
Hence we have equality on each inequality, i.e. $[\mathbb{L}:\mathbb{K}]=[\mathbb{L}:\mathbb{K}]_s$ and $\mathbb{L}/\mathbb{K}$ is separable.\\
By Theorem 2.5 we know that $\mathbb{L}/\mathbb{K}$ is simple, say $\mathbb{L}=\mathbb{K}[\alpha]$ for some $\alpha \in \mathbb{L}$.\\Let $m_{\alpha} \in \mathbb{K}[X]$ be the minimal polynomial of $\alpha$ over $\mathbb{K}$. Moreover let $\beta \in \overline{\mathbb{K}}$ be another zero of $m_{\alpha}$. Then there exists $\sigma \in \rm{Hom}_{\mathbb{K}}(\mathbb{L}, \overline{\mathbb{K}})$ such that $\sigma(\alpha)=\beta$. By the (in-)equality above we know $\rm{Aut}_{\mathbb{K}}(\mathbb{L})=\rm{Hom}_{\mathbb{K}}(\mathbb{L}, \overline{\mathbb{K}})$, hence $\sigma(\beta) \in \mathbb{L}$. Since $\beta$ was an arbitrary zero of $m_{\alpha}$, $f$ splits over $\mathbb{L}$, i.e. $\mathbb{L}$ is the splitting field of $f$ over $\mathbb{K}$. Thus $\mathbb{L}/\mathbb{K}$ is normal and finally Galois.
\end{compactitem}

\titleformat{\subsection}{\normalfont\normalsize\bfseries}{}{0em}{#1}
\subsection*{Example} %Example
\titleformat{\subsection}{\normalfont\normalsize\bfseries}{}{0em}{#1 \thesubsection}
All quadratic field extensions are normal. Moreover, if $\rm{char}(\mathbb{K}) \neq 2$, then all quadratic field extensions of $\mathbb{K}$ are Galois.

\subsection{Remark} % Remark 3.5
Let $\mathbb{L}/\mathbb{K}$ be a Galois extension and $\mathbb{K}\subseteq \mathbb{E} \subseteq \mathbb{L}$ an intermediate field.
\begin{compactenum}
\item Then $\mathbb{L}/\mathbb{E}$ is Galois and $$\rm{Gal}(\mathbb{L}/\mathbb{E}) \leqslant \rm{Gal}(\mathbb{L}/\mathbb{K})$$
\item If $\mathbb{E}/\mathbb{K}$ is Galois, then $\rm{Gal}(\mathbb{L}/\mathbb{E}) \trianglelefteqslant \rm{Gal}(\mathbb{L}/\mathbb{K})$ is a normal subgroup and 
$$\slant{\rm{Gal}(\mathbb{L}/\mathbb{K})}{\rm{Gal}(\mathbb{L}/\mathbb{E})} \cong \rm{Gal}(\mathbb{E}/\mathbb{K})$$
\end{compactenum}
\textit{proof.}
\begin{compactenum}
\item Clearly $\mathbb{L}/\mathbb{E}$ is normal, since $\mathbb{L}$ is the splitting field for the same polynomials as in $\mathbb{L}/\mathbb{K}$.\\
Let now $\alpha \in \mathbb{L}$. Then the minimal polynomial $m_{\alpha}$ of $\alpha$ over $\mathbb{E}$ divides the minimal polynomial $m_{\alpha}'$ of $\alpha$ over $\mathbb{K}$, since $\mathbb{K} \subseteq \mathbb{E}$. Since $m_{\alpha}'$ has no multiple roots, $m_{\alpha}$ does not either and hence $\mathbb{L}/\mathbb{E}$ is separable and thus Galois.
\item Define
$$\rho: \rm{Gal}(\mathbb{L}/\mathbb{K}) \longrightarrow \rm{Gal}(\mathbb{E}/\mathbb{K}), \textrm{ }\sigma \mapsto \sigma |_{\mathbb{E}}$$
$\rho$ is well defined since $\sigma |_{\mathbb{E}} \in \rm{Hom}_{\mathbb{K}}(\mathbb{E}, \overline{\mathbb{K}})=\rm{Aut}_{\mathbb{K}}(\mathbb{E})=\textrm{Gal}(\mathbb{E}/\mathbb{K})$ as $\mathbb{E}/\mathbb{K}$ is Galois:
$$ [\mathbb{E}:\mathbb{K}] = \vert \rm{Aut}_{\mathbb{K}}(\mathbb{E}) \vert  \leqslant \vert \rm{Hom}_{\mathbb{K}}(\mathbb{E}, \overline{\mathbb{K}}) \vert \leqslant [\mathbb{E}:\mathbb{K}]$$
Moreover $\rho$ is surjective. For the kernel we get
$$\ker(\rho)=\{\sigma \in \rm{Gal}(\mathbb{L}/\mathbb{K}) \mid \sigma |_{\mathbb{E}} = \rm{id}_{\mathbb{E}} \} = \rm{Gal}(\mathbb{L}/\mathbb{E})$$
$\Longrightarrow \slant{\rm{Gal}(\mathbb{L}/\mathbb{K})}{\rm{Gal}(\mathbb{L}/\mathbb{E})} \cong \rm{Gal}(\mathbb{E}/\mathbb{K})$
\end{compactenum}

\titleformat{\subsection}{\normalfont\normalsize\bfseries}{}{0em}{#1 \thesubsection \quad \textnormal{\textit{(Main Theorem of Galois theory)}}}
\subsection{Theorem } %Theorem 3.6
\titleformat{\subsection}{\normalfont\normalsize\bfseries}{}{0em}{#1 \thesubsection}
Let $\mathbb{L}/\mathbb{K}$ be a finite Galois extension and $G:=\textrm{Gal}(\mathbb{L}/\mathbb{K})$. Then the subgroups $H\leqslant G$ correspond bijectively to the intermediate fields $\mathbb{K} \subseteq \mathbb{E} \subseteq \mathbb{L}$. Explicitly we have inverse maps
$$ \mathbb{E} \mapsto \textrm{Gal}(\mathbb{L}/\mathbb{E}) \leqslant G$$
$$ H \mapsto \mathbb{L}^H := \{ \alpha \in \mathbb{L} \mid \sigma(\alpha) =\alpha \textrm{ for all } \sigma \in H \}$$
\textit{proof.}\\
Clearly $\mathbb{L}^H$ is a field for any $H \leqslant G$. We now have to show
\begin{compactenum}
\item $\textrm{Gal}(\mathbb{L}/\mathbb{L}^H) = H$ for any $H \leqslant G$.
\item $\mathbb{L}^{\textrm{Gal}(\mathbb{L}/\mathbb{E})} = \mathbb{E}$ for any intermediate field $\mathbb{K} \subseteq \mathbb{E} \subseteq \mathbb{L}$.
\end{compactenum}
Theese prove the theorem.
\begin{compactenum}
\item We show both inclusion.
\begin{compactitem}
\item['$\supseteq$'] Clear by definition.
\item['$\subseteq$'] It suffices to show $\vert \textrm{Gal}(\mathbb{L}/\mathbb{L}^H) \vert \leqslant \vert H \vert$.
By 3.4(i) we have $$\vert \textrm{Gal}(\mathbb{L}/\mathbb{L}^H) \vert = [\mathbb{L}:\mathbb{L}^H]$$By theorem 2.5 $\mathbb{L}/\mathbb{L}^H$ is simple, say $\mathbb{L}=\mathbb{L}^H[\alpha]$. Define $$f=\prod_{\sigma \in H} (X-\sigma(\alpha))$$ with $\deg(f)=\vert H \vert$. Further, since $\textrm{id} \in H$, we have $f(\alpha)=0$. Clearly $f \in \mathbb{L}[X]$. We want to show that $f \in \mathbb{L}^H[X]$. Therefore for $\tau \in H$ define $$g^{\tau}:=\sum_{i=0}^n \tau(a_i) X^{i} \textrm{ for }g=\sum_{i=0}^n a_i X^{i}$$Then for $f$ as defined above we have
$$f^{\tau}=\prod_{\sigma \in H} \left(X- \tau\left(\sigma(\alpha)\right)\right)=\prod_{\sigma \in H} \left(X- \sigma(\alpha)\right)=f$$
hence $f \in \mathbb{L}^H[X]$. From $f(\alpha)=0$ we know that the minimal polynomial $m_{\alpha}$ of $\alpha$ over $\mathbb{L}^H$ divides $f$, thus 
$$ \vert \textrm{Gal}(\mathbb{L}/\mathbb{L}^H) \vert = [\mathbb{L}:\mathbb{L}^H] = \deg(m_{\alpha}) \leqslant \deg(f)= \vert H \vert $$
\end{compactitem}
\item Again we show both inclusions.
\begin{compactitem}
\item['$\supseteq$'] Clear by definition.
\item['$\subseteq$'] Let $H:= \textrm{Gal}(\mathbb{L}/\mathbb{E})$. Since $\mathbb{E} \subseteq \mathbb{L}^H$ it suffices to show $[\mathbb{L}^{H}:\mathbb{E}]=1$. Since $\mathbb{L}^{H}/\mathbb{E}$ is separable, this is equivalent to $[\mathbb{L}^H:\mathbb{E}]_s=1$.\\
Let now $\sigma \in \textrm{Hom}_{\mathbb{E}}(\mathbb{L}^H, \overline{\mathbb{K}})$. By proposition 2.6 we can extend $\sigma$ to some $$\tilde{\sigma}: \mathbb{L} \longrightarrow \overline{\mathbb{K}}$$ with $\tilde{\sigma}|_{\mathbb{L}^H}=\sigma$. Explicitly: Let $\mathbb{L}=\mathbb{L}^H[\alpha]$ and $f \in \mathbb{L}^H[X]$ its minimal polynomial. Choose a zero $\beta \in \overline{\mathbb{K}}$ of $f^{\sigma}$. Then by 2.6 there exists $\tilde{\sigma}: \mathbb{L} \longrightarrow \overline{\mathbb{K}}$ with $\tilde{\sigma}(\alpha)=\beta$ and $\tilde{\sigma}|_{\mathbb{L}^H} = \sigma$.\\
We get $\tilde{\sigma} \in \textrm{Gal}(\mathbb{L}/\mathbb{E})=H$ and $\sigma=\tilde{\sigma}|_{\mathbb{L}^H}=\textrm{id}_{\mathbb{E}}$ and hence $[\mathbb{L}^H:\mathbb{E}]=1$.
\end{compactitem}
\end{compactenum}

\subsection{Remark} %Remark 3.7
An intermediate field $\mathbb{K} \subseteq \mathbb{E} \subseteq \mathbb{L}$ is Galois over $\mathbb{K}$ if and only if $\textrm{Gal}(\mathbb{L}/\mathbb{E}) \trianglelefteqslant \textrm{Gal}(\mathbb{L}/\mathbb{K})$ is a normal subgroup.\\
\textit{proof.}
\begin{compactenum}
\item['$\Rightarrow$'] If $\mathbb{E}/\mathbb{K}$ is Galois, then $\rm{Gal}(\mathbb{L}/\mathbb{E})=\ker(\rho)$ is a normal subgroup by 3.5.
\item['$\Leftarrow$']  Conversely let $\textrm{Gal}(\mathbb{L}/\mathbb{E})=:H \trianglelefteqslant \rm{Gal}(\mathbb{L}/\mathbb{K})$ be a normal subgroup. By 3.4 it suffices to show $\rm{Hom}_{\mathbb{K}}(\mathbb{E}, \overline{\mathbb{K}})=\rm{Aut}_{\mathbb{K}}(\mathbb{E})$. Let now $\sigma \in \rm{Hom}_{\mathbb{K}}(\mathbb{E}, \overline{\mathbb{K}})$ and $\alpha \in \mathbb{E}$. Extend $\sigma$ to $\tilde{\sigma}:\mathbb{L} \longrightarrow \overline{\mathbb{K}}$. Then $\tilde{\sigma} \in \rm{Gal}(\mathbb{L}/\mathbb{K})$. By the theorem it suffices to show that $\sigma(\alpha) \in \mathbb{L}^{\rm{Gal}(\mathbb{L}/\mathbb{E})}=\mathbb{E}$, i.e. $\sigma(\mathbb{E}) \subseteq \mathbb{E}$. Let $\tau \in \rm{Gal}(\mathbb{L}/\mathbb{L}^H)$. Then by using the properties of normal subgroups we obtain
$$ \tau\left(\sigma(\alpha)\right)= \tau\left(\tilde{\sigma}(\alpha)\right)=\left(\tilde{\sigma} \circ \tau'\right)(\alpha)= \tilde{\sigma}(\alpha)=\sigma(\alpha)$$

\end{compactenum}

\subsection{Example} %Example 3.8
Let $\mathbb{K}=\mathbb{Q}$, $f= X^5-4X+2 \in \mathbb{Q}[X]$. Further let $\mathbb{L}=\mathbb{L}(f)$ be the splitting field of $f$ over $\mathbb{Q}$. What is $\rm{Gal}(\mathbb{L}/\mathbb{Q})$?.\\
We first want to show that $f$ is irreducible. But this immediately follows by By Eisenstein's criterion for irreducibility with $p=2$.\\
Thus $\mathbb{L}$ is an extension of $\slant{\mathbb{Q}}{\langle f \rangle}$. Therefore $[\mathbb{L}:\mathbb{Q}]$ is multiple of $[\slant{\mathbb{Q}}{\langle f \rangle}]=5$, hence $\vert \rm{Gal}(\mathbb{L}/\mathbb{Q}) \vert$ is divisible by $5$. By Lagrange's theorem we know that $\rm{Gal}(\mathbb{L}/\mathbb{Q})$ contains an element of order $5$.\\
Further note that $f$ has exactly $3$ zeroes in $\mathbb{R}$.
With $$\lim_{x \to \infty} f(x)=- \infty <0, \textrm{ }f(0)=2>0 \textrm{ }f(1)=-1<0 \textrm{ }\lim_{x \to - \infty} f(x)= \infty >0$$ we see by the intermediate value theorem that $f$ has at least $3$ zeroes.
Moreover $$f'=5X^4-4=5\cdot \left(X^4-\frac{4}{5}\right)=5 \cdot \left(X^2- \frac{2}{\sqrt{5}}\right) \cdot \left(X^2+ \frac{2}{\sqrt{5}}\right)$$Obviously, since the second factor has not real zeroes, the derivative of $f$ has $2$ zeroes, hence $f$ has at most $3$ zeroes. Together we obtain that $f$ has exactly $3$ zeroes. Since $f$ splits over $\mathbb{C}$, $f$ has two more conjugate zeroes in $\mathbb{C}$, say $\beta, \overline{\beta}$. Hence we know that the conjugation in $\mathbb{C}$ must be an element of $\rm{Gal}(\mathbb{L}/\mathbb{Q})$.\\
To sum it up, we know: $\rm{Gal}(\mathbb{L}/\mathbb{Q})$ is isomorphic to a subgroup of $S_5$, contains the conjugation, which corresponds to a transposition and moreover an element of order $5$, i.e. a $5-cycle$. But these two elements generate the whole group $S_5$. Hence we have
$\rm{Gal}(\mathbb{L}/\mathbb{Q}) \cong S_5$.

\titleformat{\subsection}{\normalfont\normalsize\bfseries}{}{0em}{#1 \thesubsection \quad \textnormal{\textit{(Cyclotomic fields)}}}
\subsection{Proposition } %Proposition 3.9
\titleformat{\subsection}{\normalfont\normalsize\bfseries}{}{0em}{#1 \thesubsection}
Let $\mathbb{K}$ be a field, $n \in \mathbb{N}$, $\rm{char}(\mathbb{K}) \nmid n$ and $\mathbb{L}_n$ the splitting field of the polynomial $f=X^n-1$.\\
Then $\mathbb{L}_n/\mathbb{K}$ is Galois and $\rm{Gal}(\mathbb{L}_n/\mathbb{K})$ is isomorphic to a subgroup of $\left(\slant{\mathbb{Z}}{n \mathbb{Z}}\right)^{\times}$.\\
\textit{proof.}\\
We have $f_n?=n X^{n-1}$ and $f?=0 \Leftrightarrow X=0$ but $f_n(0)\neq 0$, hence $f_n?$ and $f_n$ are coprime. Thus $f_n$ is separable. Since $\mathbb{L}_n$ is the splitting field of $f_n$ by definition, $\mathbb{L}_n/\mathbb{K}$ is normal, thus Galois.\\
The zeroes of $f_n$ form a group $\mu_n(\mathbb{K})$ under multiplication. By proposition 2.3 $\mu_n(\mathbb{K})$ is cyclic. Let $\zeta_n$ be a generator of $\mu_n(\mathbb{K})$. Define a map
$$ \chi_n: \rm{Gal}(\mathbb{L}_n/\mathbb{K}) \longrightarrow \left(\slant{\mathbb{Z}}{n \mathbb{Z}}\right)^{\times}\textrm{ }\sigma \mapsto k\textrm{ if }\sigma(\zeta_n)=\zeta_n^k$$
where $k$ is relatively coprime to $n$. We obtain that $\chi_n$ is a homomorphism of groups since for $\sigma_1. \sigma_2 \in \rm{Gal}(\mathbb{L}_n/\mathbb{K})$ we have $ \sigma_2 \sigma_1(\zeta_n)=\sigma_2 \left(\zeta_n^{k_1}\right)=\left(\zeta_n^{k_1}\right)^{k_2}=\zeta_n^{k_1 k_2}$
and hence 
\begin{center}$ \chi_n \left( \sigma_1 \sigma_2 \right)=k_1 \cdot k_2 = \chi_n(\sigma_1) \cdot \chi_n(\sigma_2)$\end{center}
Moreover $\chi_n$ is injective, since
\begin{center}$ \chi_n(\sigma)=1 \Leftrightarrow \sigma(\zeta_n)= \zeta_n \Leftrightarrow \sigma= \rm{id}$\end{center}
This proofs the proposition. Recall that $\vert \left( \slant{\mathbb{Z}}{n \mathbb{Z}}\right)^{\times} \vert = \phi(n)$ Where $\phi$ is Euler's $\phi$-function.
\newpage

%SECTION 4

\renewcommand*\thesection{\S\ \arabic{section}\quad}
\section{Solvability of equations by radicals}
\renewcommand*\thesection{\arabic{section}}

\subsection{Definition + Remark} %Deinition + Remark 4.1
Let $\mathbb{K}$ be a field, $f \in \mathbb{K}[X]$ separable.
\begin{compactenum}
\item Let $\mathbb{L}(f)$ be the splitting field of $f$ over $\mathbb{K}$. The \textit{Galois group of the equation }$f=0$ is defined by
$$ \rm{Gal}(f):=\rm{Gal}(\mathbb{L}(f)/\mathbb{K})$$
\item There exists an injective homomorphism of groups $\rm{Gal}(f)\longrightarrow S_n$ where $n:= \deg(f)$.
\item If $\mathbb{L}/\mathbb{K}$ is a finite, separable field extension, the $\rm{Aut}_{\mathbb{K}}(\mathbb{L})$ is isomorphic to a subgroup of $S_n$, where $n= [\mathbb{L}:\mathbb{K}]$.
\end{compactenum}
\textit{proof.}
\begin{compactenum}
\item[(ii)] Clear, since the automorphisms permute the zeroes of $f$, of which we have at most $n$.
\item[(iii)] We know $\mathbb{L}/\mathbb{K}$ is simple, say $\mathbb{L}=\mathbb{K}[\alpha]$ for some $\alpha \in \mathbb{L}$. Let $m_{\alpha}$ be the minimal polynomial of $\alpha$ over $\mathbb{K}$. Then $\deg(f)=n$. Every $\sigma \in \rm{Aut}(\mathbb{L}/\mathbb{K})$ maps $\alpha$ to a zero of $f$ and the same for every zero of $f$. Hence the claim follows.
\end{compactenum}

\subsection{Definition} % Definition 4.2
\begin{compactenum}
\item A simple field extension $\mathbb{L}=\mathbb{K}[\alpha]$ of a field $\mathbb{K}$ is called an \textit{elementary radical extension} if either
\begin{compactenum}
\item $\alpha$ is a root of unity, i.e. a zero of the polynomial $X^n-1$ for some $n \in \mathbb{N}$.
\item $\alpha$ is a root of $X^n- \gamma$ for some $\gamma \in \mathbb{K}, n \in \mathbb{N}$ such that $\rm{char}(\mathbb{K}) \nmid n$.
\item $\alpha$ is a root of $X^p-X-\gamma$ for somme $\gamma \in \mathbb{K}$ where $p=\rm{char}(\mathbb{K})$.
\end{compactenum}
In the following, we will denote $(1), (2)$ and $(3)$ as the three \textit{types} of elementary radical extensions.
\item  A finite field extension $\mathbb{L}/\mathbb{K}$ is called a \textit{radical extension}, if there is a field extension $\mathbb{L}'/\mathbb{L}$ and a chain of field extension 
$$ \mathbb{K}=\mathbb{L}_0 \subseteq \mathbb{L}_1 \subseteq \dots \subseteq \mathbb{L}_m = \mathbb{L}'$$
such that $\mathbb{L}_i/\mathbb{L}_{i-1}$ is an elementary radical extension for every $1 \leqslant i \leqslant m$.
\end{compactenum}

\subsection{Example} %Example 4.3
Let $\mathbb{K}=\mathbb{Q}$, $f=X^3-3X+1$.\\
The zeroes of $f$ (in $\mathbb{C}$) are 
$$ \alpha_1= \zeta + \zeta^{-1} \in \mathbb{R}, \textrm{ }\alpha_2=\zeta^2 + \zeta^{-2}\textrm{ and }\alpha_3=\zeta^4 + \zeta^{-4}$$
where $\zeta = e^{\frac{2\pi i}{9}}$ is a primitive ninth root of unity. We show this exemplarily for $\alpha_1$. We have
$$f(\alpha_1)\ =\ \left(\alpha_1^3-3 \alpha_1+1\right)\ =\ \zeta^3+ 3\zeta + 3\zeta^{-1}+\zeta^{-3} -3 \zeta - 3\zeta^{-1}+1\ =\ \zeta^3+ \zeta{-3}+1\ =\ 0$$
where we use $\zeta^{-3}= \overline{\zeta^{-3}}$ and since $z+\overline{z}=2 \cdot \mathfrak{Re}(z)$ for any $z \in \mathbb{C}$ we have
$$\zeta^3+ \zeta^{-3}\ =\ 2 \cdot \mathfrak{Re}\left(\zeta^3\right)\ =\ 2 \cdot \mathfrak{Re} \left(e^{\frac{2 \pi i}{3}}\right)\ =\ 2 \cdot \mathfrak{Re} \left(\cos\frac{2 \pi}{3} + i \cdot \sin \frac{2 \pi}{3}\right)\ 
=\ 2 \cdot \cos\frac{2 \pi}{3} \ =\ 2 \cdot \left(- \frac{1}{2}\right)\ =\ -1$$
Further we have $$\alpha_1^2=\zeta^2 + 2 \zeta^{-2}+2=\alpha_2+2,$$hence $\alpha_2 \in \mathbb{Q}(\alpha_1)$ and $\alpha_1 + \alpha_2 + \alpha_3 =0$, hence $\alpha_3 \in \mathbb{Q}(\alpha_1, \alpha_2) = \mathbb{Q}(\alpha_1)$.\\
This means that $\mathbb{Q}(\alpha_1)$ contains all the zeroes of $f$, i.e. is a splitting field of $f$.
We conclude $$\mathbb{Q}(\alpha_1) \cong \slant{\mathbb{Q}}{\langle f \rangle}, \qquad [\mathbb{Q}(\alpha_1):\mathbb{Q}]=3.$$
From the $f$ we see that $\mathbb{Q}(\alpha_1)/\mathbb{Q}$ is not an elementary radical extension, but a radical extension, since for $\mathbb{Q}(\zeta)$ we have $\mathbb{Q}(\alpha_1) \subseteq \mathbb{Q}(\zeta)$ and $\mathbb{Q}(\zeta)/\mathbb{Q}$ is an elementary radical extension. 

\subsection{Definition} %Definition 4.4
Let $\mathbb{K}$ be afield, $f \in \mathbb{K}[X]$ a separable, non-constant polynomial. We say \textit{f is solvable by radicals}, if the splitting field $\mathbb{L}(f)$ is a radical extension.

\subsection{Remark} %Remark 4.5
Let $\mathbb{L}/\mathbb{K}$ be an elementary field extension, referring to Definition 4.1 of type
\begin{compactenum}
\item $\mathbb{L}=\mathbb{K}[\zeta]$ for some root of unity $\zeta$ (primitive for some suitable $n \in \mathbb{N}, \textrm{char}(\mathbb{K}) \nmid n$). Then $\mathbb{L}/\mathbb{K}$ is Galois with abelian Galois group $$\rm{Gal}(\mathbb{L}/\mathbb{K}) \cong \left(\slant{\mathbb{Z}}{n \mathbb{Z}}\right)^{\times}$$
\item $\mathbb{L}=\mathbb{K}[\alpha]$ where $\alpha$ is a root of $X^n-\gamma$ for some $\gamma \in \mathbb{K}, n \in \mathbb{N}, \rm{char}(\mathbb{K}) \nmid n$. If $\mathbb{K}$ contains the $n$-th roots of unity, i.e. $\mu_n(\overline{\mathbb{K}})$, then $\mathbb{L}/\mathbb{K}$ is Galois with cyclic Galois group.
\item $\mathbb{L}=\mathbb{K}[\alpha]$, where $\alpha$ is a root of $X^p-X-\gamma$ for some $\gamma \in \mathbb{K}^{\times}$. Then $\mathbb{L}/\mathbb{K}$ is Galois with Galois group $$\rm{Gal}(\mathbb{L}/\mathbb{K}) \cong \slant{\mathbb{Z}}{p \mathbb{Z}}$$
\end{compactenum}
\textit{proof.}
\begin{compactenum}
\item We proved this in proposition 3.9.
\item Let $\zeta \in \mathbb{K}$ be a primitive $n$-th root of unity. Then $\zeta^{i} \cdot \alpha$ is a zero of $X^n-\gamma$, where we assume $n$ to be minimal sucht that $X^n-\gamma$ is irreducible. Then $\mathbb{L}$ contains all roots of $X^n-\gamma$, i.e. $\mathbb{L}/\mathbb{K}$ is normal and thus Galois with $$ \vert \textrm{Gal}(\mathbb{L}/\mathbb{K}) \vert = [\mathbb{L}:\mathbb{K}]=\textrm{deg}(X^n-\gamma)=n$$ Since the automorphism $\sigma \in \textrm{Gal}(\mathbb{L}/\mathbb{K})$ that maps $\alpha \mapsto \zeta \cdot \alpha$ has order $n$, $\textrm{Gal}(\mathbb{L}/\mathbb{K})$ is cyclic.
\item $f=X^p-X-\gamma$ has $p$ zeroes in $\mathbb{L}=\mathbb{K}[\alpha]$. Since 
$f(\alpha)=0$, we have $$f(\alpha+1)=\left(\alpha+1\right)^p-\left(\alpha+1\right)-\gamma=\alpha^p+1-\alpha-1-\gamma=\alpha^p-\alpha-\gamma =f(\alpha)=0$$
Hence $\mathbb{L}$ is the splitting field of $f$ and $\mathbb{L}/\mathbb{K}$ is normal. Moreover $f'=-1 \neq 0$, hence $\mathbb{L}/\mathbb{K}$ is separable and thus Galois with $$\vert \textrm{Gal}(\mathbb{L}/\mathbb{K}) \vert = [\mathbb{L}:\mathbb{K}]=\deg(f)=p$$
Further we obtain that $\textrm{Gal}(\mathbb{L}/\mathbb{K})\ni \sigma: \ \alpha \mapsto \alpha +1$ has order $p$, hence $\textrm{Gal}(\mathbb{L}/\mathbb{K})$ is cyclic and thus $$\textrm{Gal}(\mathbb{L}/\mathbb{K}) \cong \slant{\mathbb{Z}}{p \mathbb{Z}}$$
\end{compactenum}

\subsection{Remark} %Remark 4.6
Let $\mathbb{L}/\mathbb{K}$ be an elementary radical extension of type (ii), i.e. $\mathbb{L}=\mathbb{K}[\alpha]$, where $\alpha$ is the root of $f=X^n-\gamma$ for some $\gamma \in \mathbb{K}, n \geqslant 1, \rm{char}(\mathbb{K}) \nmid n$. $X^n-\gamma$ is irreducible\\
Let $\mathbb{F}$ be a splitting field of $X^n-1$ over $\mathbb{K}$ and $\mathbb{L}\mathbb{F}=\mathbb{K}(\alpha, \zeta)$ be the \textit{compositum} of $\mathbb{L}$ and $\mathbb{F}$, i.e. the smallest subfield of $\overline{\mathbb{K}}$ containing $\mathbb{L}$ and $\mathbb{F}$. 
$$\begin{xy}
\xymatrix{
&\tilde{\mathbb{L}}=\mathbb{L}\mathbb{F} &\\
\mathbb{L}=\mathbb{K}[\alpha] \ar@{-}[ru] && \mathbb{K}[\zeta]=\mathbb{F} \ar@{-}[lu] \\
& \mathbb{K} \ar@{-}[lu] \ar@{-}[ru]&
}
\end{xy}$$
$\tilde{\mathbb{L}}$ is a splitting field of $X^n-\gamma$ over $\mathbb{F}$, hence $\tilde{\mathbb{L}}/\mathbb{F}$ is Galois and by 4.4(ii), $\textrm{Gal}(\tilde{\mathbb{L}}/\mathbb{F})$ is cyclic.\\
Moreover $\mathbb{F}/\mathbb{K}$ is Galois and $\rm{Gal}(\mathbb{F}/\mathbb{K})$ is abelian. Hence $\tilde{\mathbb{L}}/\mathbb{K}$ is Galois and\\$$\slant{\textrm{Gal}(\tilde{\mathbb{L}}/\mathbb{K})}{\textrm{Gal}(\tilde{\mathbb{L}}/\mathbb{F})} \cong \textrm{Gal}(\mathbb{F}/\mathbb{K})$$
i.e. we have a short exact sequence
$$1 \longrightarrow \underbrace{\rm{Gal}(\tilde{\mathbb{L}}/\mathbb{F})}_{\textit{cyclic}} \overset{\textrm{inj.}}{\longrightarrow} \textrm{Gal}(\tilde{\mathbb{L}}/\mathbb{K}) \overset{\textrm{surj.}}{\longrightarrow} \underbrace{\textrm{Gal}(\mathbb{F}/\mathbb{K})}_{\textit{abelian}} \longrightarrow 1$$

\titleformat{\subsection}{\normalfont\normalsize\bfseries}{}{0em}{#1}
\subsection*{Example} %Example
\titleformat{\subsection}{\normalfont\normalsize\bfseries}{}{0em}{#1 \thesubsection}
Let $\mathbb{K}=\mathbb{Q}$, $f=X^3-2$. Then $\mathbb{L}=\mathbb{Q}[\alpha]$ with $\alpha=\sqrt[3]{2}$ and $\mathbb{F}=\mathbb{Q}[\zeta]$ with $\zeta=e^{\frac{2\pi}{3}}$.\\
Then $\tilde{\mathbb{L}}=\mathbb{L}(f)$ with $[\tilde{\mathbb{L}}:\mathbb{Q}]=6$ We have $$\rm{Gal}(\tilde{\mathbb{L}}/\mathbb{F}) \cong \slant{\mathbb{Z}}{3 \mathbb{Z}}, \textrm{ } \rm{Gal}(\mathbb{F}/\mathbb{K}) \cong \slant{\mathbb{Z}}{2 \mathbb{Z}}, \textrm{ } \rm{Gal}(\tilde{\mathbb{L}}/\mathbb{Q}) \cong S_3$$

\subsection{Definition} %Definition 4.7
A group $G$ is called \textit{solvable}, if there exists a chain of subgroups $$1\ =\ G_0 \ \triangleleft G_1\ \triangleleft \ \dots\ \triangleleft \ G_n=G$$ where $G_{i-1} \triangleleft G_i$ is a normal subgroup and $\slant{G_i}{G_{i-1}}$ is abelian for all $1 \leqslant i \leqslant n$.

\titleformat{\subsection}{\normalfont\normalsize\bfseries}{}{0em}{#1}
\subsection*{Example} %Example
\titleformat{\subsection}{\normalfont\normalsize\bfseries}{}{0em}{#1 \thesubsection}
\begin{compactenum}
\item Every abelian group is solvable.
\item $S_4$ is solvable by $$1 \ \triangleleft\ V_4 \ \triangleleft \ A_4 \ \triangleleft\ S_4$$ where $V_4=\{\rm{id}, (12)(34), (13)(24), (14)(23)\}$. For the quotients we have $$\slant{V_4}{\{1\}} \cong \slant{\mathbb{Z}}{2 \mathbb{Z}} \times \slant{\mathbb{Z}}{2 \mathbb{Z}}, \qquad \slant{A_4}{V_4}\cong \slant{\mathbb{Z}}{3 \mathbb{Z}}, \qquad \slant{S_4}{A_4} \cong \slant{\mathbb{Z}}{2 \mathbb{Z}}$$
\item $S_5$ is not solvable, since $A_5$ is simple (EAZ 6.6) but the quotient $\slant{A_5}{\{1\}}$ is not abelian.
\item If $G$, $H$ are solvable groups, then the direct product $G \times H$ is solvable.
\end{compactenum}

\subsection{Proposition} % Proposition 4.8
\begin{compactenum}
\item Let $G$ be a solvable group. Then
\begin{compactenum}
\item Every subgroup $H \leqslant G$ is solvable.
\item Every homomorphic image of $G$ is solvable.
\end{compactenum}
\item Let $$1 \longrightarrow G' \longrightarrow G \longrightarrow G'' \longrightarrow 1$$be a short exact sequence. Then $G$ is solvable if and only if $G'$ and $G''$ are solvable.
\end{compactenum}
\textit{proof.}
\begin{compactenum}
\item \begin{compactenum}
\item Let $G$ be solvable, i.e. we have a chain $1=G_0 \triangleleft G_1 \triangleleft \dots \triangleleft G_n=G$. Let $G' \leqslant  G$ a subgroup. Then $$1 \ \triangleleft \ G_1 \cap G' \ \triangleleft \ \dots\ \triangleleft \ G_n \cap G' \ = \ G'$$ is a chain of subgroups of $G'$ and we have $G_i \cap G' \triangleleft G_{i+1} \cap G'$ and moreover $$\slant{\left(G_{i+1} \cap G'\right)}{\left(G_{i} \cap G'\right)} \cong G_i \slant{\left(G_{i+1} \cap G'\right)}{G_i} \leqslant \slant{G_{i+1}}{G_i}$$
Hence we have abelian quotients and $G'$ is solvable.
\item Let $H$ be a group and $\phi: G \longrightarrow H$ be a surjective homomorphism of groups. Let $$1 \ \triangleleft \ G_1 \ \triangleleft \ \dots\ \triangleleft \ G_n\ =\ G$$ Let $H_i:=\phi(G_i)$. Then $H_i$ is normal in $H_{i+1}$. It remains to show that the quotients are abelian. Consider
$$\begin{xy}
\xymatrix{
G_i \ar[dd]_{\phi} \ar[rr] && G_{i+1} \ar[dd]_{\phi} \ar[rr]^{\pi_G} \ar@{-->}[rrdd]_{\tilde{\phi}} && \slant{G_{i+1}}{G_i} \ar[dd]^{\overline{\phi}} \\ &&&& \\
H_i \ar[rr] && H_{i+1} \ar[rr]_{\pi_H} && \slant{H_{i+1}}{H_i}
}
\end{xy}$$
(We have $G_i \subseteq \ker(\tilde{\phi})$, since $\phi(G_i)=H_i=\ker(\pi_H)$.
Hence $\tilde{\phi}$ factors to $$\overline{\phi}: \underbrace{\slant{G_{i+1}}{G_i}}_{\textit{abelian}} \underbrace{\longrightarrow}_{\Rightarrow} \underbrace{\slant{H_{i+1}}{H_i}}_{\textit{abelian!}}$$
And we get $\overline{\phi}(a) \overline{\phi}(b)=\overline{\phi}(ab)=\overline{\phi}(ba)=\overline{\phi}(b) \overline{\phi}(a)$, hence the quotient is abelian and $H=\phi(G)$ is solvable.
\end{compactenum}
\item \begin{compactitem}
\item['$\Rightarrow$'] Clear.
\item['$\Leftarrow$'] Let $$1 \triangleleft G_1 \triangleleft \dots \triangleleft G_m =G'$$ and $$1 \triangleleft H_{m+1} \triangleleft \dots \triangleleft H_{m+k}=G''$$ chains of subgroups with abelian quotients. Define $$G_i:= \pi^{-1} \left(H_i\right)_{m+1 \leqslant i \leqslant m+k}, \textrm{ } \pi:G \longrightarrow G''$$
Then $G_i$ is normal in $G_{i+1}$ and we have 
\noindent
$$G_{m+0}=\pi^{-1}(\{1\})=G'=G_m$$
\noindent
For $m+1 \leqslant i \leqslant m+k$ we have $$\slant{G_{i+1}}{G_i} = \pi^{-1}\left(\slant{H_{i+1}}{H_i}\right) \cong \slant{H_{i+1}}{H_i}$$ and hence the chain $$1 \triangleleft G_1 \triangleleft \dots \triangleleft G_m=G' \triangleleft G_{m+1} \triangleleft \dots \triangleleft G_{m+k}=G$$ reveals the solvability of $G$.
\end{compactitem}
\end{compactenum}

\subsection{Lemma}%Lemma 4.9
A finite separable field extension $\mathbb{L}/\mathbb{K}$ is a radical extension if and only if there exists a finite Galois extension $\mathbb{L}'/\mathbb{K}$, $\mathbb{L} \subseteq \mathbb{L}'$ such that $\rm{Gal}(\mathbb{L}'/\mathbb{K})$ is solvable.\\
\textit{proof.}
\begin{compactitem}
\item['$\Rightarrow$'] Let $$\mathbb{K}=\mathbb{K}_0=\mathbb{L}_0 \subseteq \mathbb{L}_1 \subseteq \dots \subseteq \mathbb{L}_n$$ a chain as in definition 4.7 with $\mathbb{L} \subseteq \mathbb{L}_n$. we prove the statement by induction.
\begin{compactitem}
\item[\bf{n=1}] This is exactly remark 4.5, 4.6
\item[\bf{n>1}] By induction hypothesis $\mathbb{L}_{n-1}/\mathbb{K}$ is solvable. Moreover $\mathbb{L}_n/\mathbb{L}_{n-1}$ is solvable, too. This is equivalent to the fact, that\\
$\mathbb{L}_{n-1}$ is contained in a Galois extension $\tilde{\mathbb{L}}_{n-1}/\mathbb{K}$ such that $\rm{Gal}(\tilde{\mathbb{L}}/\mathbb{K})$ is solvable and\\
$\mathbb{L}_n$ is contained in a Galois extension $\tilde{\mathbb{L}}/\mathbb{L}_{n-1}$ such that $\rm{Gal}(\tilde{\mathbb{L}}/\mathbb{L}_{n-1})$ is solvable.\\
We have a diagramm
\begin{center}
\begin{tabular}{*{7}{l}}
&& $\tilde{\mathbb{L}}_{n-1}$ & $\subseteq$ &$ \tilde{\mathbb{L}}\mathbb{L}_{n-1}$ &$ := $& $\mathbb{M} $\\
&& \tabrotate{$\subseteq$} &&&& \tabrotate{$\subseteq$}\\
$\mathbb{K}$ & $\subseteq$ &$ \mathbb{L}_{n-1} $&$ \subseteq$ &$ \mathbb{L}_n $&$ \subseteq $& $\tilde{\mathbb{L}} $\\
\end{tabular}\\
\end{center}
We obtain, that $\mathbb{M}$ is Galois over $\mathbb{L}_{n-1}$, since $\tilde{\mathbb{L}}, \tilde{\mathbb{L}}_{n-1}$ are Galois over $\mathbb{L}_{n-1}$, hence by
$$\iota: \rm{Gal}(\mathbb{M}/\tilde{\mathbb{L}}_{n-1}) \longrightarrow \rm{Gal}(\tilde{\mathbb{L}}/\mathbb{L}_{n-1}), \textrm{ } \sigma \mapsto \sigma |_{\tilde{\mathbb{L}}}$$ an injective homomorphism of groups is given, hence $$\rm{Gal}(\mathbb{M}/\tilde{\mathbb{L}}_{n-1}) \leqslant \rm{Gal}(\tilde{\mathbb{L}}/\mathbb{L}_{n-1})$$ is solvable as a subgroup of a solvable group.\\
Let now $\tilde{\mathbb{M}}/\mathbb{M}$ be a minimal extension, such that $\tilde{\mathbb{M}}/\mathbb{K}$ is Galois. Explicitly, $\tilde{\mathbb{M}}$ is defined as the \textit{normal hull} of $\mathbb{M}$, i.e. the splitting field of the minimal polynomial of a primitive element of $\mathbb{M}/\mathbb{K}$.\\
Now we want to show that $\rm{Gal}(\mathbb{M}/\mathbb{K}$ is solvable.This finishes the proof of the sufficiency of our Lemma. Consider the short exact sequence $$1 \longrightarrow \rm{Gal}(\tilde{\mathbb{M}}/\tilde{\mathbb{L}}_{n-1}) \longrightarrow \rm{Gal}(\mathbb{M}/\mathbb{K}) \longrightarrow \rm{Gal}(\tilde{\mathbb{L}}_{n-1}/\mathbb{K}) \longrightarrow 1$$ By proposition 4.8 and our induction hypothesis it suffices to show that $\rm{Gal}(\tilde{\mathbb{M}}/\tilde{\mathbb{L}}_{n-1})$ is solvable. Therefore observe that $\tilde{\mathbb{M}}$ is generated over $\mathbb{K}$ by the $\sigma(\mathbb{K})$ for $\sigma \in \rm{Hom}_{\mathbb{K}}(\mathbb{M}, \overline{\mathbb{K}})$, where $\overline{\mathbb{K}}$ denotes an algebraic closure of $\mathbb{K}$. For any $\sigma \in \rm{Hom}_{\mathbb{K}}(\mathbb{M},\overline{\mathbb{K}})$, $\sigma(\mathbb{M})/\sigma(\mathbb{L}_{n-1})=\sigma(\mathbb{M})/\tilde{\mathbb{L}}_{n-1}$ is Galois. Hence
$$\Phi: \rm{Gal}(\tilde{\mathbb{M}}/\tilde{\mathbb{L}}_{n-1}) \longrightarrow \prod_{\sigma \in \rm{Hom}_{\mathbb{K}}(\mathbb{M}, \overline{\mathbb{K}})} \rm{Gal}\left(\sigma(\mathbb{M})/\tilde{\mathbb{L}}_{n-1}\right), \textrm{ } \tau \mapsto \left(\tau |_{\sigma(\mathbb{M})}\right)_{\sigma}$$
is injective.
\\Hence $\rm{Gal}(\tilde{\mathbb{M}}/\tilde{\mathbb{L}}_{n-1})$ is solvable as a subgroup of a product of solvable groups.
\end{compactitem}
\item['$\Leftarrow$'] Let now $\tilde{\mathbb{L}}/\mathbb{L}$ finite such that $\rm{Gal}(\tilde{\mathbb{L}}/\mathbb{K})$ is solvable. Let 
$$1 \triangleleft \ G_1\ \triangleleft\ \dots \ \triangleleft\ G_n \ =\ G$$ be a chain of subgroups as in definition 4.7. By the main theorem we have bijectively correspond intermediate fields
$$ \tilde{\mathbb{L}}=\mathbb{L}_n \supseteq \mathbb{L}_{n-1} \supseteq \dots \supseteq \mathbb{L}_0 = \mathbb{K}$$
where $\mathbb{L}_{i+1}/\mathbb{L}_i$ is Galois and $\rm{Gal}(\mathbb{L}_{i+1}/\mathbb{L}) \cong \slant{\mathbb{Z}}{p \mathbb{Z}}$ for all $1 \leqslant i \leqslant n-1$. We now have to differ between three cases.
\begin{compactitem}
\item[\textbf{case 1}] $p_i=\rm{char}(\mathbb{K})$. Then $\mathbb{L}_{i+1}/\mathbb{L}_i$ is an elementary radical extension of type (iii), i.e. $\mathbb{L}/\mathbb{K}$ is a radical extension.
\item[\textbf{case 2}] $p_i \neq \rm{char}(\mathbb{K})$ \textit{and} $\mathbb{L}_{i}$ contains a primitive $p_i$-th root of unity. Then $\mathbb{L}_{i+1}/\mathbb{L}_i$ is an elementary radical extension of type (ii), i.e. $\mathbb{L}/\mathbb{K}$ is a radical extension.
\item[\textbf{case 3}] $p_i \neq \rm{char}(\mathbb{K})$ \textit{and} $\mathbb{L}_{i}$ does not contain any primitive $p_i$-th root of unity. Then define
$$d:=\prod_{p \in \mathbb{P}, p \mid \vert G \vert} p$$
And let $\mathbb{F}$ be the splitting field of $X^d-1$ over $\mathbb{K}$.  Then $\mathbb{F}/\mathbb{K}$ is an elementary radical extension of type (i).\\
Let $\mathbb{L}':=\tilde{\mathbb{L}}\mathbb{F}$ be the composite of $\tilde{\mathbb{L}}$ and $\mathbb{F}$ in $\overline{\mathbb{K}}$. Then $\mathbb{L}'/\mathbb{F}$ is Galois by remark 4.5. Let $G'=\rm{Gal}(\mathbb{L}'/\mathbb{F})$. Consider the map
$$\Psi: \rm{Gal}(\mathbb{L}'/\mathbb{F}) \longrightarrow \rm{Gal}(\tilde{\mathbb{L}}/\mathbb{K}), \textrm{ } \sigma \mapsto \sigma |_{\tilde{\mathbb{L}}}$$
$\Psi$ is a well defined injective homomorphism of groups, hence $\rm{Gal}(\mathbb{L}'/\mathbb{F}) \leqslant \rm{Gal}(\tilde{\mathbb{L}}/\mathbb{K})$ is solvable as a subgroup of a solvable group. Let 
$$1 \triangleleft \ G_1\ \triangleleft\ \dots \ \triangleleft\ G_n \ =\ G'$$
a chain of subgroups as in definition 4.7. Let further be 
$$\mathbb{K} \subseteq \mathbb{F}=\mathbb{L}_0 \subseteq \mathbb{L}_1 \subseteq \dots \subseteq \mathbb{L}_n=\mathbb{L}'$$
be the corresponding chain of intermediate fields, i.e $\mathbb{L}_{i}/\mathbb{L}_{i-1}$ is Galois and $\rm{Gal}(\mathbb{L}_{i}/\mathbb{L}_{i-1}) \cong \slant{\mathbb{Z}}{p \mathbb{Z}}$ for $1 \leqslant i \leqslant n$.
Hence, $\mathbb{L}_{i}/\mathbb{L}_{i-1}$ is a radical extension of type (ii).
Thus $\mathbb{L}/\mathbb{K}$ is a radical extension, which finishes the proof.
\end{compactitem}
\end{compactitem}

\subsection{Theorem}%Theorem 4.10
Let $f \in \mathbb{K}[X]$ be a separable non-constant polynomial. Then $f$ is solvable by radicals if and only if $\rm{Gal}(f)=\rm{Gal}(\mathbb{L}(f)/\mathbb{K})$ is solvable.\\
\textit{proof.}\\
Let $f$ be solvable by radicals, i.e. $\mathbb{L}(f)/\mathbb{K}$ be a radical field extension.\\
$\Longleftrightarrow \ $ $\mathbb{L}(f)$ is contained in some Galois extension $\tilde{\mathbb{L}}/\mathbb{K}$ and $\rm{Gal}(\tilde{\mathbb{L}}/\mathbb{K})$ is solvable.\\
$\Longleftrightarrow \ $ In $\mathbb{K}\subseteq \mathbb{L}(f) \subseteq \tilde{\mathbb{L}}$ all extensions are Galois.\\
$\overset{3.5}{\Longleftrightarrow} \ \rm{Gal}(\mathbb{L}(f)/\mathbb{K}) \cong \slant{\rm{Gal}(\tilde{\mathbb{L}}/\mathbb{K})}{\rm{Gal}(\tilde{\mathbb{L}}/\mathbb{L}(f))}$\\
$\overset{4.8}{\Longleftrightarrow} \ \rm{Gal}(\mathbb{L}(f)/\mathbb{K})$ is solvable.

\subsection{Theorem} %Theorem 4.11
Let $G$ be a group, $\mathbb{K}$ a field. Then the subset $\rm{Hom}(G,\mathbb{K}^{\times}) \subseteq \rm{Maps}(G,\mathbb{K})$ is linearly independant in the $\mathbb{K}$-vector space $\rm{Maps}(G,\mathbb{K})$. \\
\textit{proof.}\\
Suppose $\rm{Hom}(G,\mathbb{K}^{\times})$ is linearily dependant. Then let $n>0$ minimal, such that there exist distinct elements $\chi_1, \dots \chi_n \in \rm{Hom}(G,\mathbb{K}^{\times})$ and $\lambda_1, \dots \lambda_n \in \mathbb{K}^{\times}$ such that
$$ \sum_{i=0}^n \lambda_i \chi_i =0.$$
The $\chi_i$ are called \textit{characters}. Clearly we have $n \geqslant 2$. Choose $g \in G$ such that $\chi_1(g) \neq \chi_2(g)$.
For any $h \in G$ we have
$$0 = \sum_{i=0}^n \lambda_i \chi_i (gh)= \sum_{i=0}^n \underbrace{\lambda_i \chi_i(g)}_{=:\mu_i} \chi_i(h)=\sum_{i=0}^n \mu_i \chi_i(h)$$
Then we get 
$$0 =\sum_{i=0}^n \mu_i \chi_i(h)= \sum_{i=0}^n \lambda_i \chi_i(g)\chi_i(h) \textrm{ } \Rightarrow \sum_{i=0}^n \underbrace{\left(\mu_i-\lambda_i \chi_1(g)\right)}_{=:\nu_i}\chi_i(h)=0$$
Consider
$$\nu_1=\mu_1-\lambda_1 \chi_1(g)=\lambda_1 \chi_1(g)-\lambda_1 \chi_1(g)=0$$
$$\nu_2=\mu_2-\lambda_2 \chi_1(g)=\lambda_2 \chi_2(g)-\lambda_2 \chi_1(g)=\underbrace{\lambda_2}_{\neq 0} \cdot \underbrace{\left(\chi_2(g)-\chi_1(g)\right)}_{\neq 0} \neq 0$$
Hence $\chi_2, \dots \chi_n$ are linearily dependent. This is a contradiction to the minimality of $n$.
\subsection{Proposition} %Proposition 4.12
Let $\mathbb{L}/\mathbb{K}$ be a Galois extension such that $G:= \rm{Gal}(\mathbb{L}/\mathbb{K})=\langle \sigma \rangle$ is cyclic of order $d$ for some $\sigma \in G$, where $\rm{char}(\mathbb{K}) \nmid d$. Let $\zeta_d \in \mathbb{K}$ be a primitive $d$-th root of unity.\\
Then there exsits $\alpha \in \mathbb{L}^{\times}$ such that $\sigma(\alpha)=\zeta \cdot \alpha$.\\
\textit{proof.}\\
Let $$f:\mathbb{L} \longrightarrow \mathbb{L}, \qquad f(X)\ =\ \sum_{i=0}^{d-1} \zeta^{-i} \cdot \sigma^{i}(X)$$
Applying Theorem 4.10 on $G=\mathbb{L}^{\times}$ and $\mathbb{K}=\mathbb{L}$ shows $f\neq 0$.
Then let $\gamma \in \mathbb{L}$ such that $\alpha:=f(\gamma) \neq 0$. Then we have
\begin{alignat*}{13}
\sigma(\alpha)=\sigma \left(f(\gamma)\right) &=&& \ \sigma \left(\sum_{i=0}^{d-1} \zeta^{-i} \cdot \sigma^{i}(\gamma)\right) \ &&=&&\ \sum_{i=0}^{d-1} \zeta^{-i} \cdot \sigma^{i+1}(\gamma)\ =\ \zeta \cdot \sum_{i=0}^{d-1}\zeta^{-(i+1)}\cdot \sigma^{i+1}(\gamma)\\
&= && \ \zeta \cdot \sum_{i=1}^d \zeta^{-i} \cdot \sigma^{i}(\gamma) \ &&= &&\ \zeta \left(\left(\sum_{i=1}^{d-1}\zeta^{-i} \cdot \sigma^{i}(\gamma)\right)+\gamma \right)  \\
&=&&\ \zeta \cdot f(\gamma) \ =\ \zeta \cdot \alpha 
\end{alignat*}
\textit{Remark:} The claim follows from Proposition 5.2 by insertig $\beta=\zeta$. 

\subsection{Corollary} %Corollary 4.13
Let $\mathbb{L}/\mathbb{K}$ be a Galois extension, such that $G:=\rm{Gal}(\mathbb{L}/\mathbb{K})=\langle \sigma \rangle$ is cyclic of order $d$ for some $\sigma \in G$, where $\rm{char}(\mathbb{K}) \nmid d$. Assume $\mathbb{K}$ contains a primitive $d$-th root of unity.\\ Then $\mathbb{L}/\mathbb{K}$ is an elementary radical extension of type (ii).\\
\textit{proof.}\\
Let $\zeta_d\in \mathbb{K}$ be a primitive $d$-th root of unity and $\alpha \in \mathbb{L}^{\times}$ such that $\sigma(\alpha)=\zeta \cdot \alpha$.\\
We have $$\sigma^{i}(\alpha)=\zeta^{i} \cdot \alpha \qquad \textrm{ for }1 \leqslant i \leqslant d$$ The minimal polynomial of $\alpha$ over $\mathbb{K}$ has at least $d$ zeroes, namely $\alpha, \sigma(\alpha), \dots \sigma^{d-1}(\alpha)$. Thus $\mathbb{L}=\mathbb{K}[\alpha]$.\\
Moreover we have $$\sigma(\alpha^d)=\left(\sigma(\alpha)\right)^d=\left(\zeta \cdot \alpha \right)^d=\alpha^d,$$hence $$\alpha^d \in \mathbb{L}^{\langle \sigma \rangle}=\mathbb{L}^{\rm{Gal}(\mathbb{L}/\mathbb{K})}=\mathbb{K}$$
where the last equation follows by the main theorem.\\
Define $\gamma:=\alpha^d$. Then the minimal polynomial of  $\alpha$ over $\mathbb{K}$ is $X^d-\gamma \in \mathbb{K}[X]$, which proves the claim.

\subsection{Proposition} %Proposition 4.14
Let $\mathbb{L}/\mathbb{K}$ be a Galois extension of degree $p=\rm{char}(\mathbb{K})$ with cyclic Galois group $\rm{Gal}(\mathbb{L}/\mathbb{K}) \cong \slant{\mathbb{Z}}{p \mathbb{Z}} = \langle \sigma \rangle$.
Then there exists $\alpha \in \mathbb{L}^{\times}$ such that $\sigma(\alpha)=\alpha+1$.\\
\textit{proof.}\\
The proof follows by Proposition 5.4 by setting $\beta=-1$. 

\subsection{Corollary} %Corollary 4.15
Let $\mathbb{L}/\mathbb{K}$ be a Galois extension of degree $p=\rm{char}(\mathbb{K})$ with cyclic Galois group $\rm{Gal}(\mathbb{L}/\mathbb{K}) \cong \slant{\mathbb{Z}}{p \mathbb{Z}}=\langle \sigma \rangle$.\\ Then $\mathbb{L}/\mathbb{K}$ is an elementary radical extension of type (iii).\\
\textit{proof.}\\
Let $\alpha \in \mathbb{L}^{\times}$ such that $\sigma(\alpha)=\alpha+1$.\\
We have $$\sigma^{i}(\alpha)=\alpha+i \qquad \textrm{ for }1 \leqslant i \leqslant p$$ Thus we have $\mathbb{L}=\mathbb{K}[\alpha]$.\\
Moreover we have
$$\sigma(\alpha^p-\alpha)=\sigma^p(\alpha)-\sigma(\alpha)=\left(\alpha+1\right)^p-(\alpha+1)=\alpha^p+1-\alpha-1=\alpha^p-\alpha$$
Thus again we have $\alpha^p \in \mathbb{K}$. Define $\gamma:= \alpha^p-\alpha$. Then the minimal polynomial of $\alpha$ over $\mathbb{K}$ is $X^p-X-\gamma$, which proves the claim.

%SECTION 5 

\renewcommand*\thesection{\S\ \arabic{section}\quad}
\section{Norm and trace}
\renewcommand*\thesection{\arabic{section}}

\subsection{Definition + Remark} %Definition + Remark 5.1
Let $\mathbb{L}/\mathbb{K}$ be a finite separable field extension, $[\mathbb{L}:\mathbb{K}]=n$. Let $\rm{Hom}_{\mathbb{K}}(\mathbb{L},\overline{\mathbb{K}})=\{\sigma_1, \dots \sigma_n\}$.
\begin{compactenum}
\item For $\alpha \in\mathbb{L}$ we define the \textit{norm} of $\alpha$ over $\mathbb{K}$ by
$$N_{\mathbb{L}/\mathbb{K}}(\alpha):=\prod_{i=1}^n \sigma_i(\alpha)$$
\item $N_{\mathbb{L}/\mathbb{K}} \in \mathbb{K}$ for all $\alpha \in \mathbb{L}$.
\item $N_{\mathbb{L}/\mathbb{K}}: \mathbb{L}^{\times} \longrightarrow \mathbb{K}^{\times}$ is a homomorphism of groups.
\end{compactenum}
\textit{proof.}
\begin{compactenum}
\item[(ii)] Let $\alpha \in \mathbb{L}$.
Assume first that $\mathbb{L}/\mathbb{K}$ is Galois. Then $\rm{Hom}_{\mathbb{K}}(\mathbb{L},\overline{\mathbb{K}})=\rm{Aut}_{\mathbb{K}}(\mathbb{L})=\rm{Gal}(\mathbb{L}/\mathbb{K})$.
For $\tau \in \rm{Gal}(\mathbb{L}/\mathbb{K})$ we have
$$\tau\left(N_{\mathbb{L}/\mathbb{K}}\right)=\tau \left( \prod_{i=1}^n \sigma_i(\alpha) \right) = \prod_{i=1}^n \underbrace{(\tau \sigma_i)}_{\in \rm{Gal}(\mathbb{L}/\mathbb{K})}(\alpha)=N_{\mathbb{L}/\mathbb{K}}$$
Hence $N_{\mathbb{L}/\mathbb{K}} \in \mathbb{L}^{\rm{Gal}(\mathbb{L}/\mathbb{K})} = \mathbb{K}$.
Now consider the general case. Let $\tilde{\mathbb{L}} \supseteq \mathbb{L}$ be the normal hull of $\mathbb{L}$ over $\mathbb{K}$.  Recall that $\tilde{\mathbb{L}}$ is the composition of the $\sigma_i(\mathbb{L})$, i.e. $$\tilde{\mathbb{L}}=\prod_{i=1}^n \sigma_i(\mathbb{L})$$Then $\tilde{\mathbb{L}}/\mathbb{K}$ is Galois an for $\tau \in \rm{Gal}(\tilde{\mathbb{L}}/\mathbb{K})$ we have
$$\tau\left(N_{\mathbb{L}/\mathbb{K}}(\alpha)\right)=\prod_{i=1}^n \underbrace{(\tau \sigma_i)}_{\in \rm{Hom}_{\mathbb{K}}(\mathbb{L},\overline{\mathbb{K}})}(\alpha)=\prod_{i=1}^n \sigma_i(\alpha)=N_{\mathbb{L}/\mathbb{K}}(\alpha)$$
Hence $N_{\mathbb{L}/\mathbb{K}}(\alpha) \in \tilde{\mathbb{L}}^{\rm{Gal}(\tilde{\mathbb{L}}/\mathbb{K})}=\mathbb{K}$.
\item[(iii)] We have $N_{\mathbb{L}/\mathbb{K}}(\alpha)=0 \Longleftrightarrow \sigma_i(\alpha)=0 \textrm{ for some } 1\leqslant i \leqslant n \Leftrightarrow \alpha=0$.\\
Moreover
\begin{alignat*}{5}
N_{\mathbb{L}/\mathbb{K}}(\alpha \cdot \beta)\ &=&& \ \prod_{i=1}^n \sigma_i(\alpha \beta)=\prod_{i=1}^n \sigma_1(\alpha) \sigma_i(\beta)=\left(\prod_{i=1}^n \sigma_i(\alpha)\right) \cdot \left(\prod_{i=1}^n \sigma_i(\beta)\right)\\
&=&& \ N_{\mathbb{L}/\mathbb{K}}(\alpha) \cdot N_{\mathbb{L}/\mathbb{K}}(\beta)
\end{alignat*}

\end{compactenum}
\allowdisplaybreaks[1]
\titleformat{\subsection}{\normalfont\normalsize\bfseries}{}{0em}{#1}
\subsection*{Example}%Example 
\titleformat{\subsection}{\normalfont\normalsize\bfseries}{}{0em}{#1 \thesubsection}
\begin{compactenum}
\item Let $\alpha \in \mathbb{K}$. Then\\
$$N_{\mathbb{L}/\mathbb{K}}(\alpha)=\prod_{i=1}^n \sigma_i(\alpha)=\prod_{i=1}^n \alpha = \alpha^n.$$
\item Let $\mathbb{K}=\mathbb{R}$, $\mathbb{L}=\mathbb{C}$. Then\\
$\Rightarrow \rm{Hom}_{\mathbb{R}}(\mathbb{C}, \overline{\mathbb{R}})=\rm{Gal}(\mathbb{C}/\mathbb{R})=\{\rm{id}, z \mapsto \overline{z}\}$ And thus
$N_{\mathbb{L}/\mathbb{K}}(z)=z\overline{z}=\vert z\vert^2$.
\item Let $\mathbb{K}=\mathbb{Q}$, $\mathbb{L}=\mathbb{Q}[\sqrt{d}]$ for $d \in \mathbb{Z}$ squarefree. We have $[\mathbb{Q}[\sqrt{d}]:\mathbb{Q}]=2$ and 
$$\rm{Gal}(\mathbb{Q}[\sqrt{d}]/\mathbb{Q})=\{\rm{id}, \sqrt{d} \mapsto -\sqrt{d}\}=\{a+b\sqrt{d} \mapsto a+b\sqrt{d}, a+b\sqrt{d} \mapsto a-b\sqrt{d}\}$$
Then we have
$$N_{\mathbb{Q}[\sqrt{d}]/\mathbb{Q}}(a+b\sqrt{d})=\left(a+b\sqrt{d}\right)\left(a-b\sqrt{d}\right)=a^2-db^2$$
\begin{compactitem}
\item $d<0$: $d=-\tilde{d}$, hence $a^2+\tilde{d}b^2\overset{!}{=}1 \Rightarrow $ either $ a=\pm 1, b=0$ or $a=0, b=\pm 1, \tilde{d}=1$.
\item $d>0$: Infinitely many solutions for $a^2-bd^2=1$.
\end{compactitem}
\end{compactenum}

\allowdisplaybreaks[1]
\titleformat{\subsection}{\normalfont\normalsize\bfseries}{}{0em}{#1 \thesubsection \quad \textnormal{\textit{(Hilbert's theorem 90 - multiplicative version)}}}
\subsection{Proposition } %Proposition 5.2
\titleformat{\subsection}{\normalfont\normalsize\bfseries}{}{0em}{#1 \thesubsection}
Let$\mathbb{L}/\mathbb{K}$ a finite Galois extension with cyclic Galois group $\rm{Gal}(\mathbb{L}/\mathbb{K})=\langle \sigma \rangle$, $n=[\mathbb{L}:\mathbb{K}]$. Let $\beta \in \mathbb{L}$ with $N_{\mathbb{L}/\mathbb{K}}(\beta)=1$.\\
Then there exists $\alpha \in \mathbb{L}^{\times}$ such that $\beta= \frac{\alpha}{\sigma(\alpha)}$.\\
\textit{proof.}\\
Define
$$f=\rm{id}_{\mathbb{L}}+\beta  \sigma+ \beta  \sigma(\beta) \sigma^2 + \ldots + \beta  \sigma(\beta)  \sigma^2(\beta) \cdot \cdot \cdot \sigma^{n-2}(\beta) \sigma^{n-1}=\sum_{j=0}^{n-1} \sigma^j  \prod_{i=1}^j \sigma^{i-1}(\beta)$$
Then by Theorem 4.10 $f\neq 0$. Choose $\gamma \in \mathbb{L}$ such that $\alpha:=f(\gamma)\neq 0$. Then we have
\allowdisplaybreaks[1]
\begin{alignat*}{7}
\beta \cdot \sigma(\alpha)\ = \beta \cdot \sigma \left(f(\gamma)\right)&=&&\ \beta \cdot \left(\sigma\left(\gamma+ \beta \sigma(\gamma) + \ldots + \prod_{i=0 }^{n-2} \sigma^{i}(\beta) \sigma^{n-1}(\gamma)\right)\right)\\
&=&& \ \beta \cdot \left(\sigma(\gamma)+\sigma(\beta) \sigma^2(\gamma) + \ldots + \prod_{i=0}^{n-2} \sigma^{i+1}(\beta) \sigma^n(\gamma)\right)\\
&=&& \ \beta \cdot \left(\sigma(\gamma)+\sigma(\beta) \sigma^2(\gamma) + \ldots +  \frac{1}{\beta} N_{\mathbb{L}/\mathbb{K}}(\beta) \cdot \gamma\right)\\
&=&&\ \beta \cdot \left(\sigma(\gamma)+\sigma(\beta) \sigma^2(\gamma) + \ldots + \gamma\right)\\
&=&&\ \gamma+ \beta \sigma(\gamma)+ \beta \sigma(\beta) \sigma^2(\gamma)+ \ldots + \beta \cdot \prod_{i=1}^{n-2} \sigma^{i}(\beta) \sigma^{n-1}(\gamma)\\
&=&&\ f(\gamma)=\alpha
\end{alignat*}

\subsection{Definition + Remark}%Definition + Remark 5.3
Let $\mathbb{L}/\mathbb{K}$ be a finite separable field extension, $[\mathbb{L}:\mathbb{K}]=n$. Let $\rm{Hom}_{\mathbb{K}}(\mathbb{L},\overline{\mathbb{K}})=\{\sigma_1, \dots \sigma_n\}$.
\begin{compactenum}
\item For $\alpha \in \mathbb{L}$, 
$$tr_{\mathbb{L}/\mathbb{K}}(\alpha):=\sum_{i=0}^n \sigma_i(\alpha)$$
is called the \textit{trace} of $\alpha$ over $\mathbb{K}$.
\item $tr_{\mathbb{L}/\mathbb{K}}(\alpha) \in \mathbb{K}$ for all $\alpha \in \mathbb{L}$.
\item $tr_{\mathbb{L}/\mathbb{K}}: \mathbb{L} \longrightarrow \mathbb{K}$ is $\mathbb{K}$-linear.
\end{compactenum}
\textit{proof.}
\begin{compactenum}
\item[(ii)] As in proof 5.1, $tr_{\mathbb{L}/\mathbb{K}}(\alpha)$ is invariant under $\rm{Gal}(\tilde{\mathbb{L}}/\mathbb{K})$.
\item[(iii)] Clear.
\end{compactenum}

\titleformat{\subsection}{\normalfont\normalsize\bfseries}{}{0em}{#1}
\subsection*{Examples} %Example 
\titleformat{\subsection}{\normalfont\normalsize\bfseries}{}{0em}{#1 \thesubsection}
\begin{compactenum}
\item Let $\alpha \in \mathbb{K}$. Then
$$tr_{\mathbb{L}/\mathbb{K}}(\alpha)=\sum_{i=0}^n \sigma_i(\alpha)=\sum_{i=0}^n \alpha = n \cdot \alpha.$$
\item Let $\mathbb{K}=\mathbb{R}$, $\mathbb{L}=\mathbb{C}$. Then $tr_{\mathbb{C}/\mathbb{R}}(z)=z+\overline{z}= 2 \cdot \mathfrak{Re}(z)$.
\end{compactenum}

\titleformat{\subsection}{\normalfont\normalsize\bfseries}{}{0em}{#1 \thesubsection \quad \textnormal{\textit{(Hilbert's theorem 90 - additive version)}}}
\subsection{Proposition } %Proposition 5.4
\titleformat{\subsection}{\normalfont\normalsize\bfseries}{}{0em}{#1 \thesubsection}
Let $\mathbb{L}/\mathbb{K}$ be a Galois extension with cyclic Galois group $\textrm{Gal}(\mathbb{L}/\mathbb{K})=\langle \sigma \rangle$ and \linebreak $[\mathbb{L}:\mathbb{K}]=\textrm{char}(\mathbb{K})=p\in\mathbb{P}$.\\
Then for every $\beta \in \mathbb{L}$ with $tr_{\mathbb{L}/\mathbb{K}}(\beta)=0$ there exists $\alpha \in \mathbb{L}$ such that $\beta=\alpha-\sigma(\alpha)$.\\
\textit{proof.}\\
Define
$$g=\beta \cdot \sigma + \left(\beta+\sigma(\beta)\right)\cdot \sigma^2+ \ldots + \left(\sum_{i=0}^{p-2} \sigma^{i}(\beta)\right) \cdot \sigma^{p-1}=\sum_{i=0}^{p-2} \left(\sum_{j=0}^{i} \sigma^{j}(\beta)\right)\cdot \sigma^{i+1}$$
Let now $\gamma \in \mathbb{L}$ such that $tr_{\mathbb{L}/\mathbb{K}}(\gamma) \neq 0$ (existing by 4.11). Then for
$$\alpha:=\frac{1}{tr_{\mathbb{L}/\mathbb{K}}(\gamma)} \cdot g(\gamma)$$
we have
\begin{alignat*}{5}
\alpha-\sigma(\alpha) \ &=&& \ \frac{1}{tr_{\mathbb{L}/\mathbb{K}}(\gamma)} \cdot \left(g(\gamma)-\sigma\left(g(\gamma)\right)\right) \\
&=&& \ \frac{1}{tr_{\mathbb{L}/\mathbb{K}}(\gamma)}  \left(\left(\sum_{i=0}^{p-2}\left(\sum_{j=0}^{i} \sigma^{j}(\beta)\right) \sigma^{i+1}(\gamma)\right)- \left(\sum_{i=0}^{p-2}\left(\sum_{j=0}^{i} \sigma^{j+1}(\beta)\right) \sigma^{i+2}(\gamma)\right)\right) \\
&=&& \  \frac{1}{tr_{\mathbb{L}/\mathbb{K}}(\gamma)}  \left(\left(\sum_{i=0}^{p-2}\left(\sum_{j=0}^{i} \sigma^{j}(\beta)\right) \sigma^{i+1}(\gamma)\right)- \left(\sum_{i=1}^{p-1}\left(\sum_{j=1}^{i} \sigma^{j}(\beta)\right) \sigma^{i+1}(\gamma)\right)\right) \\
&=&& \ \frac{1}{tr_{\mathbb{L}/\mathbb{K}}(\gamma)} \cdot \left(\sum_{i=0}^{p-1} \beta \cdot \sigma^{i}(\gamma) \right)
\ = \ \beta
\end{alignat*}

\subsection{Proposition} %Proposition 5.5
Let $\mathbb{L}/\mathbb{K}$ be a finite separable extension, $\alpha \in \mathbb{L}$. Consider the $\mathbb{K}$-linear map
$$ \phi_{\alpha}: \mathbb{L} \longrightarrow \mathbb{L}, \quad x \mapsto \alpha \cdot x$$
Then
\begin{compactenum}
\item $N_{\mathbb{L}/\mathbb{K}}(\alpha)=\det(\phi_{\alpha})$.
\item $tr_{\mathbb{L}/\mathbb{K}}(\alpha)=\rm{tr}(\phi_{\alpha})$.
\end{compactenum}
\textit{proof.}\\
Let $$f=\sum_{i=0}^d a_i X^{i}$$ be the minimal polynomial of $\alpha$ over $\mathbb{K}$. Then
$$\left(f \circ \phi_{\alpha}\right)(x)=f\left(\phi_{\alpha}(x)\right)=\sum_{i=0}^d a_i \phi_{\alpha}^{i}(x)=\sum_{i=0}^d a_i \alpha^{i} \cdot x = x \cdot \sum_{i=0}^d a_i \alpha^{i} = x \cdot f(\alpha)=0$$
For arbitrary $x \in \mathbb{L}$, hence $f(\phi_{\alpha})=0$.
\begin{compactitem}
\item[\textbf{case 1.1}] Assume first $\mathbb{L}=\mathbb{K}[\alpha]$ for some $\alpha \in \mathbb{K}$. Then $[\mathbb{L}:\mathbb{K}]= \deg(f)=d$, so $\{1, \alpha, \ldots, \alpha^{d-1}\}$ is a $\mathbb{K}$-basis of $\mathbb{L}$. Then we have a transformation matrix of $\phi_{\alpha}$ with respect to the basis $\{1, \alpha, \ldots, \alpha^{d-1}\}$\\
$$D=\begin{pmatrix}[rrrrr] 0 & 0 & 0 & 0 & a_0 \\ 1 & 0 && \vdots & -a_1 \\ 0 & 1 & & \vdots &  \vdots \\ \vdots & \vdots & \ddots & 0 & \vdots \\ 0 &  \ldots & 0 & 1 & -a_{d-1} \end{pmatrix}$$
So we have $\textrm{tr}(\phi_{\alpha})=-a_{d-1}$ and $\det(\phi_{\alpha})=(-1)^d \cdot a_0$.\\
We know that $f$ splits over $\overline{\mathbb{K}}$, say $$f=\prod_{i=1}^d (X-\lambda_i)=\prod_{i=1}^d \left(X-\sigma_i(\alpha)\right)$$
Then we easily see
$$\det(\phi_{\alpha})=(-1)^d \cdot a_0=(-1)^d \cdot f(0)=(-1)^d \cdot \prod_{i=1}^d \left(0-\sigma_i(\alpha)\right)=\prod_{i=1}^d \sigma_i(\alpha)=N_{\mathbb{L}/\mathbb{K}}(\alpha)$$
$$\textrm{tr}(\phi_{\alpha})=-a_{d-1}=tr_{\mathbb{L}/\mathbb{K}}(\alpha)$$
\item[\textbf{case 1.2}] For the case $\alpha \in \mathbb{K}$, $\phi_{\alpha}$ is represented by the diagonal matrix $\begin{pmatrix}[rrr] \alpha & & 0 \\ & \ddots & \\ 0 & & \alpha \end{pmatrix} \in \mathbb{K}^{d \times d}$.
\\ We obtain 
$$\textrm{tr}(\phi_{\alpha})=d \cdot \alpha = tr_{\mathbb{L}/\mathbb{K}}(\alpha) \qquad \det(\phi_{\alpha})=\alpha^d = tr_{\mathbb{L}/\mathbb{K}}(\alpha$$
\item[\textbf{case 2}] For the general case we have $\mathbb{K}\subseteq \mathbb{K}(\alpha) \subseteq \mathbb{L}$.\\
\begin{compactitem}
\item[\textbf{Claim (a)}] We have $$N_{\mathbb{L}/\mathbb{K}}(\alpha)=N_{\mathbb{K}(\alpha])\mathbb{K}} \left(N_{\mathbb{L}/\mathbb{K}(\alpha)}(\alpha)\right), \qquad tr_{\mathbb{L}/\mathbb{K}}(\alpha)=tr_{\mathbb{K}(\alpha)/\mathbb{K}}\left(tr_{\mathbb{L}/\mathbb{K}(\alpha)}(\alpha)\right)$$
\item[\textbf{Claim (b)}]  We have 
$$\det(\phi_{\alpha})=\left(\det\left(\phi_{\alpha}|_{\mathbb{K}(\alpha)}\right)\right)^{[\mathbb{L}:\mathbb{K}(\alpha)]} \qquad \textrm{tr}(\phi_{\alpha})=[\mathbb{L}:\mathbb{K}(\alpha)] \cdot \textrm{tr}\left(\phi_{\alpha}|_{\mathbb{K}(\alpha)}\right).$$
\end{compactitem}
Assuming Claim (a) and (b), we get
\begin{alignat*}{5}
\det(\phi_{\alpha})\ &=&&\ \left(\det\left(\phi_{\alpha}|_{\mathbb{K}(\alpha)}\right)\right)^{[\mathbb{L}:\mathbb{K}(\alpha)]} \ \overset{1.1}{=}\ \left(N_{\mathbb{K}(\alpha)/\mathbb{K}}\right)^{[\mathbb{L}:\mathbb{K}(\alpha)]}\ =\ N_{\mathbb{K}(\alpha)/\mathbb{K}}\left(\alpha^{[\mathbb{L}:\mathbb{K}(\alpha)]}\right)\\
\ &\overset{1.2}{=}&&\ N_{\mathbb{K}(\alpha)/\mathbb{K}}\left(N_{\mathbb{L}/\mathbb{K}(\alpha)}(\alpha)\right)\\
\ &\overset{(a)}{=}&&\ N_{\mathbb{L}/\mathbb{K}}(\alpha)
\end{alignat*}
And analogously
$\rm{tr}(\phi_{\alpha})= \it{tr}_{\mathbb{L}/\mathbb{K}}(\alpha)$.
\end{compactitem}
Let's now proof the claims.
\begin{compactitem}
\item[\textbf{(b)}]  Let $x_1, \ldots x_d$ be a basis of $\mathbb{K}(\alpha)/$ as a $\mathbb{K}$-vector space and $y_1, \ldots y_m$ a basis of $\mathbb{L}$ as a $\mathbb{K}(\alpha)$-vector space.\\
Then the $x_iy_j$ for $1 \leqslant i \leqslant d, \ 1 \leqslant j \leqslant m$ form a $\mathbb{K}$-basis for $\mathbb{L}$. \\
Let now $D \in \mathbb{K}^{d \times d}$ be the matrix representing $\phi_{\alpha}|_{\mathbb{K}(\alpha)}$. Then we have\\
$$\alpha x_i y_j = \underbrace{\left(\alpha x_i\right)}_{\in \mathbb{K}(\alpha)}y_j = \left(D \cdot x_i\right) y_j$$
Hence $\phi_{\alpha}$ is represented by
$$\tilde{D}= \begin{pmatrix}[rrrr] A & 0 & \ldots & 0 \\ 0 & A & & \vdots \\ \vdots & \vdots & \ddots & \vdots \\ 0 & 0 & \ldots & A \end{pmatrix}$$
\item[\textbf{(a)}] This is an exercise.
\end{compactitem}

\subsection{Definition + Remark} %Definition + Remark 5.6
Let $\mathbb{L}/\mathbb{K}$ be a finite field extension, $r=[\mathbb{L}:\mathbb{K}]_s=\vert \rm{Hom}_{\mathbb{K}}(\mathbb{L}, \overline{\mathbb{K}}) \vert$. Let $q=\frac{[\mathbb{L}:\mathbb{K}]}{[\mathbb{L}:\mathbb{K}]_s}$.
\begin{compactenum}
\item For $\alpha \in \mathbb{L}$ define 
$$N_{\mathbb{L}/\mathbb{K}}(\alpha)=\det(\phi_{\alpha}) \qquad tr_{\mathbb{L}/\mathbb{K}}(\alpha)=\rm{tr}(\phi_{\alpha})$$
\item Let $\rm{Hom}_{\mathbb{K}}(\mathbb{L}, \overline{\mathbb{K}})=\{\sigma_1, \ldots, \sigma_r\}$. Then
$$N_{\mathbb{L}/\mathbb{K}}(\alpha)=\left(\prod_{i=1}^r \sigma^{i}(\alpha)\right)^q, \qquad tr_{\mathbb{L}/\mathbb{K}}(\alpha)=\left(\sum_{i=1}^r \sigma_{i}(\alpha)\right) \cdot q $$
\end{compactenum}
\textit{proof.}\\
Copy the proof of 5.5. Recall that the minimal polynomial of $\alpha$ over $\mathbb{K}$ is 
$$m_{\alpha}=\prod_{i=1}^r \left(X-\sigma_i(\alpha)\right)^q$$

% SECTION 6

\renewcommand*\thesection{\S\ \arabic{section}\quad}
\section{Normal series of groups}
\renewcommand*\thesection{\arabic{section}}

\subsection{Defintion} %Definition 6.1
Let $G$ be a group.
\begin{compactenum}
\item A series $$G=G_0 \ \triangleright \ G_1 \ \triangleright \ \ldots \ \triangleright \ G_n$$
of subgroups is called a \textit{normal series} for $G$, if $G_{i}\triangleleft G_{i-1}$ is a normal subgroup in $G_{i-1}$ and $G_{i}\neq G_{i-1}$ for $1 \leqslant i \leqslant n$. The groups $H_i:=\slant{G_{i-1}}{G_i}$ are called \textit{factors} of the series.
\item A normal series as above is called a \textit{composition series} for $G$, if all its factors are simple groups and $G_n=\{e\}$.
\end{compactenum}

\titleformat{\subsection}{\normalfont\normalsize\bfseries}{}{0em}{#1}
\subsection*{Example} %Example
\titleformat{\subsection}{\normalfont\normalsize\bfseries}{}{0em}{#1 \thesubsection}
\begin{compactenum}
\item For $G=S_4$ we have a composition series
$$G=S_4 \ \triangleright \ A_4 \ \triangleright \ V_4 \ \triangleright \ T_4 \ \triangleright \ \{e\}$$
where $T_4=\{\textrm{id}, \sigma\} \cong \slant{\mathbb{Z}}{2 \mathbb{Z}}$ for some transposition $\sigma \in S_4$. \\
We have quotients
$$\slant{S_4}{A_4}=\slant{\mathbb{Z}}{2 \mathbb{Z}}, \quad \slant{A_4}{V_4}=\slant{\mathbb{Z}}{3 \mathbb{Z}}, \quad \slant{V_4}{T_4}=\slant{\mathbb{Z}}{2 \mathbb{Z}}, \quad \slant{T_4}{\{e\}}=\slant{\mathbb{Z}}{2 \mathbb{Z}}$$
\item $\mathbb{Z}$ has no composition series.
\item Every normal series is a composition series.
\item Every finite group has a composition series.
\end{compactenum}

\subsection{Remark} %Remark 6.2
If $G=G_0 \ \triangleright \ G_1 \ \triangleright \ \ldots \ \triangleright \ G_n= \{e\}$ is a normal composition series for a finite group $G$, then we have
$$\vert G \vert = \prod_{i=1}^{n} \vert \slant{G_{i-1}}{G_i} \vert $$

\pagebreak

\subsection{Definition + Remark} %Definition + Remark 6.3
Let $G$ be a group.
\begin{compactenum}
\item For subgroups $H_1, H_2 \leqslant G$ let $[H_1, H_2]$ denote the subgroup of $G$ generated by all \textit{commutators}
$$[h_1, h_2]=h_1h_2 h_1^{-1}h_2^{-1}\qquad \textrm { with } h_i \in H_i \textrm{  for } i \in \{1,2\}$$
\item $[G,G]=G'$ is called the \textit{derived} or \textit{commutator subgroup} of $G$.
\item $G' \triangleleft G$ and $G^{\rm{ab}}:=\slant{G}{G'}$ is abelian.
\item Let $A$ be an abelian group and $\phi:G \longrightarrow A$ a homomorphism of groups. Let $\pi: G \longrightarrow G^{\rm{ab}}$ denote the residue map. Then $G' \subseteq \ker(\phi)$, thus $\phi$ factors to a unique homomorphism
$$\overline{\phi}: G^{\rm{ab}} \longrightarrow A \qquad \textrm{ such that } \phi= \overline{\phi} \circ \pi$$
\item The chain 
$$G \ \triangleright G' \ \triangleright \ G''=[G', G'] \ \triangleright \ \ldots \ \triangleright \ G^{(n+1)}=[G^n, G^n]$$
is called the \textit{derived series} of $G$.
\item $G$ is solvable if and only if its derived series stops at $\{e\}$.
\end{compactenum}
\textit{proof.}
\begin{compactenum}
\item[(iii)] For $g \in G$, $a,b \in G$ we have
$$g [ab]g^{-1}=gaba^{-1}b^{-1}g^{-1}=ga\underbrace{g^{-1}g}_{=e}b\underbrace{g^{-1}g}_{=e}a^{-1}\underbrace{g^{-1}g}_{=e}b^{-1}g^{-1}=[gag^{-1}, gbg^{-1}] \in G'$$
Moreover
$$e=[\overline{a}, \overline{b}]=\overline{[a,b]}=\overline{aba^{-1}b^{-1}} \quad \Longleftrightarrow \quad \overline{ab}=\overline{a}\overline{b}=\overline{b}\overline{a}=\overline{ba}$$
\item[(iv)] Let $A$ be an abelian group, $\phi:G \longrightarrow A$ a himomorphism. For $x,y \in G$ we have
$$\phi([x,y])=\phi(xyx^{-1}y^{-1})=\phi(x)=\phi(y)\phi(x)^{-1}\phi(y)^{-1}=e \quad \Longrightarrow \quad G' \subseteq \ker(\phi)$$
\item[(vi)] \begin{compactitem}
\item['$\Leftarrow$'] If the derived series of $G$ stops at $\{e\}$, $G$ has a normal series with abelian factors and is solvable.
\item['$\Rightarrow$'] Let now $G=G_0 \ \triangleright \ \ldots \ \triangleright G_n=\{e\}$ be a normal series with abelian factors. We have to show that $G^{(n)}=\{e\}$.
\begin{compactenum}
\item[\textbf{Claim (a)}] We have $G^{(i)} \subseteq G_i$ for $0 \leqslant i \leqslant n$.
\end{compactenum}
Then we see $G^{(n)} \subseteq G_n = \{e\}$ an hence the derived series of $G$ stops at $\{e\}$.\\ It remains to prove the claim.
\begin{compactenum}
\item[\textbf{(a)}] We have $\pi_i:G_i \longrightarrow \slant{G_i}{G_{i+1}}$ is a homomorphism from $G$ to an abelian group. Then by part (iv), we have $G_i^{(1)}=G_i' \subseteq \ker(\pi_i)=G_{i+1}$.\\
By induction on $n$ we have $G^{(i)}=(G^{(i-1)})' \subseteq G_i$, hence $\left(G^{(i)}\right)' \subseteq G_i?$.\\
Thus we get 
$$G^{(i+1)} = \left(G^{(i)}\right)' \subseteq G_i' \subseteq \ker(\pi_I) = G_{i+1}$$
\end{compactenum}
\end{compactitem}
\end{compactenum}

\subsection{Proposition} %Proposition 6.4
A finite group $G$ is solvable if and only if the factors of its composition series are cyclic of prime order.\\
\textit{proof.}
\begin{compactenum}
\item['$\Rightarrow$'] Let 
$$G \ = \ G_1 \ \triangleright \ G_2 \ \triangleright \ \ldots \ \triangleright G_m \ = \ \{1\}$$
be a normal series of $G$ with abelian quotients $\slant{G_i-1}{G_{i}}$ for $1 \leqslant i \leqslant m$. Refine it to a composition series
$$G \ = \ G_0 \ = H_{0,0} \triangleright H_{0,1}  \triangleright  \ldots  \triangleright H_{0,d_0}  =  \ G_1 \ = H_{1,0}  \triangleright \ldots  \triangleright  H-{1,d_1} = G_2 \ \triangleright \ \ldots \ \triangleright \ G_m = \{1\}$$
Then we have
$$\slant{H_{i,j}}{H_{i,j+1}} \ \cong\ \bigslant{\slant{H_{i,j}}{G_{i+1}}}{\slant{H_{i,j+1}}{G_{i+1}}} \ \subseteq \ \bigslant{\slant{G_i}{G_{i+1}}}{\slant{H_{i,j+1}}{G_{i+1}}}$$
hence $\slant{H_{i,j}}{H_{i,j+1}}$ is isomorphic to a subgroup of a factor group of an abelian group, thus abelian.
\item['$\Leftarrow$'] Since the factor groups of the composition series are isomorphic to $\slant{\mathbb{Z}}{p \mathbb{Z}}$ for some primes $p$, the quotients are abelian, thus $G$ is solvable.
\end{compactenum}


\titleformat{\subsection}{\normalfont\normalsize\bfseries}{}{0em}{#1 \thesubsection \quad \textnormal{\textit{(Jordan-Hölder)}}}
\subsection{Theorem } %Theorem 6.5
\titleformat{\subsection}{\normalfont\normalsize\bfseries}{}{0em}{#1 \thesubsection}
Let $G$ be a group and
$$G=G_0 \ \triangleright \ G_1 \ \triangleright \ \ldots \ \triangleright \ G_n = \{e\}$$
$$G=H_0 \ \triangleright \ H_1 \ \triangleright \ \ldots \ \triangleright \ H_m = \{e\}$$
be two composition series of $G$.\\
Then $n=m$ and there ist $\sigma \in S_n$ such that 
$$\slant{H_i}{H_{i+1}} \ \cong\ \slant{G_{\sigma(i)}}{G_{\sigma(i)+1}} \qquad \textrm{ for } 0 \leqslant i \leqslant n-1$$
\textit{proof.}\\
We prove the statement by induction on $n$.
\begin{compactitem}
\item[\textbf{n=1}] $G$ is simple and thus $H_1= \{e\}$.
\item[\textbf{n>1}] Let $\overline{G}:= \slant{G}{G_1}$ and $\pi:G \longrightarrow \overline{G}$ be the residue map.\\
Then $\overline{H}_i= \pi(H_i) \trianglelefteq \overline{G}$ is a normal subgroup. Since $\overline{G}$ is simple, hence we have $\overline{H}_i \in \{\{e\}, \overline{G}\}$. If $\overline{H}_1=\overline{G}$, then $\overline{H}_2$ is a normal subgroup of $\overline{H}_1=\overline{H}$, and so on. Hence we find $j \in \{1, \ldots m \}$ such that
$$\overline{H}_i= \overline{G} \ \textrm{ for } 0 \leqslant 1 \leqslant j \ \textrm{ and } \overline{H}_i=\{e\} \ \textrm{ for } j+1 \leqslant i \leqslant m.$$
Define $C_i:=H_i \cap G_1 < G_1 $ for $0 \leqslant i \leqslant m$. 
\begin{compactenum}
\item[\textbf{Claim (a)}] If $j \leqslant m-2$, then we have a composition series for $G_1$:
$$G_1=C_0 \ \triangleright \ C_1 \ \triangleright \ \ldots \ \triangleright \ C_j \ \triangleright \ C_{j+2} \ \triangleright \ \ldots \ \triangleright \ C_m= \{e\}$$
If $j=m-1$, we have a composition series for $G_1$:
$$G_1=C_0 \ \triangleright \ C_1 \ \triangleright \ \ldots \ \triangleright C_{m-1}=\{e\}$$
\end{compactenum}
Clearly $G_1 \ \triangleright \ G_2 \ \triangleright \ \ldots \ \triangleright \ G_n =\{e\}$ is a composition series, too.\\
By induction hypothesis we have $n-1=m-1$, hence $n=m$. Moreover we have for $i \neq j$
$$\left.\begin{matrix} \slant{C_i}{C_{i+1}} \cong \slant{G_{\sigma(i)}}{G_{\sigma(i)+1}} \\ \slant{C_j}{C_{j+2}} \cong \slant{G_{\sigma(j)}}{G_{\sigma(j)+1}} \end{matrix} \right\} (*)$$
For some $\sigma: \{0,1, \ldots, j,j+2,j+3, \ldots, n-1\} \longrightarrow \{1, \ldots, n-1\}$
\begin{compactenum}
\item[\textbf{Claim (b)}] We have
\begin{compactenum}
\item $C_{j+1}=C_j$
\item $\slant{C_i}{C_{i+1}} \cong \slant{H_i}{H_{i+1}}$ for $i \neq j$.
\item $\slant{H_j}{H_{j+1}} \cong \overline{G} = \slant{G}{G_1}$.
\end{compactenum}
\end{compactenum}
By $(*)$ and Claim (a),(b) the theorem is proved.
\end{compactitem}
It remains to show the Claims.
\begin{compactenum}
\item[\textbf{(a)}] $C_{i+1}$ is a normal subgroup of $C_i$, $C_{i+1}=H_{i+1} \cap G_1$.\\
$C_{j+1}$ is normal in $C_j=C_{j+1}$ by Claim (b)(2).\\
$\slant{C_{i}}{C_{i+1}} \cong \slant{H_i}{H_{i+1}}$ for $i \neq j$ is simple by Claim (b)(2).\\
$\slant{C_j}{C_{j+2}} = \slant{C_j}{C_{j+1}} = \slant{H_j}{H_{j+1}}$ is simple, too. 
\item[\textbf{(b)}] \begin{compactenum}
\item We have $H_{j+1} \subseteq G_1$, hence $H_{j+1}\cap G_1=H_{j+1}=C_{j+1}$. $C_j=H_j \cap G_1$ is normal subgroup of $H_j$.\\
Thus $H_j \triangleright C_j \triangleright C_{j+1}=H_{j+1}$. Since $\slant{H_i}{H_{i+1}}$ is simple, we must have $C_j=C_{j+1}$.
\item \begin{compactenum}
\item[\bf{i>j}] Then $C_i=H_i \cap G_1=H_i$ since $H_i \subseteq G_1$.
\item[\bf{i<j}] We have $\overline{H}_i=\overline{G}=\slant{G}{G_1}$.\\
Then we have $G_1 H_i=G \ (*)$, since:
\begin{compactitem}
\item['$\subseteq$'] Clear.
\item['$\supseteq$'] For $g \in G, \overline{g} \in \overline{G}$ its image there exists $h \in H_i$ such that
$$ \overline{h}=\overline{g} \Longrightarrow \overline{h}^{-1}\overline{g} \in G_1 \Longleftarrow \overline{h}^{-1}\overline{g}=g_1 \in G_1 \Longrightarrow g=hg_1 \in H_i G_1$$
\end{compactitem}
With the isomorphism theorem we obtain 
$$\slant{C_i}{C_{i+1}} = \slant{C_i}{H_{i+1} \cap G_i} = \slant{C_i}{H_{i+1}\cap C_i} \cong \slant{C_i H_{i+1}}{H_{i+1}}$$
Therefore it remains to show that $C_i H_{i+1}=H_i$.
\begin{compactitem}
\item['$\subseteq$'] Since $C_i, H_{i+1} \subseteq H_i$ we also have $C_i H_{i+1} \subseteq H_i$
\item['$\supseteq$'] Let $x \in H_i$. by $(*)$ we have $H_{i+1}G_i=G$.\\
Hence there exists $g \in G_1, h \in H_{i+1}$ such that $x=gh$.\\
Then we have $g=x h^{-1} \in H_i H_{i+1}=H_i$, i.e. $g \in G_i \cap H_{i}=C_1$ and thus $x \in C_i H_{i+1}$.
\end{compactitem}
\end{compactenum}
\item We have 
$$\slant{H_i}{H_{i+1}}= \slant{H_i}{C_{j+1}}=\slant{H_j}{C_j}=\slant{H_j}{H_j \cap G_1} = \slant{G_1 H_j}{G_1}\overset{(*)}{=} \slant{G}{G_1}$$
\end{compactenum}
\end{compactenum}

%CHAPTER II


\chapter{Valuation theory}
\thispagestyle{empty}

\setcounter{section}{6}

% SECTION 7 

\renewcommand*\thesection{\S\ \arabic{section}\quad}
\section{Discrete valuations}
\renewcommand*\thesection{\arabic{section}}

\subsection{Example} %Exmaple 7.1
Let $P \in \mathbb{N}$ prime. For $x \in \mathbb{Z}\setminus \{0\}$ let 
$$\nu_p(x)=\max\{k \in \mathbb{N} \ \big\vert\ p^k \mid x \}$$
Then $p^{\nu_p(x)} \mid x, \quad p^{\nu_p(x)+1} \nmid x$. Example: $\nu_2(12)=2$.\\
Write $x=p^{\nu_p(x)} \cdot x'$ where $p \nmid x'$.\\
For $\frac{x}{y} \in \mathbb{Q}^{\times}$ define
$$\nu_p\left(\frac{x}{y}\right)=\nu_p(x)-\nu_p(y)$$
This defines a map $\nu_p: \mathbb{Q} \longrightarrow \mathbb{Z}$, such that
\begin{compactenum}
\item $v_p(ab)=\nu_p(a)+\nu_p(b)$ (clear)
\item $v_p(a+b) \geqslant \min\{\nu_p(a), \nu_p(b)\}$, since:
Write $a=p^{\nu_p(a)} \cdot a', b=p^{\nu_p(b)}\cdot b'$. Let w.l.o.g $\nu_p(b) \leqslant \nu_p(a)$. Then we have
$$a+b=p^{\nu_p(a)} \cdot a' + p^{\nu_p(b)} \cdot b'=p^{\nu_p(b)} \cdot \left(b'+a' \cdot p^{\nu_p(a)-\nu_p(b)}\right)$$
Hence $p^{\nu_p(b)} \mid a+b$ and thus $\nu_p(a+b) \geqslant \nu_p(b)=\min\{\nu_p(a), \nu_p(b)\}$
\end{compactenum}

\subsection{Definition} %Definition 7.2
Let $\mathbb{K}$ be afield. A \textit{discrete valuation} on $\mathbb{K}$ is a surjectove group homomorphism
$$\nu_ \mathbb{K}^{\times} \longrightarrow (\mathbb{Z},+)$$
satisfying
$$\nu(x+y) \geqslant \min\{\nu(x), \nu(y)\} \qquad \textrm{ for all } x,y \in \mathbb{K}^{\times}, \ x \neq -y$$

\subsection{Remark} %Remark 7.3
Let $R$ be a factorial domain, $\mathbb{K}=\rm{Quot}(R)$. Let further be $p \in R \setminus \{0\}$ be a prime element. Then 
$$\nu_p: \mathbb{K}^{\times} \longrightarrow \mathbb{Z}$$ can be defined as in Example 7.1: Write
$$x=e \cdot \prod_{p \in \mathbb{P}} p^{\nu_p(x)}, \qquad e\in R^{\times}$$
where $\mathbb{P}$ denotes set of representatives of prime elements of $R$. Then $\nu_p$ is a discrete valuation on $\mathbb{K}$.

\subsection{Example} %Example 7.4
Let $\mathbb{K}$ be a field, $a \in \mathbb{K}$, $R=\mathbb{K}[X]$ and $p_a=X-a \in \mathbb{K}[X]$.\\
For $f \in \mathbb{K}[X]$ define $\nu_{p_a}(f)=n$ if $f$ has an $n$-fold root in $a$, i.e. $f=(X-a)^n \cdot g$ for some $0 \neq g \in \mathbb{K}[X]$.\\
Then $\nu_{p_a}$ is a discrete valuation on $\mathbb{K}(X)=\rm{Quot}(\mathbb{K}[\mathit{X}])$ satisfying $\nu_p |_{\mathbb{K}}=0$.

\subsection{Remark} % Remark 7.5
There is no discrete valuation on $\mathbb{C}$.\\
\textit{proof.}\\
Assume there exists a discrete valuation on $\mathbb{C}$, say $\nu:\mathbb{C}^{\times} \longrightarrow \mathbb{Z}$. Since $\nu$ is surjective, there exists $z \in \mathbb{C}^{\times}$ such that $\nu(z)=1$.\\
Let now $y \in \mathbb{C}^{\times}$ such that $y^2=z$. Then we have
$$1= \nu(z)=\nu(y^2)= \nu(y \cdot y)= \nu(y)+\nu(y)=2 \nu(y) \quad \Longleftrightarrow \quad \nu(y)=\frac{1}{2} \notin \mathbb{Z}$$
which is a contradiction.

\subsection{Example} %Example 7.6
Let $\nu: \mathbb{Q}^{\times} \longrightarrow \mathbb{Z}$ be a nontrivial discrete valuation. Then there exists $a \in \mathbb{Z}$ such that $\nu(a)\neq 0$ and hence we find $p \in \mathbb{P}$: $\nu(p) \neq 0$.\\
If $\nu(q)=0$ for all $q \in \mathbb{P}$, then $\nu=\nu_p$.\\
Assume we have $\nu(p) \neq 0 \neq \nu(q)$ for some $p \neq q \in \mathbb{P}$ and write $1=ap+bq$ for suitable $a,b \in \mathbb{Z}$. Then 
$$0=\nu(1) = \nu(ap+bq) \geqslant \min\{\nu(ap), \nu(bq)\}
= \min\{\underbrace{\nu(a)}_{\geqslant 0 \ (*)} + \nu(p), \underbrace{\nu(b)}_{\geqslant 0 \ (*)} + \nu(q) \} \geqslant \min\{\nu(p), \nu(q)\} > 0$$
Hence a contradiction, i.e. we have $\nu(p) \neq 0$ for at most one $p \in \mathbb{P}$, thus $\nu=\nu_p$. \\
$(*)$ obtain that we have $\nu(1)=\nu(1 \cdot 1 )=\nu(1)+\nu(1) \ \Rightarrow \nu(1)=0$ and by induction $$\nu(a)=\nu(1+(a-1))\geqslant \min\{\nu(1), \nu(a-1)\} \geqslant 0$$

\subsection{Proposition} % Proposition 7.7
Let $\mathbb{K}$ be a field and $\nu:\mathbb{K}^{\times} \longrightarrow \mathbb{Z}$ be a discrete valuation on $\mathbb{K}$.
\begin{compactenum}
\item $\nu(1)=\nu(-1)=0$.
\item $\mathcal{O}_{\nu}:= \{ x \in \mathbb{K}^{\times} \mid \nu(x) \geqslant 0\} \cup \{0\}$ is a ring, called the \textit{valuation ring} of $\nu$. 
\item $\mathfrak{m}_{\nu}:= \left\{x \in \mathbb{K}^{\times} \mid \nu(x) > 0 \right\} \cup \{0\} \triangleleft \mathcal{O}_{\nu}$ is an ideal in $\mathcal{O}_{\nu}$, called the \textit{valuation ideal} of $\nu$. More precisely, $\mathfrak{m}_{\nu}$ is the only maximal ideal in $\mathcal{O}_{\nu}$, i.e. $\mathcal{O}_{\nu}$ is a local ring.
\item $\mathfrak{m}_{\nu}$ is a principal ideal.
\item $\mathcal{O}_{\nu}$ is a principal ideal domain. More precisely, any ideal $I \neq \{0\}$ in $\mathcal{O}_{\nu}$ is of the form $I=\langle t^d \rangle$ for some $d \in \mathbb{N}$ and $t \in \mathfrak{m}_{\nu}$ with $\nu(t)=1$.
\item We have $\mathbb{K}=\rm{Quot}(R)$ and for $x \in \mathbb{K}^{\times}$: $x \in \mathcal{O}_{\nu}$ or $\frac{1}{x} \in \mathcal{O}_{\nu}$.
\end{compactenum}
\textit{proof.}
\begin{compactenum}
\item[(ii)] This is strict calculating, which may be verified by the reader.
\item[(iii)] $\mathfrak{m}_{\nu}$ is an ideal, since for $x,y \in \mathfrak{m}_{\nu}, \alpha \in \mathcal{O}_{\nu}$ we have
$$\nu(x+y) \geqslant \min\{\nu(x), \nu(y) \}>0, \qquad \nu(\alpha x)= \underbrace{\nu(\alpha)}_{\geqslant 0}+\nu(x) \geqslant \nu(x) >0$$
Let now $x \in \mathcal{O}_{\nu}$ with $\nu(x)=0$. Then $$\nu\left(\frac{1}{x}\right)= \nu(1)-\nu(x)=-\nu(x)=0,$$hence $x \in \mathcal{O}_{\nu}^{\times}$.
Thus we have $\mathfrak{m}_{\nu}=\mathcal{O}_{\nu} \setminus \mathcal{O}_{\nu}^{\times}$ and the claim follows.
\item[(iv)] Let $t \in \mathfrak{m}_{\nu}$ such that $\nu(t)=1$. Then for $x \in \mathfrak{m}_{\nu}$ let $\nu(x)=d>0$. \\
Then we have $$\nu\left(x \cdot t^{-d}\right)=\nu(x)+\nu\left(\frac{1}{t^d}\right)=d+0-d=0$$
Define $e:= x \cdot t^{-d} \in \mathcal{O}_{\nu}^{\times}$. Then $x=e \cdot t^d$, hence $\mathfrak{m}_{\nu}=\langle t \rangle$.
\item[(v)] Let $\{0\}\neq I \neq \mathcal{O}_{\nu}$ be an ideal in $\mathcal{O}_{\nu}$.\\
Let $d:= \min\{ \nu(x) \mid x \in I \setminus \{0\} \} >0$. 
\begin{compactitem}
\item['$\supseteq$'] Let $x \in I$ such that $\nu(x)=d$. By part (iv) we have $x=e \cdot t^d$ for some $e \in \mathcal{O}_{\nu}^{\times}$, hence we have $t^d \in I$; thus $\langle t^d \rangle \subseteq I$.
\item['$\subseteq$'] Let now $y \in I \setminus \{0\}$ and write $y= e \cdot t^{\nu(y)}$ for some $e \in \mathcal{O}_{\nu}^{\times}$ and $\nu(y)>d$.\\
Then $y=t^d \cdot e \cdot t^{\nu(y)-d}$, hence $y \in \langle t^d \rangle$ and thus $I \subseteq \langle t^d \rangle$.
\end{compactitem}
\item[(vi)] If $\nu(x) \geqslant 0$, then $x \in \mathcal{O}_{\nu}$. If $\nu(x)<0$, we have
$$\nu\left(\frac{1}{x}\right)= \nu(1)-\nu(x)=-\nu(x)>0, \quad \textrm{ hence } \frac{1}{x} \in \mathfrak{m}_{\nu} \subseteq \mathcal{O}_{\nu}$$
\end{compactenum}

\subsection{Definition} %Definition 7.8
An integral domain $R$ is called a \textit{discrete valuation ring}, if there exists a discrete valuation $\nu$ of $\mathbb{K}=\rm{Quot}(R)$ such that $R=\mathcal{O}_{\nu}$.

\subsection{Proposition} %Proposition 7.9
Let $R$ be a lokal integral domain. Then the following statements are equivalent.
\begin{compactenum}
\item $R$ is a discrete valuation ring.
\item $R$ is a principal ideal domain.
\item There exists $t \in R\setminus \{0\}$ such that every $x \in R \setminus \{0\}$ can uniquely be written in the form $$x=e \cdot t^d \qquad \textrm{ for some } e \in R^{\times}, \ d \geqslant 0$$
\end{compactenum}
\textit{proof.}
\begin{compactitem}
\item['(i) $\Rightarrow$ (ii)'] This follows by 7.7.
\item['(ii) $\Rightarrow$ (iii)'] We know that principal ideal domains are factorial. Let $t \in R$ be a generator of the maximal ideal $\mathfrak{m}$ of $R$. Then $t$ is prime, since any maximal ideal is also prime. Let now $p \in R\setminus \{0\}$ a prime element. Then $p \notin R^{\times}$, hence $p \in \mathfrak{m}$, thus we can write $p=t \cdot x$ for some $x \in R$.  Since $p$ is prime, hence irreducible, we have $x \in R^{\times} \ \Rightarrow \langle p \rangle = \langle t \rangle$.\\
Thus we have $p=t$ and we have only one prime element in $R$. The unique prime factorization in factorial domains gives us $x= e \cdot t^d$ for some $ e \in R^{\times}$ and $d \geqslant 0$. 
\item['(iii)$\Rightarrow$(i)'] For $x= e \cdot t^d \in R \setminus \{0\}$, $e \in R^{\times}, d \geqslant 0$ define $\nu(x)=d$. 
We claim that $\nu$ is discrete valuation. We have
$$\nu(xy)=\nu\left( e t^d \cdot e' t^{d'}\right)=\nu\left(ee't^{d+d'}\right)=\nu\left(e''t^{d+d'}\right)=d+d'$$
Let w.l.o.g. $d \leqslant d'$. Then
$$\nu(x+y)=\nu\left(et^d+e't^{d'}\right) = \nu \left(t^d \left(e+e' t^{d'-d}\right)\right) \geqslant d = \min\{d,d'\}$$
We extend $$\nu: \mathbb{K}^{\times} \longrightarrow \mathbb{Z},  \qquad \nu\left(\frac{x}{y}\right)=\nu(x)-\nu(y)$$ This is well defined:\\
For $\frac{x}{y}=\frac{x'}{y'}$ we have $xy'=x'y$ and $\nu(x'y)=\nu(x)+\nu(y')=\nu(x')+\nu(y)$, thus  $$\ \nu\left(\frac{x}{y}\right)=\nu(x)-\nu(y)=\nu(x')-\nu(y')=\nu\left(\frac{x'}{y'}\right)$$
Finally we have $\nu(t)=1$, hence $\nu:\mathbb{K}^{\times} \longrightarrow \mathbb{Z}$ is surjective.\\
Thus $\nu$ is a discrete valuation on $\mathbb{K}$ and $R=\mathcal{O}_{\nu}$. 
\end{compactitem}

\subsection{Definition + Proposition} % Definition + Proposition 7.10
Let $R$ be a local ring with maximal ideal $\mathfrak{m}$. 
\begin{compactenum}
\item $\mathbb{K}:= \slant{R}{\mathfrak{m}}$ is called the \textit{residue field} of $R$.
\item $\slant{\mathfrak{m}}{\mathfrak{m}^2}$ has a structure of a $\mathbb{K}$-vector space.
\item If $R$ is a discrete valuation ring, then $\dim_{\mathbb{K}}(\slant{\mathfrak{m}}{\mathfrak{m}^2})=1$.
\end{compactenum}
\pagebreak 
\textit{proof.}
\begin{compactenum}
\item[(ii)] For $a \in R$, $x \in \mathfrak{m}$ define $\overline{a} \overline{x}=\overline{ax}$, where $\overline{a}, \overline{x}$ are the images of $a,x$ in $\mathbb{K}$.\\
This is well defined: Let $a' \in R$ with $\overline{a'}=\overline{a}$ and $x'\in \mathfrak{m}$ with $\overline{x'}=\overline{x}$. We have to show that $$\overline{a'x'}=\overline{ax} \ \Longleftrightarrow \ a'x' - ax \in \mathfrak{m}^2$$
We have $\overline{a'}=\overline{a}$, hence $a'=a+y$ for some $y \in \mathfrak{m}$. Analogously we have $\overline{x'}=\overline{x}$, hence $x'=x+$ for some $z \in \mathfrak{m}^2$. Thus we have
$$a'x'=(a+y)(b+z)=ax+az+xy+yz \equiv ax \mod \mathfrak{m}^2$$
\end{compactenum}


%SECTION 8

\renewcommand*\thesection{\S\ \arabic{section}\quad}
\section{The Gauss Lemma}
\renewcommand*\thesection{\arabic{section}}

Let $R$ be a UFD (unique factorization domain), $\mathbb{P}$ a set of representatives of the primes in $R$ with respect to \textit{associateness}, i.e. $x \sim y \ \Leftrightarrow y=u\cdot x$ for some $u \in R^{\times}$.\\
Every $x \in R\setminus \{0\}$ has a unique factorization 
$$x=u \cdot \prod_{p \in \mathbb{P}} p^{\nu_p(x)}, \qquad  \nu_p(x) \geqslant 0 \textrm{ for } p \in \mathbb{P} \textrm, \ u \in R^{\times} $$
where $\nu_p: \mathbb{K}^{\times} \longrightarrow \mathbb{Z}$ is a discrete valuation on $\mathbb{K}=\textrm{Quot}(R)$.

\subsection{Definition + Proposition} %Definition + Proposition 8.1
Let $R$ be a factorial domain, $\mathbb{K}=\textrm{Quot}(R)$ and $$f = \sum_{i=0}^n a_i X^{i} \in \mathbb{K}[X] \setminus \{0\}, \qquad a_n \neq 0$$
\begin{compactenum}
\item For $p \in \mathbb{P}$ let $\nu_p(f)= \min\{\nu_p(a_i) \mid 0 \leqslant i \leqslant n\}$
\item $f$ is called \textit{primitive}, if $\nu_p(f)=0$ for all $p \in \mathbb{P}$. 
\item If $f$ is primitive, then $f \in R[X]$.
\item If $f \in R[X]$ is monic, i.e. $a_n=1$, then $f$ is primitive.
\item There exists $c \in \mathbb{K}^{\times}$ such that $c \cdot f$ is primitive.
\end{compactenum}
\textit{proof.}
\begin{compactenum}
\item[(iii)] For some primitive $$f=\sum_{i=0}^n a_iX^{i} \in \mathbb{K}[X]$$ we have $\min_{1\leqslant i \leqslant n}\{\nu_p(a_i)\}=0$, i.e. $\nu_p(a_i) \geqslant 0$ for all $1 \leqslant i \leqslant n$. Thus $a_i \in R$.
\item[(iv)] If $a_i \in R$ we have $\nu_p(a_i) \geqslant 0$ for all $1 \leqslant i \leqslant n$. Moreover $\nu_p(a_n)=\nu_p(1)=0$, hence $\nu_p(f)=\min_{1\leqslant i \leqslant n}\{\nu_p(a_i)\}=0$. thus $f$ is primitive.
\item[(v)] For $\nu_p(f):=d$ choose $c:=p^{-d} \in \mathbb{K}^{\times}$. Then 
$$\nu_p(c\cdot f)=\nu_p(c)+\nu_p(f)=\nu_p(p^{-d})+d=-d+d=0$$
Thus $c\cdot f$ is primitive.
\end{compactenum}

\titleformat{\subsection}{\normalfont\normalsize\bfseries}{}{0em}{#1 \thesubsection \quad \textnormal{\textit{(Gauss Lemma)}}}
\subsection{Proposition} %Proposition 8.2
For $f,g \in \mathbb{K}[X]$ and $p \in \mathbb{P}$ we have 
$$\nu_p(f\cdot g)=\nu_p(f) + \nu_p(g)$$
\textit{proof.}\\
Write 
$$f=\sum_{i=0}^n a_i X^{i}, \qquad g=\sum_{j=0}^m b_j X^{j}, \qquad f \cdot g = \sum_{k=0}^{m+n}c_k X^{k}, \quad c_k=\sum_{i=0}^k a_i b_{k-i}$$
\begin{compactitem}
\item[\textbf{case 1}] Assume $m=0$, i.e. $g=b_0 \in \mathbb{K}^{\times}$. Then $c_k=a_k \cdot b_0$, hence 
$$\nu_p(c_k)=\nu_p(a_k) + \nu_p(b_0).$$
Then
$$\nu_p(f \cdot g) \  = \ \min_{0\leqslant k \leqslant n} \nu_p(c_k)=\min_{0 \leqslant k \leqslant n}\{\nu_p(a_k)+\nu_p(b_0)\}=\nu_p(b_0)+\min_{0\leqslant k \leqslant n}\{\nu_p(a_k)\}\ = \ \nu_p(g)+\nu_p(f)$$
\item[\textbf{case 2}] Assume $\nu_p(f)=0=\nu_p(g)$, i.e. $f,g$ are primitive. Clearly $\nu_p(fg)\geqslant 0$. To show: $\nu_p(fg)=0$.\\
Let $i_0:=\max\{i \mid \nu_p(a_i)=0\}$ and $j_0:= \max\{j \mid \nu_p(b_j)=0\}$. Then
$$c_{i_0+j_0}=\sum_{i=0}^{i_0+j_0}a_i b_{i_0+j_0-i}=\underbrace{\sum_{i=0}^{i_0-1} a_i b_{i_0+j_0-i}}_{(A)} + a_{i_0+j_0}+ \underbrace{\sum_{i=i_0+1}^{i_0+j_0} a_i b_{i_0+j_0-i}}_{(B)}$$
We have $\nu_p(a_{i_0}b_{j_0})=\nu_p(a_{i_0})+\nu_p(b_{j_0})=0$. Consider (A).\\
We have $i_0+j_0-i>j_0$, hence $\nu_p(b_{i_0+j_0-i}) \geqslant 1$ for $0 \leqslant i \leqslant i_0-1$. Then 
\begin{alignat*}{11}
\nu_p(A)\ &=&&\ \nu_p\left(\sum_{i=0}^{i_0-1}a_i b_{i_0+j_0-i}\right) \quad &&\geqslant && \quad \min_{0 \leqslant i \leqslant i_0-1} \{\nu_p(a_ib_{i_0+j_0-1})\} \ &&=&& \ \min_{0 \leqslant i \leqslant i_0-1} \{v_p(a_i)+\nu_p(b_{i_0+j_0-1})\}\\
&&&\quad &&\geqslant&&\quad \min_{0 \leqslant i \leqslant i_0-1}\{\nu_p(b_{i_0+j_0-1})\}\\
&&&&&\geqslant&& \quad 1\\
\nu_p(B)\ &=&&\ \nu_p\left(\sum_{i=i_0+1}^{i_0+j_0} a_i b_{i_0+j_0-i}\right) \ &&\geqslant&& \quad 1
\end{alignat*}
Since we have 
$$0=\nu_p(a_{i_0}b_{j_0})\geqslant \min\{\nu_p(c_{i_0+j_0}), \nu_p(A), \nu_p(B)\}=\nu_p(c_{i_0+j_0})=0$$
we get $\nu_p(c_{i_0+j_0})=0$. Hence we obtain
$$\nu_p(fg)=\min\{\nu_p(c_i) \mid 0 \leqslant i \leqslant m+n \}=\nu_p(c_{i_0+j_0})=0$$
\item[\textbf{case 3}] Consider now the general case, i.e. $f,g$ are arbitrary.
Multiply $f$ and $g$ by suitable constants $a$ and $b$, such that $\tilde{f}:=af$ and $\tilde{g}:=bg$ are primitive. Then by the first two cases we have
\begin{alignat*}{8}
\nu_p(fg)\quad&=&&\quad \nu_p\left(\frac{1}{a}\frac{1}{b} \tilde{f} \tilde{g}\right) \overset{1}{=}\nu_p\left(\frac{1}{a} \frac{1}{b}\right) + \nu_p(\tilde{f} \tilde{g}) \ &&\overset{2}{=}&& \quad \nu_p\left(\frac{1}{a}\right)+ \nu_p\left(\frac{1}{b}\right)+ \underbrace{\nu_p(\tilde{f})}_{=0}+ \underbrace{\nu_p(\tilde{g})}_{=0}\\
&=&& \quad \nu_p\left(\frac{1}{a}\right)+\nu_p(\tilde{f})+ \nu_p\left(\frac{1}{b}\right)+ \nu_p(\tilde{g}) \ &&=&& \quad \nu_p\left(\frac{1}{a} \tilde{f}\right)+ \nu_p \left(\frac{1}{b} \tilde{g}\right)\\
&=&&  \quad\nu_p(f) + \nu_p(g)
\end{alignat*}
\end{compactitem}

\titleformat{\subsection}{\normalfont\normalsize\bfseries}{}{0em}{#1 \thesubsection\quad \textnormal{\textit{(Eisenstein's criterion for irreducibility)}}}
\subsection{Theorem} %Theorem 8.3
\titleformat{\subsection}{\normalfont\normalsize\bfseries}{}{0em}{#1 \thesubsection}
Let $R$ be a factorial domain, $p \in \mathbb{P}$ and $$f= \sum_{i=0}^n a_i X^{i} \quad \in R[X]\setminus \{0\}$$
Assume that $f$ is primitive and we have
\begin{compactenum}
\item $\nu_p(a_0)=1$,
\item $\nu_p(a_i)\geqslant 1 $ or $a_i=0$ for $1 \leqslant i \leqslant n-1$ and
\item $\nu_p(a_n)=0$
\end{compactenum}
Then $f$ is irreducible over $R[X]$.\\
\textit{proof.}\\
Assume that $f=g\cdot h$ with some $g,h \in R[X]$. Write 
$$g= \sum_{i=0}^r b_i X^{i}, \qquad h=\sum_{j=0}^s c_i X^{j}, \qquad \textrm{ with } r+s=n$$
Then we have $a_0=b_0c_0$. W.l.o.g. $\nu_p(b_0)=1$ and $\nu_p(c_0)=0$.\\
Further $a_n=b_r c_s$, thus we must have $\nu_p(b_r)=\nu_p(c_s)=0$ for $\nu_p(a_n)=0$. \\
Let now $$d:=\max\{i \mid \nu_p(b_j) \geqslant 1 \textrm{ for } 0 \leqslant j \leqslant i \}$$
Obviously $0 \leqslant d \leqslant r-1$. Consider
$$a_{d+1}\ =\ \underbrace{b_{d+1}c_0}_{=:A}\ +\ \underbrace{\sum_{i=0}^d b_i c_{d+1-i}}_{=:B}.$$
We have 
$$\nu_p(A)=\nu_p(b_{d+1})+\nu_p(c_0)=0+0=0$$
$$\nu_p(B) \geqslant \min_{0 \leqslant i \leqslant d}\{\nu_p(b_ic_{d+1-1})\geqslant 1$$
And thus $\nu_p(a_{d+1})=0$. But this implies $d+1=n \Leftrightarrow n-1 =d \leqslant r-1 \Rightarrow n\leqslant r \Rightarrow n=r$.
Then we have $s=0$, thus $h=c_0$ is constant. Further for $q \in \mathbb{P}$ we have
$$0=\nu_q(f)=\nu_q(gc_o)=\underbrace{\nu_q(g)}_{\geqslant 0}+\nu_q(c_0)$$
i.e. $\nu_q(c_0)=0$, hence $c_0 \in R^{\times}$ and $f$ is irreducible.

\titleformat{\subsection}{\normalfont\normalsize\bfseries}{}{0em}{#1 \thesubsection\quad \textnormal{\textit{(Gauss)}}}
\subsection{Theorem} %Theorem 8.4
\titleformat{\subsection}{\normalfont\normalsize\bfseries}{}{0em}{#1 \thesubsection}
Let $R$ be a factorial domain. Then $R[X]$ is factorial.\\
\textit{proof.}\\
Let $f \in R[X]\setminus \{0\} \subseteq \mathbb{K}[X]$ where $\mathbb{K}=\textrm{Quot}(R)$.\\
Since $\mathbb{K}[X]$ is factorial, we can write
$$f=c \cdot f_1 \cdots f_n, \quad f_i \in \mathbb{K}[X] \textrm{ prime }, \ c \in \mathbb{K}^{\times}$$
W.l.o.g the. $f_i$ are primitive, otherse multiply them by suitable constants. In particular we have $f_i \in R[X]$.\\
Note that $c \in R$: For $p \in \mathbb{P}$, we have $$0=\nu_p(f)=\nu_p(c)+ \sum_{i=1}^n \nu_p(f_i)=\nu_p(c).$$
Write $c= \epsilon \cdot p_1 \cdots p_r$ with some $\epsilon \in R^{\times}$ and $p_i \in \mathbb{P}$.Then by
\begin{compactenum}
\item[\textbf{Claim (a)}] $f_i \in R[X]$ are prime for $1 \leqslant i \leqslant n$.
\item[\textbf{Claim (b)}] $p_i \in R[X]$ are prime for $1 \leqslant i \leqslant r$.
\end{compactenum}
we have found a factorization of $f$ into prime elements and hence $R[X]$ is factorial. Now prove the claims.
\begin{compactenum}
\item[\textbf{(a)}] Let $g,h \in R[X]$ such that $gh \in \langle f_i \rangle=f_i R[X]$.\\
May assume that $g \in f_i\mathbb{K}[X]$, i.e. $g=f_i \tilde{g}$ for some $\tilde{g} \in \mathbb{K}[X]$. For $p \in \mathbb{P}$ we obtain
$$0 \leqslant \nu_p(g)=\underbrace{\nu_p(f_i)}_{=0}+\nu_p(\tilde{g})=\nu_p(\tilde{g})$$
Thus we get $\tilde{g} \in R[X]$, which implies $g=f_i \tilde{g} \in f_i R[X]=\langle f_i \rangle$.
\item[\textbf{(b)}] Because $\pi:R \longrightarrow \slant{R}{\langle p \rangle}$ induces a map $\psi: R[X] \longrightarrow \slant{R}{\langle p \rangle}[X]$ with $\ker(\psi)=pR[X]$ we have
We have $$\slant{R[X]}{p R[X]} \cong \slant{R}{pR}[X].$$
Since $\slant{R}{pR}$ is an integral domain, $\langle p \rangle$ is prime.
\end{compactenum}

\subsection{Corollary} %Corollary 8.5
Let $\mathbb{K}$ be a field. Then $\mathbb{K}[X_1, \ldots X_n]$ is factorial for any $n \in \mathbb{N}$.\\


\subsection{Corollary} %Corollary 8.6
Let $R$ be a factorial domain, $\mathbb{K}=\textrm{Quot}(R)$.\\
If $f \in R[X]$ is irreducible over $R[X]$, then $f$ is irreducible over $\mathbb{K}[X]$\\
\textit{proof.}\\
Let $0\neq f = c \cdot f_1 \cdots f_n$ be decomposition of $f$ in $\mathbb{K}[X]$, i.e. $c \in \mathbb{K}^{\times}$ and $f_i \in \mathbb{K}[X]$ irreducible for $1 \leqslant i \leqslant n$. We may assume that the $f_i$ are primitive, hence contained in  $R[X]$, since we can multiply them by suitable constants. We still have to show $c \in R$. Since $f \in \mathbb{K}[X]$, i.e. $\nu_p(f) \geqslant 0$ we have
$$\nu_p(f)=\nu_p(c \cdot f_1 \cdots f_n)=\nu_p(c) + \sum_{i=1}^n \underbrace{\nu_p(f_i)}_{=0}=\nu_p(c)\overset{!}{\geqslant} 0$$
Thus $c \in R$. Then the decomposition from above is in $R$ - but since $f$ is irreducible in $R$, we have $n=1$ and $c \in R^{\times}$.

%SECTION 9

\renewcommand*\thesection{\S\ \arabic{section}\quad}
\section{Absolute values}
\renewcommand*\thesection{\arabic{section}}

\subsection{Definition} %Defintion 9.1
Let $\mathbb{K}$ be a field. A map $$| \cdot |: \mathbb{K} \longrightarrow \mathbb{R}_{\geqslant 0}$$ is called an \textit{absolute value}, if
\begin{compactenum}
\item \textit{positive definiteness:} $|x|=0 \ \Longleftrightarrow \ x=0$
\item \textit{multiplicativeness:} $|xy|=|x|\cdot |y|$ for all $x,y \in \mathbb{K}$.
\item \textit{triangle inequality:} $|x+y| \leqslant |x|+|y|$ for all $x,y \in \mathbb{K}$.
\end{compactenum}

\titleformat{\subsection}{\normalfont\normalsize\bfseries}{}{0em}{#1}
\subsection*{Example} %Example
\titleformat{\subsection}{\normalfont\normalsize\bfseries}{}{0em}{#1 \thesubsection}
\begin{compactenum}
\item The 'normal' absolute value $|\cdot|_{\infty}$ on $\mathbb{C}$ and on any of its subfields denotes an absolute value.
\item Let $\nu_\mathbb{K}^{\times} \longrightarrow \mathbb{Z}$ be a discrete valuation, $\rho \in (0,1)$. Then 
$$|\cdot |_{\nu}:\mathbb{K} \longrightarrow \mathbb{R}, \ x \mapsto \begin{cases} \rho^{\nu(x)} & x \neq 0 \\ 0 & x=0 \end{cases}$$
is an absolute value on $\mathbb{K}$, since
\begin{compactenum}
\item Trivial, since $|0|=0$ and $\rho^x \neq 0$ for any $x \in \mathbb{Z}$.
\item Clearly $|xy|_{\nu}=\rho^{\nu(xy)}=\rho^{\nu(x)+\nu(y)}=\rho^{\nu(x)} \rho^{\nu(y)}=|x|_{\nu}|y|_{\nu}$.
\item Further 
$$|x+y|_{\nu}\ = \ \rho^{\nu(x+y)} \leqslant \rho^{\min\{\nu(x), \nu(y)\}} = \max\{\rho^{\nu(x)}, \rho^{\nu(y)}\}=\max\{|x|_{\nu}, |y|_{\nu}\} \ \leqslant \  |x|_{\nu}+|y|_{\nu}$$
\end{compactenum}
\item For the $p$-adic valuation $\nu_p$ on $\mathbb{Q}$ we choose $\rho:=\frac{1}{p}$. Then $|x|_p=p^{-\nu_p(x)}$ is an absolute value.
\end{compactenum}

\subsection{Remark + Definition} %Remark + Definition 9.2
Let $\mathbb{K}$ be a field, $|\cdot|$ an absolute value on $\mathbb{K}$. 
\begin{compactenum}
\item $|1|=|-1|=1$ and $|x|=|-x|$ for all $x \in \mathbb{K}$.
\item The absolute value is called trivial, if $|x|=1$ for all $x \in \mathbb{K}$.
\end{compactenum}
\textit{proof.}\\
We have $|1|=|1\cdot 1|=|1| \cdot |1|$, hence $|1|=1$. Moreover $|-1|=|1\cdot (-1)|=|1| \cdot |-1|$, hence $|-1|=1$.
For $x \in \mathbb{K}$ we get
$$|-x|=|(-1) \cdot x|=|-1| \cdot |x|=|x|$$

\subsection{Proposition + Definition} %Proposition + Definiiton 9.3
Let $\mathbb{K}$ be a field with $\textrm{char}(\mathbb{K})=0$, i.e. $\mathbb{K}\supseteq \mathbb{Q}$ and $|\cdot |$ an absolute value on $\mathbb{K}$. 
\begin{compactenum}
\item $|\cdot|$ is called \textit{archimedean}, if $|n|>1$ for all $n \in \mathbb{Z}\setminus \{-1,0,1\}$.
\item $|\cdot|$ is called \textit{nonarchimedean}, if $|n| \leqslant 1$ for all $n \in \mathbb{Z}$.
\item $|\cdot |$ is either archimedean or nonarchimedean.
\item The $p$-adic absolute value on $\mathbb{Q}$ is nonarchimedean.
\end{compactenum}
\textit{proof (iii).}\\
Since $|n|=|-n|$, it suffices to check $n \in \mathbb{N}$.
Let $a \in \mathbb{N} \subseteq \mathbb{K}$ with $|a|>1$. Assume there exists $b \in \mathbb{N}_{>1}$ with $|b| \leqslant 1$. Write
$$a= \sum_{i=0}^N \alpha_i b^{i} \qquad \alpha_i \in \{0, \ldots b-1\}, \ |N|=\lfloor \log_b(a) \rfloor$$
Then we have 
$$|a| \ \leqslant\ \sum_{i=0}^{\lfloor \log_b(a)\rfloor} |\alpha_i | |b|^{i} \ \leqslant\ \log_b(a) \cdot \max_{0 \leqslant i \leqslant \lfloor \log_b(a) \rfloor}\{|\alpha_i|\}\ =:\ \log_b(a) \cdot c$$
$$|a^n|\leqslant \log_b(a^n) \cdot c \ = \ n \cdot \log_b(a) \cdot c$$
and $|a^n|$ grows linearly in $n$. Likewise we get for $n \in \mathbb{N}$
$$a^n= \sum_{i=0}^{\lfloor \log_b(a^n) \rfloor} \alpha_i^{(n)} b^{i}, \qquad \alpha_i^{(n)} \in \{0, \ldots b-1\}$$



$$|a^n|=|a|^n\ \leqslant \ \left(\log_b(a) \cdot c\right)^n$$
which grows exponentially in $n$, which is a contradiction. Hence the claim follows.

\subsection{Remark} %Remark 9.4
An absolute value $|\cdot|$ on a field $\mathbb{K}$ induces a metric
$$d(x,y):=|x-y|, \qquad x,y \in \mathbb{K}$$
Therefore, $\mathbb{K}$ as a topology and aspects as 'convergence' and 'cauchy sequences' are meaningful.

\subsection{Definition + Remark} %Definition + Remark 9.5
\begin{compactenum}
\item Two absolute values $|\cdot|_1, |\cdot|_2$ on $\mathbb{K}$ are called \textit{equivalent}, if there exists $s \in \mathbb{R}$, such that $|x|_1=|x|_2^s$ for all $x \in \mathbb{K}$. In this case, we write $|\cdot |_1 \sim |\cdot |_2$.
\item Two absolutes values $|\cdot|_1, |\cdot|_2$ are equivalent if and only if the induce the same topology on $\mathbb{K}$.
\end{compactenum}
\textit{proof.}\\
Is left for the reader as an exercise.

\subsection{Example} %Example 9.6
The $p$-adic absolute values on $\mathbb{Q}$ are not equivalent for $p \neq q \in \mathbb{P}$. Consider
$$|p^n|_p = p^{-n} \ \xrightarrow{n \to \infty}\ 0, \qquad |p^n|_q=1 \ \textrm{ for all } n \in \mathbb{N}$$
Moreover we have $|\cdot |p \nsim |\cdot |_{\infty}$, since by the transittivity of equivalence of absolute values, we have
$$|\cdot |_p \sim |\cdot |_{\infty} \sim |\cdot |_{q}$$
which is not true.

\titleformat{\subsection}{\normalfont\normalsize\bfseries}{}{0em}{#1 \thesubsection\quad \textnormal{\textit{(Ostrowski)}}}
\subsection{Theorem} %Theorem 9.7
\titleformat{\subsection}{\normalfont\normalsize\bfseries}{}{0em}{#1 \thesubsection}
Any nontrivial absolute value $| \cdot|$ on $\mathbb{Q}$ is equivalent either to the standard absolute value $|\cdot|_{\infty}$ on $\mathbb{Q}$ or to a $p$-adic absolute value $|\cdot|_p$ for some $p \in \mathbb{P}$.\\
\textit{proof.}\
\begin{compactenum}
\item[\textbf{case 1}] Assume $| \cdot |$ is nonarchimedean. We want to show, that in this case $|\cdot | \sim |\cdot |_p$ for some $p \in \mathbb{P}$.\\
Since $| \cdot|$ is non-trivial, there exists $x \in \mathbb{N}$ such that $$|x|=\bigg \vert \prod_{p \in \mathbb{P}} p^{\nu_p(x)}\bigg \vert=\prod_{p \in \mathbb{P}} |p|^{\nu_p(x)} \neq 1 $$ for at least one $x \in \mathbb{Q}$, hence, we have $|p| \neq 1$ for at least one $p \in \mathbb{P}$, i.e. $|p|<1$.\\
Assume there is another prime $q \neq p$ with $|q|<1$. Then we find $N \in \mathbb{N}$, such that
$$\vert p \vert ^N \lneq \frac{1}{2}, \qquad \vert q \vert ^N \lneq \frac{1}{2}$$
Moreover, since $p^N, q^N$ are coprime, we  can write
$$1=a \cdot p^N+b \cdot q^N \qquad \textrm{ for suitable } a,b \in \mathbb{Z}$$
So the contradiction follows by
$$1=|1|=\big\vert ap^N+bq^N \big\vert \leqslant \underbrace{\vert a\vert}_{\leqslant 1} \underbrace{\big\vert p^N\big\vert}_{<\frac{1}{2}} + \underbrace{\vert b \vert}_{\leqslant 1} \underbrace{\big\vert q^N \big\vert}_{< \frac{1}{2}} < 1$$
Hence we have $|q|=1$ for any $q\neq p \in \mathbb{P}$. Let now $s:= -\log_p|p|$. For $x \in \mathbb{Q}^{\times}$ we obtain
$$|x|=\bigg\vert \prod_{\tilde{p}\in \mathbb{P}}\tilde{p}^{\nu_{\tilde{p}}(x)} \bigg\vert = \prod_{\tilde{p} \in \mathbb{P}} \vert \tilde{p}\vert ^{\nu_{\tilde{p}}(x)}=|p|^{\nu_p(x)}=p^{-s \cdot \nu_p(x)}=\left(p^{-\nu_p(x)}\right)^s=|x|_p^s$$
Hence $|\cdot| \sim |\cdot|_p$.
\item[\textbf{case 2}] Let now $|\cdot|$ be archimedean. We now have to show $|\cdot| \sim |\cdot|_{\infty}$.
For $n \in \mathbb{N}_{\geqslant 2}$ we have
$$1 < |n| = \bigg\vert\sum_{i=1}^n 1\bigg\vert \leqslant \sum_{i=1}^n \vert 1 \vert = n$$
For any $a \in \mathbb{N}_{\geqslant 2}$ we find $s:=s(a) \in \mathbb{R}_{<0}$ such that
$$|a|=|a|_{\infty}^s=a^s$$
namely
$$s=\log_a(|a|)=\frac{\log(|a|)}{\log(a)}$$
\begin{compactenum}
\item[\textbf{Claim (a)}] We have $$\frac{\log(|a|)}{\log(a)}=\frac{\log(|2|)}{\log(2)}$$
\end{compactenum}
Since now $s$ is independent of $a$, we have $|\cdot|\sim |\cdot |_{\infty}$.\\
Prove now the claim:
\begin{compactenum}
\item[\textbf{(a)}]
For $n \in \mathbb{N}$ write 
$$2^n= \sum_{i=0}^N \alpha_i a^{i} \ \textrm{ with }\alpha_i \in \{0, \ldots a-1\} \textrm{ and } N \leqslant \log_a2^n=n \cdot \frac{\log(2)}{\log(a)}$$
Then we have
$$\vert 2\vert ^n= \vert2^n \vert \leqslant \sum_{i=0}^N \underbrace{\vert \alpha_i \vert}_{\leqslant \alpha_i <a} \overbrace{\vert a \vert ^{i}}{\leqslant \vert a \vert ^N} \leqslant (N+1) ^\cdot a \cdot \vert a \vert ^N$$
Hence we get
\begin{alignat*}{5}
n \cdot \log(|2|) \ &\leqslant && \ \log(N+1)+\log(a)+N\log(|a|)\\
& \leqslant && \  \log\left(n \cdot \frac{\log(2)}{\log(a)}+1\right)+\log(a)+n\cdot \frac{\log(2)}{\log(a)}\cdot \log(|a|)
\end{alignat*}
Multiplying the equation by $\frac{1}{n} \cdot \frac{1}{\log(2)}$ gives us
$$\frac{\log(|2|)}{\log(2)} \leqslant \frac{1}{n} \cdot \log\left(n \cdot \frac{\log(2)}{\log(a)}+1\right)+ \frac{\log(|a|)}{\log(a)}$$ and thus 
$$\frac{\log(|2|)}{\log(2)} \leqslant \frac{\log(|a|)}{\log(a)}$$
Swapping the roles of $a$ and $2$ in the equation above gives us the other inequality. Hence we have equality, which proves the claim.
\end{compactenum}
\end{compactenum}

\subsection{Proposition} %Proposition 9.8
Let $|\cdot|$ be a nonarchimedean absolute value on a field $\mathbb{K}$.
\begin{compactenum}
\item $|x+y| \leqslant \max\{|x|, |y|\}$ for all $x,y \in \mathbb{K}$.
\item If $|x| \neq |y|$, then equality holds in (i).
\end{compactenum}
\textit{proof.}
\begin{compactenum}
\item If $x=0$, we have $|y+x|=|y|\leqslant \max\{0,|y|\}=\max\{|x|,|y|\}$.\\
Thus assume $x \neq 0$. We have $|x+y|=|x|\big\vert 1+\frac{y}{x}\big \vert$.\\
It suffices to show $|x+1| \leqslant \max\{1, |x|\}$. Then we get
$$|x+y|=|y| \cdot \Big\vert 1+\frac{x}{y}\Big\vert \leqslant |y| \cdot \max\left\{\Big\vert \frac{x}{y}\Big \vert, \vert1\vert\right\} \leqslant \max\{|x|, |y|\}$$
For $n \in \mathbb{N}$ we have
$$(x+1)^n=\sum_{k=0}^n \begin{pmatrix}n \\k \end{pmatrix} x^k$$
Then we have
$$|x+1|^n\ =\ \vert(x+1)^n \vert \ =\ \bigg\vert \sum_{k=0}^n \begin{pmatrix}n \\k\end{pmatrix} x^k \bigg \vert\ \leqslant \ \sum_{k=0}^n \underbrace{\bigg\vert \begin{pmatrix}n \\k \end{pmatrix} \bigg \vert}_{\leqslant 1} {\underbrace{\vert x \vert}_{\leqslant 1} }^k \ \leqslant \ n+1$$
Hence $$|x+1| \leqslant \sqrt[n]{n+1} \qquad \textrm{ for all }n \in \mathbb{N}$$thus $|1+x| \leqslant 1$. Since we clearly have $|x+1| \leqslant |x|$, we all in all have $$|x+1| \leqslant \max|\{|x|, 1\}.$$
\item Let $z=x+y$ and assume $|x|<|y|$. We have to show $|z|=|y|$. Assume $|z|<|y|$. Then
$$|y|=|z-x| \overset{(i)}{\leqslant} \max\{|z|, |-x|\} < |y| \quad \lightning$$
\end{compactenum}

\subsection{Proposition} %Proposition 9.9
Let $|\cdot|$ be an a nonarchimedean absolute value on a field $\mathbb{K}$. Then 
\begin{compactenum}
\item We have a local ring $$\overline{\mathcal{B}}_1(0):=\{x \in \mathbb{K} \big \vert |x|\leqslant 1 \}=:\mathcal{O}_{\mathbb{K}}$$with maximal ideal $$\mathcal{B}_1(0):=\{x \in \mathbb{K} \big \vert |x| < 1 \}=:\mathfrak{m}_{\mathbb{K}}$$
\item Every point in  ball is its center.
\item Balls are either disjoint or one of them is contained in the other one.
\item All triangles are isosceles.
\end{compactenum}
\textit{proof.}
\begin{compactenum}
\item By 9.8(i), $\mathcal{B}_1(0)$ is closed under Addition. The remaining is calculating.
\item Let $z \in \overline{\mathcal{B}}_r(x)$. To show: $\overline{\mathcal{B}}_r(z)=\overline{\mathcal{B}}_r(x)$.
\begin{compactitem}
\item['$\subseteq$'] Let $y \in \overline{\mathcal{B}}_r(z)$, i.e. we have $|y-z|\leqslant r$. Then
$$|y-x|=|y-z+z-x| \leqslant \max\{|y-z|, |z-x|\} \leqslant r \quad \Rightarrow \quad y \in \overline{\mathcal{B}}_r(x) $$
Thus we have $\overline{\mathcal{B}}_r(z) \subseteq \overline{\mathcal{B}}_r(x)$.
\item['$\supseteq$'] Follows by symmetry.
\end{compactitem}
\item Let $\mathcal{B}:= \overline{\mathcal{B}}_r(x)$, $\mathcal{B}':=\overline{\mathcal{B}}_{r'}(x')$ and $y \in \mathcal{B} \cap \mathcal{B'}$. W.l.o.g. $r \leqslant r'$.\\
Then for $z \in \mathcal{B}$ we have
$$|z-x'|=|z-x+x-y+y-x'| \leqslant \max\{|z-x|, |x-y|, |y-x'|\} = \max\{r,r,r'\}=r'$$
which implies $z \in \mathbb{B}'$. Hence we have $\mathcal{B} \subseteq \mathcal{B}'$. 
\item Follows from 9.8(ii).
\end{compactenum}

\subsection{Corollary}%Corollary 9.10
Let $\mathbb{K}$ be a field, $|\cdot|$ a nonarchimedean absolute value on $\mathbb{K}$.
\begin{compactenum}
\item All balls are closed and open, considering the topology on $\mathbb{K}$ induced by the metric $d(x,y)=|x-y|$.
\item $\mathbb{K}$ is totally disconnected, i.e. no subset of $\mathbb{K}$ containing more than on element is connected.
\end{compactenum}
\textit{proof.}
\begin{compactenum}
\item Let $\mathcal{B}:= \overline{\mathcal{B}}_r(x)$ be a closed ball for some $x \in \mathbb{K},$ $r \in \mathbb{R}_{\geqslant 0}$. Then $\mathcal{B}$ topologically clearly is  closed . Let now $y \in \mathcal{B}$. Then $\mathcal{B}_r(y) \subseteq \mathcal{B}$ by 9.9(ii), i.e. $\mathcal{B}$ is open.\\
Let now $\mathcal{B}:=\mathcal{B}_r(x)$ be an open ball and $y \in \mathbb{K}$ a boundary point. Thus for all $s>0$ we find $z \in \mathcal{B}_{s}(x) \cap \mathcal{B}_r(x)$. Choose $s \leqslant r$. Then 
$$d(x,y)\leqslant \max\{d(y,z), d(x,z)\} < \max\{s,r\}=r$$
Thus $y \in \mathcal{B}_r(x)$, hence $\mathcal{B}_r(x)$ is contains its boundary and is closed.
\item Let $X \subseteq \mathbb{K}$ be a subset with $x \neq y \in X$. Then for $r:= |x-y| >0$ we get
$$X=\left(\overline{\mathcal{B}}_{\frac{r}{2}}(x) \cap X\right) \cup \left(X \setminus \overline{\mathcal{B}}_{\frac{r}{2}}(x)\right)$$
which is a decomposition of $X$ into two nonempty, disjoint open subset, i.e. the claim follows.
\end{compactenum}

\titleformat{\subsection}{\normalfont\normalsize\bfseries}{}{0em}{#1 \thesubsection\quad \textnormal{\textit{(Geometry on $(\mathbb{Q}, |\cdot|_p)$)}}}
\subsection{Example }%Example 9.11
\titleformat{\subsection}{\normalfont\normalsize\bfseries}{}{0em}{#1 \thesubsection}
The unit disc in $(\mathbb{Q}, |\cdot|_p)$ is
$$\left\{ \frac{a}{b} \in \mathbb{Q} \ \big\vert\ p\nmid b \right\} =: \mathbb{Z}_{\langle p\rangle}$$
The maximal ideal is
$$\left\{\frac{a}{b} \in \mathbb{Q} \ \big\vert\ p \nmid b, p\mid a \right\}=p\cdot \mathbb{Z}_{\langle p\rangle}=\overline{\mathcal{B}}_{\frac{1}{p}}(0)$$
We have
$$\left\{x \in \mathbb{Q} \ \big\vert\ |x|_p < 1\right\}=\left\{x \in \mathbb{Q} \ \big\vert\ |x|_{\infty} < \frac{1}{p} \right\}$$
Moreover
$$\slant{\mathbb{Z}_{\langle p\rangle}}{p \mathbb{Z}_{\langle p\rangle}} \cong \slant{\mathbb{Z}}{p\mathbb{Z}}=\mathbb{F}_p=\{\overline{0}, \overline{1}, \ldots, \overline{p-1}\}$$
$\overline{\mathcal{B}}_1(0)$ is the disjoint union of the $\overline{\mathcal{B}}_{\frac{1}{p}}(i)$ for $0\leqslant i \leqslant p-1$, where $\overline{\mathcal{B}}_{\frac{1}{p}}(i)=i+p\mathbb{Z}_{\langle p\rangle}$.

%SECTION 10

\renewcommand*\thesection{\S\ \arabic{section}\quad }
\section{Completions, \textit{p}-adic numbers and Hensel's Lemma}
\renewcommand*\thesection{\arabic{section}}

\subsection{Remark} %Remark 10.1
Let $|\cdot|$ be an absolute value on a field $\mathbb{K}$. Let
$$\mathcal{C}:= \{(a_n)_{n \in \mathbb{N}} \ \big \vert \ (a_n) \textrm{ is Cauchy sequence in }(\mathbb{K}, |\cdot |) \}$$
be th ring (!) of Cauchy sequences in $\mathbb{K}$ and 
$$\mathcal{N}:= \left\{(a_n)_{n \in \mathbb{N}}\ \big\vert \ \lim_{n \to \infty}a_n=0\right\} \trianglelefteqslant \mathcal{C}$$
the ideal (!) of Cauchy sequences converging to $0$.
Then
\begin{compactenum}
\item $\mathcal{N}$ is a maximal ideal.
\item $\mathbb{K}':=\slant{\mathcal{C}}{\mathcal{N}}$ is a field extension of $\mathbb{K}$.
\item $\vert \overline{(a_n)_{n \in \mathbb{N}}} \vert := \lim_{n \to \infty} (a_n) \in \mathbb{R}_{\geqslant 0}$ is an absolute value on $\mathbb{K}'$ extending $|\cdot|$.
\item $\mathbb{K}'$ is complete with respect to $|\cdot|$.
\end{compactenum}

\subsection{Remark}%Remark 10.2
If $|\cdot|$ is nonarchimedean, for every Cauchy sequence $(a_n)_{n \in \mathbb{N}} \notin \mathcal{N}$ we have $|a_m|=|a_n|$ for all $m,n \gg 0.$
\pagebreak\\
\textit{proof.}\\
Since $(a_n) \notin \mathcal{N}$, $0$ is not an accumulation point of $(a_n)$.\\
$\Longrightarrow$ $|a_n| \geqslant \epsilon$ for some $\epsilon >0$ and all $n \geqslant n_0(\epsilon)=:n_0$.\\
Thus for $n,m \geqslant n_0$ we have $|a_n-a_m| < \epsilon$. This implies by 9.8 (ii)
$$|a_n-a_m| \lneq \max\{|a_n|, |a_m|\} \ \Longrightarrow \ |a_n|=|a_m|$$

\subsection{Definition}%Definition 10.3
Let $\mathbb{K}=\mathbb{Q}$, $|\cdot|=|\cdot|_p$ for some $p \in \mathbb{P}$. Then the field $\mathbb{K}'$ on 10.1 is called the field of $p$\textit{-adic numbers} and denoted by $\mathbb{Q}_p$. The valuation ring is called the ring of $p$-\textit{adic integers} and is denoted by $\mathbb{Z}_p$.

\subsection{Remark}%Remark 10.4
\begin{compactenum}
\item $\mathbb{Z} \subset \mathbb{Z}_{\langle p\rangle} \subset \mathbb{Z}_p$.
\item The maximal ideal in $\mathbb{Z}_p$ is $p\mathbb{Z}_p$.
\item $\slant{\mathbb{Z}_p}{p\mathbb{Z}_p} \cong \slant{\mathbb{Z}}{p\mathbb{Z}} =\mathbb{F}_p$.
\item $\mathbb{Z}_p$ is a discrete valuation ring. 
\end{compactenum}
\textit{proof.}
\begin{compactenum}
\item The first inclusion is clear. For the second one consider $x =\frac{r}{s} \in \mathbb{Z}_{\langle p \rangle}$. Then by definition of localization we have $p \nmid s$ and hence
$$\vert x \vert=\Big\vert \frac{r}{s}\Big\vert=\frac{|r|}{|s|}=|r|\leqslant 1$$
and thus $x\in \mathbb{Z}_{p}$.Now prove that $\mathbb{Z}$ is dence in $\mathbb{Z}_p$:\\
Let $x \in \mathbb{Z}_p$ with $p$-adic expansion
$$x=\sum_{i=0}^{\infty}a_i p^{i}, \qquad a_i \in \{0,1, \ldots, p-1\}$$
Define a sequence $(x_n)_{n \in \mathbb{N}}$ by
$$x_n:=\sum_{i=0}^n a_i p^{i} \in \mathbb{Z}$$
Then we have
$$\vert x-x_n\vert =\Big\vert \sum_{i=n+1}^{\infty}\Big\vert = \max_{i\geqslant n+1}\{\vert p^{i} \vert \}=\big\vert p^{n+1} \big\vert =p^{-(n+1)} \xrightarrow{n \to \infty} 0 $$
Hence $\mathbb{Z}$ is dence in $\mathbb{Z}_p$. 
\item Recall that the maximal ideal is given by
$$\mathfrak{m}=\{x \in \mathbb{Z}_p \ \mid \ |x|<1 \} \ \overset{!}{=} p \mathbb{Z}_p$$
\begin{compactenum}
\item['$\subseteq$'] Let $x \in \mathfrak{m}$, i.e. $|x|<1$. Thus we have $|x|< \big\vert \frac{1}{p} \big\vert$.\\
This implies
$$\vert p^{-1}  x \vert \leqslant 1 \ \Longleftrightarrow \ p^{-1} x \in \mathbb{Z}_p$$
and thus $p^{-1}x = y$ for some $y \in \mathbb{Z}_p$. Then we have $x=p y \in p\mathbb{Z}_p$.
\item['$\supseteq$'] Let $x \in p\mathbb{Z}_p$, i.e. we can write $x=py$ for some $y \in \mathbb{Z}_p$. Then \\
$|x|=|py|=|p||y|<1$ and hence $x\in \mathfrak{m}$.
\end{compactenum}
\item Consider the surjective homomorphism
$$\psi_p: \mathbb{Z}_p \longrightarrow \slant{\mathbb{Z}}{p \mathbb{Z}}, \quad x=\sum_{i=0}^n a_ip^{i} \mapsto a_0$$
We have
$$\ker(\psi_p)=\{x \in \mathbb{Z}_p  \ \mid \ a_0 \equiv 0 \mod p\}=p\mathbb{Z}_p$$
Thus we get
$\slant{\mathbb{Z}_p}{p\mathbb{Z}_p} \cong \slant{\mathbb{Z}}{p\mathbb{Z}}$ by homomorphism theorem.
\item The absolute value $\vert \cdot \vert=\vert \cdot \vert_p$ on $\mathbb{Q}_p$ induces a discrete valuation $\nu$ on $\mathbb{Q}_p^{\times}$. With respect to this valuation we have
$$\mathcal{O}_{\nu}=\{x \in \mathbb{Q}_p \ \mid \ \nu(x) \geqslant 0\} \cup \{0\} = \{x \in \mathbb{Q}_p \ \mid \ |x|\leqslant 1 \} = \mathbb{Z}_p$$
\end{compactenum}

\subsection{Proposition}% Proposition 10.5
\begin{compactenum}
\item Any $x \in \mathbb{Z}_p$ can uniquely be written in the form 
$$x= \sum_{i=0}^{\infty} a_i p^{i}, \qquad a_i \in \{0,1, \ldots, p-1\}.$$
\item Any $x \in \mathbb{Q}_p$ can uniquely be written in the form 
$$x=\sum_{i=-m}^{\infty} a_i p^{i}, \qquad m \in \mathbb{Z}, \ a_i \in \{0,1, \ldots, p-1\},\ a_m\neq0.$$
\end{compactenum}
\textit{proof.}
\begin{compactenum}
\item We first obtain, that any series
$$\sum_{i=0}^{\infty}a_i p^{i}, \qquad a_i \in \{0, \ldots, p-1\}$$
converges, since for $n>m$ we have
$$\bigg\vert \sum_{i=0}^n a_i p^{i}-\sum_{i=0}^m a_ip^{i} \bigg\vert = \bigg\vert\sum_{i=n+1}^m a_ip^{i} \bigg\vert = \big\vert p^{m+1}\big\vert \underbrace{\bigg\vert \sum_{i=n+1}^m a_i p^{i-(m+1)} \bigg\vert}_{\leqslant 1} \leqslant \big\vert p^{m+1} \big\vert$$
\begin{compactenum}
\item[\textbf{uniqueness}] Let
$$x=\sum_{i=0}^{\infty} a_i p^{i} = \sum_{i=0}^{\infty} b_i p^{i}, \qquad a_i, b_i \in \{0,1, \ldots, p-1\}$$
representations of $x \in \mathbb{Q}_p$. Assume they are different, the let $i_o:= \min\{i \in \mathbb{N}_0 \mid a_i \neq b_i\}$. Then 
$$0 \ = \ \bigg\vert \sum_{i=0}^{\infty} a_i p^{i} - \sum_{i=0}^{\infty} b_i p^{i} \bigg\vert \ =\ \Bigg\vert \underbrace{p^{i_0} (a_{i_0}-b_{i_0})}_{=:A} \ + \ p^{i_0+1} \cdot \underbrace{\left(\sum_{i=i_0+1}^{\infty} a_i p^{i-(i_0+1)} - \sum_{i=i_0+1}^{\infty} b_i p^{i-(i_0+1)} \right)}_{=:B}\Bigg\vert $$
We obtain
$\nu_p(A)=p^{-i_0}$
and
$$ B \in \mathbb{Z}_p, \quad \nu_p\left(p^{i_0+1} \cdot B\right) = \nu_p\left(p^{i_0+1}\right) \underbrace{\nu_p(B)}_{\leqslant 1} \leqslant \nu_p\left(p^{i_0+1}\right)=p^{-(i_0+1)}$$
So all in all
$$0=\big\vert A+ p^{i_0+1} \cdot B \big\vert \overset{9.8(ii)}{=} \max\{p^{-i_0}, p^{-(i_0+1)}\} = p^{-i_0} \ \lightning $$
\item[\textbf{existence}] Look at $\overline{x} \in \slant{\mathbb{Z}_p}{p \mathbb{Z}_p} = \mathbb{F}_p$.\\
Let $a_0$ be the representative of $x$ in $\{0,1, \ldots, p-1\}$. Then we have
$$|x-a_0| < 1 \ \Leftrightarrow \ |x-a_0| \leqslant \frac{1}{p}.$$
In the next step, let $a_1$ be the representative of $\frac{1}{p}(x-a_0)$ in $\{0,1, \ldots, p-1\}$. Then we have 
$$\bigg\vert \frac{1}{p} (x-a_0) - a_1 \bigg\vert = \bigg\vert \frac{1}{p} \bigg\vert \vert x-a_0-a_1p \vert \leqslant \frac{1}{p}$$
And thus
$$\vert x-a_0-a_1 p \vert \leqslant \frac{1}{p^2}$$
Inductively we let $a_n$ be the representative of $$\frac{1}{p^n}(x-a_0-a_1p- \ldots -a_{n-1}p^{n-1} )=\frac{1}{p^n} \left(x- \sum_{i=0}^{n-1} a_i p^{i} \right)$$
in $\{0,1, \ldots, p-1\}$. Then we have
$$\bigg\vert x- \sum_{i=0}^{n-1} a_i p^{i} \bigg\vert \leqslant \frac{1}{p^{n+1}}$$
and finally
$$\lim_{n \to \infty} \bigg\vert x- \sum_{i=0}^{n-1} a_i p^{i} \bigg\vert \leqslant \lim_{n \to \infty} \frac{1}{p^{n+1}} =0 \ \Longrightarrow \ x = \sum_{i=0}^{\infty} a_i p^{i}$$
\end{compactenum}
\item  If $|x|=p^m$ for some $m \in \mathbb{Z}$, we have $$\vert x \cdot p^m\vert = \vert d \vert \cdot \vert p^m \vert = p^m \cdot p^{-m}=1, \qquad \textrm{ i.e. }x \cdot p^m\in \mathbb{Z}_p^{\times}$$
By part (i) we conclude $$x \cdot p^m= \sum_{i=0}^{\infty} a_i p^{i}, \quad a_0 \neq 0$$
Thus we have
$$x= \frac{1}{p^m} \cdot x \cdot p^m = \frac{1}{p^m} \cdot \sum_{i=0}^{\infty} a_i p^{i} = \sum_{i=-m}^{\infty} a_{i+m}p^{i}$$
\end{compactenum}

\subsection{Remark}%Remark 10.6
What is $-1$ in $\mathbb{Q}_p$? We have\\
$a_0=p-1$, since $\overline{p-1}-\overline{(-a)}=\overline{p}=0$.\\
$a_1$ is the representative of $\frac{1}{p}\left(-1-(p-1)\right)=-1$, i.e. $a_1=p-1$.\\
$a_2$ is the representative of $\frac{1}{p^2}\left(-1-(p-1)-(p-1)p\right)=-1$, i.e. $a_2=p-1$.\\
Inductively we have $a_n=p-1$ for all $n \in \mathbb{N}_0$, so we get
$$-1\ =\ \sum_{i=0}^{\infty} a_i p^{i}\ =\ \sum_{i=0}^{\infty} (p-1)p^{i}$$
Obtain
$$\sum_{i=0}^{\infty} (p-1)p^{i}=(p-1) \sum_{i=0}^{\infty} p^{i} = (p-1) \cdot \frac{1}{1-p} = \frac{p-1}{1-p}=-1$$

\subsection{Remark}%Remark 10.7
Let
$$x= \sum_{i=0}^{\infty} a_i p^{i}, \qquad y=\sum_{i=0}^{\infty} b_i p^{i}$$
$p$-adic integers. Then 
$$x+y = \sum_{i=0}^{\infty} c_i p^{i}$$
with coefficients
$$c_0=\begin{cases} a_0+b_0 & \textrm{ if }\ a_0+b_0 <p \\ a_0+b_0-p & \textrm{ if } a_0+b_0 \geqslant p\end{cases}$$
$$c_1=\begin{cases} a_1+b_1 & \textrm{ if }\ a_0+b_0 <p \ \textrm{ and } \ a_1+b_1 < p \\
a_1+b_1-p & \textrm{ if } \ a_0+b_0<p \ \textrm{ and } \ a_1+b_1 \geqslant p \\
a_1+b_1+1 & \textrm{ if } \ a_0+b_0 \geqslant p \ \textrm{ and } \ a_1+b_1 +1 <p \\
a_1+b_1+1-p & \textrm{ if } \ a_0+b_0 \geqslant p \ \textrm{ and } \ a_1+b_1+1 \geqslant p \end{cases}$$
Inductively let 
$$\epsilon_0:=0,\qquad \epsilon_i:= \begin{cases} 0 & \textrm{ if } \ a_i+b_i + \epsilon_{i-1} < p \\ 1 & \textrm{ if } \ a_i + b_i + \epsilon_{i-1}\geqslant p \end{cases} \quad \textrm{ for } i \geqslant 1$$ 
Then we have
$$c_i=\begin{cases} a_i+b_i+\epsilon_i & \textrm{ if } \ a_i + b_i + \epsilon_i < p \\ a_i + b_i + \epsilon_i - p & \textrm{ if } \ a_i+b_i+\epsilon_i \geqslant p \end{cases}$$


\subsection{Remark}%Remark 10.8
\begin{compactenum}
\item $\sqrt{p} \notin \mathbb{Q}_p$, since $\vert \sqrt{p} \vert = \sqrt{\vert p \vert }= \sqrt{\frac{1}{p}} \in \left(\frac{1}{p}, 1\right)$, which is not possible.
\item Let $a \in \mathbb{Z}_p^{\times}$ with image $\overline{a} \in \mathbb{F}_p^{\times} \setminus \mathbb{F}_p^{\times^2}$, where $$\mathbb{F}_p^{\times^2}=\{x \in \mathbb{F}_p \mid \textrm{ there exists } y \in \mathbb{F}_p: y^2=x \}$$
denotes the set of squares in $\mathbb{F}_p^{\times}$. Then $\sqrt{a} \notin \mathbb{Q}_p$.\\
Assume there exists $b \in \mathbb{Q}_p$, such that $b^2=a$. Then 
$$\vert b \vert = \sqrt{\vert a \vert } =1 \quad \Rightarrow \quad b \in \mathbb{Z}_p^{\times}$$
Bt then $\overline{b} \in \mathbb{F}_p$ satisfies $\overline{b}^2\equiv a$, which is a contradiction, since $a \notin \mathbb{F}_p^{\times^2}$.
\item Let now $\overline{\mathbb{Q}}_p$ be the algebraic closure of $\mathbb{Q}_p$ with valuation ring $\overline{\mathbb{Z}}_p$ and maximal ideal $\overline{\mathfrak{m}}_p$.\\
Then $\slant{\overline{\mathbb{Z}}_p}{\overline{\mathfrak{m}}}$ is algebraically closed.\\
Moreover $\mathbb{Q}_p$ is complete with respect to $|\cdot|_p$. The completion $\mathbb{C}_p$ of $\overline{\mathbb{Q}}_p$ is complete and algebraically closed, but:
\begin{compactenum}
\item $|\cdot|_p$ is not a discrete valuation.
\item $\overline{\mathbb{Z}}_p$ is not a discrete valuation ring.
\item $\overline{\mathfrak{m}}_p$ is not a principal ideal.
\end{compactenum}
\end{compactenum}


\titleformat{\subsection}{\normalfont\normalsize\bfseries}{}{0em}{#1 \thesubsection\quad \textnormal{\textit{(Hensel's Lemma)}}}
\subsection{Theorem } %Theorem 10.9
\titleformat{\subsection}{\normalfont\normalsize\bfseries}{}{0em}{#1 \thesubsection}
Let 
$$f= \sum_{i=0}^n a_i X^{i} \in \mathbb{Z}_p[X], \qquad \overline{f}=\sum_{i=0}^n \overline{a_i} X^{i} \in \mathbb{F}[X]$$
where $\overline{f}$ is the reduction of $f$ in $\mathbb{F}[X]$. \\
Suppose that $\overline{f}=f_1 \cdot f_2$ with $f_1, f_2 \in \mathbb{F}_p[X]$ relatively prime.\\
Then there exist $g,h \in \mathbb{Z}_p[X]$, such that 
$$f= g \cdot h, \quad \overline{g}=f_1, \overline{h}=f_2, \quad \deg(f_1)=\deg(g)$$
\pagebreak\\
\textit{proof.}\\
Let $d:= \deg(f), m:= \deg(f_1)$. Then $\deg(f_2) \leqslant d-m$.\\
Choose $g_0, h_0 \in \mathbb{Z}_p[X]$ such that $\overline{g_0}=f_1, \overline{h_0}=f_2, \deg(g_0)=m, \deg(h_0)=d-m$.\\
\textit{Strategy:} Find $g_1=g_0+pc_1$, $h_1=h_0+pd_1$ with some $c_1, d_1 \in \mathbb{Z}_p[X]$, such that 
$$f-g_1h_1 \in p^2 \mathbb{Z}_p[X]$$
Therefore we have a 
\begin{compactenum}
\item[\textbf{Claim (a)}] For $n \geqslant 1 $ there exists $c_n, d_n \in \mathbb{Z}_p[X]$ with $\deg(c_n) \leqslant m, \deg(d_n) \leqslant d-m$ and 
$$f-g_nh_n \in p^{n+1} \mathbb{Z}_p[X], \qquad \textrm{where } g_n= g_{n-1}+p^n c_n , \quad h_n=h_{n-1}+p^nd_n$$
\end{compactenum}
Assuming (a), write
$$g_n= \sum_{i=0}^m g_{n,i} X^{i}, \qquad h_n= \sum_{i=0}^{d-m} h_{n,i} X^{i}$$
By construction, the $(g_{n,i})$ converge to some $\alpha_i \in \mathbb{Z}_p$ and the $(h_{n,i})$ converge to some $\beta_i \in \mathbb{Z}_p$. Let 
$$g:=\sum_{i=0}^m \alpha_i X^{i}, \qquad h:=\sum_{i=0}^{d-m} \beta_i X^{i}$$
Observe, that $\deg(g)=m, \deg(h)=d-m$. Obviously we have $$f=g \cdot h$$
It remains to show the claim.
\begin{compactenum}
\item[\textbf{(a)}] $c_n, d_n$ have to satisfy
\begin{alignat*}{5}
f-g_nh_n \ &=&& \ f-\left(g_{n-1}+p^n c_n\right)\cdot \left(h_{n-1}+p^nd_n\right)\\
&=&& \ f- g_{n-1}h_{n-1} - p^n \cdot \left( g_{n-1}d_n+h_{n-1}c_n + p^n c_n d_n \right)\\
&\overset{!}{\in}&& \ p^{n+1} \mathbb{Z}_p[X]
\end{alignat*}
where $f-g_{n-1}h_{n-1} \in p^n \mathbb{Z}_p[X]$ by hypothesis. We get 
$$\tilde{f}_n := \frac{1}{p^n} (f-g_{n-1}h_{n-1}) \equiv c_n h_{n-1} + d_n g_{n-1} \mod p \ (*)$$
Since $f_1, f_2$ are relatively prime and $g_j \equiv g_k \mod p$ for any $j,k$, we find integers $a,b \in \mathbb{Z}$, such that
$$af_1, bf_2=1 \quad \Longrightarrow \quad ag_{n-1}+bh_{n-1} \equiv 1 \mod p$$
Multiplying the equation by $\tilde{f}_n$ gives us 
$$\tilde{f}_n \ \equiv \ \underbrace{a \tilde{f}_n}_{=:\tilde{d}_n} g_{n-1} \ + \ \underbrace{b \tilde{f}_{n}}_{=:\tilde{c}_n} h_{n-1} \ \mod p \ (**)$$
Further $\mathbb{Z}_p[X]$ is euclidean, thus we can choose $q_n, r_n \in \mathbb{Z}_p[X]$, $\deg(r_n)<m$ such that
$$b \tilde{f}_n=q_n g_{n-1} + r_n$$
By $(**)$ we have
$$g_{n-1}\left(a \tilde{f}_n + q_n h_{n-1}\right) + r_n \equiv \tilde{f}_n \ \mod p$$
Let now $c_n=r_n, d_n=a\tilde{f}_n+q_nh_{n-1}$. All the terms are divisible by $p$. Then
$$d_n \ \equiv\ a \tilde{f}_n \ + \ q_nh_{n-1} \ \mod p$$
Thus $(*)$ holds and we have
$$\deg(d_n)=\deg(\overline{d_n})\leqslant \deg\left(\underbrace{\overbrace{\overline{\tilde{f}}_n}^{\leqslant d}-\overbrace{\overline{c}_n}^{<m}\overbrace{\overline{h}_{n-1}}^{<d-m}}_{\leqslant d}\right)-\underbrace{\deg(\overline{g}_{n-1})}_{=m} \leqslant d-m$$
Since $\overline{d}_n\overline{g}_{n-1}=\overline{\tilde{f}}_n-\overline{c}_n\overline{h}_{n-1}$. Thus, the claim is proved. 
\end{compactenum}

\subsection{Corollary}%Corollary 10.10
Let $p\in \mathbb{P}$ odd. Then
$a \in \mathbb{Z}_p^{\times}$ is a square if and only if $\overline{a} \in \mathbb{F}_p^{\times}$ is a square.

\subsection{Proposition}%Corollary 10.11
$a \in \mathbb{Q}$ is a square if and only if $a>0$ and $a$ is a square in $\mathbb{Q}_p$ for all $p \in \mathbb{P}$. 
\\
\textit{Remark: This is a special case of the 'Hasse-Minkowski-Theorem'. 
}

%CHAPTER III


\chapter{Rings and modules}
\thispagestyle{empty}

\setcounter{section}{10}
\renewcommand*\thesection{\S\ \arabic{section}\quad }
\section{Multilinear Algebra}
\renewcommand*\thesection{\arabic{section}}

In this section, $R$ will always be a commutative, unitary ring.

\subsection{Reminder} %Reminder 11.1
\begin{compactenum}
\item An $R$\textit{-module} is an abelian group $(M,+)$ together with a scalar multiplication
$$\cdot: R \times M \longrightarrow M$$
with the usual properties of a vector space, i.e. we have for any $x,y \in M, r,s \in R$
\begin{compactenum}
\item $r \cdot (s \cdot x)= (r \cdot s) \cdot x$
\item $(r+s)\cdot x= r\cdot x + s \cdot x$
\item $r \cdot (x+y)=r \cdot x + r \cdot y$
\item $1_R \cdot x = x$
\end{compactenum}
\item A map
$$\phi:M \longrightarrow M'$$
of $R$-modules $M, M'$ is called $R$-\textit{linear} or $R$-\textit{module homomorphism}, if 
$$\phi(rx+sy)=r \phi(x)+s\phi(y) \qquad \textrm{ for all } r,s \in R, x,y \in M$$
\item  A subset $S\subseteq M$ of an R-module is called an $R$-\textit{submodule of M}, if $S$ is an $R$-module.
\item $R$ is an $R$-module, the submodules are the ideals of $R$.
\item If $\phi:M \longrightarrow M'$ is $R$-linear, then 
$$\ker(\phi)= \{m \in M\ \mid\ \phi(m)=0\}$$
$$\textrm{im}(\phi)=\{m' \in M'\ \mid\ \phi(m)=m' \textrm{ for some } m \in M\}$$
are $R$-submodules.
\item If $M \subseteq M'$ is a submodule, then the factor group $\slant{M}{M'}$ isn $R$-module by
$$a \cdot \overline{m}= \overline{a\cdot m}$$
\item For an $R$-linear map $\phi: M \longrightarrow M''$, we have
$$ \textrm{im}(\phi) \cong \slant {M}{\ker(\phi)}$$
\item An $R$-module $M$ is called \textit{free}, if there exists a subset $X \subseteq M$, such that every $y \in M$ has a unique representation
$$y = \sum_{x \in X} a_x \cdot x \qquad a_x \in R, \ a_x\neq0 \textrm{ only for finitely many } x \in X$$
In this case, $X$ is called the rank of $M$.
\item Not every $R$-module is free. \\
Let $0 \lneq I \lneq R$ be a proper ideal. Then $\slant{R}{I}$ is not free:\\
Let $X \subseteq R$, such that $\overline{X} \subseteq \slant{R}{I}$ generates the $R$-module $\slant{R}{I}$. \\
Let $x \in X$ and $a\in I \setminus \{0\}$. Then we have
$$x \cdot \overline{x} = \overline{a \cdot x}=\overline{0}=\overline{0\cdot x}= 0 \cdot \overline{x}$$
hence we have found two different reapersentations of $0$. Thus $\slant{R}{I}$ is not free.
\item For any $n \in \mathbb{N}$, $n\mathbb{Z}$ is a free module
\item If $I \leqslant R$ is not a principle ideal, then $I$ is not a free $R$-module., since for $x,y \in I$ with $y \notin \langle x \rangle$ we have $xy-yx=0$. Again we have a nontrivial representation of $0$ and $I$ is not free.
\end{compactenum}

\subsection{Definition + Proposition}%Definition + Proposition 11.2
\newcounter{temp}
Let $R$ be a ring, $M, M'$ $R$-modules.
\begin{compactenum}
\item $$\textrm{Hom}_R(M,M')=\{ \phi:M \longrightarrow M' \mid \phi \textrm{ is $R$-linear } \}$$
is an $R$-module.
\item $M^{*}=\textrm{Hom}_R(M,R)$ is called the \textit{dual module} of M.
\setcounter{temp}{\value{enumi}}
\end{compactenum}
Let now $$0 \longrightarrow M' \overset{\alpha}{\longrightarrow} M \overset{\beta}{\longrightarrow} M'' \longrightarrow 0$$
be a short exact sequence of $R$-modules $M,M',M''$, i.e. we have $\ker(\beta)=\textrm{im}(\alpha), \ker(\alpha)= \{0\}, \textrm{im}(\beta)=M''$ and let $N$ be a further $R$-module.
\begin{compactenum}
\setcounter{enumi}{\value{temp}}
\item Then we have a short exact sequence
$$\begin{matrix}[ccrcccl]0 &\longrightarrow& \textrm{Hom}_R(N,M')& \overset{\alpha_*}{\longrightarrow} &\textrm{Hom}_R(N,M)&\overset{\beta_*}{\longrightarrow} &\textrm{Hom}_R(N,M'') \\ && \phi & \mapsto & \alpha \circ \phi, \quad \psi & \mapsto & \beta \circ \psi \end{matrix}$$
\item We have s short exact sequence
$$\begin{matrix}[ccrcccl]0 & \longrightarrow & \textrm{Hom}_R(M'',N) & \overset{\beta^*}{\longrightarrow} & \textrm{Hom}_R(M,N) & \overset{\alpha^*}{\longrightarrow} & \textrm{Hom}_R(M',N) \\ && \phi & \mapsto & \phi \circ \beta, \quad \psi & \mapsto & \psi \circ \alpha \end{matrix}$$
\item $N$ is called a \textit{projective} module, if $\beta_*$ is surjective for all short exact sequences as in (iii).
\item $N$ is called an \textit{injective}  module, if $\alpha^*$ is surjective for all short exact sequences an in (iv).
\end{compactenum}
\textit{proof.}
\begin{compactenum}
\item This is clear: For all $ \phi, \phi_1, \phi_2 \in \textrm{Hom}_R(M,M'), a \in R$ we have
$$\left(\phi_1+\phi_2\right)(x)= \phi_1(x)+\phi_2(x), \qquad \left(a \cdot \phi\right)(x)=a \cdot \phi(x) \quad$$
\item[(iii)]
$\alpha_*$ is $R$-linear: we have
\begin{alignat*}{6}
\alpha_*(\phi_1+\phi_2)(x)\ &=&&\ \left(\alpha \circ (\phi_1+\phi_2)\right)(x)\ &&=&&\  \alpha \left(\phi_1(x)+\phi_2(x)\right)\ &&=&& \ \alpha \left(\phi_1(x)\right) + \alpha\left(\phi_2(x)\right)  \\ &=&& \ \alpha_*(\phi_1)(x)+\alpha_*(\phi_2)(x)\ &&=&& \ \left(\alpha_*(\phi_1)+ \alpha_*(\phi_2)\right)(x)
\end{alignat*}
$\alpha_*$ is injective: we have
\begin{alignat*}{5}
\alpha_*(\phi)=0 \ \Longleftrightarrow \ (\alpha \circ \phi)(x)=0 \textrm{ for all } x \in N \ &\Longleftrightarrow \ \alpha\left(\phi(x)\right)=0  \ &&\overset{\alpha \textrm{ inj.}}{\Longleftrightarrow} \ \phi(x)=0 \textrm{ for all } x \in N \\
&&& \Longleftrightarrow \ \phi=0
\end{alignat*}
Now we still have to show $\ker(\beta_*)=\textrm{im}(\alpha_*)$.
\begin{compactitem}
\item['$\supseteq$'] For $\phi \in \textrm{Hom}_R(N,M')$ we have $\beta_*(\alpha \circ \phi)= \beta \circ \alpha \circ \phi = 0 \circ \phi = 0$, i.e. $\alpha \circ \phi = \alpha_*(\phi) \in \ker(\beta_*)$.
\item['$\subseteq$'] Let $\phi:N \longrightarrow M$, $\phi \in \ker(\beta_*)$, i.e. $\beta \circ \phi=0$. \\
We have to show, that there exists $\phi' \in \textrm{Hom}_R(N,M')$ such that $\phi=\alpha_*(\phi')=\alpha \circ \phi'$.\\
Let $x \in N$. Then $\phi(x) \in \ker(\beta)=\textrm{im}(\alpha)$.\\
$\Rightarrow$ there exists $z \in M'$ such that $\phi(x)=\alpha(z)$ and $z$ is unique, since $\alpha$ is injective.\\
Define $\phi'(x):=z$. Then we have $\alpha \circ \phi'=\phi$. \\
It remains to show that $\phi'$ is $R$-linear. We have\\
$\phi'(x_1+x_2)=z$ and with $\alpha(z)=\phi(x_1+x_2)=\phi(x_1)+\phi(x_2)$ we again have $\alpha(z)= \phi(z_1)+\phi(z_2)$ for some suitable, but unique $z_1, z_2 \in M'$. Since we have
$$\alpha(z)=\phi(x_1+x_2)= \phi(x_1)+\phi(x_2)=\alpha(z_1)+\alpha(z_2)=\alpha(z_1+z_2)$$
and $\alpha$ is injective, we have $z=z_1+z_2$, thus
$$\phi'(x_1+x_2)=z=z_1+z_2=\phi'(x_1)+\phi'(x_2)$$
Moreover for $a \in R$ we have $\phi'(ax)=w$ with $\alpha(w)=\phi(ax)=a\cdot \phi(x)=a \cdot \alpha(z)$. Thus
$$\alpha\left(\phi'(ax)\right)=\alpha(w)=\phi(ax)=a \cdot \phi(x)=a \cdot \alpha(z)=a \cdot \alpha\left(\phi'(x)\right) \ \overset{\alpha \textrm{ inj.}}{\Longrightarrow} \ \phi'(ax)=a \cdot \phi'(x)$$
\end{compactitem}
\end{compactenum}

\subsection{Remark}%Remark 11.3
\begin{compactenum}
\item An $R$-module $N$ is projective if and only if for every surjective $R$-linear map $\beta:M \longrightarrow M''$ and every $R$-linear map $\phi:N \longrightarrow M''$ there is an $R$-linear map $\tilde{\phi}:N \longrightarrow M$, such that the diagram below commutes, i.e. $\phi=\beta \circ \tilde{\phi}$.
$$\begin{xy}
\xymatrix{
&& M \ar[dd]^{\beta} \\ &&\\ N \ar@{-->}[rruu]^{\tilde{\phi}} \ar[rr]_{\phi} && M''  
}
\end{xy}$$
\item Free modules are projective.
\end{compactenum}

\subsection{Definition}%Definition 11.4
Let $M, M_1, M_2$ be $R$-modules. A map
$$\Phi: M_1 \times M_2 \longrightarrow M$$
is called \textit{bilinear}, if\\
$\Phi_{x_0}: M_2 \longrightarrow M, \quad y \mapsto \Phi(x_0,y)$ is linear for all $x_0 \in M_1$ and\\
$\Phi_{y_0}: M_1 \longrightarrow M, \quad x \mapsto \Phi(x,y_0)$ is linear for all $y_0 \in M_2$.

\subsection{Definition}%Definiiton 11.5
Let $M_1, M_2$ be $R$-modules. A \textit{tensor prodcut} of $M_1$ and $M_2$ is an $R$-module $T$ together with a bilinear map
$$\tau: M_1 \times M_2 \longrightarrow T,$$
such that for every bilinear map $\Phi: M_1 \times M_2 \longrightarrow M$ for any $R$-module $M$ there is a unique linear map $\phi:T \longrightarrow M$, such that the following diagram becomes commutative.
$$\begin{xy}
\xymatrix{
M_1 \times M_2 \ar[rrrr]^{\tau} \ar[rrdd]_{\Phi} &&&& T \ar@{-->}[lldd]^{\phi} \\ &&&&\\ && M &&
}
\end{xy}$$

\subsection{Remark}%Remark 11.6
Let $(T,\tau)$ and $(T', \tau')$ be tensor products of $R$-modules $M_1$ and $M_2$.\\
Then there exists a unique isomorphism
$h: T \longrightarrow T'$, such that $$\tau'=h \circ \tau$$
\pagebreak\\
\textit{proof.}\\
Consider
$$\begin{xy}?
\xymatrix{
**[c] M_1 \times M_2 \ar[rr]^{\tau} \ar[rd]_{\tau'} && T \ar@<4pt> [ld]^{h}  \\ & T'  \ar@<1pt>[ru]^g&
}
\end{xy}$$
Existence and uniqueness of the linear maps $g$ and $h$ come from Definition 11.5. It remains to show, that $h \circ g = \textrm{id}_{T'}$ and $g \circ h = \textrm{id}_{T}$.\\
For this, look at 
$$\begin{xy}
\xymatrix{
M_1 \times M_2 \ar[rr]^{\tau} \ar[rd]_{\tau'} && T \ar@{-->}[ld]^{g\ \circ \ h\ \overset{!}{=}\ \textrm{id}_T} \\ & T &
}
\end{xy}$$
We have $(g \circ h) \tau = g \circ ( h \circ \tau )= g \circ \tau'=\tau$. By the uniqueness we get $\textrm{id}_{T}=g \circ h $.
\\Similarly we get $\textrm{id}_{T'}= h \circ g$.\\

\subsection{Corollary}%Corollary 11.7
The tensor product $(T, \tau)$ of $R$-modules $M_1$, $M_2$ is unique up to isomorphism. The standard notation is
$$T= M_1 \otimes_R M_2, \qquad \tau(x,y)=x \otimes y$$

\subsection{Example}%Example 11.8
Let $M_1, M_2$ be free $R$-modules with bases $\{e_i\}_{i \in I}, \{f_j\}_{j \in J}$. Let $T$ be the free $R$-module with basis $\{g_{ij}\}_{(i,j) \in I \times J}$ and $$\tau: M_1 \times M_2 \longrightarrow T, \ (e_i, f_j) \mapsto g_{ij} \quad \textrm{ for all } (i,j) \in I \times J,$$
i.e. for elements in $M_1, M_2$ we have
$$\tau \left(\sum_{i\in I} a_i e_i, \ \sum_{j \in J} b_j f_j \right) = \sum_{(i,j) \in I \times J} a_i b_j g_{ij}$$
Then $(T, \tau)$ is the tensor product of $M_1, M_2$.\\
\textit{proof.}\\
Let $\Phi: M_1 \times M_2 \longrightarrow M$ be bilinear.\\
Define $$\phi: T \longrightarrow M, \ g_{ij} \mapsto \Phi(e_i, f_j).$$
Obviously $\phi$ is linear and satisfies $\Phi=\phi \circ \tau$.\\
Now consider a special case and let $|I|=n, |J|=m$. \\
Identify $M_1$ via $ (e_1, \ldots e_n)$ with $R^n$ andn $M_2$ via $(f_1, \ldots f_m)$ with $R^m$. \\
Then $T$ is identified with $R^{n\times m}$ via
$$g_{ij}=E_{ij}=\begin{pmatrix} 0 & \ldots & 0 & \ldots & 0  \\ \vdots & & 1 & & \vdots \\ 0 & \ldots & 0 & \ldots & 0 \end{pmatrix}$$
where the only nonzero entry is in the $i$-th row and $j$-th column. Then $\tau: R^n \times R^m \longrightarrow R^{n\times m}$ is given by
$$\begin{pmatrix} a_1 \\ \vdots \\ a_n \end{pmatrix} \cdot \begin{pmatrix} b_1 \\ \vdots \\ b_m \end{pmatrix} = \begin{pmatrix} a_1b_1 & \ldots & a_1b_m \\ \vdots & & \vdots \\ a_nb_1 & \ldots & a_nb_m \end{pmatrix} = \begin{pmatrix} a_1 \\ \vdots \\ a_n \end{pmatrix} \cdot \begin{pmatrix} b_1 & \ldots & b_m \end{pmatrix}$$

\subsection{Theorem}%Theorem 11.9
For any two $R$-modules $M_1, M_2$ there exists a tensor product $(T, \tau)=(M_1 \otimes_R M_2, \otimes)$.\\
\textit{proof.}\\
Let $F$ be the free $R$-module with basis $M_1 \times M_2$ and $Q$ be the submodule generated by all the elements
$$(x+x',y)-(x,y)-(x',y), \quad (x,y+y')-(x,y)-(x,y'), \quad (ax,y)-a(x,y), \quad (x,ay)-a(x,y)$$
where $a \in R, x,x' \in M_1, y,y' \in M_2$. 
Define $$T:= \slant{F}{Q}, \qquad \tau: M_1 \times M_2 \longrightarrow T, \ (x,y) \mapsto \overline{(x,y)}$$
Then by the construction of $Q$, $\tau$ is bilinear.\\
Let now be $M$ a further $R$-module and $\Phi: M_1 \times M_2 \longrightarrow M$ a bilinear map.
Define $$\tilde{\phi}: F \longrightarrow M, \quad (x,y) \mapsto \Phi(x,y)$$
Clearly $\tilde{\phi}$ is linear. Moreover we have $Q \subseteq \ker(\phi)$, since $\Phi$ is bilinear.
By the isomorphism theorem, $\tilde{\phi}$ factors to a linear map
$$\phi: T \longrightarrow M, \quad \textrm{ satisfying } \phi\left(\overline{(x,y)}\right)= \Phi(x,y)$$
The uniqueness of $\phi$ follows by the fact that $T$ is generated by the $\overline{(x,y)}$ for $x \in M_1, y \in M_2$. 

\titleformat{\subsection}{\normalfont\normalsize\bfseries}{}{0em}{#1}
\subsection*{Example} %Example
\titleformat{\subsection}{\normalfont\normalsize\bfseries}{}{0em}{#1 \thesubsection}
We want to find out what is
$$\slant{\mathbb{Z}}{2 \mathbb{Z}} \otimes_{\mathbb{Z}} \slant{\mathbb{Z}}{3 \mathbb{Z}}$$
Let $\Phi: \slant{\mathbb{Z}}{2 \mathbb{Z}} \times \slant{\mathbb{Z}}{3 \mathbb{Z}} \longrightarrow A$ bilinear for some $\mathbb{Z}$-module $A$. Then we see
$$\Phi(\overline{1}, \overline{1})=\Phi(\overline{3}, \overline{1})=\Phi\left(3 \cdot (\overline{1}, \overline{1})\right)= 3 \cdot \Phi(\overline{1}, \overline{1})=\Phi(\overline{1}, \overline{3})=\Phi(\overline{1},\overline{0})=0\cdot \Phi(\overline{1}, \overline{1}) =0$$
Hence $\Phi=0$, since $(\overline{1}, \overline{1})$ generates $\slant{\mathbb{Z}}{2 \mathbb{Z}} \times \slant{\mathbb{Z}}{3 \mathbb{Z}}$. 

\subsection{Proposition} %Proposition 11.10
For $R$-modules $M, M_1, M_2, M_3$ we have the following properties.
\begin{compactenum}
\item $M \otimes_R R \cong M$.
\item $M_1 \otimes_R M_2 \cong M_2 \otimes_R M_1$.
\item $\left(M_1 \otimes_R M_2 \right) \otimes_R M_3 \cong M_1 \otimes_R \left(M_2 \otimes_R M_2 \right)$.
\end{compactenum}
\textit{proof.}
\begin{compactenum}
\item Let $\tau: M \times R \longrightarrow M, \ (x,a) \mapsto a \cdot x$.
$\tau$ is bilinear. We now can verify the universal property of the tensor product.
Let 
$\Phi: M \times R \longrightarrow N$
be bilinear of some $R$-module $N$. Define
$$\phi: M \longrightarrow N, \quad x \mapsto \Phi(x,1)$$
Then $\phi$ is $R$-linear: For $x,y \in M, \alpha \in R$ we have 
$$\phi(\alpha \cdot x)= \Phi(\alpha \cdot x,1)= \alpha \cdot \Phi(x,1)= \alpha \cdot \phi(x)$$
$$\phi(x+y)= \Phi(x+y,1)= \Phi(x,1)+ \Phi(y,1)= \phi(x)+\phi(y)$$
and
$$\phi\left(\tau(x,a)\right)= \phi(a\cdot x)=a \cdot \Phi(x,1)= \Phi(x,a)$$
\item The isomorphism 
$$M_1 \times M_2 \overset{\cong}{\longrightarrow} M_2 \times M_1, \quad (x,y) \mapsto (y,x)$$
induces an isomorphism $M_1 \otimes_R M_2 \cong M_2 \otimes_R M_1$.
\item For fixed $z \in M_3$ define
$$\Phi_z: M_1 \times M_2 \longrightarrow M_1 \otimes_R \left(M_2 \otimes_R M_3 \right), \quad (x,y) \mapsto x \otimes (y \otimes z)=\tau_{1(23)}\left(\tau_{23}(x,y)\right)$$
$\Phi_z$ is bilinear. Then $\Phi_z$ induces a linear map
$$\phi_z: M_1 \otimes_R M_2 \longrightarrow M_1 \otimes_R \left(M_2 \otimes_R M_3 \right)$$
Define
$$\Psi: \left(M_1 \otimes_R M_2 \right) \times M_3 \longrightarrow M_1 \otimes_R \left(M_2 \otimes_R M_3 \right), \quad (x \otimes y, z) \mapsto \phi_z(x \otimes y)$$
$\Psi$ is bilinear. Then $\Psi$ induces a linear map
$$\psi: \left(M_1 \otimes_R M_2\right) \otimes_R M_3 \longrightarrow M_1 \otimes_R \left(M_2 \otimes_R M_3 \right)$$
Do this again the other way round and we find a linear map
$$\tilde{\psi}: M_1 \otimes_R \left(M_2 \otimes_R M_3 \right) \longrightarrow \left(M_1 \otimes_R M_2\right) \otimes_R M_3$$
By the uniqueness we obtain as in Remark 11.6 that $\psi \circ \tilde{\psi} = \tilde{\psi} \circ \psi = \textrm{id}$, hence the claim follows.
\end{compactenum}

\subsection{Definition + Remark} %Definition + Proposition 11.11
Let $M, M_1, \ldots M_n$ be $R$-modules. 
\begin{compactenum}
\item A map 
$$\Phi: M_1 \times \ldots \times M_n = \prod_{i=1}^n M_i \longrightarrow M$$
is called \textit{multilinear}, if for any $1 \leqslant i \leqslant n$ and all choices of $x_j \in M_j$ for $j \neq i$ the map 
$$\Phi_i: M_i \longrightarrow M, \quad x \mapsto \Phi(x_1, \ldots, x_{i-1},x,x_{i+1},\ldots, x_n)$$
is linear.
\item The map
$$\tau_{M_1, \ldots M_n}: \prod_{i=1}^n M_i \longrightarrow \bigotimes_{i=1}^n M_i, \qquad (x_1, \ldots, x_n) \mapsto x_1 \otimes \ldots \otimes x_n$$
is multilinear.
\item For every multilinear map $$\Phi: \prod_{i=1}^n M_i \longrightarrow M$$
there exists a unique linear map
$$\phi: \bigotimes_{i=1}^n M_i \longrightarrow M$$
such that $\Phi= \phi \circ \tau_{M_1, \ldots M_n}$.
\end{compactenum}

\subsection{Definiton} %Definition 11.12
Let $M,N$ be $R$-modules, $$\Phi:M^n= \prod_{i=1}^n M \longrightarrow N$$
a multilinear map.
\begin{compactenum}
\item $\Phi$ is called \textit{symmetric}, if for any $\sigma \in S_n$ we have
$$\Phi(x_1,\ldots x_n) = \Phi(x_{\sigma(1)}, \ldots x_{\sigma(n)})$$
\item $\Phi$ is called \textit{alternating}, if
$$x_i = x_j \textrm{ for some } i\neq j \ \ \Longrightarrow \ \ \Phi(x_1, \ldots x_n)=0$$
If $\textrm{char}(R) \neq 2$, this is equivalent to 
$$\Phi(x_1, \ldots, x_i, \ldots, x_j, \ldots, x_n)=- \Phi(x_1, \ldots, x_j, \ldots, x_i, \ldots, x_n)$$
\end{compactenum}

\subsection{Proposition} %Proposition 11.13
Let $M$ be an $R$-module, $n \geqslant 1$. 
\begin{compactenum}
\item There exists an $R$-module $S^n(M)$, called the $n$\textit{-th symmetric power} of $M$ and a symmetric multilinear map
$$\sigma_M^n:M^n \longrightarrow S^n(M)$$
such that for all symmetric, multilinear maps $\Phi: M^n \longrightarrow N$ for any $R$-module $N$ there exists a unique linear map
$$\phi:S^n(M) \longrightarrow N \qquad \textrm{ satisfying } \Phi = \phi \circ \sigma_M^n$$
\item There exists an $R$-module $\Lambda^n(M)$, called the $n$-\textit{th exterior power} of $M$ and an alternating multilinear map 
$$\lambda_M^n: M^n \longrightarrow \Lambda^n(M)$$
such that for all alternating, multilinear maps $\Phi: \Lambda^n(M) \longrightarrow N$ for any $R$-module $N$ there exists a unique linear map 
$$\phi:\Lambda^n(M) \longrightarrow N \qquad \textrm{ satisfying } \Phi=\phi \circ \lambda_M^n$$
\end{compactenum}
\textit{proof.}
\begin{compactenum}
\item Let $T^n(M)=M \otimes_R \ldots \otimes_R M$.\\
Let now $J_n(M)$ be the submodule of $T^n(M)$ generated by all elements
$$\left(x_1 \otimes \ldots \otimes x_n\right)-\left(x_{\sigma(1)} \otimes \ldots \otimes x_{\sigma(n)}\right), \quad x_i \in M, \sigma \in S_n$$
Deifne $$S^n(M):=\slant{T^n(M)}{J_n(M)}, \qquad \sigma_M^n := \rm{proj} \circ \tau_{M,\ldots M}$$
Then $\sigma_M^n$ is multilinear and symmetric by construction. Given a multilinear and symmetric map
$\Phi: M^n \longrightarrow N,$
define $\phi$ as follows: Let $\tilde{\phi}:T^n(M) \longrightarrow N$ be the linear map induced by $\Phi$ and observe that $J_n(M) \subseteq \ker(\tilde{\phi})$. Hence $\tilde{\phi}$ factors to a linear map 
$$\phi: S^n(M)=\slant{S^n(M)}{J_n(M)} \longrightarrow N$$
satisfying $\phi \circ \sigma_M^n=\Phi$. 
\item Similarily let $I_n(M)$ be the submodule of $T^n(M)$ generated by all the elements
$$x_1 \otimes \ldots \otimes x_n, \qquad x_i \in M \ \textrm{ with } x_i =x_j \ \textrm{ for some }i \neq j$$
Analogously we define 
$$\Lambda^n(M):=\slant{T^n(M)}{I_n(M)}, \qquad \lambda_M^n:= \rm{proj} \circ \tau_{M, \ldots, M}$$
and receive the required properties.
\end{compactenum}

\subsection{Proposition} %Proposition 11.14
Let $M$ be a free $R$-module of rank $r$ and $\{e_1, \ldots, e_r\}$ a basis of $M$.\\
Then $\Lambda^n(M)$ is a free $R$-module with basis
$$ \rm{proj}(e_{i_1} \otimes \ldots \otimes e_{i_n}) =: e_{i_1}\wedge \ldots \wedge e_{i_n}, \qquad 1 \leqslant i_1 < \ldots < i_n \leqslant r$$
In particular, $\Lambda^n(M)=0$ for $n>r$ and $\rm{rank}\left(\Lambda^r(M)\right)=1$.\\
\textit{proof.}\\
By definition we have $e_{i_1} \wedge \ldots \wedge e_{i_n}=0$ if $i_k =i_j$ for some $k \neq j$, hence we have $\Lambda^n(M)=0$ for $n > r$, as at least on of the $e_k$ must appear twice.
\begin{compactitem}
\item[\textbf{generating}] Clearly the $e_{i_1} \wedge \ldots \wedge e_{i_n}, i_k \in \{1, \ldots, r\}$ generate $\Lambda^n(M)$. We have to show that we can leave out some of them.\\
Further, $e_{i_{\sigma(1)}} \wedge \ldots \wedge e_{i_{\sigma(n)}}$ is a multiple by $\pm 1 $ of $e_{i_1} \wedge \ldots \wedge e_{i_n}$.\\
$\Longrightarrow$ The $e_{i_1} \wedge \ldots \wedge e_{i_n}$ with $1 \leqslant i_1 < i_2 < \ldots < i_n \leqslant r$ generate $\Lambda^n(M)$. 
\item[\textbf{linear independence}] Assume
$$\sum_{1\leqslant i_1<...<i_n\leqslant r}a_{i_1, \ldots, i_n} e_{i_1} \wedge \ldots \wedge e_{i_n} = 0 \ (*)$$
For fixed $j:=(j_1, \ldots j_n), 1 \leqslant j_1 <...<j_n \leqslant r$ choose $\sigma_j \in S_r$, such that $\sigma_j(k)=j_k$ for $1 \leqslant k \leqslant n$. \\
Then we obtain 
$$e_{i_1} \wedge \ldots \wedge e_{i_n} \wedge e_{\sigma_j(n+1)} \wedge \ldots \wedge e_{\sigma_j(r)} = \begin{cases}\ \pm \ e_1 \wedge \ldots \wedge e_r, & \ \textrm{ if } \ i_k=j_k \ \textrm{ for all } k \\ \ 0 & \ \textrm{ otherwise } \end{cases}$$
By $(*)$ we get 
$$0\ = \ \left(\sum_{1 \leqslant i_1<\ldots i_n \leqslant r} a_{i_1,\ldots, i_n} e_{i_1} \wedge \ldots \wedge e_{i_n} \right) \wedge e_{\sigma_j(n+1)} \wedge \ldots \wedge e_{\sigma_j(r)} \ = \ a_j e_{j_1} \wedge \ldots \wedge e_{j_r}$$
And thus $a_j=0$. 
\end{compactitem}

\subsection{Example} %Example 11.15
Let $M=R^n, \Lambda^k(M)$ is the free $R$-module with basis 
$$e_{i_1} \wedge \ldots \wedge e_{i_k}, \quad 1 \leqslant i_1 < \ldots < i_k \leqslant n$$
and we have $e_1 \wedge e_2=-e_2 \wedge e_1$. \\
What is $\Lambda^n(R^n)=\Lambda^n(M)$? And what is $\lambda_k^M$?\\
First we obtain $\Lambda^n(R^n)=(e_1 \wedge \ldots \wedge e_n)R \cong R$. Then 
$$M^n=(R^n)^n=R^{n \times n}, \quad (a_1, \ldots a_n)=A \in R^{n \times n}, \quad a_i = \begin{pmatrix} a_{1i} \\ \vdots \\ a_{ni} \end{pmatrix}=\sum_{j=1}^n a_{ji}e_j \in R^n=M$$
For $\lambda_n^M$ we get
\begin{alignat*}{8}
\lambda_n^M\ =\ \lambda_n^{R^n}=\lambda_n(A)\ &=&&\quad \lambda_n \left( \sum_{j=1}^n a_{j1}e_j, \ldots, \sum_{j=1}^n a_{jn}e_j \right)\ =\ \sum_{j=1}^n a_{j1} e_j \wedge \ldots \wedge \sum_{j=1}^n a_{jn}e_j \\
&=&&\quad \sum_{j=1}^n a_{j1} \left( e_1 \wedge \sum_{j=1}^n a_{j2}e_j \wedge \ldots \wedge \sum_{j=1}^n a_{jn}e_j \right)\ =\ \sum_{j=1}^n a_{j1} \cdot \cdot \cdot \sum_{j=1}^n a_{jn} \left(e_1 \wedge \ldots \wedge e_n \right)\\
&=&&\quad \sum_{\sigma \in S_n} a_{\sigma(1)1} \cdot \cdot \cdot a_{\sigma(n)n} \cdot e_1 \wedge \ldots \wedge e_n \cdot \rm{sgn}(\sigma)\\
&=&&\quad \det(A) \cdot e_1 \wedge \ldots \wedge e_n
\end{alignat*}

\subsection{Definition} %Definition 11.16
Let $M$ be a $R$-module, define
$$T(M):= \bigoplus_{n=0}^{\infty} T^n(M), \qquad T^0(M):=R, \ T(M):=M$$
$$S(M):= \bigoplus_{n=0}^{\infty} S^n(M). \qquad S^0(M):=R, \ S(M):=M$$
$$\Lambda(M):= \bigoplus_{n=0}^{\infty} \Lambda^n(M), \qquad \Lambda^0(M);=R, \ \Lambda(M):=M$$
On $T^n(M)$ define a multiplication
$$\begin{matrix}[rcl]\cdot: T^n(M) \times T^m(M) &\longrightarrow &T^{n+m}(M), \\ (x_1 \otimes \ldots \otimes x_n) \cdot (y_1 \otimes \ldots\otimes y_m) & \mapsto & x_1 \otimes \ldots \otimes x_n \otimes y_1 \otimes \ldots \otimes y_m\end{matrix}$$
Similarly do it for $S(M)$ and $\Lambda(M)$. Then we have $R$-algebra-structures and feel free to define
\begin{compactenum}
\item the \textit{tensor algebra} $T(M)$,
\item the \textit{symmetric algebra} $S(M)$
\item the \textit{exterior algebra} $\Lambda(M)$.
\end{compactenum}

\subsection{Definition} %Definition 11.17
Let $R$ be a ring.
\begin{compactenum}
\item An $R$-algebra is a ring $R'$ together with a ring homomorphism $\alpha: R \longrightarrow R'$. In particular $R'$ is an $R$-module. If $\alpha$ is injective, $R'/R$ is called a \textit{ring extension}.
\item A homomorphism of $R$-algebras $R', R''$ is an $R$-linear map $\phi: R' \longrightarrow R''$, which is a ring homomorphism.
\end{compactenum}

\titleformat{\subsection}{\normalfont\normalsize\bfseries}{}{0em}{#1}
\subsection*{Example} %Example
\titleformat{\subsection}{\normalfont\normalsize\bfseries}{}{0em}{#1 \thesubsection}
\begin{compactenum}
\item $R[X_1,\ldots X_N]$ is an $R$-algebra for every $n \in \mathbb{N}$. 
\item If $R'$ is an $R$-algebra and $I \trianglelefteqslant R'$ an ideal, then $\slant{R'}{I}$ is an $R$-algebra.
\end{compactenum}

\subsection{Remark} %Remark 11.18
Let $R'$ be an $R$-algebra, $F$ a free $R$-module. Then $F':= F \otimes_R R'$ is a free $R'$-module.\\
\textit{proof.}\\
Let $\{e_i\}_{i\in I}$ be basis of $F$. Let us show, that $\{e_1 \otimes 1\}_{i\in I}$ is basis of $F'$ as an $R$-module, where $F'$ is an $R'$ module by 
$$b \cdot (x \otimes a) := x \otimes b \cdot a, \qquad a,b \in R, \ x \in F$$
Check the universal property of the free $R'$-module with basis $\{e_i \otimes 1\}_{i \in I}$ for $F \otimes_R R'$.\\
Let $M'$ be an $R$-module and $f: \{e_i \otimes 1 \}_{i \in I} \longrightarrow M'$ be a map. \\
We have to show: There exists an $R'$-linear map $\phi: F' \longrightarrow M'$ with $\phi(e_i \otimes 1 ) = f (e_i \otimes 1)$.\\
Note that the $\{e_i \otimes 1 \}$ generate $F'$ as an $R'$-module, since $e_i \otimes a =a \cdot (e_i \otimes a )$ for $a \in R'$.\\
Let $\tilde{\phi}: F \longrightarrow M'$ be the unique $R$-linear map satisfying $\tilde{\phi}(e_i)= f(e_i\otimes 1)$.\\ Then define 
$$\phi: F \otimes_R R' \longrightarrow M', \quad x \otimes a \mapsto a \cdot \tilde{\phi}(x)$$
Then $\phi$ is $R'$-linear an we have 
$$\phi(e_i \otimes 1)= 1 \cdot \tilde{\phi}(e_i)= \tilde{\phi}(e_i)=f(e_i \otimes 1)$$

\subsection{Proposition} %Proposition 11.19
Let $R$ be a ring, $R', R''$ two $R$-algebras. 
\begin{compactenum}
\item $R' \otimes_R R''$ is an $R$-algebra with multiplication
$$(a_1 \otimes b_1) \cdot (a_2 \otimes b_2):= (a_1 a_2) \otimes (b_1 b_2)$$
\item There are $R$-algebra homomorphisms
$$\sigma': R' \longrightarrow R' \otimes_R R'', \qquad a \mapsto a \otimes 1$$
$$\sigma'':R'' \longrightarrow R'' \otimes_R R'', \qquad b \mapsto 1 \otimes b$$
\item For any $R$-algebra $A$ and $R$-algebra homomorphisms $\phi':R' \longrightarrow A, \phi'': R'' \longrightarrow A$, there is a unique $R$-algebra homomorphism
$$\phi: R' \otimes_R R'' \longrightarrow A$$
satisfying $\phi'=\phi \circ \sigma'$ and $\ \phi''=\phi \circ \sigma''$, i.e. making the following diagram commutative
$$\begin{xy}
\xymatrix{
&& R' \otimes_R R'' \ar@{-->}[dd]^{\phi} \\ R' \ar[rru]^{\sigma'} \ar[rrd]_{\phi'} & R'' \ar[ru]_{\sigma''} \ar[rd]^{\phi''} & \\ && A 
}
\end{xy}$$
\end{compactenum}
\textit{proof.}\\
Defining $$\tilde{\phi}: R' \times R'' \longrightarrow A, \qquad (x,y) \mapsto \phi'(x) \cdot \phi''(y)$$
gives us $\phi$, which satisfies the required properties.

%SECTION 12

\renewcommand*\thesection{\S\ \arabic{section}\quad }
\section{Hilbert's basis theorem}
\renewcommand*\thesection{\arabic{section}}

\subsection{Definition} %Definition 12.1
Let $R$ be a ring, $M$ and $R$-module.
\begin{compactenum}
\item $M$ is called \textit{noetherian}, if any ascending chain of submodules $M_0 \subset M_1 \subset \ldots$ becomes stationary.
\item $R$ is called \textit{noetherian}, if $R$ is noetherian as an $R$-module, i.e. if every ascending chain of ideals becomes stationary.
\end{compactenum}

\titleformat{\subsection}{\normalfont\normalsize\bfseries}{}{0em}{#1 }
\subsection*{Example} %Example
\titleformat{\subsection}{\normalfont\normalsize\bfseries}{}{0em}{#1 \thesubsection}
\begin{compactenum}
\item Let $R=\mathbb{K}$ be a field. A $\mathbb{K}$-vector space is noetherian if and only if $\dim(V)<\infty$. 
\item $\mathbb{Z}$ is noetherian.
\item Principle ideal domains are noetherian.
\end{compactenum}

\subsection{Proposition} %Proposition 12.2
Let 
$$ 0 \longrightarrow M' \overset{\alpha}{\longrightarrow} M \overset{\beta}{\longrightarrow} M'' \longrightarrow 0$$
be a short exact sequence. Then $M$ is noetherian if and only if $M'$ and $M''$ are noetherian.\\
\textit{proof.}
\begin{compactitem}
\item['$\Rightarrow$']Let $M$ be noetherian.
\begin{compactitem}
\item[\textbf{for M'.}] Let $M_0' \subset M_1' \subset \ldots $ be an ascending chain of submodules in $M'$. Then \linebreak $\alpha(M_0') \subset \alpha(M_1') \subset \ldots $ is an ascending chain in $M$. Since $M$ is noetherian, there exists some $n \in \mathbb{N}$, such that $\alpha(M_i')=\alpha(M_n')$ for all $i \geqslant n$. Since $\alpha$ is injective, we have $M_i'=M_n?$ for $i \geqslant n$, hence $M'$ is noetherian.
\item[\textbf{for M''}] Let $M_0'' \subset M_1'' \subset \ldots$ be an ascending chain of submodules in $M''$. Then  \linebreak  $\beta^{-1}(M_0)'' \subset \beta^{-1}(M_1'') \subset \ldots$ is an ascending chain in $M$, hence becomes stationary. Since $\beta$ is surjective, $\beta\left(\beta^{-1}(M_i'')\right)=M_i''$ and thus $M_0'' \subset M_1'' \subseteq \ldots $ becomes stationary.
\end{compactitem}
\item['$\Leftarrow$'] Let $M_0 \subset M_1 \subset \ldots $ be an ascending chain in $M$.\\
Let $M_i':= \alpha^{-1}(M_i) \cong M_i \cap M'$ and $M_i'' := \beta(M_i)$. By assumption, there exists $n \in \mathbb{N}$, such that $M_i'=M_n'$ and $M_i'' = M_n''$ for all $i \geqslant n$. Then for $i \geqslant n$ we have
$$\begin{xy}
\xymatrix{
0 \ar[rr] && M_n' \ar[rr]^{\alpha} \ar@2{-}[d] && M_n \ar[rr]^{\beta} \ar[d]^{\gamma} && M_n'' \ar[rr] \ar@2{-}[d] && 0 &  \textrm{exact} \\ 0 \ar[rr] && M_i' \ar[rr]_{\alpha} && M_i \ar[rr]_{\beta} && M_i'' \ar[rr] && 0 & \textrm{exact}
}
\end{xy}$$
Where $\gamma$ is injective as an embedding. It remains to show that $\gamma$ is surjective.\\
Let $z \in M_i$. Since $\beta$ is surjective, there exists $x \in M_n$, such that $\beta(x)=\beta(z)$.\\
Then $\beta\left(\gamma(x)-z\right)=0 \Rightarrow \gamma(x)-z = \alpha(y)$ for some $y \in M_i'=M_n'$. Let $\tilde{x}:=x - \alpha(y)$. Then
$$\gamma(\tilde{x})= \gamma(x)- \gamma\left(\alpha(y)\right)=\gamma(x)-\gamma(x)+z=z$$
hence $\gamma$ is surjective, thus bijective and we have $M_i=M_n$ for $i \geqslant n$. 
\end{compactitem}

\subsection{Corollary} %Corollary 12.3
Let $R$ be noetherian. 
\begin{compactenum}
\item Any free $R$-module $F$ of finite rank $n$ is noetherian.
\item Any finitely generated $R$-module $M$ is noetherian.
\end{compactenum}
\textit{proof.}
\begin{compactenum}
\item Prove this by induction on $n$. 
\begin{compactitem}
\item[\textbf{n=1}] Clear.
\item[\textbf{n>1}] Let $e_1, \ldots e_n$ be a basis of $F$ and le $F'$ be the submodule generated by $e_1, \ldots e_{n-1}$. Then $F'$ is free of rank $n-1$, thus noetherian by induction hypothesis. Moreover $\slant{F}{F'}$ is free with generator $e_n$. Thus we have a short exact sequence
$$0 \longrightarrow F' \longrightarrow F \longrightarrow \slant{F}{F'} \longrightarrow 0$$
with $F', \slant{F}{F'}$ noetherian, hence by 12.2, $F$ is noetherian.
\end{compactitem}
\item If $M$ is generated by $x_1, \ldots x_n$, there is a surjective, $R$-linear map $\phi: F \longrightarrow M$, sending the $e_i$ to $x_i$, where $F$ is the free $R$-module with basis $e_1, \ldots e_n$. Again by 12.2, $M$ is noetherian.
\end{compactenum}

\subsection{Proposition} % Proposition 12.4
For an $R$-module $M$ the following statements are equivalent:
\begin{compactenum}
\item $M$ is noetherian.
\item Any nonempty family of submodules of $M$ has a maximal element with respect to '$\subseteq$'.
\item Every submodule of $M$ is finitely generated.
\end{compactenum}
\textit{proof.}
\begin{compactenum}
\item['(i)$\Rightarrow$(ii)'] Let $\mathcal{M} \neq \emptyset$ be a set of submodules of $M$. Let $M_0 \in \mathcal{M}$. If $M_0$ is not maximal, there is $M_1 \in \mathcal{M}$ with $M_0 \subsetneq M_1$. If $M_1$ is not maximal, there is $M_2 \in \mathcal{M}$ with $M_1 \subsetneq M_2$. Since $M$ is noetherian, we come to a maximal submodule $M_n$ after finitely many step.
\item['(ii)$\Rightarrow$(iii)'] Let $N \subseteq M$ be a submodule. Let $\mathcal{M}$ be the set of finitely generated submodules of $N$. Since $\langle 0 \rangle \in \mathcal{M}$, we have $\mathcal{M} \neq \emptyset$ and thus there exists a maximal element $N_0 \in \mathcal{M}$. If $N_0 \neq N$, let $x \in N \setminus N_0$ and $N':=N_0 + \langle x \rangle $ be the submodule generated by $N_0$ and $x$. Then clearly $N' \in \mathcal{M}$, which is a contradiction to the maximality of $N_0$. Hence $N_0=N$ and $N$ is finitely generated. 
\item['(iii)$\Rightarrow$(i)'] Let $M_0 \subseteq M_1 \subseteq \ldots$ be an ascending chain of submodules in $M$. Let $N:= \bigcup_{n \in \mathbb{N}_0} M_n$. By assumption, $N$ is finiteley generated, say by $x_1, \ldots x_n$. Then there exists $i_0 \in \mathbb{N}$, such that $x_k \in M_{i_0}$ for all $1 \leqslant k \leqslant n$. Thus we have $M_i=M_{i_0}$ for $i \geqslant i_0$, i.e. th chain becomes stationary and $M$ is noetherian.
\end{compactenum}

\subsection{Corollary} %Corollary 12.5
$R$ is noetherian if and only if every ideal $I \trianglelefteqslant R$ can be generated by finitely many elements. In particular, every principle ideal domain is noetherian.\\
\textit{proof.}\\
Follows from Proposition 12.4

\titleformat{\subsection}{\normalfont\normalsize\bfseries}{}{0em}{#1 \thesubsection\quad \textnormal{\textit{(Hilbert's basis theorem)}}}
\subsection{Theorem } %Theorem 12.6
\titleformat{\subsection}{\normalfont\normalsize\bfseries}{}{0em}{#1 \thesubsection}
If $R$ is noetherian, $R[X]$ is also noetherian.\\
\textit{proof.}\\
Let $J \trianglelefteqslant R[X]$ be an ideal.\\
Assume that $J$ is not finitely generated.\\
Let $f_1$ be an element of $J \setminus \{0\}$ of minimal degree. Then $\langle f_1 \rangle \neq J$.\\
Inductively let $J_i:= \langle f_1, \ldots f_i \rangle$ and pick $f_{i+1} \in J \setminus J_{i}$ of minimal degree.\\
Let $a_i$ be the leading coefficient of $f_i$, i.e. we have
$$f_i = a_i X^{\deg(f_i)} + \sum_{j=1}^{\deg(f_i)-1} b_j X^{j}$$
The ideal $I \trianglelefteqslant R$ generated by the $a_i$ for $i \in \mathbb{N}$, is finitely generated by assumption.\\
Then we find $n \in \mathbb{N}$ such that $a_{n+1} \in \langle a_1, \ldots, a_n \rangle$, i.e.
$$a_{n+1}=\sum_{i=1}^n \lambda_i a_i$$
for suitable $\lambda_i \in R$. Let $d_i := \deg(f_i)$. Note, that $d_{i+1}\geqslant d_i$ for all $1 \leqslant i \leqslant n$.
Let now
$$\rho:= \sum_{i=1}^n \lambda_i f_i X^{d_{n+1}-d_i}$$
Then the leading coefficient of $\rho$ is 
$$a_{d_{n+1}}=\sum_{i=1}^n \lambda_i a_i$$
Hence $\deg(\rho-f_{n+1}) < d_{n+1}, \rho-f_{n+1} \notin J_n$, since $\rho \in J_n$, so $f_{n+1}$ would be in $J_n$. This contradicts the choice of $f_{n+1}$!\\
Hence our assumption was false and $J$ is finitely generated and by Corollary 12.5 $R[X]$ is noetherian.

\subsection{Corollary} %Corollary 12.7
Let $R$ be noetherian. Then 
\begin{compactenum}
\item $R[X_1, \ldots X_n]$ is noetherian for any $n \in \mathbb{N}$. 
\item Any finitely generated $R$-algebra is noetherian.
\end{compactenum}

%SECTION 13

\renewcommand*\thesection{\S\ \arabic{section}\quad }
\section{Integral ring extensions}
\renewcommand*\thesection{\arabic{section}}

\subsection{Definition} %Defintiion 13.1
Let $R$ be ring, $S$ an $R$-algebra.
\begin{compactenum}
\item If $R\subseteq S$, $S/R$ is called a \textit{ring extension}.
\item If $R\subseteq S$, $b \in S$ is called \textit{integral over} $S$, if there exists a monic polynomial $f \in R[X]\setminus \{0\}$ such that $f(b)=0$.
\item $S/R$ is called an \textit{integral ring extension}, if every $b \in S$ is integral over $R$.
\end{compactenum}

\titleformat{\subsection}{\normalfont\normalsize\bfseries}{}{0em}{#1}
\subsection*{Example} %Example
\titleformat{\subsection}{\normalfont\normalsize\bfseries}{}{0em}{#1 \thesubsection}
\begin{compactenum}
\item If $R= \mathbb{K}$ is a field, then \textit{integral} is equivalent to \textit{algebraic}.
\item $\sqrt{2}$ is integral over $\mathbb{Z}$, since $f=X^2-2$ is monic with $f(\sqrt{2})=0$.
\item $\frac{1}{2}$ is not integral over $\mathbb{Z}$.\\
Assume $\frac{1}{2}$ is integral over $\mathbb{Z}$. Then there exists some monic $f \in R[X]$, such that $f\left(\frac{1}{2}\right)=0$, i.e. we have
$$\left(\frac{1}{2}\right)^n+g\left(\frac{1}{2}\right) =0 \ (*)$$
for some $g \in \mathbb{Z}[X]$.Then $2^{n-1} \cdot g\left(\frac{1}{2}\right) \in \mathbb{Z}$. Multiplying $(*)$ by $2^{n-1}$ gives us
$$2^{n-1} \cdot \left(\left(\frac{1}{2}\right)^n+g\left(\frac{1}{2}\right)\right)=0$$
and hence
$$ \frac{1}{2}=-2^{n-1} \cdot g\left(\frac{1}{2}\right) \in \mathbb{Z} \quad \lightning$$
Thus $\frac{1}{2}$ is not integral over $\mathbb{Z}$. More generally, we easily see that any $q \in \mathbb{Q} \setminus \mathbb{Z}$ is not integral over $\mathbb{Z}$.
\end{compactenum}

\subsection{Lemma} %Lemma 13.2
Let $S/R$ be a ring extension, $b \in S$. If $R[b]$ is contained in a subring $S' \subseteq S$ which is finitely generated as an $R$-module, then $b$ is integral over $R$.\\
\textit{proof.}\\
Let $s_1, \ldots, s_n$ be generators of $S'$. Since $b \cdot s_i \in S$ (we have $b \in R[b] \subseteq S$), we find $a_{ik} \in R$, such that 
$$b\cdot s_i = \sum_{k=1}^n a_{ik} s_k \ \Longleftrightarrow \ 0=\sum_{k=1}^n (a_ik-\delta_{ik})s_k \quad (*)$$
\begin{compactenum}
\item[\textbf{Claim (a)}] Let $A$ be the coefficient matrix of $(*)$. Then $\det(A)=0$
\end{compactenum}
Since the determinant is a monic polynomial in $b$ of degree $n$ with coefficients in $R$, $b$ is integral over $R$. It remains to show the claim.
\begin{compactenum}
\item[\textbf{(a)}] Let $A^{\#}$ be the adjoint matrix
$$A_{ji}^{\#}=\det(A_{ij} \cdot (-1)^{i+j}$$
where $A_{ij}$ is obtained from $A$ by deleting the $i$-the row and $j$-th column. Recall
$$A^{\#} A = \det(A) \cdot E_n$$
By $(*)$ we have
$$A \cdot \begin{pmatrix}s_1 \\ \vdots \\ s_n \end{pmatrix} =0$$
hence we have
$$A^{\#}\cdot A \cdot \begin{pmatrix}s_1 \\ \vdots \\ s_n \end{pmatrix} =0 \ \Longrightarrow \ \det(A) \cdot s_i = 0 \quad \textrm{ for all } 1 \leqslant i \leqslant n.$$
Since $S'$ is a subring of $S$, we have $1\in S'$, hence there exist $\lambda_1, \ldots, \lambda_n \in R$ with
$$1 = \sum_{i=1}^n \lambda_i s_i.$$
Finally
$$\det(A)=\det(A) \cdot 1 = \det(A)\cdot \sum_{i=1}^n \lambda_i s_i = \sum_{i=1}^n \det(A) \cdot \lambda_i \cdot s_i = 0$$
\end{compactenum}

\subsection{Proposition} %Proposition 13.3
Let $S/R$ be a ring extension. Define
$$\overline{R}:=\{b \in S \mid b \textrm{ is integral over }R\} \supseteq R$$
Then $\overline{R}$ is a subring of $S$, called the \textit{integral closure} of $R$ in $S$.
\pagebreak\\
\textit{proof.}\\
Let $b_1, b_2 \in \overline{R}$. We have to show, that $b_1 \pm b_2 \in \overline{R}$, $b_1b_2 \in \overline{R}$.\\
Let $R[b_1]$ be the smallest subring of $S$ containing $R$ and $b_1$. Then $R$ is finitely generated as an $R$-module by $1,b_1, b_1^2, \ldots, b_1^{n-1}$, where $n$ denotes the degree of the 'minimal polynomial' of $f$.\\
Thus  $R[b_1, b_2]=\left(R[b_1]\right)[b_2]$ is also finitely generated as an $R[b_1]$-module. This implies, that $R[b_1,b_2]$ is also finitely generated as an $R$-module and by Lemma 13.2, $R[b_1,b_2]/R$ is an integral ring extension. In particular, $b_1 \pm b_2$ and $b_1b_2$ are integral over $R$.

\subsection{Definition} %Definition 13.4
Let $S/R$ be a ring extension, $\overline{R}$ the integral closure of $R$ in $S$.
\begin{compactenum}
\item $R$ is called \textit{integrally closed} in $S$, if $\overline{R}=R$.
\item Let $R$ be an integral domain. The integral closure of $R$ in $\textrm{Quot}(R)$ is called the \textit{normalization} of $R$. $R$ is called \textit{normal}, if it agrees with its normalization.
\end{compactenum}

\subsection{Proposition} %Proposition 13.5
Any factorial domain $R$ is normal.\\
\textit{proof.}\\
Let $x=\frac{a}{b} \in \textrm{Quot}(R), a,b \in R, b \neq 0$ relatively prime.\\
Suppose, $x$ is integral over $R$, i.e. there exist $\alpha_0, \ldots, \alpha_{n-1} \in R$, such that 
$$x^n+ \alpha_{n-1}x^{n-1}+ \ldots + \alpha_1 x+\alpha_0=0$$
Multiplying by $b^n$ gives us
$$a^n+\alpha_{n-1}a^{n-1}b + \ldots + \alpha_1 a b^{n-1}+ \alpha_0 b^n =0$$
and hence
$$a^n= b \cdot \underbrace{\left(-\alpha_{n-1}a^{n-1}- \ldots - \alpha_1 a b^{n-2} - \alpha_0 b^{n-1}\right)}_{\in R} \quad \Longleftrightarrow\ b \mid a^n$$
Since $a$ and $b$ are coprime, we have $b \in R^{\times}$. Thus $x = \frac{a}{b}=ab^{-1} \in R$ and $R$ is normal.

\subsection{Definition} %Definition 13.6
Let $R$ be a ring.
\begin{compactenum}
\item For a prime ideal $\mathfrak{p} \trianglelefteqslant R$ we define
$$ht(\mathfrak{p}):=\sup\{n \in \mathbb{N}_0 \ \big \vert\ \textrm{ there exist prime ideals } \mathfrak{p}_0, \mathfrak{p}_1, \ldots, \mathfrak{p}_n, \textrm{ with } \mathfrak{p}_n=\mathfrak{p} \textrm{ and } \mathfrak{p}_0 \subsetneq \ldots \subsetneq \mathfrak{p}_n\}$$
to be the \textit{height} of $\mathfrak{p}$.
\item The \textit{Krull-dimension} of $R$ is
$$\dim(R):=\dim_{\textrm{Krull}}(R)= \sup\{ht(\mathfrak{p}) \ \big \vert\ \mathfrak{p} \trianglelefteqslant R \textrm{ prime }\}$$
\end{compactenum}

\titleformat{\subsection}{\normalfont\normalsize\bfseries}{}{0em}{#1}
\subsection*{Example} %Example
\titleformat{\subsection}{\normalfont\normalsize\bfseries}{}{0em}{#1 \thesubsection}
\begin{compactenum}
\item Since $\langle 0 \rangle \subsetneq \langle X_1 \rangle \subsetneq \langle X_1, X_2 \rangle \subsetneq \ldots \subsetneq \langle X_1, \ldots, X_n \rangle$, we have $\dim\left(\mathbb{K}[X_1, \ldots, X_n]\right) \geqslant n$.\item $\dim(\mathbb{K})=0$ for any field $\mathbb{K}$, since $\langle 0 \rangle$ is the only prime ideal.
\item $\dim(\mathbb{Z})=1$, since $\langle 0 \rangle \subsetneq \langle p \rangle$ is a maximal chain of prime ideals for $p \in \mathbb{P}$.
\item $\dim(R) =1$ for any principle ideal domain which is not a field:\\
Assume $p,q$ are prime element with $\langle p \rangle \subseteq \langle q \rangle$. Then $p=q \cdot a$ for some $a \in R$. Since $p$ is irreducible, we have $a \in R^{\times}$ and hence $\langle p \rangle = \langle q \rangle$.
\item $\dim(\mathbb{K}[X])=1$ for any field $\mathbb{K}$: 
\end{compactenum}

\titleformat{\subsection}{\normalfont\normalsize\bfseries}{}{0em}{#1 \thesubsection\quad \textnormal{\textit{(Going up)}}}
\subsection{Proposition} %Proposition 13.7
\titleformat{\subsection}{\normalfont\normalsize\bfseries}{}{0em}{#1 \thesubsection}
Let $S/R$ be an integral ring extension and
$$\mathfrak{p}_0 \subsetneq \mathfrak{p}_1 \subsetneq \ldots \subsetneq \mathfrak{p}_n$$
a chain of prime ideals in $R$. Then there exists a chain of prime ideals
$$\mathfrak{P}_0 \subsetneq \mathfrak{P}_1 \subsetneq \ldots \subsetneq \mathfrak{P}_n$$
in $S$, such that $\mathfrak{p}_i = \mathfrak{P}_i \cap R$.\\
\textit{proof.}\\
Do this by induction on $n$. 
\begin{compactitem}
\item[\textbf{n=0}] Let $\mathfrak{p} \triangleleft R$ be a prime ideal. We have to find a prime ideal $\mathfrak{P} \triangleleft S$ with $\mathfrak{P} \cap R=\mathfrak{p}$. Let
$$\mathcal{P}:= \{I \triangleleft S \textrm{ ideal } \big \vert I \cap R = \mathfrak{p} \}$$
\begin{compactenum}
\item[\textbf{Claim (a)}] $\mathfrak{p}S \in \mathcal{P}$.
\end{compactenum}
Then $\mathcal{P}$ is nonempty. Zorn's lemma provides us then a maximal element $\mathfrak{m} \in \mathcal{P}$.
\begin{compactenum}
\item[\textbf{Claim (b)}] $\mathfrak{m} \triangleleft S$ is a prime ideal.
\end{compactenum}
This proves the claim. It remains to show the Claims.
\begin{compactenum}
\item[\textbf{(b)}] Suppose $b_1, b_2 \in S$ with $b_1b_2 \in \mathfrak{m}$. Assume $b_1, b_2 \in S\setminus \mathfrak{m}$.\\
Then $\mathfrak{m}+\langle b_i \rangle \notin \mathcal{P}$, hence $\left(\mathfrak{m}+\langle b_i \rangle \right) \supsetneq \mathfrak{p}$ for $i \in \{1,2\}$.
$\Longrightarrow$ Thus there exists $p_i \in \mathfrak{m}, s_i \in S$ such that $r_i:=p_i+b_is_i \in R\setminus \mathfrak{p}$.  Then we have
$$r_1r_2=(p_1+b_1s_1)(p_2+b_2s_2)=\underbrace{p_1p_2+p_1b_2s_2+b_1s_1p_2}_{\in \mathfrak{m}}\ \ +\underbrace{b_1b_2}_{\in \mathfrak{m} \textrm{ by ass. }}s_1s_2\ \in \ \mathfrak{m}$$
Clearly $r_1r_2 \in R$, hence $r_1r_2 \in \mathfrak{m} \cap R = \mathfrak{p}$, which is a contradiction, since $\mathfrak{p}$ is prime.
\item[\textbf{(a)}] We have to show $\mathfrak{p}S \cap R = \mathfrak{p}$. We prove both inclusions.
\begin{compactitem}
\item['$\supseteq$'] This is clear by definition.
\item['$\subseteq$'] Let now
$$b=\sum_{i=0}^n p_i t_i, \qquad p_ \in \mathfrak{p}, \ t_i \in S$$
Since the $t_i$ are integral over $R$, $R[t_1, \ldots t_n]=:S'$ is finitely generated. Let $s_1, \ldots, s_m$ be generators of $S'$ as an $R$-module. Since $b \in \mathfrak{p}S'$, we have
$$bs_i=\sum_{k=0}^m a_{ki}s_k$$
for suitable $a_{ik} \in \mathfrak{p}$. Then as in lemma 13.3 we have $$\det(a_{ik}-\delta_{ik}b)=0$$ and thus $b$ is a zero of monic polynomial with coefficients in $\mathfrak{p}$, i.e. $b$ satisfies an equation 
$$b^n+a_{n-1}b^{n-1} + \ldots + a_1b+a_0=0 \qquad \textrm{ with } a_i \in \mathfrak{p},$$
Write
$$b^n=-\sum_{i=0}^{n-1} a_i b^{i} \in \mathfrak{p},$$
since $b^{i} \in \mathfrak{p}$. Since $\mathfrak{p}$ is prime, we must have $b \in \mathfrak{p}$ and hence the required inclusion.\\
\end{compactitem}
\end{compactenum}
\item[\textbf{n>1}] By induction hypothesis we have a chain
$$\mathfrak{P}_0 \subsetneq \mathfrak{P}_1 \subsetneq \ldots \subsetneq \mathfrak{P}_{n-1}$$
satisfying $\mathfrak{P}_i \cap R = \mathfrak{p}_i$. Moreover we find $\mathfrak{P}_n \triangleleft S$ such that $\mathfrak{P}_n \cap R=\mathfrak{p}_n$. It remains to show $\mathfrak{P}_{n-1} \subsetneq \mathfrak{P}_n$.
For $x \in \mathfrak{P}_{n-1}$ we have $x \in R \cap \mathfrak{p}_{n-1}$, i.e. $x \in \mathfrak{p}_{n-1}\subset \mathfrak{p}_n$. Thus $x \in \mathfrak{p}_{n} \cap R = \mathfrak{P}_n$. Assume now $\mathfrak{P}_{n-1}=\mathfrak{P}_{n}$. Let $x\in \mathfrak{p}_{n}$. Then $$x \in \mathfrak{p}_n \in \mathfrak{p}_n\cap R=\mathfrak{P}_n=\mathfrak{P}_{n-1}=\mathfrak{p}_{n-1} \cap R, \quad \Longrightarrow \quad x \in \mathfrak{p}_{n-1}$$
and thus $\mathfrak{p}_n \subseteq \mathfrak{p}_{n-1}$, hence $\mathfrak{p}_n=\mathfrak{p}_{n-1}$, a contradiction.
\end{compactitem}

\subsection{Theorem} %Theorem 13.8
Let $S/R$ be an integral ring extension. Then $\dim(R)=\dim(S)$.\\
\textit{proof.}
\begin{compactenum}
\item['$\leqslant$'] Follows from Proposition 13.7
\item['$\geqslant$'] Let $\mathfrak{P}_0 \ \subsetneq \mathfrak{P}_1 \subsetneq \ldots \subsetneq \mathfrak{P}_n$ be chain of prime ideals in $S$ and define $\mathfrak{p}_i:=\mathfrak{P}_i \cap R$.\\
Then $\mathfrak{p}_i$ is prime and we have $\mathfrak{p}_i \subseteq \mathfrak{p}_{i+1}$. It remains to show, that $\mathfrak{p}_i \neq \mathfrak{p}_{i+1}$.\\
Define $S':=\slant{S}{\mathfrak{P}_i}$ and $R':=\slant{R}{\mathfrak{p}_i}$. Then $S'/R'$ is integral (!).\\
We have to show that $\overline{\mathfrak{P}}_{i+1} \cap R=\overline{\mathfrak{p}}_{i+1}:=$ image of $\mathfrak{p}_{i+1}$ in $S'$ is not $\langle 0 \rangle$.\\
Let $b \in \mathfrak{P}_{i+1} \setminus \{0\}$. Since $b$ is integral over $R'$, there exist $a_0, \ldots, a_{n-1} \in R$, such that
$$b^n+a_{n-1}b^{n-1}+ \ldots +a_1b+a_0=0$$
Let further $n$ be minimal with this property. Write
$$a_0=-b \cdot \underbrace{\left( a_1+a_2b+ \ldots + a_{n-1}b^{n-2}+b^{n-1}\right)}_{=:c} \in \overline{\mathfrak{P}}_{i+1} \cap R=\overline{\mathfrak{p}}_{i+1}$$
But $c\neq 0$ by the choice of $n$ and $b \neq 0$. Since $R'=\slant{R}{\mathfrak{p}}$ is an integral domain, we have
$$\overline{0} \neq a_0 \in \overline{\mathfrak{p}}_{i+1} \ \Longrightarrow \ \overline{\mathfrak{p}}_{i+1} \neq \langle 0 \rangle$$
\end{compactenum}

\titleformat{\subsection}{\normalfont\normalsize\bfseries}{}{0em}{#1 \thesubsection\quad \textnormal{\textit{(Noether normalization)}}}
\subsection{Theorem} %Theorem 13.9
\titleformat{\subsection}{\normalfont\normalsize\bfseries}{}{0em}{#1 \thesubsection}
Let $\mathbb{K}$ be a field. Then every finitely generated $\mathbb{K}$-algebra is an integral extension of a polynomial ring over $\mathbb{K}[X]$.\\
\textit{proof.}\\
Let $a_1, \ldots a_n$ be generators of $A$ as a $\mathbb{K}$-algebra. Prove the theorem by induction.
\begin{compactenum}
\item[\textbf{n=1}] If $a_1$ is transcendental over $\mathbb{K}$, then $A \cong \mathbb{K}[X]$. Otherwise $A \cong \slant{\mathbb{K}[X]}{\langle f \rangle}$, where $f$ denotes the minimal polynomial of $a_1$ over $\mathbb{K}$. Thus $A$ is integral over $\mathbb{K}$.
\item[\textbf{n>1}] If $a_1, \ldots a_n$ are algebraically independent, $A \cong \mathbb{K}[X_1, \ldots X_n]$. Otherwise there exists some polynomial\newline $F \in \mathbb{K}[X_1, \ldots X_n] \setminus \{0\}$ such that $F(a_1, \ldots a_n)=0$. 
\begin{compactenum}
\item[\textbf{case 1}] Assume we have
$$F= X_n^m+ \sum_{i=1}^{m-1} g_i X_n^i$$
with $g_i \in \mathbb{K}[X_1, \ldots X_n]$. Then $F(a_1, \ldots a_n)=0$, hence $a_n$ is integral over $A':=\mathbb{K}[a_1, \ldots, a_{n-1}]$. By induction hypothesis, $A'$ is integral over some polynomial ring, so is $A$.
\item[\textbf{case 2}] For the general case write
$$F=\sum_{i=0}^m  F_i,$$
where $F_i$ is homogenous of degree $i$, i.e. the sum of the exponents of any monomial in $f_i$ is equal to $i$. Then replace $a_i$ by $b_i:=a_i- \lambda a_n$ (*) with suitable $\lambda_i \in \mathbb{K}$, $1 \leqslant i \leqslant n-1$. Then 
$$A \cong \mathbb{K}[b_1, \ldots, b_{n-1}, a_n]$$
For any monomial $a_1^{d_1} \cdot \cdot \cdot a_n^{d_n}$ we find 
$$a_1^{d_1} \cdot \cdot \cdot a_n^{d_n} = \left(b_1+\lambda_1 a_n\right)^{d_1} \cdot \cdot \cdot \left(b_{n-1}+\lambda_{n-1}a_n\right)^{d_{n-1}} \cdot a_n^{d_n} = \left(\prod_{i=1}^{n-1} \lambda_i^{d_i}\right) \cdot a_n^{\sum_{i=1}^n d_i} \ + \ \mathcal{O}(a_n)$$
where $\mathcal{O}(a_n)$ denotes terms of lower degree in $a_n$. Then for $d:= \sum_{i=1}^n d_i$ we obtain
$$F_d(a_1, \ldots a_n)= a_n^d \cdot F_d(\lambda_1, \ldots \lambda_{n-1}, 1 ) \ + \ \mathcal{O}(a_n)$$
and thus
$$F(a_1, \ldots, a_n)=a_n^m F_m(\lambda_1, \ldots, \lambda_{n-1},1) \ + \ \mathcal{O}(a_n)$$
Choose now $\lambda_1, \ldots, \lambda_{n-1} \in \mathbb{K}$, such that $F_m(\lambda_1, \ldots, \lambda_{n-1},1) \neq 0$. If $\mathbb{K}$ is infinite, this is always possible. In the finite case, go back to $(*)$ and use $b_i:=a_i+a_n^{\mu_i}$ instead and repeat the procedure.\\
Then by the first case and induction hypothesis the claim follows.
\end{compactenum}
\end{compactenum}


%SECTION 14

\renewcommand*\thesection{\S\ \arabic{section}\quad }
\section{Dedekind domains}
\renewcommand*\thesection{\arabic{section}}

\subsection{Definition} %Definition 14.1
A noetherian integral domain $R$ of dimension $1$ is called a \textit{Dedekind domain}, if every nonzero ideal $I \triangleleft R$ has a unique representation as a product of prime ideals
$$I \ = \ \mathfrak{p}_1 \cdot \cdot \cdot \mathfrak{p}_r$$

\subsection{Definition + Remark} %Definition + Remark 14.2
Let $R$ be a noetherian integral domain, $\mathbb{K}:= \textrm{Quot}(R)$ and $\langle 0 \rangle \neq I \subseteq \mathbb{K}$ an $R$-module.
\begin{compactenum}
\item $I$ is called a \textit{fractional ideal}, if there exists $a \in R\setminus \{0\}$, such that $a \cdot I \subseteq R$. 
\item $I$ is a fractional ideal if and only if $I$ is finitely generated as an $R$-module.
\item For a fractional ideal $I$ let
$$I^{-1}:=\{x \in \mathbb{K} \big\vert x \cdot I \subseteq R \}$$
Then $I^{-1}$ is a fractional ideal.
\item $I$ is called \textit{invertible}, if $I \cdot I^{-1}=R$, where $I \cdot I^{-1}$ denotes the $R$-module generated by all products $x\cdot y$ with $x \in I, y \in I^{-1}$.
\end{compactenum}
\textit{proof.}
\begin{compactenum}
\item[(ii)] \begin{compactenum}
\item['$\Rightarrow$'] If $a \cdot I \subseteq R$, then $a \cdot I$ is an ideal in $R$. since $R$ is noetherian, $a \cdot I$ is finitely generated, say by $x_1, \ldots, x_n$. Then $I$ is generated by $\frac{x_1}{a}, \ldots, \frac{x_n}{a}$. 
\item['$\Leftarrow$'] Let $y_1, \ldots, y_m$ be generators of $I$. Write $y_i = \frac{r_i}{a_i}$ with $r_i, a_i \in R\setminus 0$. Define
$$a:= \prod_{i=1}^n a_i$$
Then for any generator we have $a \cdot y_i = r \cdot a_1 \cdot \ldots a_{i-1} \cdot a_{i+1} \cdot  \ldots \cdot a_m \in R$, hence $a \cdot I \subseteq R$.
\end{compactenum} 
\end{compactenum}

\titleformat{\subsection}{\normalfont\normalsize\bfseries}{}{0em}{#1}
\subsection*{Example} %Example
\titleformat{\subsection}{\normalfont\normalsize\bfseries}{}{0em}{#1 \thesubsection}
Every principle ideal $I \neq \langle 0 \rangle$ is invertible:\\
Let $I=\langle a \rangle \trianglelefteqslant R$. Then $I^{-1}=\frac{1}{a} R$, since we have
$$I \cdot I^{-1}=\langle a \rangle \cdot \frac{1}{a} R=aR \cdot \frac{1}{a}R=R$$

\subsection{Proposition} %Proposition 14.3
Let $R$ be a Dedekind domain. Then every nonzero ideal $I \trianglelefteqslant R$ is invertible.\\
\textit{proof.}\\
Let $\langle 0 \rangle \neq I \triangleleft R$ be a proper ideal. Then by assumption we can write
$$I= \mathfrak{p}_1 \ \cdot \cdot \cdot \cdot \mathfrak{p}_r$$
with prime ideal $\mathfrak{p}_i \triangleleft R$. \\
If each $\mathfrak{p}_i$ is invertible, then we have
$$I \cdot \mathfrak{p}_r^{-1} \cdot \cdot \cdot \mathfrak{p}_1^{-1}=R,$$hence $I$ is invertible. Thus we may assume that $I= \mathfrak{p}$ is prime.\\
Let $a \in \mathfrak{p} \setminus \{0\}$ an write
$$\langle a \rangle = \mathfrak{p}_1 \ \cdot \cdot \cdot \mathfrak{p}_m$$
with prime ideals $\mathfrak{p}_i \triangleleft R$. Then $\langle a \rangle \subseteq \mathfrak{p}$, i.e. $\mathfrak{p}_i \subseteq \mathfrak{p}$ for some $1 \leqslant i \leqslant m$, say $i=1$.
Since the ideals were proper and $\dim(R)=1$, we have $\mathfrak{p}_1=\mathfrak{p}$ and $\mathfrak{p}^{-1}=\mathfrak{p}_1^{-1}=\frac{1}{a} \cdot \mathfrak{p}_2 \cdot \cdot \cdot \mathfrak{p}_m$, since $\mathfrak{p}_1 \mathfrak{p}_{1}^{-1}=\frac{1}{a} \langle a \rangle = \langle 1 \rangle = R$.

\subsection{Corollary} %Corollary 14.4
The fractional ideals in a Dedekind domain $R$ form a group.\\
\textit{proof.}\\
Let $\langle 0 \rangle \neq I \subseteq \mathbb{K}=\textrm{Quot}(R)$ ba a fractional ideal. Choose $a \in R$ such that $a \cdot I \subseteq R$. \\
By Proposition 14.3, $a \cdot I$ is invertible, i.e. there exists a fractional ideal $I'$, such that 
$$(a \cdot I) \cdot I' = R \ \Longrightarrow \ I \cdot (a \cdot I') =R$$
where $R$ is neutral element of the group.

\subsection{Proposition} %Proposition 14.5
Every Dedekind domain $R$ is normal. \\
\textit{proof.}\\
Let $x \in \mathbb{K}:= \rm{Quot}(R)$ be integral over $R$, i.e. we can write
$$x^n+a_{n-1}X^{n-1} + \ldots a_1x+a_0=0, \qquad a_i \in R$$
By the proof of Proposition 13.3, $R[x]$ is a finitely generated $R$-module, hence $R[x]$ is a fractional ideal by Remark 14.2. Further by Corollary 14.4 $R[x]$ is invertible, i.e. we can find $I \trianglelefteqslant \mathbb{K}$, such that $I \cdot R[x]=R$.\\
On the other hand $R[x]$ is a ring, i.e. $R[x] \cdot R[x] = R[x]$. Multiplying the equation by $I$ gives us $x \in R$. In particular we have
$$R=I \cdot R[x]=I \cdot (R[x] \cdot R[x])=(I \cdot R[x])\cdot R[x]=R \cdot R[x]=R[x]$$

\subsection{Proposition} %Proposition 14.6
Let $R$ be noetherian integral domain of dimension $1$.\\
Then $R$ is a Dedekind domain if and only if $R$ is normal.\\
\textit{proof.}
\begin{compactitem}
\item['$\Rightarrow$'] This is Proposition 14.5
\item['$\Leftarrow$'] We claim
\begin{compactenum}
\item[\textbf{claim (a)}] For every prime ideal $\langle 0 \rangle \neq \mathfrak{p} \triangleleft R$ the localization $R_{\mathfrak{p}}$ is a discrete valuation ring.
\item[\textbf{claim (b)}] Every nonzero ideal in $R$ is invertible.
\end{compactenum}
Then let $\langle 0 \rangle \neq I \neq R$ be an ideal in $R$.\\
Then $I \subseteq \mathfrak{m}_0$ for a maximal ideal $\mathfrak{m}_0 \triangleleft R$. By claim (b), $\mathfrak{m}_0$ is invertble. Define $I_1 := \mathfrak{m}_0^{-1} \cdot I$.\\
Then $I_1 \subseteq \mathfrak{m}_0^{-1} \cdot \mathfrak{m}_0 = R$ is an ideal.\\
If $I_1=R$, then $I=\mathfrak{m}_0$. Otherwise let $\mathfrak{m}_1$ be a maximal ideal containing $I_1$ and define $I_2:=\mathfrak{m}_1^{-1}\cdot I_1 \trianglelefteqslant R$.\\
If $I_1=I$, then $\mathfrak{m}_0^{-1} \cdot I=I \overset{\rm{invert.}}{\Longrightarrow} \mathfrak{m}_0^{-1}=R$, which is a contradiciton.\\
By this way we obtain a chain of ideals
$$I \subsetneq I_1 \subsetneq I_2 \subsetneq \ldots \subsetneq I_n$$
Since $R$ is noetherian, there exists $n \in \mathbb{N}$; such that $I_n=R$. \\
Then $$R=I_n=\mathfrak{m}_{n-1}^{-1} \cdot I_{n-1}=\mathfrak{m}_{n-1}^{-1} \cdot \mathfrak{m}_{n-1}^{-1} \cdot I_{n-2} = \mathfrak{m}_{n-1}^{-1} \cdot \cdot \cdot \mathfrak{m}_0^{-1} \cdot I$$
Thus
$$I=\mathfrak{m}_0 \cdot \mathfrak{m}_1 \cdot \cdot \cdot \mathfrak{m}_{n-2} \cdot \mathfrak{m}_{n-1}$$
with maximal, thus prime ideals $\mathfrak{m}_i$. Hence $R$ is a Dedekind domain.
\end{compactitem}
It remains to show the claims.
\begin{compactenum}
\item[\textbf{(b)}] Let $\langle 0 \rangle \neq I \trianglelefteqslant R$ be an ideal. We have to show
$$I \cdot I^{-1}=R\qquad \textrm{ for } I^{-1}=\{x \in \mathbb{K} \mid x \cdot I \subseteq R\}$$
\begin{compactitem}
\item['$\subseteq$'] Clear.
\item['$\supseteq$'] Assume $I\cdot I^{-1} \neq R$. Then there exists a maximal ideal $\mathfrak{m}\triangleleft R$ such that $I\cdot I^{-1} \subseteq \mathfrak{m}$. By claim (a), $R_{\mathfrak{m}}$ is a principal ideal domain, thus $I\cdot R_{\mathfrak{m}}$ is generated by one element, say $\frac{a}{s}$ for some $a \in I, s \in R\setminus \mathfrak{m}$.
Let now $b_1, \ldots, b_n$ be generators of $I$ as an ideal in $R$. Then 
$$\frac{b_i}{1}=\frac{a}{s} \cdot \frac{r_i}{s_i}, \quad r_i \in R, s_i \in R\setminus \mathfrak{m}, \ \textrm{ for } 1 \leqslant i \leqslant n$$
Define $t:= s \cdot s_1 \cdot \cdot \cdot s_n \in R\setminus \mathfrak{m}$.\\
We have $\frac{t}{a} \in I^{-1}$, since
$$\frac{t}{a} \cdot b_i= \frac{t}{a} \cdot \frac{a}{s} \cdot \frac{r_i}{s_i}=r_i \cdot s_1 \cdot \cdot \cdot s_{i-1} \cdot s_{i+1} \cdot \cdot \cdot s_n \in R$$
for $1 \leqslant i \leqslant n$. But then 
$$t=\frac{t}{a} \cdot a \in I^{-1} \cdot I \subseteq \mathfrak{m} \quad \lightning$$
\end{compactitem}
\item[\textbf{(a)}] We will only give a proof sketch. The strategy is as follows:
\begin{compactenum}
\item[(i)] Ot suffices to show, that $\mathfrak{m}:=\mathfrak{p}R_{\mathfrak{p}}$ is a principal ideal.
\item[(ii)] Show that $\mathfrak{m}^n \neq \mathfrak{m}$.
\item[(iii)] Show that $\mathfrak{m}$ is invertible.
\end{compactenum}
Then pick $t \in \mathfrak{m}^2 \setminus \mathfrak{m}$ and obtain $t\cdot \mathfrak{m}^{-1}=R_{\mathfrak{m}}$. This is true, since otherwise, as $\mathfrak{m}$ is the only maximal ideal in $R_{\mathfrak{p}}$, we would have $t \cdot \mathfrak{m}^{-1} \subseteq \mathfrak{m}$ and thus $t \in \mathfrak{m}^2$, which implies $\mathfrak{m}=\mathfrak{m}^2$. Then we have
$$\langle t \rangle = t \cdot R = t \cdot (\mathfrak{m} \cdot \mathfrak{m}^{-1})=R_{\mathfrak{p}} \cdot \mathfrak{m} = \mathfrak{m}$$


\end{compactenum}

\subsection{Theorem} %Theorem 14.7
Let $R$ be a Dedekind domain, $\mathbb{L}/\mathbb{K}$ a finite separable field extension of $\mathbb{K}:= \textrm{Quot}(R)$ and $S$ the integral closure of $R$ in $\mathbb{L}$. Then $S$ is a Dedekind domain.\\
\textit{proof.}\\
We will show all the required properties of a Dedekind domain.\\
\textit{integral domain.} This is clear.\\
\textit{dimension 1.} We know that $S/R$ is integral and Proposition 13.7 gives us $\dim(S)=1$. \\
\textit{normal.} If $x \in \mathbb{L}$ is integral over $S$, $x$ is integral over $R$, thus $x \in S$. \\
\textit{noetherian.} This is the only hard work in the proof. \\
Let $N:=[\mathbb{L}:\mathbb{K}]$. Since $\mathbb{L}/\mathbb{K}$ is separable, there exists $\alpha \in \mathbb{L}$ such that $\mathbb{L}=\mathbb{K}(\alpha)$. Moreover we have $\big\vert \textrm{Hom}_{\mathbb{K}}(\mathbb{L}, \overline{\mathbb{K}}) \big\vert =n$, say $\textrm{Hom}_{\mathbb{K}}(\mathbb{L}, \overline{\mathbb{K}})= \{\rm{id}=\sigma_1, \ldots \sigma_n \}$.
\begin{compactenum}
\item[\textbf{claim (a)}] $\alpha$ can be chosen in $S$. 
\end{compactenum}
Then let
$$D:= \begin{pmatrix}[cccc] 1 & \alpha & \ldots & \alpha^{n-1} \\ 1 & \sigma_2(\alpha) & \ldots & \sigma_2(\alpha^{n-1}) \\ \vdots & \vdots & & \vdots \\ 1 & \sigma_n(\alpha) & \ldots & \sigma_n(\alpha^{n-1}) \end{pmatrix}=\left(\sigma_i(\alpha^j)\right)_{(i,j) \in \{1, \ldots, n \} \times \{0, \ldots, n-1 \}}$$
and $d:= \left(\det(D)\right)^2$. $d:= d_{\mathbb{L}/\mathbb{K}}(\alpha)$ is called the \textit{discriminant of }$\mathbb{L}/\mathbb{K}$\textit{ w.r.t.} $\alpha$. 
\begin{compactenum}
\item[\textbf{claim (b)}]We have
 \begin{compactenum}
\item[(i)] $d \neq 0$
\item[(ii)] $S$ is contained in the $R$-module generated by $\frac{1}{d}, \frac{\alpha}{d}, \ldots, \frac{\alpha^{n-1}}{d}$.
\end{compactenum}
\end{compactenum}
Then $S$ is submodule of a finitely generated $R$-module, and since $R$ is noetherian, $S$ is noetherian as an $R$-module, thus also as an $S$-module. This proves \textit{noetherian}. Now prove the claims.
\begin{compactenum}
\item[\textbf{(a)}] Let $\tilde{\alpha} \in \mathbb{L}$ ba a primitive element, i.e. $\mathbb{L}=\mathbb{K}(\tilde{\alpha})$. Let 
$$f = X^n- \sum_{i=0}^{n-1} c_i X^{i}$$
be the minimal polynomial of $\tilde{\alpha}$ over $\mathbb{K}$. Writr $c_i = \frac{a_i}{b_i}$ for suitable $a_i, b_i \in R, b_i \neq 0$. Now define
$$b:= \prod_{i=0}^{n-1} b_i, \qquad \alpha:= b \cdot \tilde{\alpha}$$
Since we have
$$\alpha^n=b^n \tilde{\alpha}^n=b^n \cdot \sum_{i=0}^{n-1} c_i \tilde{\alpha}^{i} =\sum_{i=0}^{n-1} c_i \cdot \frac{\alpha^{i}}{b^{i}} b^n$$
we obtain
$$\alpha^n = b^n \cdot \tilde{\alpha}^n = \sum_{i=0}^{n-1}c_i? \alpha^{i}, \quad c_i?=c_i \cdot b^{n-i} \in R$$
Thus $\alpha$ is integral over $R$, i.e. $\alpha \in S$. We easily see $\mathbb{K}(\alpha)=\mathbb{K}(\tilde{\alpha})$, hence the claim is proved.
\item[\textbf{(b)}] \begin{compactenum}
\item[(i)] We have
$$d= \left(\det(D)\right)^2 = \prod_{1 \leqslant i < j \leqslant n} \left(\sigma_i(\alpha)-\sigma_j(\alpha)\right)^2 \neq 0$$
Since otherwise we would have $\sigma_i(\alpha)=\sigma_j(\alpha)$, i.e.e $\sigma_i = \sigma_j$, which is not possible.
\item[(ii)] Let $\beta \in S$. Write 
$$\beta=\sum_{i=0}^{n-1} c_{i+1}\alpha^{i}, \quad c_i \in \mathbb{K}$$
We have to show: $c_i \in \frac{1}{d} R$ for all $1 \leqslant i \leqslant n$. Therefore we need
\begin{compactenum}
\item[\textbf{claim (c)}] There is a matrix $A \in R^{n \times n}$ and $b \in R^n$, such that
$$A \cdot \begin{pmatrix} c_1 \\ \vdots \\ c_n \end{pmatrix} = b \qquad \textrm{ and } \ \det(A)=d$$
\end{compactenum} 
Then by Cramer?s rule and Claim (c) we have
$$c_i= \frac{\det(A_i)}{\det(A)} = \frac{\det(A_i)}{d} \in \frac{1}{d} \in R$$
where $A_i$ is obtained by replacing the $i$-th column of $A$ by $b$. This proves claim (b).
\end{compactenum}
\item[\textbf{(c)}] Recall that 
$$tr_{\mathbb{L}/\mathbb{K}}: \mathbb{L} \longrightarrow \mathbb{K}, \quad \beta \mapsto \sum_{i=1}^n \sigma_i(\beta)$$
is a $\mathbb{K}$-linear map.For $\beta$ as above we find for $1 \leqslant i \leqslant n$
$$(*) \ tr_{\mathbb{L}/\mathbb{K}}(\underbrace{\alpha^{i-1}\beta}_{\in S})= \sum_{j=1}^n tr_{\mathbb{L}/\mathbb{K}}(\alpha^{i-1} \alpha^{j-1} c_j)=\sum_{j=1}^n tr_{\mathbb{L}/\mathbb{K}}(\alpha^{i-1}\alpha^{j-1})c_j \ \in \ \mathbb{K} \cap S = R$$
where the last equality holds since $R$ is normal and by Proposition 14.5. Let now 
$$A=\left(a_{ij}\right)_{(i,i) \in \{1,\ldots, n\} \times \{1, \ldots, n \}}, \quad a_{ij}=tr_{\mathbb{L}/\mathbb{K}}(\alpha^{i-1}, \alpha^{j-1})$$
and
$$b=\begin{pmatrix} b_1 \\ \vdots \\ b_n \end{pmatrix}, \quad b_i=Tr_{\mathbb{L}/\mathbb{K}}(\alpha^{i-1}\beta)$$
Then by $(*)$ we have
$$A \cdot \begin{pmatrix} c_1 \\ \cdots \\ c_n \end{pmatrix} = b,$$
i.e.e the first part of the claim. Moreover we have
$D^TD=\left(\tilde{a}_{ij}\right)$, where
$$\tilde{a}_{ij}=\sum_{k=1}^n \sigma_k(\alpha^{i-1}) \sigma_k(\alpha^{j-1})=\sum_{k=1}^n \sigma_k(\alpha^{i-1}\alpha^{j-1})=tr_{\mathbb{L}/\mathbb{K}}(\alpha^{i-1}, \alpha^{j-1}) = a_{ij}$$
Hence $D^TD=A$ and by $\det(D)=\det(D^T)$ we have $$\det(D)^2=\det(D\cdot D)=\det(D \cdot D^T)=\det(A)=d$$
\end{compactenum}
We have now shown that $S$ is an integral domain, of dimension $1$, noetherian and normal. By Proposition 14.6 the theorem is proved.

\newpage

\printindex

\end{spacing}
\end{document}
