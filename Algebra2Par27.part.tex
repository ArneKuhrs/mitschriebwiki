\section{Krull-Dimension}

\begin{Def}
\label{2.24}
Sei $R$ ein Ring. 
\begin{enumerate}

\item Eine Folge $\mathfrak{p}_0, \mathfrak{p}_1, \dots ,\mathfrak{p}_n$
von Primidealen in $R$ hei\ss t \emp{Primidealkette}\index{Primidealkette} zu $\mathfrak{p}=\mathfrak{p_n}$
der L\"ange $n$, wenn $\mathfrak{p}_{i-1}\subsetneq \mathfrak{p}_i$ f\"ur $i=1,\ldots, n$.

\item F\"ur ein Primideal $\mathfrak{p}\subset R$ hei\ss t
$$ \hoe{\mathfrak{p}}\defeqr\sup\{n\in\NN:
\text{es gibt Primidealkette der L\"ange}\ n\ \text{zu}\ \mathfrak{p}\} $$

die \emp{H\"ohe}\index{Primideal!H\"ohe}\index{$\hoe{\mathfrak{p}}$ (Höhe von $\mathfrak{p}$)}
von $\mathfrak{p}$.

\item $\dim{R} \defeqr\sup\{\hoe{\mathfrak{p}}:\mathfrak{p}
\text{ Primideal in }R\}$
hei\ss t \emp{Krull-Dimension}\index{Krull-Dimension} von $R$.

\end{enumerate}
\end{Def}


\begin{nnBsp}
\begin{enumerate}
\item $R=k$ K\"orper $\Rightarrow \dim{k}=0$.
\item $R=\ZZ$: $\dim{\ZZ}=1$.
\item $R=k[X]$: $\dim{k[X]} =1$.
\item $R=k[X,Y]$: $\dim{k[X,Y]} =2$.\\
$\geq 2$ ist klar, da $(0)\subsetneq(X)\subsetneq(X,Y)$. Aber warum $=2$?
\end{enumerate}
\end{nnBsp}

\begin{Bem}
\label{2.25}
Sei $R$ ein nullteilerfreier Ring. Dann gilt: 
\begin{enumerate}
\item Sind $p$, $q$ Primelemente, $p\neq 0\neq q$ mit $(p)\subseteq (q)$, so ist
$(p)=(q)$.
\item Ist $R$ Hauptidealring, so ist $R$ K\"orper oder $\dim{R}=1$
\end{enumerate}
\end{Bem}
\begin{Bew}
\begin{enumerate}

\item $(p)\subseteq (q)\Rightarrow p\in(q)$, d.h. $p=q\cdot r$ f\"ur ein $r\in R$.
Da $R$ nullteilerfrei, ist $p$ irreduzibel, also $r\in R^{\times}\Rightarrow (p)=(q)$

\item $\dim{R}\leq 1$ nach (a). Sei $R$ kein K\"orper, also gibt es ein $p\in R$ 
($p\neq 0$) mit $p\notin R^{\times}$. Da $R$ nullteilerfrei, ist $(0)$ Primideal;
$p$ ist in einem maximalen Ideal $m$ enthalten ($m=(q)$)
$\Rightarrow (0)\subsetneq m$ ist Kette der L\"ange 1
$\Rightarrow \dim{R}\geq 1 \Rightarrow \dim{R}=1$

\end{enumerate}
\end{Bew}

\begin{Satz}
\label{Satz10}
Sei $S/R$ eine ganze Ringerweiterung. Dann gilt:
\begin{enumerate}

\item Zu jedem Primideal $\mathfrak{p}$ in $R$ gibt es ein Primideal $\mathfrak{P}$ in $S$
mit $\mathfrak{P}\cap R=\mathfrak{p}$

\item Zu jeder Primidealkette $\mathfrak{p}_0\subsetneq \mathfrak{p}_1\subsetneq \cdots
\mathfrak{p}_n$ in $R$ gibt es eine Primidealkette 
$\mathfrak{P}_0\subsetneq \mathfrak{P}_1\subsetneq \cdots \mathfrak{P}_n$ in $S$
mit $\mathfrak{P}_i\cap R=\mathfrak{p}_i$ ($i=0,\ldots, n$)

\item \label{Satz10c}$\dim{R}= \dim{S}$.

\end{enumerate}
\end{Satz}

\begin{Bew}
\begin{enumerate}
\item \textbf{Beh. 1: } $\mathfrak{p}\cdot S \cap R =\mathfrak{p}$

Dann sei $N\defeqr R\backslash\mathfrak{p}$ und $\mathcal{P}\defeqr
\{I\subseteq S\ \text{Ideal}: I\cap N=\varnothing, \mathfrak{p}\cdot S \subseteq I\}$

Nach Beh. 1 ist $\mathcal{P}\neq \varnothing$. Nach Zorn gibt es ein maximales Element
$\mathfrak{P}$ in $\mathcal{P}$. Die Aussage folgt also aus Beh 2.:

\textbf{Beh. 2: } $\mathfrak{P}$ ist Primideal

\textbf{Bew. 2: } Seien $b_1, b_2\in S\backslash \mathfrak{P}$ mit $b_1\cdot b_2\in \mathfrak{P}$.
Dann sind $\mathfrak{P}+(b_1)$ und $\mathfrak{P}+(b_2)$ nicht in $\mathcal{P}$. Es gibt
also $s_i\in S$ und $p_i\in \mathfrak{P}$ ($i=1,2$) mit $p_i+s_i\cdot b_i\in N$.
$\Rightarrow (p_1+s_1b_1)(p_2+s_2b_2)\in N\cap \mathfrak{P}=\varnothing$. Widerspruch.

\textbf{Bew. 1: } Sei $b\in \mathfrak{p}\cdot S\cap R$, $b=p_1t_1+\ldots+b_k t_k$ mit 
$p_i\in \mathfrak{p}, t_i\in S$. Da $S$ ganz ist \"uber $R$, ist 
$S'\defeqr R[t_1, \ldots, t_k]\subseteq S$ endlich erzeugbarer $R$-Modul.

Seien $s_1,\ldots, s_n$ $R$-Modul Erzeuger von $S'$. F\"ur jedes $i$ hat $b\cdot s_i$ eine
Darstellung $b\cdot s_i=\sum_{k=1}^{n}a_{ik}s_k$ mit $a_{ik}\in \mathfrak{p}$
(weil $b\in \mathfrak{p}\cdot S'$). Es folgt: $b$ ist Nullstelle eines Polynoms
vom Grad $n$ mit Koeffizienten in $\mathfrak{p}$: 

$$b^n+ \underbrace{\sum_{i=0}^{n-1}\alpha_i b^i}_{\in \mathfrak{p}}=0, \alpha_i\in \mathfrak{p}$$

Nach Voraussetzung: $b\in R\Rightarrow b^n\in \mathfrak{p}\Rightarrow b\in \mathfrak{p}
\Rightarrow \mathfrak{p}\cdot S \cap R\subseteq \mathfrak{p}$.

\item Induktion \"uber $n$: $n=0$ ist (a). $n\geq 1$:

Nach Induktionsvoraussetzung gibt es eine Kette

$$\mathfrak{P}_0\subsetneq\ldots\subsetneq\mathfrak{P}_{n-1}$$ 
in $S$ mit $\mathfrak{P}_i\cap R=\mathfrak{p}_i$ ($i=0,\ldots, n-1$). Sei
$S'\defeqr S/\mathfrak{P}_{n-1}$, $R'\defeqr R/\mathfrak{p}_{n-1}$. Dann 
ist $S'/R'$ ganze Ringerweiterung. Nach (a) gibt es in $S'$ ein Primideal $\mathfrak{P}_{n}'$
mit $\mathfrak{P}_n'\cap R'=\mathfrak{p}_n'\defeqr \mathfrak{p}_n/\mathfrak{p}_{n-1}$.
Dann gilt f\"ur $\mathfrak{P}_n\defeqr pr^{-1}(\mathfrak{P}_n')$ ($pr:S\to S'$ kan. Proj.):
$\mathfrak{P}_n\cap R=\mathfrak{p}_n$ und $\mathfrak{P_n}\neq\mathfrak{P}_{n-1}$.

\item Aus (b) folgt: $\dim{S} \geq\dim{R}$. Es bleibt zu
zeigen: $\dim{S} \leq \dim{R}$.

Sei $\mathfrak{P}_0\subsetneq \ldots \subsetneq \mathfrak{P}_n$ Kette in $S$, 
$\mathfrak{p}_i\defeqr \mathfrak{P}_i\cap R$, $i=0,\ldots,n$.\\
klar: $\mathfrak{p}_i$ ist Primideal in $R$, $\mathfrak{p}_{i-1}\subseteq \mathfrak{p}_i$.
Noch zu zeigen: $\mathfrak{p}_{i-1}\neq \mathfrak{p}_i$ f\"ur alle $i$. Gehe \"uber zu
$R/\mathfrak{p}_{i-1}$ und $S/\mathfrak{P}_{i-1}$, also \OE\ $\mathfrak{p}_{i-1}=(0)$,
$\mathfrak{P}_{i-1}=(0)$.

\textbf{Annahme}: $\mathfrak{p}_i=(0)$.

Sei $b\in \mathfrak{P}_{i}\backslash \{0\}$. $b$ ist ganz \"uber $R$: $b^n+a_{n-1}b^{n-1}+\cdots+a_1b+a_0=0$. Sei $n$ der minimale Grad einer solchen Gleichung.
Es ist $a_0=-b(b^{n-1}+a_{n-1}b^{n-2}+\cdots+a_1)\in R\cap \mathfrak{P}_i=\mathfrak{p}_i=(0)$.
$\Rightarrow 0=-b(b^{n-1}+a_{n-1}b^{n-2}+\cdots+a_1)$
Da $S$ nullteilerfrei ist, muss gelten: $b^{n-1}+a_{n-1}b^{n-2}+\cdots+a_1=0$. Widerspruch zur
Wahl von $n$.

\end{enumerate}
\end{Bew}

\begin{Folg}
\label{2.26}
Sei $S/R$ ganze Ringerweiterung, $\mathfrak{p}$ bzw. $\mathfrak{P}$ Primideale in $R$ bzw. $S$.
Ist $\mathfrak{p}=\mathfrak{P}\cap R$, so gilt:
\[
\mathfrak{P}\ \text{maximal} \Leftrightarrow\ \mathfrak{p}\ \text{maximal}.
\]
\end{Folg}

\begin{Bew}
``$\Leftarrow$'': Sei $\mathfrak{P}'$ maximales Ideal in $S$ mit 
$\mathfrak{P}\subseteq \mathfrak P'$. Dann ist $\mathfrak{P}'\cap R=\mathfrak{p}$ weil
$\mathfrak{p}$ maximal $\Rightarrow\mathfrak{P'}=\mathfrak{P}$. Nach dem Beweis
von Teil (c) des Satzes.

``$\Rightarrow$'': Sei $\mathfrak{p}'$ maximales Ideal mit $\mathfrak{p}\subseteq \mathfrak{p'}$.
Nach (b) gibt es ein Primideal $\mathfrak{P}'$ in $S$ mit $\mathfrak{P}'\cap R=\mathfrak{p}'$
und $\mathfrak{P}\subseteq \mathfrak{P}'\Rightarrow \mathfrak{P'}=\mathfrak{P}\Rightarrow
\mathfrak{p}'=\mathfrak{p}$.

\end{Bew}

\begin{Satz}
\label{11}
Sei $k$ K\"orper, $A$ endlich erzeugbare $k$-Algebra.
\begin{enumerate}
\item In $A$ gibt es algebraisch unabh\"angige Elemente $x_1, \ldots, x_d$ (f\"ur ein
$d\geq 0$), sodass $A$ ganz ist \"uber $k[x_1, \ldots, x_d]$ (isomorph zu 
$k[X_1,\ldots,X_d]$, da algebraisch unabh\"angig).
\item Ist $I\subseteq A$ ein echtes Ideal, so k\"onnen in a) die $x_i$ so
gew\"ahlt werden, dass $I\cap k[x_1, \ldots, x_d]=(x_{\delta+1},\ldots,x_d)$ f\"ur ein
$\delta \leq d$.
\item $\dim{k[x_1,\ldots,x_d]}=d$ ($\Rightarrow \dim{A} = d$).
\end{enumerate}
\end{Satz}

\begin{Bew}
\begin{enumerate}
\item[(c)] ``$\geq$'': klar\\
``$\leq$'': Sei $(0)\subsetneq \mathfrak{p}_1\subsetneq \cdots\subsetneq \mathfrak{p}_m$ Primidealkette
in $A$. \OE\ (Satz 10) $A=k[x_1,\ldots, x_n]$. Nach (b) existiert eine Einbettung 
$B\defeqr k[y_1,\ldots,y_d]\hookrightarrow A$ mit $\mathfrak{p}_1\cap k[y_1,\ldots, y_d]
=(y_{\delta+1},\ldots, y_d)$.\\
\textbf{Beh.:} $\delta\leq d-1$ (d.h. $\mathfrak{p}_1\cap k[y_1,\ldots,y_d]\neq \{0\}$).

Denn: Sonst $A$ ganz \"uber $B$ $\Rightarrow \mathfrak{p}_1=(0)$ (Satz 10, Beweis Teil (c)).
Sei nun $A_1\defeqr A/\mathfrak{p}_1, B_1\defeqr B/(\mathfrak{p}_1\cap B)
\cong k[y_1,\ldots, y_\delta]$. $A_1$ ist ganz \"uber $B_1$, also ist nach Satz 10 (c)
$\dim{A_1}=\dim{B_1} \stackrel{I.V.}{=}\delta$.

Weiter ist $0=\mathfrak{p}_1/\mathfrak{p}_1\subsetneq \mathfrak{p}_2/\mathfrak{p}_1
\subsetneq \cdots \subsetneq \mathfrak{p}_m/\mathfrak{p}_1$ Primidealkette in $A_1$.
$\Rightarrow m-1\leq \delta \leq d-1 \Rightarrow m\leq d$.

\item[(a)] Sei $A=k[a_1,\ldots, a_n]$  (endl. Erzeug.-Syst.)\\
Ind. \"uber $n$: 

$n=1$: $A=k[a]$; ist $a$ transzendent, so ist $A\cong k[X]$. Sonst: $A\cong k[X]/(f)$
f\"ur ein irreduzibles $f\in k[X]$, also endliche K\"orpererweiterung von $k$.

$n> 1$: Sind $a_1,\ldots, a_n$ algebraisch unabh\"angig, so ist $A\cong k[X_1,\ldots, X_n]$.
Andernfalls gibt es $F\in k[X_1, \ldots, X_n]\setminus k$ mit $F(a_1, \ldots,
a_n)=0$.

\textbf{1. Fall}: $F=X_n^m+\sum_{i=0}^{m-1}g_i X_n^i$ f\"ur ein $m\geq 1$ und
$g_i\in k[X_1,\ldots, X_{n-1}]$.

Aus $F(a_1, \ldots, a_n)=0$ folgt, dass $a_n$ ganz \"uber $k[a_1, \ldots,
a_{n-1}]\defeql A'$ ist.
Nach Induktionsvoraussetzung existieren algebraisch unabh\"angige Elemente 
$x_1,\ldots,x_d$ in
$k[a_1,\ldots, a_{n-1}]$, sodass $A'$ ganz \"uber $k[x_1,\ldots,x_d]$. $A$ ist also ganz
\"uber $k[x_1, \ldots, x_d]$, da $A=A'[a_n]$.

\textbf{2. Fall}: $F$ beliebig, $F=\sum_{i=0}^{m}F_i$ homogen vom Grad $i$. 

Ersetze $a_i$ durch $b_i\defeqr a_i-\lambda_i a_n$ ($i=1,\ldots, n-1$, $\lambda_i\in k$
``geeignet''). Dann sind $b_1,\ldots, b_{n-1},a_n$ auch $k$-Algebra-Erzeuger von $A$.
Das Monom $a_1^{\nu_1}\cdots a_n^{\nu_n}$ geht \"uber in 
\[
a_n^{\nu_n}\prod_{i=1}^{n-1}(b_i+\lambda_i a_n)^{\nu_i}
=a_n^{\nu_n}\prod_{i=1}^{n-1} \lambda_i^{\nu_i} a_n^{\nu_i}+
\text{Terme niedriger Ordnung in } a_n
\]
$\Rightarrow F_m(a_1,\ldots,
a_n)=F_m(\lambda_1,\ldots,\lambda_{n-1},1)a_n^m+$Terme niedriger Ordnung in 
$a_n$ $\Rightarrow F(a_1,\ldots,
a_n)=F(\lambda_1,\ldots,\lambda_{n-1},1)a_n^m+$Terme niedriger Ordnung in $a_n$.

Ist $F_m(\lambda_1,\ldots, \lambda_{n-1},1)\neq 0$, so weiter wie in Fall 1.\\
Ist $k$ unendlich, so kann man immer $\lambda_1,\ldots,\lambda_n$ finden, sodass
\[
F_m(\lambda_1,\ldots,\lambda_{n-1},1)\neq 0.
\]
Ist $k$ endlich, so hilft es, $a_i$ durch $b_i=a_i-a_n^{\mu_i}$ zu ersetzen.

\item[(b)] \OE: $A=k[x_1,\ldots,x_d]$ (betrachte $I'=I\cap k[x_1,\ldots,x_d]$).

\textbf{1. Fall}: $I=(f)$ Hauptideal, $f\neq 0$.

Setze $y_d\defeqr f$, $y_i=x_i-\lambda_ix_d$ f\"ur geeignete $\lambda_i\in k$.
Dann ist $f-y_d=0$ normiertes Polynom in $x_d$ \"uber $k[y_1,\ldots,y_d]$ (vgl. (a)).

\textbf{Beh.}: $I\cap k[y_1,\ldots, y_d]=(y_d)$

Denn: Sei $g\in I\cap k[y_1,\ldots, y_d]$, d.h. $g=h\cdot f $ f\"ur ein $h\in k[x_1,\ldots,x_d]$.
$h$ ist ganz \"uber $k[y_2,\ldots, y_d]
\Rightarrow h^m+b_{m-1}h^{m-1}+\cdots+b_1h+b_0=0$
($m\geq 1$, $b_i\in k[y_1,\ldots, y_d]$) $\Rightarrow g^m+\underbrace{b_{m-1}fg^{m-1}+\cdots+
b_1f^{m-1}g+b_0f^m}_{=y_d\cdot\ldots}=0$

$y_d$ teilt also $g^m$, d.h. $g^m\in (y_d)\stackrel{\text{prim}}{\Rightarrow} g\in (y_d)$

\textbf{2. Fall}

Sei $I$ beliebig. Induktion \"uber $d$:

$d=1$:	$A = k[X]$ $\Rightarrow$ jedes Ideal ist Hauptideal.

$d>1$: Sei $f \in I$, $f \neq 0$.

Dann gibt es nach Fall 1 eine Einbettung $k[y_1, \ldots, y_d] \hookrightarrow A$ mit $f = y_d$.

$I' := I \cap k[y_1, \ldots, y_{d-1}]$

Nach Ind.Vor. gibt es Einbettung $k[z_1, \ldots, z_{d-1}] \hookrightarrow k[y_1, \ldots, y_{d-1}]$ mit $I' \cap k[z_1, \ldots, z_{d-1}] \subset (z_{\delta+1}, \ldots, z_{d-1})$ f"ur ein $\delta \leq d-1$.

$\Rightarrow$ $I \cap k[z_1, \ldots, z_{d-1}, z_{d_n}] = (z_{\delta+1}, \ldots, z_{d-1}, y_d)$.

\begin{nnFolg}
F\"ur jede endlich erzeugte nullteilerfreie $k$-Algebra $A$ \"uber einem K\"orper $k$ gilt:
\[
\trdeg{\Quot{A}} = \dim A.
\]
Dabei ist $\trdeg{K}$ (der \emp{Transzendenzgrad}\index{Transzendenzgrad} von
$K$ \"uber $k$) die maximale Anzahl \"uber $k$ algebraisch unabh\"angiger
Elemente in $K$.
\end{nnFolg}

\end{enumerate}
\end{Bew}
