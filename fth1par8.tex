\documentclass{article}
\newcounter{chapter}
\setcounter{chapter}{8}
\usepackage{ana}
\def\gdw{\equizu}
\def\Arg{\text{Arg}}
\def\MdD{\mathbb{D}}
\def\Log{\text{Log}}
\def\Tr{\text{Tr}}
\def\abnC{\ensuremath{[a,b]\to\MdC}}
\def\wegint{\ensuremath{\int\limits_\gamma}}
%iso-int
\def\iint{\ensuremath{\int\limits}}
\title{Komplexe Wegintegrale}
\author{Matej Belica, Bernhard Konrad, Florian Mickler}
% Wer nennenswerte �nderungen macht, schreibt euch bei \author dazu

\begin{document}
\maketitle

Im Folgenden sei $I= [a,b] \subseteq \MdR \; (a<b)$ und $\varphi,\psi: I \rightarrow \MdC$ Funktionen.

\begin{definition}
\begin{liste}
\item Ist $\varphi$ auf $I$ stetig, so setze: $\int_a^b \varphi(t)dt := \int_a^b \Re \varphi(t)dt + i \int_a^b \Im \varphi(t)dt$; \quad $\int_b^a \varphi(t)dt := -\int_a^b \varphi(t)dt; \quad \int_a^a \varphi(t)dt := 0$
\item $\varphi$ hei"st auf $I$ \begriff{differenzierbar} (db) $\Leftrightarrow \Re \varphi, \, \Im \varphi$ sind auf $I$ differenziebar.
In diesem Fall: $\varphi' := (\Re \varphi)' + i(\Im \varphi)'$
\item $\varphi$ hei"st auf $I$ \begriff{stetig differenzierbar} $\Leftrightarrow \Re \varphi, \, \Im \varphi$ sind auf $I$ setig differenziebar.\\
\end{liste}
\end{definition}

%Satz 8.1
\begin{samepage}\begin{satz}
Sei $D \subseteq \MdC$ offen, $f\in H(\MdC), \, \varphi(I) \subseteq D$ und $\varphi$ auf $I$ differenziebar.\\
 Dann ist $f \circ \varphi: I \rightarrow \MdC$ differenziebar auf $I$ und $(f \circ \varphi)'(t) = f'(\varphi(t))\varphi'(t) \, \forall t \in I.$
\end{satz}\end{samepage}

\begin{beweis}
"Ubung!
\end{beweis}

%Satz 8.2)
\begin{samepage}\begin{satz}
\begin{liste}
\item Sei $\varphi$ stetig auf $I$ und $\Phi: I \rightarrow \MdC$ definiert durch $\Phi(s):= \int_a^s \varphi(t)dt$. Dann ist $\Phi$ stetig differenziebar auf $I$ und $\Phi' = \varphi$ auf $I$
\item Sei $\varphi$ auf $I$ stetig differenziebar $\Rightarrow \int_a^b \varphi'(t)dt = \varphi(b) - \varphi(a)$
\end{liste}
\end{satz}\end{samepage}

\begin{beweis}
"Ubung!
\end{beweis}


\begin{definition}
Sei $\gamma: [a,b] \rightarrow \MdC$ ein Weg (also $\gamma$ stetig).
\begin{liste}
\item $\Tr(\gamma):= \gamma([a,b])$ \quad \begriff{Tr"ager von $\gamma$}
\item $\gamma$ hei"st \begriff{geschlossen} $:\Leftrightarrow \gamma(a) = \gamma(b)$
\item $\gamma$ hei"st \begriff{glatt} $:\Leftrightarrow \gamma$ ist auf $[a,b]$ stetig differenziebar.
\end{liste}
\end{definition}

\begin{definition}
Sei $n \in \MdN, a_1,\, \dots,\, a_{n+1} \in \MdR, \; a_1<a_2<\dots<a_{n+1}$ und $\gamma_j:=[a_j,a_{j+1}] \rightarrow \MdC$ seien Wege ($j=1,\dots,n$) mit $\gamma_j(a_{j+1}) = \gamma_{j+1}(a_{j+1}) \, (j=1,\dots,n-1)$. \\
Definiere $\gamma:[a_1,a_{n+1}] \rightarrow \MdC$ durch $\gamma(t) := \gamma_j(t)$, falls $t\in [a_j,a_{j+1}].$\\ 
Dann ist $\gamma$ ein Weg und man schreibt: $\gamma = \gamma_1 \oplus \gamma_2 \oplus \dots \oplus \gamma_n$\\
$\gamma$ hei"st \begriff{st"uckweise glatt} $: \Leftrightarrow \gamma_1,\ldots,\gamma_n$ sind glatt.
\end{definition}


\begin{bemerkungen}
\item Sei $\gamma:[a,b] \rightarrow \MdC$ ein Weg. $\gamma$ ist st"uckweise glatt $\Leftrightarrow \exists \, a_1,\dots,a_{n+1} \in [a,b]: \, a=a_1<a_2<\dots<a_{n+1}=b$ und $\gamma_{|[a_j,a_{j+1}]}$ ist glatt ($j=1,\dots,n$)
\item st"uckweise glatte Wege sind rektifizierbar.
\item glatt $\Rightarrow$ st"uckweise glatt
\end{bemerkungen}

\begin{beispiel}
$\gamma_1(t) := t \, (t \in [0,1]), \; \gamma_2(t):= 1+ i(t-1) \, (t \in [1,2]). \\
\gamma := \gamma_1 \oplus \gamma_2, \;\, \gamma_1, \gamma_2$ sind glatt, \, $\gamma_1'(t) = 1 \not= \gamma_2'(1) = i \Rightarrow \gamma$ in 1 nicht differenziebar.
\end{beispiel}

Wie in $\MdR$ zeigt man: Ist $\phi:\abnC$ stetig, so gilt: $|\iint_a^b\phi(t)dt|\le\iint_a^b|\phi(t)|dt$.\\
F"ur den Rest des Paragraphen sei $\gamma:\abnC$ stets ein Weg (also stetig).

\begin{definition}
$\gamma^-:\abnC,\gamma^-(t):=\gamma(b+a-t);\gamma^-$ hei"st der zu $\gamma$ \begriff{inverse Weg}.\\
Klar: $\Tr(\gamma)=\Tr(\gamma^-)$
\end{definition}
\begin{definition}
  Ist $\gamma$ glatt, so setze $L(\gamma):=\iint_a^b|\gamma'(t)|dt$. \\
  Ist $\gamma = \gamma_1 \oplus \cdots \oplus \gamma_n$ st"uckweise glatt (mit $\gamma_1,\ldots,\gamma_n$ glatt), so setze:\\
  $L(\gamma):=L(\gamma_1)+\dots+L(\gamma_n)$\\
  $L(\gamma)$ hei"st \begriff{Wegl"ange} von $\gamma$.
\end{definition}

\begin{beispiele}
    \item Seien $z_1,z_2\in\MdC,\gamma(t):=z_1+t(z_2-z_1)(t\in[0,1]),\gamma$ ist glatt.\\
      $\gamma'(t)=z_2-z_1$. 
      \folgt $L(\gamma)=\iint_0^1|z_2-z_1|dt=|z_2-z_1|$
    \item Sei $z_0\in\MdC,r>0$ und $\gamma(t):=z_0+re^{it}(t\in[0,2\pi])$\\
     
      $\gamma$ ist glatt, $\gamma'(t)=rie^{it}$, $|\gamma'(t)|=r$\folgt $L(\gamma) = \iint_0^{2\pi}rdt=2\pi r$.
\end{beispiele}

\begin{definition}
Sei $[\alpha,\beta]\subseteq\MdR$ und $h:[\alpha,\beta]\to[a,b]$ stetig differenzierbar, bijektiv und $h(\alpha)=a,h(\beta)=b$. Ist $\gamma$ st"uckweise glatt, so setze $\Gamma:=\gamma \circ h$, also $\Gamma(s)=\gamma(h(s))(s\in[\alpha,\beta])$.\\
Dann ist $\Gamma$ ein st"uckweise glatter Weg mit $\Tr(\Gamma)=\Tr(\gamma)$.\\
$h$ hei"st eine \begriff{Parametertransformation}. Man schreibt $\Gamma \sim \gamma$.
\end{definition}

\begin{definition}
Sei $f \in C(\Tr(\gamma));$\\
Ist $\gamma$ glatt, so setze $\wegint f(z)dz:=\iint_a^bf(\gamma(t))\gamma'(t)dt$\\
Ist $\gamma=\gamma_1 \oplus \cdots \oplus \gamma_n$ st"uckweise glatt mit $\gamma_1,\ldots,\gamma_n$ glatt, so setze $\wegint f(z)dz:=\sum\limits_{j=1}^n\iint_{\gamma_j} f(z)dz$.\\
$\wegint f(z)dz$ heisst ein (komplexes) \begriff{Wegintegral} (von f l"angs $\gamma$)
\end{definition}

\begin{beispiel}
$\gamma(t):=3e^{it}(t\in[0,2\pi])$ 
 \begin{liste}
 \item $f(z)=\overline{z},\wegint \overline{z}dz=\iint_0^{2\pi} 3e^{-it}i3e^{it}dt=18\pi i$.
 \item $f(z)=z^2,\wegint z^2dz=\iint_0^{2\pi} 9e^{2it}i3e^{it}dt=0$.
 \end{liste}
\end{beispiel}

Wie in der Analysis zeigt man:


%Satz 8.3
\begin{samepage}\begin{satz}
$\gamma$ sei st"uckweise glatt, $f,g:\Tr(\gamma)\to\MdC$ seien stetig, $\alpha,\beta\in\MdC$ und $\Gamma$ sei ein st"uckweise glatter Weg mit $\Gamma \sim \gamma$.
 \begin{liste}
  \item $\wegint(\alpha f(z)+\beta g(z))dz=\alpha\wegint f(z)dz + \beta\wegint g(z)dz$
  \item $\iint_\Gamma f(z)dz=\wegint f(z)dz$ (und $L(\Gamma) = L(\gamma)$)
  \item $\iint_{\gamma^-}f(z)dz=-\wegint f(z)dz$
 \end{liste}
\end{satz}\end{samepage}


%Satz 8.4
\begin{samepage}
\begin{satz}
$\gamma$ und $f$ seien wie in 8.3. $f_n$ sei eine Folge in $C(\Tr(\gamma))$ und es sei $M:=\max_{z\in\Tr(\gamma)}|f(z)|$. Dann: 
\begin{liste}
\item $|\wegint f(z)dz| \le M L(\gamma)$\\
\item Konvergiert die Folge $(f_n)$ auf $\Tr(\gamma)$ gleichmaessig gegen f, so gilt: \\ \\ 
\centerline{$\wegint f_n(z)dz \to\wegint f(z)dz (n\to\infty)$}\\
(also: $\lim_{n\to\infty}\wegint f_n(z)dz = \wegint(\lim_{n\to\infty}f_n(z))dz$ )
\end{liste}
\end{satz}
\end{samepage}

\begin{beweis}
\begin{liste}
   \item $|\wegint f(z)dz| = |\iint_a^bf(\gamma(t))\gamma'(t)dt|\le \iint_a^b|f(\gamma(t))||\gamma'(t)|dt\le\iint_a^b M |\gamma'(t)dt=M L(\gamma)$.
   \item $M_n:=\max_{z\in\Tr(\gamma)}|f_n(z)-f(z)|$. Vorraussetzung \folgt $M_n\to 0(n\to\infty)$.\\
        $|\wegint f_n(z)dz-\wegint f(z)dz|=|\wegint(f_n(z)-f(z))dz| \stackrel{(1)}{\leq} M_n L(\gamma)$.
\end{liste}
\end{beweis}

\begin{definition}
Sei $\emptyset \neq D \subseteq \MdC$, $D$ offen und $f\in C(D)$. $f$ besitzt auf D eine \begriff{Stammfunktion} (SF) :\gdw $\exists F\in H(D):F'=f$ auf $D$.
\end{definition}

%Satz 8.5
\begin{samepage}\begin{satz}
Sei $\emptyset \neq D \subseteq \MdC$, $D$ offen und $f\in C(D)$ besitze auf $D$ die Stammfunktion $F$ und $\gamma$ sei ein st"uckweiser glatter Weg mit $\Tr(\gamma)\subseteq D$. Dann:\\
\\ \centerline{$\wegint f(z)dz = F(\gamma(b))-F(\gamma(a))$}
\end{satz}\end{samepage}

\begin{beweis}
O.B.d.A: $\gamma$ glatt. $\wegint f(z)dz = \iint_a^bf(\gamma(t))\gamma'(t)dt=\iint_a^b F'(\gamma(t))\gamma'(t)dt=\iint_a^b(F\circ \gamma)'(t)dt \gleichnach{8.2} (F \circ \gamma)(b) - (F\circ \gamma)(a)$.
\end{beweis}

%Folgerung 8.6
\begin{folgerung}
 $D$, $f$ und $\gamma$ seien wie in 8.5. Ist $\gamma$ geschlossen \folgt $\wegint f(z)dz = 0$.
\end{folgerung}

%Beispiel 8.7
\begin{wichtigesbeispiel}
Sei $z_0 \in \MdC$, $r>0$. $\gamma(t)=z_0+re^{it} (t\in[0,2\pi])$; $\gamma$ ist geschlossen. \\
F"ur $k\in\MdZ$ sei $f_k(z):=(z-z_0)^k $ $(z\in\MdC \backslash \{z_0\})$\\
\begin{liste}
   \item Sei $k\neq -1$. $f_k$ hat auf $\MdC \backslash \{z_0\}$ die Stammfunktion $z\mapsto \frac{1}{k+1}(z-z_0)^{k+1}$.\\
     8.6\folgt $\wegint(z-z_0)^kdz=0$
   \item Sei $k=-1$: $\wegint \frac{dz}{z-z_0} = \iint_0^{2\pi} \frac1{re^{it}}ire^{it}dt = 2\pi i$.
\end{liste}
Also: 
$$\frac1{2\pi i} \wegint \frac{dz}{z-z_0} = 1$$ \\ 
Wegen 8.6: Die Funktion $z\mapsto\frac1{z-z_0}$ hat auf $\MdC \backslash \{z_0\}$ keine Stammfunktion!\\
Aber:  im Falle $z_0=0$ hat die Funktion $z \mapsto \frac1z$ die Stammfunktion $\Log z$ auf $\MdC_-$.
\end{wichtigesbeispiel}

%satz 8.8
\begin{samepage}\begin{satz}
   Sei $[a_j,b_j]\subseteq \MdR$ und $\gamma_j:[a_j,b_j]\to\MdC$ glatte Wege mit $\gamma_j(b_j)=\gamma_{j+1}(a_{j+1}) $ $ (j=1,\ldots,n)$ \\
   Dann exisitert ein st"uckweise glatter Weg $\gamma$ mit: $\Tr(\gamma) = \bigcup\limits_{j=1}^n \Tr(\gamma_j),L(\gamma)=L(\gamma_1)+\cdots+L(\gamma_n)$ und $\wegint f(z)dz = \iint_{\gamma_1} f(z)dz+\cdots+\iint_{\gamma_n} f(z)dz $ $\forall f\in C(\Tr(\gamma))$
   
\end{satz}\end{samepage}

\begin{beweis}
mit 8.3(2).
\end{beweis}
\begin{beispiel}
$\gamma_1(t)=t, t\in[0,1]);$ $\gamma_2(t)=1+it, t\in[0,1]$\\
$\tilde\gamma_2:=1+i(t-1), t\in[1,2]$. Dann: $\tilde\gamma_2 \sim \gamma_2;$ $\gamma:=\gamma_1\oplus\gamma_2:[0,2]\to\MdC$.
\end{beispiel}
%satz 8.9
\begin{samepage}\begin{satz}
  Sei $(K_n)$ eine Folge kompakter Mengen in $\MdC$ mit: $K_n\neq\emptyset $ $ (\forall n\in\MdN), K_1\supseteq K_2\supseteq K_3\supseteq \ldots$ und $d(K_n):=\max\limits_{z,w\in K_n}|z-w| \to 0 $ $(n\to\infty)$. Dann: \\
  \centerline{ $\bigcap\limits_{n\in\MdN} K_n \neq \emptyset.$}
\end{satz}\end{samepage}
\begin{beweis}
W"ahle in jedem $K_n$ ein $z_n$. F"ur n,k $\in \MdN:z_n,z_{n+k}\in K_n$\\
Dann: $|z_n-z_{n+k}|\subseteq d(K_n) \folgt (z_n)$ ist eine Cauchy-Folge \folgt $\exists z_0\in\MdC:z_n\to z_0$.\\ \\
Sei $N\in\MdN$. $z_n \in K_n \forall n\ge N$. $K_N$ abgeschlossen \folgt $z_0\in K_N$.
\end{beweis}
\end{document}
