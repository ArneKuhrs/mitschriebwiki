\documentclass[a4paper,twoside,DIV15,BCOR12mm]{scrbook}

\usepackage{mathe}
\usepackage{saetze-hug}
\usepackage{enumerate}
\usepackage{dsfont}

\newcommand{\A}{\mathcal A}
\newcommand{\ind}{\mathds 1}
\DeclareMathOperator{\im}{im}
% Maß-Restriktions-Symbol
\newcommand{\MR}{\lfloor}

\pdfinfo{
	/Author (Die Mitarbeiter von http://mitschriebwiki.nomeata.de/)
	/Title   (Geometrische Maßtheorie)
	/Subject (Geometrische Maßtheorie)
}

\author{PD. Dr. Daniel Hug}
\publishers{Die Mitarbeiter von \url{http://mitschriebwiki.nomeata.de/}}
\title{Geometrische Maßtheorie}
\date{Wintersemester 2009/2010}
\makeindex

\begin{document}

\maketitle

\tableofcontents

\chapter*{Vorwort}

\section*{Über dieses Skriptum}
Dies ist ein Mitschrieb der Vorlesung \glqq Geometrische Maßtheorie\grqq\
von Herrn PD. Dr. Daniel Hug im Wintersemester 2009/2010 an der Universität Karlsruhe (TH).
 Die Mitschriebe der Vorlesung werden mit ausdrücklicher Genehmigung von Herrn Hug hier veröffentlicht,
Herr Hug ist für den Inhalt nicht verantwortlich.

\section*{Wer}
Beteiligt am Mitschrieb ist Joachim Breitner.

\section*{Wo}
Alle Kapitel inklusive \LaTeX-Quellen können unter \url{http://mitschriebwiki.nomeata.de} abgerufen werden.
Dort ist ein \emph{Wiki} eingerichtet und von Joachim Breitner um die \LaTeX-Funktionen erweitert.
Das heißt, jeder kann Fehler nachbessern und sich an der Entwicklung
beteiligen. Auf Wunsch ist auch ein Zugang über \emph{Subversion} möglich.

% \documentclass[a4paper,twoside,DIV15,BCOR12mm]{scrbook}

\usepackage{mathe}
\usepackage{saetze-hug}
\usepackage{enumerate}
\usepackage{dsfont}

\newcommand{\A}{\mathcal A}
\newcommand{\ind}{\mathds 1}
\DeclareMathOperator{\im}{im}
% Maß-Restriktions-Symbol
\newcommand{\MR}{\lfloor}

\pdfinfo{
	/Author (Die Mitarbeiter von http://mitschriebwiki.nomeata.de/)
	/Title   (Geometrische Maßtheorie)
	/Subject (Geometrische Maßtheorie)
}

\author{PD. Dr. Daniel Hug}
\publishers{Die Mitarbeiter von \url{http://mitschriebwiki.nomeata.de/}}
\title{Geometrische Maßtheorie}
\date{Wintersemester 2009/2010}
\makeindex

\begin{document}

\maketitle

\tableofcontents

\chapter*{Vorwort}

\section*{Ãœber dieses Skriptum}
Dies ist ein Mitschrieb der Vorlesung \glqq Geometrische Maßtheorie\grqq\
von Herrn PD. Dr. Daniel Hug im Wintersemester 2009/2010 an der Universität Karlsruhe (TH).
 Die Mitschriebe der Vorlesung werden mit ausdrücklicher Genehmigung von Herrn Hug hier veröffentlicht,
Herr Hug ist für den Inhalt nicht verantwortlich.

\section*{Wer}
Beteiligt am Mitschrieb ist Joachim Breitner.

\section*{Wo}
Alle Kapitel inklusive \LaTeX-Quellen können unter \url{http://mitschriebwiki.nomeata.de} abgerufen werden.
Dort ist ein \emph{Wiki} eingerichtet und von Joachim Breitner um die \LaTeX-Funktionen erweitert.
Das heißt, jeder kann Fehler nachbessern und sich an der Entwicklung
beteiligen. Auf Wunsch ist auch ein Zugang über \emph{Subversion} möglich.

\chapter{Grundlagen: Maß und Integral}

\section{Äußere Maße und Meßbarkeit}

\begin{definition}
Sei $X$ eine Menge. Eine Abbildung
\[
\mu : \mathcal P(X) \to [0,\infty]
\]
heißt \emph{äußeres Maß} auf $X$, falls gilt:
\begin{enumerate}
\item $\mu(\emptyset) = 0$
\item Für $A,A_n \subset X$, $i\in \MdN$ mit $A\subset \bigcup_{i\ge 1} A_i$ gilt 
\[
\mu(A) \le \sum_{i\ge1} \mu(A_i).
\]
\end{enumerate}
\end{definition}

Beobachte folgende einfache Folgerungen der Definition:

\begin{itemize}
\item $A\subset B \subset X \implies \mu(A)\le \mu(B)$
\item $A\subset B \cup \emptyset \cup \emptyset \cup \ldots \implies \mu(A) \le \mu(B) + \mu(\emptyset) + \mu(\emptyset) + \cdots = \mu(B)$
\end{itemize}

\begin{beispieleX}
\begin{align*}
\mu_1(A) &= 
\begin{cases}
\#A, & A \text{ endlich} \\
0, & \text{sonst}
\end{cases}
&
\mu_2(A) &= 
\begin{cases}
1, & A \ne \emptyset \\
0, & \text{sonst}
\end{cases} \\
\mu_3(A) &= 
\begin{cases}
\infty, & A \ne \emptyset \\
0, & \text{sonst}
\end{cases}
&
\mu_4(A) &= 
\begin{cases}
\infty, & A^c \text{ endlich} \\
0, & \text{sonst}
\end{cases} \\
\mu_5(A) &= 
\begin{cases}
0, & \text{$A$ abzählbar} \\
1, & \text{sonst}
\end{cases}
\end{align*}
\end{beispieleX}

Für die Konstruktion eines äußeren Maßes aus Rohdaten ist folgender Satz nützlich:

\begin{satz}
Sei $\mathcal E \subset \mathcal P (X)$ mit $\emptyset \in \mathcal E$, sei $\eta: \mathcal E \to [0,\infty]$ mit $\eta(\emptyset)=0$. Dann wird durch
\[
\mu(A) \da \inf\left\{\sum_{i=1}^\infty \eta(A_i) : A_i\in \mathcal E, i\in \MdN, A\subset\bigcup_{i\ge1}A_i\right\}
\]
($\inf\emptyset = \infty$) für $A\subset X$ ein äußeres Maß erklärt, das von $(\mathcal E, \eta)$ induzierte äußere Maß.
\end{satz}

\begin{beweis}
Es ist $0\le \mu(\emptyset) \le \sum_{i=1}^\infty \eta(\emptyset)=0$, da $\emptyset \subset \bigcup_{i=1}^\infty\emptyset$ und $\emptyset\in\mathcal E$.

Seien $A, A_i\subset X$ und $A\subset \bigcup_{i\ge1}A_i$. Wir müssen zeigen: $\mu(A) \le \sum_{i\ge 1}\mu(A_i)$.

Ist für ein $i\in\MdN$ bereits $\mu(A_i)=\infty$, so sind wir fertig. Sei also $\mu(A_i)<\infty$ für alle $i\in\MdN$. Sei $\varepsilon>0$. Dann existiert $E_{ij}\in \mathcal E$, $j\in \MdN$ mit $A_i\subset \bigcup_{j\ge 1} E_{ij}$ und
\[
\mu(A_i) + \frac{\varepsilon}{2^i}\ge \sum_{j\ge 1}\eta(E_{ij}) \quad \text{für $i\in \MdN$}
\]
Also gilt
\[
A \subset \bigcup_{i\ge 1} A_i \subset \bigcup_{i,j\ge 1} E_{ij},
%\quad E_{ij}\in \mathcal E
\]
und daraus folgt
\begin{align*}
\mu(A) &\le \sum_{i,j\ge 1}\eta(E_{ij}) \\
&= \sum_{i=1}^\infty \sum_{j=1}^\infty \eta(E_{ij}) \\
&\le \sum_{i=1}^\infty \mu(A_i) + \frac\varepsilon{2^i} \\
&\le \sum_{i=1}^\infty \mu(A_i) + \varepsilon.
\end{align*}
Mit $\varepsilon \to 0$ ergibt dies
\[
\mu(A) \le \sum_{i=1}^\infty \mu(A_i).
\]
\end{beweis}

\begin{definition}
Sei $\mu$ ein äußeres Maß auf $X$. Eine Menge $A\subset X$ heißt $\mu$-messbar, falls für alle $M\subset X$ gilt:
\[
\mu(M) = \mu(M\cap A) + \mu(M \cap A^c).
\]
Die Menge aller $\mu$-messbaren Mengen wird mit $\A_\mu$ bezeichnet.
\end{definition}

Es genügt bereits: $A$ ist $\mu$-messbar genau dann, wenn für alle $M\subset X$ gilt:
\[
\mu(M) \ge \mu(M\cap A) + \mu(M\cap A^c).
\]

Denn wegen
\[
M\subset (M\cap A) \cup (M\cap A^c) \cup \emptyset \cup \emptyset\cdots
\]
gilt 
\[
\mu(M)\le \mu(M\cap A) + \mu(M\cap A^c) + \mu(\emptyset) + \cdots.
\]

Es gilt stets $\emptyset, X\in\A_\mu$.

\begin{bemerkung}
Für $Y\subset X$ ist $\mu \MR Y$ das durch
\[
(\mu \MR Y)(M) \da \mu(M\cap Y),\quad M\subset X
\]
erklärte äußere Maß. Ferner ist $\A_\mu \subset \A_{\mu \MR Y}$. Denn für $A\in\A_\mu$ und $M\subset X$ ist
\begin{align*}
\mu_{\MR Y}(M)
&= \mu(Y\cap M) = \mu(Y\cap M\cap A) + \mu(Y\cap M \cap A^c) \\
&= (\mu \MR Y) (M\cap A) + (\mu \MR Y)(M\cap A^c).
\end{align*}
Ferner gilt
\[
A \in \A_\mu \iff \mu= (\mu \MR A) + (\mu \MR  A^c).
\]
\end{bemerkung}

\begin{proposition}
Für ein äußeres Maß $\mu$ auf $X$ gelten die folgenden Aussagen:
\begin{enumerate}[a)]
\item $\emptyset,\, X\in \A_\mu$ sowie $A\in \A_\mu \iff A^c\in\A_\mu$.
\item Für $A\subset X$ mit $\mu(A)=0$ gilt $A\in\A_\mu$.
\item Für $A_i\in\A_\mu$, $i\in\MdN$ gilt $\bigcup_{i\ge1} A_i\in\A_\mu$ und $\bigcap_{i\ge1} A_i\in\mathcal A_\mu$.
\item Für $A\in\A_\mu$, $B\subset X$ gilt
\[
\mu(A\cap B) + \mu(A\cup B) = \mu(A) + \mu(B).
\]
\item Für $A_i\in\A_\mu$, $i\in\MdN$, paarweise disjunkt, gilt
\[
\mu(\bigcup_{i=1}^\infty A_i) = \sum_{i=1}^\infty \mu(A_i).
\]
\item Für $A_i\in \A_\mu$, $i\in\MdN$ und $A_i\subset A_{i+1}$ für alle $i\in\MdN$ gilt
\[
\mu(\bigcup_{i=1}^\infty A_i) = \lim_{i\to\infty}\mu(A_i).
\]
\item Für $A_i \in \A_\mu$, $i\in \MdN$ mit $\mu(A_1) < \infty$ und $A_i\supset A_{i+1}$ für alle $i\in\MdN$ gilt:
\[
\mu(\bigcap_{i=1}^\infty A_i) = \lim_{i\to\infty}\mu(A_i).
\]
\end{enumerate}
\end{proposition}

\begin{beweis}
\begin{enumerate}
\item[c)] Seien $A_1,A_2\in \A_\mu$, $M\supset X$. Dann folgt
\begin{align*}
\mu(M) &= \mu(M\cap A_1) + \mu(M\cap A_1^c) \\
&= \mu(M\cap A_1) + \mu(M\cap A_1^c\cap A_2) + \mu(M\cap A_1^c \cap A_2^c) \\
&\ge \mu(M \cap (A_1 \cup (A_1^c \cap A_2))) + \mu(M \cap A_1^c \cap A_2^c) \\
&= \mu(M\cap (A_1 \cup A_2)) + \mu(M\cap (A_1\cup A_2)^c).
\end{align*}
Daraus folgt $A_1\cup A_2 \in \A_\mu$. Per Induktion sieht man dann, dass für $A_1,\ldots,A_n\in\A_\mu$ gilt: $\bigcup_{i=1}^n A_i \in \A_\mu$.
\item[e)] Sind $A_1,\ldots,A_n\in\A_\mu$ und paarweise disjunkt, dann gilt
\begin{align*}
\mu(A_1\cup A_2) = \mu( (A_1\cup A_2)\cap A_1) + \mu( (A_1\cup A_2)\cap A_1^c) 
= \mu(A_1) + \mu(A_2),
\end{align*}
woraus
\[
\mu(\bigcup_{i=1}^n A_i) = \sum_{i=1}^n \mu(A_i)
\]
folgt. Wegen
\[
\sum_{i=1}^n \mu(A_i) \le \mu(\bigcup_{i=1}^\infty A_i)
\]
gilt
\[
\sum_{i=1}^\infty \mu(A_i) \le \mu(\bigcup_{i=1}^\infty A_i) \le \sum_{i=1}^\infty \mu(A_i)
\]
und damit Gleichheit.
\item[f)] Wir definieren $B_1 \da A_1$, $B_2 \da A_2\setminus A_1$, $B_3 \da A_3\setminus A_2\ldots$ Nun ist $B_i\in \A_\mu$ für alle $i\in\MdN$ und die $B_i$ sind paarweise disjunkt. Es folgt
\begin{align*}
\lim_{k\to\infty} \mu(A_k) 
&= \lim_{k\to\infty} \mu(\bigcup_{i=1}^k B_i) \\
&= \lim_{k\to\infty} \sum_{i=1}^k \mu(B_i)\\
&= \sum_{i=1}^\infty \mu(B_i) \tag{nach e)}\\
&= \mu(\bigcup_{i=1}^\infty B_i) \\
&= \mu(\bigcup_{i=1}^\infty A_i).
\end{align*}
\item[g)] Es ist
\begin{align*}
\mu(A_1) = \mu(A_2\cup (A_1\setminus A_2)) = \mu(A_2) + \mu(A_1\setminus A_2),
\end{align*}
das heißt
\[
\mu(A_1\setminus A_2) = \mu(A_1) - \mu(A_2).
\]
Damit zeigt man
\begin{align*}
\mu(A_1) -  \mu(\bigcap_{i\ge 1} A_i) 
&= \mu(A_1\setminus \bigcap_{i\ge1}A_i)  \\
&= \mu(A_1 \cap (\bigcap_{i\ge1}A_i)^c) \\
&= \mu(A_1 \cap (\bigcap_{i\ge1}A_i^c)) \\
&= \mu(\bigcup_{i\ge 1}(A_1\cap A_i^c)) \\
&= \lim_{i\to\infty} \mu(\underbrace{\A_1\cap A_i^c}_{= A_1\setminus A_i})\tag{nach f)} \\
&= \lim_{i\to\infty} (\mu(A_1) -\mu(A_i)) \\
&= \mu(A_1) - \lim_{i\to\infty}\mu(A_i)
\end{align*}
und damit die Behauptung.
\item[c)] Sei $M\subset X$. Wir definieren $C_k \da \bigcup_{i=1}^k A_i \in \A_\mu$. Damit gilt $C_1\subset C_2\subset \cdots$.

Sei ohne Beschränkung der Allgemeinheit $\mu(M) <\infty$. Dann gilt
\begin{align*}
\infty > \mu(M) &= (\mu \MR{}M)(X)\\
&= (\mu \MR{}M)(C_k) + (\mu \MR{}M)(C_k^c) \\
&= \lim_{k\to\infty} (\mu \MR M)(C_k) + \lim_{k\to\infty}(\mu \MR M)(C_k^c) \\
&= (\mu \MR M)(\bigcup_{i\ge 1}C_i) + (\mu \MR M)(\bigcap_{i\ge 1}C_i^c) \\
&= (\mu \MR M)(\bigcup_{i\ge 1}C_i) + (\mu \MR M)( (\bigcup_{i\ge 1}C_i)^c) \\
&= \mu(M\cap (\bigcup_{i\ge 1}A_i)) + \mu(M\cap (\bigcup_{i\ge 1}A_i)^c) 
\end{align*}
und somit $\bigcup_{i\ge 1}A_i \in \A_\mu$.

\item[d)] Für $A\in\A_\mu$ und $B\subset X$ gilt:
\begin{align*}
\mu(A\cup B) &= \mu( (A\cup B) \cap A) + \mu( (A\cup B)\cap A^c) \\
&= \mu(A) + \mu(B\cap A^c)\\
\intertext{sowie}
\mu(B) &= \mu(B\cap A) + \mu(B\cap A^c).
\intertext{Hiermit so erhält man}
\mu(A) + \mu(B) &= \mu(A) + \mu(B\cap A) + \mu(B\cap A^c)\\
&= \mu(B\cap A) + \mu(A\cup B).
\end{align*}
\end{enumerate}
\end{beweis}

Hinweis: Es ist $\A_\mu$ eine (bezüglich $\mu$ vollständige) $\sigma$-Algebra und $\mu$ ist ein $\sigma$-additives Maß auf $\A_\mu$, wobei „$\A_\mu$ ist $\mu$-vollständig“ heißt, dass jede $\mu$-Nullmenge in $\A_\mu$ liegt. $(X,\A_\mu)$ ist ein messbarer Raum und $(X,\A_\mu,\mu)$ ist ein Maßraum.

\begin{definition}
Sei $\A$ eine $\sigma$-Algebra auf $X$. Ein äußeres Maß $\mu$ auf $X$ heißt $\A$-regulär, falls $\A\subset \A_\mu$ gilt und zu jeder Menge $M\subset X$ ein $A\in\A$ existiert mit $M\subset A$ und $\mu(M) = \mu(A)$. Das äußere Maß $\mu$ heißt regulär, falls $\mu$ ein $\A_\mu$-reguläres Maß ist.
\end{definition}

\begin{proposition}
Sei $\A$ eine $\sigma$-Algebra in $X$, $\mu$ ein $\A$-reguläres äußeres Maß auf $X$. Dann gilt:
\begin{enumerate}[a)]
\item Ist $M_i\subset X$, $M_i\subset M_{i+1}$ für alle $i\in\MdN$, so ist
\[
\mu(\bigcup_{i\ge 1}M_i) = \lim_{i\to\infty}\mu(M_i).
\]
\item Zu jedem $M\subset X$ mit $\mu(M)<\infty$ existiert ein $A\in\A$, so dass für alle $B\in \A_\mu$ gilt:
\[
\mu(B\cap M) = \mu(B\cap A)
\]
\item Ist $M_1\cup M_2\in \A$ und $\mu(M_1\cup M_2) = \mu(M_1)+\mu(M_2) <\infty$, so existiereren $A_1,A_2\in\A$ mit $M_i\subset A_i$, $i=1,2$ und $\mu(A_i\setminus M_i) = 0$. Insbesondere ist $M_1,M_2\in \A_\mu$.
\end{enumerate}
\end{proposition}

\begin{beweis}
\begin{enumerate}[a)]
\item Zu jedem $i\in\MdN$ finden wir ein $A_i\in \A$ so dass $M_i\subset A_i$ und $\mu(M_i) =\mu(A_i)$. Dazu definieren wir $B_i \da \bigcap_{j\ge i} A_j$. Damit gilt $M_i \subset B_i\subset A_i$, $B_i\subset B_{i+1}$ und $B_i\in\A$, $i\in\MdN$. Es folgt
\begin{align*}
\mu(\bigcup_{i\ge 1} M_i) &\le \mu(\bigcup_{i\ge1}B_i) \\
&= \lim_{i\to\infty} \mu(B_i) \\
&\le \lim_{i\to\infty} \mu(A_i) \\
&= \lim_{i\to\infty} \mu(M_i) \\
&\le \lim_{i\to\infty} \mu(\bigcup_{i\ge 1} M_i).
\end{align*}
\item  Zu $M$ existiert ein $A\in\A$ mit $M\subset A$ und $\mu(M) = \mu(A)$. Sei $B\in \A_\mu$. Dann folgt
\begin{align*}
\mu(A) = \mu(M) &= \mu(M\cap B) + \mu(M\cap B^c) \\
&\le \mu(A\cap B) + \mu(M \cap B^c) \\
&\le \mu(A\cap B) + \mu(A \cap B^c) = \mu(A),
\end{align*}
woraus Gleichheit in obigern Ungleichungen folgt. Wegen $\mu(M)<\infty$ ist auch $\mu(M\cap B^c)<\infty$, und wir können dies von zwei obigen Termen abziehen und erhalten
\[
\mu(M\cap B) = \mu(A\cap B).
\]
\item  Zu $M_1$ existiert $\tilde A_1\in\A$ mit $M_1 \subset \tilde A_1$ und $\mu(M_1) = \mu(\tilde A_1)$. Wir definieren $A_1 \da \tilde A_1 \cap (M_1\cup M_2)$. Für diese Menge gilt nun $M_1\subset A_1 \subset M_1\cup M_2$. Wir folgern 
\[
\mu(M_1) \le \mu(A_1) \le \mu (\tilde A_1) = \mu(M_1)
\]
und wegen $M_1\cup M_2 = A_1\cup M_2$ weiter
\begin{align*}
\mu(A_1\cap M_2) + \mu(A_1 \cup M_2) &= \mu(A_1) + \mu(M_2) \\
&= \mu(M_1) + \mu(M_2) \\
&= \mu(M_1\cup M_2) \\
&= \mu(A_1 \cup M_2) < \infty,
\end{align*}
woraus $\mu(A_1\cap M_2) = 0$ folgt.

Nun ist $A_1\setminus M_1\subset A_1\cap M_2$, also gilt $\mu(A_1\setminus M_1) = 0$ und somit $A_1\setminus M_1\in \A_\mu$. Damit gilt dann $M_1 = A_1\cap (A_1\setminus M_1)^c\in \A_\mu$.
\end{enumerate}
\end{beweis}

\begin{satz}
Sei $\A$ eine $\sigma$-Algebra in $X$ und $\nu$ ein Maß auf $\A$. Dann wird durch
\[
\mu(M) \da \inf\left\{\nu(A) : A\in\A,\, M\subset A\right\}
\]
für $M\subset X$ ein $\A$-reguläres äußeres Maß auf $X$ erklärt mit $\mu|_{\A} = \nu$. Ist $M\in\A_\mu$ und $\mu(M)<\infty$, so existiert ein $A\in\A$ mit $M\subset A$ und $\mu(A\setminus M) = 0$.
\end{satz}

\begin{beweis}
Für $M\subset X$ sieht man leicht, dass
\begin{align*}
\mu(M) &= \inf\left\{\sum_{i=1}^\infty \nu(A_i) : A_i \in \A,\, i\in \MdN,\, M\subset \bigcup_{i=1}^\infty A_i\right\}.
\end{align*}
Also ist $\mu$ das von $(\A,\nu)$ induzierte äußere Maß. Da $\nu$ monoton ist und nach der Definition von $\mu$ ist $\mu|_\A = \nu$.

Um die $\A$-Regularität zu zeigen, nehmen wir ein $A\in\A$ und ein $M\subset X$. Für $B\in\A$ mit $M\subset B$ gilt:
\begin{align*}
\mu(M\cap A) + \mu(M\cap A^c) 
&\le \nu(B\cap A) + \nu(B\cap A^c) \\
&= \nu(B)
\end{align*}
und daher
\[
\mu(M\cap A) + \mu(M\cap A^c) \le \mu(M).
\]
also ist $A\in\A_\mu$. Sei nun $M\subset X$ beliebig und ohne Beschränkung der Allgemeinheit $\mu(M)<\infty$. Es existiert also eine Folge $A_i\in\A$, $i\in\MdN$ mit $M\subset A_i$ und $\nu(A_i) \to \mu(M)$. Setze $A\da \bigcap_{i\ge 1} A_i$. Dann gilt $A \in \A$, $M\subset A$, sowie
\begin{align*}
\mu(M) = \lim_{i\to\infty} \nu(A_i) \ge \nu(\bigcap_{i=1}^\infty A_i) = \mu(\bigcap _{i=1}^\infty A_i) = \mu(A) \ge \mu(M),
\end{align*}
woraus $\mu(M) = \mu(A)$ folgt.

Sei schließlich $M\in\A_\mu$ mit $\mu(M)<\infty$. Es gibt ein $A\in\A$ mit $M\subset A$ und $\mu(M) = \mu(A)<\infty$. Dann folgt
\begin{align*}
\infty > \mu(A) &= \mu(A\cap M) +\mu(A\cap M^c) \\
&= \mu(M) + \mu(A\cap M^c) \\
&= \mu(A) + \mu(A\setminus M).
\end{align*}
Wegen $\mu(A)=\mu(M)<\infty$ gilt also $\mu(A\setminus M) =0$.
\end{beweis}


\begin{anwendung}
Sei $\vartheta$ ein beliebiges äußeres Maß auf $X$. Dann ist $\vartheta|_{\A_\vartheta}$ ein Maß. Durch
\[
\mu(M) \da \inf\{\vartheta(A) : A\in\A_\vartheta,\, M\subset A\}
\]
wird also ein $\A_\vartheta$-reguläres äußeres Maß auf $X$ erklärt, das $\vartheta$ fortsetzt (also $\mu|_{\A_\vartheta} = \vartheta|_{\A_\vartheta}$).
\end{anwendung}

\begin{definition}
Seien $X,Y$ Mengen, $\mu$ ein äußeres Maß auf $X$ und $f:X\to Y$. Dann wird durch
\[
(f\mu)(M) \da \mu(f^{-1}(M))
\]
für $M\subset Y$ ein äußeres Maß $f\mu$ auf $Y$ erklärt. Man nennt $f\mu$ das \emph{Bild} von $\mu$ unter $f$ oder auch \emph{„push forward“} von $\mu$ bezüglich $f$ und schreibt hierfür auch $f_\#\mu$.
\end{definition}

\begin{bemerkung}
Für $B\subset Y$ gilt
\[
f^{-1}(B) \in \A_\mu \iff \forall M\subset X: B \in \A_{f(\mu \MR M)}.
\]
Seien hierzu $M\subset X$, $A,B\subset Y$. Dann gilt
\begin{align*}
&\phantom{=\ \ }\mu(M\cap f^{-1}(A) \cap f^{-1}(B)) + \mu(M\cap f^{-1}(A) \cap f^{-1}(B)^c) \\
&= (\mu \MR M)(f^{-1}(A\cap B)) + (\mu \MR M)(f^{-1}(A\cap B^c)) \\
&= f(\mu \MR M)(A\cap B) + f(\mu \MR M) (A\cap B^c).
\end{align*}
Insbesondere gilt: Ist $f^{-1}(A) \in \A_\mu$, so ist $A\in\A_{f(\mu)}$.
\end{bemerkung}

\begin{sprechweisen}
Sei $\mu$ ein äußeres Maß auf $X$. Eine Menge $N\subset X$ heißt $\mu$-Nullmenge, falls $\mu(N)=0$. Eine Eigenschaft $\mathcal E$ gilt für $\mu$-fast-alle $x\in X$ bzw. $\mu$-fast-überall, falls 
\[
\mu(\{x\in X : \mathcal E\text{ gilt für $x$ nicht}\}) = 0.
\]
Mit $\mathbb F_\mu(X,Y)$ wird die Menge aller Abbildungen $f:D\to Y$ bezeichnet mit $D\subset X$ und $\mu(X\setminus D) = 0$.
\end{sprechweisen}

\begin{definition}
Seien $X,Y$ Mengen und $\mu$ ein äußeres Maß auf $X$ und $\mathcal C$ eine $\sigma$-Algebra in $Y$. Dann heißt $f\in\mathbb F_\mu(X,Y)$ $\mu$-messbar bezüglich $\mathcal C$, falls $f^{-1}(\mathcal C)\subset \A_\mu$.
\end{definition}

Beachte, dass für $f:D\to Y$ mit $\mu(X\setminus D)=0$ gilt: $D=f^{-1}(Y)\in \A_\mu$.

\begin{lemma}
Seien $X,Y$ Mengen, $\mu$ ein äußeres Maß auf $X$ und $\mathcal E\subset \mathcal P(Y)$.  Für $f\in\mathbb F_\mu(X,Y)$ sind äquivalent:
\begin{enumerate}[a)]
\item $f^{-1}(\mathcal E) \subset \A_\mu$
\item $f$ ist $\mu$-messbar bezüglich $\sigma(\mathcal E)$.
\end{enumerate}
\end{lemma}

\begin{definition}
Ist $(X,\mathcal T)$ ein topologischer Raum, so nennt man die von den offenen Mengen erzeugte $\sigma$-Algebra $\sigma(\mathcal T)$ die Borelsche $\sigma$-Algebra des topologischen Raumes $(X,\mathcal T)$ mit der Notation $\mathfrak B(X)$.

Spezielle Borelsche Algebren sind $\mathfrak B(\MdR)$, $\mathfrak B(\MdR^n)$, $\mathfrak B(\bar\MdR) \da \{B\in \bar\MdR : B\cap \MdR \in \mathfrak B(\MdR)\}$.
\end{definition}

\begin{definition}
Sei $X$ eine Menge, $\mu$ ein äußeres Maß auf $X$ und $f\in\mathbb F_\mu(X,\bar\MdR)$. Man nennt $f$ eine $\mu$-messbare Abbildung, falls $f$ dies bezüglich $\mathfrak B(\bar\MdR)$ ist.
\end{definition}

Im Folgenden schreiben wir für eine Relation $\varrho$ auf $\bar\MdR$, Mengen $D, D'\subset X$ und Abbildungen $f: D\to \bar\MdR$, $g:D'\to\bar\MdR$:
\[
\{ f\mathrel{\varrho} g\} \da \{ x\in D\cap D' : f(x)\mathrel{\varrho} g(x) \}
\]

\begin{lemma}
Sei $\mu$ ein äußeres Maß auf $X$ und $f\in \mathbb F_\mu(X,\bar\MdR)$. Genau dann ist $f$ eine $\mu$-messbare Abbildung, wenn eine der folgenden Bedingungen für alle $a\in\MdR$ erfüllt ist:
\begin{align*}
 \{f > a\} \in \A_\mu, && \{f\ge a\} \in \A_\mu, && \{f<a\} \in \A_\mu, && \{f\le a\}\in\A_\mu.
\end{align*}
\end{lemma}

\begin{lemma}
Sei $\mu$ ein äußeres Maß auf $X$, seien $f,g,f_n\in \mathbb F_\mu(X,\bar\MdR)$, $n\in\MdN$, $\mu$-messbar. Dann gilt
\begin{enumerate}[(a)]
\item $\{f<g\}$, $\{f\le g\}$, $\{f=g\}$, $\{f\ne g\}$ sind $\mu$-messbare Mengen.
\item Die Funktionen
\begin{align*}
&f+ g, && f-g, && f \cdot g \text{ (falls $\mu$-fast-überall definiert)}, \\
&\sup_n f_n, && \inf_n f_n, && \\
&f^+ \da \max\{f,0\}, && f^- \da -\min\{f,0\}, && |f|, \\
&\limsup_n f_n, && \liminf_n f_n
\end{align*}
sind $\mu$-messbar.
\end{enumerate}
\end{lemma}

\begin{satz}
Ist $\mu$ ein äußeres Maß auf $X$, so ist $f\in \mathbb F_\mu(X,\bar\MdR)$ genau dann $\mu$-messbar, wenn für alle $M\subset X$, $a,b\in\MdR$ mit $a<b$ gilt
\[
\mu(M) \ge \mu(M\cap \{f \le a\}) + \mu(M\cap \{f\ge b\}).
\]
\end{satz}

\begin{beweis}
Sei $f$ zunächst $\mu$-messbar. Dann gilt mit $a<b$, $M\subset X$: 
\begin{align*}
\mu(M) &\ge \mu(M\cap\{f\le a\} ) + \mu(M\cap \{f> a\}) \\
&\ge \mu(M\cap\{f\le a\} ) + \mu(M\cap \{f\ge b\}) 
\end{align*}

Jetzt gelte die Bedingung des Satzes für alle $M\subset X$, $a<b$. Zu zeigen ist: $\{f\le r\}\in \A_\mu$ für beliebige $r\in\MdR$. Sei $M\subset X$ beliebig mit $\mu(M) <\infty$. Für $i\in \MdN$ sei
\[
A_i \da M\cap \{r + \frac1{i+1} \le f \le r+\frac 1i\}.
\]
Wir zeigen mit vollständiger Induktion, dass 
\[
\mu(\bigcup_{i=0}^n A_{2i+1})  \ge \sum_{0=1}^n \mu(A_{2i+1})
\]
gilt.

Für $n=0$ ist dies klar. Die Ungleichung gelte für ein $n\in\MdN$.
\begin{align*}
\mu(\bigcup_{i=0}^{n+1} A_{2i+1}) 
&\ge \mu(\bigcup_{i=0}^{n+1} A_{2i+1} \cap \{f\ge \underbrace{r+ \frac1{2n+2}}_{b}\}) + 
     \mu(\bigcup_{i=0}^{n+1} A_{2i+1} \cap \{f\le \underbrace{r+\frac1{2n+3}}_{a}\}) \\
&= \mu(\bigcup_{i=0}^n A_{2i+1}) + \mu(A_{2n+3}) \\
&\ge \sum_{i=0}^n \mu(A_{2i+1}) + \mu(A_{2n+3}) \\
&\ge \sum_{i=0}^{n+1} \mu(A_{2i+1})
\end{align*}

Analog zeigt man 
\[
\mu(\bigcup_{i=1}^n A_{2i})  \ge \sum_{i=1}^n \mu(A_{2i})
\]
und erhält zusammen
\[
\sum_{i=1}^\infty \mu(A_i) \le 2 \mu(M) <\infty.
\]

Sei $\ep>0$. Dann gibt es ein $n\in\MdN$ mit $\sum_{i\ge n}\mu(A_i) <\ep$. Zunächst schätzen wir ab
\begin{align*}
\mu(M\cap \{r < f < r+\frac1n\}) 
&\le \mu(M\cap \{r<f\le r + \frac1n\}) \\
&= \mu(M\cap \bigcup_{i=n}^\infty \{r +\frac 1{i+1} \le f \le r+\frac 1i\}) \\
&= \mu(\bigcup_{i=n}^\infty A_i) \\
&\le \sum_{i\ge n} \mu(A_i) < \ep
\end{align*}
und damit
\begin{align*}
&\phantom{=\ \ }\mu(M\cap \{f\le r\}) + \mu(M\cap \{f>r\})  \\
&\le \mu(M\cap \{f\le r\}) + \mu(M\cap \{r<f<r+\frac1n\}) + \mu(M\cap \{f\ge r+\frac 1n\}) \\
&\le \mu(M) + \ep.
\end{align*}
\end{beweis}

\begin{satz}
Seien $\mu$ ein äußeres Maß auf $X$, $f: X \to [0,\infty]$ eine $\mu$-messbare Abbildung und $(r_n)_{n\in\MdN}$ eine Folge in $[0,\infty)$ mit $\lim_{n\to\infty} r_n = 0$ und $\sum_{n=1}^\infty r_n = \infty$. Dann gibt es eine Folge $(A_n)_{n\in\MdN}$ $\mu$-messbarer Mengen mit
\[
f = \sum_{n\ge1} r_n \ind_{A_n}.
\]
\end{satz}

\begin{beweis}
Setze $A_1 \da \{ f \ge r_1 \}$ und allgemein $A_n \da \{f \ge r_n + \sum_{j=1}^{n-1} r_j \ind_{A_j}\}$, $n\ge 1$.

\textbf{Behauptung:} Es ist $f \ge \sum_{i=1}^n r_i \ind_{A_i}$, $n\in\MdN$. Dies gilt für $n=1$, und wenn es für ein $n\in\MdN$ gilt, dann folgt: Ist $x\notin A_{n+1}$, dann ist
\begin{align*}
f(x) \ge \sum_{i=1}^n r_i \ind_{A_i}(x) + \underbrace{r_{n+1}\ind_{A_{n+1}}(x)}_{=0}
\end{align*}
nach Induktionsvoraussetzung. Ist dagegen $x\in A_{n+1}$, so gilt nach der Definition von $A_{n+1}$
\begin{align*}
f(x) \ge \sum_{i=1}^n r_i \ind_{A_i}(x) + r_j = \sum_{i=1}^n r_i \ind_{A_i}(x) + r_j \underbrace{\ind_{A_{n+1}}(x)}_{=1}.
\end{align*}

Folglich ist $f\ge \sum_{i=1}^\infty r_i \ind_{A_i}$.

\textbf{Annahme:} Es gelte $f(x) > \sum_{i=1}^\infty r_i \ind_{A_i}(x)$ für ein $x\in X$.

Also ist $\sum_{i=1}^\infty r_i \ind_{A_i}(x)<\infty$. Da $\sum_{i=1}^\infty r_i = \infty$ gilt, muss es eine Folge natürlicher Zahlen $(j_k)_{k\in\MdN}$ geben mit $\ind_{A_{j_k}}(x)=0$ für alle $k\in\MdN$.
Wegen $\lim_{k\to\infty} r_{j_k} = 0$ gibt es ein $k\in\MdN$ mit
\begin{align*}
r_{j_k} < f(x) - \sum_{j=1}^\infty r_j \ind_{A_j}(x)
\end{align*}
und damit 
\begin{align*}
f(x) &> \sum_{j=1}^\infty r_j \ind_{A_j}(x) + r_{j_k} \\
&\ge \sum_{j=1}^{j_k-1} r_j \ind_{A_j}(x) + r_{j_k}.
\end{align*}
Das bedeutet $x\in A_{j_k}$ im Widerspruch zu $\ind_{A_{j_k}}(x) \ne 0$.
\end{beweis}

\section{Integration}

In diesem Abschnitt wird generell vorausgesetzt, dass $X$ eine Menge und $\mu$ ein äußeres Maß auf $X$ ist.

\begin{definition}
Eine $\mu$-Treppenfunktion auf $X$ ist eine $\mu$-messbare Abbildung $h\in \mathbb F_\mu(X,\MdR)$ mit abzählbarer Wertemenge $\im(h)$ und
\[
\sum_{\substack{r\in \im(h)\\r< 0}} r \cdot \mu(\{h=r\}) > -\infty \quad\text{ oder }\quad
\sum_{\substack{r\in \im(h)\\r> 0}} r \cdot \mu(\{h=r\}) < \infty.
\]
Ist $h$ eine $\mu$-Treppenfunktion auf $X$, so wird durch
\[
\int hd\mu = \sum_{r\in \im(h)} r \cdot \mu(\{h=r\})
\]
das $\mu$-Integral von $h$ erklärt, wobei „$0\cdot \infty\da 0$“ gelte.
\end{definition}

\begin{bemerkung}
\begin{enumerate}
\item  Es gilt
\[
\int h d\mu = \int h^+ d\mu - \int h^- d\mu.
\]
\item $h=\ind_A$, $A\in\A_\mu$ ist eine $\mu$-Treppenfunktion, $\int \ind_A d\mu=\mu(A)$.
\end{enumerate}
\end{bemerkung}

\begin{lemma}
\label{lem1.10}
Seien $h,g$ $\mu$-Treppenfunktionen auf $X$. Es gelte $\int h^+ d\mu <\infty$ und $\int g^+ d\mu<\infty$  oder $\int h^- d\mu <\infty$ und $\int g^- d\mu<\infty$. Dann ist $h+g$ eine $\mu$-Treppenfunktion und es gilt
\[
\int (h+g) d\mu = \int h d\mu + \int g d\mu.
\]
\end{lemma}

\begin{beweis}
Es gilt zunächst $h+g \in \mathbb F_\mu(X,\MdR)$. Zur Additivität: Wir definieren $A(r,s) \da \{h=r\} \cap \{g=s\}$ für $r,s\in\MdR$. Die Voraussetzungen des Lemmas stellen sicher, 
dass die nachfolgend vorgenommenen Vertauschungen der Summationsreihenfolge zul{\"a}ssig sind. Es gilt damit
\begin{align*}
\int hd\mu + \int gd\mu 
&= \sum_{r\in\im(h)} r\cdot \mu(\{h=r\}) + \sum_{s\in \im(g)} s \cdot \mu(\{g=s\}) \\
&= \sum_{r\in\im(h)} r \cdot \sum_{s\in \im(g)} \mu(A(r,s)) + \sum_{s\in\im(g)} s \cdot \sum_{r\in \im(h)} \mu(A(r,s)) \\
&= \sum_{\substack{r\in\im(h)\\s\in\im(g)}} (r+s)\cdot \mu(A(r,s)) \\
&= \sum_{t\in \im(g+h)} t \cdot \sum_{\substack{r\in \im(h) \\s\in\im(g)\\r+s=t}} \mu(A(r,s)) \\
&= \sum_{t\in \im(g+h)} t \cdot \mu\big(\bigcup_{\substack{r\in \im(h) \\s\in\im(g)\\r+s=t}} A(r,s)\big) \\
&= \sum_{t\in \im(g+h)} t \cdot \mu(\{g+h=t\}) \\
&= \int (h+g) d\mu .
\end{align*}
Ãœbung: Zeige, dass $\int (g+h)^+ d\mu<\infty$ oder $\int (g+h)^- d\mu<\infty$ gilt.
\end{beweis}

\begin{bemerkung}
Sei $h$ eine $\mu$-Treppenfunktion mit $h\ge 0$, dann gilt $\int hd\mu \ge 0$. Mit Lemma \ref{lem1.10} folgt für $\mu$-Treppenfunktionen $h,g$:
\[
h \le g \implies \int h d\mu \le \int g d\mu
\]
\end{bemerkung}

\begin{definition}
Sei $f\in\mathbb F_\mu(X,\bar\MdR)$. Eine $\mu$-Oberfunktion (bzw. $\mu$-Unterfunktion) von $f$ ist eine $\mu$-Treppenfunktion $h$ auf $X$ mit $f\le h$ $\mu$-fast-überall auf $X$ (bzw. $h\le f$ $\mu$-fast-überall auf $X$).

Durch
\[
\int^* fd\mu \da \inf\left\{\int hd\mu : \text{$h$ ist eine $\mu$-Oberfunktion von $f$}\right\}
\]
wird das $\mu$-Oberintegral von $f$ erklärt.
Analog wird durch
\[
\int_* fd\mu \da \sup\left\{\int hd\mu : \text{$h$ ist eine $\mu$-Unterfunktion von $f$}\right\}
\]
das $\mu$-Unterintegral von $f$ erklärt.
\end{definition}

\begin{lemma}
\label{lem1.11}
Für $f,g\in\mathbb F_\mu(X,\bar\MdR)$ gelten die folgenden Aussagen:
\begin{enumerate}
\item $\int_* fd\mu = - \int^* (-f)d\mu$.
\item Gilt $\mu$-fast-überall $f\le g$, so ist $\int^* fd\mu \le \int^* gd\mu$ und $\int_* fd\mu \le \int_* gd\mu$.
\item Gilt $\mu$-fast-überall $f\ge 0$, so ist $\int^* fd\mu \ge 0$ und $\int_* fd\mu \ge 0$.
\item Gilt $\int^* fd\mu <\infty$, so auch $\int^*f^+d\mu<\infty$ und $f<\infty$ $\mu$-fast-überall.
\item Für $c\in(0,\infty)$ gilt $\int^*(cf)d\mu = c\cdot \int^*fd\mu$ und $\int_*(cf)d\mu = c\cdot \int_*fd\mu$.
\item Ist $\int^* fd\mu <\infty$ und $\int^*g d\mu<\infty$, so ist $f+g \in \mathbb F_\mu(X,\bar\MdR)$ und $$\int^*(f+g)d\mu \le \int^*fd\mu + \int^*gd\mu.$$

Analog gilt: Ist $\int_* fd\mu >-\infty$ und $\int_*g d\mu>-\infty$, so ist $f+g \in \mathbb F_\mu(X,\bar\MdR)$ und $$\int_*(f+g)d\mu \ge \int_*fd\mu + \int_*gd\mu.$$
\item $\int_* fd\mu \le \int^* fd\mu$.
\end{enumerate}
\end{lemma}

\begin{beweis} \"Ubung
\end{beweis}

\begin{bemerkung}
Ist $h$ eine $\mu$-Treppenfunktion, so ist $h$ eine $\mu$-Oberfunktion und eine $\mu$-Unterfunktion von $h$. Das heißt insbesondere
\[
\int hd\mu \le \int_* h d\mu \le \int^* h d\mu \le \int h d\mu.
\]
\end{bemerkung}

\begin{definition}
Ist $f\in \mathbb F_\mu(X,\bar\MdR)$ eine $\mu$-messbare Abbildung und stimmt das $\mu$-Oberintegral mit dem $\mu$-Unterintegral von $f$ überein, so wird durch
\[
\int fd\mu \da \int^* fd\mu = \int_* fd\mu
\]
das $\mu$-Integral von $f$ erkärt. Man sagt in diesem Fall, dass das $\mu$-Integralvon $f$ existiert. Ist $\int fd\mu \in \MdR$, so heißt $f$ $\mu$-integrierbar.
\end{definition}

\begin{satz}
\label{satz1.12}
Sei $f\in \mathbb F_\mu(X,\bar\MdR)$ nicht-negativ und $\mu$-messbar. Dann existiert das $\mu$-Integral von $f$. Es gilt $\int fd\mu \ge 0$ und 
\[
\int fd\mu = \sup\left\{ \int hd\mu : \text{$h$ ist $\mu$-Unterfunktion, $\im(h)$ ist endlich}\right\}.
\]
\end{satz}

\begin{beweis}
Ist $\mu(\{f=\infty\})>0$, dann ist für jedes $n\in\MdN$ die Funktion $n\cdot \ind_{\{f=\infty\}}$ eine $\mu$-Unterfunktion von $f$ und 
\[
\int^*fd\mu \ge \int_* fd\mu \ge \int n \cdot\ind_{\{f=\infty\}} d\mu = n \cdot \mu(\{f=\infty\})\to \infty\text{ für }n\to\infty.
\]
Also ist $\int^* fd\mu = \int_* fd\mu = \infty$.

Sei jetzt $f<\infty$ $\mu$-fast-überall. Für $t\in(1,\infty)$ sei
\[
U_t \da \sum_{n\in\MdZ} t^n \cdot \ind_{\{t^n \le f < t^{n+1}\}}. 
\]
Offenbar ist
$$
U_t\le f\le t U_t
$$
$\mu$-fast-\"uberall, d.h.\ 
$U_t$ ist eine $\mu$-Unterfunktion von $f$, $tU_t$ eine $\mu$-Oberfunktion von $f$. Damit gilt
\begin{align*}
\int^* fd\mu \le \int t U_t d\mu = t \cdot \int U_t d\mu \le t \int_* f d \mu.
\end{align*}
Ist $\int_* fd\mu <\infty$, dann folgt $\int^* fd\mu \le \int_* fd\mu$ aus $t\to 1$. Ist dagegen $\int_* fd\mu = \infty$, so ist $\int^* fd\mu = \int_* fd\mu = \infty$.
\end{beweis}



\begin{satz}
\label{satz1.13}
Seien $f,g\in\mathbb F_\mu(X,\bar\MdR)$ $\mu$-messbar. Dann gilt:
\begin{enumerate}
\item Sei $c\in \MdR\setminus\{0\}$. Es existiert $\int fd\mu$ genau dann, wenn $\int(cf)d\mu$ existiert. In diesem Fall ist 
\[
\int (cf) d\mu = c \cdot \int fd\mu.
\]
\item Angenommen, es existieren $\int fd \mu$ und $\int gf\mu$ und $(\int fd\mu, \int gd\mu) \ne (\pm \infty, \mp \infty)$. Dann ist $f+g\in \mathbb F_\mu(X,\bar\MdR)$, $\int(f+g)d\mu$ existiert und
\[
\int (f+g)d\mu = \int fd\mu + \int gd\mu.
\]
\item Ist $f\le g$ $\mu$-messbar und existiert $\int gd\mu$ (bzw. $\int fd\mu)$, so existiert auch $\int fd\mu$ (bzw. $\int gd\mu$), und es gilt in jedem Fall
\[
\int fd\mu \le \int gd\mu.
\]
\end{enumerate}
\end{satz}

\begin{beweis}
\begin{enumerate}
\item Es existiere $\int fd\mu$. Sei $c>0$. Dann folgt
\begin{align*}
\int^* (cf)d\mu 
&= c\cdot \int^*fd\mu = c\cdot \int_* fd\mu = \int_* (cf)d\mu.
\end{align*}
Sei $c<0$. Dann folgt
\begin{multline*}
\int^* (cf)d\mu 
= \int^*(-c)(-f)d\mu
= (-c)\cdot \int^*(-f)d\mu
= (-c) \cdot (-1) \int_* fd\mu 
\\
= c\cdot \int fd\mu 
= c\int^* fd\mu 
= (-c)\int_*(-f) d\mu
= \int_*(-c)(-f) d\mu
= \int_* (cf) d\mu .
\end{multline*}
\item Seien $f,g$ $\mu$-integrierbar. Dann ist $f,g<\infty$ $\mu$-fast-überall, und somit ist $f+g\in \mathbb F_\mu(X,\bar\MdR)$. Ferner gilt
\begin{multline*}
\int fd\mu + \int gd\mu 
= \int^* fd\mu + \int^* gd\mu
\ge \int^*(f+g)d\mu \\
\ge \int_*(f+g)d\mu
\ge \int_* fd\mu + \int_* gd\mu
= \int fd\mu + \int gd\mu,
\end{multline*}
woraus die Aussage folgt.

Sei nun $\int fd\mu = \infty$. Nach Voraussetzung gilt dann $\int gd\mu >-\infty$ und damit $g>-\infty$ $\mu$-fast-überall. Das heißt $f+g\in\mathbb F_\mu(X,\bar\MdR)$. Ferner gilt
\begin{align*}
\infty \ge \int^*(f+g)d\mu \ge \int_*(f+g)d\mu \ge \int_*f d\mu + \int_* gd\mu = \infty.
\end{align*}
Analog kann der Fall $\int fd\mu=-\infty$ gezeigt werden.
\item Sei $\int gd\mu <\infty$, das heißt $f\le g <\infty$ $\mu$-fast-überall. Dann ist 
$(f-g)\ind_{\{g>-\infty\}}\in F_\mu(X,\bar\MdR)$ nicht positiv. Wegen $f\le g <\infty$ $\mu$-fast-überall 
ist $g+(f-g)\ind_{\{g>-\infty\}}=f$ und damit ergibt (2)
\begin{align*}
\int gd\mu \ge \int gd\mu + \int (f-g)\ind_{\{g>-\infty\}}d\mu = \int fd\mu.
\end{align*}
% Was ist mit \int_ fd\mu = -\infty?
\end{enumerate}
\end{beweis}

\begin{satz}
Sei $f\in\mathbb F_\mu(X,\bar\MdR)$ $\mu$-messbar. Dann gilt:
\begin{enumerate}
\item $\int f^+ d\mu$, $\int f^- d\mu$ existieren stets. Es existiert $\int fd\mu$ genau dann, wenn $\int f^+d \mu <\infty$ oder $\int f^- d\mu<\infty$. In diesem Fall ist 
\[
\int fd\mu = \int f^+d\mu -\int f^-d\mu.
\]
Ferner ist
\[
\Big|\int fd\mu\Big| \le \int |f|d\mu .
\]
\item Ist $f$ $\mu$-integrierbar, so auch $|f|$. 
\end{enumerate}
\end{satz}

\begin{beweis}
\begin{enumerate}
\item Es existiere $\int fd\mu$. Ist $\int fd\mu <\infty$, so ist $\int f^+ d\mu <\infty$ nach Lemma \ref{lem1.11} (4). Ist dagegen $\int fd\mu =\infty$, so ist $\int (-f)d\mu = -\infty$ und daher $\int f^- d\mu = \int (-f)^+d\mu <\infty$.

Umgekehrt sei $\int f^+d\mu <\infty$ oder $\int f^- d\mu <\infty$. Dann existiert nach Satz \ref{satz1.13} (2) das Integral $\int (f^+-f^-)d\mu$ wegen Satz \ref{satz1.13} (1) ist $\int fd\mu = \int f^+d\mu - \int f^-d\mu$.

Stets existiert das Integral von $|f|$ und 
\begin{align*}
\int |f|d\mu = \int (f^+ + f^-) d\mu = \int f^+ d\mu + \int f^- d\mu 
\ge \Big|\int f^+ d\mu - \int f^-d\mu \Big| = \Big|\int fd\mu \Big |.
\end{align*}
\item Ist $f$ $\mu$-integrierbar, so folgt aus (1), dass $\int f^+d \mu <\infty$ \emph{und} $\int f^-d \mu<\infty$.
\end{enumerate}
\end{beweis}

\begin{satz}[Lemma von Fatou]
\label{satz1.15}
Sei $f_n \in \mathbb F_n(X,\bar\MdR)$, $n\in\MdN$, mit $f_n\ge 0$ und $\mu$-messbar. Dann gilt
\begin{align*}
\int \liminf_{n\to\infty} f_n d\mu \le \liminf_{n\to\infty} \int f_n d\mu.
\end{align*}
\end{satz}

\begin{beweis}
Sei $\varepsilon \in (0,1)$. Sei $h$ eine $\mu$-Unterfunktion von $\liminf_{n\to\infty} f_n$ mit $\im(h) = \{r_1,\ldots,r_m\}\subset [0,\infty)$ (vergleiche Satz \ref{satz1.12}). Für $i=1,\ldots,m$ und $n\in\MdN$ sei 
\begin{align*}
A_{i,n}\da \{h=r_i\} \cap \{\inf_{k\ge n} f_k \ge \ep\cdot r_i\} \in A_\mu.
\end{align*}
Es gilt $A_{i,n}\subset A_{i,n+1}$ für $i=1,\ldots,m$ und $n\in\mathbb{N}$. Für $\mu$-fast-alle $x\in X$ mit $h(x) = r_i$ gilt:
\begin{align*}
\ep r_i < r_i \le \liminf_{n\to\infty} f_n(x) = \sup_{n\in\MdN} \inf_{k\ge n} f_k(x).
\end{align*}
Es gibt ein $n\in\MdN$ mit $\ep r_i < \inf_{k\ge n} f_k(x),$ das heißt $x\in A_{i,n}$. Also
\begin{align*}
\mu(\{h=r_i\}) = \mu(\bigcup_{n\in\MdN} A_{i,n}).
\end{align*}

Die Mengen $A_{i,n}$, $i=1,\ldots,m$, sind paarweise disjunkt und aus ihrer Definition folgt, dass
\begin{align*}
\sum_{i=1}^m \ep r_i \ind_{A_{i,n}}
\end{align*}
eine $\mu$-Unterfunktion von $f_n$ ist.

Hiermit gilt
\begin{align*}
\liminf_{n\to\infty} \int f_n d\mu 
&\ge \liminf_{n\to\infty} \sum_{i=1}^m \ep r_i \mu(A_{i,n})\\
&= \sum_{i=1}^m \ep r_i \liminf_{n\to\infty} \mu(A_{i,n})\\
&= \sum_{i=1}^m \ep r_i \mu( \bigcup_{n\in\MdN} A_{i,n})\\
&= \ep \sum_{i=1}^m r_i \mu(\{h=r_i\}) \\
&= \ep \int hd\mu.
\end{align*}
Es folgt
\begin{align*}
\int \liminf_{n\to\infty} f_n d\mu \le \frac1\ep \liminf_{n\to\infty} \int f_n d\mu
\end{align*}
für ein beliebiges $\ep \in (0,1)$. Lässt man $\ep$ gegen $1$ gehen, so folgt die Behauptung.
\end{beweis}

\begin{satz}
\label{satz1.16}
Ist $f_n\in\mathbb F_\mu(X,\bar\MdR)$ $\mu$-messbar, $n\in\MdN$, $0\le f_1\le f_2\le \cdots$, so gilt
\begin{align*}
\lim_{n\to\infty} \int f_n d\mu = \int \lim_{n\to\infty} f_n  d\mu.
\end{align*}
\end{satz}

\begin{beweis}
Grenzwerte und Integrale existieren offenbar. Es gilt 
\begin{align*}
\lim_{k\to\infty}\int f_k d\mu 
&\le \int \lim_{n\to\infty} f_n d\mu \\
&\le \limsup_{n\to\infty} \int f_n d\mu. \tag{nach Satz \ref{satz1.15}}
\end{align*}
\end{beweis}

\begin{satz}[Lebesgue]
\label{satz1.17}
Sei $f_n\in\mathbb F_\mu(X,\bar\MdR)$, $n\in\MdN$ eine konvergente Folge $\mu$-messbarer Funktionen. Es existiere eine $\mu$-integrierbare Funktion $g\in\mathbb F_\mu(X,\bar\MdR)$ mit $|f_n|\le g$ $\mu$-fast-überall für jedes $n\in\MdN$. Dann sind $f_n$ und $f\da \lim_{n\to\infty} f_n$ $\mu$-integrierbar und 
\begin{align*}
\lim_{n\to\infty} \int f_n d\mu = \int f d\mu.
\end{align*}
Schärfer gilt sogar
\begin{align*}
\lim_{n\to\infty} \int |f-f_n| d\mu = 0.
\end{align*}
\end{satz}

\begin{beweis}
Wegen $|f_n|,|f| \le g$ ist $\int |f_n| d\mu <\infty$ und $\int |f|d\mu <\infty$. Die Folge $(2g-|f_n-f|)_{n\in\MdN}$ nichtnegativer Funktionen in $\mathbb F_\mu(X,\bar\MdR)$ konvergiert für $n\to\infty$ $\mu$-fast-überall gegen $2g$. Mit dem Lemma von Fatou (Satz \ref{satz1.15}) folgt:
\begin{multline*}
\int 2g d\mu - \limsup_{n\to\infty} \int |f_n -f|d\mu
= \liminf_{n\to\infty} \int (2g - |f_n -f|) d\mu \\
\ge \int \underbrace{\liminf_{n\to\infty}(2g - |f_n -f|)}_{=2g} d\mu = \int 2gd\mu
\end{multline*}
also 
\begin{align*}
\lim_{n\to\infty} \int |f_n -f| d\mu = 0.
\end{align*}
\end{beweis}

\begin{notation}
Für $A\subset X$ und $f\in\mathbb F_\mu (X,\bar\MdR)$ sei
\begin{align*}
\int^*_A fd\mu \da \int^* \ind_{A}f d\mu && \text{und} &&
\int_{*A} fd\mu \da \int_* \ind_{A}f d\mu.
\end{align*}
Existiert das $\mu$-Integral von $\ind_A \cdot f$, so setzt man
\begin{align*}
\int_A f d\mu \da \int \ind_A fd\mu.
\end{align*}
\end{notation}

\begin{lemma}
\label{lem1.18}
(Ãœbungsblatt 2, Aufgabe 4) Sei $f_n\in\mathbb F_\mu(X,\bar\MdR)$, $n\in\MdN$, eine Folge nichtnegativer Funktionen. Dann gilt
\begin{align*}
\int^* \sum_{n=1}^\infty {f_n} d\mu \le \sum_{n=1}^\infty \int^* f_n d\mu.
\end{align*}
\end{lemma}

\begin{satz}
\label{satz1.19}
Sei $g\in \mathbb F_\mu(X,\bar\MdR)$ eine nichtnegative Funktion. Dann wird  durch
\begin{align*}
\psi(A) \da \int_A^* gd\mu, \  A\subset X,
\end{align*}
 ein äußeres Maß auf $X$ definiert. Es gilt $\A_\mu \subseteq \A_\psi$. Ist $g$ sogar $\mu$-messbar und $f\in\mathbb F_\mu(X,\bar\MdR)$ $\mu$-messbar, dann existiert $\int fd\psi$ genau dann, wenn $\int fgd\mu$ existiert. In diesem Fall gilt
\begin{align*}
\int fd\psi = \int fg d\mu.
\end{align*}
\end{satz}

\begin{beweis}
Es gilt $\psi(\emptyset)=0$. Die Subsigmaadditivität folgt direkt aus Lemma \ref{lem1.18}.

Sei $A\in\A_\mu$ und $M\subset X$ mit $\psi(M)<\infty$. Sei $\ep>0$ beliebig. Dann existiert eine $\mu$-Oberfunktion $h$ zu $\ind_M\cdot g$ mit $h\ge 0$ und
\[
\int h d\mu \le \int^* \ind_M g d \mu + \ep.
\]
Dann ist $\ind_A\cdot h$ eine $\mu$-Oberfunktion zu $\ind_{M\cap A}\cdot g$ und $\ind_{A^c}\cdot h$ ist eine $\mu$-Oberfunktion zu $\ind_{M\cap A^c}\cdot g$. Daher folgt
\begin{align*}
\psi(M\cap A)+\psi(M\cap A^c)
&= \int^* \ind_{M\cap A} gd\mu + \int^* \ind_{M\cap A^c} gd\mu \\
&\le \int \ind_A \cdot h d\mu + \int \ind_{A^c} \cdot h d\mu \\
&= \int h d\mu \le \int^* \ind_M g d\mu + \ep = \psi(M) + \ep.
\end{align*}

Sei nun $g$ $\mu$-messbar und $f\in \mathbb F_\mu(X,\bar\MdR)$ und sei zunächst $f\ge 0$. Also existieren $\int fd\psi$ und $\int gfd\mu$. Es gibt eine Folge $(r_j)_{j\in\MdN}$ in $(0,\infty)$ und $(A_j)_{j\in\MdN}$ in $A_\mu$ mit $f = \sum_{j=1}^\infty r_j\ind_{A_j}$. Zweimalige Anwendung des 
Satzes von der monotonen Konvergenz (Satz \ref{satz1.16}) ergibt 
\begin{align*}
\int fd\psi 
&= \int \sum_{j=1}^\infty r_j \ind_{A_j} d\psi \\
&= \sum_{j=1}^\infty r_j \int \ind_{A_j} d\psi \\
&= \sum_{j=1}^\infty r_j \psi(A_j) \\
&= \sum_{j=1}^\infty r_j \int \ind_{A_j} gd\mu \\
&= \int \left( \sum_{j=1}^\infty r_j \ind_{A_j}\right) g d\mu \\
&= \int fgd\mu.
\end{align*}
Sei nun $f$ eine beliebige $\mu$-messbare Funktion. Wegen $(fg)^\pm = f^\pm \cdot g$ gilt $\int f^\pm d\psi < \infty$ genau dann, wenn $\int (f g)^{\pm} d\mu<\infty$. Somit existiert $\int f d\psi$ genau dann, wenn $\int fg d\mu$ existiert und
\begin{multline*}
\int fd\psi = \int f^+ d\psi - \int f^- d\psi = \int f^+gd\mu - \int f^-gd\mu = \int (f^+ - f^-)g d\mu = \int fgd\mu.
\end{multline*}
\end{beweis}

\begin{satz}
Sei $f\in \mathbb F_\mu(X,\bar\MdR)$ $\mu$-integrierbar. Dann gibt es zu jedem $\ep >0$ ein $\delta > 0$ derart, dass für alle $A\in \A_\mu$ mit $\mu(A)<\delta$ gilt
\[
\int_A |f|d\mu <\ep.
\]
\end{satz}

\begin{beweis}
Betrachte $g_n\da \min\{|f|,n\}$, $n\in\MdN$. Es gilt $0\le g_n\nearrow |f|$ für $n\to\infty$. Damit ist
\begin{align*}
\lim_{n\to \infty} \int g_n d\mu = \int |f| d\mu <\infty.
\end{align*}
Zu einem vorgegebenem $\ep>0$ gibt es ein $N\in\mathbb{N}$ mit 
\begin{align*}
0\le \int |f|d\mu - \int g_N d\mu < \frac\ep2.
\end{align*}
Für $A\in\A_\mu$ mit $\mu(A) < \frac\ep{2N} \ad \delta$ folgt nun
\begin{align*}
\int_A |f| d\mu
&= \int_A\underbrace{(|f|-g_N)}_{\ge 0} d\mu + \int _A g_N d\mu \\
&\le \int\underbrace{(|f|-g_N)}_{\ge 0} d\mu + N \cdot \mu(A) \\
&< \frac\ep2 + \frac\ep2 = \ep.
\end{align*}
\end{beweis}

Seien $X,Y\neq\emptyset$ Mengen mit äußeren Maßen $\mu$ auf $X$, $\nu$ auf $Y$. Durch
\[
(\mu \times \nu)(M) \da \inf\{ \sum_{i=1}^\infty \mu(A_i)\nu(B_i): A_i\in \A_\mu, B_i\in \A_\nu,\, i\in\MdN,\, M\subset \bigcup_{i=1}^\infty (A_i \times B_i)\}
\]
wird ein äußeres Maß auf $X\times Y$ erklärt, nämlich das von $\mathcal E_0 \da \{A \times B : A\in\A_\mu,\, B\in\A_\nu\}$ und $\lambda$ mit $\lambda(A\times B) \da \mu(A)\cdot \nu(B)$ für $A\in\A_\mu$, $B\in\A_\nu$ induzierte äußere Maß.

\begin{satz}
Seien $X,Y\ne \emptyset$ Mengen mit äußeren Maßen $\mu$ auf $X$ und $\nu$ auf $Y$. Dann gelten folgende Aussagen:
\begin{enumerate}
\item $\A_\mu \otimes \A_\nu \subset \A_{\mu\times \nu}$ und $(\mu\times \nu)(A\times B) = \mu(A) \cdot \nu(B)$ für $A\in\A_\mu$, $B\in\A_\nu$.
\item Das Maß $\mu\times\nu$ ist $\A_\mu \otimes \A_\nu$-regulär.
\item Existiert das $(\mu \times \nu)$-Integral von $f\in\mathbb F_{\mu\times\nu}(X\times Y,\bar\MdR)$ und gilt $\{f\ne 0\}\subset \bigcup_{i=1}^\infty M_i$, $M_i\in\A_{\mu\times\nu}$, $(\mu\times\nu)(M_i)<\infty$, $i\in\MdN$, so ist:

$f(\cdot,y) \in\mathbb F_\mu(X,\bar\MdR)$ ist $\mu$-messbar für $\nu$-fast-alle $y\in Y$. Es ist $\int f(x,y)\mu(dx)$ $\nu$-messbar und $\iint f(x,y)\mu(dx)\nu(dy)$ existiert (und symmetrisch in $x$ und $y$) und schließlich:
\[
\int f d(\mu\times\nu) = \iint f(x,y)\mu(dx)\nu(dy) = \iint f(x,y) \nu(dy) \mu(dx).
\]
\end{enumerate}
\end{satz}

\begin{beweis}
Wir setzen
\begin{multline*}
\mathcal E \da \{ M\subset X\times Y: \ind_M(\cdot,y) \in \mathbb F_\mu(X,\bar\MdR) \text{ ist $\mu$-messbar für $\nu$-fast-alle $y\in Y$,} \\
\int\ind_M(x,y) \mu(dx) \in \mathbb F_\nu(Y,\bar\MdR)\text{ ist $\nu$-messbar}\}.
\end{multline*}
Für $M\in\mathcal E$ sei
\[
\varrho(M) \da \iint \ind_M (x,y)\mu(dx)\nu(dy).
\]
Wir zeigen zwei Hilfsbehauptungen:
\begin{enumerate}
\item[($\alpha$)] Ist $M_j\in\mathcal E$, $j\in\MdN$, eine Folge paarweise disjunkter Mengen, so ist $\bigcup_{j=1}^\infty M_j\in\mathcal E$, denn:

\[
\ind_{\bigcup_{j=1}^\infty M_j}(\cdot, y)=\sum_{j=1}^\infty \ind_{M_j}(\cdot, y)
\]
ist $\mu$-messbar für $\nu$-fast-alle $y\in Y$ und \[
\int\ind_{\bigcup_{j=1}^\infty M_j} (x,y)\mu(dx) = \sum_{j=1}^\infty \int \ind_{M_j}(x,y) \mu(dx)
\]
ist $\nu$-messbar.

\item[($\beta$)] Ist $M_j\in\mathcal E$, $j\in\MdN$, $M_1\supset M_2 \supset \cdots$ sowie $\varrho(M_1) <\infty$, so gilt $\bigcap_{j\ge 1}M_j \in\mathcal E$, denn:

\[
\ind_{\bigcap_{j=1}^\infty M_j}(\cdot, y) = \lim_{j\to\infty}\ind_{M_j}(\cdot,y)
\]
ist $\mu$-messbar für $\nu$-fast-alle $y\in Y$ und 
\[
\int\ind_{\bigcap_{j=1}^\infty M_j} (x,y)\mu(dx) = \lim_{j\to\infty} \int \ind_{M_j} (x,y)\mu(dx)
\]
ist $\nu$-messbar.
\end{enumerate}

Betrachte nun folgende Mengensysteme:
\begin{align*}
\mathcal E_0 &\da \{A\times B: A\in\A_\mu,\, B\in\A_\nu\}, \\
\mathcal E_1 &\da \{\bigcup_{i=1}^\infty G_i: G_i\in\mathcal E_0\}, \\
\mathcal E_2 &\da \{\bigcap_{j\ge 1}^\infty H_j: H_j\in\mathcal E_1\}.
\end{align*}

Für $A\times B\in\mathcal E_0$ ist $\ind_{A\times B}(\cdot, y) = \ind_A \cdot \ind_B(y)$ $\mu$-messbar für alle $y\in Y$ und $\int \ind_{A\times B}(x,y) \mu(dx) = \mu(A) \cdot \ind_B(y)$ ist $\nu$-messbar. Also ist $A\times B\in \mathcal E$, und damit $\mathcal E_0\subset\mathcal E$.

Für $A\times B\in \mathcal E_0$, $C\times D\in\mathcal E_0$ ist 
\begin{align*}
(A\times B)\cap (C\times D) &= (A\cap C) \times (B\cap D) \in \mathcal E_0
\intertext{und}
(A\times B)\setminus (C\times D) &= \big(\underbrace{(A\setminus C) \times B}_{\in\mathcal E_0}\big) \stackrel{\bullet}\cup \big(\underbrace{(A\cap C) \times (B\setminus D)}_{\in\mathcal E_0}\big).
\end{align*}
Jede abzählbare Vereinigung von Mengen aus $\mathcal E_0$ kann als abzählbare Vereinigung von paarweise disjunkten Mengen aus $\mathcal E_0$ erhalten werden, das heißt $\mathcal E_1\subset \mathcal E$ nach ($\alpha$).

Da $\mathcal{E}_1$ stabil bez{\"u}glich der Bildung endlicher Durchschnitte ist, folgt mit Hilfe von $(\beta)$
\[
\{\bigcap_{i=1}^\infty H_i : H_i\in\mathcal E_1,\, i\in\MdN,\, \varrho(H_1)<\infty\} \subset \mathcal E.
\]

\textbf{Behauptung:} Für $M\subset X\times Y$ gilt 
\[
(\mu\times\nu)(M) = \inf\{\varrho(V):M\subset V, V\in \mathcal E_1\}
\]
und es gibt zu $M$ ein $W\in\mathcal E_2$ mit $M\subset W$ und $(\mu\times\nu)(M) = (\mu\times\nu)(W) = \varrho (W)$.

\textbf{Nachweis:} F{\"u}r $i\in\mathbb{N}$ sei $A_i\times B_i  \in \mathcal E_0$ mit $M\subset \bigcup_{i=1}^\infty (A_i\times B_i)\ad V \in\mathcal E_1$. Dann gilt 
$$\ind_V \le \sum_{i=1}^\infty \ind_{A_i\times B_i},
$$ 
wobei Gleichheit gilt, falls die Mengen $A_i\times B_i$ paarweise disjunkten sind. Somit erh{\"a}lt man 
$$\varrho(V) \le \sum_{i=1}^\infty \varrho(A_i\times B_i) = \sum_{i=1}^\infty \mu(A_i) \nu(B_i),
$$ 
wobei auch hier Gleichheit gilt, falls die Mengen $A_i\times B_i$, $i\in\mathbb{N}$, paarweise disjunkt sind.

Der erste Teil der Behauptung folgt somit aus
\begin{align*}
(\mu\times\nu)(M) &= \inf\{\sum_{i=1}^\infty \mu(A_i)\nu(B_i) : A_i\times B_i \in \mathcal E_0, i\in\MdN, M\subset\bigcup_{i=1}^\infty(A_i\times B_i)\} \\
&= \inf\{\varrho(V): M\subset V, V\in\mathcal E_1\}.
\end{align*}

Ist $(\mu\times \nu)(M)<\infty$, so existieren $V_i\in\mathcal E_1$, $i\in\MdN$, $M\subset V_i$ mit
\[
\lim_{i\to\infty} \varrho(V_i) = (\mu\times\nu)(M).
\]
Setze $M\subset W \da \bigcap_{i=1}^\infty V_i \in\mathcal E_2$. Es gilt
\[
(\mu\times\nu)(M) \le (\mu\times \nu)(W) \le \lim_{i\to\infty}\varrho(V_i) = \varrho(W) = (\mu\times\nu)(M).
\]
Ist $(\mu\times\nu)(M) =\infty$, so setze $W\da X\times Y\in\mathcal E_2$.

Nun beweisen wir die eigentlichen Aussagen des Satzes:
\begin{enumerate}
\item Sei $A\times B\in\mathcal E_0$. Zun{\"a}chst gilt offenbar

Für ein beliebiges $V\in\mathcal E_1$ mit $A\times B\subset V$ gilt
\[
(\mu\times\nu)(A\times B) =\inf\{\varrho(V):A\times B\subset V,V\in\mathcal{E}_1\}= \varrho(A\times B) = \mu(A) \nu(B).
\]
Für $T\subset X\times Y$ und $U\in \mathcal E_1$ mit $T\subset U$ Sind
$U\cap (A\times B)$ und $U\cap (A\times B)^c$ disjunkte Mengen in $\mathcal E_1$. Wir  erhalten so
\begin{multline*}
(\mu\times \nu) (T\cap (A\times B)) + (\mu\times\nu)(T\cap (A\times B)^c)
\\ \le \varrho(U\cap (A\times B)) + \varrho(U\cap (A\times B)^c) = \varrho(U).
\end{multline*}
Bildet man das Infimum über alle $U\in\mathcal E_1$ mit $U\supset T$, so ergibt diese Ungleichung
\[
(\mu\times \nu) (T\cap (A\times B)) + (\mu\times\nu)(T\cap (A\times B)^c) \le (\mu\times \nu)(T),
\]
woraus $A\times B\in\A_{\mu\times\nu}$ folgt.
\item Ist $M\subset X\times Y$ und $(\mu\times\nu)(M)<\infty$, so gibt es $W\in\mathcal E_2$ mit $\varrho(W)<\infty$ und mit der gewünschten Eigenschaft $(\mu\times\nu)(M) = (\mu\times\nu)(W)$.
\item Sei $f=\ind_M$, $M\in\A_{\mu\times\nu}$ und $(\mu\times\nu)(M)<\infty$. Zu $M$ existiert ein $W\in\mathcal E_2$ mit $M\subset W$ und $(\mu\times\nu)(M) = (\mu\times\nu)(W) =\varrho(W)$.

Fall 1: $(\mu\times\nu)(M) = 0$. Dann gilt $\rho(W)=0$ und $\ind_M(\cdot,y)=0$ $\mu$-fast-überall für $\nu$-fast-alle $y\in Y$. Insbesondere ist $M\in\mathcal E$ und $\rho(M)=0$.

Fall 2: $(\mu\times\nu)(M) > 0$. Dann gilt $(\mu\times\nu)(W\setminus M) = 0$, $M\subset W$. Fall 1 liefert $W\setminus M\in\mathcal E$ und $\rho(W\setminus M)=0$. Also ist $\ind_M(\cdot, y) = (\ind_W - \ind_{W\setminus M})(\cdot,y)$ $\mu$-messbar für $\nu$-fast-alle $y\in Y$ und  $\ind_M(\cdot,y) = \ind_W(\cdot,y)$ $\mu$-fast-überall für $\nu$-fast-alle $y\in Y$. Insbesondere ist $M\in\mathcal E$ und $\varrho(M) = \varrho(W) = (\mu\times\nu)(M)$.
\end{enumerate}
\end{beweis}


\end{document}
