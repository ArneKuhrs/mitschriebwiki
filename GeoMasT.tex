\documentclass[a4paper,twoside,DIV15,BCOR12mm]{scrbook}

\usepackage{mathe}
\usepackage{saetze-hug}
\usepackage{enumerate}

\newcommand{\A}{\mathcal A}

\pdfinfo{
	/Author (Die Mitarbeiter von http://mitschriebwiki.nomeata.de/)
	/Title   (Geometrische Maßtheorie)
	/Subject (Geometrische Maßtheorie)
}

\author{PD. Dr. Daniel Hug}
\publishers{Die Mitarbeiter von \url{http://mitschriebwiki.nomeata.de/}}
\title{Geometrische Maßtheorie}
\date{Wintersemester 2009/2010}
\makeindex

\begin{document}

\maketitle

\tableofcontents

\chapter*{Vorwort}

\section*{Über dieses Skriptum}
Dies ist ein Mitschrieb der Vorlesung \glqq Geometrische Maßtheorie\grqq\
von Herrn PD. Dr. Daniel Hug im Wintersemester 2009/2010 an der Universität Karlsruhe (TH).
 Die Mitschriebe der Vorlesung werden mit ausdrücklicher Genehmigung von Herrn Hug hier veröffentlicht,
Herr Hug ist für den Inhalt nicht verantwortlich.

\section*{Wer}
Beteiligt am Mitschrieb ist Joachim Breitner.

\section*{Wo}
Alle Kapitel inklusive \LaTeX-Quellen können unter \url{http://mitschriebwiki.nomeata.de} abgerufen werden.
Dort ist ein \emph{Wiki} eingerichtet und von Joachim Breitner um die \LaTeX-Funktionen erweitert.
Das heißt, jeder kann Fehler nachbessern und sich an der Entwicklung
beteiligen. Auf Wunsch ist auch ein Zugang über \emph{Subversion} möglich.

\chapter{Grundlage: Maß und Integral}

\section{Äußere Maße und Meßbarkeit}

\begin{definition}
Sei $X$ eine Menge. Eine Abbildung
\[
\mu : \mathcal P(X) \to [0,\infty]
\]
heißt \emph{äußeres Maß} auf $X$, falls gilt:
\begin{enumerate}
\item $\mu(\emptyset) = 0$
\item Für $A,A_n \subset X$, $i\in \MdN$ mit $A\subset \bigcup_{i\ge 1} A_i$ gilt 
\[
\mu(A) \le \sum_{i\ge1} \mu(A_i).
\]
\end{enumerate}
\end{definition}

Beobachte folgende einfache Folgerungen der Definition:

\begin{itemize}
\item $A\subset B \subset X \implies \mu(A)\le \mu(B)$
\item $A\subset B \cup \emptyset \cup \emptyset \cup \ldots \implies \mu(A) \le \mu(B) + \mu(\emptyset) + \mu(\emptyset) + \cdots = \mu(B)$
\end{itemize}

\begin{beispiel}
\begin{align*}
\mu_1(A) &= 
\begin{cases}
\#A, & A \text{ endlich} \\
0, & \text{sonst}
\end{cases}
&
\mu_2(A) &= 
\begin{cases}
1, & A \ne 0\\
0, & \text{sonst}
\end{cases} \\
\mu_3(A) &= 
\begin{cases}
\infty, & A \ne \emptyset \\
0, & \text{sonst}
\end{cases}
&
\mu_4(A) &= 
\begin{cases}
\infty, & A^c \text{ endlich} \\
0, & \text{sonst}
\end{cases} \\
\mu_5(A) &= 
\begin{cases}
0, & \text{$A$ abzählbar oder $A^c$ abzählbar} \\
1, & \text{sonst}
\end{cases}
\end{align*}
\end{beispiel}

Für die Konstruktion eines äußeren Maßes aus Rohdaten ist folgender Satz nützlich:

\begin{satz}
Sei $\mathcal E \subset \mathcal P (X)$ mit $\emptyset \in \mathcal E$, sei $\eta: \mathcal E \to [0,\infty]$ mit $\eta(\emptyset)=0$. Dann wird durch
\[
\mu(A) \da \inf\{\sum_{i=1}^\infty \eta(A_i) \mid A_i\in \mathcal E, i\in \MdN, A\subset\bigcup_{i\ge1}A_i\}
\]
($\inf\emptyset = \infty$) für $A\subset X$ ein äußeres Maß erklärt, das von $(\mathcal E, \eta)$ induzierte äußere Maß.
\end{satz}

\begin{beweis}
Es ist $0\le \mu(\emptyset) \le \sum_{i=1}^\infty \eta(\emptyset)$, da $\emptyset \subset \bigcup_{i=1}^\infty\emptyset$ und $\emptyset\in\mathcal E$.

Seien $A, A_i\subset X$ und $A\subset \bigcup_{i\ge1}A_i$. Wir müssen zeigen: $\mu(A) \le \sum_{i\ge 1}A_i$.

Ist für ein $i\in\MdN$ bereits $\mu(A_i)=\infty$, so sind wir fertig. Sei also $\mu(A_i)<\infty$ für alle $i\in\MdN$. Sei $\varepsilon>0$, dann existiert ein $E_{ij}\in \mathcal E$, $j\in \MdN$ mit $A_i\subset \bigcup_{j\ge 1} E_{ij}$ und
\[
\mu(A_i) + \frac{\varepsilon}{2^i}\ge \sum_{i\ge j}\mu(E_{ij}).
\]
Also gilt
\[
A \subset \bigcup_{i\ge j} Ai \subset \bigcup_{i,j\ge 1} E_{ij}, \quad E_{ij}\in \mathcal E
\]
und daraus folgt
\begin{align*}
\mu(A) &\le \sum_{i,j\ge 1}\eta(E_{ij}) \\
&= \sum_{i=1}^\infty \sum_{j=1}^\infty \mu(E_{ij}) \\
&\le \sum_{i=1}^\infty \mu(A_i) + \frac\varepsilon{2^i} \\
&\le \sum_{i=1}^\infty \mu(A_i) + \varepsilon
\end{align*}
was mit $\varepsilon \to 0$ bedeutet dass
\[
\mu(A) \le \sum_{i=1}^\infty \mu(A_i)
\]
\end{beweis}

\begin{definition}
Sei $\mu$ das äußere Maß auf $X$. Eine Menge $A\subset X$ heißt $\mu$-messbar, falls für alle $M\subset X$ gilt:
\[
\mu(M) = \mu(M\cap A) + \mu(M \cap A^c).
\]
Die Menge aller $\mu$-messbaren Mengen wird mit $\mathcal A_\mu$ bezeichnet.
\end{definition}

Es genügt bereits: $A$ ist $\mu$-messbar genau dann wenn
\[
\mu(M) \ge \mu(M\cap A) + \mu(M\cap A^c) \quad \forall M\subset X
\]

Denn wegen
\[
M\subset (M\cap A) \cup (M\cap A^c) \cup \emptyset \cup \emptyset\cdots
\]
gilt 
\[
\mu(M)\le \mu(M\cap A) + \mu(M\cap A^c) + \mu(\emptyset) + \cdots.
\]

Es gilt stets $\emptyset, X\in\mathcal A_\mu$.

\begin{bemerkung}
Sei $Y\subset X$. $\mu LY$ ist das durch
\[
(\mu LY)(M) \da \mu(M\cap Y)\quad M\subset X
\]
erklärte äußere Maß. Ferner ist $\mathcal A_\mu \subset \mathcal A_{\mu LY}$, denn: für $A\in\mathcal A_\mu$ und $M\subset X$ ist
\begin{align*}
\mu_{LY}(M)
&= \mu(Y\cap M) = \mu(Y\cap M\cap A) + \mu(Y\cap M \cap A^c) \\
&= (\mu LY) (M\cap A) + (\mu LY)(M\cap A^c)
\end{align*}
Es gilt
\[
A \in \mathcal A_\mu \iff \mu= (\mu LA) + (\mu L A^c)
\]
\end{bemerkung}

\begin{proposition}
Für ein äußeres Mapß $\mu$ auf $X$ gelten die folgenden Aussagen:
\begin{enumerate}[a)]
\item $\emptyset,\, X\in \mathcal A_\mu$ sowie $A\in \mathcal A_\mu \iff A^c\in\mathcal A_\mu$.
\item Für $A\subset X$ mit $\mu(A)=0$ gilt $A\in\mathcal A_\mu$.
\item Für $A_i\in\mathcal A_\mu$, $i\in\MdN$ gilt $\bigcup_{i\ge1} A_i\in\mathcal M_\mu$ und $\bigcap_{i\ge1} A_i\in\mathcal M_\mu$.
\item Für $A_\in\mathcal A_\mu$, $B\in X$ gilt
\[
\mu(A\cap B) + \mu(A\cup B) = \mu(A) + \mu(B).
\]
\item Für $A_i\in\mathcal A_\mu$, paarweise disjunkt, gilt
\[
\mu(\bigcup_{i=1}^\infty A_i) = \sum_{i=1}^\infty A_i.
\]
\item Für $A_i\in \mathcal A_\mu$, $i\in\MdN$ und $A_i\subset A_{i+1}$ für alle $i\in\MdN$ gilt
\[
\mu(\bigcup_{i=1}^\infty A_i) = \lim_{i\to\infty}\mu(A_i).
\]
\item Für $A_i \in \mathcal A_\mu$, $i\in \MdN$ mit $\mu(A_1) < \infty$ und $A_i\supset A_{i+1}$ für alle $i\in\MdN$ gilt:
\[
\mu(\bigcap_{i=1}^\infty A_i) = \lim_{i\to\infty}\mu(A_i).
\]
\end{enumerate}
\end{proposition}

\begin{beweis}
\begin{enumerate}
\item[c)] $A_1,A_2\in \mathcal A_\mu$.
\begin{align*}
\mu(M) &= \mu(M\cap A_1) + \mu(M\cap A_1^c) \\
&= \mu(M\cap A_1) + \mu(M\cap A_1^c\cap A_2) + \mu(M\cap A_1^c \cap A_2^c) \\
&\ge \mu(M \cap (A_1 \cup (A_1^c \cap A_2))) + \mu)(M \cap A_1^c \cap A_2^c) \\
&= \mu(M\cap (A_1 \cup A_2)) + \mu(M\cap (A_1\cup A_2)^c)
\end{align*}
Daraus folgt, dass $A_1\cup A_2$ $\mu$-messbar ist. Per Induktion sieht man dann, dass für $A_1,\ldots,A_n\in\mathcal A_\mu$ gilt: $\bigcup_{i=1}^n A_i \in \mathcal A_\mu$.
\item[e)] Sind $A_1,\ldots,A_n\in\mathcal A_\mu$ und paarweise disjunkt. Dann gilt
\begin{align*}
\mu(A_1\cup A_2) = \mu( (A_1\cup A_2)\cap A_1) + \mu( (A_1\cup A_2)\cap A_1^c) 
= \mu(A_1) + \mu(A_2)
\end{align*}
woraus folgt dass
\[
\mu(\bigcup_{i=1}^n A_i) = \sum_{i=1}^n \mu(A_i).
\]
Wegen
\[
\sum_{i=1}^n \mu(A_i) \le \mu(\bigcup_{i=1}^n A_i) \quad \forall n\in\MdN
\]
gilt
\[
\sum_{i=1}^\infty \mu(A_i) \le \mu(\bigcup_{i=1}^\infty A_i) \le \sum_{i=1}^\infty \mu(A_i)
\]
und damit Gleichheit.
\item[f)] Wir definieren $B_1 \da A_1$, $B_2 \da A_2\setminus A_1$, $B_3 \da A_3\setminus A_2\ldots$ Es gilt nun dass $B_i\in \mathcal A_\mu$ für alle $i\in\MdN$ und die $B_i$ sind paarweise disjunkt. Nun kann ausrechnen dass
\begin{align*}
\lim_{k\to\infty} \mu(A_k) 
&= \lim_{k\to\infty} \mu(\bigcup_{i=1}^k B_i) \\
&= \lim_{k\to\infty} \sum_{i=1}^k \mu(B_i)\\
&= \sum_{i=1}^\infty \mu(B_i)\\
&= \mu(\bigcup_{i=1}^\infty B_i) \\
&= \mu(\bigcup_{i=1}^\infty A_i)
\end{align*}
\item[g)] Es ist
\begin{align*}
\mu(A_1) = \mu(A_2\cup A_1\setminus A_2) = \mu(A_2) + \mu(A_1\setminus A_2),
\end{align*}
das heißt
\[
\mu(A_1\setminus A_2) = \mu(A_1) - \mu(A_2).
\]
Damit zeigt man
\begin{align*}
\mu(A_1\setminus \bigcup_{i\ge1}A_i) 
&= \mu(A_1 \cap (\bigcap_{i\ge1}A_i)^c) \\
&= \mu(A_1 \cap (\bigcup_{i\ge1}A_i^c) \\
&= \mu(\bigcup_{i\ge 1}(A_1\cap A_i^c)) \\
\text{(nach f)) }&= \lim_{i\to\infty} \mu(\underbrace{\A_1\cap A_i^c}_{= A_1\setminus A_i}) \\
&= \lim_{i\to\infty} (\mu(A_1) -\mu(A_i)) \\
&= \mu(A_1) - \lim_{i\to\infty}\mu(A_i)
\end{align*}
\item[c)] Sei $M\subset X$. Wir definieren $C_k \da \bigcup_{i=1}^k A_i \in \A_\mu$. Damit gilt $C_1\subset C_2\subset \cdots$.

Sei ohne Beschränkung der Allgemeinheit $\mu(M) <\infty$. Dann gilt
\begin{align*}
\infty &> \mu(M) = (\mu LM)(X)\\
&= (\mu LM)(C_k) + (\mu LM)(C_k^c) \\
&= \lim_{k\to\infty} (\mu LM)(C_k) + \lim_{k\to\infty}(\mu LM)(C_k^c) \\
&= (\mu LM)(\bigcup_{i\ge 1}C_i) + (\mu LM)(\bigcap_{i\ge 1}C_i^c) \\
&= (\mu LM)(\bigcup_{i\ge 1}C_i) + (\mu LM)( (\bigcup_{i\ge 1}C_i)^c) \\
&= \mu(M\cap (\bigcup_{i\ge 1}A_i)) + \mu(M\cap (\bigcup_{i\ge 1}A_i)^c) 
\end{align*}
und somit $\bigcup_{i\ge 1}A_i \in \A_\mu$.

\item[d)] Für $A\in\A_\mu$ und $B\subset X$ gilt:
\begin{align*}
\mu(A\cup B) &= \mu( (A\cup B) \cap A) + \mu( (A\cup B)\cap A^c) \\
&= \mu(A) + \mu(B\cap A^c)\\
\text{sowie}\quad 
\mu(B) &= \mu(B\cap A) + \mu(B\cap A^c).
\intertext{Setzt man dies in die folgende Gleichung ein, so erhält man}
\mu(A) + \mu(B) &= \mu(A) + \mu(B\cap A) + \mu(B\cap A^c)\\
&= \mu(B\cap A) + \mu(A\cup B).
\end{align*}
\end{enumerate}
\end{beweis}

Hinweis: Es ist $\A_\mu$ eine (bezüglich $\mu$ vollständige) $\sigma$-Algebra und $\mu$ ist ein $\sigma$-additives Maß auf $\A_\mu$, wobei „$\A_\mu$ ist $\mu$-vollständig“ heißt, dass jede $\mu$-Nullmenge in $\A_\mu$ liegt. $(X,\A_\mu)$ ist ein messbarer Raum und $(X,\A_\mu,\mu)$ ist ein Meßraum.

\begin{definition}
Sei $\A$ eine $\sigma$-Algebra auf $X$. Ein äußeres Maß $\mu$ auf $X$ heißt $\A$-regulär, falls $\A\subset \A_\mu$ gilt und zu jeder Menge $M\subset X$ ein $A\in\A$ existiert mit $M\subset A$ und $\mu(M) = \mu(A)$. Das äußere Maß $\mu$ heißt regulär, falls $\mu$ ein $\A_\mu$-reguläres Maß ist.
\end{definition}

\begin{proposition}
Sei $\mathcal A$ eine $\sigma$-Algebra in $X$, $\mu$ ein $\A$-reguläres äußeres Maß auf $X$. Dann gilt:
\begin{enumerate}[a)]
\item Ist $M_i\subset X$, $M_i\subset M_{i+1}$ für alle $i\in\MdN$, so ist
\[
\mu(\bigcup_{i\ge 1}) = \lim_{i\to\infty}\mu(M_i)
\]
\item Zu jedem $M\in X$ mit $\mu(M)<\infty$ existiert ein $A\in\A$, so dass für alle $B\in \A_\mu$ gilt:
\[
\mu(B\cap M) = \mu(B\cap A)
\]
\item Ist $M_1\cup M_2\in \A$ und $\mu(M_1\cup M_2) = \mu(M_1)+\mu(M_2) <\infty$, so existiereren $A_1,A_2\in\A$ mit $M_i\subset A_i$, $i=1,2$ und $\mu(A_i\setminus M_i) = 0$. Insbesondere ist $M_1,M_2\in \A_\mu$.
\end{enumerate}
\end{proposition}

\begin{beweis}
\begin{enumerate}[a)]
\item Zu jedem $i\in\MdN$ finden wir ein $A_i\in \A$ so dass $M_i\subset A_i$ und $\mu(M_i) =\mu(A_i)$. Dazu definieren wir $B_i \da \bigcap_{j\ge i} A_j$. Damit gilt $M_i \subset B_i\subset A_i$, $B_i\subset B_{i+1}$ und $B_i\in\A$, $i\in\MdN$. Es folgt:
\begin{align*}
\mu(\bigcup_{i\ge 1} M_i) &\le \mu(\bigcup_{i\ge1}B_i) \\
&= \lim_{i\to\infty} \mu(B_i) \\
&\le \lim_{i\to\infty} \mu(A_i) \\
&\le \lim_{i\to\infty} \mu(M_i) \\
&\le \lim_{i\to\infty} \mu(\bigcup_{i\ge 1} M_i)
\end{align*}
\item  Zu $M$ existiert ein $A\in\A$ mit $M\subset A$ und $\mu(M) = \mu(A)$. Sei $B\in \A_\mu$. Dann folgt:
\begin{align*}
\mu(A) = \mu(M) &= \mu(M\cap B) + \mu(M\cap B^c) \\
&\le \mu(A\cap B) + \mu(M \cap B^c) \\
&\le \mu(A\cap B) + \mu(A \cap B^c) = \mu(A) \\
\end{align*}
woraus Gleichheit in obiger Ungleichung folgt. Wegen $\mu(M)<\infty$ ist auch $\mu(M\cap B^c)<\infty$, und wir können dies von zwei obigen Termen abziehen und erhalten
\[
\mu(M\cap B) = \mu(A\cap B).
\]
\item  Zu $M_1$ existiert $\tilde A_1\in\A$ mit $M_1 \subset \tilde A_1$ und $\mu(M_1) = \mu(\tilde A_1)$. Wir definieren $A_1 \da \tilde A_1 \cap (M_1\cup M_2)$. Für diese Menge gilt nun $M_1\subset A_1 \subset M_1\cup M_2$. Wir folgern 
\[
\mu(M_1) \le \mu(A_1) \le \mu (\tilde A_1) \le \mu(M_1)
\]
und
\begin{align*}
\mu(A_1\cap M_2) + \mu(A_1 \cup M_2) &= \mu(A_1) + \mu(M_2) \\
&= \mu(M_1) + \mu(M_2) \\
&= \mu(M_1\cup M_2) \\
&= \mu(A_1 \cup M_2) < \infty
\end{align*}
woraus $\mu(A_1\cap M_2) = 0$ folgt.

Nun liegt $A_1\setminus M_1\subset A_1\cap M_2$, also gilt $\mu(A_1\setminus M_1) = 0$ und somit $A_1\setminus M_1\in \A_\mu$. Damit gilt dann $M_1 = A_1\cap (A_1\setminus M_1)^c\in \A_\mu$.
\end{enumerate}
\end{beweis}

\end{document}
