\documentclass[a4paper,twoside,DIV15,BCOR12mm]{scrbook}
\usepackage{mathe}
\usepackage{saetze-hug}
\usepackage{enumerate}
\usepackage{dsfont}
\newcommand{\A}{\mathcal A}
\newcommand{\borel}{\mathfrak B}
\newcommand{\ind}{\mathds 1}
\newcommand{\HR}{\mathcal H}
\newcommand{\HM}{\mathcal H}
%\renewcommand{\mathbb}{\mathds}
\DeclareMathOperator{\im}{im}
\DeclareMathOperator{\id}{id}
\DeclareMathOperator{\diam}{diam}
\DeclareMathOperator{\Lip}{Lip}
\DeclareMathOperator{\epi}{epi}
\DeclareMathOperator{\supp}{supp}
\DeclareMathOperator{\GL}{GL}
\DeclareMathOperator{\Kern}{Kern}
\DeclareMathOperator{\Bild}{Bild}
% Maß-Restriktions-Symbol
\newcommand{\MR}{\lfloor}
\jot3mm
\pdfinfo{
	/Author (Die Mitarbeiter von http://mitschriebwiki.nomeata.de/)
	/Title   (Geometrische Maßtheorie)
	/Subject (Geometrische Maßtheorie)
}

\author{PD. Dr. Daniel Hug}
\publishers{Die Mitarbeiter von \url{http://mitschriebwiki.nomeata.de/}}
\title{Geometrische Maßtheorie}
\date{Wintersemester 2009/2010}
\makeindex

\begin{document}

\maketitle

\tableofcontents

\chapter*{Vorwort}

\section*{Über dieses Skriptum}
Dies ist ein Mitschrieb der Vorlesung \glqq Geometrische Maßtheorie\grqq\
von Herrn PD. Dr. Daniel Hug im Wintersemester 2009/2010 an der Universität Karlsruhe (TH).
 Die Mitschriebe der Vorlesung werden mit ausdrücklicher Genehmigung von Herrn Hug hier veröffentlicht,
Herr Hug ist für den Inhalt nicht verantwortlich.

\section*{Wer}
Beteiligt am Mitschrieb ist Joachim Breitner.

\section*{Wo}
Alle Kapitel inklusive \LaTeX-Quellen können unter \url{http://mitschriebwiki.nomeata.de} abgerufen werden.
Dort ist ein \emph{Wiki} eingerichtet und von Joachim Breitner um die \LaTeX-Funktionen erweitert.
Das heißt, jeder kann Fehler nachbessern und sich an der Entwicklung
beteiligen. Auf Wunsch ist auch ein Zugang über \emph{Subversion} möglich.

\chapter{Grundlagen: Maß und Integral}

\section{Äußere Maße und Messbarkeit}

\begin{definition}
Sei $X$ eine Menge. Eine Abbildung
\[
\mu : \mathcal P(X) \to [0,\infty]
\]
heißt \emph{äußeres Maß} auf $X$, falls gilt:
\begin{enumerate}
\item $\mu(\emptyset) = 0$
\item Für $A,A_n \subset X$, $i\in \MdN$ mit $A\subset \bigcup_{i\ge 1} A_i$ gilt 
\[
\mu(A) \le \sum_{i\ge1} \mu(A_i).
\]
\end{enumerate}
\end{definition}

Beobachte folgende einfache Folgerungen der Definition:

\begin{itemize}
\item $A\subset B \subset X \implies \mu(A)\le \mu(B)$
\item $A\subset B \cup \emptyset \cup \emptyset \cup \ldots \implies \mu(A) \le \mu(B) + \mu(\emptyset) + \mu(\emptyset) + \cdots = \mu(B)$
\end{itemize}

\begin{beispieleX}
\begin{align*}
\mu_1(A) &= 
\begin{cases}
\#A, & A \text{ endlich} \\
0, & \text{sonst}
\end{cases}
&
\mu_2(A) &= 
\begin{cases}
1, & A \ne \emptyset \\
0, & \text{sonst}
\end{cases} \\
\mu_3(A) &= 
\begin{cases}
\infty, & A \ne \emptyset \\
0, & \text{sonst}
\end{cases}
&
\mu_4(A) &= 
\begin{cases}
\infty, & A^c \text{ endlich} \\
0, & \text{sonst}
\end{cases} \\
\mu_5(A) &= 
\begin{cases}
0, & \text{$A$ abzählbar} \\
1, & \text{sonst}
\end{cases}
\end{align*}
\end{beispieleX}

Für die Konstruktion eines äußeren Maßes aus Rohdaten ist folgender Satz nützlich:

\begin{satz}
Sei $\mathcal E \subset \mathcal P (X)$ mit $\emptyset \in \mathcal E$, sei $\eta: \mathcal E \to [0,\infty]$ mit $\eta(\emptyset)=0$. Dann wird durch
\[
\mu(A) \da \inf\left\{\sum_{i=1}^\infty \eta(A_i) : A_i\in \mathcal E, i\in \MdN, A\subset\bigcup_{i\ge1}A_i\right\}
\]
($\inf\emptyset = \infty$) für $A\subset X$ ein äußeres Maß erklärt, das von $(\mathcal E, \eta)$ induzierte äußere Maß.
\end{satz}

\begin{beweis}
Es ist $0\le \mu(\emptyset) \le \sum_{i=1}^\infty \eta(\emptyset)=0$, da $\emptyset \subset \bigcup_{i=1}^\infty\emptyset$ und $\emptyset\in\mathcal E$.

Seien $A, A_i\subset X$ und $A\subset \bigcup_{i\ge1}A_i$. Wir müssen zeigen: $\mu(A) \le \sum_{i\ge 1}\mu(A_i)$.

Ist für ein $i\in\MdN$ bereits $\mu(A_i)=\infty$, so sind wir fertig. Sei also $\mu(A_i)<\infty$ für alle $i\in\MdN$. Sei $\varepsilon>0$. Dann existiert $E_{ij}\in \mathcal E$, $j\in \MdN$ mit $A_i\subset \bigcup_{j\ge 1} E_{ij}$ und
\[
\mu(A_i) + \frac{\varepsilon}{2^i}\ge \sum_{j\ge 1}\eta(E_{ij}) \quad \text{für $i\in \MdN$}
\]
Also gilt
\[
A \subset \bigcup_{i\ge 1} A_i \subset \bigcup_{i,j\ge 1} E_{ij},
%\quad E_{ij}\in \mathcal E
\]
und daraus folgt
\begin{align*}
\mu(A) &\le \sum_{i,j\ge 1}\eta(E_{ij}) \\
&= \sum_{i=1}^\infty \sum_{j=1}^\infty \eta(E_{ij}) \\
&\le \sum_{i=1}^\infty (\mu(A_i) + \frac\varepsilon{2^i}) \\
&\le \left( \sum_{i=1}^\infty \mu(A_i) \right) + \varepsilon.
\end{align*}
Mit $\varepsilon \to 0$ ergibt dies
\[
\mu(A) \le \sum_{i=1}^\infty \mu(A_i).
\]
\end{beweis}

\begin{definition}
Sei $\mu$ ein äußeres Maß auf $X$. Eine Menge $A\subset X$ heißt $\mu$-messbar, falls für alle $M\subset X$ gilt:
\[
\mu(M) = \mu(M\cap A) + \mu(M \cap A^c).
\]
Die Menge aller $\mu$-messbaren Mengen wird mit $\A_\mu$ bezeichnet.
\end{definition}

Es genügt bereits: $A$ ist $\mu$-messbar genau dann, wenn für alle $M\subset X$ gilt:
\[
\mu(M) \ge \mu(M\cap A) + \mu(M\cap A^c).
\]

Denn wegen
\[
M\subset (M\cap A) \cup (M\cap A^c) \cup \emptyset \cup \emptyset\cdots
\]
gilt 
\[
\mu(M)\le \mu(M\cap A) + \mu(M\cap A^c) + \mu(\emptyset) + \cdots.
\]

Es gilt stets $\emptyset, X\in\A_\mu$.

\begin{bemerkung}
Für $Y\subset X$ ist $\mu \MR Y$ das durch
\[
(\mu \MR Y)(M) \da \mu(M\cap Y),\quad M\subset X
\]
erklärte äußere Maß. Ferner ist $\A_\mu \subset \A_{\mu \MR Y}$. Denn für $A\in\A_\mu$ und $M\subset X$ ist
\begin{align*}
\mu_{\MR Y}(M)
&= \mu(Y\cap M) = \mu(Y\cap M\cap A) + \mu(Y\cap M \cap A^c) \\
&= (\mu \MR Y) (M\cap A) + (\mu \MR Y)(M\cap A^c).
\end{align*}
Ferner gilt
\[
A \in \A_\mu \iff \mu= (\mu \MR A) + (\mu \MR  A^c).
\]
\end{bemerkung}

\begin{proposition}
Für ein äußeres Maß $\mu$ auf $X$ gelten die folgenden Aussagen:
\begin{enumerate}[a)]
\item $\emptyset,\, X\in \A_\mu$ sowie $A\in \A_\mu \iff A^c\in\A_\mu$.
\item Für $A\subset X$ mit $\mu(A)=0$ gilt $A\in\A_\mu$.
\item Für $A_i\in\A_\mu$, $i\in\MdN$ gilt $\bigcup_{i\ge1} A_i\in\A_\mu$ und $\bigcap_{i\ge1} A_i\in\mathcal A_\mu$.
\item Für $A\in\A_\mu$, $B\subset X$ gilt
\[
\mu(A\cap B) + \mu(A\cup B) = \mu(A) + \mu(B).
\]
\item Für $A_i\in\A_\mu$, $i\in\MdN$, paarweise disjunkt, gilt
\[
\mu(\bigcup_{i=1}^\infty A_i) = \sum_{i=1}^\infty \mu(A_i).
\]
\item Für $A_i\in \A_\mu$, $i\in\MdN$ und $A_i\subset A_{i+1}$ für alle $i\in\MdN$ gilt
\[
\mu(\bigcup_{i=1}^\infty A_i) = \lim_{i\to\infty}\mu(A_i).
\]
\item Für $A_i \in \A_\mu$, $i\in \MdN$ mit $\mu(A_1) < \infty$ und $A_i\supset A_{i+1}$ für alle $i\in\MdN$ gilt:
\[
\mu(\bigcap_{i=1}^\infty A_i) = \lim_{i\to\infty}\mu(A_i).
\]
\end{enumerate}
\end{proposition}

\begin{beweis}
\begin{enumerate}
\item[c)] Seien $A_1,A_2\in \A_\mu$, $M\supset X$. Dann folgt
\begin{align*}
\mu(M) &= \mu(M\cap A_1) + \mu(M\cap A_1^c) \\
&= \mu(M\cap A_1) + \mu(M\cap A_1^c\cap A_2) + \mu(M\cap A_1^c \cap A_2^c) \\
&\ge \mu(M \cap (A_1 \cup (A_1^c \cap A_2))) + \mu(M \cap A_1^c \cap A_2^c) \\
&= \mu(M\cap (A_1 \cup A_2)) + \mu(M\cap (A_1\cup A_2)^c).
\end{align*}
Daraus folgt $A_1\cup A_2 \in \A_\mu$. Per Induktion sieht man dann, dass für $A_1,\ldots,A_n\in\A_\mu$ gilt: $\bigcup_{i=1}^n A_i \in \A_\mu$.
\item[e)] Sind $A_1,\ldots,A_n\in\A_\mu$ und paarweise disjunkt, dann gilt
\begin{align*}
\mu(A_1\cup A_2) = \mu( (A_1\cup A_2)\cap A_1) + \mu( (A_1\cup A_2)\cap A_1^c) 
= \mu(A_1) + \mu(A_2),
\end{align*}
woraus
\[
\mu(\bigcup_{i=1}^n A_i) = \sum_{i=1}^n \mu(A_i)
\]
folgt. Wegen
\[
\sum_{i=1}^n \mu(A_i) \le \mu(\bigcup_{i=1}^\infty A_i)
\]
gilt
\[
\sum_{i=1}^\infty \mu(A_i) \le \mu(\bigcup_{i=1}^\infty A_i) \le \sum_{i=1}^\infty \mu(A_i)
\]
und damit Gleichheit.
\item[f)] Wir definieren $B_1 \da A_1$, $B_2 \da A_2\setminus A_1$, $B_3 \da A_3\setminus A_2\ldots$ Nun ist $B_i\in \A_\mu$ für alle $i\in\MdN$ und die $B_i$ sind paarweise disjunkt. Es folgt
\begin{align*}
\lim_{k\to\infty} \mu(A_k) 
&= \lim_{k\to\infty} \mu(\bigcup_{i=1}^k B_i) \\
&= \lim_{k\to\infty} \sum_{i=1}^k \mu(B_i)\\
&= \sum_{i=1}^\infty \mu(B_i) \tag{nach e)}\\
&= \mu(\bigcup_{i=1}^\infty B_i) \\
&= \mu(\bigcup_{i=1}^\infty A_i).
\end{align*}
\item[g)] Es ist
\begin{align*}
\mu(A_1) = \mu(A_2\cup (A_1\setminus A_2)) = \mu(A_2) + \mu(A_1\setminus A_2),
\end{align*}
das heißt
\[
\mu(A_1\setminus A_2) = \mu(A_1) - \mu(A_2).
\]
Damit zeigt man
\begin{align*}
\mu(A_1) -  \mu(\bigcap_{i\ge 1} A_i) 
&= \mu(A_1\setminus \bigcap_{i\ge1}A_i)  \\
&= \mu(A_1 \cap (\bigcap_{i\ge1}A_i)^c) \\
&= \mu(A_1 \cap (\bigcup_{i\ge1}A_i^c)) \\
&= \mu(\bigcup_{i\ge 1}(A_1\cap A_i^c)) \\
&= \lim_{i\to\infty} \mu(\underbrace{\A_1\cap A_i^c}_{= A_1\setminus A_i})\tag{nach f)} \\
&= \lim_{i\to\infty} (\mu(A_1) -\mu(A_i)) \\
&= \mu(A_1) - \lim_{i\to\infty}\mu(A_i)
\end{align*}
und damit die Behauptung.
\item[c)] Sei $M\subset X$. Wir definieren $C_k \da \bigcup_{i=1}^k A_i \in \A_\mu$. Damit gilt $C_1\subset C_2\subset \cdots$.

Sei ohne Beschränkung der Allgemeinheit $\mu(M) <\infty$. Dann gilt
\begin{align*}
\infty > \mu(M) &= (\mu \MR{}M)(X)\\
&= (\mu \MR{}M)(C_k) + (\mu \MR{}M)(C_k^c) \\
&= \lim_{k\to\infty} (\mu \MR M)(C_k) + \lim_{k\to\infty}(\mu \MR M)(C_k^c) \\
&= (\mu \MR M)(\bigcup_{i\ge 1}C_i) + (\mu \MR M)(\bigcap_{i\ge 1}C_i^c) \\
&= (\mu \MR M)(\bigcup_{i\ge 1}C_i) + (\mu \MR M)( (\bigcup_{i\ge 1}C_i)^c) \\
&= \mu(M\cap (\bigcup_{i\ge 1}A_i)) + \mu(M\cap (\bigcup_{i\ge 1}A_i)^c) 
\end{align*}
und somit $\bigcup_{i\ge 1}A_i \in \A_\mu$.

\item[d)] Für $A\in\A_\mu$ und $B\subset X$ gilt:
\begin{align*}
\mu(A\cup B) &= \mu( (A\cup B) \cap A) + \mu( (A\cup B)\cap A^c) \\
&= \mu(A) + \mu(B\cap A^c)\\
\intertext{sowie}
\mu(B) &= \mu(B\cap A) + \mu(B\cap A^c).
\intertext{Hiermit so erhält man}
\mu(A) + \mu(B) &= \mu(A) + \mu(B\cap A) + \mu(B\cap A^c)\\
&= \mu(B\cap A) + \mu(A\cup B).
\end{align*}
\end{enumerate}
\end{beweis}

Hinweis: Es ist $\A_\mu$ eine (bezüglich $\mu$ vollständige) $\sigma$-Algebra und $\mu$ ist ein $\sigma$-additives Maß auf $\A_\mu$, wobei „$\A_\mu$ ist $\mu$-vollständig“ heißt, dass jede $\mu$-Nullmenge in $\A_\mu$ liegt. $(X,\A_\mu)$ ist ein messbarer Raum und $(X,\A_\mu,\mu)$ ist ein Maßraum.

\begin{definition}
Sei $\A$ eine $\sigma$-Algebra auf $X$. Ein äußeres Maß $\mu$ auf $X$ heißt $\A$-regulär, falls $\A\subset \A_\mu$ gilt und zu jeder Menge $M\subset X$ ein $A\in\A$ existiert mit $M\subset A$ und $\mu(M) = \mu(A)$. Das äußere Maß $\mu$ heißt regulär, falls $\mu$ ein $\A_\mu$-reguläres Maß ist.
\end{definition}

\begin{proposition}\label{prop1.3}
Sei $\A$ eine $\sigma$-Algebra in $X$, $\mu$ ein $\A$-reguläres äußeres Maß auf $X$. Dann gilt:
\begin{enumerate}[a)]
\item Ist $M_i\subset X$, $M_i\subset M_{i+1}$ für alle $i\in\MdN$, so ist
\[
\mu(\bigcup_{i\ge 1}M_i) = \lim_{i\to\infty}\mu(M_i).
\]
\item Zu jedem $M\subset X$ mit $\mu(M)<\infty$ existiert ein $A\in\A$, so dass für alle $B\in \A_\mu$ gilt:
\[
\mu(B\cap M) = \mu(B\cap A)
\]
\item Ist $M_1\cup M_2\in \A$ und $\mu(M_1\cup M_2) = \mu(M_1)+\mu(M_2) <\infty$, so existiereren $A_1,A_2\in\A$ mit $M_i\subset A_i$, $i=1,2$ und $\mu(A_i\setminus M_i) = 0$. Insbesondere ist $M_1,M_2\in \A_\mu$.
\end{enumerate}
\end{proposition}

\begin{beweis}

\begin{enumerate}[a)]
\item Zu jedem $i\in\MdN$ finden wir ein $A_i\in \A$ so dass $M_i\subset A_i$ und $\mu(M_i) =\mu(A_i)$. Dazu definieren wir $B_i \da \bigcap_{j\ge i} A_j$. Damit gilt $M_i \subset B_i\subset A_i$, $B_i\subset B_{i+1}$ und $B_i\in\A$, $i\in\MdN$. Es folgt
\begin{align*}
\mu(\bigcup_{i\ge 1} M_i) &\le \mu(\bigcup_{i\ge1}B_i) \\
&= \lim_{i\to\infty} \mu(B_i) \\
&\le \lim_{i\to\infty} \mu(A_i) \\
&= \lim_{i\to\infty} \mu(M_i) \\
&\le \lim_{i\to\infty} \mu(\bigcup_{i\ge 1} M_i).
\end{align*}
\item  Zu $M$ existiert ein $A\in\A$ mit $M\subset A$ und $\mu(M) = \mu(A)$. Sei $B\in \A_\mu$. Dann folgt
\begin{align*}
\mu(A) = \mu(M) &= \mu(M\cap B) + \mu(M\cap B^c) \\
&\le \mu(A\cap B) + \mu(M \cap B^c) \\
&\le \mu(A\cap B) + \mu(A \cap B^c) = \mu(A),
\end{align*}
woraus Gleichheit in obigern Ungleichungen folgt. Wegen $\mu(M)<\infty$ ist auch $\mu(M\cap B^c)<\infty$, und wir können dies von zwei obigen Termen abziehen und erhalten
\[
\mu(M\cap B) = \mu(A\cap B).
\]
\item  Zu $M_1$ existiert $\tilde A_1\in\A$ mit $M_1 \subset \tilde A_1$ und $\mu(M_1) = \mu(\tilde A_1)$. Wir definieren $A_1 \da \tilde A_1 \cap (M_1\cup M_2)$. Für diese Menge gilt nun $M_1\subset A_1 \subset M_1\cup M_2$. Wir folgern 
\[
\mu(M_1) \le \mu(A_1) \le \mu (\tilde A_1) = \mu(M_1)
\]
und wegen $M_1\cup M_2 = A_1\cup M_2$ weiter
\begin{align*}
\mu(A_1\cap M_2) + \mu(A_1 \cup M_2) &= \mu(A_1) + \mu(M_2) \\
&= \mu(M_1) + \mu(M_2) \\
&= \mu(M_1\cup M_2) \\
&= \mu(A_1 \cup M_2) < \infty,
\end{align*}
woraus $\mu(A_1\cap M_2) = 0$ folgt.

Nun ist $A_1\setminus M_1\subset A_1\cap M_2$, also gilt $\mu(A_1\setminus M_1) = 0$ und somit $A_1\setminus M_1\in \A_\mu$. Damit gilt dann $M_1 = A_1\cap (A_1\setminus M_1)^c\in \A_\mu$.
\end{enumerate}

\end{beweis}

\begin{satz}\label{satz:1.4}
Sei $\A$ eine $\sigma$-Algebra in $X$ und $\nu$ ein Maß auf $\A$. Dann wird durch
\[
\mu(M) \da \inf\left\{\nu(A) : A\in\A,\, M\subset A\right\}
\]
für $M\subset X$ ein $\A$-reguläres äußeres Maß auf $X$ erklärt mit $\mu|_{\A} = \nu$. Ist $M\in\A_\mu$ und $\mu(M)<\infty$, so existiert ein $A\in\A$ mit $M\subset A$ und $\mu(A\setminus M) = 0$.
\end{satz}

\begin{beweis}
Für $M\subset X$ sieht man leicht, dass
\begin{align*}
\mu(M) &= \inf\left\{\sum_{i=1}^\infty \nu(A_i) : A_i \in \A,\, i\in \MdN,\, M\subset \bigcup_{i=1}^\infty A_i\right\}.
\end{align*}
Also ist $\mu$ das von $(\A,\nu)$ induzierte äußere Maß. Da $\nu$ monoton ist und nach der Definition von $\mu$ ist $\mu|_\A = \nu$.

Um die $\A$-Regularität zu zeigen, nehmen wir ein $A\in\A$ und ein $M\subset X$. Für $B\in\A$ mit $M\subset B$ gilt:
\begin{align*}
\mu(M\cap A) + \mu(M\cap A^c) 
&\le \nu(B\cap A) + \nu(B\cap A^c) \\
&= \nu(B)
\end{align*}
und daher
\[
\mu(M\cap A) + \mu(M\cap A^c) \le \mu(M).
\]
also ist $A\in\A_\mu$. Sei nun $M\subset X$ beliebig und ohne Beschränkung der Allgemeinheit $\mu(M)<\infty$. Es existiert also eine Folge $A_i\in\A$, $i\in\MdN$ mit $M\subset A_i$ und $\nu(A_i) \to \mu(M)$. Setze $A\da \bigcap_{i\ge 1} A_i$. Dann gilt $A \in \A$, $M\subset A$, sowie
\begin{align*}
\mu(M) = \lim_{i\to\infty} \nu(A_i) \ge \nu(\bigcap_{i=1}^\infty A_i) = \mu(\bigcap _{i=1}^\infty A_i) = \mu(A) \ge \mu(M),
\end{align*}
woraus $\mu(M) = \mu(A)$ folgt.

Sei schließlich $M\in\A_\mu$ mit $\mu(M)<\infty$. Es gibt ein $A\in\A$ mit $M\subset A$ und $\mu(M) = \mu(A)<\infty$. Dann folgt
\begin{align*}
\infty > \mu(A) &= \mu(A\cap M) +\mu(A\cap M^c) \\
&= \mu(M) + \mu(A\cap M^c) \\
&= \mu(A) + \mu(A\setminus M).
\end{align*}
Wegen $\mu(A)=\mu(M)<\infty$ gilt also $\mu(A\setminus M) =0$.
\end{beweis}


\begin{anwendung}
Sei $\vartheta$ ein beliebiges äußeres Maß auf $X$. Dann ist $\vartheta|_{\A_\vartheta}$ ein Maß. Durch
\[
\mu(M) \da \inf\{\vartheta(A) : A\in\A_\vartheta,\, M\subset A\}
\]
wird also ein $\A_\vartheta$-reguläres äußeres Maß auf $X$ erklärt, das $\vartheta$ fortsetzt (also $\mu|_{\A_\vartheta} = \vartheta|_{\A_\vartheta}$).
\end{anwendung}

\begin{definition}
Seien $X,Y$ Mengen, $\mu$ ein äußeres Maß auf $X$ und $f:X\to Y$. Dann wird durch
\[
(f\mu)(M) \da \mu(f^{-1}(M))
\]
für $M\subset Y$ ein äußeres Maß $f\mu$ auf $Y$ erklärt. Man nennt $f\mu$ das \emph{Bild} von $\mu$ unter $f$ oder auch \emph{„push forward“} von $\mu$ bezüglich $f$ und schreibt hierfür auch $f_\#\mu$.
\end{definition}

\begin{bemerkung}
Für $B\subset Y$ gilt
\[
f^{-1}(B) \in \A_\mu \iff \forall M\subset X: B \in \A_{f(\mu \MR M)}.
\]
Seien hierzu $M\subset X$, $A,B\subset Y$. Dann gilt
\begin{align*}
&\phantom{=\ \ }\mu(M\cap f^{-1}(A) \cap f^{-1}(B)) + \mu(M\cap f^{-1}(A) \cap f^{-1}(B)^c) \\
&= (\mu \MR M)(f^{-1}(A\cap B)) + (\mu \MR M)(f^{-1}(A\cap B^c)) \\
&= f(\mu \MR M)(A\cap B) + f(\mu \MR M) (A\cap B^c).
\end{align*}
Insbesondere gilt: Ist $f^{-1}(A) \in \A_\mu$, so ist $A\in\A_{f(\mu)}$.
\end{bemerkung}

\begin{sprechweisen}
Sei $\mu$ ein äußeres Maß auf $X$. Eine Menge $N\subset X$ heißt $\mu$-Nullmenge, falls $\mu(N)=0$. Eine Eigenschaft $\mathcal E$ gilt für $\mu$-fast-alle $x\in X$ bzw. $\mu$-fast-überall, falls 
\[
\mu(\{x\in X : \mathcal E\text{ gilt für $x$ nicht}\}) = 0.
\]
Mit $\mathbb F_\mu(X,Y)$ wird die Menge aller Abbildungen $f:D\to Y$ bezeichnet mit $D\subset X$ und $\mu(X\setminus D) = 0$.
\end{sprechweisen}

\begin{definition}
Seien $X,Y$ Mengen und $\mu$ ein äußeres Maß auf $X$ und $\mathcal C$ eine $\sigma$-Algebra in $Y$. Dann heißt $f\in\mathbb F_\mu(X,Y)$ $\mu$-messbar bezüglich $\mathcal C$, falls $f^{-1}(\mathcal C)\subset \A_\mu$.
\end{definition}

Beachte, dass für $f:D\to Y$ mit $\mu(X\setminus D)=0$ gilt: $D=f^{-1}(Y)\in \A_\mu$.

\begin{lemma}
Seien $X,Y$ Mengen, $\mu$ ein äußeres Maß auf $X$ und $\mathcal E\subset \mathcal P(Y)$.  Für $f\in\mathbb F_\mu(X,Y)$ sind äquivalent:
\begin{enumerate}[a)]
\item $f^{-1}(\mathcal E) \subset \A_\mu$
\item $f$ ist $\mu$-messbar bezüglich $\sigma(\mathcal E)$.
\end{enumerate}
\end{lemma}

\begin{definition}
Ist $(X,\mathcal T)$ ein topologischer Raum, so nennt man die von den offenen Mengen erzeugte $\sigma$-Algebra $\sigma(\mathcal T)$ die Borelsche $\sigma$-Algebra des topologischen Raumes $(X,\mathcal T)$ mit der Notation $\borel(X)$.

Spezielle Borelsche Algebren sind $\borel(\MdR)$, $\borel(\MdR^n)$, $\borel(\bar\MdR) \da \{B\in \bar\MdR : B\cap \MdR \in \borel(\MdR)\}$.
\end{definition}

\begin{definition}
Sei $X$ eine Menge, $\mu$ ein äußeres Maß auf $X$ und $f\in\mathbb F_\mu(X,\bar\MdR)$. Man nennt $f$ eine $\mu$-messbare Abbildung, falls $f$ dies bezüglich $\borel(\bar\MdR)$ ist.
\end{definition}

Im Folgenden schreiben wir für eine Relation $\varrho$ auf $\bar\MdR$, Mengen $D, D'\subset X$ und Abbildungen $f: D\to \bar\MdR$, $g:D'\to\bar\MdR$:
\[
\{ f\mathrel{\varrho} g\} \da \{ x\in D\cap D' : f(x)\mathrel{\varrho} g(x) \}
\]

\begin{lemma}
Sei $\mu$ ein äußeres Maß auf $X$ und $f\in \mathbb F_\mu(X,\bar\MdR)$. Genau dann ist $f$ eine $\mu$-messbare Abbildung, wenn eine der folgenden Bedingungen für alle $a\in\MdR$ erfüllt ist:
\begin{align*}
 \{f > a\} \in \A_\mu, && \{f\ge a\} \in \A_\mu, && \{f<a\} \in \A_\mu, && \{f\le a\}\in\A_\mu.
\end{align*}
\end{lemma}

\begin{lemma}
Sei $\mu$ ein äußeres Maß auf $X$, seien $f,g,f_n\in \mathbb F_\mu(X,\bar\MdR)$, $n\in\MdN$, $\mu$-messbar. Dann gilt
\begin{enumerate}[\quad(a)]
\item $\{f<g\}$, $\{f\le g\}$, $\{f=g\}$, $\{f\ne g\}$ sind $\mu$-messbare Mengen.
\item Die Funktionen
\begin{align*}
&f+ g, && f-g, && f \cdot g \text{ (falls $\mu$-fast-überall definiert)}, \\
&\sup_n f_n, && \inf_n f_n, && \\
&f^+ \da \max\{f,0\}, && f^- \da -\min\{f,0\}, && |f|, \\
&\limsup_n f_n, && \liminf_n f_n
\end{align*}
sind $\mu$-messbar.
\end{enumerate}
\end{lemma}

\begin{satz}
\label{satz:1.8}
Ist $\mu$ ein äußeres Maß auf $X$, so ist $f\in \mathbb F_\mu(X,\bar\MdR)$ genau dann $\mu$-messbar, wenn für alle $M\subset X$, $a,b\in\MdR$ mit $a<b$ gilt
\[
\mu(M) \ge \mu(M\cap \{f \le a\}) + \mu(M\cap \{f\ge b\}).
\]
\end{satz}

\begin{beweis}
Sei $f$ zunächst $\mu$-messbar. Dann gilt mit $a<b$, $M\subset X$: 
\begin{align*}
\mu(M) &\ge \mu(M\cap\{f\le a\} ) + \mu(M\cap \{f> a\}) \\
&\ge \mu(M\cap\{f\le a\} ) + \mu(M\cap \{f\ge b\}) 
\end{align*}

Jetzt gelte die Bedingung des Satzes für alle $M\subset X$, $a<b$. Zu zeigen ist: $\{f\le r\}\in \A_\mu$ für beliebige $r\in\MdR$. Sei $M\subset X$ beliebig mit $\mu(M) <\infty$. Für $i\in \MdN$ sei
\[
A_i \da M\cap \{r + \frac1{i+1} \le f \le r+\frac 1i\}.
\]
Wir zeigen mit vollständiger Induktion, dass 
\[
\mu(\bigcup_{i=0}^n A_{2i+1})  \ge \sum_{0=1}^n \mu(A_{2i+1})
\]
gilt.

Für $n=0$ ist dies klar. Die Ungleichung gelte für ein $n\in\MdN$.
\begin{align*}
\mu(\bigcup_{i=0}^{n+1} A_{2i+1}) 
&\ge \mu(\bigcup_{i=0}^{n+1} A_{2i+1} \cap \{f\ge \underbrace{r+ \frac1{2n+2}}_{b}\}) + 
     \mu(\bigcup_{i=0}^{n+1} A_{2i+1} \cap \{f\le \underbrace{r+\frac1{2n+3}}_{a}\}) \\
&= \mu(\bigcup_{i=0}^n A_{2i+1}) + \mu(A_{2n+3}) \\
&\ge \sum_{i=0}^n \mu(A_{2i+1}) + \mu(A_{2n+3}) \\
&\ge \sum_{i=0}^{n+1} \mu(A_{2i+1})
\end{align*}

Analog zeigt man 
\[
\mu(\bigcup_{i=1}^n A_{2i})  \ge \sum_{i=1}^n \mu(A_{2i})
\]
und erhält zusammen
\[
\sum_{i=1}^\infty \mu(A_i) \le 2 \mu(M) <\infty.
\]

Sei $\ep>0$. Dann gibt es ein $n\in\MdN$ mit $\sum_{i\ge n}\mu(A_i) <\ep$. Zunächst schätzen wir ab
\begin{align*}
\mu(M\cap \{r < f < r+\frac1n\}) 
&\le \mu(M\cap \{r<f\le r + \frac1n\}) \\
&= \mu(M\cap \bigcup_{i=n}^\infty \{r +\frac 1{i+1} \le f \le r+\frac 1i\}) \\
&= \mu(\bigcup_{i=n}^\infty A_i) \\
&\le \sum_{i\ge n} \mu(A_i) < \ep
\end{align*}
und damit
\begin{align*}
&\phantom{=\ \ }\mu(M\cap \{f\le r\}) + \mu(M\cap \{f>r\})  \\
&\le \mu(M\cap \{f\le r\}) + \mu(M\cap \{r<f<r+\frac1n\}) + \mu(M\cap \{f\ge r+\frac 1n\}) \\
&\le \mu(M) + \ep.
\end{align*}
\end{beweis}

\begin{satz}
Seien $\mu$ ein äußeres Maß auf $X$, $f: X \to [0,\infty]$ eine $\mu$-messbare Abbildung und $(r_n)_{n\in\MdN}$ eine Folge in $[0,\infty)$ mit $\lim_{n\to\infty} r_n = 0$ und $\sum_{n=1}^\infty r_n = \infty$. Dann gibt es eine Folge $(A_n)_{n\in\MdN}$ $\mu$-messbarer Mengen mit
\[
f = \sum_{n\ge1} r_n \ind_{A_n}.
\]
\end{satz}

\begin{beweis}
Setze $A_1 \da \{ f \ge r_1 \}$ und allgemein $A_n \da \{f \ge r_n + \sum_{j=1}^{n-1} r_j \ind_{A_j}\}$, $n\ge 1$.

\textbf{Behauptung:} Es ist $f \ge \sum_{i=1}^n r_i \ind_{A_i}$, $n\in\MdN$. Dies gilt für $n=1$, und wenn es für ein $n\in\MdN$ gilt, dann folgt: Ist $x\notin A_{n+1}$, dann ist
\begin{align*}
f(x) \ge \sum_{i=1}^n r_i \ind_{A_i}(x) + \underbrace{r_{n+1}\ind_{A_{n+1}}(x)}_{=0}
\end{align*}
nach Induktionsvoraussetzung. Ist dagegen $x\in A_{n+1}$, so gilt nach der Definition von $A_{n+1}$
\begin{align*}
f(x) \ge \sum_{i=1}^n r_i \ind_{A_i}(x) + r_j = \sum_{i=1}^n r_i \ind_{A_i}(x) + r_j \underbrace{\ind_{A_{n+1}}(x)}_{=1}.
\end{align*}

Folglich ist $f\ge \sum_{i=1}^\infty r_i \ind_{A_i}$.

\textbf{Annahme:} Es gelte $f(x) > \sum_{i=1}^\infty r_i \ind_{A_i}(x)$ für ein $x\in X$.

Also ist $\sum_{i=1}^\infty r_i \ind_{A_i}(x)<\infty$. Da $\sum_{i=1}^\infty r_i = \infty$ gilt, muss es eine Folge natürlicher Zahlen $(j_k)_{k\in\MdN}$ geben mit $\ind_{A_{j_k}}(x)=0$ für alle $k\in\MdN$.
Wegen $\lim_{k\to\infty} r_{j_k} = 0$ gibt es ein $k\in\MdN$ mit
\begin{align*}
r_{j_k} < f(x) - \sum_{j=1}^\infty r_j \ind_{A_j}(x)
\end{align*}
und damit 
\begin{align*}
f(x) &> \sum_{j=1}^\infty r_j \ind_{A_j}(x) + r_{j_k} \\
&\ge \sum_{j=1}^{j_k-1} r_j \ind_{A_j}(x) + r_{j_k}.
\end{align*}
Das bedeutet $x\in A_{j_k}$ im Widerspruch zu $\ind_{A_{j_k}}(x) \ne 0$.
\end{beweis}

\section{Integration}

In diesem Abschnitt wird generell vorausgesetzt, dass $X$ eine Menge und $\mu$ ein äußeres Maß auf $X$ ist.

\begin{definition}
Eine $\mu$-Treppenfunktion auf $X$ ist eine $\mu$-messbare Abbildung $h\in \mathbb F_\mu(X,\MdR)$ mit abzählbarer Wertemenge $\im(h)$ und
\[
\sum_{\substack{r\in \im(h)\\r< 0}} r \cdot \mu(\{h=r\}) > -\infty \quad\text{ oder }\quad
\sum_{\substack{r\in \im(h)\\r> 0}} r \cdot \mu(\{h=r\}) < \infty.
\]
Ist $h$ eine $\mu$-Treppenfunktion auf $X$, so wird durch
\[
\int hd\mu = \sum_{r\in \im(h)} r \cdot \mu(\{h=r\})
\]
das $\mu$-Integral von $h$ erklärt, wobei „$0\cdot \infty\da 0$“ gelte.
\end{definition}

\begin{bemerkung}
\begin{enumerate}
\item  Es gilt
\[
\int h d\mu = \int h^+ d\mu - \int h^- d\mu.
\]
\item $h=\ind_A$, $A\in\A_\mu$ ist eine $\mu$-Treppenfunktion, $\int \ind_A d\mu=\mu(A)$.
\end{enumerate}
\end{bemerkung}

\begin{lemma}
\label{lem1.10}
Seien $h,g$ $\mu$-Treppenfunktionen auf $X$. Es gelte $\int h^+ d\mu <\infty$ und $\int g^+ d\mu<\infty$  oder $\int h^- d\mu <\infty$ und $\int g^- d\mu<\infty$. Dann ist $h+g$ eine $\mu$-Treppenfunktion und es gilt
\[
\int (h+g) d\mu = \int h d\mu + \int g d\mu.
\]
\end{lemma}

\begin{beweis}
Es gilt zunächst $h+g \in \mathbb F_\mu(X,\MdR)$. Zur Additivität: Wir definieren $A(r,s) \da \{h=r\} \cap \{g=s\}$ für $r,s\in\MdR$. Die Voraussetzungen des Lemmas stellen sicher, 
dass die nachfolgend vorgenommenen Vertauschungen der Summationsreihenfolge zulässig sind. Es gilt damit
\begin{align*}
\int hd\mu + \int gd\mu 
&= \sum_{r\in\im(h)} r\cdot \mu(\{h=r\}) + \sum_{s\in \im(g)} s \cdot \mu(\{g=s\}) \\
&= \sum_{r\in\im(h)} r \cdot \sum_{s\in \im(g)} \mu(A(r,s)) + \sum_{s\in\im(g)} s \cdot \sum_{r\in \im(h)} \mu(A(r,s)) \\
&= \sum_{\substack{r\in\im(h)\\s\in\im(g)}} (r+s)\cdot \mu(A(r,s)) \\
&= \sum_{t\in \im(g+h)} t \cdot \sum_{\substack{r\in \im(h) \\s\in\im(g)\\r+s=t}} \mu(A(r,s)) \\
&= \sum_{t\in \im(g+h)} t \cdot \mu\big(\bigcup_{\substack{r\in \im(h) \\s\in\im(g)\\r+s=t}} A(r,s)\big) \\
&= \sum_{t\in \im(g+h)} t \cdot \mu(\{g+h=t\}) \\
&= \int (h+g) d\mu .
\end{align*}
Übung: Zeige, dass $\int (g+h)^+ d\mu<\infty$ oder $\int (g+h)^- d\mu<\infty$ gilt.
\end{beweis}

\begin{bemerkung}
Sei $h$ eine $\mu$-Treppenfunktion mit $h\ge 0$, dann gilt $\int hd\mu \ge 0$. Mit Lemma \ref{lem1.10} folgt für $\mu$-Treppenfunktionen $h,g$:
\[
h \le g \implies \int h d\mu \le \int g d\mu
\]
\end{bemerkung}

\begin{definition}
Sei $f\in\mathbb F_\mu(X,\bar\MdR)$. Eine $\mu$-Oberfunktion (bzw. $\mu$-Unterfunktion) von $f$ ist eine $\mu$-Treppenfunktion $h$ auf $X$ mit $f\le h$ $\mu$-fast-überall auf $X$ (bzw. $h\le f$ $\mu$-fast-überall auf $X$).

Durch
\[
\int^* fd\mu \da \inf\left\{\int hd\mu : \text{$h$ ist eine $\mu$-Oberfunktion von $f$}\right\}
\]
wird das $\mu$-Oberintegral von $f$ erklärt.
Analog wird durch
\[
\int_* fd\mu \da \sup\left\{\int hd\mu : \text{$h$ ist eine $\mu$-Unterfunktion von $f$}\right\}
\]
das $\mu$-Unterintegral von $f$ erklärt.
\end{definition}

\begin{lemma}
\label{lem1.11}
Für $f,g\in\mathbb F_\mu(X,\bar\MdR)$ gelten die folgenden Aussagen:
\begin{enumerate}
\item $\int_* fd\mu = - \int^* (-f)d\mu$.
\item Gilt $\mu$-fast-überall $f\le g$, so ist $\int^* fd\mu \le \int^* gd\mu$ und $\int_* fd\mu \le \int_* gd\mu$.
\item Gilt $\mu$-fast-überall $f\ge 0$, so ist $\int^* fd\mu \ge 0$ und $\int_* fd\mu \ge 0$.
\item Gilt $\int^* fd\mu <\infty$, so auch $\int^*f^+d\mu<\infty$ und $f<\infty$ $\mu$-fast-überall.
\item Für $c\in(0,\infty)$ gilt $\int^*(cf)d\mu = c\cdot \int^*fd\mu$ und $\int_*(cf)d\mu = c\cdot \int_*fd\mu$.
\item Ist $\int^* fd\mu <\infty$ und $\int^*g d\mu<\infty$, so ist $f+g \in \mathbb F_\mu(X,\bar\MdR)$ und $$\int^*(f+g)d\mu \le \int^*fd\mu + \int^*gd\mu.$$

Analog gilt: Ist $\int_* fd\mu >-\infty$ und $\int_*g d\mu>-\infty$, so ist $f+g \in \mathbb F_\mu(X,\bar\MdR)$ und $$\int_*(f+g)d\mu \ge \int_*fd\mu + \int_*gd\mu.$$
\item $\int_* fd\mu \le \int^* fd\mu$.
\end{enumerate}
\end{lemma}

\begin{beweis} Übung
\end{beweis}

\begin{bemerkung}
Ist $h$ eine $\mu$-Treppenfunktion, so ist $h$ eine $\mu$-Oberfunktion und eine $\mu$-Unterfunktion von $h$. Das heißt insbesondere
\[
\int hd\mu \le \int_* h d\mu \le \int^* h d\mu \le \int h d\mu.
\]
\end{bemerkung}

\begin{definition}
Ist $f\in \mathbb F_\mu(X,\bar\MdR)$ eine $\mu$-messbare Abbildung und stimmt das $\mu$-Oberintegral mit dem $\mu$-Unterintegral von $f$ überein, so wird durch
\[
\int fd\mu \da \int^* fd\mu = \int_* fd\mu
\]
das $\mu$-Integral von $f$ erkärt. Man sagt in diesem Fall, dass das $\mu$-Integral von $f$ existiert. Ist $\int fd\mu \in \MdR$, so heißt $f$ $\mu$-integrierbar.
\end{definition}

\begin{satz}
\label{satz1.12}
Sei $f\in \mathbb F_\mu(X,\bar\MdR)$ nicht-negativ und $\mu$-messbar. Dann existiert das $\mu$-Integral von $f$. Es gilt $\int fd\mu \ge 0$ und 
\[
\int fd\mu = \sup\left\{ \int hd\mu : \text{$h$ ist $\mu$-Unterfunktion, $\im(h)$ ist endlich}\right\}.
\]
\end{satz}

\begin{beweis}
Ist $\mu(\{f=\infty\})>0$, dann ist für jedes $n\in\MdN$ die Funktion $n\cdot \ind_{\{f=\infty\}}$ eine $\mu$-Unterfunktion von $f$ und 
\[
\int^*fd\mu \ge \int_* fd\mu \ge \int n \cdot\ind_{\{f=\infty\}} d\mu = n \cdot \mu(\{f=\infty\})\to \infty\text{ für }n\to\infty.
\]
Also ist $\int^* fd\mu = \int_* fd\mu = \infty$.

Sei jetzt $f<\infty$ $\mu$-fast-überall. Für $t\in(1,\infty)$ sei
\[
U_t \da \sum_{n\in\MdZ} t^n \cdot \ind_{\{t^n \le f < t^{n+1}\}}. 
\]
Offenbar ist
$$
U_t\le f\le t U_t
$$
$\mu$-fast-überall, d.h.\ 
$U_t$ ist eine $\mu$-Unterfunktion von $f$, $tU_t$ eine $\mu$-Oberfunktion von $f$. Damit gilt
\begin{align*}
\int^* fd\mu \le \int t U_t d\mu = t \cdot \int U_t d\mu \le t \int_* f d \mu.
\end{align*}
Ist $\int_* fd\mu <\infty$, dann folgt $\int^* fd\mu \le \int_* fd\mu$ aus $t\to 1$. Ist dagegen $\int_* fd\mu = \infty$, so ist $\int^* fd\mu = \int_* fd\mu = \infty$.
\end{beweis}



\begin{satz}
\label{satz1.13}
Seien $f,g\in\mathbb F_\mu(X,\bar\MdR)$ $\mu$-messbar. Dann gilt:
\begin{enumerate}
\item Sei $c\in \MdR\setminus\{0\}$. Es existiert $\int fd\mu$ genau dann, wenn $\int(cf)d\mu$ existiert. In diesem Fall ist 
\[
\int (cf) d\mu = c \cdot \int fd\mu.
\]
\item Angenommen, es existieren $\int fd \mu$ und $\int gf\mu$ und $(\int fd\mu, \int gd\mu) \ne (\pm \infty, \mp \infty)$. Dann ist $f+g\in \mathbb F_\mu(X,\bar\MdR)$, $\int(f+g)d\mu$ existiert und
\[
\int (f+g)d\mu = \int fd\mu + \int gd\mu.
\]
\item Ist $f\le g$ $\mu$-messbar und existiert $\int gd\mu$ (bzw. $\int fd\mu)$, so existiert auch $\int fd\mu$ (bzw. $\int gd\mu$), und es gilt in jedem Fall
\[
\int fd\mu \le \int gd\mu.
\]
\end{enumerate}
\end{satz}

\begin{beweis}
\begin{enumerate}
\item Es existiere $\int fd\mu$. Sei $c>0$. Dann folgt
\begin{align*}
\int^* (cf)d\mu 
&= c\cdot \int^*fd\mu = c\cdot \int_* fd\mu = \int_* (cf)d\mu.
\end{align*}
Sei $c<0$. Dann folgt
\begin{multline*}
\int^* (cf)d\mu 
= \int^*(-c)(-f)d\mu
= (-c)\cdot \int^*(-f)d\mu
= (-c) \cdot (-1) \int_* fd\mu 
\\
= c\cdot \int fd\mu 
= c\int^* fd\mu 
= (-c)\int_*(-f) d\mu
= \int_*(-c)(-f) d\mu
= \int_* (cf) d\mu .
\end{multline*}
\item Seien $f,g$ $\mu$-integrierbar. Dann ist $f,g<\infty$ $\mu$-fast-überall, und somit ist $f+g\in \mathbb F_\mu(X,\bar\MdR)$. Ferner gilt
\begin{multline*}
\int fd\mu + \int gd\mu 
= \int^* fd\mu + \int^* gd\mu
\ge \int^*(f+g)d\mu \\
\ge \int_*(f+g)d\mu
\ge \int_* fd\mu + \int_* gd\mu
= \int fd\mu + \int gd\mu,
\end{multline*}
woraus die Aussage folgt.

Sei nun $\int fd\mu = \infty$. Nach Voraussetzung gilt dann $\int gd\mu >-\infty$ und damit $g>-\infty$ $\mu$-fast-überall. Das heißt $f+g\in\mathbb F_\mu(X,\bar\MdR)$. Ferner gilt
\begin{align*}
\infty \ge \int^*(f+g)d\mu \ge \int_*(f+g)d\mu \ge \int_*f d\mu + \int_* gd\mu = \infty.
\end{align*}
Analog kann der Fall $\int fd\mu=-\infty$ gezeigt werden.
\item Sei $\int gd\mu <\infty$, das heißt $f\le g <\infty$ $\mu$-fast-überall. Dann ist 
$(f-g)\ind_{\{g>-\infty\}}\in F_\mu(X,\bar\MdR)$ nicht positiv. Wegen $f\le g <\infty$ $\mu$-fast-überall 
ist $g+(f-g)\ind_{\{g>-\infty\}}=f$ und damit ergibt (2)
\begin{align*}
\int gd\mu \ge \int gd\mu + \int (f-g)\ind_{\{g>-\infty\}}d\mu = \int fd\mu.
\end{align*}
% Was ist mit \int_ fd\mu = -\infty?
\end{enumerate}
\end{beweis}

\begin{satz}
Sei $f\in\mathbb F_\mu(X,\bar\MdR)$ $\mu$-messbar. Dann gilt:
\begin{enumerate}
\item $\int f^+ d\mu$, $\int f^- d\mu$ existieren stets. Es existiert $\int fd\mu$ genau dann, wenn $\int f^+d \mu <\infty$ oder $\int f^- d\mu<\infty$. In diesem Fall ist 
\[
\int fd\mu = \int f^+d\mu -\int f^-d\mu.
\]
Ferner ist
\[
\Big|\int fd\mu\Big| \le \int |f|d\mu .
\]
\item Ist $f$ $\mu$-integrierbar, so auch $|f|$. 
\end{enumerate}
\end{satz}

\begin{beweis}
\begin{enumerate}
\item Es existiere $\int fd\mu$. Ist $\int fd\mu <\infty$, so ist $\int f^+ d\mu <\infty$ nach Lemma \ref{lem1.11} (4). Ist dagegen $\int fd\mu =\infty$, so ist $\int (-f)d\mu = -\infty$ und daher $\int f^- d\mu = \int (-f)^+d\mu <\infty$.

Umgekehrt sei $\int f^+d\mu <\infty$ oder $\int f^- d\mu <\infty$. Dann existiert nach Satz \ref{satz1.13} (2) das Integral $\int (f^+-f^-)d\mu$ wegen Satz \ref{satz1.13} (1) ist $\int fd\mu = \int f^+d\mu - \int f^-d\mu$.

Stets existiert das Integral von $|f|$ und 
\begin{align*}
\int |f|d\mu = \int (f^+ + f^-) d\mu = \int f^+ d\mu + \int f^- d\mu 
\ge \Big|\int f^+ d\mu - \int f^-d\mu \Big| = \Big|\int fd\mu \Big |.
\end{align*}
\item Ist $f$ $\mu$-integrierbar, so folgt aus (1), dass $\int f^+d \mu <\infty$ \emph{und} $\int f^-d \mu<\infty$.
\end{enumerate}
\end{beweis}

\begin{satz}[Lemma von Fatou]
\label{satz1.15}
Sei $f_n \in \mathbb F_n(X,\bar\MdR)$, $n\in\MdN$, mit $f_n\ge 0$ und $\mu$-messbar. Dann gilt
\begin{align*}
\int \liminf_{n\to\infty} f_n d\mu \le \liminf_{n\to\infty} \int f_n d\mu.
\end{align*}
\end{satz}

\begin{beweis}
Sei $\varepsilon \in (0,1)$. Sei $h$ eine $\mu$-Unterfunktion von $\liminf_{n\to\infty} f_n$ mit $\im(h) = \{r_1,\ldots,r_m\}\subset [0,\infty)$ (vergleiche Satz \ref{satz1.12}). Für $i=1,\ldots,m$ und $n\in\MdN$ sei 
\begin{align*}
A_{i,n}\da \{h=r_i\} \cap \{\inf_{k\ge n} f_k \ge \ep\cdot r_i\} \in A_\mu.
\end{align*}
Es gilt $A_{i,n}\subset A_{i,n+1}$ für $i=1,\ldots,m$ und $n\in\mathbb{N}$. Für $\mu$-fast-alle $x\in X$ mit $h(x) = r_i$ gilt:
\begin{align*}
\ep r_i < r_i \le \liminf_{n\to\infty} f_n(x) = \sup_{n\in\MdN} \inf_{k\ge n} f_k(x).
\end{align*}
Es gibt ein $n\in\MdN$ mit $\ep r_i < \inf_{k\ge n} f_k(x),$ das heißt $x\in A_{i,n}$. Also
\begin{align*}
\mu(\{h=r_i\}) = \mu(\bigcup_{n\in\MdN} A_{i,n}).
\end{align*}

Die Mengen $A_{i,n}$, $i=1,\ldots,m$, sind paarweise disjunkt und aus ihrer Definition folgt, dass
\begin{align*}
\sum_{i=1}^m \ep r_i \ind_{A_{i,n}}
\end{align*}
eine $\mu$-Unterfunktion von $f_n$ ist.

Hiermit gilt
\begin{align*}
\liminf_{n\to\infty} \int f_n d\mu 
&\ge \liminf_{n\to\infty} \sum_{i=1}^m \ep r_i \mu(A_{i,n})\\
&= \sum_{i=1}^m \ep r_i \liminf_{n\to\infty} \mu(A_{i,n})\\
&= \sum_{i=1}^m \ep r_i \mu( \bigcup_{n\in\MdN} A_{i,n})\\
&= \ep \sum_{i=1}^m r_i \mu(\{h=r_i\}) \\
&= \ep \int hd\mu.
\end{align*}
Es folgt
\begin{align*}
\int \liminf_{n\to\infty} f_n d\mu \le \frac1\ep \liminf_{n\to\infty} \int f_n d\mu
\end{align*}
für ein beliebiges $\ep \in (0,1)$. Lässt man $\ep$ gegen $1$ gehen, so folgt die Behauptung.
\end{beweis}

\begin{satz}
\label{satz1.16}
Ist $f_n\in\mathbb F_\mu(X,\bar\MdR)$ $\mu$-messbar, $n\in\MdN$, $0\le f_1\le f_2\le \cdots$, so gilt
\begin{align*}
\lim_{n\to\infty} \int f_n d\mu = \int \lim_{n\to\infty} f_n  d\mu.
\end{align*}
\end{satz}

\begin{beweis}
Grenzwerte und Integrale existieren offenbar. Es gilt 
\begin{align*}
\lim_{k\to\infty}\int f_k d\mu 
&\le \int \lim_{n\to\infty} f_n d\mu \\
&\le \limsup_{n\to\infty} \int f_n d\mu. \tag{nach Satz \ref{satz1.15}}
\end{align*}
\end{beweis}

\begin{satz}[Lebesgue]
\label{satz1.17}
Sei $f_n\in\mathbb F_\mu(X,\bar\MdR)$, $n\in\MdN$ eine konvergente Folge $\mu$-messbarer Funktionen. Es existiere eine $\mu$-integrierbare Funktion $g\in\mathbb F_\mu(X,\bar\MdR)$ mit $|f_n|\le g$ $\mu$-fast-überall für jedes $n\in\MdN$. Dann sind $f_n$ und $f\da \lim_{n\to\infty} f_n$ $\mu$-integrierbar und 
\begin{align*}
\lim_{n\to\infty} \int f_n d\mu = \int f d\mu.
\end{align*}
Schärfer gilt sogar
\begin{align*}
\lim_{n\to\infty} \int |f-f_n| d\mu = 0.
\end{align*}
\end{satz}

\begin{beweis}
Wegen $|f_n|,|f| \le g$ ist $\int |f_n| d\mu <\infty$ und $\int |f|d\mu <\infty$. Die Folge $(2g-|f_n-f|)_{n\in\MdN}$ nichtnegativer Funktionen in $\mathbb F_\mu(X,\bar\MdR)$ konvergiert für $n\to\infty$ $\mu$-fast-überall gegen $2g$. Mit dem Lemma von Fatou (Satz \ref{satz1.15}) folgt:
\begin{multline*}
\int 2g d\mu - \limsup_{n\to\infty} \int |f_n -f|d\mu
= \liminf_{n\to\infty} \int (2g - |f_n -f|) d\mu \\
\ge \int \underbrace{\liminf_{n\to\infty}(2g - |f_n -f|)}_{=2g} d\mu = \int 2gd\mu
\end{multline*}
also 
\begin{align*}
\lim_{n\to\infty} \int |f_n -f| d\mu = 0.
\end{align*}
\end{beweis}

\begin{notation}
Für $A\subset X$ und $f\in\mathbb F_\mu (X,\bar\MdR)$ sei
\begin{align*}
\int^*_A fd\mu \da \int^* \ind_{A}f d\mu && \text{und} &&
\int_{*A} fd\mu \da \int_* \ind_{A}f d\mu.
\end{align*}
Existiert das $\mu$-Integral von $\ind_A \cdot f$, so setzt man
\begin{align*}
\int_A f d\mu \da \int \ind_A fd\mu.
\end{align*}
\end{notation}

\begin{lemma}
\label{lem1.18}
(Übungsblatt 2, Aufgabe 4) Sei $f_n\in\mathbb F_\mu(X,\bar\MdR)$, $n\in\MdN$, eine Folge nichtnegativer Funktionen. Dann gilt
\begin{align*}
\int^* \sum_{n=1}^\infty {f_n} d\mu \le \sum_{n=1}^\infty \int^* f_n d\mu.
\end{align*}
\end{lemma}

\begin{satz}
\label{satz1.19}
Sei $g\in \mathbb F_\mu(X,\bar\MdR)$ eine nichtnegative Funktion. Dann wird  durch
\begin{align*}
\psi(A) \da \int_A^* gd\mu, \  A\subset X,
\end{align*}
 ein äußeres Maß auf $X$ definiert. Es gilt $\A_\mu \subseteq \A_\psi$. Ist $g$ sogar $\mu$-messbar und $f\in\mathbb F_\mu(X,\bar\MdR)$ $\mu$-messbar, dann existiert $\int fd\psi$ genau dann, wenn $\int fgd\mu$ existiert. In diesem Fall gilt
\begin{align*}
\int fd\psi = \int fg d\mu.
\end{align*}
\end{satz}

\begin{beweis}
Es gilt $\psi(\emptyset)=0$. Die Subsigmaadditivität folgt direkt aus Lemma \ref{lem1.18}.

Sei $A\in\A_\mu$ und $M\subset X$ mit $\psi(M)<\infty$. Sei $\ep>0$ beliebig. Dann existiert eine $\mu$-Oberfunktion $h$ zu $\ind_M\cdot g$ mit $h\ge 0$ und
\[
\int h d\mu \le \int^* \ind_M g d \mu + \ep.
\]
Dann ist $\ind_A\cdot h$ eine $\mu$-Oberfunktion zu $\ind_{M\cap A}\cdot g$ und $\ind_{A^c}\cdot h$ ist eine $\mu$-Oberfunktion zu $\ind_{M\cap A^c}\cdot g$. Daher folgt
\begin{align*}
\psi(M\cap A)+\psi(M\cap A^c)
&= \int^* \ind_{M\cap A} gd\mu + \int^* \ind_{M\cap A^c} gd\mu \\
&\le \int \ind_A \cdot h d\mu + \int \ind_{A^c} \cdot h d\mu \\
&= \int h d\mu \le \int^* \ind_M g d\mu + \ep = \psi(M) + \ep.
\end{align*}

Sei nun $g$ $\mu$-messbar und $f\in \mathbb F_\mu(X,\bar\MdR)$ und sei zunächst $f\ge 0$. Also existieren $\int fd\psi$ und $\int gfd\mu$. Es gibt eine Folge $(r_j)_{j\in\MdN}$ in $(0,\infty)$ und $(A_j)_{j\in\MdN}$ in $A_\mu$ mit $f = \sum_{j=1}^\infty r_j\ind_{A_j}$. Zweimalige Anwendung des 
Satzes von der monotonen Konvergenz (Satz \ref{satz1.16}) ergibt 
\begin{align*}
\int fd\psi 
&= \int \sum_{j=1}^\infty r_j \ind_{A_j} d\psi \\
&= \sum_{j=1}^\infty r_j \int \ind_{A_j} d\psi \\
&= \sum_{j=1}^\infty r_j \psi(A_j) \\
&= \sum_{j=1}^\infty r_j \int \ind_{A_j} gd\mu \\
&= \int \left( \sum_{j=1}^\infty r_j \ind_{A_j}\right) g d\mu \\
&= \int fgd\mu.
\end{align*}
Sei nun $f$ eine beliebige $\mu$-messbare Funktion. Wegen $(fg)^\pm = f^\pm \cdot g$ gilt $\int f^\pm d\psi < \infty$ genau dann, wenn $\int (f g)^{\pm} d\mu<\infty$. Somit existiert $\int f d\psi$ genau dann, wenn $\int fg d\mu$ existiert und
\begin{multline*}
\int fd\psi = \int f^+ d\psi - \int f^- d\psi = \int f^+gd\mu - \int f^-gd\mu = \int (f^+ - f^-)g d\mu = \int fgd\mu.
\end{multline*}
\end{beweis}

\begin{satz}
Sei $f\in \mathbb F_\mu(X,\bar\MdR)$ $\mu$-integrierbar. Dann gibt es zu jedem $\ep >0$ ein $\delta > 0$ derart, dass für alle $A\in \A_\mu$ mit $\mu(A)<\delta$ gilt
\[
\int_A |f|d\mu <\ep.
\]
\end{satz}

\begin{beweis}
Betrachte $g_n\da \min\{|f|,n\}$, $n\in\MdN$. Es gilt $0\le g_n\nearrow |f|$ für $n\to\infty$. Damit ist
\begin{align*}
\lim_{n\to \infty} \int g_n d\mu = \int |f| d\mu <\infty.
\end{align*}
Zu einem vorgegebenem $\ep>0$ gibt es ein $N\in\mathbb{N}$ mit 
\begin{align*}
0\le \int |f|d\mu - \int g_N d\mu < \frac\ep2.
\end{align*}
Für $A\in\A_\mu$ mit $\mu(A) < \frac\ep{2N} \ad \delta$ folgt nun
\begin{align*}
\int_A |f| d\mu
&= \int_A\underbrace{(|f|-g_N)}_{\ge 0} d\mu + \int _A g_N d\mu \\
&\le \int\underbrace{(|f|-g_N)}_{\ge 0} d\mu + N \cdot \mu(A) \\
&< \frac\ep2 + \frac\ep2 = \ep.
\end{align*}
\end{beweis}

Seien $X,Y\neq\emptyset$ Mengen mit äußeren Maßen $\mu$ auf $X$, $\nu$ auf $Y$. Durch
\[
(\mu \times \nu)(M) \da \inf\{ \sum_{i=1}^\infty \mu(A_i)\nu(B_i): A_i\in \A_\mu, B_i\in \A_\nu,\, i\in\MdN,\, M\subset \bigcup_{i=1}^\infty (A_i \times B_i)\}
\]
wird ein äußeres Maß auf $X\times Y$ erklärt, nämlich das von $\mathcal E_0 \da \{A \times B : A\in\A_\mu,\, B\in\A_\nu\}$ und $\lambda$ mit $\lambda(A\times B) \da \mu(A)\cdot \nu(B)$ für $A\in\A_\mu$, $B\in\A_\nu$ induzierte äußere Maß.

\begin{satz}
Seien $X,Y\ne \emptyset$ Mengen mit äußeren Maßen $\mu$ auf $X$ und $\nu$ auf $Y$. Dann gelten folgende Aussagen:
\begin{enumerate}
\item $\A_\mu \otimes \A_\nu \subset \A_{\mu\times \nu}$ und $(\mu\times \nu)(A\times B) = \mu(A) \cdot \nu(B)$ für $A\in\A_\mu$, $B\in\A_\nu$.
\item Das Maß $\mu\times\nu$ ist $\A_\mu \otimes \A_\nu$-regulär.
\item Existiert das $(\mu \times \nu)$-Integral von $f\in\mathbb F_{\mu\times\nu}(X\times Y,\bar\MdR)$ und gilt $\{f\ne 0\}\subset \bigcup_{i=1}^\infty M_i$, $M_i\in\A_{\mu\times\nu}$, $(\mu\times\nu)(M_i)<\infty$, $i\in\MdN$, so ist:

$f(\cdot,y) \in\mathbb F_\mu(X,\bar\MdR)$ ist $\mu$-messbar für $\nu$-fast-alle $y\in Y$. Es ist $\int f(x,y)\mu(dx)$ $\nu$-messbar und $\iint f(x,y)\mu(dx)\nu(dy)$ existiert (und symmetrisch in $x$ und $y$) und schließlich:
\[
\int f d(\mu\times\nu) = \iint f(x,y)\mu(dx)\nu(dy) = \iint f(x,y) \nu(dy) \mu(dx).
\]
\end{enumerate}
\end{satz}

\begin{beweis}
Wir setzen
\begin{multline*}
\mathcal E \da \{ M\subset X\times Y: \ind_M(\cdot,y) \in \mathbb F_\mu(X,\bar\MdR) \text{ ist $\mu$-messbar für $\nu$-fast-alle $y\in Y$,} \\
\int\ind_M(x,y) \mu(dx) \in \mathbb F_\nu(Y,\bar\MdR)\text{ ist $\nu$-messbar}\}.
\end{multline*}
Für $M\in\mathcal E$ sei
\[
\varrho(M) \da \iint \ind_M (x,y)\mu(dx)\nu(dy).
\]
Wir zeigen zwei Hilfsbehauptungen:
\begin{enumerate}
\item[($\alpha$)] Ist $M_j\in\mathcal E$, $j\in\MdN$, eine Folge paarweise disjunkter Mengen, so ist $\bigcup_{j=1}^\infty M_j\in\mathcal E$, denn:

\[
\ind_{\bigcup_{j=1}^\infty M_j}(\cdot, y)=\sum_{j=1}^\infty \ind_{M_j}(\cdot, y)
\]
ist $\mu$-messbar für $\nu$-fast-alle $y\in Y$ und \[
\int\ind_{\bigcup_{j=1}^\infty M_j} (x,y)\mu(dx) = \sum_{j=1}^\infty \int \ind_{M_j}(x,y) \mu(dx)
\]
ist $\nu$-messbar.

\item[($\beta$)] Ist $M_j\in\mathcal E$, $j\in\MdN$, $M_1\supset M_2 \supset \cdots$ sowie $\varrho(M_1) <\infty$, so gilt $\bigcap_{j\ge 1}M_j \in\mathcal E$, denn:

\[
\ind_{\bigcap_{j=1}^\infty M_j}(\cdot, y) = \lim_{j\to\infty}\ind_{M_j}(\cdot,y)
\]
ist $\mu$-messbar für $\nu$-fast-alle $y\in Y$ und 
\[
\int\ind_{\bigcap_{j=1}^\infty M_j} (x,y)\mu(dx) = \lim_{j\to\infty} \int \ind_{M_j} (x,y)\mu(dx)
\]
ist $\nu$-messbar.
\end{enumerate}

Betrachte nun folgende Mengensysteme:
\begin{align*}
\mathcal E_0 &\da \{A\times B: A\in\A_\mu,\, B\in\A_\nu\}, \\
\mathcal E_1 &\da \{\bigcup_{i=1}^\infty G_i: G_i\in\mathcal E_0\}, \\
\mathcal E_2 &\da \{\bigcap_{j\ge 1}^\infty H_j: H_j\in\mathcal E_1\}.
\end{align*}

Für $A\times B\in\mathcal E_0$ ist $\ind_{A\times B}(\cdot, y) = \ind_A \cdot \ind_B(y)$ $\mu$-messbar für alle $y\in Y$ und $\int \ind_{A\times B}(x,y) \mu(dx) = \mu(A) \cdot \ind_B(y)$ ist $\nu$-messbar. Also ist $A\times B\in \mathcal E$, und damit $\mathcal E_0\subset\mathcal E$.

Für $A\times B\in \mathcal E_0$, $C\times D\in\mathcal E_0$ ist 
\begin{align*}
(A\times B)\cap (C\times D) &= (A\cap C) \times (B\cap D) \in \mathcal E_0
\intertext{und}
(A\times B)\setminus (C\times D) &= \big(\underbrace{(A\setminus C) \times B}_{\in\mathcal E_0}\big) \stackrel{\bullet}\cup \big(\underbrace{(A\cap C) \times (B\setminus D)}_{\in\mathcal E_0}\big).
\end{align*}
Jede abzählbare Vereinigung von Mengen aus $\mathcal E_0$ kann als abzählbare Vereinigung von paarweise disjunkten Mengen aus $\mathcal E_0$ erhalten werden, das heißt $\mathcal E_1\subset \mathcal E$ nach ($\alpha$).

Da $\mathcal{E}_1$ stabil bezüglich der Bildung endlicher Durchschnitte ist, folgt mit Hilfe von $(\beta)$
\[
\{\bigcap_{i=1}^\infty H_i : H_i\in\mathcal E_1,\, i\in\MdN,\, \varrho(H_1)<\infty\} \subset \mathcal E.
\]

\textbf{Behauptung:} Für $M\subset X\times Y$ gilt 
\[
(\mu\times\nu)(M) = \inf\{\varrho(V):M\subset V, V\in \mathcal E_1\}
\]
und es gibt zu $M$ ein $W\in\mathcal E_2$ mit $M\subset W$ und $(\mu\times\nu)(M) = (\mu\times\nu)(W) = \varrho (W)$.

\textbf{Nachweis:} Für $i\in\mathbb{N}$ sei $A_i\times B_i  \in \mathcal E_0$ mit $M\subset \bigcup_{i=1}^\infty (A_i\times B_i)\ad V \in\mathcal E_1$. Dann gilt 
$$\ind_V \le \sum_{i=1}^\infty \ind_{A_i\times B_i},
$$ 
wobei Gleichheit gilt, falls die Mengen $A_i\times B_i$ paarweise disjunkten sind. Somit erhält man 
$$\varrho(V) \le \sum_{i=1}^\infty \varrho(A_i\times B_i) = \sum_{i=1}^\infty \mu(A_i) \nu(B_i),
$$ 
wobei auch hier Gleichheit gilt, falls die Mengen $A_i\times B_i$, $i\in\mathbb{N}$, paarweise disjunkt sind.

Der erste Teil der Behauptung folgt somit aus
\begin{align*}
(\mu\times\nu)(M) &= \inf\{\sum_{i=1}^\infty \mu(A_i)\nu(B_i) : A_i\times B_i \in \mathcal E_0, i\in\MdN, M\subset\bigcup_{i=1}^\infty(A_i\times B_i)\} \\
&= \inf\{\varrho(V): M\subset V, V\in\mathcal E_1\}.
\end{align*}

Ist $(\mu\times \nu)(M)<\infty$, so existieren $V_i\in\mathcal E_1$, $i\in\MdN$, $M\subset V_i$ mit
\[
\lim_{i\to\infty} \varrho(V_i) = (\mu\times\nu)(M).
\]
Setze $M\subset W \da \bigcap_{i=1}^\infty V_i \in\mathcal E_2$. Es gilt
\[
(\mu\times\nu)(M) \le (\mu\times \nu)(W) \le \lim_{i\to\infty}\varrho(V_i) = \varrho(W) = (\mu\times\nu)(M).
\]
Ist $(\mu\times\nu)(M) =\infty$, so setze $W\da X\times Y\in\mathcal E_2$.

Nun beweisen wir die eigentlichen Aussagen des Satzes:
\begin{enumerate}
\item Sei $A\times B\in\mathcal E_0$. Zunächst gilt offenbar

Für ein beliebiges $V\in\mathcal E_1$ mit $A\times B\subset V$ gilt
\[
(\mu\times\nu)(A\times B) =\inf\{\varrho(V):A\times B\subset V,V\in\mathcal{E}_1\}= \varrho(A\times B) = \mu(A) \nu(B).
\]
Für $T\subset X\times Y$ und $U\in \mathcal E_1$ mit $T\subset U$ Sind
$U\cap (A\times B)$ und $U\cap (A\times B)^c$ disjunkte Mengen in $\mathcal E_1$. Wir  erhalten so
\begin{multline*}
(\mu\times \nu) (T\cap (A\times B)) + (\mu\times\nu)(T\cap (A\times B)^c)
\\ \le \varrho(U\cap (A\times B)) + \varrho(U\cap (A\times B)^c) = \varrho(U).
\end{multline*}
Bildet man das Infimum über alle $U\in\mathcal E_1$ mit $U\supset T$, so ergibt diese Ungleichung
\[
(\mu\times \nu) (T\cap (A\times B)) + (\mu\times\nu)(T\cap (A\times B)^c) \le (\mu\times \nu)(T),
\]
woraus $A\times B\in\A_{\mu\times\nu}$ folgt.
\item Ist $M\subset X\times Y$ und $(\mu\times\nu)(M)<\infty$, so gibt es $W\in\mathcal E_2$ mit $\varrho(W)<\infty$ und mit der gewünschten Eigenschaft $(\mu\times\nu)(M) = (\mu\times\nu)(W)$.
\item Sei $f=\ind_M$, $M\in\A_{\mu\times\nu}$ und $(\mu\times\nu)(M)<\infty$. Zu $M$ existiert ein $W\in\mathcal E_2$ mit $M\subset W$ und $(\mu\times\nu)(M) = (\mu\times\nu)(W) =\varrho(W)$.

Fall 1: $(\mu\times\nu)(M) = 0$. Dann gilt $\varrho(W)=0$ und $\ind_M(\cdot,y)=0$ $\mu$-fast-überall für $\nu$-fast-alle $y\in Y$. Insbesondere ist $M\in\mathcal E$ und $\varrho(M)=0$.

Fall 2: $(\mu\times\nu)(M) > 0$. Dann gilt $(\mu\times\nu)(W\setminus M) = 0$, $M\subset W$. Fall 1 liefert $W\setminus M\in\mathcal E$ und $\varrho(W\setminus M)=0$. Also ist $\ind_M(\cdot, y) = (\ind_W - \ind_{W\setminus M})(\cdot,y)$ $\mu$-messbar für $\nu$-fast-alle $y\in Y$ und  $\ind_M(\cdot,y) = \ind_W(\cdot,y)$ $\mu$-fast-überall für $\nu$-fast-alle $y\in Y$. Insbesondere ist $M\in\mathcal E$ und $\varrho(M) = \varrho(W) = (\mu\times\nu)(M)$.
\end{enumerate}
\end{beweis}

\chapter{Äußere Maße auf metrischen Räumen}

\section{Regularität und metrische äußere Maße}

Sei $(X,d)$ ein metrischer Raum mit der von $d$ induzierten Topologie $\mathcal T$. Sei $\borel(X)$ die Borel-$\sigma$-Algebra. Ist $\mu$ ein äußeres Maß auf $X$ und sind alle offenen (alle abgeschlossenen) Mengen in $\A_\mu$, so gilt $\borel(X)\subset \A_\mu$. Wir nennen $\mu$ Borel-regulär, wenn $\mu$ $\borel(X)$-regulär ist, das heißt $\borel(X)\subset \A_\mu$ und zu jedem $M\subset X$ existiert ein $B\in\borel(X)$ mit $M\subset B$ und $\mu(M)=\mu(B)$.


\begin{lemma}
\label{lem:2.1}
Sei $(X,d)$ ein metrischer Raum und $\A\subset \mathcal P(X)$. Für jede Folge $(A_i)_{i\in\MdN}$ in $\A$ gelte $\bigcup_{i=1}^\infty A_i \in \A$ und $\bigcap_{i=1}^\infty A_i \in \A$. Enthält $\A$ ferner alle offenen Mengen oder alle abgeschlossenen Mengen, so gilt $\borel(X)\subset \A$.
\end{lemma}

\begin{beweis}
Übungsblatt 2, Aufgabe 2 (a).
\end{beweis}

\begin{satz}
\label{satz:2.2}
Sei $\mu$ ein äußeres Maß auf dem metrischen Raum $(X,d)$ mit $\borel(X)\subset\A_\mu$ und $A\in\A_\mu$. Dann gilt:
\begin{enumerate}
\item Ist $\mu$ Borel-regulär und $\mu(A)<\infty$, so existieren $B_1,B_2\in\borel(X)$ mit $B_2\subset A\subset B_1$ und $\mu(B_1\setminus B_2)=0$.
\item Gilt $\mu(A)<\infty$ und ist $A\in \borel(X)$ oder $\mu$ Borel-regulär, so existiert zu jedem $\ep>0$ eine abgeschlossene Menge $C\subset A$ mit $\mu(A\setminus C)<\ep$. In diesem Fall gilt
\[
\mu(A) = \sup\{\mu(C): C\subset A,\, C\text{ abgeschlossen}\}.
\]
\item Gibt es eine abzählbare, offene Überdeckung $(U_i)_{i\in\MdN}$ von $A$ mit $\mu(U_i)<\infty$ für alle $i\in\MdN$ und ist $A\in\borel(X)$ oder $\mu$ Borel-regulär, so existiert zu jedem $\ep>0$ eine offene Menge $U\supset A$ mit $\mu(U\setminus A)<\ep$. In diesem Fall gilt
\[
\mu(A) = \inf\{\mu(U): U\supset A,\, U\text{ offen}\}.
\]
\end{enumerate}
\end{satz}

\begin{beweis}
Teil (1) und (2): Übungsblatt 2, Aufgabe 2. Zu (3):

Seien $A$ und $(U_i)_{i\in\MdN}$ wie vorausgesetzt. Sei $\ep>0$. Zu jedem $i\in\MdN$ existiert nach (2) eine abgeschlossene Menge $C_i\subset U_i\setminus A$ mit $\mu( (U_i\setminus A)\setminus C_i)< \frac\ep{2^i}$. Setze $U\da \bigcup_{i=1}^\infty (U_i \setminus C_i)$. Dann ist $U$ offen und $A = \bigcup_{i=1}^\infty (A\cap U_i) \subset \bigcup_{i=1}^\infty (U_i\setminus C_i) = U$. Ferner ist
\begin{align*}
\mu(U\setminus A) 
= \mu(\bigcup_{i=1}^\infty (U_i\setminus C_i) \setminus A)
\le \sum_{i=1}^\infty \mu( (U_i\setminus C_i) \setminus A) 
\le \sum_{i=1}^\infty \mu( (U_i\setminus A) \setminus C_i) 
\le \sum_{i=1}^\infty \frac \ep{2^i} = \ep.
\end{align*}
\end{beweis}

\begin{definition}
Seien $X$ eine Menge, $(Y,\mathcal T)$ ein topologischer Raum, $\mu$ ein äußeres Maß auf $X$ und $f\in\mathbb F_\mu(X,Y)$. Man nennt $f$ $\mu$-messbar, falls $f$ $\mu$-messbar bezüglich $\borel(Y)$ ist, das heißt $f^{-1}(\borel(Y))\subset \A_\mu$.
\end{definition}

Für einen metrischen Raum $(X,d)$, für Mengen $A,B\subset X$ und $x\in X$ sei 
$$
d(x,A) \da \inf\{d(x,y): y\in A\},\qquad d(A,B)\da \inf\{d(x,y),x\in A,\, y\in B\}.
$$

\begin{satz}
\label{satz:2.3}
Seien $X$ eine Menge, $(Y,d)$ ein metrischer Raum, $\mu$ ein äußeres Maß auf $X$ sowie $f\in\mathbb F_\mu(X,Y)$. Dann sind äquivalent:
\begin{enumerate}
\item $f$ ist $\mu$-messbar.
\item Für jede Menge $M\subset X$ und alle Mengen $A,B\subset Y$ mit $d(A,B)>0$ gilt
\[
\mu(M) \ge \mu(M\cap f^{-1}(A)) + \mu(M \cap f^{-1}(B)).
\]
\end{enumerate}
\end{satz}

\begin{beweis}
(1)$\implies$(2): Sei $f$ $\mu$-messbar. Seien $M\subset X$, $A,B\subset Y$ mit $d(A,B)>0$. Es gibt $U_A\supset A$, $U_A$ offen mit $U_A\cap B=\emptyset$. Somit gilt $f^{-1}(B) \subset f^{-1}(U_A^c) = f^{-1}(U_A)^c$ und $f^{-1}(A)\subset f^{-1}(U_A)$. Nach Voraussetzung ist $f^{-1}(U_A)\in\A_\mu$. Es folgt
\begin{align*}
\mu(A) \ge \mu(M\cap f^{-1}(U_A)) + \mu (M\cap f^{-1}(U_A)^c)
\ge \mu(M\cap f^{-1}(U_A)) + \mu(M\cap f^{-1}(B)).
\end{align*}

(2)$\implies$(1): Zeige $f^{-1}(C) \in \A_\mu$ für eine beliebige abgeschlossene Menge $C\subset Y$. Für $y\in Y$ sei $g(y) \da d(y,C)$. Ohne Beschränkung der Allgemeinheit sei $C\ne \emptyset$. Für $a,b\in \MdR$ mit $a<b$ gilt 
\[
d(\underbrace{\{g\le a\}}_{\ad A}, \underbrace{\{g\ge b\}}_{\ad B})  \ge b - a > 0.
\]
Für $M\subset X$ erhält man nun
\begin{align*}
\mu(M) \ge \mu(M\cap f^{-1}(A)) + \mu(M\cap f^{-1}B(B)) 
= \mu(M \cap \{g\circ f\le a\}) + \mu(M\cap \{g\circ f \ge b\}).
\end{align*}
Da $g\circ f : X\to \MdR$ die Bedingung von Satz \ref{satz:1.8} erfüllt, folgt die Messbarkeit von $g\circ f$, das heißt insbesondere ist $\{g \circ f = 0\}\in \A_\mu$. Ferner ist 
\begin{align*}
f^{-1}(C) = \{x\in X: d(f(x),C)=0\} = \{g\circ f= 0\} \in \A_\mu,
\end{align*}
wobei die Abgeschlossenheit von $C$ verwendet wurde.
\end{beweis}

\begin{satz}[Kriterium von Carathéodory]
\label{satz:2.4}
Für ein äußeres Maß $\mu$ auf $(X,d)$ sind äquivalent:
\begin{enumerate}
\item $\borel(X) \subset \A_\mu$.
\item Für $A,B\subset X$ mit $d(A,B)>0$ gilt $\mu(A\cup B) \ge \mu(A) + \mu(B)$.
\end{enumerate}
\end{satz}

\begin{beweis}
(1)$\implies$(2): Ist $\borel(X)\subset \A_\mu$, so ist $\id_X: X\to X$ $\mu$-messbar. Für Mengen $A,B\subset X$ mit $d(A,B)>0$ folgt nun aus Satz \ref{satz:2.3}, (1)$\implies$(2), dass 
\begin{align*}
\mu(A\cup B) \ge \mu( \underbrace{(A\cup B) \cap \id_X^{-1}(A)}_{=A} ) + \mu( \underbrace{(A\cup B) \cap \id_X^{-1}(B)}_{=B}) = \mu(A) + \mu(B).
\end{align*}

(2)$\implies$(1): Seien $M\subset X$, $A,B\subset X$ mit $d(A,B)>0$. Dann ist auch $d(M\cap A, M\cap B)>0$ und daher
\begin{align*}
\mu(M)
&\ge \mu( M\cap (A\cup B) )\\
&= \mu( (M\cap A)\cup (M\cap B) )\\
&\ge \mu(M\cap A) + \mu(M\cap B)\\
&= \mu(M \cap \id_X^{-1}(A)) + \mu(M \cap \id_X^{-1}(B)).
\end{align*}
Nach Satz \ref{satz:2.3}, (2)$\implies$(1), ergibt dies die $\mu$-Messbarkeit von $\id_X$, das heißt $\borel(X)\subset \A_\mu$.
\end{beweis}

\section{Vitali-Relationen}

In dem metrischen Raum $(X,d)$ sei 
\begin{multline*}
\mathbb M(X) \da \{\mu: \text{$\mu$ ist ein Borel-reguläres Maß auf $X$ und}\\\text{$\mu(A)<\infty$ für alle beschänkten Mengen $A\subset X$}\}.
\end{multline*}

\begin{motivation}
Seien $a,b\in\MdR$, $a<b$ und $f:[a,b]\to \MdR$ monoton wachsend. Für $x\in[a,b)$ sei 
\[
\bar L_f(x) \da \limsup_{h\searrow0} \frac{f(x+h) - f(x)}{h}.
\]
Mit $\lambda$ wird das äußere Lebesguemaß auf $\MdR$ bezeichnet, das heißt
\[
\lambda(M) \da\inf\{\sum_{i=1}^\infty (b_i-a_i): a_i < b_i, M\subset \bigcup_{i=1}^\infty [a_i, b_i]\}.
\]
\end{motivation}

\textbf{Behauptung:} Für $\lambda$-fast-alle $x\in[a,b]$ ist $\bar L_f(x)<\infty$.

Zum Nachweis sei
\[
\yen  \da \{[x,x+h]: x\in[a,b],\, 0<h<b-x\}.
\]
Für $t>1$ sei
\[
\yen_t \da \{[x,x+h] \in \yen: f(x+h) - f(x) \ge t\cdot h\}.
\]
Wir wollen zeigen, dass die Menge
\[
A\da \{x\in [a,b): \limsup_{h\searrow0} \frac{f(x+h)-f(x)}h = \infty\}
\]
Lebesguemaß Null hat.

Für $x\in A$, $t>1$ gilt:
\[
\inf\{h>0: [x,x+h ] \in \yen_t\} = 0.
\]
Falls es gelingt, eine disjunkte Folge $([x_i, x_i+h_i])_{i\in\MdN}$ in $\yen_t$ zu finden mit
\[
\lambda( A \setminus  \underbrace{\bigcup_{i=1}^\infty [x_i, x_i+h_i]}_{\ad S}) = 0,
\]
dann folgt
\begin{multline*}
\lambda(A) = \lambda(A\cap S) + \lambda(A\cap S^c) = \lambda(A\cap S)\\
\le \lambda(S) = \sum_{i=1}^\infty h_i < \sum_{i=1}^\infty \frac{f(x_i+h_i)- f(x_i)}t \le \frac{f(b)-f(a)}t.
\end{multline*}
Da $t>1$ beliebig ist, folgt $\lambda(A) = 0$.

Wichtig war hierbei:
\begin{enumerate}
\item Die betrachteten Intervalle sind beliebig kurz.
\item Bis auf eine Nullmenge lässt sich eine abzählbare disjunkte Teilüberdeckung finden.
\item Oft genügt es auch, den Grad der Mehrfachüberlappung beschränken zu können.
\end{enumerate}

\begin{definition}
Sei $\mu$ ein äußeres Maß auf $(X,d)$, $M\subset X$, $\A\subset \mathcal P(X)$. Man nennt $\A$ eine (offene, abgeschlossene, Borelsche) $\mu$-Überdeckung von $M$ wenn $\mu(M\setminus \bigcup_{A\in\A} A) = 0$ und $\A$ aus (offenen, abgeschlossenen, Borelschen) Mengen besteht.
\end{definition}

\begin{definition}
Sei $\mu\in \mathbb M(X)$. Eine $\mu$-Vitali-Relation $V$ auf $X$ ist eine Teilmenge des Mengensystems $\{(x,S): x\in S, S\in \borel(X)\}$ mit folgenden Eigenschaften:
\begin{enumerate}
\item $\inf\{\diam(S): (x,S)\in V\}=0$ für alle $x\in X$.
\item Für $Z\subset X$, $C\subset V$ mit $\inf\{\diam(S): (z,S)\in C\}=0$ für alle $z\in Z$ enthält die Familie $C(Z) \da \{S: (z,S)\in C \text{ für ein $z\in Z$}\}$ eine abzählbare disjunkte $\mu$-Überdeckung von $Z$.
\end{enumerate}
\end{definition}

Im Folgenden werden Methoden und Kriterien zum Nachweis von Eigenschaft (2) vorgestellt und  Vitali-Relationen bei der Differentation von Maßen verwendet.

\begin{beispiel}
Seien $X=\MdR$, $V\da \{(x,[x,x+h]): [x,x+h] \in \yen\}$. Dann ist (1) erfüllt. Später wird (2) für $\mu=\lambda$ gezeigt, so dass $V$ eine $\lambda$-Vitali-Relation ist. Wähle $Z = A$ wie oben, $C\da \{(x,[x,x+h]): [x,x+h] \in \yen_t\}$. Dann ist die Voraussetzung für $Z$ aus (2) erfüllt und es existiert eine abzählbare disjunkte $\lambda$-Überdeckung von $Z$, das heißt es existiert $([x_i,x_i+h_i])_{i\in\MdN}$ paarweise disjunkt mit $[x_i,x_i+h_i]\in \yen_t$ und $\lambda(A\setminus\bigcup_{i=1}^\infty [x_i,x_i+h_i])=0$.
\end{beispiel}

Rahmensituationen, in denen (2) verifiziert werden kann:
\begin{enumerate}
\item Der metrische Raum $(X,d)$ ist allgemein, $\mu$ erfüllt eine diametrische Regularitätsbedingung.
\item Der Raum $(X,d)$ ist spezieller, geeignet ist etwa $(\MdR^n,\|\cdot\|)$, $\mu$ ist ein beliebiges Radonmaß.
\end{enumerate}

\begin{satz}
\label{satz:2.5}
Sei $\mathcal S\subset\mathcal P(X)$ mit $\sup\{\diam(\mathcal S): S\in \mathcal S\}<\infty$, $1<t<\infty$ und $\emptyset\notin \mathcal S$. Für $T\in\mathcal S$ sei $\mathcal S_T^t \da\{S\in \mathcal S: S\cap T \ne \emptyset,\, \diam(S) \le t \cdot \diam(T)\}$. Dann existiert eine disjunkte Familie $\mathcal T\subset \mathcal S$ mit $\mathcal S = \bigcup_{T\in\mathcal T} \mathcal S_T^t$.
\end{satz}

\begin{beweis}
Definiere
\begin{multline*}
\Omega \da \{\mathcal R\subset \mathcal S: \mathcal R \text{ ist eine disjunkte Familie von Mengen, für alle $S\in\mathcal S$ gilt:}\\
S\cap \bigcup_{R\in\mathcal R}R = \emptyset \text{ oder }
S\in \mathcal S_R^t \text{ für ein } R\in\mathcal R\}.
\end{multline*}
Dann ist $(\Omega,\subset)$ eine geordnete Menge. Nach Voraussetzung gibt es ein $T\in \mathcal S$ mit $\sup\{\diam(S): S\in \mathcal S\} < t \cdot \diam (T)$. Daher ist $\{T\}\in\Omega$.

\textbf{Behauptung:} Jede linear geordnete Teilmenge $\Lambda \subset \Omega$ hat eine obere Schranke.

\textbf{Nachweis:} Die obere Schranke ist $\bigcup_{\mathcal R\in \Lambda}\mathcal R$. Dies ist eine disjunkte Familie: Wenn $R,R'\in \bigcup_{\mathcal R\in\Lambda}\mathcal R$, so ist $R\in \mathcal R$, $R'\in \mathcal R'$ mit $\mathcal R,\mathcal R'\in\Lambda$. Ohne Beschränkung der Allgemeinheit sei $\mathcal R'\supset \mathcal R$, also $R,R'\in \mathcal R'$. Da aber $\mathcal R'$ eine disjunkte Familie ist, ist $R\cap R'=\emptyset$ oder $R=R'$. Sei $S\in \mathcal S$. Ist nun $S\cap(\bigcup_{\mathcal{R}\in\Lambda} \bigcup_{R\in\mathcal R} R) \ne \emptyset$, so ist für ein $\mathcal R'\in\Lambda$ schon $S\cap \bigcup_{R\in\mathcal R'} R \ne \emptyset$. Wegen $\mathcal R'\in\Omega$ ist dann $S\in\mathcal S_{R'}^t$ für ein $R'\in\mathcal R'$ und damit auch $S\in\mathcal S_R^t$ für ein $R\in\bigcup_{\mathcal R\in\Lambda}\bigcup_{R\in\mathcal R}\mathcal S_R^t$. Folglich ist $\bigcup_{\mathcal R\in\Lambda}\mathcal R\in \Omega$.

Aufgrund des Zornschen Lemmas existiert ein maximales Element $\mathcal T\in\Omega$. Sei
\[
\mathcal K\da \{S\in\mathcal S: S\cap \bigcup_{T\in\mathcal T}T = \emptyset\}.
\]
Angenommen, $K\ne \emptyset$. Dann existiert ein $K\in\mathcal K$ mit $\sup\{\diam(S):S\in\mathcal K\}<t\cdot\diam(K)$. Es ist $\mathcal T\cup\{\mathcal K\}$ eine disjunkte Familie. Ist schließlich $S\in\mathcal S$ und $S\cap \bigcup\{T: T\in\mathcal T\cup\{\mathcal K\}\} \ne \emptyset$. Dann gilt $S\cap \bigcup_{T\in\mathcal T}T \ne \emptyset$, und somit $S\in \mathcal S_T^t$ für ein $T\in\mathcal T$, oder $S\cap K\ne\emptyset$ und $S\in\mathcal K$, $\diam(S)<t\cdot \diam(K)$, so dass $S\in\mathcal S_K^t$. In jedem Fall ist $\mathcal T\cup\{\mathcal K\}\in\Omega$ im Widerspruch zur Maximalität von $\mathcal T$.

Folglich ist $\mathcal K=\emptyset$. Für jedes $S\in\mathcal S$ existiert also ein $T\in\mathcal T$ mit $S\cap T\ne\emptyset$. Da $\mathcal T\in\Omega$, folgt $S\in\mathcal S_T^t$ für ein $T\in\mathcal T$, also ist $\mathcal S=\bigcup_{T\in\mathcal T}\mathcal S_T^t$.
\end{beweis}

Dieser Satz ist ein entscheidentes Hilfsmittel im Beweis des folgenden Satzes.

\begin{satz}
\label{satz:2.6}
Seien $\mu\in\mathbb M(X)$, $A\subset X$, $A\ne \emptyset$ sowie $t>1$ und $r>0$. Sei $\mathcal S$ eine abgeschlossene Überdeckung von $A$ mit
\begin{enumerate}
\item $\sup\{\diam(S): S\in\mathcal S\}<\infty$,
\item $\inf\{\diam(S): S\in\mathcal S,\, x\in S\}=0$ für alle $x\in A$,
\item $\mu(\bigcup_{S\in\mathcal S_B^t} S) < r\cdot \mu(B)$ für alle $B\in\mathcal S$.
\end{enumerate}
Dann gibt es zu jeder offenen Menge $U\subset X$ eine abzählbare, disjunkte $\mu$-Überdeckung $\mathcal C$ von $A\cap U$ mit $\mathcal C\subset\mathcal S$ und $\bigcup_{C\in\mathcal C}C\subset U$.
\end{satz}

\begin{beweis}
Sei zunächst $A$ beschränkt. Sei $U\subset X$ offen. Sei ohne Beschränkung der Allgemeinheit $A\cap U\ne \emptyset$ und $U$ beschränkt. Wir betrachten $\mathcal U \da \{S\in\mathcal S: \emptyset \ne S\subset U\}$. Nach Satz \ref{satz:2.5} existiert eine disjunkte Familie $\mathcal C\subset \mathcal U$ mit der Eigenschaft $\mathcal U = \bigcup_{C\in\mathcal C} \mathcal U_C^t \subset \bigcup_{C\in\mathcal C}\mathcal S_C^t$. Wegen $\mu\in\mathbb M(X)$ ist $\mu(U)<\infty$ und wegen (3) ist $\mu(B) >0$ für $B\in\mathcal C$. Da $\bigcup_{C\in\mathcal C} C \subset \mathcal U$ und $\mathcal C$ disjunkt, folgt die Abzählbarkeit von $\mathcal C$.

Weiterhin gilt
\[
\sum_{C\in\mathcal C} \mu(\bigcup_{C'\in\mathcal S_C^t} C') < \sum_{C\in\mathcal C} r\cdot \mu(C) \le r\cdot \mu(U) <\infty.
\]
Somit gibt es zu jedem $\ep>0$ eine endliche Menge $\mathcal D \subset \mathcal C$ mit
\[
\sum_{C\in\mathcal C\setminus\mathcal D} \mu(\bigcup_{C'\in\mathcal S_C^t} C') < \ep.
\]
Sei nun $x\in A\cap U \setminus \bigcup_{D\in\mathcal D} D$. Da $\bigcup_{D\in\mathcal D}D$ abgeschlossen ist und nach (2) gibt es ein $S\in\mathcal S$ mit $S\cap \bigcup_{D\in\mathcal D}D =\emptyset$, $x\in S\subset U$. Insbesondere ist $S\in \mathcal U \subset \bigcup_{C\in\mathcal C}\mathcal S_C^t$. Dann gilt aber $S\in \bigcup_{C\in\mathcal C\setminus\mathcal D} \mathcal S_C^t$. Nun kann man abschätzen:
\begin{align*}
\mu(A\cap U\setminus \bigcup_{C\in\mathcal C}C)
&\le \mu(A\cap U \setminus\bigcup_{C\in\mathcal D} D) \\
&\le \mu(\bigcup_{C\in\mathcal C \setminus\mathcal D} \bigcup_{C'\in\mathcal S_C^t} C') \\
&\le \sum_{C\in\mathcal C\setminus\mathcal D} \mu(\bigcup_{C'\in\mathcal S_C^t} C') < \ep .
\end{align*}
Da $\ep>0$ beliebig ist folgt $\mu(A\cap U\setminus \bigcup_{C\in\mathcal C}C)=0$.

Sei nun $A$ beliebig. Wähle $a\in A$. Induktiv werden beschränkte Mengen $A_j\subset X$, offene Mengen $W_j\subset X$ und endliche disjunkte Folgen $\mathcal C_j\subset\mathcal S$ abgeschlossener Mengen konstruiert. Setze hierzu $W_0\da U$, $A_0\da \emptyset$, $\mathcal C_0\da \emptyset$ und falls $W_{j-1}$, $A_{j-1}$ und $\mathcal C_{j-1}$ für $j>0$ schon definiert sind, so sei $W_j \da W_{j-1} \setminus \bigcup_{C\in\mathcal C_{j-1}}C$, $A_j \da \{x\in A: d(a,x)\le j\}$. Da $A_j$ beschränkt ist und $W_{j-1}\setminus \bigcup_{C\in\mathcal C_{j-1}}C = W_j$ offen, gibt es eine endliche disjunkte Teilfamilie $\mathcal C_j\subset \mathcal S$ mit $\bigcup_{C\in\mathcal C_j} C \subset W_j$ und $\mu(W_j\cap A_j \setminus \bigcup_{C\in\mathcal C_j} C) <\frac1{2^j}$. Die Mengen $W_j = \mathcal U\setminus \bigcup_{i=1}^{j-1} \bigcup_{C\in\mathcal C_i} C$, $j\in\MdN$, sind absteigend, $(A_j)_{j\in\MdN}$ sind aufsteigend, $\mathcal C\da \bigcup_{j\ge 1}\mathcal C_j$ ist nach Konstruktion eine abzählbare disjunkte Mengenfolge in $\mathcal S$. Ferner ist $U\setminus \bigcup_{C\in\mathcal C}C\subset W_i$ für jedes $i\in\MdN$. Für $k\in\MdN$ beliebig gibt also:
\begin{align*}
\mu(U\cap A \setminus\bigcup_{C\in\mathcal C} C)
&= \mu(\bigcup_{j=1}^\infty ( (U\setminus \bigcup_{C\in\mathcal C}C)\cap A_j )) \\
&= \mu(\bigcup_{j=k}^\infty ( (U\setminus \bigcup_{C\in\mathcal C}C)\cap A_j )) \\
&= \mu(\bigcup_{j=k}^\infty (W_{j+1} \cap A_j) )\\
&\le \mu(\bigcup_{j=k}^\infty (W_j \cap A_j \setminus \bigcup_{C\in\mathcal C_j}C)) \\
&\le \sum_{j=k}^\infty \frac1{2^j}.
\end{align*}
Da $k\in\MdN$ beliebig, folgt $\mu(U\cap A \setminus\bigcup_{C\in\mathcal C} C) = 0$.
\end{beweis}

\begin{beispiel}
Sei $V=\{(x,[x,x+h]): x\in [a,b),\, 0<h<b-x\}$. Dann gilt:
\begin{enumerate}
\item $\inf\{\diam([x,x+h]) : 0 < h < b-x\}=0$ für alle $x\in[a,b)$.
\item Seien $Z\subset [a,b)$, $C\subset V$ mit $\inf\{\diam(S): (z,S) \in C\}=0$ für alle $z\in Z$.

Betrachte $C(Z) \da \{S: (z,S)\in C\text{ für ein }z\in Z\}$.

Wir werden Satz \ref{satz:2.6} verwenden und weisen die folgenden Voraussetzungen nach:
\begin{enumerate}
\item $\sup\{\diam(S): S\in C(Z)\} \le b-a < \infty$.
\item $\inf\{\diam(S): S\in C(Z),\, x\in S\} = \inf\{\diam(S) : (z,S) \in C \text{ für ein $z\in Z$}, x\in S\} \le \inf\{\diam(S): (x,S)\in C\} = 0$ für alle $x\in Z$.
\item Sei $B \da [x,x+h] \in C(Z) \ad \mathcal S$. Dann ist $\mathcal S$ eine abgeschlossene Überdeckung von $Z$. Ferner: $[y,y+\bar h]\in\mathcal S_B^t$ impliziert $[y,y+\bar h] \cap [x,x+h] \ne \emptyset$ und $\diam([y,y+\bar h]) \le t\cdot \diam([x,x+h])$, das heißt $\bar h \le h$, also gilt $[y,y+\bar h] \subset [x-th,x+h+th]$ und damit $\lambda(\bigcup_{S\in\mathcal S_B^t}S) \le \lambda([x-th,(1+t)h]) = (1+2t)h = (1+2t)\lambda(B)$.
\end{enumerate}
Nach Satz \ref{satz:2.6} existiert eine abzählbare disjunkte $\lambda$-Überdeckung von $Z$ mit Mengen aus $\mathcal S = C(Z)$. 
\end{enumerate}
Somit ist $V$ eine $\lambda$-Vitali-Relation.
\end{beispiel}

\begin{definition}
Eine metrische abgeschlossene Kugel in $(X,d)$ ist $\mathbb B(x,r) := \{z\in X: d(z,x) \le r\}$ für $r>0$.
\end{definition}

\begin{korollar}
Seien $\mu \in \mathcal M(X)$, $A\subset X$, $r>0$, $t>1$. Sei $\mathcal S$ eine Überdeckung von $A$ durch abgeschlossene Kugeln mit
\begin{enumerate}
\item $\sup\{\diam(S): S\in\mathcal S\} <\infty$,
\item $\inf\{\diam(\mathbb B(y,s)): y\in A, x\in\mathbb B(y,s)\in \mathcal S\} = 0$ für alle $x\in A$,
\item für alle $x\in A$ und alle $\mathbb B(x,s)\in\mathcal S$ gilt: $\mu(\mathbb B(x,(1+2t)s)) < r \cdot \mu(\mathbb B(x,s))$.
\end{enumerate}
Dann gibt es zu jeder offenen Menge $U\subset X$ eine abzählbare disjunkte $\mu$-Überdeckung $\mathcal C\subset \mathcal S$ von $U\cap A$, mit $C\subset U$ für alle $C\in\mathcal C$ (wobei alle $C\in\mathcal C$ ihren Mittelpunkt in $A$ haben).
\end{korollar}

\begin{beweis}
Sei $\mathcal S' \da \{\mathbb B(x,s)\in \mathcal S: x\in A\}$. Dann ist $\mathcal S'$ eine abgeschlossene Überdeckung von $A$ wegen (2). Bedingung (1) in Satz \ref{satz:2.6} folgt aus der Voraussetzung (1) hier, die Bedingung (2) in Satz \ref{satz:2.6} folgt aus der Voraussetzung (2) hier. Wir weisen die Bedingung (3) in Satz \ref{satz:2.6} nach. Hierzu sei $\mathbb B(x,s)\in\mathcal S'$, das heißt $x\in A$.

\textbf{Behauptung:} \[
\bigcup_{S\in\mathcal S'^t_{\mathrlap{\mathbb B(x,s)}}} S \subset \mathbb B(x,(1+2t)s)
\]

\textbf{Nachweis:}
Sei $S=\mathbb B(y,\varrho)\in \mathcal S'^t_{\mathbb B(x,s)}$. Dann ist $\mathbb B(y,\varrho) \cap \mathbb B(x,s) \ne \emptyset$ und außerdem gilt $\diam(\mathbb B(y,\varrho)) \le t \cdot \diam(\mathbb B(x,s))$. Sei $\bar z\in \mathbb B(y,\varrho)$ und $z_0\in\mathbb B(y,\varrho)\cap \mathbb B(x,s)$. Dann gilt 
$$d(\bar z, x)\le d(\bar z, z_0) + d(z_0,x)\le \diam(\mathbb B(y,\varrho)) + s \le t \cdot 2s +s = (1+2t)s.
$$ 
Hiermit:
\[
\mu(\bigcup_{S\in\mathcal S'^t_{\mathrlap{\mathbb B(x,s)}}} S )
\le \mu ( \mathbb B(x,(1+2t)s) 
< r\cdot \mu(\mathbb B(x,s)).
\]
Die Aussage des Korollars folgt nun mit Satz \ref{satz:2.6}.
\end{beweis}

\begin{bemerkungen}
\item Die Bedingung (2) ist insbesondere dann erfüllt, wenn $\inf\{\diam(\mathbb B(s,x) ): \mathbb B(x,s)\in\mathcal S\} = 0$ für alle $x\in A$. Falls jetzt $\mu$ eine diametrische Regularitätsbedingung erfüllt, dann ist $V \da \{(x,\mathbb B(x,s)): x\in X,\, s>0\}$ eine $\mu$-Vitali-Relation.

In Satz \ref{satz:2.8} werden wir einen Überdeckungssatz formulieren, mit dessen Hilfe zum Beispiel für $X=\MdR^n$, $d=\|\cdot\|$, gezeigt werden kann dass $V$ für jedes (!) Maß $\mu\in\mathbb M(\MdR^n)$ eine $\mu$-Vitali-Relation ist.

\item Ist $\mu\in\mathbb M(X)$ und gibt es $0<u<v$, $n\in\MdN$ mit
\[
0 < u < \frac{\mu(\mathbb B(x,s))}{s^n} < v
\]
für alle $a\in X$, $s>0$, so folgt 
\begin{align*}
\mu(\mathbb B(x,(2t+1)s))
&\le s^n \cdot v \cdot (2t+1)^n\\
&= s^n \cdot u\cdot \frac vu \cdot (2t+1)^n \\
&< \mu(\mathbb B(x,s)) \cdot \underbrace{\frac vu \cdot (2t+1)^n}_{\ad r} \\
&= r \cdot \mu(\mathbb B(x,s)).
\end{align*}
\end{bemerkungen}

\begin{satz}[Besicovitch]
\label{satz:2.8}
Zu $n\in\MdN$ existiert ein $N_n\in\MdN$ so dass gilt: Ist $\mathcal F$ eine Familie abgeschlossener Kugeln mit positiven Radien in $\MdR^n$ und $\sup\{\diam(B):B\in\mathcal F\}<\infty$ und ist $A$ die Menge der Mittelpunkte dieser Kugeln, dann gibt es $\mathcal G_1,\ldots,\mathcal G_{N_n}\subset \mathcal F$ derart, dass $\mathcal G_i$ eine abzählbare disjunkte Teilfamilie von $\mathcal F$ ist und
\[
A\subset \bigcup_{i=1}^{N_n} \bigcup_{B\in\mathcal G_i}B.
\]
\end{satz}

\begin{beweis} 
Übung bzw. Literatur (Evans \& Gariepy, Mattila, allgemeiner Federer)
\end{beweis}

\begin{korollar}
\label{kor:2.9}
Seien $\mu$ ein Borel-reguläres Maß auf $\MdR^n$ und $\mathcal F$ eine Familie abgeschlossener Bälle mit positiven Radien in $\MdR^n$. Sei $A$ die Menge der Mittelpunkte dieser Bälle. Sei $\mu(A)<\infty$ und $\inf\{r>0: \mathbb B(a,r)\in\mathcal F\}=0$ für alle $a\in A$. Dann gibt es zu jeder offenen Menge $ U \subset\MdR^n$ eine abzählbare disjunkte Teilfamilie $\mathcal G\subset\mathcal F$, so dass $B\subset U$ für alle $B\in \mathcal G$ und $\mu(A\cap U \setminus \bigcup_{B\in\mathcal G}B)=0$, das heißt, $\mathcal G$ ist eine $\mu$-Überdeckung von $A\cap U$.
\end{korollar}

\begin{beweis}
Wähle $\theta \in (1-\frac1{N_n}, 1)$.

\textbf{Behauptung:} Es gibt $M_1\in\MdN$ und disjunkte Bälle $B_1,\ldots,B_{M_1}\in\mathcal F$ mit $B_i \subset  U$ und 
$$\mu(A\cap U\setminus \bigcup_{i=1}^{M_1} B_i) \le \theta \cdot \mu(A\cap U).
$$


\textbf{Nachweis:} Sei $\mathcal F_1 \da \{B\in\mathcal F : \diam(B) \le 1,\, B\subset U\}$. Die Menge der Mittelpunkte von Bällen aus $\mathcal F_1$ ist gerade $A\cap U$. Wegen Satz \ref{satz:2.8} gibt es zu $\mathcal F_1$ Teilfamilien $\mathcal G_1,\ldots,\mathcal G_{N_n}\subset\mathcal F_1$, wobei $\mathcal G_i$ abzählbar viele disjunkte Bälle enthält mit $A\cap U\subset A\cap U \cap \bigcup_{i=1}^{N_n}\bigcup_{B\in\mathcal G_i}B$. Hieraus folgt
\begin{align*}
\mu(A\cap U)
\le \mu(A\cap U \cap \bigcup_{i=1}^{N_n} \bigcup_{B\in\mathcal G_i} B) 
\le \sum_{i=1}^{N_n}\mu(A\cap U \cap \bigcup_{B\in\mathcal G_i} B).
\end{align*}
Es gibt ein $i\in\{1,\ldots,N_n\}$, so dass
\begin{align*}
\mu(A\cap U \cap \bigcup_{B\in\mathcal G_i} B) \ge \frac1{N_n} \mu(A\cap U).
\end{align*}
Da $\mu$ Borel-regulär ist, gibt es $M_1\in\MdN$ und $B_1,\ldots,B_{M_n}\in \mathcal G_i$ mit
\begin{align*}
\mu(A\cap U \cap \bigcup_{i=1}^{M_1} B_i) 
\ge (1-\theta)\mu(A\cap B).
\end{align*}
Nun folgt
\begin{align*}
\infty > \mu(A\cap U)
&= \mu(A\cap U \cap \bigcup_{i=1}^{M_1} B_i) + \mu(A\cap U \setminus \bigcup_{i=1}^{M_1} B_i) \\
&\ge (1-\theta)\mu(A\cap U) + \mu(A\cap U\setminus \bigcup_{i=1}^{M_1} B_i),
\end{align*}
also die Behauptung.

Im Folgenden wenden wir die bewiesene Behauptung wiederholt an. 
Setze  $U_2 \da U\setminus\bigcup_{i=1}^{M_1} B_i$ sowie $\mathcal F_2\da\{B\in\mathcal F: \diam(B)\le 1,\, B\subset U_2\}$. Wie eben findet man $M_2\in\MdN$, $M_2\ge M_1$ und disjunkte abgeschlossene Bälle $B_{M_1+1},\ldots,B_{M_2}\in\mathcal F_2$ mit
\begin{align*}
\mu(A\cap U \setminus \bigcup_{i=1}^{M_2} B_i) 
&= \mu(A\cap  U_2 \setminus\bigcup_{i=M_1+1}^{M_2} B_i) \\
&\le \theta \mu(A\cap U_2) \\
&= \theta \mu(A\cap U \setminus \bigcup_{i=1}^{M_1} B_i)\\
&\le \theta^2 \mu(A\cap U).
\end{align*}
Man erhält so eine Folge disjunkter Bälle in $\mathcal F$, die in $U$ enthalten ist, so dass
\begin{align*}
\mu(A\cap U \setminus\bigcup_{i=1}^{M_k} B_i) \le \theta^k\cdot \underbrace{\mu(A\cap U)}_{<\infty} \to 0 \text{ für } k\to \infty.
\end{align*}
\end{beweis}

\section{Differentiation von Maßen}

\begin{definition}
Seien $\mu\in\mathbb{M} (X)$, $V$ eine $\mu$-Vitali-Relation auf $X$, $x\in X$. Seien $\mathcal{D}\subset \mathcal P(X)$ und $f:\mathcal D\to \MdR$. Dann wird für jede Teilfamilie $C\subset V$ mit $\inf\{\diam(S): (x,S)\in C\}=0$ erklärt:
\begin{align*}
C\text{-}\limsup_x f &\da C\text{-}\limsup_{S\to x} f(S) \da
\lim_{\ep\searrow 0} \sup\{f(S):(x,S)\in C,\, S\in\mathcal D,\, \diam(S)<\ep\}\\
\intertext{und}
C\text{-}\liminf_x f &\da C\text{-}\liminf_{S\to x} f(S) \da
\lim_{\ep \searrow0} \inf\{f(S):(x,S)\in C,\, S\in\mathcal D,\, \diam(S)<\ep\}.
\end{align*}
Falls $C\text{-}\limsup_x f = C\text{-}\liminf_x f$ gilt, so wird erklärt:
\begin{align*}
C\text{-}\lim_x f \da C\text{-}\lim_{S\to x} f(S)\da C\text{-}\limsup_x f = C\text{-}\liminf_x f.
\end{align*}
Insbesondere: Seien $\mu,\nu\in\mathbb M(X)$ und $V$ eine $\mu$-Vitali-Relation. Dann wird erklärt:
\begin{align*}
\mathbb D(\nu,\mu,V,x) \da V\text{-}\lim_x \frac\nu\mu,
\end{align*}
falls dieser Grenzwert existiert.
\end{definition}

\begin{satz}
\label{satz:2.10}
Seien $\mu,\nu\in\mathbb{M}(X)$. Für $M\subset X$ sei
\begin{align*}
\nu_\mu(M) \da \inf\{\nu(A) : A\in \borel(X),\, \mu(M\setminus A) =0 \}.
\end{align*}
Dann ist $\nu_\mu\in\mathbb{M} (X)$. Es gibt ein $B\in\borel(X)$ mit $\nu_\mu = \nu\MR B$ und $\mu(B^c) = 0$. Ferner ist $\nu_\mu=\nu$ genau dann, wenn jede $\mu$-Nullmenge auch eine $\nu$-Nullmenge ist.
\end{satz}

\begin{beweis}
Sei $M\subset X$. Wir zeigen zunächst, dass es $A\in\borel(X)$ gibt mit $\mu(M\setminus A)=0$ und $\nu_\mu(M) = \nu(A)$. Hierzu sei $\nu_\mu(M)<\infty$ (sonst wähle $A=X$). Zu $k\in\MdN$ existiert $A_k\in\borel(X)$ mit $\mu(M\setminus A_k)=0$ und
\[
0 \le \nu(A_k) - \nu_\mu(M) \le \frac1k.
\]
Setze $A\da\bigcap_{k=1}^\infty A_k \in \borel(X)$. Es gilt:
\begin{align*}
\mu(M\setminus A) = \mu(\bigcup_{k=1}^\infty (M\setminus A_k)) \le \sum_{k=1}^\infty  \mu(M\setminus A_k) = 0
\end{align*}
und damit
\begin{align*}
\nu_\mu(M) \le \nu(A) \le \nu(A_k) \le \nu_\mu(M) + \frac1k \to \nu_\mu(M)
\end{align*}
für $k\to\infty$.

\textbf{Behauptung:} $\nu_\mu$ ist ein äußeres Maß.

Hierzu ist 
\begin{align*}
0 \le \nu_\mu(\emptyset) \le \nu(\emptyset) = 0.
\end{align*}
Seien $M,M_1,M_2,\ldots \subset X$ mit $M\subset \bigcup_{i=1}^\infty M_i$. Zu jedem $i\in\MdN$ gibt es $A_i \in\borel(X)$ mit $\mu(M_i\setminus A_i) = 0$ und $\nu_\mu(M_i) = \nu(A_i)$. Damit ist
\begin{align*}
\mu(M \setminus\bigcup_{i=1}^\infty A_i) 
&\le \mu( (\bigcup_{i=1}^\infty M_i) \setminus (\bigcup_{i=1}^\infty A_i))\\
&\le \mu( \bigcup_{i=1}^\infty (M_i \setminus A_i) )\\
&\le \sum_{i=1}^\infty \mu(M_i\setminus A_i) = 0.
\end{align*}
Folglich ist
\begin{align*}
\nu_\mu(M)
\le \nu(\bigcup_{i=1}^\infty A_i) 
\le \sum_{i=1}^\infty \nu(A_i) = \sum_{i=1}^\infty \nu_\mu(M_i).
\end{align*}

\textbf{Behauptung:} $\borel(X) \subset \A_{\nu_\mu}$.

Sei $A\in\borel(X)$, $M\subset X$. Zu $M$ existiert $C\in\borel(X)$ mit $\mu(M\setminus C) = 0$ und $\nu_\mu(M) = \nu(C)$. Es gilt nun
\begin{align*}
\mu( (M\cap A) \setminus (C\cap A) )
&= \mu( (M\setminus C) \cap A ) 
\le \mu(M\setminus C) = 0 \\
\intertext{und ebenso}
\mu( (M\cap A^c) \setminus (C\cap A^c) )
&= \mu( (M\setminus C) \cap A^c ) 
\le \mu(M\setminus C) = 0.
\end{align*}
Daher ist
\begin{align*}
\nu_\mu(M) = \nu(C) = \nu(C \cap A) + \nu(C\cap A^c) 
\ge \nu_\mu(M\cap A) + \nu_\mu(M \cap A^c).
\end{align*}
Dies zeigt $A\in \A_{\nu_\mu}$.

Sei $A\in\borel(X)$ beschänkt.  Dann existiert ein $C\in\borel(X)$ mit $C\subset A$, $\mu(A\setminus C) = 0$ und $\nu_\mu(A) = \nu(C)$.

\textbf{Behauptung:} Für $M\subset A$ gilt $\nu_\mu(M) = \nu(M\cap C)$.

Fall 1: $M\subset A$, $M\in\borel(X)$. Es gilt 
\begin{align*}
\mu(M\setminus \underbrace{(C\cap M)}_{\in\borel(X)}) &= \mu( (M\cap A)\setminus (C\cap M)) \le \mu(A\setminus C) = 0
\intertext{und ebenso}
\mu((M^c\cap A)\setminus \underbrace{(C\cap M^c)}_{\in\borel(X)}) &\le  \mu(A\setminus C) = 0.
\end{align*}
Daher  ist $\nu_\mu(M) \le \nu(M\cap C)$ und $\nu_\mu(M^c \cap A) \le \nu(M^c\cap C)$ und weiter
\begin{align*}
\nu_\mu(M) + \nu_\mu(M^c\cap A)
&= \nu_\mu(M\cap A) + \nu_\mu(M^c\cap A) \\
&\ge \nu_\mu(A) \\
&= \nu(C) \\
&= \nu(C\cap M) + \nu(C\cap M^c) \tag{wegen $M\in\borel(X)\subset \A_\nu$} \\
&\ge\nu(M\cap C) + \nu_\mu(M^c\cap A) \\
&\ge \nu_\mu(M) + \nu_\mu(M^c\cap A).
\end{align*}
Es besteht also überall Gleichheit, insbesondere ist
\begin{align*}
\infty > \nu(M\cap C ) + \nu( M^c\cap C) = \nu_\mu(M) + \nu_\mu(M^c\cap A)
\end{align*}
Da der erste (bzw. zweite) Summand links größer gleich dem ersten (bzw. zweiten) Summand rechts ist, folgt insbesondere $\nu_\mu(M) = \nu(M\cap C)$.


Fall 2: $M\subset A$ beliebig. Dann existieren $D_1,D_2\in\borel(X)$ mit $M\subset D_i$ und $\nu(M\cap S) = \nu(D_1\cap S)$ und $\mu(M\cap S) = \mu(D_2\cap S)$ für alle $S\in\borel(X)$, da $\borel(X) \subset \mathcal A_\mu,A_\nu$ (vgl. Proposition \ref{prop1.3} (b)). Setze $D\da D_1\cap D_2\cap A\in\borel(X)$. Es gilt $M\subset D\subset A$. Ferner ist
\begin{align*}
\nu(M\cap S) \le \nu(D\cap S) \le \nu(D_1\cap S) = \nu(M\cap S)
\end{align*}
also $\nu(D\cap S) = \nu(M\cap S)$ und ebenso $\mu(D\cap S) = \mu(M\cap S)$. Insbesondere ist dann auch $\mu(M\setminus S) = \mu(D\setminus S)$. Wegen $D\in\borel(X)$, $D\subset A$ ist $\nu_\mu(D) = \nu(C\cap D)$ nach Fall 1. Zusammen erhält man
\begin{align*}
\nu_\mu(M)
&= \inf \{\nu(S) : S\in\borel(X),\, \mu(M\setminus S) = 0\}\\
&= \inf \{\nu(S) : S\in\borel(X),\, \mu(D\setminus S) = 0\}\\
&= \nu_\mu(D) = \nu(C\cap D) = \nu(M\cap C).
\end{align*}

Wir zerlegen nun $X$ in der Form $X=\bigcup_{i\ge 1}A_i$ mit beschränkten disjunkten Borelmengen $A_i$. 
Zu jedem $i\in\mathbb{N}$ existiert $C_i\in \borel(X)$ mit $C_i\subset A_i$, $\mu(A_i\setminus C_i)=0$ und  $\nu_\mu(M)=\nu(M\cap C_i)$ für alle $M\subset A_i$. Setze 
$$B:=\bigcup_{i\ge 1}C_i\in \borel(X).$$
Dann folgt
$$
\mu(B^c)=\mu(X\setminus B)=\mu((\bigcup_{i\ge 1}A_i) \setminus(\bigcup_{i\ge 1}C_i))\le\mu(\bigcup_{i\ge 1}(A_i\setminus C_i))=0.
$$
Es folgt für $M\subset X$, wobei für die zweiten Gleichung $\borel (X)\subset\mathcal{A}_{\nu_\mu}\subset
\mathcal{A}_{(\nu_\mu\llcorner M)}$ verwendet wird,
\begin{align*}
\nu_\mu(M)&=(\nu_\mu\MR M)(\bigcup_{i\ge 1}A_i)=\sum_{i\ge 1}(\nu_\mu\MR M)(A_i)\\
&=\sum_{i\ge 1}\nu_\mu(\underbrace{M\cap A_i}_{\subset A_i})=\sum_{i\ge 1}\nu(\underbrace{M\cap A_i\cap C_i}_{=M\cap C_i})\\
&=\sum_{i\ge 1}(\nu\MR M)(C_i)=(\nu\MR M)(\bigcup_{i\ge 1} C_i)=(\nu\llcorner M)(B)\\
&=\nu(M\cap B).
\end{align*}
Dies zeigt $\nu_\mu=\nu\MR B$.

Zu $M\subset X$ existiert $C\in\borel (X)$ mit $M\cap B\subset C$ und $\nu(M\cap B)=\nu(C)$. Damit ist 
$M\subset B^c\cup C\in \borel(X)$ und
$$
\nu_\mu(M)\le \nu_\mu(B^c\cup C)=\nu(B\cap C)\le\nu(C)=\nu(M\cap B)=\nu(M),
$$
das heißt
$$
\nu_\mu(M)=\nu_\mu(B^c\cup C).
$$
Dies schließt den Nachweis ab, dass $\nu_\mu$ Borel-regulär ist. Ist  $M$ beschränkt, 
so ist $\nu_\mu(M)=\nu(M\cap B)<\infty$. 

Ist jede $\mu$-Nullmenge eine $\nu$-Nullmenge, so ist $\nu(B^c)=0$, 
also $\nu=\nu\MR B=\nu_\mu$. Ist schließlich $M\subset X$ und $\mu(M)=0$, so ist 
$\nu_\mu(M)=\nu(\emptyset)=0$. Aus $\nu_\mu=\nu$ folgt damit $\nu(M)=0$, so dass jede $\mu$-Nullmenge  eine $\nu$-Nullmenge.
\end{beweis}

\begin{bemerkung}
Für \(\mu, \nu \in \mathbb{M}(X)\) schreibt man \(\nu \ll \mu\) und sagt \(\nu\) ist absolut stetig bezüglich \(\mu\), falls für alle \(M \subset X\) gilt: 
\[
\mu(M) = 0 \Rightarrow \nu(M) = 0.
\]
Das Maß \(\nu_\mu\) ist absolut stetig bezüglich \(\mu\) und \(\nu_\mu = \nu\MR B\) mit \(\mu(B^c) = 0,$ $B \in \borel(X)\). Also ist
\[
\nu = \nu_\mu + \underbrace{(\nu - \nu_\mu)}_{=: \nu_\mu^\bot}
\]
wobei \(\nu_\mu^\bot \in \mathbb{M}(X)\) wegen \(\nu_\mu^\bot = \nu\MR B^c\), d.h. \(\nu_\mu^\bot \bot \nu_\mu \). 
Letzteres bedeutet, dass es eine Menge $B\subset X$ gibt mit \(\nu_\mu(B^c) = \nu_\mu^\bot(B) = 0\).
\end{bemerkung}

\begin{lemma}
\label{lem2.11}
Seien \(\mu, \nu, \tau \in \mathbb{M}(X), V\) eine \(\mu-\)Vitali-Relation, \(c \in (0,\infty)\) und \[A \subset \{ x \in X : V\text{-}\liminf_x \frac{\nu(S)}{\tau(S)} < c \}.\]
Dann gilt: 
$$
\nu_\mu(A) \leq c \cdot \tau_\mu(A).
$$
\end{lemma}

\begin{beweis}
Sei \(\varepsilon > 0\). Aufgrund des Beweises von Satz \ref{satz:2.10} gibt es zu $A$ ein \(B \in \borel(X)\) mit \(\mu(A \setminus B)=0\) und \(\tau_\mu(A) = \tau(B)\).
Wegen Satz \ref{satz:2.2} gibt es zu $B$ eine offene Menge \(U \subset X\) mit \(B \subset U\) und \(\tau(U \setminus B) \leq \varepsilon\). Also ist \(\mu(A \setminus U) \leq \mu(A \setminus B)=0\) und
\[
\tau(U) = \tau(U \cap B) + \tau(U \setminus B) \leq \tau(B) + \varepsilon = \tau_\mu(A) + \varepsilon.
\]
Sei 
$$
C \da \{ (x,S) \in V : S \subset U , \frac{\nu(S)}{\tau(S)} < c \}.
$$ 
Für \(z\in A\cap U\) gilt: \(V\text{-}\liminf\frac{\nu(S)}{\mu(S)} < c\), also \(\inf\{\diam(S):(z,S)\in C\}=0\) für alle \(z \in A \cap U\). 

Da \(V\) eine \(\mu-\)Vitali-Relation ist, enthält \(C(A\cap U)\) eine abzählbare, disjunkte \(\mu\)-Überdeckung \(\mathcal{S}\) von \(A\cap U\). Aus \(\mu(A \setminus U)=0\), \(\mu(A\cap U \setminus \bigcup_{S\in \mathcal{S}} S) = 0\), \(\nu_\mu \ll \mu\) und \(\nu_\mu \leq \nu\) folgt
\begin{align*}
\nu_\mu(A) &= \nu_\mu(A\cap U \cap \bigcup_{S\in\mathcal{S}}S) \\
&\leq \nu_\mu(\bigcup_{S\in\mathcal{S}} S) \\
&\leq \nu(\bigcup_{S\in\mathcal{S}} S) \\
&\leq \sum_{S\in\mathcal{S}}\underbrace{\nu(S)}_{\leq c\tau(S)}\\
&\leq c\sum_{S\in\mathcal{S}} \tau(S) = c\cdot \tau(\underbrace{\bigcup_{S\in\mathcal{S}}S}_{\subset U})\\
&\leq c\cdot\tau(U) \\
&\leq c\cdot(\tau_\mu(A)+\varepsilon) \\
&= c\cdot\tau_\mu(A)+c\cdot\varepsilon.
\end{align*}
Da \(c<\infty\) und \(\varepsilon>0\) beliebig war, folgt die Behauptung.
\end{beweis}

\begin{bemerkungen}
	\item Sind \(\nu,\tau \ll \mu\), so gilt \(\nu_\mu=\nu, \tau_\mu=\tau\) und in Lemma \ref{lem2.11} lautet die Folgerung \(\nu(A) \leq c\cdot\tau(A)\).
	\item Es gilt \(\mu_\mu=\mu\).
\end{bemerkungen}

\begin{korollar}
\label{kor:2.12}
Seien \(\mu,\nu \in \mathbb{M}(X)\), \(V\) eine \(\mu-\)Vitali-Relation, \(c>0\). Seien ferner
\[
A\subset \{x\in X:\liminf_x\frac{\nu(S)}{\mu(S)}<c\}
\quad\text{und}\quad
B\subset\{x\in X:\limsup_x\frac{\nu(S)}{\mu(S)}>c\}.
\]
Dann gilt:
\[
\nu_\mu(A) \leq c\cdot\mu(A), \quad \nu_\mu(B) \geq c\cdot\mu(B).
\]
\end{korollar}
\begin{beweis}
Wegen Lemma \ref{lem2.11} gilt \(\nu_\mu(A) \leq c\mu_\mu(A) = c\mu(A)\). Ferner gilt:
\[
V\text{-}\liminf_x\frac{\mu(S)}{\nu(S)} = \frac1{V\text{-}\limsup_x\frac{\nu(S)}{\mu(S)}} < \frac1c
\]
und damit aufgrund von Lemma \ref{lem2.11} 
$\mu(B) = \mu_\mu(B) \leq \frac1c \nu_\mu(B)$, so dass $\nu_\mu(B) \geq c\mu(B)$.
\end{beweis}

\begin{lemma}
\label{lem2.13}
Seien \(\mu,\nu \in \mathbb{M}(X)\) und \(V\) eine \(\mu\)-Vitali-Relation. Dann ist \(\mathbb{D}(\nu,\mu,V,\cdot) \in \mathbb{F}_\mu(X,\bar{\mathbb{R}})\) und \(0 \leq \mathbb{D}(\nu,\mu,V,\cdot) < \infty\) \(\mu-\)fast-überall in \(X\).
\end{lemma}
\begin{beweis}
Seien
\begin{align*}
C & \da \{ (x,S) \in V : \mu(S) = 0 \}, \\
P & \da \{ x \in X : \inf\{\diam S : (x,S) \in C\} = 0 \}, \\
Q & \da \{ x \in X : \limsup_x\frac\nu\mu = \infty \}
\end{align*}
und für \(a,b \in \mathbb{R}\) mit \(a < b\) sei
\begin{align*} 
R(a,b) & \da \{ x \in X : V\text{-}\liminf_x\frac\nu\mu<a<b<V\text{-}\limsup_x\frac\nu\mu \}.
\end{align*}

Da \(V\) eine \(\mu\)-Vitali-Relation ist, enthält \(C(P)\) eine abzählbare, disjunkte \(\mu\)-Überdeckung \(\mathcal{S}\) von \(P\), so dass
\[
0 \leq \mu(P) \leq \mu(P\cap\bigcup_{S\in\mathcal{S}}S) \leq \mu(\bigcup_{S\in\mathcal{S}}S) = \sum_{S\in\mathcal{S}}\underbrace{\mu(S)}_{=0} = 0 ,
\]
das heißt \(\mu(P) = 0\).

Seien nun \(a<b,\; a,b\in\mathbb{Q}\) und \(A \subset Q\), \(B \subset R(a,b)\) beschränkt. 
Für ein beliebiges \(c > 0\) gilt:
\[
\infty > \nu_\mu(A) \geq c\cdot\mu(A),
\]
also $\mu(A) = 0$. 
Ebenso gilt
\begin{align*}
b\cdot\mu(B) \leq \nu_\mu(B) \leq a\cdot\mu(B)
\end{align*}
für $a < b$. Wegen $\mu(B)<\infty$ folgt \(\mu(B)=0\).

Es folgt \(\mu(Q) = \mu(R(a,b)) = 0\) und damit die Behauptung.
\end{beweis}

\begin{lemma}
\label{lem2.14}
Sei \(\mu,\nu \in \mathbb{M}(X)\), \(V\) eine \(\mu\)-Vitali-Relation. Dann ist \(\mathbb{D}(\nu,\mu,V,\cdot)\) eine \(\mu\)-messbare Funktion.
\end{lemma}
\begin{beweis}
Seien \(a<b\),
\[
M_1 \da \{ \mathbb{D}(\nu,\mu,V,\cdot) < a \} \quad\text{und}\quad  M_2 \da \{ \mathbb{D}(\nu,\mu,V,\cdot) > b \}.
\]
Seien weiter \(A_i \subset M_i\) beschränkt (\(i=1,2\)).
Es existieren \(B_i \in \borel(X)\) mit \(A_i \subset B_i\), so dass für alle $ C\in\borel(X)$ gilt
\begin{align*}
 \mu(A_i\cap C) = \mu(B_i \cap C) \quad\text{und}\quad \nu_\mu(A_i\cap C) = \nu_\mu(B_i \cap C)\tag{$\ast$}.
\end{align*}
Wegen \(a<b\) gilt mit Korollar \ref{kor:2.12} und \((\ast)\)
\begin{align*}
a \cdot \mu(B_1 \cap B_2)
&= a\cdot\mu(A_1 \cap B_2) \\
&\geq \nu_\mu(A_1 \cap B_2) \\
&= \nu_\mu(B_1 \cap B_2) \\
&= \nu_\mu(B_1 \cap A_2) \\
&\geq b\cdot\mu(B_1 \cap A_2) \\
&= b\cdot\underbrace{\mu(B_1\cap B_2)}_{<\infty}
\end{align*}
und damit $\mu(B_1 \cap B_2) = 0$.

Nun folgt:
\begin{align*}
\mu(A_1 \cap A_2) &= \mu( \underbrace{(A_1 \cup A_2)}_{\supset A_1} \cap B_1 ) + \mu( \underbrace{(A_1\cup A_2)\cap B_1^c}_{=A_2\cap B_1^c} ) \\
&\geq \mu(A_1) + \mu(A_2 \cap B_1^c) \\ &= \mu(A_1) + \mu(A_2 \cap B_1^c \cap B_2) \\
&= \mu(A_1) + \mu(\underbrace{A_2}_{\subset B_2} \cap B_2) \\
&= \mu(A_1) + \mu(A_2).
\end{align*}
Sei \(T \subset X\) beliebig, \(y \in X\). Für $i=1,2$ setze
\[
A_{i,n} \da M_i \cap T \cap B(y,n) .
\]
Es ist \(A_{i,n} \subset M_i\), \(A_{i,n}\nearrow M_i \cap T\) und \(A_{1,n} \cup A_{2,n} \subset T\). Somit 
erhält man
\begin{align*}
\mu(T) &\geq \lim_{n\rightarrow\infty} \mu(A_{1,n} \cup A_{2,n})\\
&\geq \lim_{n\rightarrow\infty} \mu(A_{1,n}) + \lim_{n\rightarrow\infty} \mu(A_{2,n}) \\
&= \mu(M_1 \cap T) + \mu(M_2 \cap T).
\end{align*}
Die Behauptung folgt nun aus Satz \ref{satz:1.8}.
\end{beweis}

\begin{satz}
\label{satz:2.15}
Seien $\nu,\mu\in\mathbb M(X)$ und $V$ eine $\mu$-Vitali-Relation. Dann gilt $\A_\mu \subset \A_{\nu_\mu}$ und für $A\in\A_\mu$ ist
\[
\nu_\mu(A) = \int_A \mathbb D(\nu,\mu,V,x)\mu(dx).
\]
\end{satz}

\begin{beweis}
Sei $A\in\A_\mu$ beschränkt. Dann gibt es eine Borelmenge $B\in\borel(X)$ mit $A\subset B$ und $\mu(A) = \mu(B)$, also $\mu(B\setminus A)=0$.  Dann ist auch $\nu_\mu(B\setminus A) = 0$, das heißt $B\setminus  A \in \mathcal \A_{\nu_\mu}$ und folglich $A = B \setminus (B\setminus A) \in \A_{\nu_\mu}$, da $\borel(X)\subset\A_{\nu_\mu}$ (vgl.
Satz \ref{satz:2.10}). Wegen $A = \bigcup_{i=1}^\infty A_i$ mit beschränkten $A_i \in \A_\mu$ folgt allgemein $A\in\A_{\nu_\mu}$, das heißt $A_\mu\subset\A_{\nu_\mu}$.

Sei nun $Z_0 \da \{ \mathbb D(\nu,\mu,V,\cdot) = 0 \}$, $Z_\infty \da \{ \mathbb D(\nu,\mu,V,\cdot) = \infty \}$. Da aufgrund von Lemma \ref{lem2.13} $\mu(Z_\infty)  = 0$ ist, ist $\nu_\mu(Z_\infty) = 0$ und daher ist
\[
\nu_\mu(Z_\infty) = 0 = \int_{Z_\infty} \mathbb D(\nu,\mu,V,x) \mu(dx).
\]

Ferner ist $Z_0 = \bigcup_{n=1}^\infty Z_n$ mit $Z_n$ beschränkt. Aus Korollar \ref{kor:2.12} erhält man für $\ep>0$ die Abschätzung $\nu_\mu(Z_n) \le \ep \cdot \underbrace{\mu(Z_n)}_{<\infty}$, das heißt $\nu_\mu(Z_n)=0$, also auch $\nu_\mu(Z_0)=0$. Daher ist
\[
\nu_\mu(Z_0) = 0 = \int_{Z_0} \underbrace{\mathbb D(\nu,\mu,V,x)}_{=0} \mu(dx).
\]

Sei schließlich $t\in(1,\infty)$, $n\in\MdZ$ und $P_n^t \da \{t^n \le \mathbb D(\nu,\mu,V,\cdot) < t^{n+1}\}$. Man erhält
\[
A\setminus(Z_0\cup Z_\infty) = \bigcup_{n\in\MdZ} \underbrace{(P_n^t \cap A)}_{\in\A_\mu}
\]
mit disjunkter Vereinigung und daher 
\[
\nu_\mu(A) = \sum_{n\in\MdZ}\nu_\mu(P_n^t \cap A).
\]
Andererseits ist wegen $\mu(Z_\infty)=0$
\begin{align*}
t^{-1} \int_A \mathbb D(\nu,\mu,V,x)\mu(dx)
&= \sum_{n\in\MdZ} t^{-1} \int_{P_n^t\cap A} \underbrace{\mathbb D(\nu,\mu,V,x)}_{<t^{n+1}}\mu(dx) \\
&\le \sum_{n\in\MdZ} t^n \cdot \mu(P_n^t \cap A) \\
&\le \sum_{n\in\MdZ} \nu_\mu(P_n^t \cap A) = \mu_\nu(A) \\
&\le \sum_{n\in\MdZ} t^{n+1} \mu(P_n^t \cap A) \\
&\le \sum_{n\in\MdZ} t \int_{P_n^t \cap A} \mathbb D(\nu,\mu,V,x)\mu(dx) \\
&= t \cdot \int_A \mathbb D(\nu,\mu,V,x)\mu(dx).
\end{align*}
Da $t>1$ beliebig war, folgt die Behauptung.
\end{beweis}

\begin{satz}
\label{satz:2.16}
Sind $\mu,\nu\in\mathbb M(X)$ und $V_1,V_2$ zwei beliebige $\mu$-Vitali-Relationen auf $X$, dann gilt
\[
\mathbb D(\nu,\mu,V_1,\cdot) = \mathbb D(\nu,\mu,V_2,\cdot)
\]
$\mu$-fast-überall.
\end{satz}

\begin{beweis}
Sei $y\in X$ fest. Für $i,j\in\{1,2\}$, $i\ne j$, sei
\[
A_{ij}^n \da \{x\in \mathbb B(y,n): \mathbb D(\nu,\mu,V_i,x) \ge \mathbb D(\nu,\mu,V_j,x) + \frac1n\}.
\]
Hiermit ist
\[
\frac 1n\cdot \mu(A_{ij}^n) \le \int_{A_{ij}^n}(\mathbb D(\nu,\mu,V_i,x) - \mathbb D(\nu,\mu,V_j,x)) \mu(dx) = \nu_\mu(A_{ij}^n) - \nu_\mu(A_{ij}^n) = 0.
\]
Dies zeigt $\mu(A_{ij}^n)=0$ für alle $n\in\MdN$. Die Behauptung folgt nun aus 
\[
\mu(\mathbb D(\nu,\mu,V_1,\cdot) \ne \mathbb D(\nu,\mu,V_2,\cdot)) =  \mu(\bigcup_{n\in\MdN} A_{12}^n \cup \bigcup_{n\in\MdN} A_{21}^n) = 0.
\]
\end{beweis}

\begin{satz}[Lebesguescher Dichtesatz]
\label{satz:2.17}
Seien $\mu\in\mathbb M(x)$, $V$ eine $\mu$-Vitali-Relationen auf $X$ , $f:X\to\bar\MdR$ sei $\mu$-messbar und $\int_A |f|d\mu <\infty$ für jede beschränkte Menge $A\in\A_\mu$. Dann gilt:
\[
V\text-\lim_{S\to x} \frac1{\mu(S)} \cdot \int_S fd\mu = f(x)
\]
für $\mu$-fast-alle $x\in X$.
\end{satz}

\begin{beweis}
Sei zunächst $f\ge 0$. Sei $A\subset X$, $A\in \mathcal{A}_\mu$ und $\bar\nu(A) \da \int_A fd\mu$. Dann ist $\bar\nu$ ein Maß auf der $\sigma$-Algebra $\mathcal{A}_\mu$. Setze
\[
\nu(M) \da \inf\{\bar\nu(A):  A\in \A_\mu,\, M\subset A \}.
\]
Dann ist $\nu$ ein $\A_\mu$-reguläres äußeres Maß (vgl. Satz \ref{satz:1.4}). Ferner ist $\borel(X) \subset \A_\mu \subset A_\nu$. Ist $M\subset X$, so gibt es ein $A\in \A_\mu$ mit $M\subset A$ und $\nu(M)=\nu(A)$. Zu $A\in \A_\mu$ existiert $B\in\borel(X)$ mit $A\subset B$ und $\mu(B\setminus A)=0$ (vgl.\ Satz \ref{satz:2.2} (3)). Daraus folgt
\[
\nu(M) = \nu(A) = \int_Afd\mu = \int_A fd\mu + \int_{B\setminus A} fd\mu = \int_B fd\mu = \nu(B),
\]
da $B\in\borel(X)$ mit $M\subset A \subset B$. Also ist $\nu\in\mathbb M(X)$, denn für ein beschränktes $A\in\A_\mu$ ist $\nu(A) = \int_A fd\mu < \infty$ nach Voraussetzung.

Offenbar ist $\nu \ll \mu$ und daher $\nu_\mu = \nu$. Wegen $A\in\A_\mu$ ist aufgrund von Satz \ref{satz:2.15}
\[
\int_A fd\mu = \nu(A) = \nu_\mu(A) = \int_A \mathbb D(\nu,\mu,V,x)\mu(dx)
\]
für alle $A\in\A_\mu$. Dies zeigt $f = \mathbb D(\nu,\mu,V,\cdot)$ $\mu$-fast-überall, das heißt, für $\mu$-fast-alle $x\in X$ gilt
\[
f(x) = \mathbb D(\nu,\mu,V,x) = V\text-\lim_{S\to x} \frac{\nu(S)}{\mu(S)} =  V\text-\lim_{S\to x} \frac1{\mu(S)} \int_S fd\mu,
\]
wobei $S\in\borel(X)\subset\mathcal{A}_\mu$ benutzt wurde.

Für die allgemeine Aussage wird die Zerlegung $f=f^+-f^-$ verwendet.
\end{beweis}

\begin{definition}
Seien $\mu\in\mathbb M(X)$, $V$ eine $\mu$-Vitali-Relation, $M\subset X$ und $x\in x$. Dann nennt man
\[
V\text-\lim_{S\to x}\frac{\mu(M\cap S)}{\mu(S)} = V\text-\lim_{S\to x} \frac{(\mu\MR M)(S)}{\mu(S)}
\]
die $(\mu,V)$-Dichte von $M$ in $x$, falls der Limes existiert.
\end{definition}

\begin{satz}
\label{satz:2.18}
Seien $\mu\in\mathbb M(X)$, $V$ eine $\mu$-Vitali-Relation, $M\subset X$. Dann existiert die $(\mu,V)$-Dichte 
von $M$ für $\mu$-fast-alle $x\in X$. Setze
\[
P \da \{x\in X: V\text-\lim_{S\to x}\frac{\mu(S\cap M)}{\mu(S)} = 1\} \quad\text{und}\quad
Q \da \{x\in X: V\text-\lim_{S\to x}\frac{\mu(S\cap  M)}{\mu(S)} = 0\}.
\]
Dann gilt $P,Q\in\A_\mu$, $\mu(M\cap P^c)$ und $\mu(Q\cap M) =0$.

Ferner sind äquivalent:
\begin{enumerate}
\item $M\in \A_\mu$,
\item $\mu(P\cap M^c) = 0$,
\item $\mu(M^c\cap Q^c) = 0$.
\end{enumerate}
\end{satz}

\begin{beweis}
Zu $M\subset X$ existiert $A\in\A_\mu$ mit $M\subset A$ und $\mu(M\cap B) = \mu(A\cap B)$ für alle $B\in\A_\mu$ 
(vgl. Aufgabe 2, Übungsblatt 1). Es gilt $\mu\MR A\in\mathbb M(X)$ und daher ist $\{\mathbb D(\mu\MR A,\mu,V,\cdot)= 1\} \in \A_\mu$. Ferner existiert wegen $S\in\borel(X)$ der Limes
\[
\mathbb D(\mu\MR A, \mu, V, x) = V\text-\lim_{S\to x} \frac{(\mu\MR A)(S)}{\mu(S)} = V\text-\lim_{S\to x}\frac{\mu(M\cap S)}{\mu(S)} 
\]
für $\mu$-fast-alle $x\in X$. Dies zeigt insbesondere $P,Q\in\A_\mu$.

Für $\mu$-fast-alle $x\in X$ gilt wegen Satz \ref{satz:2.17} 
\[
\ind_A(x)  = V\text-\lim_{S\to x} \frac1{\mu(S)} \int_S \ind_A(x) \mu(dx) = V\text-\lim_{S\to x} \frac{\mu(A\cap S)}{\mu(S)} = \mathbb D(\mu\MR A,\mu,V,x).
\]
Sei nun $x\notin P$. Dann ist $x\notin A$ für $\mu$-fast-alle $x\in X$, das heißt $\mu(A\setminus P) = 0$. In  analoger Weise sieht man $\mu(P\setminus A)=0$ ein. Wegen $M\subset A$ ist $\mu(M\setminus P) \le \mu(A\setminus P) =0$. Hieraus folgt $\mu(M\cap P^c)=0$ und damit $M\setminus P\in\A_\mu$. 

Ist $M\in\A_\mu$, so wähle $A=M$, das heißt $\mu(P\setminus M)=0$. 

Ist  dagegen $\mu(P\setminus M)=0$, so ist $P\setminus M\in\A_\mu$ und $M=(M\setminus P) \cup (M\cap P) = (M\setminus P) \cup (P\setminus (P\setminus M)) \in \A_\mu$. 

Die Argumentation für $Q$ verläuft analog.
\end{beweis}

\section{Hausdorffmaße und Hausdorffdimension}

\begin{definition}
Sei \((X,d)\) ein metrischer Raum. Wir definieren für \(\delta > 0\) und \(M \subset X\)
\begin{align*}
& \Omega_\delta(M) \da \{ (A_i)_{i\in\MdN} : A_i \subset X, \diam(A_i) \leq \delta, i\in \MdN, M \subset \bigcup_{i\geq 1} A_i \}, \\
& \tilde\Omega_\delta(M) \da \{ (A_i)_{i\in\MdN} : X \supset A_i \text{ offen}, \diam(A_i) \leq \delta, i\in \MdN, M \subset \bigcup_{i\geq 1} A_i \} \\
\intertext{und}
& \bar\Omega_\delta(M) \da \{ (A_i)_{i\in\MdN} : X \supset A_i \text{ abgeschlossen}, \diam(A_i) \leq \delta, i\in \MdN, M \subset \bigcup_{i\geq 1} A_i \},
\end{align*}
wobei \(\diam(\emptyset) \da 0\).
% Analog \(\tilde\Omega_\delta(M)\) und \(\bar\Omega_\delta(M)\), wobei \(A_i\) offen bzw. abgeschlossen gewählt werden.
\end{definition}

\begin{lemma}
\label{lem:2.19}

Sei \(s\in [0,\infty)\). Für \(\delta>0\) und \(M \subset X\) sei
\begin{align*}
\mu_\delta^s (M) \da \inf\{ \sum_{A \in \mathcal{A}}{(\diam(A))}^s : \mathcal{A} \in \Omega_\delta(M) \}, \\
\tilde\mu_\delta^s (M) \da \inf\{ \sum_{A \in \mathcal{A}}{(\diam(A))}^s : \mathcal{A} \in \tilde\Omega_\delta(M) \} \\
\intertext{und}
\bar\mu_\delta^s (M) \da \inf\{ \sum_{A \in \mathcal{A}}{(\diam(A))}^s : \mathcal{A} \in \bar\Omega_\delta(M) \}.
\end{align*}
Dann gilt:
\begin{enumerate}[(1)]
\item \(\mu_\delta^s\), \(\tilde\mu_\delta^s\), \(\bar\mu_\delta^s\) sind äußere Maße auf \(X\),
\item \(\tilde\mu_\varepsilon^s(M) \leq \bar\mu_\delta^s(M) = \mu_\delta^s(M) \leq \tilde\mu_\delta^s(M)\) für alle \(0 < \delta < \varepsilon\).
\end{enumerate}

\end{lemma}
Beweis: Übung.

\begin{satz}
\label{satz:2.20}

Sei \(s \in [0,\infty)\). Für \(M \subset X\) wird durch
\[
\mu^s(M) \da \sup\{\mu_\delta^s(M) : \delta > 0\}
\]
ein Borel-reguläres, äußeres Maß auf \(X\)  erklärt, das \(s\)-dimensionale Hausdorffmaß auf \((X,d)\). Es gilt:
\begin{enumerate}[(1)]
\item \(\mu^s(M) = \sup\{\bar\mu_\delta^s(M) : \delta > 0\} = \sup\{\tilde\mu_\delta^s(M) : \delta > 0\}\).
\item \(\mu^s(M) = \lim_{\delta \searrow 0} \mu_\delta^s(M) = \lim_{\delta\searrow0} \bar\mu_\delta^s(M) = \lim_{\delta\searrow0} \tilde\mu_\delta^s(M)\).
\end{enumerate}

\end{satz}

\begin{beweis}
Die Aussagen (1), (2) sind klar. Ebenso ist offenbar \(\mu^s(\emptyset) = 0\). Sei \(A \subset \bigcup_{i\geq1} A_i,\;A,A_i\subset X\). Dann ist für \(\delta>0\):
\[
\mu_\delta^s(A) \leq \sum_{i\geq1} \mu_\delta^s(A_i) \leq \sum_{i\geq1} \mu^s(A_i).
\]
Hieraus folgt aber
\[
\mu^s(A) \leq \sum_{i\geq1} \mu^s(A_i).
\]
Wir zeigen \(\borel(X) \subset \mathcal{A}_{\mu^s}\) mit Hilfe von Satz \ref{satz:2.4}. Hierzu seien \(A,B \subset X\) mit $d(A,B)>0$ und o.B.d.A. sei \(\mu(A \cup B) < \infty\). Wir setzen \(\delta \da \frac12 d(A,B) > 0\). Sei \((M_i)_{i\in\MdN} \in \Omega_\delta(M)\). Es folgt
\[
\underbrace{\{ M_i : M_i \cap A \neq \emptyset \}}_{\in \Omega_\delta(A)}  \; \cap \; \underbrace{\{ M_i : M_i \cap B \neq \emptyset \}}_{\in \Omega_\delta(B)} = \emptyset,
\]
und daher
\[
\sum_{i\geq1} {(\diam(M_i))}^s \geq \sum_{\substack{i\geq1\\ M_i\cap A \neq\emptyset}} {(\diam(M_i))}^s + \sum_{\substack{i\geq1\\ M_i \cap B \neq \emptyset}} {(\diam(A_i))}^s \geq \mu_\delta^s(A) + \mu_\delta^s(B).
\]
Das zeigt
\[
\mu^s(A \cup B) = \lim_{\delta\searrow0}\mu_\delta^s(A\cup B) \geq \lim_{\delta\searrow0}\mu_\delta^s(A) + \lim_{\delta\searrow0}\mu_\delta^s(B) = \mu^s(A)+\mu^s(B).
\]
Sei jetzt \(M \subset X\) und o.B.d.A. \(\mu^s(M)<\infty\). Zu jedem \(n\in\MdN\) existiert \(\mathcal{C}^n \in \tilde\Omega_{\frac1n}(M)\) mit 
$$
\sum_{C \in \mathcal{C}^n} {(\diam(C))}^s \leq \tilde\mu_{\frac1n}^s(M)+\frac1n.
$$
Setze
\[
B \da \bigcap_{n\in\MdN}\bigcup_{C\in\mathcal{C}^n}C.
\]
Dann ist \(M \subset B \in \borel(X)\) und 
$$
\tilde\mu^s_{\frac1n}(B) \leq \tilde\mu^s_{\frac1n}(\bigcup_{C\in\mathcal C^n}C) \leq \sum_{C\in\mathcal C^n} {(\diam(C))}^s \leq \tilde\mu^s_{\frac1n}(M)+\frac1n,\qquad  n\in\MdN.
$$
Insgesamt erhält man
$$
\mu^s(M) \leq \mu^s(B) = \lim_{n\rightarrow\infty} \tilde\mu^s_{\frac1n}(B) \leq \lim_{n\rightarrow\infty}(\tilde\mu^s_{\frac1n}(M)+\frac1n) = 
\lim_{n\rightarrow\infty} \tilde\mu^s_{\frac1n}(M) + 0 = \mu^s(M),
$$
d.h. \(\mu^s(M) = \mu^s(B)\), was die Borel-Regularität ergibt.
\end{beweis}

\begin{proposition}
\label{prop:2.21}
Zu \(M \subset X\) gibt es genau ein \(s\in[0,\infty]\) mit
\[
\mu^p(M) = \begin{cases} 0, & p>s, \\ \infty, & p<s. \end{cases}
\]
Man nennt diese Zahl \(s\) die Hausdorffdimension von \(M\), \(\dim_H(M)\). Also:
\[
\dim_H(M) \da \inf\{p\geq0 : \mu^p(M)=0\}.
\]

\end{proposition}

\begin{beweis}
Seien \(p,q \in [0,\infty)\) mit \(p<q\). Sei \(\mu^p(M)<\infty\). Dann existiert zu \(\delta>0\) eine Folge \((A_i)_{i\in\MdN} \in \Omega_\delta(M)\) mit \(\sum_{i\geq1} {(\diam(A_i))}^p \leq \mu_\delta^p(M)+1 \leq \mu^p(M)+1\). Hieraus folgt
\begin{align*}
\mu_\delta^q(M) &
\leq \sum_{i\geq1} {(\diam(A_i))}^q \\ &
= \sum_{i\geq1} (\diam(A_i))^p \cdot (\underbrace{\diam(A_i)}_{\leq\delta})^{q-p} \\ &
\leq (\sum_{i\geq1} (\diam(A_i))^p)\cdot \delta^{q-p} \\ &
\leq \delta^{q-p}\cdot (\underbrace{\mu^p(M)+1}_{<\infty}),
\end{align*}
und damit
\[
0 \leq \mu^p(M) = \lim_{\delta\searrow0} \mu_\delta^q(M) \leq \lim_{\delta\searrow0}(\delta^{q-p})\cdot(\mu^p(M)+1) = 0.
\]
Sei nun \(s \da \inf\{p\geq0 : \mu^p(M) = 0\}\). Ist \(p>s\), so gibt es ein \(q\in(s,p)\) mit \(\mu^q(M) = 0\) und daher \(\mu^p(M)=0\).
\par
Sei jetzt \(p<s\). Wäre \(\mu^p(M)<\infty\), so würde \(\mu^q(M)=0\) für alle \(q>p\) gelten, im Widerspruch zur Definition von \(s\). Also ist \(\mu^p(M)=\infty\) für \(p<s\).
\end{beweis}

\begin{beispiel}
$$
\mu^0(M) = \begin{cases} \#M, & M \text{ endlich}, \\ \infty, & \text{sonst}. \end{cases}
$$
\par
und \(\dim_H(M) = 0\) für \(\#M < \infty\).
\end{beispiel}

\begin{definition}
Seien \((X,d)\) und \((\bar X,\bar d)\) metrische Räume. Eine Abbildung \(f: X \rightarrow \bar X\) heißt Isometrie, falls \(\bar d(f(x),f(y)) = d(x,y)\)für alle $ x,y\in X$. 
\par
Zu \(f:X \rightarrow \bar X\) wird
\[
\Lip(f) \da \sup\left\{\frac{\bar d(f(x),f(y))}{d(x,y)} : x,y \in X, x\neq y\right\}
\]
erklärt. Ist \(\Lip(f)<\infty\), so nennt man \(f\) eine Lipschitzfunktion und \(\Lip(f)\) die Lipschitz-Konstante von \(f\).
\end{definition}

\begin{lemma}
\label{lem:2.22}

Ist \(f: (X,d) \rightarrow (\bar X, \bar d)\) eine Lipschitzfunktion, 
so gilt \(\bar\mu^s(f(M)) \leq \Lip(f)^s \cdot \mu^s(M)\) und \(\dim_H(f(M)) \leq \dim_H(M)\).

\end{lemma}

\begin{beweis}
Sei \((A_i)_{i\in\MdN} \in \Omega_\delta(M)\). Dann gilt \(f(M) \subset \bigcup_{i\geq1} f(A_i)\) und
\begin{align*}
\diam(f(A_i)) & = \sup\{\bar d(f(x),f(y)) : x,y \in A_i\} \\
& \leq \sup \{ \Lip(f) \cdot d(x,y) : x,y \in A_i \} \\
& = \Lip(f) \cdot \diam(A_i) \leq \Lip(f) \cdot \delta,
\end{align*}
d.h. \((f(A_i))_{i\in\MdN} \in \Omega_{\Lip(f)\cdot\delta}(f(M))\) und
$$
\mu_{\Lip(f)\cdot\delta}^s(f(M))  \leq \sum_{i\geq1}(\diam(f(A_i)))^s 
 \leq \Lip(f)^s \cdot \sum_{i\geq1} (\diam(A_i))^s.
$$
Dies zeigt
\[
\mu_{\Lip(f)\cdot\delta}^s(f(M)) \leq \Lip(f)^s\cdot \mu_\delta^s(M).
\]
Aus \(\delta\searrow0\) folgt die Behauptung.
\end{beweis}

Sei nun \(K(X) \da \{f:X\rightarrow X: \Lip(f)<1\}\). Für \(m \geq 2\) induziert \(\Psi \da (\psi_1,\dots,\psi_m) \in K(X)^m\) eine Abbildung
\[
\Psi^\ast: \mathcal P(X) \rightarrow \mathcal P(X), \quad M \mapsto \bigcup_{i=1}^m \psi_i(M).
\]
Man erklärt \(\Psi^{\ast k} \da \underbrace{\Psi^\ast \circ \dots \circ \Psi^\ast}_{k \text{ mal}}\) für  $k\in\MdN$. Offenbar gilt \(M \subset M' \Rightarrow \Psi^{\ast k}(M) \subset \Psi^{\ast k}(M')\).
\par
Ist \(A_i \subset X, \; i\in\MdN\), so sei
\[
\limsup_{i\rightarrow\infty} A_i \da \{x \in X : x \text{ liegt in unendlich vielen } A_i\} = \bigcap_{n\in\MdN}\bigcup_{i\geq n} A_i.
\]

\begin{lemma}
\label{lem:2.23}
Seien \(\Psi = (\psi_1,\dots,\psi_m) \in K(X)^m\) und \(\emptyset \neq C \subset X\) kompakt mit \(\Psi^\ast(C) \subset C\). Dann gilt für eine beliebige beschränkte Menge \(M \subset X\) die Inklusion
\[
\limsup_{k\rightarrow\infty} \Psi^{\ast k}(M) \subset C.
\]
\end{lemma}

\begin{beweis}
Sei \(M \subset X\) beschränkt. Sei \(y \in \limsup_{k\rightarrow\infty} \Psi^{\ast k}(M)\), d.h. \(y \in \bigcap_{n\geq1}\bigcup_{k\geq n} \Psi^{\ast k} (M)\). Zu jedem \(n\in\MdN\) existiert also ein \(k\geq n\) mit \(y\in\Psi^{\ast k}(M)\), d.h. \(\exists y_k \in M\) und \(i_1,\dots,i_k \in \{1,\dots,m\}\) mit \(y = \psi_{i_k}\circ\dots\circ\psi_{i_1}(y_k)\). 
\par
Sei \(z \in C\) beliebig. Dann gilt \(\psi_{i_k}\circ\dots\circ\psi_{i_1}(z) \in C\) sowie
\begin{align*}
d(y,C) & \leq d(y,\psi_{i_k}\circ\dots\circ\psi_{i_1}(z)) \\
& \leq \Lip(\psi_{i_k}) \cdot d(\psi_{i_{k-1}}\circ\dots\circ\psi_{i_1}(y_k), \psi_{i_{k-1}}\circ\dots\circ\psi_{i_1}(z)) \\
& \leq \Lip(y_{i_k}) \cdots \Lip(y_{i_1}) \cdot d(y_k, z) \\
& \leq {\underbrace{\max\{\Lip(\psi_i): 1 \leq i \leq m\}}_{\ad r < 1}}^k \cdot d(y_k,z) \\
& \le r^k \cdot \underbrace{\diam(M\cup C)}_{<\infty}.
\end{align*}
Mit \(n\rightarrow\infty \) (und damit \( k\rightarrow\infty\)) folgt \(d(y,C)=0\), d.h. \(y\in C\), da \(C\) abgeschlossen ist.
\end{beweis}

\begin{satz}
\label{satz:2.24}
Sei $(X,d)$ ein vollständiger metrischer Raum und $\Psi=(\psi_1,\ldots,\psi_m)\in K(X)^m$. Dann gibt es genau eine kompakte Menge $\emptyset \ne E\subset X$ mit $\Psi^*(E) = E$. Ist $s\in[0,\infty)$ die eindeutig bestimmte Lösung der Gleichung $\sum_{i=1}^m \Lip(\psi_i)^s = 1$, so gilt $\mu^s(E)<\infty$ und daher $\dim_H(E) \le s$.
\end{satz}

\begin{beweis}
Nach dem Banachschen Fixpunktsatz hat jedes $\psi_i$ einen Fixpunkt $x_i\in X$, das heißt $\psi_i(x_i) = x_i$. Setze $F \da \{x_1,\ldots,x_m\}\subset X$ und
\[
E\da\overline{\bigcup_{k=1}^\infty \Psi^{*k}(F)}.
\]

Zunächst ist $\Psi^{*k}(F)\subset \Psi^{*(k+1)}(F)$ für $k\in \MdN$, da 
\[
\psi_{i_1}\circ \cdots \circ \psi_{i_k}(x_j) = \psi_{i_1}\circ \cdots\circ \psi_{i_k}\circ \psi_j (x_j) \in \Psi^{*(k+1)}(F).
\]
Ferner gilt 
\begin{align*}
\Psi^*(\bigcup_{k=1}^\infty \Psi^{*k}(F))
&= \bigcup_{i=1}^m \psi_i(\bigcup_{k=1}^\infty \Psi^{*k}(F))\\
&= \bigcup_{k=1}^\infty \bigcup_{i=1}^m \psi_i(\Psi^{*k}(F))\\
&= \bigcup_{k=1}^\infty \Psi^*(\Psi^{*k}(F))\\
&= \bigcup_{k=1}^\infty \Psi^{*(k+1)}(F) \\
&= \bigcup_{k=1}^\infty \Psi^{*k}(F).
\end{align*}
Es folgt für $\Psi^*(E)$ wegen der Stetigkeit der $\psi_i$:
\begin{align*}
\Psi^*(E)
& =\Psi^*\left(\overline{\bigcup_{k=1}^\infty \Psi^{*k}(F)}\right)\\
&= \bigcup_{i=1}^m\psi_i\left(\overline{\bigcup_{k=1}^\infty \Psi^{*k}(F)}\right) \\
&\subset \bigcup_{i=1}^m\overline{\psi_i\left(\bigcup_{k=1}^\infty \Psi^{*k}(F)\right)} \\
&\subset \bigcup_{i=1}^m\overline{\Psi^*\left(\bigcup_{k=1}^\infty \Psi^{*k}(F)\right)} \\
&= \overline{\bigcup_{k=1}^\infty \Psi^{*k}(F)} = E.
\end{align*}
Wir zeigen als nächstes, dass $E$ totalbeschränkt ist. Sei hierzu
\[
r\da \max\{\Lip(\psi_i): i\in\{1,\ldots,m\}\} < 1.
\]
Sei $\ep>0$. Dann gibt es ein $k\in \MdN$ mit
\[
\underbrace{\diam(\Psi^*(F))}_{<\infty} \cdot \sum_{i=k}^\infty r^i < \frac\ep2.
\]
Sei $x\in E$. Nach Definition von $E$ gibt es ein $p\in\MdN$, ohne Beschränkung der Allgemeinheit $p\ge k+1$, und ein $y\in\Psi^{*p}(F)$ mit $d(x,y)<\frac\ep2$. Also gibt es Indizes $i_1,\ldots,i_p\in \{1,\ldots,m\}$ und $j\in\{1,\ldots,m\}$ mit $y = \psi_{i_p}\circ\cdots\circ\psi_{i_1}(x_j)$. Wiederholte Anwendung der Dreiecksungleichung ergibt
\begin{align*}
d(x,\Psi^{*k}(F))
&\le d(x,y) + d(y,\Psi^{*k}(F))\\
&\le d(x,y) + \sum _{l=1}^{p-k} d(\psi_{i_p}\circ \cdots \circ \psi_{i_l}, \psi_{i_p} \circ \cdots\circ \psi_{i_{l+1}}(x_j))\\
&\le d(x,y) + \sum_{l=1}^{p-k} r^{p-l} \underbrace{d(\psi_{i_l}(x_j), x_j)}_{\le \diam(\Psi^*(F))}\\
&\le \frac\ep2 + \frac\ep2 = \ep .
\end{align*}
Dies zeigt, dass $E$ total beschränkt ist. Als abgeschlossene Menge in einem vollständigen metrischen Raum 
ist $E$ selbst vollständig, und somit ist $E$ sogar kompakt.

Hiermit zeigen wir nun $E\subset \Psi^*(E)$. Sei $z\in E$. Es existieren dann $z_i \in \bigcup_{k=1}^\infty \Psi^{*k}(F)$ mit $z_i \to z$. Hierbei ist $z_i = \psi_{l(i)}(y)$ mit $y_i\in \bigcup_{k=0}^\infty \Psi^{*k}(F) = \bigcup_{k=1}^\infty \Psi^{*k}(F) \subset E$ und $l(i)\in\{1,\ldots,m\}$. Es existiert eine Teilfolge von $(y_i)_{i\in\MdN}$, ohne Beschränkung der Allgemeinheit die Folge selbst, mit $y_i \to y\in E$. Ferner gibt es ein $l\in\{1,\ldots,m\}$ mit $l(i)=l$ für unendlich viele $i\in\MdN$. Hieraus folgt: $z \leftarrow z_i = \psi_l(y_i) \to \psi_l(y)$. Daher ist $z = \psi_l(y) \in \Psi^*(E)$, das heißt $E\subset \Psi^*(E)$.

Zur Eindeutigkeit: Sei $\emptyset \ne \tilde E\subset X$ kompakt und $\Psi^*(\tilde E) = \tilde E$. Aus Lemma \ref{lem:2.23} folgt:
\[
E = \limsup_{k\to\infty}\Psi^{*k}(E) \subset \tilde E = \limsup_{k\to\infty}\Psi^{*k}(\tilde E) \subset E,
\]
das heißt $E=\tilde E$.

Zur Hausdorffdimension: Ist $\delta >0$, so existiert zunächst $k\in\MdN$ mit $r^k\cdot \diam(E)<\delta$. Wegen $\Psi^{*k}(E)=E$ folgt $\{\psi_{i_k}\circ\cdots\circ\psi_{i_1}(E) : i_1,\ldots,i_k\in\{1,\ldots,m\}\}\in\Omega_\delta(E)$ und damit
\begin{align*}
\mu^s_\delta(E)
&\le \sum_{\mathclap{i_1,\ldots,i_k=1}}^m \diam(\psi_{i_k}\circ\cdots\circ\psi_{i_1}(E))^s \\
&\le \sum_{\mathclap{i_1,\ldots,i_k=1}}^m \Lip(\psi_{i_k})^s\cdots \Lip(\psi_{i_1})^s (\diam(E))^s\\
&= (\diam(E))^s \cdot (\underbrace{\Lip(\psi_1)^s+\cdots+ \Lip(\psi_m)^s}_{=1})^k = \diam(E)^s.
\end{align*}
Dies impliziert
\[
\mu^s(E) = \lim_{\delta\searrow0}\mu_\delta^s(E) \le (\diam(E))^s < \infty.
\]
\end{beweis}

\textbf{Zusatz:} Sei $\emptyset \ne C\subset X$ kompakt mit $\Psi^*(C)\subset C$. Dann ist $E\da \bigcap_{k=1}^\infty \Psi^{*k}(C)$.

In der Tat ist
$E = \limsup_{k\to\infty} \Psi^{*k}(E) \subset C$ nach Lemma \ref{lem:2.23}, $\Psi^{*k}(E) = E$, $\Psi^{*(k+1)}(C) \subset \Psi^{*k}(C)$ und damit, wiederum mit Lemma \ref{lem:2.23},
\[
E = \bigcap_{k=1}^\infty \Psi^{*k}(E) \subset \bigcap_{k=1}^\infty \Psi^{*k}(C) = \limsup_{k\to\infty} \Psi^{*k}(C) \subset E.
\]

\begin{satz}[Hutchinson, 1981]
\label{satz:2.25}
Für $\Psi= (\psi_1,\ldots,\psi_m)\in K(\MdR^n)^m$ seien die folgenden Eigenschaften erfüllt:
\begin{enumerate}
\item Für $i=1,\ldots,m$ gibt es $r_i\in(0,1)$ mit $|\psi_i(x) - \psi_i(y)| = r_i|x-y|$ für alle $x,y\in\MdR^n$.
\item Es gibt eine beschränkte, offene Menge $\emptyset \ne U\subset\MdR^n$ derart, dass $\psi_1(U),\ldots,\psi_m(U)$ paarweise disjunkt sind und $\Psi^*(U) \subset U$.
\end{enumerate}
Dann gilt für die durch $\sum_{i=1}^m r_i^s = 1$ festgelegte Zahl $s>0$ und das $\Psi^*$-invariante Kompaktum $E\subset \MdR^n$ die Abschätzung $0<\mu^s(E) < \infty$, das heißt insbesondere $\dim_H(E)= s$. Man nennt $s$ die Ähnlichkeitsdimension von $R$.
\end{satz}

\begin{beweis}
Übung.
\end{beweis}

Konstruktion von Mengen mit Hausdorffdimension $s\in[0,n]$. Für $r\in(0,\frac12]$ und für $\alpha=(\alpha_1,\ldots,\alpha_n)^\top\in\{0,1\}^n$ sei $\psi_\alpha : \MdR^n \to \MdR$ 
erklärt durch 
$$
\psi_\alpha(x) \da r\cdot x + \sum_{k=1}^n (1-r)\alpha_ke_k,
$$ 
wobei $e_1,\ldots,e_n$ die Standardbasis des $\MdR^n$ ist. Nun sei $\Psi=(\psi_\alpha: \alpha\in\{0,1\}^n)$. Setze $U\da (0,1)^n$. Für $x\in U$ ist 
$$
0 < (\psi_\alpha(x))_k = r\cdot x_k + (1-r)\alpha_k < r + (1-r) = 1,
$$ das heißt $\Psi(x) \in U$.

Seien $\alpha,\beta \in\{0,1\}^n$, $\alpha \ne \beta$. Dann gibt es $k\in\{1,\ldots,n\}$ mit $|\alpha_k - \beta_k| = 1$. Seien $x,y\in U$. Wegen $0<r\le \frac12$ gilt
\begin{align*}
|(\psi_\alpha(x))_k - (\psi_\beta(y))_k|
&= |rx_k + (1-r)\alpha_k - ry_k - (1-r)\beta_k|\\
&\ge |(1-r)(\alpha_k - \beta_k)| - |r(x_k-y_k)|\\
&> (1-r) - r \ge 0,
\end{align*}
das heißt $(\psi_\alpha(x))_k \ne (\psi_\beta(x))_k$. Ferner gilt $|\psi_\alpha(x) - \psi_\beta(x)| = r |x-y|$, $x,y\in \MdR^n$.  Nach Satz \ref{satz:2.25} erhält man für die $\Psi^*$-invariante Menge $E$ die Abschätzungen $0<\mu^s(E)<\infty$. 

Genauer: Wegen $r_1=\ldots=r_{2^n}=r\in(0,\frac12]$ ist $1 = \sum_{i=1}^{2^n} r_i^s = 2^n \cdot r^s $ und damit
\[
s = -\frac{n \ln 2}{\ln r}.
\]

\begin{beispiele}
\item Für $n=1$, $r=\frac13$ ist $\dim_H(E_{\frac13}^1) = \frac{\log 2}{\log 3}$.
\item Für $n=2$, $r=\frac14$ ist $\dim_H(E_{\frac14}^2) = 1$.
\end{beispiele}

\section{Hausdorffmaße auf euklidischen Räumen}

\begin{definition}
Das $n$-dimensionale äußere Lebesguemaß $\lambda^n$ auf $\MdR^n$ ist erklärt durch
\begin{multline*}
\lambda^n(M) \da \inf\{\sum_{i=1}^\infty \prod_{j=1}^n (b_i^{(j)} - a_i^{(j)}) : -\infty < a_i^{(j)} < b_i^{(j)} < \infty : \\ i\in\MdN\,,j=1,\ldots,n,\,
M\subset \bigcup_{i=1}^\infty (a_i^{(1)}, b_i^{(1)})\times \cdots \times (a_i^{(n)},b_i^{(n)})\}.
\end{multline*}
\end{definition}

Bei der Definition kann man sich auf Überdeckungsmengen mit Durchmesser kleiner $\ep$ für jedes $\ep>0$ beschränken oder halboffene bzw. abgeschlossene Intervalle verwenden. Wie beim Hausdorffmaß zeigt man dann, dass $\lambda^n$ Borel-regulär ist. Ferner ist $\lambda^n$ das einzige Borel-reguläre und translationsinvariante äußere Maß auf $\MdR^n$, das $[0,1]^n$ den Wert 1 zuordnet. Außerdem gilt
\[
\lambda^n = \underbrace{\lambda^1\otimes \cdots \otimes \lambda^1}_{\text{$n$-mal}},
\]
\begin{align*}
\lambda^n([a_1,b_1]\times \cdots \times [a_n,b_n]) = \prod_{j=1}^n (b_j-a_j)
\end{align*}
und
\begin{align*}
\lambda^n(r\cdot M) = r^n\cdot \lambda^n(M)
\end{align*}
für $r\ge 0$, $M\subset \MdR^n$.

\begin{satz}[Brunn-Minkowski-Ungleichung]
\label{satz:2.26}
Seien $A,B\subset \MdR^n$ nicht leere Mengen. Dann gilt \[
\lambda^n(A+B)^{\frac1n} \ge \lambda^n(A)^{\frac1n} + \lambda^n(B)^{\frac1n},
\]
wobei $A+B \da \{x+y : x\in A,\, y\in B\}$.
\end{satz}

\begin{beispiel}
Sei $B= sA$, $s>0$.
\begin{multline*}
\lambda^n(A + sA)^{\frac1n} = \lambda^n( (1+s)A )^{\frac1n} = (1+s)\lambda^n(A)^{\frac1n} =
\\
\lambda^n(A)^{\frac1n} + (s^n\lambda^n(A))^{\frac1n} = \lambda^n(A)^{\frac1n} + \lambda^n(sA)^{\frac1n} = \lambda^n(A)^{\frac1n} + \lambda^n(B)^{\frac1n}.
\end{multline*}
\end{beispiel}

\begin{beweis}
Seien zunächst $A=I_1\times \cdots \times I_n$, $B=J_1\times\cdots\times J_n$ mit beschränken Intervallen $I_j,J_j\subset\MdR$ mit nichtleerem Inneren. Dann gilt
$A + B = (I_1 + J_1) \times \cdots \times (I_n + J_n)$. Wir setzen $\lambda \da \lambda^1$, $u_k = \frac{\lambda(I_k)}{\lambda(I_k + J_k)}$ und $v_k = \frac{\lambda(J_k)}{\lambda(I_k + J_k)}$ für $k=1,\ldots,n$. Dann folgt
\begin{align*}
\frac{\lambda^n(A)^{\frac1n} + \lambda^n(B)^{\frac1n}}{\lambda^n(A+B)^{\frac1n}}
&= \frac{(\prod_{k=1}^n \lambda(I_k))^{\frac1n} + (\prod_{k=1}^n \lambda(J_k))^{\frac1n}}{\prod_{k=1}^n \lambda(I_k+J_k)^{\frac1n}} \\
&=\prod_{k=1}^n u_k^{\frac1n} + \prod_{k=1}^n v_k^{\frac1n} \\
&= e^{\sum_{k=1}^n \frac1n \ln u_k} + e^{\sum_{k=1}^n \frac1n \ln v_k} \\
&\le e^{\ln(\sum_{k=1}^n \frac1n u_k)} + e^{\ln(\sum_{k=1}^n \frac1n v_k)}\\
&= \frac1n \sum_{k=1}^n \underbrace{(u_k + v_k)}_{=1} = 1.
\end{align*}
Hier wurde verwendet, dass $\ln$ konkav und die Exponentialfunktion monoton wachsend ist. 


Seien nun $\mathcal G, \mathcal F$ endliche Familien von paarweise disjunkten Elementen aus
\[
\{ [a_1,b_1) \times \cdots \times [a_n,b_n) : -\infty < a_i < b_i < \infty \ \forall i=1,\ldots,n\}.
\]
Sei $A \da \bigcup_{P\in\mathcal G}P$, $B\da\bigcup_{Q\in\mathcal F}Q$. Durch vollständige Induktion über $\#\mathcal G + \#\mathcal F$ wird nun gezeigt, dass die Brunn-Minkowski-Ungleichung für solche Mengen $A,B$ gilt. Der Induktionsanfang ist bereits oben gegeben.

Die Brunn-Minkowski-Ungleichung gelte für $\#\mathcal G\ge 1$, $\#\mathcal F\ge 1$ mit $\#\mathcal G + \#\mathcal F \le p$ für ein $p\ge 2$. Sei nun $\#\mathcal G + \#\mathcal F \leq  p+1$ (ohne Beschränkung der Allgemeinheit $\#\mathcal G>1$). Wähle ein $i\in\{1,\ldots,m\}$ und $a\in\MdR$ derart, dass 
\begin{align*}
A_1 \da \{x = (x_1,\dots,x_n)\in A : x_i < a\}, &&
A_2 \da \{x = (x_1,\dots,x_n)\in A : x_i \ge a\}
\end{align*}
jeweils mindestens ein Element aus $\mathcal G$ enthalten. Wähle dann $b\in\MdR$ derart, dass für
\begin{align*}
B_1 \da \{x = (x_1,\dots,x_n)\in A : x_i < b\}, && 
B_2 \da \{x = (x_1,\dots,x_n)\in A : x_i \ge b\}
\end{align*}
gilt:
\begin{align*}
\frac{\lambda^n(B_i)}{\lambda^n(B)} = \frac{\lambda^n(A_i)}{\lambda^n(A)}
\end{align*}
für $i=1,2$.

Sei $\mathcal G_i \da \{P \cap A_i: P\in\mathcal G,\, P\cap A_i\ne \emptyset\}$ und $\mathcal F_i \da \{Q \cap B_i: Q\in\mathcal F,\, Q\cap B_i\ne \emptyset\}$ für $i=1,2$. Dann ist $A_i = \bigcup_{P\in \mathcal G_i}P$, $B_i = \bigcup_{G\in \mathcal F_i}Q$, $\#\mathcal G_i < \#\mathcal G$ und $\#\mathcal F_i \le \#\mathcal F$ für $i=1,2$. Folglich gilt die Induktionsannahme für $(A_i,B_i)$, $i=1,2$. Die Mengen $A_1 + B_1$ und $A_2+B_2$ werden durch die Hyperebene $\{x\in\MdR^n: x_i = a+b\}$ getrennt. Also ist wegen
\[
A+B = (A_1\cup A_2) + (B_1\cup B_2) \supset (A_1 + B_1) \cup (A_2 + B_2)
\]
und der Induktionsannahme
\begin{align*}
\lambda^n(A+B)
&\ge \lambda^n(A_1+B_1) + \lambda^n(A_2+B_2) \\
&\ge \left(\lambda^n(A_1)^{\frac 1n} + \lambda^n(B_1)^{\frac 1n}\right)^n + \left(\lambda^n(A_2)^{\frac 1n} + \lambda^n(B_2)^{\frac 1n}\right)^n \\
&= \left(\left(\frac{\lambda^n(A)}{\lambda^n(B)}\lambda^n(B_1)\right)^{\frac 1n} + \lambda^n(B_1)^{\frac 1n}\right)^n + \left(\left(\frac{\lambda^n(A)}{\lambda^n(B)}\lambda^n(B_2)\right)^{\frac 1n} + \lambda^n(B_2)^{\frac 1n}\right)^n\\
&= \lambda^n(B_1)\left( \frac{\lambda^n(A)^{\frac 1n}}{\lambda^n(B)^{\frac 1n}} + 1 \right)^n +\lambda^n(B_2)\left( \frac{\lambda^n(A)^{\frac 1n}}{\lambda^n(B)^{\frac 1n}} + 1 \right)^n\\
&= \underbrace{(\lambda^n(B_1) + \lambda^n(B_2))}_{=\lambda^n(B)} \cdot \left( \frac{\lambda^n(A)^{\frac 1n} + \lambda^n(B)^{\frac 1n}}{\lambda^n(B)^{\frac 1n}}\right)^n \\
&= \left( \lambda^n(A)^{\frac1n} + \lambda^n(B)^{\frac1n} \right)^n,
\end{align*}
d.h. die behauptete Ungleichung im betrachteten Fall.

Seien schließlich $A,B$ nichtleer und kompakt. Setze für $p\in\MdN$
\begin{align*}
\mathcal C_p &\da \{[0,2^{-p}]^n + 2^{-p}\cdot z : z\in\MdZ^n\}, \\
A_p &\da \bigcup_{\substack{C\in\mathcal C_p \\ C\cap A \ne \emptyset}} C, \\
B_p &\da \bigcup_{\substack{C\in\mathcal C_p \\ C\cap B \ne \emptyset}} C.
\end{align*}
Dann gilt $A_p \searrow A$, $B_p \searrow B$ für $p\to\infty$. Außerdem ist $A+B$ kompakt und $A_p + B_p \searrow A+ B$. Es folgt
\[
\lambda^n(A+B) = \lim_{p\to\infty} \lambda^n(A_p + B_p) 
\ge \lim_{p\to\infty} \left( \lambda^n(A_p)^{\frac 1n} + \lambda_n(B_p)^{\frac1n} \right)^n
=  \left( \lambda^n(A)^{\frac 1n} + \lambda_n(B)^{\frac1n} \right)^n.
\]
Für beliebige beschränkte Mengen folgt die Behauptung nun aus der inneren Regularität 
des Lebesguemaßes (vgl. Satz \ref{satz:2.2}), die allgemeine Aussage erhält man 
mit Hilfe von Proposition \ref{prop1.3} a).
\end{beweis}

\begin{satz}[Isodiametrische Ungleichung]
\label{satz:2.27}
Für jede Menge$A\subset \MdR^n$ gilt
\[
\lambda^n(A)  \le \frac{\lambda^n(B(0,1))}{2^n} \cdot \diam(A)^n.
\]
\end{satz}

\begin{beweis}
Sei ohne Beschränkung der Allgemeinheit $A$ beschränkt und kompakt, da $\diam(A) = \diam(\bar A)$ und $\lambda^n(A) \le \lambda^n(\bar A)$. Setze $A-A \da A + (-A) = \{ x-y : x,y\in A\}$. Aus Satz \ref{satz:2.26} folgt
\begin{align*}
\lambda^n(\frac12(A-A)) = \frac1{2^n} \lambda^n(A + (-A)) \ge \frac 1{2^n} \left(\lambda^n(A)^{\frac1n} + {\underbrace{\lambda^n(-A)}_{=\lambda^n(A)}}^{\frac1n}\right)^n = \lambda^n(A).
\end{align*}
Ist $z\in\frac12(A-A)$, also $z=\frac12(x-y)$ mit $x,y\in A$, so ist $|z| = \frac 12 |x-y| \le \frac12 \diam(A)$. Folglich ist $\frac12(A-A) \subset B(0,\frac12\diam(A))$, also
\begin{align*}
\lambda^n(A) \le \lambda^n(\frac12(A-A)) \le \lambda^n(B(0,\frac12\diam(A))) \le \frac1{2^n} \diam(A)^n \cdot \lambda^n(B(0,1)).
\end{align*}
\end{beweis}

\begin{bemerkung}
\begin{itemize}
\item Ein alternativer Beweis der vorangehenden beiden Ungleichungen kann mit Hilfe 
von Steinersymmetrisierung erfolgen.
\item In den obigen beiden Ungleichungen ist es natürlich, nach dem Gleichheitsfall 
zu fragen. Die Schwierigkeit bei der Beantwortung der Frage hängt dann wesentlich von 
der betrachteten Mengenklasse ab.
\end{itemize}
\end{bemerkung}

\begin{definition}
Für \(\delta > 0, p>0\) und \(M \subseteq \mathbb R^n\) sei
\[
\HM^p_\delta \da \inf\left\{ \sum_{A\in\mathcal A} \alpha(p) 2^{-p} \diam(A)^p \colon \mathcal A \in \Omega_\delta(M) \right\}
\]
und
\[
\HM^p(M) \da \lim_{\varepsilon\searrow0} \HM^p_\varepsilon(M).
\]
Notation: Für \(p>0\) sei
\[
\alpha(p) \da \frac{\pi^{\frac{p}2}}{\Gamma(\frac{p}2+1)} .
\]
\end{definition}

Für $p\in\mathbb{N}$ ist dann  \(\alpha(p) = \lambda^p(B^p(0,1))\) mit \(B^p(0,1) \da \{ x \in \mathbb R^p \colon \|x\| \leq 1 \}\).

\begin{lemma}\label{lem:2.28}
Im euklidischen Raum \(\mathbb R^n\) gilt: \(\HM^n \ll \lambda^n\).
\end{lemma}
\begin{beweis}
Ein allgemeines Argument folgt aus Übungsblatt 6, Nr. 1.
\par
Die überdeckenden Mengen \(A=(a_1,b_1) \times \dots \times (a_n,b_n)\) bei der Definition des Maßes \(\lambda^n\) können so gewählt werden, dass \(\diam(A) < \delta\) für ein vorgegebenes \(\delta > 0\) und
\begin{equation}\label{lowb}
 |a_i-b_i| \leq 2 \min\{|b_j-a_j| \colon 1 \leq j \leq n\} = 2|b_{i_0}-a_{i_0}|, \; i_0 \in \{1,\dots,n\}.
\end{equation}
Dann gilt:
\[
\lambda^n(A) = \prod_{i=1}^n(b_i-a_i) \geq (b_{i_0}-a_{i_0})^n
\]
und
\begin{align*}
\diam(A)^n &= \left( \sum_{i=1}^n (b_i-a_i)^2 \right)^{\frac{n}2} \\
&\leq 2^n {\sqrt n}^n |b_{i_0}-a_{i_0}|^n \\
&\leq 2^n n^{\frac{n}2} \lambda^n(A).
\end{align*}
Sei nun \(M \subseteq \mathbb R^n\) mit \(\lambda^n(M)=0\). Sei \(\delta > 0\) fest. Zu \(\varepsilon > 0\) gibt es \(A_i \subseteq \mathbb R^n\), wie oben beschrieben, d.h. \(\diam(A_i) < \delta\) und mit \eqref{lowb}, \(M \subseteq \bigcup_{i=1}^\infty A_i\) und \(\sum_{i=1}^\infty \lambda(A_i) < \varepsilon\). Dann folgt
$$ \sum_{i=1}^\infty \diam(A_i)^n \leq 2^n \cdot n^{\frac{n}2} \cdot \varepsilon 
$$
also
$$ 
\HM^n_\delta(M) \leq \alpha(n) \cdot  n^{\frac{n}2} \cdot \varepsilon. 
$$
Da \(\varepsilon\) beliebig war, ist \(\HM^n_\delta(M)=0\). Folglich ist auch \(\HM^n(M)=0\).
\end{beweis}

\begin{satz}
In \(\mathbb R^n\) gilt: \(\HM^n = \lambda^n = \HM^n_\delta\) für \(\delta > 0\).
\end{satz}
\begin{beweis}
Sei \(M \subseteq \mathbb R^n\) und $\delta>0$. Zu \(\varepsilon > 0\) existiert eine offene Menge \(U\) mit \(M \subseteq U\) und \(\lambda^n(U) \leq \lambda^n(M) + \varepsilon\) (vgl. Satz \ref{satz:2.2} c)).
\par
Sei \(\mathcal S\) die Menge aller Kugeln in \(\mathbb R^n\) mit Radius kleiner als \(\frac{\delta}2\). Wegen Korollar \ref{kor:2.9} und Übung Nr. 1 auf Blatt 4 gibt es eine abzählbare Folge \((B_i)_{i\in\mathbb N}\) in \(\mathcal S\) paarweise disjunkter Kugeln in \(U\) mit \(\lambda^n(M \setminus \bigcup_{i=1}^\infty B_i) = 0\). Mit Lemma \ref{lem:2.28} folgt
\begin{align*}
\HM^n_\delta(M) &\leq \underbrace{\HM^n_\delta(M\setminus \bigcup_{i=1}^\infty B_i)}_{=0} + \HM^n_\delta(\bigcup_{i=1}^\infty B_i) \\
&\leq \sum_{i\geq1} \underbrace{ \alpha(n) \cdot 2^{-n} \diam(B_i)^n }_{= \lambda^n(B_i)} \\
&= \lambda^n(\bigcup_{i\geq1} B_i) \\
&\leq \lambda^n(U) \\
&\leq \lambda^n(M) + \varepsilon
\end{align*}
folglich also
\[
\HM^n_\delta(M) \leq \HM^n(M) \leq \lambda^n(M).
\]
Sei \((A_i)_{i\in\mathbb N} \in \Omega_\delta(M)\). Dann ist mit Satz \ref{satz:2.27}
\[
\lambda^n(M) \leq \lambda^n(\bigcup_{i\geq1} A_i) \leq \sum_{i\geq1} \lambda^n(A_i) \leq \sum_{i\geq1}. \alpha(n)\cdot 2^{-n} \diam(A_i)^n
\]
Wir erhalten
\[
\lambda^n(M) \leq \HM^n_\delta(M) \leq \HM^n(M) \leq \lambda^n(M),
\]
das heißt die Behauptung.
\end{beweis}

\chapter{Lipschitzfunktionen und Rektifizierbarkeit}

\section{Fortsetzbarkeit und Differenzierbarkeit von Lipschitzfunktionen}

\begin{definition}
Seien \((X,d), (\bar X,\bar d)\) metrische Räume, \(f: A\rightarrow \bar X, \emptyset \neq A \subseteq X\). 
Setze
\[
\Lip(f) \da \sup \{ \frac{\bar d(f(x),f(y))}{d(x,y)} \colon x,y \in A, x\neq y\}.
\]
Man nennt \(f\) eine \emph{Lipschitz-Abbildung}, falls \(\Lip(f) < \infty\).
\end{definition}

\begin{satz}
Sei \((X,d)\) metrischer Raum, \(\emptyset \neq A \subseteq X\)
\begin{enumerate}[(1)]
\item Ist \(f: A \rightarrow \mathbb R\) Lipschitzfunktion, so ist
\[
g(x) \da \inf\{f(z)+Lip(f) \cdot d(x,z) \colon z\in A\} \quad (x\in X)
\]
eine Lipschitzfunktion mit \(\Lip(g) = \Lip(f)\) und \(g|_A=f\).
\item Ist \(f: A \rightarrow \mathbb R^n\) eine Lipschitzfunktion, so gibt es eine Lipschitzfunktion \(h: X \rightarrow \mathbb R^n\) mit \(\Lip(h) \leq \sqrt n \Lip(f)\) und \(h|_A = f\).
\end{enumerate}
\end{satz}

\begin{bemerkung}
\begin{enumerate}[(1)]
\item Johnson, Lindenstrauss und Schechtman '86 zeigen, dass im Allgemeinen nicht \(\Lip(h) = \Lip(f)\) möglich ist. 
Siehe auch Lang '99.
\item Ist \(f: A \subseteq \mathbb R^n \rightarrow \mathbb R^n\) Lipschitzfunktion, so existiert eine Lipschitz-Fortsetzung mit gleicher Lipschitzkonstante (Satz von Kirszbraun '34, Valentine '45).
\end{enumerate}
\end{bemerkung}

\begin{beweis}
\begin{enumerate}[(1)]
\item Sei \(f: A\rightarrow \mathbb R\). Für \(x,y \in X\) ist
\begin{align*}
g(x) &\leq \inf\{f(x)+(d(x,y)+d(y,z))\Lip(f) \colon z \in A\} \\
&\leq g(y) + \Lip(f)\cdot d(x,y)
\end{align*}
und aus Symmetriegründen also \(|g(x)-g(y)| \leq \Lip(f) d(x,y)\). Insbesondere ist \(\Lip(g) \leq \Lip(f)\). Für \(x\in A\) gilt
\[
f(x) + \Lip(f) d(x,x) = f(x) \leq f(z)+ \Lip(f) d(x,z)
\]
also \(g(x) \leq f(x) \leq g(x)\), d.h. \(g(x) = f(x)\) für \(x\in A\). Somit ist \(\Lip(g) \geq \Lip(f)\).
\item Wir haben \(f=(f_1,\ldots,f_n)^T, f_i: A \rightarrow \mathbb R, \Lip(f_i) \leq \Lip(f)\). Zu \(i\in \{1,\ldots,n\}\) gibt es nach (1) eine Lipschitz-Fortsetzung \(h_i: X \rightarrow \mathbb R\) von \(f_i\) mit \(\Lip(h_i)=\Lip(f_i)\). Dann ist \(h\da (h_1,\ldots,h_n)^T\) eine Lipschitz-Fortsetzung von \(f\) und \(\Lip(h) \leq \sqrt n \Lip(f)\), denn
\begin{align*}
\|h(x)-h(y)\| &= \left( \sum_{i=1}^n (h_i(x)-h_i(y))^2 \right)^{\frac12} \\
&\leq \left( \sum_{i=1}^n \Lip(h_i)^2 d(x,y)^2 \right)^{\frac12} \\
&\leq \Lip(f) \cdot (n d(x,y)^2)^{\frac12} \\
&= \sqrt n \Lip(f) d(x,y)
\end{align*}
\end{enumerate}
\end{beweis}

\begin{satz}[Kirszbraun, Valentine]
Seien \((\HR_1, \langle,\rangle_1)\) und \((\HR_2, \langle,\rangle_2)\) Hilberträume, \(\emptyset \neq D \subseteq \HR_1\),
\(f:D\rightarrow \HR_2\) eine Lipschitz-Abbildung. Dann existiert eine Lipschitz-Fortsetzung \(h\) von \(f\) mit \(\Lip(h)=\Lip(f)\) und \(h|_D = f\).
\end{satz}
\begin{beweis}[Reich und Simons '05]
Es genügt, den Fall \(\HR_1 = \HR_2 = \HR\) zu betrachten, sowie \(\Lip(f)=1\).

\emph{Skizze}: \(\HR \da \HR_1 \oplus \HR_2 \da \{ (x,y) \colon x \in \HR_1, y \in \HR_2\}, \langle (x_1,y_1), (x_2,y_2) \rangle \da \langle x_1,x_2\rangle_1 + \langle y_1,y_2 \rangle_2\), \((x_i,y_i) \in \HR_1 \oplus \HR_2\).
Zu \(f:D \subseteq \HR_1 \rightarrow \HR_2\) betrachte \(\tilde f: D \oplus \HR_2 \rightarrow \HR_1 \oplus \HR_2\), \((x,y) \mapsto (0,f(x))\), usw. (Übung)

Sei nun \(f: D \subseteq \HR \rightarrow \HR\) mit \(\Lip(f)=1\). Zu \(f\) gebe es keine echte 1-Lipschitz-Fortsetzung. Seien
\[\HR^2 \da \HR \oplus \HR = \HR \times \HR\]
und \[\chi: \HR^2 \rightarrow (-\infty, \infty], \qquad\chi(x,y) \da \sup\{ \|y-f(z)\|^2 - \|x-z\|^2 \colon z \in D\}.\]

\textbf{Lemma A:}
Es gilt \(\chi \geq 0\) und \(\{\chi=0\} = G(f) \da \{(z,f(z)) \colon z \in D\}\).

\textbf{Beweis des Lemmas:}
Seien \(x,y \in \HR\). Ist \(x \in D\), so ergibt die Wahl \(z \da x \in D\)
\[\chi(x,y) \geq \|y-f(x)\|^2 - \|x-x\|^2 = \|y-f(x)\|^2 \geq 0.\] 
Sei nun \(x \notin D\). Sei \(\tilde f\) die Fortsetzung von \(f\) auf \(D \cup \{x\}\) mit \(\tilde f(x) \da y\). Da \(\tilde f\) keine 1-Lipschitz-Abbildung ist, muss es ein \(z \in D\) geben mit
\[
\|y-f(z)\|^2 = \|\tilde f(x) - \tilde f(z)\|^2 > \|x-z\|^2,
\]
also ist \(\chi(x,y) > 0\). Damit ist \(\chi \geq 0 \) und \(\{\chi = 0\} \subseteq G(f)\) gezeigt.
\par
Sei nun \((x,y) \in G(f)\), d.h. \(x\in D\) und \(y=f(x)\). Also ist für $z\in D$
\[
\|y-f(z)\|^2 = \|f(x)-f(z)\|^2 \leq \|x-z\|^2.
\]
Dies ergibt \(\chi(x,y) \leq 0\) und daher \( \chi(x,y) = 0\), d.h. \(G(f) \subseteq \{\chi=0\}\).

\textbf{Lemma B:}
Definiere die Abbildung \(\varphi: \HR^2 \rightarrow (-\infty, \infty]\),\quad \(\varphi(x,y) \da \frac14 \chi(x+y,x-y) + \langle x,y \rangle\) für \( x,y\in\HR\).
\begin{enumerate}[(1)]
\item Für \(x,y \in \HR\) gilt
\[
4\cdot \varphi(x,y) = \sup\{ \|f(z)\|^2 - \|z\|^2 + 2 \langle x,z-f(z) \rangle + 2 \langle y,z+f(z) \rangle \colon z \in D\}.
\]
\item Für \(z \in D\) gilt
\[
4 \cdot \varphi(\frac{z+f(z)}2, \frac{z-f(z)}2) = \|z\|^2 - \|f(z)\|^2.
\]
\item \(\varphi\) ist eine eigentliche (\(\varphi \not\equiv \infty\)), unterhalbstetige, konvexe Funktion mit \(\varphi^\ast(x,y) \geq \varphi(y,x)\) für \(x,y\in \HR\). Hierbei ist \(\varphi^\ast\) die zu \(\varphi\) konjugierte Funktion, die erklärt ist durch
\[
\varphi^\ast(\zeta) \da \sup\{ \langle \zeta,\xi \rangle - \varphi(\xi) \colon \xi \in \HR^2 \},\qquad  \zeta \in \HR^2.
\]
\end{enumerate}

\textbf{Beweis des Lemmas:}
\begin{enumerate}[(1)]
\item Für \(x,y\in \HR, z \in D\) gilt:
\begin{align*}
&\|x-y-f(z)\|^2 - \|x+y-z\|^2\\
=& -4\langle x,y\rangle - 2\langle x-y, f(z)\rangle + 2\langle x+y, z\rangle + \|f(z)\|^2 - \|z\|^2 \\
=& -4\langle x,y\rangle + 2\langle x,z-f(z)\rangle + 2\langle y,z+f(z)\rangle + \|f(z)\|^2-\|z\|^2.
\end{align*}
Bildung des Supremums über $z\in D$ ergibt die Behauptung.
\item Direkt durch Einsetzen in die Definition erhält man für $z\in D$
\[
4\varphi(\frac{z+f(z)}2, \frac{z-f(z)}2) = \underbrace{\chi(z,f(z))}_{=0} + \langle z+f(z),z-f(z)\rangle = \|z\|^2 - \|f(z)\|^2.
\]
\item Aus der rechten Seite von (1) erkennt man, dass $\varphi$ als Supremum von stetigen, konvexen Funktionen  konvex und unterhalbstetig ist. Aus (2) folgt, dass $\varphi\not\equiv \infty$ gilt. Für $x,y\in\HR$ und $z\in D$ 
gilt
\begin{align*}
\varphi^*(x,y)&\ge \left\langle \left(\frac{z+f(z)}{2}, \frac{z-f(z)}{2}\right),(x,y)\right\rangle -\varphi\left(\frac{z+f(z)}{2}, \frac{z-f(z)}{2}\right)\\
&=\frac{1}{4}\left(2\langle x,z+f(z)\rangle +2\langle y,z-f(z)\rangle+\|f(z)\|^2-\|z\|^2\right),
\end{align*}
wobei zuletzt (2) verwendet wurde. Aufgrund von (1) ist das Supremum über $z\in D$ auf der rechten Seite 
gerade $\varphi(y,x)$. 
\end{enumerate}

\textbf{Lemma C:} Sei $\tilde\HR$ ein Hilbertraum, $h(\zeta) \da \frac12\|\zeta\|^2$, $\zeta\in\tilde\HR$. Sei ferner $\psi : \tilde\HR \to (-\infty,\infty]$ eine eigentliche, unterhalbstetige konvexe Funktion. Gilt $\psi(\zeta) + h(\zeta) \ge 0 $ für alle $\zeta\in\tilde\HR$, dann gibt es ein $v\in\tilde\HR$ mit $\psi^*(v)+ h(v) \le 0$.

\textbf{Beweis des Lemmas:} Nach Voraussetzung gilt $\psi(\zeta) \ge -h(\zeta)$ für $\zeta\in\tilde\HR$. Sei
\[
\epi(\psi) \da \{ (\zeta,t) \in\tilde\HR\times\MdR: \psi(\zeta)\le t\}, \quad 
\widetilde\epi(-h) \da \{ (\zeta,s) \in\tilde\HR\times\MdR: s \le - h(\zeta) \}.
\]
Dann sind $\epi(\psi)$, $\widetilde\epi(-h)$ abgeschlossene, konvexe, nicht leere Mengen, die keine gemeinsamen inneren Punkt haben. Dann gibt es eine „nicht vertikale“ trennende Hyperebene, das heißt es gibt ein $\alpha \in \MdR$ und $v\in\tilde\HR$ mit:
\[
\psi(\zeta) \ge \langle \zeta, v \rangle + \alpha \ge -h(\zeta)
\]
für alle $\zeta\in\tilde\HR$.
Insbesondere 
\begin{align*}
\psi^*(v) &= \sup\{\langle\zeta,v\rangle - \psi(\zeta) : \zeta \in \tilde\HR\} \\
&\le -\alpha \\
&\le \inf\{\langle \zeta,v\rangle + h(\zeta) : \zeta \in \tilde\HR\}\\
&\le \langle -v, v\rangle + \frac12 \|v\|^2\\
&= -\frac12 \|v\|^2 = -h(v),
\end{align*}
also $\psi^*(v)+ h(v) \le 0$.

\textbf{Lemma D}: Hat $f:D\to \HR$ keine echte 1-Lipschitzfortsetzung, so gilt $D=\HR$.

\textbf{Beweis des Lemmas:} Sei $h(x,y) \da \frac12\|(x,y)\|^2$, $(x,y)\in\HR^2$. Es gilt für $x,y\in\HR$:
\begin{align*}
4(\varphi(x,y) + h(x,y)) &= \chi(x+y,x-y) + 4\langle x,y\rangle + 2\|x\|^2 + 2\|y\|^2 \\
&= \chi(x+y,x-y) + 2\|x+y\|^2 \ge 0.
\end{align*}
Wegen Lemma B gilt somit
\[
\varphi^*(y,x) + h(y,x) \ge \varphi(x,y) + h(x,y) \ge0
\]
für alle $x,y\in\HR$. Nach Lemma C gibt es ein $(x_0,y_0)\in\HR^2$ mit
$\varphi(y_0,x_0) + h(y_0,x_0) \le 0$, das heißt $\varphi^*(y_0,x_0) + h(y_0,x_0) = 0 = \varphi(x_0,y_0)+h(x_0,y_0)$ also $\chi(x_0+y_0, x_0-y_0) + 2\|x_0+y_0\|^2 = 0$ und somit $x_0=-y_0$ und $\chi(0,2x_0)= 0$. Dies führt wegen Lemma A auf $(0,2x_0)\in G(f)$, das heißt $0\in D$.

Sei nun $x_1\in\HR$ beliebig. Definiere $f_{x_1} : D-x_1 \to \HR$ durch $f_{x_1}(x) \da f(x+x_1)$. Da $f_{x_1}$ keine echte 1-Lipschitzfortsetzung hat, muss $0\in  = D-x_1$ gelten, also $x_1\in D$ und somit $D=\HR$.

\textbf{Beweis des Satzes:}
Sei $f:D \to \HR$ eine 1-Lipschitzabbildung mit $\emptyset \ne D \subset \HR$. Betrachte
\[
\mathcal S \da \{(E,g) : D\subset E,\, g|_D = f,\, g\text{ ist 1-Lipschitzabbildung}\}.
\]
Durch 
\[
(E,g)\prec(\tilde E, \tilde g) :\iff E\subset \tilde E \text{ und } \tilde g|_E = g
\]
für $(E,g),(\tilde E,\tilde g)\in \mathcal S$ wird auf $\mathcal S$ eine Ordnung eingeführt, in der jede Kette eine obere Schranke besitzt. Nach dem Zornschen Lemma gibt es ein maximales Element in $\mathcal S$, etwa $(E,g)$. Dann ist wegen Lemma D aber $E=\HR$, das heißt $(\HR,g)$ ist die gesuchte 1-Lipschitzfortsetzung von $f$.
\end{beweis}

Eine Abbildung $f:\MdR^m \to \MdR^n$ heißt differenzierbar an der Stelle $x\in\MdR^m$, falls es eine lineare Abbildung $L:\MdR^m\to\MdR^n$ gibt mit
\[
\lim_{y\to x} \frac{f(y)-f(x)- L(y-x)}{\|y-x\|} = 0.
\]
In diesem Fall schreiben wir $Df(x) \da Df_x \da L$. Ferner ist
\[
 D_uf(x)\da Df_x(u) = \lim_{t\to0}\frac{f(x+tu) - f(x)}{t}
\]
für $u\in\MdR^m$.



\begin{satz}[Rademacher]
Jede Lipschitzfunktion $f:\MdR^m \to \MdR^n$ ist $\lambda^m$-fast-überall differenzierbar.
\end{satz}

\begin{beweis}
Sei stets ohne Beschränkung der Allgemeinheit $n=1$.
\begin{enumerate}[{Teil} 1:]
\item $m=1$ und $f$ ist monoton wachsend. Definiere
\[
\nu_f(M) \da \inf\{\sum_{i=1}^\infty (\underbrace{f(b_i) - f(a_i)}_{\ge 0}): a_i < b_i,\, M\subset \bigcup_{i=1}^\infty (a_i,b_i)\}.
\]
Wie im Fall $\lambda^1$ ist $\nu_f\in\mathbb M (\MdR)$. Beachte hierzu $\nu_f(M) \le \Lip(f) \cdot \lambda^1(M)$. Dies zeigt auch $\nu_f \ll \lambda^1$ und daher $(\nu_f)_{\lambda^1} = \nu_f$. Das System
\[
V \da \{ (x,[a,b]): x\in[a,b]\}
\]
ist eine $\lambda^1$-Vitali-Relation. Also ist $\lambda^1$-fast-überall $\mathbb D(\nu_f,\lambda^1,V,\cdot) \in [0,\infty)$  und
\[
\mathbb D(\nu_f,\lambda^1,V,x)
= V\text-\lim_{[a,b]\to x} \frac{\nu_f([a,b])}{\lambda^1([a,b])}
= \lim_{y\to x} \frac{f(y) - f(x) }{y-x} = f'(x)
\]
für $\lambda^1$-fast-alle $x\in\MdR$.

\item $m=1$, $f:\MdR\to\MdR$ nicht notwendigerweise monoton. Für $x\in \MdR$ sei
\[
g(x) \da 
\begin{cases}
\phantom{-}\sup\{\sum_{i=1}^m|f(a_i)-f(a_{i-1})| : 0 = a_0 \le a_1 \le \cdots \le a_m = x, \, m\in\MdN\}, & x\ge 0,\\
-\sup\{\sum_{i=1}^m|f(a_i)-f(a_{i-1})| : x = a_0 \le a_1 \le \cdots \le a_m = 0, \, m\in\MdN\}, & x\ge 0.
\end{cases}
\]
Damit ist $g$ monoton wachsend. Für $x>y$ gilt
\begin{align*}
g(x)
&= g(y) + \sup\{\sum_{i=1}^m |f(a_i) - f(a_{i-1})|: y = a_0\le a_1\le \cdots \le a_m = x,\, m\in\MdN\} \\
&\le g(y) + \Lip(f) \cdot \sup\{\sum_{i=1}^m |a_i-a_{i-1}|: y = a_0\le a_1\le \cdots \le a_m = x,\, m\in\MdN\} \\
&= g(y) + \Lip(f) \cdot |y-x|
\end{align*}
also $|g(x) - g(y)| \le \Lip(f) \cdot |x-y|$. Außerdem erhalten wir für $x>y$, dass
\[
g(x)\ge g(y) + |f(x)-f(y)| \ge g(y) + f(x) - f(y),
\]
und somit $(g-f)(x) \ge (g-f)(y)$, das heißt $g-f$ ist ebenfalls monoton wachsend und ebenfalls Lipschitz. Nach Teil 1 sind daher $g$ und $g-f$ $\lambda^1$-fast-überall differenzierbar. Somit auch $f=g-(g-f)$.

\textbf{Nebenbemerkung:} Aus dem Beweis folgt zunächst im Fall einer monotonen Lipschitzfunktion 
\begin{align*}
f(b)-f(a) &= \nu_f([a,b]) = (\nu_f)_{\lambda^1}([a,b])\\
& = \int_{[a,b]} \mathbb D(\nu_f, \lambda^1, V, x)\lambda(dx)\\
& = \int_{[a,b]} f'(x) \lambda(dx).
\end{align*}
Die Gleichung
$$
f(b)-f(a)= \int_{[a,b]} f'(x) \lambda(dx)
$$
erhält man dann sogar für jede Lipschitzfunktion aufgrund der in Teil 2 beschriebenen Zerlegung. 
Dies ist der Hauptsatz der Differenzial- und Integrationsrechnung für Lipschitzfunktionen.

\item $m\ge 1$. Seien $e\in S^{m-1}$. Sei $N_e$ die Menge aller $x\in\MdR^m$, für die $t \mapsto f(x+te)$ in $t=0$ nicht differenzierbar ist. Dann ist $N_e$ eine Borelmenge und $\HM^1(N_e \cap (x+\MdR e)) = 0$ nach Teil 2. Der Satz von Fubini zeigt $\lambda^m(N_e) = 0$ für beliebige $e\in S^{m-1}$.

Für $\lambda^m$-fast-alle $x\in\MdR^m$ existiert dann 
\[
\nabla f(x) \da (D_1 f(x), \ldots, D_mf(x)),
\]
wobei $D_if(x) \da D_{e_i}f(x)$ für die Standardbasis $(e_1,\ldots,e_m)$ des $\MdR^m$. Für festes $e\in S^{m-1}$ existiert $D_ef(x)$ für $\lambda^m$-fast-alle $x\in\MdR^m$. Sei $\varphi \in\mathcal C^\infty_c(\MdR^m)$. Für $t>0$ und $e\in S^{m-1}$ gilt
\[
\int_{\MdR^m} \frac{f(x+t e) - f(x)}{t} \cdot \varphi(x) \lambda^m(dx)
= 
-\int f(x) \cdot \frac{\varphi(x) - \varphi(x-t e)}{t}\varphi(x) \lambda^m(dx).
\]
Wegen $|\varphi| \le \|\varphi\|_\infty \cdot \ind_{\supp(\varphi)}$ und da $f$ Lipschitz-stetig ist, erhält man   mit dem Satz von der majorisierten Konvergenz:
\begin{align*}
\int_{\MdR^m} D_ef(x) \varphi(x) \lambda^m(dx)
&= -\int f(x) D_e\varphi(x) \lambda^m(dx) \\
&= - \sum_{j=1}^m \langle e, e_j\rangle \cdot \int f(x) \cdot D_j \varphi(x)\lambda^m(dx) \\
&= - \sum_{j=1}^m \langle e, e_j\rangle \cdot (-1) \int_{\MdR^m} D_{e_j} f(x) \varphi(x) \lambda^m(dx) \\
&= \int_{\MdR^m} \langle e, \nabla f(x) \rangle \varphi(x) \lambda^m(dx).
\end{align*}
Hieraus liest man $D_ef(x) = \langle e,\nabla f(x) \rangle$ für $\lambda^m$-fast-alle $x\in\MdR^m$ ab, und zwar   
für ein beliebiges (aber festes) $e\in S^{m-1}$.

Sei $E\subset S^{m-1}$ eine abzählbare dichte Teilmenge. Dann ist das Komplement der Menge 
\[
A \da \bigcap_{e\in E}\{x\in\MdR^m : D_ef(x) = \langle e, \nabla f(x) \rangle\}
\]
eine $\lambda^m$-Nullmenge. Wir zeigen, dass $f$ in $x\in A$ differenzierbar ist. Sei also $x\in A$. Für $t>0$ und $e\in S^{m-1}$ sei
\[
\Delta_t(e) \da \frac{f(x+t e)- f(x)}{t} - \langle e,\nabla f(x) \rangle.
\]

Wir zeigen: $\Delta_t(e) \to 0$ für $t\to 0$ gleichmäßig in $e\in S^{m-1}$.

Setze $M\da 2\max\{\Lip(f), \|\nabla f(x)\|, 1\}$. Dann gilt für $e,\bar e\in S^{m-1}$:
\begin{align*}
|\Delta_t (e) - \Delta_t(\bar e)|
\le | \frac{f(x + t e) - f(x + t \bar e)}t | + \|\nabla f(x) \| \cdot \|e-\bar e\|
\le M \cdot \|e-\bar e\|.
\end{align*}
Ferner gilt $\Delta_t(e) \to 0$ für alle $e\in E$, wenn $t \to 0$. 
Sei nun $\ep>0$ beliebig vorgegeben. Dann gibt es eine endliche Teilmenge $\bar E\subset E$ mit $d(e,\bar E) \le \frac\ep{2M}$ für jedes $e\in S^{m-1}$. Bei fest gewählter Menge $\bar{E}$ gibt es ein $\delta>0$, so dass $|\Delta_{t} (\bar e) | \le \frac\ep2$ für alle $\bar e\in\bar E$ und $0<t\le \delta$. Für $e\in S^{m-1}$ und $0<t\le\delta$ gilt dann:
\begin{align*}
|\Delta_{t} (e)|
&\le \min_{\bar e\in\bar E} \{ |\Delta_{t}(e) - \Delta_{t}(\bar e)| + |\Delta_{t}(\bar e)|\} \\
&\le \min_{\bar e\in\bar E} \{ M \cdot \|e-\bar e\| + \frac\ep2\} \\
&\le \frac\ep{2M} \cdot M + \frac\ep2 = \ep.
\end{align*}
Dies zeigt $\Delta_t\to 0$ gleichmäßig für $t\to 0$, und daraus folgt die Behauptung.
\end{enumerate}
\end{beweis}

\section{Die Flächenformel}
Sei $U\subset\MdR^n$ offen und $f:U\to\MdR^n$ ein $C^1$-Diffeomorphismus. Dann ist für eine beliebige 
$\lambda^n$-messbare Funktion $g:\MdR^n\to [0,\infty]$
\begin{equation}
\int_U g\circ f(x)\cdot Jf(x) \lambda^n(dx)=\int_{f(U)}g(y)\lambda^n(dy),\tag{$*$}
\end{equation}
wobei $Jf(x):=|\det(Df(x))|$.

\textbf{Ansätze für Verallgemeinerungen:}
\begin{itemize}
\item Abbildungen $f:\MdR^m\to\MdR^n$ mit $m,n\in\MdN$ oder sogar allgemeiner $f:M\to N$ mit gewissen $m$-dimensionalen bzw. $n$-dimensionalen Mengen $M,N$.
\item $f$ nicht notwendig injektiv.
\item $f$ nicht notwendig differenzierbar, aber Lipschitz.
\end{itemize}

Ist speziell $A\subset U$ eine Borelmenge und $g(y):=\ind_{f(A)}(y)$, dann geht $(*)$ über in
\[
\int_A Jf(x)\lambda^n(dx)=\lambda^n(f(A)).
\]

Für eine lineare Abbildung ist dies fast trivial. Umformulierung ergibt
\[
\int_A Jf(x)\lambda^n(dx)=\int_{\MdR^n}\#\left(A\cap f^{-1}\{y\}\right)\lambda^n(dy),
\]
wobei
$$
\#\left(A\cap f^{-1}\{y\}\right)=\begin{cases} 1,&y\in f(A),\\0,&\text{ sonst}, 
\end{cases}
$$
da $f$ injektiv ist.

\begin{beispiel}
Wie sieht es dagegen in der folgenden Situation aus?
Es seien $n=1$, $f:[-1,1]\to\MdR$ mit $f(t) = 1-t^2$. Es ist $f'(t) = -2t$ und damit $Jf(t)=2|t|$.
\[
\int_{[-1,1]}Jf(y) dy = \int_{[-1,1]} 2|t|dt = 2 \cdot d\int_0^1 tdt = 4 \cdot \frac12 \cdot 1 = 2.
\]
Aber es ist $f([-1,1]) = [0,1]$ und $\int_{f(A)}dt = \lambda(f(A)) = 1$. Man beachte jedoch
\[
\int \underbrace{\#( [-1,1] \cap f^{-1}(\{t\}))}_{
=\begin{cases}
2, & t\in(0,1) \\
1, & t\in\{0,1\} \\
0, & \text{sonst}
\end{cases}
}=  2\cdot 1.
\]
\end{beispiel}

\begin{definition}
Eine lineare Abbildung $\varrho: \MdR^m \to \MdR^n$ heißt lineare Isometrie, falls
\[
\langle \varrho(x), \varrho(y) \rangle = \langle x,y \rangle
\]
für alle $x,y\in\MdR^m$.
\end{definition}

\begin{bemerkungen}
\item Eine lineare Isometrie ist injektiv und $\|\varrho(x) - \varrho(y)\| = \|x-y\|$ für alle $x,y\in\MdR^m$.
\item $\HM^t(\varrho(A)) = \HM^t(A)$ für alle $A\subset \MdR^m$ und $t\ge 0$.
\end{bemerkungen}

\begin{lemma}
\label{lem:3.4}
Sei $m\le n$ und $f:\MdR^m \to \MdR^n$ eine lineare Abbildung. Dann gibt es eine symmetrische lineare Abbildung $\sigma: \MdR^m\to\MdR^m$ und eine lineare Isometrie $\varrho:\MdR^m\to\MdR^n$ mit $f =\varrho \circ \sigma$. Ist $(a_1,\ldots,a_m)$ eine Orthonormalbasis von $\MdR^m$, so gilt
\[
\llbracket f \rrbracket \da |\det(\sigma)| = \sqrt{ \det(\langle f(a_i),f(a_j)\rangle)_{i,j=1,\ldots,m}},
\]
und $\HM^m(f(A)) = \llbracket f \rrbracket \cdot \HM^m(A)$ für $A\subset \MdR^m$.
\end{lemma}

\begin{beweis}
Wir definieren eine Hilfsabbildung $\varphi = f^* \circ f : \MdR^m\to\MdR^m$. Sie ist symmetrisch, da 
\[
\langle \varphi(x),y \rangle = \langle f^* \circ f(x),y\rangle = \langle f(x),f(y)\rangle,
\]
und $\varphi$ ist positiv semidefinit. Somit existiert eine Orthonormalbasis $a_1,\ldots,a_m$ aus Eigenvektoren von $\varphi$ mit Eigenwerten $\lambda_1,\ldots,\lambda_m\ge 0$. Wir setzten $b_i \da \frac{1}{\sqrt{\lambda_i}} \cdot f(a_i)$, falls $\lambda_i \ne 0$. Beachte dabei
\begin{align*}
\langle b_i,b_j \rangle 
= \frac1{\sqrt{\lambda_i\lambda_j}} \cdot \langle f(a_i),f(a_j)\rangle 
= \sqrt{\frac{\lambda_i}{\lambda_j}} \cdot \langle a_i,a_j\rangle = \delta_{ij},
\end{align*}
falls $\lambda_i,\lambda_j\ne 0$. Wir ergänzen diese Vektoren zu einer Orthonormalbasis $(b_1,\ldots,b_m)$ des $\MdR^m$.

Wir definieren nun $\sigma(a_i) \da \sqrt{\lambda_i}\cdot a_i$, $i=1,\ldots,m$. $\sigma$ ist symmetrisch, da $(a_1,\ldots,a_m)$ eine Orthonormalbasis ist. Wir definieren weiter $\varrho(a_i) \da b_i$, $i=1\ldots,m$. $\varrho$ ist eine lineare Isometrie. Nun gilt für $\lambda_i \ne 0$
\begin{align*}
\varrho\circ\sigma(a_i) = \varrho(\sqrt{\lambda_i}\cdot a_i) = \sqrt{\lambda_i} \cdot b_i  = \sqrt{\lambda_i} \cdot \frac{1}{\sqrt{\lambda_i}} \cdot f(a_i) = f(a_i),
\end{align*}
und für $\lambda_i = 0$ ist $\varrho\circ\sigma(a_i)=0$ und es gilt
\begin{align*}
\langle f(a_i),f(a_i)\rangle = \langle\varphi(a_i),a_i\rangle = \langle \lambda_i a_i,a_i\rangle = 0
\end{align*}
also $f(a_i) = 0$.

Seien $\varrho$, $\sigma$ und $(a_1,\ldots,a_m)$ wie in den Voraussetzungen des Lemmas gewählt:
\begin{align*}
\langle f(a_i), f(a_j)\rangle = \langle \varrho\circ \sigma(a_i), \varrho\circ\sigma(a_j) \rangle = \langle \sigma(a_i),\sigma(a_j)\rangle.
\end{align*}
Sei $S$ die beschreibende Matrix von $\sigma$ bezüglich $(a_1,\ldots,a_m)$. Dann:
\begin{align*}
(\det(\sigma))^2
&= \det(S^\top)\cdot \det(S) = \det(S^\top\cdot S) = \\
&= \det( (\langle \sigma(a_i), \sigma(a_j)\rangle)_{i,j=1,\ldots,m}) \\
&= \det( (\langle f(a_i), f(a_j)\rangle)_{i,j=1,\ldots,m}).
\end{align*}

Ferner gilt für $A\subset\MdR^m$:
\begin{align*}
\HM^m(f(A)) = \HM^m(\varrho\circ\sigma(A)) = \HM^m(\sigma(A)) = |\det(\sigma)| \cdot \HM^m(A)  = \llbracket f \rrbracket \cdot \HM^m(A).
\end{align*}
\end{beweis}

\begin{bemerkungen}
\item Sei
\[
\Lambda(n,m) \da \{\alpha:\{1,\ldots,m\} \to \{1,\ldots,n\} : \text{$\alpha$ ist streng monoton wachsend}\}
\]
für $m\le n$. Für $\alpha \in\Lambda(n,m)$ sei $p_\alpha: \MdR^n \to\MdR^m$ mit
\[
p_\alpha( (x_1,\ldots,x_n)^\top ) \da (x_{\alpha(1)},\ldots,x_{\alpha(m)})^\top
\]
erklärt. Dann gilt
\[
\llbracket f \rrbracket^2 = \sum_{\alpha\in \Lambda(n,m)}  \det(p_\alpha \circ f)^2.
\]
Dies ist der verallgemeinerte Satz des Pythagoras.
\item Sei $f:\MdR^m \to \MdR^n$ linear mit $f=\varrho\circ\sigma$ und $a_1,\ldots,a_m$ eine Orthonormalbasis. Dann ist
\begin{align*}
\llbracket f \rrbracket 
&= \det(\sigma(a_1),\ldots,\sigma(a_m)) \\
&\le \|\sigma(a_1)\|\cdot \cdots \cdot \|\sigma(a_m)\| \\
&= \|f(a_1)\|\cdot \cdots \cdot \|f(a_m)\|.
\end{align*}
\end{bemerkungen}

\begin{proposition}
\label{prop:3.5}
Seien $m\le n$, $f:\MdR^m\to\MdR^n$ eine Lipschitzabbildung und $A\in \borel(\MdR^m)$. Dann ist die Abbildung
\[
y\mapsto \#(A\cap f^{-1}(\{y\}))
\]
eine $\HM^m$-messbare Abbildung $\MdR^n\to [0,\infty]$ und
\[
\int_{\MdR^n} \#(A\cap f^{-1}(\{y\}))\, \HM^m(dy) \le \Lip(F)^m \cdot \HM^m(A).
\]
\end{proposition}

\begin{bemerkung}
Eine entsprechende Aussage gilt allgemein für $f:(X,d)\to (Y,\bar d)$, wobei $(X,d)$ ein polnischer, das heißt separabler, vollständiger metrischer Raum ist.
\end{bemerkung}

\begin{beweis}
Betrachte Folgen von Zerlegungen für $i\in \MdN$:
\begin{align*}
\mathcal C_i &\da  \{[0,2^{-i})^m + 2^i \cdot z: z \in \MdZ^m\},\\
\mathcal B_i &\da  \{C\cap A : C \in \mathcal C_i\}.
\end{align*}
Dies ist eine abzählbare disjunkte Zerlegung des $\MdR^m$ mit
\[
\sup\{\diam(C): C\in\mathcal C_i\} \to 0
\]
für $i\to \infty$, und ferner ist jedes $C\in\mathcal C_i$ disjunkte Vereinigung von gewissen Mengen 
$\tilde C\in\mathcal C_{i+1}$.

Zu $B\in\mathcal B_i$ existiert eine Folge $(K_j)_{j\in \MdN}$ kompakter Mengen mit $K_j \subset B$ und $\HM^m(B\setminus K_j) < \frac 1j$ und damit $\HM^m(B\setminus\bigcup_{j=1}^\infty K_j) = 0$.
Wegen 
\begin{align*}
f(B) \setminus \bigcup_{j=1}^\infty f(K_j) = f(B) \setminus f(\bigcup_{j=1}^\infty K_j) \subset f(B\setminus \bigcup_{j=1}^\infty K_j)
\end{align*}
folgt
\begin{align*}
0 \le \HM^m(f(B) \setminus \bigcup_{j=1}^\infty f(K_j)) \le \Lip(f)^m \cdot \HM^m(B\setminus \bigcup_{j=1}^\infty K_j) = 0.
\end{align*}

Da $f(K_j) \in \borel(\MdR^n) \subset \A_{\HM^m}$ und $f(B)\setminus \bigcup_{j=1}^\infty f(K_j) \in \A_{\HM^m}$ folgt $f(B) \in \A_{\HM^m}$. Wegen
\begin{align*}
\sum_{B\in\mathcal B_i} \ind_{f(B)} \nearrow \#(A\cap f^{-1}(\{y\}))
\end{align*}
für $i\to \infty$  folgt die Messbarkeitsbehauptung.

Mit dem Satz von der monotonen Konvergenz erhält man schließlich
\begin{align*}
\int_{\MdR^n} \#(A \cap f^{-1}(\{y\}) )\, \HM^m(dy) 
&= \int_{\MdR^n} \lim_{i\to\infty} \sum_{B\in\mathcal B_i} \ind_{f(B)}(y) \HM^m(dy) \\
&= \lim_{i\to\infty} \int_{\MdR^n}\sum_{B\in\mathcal B_i}\ind_{f(B)}(y) \HM^m(dy) \\
&= \lim_{i\to\infty} \sum_{B\in\mathcal B_i} \underbrace{\HM^m(f(B))}_{\le \Lip(f)^m\cdot \HM^m(B)}\\
&\le \Lip(f)^m \cdot \lim_{i\to\infty} \HM^m(\underbrace{\bigcup_{B\in\mathcal B_i} B}_{=A}) \\
&=   \Lip(f)^m \cdot \HM^m(A).
\end{align*}
\end{beweis}

\begin{lemma}
\label{lem:3.6}
Ist $f:\MdR^m\to\MdR^n$ stetig, so ist
\[
\{ x\in\MdR^m: f \text{ ist in $x$ differenzierbar}\} \in\borel(\MdR^m).
\]
\end{lemma}

\begin{beweis}
Übung
\end{beweis}

\begin{definition}
Sei $m\le n$ und $f:\MdR^m\to\MdR^n$ differenzierbar in $x\in\MdR^m$. Dann heißt
\[
Jf(x) \da \llbracket Df_x \rrbracket
\]
die Jakobische (Jakobi-Determinante) von $f$ in $x$.
\end{definition}

Unser Ziel im folgenden ist es, $\{x\in\MdR^m: Jf(x) \ne 0\}$ in Teile zu zerlegen, auf denen $f$ injektiv ist und auf denen $Df_x$ kontrolliert ist.

\begin{lemma}
\label{lem:3.7}
Sei $m\le n$ und $f:\MdR^m\to\MdR^n$ eine stetige Abbildung. Sei $t>1$. Dann existiert eine abzählbare Borel-Überdeckung $\mathcal E$ der Menge
\[
B \da \{x\in\MdR^n: f \text{ ist differenzierbar in $x$, } Jf(x) \ne 0\}
\]
derart, dass für jedes $E\in\mathcal E$ gilt:
\begin{enumerate}
\item $f|_E$ ist injektiv.
\item Es existiert ein Automophismus $\tau_E : \MdR^m \to \MdR^m$ mit 
\begin{enumerate}
\item $\Lip(f|_E \circ \tau_E^{-1}) \le t$ und $\Lip(\tau_E \circ (f|_E)^{-1}) \le t$.
\item $t^{-1} \cdot\|\tau_E(v)\| \le \|Df_b(v)\| \le t \cdot\|\tau_E(v)\|$ für alle $b\in B$ und $v\in \MdR^m$.
\item $t^{-m} \cdot |\det(\tau_E)| \le J(f)|_E \le t^m \cdot |\det(\tau_E)|$.
\end{enumerate}
\end{enumerate}
\end{lemma}

\begin{beweis}
Sei $\ep>0$ mit $t^{-1}+ \ep < 1< t-\ep$. Wähle abzählbare dichte Teilmengen $C\subset \MdR^m$ und $T\subset \GL(m,\MdR)$. Für $c\in C$, $\tau\in T$ und $i\in\MdN$ sei $E(c,\tau,i)$ die Menge aller $b\in B \cap B(c,\frac1i)$, für die gilt:
\begin{enumerate}[\quad(a)]
\item $ (t^{-1} +\ep)\cdot \|\tau(v)\| \le \|Df_b(v)\| \le (t-\ep) \cdot \|\tau(v)\|$ für alle $v\in\MdR^m$ und
\item $\|f(a)- f(b) - Df_b(a-b)\| \le \ep \cdot \|\tau(a-b)\|$ für alle $a\in B(c,\frac1i)$.
\end{enumerate}
Sei $b\in E(c,\tau,i)$ und $a\in B(c,\frac1i)$. Dann gilt 
\begin{align*}
\|f(a) - f(b)\| &\le \|Df_b(a-b)\| + \ep \cdot \|\tau(a-b)\|\\
&\le (t-\ep)\|\tau(a-b)\| +\ep\cdot \|\tau (a-b)\| \\
&= t\cdot \|\tau(a-b)\| = t\cdot \|\tau(a) - \tau(b)\|
\end{align*}
und
\begin{align*}
\|f(a) - f(b)\| &\ge \|Df_b(a-b)\| - \ep \cdot \|\tau(a-b)\| \\
&\ge (t^{-1} + \ep )\|\tau(a-b)\| - \ep \cdot \|\tau(a-b)\| \\
&=  t^{-1} \cdot \|\tau(a)-\tau(b)\|.
\end{align*}
Zusammen erhält man für $a,b\in E(c,\tau,\frac1i)$:
\begin{align*}
t^{-1}\cdot \|\tau(a)-\tau(b)\| \le \|f(a) - f(b)\| \le t\cdot \|\tau(a)-\tau(b)\|.
\end{align*}
Dies zeigt (1). Für $a,b\in E(c,\tau,i)$ gilt also
\begin{align*}
\|f\circ \tau^{-1}(\tau(a)) - f\circ \tau^{-1}(\tau(b))\|\le t \cdot\|\tau(a)-\tau(b)\|,
\end{align*}
das heißt
\begin{align*}
\Lip(\underbrace{f|_{E(c,\tau,i)} \circ \tau^{-1}}_{\mathclap{\text{Abbildung auf }\tau(E(c,\tau,i))}}) \le t
\end{align*}
sowie
\begin{align*}
\|\tau\circ f^{-1}(f(a)) - \tau \circ f^{-1}(f(b))\| \le t \cdot \|f(a)-f(b)\|,
\end{align*}
das heißt
\begin{align*}
\Lip(\underbrace{\tau\circ(f|_{E(c,\tau,i)})^{-1}}_{\mathclap{\text{Abbildung auf }f(E(c,\tau,i))}}) \le t.
\end{align*}

Sei $b\in E(c,\tau,i)$. Sei $(e_1,\ldots,e_m)$ die Standardbasis. Da $Df_b$ injektiv ist, ist in der Zerlegung $Df_b=\rho\circ\sigma$, wobei $\sigma:\MdR^m\to\MdR^m$ symmetrisch und $\varrho:\MdR^m\to\MdR^n$ eine lineare Isometrie ist, die Abbildung $\sigma$ ein Automophismus. Wir schätzen ab
\begin{align*}
Jf(b) &= |\det(\sigma)| = |\det(\sigma\circ\tau^{-1})|\cdot |\det(\tau)|\\
&= |\det(\sigma\circ\tau^{-1}(e_1),\ldots,\sigma\circ\tau^{-1}(e_m))| \cdot |\det(\tau)| \\
&\le \prod_{i=1}^m \|\sigma\circ\tau^{-1}(e_i)\| \cdot |\det(\tau)| \\
&= \prod_{i=1}^m \underbrace{\|Df_b(\tau^{-1}(e_i))\|}_{\le t\cdot \|\tau(\tau^{-1}(e_i))\| = t}\cdot |\det(\tau)| \\
&\le t^m \cdot |\det(\tau)|.
\end{align*}
Analog erhält man:
\begin{align*}
(Jf(b))^{-1} &= |\det(\sigma)|^{-1} = |\det(\sigma^{-1})| = |\det(\tau\circ\sigma^{-1})| \cdot |\det(\tau^{-1})| \\
&\le |\det(\tau \circ \sigma^{-1}(e_1),\ldots,\tau\circ\sigma^{-1}(e_m))| \cdot |\det(\tau)|^{-1} \\
&\le \prod_{i=1}^m \|\tau \circ \sigma^{-1} (e_i)\|\cdot|\det(\tau)|^{-1} \\
&\le \prod_{i=1}^m t \cdot \|Df_b(\sigma^{-1}(e_i))\| \cdot |\det(\tau)|^{-1} \\
&= t^m \cdot |\det(\tau)|^{-1},
\end{align*}
das heißt
\begin{align*}
Jf(b) \ge t^{-1}\cdot |\det(\tau)|.
\end{align*}

Zu zeigen ist noch, dass die Mengen $E(c,\tau,i)$ mit $c\in C, \tau\in T,i\in \MdN$ die Menge $B$ überdecken. Sei hierzu $b\in B$. Dann ist $Df_b = \varrho \circ \sigma$ mit $\varrho, \sigma$ wie oben. Wähle $\delta>0$ so, dass
\begin{align*}
(1-\delta\cdot\|\sigma^{-1}\|)^{-1} < t - \ep \quad\text{und} \quad 1+\delta\cdot\|\sigma^{-1}\| < (t^{-1}+\ep)^{-1}
\end{align*}
und $\tau\in T$ so, dass $\|\sigma-\tau\|< \delta$. Dann gilt:
\begin{align*}
\|\tau\circ\sigma^{-1}\| 
&= \|\id_{\MdR^m}+\tau \circ \sigma^{-1} - \id_{\MdR^m} \|\\
&\le 1 + \| \tau \circ \sigma - \sigma\circ \sigma^{-1}\| \\
&=   1 + \|(\tau -\sigma)\circ \sigma^{-1}\| \\
&\le 1 + \|\tau -\sigma\|\cdot\|\sigma^{-1}\| \\
&<  1+\delta\cdot\|\sigma^{-1}\| \\
&< (t^{-1}+\ep)^{-1}
\end{align*}
und ähnlich folgt
\begin{align*}
\|\sigma\circ\tau^{-1}\| &\le 1 + \|\sigma - \tau\| \cdot \|\tau^{-1}\| \\
&< 1 + \delta\cdot\|\tau^{-1}\| \\
&= 1+\delta\cdot\|\sigma^{-1} \circ \sigma\circ\tau^{-1}\| \\
&\le 1 + \delta\cdot\|\sigma^{-1}\|\cdot \|\sigma\circ\tau^{-1}\| 
\end{align*}
und damit
\begin{align*}
\|\sigma\circ\tau^{-1}\| \le (1-\delta\cdot\|\sigma^{-1}\|)^{-1} < t-\ep.
\end{align*}

Wähle $i\in\MdN$ mit
\begin{align*}
\|f(a)-f(b) - Df_b(a-b)\| \le \ep \cdot \frac{\|a-b\|}{\Lip(\tau^{-1})}
\end{align*}
für alle $a\in B(b,\frac2i)$ und wähle $c\in C$ mit $\|c-b\| < \frac1i$. Für $v\in\MdR^m$ gilt dann
\begin{align*}
(t^{-1} + \ep)\cdot \|\tau(v)\| 
&= (t^{-1} + \ep)\cdot \|\tau\circ\sigma^{-1}(\sigma(v))\| \\
&\le (t^{-1} + \ep)\cdot \|\tau\circ\sigma^{-1}\| \cdot \|\sigma(v)\| \\
&\le \|\sigma(v)\| \\
&= \|Df_b(v)\|
\end{align*}
und
\begin{align*}
\|Df_b(v)\| &= \|\sigma(v)\| = \|\sigma\circ\tau^{-1}(\tau(v))\| \\
&\le \|\sigma\circ\tau^{-1}\| \cdot \|\tau(v)\| \\
&\le (t-\ep)\|\tau(v)\|.
\end{align*}
Dies zeigt Bedingung (a).

Für $a\in B(c,\frac1i) \subset B(b,\frac2i)$ ist 
\begin{align*}
\|f(a) - f(b) - Df_b(a-b)\|
&\le \ep\cdot\frac{\|a-b\|}{\Lip(\tau^{-1})} \\
&= \ep\cdot\frac{\|\tau^{-1}(\tau(a)) - \tau^{-1}(\tau(b))\|}{\Lip(\tau^{-1})} \\
&\le \ep\cdot \|\tau(a-b)\|.
\end{align*}
Dies zeigt Bedingung (b). Somit ist eine Überdeckung gegeben.

Nun kommen wir zur Borel-Messbarkeit der überdeckenden Mengen.

Sei $M$ die Menge aller $x\in \MdR^m$, für die $f$ in $x$ differenzierbar ist. Nach einer 
Übungsaufgabe, ist diese Menge messbar. Sei $\{x_i:i\in \MdN\}$ eine dichte Teilmenge von $\MdR^m$. 
Da $\tau $ und $Df_b(\cdot)$ stetig sind, gilt
\begin{align*}
\{b\in M:\text{ (a) gilt in } b\}=&\bigcap_{j\in\MdN}\{b\in M:(t^{-1}+\ep)\|\tau(x_j)\|\le \|Df_b(x_j)\|\le (t-\ep)\|\tau(x_j)\|\}\\
\in &\mathfrak{B}(\MdR^m)
\end{align*}
und
\begin{align*}
\{b\in M:\text{ (b) gilt in } b\}=&\bigcap_{j\in\MdN\atop x_j\in B(c,1/i)}\{b\in M:
\|f(x_j)-f(b)-Df_b(x_j-b)\|\le\ep\|\tau(x_j-b)\|\}\\
\in &\mathfrak{B}(\MdR^m).
\end{align*}
Zusammen ergibt dies die behauptete Messbarkeitsaussage.
\end{beweis}

\begin{satz}
Sei $m\le n$, $f:\MdR^m\to\MdR^n$ eine Lipschitzabbildung und $A\subset\MdR^m$ eine $\lambda^m$-messbare Menge. Dann gilt
\[
\int_A Jf(x) \lambda^m(dx) = \int_{\MdR^n} \#(A\cap f^{-1}(\{y\}))\,\HM^m(dy).
\]
\end{satz}

\begin{beweis}
Die Abbildung $y\mapsto \#(A\cap f^{-1}(\{y\}))$ ist messbar, falls $A$ eine Borelmenge ist. Zu der
gegebenen $\HM$-messbaren Menge $A$ gibt es eine Borelmenge $A'$ mit $A\subset A'$ und $\HM^m(A'\setminus A)=0$. 
Nach Proposition \ref{prop:3.5} ist $y\mapsto \#((A'\setminus A)\cap f^{-1}(\{y\}))$ 
$\HM^m$ fast überall die Nullfunktion und damit $\HM^m$-messbar. Hieraus folgt die Messbarkeit 
der Abbildung $y\mapsto \#(A\cap f^{-1}(\{y\}))$ allgemein. 


Aufgrund des Satzes von Rademacher, Proposition \ref{prop:3.5}, und der Regularität des Lebesguemaßes kann man annehmen, dass $A\in\borel(\MdR^m)$ und $f$ in jedem Punkt von $A$ differenzierbar ist. Ferner kann $\lambda^m(A) < \infty$ angenommen werden.
\begin{enumerate}[{Fall} (a):]
\item $A\subset \{x\in\MdR^m: Jf(x)\ne 0\}$. Sei $t>1$ beliebig. Wähle eine Borel-Überdeckung $\mathcal E$ von $B\da \{x\in\MdR^m: Jf(x) \ne 0\}$ wie in Lemma \ref{lem:3.7}. Sei $\mathcal G$ eine abzählbare Zerlegung von $A$ derart, das für jedes $G\in\mathcal G$ gilt: $G\in \borel(\MdR^m)$ und $G\subset E$ für ein $E\in\mathcal E$ mit  $\tau_E$ wie in Lemma \ref{lem:3.7}. Dann folgt
\begin{align*}
t^{-2m}\cdot\HM^m(f(G)) 
&= t^{-2m} \cdot \HM^m( ( (f|_E)\circ\tau_E^{-1}) (\tau_E(G)) ) \\
&\le t^{-2m} \cdot \underbrace{\Lip( (f|_E)\circ\tau_E^{-1})^m}_{\le t^m} \cdot \HM^m(\tau_E(G))\\
&\le t^{-m} \cdot \HM^m(\tau_E(G))\\
&= t^{-m} \cdot |\det(\tau_E)| \cdot \HM^m(G) \\
&=\int_G t^{-m} |\det(\tau_E)| d\lambda^m \\
&\le \int_G Jfd\lambda^m.
\end{align*}
Die Abschätzungen lassen sich nun in analoger Weise fortsetzen:
\begin{align*}
\int_G Jfd\lambda^m
&\le \int_G t^m |\det(\tau_E)| d\lambda^m \\
&= t^m \cdot |\det(\tau_E)| \cdot \HM^m(G) \\
&= t^m \cdot \HM^m(\tau_E(G)) \\
&= t^m \cdot \HM^m(\tau_E \circ (f|_E)^{-1} (f|_E(G))) \\
&\le t^m \cdot \Lip(\tau_E \circ (f|_E)^{-1})^m \cdot \HM^m(f(G)) \\
&\le t^{2m} \cdot \HM^m(f(G)).
\end{align*}
Da $f|_G$ für alle $G\in\mathcal G$ injektiv ist, erhält man zunächst
\begin{align*}
\sum_{G\in\mathcal G} \HM^m(f(G))
&= \sum_{G\in\mathcal G} \int_{\MdR^n} \ind_{f(G)} d\HM^m \\
&= \sum_{G\in\mathcal G} \int_{\MdR^n} \#(G\cap f^{-1}(\{y\}))\, \HM^m(dy) \\
&= \int_{\MdR^n} \#(A\cap f^{-1}(\{y\}))\, \HM^m(dy)
\end{align*}
und hiermit
\begin{align*}
t^{-2m}\cdot \int_{\MdR^n} \#(A\cap f^{-1}(\{y\}))\, \HM^m(dy) 
&\le \int_A Jfd\lambda^m \\
&\le t^{2m} \cdot \int_{\MdR^n} \#(A\cap f^{-1}(\{y\})) \, \HM^m(dy).
\end{align*}
Dies gilt für jedes $t>1$ und somit folgt die Gleichheit.
\item $A\subset \{x: Jf(x) = 0\}$. Sei $\ep>0$. Dann werden Lipschitzabbildungen $g,p$ erklärt;
\begin{align*}
p: {}&\MdR^n\times\MdR^m \to\MdR^n, & (y,z) &\mapsto y \\
g: {}&\MdR^m\to\MdR^n\times\MdR^m,  & x&\mapsto (f(x),\ep\cdot x).
\end{align*}
Dann ist $f= p\circ g$. Für $x\in A$ gilt $Dg_x(v) = (Df_x(v), \ep\cdot v)$ für $v\in\MdR^m$. Somit ist $Dg_x$ injektiv und daher $Jg(x)\ne 0$ für $x\in A$. Da $Jf(x)=0$ für $x\in A$ ist $q \da \dim(\Kern(Df_x)) \ge 1$.  Sei $(b_1,\ldots,b_q)$ eine Orthonormalbasis von $\Kern(Df_x)$ und $(b_1,\ldots,b_m)$ eine Ergänzung zu einer Orthonormalbasis von $\MdR^m$. Somit gilt
\begin{align*}
\|Dg_x(b_i)\| = \|(0,\ep \cdot b_i)\| = \ep
\end{align*}
für $i=1,\ldots,q$ und
\begin{align*}
\|Dg_x(b_j)\| = \|(Df_x(b_j),\ep\cdot b_j)\| \le \Lip(f) + \ep
\end{align*}
für $j=q+1,\ldots,m$. Nach Fall $(a)$ ist
\begin{align*}
\HM^m(f(A)) &= \HM^m(p\circ g(A))\\
&\le \HM^m(g(A)) = \int_{A} Jgd\lambda^m\\
&\le \ep^q\cdot(\Lip(f)+\ep)^{m-q}\cdot\lambda^m(A).
\end{align*}
Dies zeigt $\HM^m(f(A))=0$ und somit $\#(A\cap f^{-1}(\{y\})) = 0$ für $\HM^m$-fast alle $y\in\MdR^n$. Dies zeigt die Gleichheit auch in diesem Fall.
\end{enumerate}
\end{beweis}

\begin{korollar}
Seien $m\le n$, $f:\MdR^m\to\MdR^n$ eine Lipschitzabbildung und $h:\MdR^m\to\MdR$ eine Funktion, 
deren $\lambda^m$-Integral existiert. Dann gilt
\begin{align*}
\int_{\MdR^m}h(x)Jf(x)\, \lambda^m(dx)&=\int_{\MdR^n}\left(\sum_{x\in f^{-1}(\{y\})}h(x)\right)
\, \HM^m(dy)\\
&=\int_{\MdR^n}\int_{f^{-1}(\{y\})}h(x)\,\HM^0(dx)\, \HM^m(dy).
\end{align*}
\end{korollar}

\begin{beweis}
Übung
\end{beweis}


\emph{Anwendung:}
Sei \(m \leq n\), \(A \subset \MdR^m\), \(f: \MdR^m \to \MdR^n\) sei \(C^1\) und injektiv. Dann gilt:
\[
\HM^m(f(A)) = \int_A Jf(x) \,\HM^m(dx) = \int_A \sqrt{g(x)}\, \HM^m(dx)
\]
mit \(g(x) = \det( (\langle D_i f(x), D_j f(y) \rangle)_{i,j=1,\dots,m} )\).
Ist \(f\) nicht injektiv, so nennt man
\[
\int_{\MdR^n} \#(A \cap f^{-1}(\{y\}))\, \HM^m(dy) \quad \geq \HM^m(f(A))
\]
die Hausdorff-Fläche von  $f|_A$.


\section{Die Koflächenformel}

Sei jetzt \(m \geq n\) und \(f: \MdR^m \to \MdR^n\). Sei ferner \(g: \MdR^m \to [0,\infty]\) eine $\HM^m$-messbare 
Abbildung. Dann besagt die Koflächenformel, dass
\[
\int_{\MdR^m} g(x) Jf(x) \,\HM^m(dx) = \int_{\MdR^n} \int_{f^{-1}(\{y\})} g(x)\, \HM^{m-n}(dx) \,\HM^n(dy)
\]
gilt. Die Jakobische $Jf$ von $f$ wird nachfolgend erklärt. Im Spezialfall \(g(x) = \ind_A(x)\) mit einer $\HM^m$-messbaren Menge \(A \subset \MdR^m\) besagt dies gerade
\[
\int_A Jf(x) \,\HM^m(dx) = \int_{\MdR^n} \HM^{m-n}(A \cap f^{-1}(\{y\})) \,\HM^n(dy).
\]
Aus diesem Spezialfall erhält man umgekehrt die allgemeine Aussage durch die üblichen Routineargumente.
\begin{beispiel}
Für die Abbildung 
\(d: \MdR^m \to [0,\infty),\, x \mapsto \|x\|\) gilt \( d^{-1}(\{r\}) = \{x\in\MdR^m \colon \|x\|=r\}\). 
Die Berechnung der Jakobischen $Jd$ von $d$ erfolgt in den Übungen.
\end{beispiel}

\begin{lemma}\label{lem:3.10}
Sei \(k\geq m\geq n\), \(A\subset\MdR^k\) und \(f: \MdR^k \to \MdR^n\) eine Lipschitz-Abbildung. Dann gilt
\[
\int_{\MdR^n}^* \HM^{m-n}(A \cap f^{-1}(\{y\}))\, \HM^n(dy) \; \leq \; \frac{\alpha(m-n) \alpha(n)}{\alpha(m)} \Lip(f)^m.
\]
\end{lemma}
\begin{beweis}
Für \(j \in \MdN\) gilt \(\HM^m_{\frac1j}(A) \leq \HM^m(A) \leq \HM^m(A) + \frac1j\). Dann existiert eine Folge \(\mathcal B_j \in \bar\Omega_{\frac1j}(A)\) mit
\[
\HM^m_{\frac1j}(A) \leq \sum_{B\in\mathcal B_j} \frac{\alpha(m)}{2^m} \diam (B)^m \leq \HM^m(A)+\frac1j.
\]
Für \(B \in \mathcal B_j\) ist \(f(B)\) kompakt, also Borelmenge. Erkläre 
$$g_B: \MdR^n \to \MdR,\qquad  y \mapsto \frac{\alpha(m-n)}{2^{m-n}}\cdot\diam(B)^{m-n}\cdot\ind_{f(B)}(y).
$$ 
Insbesondere ist \(g_B\) eine \(\HM^n\)-messbare Funktion und 
$$
\diam(B\cap f^{-1}(\{y\})) \leq \diam(B) \cdot \ind_{f(B)}(y) \leq \frac1j. 
$$
Es folgt
\[
\HM_{\frac1j}^{m-n}(A\cap f^{-1}(\{y\})) \leq \sum_{B \in \mathcal B_j} \frac{\alpha(m-n)}{2^{m-n}} \diam(B\cap f^{-1}(\{y\}))^{m-n} \leq \sum_{B \in \mathcal B_j} g_B(y).
\]
Mit dem Lemma von Fatou und dem Satz von der monotonen Konvergenz  erhält man nun
\begin{align*}
\int_{\MdR^n}^* \HM^{m-n}(A \cap f^{-1}(\{y\})) \,\HM^n(dy) &= \int_{\MdR^n}^* \lim_{j\to\infty} \HM_{\frac1j}^{m-n}(A\cap f^{-1}(\{y\})) \,\HM^n(dy) \\
&\leq \int_{\MdR^n} \liminf_{j\to\infty} \sum_{B\in\mathcal B_j} g_B(y)\, \HM^n(dy) \\
&\leq \liminf_{j\to\infty} \int_{\MdR^n} \sum_{B\in\mathcal B_j} g_B(y) \,\HM^n(dy) \\
&= \liminf_{j\to\infty} \sum_{B\in\mathcal B_j} \int_{\MdR^n} g_B(y) \,\HM^n(dy) \\
&= \liminf_{j\to\infty} \sum_{B\in\mathcal B_j} \frac{\alpha(m-n)}{2^{m-n}} \diam(B)^{m-n} \cdot \HM^n(\underbrace{f(B)}_{\subset \MdR^n}).
\end{align*}
Mit Hilfe der isodiametrichen Ungleichung ergibt dies
\begin{align*}
\int_{\MdR^n}^* \HM^{m-n}(A \cap f^{-1}(\{y\})) \,\HM^n(dy) 
&\leq \liminf_{j\to\infty} \sum_{B\in\mathcal B_j} \frac{\alpha(m-n)}{2^{m-n}}\diam(B)^{m-n} \cdot \frac{\alpha(n)}{2^n}\diam(f(B))^n \\
&\leq \liminf_{j\to\infty} \frac{\alpha(m-n)\alpha(n)}{\alpha(m)} \Lip(f)^n \cdot \underbrace{\sum_{B\in\mathcal B_j} \frac{\alpha(m)}{2^m}\diam(B)^m}_{\le \HM^m(A) + \frac1j} \\
&\le \frac{\alpha(m-n)\alpha(n)}{\alpha(m)}\Lip(f)^n\HM^m(A).
\end{align*}
\end{beweis}

\begin{lemma}\label{lem:3.11}
Sei \(m\geq n\) und \(f:\MdR^m\to\MdR^n\) eine Lipschitz-Abbildung. Sei ferner \(A \subset \MdR^m\) eine \(\HM^m\)-messbare Menge. Dann gilt:
\begin{enumerate}[(1)]
\item \(A\cap f^{-1}(\{y\})\) ist \(\HM^{m-n}\)-messbar für \(\HM^n\)-fast-alle \(y \in \MdR^n\).
\item \(y\mapsto\HM^{m-n}(A\cap f^{-1}(\{y\}))\) ist \(\HM^n\)-messbar.
\end{enumerate}
\end{lemma}
\begin{beweis}
Sei \(A\) kompakt. Dann ist \(A\cap f^{-1}(\{y\})\) kompakt, also Borelmenge. Sei \(t>0\) beliebig. Für \(j \in \MdN\) sei \(U_j\) die Menge aller \(y\in\MdR^n\), für die es eine endliche, offene Überdeckung \(\mathcal G\) von \(A \cap f^{-1}(\{y\})\) gibt mit \(\diam(G) \leq \frac1j \, (G \in \mathcal G)\) und 
$$
\sum_{G\in \mathcal G} \frac{\alpha(m-n)}{2^{m-n}} \diam(G)^{m-n} \leq t + \frac1j.
$$ 
Dann bestätigt man leicht
\[
\{y\in\MdR^n \colon \HM^{m-n}(A\cap f^{-1}(\{y\})) \leq t\} = \bigcap_{j\in\MdN} U_j.
\]
Wir zeigen, dass \(U_j\) eine offene Menge und damit eine Borelmenge ist. Sei hierzu \(y\in U_j\) und  \(\mathcal G\) eine offene, endliche Überdeckung von \(A \cap f^{-1}(\{y\})\). Da \(A\) kompakt ist, ist \(A \setminus \bigcup_{G \in \mathcal G} G\) kompakt und somit auch \(f(A\setminus \bigcup_{G\in\mathcal G} G)\). Daher ist \(f(A \setminus \bigcup_{G\in\mathcal G} \subset \MdR^n)^c\) offen.
\par
\emph{Behauptung}: Für \(z \in \MdR^n\) gilt: 
$$
z \in f(A \setminus \bigcup_{G\in\mathcal G})^c \Leftrightarrow A \cap f^{-1}(\{z\}) \subseteq \bigcup_{G\in\mathcal G} G.
$$ 
Dies ist leicht einzusehen. 


Eine zweimalige Anwendung der Behauptung zeigt 
\(y\in f(A \cap \bigcup_{G\in\mathcal G} G)^c \subseteq U_j\). Also ist \(U_j\) offen.
\par
Sei \(A \subset \MdR^m\) nun eine beliebige \(\HM^m\)-messbare Menge. Es genügt, \(\HM^m(A)<\infty\) zu betrachten. Zu $A$ gibt es eine aufsteigende Folge kompakter Mengen \( A_i  \subset A\) mit \(\HM^m(A \setminus \bigcup_{i\geq1}A_i)=0\). Lemma \ref{lem:3.10} zeigt 
\[
\HM^{m-n}((A\setminus\bigcup_{i\geq1}A_i)\cap f^{-1}(\{y\})) = 0
\]
für \(\HM^n\)-fast-alle \(y\in\MdR^n\). Hieraus folgt (1) und wegen 
$$
\HM^{m-n}(A\cap f^{-1}(\{y\})) = \lim_{i\to\infty} \HM^{m-n}(A_i\cap f^{-1}(\{y\}))
$$ 
auch (2).
\end{beweis}

\emph{Notation:} Zu \(f \in \L(\MdR^m,\MdR^n)\) ist \(f^* \in \L(\MdR^n,\MdR^m)\) die adjungierte Abbildung, die 
durch die Bedingung \(\langle f(x),y\rangle = \langle x,f^*(y)\rangle\) für \(x\in\MdR^m, y\in\MdR^n\) 
festgelegt ist.

\begin{lemma}\label{lem:3.12}
Sei \(f:\MdR^m\to\MdR^n\) eine lineare Abbildung und \(m\geq n\). Dann gilt:
\begin{enumerate}[(1)]
\item \(f^{**} = f\).
\item \(f\in\L(\MdR^m,\MdR^n),\, g\in\L(\MdR^n,\MdR^p) \; \Rightarrow \; (g\circ f)^* = f^* \circ g^* \in \L(\MdR^p,\MdR^m)\).
\item Ist \(\varrho \in \L(\MdR^n,\MdR^m)\) orthogonal, so gilt \(\varrho^* \circ \varrho = \id_{\MdR^n}\) und \(\Bild(\varrho) = \Kern(\varrho^*)^\perp\).
\item Zu \(f\) exisitiert \(\sigma \in \L(\MdR^n,\MdR^n)\) symmetrisch und \(\varrho\in \L(\MdR^n,\MdR^m)\) orthogonal mit \(f = \sigma \circ \varrho^*\).
\item Ist \(h \in \L(\MdR^m,\MdR^m)\), so gilt \(\llbracket h \rrbracket = \llbracket h^* \rrbracket = |\det(h)| = |\det(h^*)|\).
\item Setze \(\llbracket f \rrbracket \da \llbracket f^* \rrbracket\). Dann gilt:
	\begin{enumerate}[(a)]
	\item \(\llbracket f \rrbracket = 0\), falls \(\dim(f(\MdR^n)) < n\).
	\item Ist \(\dim(f(\MdR^n)) = n\) und \((b_1,\dots,b_n)\) eine Orthonormalbasis von \(\Kern(f)^\perp\), so gilt:
\[
\llbracket f \rrbracket = |\det(f(b_1),\dots,f(b_n))|.
\]
	\end{enumerate}
\item Ist \((a_1,\dots,a_n)\) eine ONB von \(\MdR^n\), so gilt:
\[
\llbracket f \rrbracket \da \sqrt{\det(f\circ f^*)} = \sqrt{
\det(\langle f^*(a_i),f^*(a_j)\rangle_{i,j=1,\dots,n}}).
\]
\end{enumerate}
\end{lemma}
\begin{beweis}
Übung
\end{beweis}

\begin{lemma}\label{lem:3.13}
Sei \(m\leq n\) und seien \(h: \MdR^m\to\MdR^n,\, \tilde h: \MdR^n\to\MdR^m\) Lipschitz-Abbildungen. Sei ferner \(E \subset \{x\in\MdR^m\colon \tilde h \circ h (x) = x \}\) eine \(\HM^m\)-messbare Menge. Dann gibt es eine \(\HM^m\)-messbare Menge \(S_E \subset E\) mit \(\HM^m(E\setminus S_e) = 0\), so dass für \(x \in S_e\) gilt:
\begin{itemize}
\item \(h\) ist in \(x\) differenzierbar.
\item \(\tilde h\) ist in \(h(x)\) differenzierbar.
\item \(D\tilde h_{h(x)} \circ D  h_x = \id_{\MdR^m}\).
\end{itemize}
\end{lemma}
\begin{beweis}
Sei \(Z \da \{ x\in \MdR^m \colon \tilde h\circ h (x) - x = 0 \} = \{ x\in\MdR^m \colon (\tilde h\circ h-\id)(x) = 0 \}\). Dann gilt (nach Übung) für \(\HM^m\)-fast-alle \(x \in Z\), dass \(D(\tilde h \circ h - \id)_x = 0\), d.h. \(D(\tilde h \circ h)_x = \id_{\MdR^m}\). Ist \(E \subset Z\), so gilt dies auch für \(\HM^m\)-fast-alle \(x\in E\). 

Sei \(F \da \{x\in\MdR^m \colon h \text{ ist differenzierbar in } x\}\), \(G \da \{y\in\MdR^n \colon \tilde h \text{ ist differenzierbar in } y\}\) und \(D \da F \cap \{x\in\MdR^m \colon h(x)\in G\}\). Dann gilt:
\[
E\setminus D = (E\setminus F) \cup (E \setminus \{x\in\MdR^m \colon h(x) \in G\}) = (E\setminus F) \cup \{x\in E \colon h(x) \notin G\},
\]
wobei aufgrund des Satzes von Rademacher \(\HM^m(E\setminus F) = 0\) und wegen \(\{x\in E \colon h(x) \notin G\} \subset \tilde h(\MdR^n\setminus G)\) auch 
$$
\HM^m(\{x\in E \colon h(x) \notin G\}) \leq \HM^m(\tilde h(\MdR^n\setminus G)) \leq \Lip(\tilde h)^m \cdot \HM^m(\MdR^n\setminus G) = 0.
$$
Hieraus folgen leicht alle Behauptungen.
\end{beweis}

\begin{lemma}\label{lem:3.14}
Sei \(m\geq n\) und sei \(f: \MdR^m\to\MdR^n\) stetig. Dann existiert eine abzählbare Borelüberdeckung \(\mathcal E\) von \(B \da \{x\in\MdR^m \colon f\text{ differenzierbar in } x \text{ und } Df_x(\MdR^m)=\MdR^n\}\) derart, dass für jedes \(E \in \mathcal E\) eine (lineare) Orthogonalprojektion \(p_E: \MdR^m \to \MdR^{m-n}\) und Lipschitzabbildungen \(h_E: \MdR^m \to \MdR^n \times \MdR^{m-n},\, \tilde h_E:\MdR^n\times \MdR^{m-n} \to \MdR^m\) existieren mit
\[
h_E(x) = (f(x),p_E(x)) \quad \text{und} \quad \tilde h_E \circ h_E(x) = x \quad \text{ für alle } x\in E.
\]
\end{lemma}

\begin{beweis}
Für $\gamma \in \Lambda(m,m-n)$ sei $p_\gamma:\MdR^{m}\to \MdR^{m-n}$ gegeben durch
\[
p_\gamma(x_1,\ldots,x_m) \da (x_{\gamma(1)},\ldots,x_{\gamma(m-n)}).
\]
Ferner sei $h_\gamma: \MdR^m \to \MdR^n \times \MdR^{m-n}$ gegeben durch $h_\gamma(x) \da (f(x),p_\gamma(x))$. Setze
\[A_\gamma \da \{x\in\MdR^m : \text{$h_\gamma$ ist differenzierbar in $x$ und $Dh_\gamma(x)$ ist injektiv}\}.\]
Die Abbildung $f$ ist differenzierbar in $x$ genau dann, wenn $h_\gamma$ in $x$ differenzierbar ist und $\Kern( (Dh_\gamma)_x) = \Kern(Df_x) \cap \Kern(p_\gamma)$. Hieraus folgt leicht
\[
B  = \bigcup_{\mathclap{\gamma\in\Lambda(m,m-n)}} A_\gamma.
\]

Nach Lemma \ref{lem:3.7} gibt es zu jedem $t>1$ eine abzählbare Borel-Überdeckung $\mathcal E_\gamma$ von $A_\gamma$ derart, dass für jedes $E\in\mathcal E_\gamma$ ein Automorphismus $\tau_E : \MdR^m \to \MdR^m$ existiert, so dass gilt:  $(h_\gamma)|_E$ ist injektiv und $\Lip( (h_\gamma)|_E \circ \tau_E^{-1}) \le t$ sowie $\Lip( \tau_E \circ ( (h_\gamma)|_E )^{-1}) \le t$.

Für $x,y\in E$ gilt dann
\begin{align*}
\|h_\gamma(x) - h_\gamma(y) \| 
&= \| (h_\gamma)|_E \circ \tau_E^{-1}( \tau_E^{}(x)) - (h_\gamma)|_E \circ \tau_E^{-1}( \tau_E^{}(y) )\| \\
&\le t \cdot \| \tau_E^{}(x) - \tau_E^{}(y)\| \le t \cdot \Lip(\tau_E^{}) \cdot \|x-y\|
\end{align*}
und für $u,v\in h_\gamma(E)$ gilt
\begin{align*}
\|(h_\gamma|_E)^{-1}(u) - (h_\gamma|_E)^{-1}(v)\| 
&= \| \tau_E^{-1} \circ \tau_E^{} \circ (h_\gamma|_E)^{-1}(u) - \tau_E^{-1} \circ \tau_E^{} \circ (h_\gamma|_E)^{-1}(v)  \| \\
&\le \Lip(\tau_E^{-1}) \cdot t \cdot \|u-v\|.
\end{align*}
Also sind $h_\gamma|_E$ und $(h_\gamma|_E)^{-1}$ Lipschitzabbildungen. Somit existieren Lipschitzfortsetzungen $h$ von $(h_\gamma|_E)$ und $\tilde h$ von $(h_\gamma|_E)^{-1}$ mit $h:\MdR^m \to \MdR^n\times\MdR^{m-n}$ und $\tilde h: \MdR^n \times \MdR^{m-n}\to \MdR^m$, wobei $h|_E = (h_\gamma)|_E$ und $\tilde h|_{h(E)} = (h_\gamma|_E)^{-1}|_{h_\gamma(E)}$. Daher ist $h(x)=(f(x),p_\gamma(x))$ für $x\in E$ und $\tilde h \circ h(x)=x$ für $x\in E$. Hieraus folgt die Behauptung.
\end{beweis}

Der folgende Hilfssatz stellt gewissermaßen eine lokale Version der Koflächenformel dar. 
Die hierbei auftretenden Annahmen sind nach dem vorangehenden Hilfssatz stets erfüllbar. 
Zusammen erhalten wir dann schließlich die allgemeine Aussage.

\begin{lemma}
\label{lem:3.15}
Sei $m>n$, $f:\MdR^m\to\MdR^n$ eine Lipschitzabbildung und $E\subset \MdR^m$ eine Borelmenge derart, dass für alle $x\in E$ die Abbildung $f$ differenzierbar in $x$ ist und $Df_x$ surjektiv. Sei $p:\MdR^m\to\MdR^{m-n}$ eine Orthogonalprojektion und $h:\MdR^m \to \MdR^n \times \MdR^{m-n}$, $\tilde h: \MdR^n\times\MdR^{m-n}\to\MdR^m$ seien Lipschitzabbildungen so dass $\tilde h \circ h(x) = x$ für $x\in E$ und $h(x) = (f(x),p(x))$ für $x\in E$. Dann gilt für alle $A\subset E$ mit $A\in \A_{\HM^m}$
\[
\int_A Jf(x) \,\HM^m(dx) = \int_{\MdR^n} \HM^{m-n}(A \cap f^{-1}(\{y\}))\, \HM^n(dy).
\]
\end{lemma}

\begin{beweis}
Nach Voraussetzung sind $h|_E$ und $\tilde h|_{h(E)}$ injektiv. Für $y\in\MdR^n$ gilt
\begin{align*}
h(E \cap f^{-1}(\{y\}))
&= \{(f(x),p(x)) : x\in E\cap f^{-1}(\{y\})\} \\
&= \{(y,p(x)) : x \in E\cap f^{-1}(\{y\})\} \\
&= \{y\}\times p(E\cap f^{-1}(\{y\})).
\end{align*}

Für $y\in\MdR^n$ setzte $\tilde h_y: \MdR^{m-n}\to\MdR^m$, $\tilde h_y (z) \da \tilde h (y,z)$ für $z\in\MdR^{m-n}$. Da $\tilde h$ auf der Menge $\{y\}\times p(E \cap f^{-1}(\{y\})) \subset h(E)$ injektiv ist, ist $\tilde h_y$ injektiv auf $p(E\cap f^{-1}(\{y\}))$. Ferner gilt
\begin{align*}
\tilde h_y(p(E\cap f^{-1}(\{y\})))
&=\tilde h(\{y\}\times p(E\cap f^{-1}(\{y\}))\\
&= \tilde h (h (E\cap f^{-1}(\{y\})))\\
&= E\cap f^{-1}(\{y\}) \tag{$*$}.
\end{align*}

Nach Lemma $\ref{lem:3.14}$ ist für $\HM^m$-fast-alle $x\in E$ die Abbildung $h$ in $x$ und die Abbildung $\tilde h$ in $h(x)$ differenzierbar und $D\tilde h_{h(x)} = (Dh_x)^{-1}$. Für $\HM^m$-fast-alle $x\in E$ existiert somit die Abbildung $L_x:\MdR^{m-n}\to\MdR^m$ mit $L_x\da D(\tilde h_{f(x)})_{p(x)}$, wobei die Existenz und die erste  Gleichung
$$
L_x= D(\tilde h_{f(x)})_{p(x)} 
= D\tilde h_{(f(x),p(x))} \circ (0_{n,n-m}, \id_{\MdR^{m-n}})=(D h_x)^{-1}\circ (0_{n,n-m}, \id_{\MdR^{m-n}})
$$
aus der Kettenregel  (betrachte hierzu: $z\mapsto (y,z)\mapsto \tilde h(y,z)$) folgt. Wegen 
$Dh_x = (Df_x,p)$ erhält man  
\begin{align*}
L_x(\MdR^{m-n}) = (Dh_x)^{-1}(\{0_{\MdR^n}\} \times \MdR^{m-n}) = \Kern(Df_x).
\end{align*}
Dies zeigt insbesondere, dass $L_x$ maximalen Rang hat. 
Sei nun $(b_1,\ldots,b_m)$ eine Orthonormalbasis von $\MdR^m$, so dass $(b_1,\ldots,b_{m-n})$ eine Orthonormalbasis von $\Kern(Df_x)$ ist, das heißt $(b_{m-n+1},\ldots,b_m)$ ist eine Orthonormalbasis von $\Kern(Df_x)^\bot$. Damit folgt
\begin{align*}
Jh(x) 
&= |\det(Dh_x(b_1),\ldots,Dh_x(b_m))|\\
&= |\det\big( (0,p(b_1)),\ldots,(0,p(b_{m-n})), (Df_x(b_{m-n+1}),p(b_{m-n+1})),\ldots,(Df_x(b_{m}),p(b_{m}))\big)|\\
&= |\det\big( (0,p(b_1)),\ldots,(0,p(b_{m-n})), (Df_x(b_{m-n+1}),0),\ldots,(Df_x(b_{m}),0)\big)|\\
&= |\det\big( p(b_1),\ldots,p(b_{m-n})\big)| \cdot Jf(x) ,
\end{align*}
wobei wir verwendet haben, dass $(0,p(b_1)),\ldots,(0,p(b_{m-n}))$ linear unabhängig sind. 
Für $v\in\MdR^{m-n}$ gilt
\begin{align*}
(0_{n,m-n}, \id_{\MdR^{m-n}})(v) = Dh_x \circ L_x(v) = (Df_x(L_x(v)), p(L_x(v))) 
\end{align*}
und daher
\begin{align*}
p\circ L_x = \id_{\MdR^{m-n}},
\end{align*}
das heißt $\det(p\circ L_x)=1$. Sei nun $\sigma:\MdR^{m-n} \to \MdR^{m-n}$ eine symmetrische und $\varrho:\MdR^{m-n}\to\MdR^m$ eine orthogonale Abbildung, so dass $L_x=\varrho \circ \sigma$. Da $L_x$ 
maximalen Rang hat, ist $\sigma$ ein Automorphismus. Daher ist $\Kern(Df_x) = L_x(\MdR^{m-n}) = \varrho\circ\sigma(\MdR^{m-n}) = \varrho(\MdR^{m-n})$. Da $\varrho$ orthogonal ist, gibt es eine Orthonormalbasis $\bar b_1,\ldots,\bar b_{m-n}$ in $ \MdR^{m-n}$ mit $\varrho(\bar b_i) = b_i$. Somit folgt
\begin{align*}
1 &= \det(p\circ L_x) = \det(p\circ\varrho\circ\sigma) = \det(p\circ \varrho) \cdot \det(\sigma) \\
&= |\det( p\circ\varrho(\bar b_1), \ldots, p\circ\varrho(\bar b_{m-n}))| \cdot \llbracket L_x\rrbracket \\
&= |\det( p(b_1), \ldots, p(b_{m-n}))| \cdot J(\tilde h_{f(x)})(p(x)).
\end{align*}
Dies ergibt schließlich
\begin{align*}
Jf(x) = Jh(x) \cdot J(\tilde h_{f(x)}) (p(x)).
\end{align*}

Da $h$ auf $E$ und $\tilde h$ auf $h(E)$ injektiv sind, folgt für $A\subset E$ mit $A\in\A_{\HM^m}$ durch zweimalige Anwendung der Flächenformel
\begin{align*}
\int_A Jf(x) \, \HM^m(dx) 
&= \int_{\MdR^m} \ind_A(x) \cdot J(\tilde h_{f(x)})(p(x)) \cdot J h(x)\, \HM^m(dx) \\
&= \int_{\MdR^n\times\MdR^{m-n}} \ind_{h(A)} (y,z) \cdot J(\tilde h_y)(z) \, (\HM^n\otimes \HM^{m-n})(d(y,z)) \\
&= \int_{\MdR^n} \int_{p(A\cap f^{-1}(\{y\}))} J(\tilde h_y)(z) \, \HM^{m-n}(dz)\, \HM^n(dy) \\
&= \int_{\MdR^n} \HM^{m-n}(\underbrace{\tilde h_y (p(A\cap f^{-1}(\{y\})))}_{\gleichwegen{(*)} A\cap f^{-1}(\{y\})}) \, \HM^n(dy) \\
&= \int_{\MdR^n} \HM^{m-n}(A\cap f^{-1}(\{y\}))\, \HM^n(dy).
\end{align*}
\end{beweis}

\begin{satz}
\label{satz:3.16}
Seien $m\ge n$ und $f:\MdR^m \to \MdR^n$ eine Lipschitzabbildung. Für jede $\HM^m$-messbare Menge $A\subset\MdR^m$ gilt 
\[
\int_A Jf(x) \,\HM^m(dx) = \int_{\MdR^n} \HM^{m-n} ( A\cap f^{-1}(\{y\})) \, \HM^n(dy).
\]
\end{satz}

\begin{korollar}
\label{kor:3.17}
Seien $m\ge n$, $f:\MdR^m \to \MdR^n$ eine Lipschitzabbildung und $h:\MdR^m\to [0,\infty]$ eine $\HM^m$-messbare Abbildung. Dann gilt
\[
\int h(x) Jf(x) \, \HM^m(dx) =
\int_{\MdR^n} \int_{f^{-1}(\{y\})} h(x) \, \HM^{m-n}(dx) \, \HM^n(dy).
\]
\end{korollar}

\begin{beweis}[von Satz \ref{satz:3.16}]
Ist $\HM^m(A) = 0$, so sind beide Seiten der behaupteten Gleichung Null (Lemma \ref{lem:3.10}). Daher können wir den Beweis wie früher auf den Fall $\HM^m(A)<\infty$, $f$ differenzierbar in allen $x\in A$ sowie $A\in \borel(\MdR^m)$ reduzieren.
\begin{enumerate}[\quad(a)]
\item Sei $Jf(x) > 0$ für alle $x\in A$. Also ist $Df_x(\MdR^m) = \MdR^n$. Dann existiert eine abzählbare Borel-Überdeckung $\mathcal E$ von $A$ derart, dass für jedes $E\in\mathcal E$ eine Orthogonalprojektion $p:\MdR^m\to\MdR^{m-n}$ und Lipschitzabbildungen $h:\MdR^m\to\MdR^n\times\MdR^{m-n}$, $\tilde h:\MdR^n\times\MdR^{m-n}\to \MdR^m$ existieren mit $h(x) = (f(x), p(x))$ und $\tilde h \circ h (x) = x$ für $x\in E$. Wähle eine abzählbare Zerlegung $\mathcal G$ von $A$, so dass für jedes $G\in\mathcal G$ ein $E\in\mathcal E$ exisitiert mit $G\subset E$. Aus Lemma \ref{lem:3.15} folgt nun
\begin{align*}
\int_A Jf(x) \,\HM^m(dx) 
&= \sum_{G\in\mathcal G} \int_G Jf(x) \, \HM^m(dx) \\
&= \sum_{G\in\mathcal G} \int_{\MdR^n} \HM^{m-n}(G \cap f^{-1}(\{y\})) \, \HM^n(dy) \\
&= \int_{\MdR^n} \HM^{m-n}( A \cap f^{-1}(\{y\}))\, \HM^n(dy).
\end{align*}
\item Sei  $Jf(x) = 0$ für alle $x\in A$, das heißt $\text{Rang}(Df_x(\MdR^m))<n$. Sei $\ep>0$ beliebig 
vorgegeben. Definiere Abbildungen $g:\MdR^m\times \MdR^n\to\MdR^n$ und $p:\MdR^m\times \MdR^n\to\MdR^n$ durch
$$
g(x,y)\da f(x)+\ep y\qquad\text{und}\qquad p(x,y)\da y\qquad \text{ für } x\in\MdR^m, y\in \MdR^n.
$$
Für $x\in A$ und $y\in \MdR^n$ erhält man
$$
Dg_{(x,y)}(v,w)=Df_x(v)+\ep w,\qquad v\in\MdR^m, w\in \MdR^n
$$
und folglich
$$
\| Dg_{(x,y)}(v,w)\|\le \|Df_x(v)\|+\ep \|w\|\le \text{Lip}(f)\|v\|+\ep\|w\|\le (\text{Lip}(f)+\ep)\|(v,w)\|,
$$
also 
$$
\| Dg_{(x,y)}(v,w)\|\le\text{Lip}(f)+\ep.
$$
Ferner ist 
$$
\text{Kern}(Dg_{(x,y)})=\left\{ \left(v,-\frac{1}{\ep}Df_x(v)\right):v\in \MdR^m\right\},
$$
also $\text{dim}(\text{Kern}(Dg_{(x,y)}))=m$ und $\text{dim}(\text{Bild}(Dg_{(x,y)}))=n$. 
Unter (b) gibt es einen Vektor $w\in Df_x(\MdR^m)^\perp$ mit $\|w\|=1$, und wir können hiermit 
$b_1:=\left(\frac{1}{\ep}Df_x^*(w),w\right)\in\MdR^m\times\MdR^n$ definieren. Dann gilt
$$
\|b_1\|^2=\left\langle \frac{1}{\ep}Df_x^*(w),\frac{1}{\ep}Df_x^*(w)\right\rangle+\langle w,w\rangle=
\frac{1}{\ep^2}\langle w,Df_x\circ Df_x^*(w)\rangle+1=1,
$$
wobei
$$ 
\|Df_x\circ Df_x^*(w)\|^2=\langle \underbrace{w}_{\in Df_x(\MdR^m)^\perp},\underbrace{Df_x\circ Df_x^*\circ Df_x\circ Df_x^*(w)}_{\in Df_x(\MdR^m)}\rangle=0,
$$
also $Df_x\circ Df_x^*(w)=0$ verwendet wurde. Wir erhalten $b_1\in \text{Kern}(Dg_{(x,y)})^\perp$, da für alle 
$v\in\MdR^m$ gilt
$$
\left\langle b_1,(v,-\frac{1}{\ep}Df_x(v))\right\rangle=\frac{1}{\ep}\langle Df_x^*(w),v\rangle -\frac{1}{\ep}\langle
w,Df_x(v)\rangle =0,
$$
und ferner
$$
\|Dg_{(x,y)}(b_1)\|=\left\|\frac{1}{\ep} Df_x\circ Df_x^*(w)+\ep w\right\|=\ep.
$$
Wir ergänzen nun $b_1$ zu einer Orthonormalbasis $(b_1,\ldots,b_n)$ von $\text{Kern}(Dg_{(x,y)})^\perp$. 
Hierfür gilt die Abschätzung
$$
Jg(x,y)=|\det(Dg_{(x,y)}(b_1),\ldots, Dg_{(x,y)}(b_n))|=
\prod_{i=1}^n\|Dg_{(x,y)}(b_i)\|\le \ep(\text{Lip}+\ep)^{n-1}.
$$
Die Translationsinvarianz des Hausdorff-Maßes zeigt, dass
$$
\int_{\MdR^n}\HM^{m-n}(A\cap f^{-1}(\{y-\ep z\}))\, \HM^n(dy)
$$
von $\ep$ und $z$ unabhängig ist. Setze $B:=A\times B^n(0,1)$, wobei $B^n(0,1)$ die Einheitskugel im $\MdR^n$ ist. 
Hierbei gilt wegen $\{(x,z):x\in A,f(x)+\ep z=y\}=\{(x,z):x\in A\cap f^{-1}(\{y-\ep z\})\}$ 
$$
B\cap g^{-1}(\{y\})\cap p^{-1}(\{z\})
=\begin{cases}
\{(x,z):x\in A\cap f^{-1}(\{y-\ep z\})\},& \text{falls } z\in B^n(0,1),\\
\emptyset,&\text{sonst}.
\end{cases}
$$
Es bezeichne $\alpha(n)$ das Volumen der $n$-dimensionalen Einheitskugel. 
Mit Hilfe von Lemma \ref{lem:3.10} (mit $k=m+n$) und Anwendung des Resultats aus Teil (a) folgt
\begin{align*}
&\int_{\MdR^n}\HM^{m-n}(A\cap f^{-1}(\{y\}))\,\HM^n(dy)\\
&=\alpha(n)^{-1}\int_{B^n(0,1)}\int_{\MdR^n}\HM^{m-n}(A\cap f^{-1}(\{y-\ep z\}))\,\HM^n(dy)\, \HM^n(dz)\\
&=\alpha(n)^{-1}\int_{\MdR^n}\int_{\MdR^n}\HM^{m-n}(B\cap g^{-1}(\{y\})\cap p^{-1}(\{z\}))\,\HM^n(dz)\, \HM^n(dy)\\
&\le \alpha(n)^{-1}\frac{\alpha(m-n)\alpha(n)}{\alpha(m)}\text{Lip}(p)^n\HM^m(B\cap g^{-1}(\{y\})).
\end{align*}
Wegen $\text{Lip}(p)=1$ und da $Dg_{(x,y)}$ vollen Rang hat, folgt schließlich mit Hilfe von 
Teil (a), der für die Abbildung $g$ anwendbar ist,
\begin{align*}
&\int_{\MdR^n}\HM^{m-n}(A\cap f^{-1}(\{y\}))\,\HM^n(dy)\\
&\le\frac{\alpha(m-n)}{\alpha(m)}\int_{\MdR^n}\HM^m(B\cap g^{-1}(\{y\}))\, \HM^n(dy)\\
&=\frac{\alpha(m-n)}{\alpha(m)}\int_{B}Jg\, d(\HM^m\otimes\HM^n)\\
&\le \frac{\alpha(m-n)\alpha(n)}{\alpha(m)}\cdot\HM^m(A)\cdot \ep\cdot(\text{Lip}(f)+\ep)^{n-1}.
\end{align*}
Da $\ep>0$ beliebig klein gewählt werden kann, ist das Integral Null.
\end{enumerate}
\end{beweis}

\end{document}
