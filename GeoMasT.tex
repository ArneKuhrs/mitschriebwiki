\documentclass[a4paper,twoside,DIV15,BCOR12mm]{scrbook}
\usepackage{mathe}
\usepackage{saetze-hug}
\usepackage{enumerate}
\usepackage{dsfont}
\usepackage{cancel}
\usepackage{graphicx}
\newcommand{\A}{\mathcal A}
\newcommand{\borel}{\mathfrak B}
\newcommand{\ind}{\mathds 1}
\newcommand{\HR}{\mathcal H}
\newcommand{\HM}{\mathscr H}
\newcommand{\MN}{\mathbf M}
\newcommand{\NS}{\mathbf N}
\newcommand{\bw}{\bigwedge\nolimits}
\newcommand{\loc}{\mathrm loc}
%\renewcommand{\mathbb}{\mathds}
\DeclareMathOperator{\im}{im}
\newcommand{\id}{\mathrm{id}}
\DeclareMathOperator{\diam}{diam}
\DeclareMathOperator{\Lip}{Lip}
\DeclareMathOperator{\epi}{epi}
\DeclareMathOperator{\supp}{supp}
\DeclareMathOperator{\spt}{spt}
\DeclareMathOperator{\GL}{GL}
\DeclareMathOperator{\Kern}{Kern}
\DeclareMathOperator{\Bild}{Bild}
\DeclareMathOperator{\Div}{div}
\newcommand{\downto}{\mathrel\searrow}
\DeclareMathOperator{\sgn}{sgn}
% Maß-Restriktions-Symbol
\newcommand{\MR}{\lfloor}
\newcommand{\overarrow}{\overrightarrow}
\jot3mm
\pdfinfo{
	/Author (Die Mitarbeiter von http://mitschriebwiki.nomeata.de/)
	/Title   (Geometrische Maßtheorie)
	/Subject (Geometrische Maßtheorie)
}

\author{PD. Dr. Daniel Hug}
\publishers{Die Mitarbeiter von \url{http://mitschriebwiki.nomeata.de/}}
\title{Geometrische Maßtheorie}
\date{Wintersemester 2009/2010}
\makeindex

\begin{document}

\maketitle

\tableofcontents

\chapter*{Vorwort}

\section*{Über dieses Skriptum}
Dies ist ein Mitschrieb der Vorlesung \glqq Geometrische Maßtheorie\grqq\
von Herrn PD. Dr. Daniel Hug im Wintersemester 2009/2010 an der Universität Karlsruhe (TH).
 Die Mitschriebe der Vorlesung werden mit ausdrücklicher Genehmigung von Herrn Hug hier veröffentlicht,
Herr Hug ist für den Inhalt nicht verantwortlich.

\section*{Wer}
Beteiligt am Mitschrieb ist Joachim Breitner.

\section*{Wo}
Alle Kapitel inklusive \LaTeX-Quellen können unter \url{http://mitschriebwiki.nomeata.de} abgerufen werden.
Dort ist ein \emph{Wiki} eingerichtet und von Joachim Breitner um die \LaTeX-Funktionen erweitert.
Das heißt, jeder kann Fehler nachbessern und sich an der Entwicklung
beteiligen. Auf Wunsch ist auch ein Zugang über \emph{Subversion} möglich.

\chapter{Grundlagen: Maß und Integral}

\section{Äußere Maße und Messbarkeit}

\begin{definition}
Sei $X$ eine Menge. Eine Abbildung
\[
\mu : \mathcal P(X) \to [0,\infty]
\]
heißt \emph{äußeres Maß} auf $X$, falls gilt:
\begin{enumerate}
\item $\mu(\emptyset) = 0$
\item Für $A,A_n \subset X$, $i\in \MdN$ mit $A\subset \bigcup_{i\ge 1} A_i$ gilt 
\[
\mu(A) \le \sum_{i\ge1} \mu(A_i).
\]
\end{enumerate}
\end{definition}

Beobachte folgende einfache Folgerungen der Definition:

\begin{itemize}
\item $A\subset B \subset X \implies \mu(A)\le \mu(B)$
\item $A\subset B \cup \emptyset \cup \emptyset \cup \ldots \implies \mu(A) \le \mu(B) + \mu(\emptyset) + \mu(\emptyset) + \cdots = \mu(B)$
\end{itemize}

\begin{beispieleX}
\begin{align*}
\mu_1(A) &= 
\begin{cases}
\#A, & A \text{ endlich} \\
0, & \text{sonst}
\end{cases}
&
\mu_2(A) &= 
\begin{cases}
1, & A \ne \emptyset \\
0, & \text{sonst}
\end{cases} \\
\mu_3(A) &= 
\begin{cases}
\infty, & A \ne \emptyset \\
0, & \text{sonst}
\end{cases}
&
\mu_4(A) &= 
\begin{cases}
\infty, & A^c \text{ endlich} \\
0, & \text{sonst}
\end{cases} \\
\mu_5(A) &= 
\begin{cases}
0, & \text{$A$ abzählbar} \\
1, & \text{sonst}
\end{cases}
\end{align*}
\end{beispieleX}

Für die Konstruktion eines äußeren Maßes aus Rohdaten ist folgender Satz nützlich:

\begin{satz}
Sei $\mathcal E \subset \mathcal P (X)$ mit $\emptyset \in \mathcal E$, sei $\eta: \mathcal E \to [0,\infty]$ mit $\eta(\emptyset)=0$. Dann wird durch
\[
\mu(A) \da \inf\left\{\sum_{i=1}^\infty \eta(A_i) : A_i\in \mathcal E, i\in \MdN, A\subset\bigcup_{i\ge1}A_i\right\}
\]
($\inf\emptyset = \infty$) für $A\subset X$ ein äußeres Maß erklärt, das von $(\mathcal E, \eta)$ induzierte äußere Maß.
\end{satz}

\begin{beweis}
Es ist $0\le \mu(\emptyset) \le \sum_{i=1}^\infty \eta(\emptyset)=0$, da $\emptyset \subset \bigcup_{i=1}^\infty\emptyset$ und $\emptyset\in\mathcal E$.

Seien $A, A_i\subset X$ und $A\subset \bigcup_{i\ge1}A_i$. Wir müssen zeigen: $\mu(A) \le \sum_{i\ge 1}\mu(A_i)$.

Ist für ein $i\in\MdN$ bereits $\mu(A_i)=\infty$, so sind wir fertig. Sei also $\mu(A_i)<\infty$ für alle $i\in\MdN$. Sei $\varepsilon>0$. Dann existiert $E_{ij}\in \mathcal E$, $j\in \MdN$ mit $A_i\subset \bigcup_{j\ge 1} E_{ij}$ und
\[
\mu(A_i) + \frac{\varepsilon}{2^i}\ge \sum_{j\ge 1}\eta(E_{ij}) \quad \text{für $i\in \MdN$}.
\]
Also gilt
\[
A \subset \bigcup_{i\ge 1} A_i \subset \bigcup_{i,j\ge 1} E_{ij},
%\quad E_{ij}\in \mathcal E
\]
und daraus folgt
\begin{align*}
\mu(A) &\le \sum_{i,j\ge 1}\eta(E_{ij}) \\
&= \sum_{i=1}^\infty \sum_{j=1}^\infty \eta(E_{ij}) \\
&\le \sum_{i=1}^\infty (\mu(A_i) + \frac\varepsilon{2^i}) \\
&\le \left( \sum_{i=1}^\infty \mu(A_i) \right) + \varepsilon.
\end{align*}
Mit $\varepsilon \to 0$ ergibt dies
\[
\mu(A) \le \sum_{i=1}^\infty \mu(A_i).
\]
\end{beweis}

\begin{definition}
Sei $\mu$ ein äußeres Maß auf $X$. Eine Menge $A\subset X$ heißt $\mu$-messbar, falls für alle $M\subset X$ gilt:
\[
\mu(M) = \mu(M\cap A) + \mu(M \cap A^c).
\]
Die Menge aller $\mu$-messbaren Mengen wird mit $\A_\mu$ bezeichnet.
\end{definition}

Es genügt bereits: $A$ ist $\mu$-messbar genau dann, wenn für alle $M\subset X$ gilt:
\[
\mu(M) \ge \mu(M\cap A) + \mu(M\cap A^c).
\]

Denn wegen
\[
M\subset (M\cap A) \cup (M\cap A^c) \cup \emptyset \cup \emptyset\cdots
\]
gilt 
\[
\mu(M)\le \mu(M\cap A) + \mu(M\cap A^c) + \mu(\emptyset) + \cdots.
\]

Es gilt stets $\emptyset, X\in\A_\mu$.

\begin{bemerkung}
Für $Y\subset X$ ist $\mu \MR Y$ das durch
\[
(\mu \MR Y)(M) \da \mu(M\cap Y),\quad M\subset X
\]
erklärte äußere Maß. Ferner ist $\A_\mu \subset \A_{\mu \MR Y}$. Denn für $A\in\A_\mu$ und $M\subset X$ ist
\begin{align*}
\mu_{\MR Y}(M)
&= \mu(Y\cap M) = \mu(Y\cap M\cap A) + \mu(Y\cap M \cap A^c) \\
&= (\mu \MR Y) (M\cap A) + (\mu \MR Y)(M\cap A^c).
\end{align*}
Ferner gilt
\[
A \in \A_\mu \iff \mu= (\mu \MR A) + (\mu \MR  A^c).
\]
\end{bemerkung}

\begin{proposition}
Für ein äußeres Maß $\mu$ auf $X$ gelten die folgenden Aussagen:
\begin{enumerate}[a)]
\item $\emptyset,\, X\in \A_\mu$ sowie $A\in \A_\mu \iff A^c\in\A_\mu$.
\item Für $A\subset X$ mit $\mu(A)=0$ gilt $A\in\A_\mu$.
\item Für $A_i\in\A_\mu$, $i\in\MdN$ gilt $\bigcup_{i\ge1} A_i\in\A_\mu$ und $\bigcap_{i\ge1} A_i\in\mathcal A_\mu$.
\item Für $A\in\A_\mu$, $B\subset X$ gilt
\[
\mu(A\cap B) + \mu(A\cup B) = \mu(A) + \mu(B).
\]
\item Für $A_i\in\A_\mu$, $i\in\MdN$, paarweise disjunkt, gilt
\[
\mu(\bigcup_{i=1}^\infty A_i) = \sum_{i=1}^\infty \mu(A_i).
\]
\item Für $A_i\in \A_\mu$, $i\in\MdN$ und $A_i\subset A_{i+1}$ für alle $i\in\MdN$ gilt
\[
\mu(\bigcup_{i=1}^\infty A_i) = \lim_{i\to\infty}\mu(A_i).
\]
\item Für $A_i \in \A_\mu$, $i\in \MdN$ mit $\mu(A_1) < \infty$ und $A_i\supset A_{i+1}$ für alle $i\in\MdN$ gilt:
\[
\mu(\bigcap_{i=1}^\infty A_i) = \lim_{i\to\infty}\mu(A_i).
\]
\end{enumerate}
\end{proposition}

\begin{beweis}
\begin{enumerate}
\item[c)] Seien $A_1,A_2\in \A_\mu$, $M\supset X$. Dann folgt
\begin{align*}
\mu(M) &= \mu(M\cap A_1) + \mu(M\cap A_1^c) \\
&= \mu(M\cap A_1) + \mu(M\cap A_1^c\cap A_2) + \mu(M\cap A_1^c \cap A_2^c) \\
&\ge \mu(M \cap (A_1 \cup (A_1^c \cap A_2))) + \mu(M \cap A_1^c \cap A_2^c) \\
&= \mu(M\cap (A_1 \cup A_2)) + \mu(M\cap (A_1\cup A_2)^c).
\end{align*}
Daraus folgt $A_1\cup A_2 \in \A_\mu$. Per Induktion sieht man dann, dass für $A_1,\ldots,A_n\in\A_\mu$ gilt: $\bigcup_{i=1}^n A_i \in \A_\mu$.
\item[e)] Sind $A_1,\ldots,A_n\in\A_\mu$ und paarweise disjunkt, dann gilt
\begin{align*}
\mu(A_1\cup A_2) = \mu( (A_1\cup A_2)\cap A_1) + \mu( (A_1\cup A_2)\cap A_1^c) 
= \mu(A_1) + \mu(A_2),
\end{align*}
woraus
\[
\mu(\bigcup_{i=1}^n A_i) = \sum_{i=1}^n \mu(A_i)
\]
folgt. Wegen
\[
\sum_{i=1}^n \mu(A_i) \le \mu(\bigcup_{i=1}^\infty A_i)
\]
gilt
\[
\sum_{i=1}^\infty \mu(A_i) \le \mu(\bigcup_{i=1}^\infty A_i) \le \sum_{i=1}^\infty \mu(A_i)
\]
und damit Gleichheit.
\item[f)] Wir definieren $B_1 \da A_1$, $B_2 \da A_2\setminus A_1$, $B_3 \da A_3\setminus A_2\ldots$ Nun ist $B_i\in \A_\mu$ für alle $i\in\MdN$ und die $B_i$ sind paarweise disjunkt. Es folgt
\begin{align*}
\lim_{k\to\infty} \mu(A_k) 
&= \lim_{k\to\infty} \mu(\bigcup_{i=1}^k B_i) \\
&= \lim_{k\to\infty} \sum_{i=1}^k \mu(B_i)\\
&= \sum_{i=1}^\infty \mu(B_i) \tag{nach e)}\\
&= \mu(\bigcup_{i=1}^\infty B_i) \\
&= \mu(\bigcup_{i=1}^\infty A_i).
\end{align*}
\item[g)] Es ist
\begin{align*}
\mu(A_1) = \mu(A_2\cup (A_1\setminus A_2)) = \mu(A_2) + \mu(A_1\setminus A_2),
\end{align*}
das heißt
\[
\mu(A_1\setminus A_2) = \mu(A_1) - \mu(A_2).
\]
Damit zeigt man
\begin{align*}
\mu(A_1) -  \mu(\bigcap_{i\ge 1} A_i) 
&= \mu(A_1\setminus \bigcap_{i\ge1}A_i)  \\
&= \mu(A_1 \cap (\bigcap_{i\ge1}A_i)^c) \\
&= \mu(A_1 \cap (\bigcup_{i\ge1}A_i^c)) \\
&= \mu(\bigcup_{i\ge 1}(A_1\cap A_i^c)) \\
&= \lim_{i\to\infty} \mu(\underbrace{\A_1\cap A_i^c}_{= A_1\setminus A_i})\tag{nach f)} \\
&= \lim_{i\to\infty} (\mu(A_1) -\mu(A_i)) \\
&= \mu(A_1) - \lim_{i\to\infty}\mu(A_i)
\end{align*}
und damit die Behauptung.
\item[c)] Sei $M\subset X$. Wir definieren $C_k \da \bigcup_{i=1}^k A_i \in \A_\mu$. Damit gilt $C_1\subset C_2\subset \cdots$.

Sei ohne Beschränkung der Allgemeinheit $\mu(M) <\infty$. Dann gilt
\begin{align*}
\infty > \mu(M) &= (\mu \MR{}M)(X)\\
&= (\mu \MR{}M)(C_k) + (\mu \MR{}M)(C_k^c) \\
&= \lim_{k\to\infty} (\mu \MR M)(C_k) + \lim_{k\to\infty}(\mu \MR M)(C_k^c) \\
&= (\mu \MR M)(\bigcup_{i\ge 1}C_i) + (\mu \MR M)(\bigcap_{i\ge 1}C_i^c) \\
&= (\mu \MR M)(\bigcup_{i\ge 1}C_i) + (\mu \MR M)( (\bigcup_{i\ge 1}C_i)^c) \\
&= \mu(M\cap (\bigcup_{i\ge 1}A_i)) + \mu(M\cap (\bigcup_{i\ge 1}A_i)^c) 
\end{align*}
und somit $\bigcup_{i\ge 1}A_i \in \A_\mu$.

\item[d)] Für $A\in\A_\mu$ und $B\subset X$ gilt:
\begin{align*}
\mu(A\cup B) &= \mu( (A\cup B) \cap A) + \mu( (A\cup B)\cap A^c) \\
&= \mu(A) + \mu(B\cap A^c)\\
\intertext{sowie}
\mu(B) &= \mu(B\cap A) + \mu(B\cap A^c).
\intertext{Hiermit so erhält man}
\mu(A) + \mu(B) &= \mu(A) + \mu(B\cap A) + \mu(B\cap A^c)\\
&= \mu(B\cap A) + \mu(A\cup B).
\end{align*}
\end{enumerate}
\end{beweis}

Hinweis: Es ist $\A_\mu$ eine (bezüglich $\mu$ vollständige) $\sigma$-Algebra und $\mu$ ist ein $\sigma$-additives Maß auf $\A_\mu$, wobei „$\A_\mu$ ist $\mu$-vollständig“ heißt, dass jede $\mu$-Nullmenge in $\A_\mu$ liegt. $(X,\A_\mu)$ ist ein messbarer Raum und $(X,\A_\mu,\mu)$ ist ein Maßraum.

\begin{definition}
Sei $\A$ eine $\sigma$-Algebra auf $X$. Ein äußeres Maß $\mu$ auf $X$ heißt $\A$-regulär, falls $\A\subset \A_\mu$ gilt und zu jeder Menge $M\subset X$ ein $A\in\A$ existiert mit $M\subset A$ und $\mu(M) = \mu(A)$. Das äußere Maß $\mu$ heißt regulär, falls $\mu$ ein $\A_\mu$-reguläres Maß ist.
\end{definition}

\begin{proposition}\label{prop1.3}
Sei $\A$ eine $\sigma$-Algebra in $X$, $\mu$ ein $\A$-reguläres äußeres Maß auf $X$. Dann gilt:
\begin{enumerate}[a)]
\item Ist $M_i\subset X$, $M_i\subset M_{i+1}$ für alle $i\in\MdN$, so ist
\[
\mu(\bigcup_{i\ge 1}M_i) = \lim_{i\to\infty}\mu(M_i).
\]
\item Zu jedem $M\subset X$ mit $\mu(M)<\infty$ existiert ein $A\in\A$, so dass für alle $B\in \A_\mu$ gilt:
\[
\mu(B\cap M) = \mu(B\cap A)
\]
\item Ist $M_1\cup M_2\in \A$ und $\mu(M_1\cup M_2) = \mu(M_1)+\mu(M_2) <\infty$, so existiereren $A_1,A_2\in\A$ mit $M_i\subset A_i$, $i=1,2$ und $\mu(A_i\setminus M_i) = 0$. Insbesondere ist $M_1,M_2\in \A_\mu$.
\end{enumerate}
\end{proposition}

\begin{beweis}

\begin{enumerate}[a)]
\item Zu jedem $i\in\MdN$ finden wir ein $A_i\in \A$ so dass $M_i\subset A_i$ und $\mu(M_i) =\mu(A_i)$. Dazu definieren wir $B_i \da \bigcap_{j\ge i} A_j$. Damit gilt $M_i \subset B_i\subset A_i$, $B_i\subset B_{i+1}$ und $B_i\in\A$, $i\in\MdN$. Es folgt
\begin{align*}
\mu(\bigcup_{i\ge 1} M_i) &\le \mu(\bigcup_{i\ge1}B_i) \\
&= \lim_{i\to\infty} \mu(B_i) \\
&\le \lim_{i\to\infty} \mu(A_i) \\
&= \lim_{i\to\infty} \mu(M_i) \\
&\le \lim_{i\to\infty} \mu(\bigcup_{i\ge 1} M_i).
\end{align*}
\item  Zu $M$ existiert ein $A\in\A$ mit $M\subset A$ und $\mu(M) = \mu(A)$. Sei $B\in \A_\mu$. Dann folgt
\begin{align*}
\mu(A) = \mu(M) &= \mu(M\cap B) + \mu(M\cap B^c) \\
&\le \mu(A\cap B) + \mu(M \cap B^c) \\
&\le \mu(A\cap B) + \mu(A \cap B^c) = \mu(A),
\end{align*}
woraus Gleichheit in obigen Ungleichungen folgt. Wegen $\mu(M)<\infty$ ist auch $\mu(M\cap B^c)<\infty$, und wir können dies von zwei obigen Termen abziehen und erhalten
\[
\mu(M\cap B) = \mu(A\cap B).
\]
\item  Zu $M_1$ existiert $\tilde A_1\in\A$ mit $M_1 \subset \tilde A_1$ und $\mu(M_1) = \mu(\tilde A_1)$. Wir definieren $A_1 \da \tilde A_1 \cap (M_1\cup M_2)$. Für diese Menge gilt nun $M_1\subset A_1 \subset M_1\cup M_2$. Wir folgern 
\[
\mu(M_1) \le \mu(A_1) \le \mu (\tilde A_1) = \mu(M_1)
\]
und wegen $M_1\cup M_2 = A_1\cup M_2$ weiter
\begin{align*}
\mu(A_1\cap M_2) + \mu(A_1 \cup M_2) &= \mu(A_1) + \mu(M_2) \\
&= \mu(M_1) + \mu(M_2) \\
&= \mu(M_1\cup M_2) \\
&= \mu(A_1 \cup M_2) < \infty,
\end{align*}
woraus $\mu(A_1\cap M_2) = 0$ folgt.

Nun ist $A_1\setminus M_1\subset A_1\cap M_2$, also gilt $\mu(A_1\setminus M_1) = 0$ und somit $A_1\setminus M_1\in \A_\mu$. Damit gilt dann $M_1 = A_1\cap (A_1\setminus M_1)^c\in \A_\mu$.
\end{enumerate}

\end{beweis}

\begin{satz}\label{satz:1.4}
Sei $\A$ eine $\sigma$-Algebra in $X$ und $\nu$ ein Maß auf $\A$. Dann wird durch
\[
\mu(M) \da \inf\left\{\nu(A) : A\in\A,\, M\subset A\right\}
\]
für $M\subset X$ ein $\A$-reguläres äußeres Maß auf $X$ erklärt mit $\mu|_{\A} = \nu$. Ist $M\in\A_\mu$ und $\mu(M)<\infty$, so existiert ein $A\in\A$ mit $M\subset A$ und $\mu(A\setminus M) = 0$.
\end{satz}

\begin{beweis}
Für $M\subset X$ sieht man leicht, dass
\begin{align*}
\mu(M) &= \inf\left\{\sum_{i=1}^\infty \nu(A_i) : A_i \in \A,\, i\in \MdN,\, M\subset \bigcup_{i=1}^\infty A_i\right\}.
\end{align*}
Also ist $\mu$ das von $(\A,\nu)$ induzierte äußere Maß. Da $\nu$ monoton ist und nach der Definition von $\mu$ ist $\mu|_\A = \nu$.

Um die $\A$-Regularität zu zeigen, nehmen wir ein $A\in\A$ und ein $M\subset X$. Für $B\in\A$ mit $M\subset B$ gilt:
\begin{align*}
\mu(M\cap A) + \mu(M\cap A^c) 
&\le \nu(B\cap A) + \nu(B\cap A^c) \\
&= \nu(B)
\end{align*}
und daher
\[
\mu(M\cap A) + \mu(M\cap A^c) \le \mu(M).
\]
also ist $A\in\A_\mu$. Sei nun $M\subset X$ beliebig und ohne Beschränkung der Allgemeinheit $\mu(M)<\infty$. Es existiert also eine Folge $A_i\in\A$, $i\in\MdN$ mit $M\subset A_i$ und $\nu(A_i) \to \mu(M)$. Setze $A\da \bigcap_{i\ge 1} A_i$. Dann gilt $A \in \A$, $M\subset A$, sowie
\begin{align*}
\mu(M) = \lim_{i\to\infty} \nu(A_i) \ge \nu(\bigcap_{i=1}^\infty A_i) = \mu(\bigcap _{i=1}^\infty A_i) = \mu(A) \ge \mu(M),
\end{align*}
woraus $\mu(M) = \mu(A)$ folgt.

Sei schließlich $M\in\A_\mu$ mit $\mu(M)<\infty$. Es gibt ein $A\in\A$ mit $M\subset A$ und $\mu(M) = \mu(A)<\infty$. Dann folgt
\begin{align*}
\infty > \mu(A) &= \mu(A\cap M) +\mu(A\cap M^c) \\
&= \mu(M) + \mu(A\cap M^c) \\
&= \mu(A) + \mu(A\setminus M).
\end{align*}
Wegen $\mu(A)=\mu(M)<\infty$ gilt also $\mu(A\setminus M) =0$.
\end{beweis}


\begin{anwendung}
Sei $\vartheta$ ein beliebiges äußeres Maß auf $X$. Dann ist $\vartheta|_{\A_\vartheta}$ ein Maß. Durch
\[
\mu(M) \da \inf\{\vartheta(A) : A\in\A_\vartheta,\, M\subset A\}
\]
wird also ein $\A_\vartheta$-reguläres äußeres Maß auf $X$ erklärt, das $\vartheta$ fortsetzt (also $\mu|_{\A_\vartheta} = \vartheta|_{\A_\vartheta}$).
\end{anwendung}

\begin{definition}
Seien $X,Y$ Mengen, $\mu$ ein äußeres Maß auf $X$ und $f:X\to Y$. Dann wird durch
\[
(f\mu)(M) \da \mu(f^{-1}(M))
\]
für $M\subset Y$ ein äußeres Maß $f\mu$ auf $Y$ erklärt. Man nennt $f\mu$ das \emph{Bild} von $\mu$ unter $f$ oder auch \emph{„push forward“} von $\mu$ bezüglich $f$ und schreibt hierfür auch $f_\#\mu$.
\end{definition}

\begin{bemerkung}
Für $B\subset Y$ gilt
\[
f^{-1}(B) \in \A_\mu \iff \forall M\subset X: B \in \A_{f(\mu \MR M)}.
\]
Seien hierzu $M\subset X$, $A,B\subset Y$. Dann gilt
\begin{align*}
&\phantom{=\ \ }\mu(M\cap f^{-1}(A) \cap f^{-1}(B)) + \mu(M\cap f^{-1}(A) \cap f^{-1}(B)^c) \\
&= (\mu \MR M)(f^{-1}(A\cap B)) + (\mu \MR M)(f^{-1}(A\cap B^c)) \\
&= f(\mu \MR M)(A\cap B) + f(\mu \MR M) (A\cap B^c).
\end{align*}
Insbesondere gilt: Ist $f^{-1}(A) \in \A_\mu$, so ist $A\in\A_{f(\mu)}$.
\end{bemerkung}

\begin{sprechweisen}
Sei $\mu$ ein äußeres Maß auf $X$. Eine Menge $N\subset X$ heißt $\mu$-Nullmenge, falls $\mu(N)=0$. Eine Eigenschaft $\mathcal E$ gilt für $\mu$-fast-alle $x\in X$ bzw. $\mu$-fast-überall, falls 
\[
\mu(\{x\in X : \mathcal E\text{ gilt für $x$ nicht}\}) = 0.
\]
Mit $\mathbb F_\mu(X,Y)$ wird die Menge aller Abbildungen $f:D\to Y$ bezeichnet mit $D\subset X$ und $\mu(X\setminus D) = 0$.
\end{sprechweisen}

\begin{definition}
Seien $X,Y$ Mengen, $\mu$ ein äußeres Maß auf $X$ und $\mathcal C$ eine $\sigma$-Algebra in $Y$. Dann heißt $f\in\mathbb F_\mu(X,Y)$ $\mu$-messbar bezüglich $\mathcal C$, falls $f^{-1}(\mathcal C)\subset \A_\mu$.
\end{definition}

Beachte, dass für $f:D\to Y$ mit $\mu(X\setminus D)=0$ gilt: $D=f^{-1}(Y)\in \A_\mu$.

\begin{lemma}
Seien $X,Y$ Mengen, $\mu$ ein äußeres Maß auf $X$ und $\mathcal E\subset \mathcal P(Y)$.  Für $f\in\mathbb F_\mu(X,Y)$ sind äquivalent:
\begin{enumerate}[a)]
\item $f^{-1}(\mathcal E) \subset \A_\mu$
\item $f$ ist $\mu$-messbar bezüglich $\sigma(\mathcal E)$.
\end{enumerate}
\end{lemma}

\begin{definition}
Ist $(X,\mathcal T)$ ein topologischer Raum, so nennt man die von den offenen Mengen erzeugte $\sigma$-Algebra $\sigma(\mathcal T)$ die Borelsche $\sigma$-Algebra des topologischen Raumes $(X,\mathcal T)$ mit der Notation $\borel(X)$.

Spezielle Borelsche Algebren sind $\borel(\MdR)$, $\borel(\MdR^n)$, $\borel(\bar\MdR) \da \{B\in \bar\MdR : B\cap \MdR \in \borel(\MdR)\}$.
\end{definition}

\begin{definition}
Sei $X$ eine Menge, $\mu$ ein äußeres Maß auf $X$ und $f\in\mathbb F_\mu(X,\bar\MdR)$. Man nennt $f$ eine $\mu$-messbare Abbildung, falls $f$ dies bezüglich $\borel(\bar\MdR)$ ist.
\end{definition}

Im Folgenden schreiben wir für eine Relation $\varrho$ auf $\bar\MdR$, Mengen $D, D'\subset X$ und Abbildungen $f: D\to \bar\MdR$, $g:D'\to\bar\MdR$:
\[
\{ f\mathrel{\varrho} g\} \da \{ x\in D\cap D' : f(x)\mathrel{\varrho} g(x) \}
\]

\begin{lemma}
Sei $\mu$ ein äußeres Maß auf $X$ und $f\in \mathbb F_\mu(X,\bar\MdR)$. Genau dann ist $f$ eine $\mu$-messbare Abbildung, wenn eine der folgenden Bedingungen für alle $a\in\MdR$ erfüllt ist:
\begin{align*}
 \{f > a\} \in \A_\mu, && \{f\ge a\} \in \A_\mu, && \{f<a\} \in \A_\mu, && \{f\le a\}\in\A_\mu.
\end{align*}
\end{lemma}

\begin{lemma}
Sei $\mu$ ein äußeres Maß auf $X$, seien $f,g,f_n\in \mathbb F_\mu(X,\bar\MdR)$, $n\in\MdN$, $\mu$-messbar. Dann gilt
\begin{enumerate}[\quad(a)]
\item $\{f<g\}$, $\{f\le g\}$, $\{f=g\}$, $\{f\ne g\}$ sind $\mu$-messbare Mengen.
\item Die Funktionen
\begin{align*}
&f+ g, && f-g, && f \cdot g \text{ (falls $\mu$-fast-überall definiert)}, \\
&\sup_n f_n, && \inf_n f_n, && \\
&f^+ \da \max\{f,0\}, && f^- \da -\min\{f,0\}, && |f|, \\
&\limsup_n f_n, && \liminf_n f_n
\end{align*}
sind $\mu$-messbar.
\end{enumerate}
\end{lemma}

\begin{satz}
\label{satz:1.8}
Ist $\mu$ ein äußeres Maß auf $X$, so ist $f\in \mathbb F_\mu(X,\bar\MdR)$ genau dann $\mu$-messbar, wenn für alle $M\subset X$, $a,b\in\MdR$ mit $a<b$ gilt
\[
\mu(M) \ge \mu(M\cap \{f \le a\}) + \mu(M\cap \{f\ge b\}).
\]
\end{satz}

\begin{beweis}
Sei $f$ zunächst $\mu$-messbar. Dann gilt mit $a<b$, $M\subset X$: 
\begin{align*}
\mu(M) &\ge \mu(M\cap\{f\le a\} ) + \mu(M\cap \{f> a\}) \\
&\ge \mu(M\cap\{f\le a\} ) + \mu(M\cap \{f\ge b\}) 
\end{align*}

Jetzt gelte die Bedingung des Satzes für alle $M\subset X$, $a<b$. Zu zeigen ist: $\{f\le r\}\in \A_\mu$ für beliebige $r\in\MdR$. Sei $M\subset X$ beliebig mit $\mu(M) <\infty$. Für $i\in \MdN$ sei
\[
A_i \da M\cap \{r + \frac1{i+1} \le f \le r+\frac 1i\}.
\]
Wir zeigen mit vollständiger Induktion, dass 
\[
\mu(\bigcup_{i=0}^n A_{2i+1})  \ge \sum_{i=1}^n \mu(A_{2i+1})
\]
gilt.

Für $n=0$ ist dies klar. Die Ungleichung gelte für ein $n\in\MdN$.
\begin{align*}
\mu(\bigcup_{i=0}^{n+1} A_{2i+1}) 
&\ge \mu(\bigcup_{i=0}^{n+1} A_{2i+1} \cap \{f\ge \underbrace{r+ \frac1{2n+2}}_{b}\}) + 
     \mu(\bigcup_{i=0}^{n+1} A_{2i+1} \cap \{f\le \underbrace{r+\frac1{2n+3}}_{a}\}) \\
&= \mu(\bigcup_{i=0}^n A_{2i+1}) + \mu(A_{2n+3}) \\
&\ge \sum_{i=0}^n \mu(A_{2i+1}) + \mu(A_{2n+3}) \\
&\ge \sum_{i=0}^{n+1} \mu(A_{2i+1})
\end{align*}

Analog zeigt man 
\[
\mu(\bigcup_{i=1}^n A_{2i})  \ge \sum_{i=1}^n \mu(A_{2i})
\]
und erhält zusammen
\[
\sum_{i=1}^\infty \mu(A_i) \le 2 \mu(M) <\infty.
\]

Sei $\ep>0$. Dann gibt es ein $n\in\MdN$ mit $\sum_{i\ge n}\mu(A_i) <\ep$. Zunächst schätzen wir ab
\begin{align*}
\mu(M\cap \{r < f < r+\frac1n\}) 
&\le \mu(M\cap \{r<f\le r + \frac1n\}) \\
&= \mu(M\cap \bigcup_{i=n}^\infty \{r +\frac 1{i+1} \le f \le r+\frac 1i\}) \\
&= \mu(\bigcup_{i=n}^\infty A_i) \\
&\le \sum_{i\ge n} \mu(A_i) < \ep
\end{align*}
und damit
\begin{align*}
&\phantom{=\ \ }\mu(M\cap \{f\le r\}) + \mu(M\cap \{f>r\})  \\
&\le \mu(M\cap \{f\le r\}) + \mu(M\cap \{r<f<r+\frac1n\}) + \mu(M\cap \{f\ge r+\frac 1n\}) \\
&\le \mu(M) + \ep.
\end{align*}
\end{beweis}

\begin{satz}
Seien $\mu$ ein äußeres Maß auf $X$, $f: X \to [0,\infty]$ eine $\mu$-messbare Abbildung und $(r_n)_{n\in\MdN}$ eine Folge in $[0,\infty)$ mit $\lim_{n\to\infty} r_n = 0$ und $\sum_{n=1}^\infty r_n = \infty$. Dann gibt es eine Folge $(A_n)_{n\in\MdN}$ $\mu$-messbarer Mengen mit
\[
f = \sum_{n\ge1} r_n \ind_{A_n}.
\]
\end{satz}

\begin{beweis}
Setze $A_1 \da \{ f \ge r_1 \}$ und $A_n \da \{f \ge r_n + \sum_{j=1}^{n-1} r_j \ind_{A_j}\}$, $n> 1$.

\textbf{Behauptung:} Es ist $f \ge \sum_{i=1}^n r_i \ind_{A_i}$, $n\in\MdN$. Dies gilt für $n=1$, und wenn es für ein $n\in\MdN$ gilt, dann folgt: Ist $x\notin A_{n+1}$, dann ist
\begin{align*}
f(x) \ge \sum_{i=1}^n r_i \ind_{A_i}(x) + \underbrace{r_{n+1}\ind_{A_{n+1}}(x)}_{=0}
\end{align*}
nach Induktionsvoraussetzung. Ist dagegen $x\in A_{n+1}$, so gilt nach der Definition von $A_{n+1}$
\begin{align*}
f(x) \ge \sum_{i=1}^n r_i \ind_{A_i}(x) + r_{n+1} = \sum_{i=1}^n r_i \ind_{A_i}(x) + r_{n+1} \underbrace{\ind_{A_{n+1}}(x)}_{=1}.
\end{align*}

Folglich ist $f\ge \sum_{i=1}^\infty r_i \ind_{A_i}$.

\textbf{Annahme:} Es gelte $f(x) > \sum_{i=1}^\infty r_i \ind_{A_i}(x)$ für ein $x\in X$.

Also ist $\sum_{i=1}^\infty r_i \ind_{A_i}(x)<\infty$. Da $\sum_{i=1}^\infty r_i = \infty$ gilt, muss es eine Folge natürlicher Zahlen $(j_k)_{k\in\MdN}$ geben mit $\ind_{A_{j_k}}(x)=0$ für alle $k\in\MdN$.
Wegen $\lim_{k\to\infty} r_{j_k} = 0$ gibt es ein $k\in\MdN$ mit
\begin{align*}
r_{j_k} < f(x) - \sum_{j=1}^\infty r_j \ind_{A_j}(x)
\end{align*}
und damit 
\begin{align*}
f(x) &> \sum_{j=1}^\infty r_j \ind_{A_j}(x) + r_{j_k} \\
&\ge \sum_{j=1}^{j_k-1} r_j \ind_{A_j}(x) + r_{j_k}.
\end{align*}
Das bedeutet $x\in A_{j_k}$ im Widerspruch zu $\ind_{A_{j_k}}(x) = 0$.
\end{beweis}

\section{Integration}

In diesem Abschnitt wird generell vorausgesetzt, dass $X$ eine Menge und $\mu$ ein äußeres Maß auf $X$ ist.

\begin{definition}
Eine $\mu$-Treppenfunktion auf $X$ ist eine $\mu$-messbare Abbildung $h\in \mathbb F_\mu(X,\MdR)$ mit abzählbarer Wertemenge $\im(h)$ und
\[
\sum_{\substack{r\in \im(h)\\r< 0}} r \cdot \mu(\{h=r\}) > -\infty \quad\text{ oder }\quad
\sum_{\substack{r\in \im(h)\\r> 0}} r \cdot \mu(\{h=r\}) < \infty.
\]
Ist $h$ eine $\mu$-Treppenfunktion auf $X$, so wird durch
\[
\int hd\mu = \sum_{r\in \im(h)} r \cdot \mu(\{h=r\})
\]
das $\mu$-Integral von $h$ erklärt, wobei „$0\cdot \infty\da 0$“ gelte.
\end{definition}

\begin{bemerkung}
\begin{enumerate}
\item  Es gilt
\[
\int h d\mu = \int h^+ d\mu - \int h^- d\mu.
\]
\item $h=\ind_A$, $A\in\A_\mu$ ist eine $\mu$-Treppenfunktion, $\int \ind_A d\mu=\mu(A)$.
\end{enumerate}
\end{bemerkung}

\begin{lemma}
\label{lem1.10}
Seien $h,g$ $\mu$-Treppenfunktionen auf $X$. Es gelte $\int h^+ d\mu <\infty$ und $\int g^+ d\mu<\infty$  oder $\int h^- d\mu <\infty$ und $\int g^- d\mu<\infty$. Dann ist $h+g$ eine $\mu$-Treppenfunktion und es gilt
\[
\int (h+g) d\mu = \int h d\mu + \int g d\mu.
\]
\end{lemma}

\begin{beweis}
Es gilt zunächst $h+g \in \mathbb F_\mu(X,\MdR)$. Zur Additivität: Wir definieren $A(r,s) \da \{h=r\} \cap \{g=s\}$ für $r,s\in\MdR$. Die Voraussetzungen des Lemmas stellen sicher, 
dass die nachfolgend vorgenommenen Vertauschungen der Summationsreihenfolge zulässig sind. Es gilt damit
\begin{align*}
\int hd\mu + \int gd\mu 
&= \sum_{r\in\im(h)} r\cdot \mu(\{h=r\}) + \sum_{s\in \im(g)} s \cdot \mu(\{g=s\}) \\
&= \sum_{r\in\im(h)} r \cdot \sum_{s\in \im(g)} \mu(A(r,s)) + \sum_{s\in\im(g)} s \cdot \sum_{r\in \im(h)} \mu(A(r,s)) \\
&= \sum_{\substack{r\in\im(h)\\s\in\im(g)}} (r+s)\cdot \mu(A(r,s)) \\
&= \sum_{t\in \im(g+h)} t \cdot \sum_{\substack{r\in \im(h) \\s\in\im(g)\\r+s=t}} \mu(A(r,s)) \\
&= \sum_{t\in \im(g+h)} t \cdot \mu\big(\bigcup_{\substack{r\in \im(h) \\s\in\im(g)\\r+s=t}} A(r,s)\big) \\
&= \sum_{t\in \im(g+h)} t \cdot \mu(\{g+h=t\}) \\
&= \int (h+g) d\mu .
\end{align*}
Übung: Zeige, dass $\int (g+h)^+ d\mu<\infty$ oder $\int (g+h)^- d\mu<\infty$ gilt.
\end{beweis}

\begin{bemerkung}
Sei $h$ eine $\mu$-Treppenfunktion mit $h\ge 0$, dann gilt $\int hd\mu \ge 0$. Mit Lemma \ref{lem1.10} folgt für $\mu$-Treppenfunktionen $h,g$:
\[
h \le g \implies \int h d\mu \le \int g d\mu
\]
\end{bemerkung}

\begin{definition}
Sei $f\in\mathbb F_\mu(X,\bar\MdR)$. Eine $\mu$-Oberfunktion (bzw. $\mu$-Unterfunktion) von $f$ ist eine $\mu$-Treppenfunktion $h$ auf $X$ mit $f\le h$ $\mu$-fast-überall auf $X$ (bzw. $h\le f$ $\mu$-fast-überall auf $X$).

Durch
\[
\int^* fd\mu \da \inf\left\{\int hd\mu : \text{$h$ ist eine $\mu$-Oberfunktion von $f$}\right\}
\]
wird das $\mu$-Oberintegral von $f$ erklärt.
Analog wird durch
\[
\int_* fd\mu \da \sup\left\{\int hd\mu : \text{$h$ ist eine $\mu$-Unterfunktion von $f$}\right\}
\]
das $\mu$-Unterintegral von $f$ erklärt.
\end{definition}

\begin{lemma}
\label{lem1.11}
Für $f,g\in\mathbb F_\mu(X,\bar\MdR)$ gelten die folgenden Aussagen:
\begin{enumerate}
\item $\int_* fd\mu = - \int^* (-f)d\mu$.
\item Gilt $\mu$-fast-überall $f\le g$, so ist $\int^* fd\mu \le \int^* gd\mu$ und $\int_* fd\mu \le \int_* gd\mu$.
\item Gilt $\mu$-fast-überall $f\ge 0$, so ist $\int^* fd\mu \ge 0$ und $\int_* fd\mu \ge 0$.
\item Gilt $\int^* fd\mu <\infty$, so auch $\int^*f^+d\mu<\infty$ und $f<\infty$ $\mu$-fast-überall.
\item Für $c\in(0,\infty)$ gilt $\int^*(cf)d\mu = c\cdot \int^*fd\mu$ und $\int_*(cf)d\mu = c\cdot \int_*fd\mu$.
\item Ist $\int^* fd\mu <\infty$ und $\int^*g d\mu<\infty$, so ist $f+g \in \mathbb F_\mu(X,\bar\MdR)$ und $$\int^*(f+g)d\mu \le \int^*fd\mu + \int^*gd\mu.$$

Analog gilt: Ist $\int_* fd\mu >-\infty$ und $\int_*g d\mu>-\infty$, so ist $f+g \in \mathbb F_\mu(X,\bar\MdR)$ und $$\int_*(f+g)d\mu \ge \int_*fd\mu + \int_*gd\mu.$$
\item $\int_* fd\mu \le \int^* fd\mu$.
\end{enumerate}
\end{lemma}

\begin{beweis} Übung
\end{beweis}

\begin{bemerkung}
Ist $h$ eine $\mu$-Treppenfunktion, so ist $h$ eine $\mu$-Oberfunktion und eine $\mu$-Unterfunktion von $h$. Das heißt insbesondere
\[
\int hd\mu \le \int_* h d\mu \le \int^* h d\mu \le \int h d\mu.
\]
\end{bemerkung}

\begin{definition}
Ist $f\in \mathbb F_\mu(X,\bar\MdR)$ eine $\mu$-messbare Abbildung und stimmt das $\mu$-Oberintegral mit dem $\mu$-Unterintegral von $f$ überein, so wird durch
\[
\int fd\mu \da \int^* fd\mu = \int_* fd\mu
\]
das $\mu$-Integral von $f$ erkärt. Man sagt in diesem Fall, dass das $\mu$-Integral von $f$ existiert. Ist $\int fd\mu \in \MdR$, so heißt $f$ $\mu$-integrierbar.
\end{definition}

\begin{satz}
\label{satz1.12}
Sei $f\in \mathbb F_\mu(X,\bar\MdR)$ nicht-negativ und $\mu$-messbar. Dann existiert das $\mu$-Integral von $f$. Es gilt $\int fd\mu \ge 0$ und 
\[
\int fd\mu = \sup\left\{ \int hd\mu : \text{$h$ ist $\mu$-Unterfunktion, $\im(h)$ ist endlich}\right\}.
\]
\end{satz}

\begin{beweis}
Ist $\mu(\{f=\infty\})>0$, dann ist für jedes $n\in\MdN$ die Funktion $n\cdot \ind_{\{f=\infty\}}$ eine $\mu$-Unterfunktion von $f$ und 
\[
\int^*fd\mu \ge \int_* fd\mu \ge \int n \cdot\ind_{\{f=\infty\}} d\mu = n \cdot \mu(\{f=\infty\})\to \infty\text{ für }n\to\infty.
\]
Also ist $\int^* fd\mu = \int_* fd\mu = \infty$.

Sei jetzt $f<\infty$ $\mu$-fast-überall. Für $t\in(1,\infty)$ sei
\[
U_t \da \sum_{n\in\MdZ} t^n \cdot \ind_{\{t^n \le f < t^{n+1}\}}. 
\]
Offenbar ist
$$
U_t\le f\le t U_t
$$
$\mu$-fast-überall, d.h.\ 
$U_t$ ist eine $\mu$-Unterfunktion von $f$, $tU_t$ eine $\mu$-Oberfunktion von $f$. Damit gilt
\begin{align*}
\int^* fd\mu \le \int t U_t d\mu = t \cdot \int U_t d\mu \le t \int_* f d \mu.
\end{align*}
Ist $\int_* fd\mu <\infty$, dann folgt $\int^* fd\mu \le \int_* fd\mu$ aus $t\to 1$. Ist dagegen $\int_* fd\mu = \infty$, so ist $\int^* fd\mu = \int_* fd\mu = \infty$.
\end{beweis}



\begin{satz}
\label{satz1.13}
Seien $f,g\in\mathbb F_\mu(X,\bar\MdR)$ $\mu$-messbar. Dann gilt:
\begin{enumerate}
\item Sei $c\in \MdR\setminus\{0\}$. Es existiert $\int fd\mu$ genau dann, wenn $\int(cf)d\mu$ existiert. In diesem Fall ist 
\[
\int (cf) d\mu = c \cdot \int fd\mu.
\]
\item Angenommen, es existieren $\int fd \mu$ und $\int gf\mu$ und $(\int fd\mu, \int gd\mu) \ne (\pm \infty, \mp \infty)$. Dann ist $f+g\in \mathbb F_\mu(X,\bar\MdR)$, $\int(f+g)d\mu$ existiert und
\[
\int (f+g)d\mu = \int fd\mu + \int gd\mu.
\]
\item Ist $f\le g$ $\mu$-messbar und existiert $\int gd\mu$ (bzw. $\int fd\mu)$, so existiert auch $\int fd\mu$ (bzw. $\int gd\mu$), und es gilt in jedem Fall
\[
\int fd\mu \le \int gd\mu.
\]
\end{enumerate}
\end{satz}

\begin{beweis}
\begin{enumerate}
\item Es existiere $\int fd\mu$. Sei $c>0$. Dann folgt
\begin{align*}
\int^* (cf)d\mu 
&= c\cdot \int^*fd\mu = c\cdot \int_* fd\mu = \int_* (cf)d\mu.
\end{align*}
Sei $c<0$. Dann folgt
\begin{multline*}
\int^* (cf)d\mu 
= \int^*(-c)(-f)d\mu
= (-c)\cdot \int^*(-f)d\mu
= (-c) \cdot (-1) \int_* fd\mu 
\\
= c\cdot \int fd\mu 
= c\int^* fd\mu 
= (-c)\int_*(-f) d\mu
= \int_*(-c)(-f) d\mu
= \int_* (cf) d\mu .
\end{multline*}
\item Seien $f,g$ $\mu$-integrierbar. Dann ist $f,g<\infty$ $\mu$-fast-überall, und somit ist $f+g\in \mathbb F_\mu(X,\bar\MdR)$. Ferner gilt
\begin{multline*}
\int fd\mu + \int gd\mu 
= \int^* fd\mu + \int^* gd\mu
\ge \int^*(f+g)d\mu \\
\ge \int_*(f+g)d\mu
\ge \int_* fd\mu + \int_* gd\mu
= \int fd\mu + \int gd\mu,
\end{multline*}
woraus die Aussage folgt.

Sei nun $\int fd\mu = \infty$. Nach Voraussetzung gilt dann $\int gd\mu >-\infty$ und damit $g>-\infty$ $\mu$-fast-überall. Das heißt $f+g\in\mathbb F_\mu(X,\bar\MdR)$. Ferner gilt
\begin{align*}
\infty \ge \int^*(f+g)d\mu \ge \int_*(f+g)d\mu \ge \int_*f d\mu + \int_* gd\mu = \infty.
\end{align*}
Analog kann der Fall $\int fd\mu=-\infty$ gezeigt werden.
\item Sei $\int gd\mu <\infty$, das heißt $f\le g <\infty$ $\mu$-fast-überall. Dann ist 
$(f-g)\ind_{\{g>-\infty\}}\in F_\mu(X,\bar\MdR)$ nicht positiv. Wegen $f\le g <\infty$ $\mu$-fast-überall 
ist $g+(f-g)\ind_{\{g>-\infty\}}=f$ und damit ergibt (2)
\begin{align*}
\int gd\mu \ge \int gd\mu + \int (f-g)\ind_{\{g>-\infty\}}d\mu = \int fd\mu.
\end{align*}
% Was ist mit \int_ fd\mu = -\infty?
\end{enumerate}
\end{beweis}

\begin{satz}
Sei $f\in\mathbb F_\mu(X,\bar\MdR)$ $\mu$-messbar. Dann gilt:
\begin{enumerate}
\item $\int f^+ d\mu$, $\int f^- d\mu$ existieren stets. Es existiert $\int fd\mu$ genau dann, wenn $\int f^+d \mu <\infty$ oder $\int f^- d\mu<\infty$. In diesem Fall ist 
\[
\int fd\mu = \int f^+d\mu -\int f^-d\mu.
\]
Ferner ist
\[
\Big|\int fd\mu\Big| \le \int |f|d\mu .
\]
\item Ist $f$ $\mu$-integrierbar, so auch $|f|$. 
\end{enumerate}
\end{satz}

\begin{beweis}
\begin{enumerate}
\item Es existiere $\int fd\mu$. Ist $\int fd\mu <\infty$, so ist $\int f^+ d\mu <\infty$ nach Lemma \ref{lem1.11} (4). Ist dagegen $\int fd\mu =\infty$, so ist $\int (-f)d\mu = -\infty$ und daher $\int f^- d\mu = \int (-f)^+d\mu <\infty$.

Umgekehrt sei $\int f^+d\mu <\infty$ oder $\int f^- d\mu <\infty$. Dann existiert nach Satz \ref{satz1.13} (2) das Integral $\int (f^+-f^-)d\mu$ wegen Satz \ref{satz1.13} (1) ist $\int fd\mu = \int f^+d\mu - \int f^-d\mu$.

Stets existiert das Integral von $|f|$ und 
\begin{align*}
\int |f|d\mu = \int (f^+ + f^-) d\mu = \int f^+ d\mu + \int f^- d\mu 
\ge \Big|\int f^+ d\mu - \int f^-d\mu \Big| = \Big|\int fd\mu \Big |.
\end{align*}
\item Ist $f$ $\mu$-integrierbar, so folgt aus (1), dass $\int f^+d \mu <\infty$ \emph{und} $\int f^-d \mu<\infty$.
\end{enumerate}
\end{beweis}

\begin{satz}[Lemma von Fatou]
\label{satz1.15}
Sei $f_n \in \mathbb F_n(X,\bar\MdR)$, $n\in\MdN$, mit $f_n\ge 0$ und $\mu$-messbar. Dann gilt
\begin{align*}
\int \liminf_{n\to\infty} f_n d\mu \le \liminf_{n\to\infty} \int f_n d\mu.
\end{align*}
\end{satz}

\begin{beweis}
Sei $\varepsilon \in (0,1)$. Sei $h$ eine $\mu$-Unterfunktion von $\liminf_{n\to\infty} f_n$ mit $\im(h) = \{r_1,\ldots,r_m\}\subset [0,\infty)$ (vergleiche Satz \ref{satz1.12}). Für $i=1,\ldots,m$ und $n\in\MdN$ sei 
\begin{align*}
A_{i,n}\da \{h=r_i\} \cap \{\inf_{k\ge n} f_k \ge \ep\cdot r_i\} \in A_\mu.
\end{align*}
Es gilt $A_{i,n}\subset A_{i,n+1}$ für $i=1,\ldots,m$ und $n\in\mathbb{N}$. Für $\mu$-fast-alle $x\in X$ mit $h(x) = r_i$ gilt:
\begin{align*}
\ep r_i < r_i \le \liminf_{n\to\infty} f_n(x) = \sup_{n\in\MdN} \inf_{k\ge n} f_k(x).
\end{align*}
Es gibt ein $n\in\MdN$ mit $\ep r_i < \inf_{k\ge n} f_k(x),$ das heißt $x\in A_{i,n}$. Also
\begin{align*}
\mu(\{h=r_i\}) = \mu(\bigcup_{n\in\MdN} A_{i,n}).
\end{align*}

Die Mengen $A_{i,n}$, $i=1,\ldots,m$, sind paarweise disjunkt und aus ihrer Definition folgt, dass
\begin{align*}
\sum_{i=1}^m \ep r_i \ind_{A_{i,n}}
\end{align*}
eine $\mu$-Unterfunktion von $f_n$ ist.

Hiermit gilt
\begin{align*}
\liminf_{n\to\infty} \int f_n d\mu 
&\ge \liminf_{n\to\infty} \sum_{i=1}^m \ep r_i \mu(A_{i,n})\\
&= \sum_{i=1}^m \ep r_i \liminf_{n\to\infty} \mu(A_{i,n})\\
&= \sum_{i=1}^m \ep r_i \mu( \bigcup_{n\in\MdN} A_{i,n})\\
&= \ep \sum_{i=1}^m r_i \mu(\{h=r_i\}) \\
&= \ep \int hd\mu.
\end{align*}
Es folgt
\begin{align*}
\int \liminf_{n\to\infty} f_n d\mu \le \frac1\ep \liminf_{n\to\infty} \int f_n d\mu
\end{align*}
für ein beliebiges $\ep \in (0,1)$. Lässt man $\ep$ gegen $1$ gehen, so folgt die Behauptung.
\end{beweis}

\begin{satz}[von der monotonen Konvergenz]
\label{satz1.16}
Ist $f_n\in\mathbb F_\mu(X,\bar\MdR)$ $\mu$-messbar, $n\in\MdN$, $0\le f_1\le f_2\le \cdots$, so gilt
\begin{align*}
\lim_{n\to\infty} \int f_n d\mu = \int \lim_{n\to\infty} f_n  d\mu.
\end{align*}
\end{satz}

\begin{beweis}
Grenzwerte und Integrale existieren offenbar. Es gilt 
\begin{align*}
\lim_{n\to\infty}\int f_n d\mu 
&\le \int \lim_{n\to\infty} f_n d\mu \\
&\le \limsup_{n\to\infty} \int f_n d\mu. \tag{nach Satz \ref{satz1.15}}
\end{align*}
\end{beweis}

\begin{satz}[Lebesgue]
\label{satz1.17}
Sei $f_n\in\mathbb F_\mu(X,\bar\MdR)$, $n\in\MdN$ eine konvergente Folge $\mu$-messbarer Funktionen. Es existiere eine $\mu$-integrierbare Funktion $g\in\mathbb F_\mu(X,\bar\MdR)$ mit $|f_n|\le g$ $\mu$-fast-überall für jedes $n\in\MdN$. Dann sind $f_n$ und $f\da \lim_{n\to\infty} f_n$ $\mu$-integrierbar und 
\begin{align*}
\lim_{n\to\infty} \int f_n d\mu = \int f d\mu.
\end{align*}
Schärfer gilt sogar
\begin{align*}
\lim_{n\to\infty} \int |f-f_n| d\mu = 0.
\end{align*}
\end{satz}

\begin{beweis}
Wegen $|f_n|,|f| \le g$ ist $\int |f_n| d\mu <\infty$ und $\int |f|d\mu <\infty$. Die Folge $(2g-|f_n-f|)_{n\in\MdN}$ nichtnegativer Funktionen in $\mathbb F_\mu(X,\bar\MdR)$ konvergiert für $n\to\infty$ $\mu$-fast-überall gegen $2g$. Mit dem Lemma von Fatou (Satz \ref{satz1.15}) folgt:
\begin{multline*}
\int 2g d\mu - \limsup_{n\to\infty} \int |f_n -f|d\mu
= \liminf_{n\to\infty} \int (2g - |f_n -f|) d\mu \\
\ge \int \underbrace{\liminf_{n\to\infty}(2g - |f_n -f|)}_{=2g} d\mu = \int 2gd\mu
\end{multline*}
also 
\begin{align*}
\lim_{n\to\infty} \int |f_n -f| d\mu = 0.
\end{align*}
\end{beweis}

\begin{notation}
Für $A\subset X$ und $f\in\mathbb F_\mu (X,\bar\MdR)$ sei
\begin{align*}
\int^*_A fd\mu \da \int^* \ind_{A}f d\mu && \text{und} &&
\int_{*A} fd\mu \da \int_* \ind_{A}f d\mu.
\end{align*}
Existiert das $\mu$-Integral von $\ind_A \cdot f$, so setzt man
\begin{align*}
\int_A f d\mu \da \int \ind_A fd\mu.
\end{align*}
\end{notation}

\begin{lemma}
\label{lem1.18}
(Übungsblatt 2, Aufgabe 4) Sei $f_n\in\mathbb F_\mu(X,\bar\MdR)$, $n\in\MdN$, eine Folge nichtnegativer Funktionen. Dann gilt
\begin{align*}
\int^* \sum_{n=1}^\infty {f_n} d\mu \le \sum_{n=1}^\infty \int^* f_n d\mu.
\end{align*}
\end{lemma}

\begin{satz}
\label{satz1.19}
Sei $g\in \mathbb F_\mu(X,\bar\MdR)$ eine nichtnegative Funktion. Dann wird  durch
\begin{align*}
\psi(A) \da \int_A^* gd\mu, \  A\subset X,
\end{align*}
 ein äußeres Maß auf $X$ definiert. Es gilt $\A_\mu \subseteq \A_\psi$. Ist $g$ sogar $\mu$-messbar und $f\in\mathbb F_\mu(X,\bar\MdR)$ $\mu$-messbar, dann existiert $\int fd\psi$ genau dann, wenn $\int fgd\mu$ existiert. In diesem Fall gilt
\begin{align*}
\int fd\psi = \int fg d\mu.
\end{align*}
\end{satz}

\begin{beweis}
Es gilt $\psi(\emptyset)=0$. Die Subsigmaadditivität folgt direkt aus Lemma \ref{lem1.18}.

Sei $A\in\A_\mu$ und $M\subset X$ mit $\psi(M)<\infty$. Sei $\ep>0$ beliebig. Dann existiert eine $\mu$-Oberfunktion $h$ zu $\ind_M\cdot g$ mit $h\ge 0$ und
\[
\int h d\mu \le \int^* \ind_M g d \mu + \ep.
\]
Dann ist $\ind_A\cdot h$ eine $\mu$-Oberfunktion zu $\ind_{M\cap A}\cdot g$ und $\ind_{A^c}\cdot h$ ist eine $\mu$-Oberfunktion zu $\ind_{M\cap A^c}\cdot g$. Daher folgt
\begin{align*}
\psi(M\cap A)+\psi(M\cap A^c)
&= \int^* \ind_{M\cap A} gd\mu + \int^* \ind_{M\cap A^c} gd\mu \\
&\le \int \ind_A \cdot h d\mu + \int \ind_{A^c} \cdot h d\mu \\
&= \int h d\mu \le \int^* \ind_M g d\mu + \ep = \psi(M) + \ep.
\end{align*}

Sei nun $g$ $\mu$-messbar und $f\in \mathbb F_\mu(X,\bar\MdR)$ und sei zunächst $f\ge 0$. Also existieren $\int fd\psi$ und $\int gfd\mu$. Es gibt eine Folge $(r_j)_{j\in\MdN}$ in $(0,\infty)$ und $(A_j)_{j\in\MdN}$ in $A_\mu$ mit $f = \sum_{j=1}^\infty r_j\ind_{A_j}$. Zweimalige Anwendung des 
Satzes von der monotonen Konvergenz (Satz \ref{satz1.16}) ergibt 
\begin{align*}
\int fd\psi 
&= \int \sum_{j=1}^\infty r_j \ind_{A_j} d\psi \\
&= \sum_{j=1}^\infty r_j \int \ind_{A_j} d\psi \\
&= \sum_{j=1}^\infty r_j \psi(A_j) \\
&= \sum_{j=1}^\infty r_j \int \ind_{A_j} gd\mu \\
&= \int \left( \sum_{j=1}^\infty r_j \ind_{A_j}\right) g d\mu \\
&= \int fgd\mu.
\end{align*}
Sei nun $f$ eine beliebige $\mu$-messbare Funktion. Wegen $(fg)^\pm = f^\pm \cdot g$ gilt $\int f^\pm d\psi < \infty$ genau dann, wenn $\int (f g)^{\pm} d\mu<\infty$. Somit existiert $\int f d\psi$ genau dann, wenn $\int fg d\mu$ existiert und
\begin{multline*}
\int fd\psi = \int f^+ d\psi - \int f^- d\psi = \int f^+gd\mu - \int f^-gd\mu = \int (f^+ - f^-)g d\mu = \int fgd\mu.
\end{multline*}
\end{beweis}

\begin{satz}
Sei $f\in \mathbb F_\mu(X,\bar\MdR)$ $\mu$-integrierbar. Dann gibt es zu jedem $\ep >0$ ein $\delta > 0$ derart, dass für alle $A\in \A_\mu$ mit $\mu(A)<\delta$ gilt
\[
\int_A |f|d\mu <\ep.
\]
\end{satz}

\begin{beweis}
Betrachte $g_n\da \min\{|f|,n\}$, $n\in\MdN$. Es gilt $0\le g_n\nearrow |f|$ für $n\to\infty$. Damit ist
\begin{align*}
\lim_{n\to \infty} \int g_n d\mu = \int |f| d\mu <\infty.
\end{align*}
Zu einem vorgegebenem $\ep>0$ gibt es ein $N\in\mathbb{N}$ mit 
\begin{align*}
0\le \int |f|d\mu - \int g_N d\mu < \frac\ep2.
\end{align*}
Für $A\in\A_\mu$ mit $\mu(A) < \frac\ep{2N} \ad \delta$ folgt nun
\begin{align*}
\int_A |f| d\mu
&= \int_A\underbrace{(|f|-g_N)}_{\ge 0} d\mu + \int _A g_N d\mu \\
&\le \int\underbrace{(|f|-g_N)}_{\ge 0} d\mu + N \cdot \mu(A) \\
&< \frac\ep2 + \frac\ep2 = \ep.
\end{align*}
\end{beweis}

Seien $X,Y\neq\emptyset$ Mengen mit äußeren Maßen $\mu$ auf $X$, $\nu$ auf $Y$. Durch
\[
(\mu \times \nu)(M) \da \inf\{ \sum_{i=1}^\infty \mu(A_i)\nu(B_i): A_i\in \A_\mu, B_i\in \A_\nu,\, i\in\MdN,\, M\subset \bigcup_{i=1}^\infty (A_i \times B_i)\}
\]
wird ein äußeres Maß auf $X\times Y$ erklärt, nämlich das von $\mathcal E_0 \da \{A \times B : A\in\A_\mu,\, B\in\A_\nu\}$ und $\lambda$ mit $\lambda(A\times B) \da \mu(A)\cdot \nu(B)$ für $A\in\A_\mu$, $B\in\A_\nu$ induzierte äußere Maß.

\begin{satz}[Fubini]
Seien $X,Y\ne \emptyset$ Mengen mit äußeren Maßen $\mu$ auf $X$ und $\nu$ auf $Y$. Dann gelten folgende Aussagen:
\begin{enumerate}
\item $\A_\mu \otimes \A_\nu \subset \A_{\mu\times \nu}$ und $(\mu\times \nu)(A\times B) = \mu(A) \cdot \nu(B)$ für $A\in\A_\mu$, $B\in\A_\nu$.
\item Das Maß $\mu\times\nu$ ist $\A_\mu \otimes \A_\nu$-regulär.
\item Existiert das $(\mu \times \nu)$-Integral von $f\in\mathbb F_{\mu\times\nu}(X\times Y,\bar\MdR)$ und gilt $\{f\ne 0\}\subset \bigcup_{i=1}^\infty M_i$, $M_i\in\A_{\mu\times\nu}$, $(\mu\times\nu)(M_i)<\infty$, $i\in\MdN$, so gilt:

$f(\cdot,y) \in\mathbb F_\mu(X,\bar\MdR)$ ist $\mu$-messbar für $\nu$-fast-alle $y\in Y$. Es ist $\int f(x,y)\mu(dx)$ $\nu$-messbar und $\iint f(x,y)\mu(dx)\nu(dy)$ existiert (und symmetrisch in $x$ und $y$) und schließlich:
\[
\int f d(\mu\times\nu) = \iint f(x,y)\mu(dx)\nu(dy) = \iint f(x,y) \nu(dy) \mu(dx).
\]
\end{enumerate}
\end{satz}

\begin{beweis}
Wir setzen
\begin{multline*}
\mathcal E \da \{ M\subset X\times Y: \ind_M(\cdot,y) \in \mathbb F_\mu(X,\bar\MdR) \text{ ist $\mu$-messbar für $\nu$-fast-alle $y\in Y$,} \\
\int\ind_M(x,y) \mu(dx) \in \mathbb F_\nu(Y,\bar\MdR)\text{ ist $\nu$-messbar}\}.
\end{multline*}
Für $M\in\mathcal E$ sei
\[
\varrho(M) \da \iint \ind_M (x,y)\mu(dx)\nu(dy).
\]
Wir zeigen zwei Hilfsbehauptungen:
\begin{enumerate}
\item[($\alpha$)] Ist $M_j\in\mathcal E$, $j\in\MdN$, eine Folge paarweise disjunkter Mengen, so ist $\bigcup_{j=1}^\infty M_j\in\mathcal E$, denn:

\[
\ind_{\bigcup_{j=1}^\infty M_j}(\cdot, y)=\sum_{j=1}^\infty \ind_{M_j}(\cdot, y)
\]
ist $\mu$-messbar für $\nu$-fast-alle $y\in Y$ und \[
\int\ind_{\bigcup_{j=1}^\infty M_j} (x,y)\mu(dx) = \sum_{j=1}^\infty \int \ind_{M_j}(x,y) \mu(dx)
\]
ist $\nu$-messbar.

\item[($\beta$)] Ist $M_j\in\mathcal E$, $j\in\MdN$, $M_1\supset M_2 \supset \cdots$ sowie $\varrho(M_1) <\infty$, so gilt $\bigcap_{j\ge 1}M_j \in\mathcal E$, denn:

\[
\ind_{\bigcap_{j=1}^\infty M_j}(\cdot, y) = \lim_{j\to\infty}\ind_{M_j}(\cdot,y)
\]
ist $\mu$-messbar für $\nu$-fast-alle $y\in Y$ und 
\[
\int\ind_{\bigcap_{j=1}^\infty M_j} (x,y)\mu(dx) = \lim_{j\to\infty} \int \ind_{M_j} (x,y)\mu(dx)
\]
ist $\nu$-messbar.
\end{enumerate}

Betrachte nun folgende Mengensysteme:
\begin{align*}
\mathcal E_0 &\da \{A\times B: A\in\A_\mu,\, B\in\A_\nu\}, \\
\mathcal E_1 &\da \{\bigcup_{i=1}^\infty G_i: G_i\in\mathcal E_0\}, \\
\mathcal E_2 &\da \{\bigcap_{j\ge 1}^\infty H_j: H_j\in\mathcal E_1\}.
\end{align*}

Für $A\times B\in\mathcal E_0$ ist $\ind_{A\times B}(\cdot, y) = \ind_A \cdot \ind_B(y)$ $\mu$-messbar für alle $y\in Y$ und $\int \ind_{A\times B}(x,y) \mu(dx) = \mu(A) \cdot \ind_B(y)$ ist $\nu$-messbar. Also ist $A\times B\in \mathcal E$, und damit $\mathcal E_0\subset\mathcal E$.

Für $A\times B\in \mathcal E_0$, $C\times D\in\mathcal E_0$ ist 
\begin{align*}
(A\times B)\cap (C\times D) &= (A\cap C) \times (B\cap D) \in \mathcal E_0
\intertext{und}
(A\times B)\setminus (C\times D) &= \big(\underbrace{(A\setminus C) \times B}_{\in\mathcal E_0}\big) \stackrel{\bullet}\cup \big(\underbrace{(A\cap C) \times (B\setminus D)}_{\in\mathcal E_0}\big).
\end{align*}
Jede abzählbare Vereinigung von Mengen aus $\mathcal E_0$ kann als abzählbare Vereinigung von paarweise disjunkten Mengen aus $\mathcal E_0$ erhalten werden, das heißt $\mathcal E_1\subset \mathcal E$ nach ($\alpha$).

Da $\mathcal{E}_1$ stabil bezüglich der Bildung endlicher Durchschnitte ist, folgt mit Hilfe von $(\beta)$
\[
\{\bigcap_{i=1}^\infty H_i : H_i\in\mathcal E_1,\, i\in\MdN,\, \varrho(H_1)<\infty\} \subset \mathcal E.
\]

\textbf{Behauptung:} Für $M\subset X\times Y$ gilt 
\[
(\mu\times\nu)(M) = \inf\{\varrho(V):M\subset V, V\in \mathcal E_1\}
\]
und es gibt zu $M$ ein $W\in\mathcal E_2$ mit $M\subset W$ und $(\mu\times\nu)(M) = (\mu\times\nu)(W) = \varrho (W)$.

\textbf{Nachweis:} Für $i\in\mathbb{N}$ sei $A_i\times B_i  \in \mathcal E_0$ mit $M\subset \bigcup_{i=1}^\infty (A_i\times B_i)\ad V \in\mathcal E_1$. Dann gilt 
$$\ind_V \le \sum_{i=1}^\infty \ind_{A_i\times B_i},
$$ 
wobei Gleichheit gilt, falls die Mengen $A_i\times B_i$ paarweise disjunkten sind. Somit erhält man 
$$\varrho(V) \le \sum_{i=1}^\infty \varrho(A_i\times B_i) = \sum_{i=1}^\infty \mu(A_i) \nu(B_i),
$$ 
wobei auch hier Gleichheit gilt, falls die Mengen $A_i\times B_i$, $i\in\mathbb{N}$, paarweise disjunkt sind.

Der erste Teil der Behauptung folgt somit aus
\begin{align*}
(\mu\times\nu)(M) &= \inf\{\sum_{i=1}^\infty \mu(A_i)\nu(B_i) : A_i\times B_i \in \mathcal E_0, i\in\MdN, M\subset\bigcup_{i=1}^\infty(A_i\times B_i)\} \\
&= \inf\{\varrho(V): M\subset V, V\in\mathcal E_1\}.
\end{align*}

Ist $(\mu\times \nu)(M)<\infty$, so existieren $V_i\in\mathcal E_1$, $i\in\MdN$, $M\subset V_i$ mit
\[
\lim_{i\to\infty} \varrho(V_i) = (\mu\times\nu)(M).
\]
Setze $M\subset W \da \bigcap_{i=1}^\infty V_i \in\mathcal E_2$. Es gilt
\[
(\mu\times\nu)(M) \le (\mu\times \nu)(W) \le \lim_{i\to\infty}\varrho(V_i) = \varrho(W) = (\mu\times\nu)(M).
\]
Ist $(\mu\times\nu)(M) =\infty$, so setze $W\da X\times Y\in\mathcal E_2$.

Nun beweisen wir die eigentlichen Aussagen des Satzes:
\begin{enumerate}
\item Sei $A\times B\in\mathcal E_0$. Zunächst gilt offenbar

Für ein beliebiges $V\in\mathcal E_1$ mit $A\times B\subset V$ gilt
\[
(\mu\times\nu)(A\times B) =\inf\{\varrho(V):A\times B\subset V,V\in\mathcal{E}_1\}= \varrho(A\times B) = \mu(A) \nu(B).
\]
Für $T\subset X\times Y$ und $U\in \mathcal E_1$ mit $T\subset U$ sind
$U\cap (A\times B)$ und $U\cap (A\times B)^c$ disjunkte Mengen in $\mathcal E_1$. Wir  erhalten so
\begin{multline*}
(\mu\times \nu) (T\cap (A\times B)) + (\mu\times\nu)(T\cap (A\times B)^c)
\\ \le \varrho(U\cap (A\times B)) + \varrho(U\cap (A\times B)^c) = \varrho(U).
\end{multline*}
Bildet man das Infimum über alle $U\in\mathcal E_1$ mit $U\supset T$, so ergibt diese Ungleichung
\[
(\mu\times \nu) (T\cap (A\times B)) + (\mu\times\nu)(T\cap (A\times B)^c) \le (\mu\times \nu)(T),
\]
woraus $A\times B\in\A_{\mu\times\nu}$ folgt.
\item Ist $M\subset X\times Y$ und $(\mu\times\nu)(M)<\infty$, so gibt es $W\in\mathcal E_2$ mit $\varrho(W)<\infty$ und mit der gewünschten Eigenschaft $(\mu\times\nu)(M) = (\mu\times\nu)(W)$.
\item Sei $f=\ind_M$, $M\in\A_{\mu\times\nu}$ und $(\mu\times\nu)(M)<\infty$. Zu $M$ existiert ein $W\in\mathcal E_2$ mit $M\subset W$ und $(\mu\times\nu)(M) = (\mu\times\nu)(W) =\varrho(W)$.

Fall 1: $(\mu\times\nu)(M) = 0$. Dann gilt $\varrho(W)=0$ und $\ind_M(\cdot,y)=0$ $\mu$-fast-überall für $\nu$-fast-alle $y\in Y$. Insbesondere ist $M\in\mathcal E$ und $\varrho(M)=0$.

Fall 2: $(\mu\times\nu)(M) > 0$. Dann gilt $(\mu\times\nu)(W\setminus M) = 0$, $M\subset W$. Fall 1 liefert $W\setminus M\in\mathcal E$ und $\varrho(W\setminus M)=0$. Also ist $\ind_M(\cdot, y) = (\ind_W - \ind_{W\setminus M})(\cdot,y)$ $\mu$-messbar für $\nu$-fast-alle $y\in Y$ und  $\ind_M(\cdot,y) = \ind_W(\cdot,y)$ $\mu$-fast-überall für $\nu$-fast-alle $y\in Y$. Insbesondere ist $M\in\mathcal E$ und $\varrho(M) = \varrho(W) = (\mu\times\nu)(M)$.
\end{enumerate}
\end{beweis}

\chapter{Äußere Maße auf metrischen Räumen}

\section{Regularität und metrische äußere Maße}

Sei $(X,d)$ ein metrischer Raum mit der von $d$ induzierten Topologie $\mathcal T$. Sei $\borel(X)$ die Borel-$\sigma$-Algebra. Ist $\mu$ ein äußeres Maß auf $X$ und sind alle offenen (alle abgeschlossenen) Mengen in $\A_\mu$, so gilt $\borel(X)\subset \A_\mu$. Wir nennen $\mu$ Borel-regulär, wenn $\mu$ $\borel(X)$-regulär ist, das heißt $\borel(X)\subset \A_\mu$ und zu jedem $M\subset X$ existiert ein $B\in\borel(X)$ mit $M\subset B$ und $\mu(M)=\mu(B)$.


\begin{lemma}
\label{lem:2.1}
Sei $(X,d)$ ein metrischer Raum und $\A\subset \mathcal P(X)$. Für jede Folge $(A_i)_{i\in\MdN}$ in $\A$ gelte $\bigcup_{i=1}^\infty A_i \in \A$ und $\bigcap_{i=1}^\infty A_i \in \A$. Enthält $\A$ ferner alle offenen Mengen oder alle abgeschlossenen Mengen, so gilt $\borel(X)\subset \A$.
\end{lemma}

\begin{beweis}
Übungsblatt 2, Aufgabe 2 (a).
\end{beweis}

\begin{satz}
\label{satz:2.2}
Sei $\mu$ ein äußeres Maß auf dem metrischen Raum $(X,d)$ mit $\borel(X)\subset\A_\mu$ und $A\in\A_\mu$. Dann gilt:
\begin{enumerate}
\item Ist $\mu$ Borel-regulär und $\mu(A)<\infty$, so existieren $B_1,B_2\in\borel(X)$ mit $B_2\subset A\subset B_1$ und $\mu(B_1\setminus B_2)=0$.
\item Gilt $\mu(A)<\infty$ und ist $A\in \borel(X)$ oder $\mu$ Borel-regulär, so existiert zu jedem $\ep>0$ eine abgeschlossene Menge $C\subset A$ mit $\mu(A\setminus C)<\ep$. In diesem Fall gilt
\[
\mu(A) = \sup\{\mu(C): C\subset A,\, C\text{ abgeschlossen}\}.
\]
\item Gibt es eine abzählbare, offene Überdeckung $(U_i)_{i\in\MdN}$ von $A$ mit $\mu(U_i)<\infty$ für alle $i\in\MdN$ und ist $A\in\borel(X)$ oder $\mu$ Borel-regulär, so existiert zu jedem $\ep>0$ eine offene Menge $U\supset A$ mit $\mu(U\setminus A)<\ep$. In diesem Fall gilt
\[
\mu(A) = \inf\{\mu(U): U\supset A,\, U\text{ offen}\}.
\]
\end{enumerate}
\end{satz}

\begin{beweis}
Teil (1) und (2): Übungsblatt 2, Aufgabe 2. Zu (3):

Seien $A$ und $(U_i)_{i\in\MdN}$ wie vorausgesetzt. Sei $\ep>0$. Zu jedem $i\in\MdN$ existiert nach (2) eine abgeschlossene Menge $C_i\subset U_i\setminus A$ mit $\mu( (U_i\setminus A)\setminus C_i)< \frac\ep{2^i}$. Setze $U\da \bigcup_{i=1}^\infty (U_i \setminus C_i)$. Dann ist $U$ offen und $A = \bigcup_{i=1}^\infty (A\cap U_i) \subset \bigcup_{i=1}^\infty (U_i\setminus C_i) = U$. Ferner ist
\begin{align*}
\mu(U\setminus A) 
= \mu(\bigcup_{i=1}^\infty (U_i\setminus C_i) \setminus A)
\le \sum_{i=1}^\infty \mu( (U_i\setminus C_i) \setminus A) 
\le \sum_{i=1}^\infty \mu( (U_i\setminus A) \setminus C_i) 
\le \sum_{i=1}^\infty \frac \ep{2^i} = \ep.
\end{align*}
\end{beweis}

\begin{definition}
Seien $X$ eine Menge, $(Y,\mathcal T)$ ein topologischer Raum, $\mu$ ein äußeres Maß auf $X$ und $f\in\mathbb F_\mu(X,Y)$. Man nennt $f$ $\mu$-messbar, falls $f$ $\mu$-messbar bezüglich $\borel(Y)$ ist, das heißt $f^{-1}(\borel(Y))\subset \A_\mu$.
\end{definition}

Für einen metrischen Raum $(X,d)$, für Mengen $A,B\subset X$ und $x\in X$ sei 
$$
d(x,A) \da \inf\{d(x,y): y\in A\},\qquad d(A,B)\da \inf\{d(x,y),x\in A,\, y\in B\}.
$$

\begin{satz}
\label{satz:2.3}
Seien $X$ eine Menge, $(Y,d)$ ein metrischer Raum, $\mu$ ein äußeres Maß auf $X$ sowie $f\in\mathbb F_\mu(X,Y)$. Dann sind äquivalent:
\begin{enumerate}
\item $f$ ist $\mu$-messbar.
\item Für jede Menge $M\subset X$ und alle Mengen $A,B\subset Y$ mit $d(A,B)>0$ gilt
\[
\mu(M) \ge \mu(M\cap f^{-1}(A)) + \mu(M \cap f^{-1}(B)).
\]
\end{enumerate}
\end{satz}

\begin{beweis}
(1)$\implies$(2): Sei $f$ $\mu$-messbar. Seien $M\subset X$, $A,B\subset Y$ mit $d(A,B)>0$. Es gibt $U_A\supset A$, $U_A$ offen mit $U_A\cap B=\emptyset$. Somit gilt $f^{-1}(B) \subset f^{-1}(U_A^c) = f^{-1}(U_A)^c$ und $f^{-1}(A)\subset f^{-1}(U_A)$. Nach Voraussetzung ist $f^{-1}(U_A)\in\A_\mu$. Es folgt
\begin{align*}
\mu(A) \ge \mu(M\cap f^{-1}(U_A)) + \mu (M\cap f^{-1}(U_A)^c)
\ge \mu(M\cap f^{-1}(U_A)) + \mu(M\cap f^{-1}(B)).
\end{align*}

(2)$\implies$(1): Zeige $f^{-1}(C) \in \A_\mu$ für eine beliebige abgeschlossene Menge $C\subset Y$. Für $y\in Y$ sei $g(y) \da d(y,C)$. Ohne Beschränkung der Allgemeinheit sei $C\ne \emptyset$. Für $a,b\in \MdR$ mit $a<b$ gilt 
\[
d(\underbrace{\{g\le a\}}_{\ad A}, \underbrace{\{g\ge b\}}_{\ad B})  \ge b - a > 0.
\]
Für $M\subset X$ erhält man nun
\begin{align*}
\mu(M) \ge \mu(M\cap f^{-1}(A)) + \mu(M\cap f^{-1}(B)) 
= \mu(M \cap \{g\circ f\le a\}) + \mu(M\cap \{g\circ f \ge b\}).
\end{align*}
Da $g\circ f : X\to \MdR$ die Bedingung von Satz \ref{satz:1.8} erfüllt, folgt die Messbarkeit von $g\circ f$, das heißt insbesondere ist $\{g \circ f = 0\}\in \A_\mu$. Ferner ist 
\begin{align*}
f^{-1}(C) = \{x\in X: d(f(x),C)=0\} = \{g\circ f= 0\} \in \A_\mu,
\end{align*}
wobei die Abgeschlossenheit von $C$ verwendet wurde.
\end{beweis}

\begin{satz}[Kriterium von Carathéodory]
\label{satz:2.4}
Für ein äußeres Maß $\mu$ auf $(X,d)$ sind äquivalent:
\begin{enumerate}
\item $\borel(X) \subset \A_\mu$.
\item Für $A,B\subset X$ mit $d(A,B)>0$ gilt $\mu(A\cup B) \ge \mu(A) + \mu(B)$.
\end{enumerate}
\end{satz}

\begin{beweis}
(1)$\implies$(2): Ist $\borel(X)\subset \A_\mu$, so ist $\id_X: X\to X$ $\mu$-messbar. Für Mengen $A,B\subset X$ mit $d(A,B)>0$ folgt nun aus Satz \ref{satz:2.3}, (1)$\implies$(2), dass 
\begin{align*}
\mu(A\cup B) \ge \mu( \underbrace{(A\cup B) \cap \id_X^{-1}(A)}_{=A} ) + \mu( \underbrace{(A\cup B) \cap \id_X^{-1}(B)}_{=B}) = \mu(A) + \mu(B).
\end{align*}

(2)$\implies$(1): Seien $M\subset X$, $A,B\subset X$ mit $d(A,B)>0$. Dann ist auch $d(M\cap A, M\cap B)>0$ und daher
\begin{align*}
\mu(M)
&\ge \mu( M\cap (A\cup B) )\\
&= \mu( (M\cap A)\cup (M\cap B) )\\
&\ge \mu(M\cap A) + \mu(M\cap B)\\
&= \mu(M \cap \id_X^{-1}(A)) + \mu(M \cap \id_X^{-1}(B)).
\end{align*}
Nach Satz \ref{satz:2.3}, (2)$\implies$(1), ergibt dies die $\mu$-Messbarkeit von $\id_X$, das heißt $\borel(X)\subset \A_\mu$.
\end{beweis}

\section{Vitali-Relationen}

In dem metrischen Raum $(X,d)$ sei 
\begin{multline*}
\mathbb M(X) \da \{\mu: \text{$\mu$ ist ein Borel-reguläres Maß auf $X$ und}\\\text{$\mu(A)<\infty$ für alle beschänkten Mengen $A\subset X$}\}.
\end{multline*}

\begin{motivation}
Seien $a,b\in\MdR$, $a<b$ und $f:[a,b]\to \MdR$ monoton wachsend. Für $x\in[a,b)$ sei 
\[
\bar L_f(x) \da \limsup_{h\downto0} \frac{f(x+h) - f(x)}{h}.
\]
Mit $\lambda$ wird das äußere Lebesguemaß auf $\MdR$ bezeichnet, das heißt
\[
\lambda(M) \da\inf\{\sum_{i=1}^\infty (b_i-a_i): a_i < b_i, M\subset \bigcup_{i=1}^\infty [a_i, b_i]\}.
\]
\end{motivation}

\textbf{Behauptung:} Für $\lambda$-fast-alle $x\in[a,b]$ ist $\bar L_f(x)<\infty$.

Zum Nachweis sei
\[
\yen  \da \{[x,x+h]: x\in[a,b],\, 0<h<b-x\}.
\]
Für $t>1$ sei
\[
\yen_t \da \{[x,x+h] \in \yen: f(x+h) - f(x) \ge t\cdot h\}.
\]
Wir wollen zeigen, dass die Menge
\[
A\da \{x\in [a,b): \limsup_{h\downto0} \frac{f(x+h)-f(x)}h = \infty\}
\]
Lebesguemaß Null hat.

Für $x\in A$, $t>1$ gilt:
\[
\inf\{h>0: [x,x+h ] \in \yen_t\} = 0.
\]
Falls es gelingt, eine disjunkte Folge $([x_i, x_i+h_i])_{i\in\MdN}$ in $\yen_t$ zu finden mit
\[
\lambda( A \setminus  \underbrace{\bigcup_{i=1}^\infty [x_i, x_i+h_i]}_{\ad S}) = 0,
\]
dann folgt
\begin{multline*}
\lambda(A) = \lambda(A\cap S) + \lambda(A\cap S^c) = \lambda(A\cap S)\\
\le \lambda(S) = \sum_{i=1}^\infty h_i < \sum_{i=1}^\infty \frac{f(x_i+h_i)- f(x_i)}t \le \frac{f(b)-f(a)}t.
\end{multline*}
Da $t>1$ beliebig ist, folgt $\lambda(A) = 0$.

Wichtig war hierbei:
\begin{enumerate}
\item Die betrachteten Intervalle sind beliebig kurz.
\item Bis auf eine Nullmenge lässt sich eine abzählbare disjunkte Teilüberdeckung finden.
\item Oft genügt es auch, den Grad der Mehrfachüberlappung beschränken zu können.
\end{enumerate}

\begin{definition}
Sei $\mu$ ein äußeres Maß auf $(X,d)$, $M\subset X$, $\A\subset \mathcal P(X)$. Man nennt $\A$ eine (offene, abgeschlossene, Borelsche) $\mu$-Überdeckung von $M$ wenn $\mu(M\setminus \bigcup_{A\in\A} A) = 0$ und $\A$ aus (offenen, abgeschlossenen, Borelschen) Mengen besteht.
\end{definition}

\begin{definition}
Sei $\mu\in \mathbb M(X)$. Eine $\mu$-Vitali-Relation $V$ auf $X$ ist eine Teilmenge des Mengensystems $\{(x,S): x\in S, S\in \borel(X)\}$ mit folgenden Eigenschaften:
\begin{enumerate}
\item $\inf\{\diam(S): (x,S)\in V\}=0$ für alle $x\in X$.
\item Für $Z\subset X$, $C\subset V$ mit $\inf\{\diam(S): (z,S)\in C\}=0$ für alle $z\in Z$ enthält die Familie $C(Z) \da \{S: (z,S)\in C \text{ für ein $z\in Z$}\}$ eine abzählbare disjunkte $\mu$-Überdeckung von $Z$.
\end{enumerate}
\end{definition}

Im Folgenden werden Methoden und Kriterien zum Nachweis von Eigenschaft (2) vorgestellt und  Vitali-Relationen bei der Differentation von Maßen verwendet.

\begin{beispiel}
Seien $X=\MdR$, $V\da \{(x,[x,x+h]): [x,x+h] \in \yen\}$. Dann ist (1) erfüllt. Später wird (2) für $\mu=\lambda$ gezeigt, so dass $V$ eine $\lambda$-Vitali-Relation ist. Wähle $Z = A$ wie oben, $C\da \{(x,[x,x+h]): [x,x+h] \in \yen_t\}$. Dann ist die Voraussetzung für $Z$ aus (2) erfüllt und es existiert eine abzählbare disjunkte $\lambda$-Überdeckung von $Z$, das heißt es existiert $([x_i,x_i+h_i])_{i\in\MdN}$ paarweise disjunkt mit $[x_i,x_i+h_i]\in \yen_t$ und $\lambda(A\setminus\bigcup_{i=1}^\infty [x_i,x_i+h_i])=0$.
\end{beispiel}

Rahmensituationen, in denen (2) verifiziert werden kann:
\begin{enumerate}
\item Der metrische Raum $(X,d)$ ist allgemein, $\mu$ erfüllt eine diametrische Regularitätsbedingung.
\item Der Raum $(X,d)$ ist spezieller, geeignet ist etwa $(\MdR^n,\|\cdot\|)$, $\mu$ ist ein beliebiges Radonmaß.
\end{enumerate}

\begin{satz}
\label{satz:2.5}
Sei $\mathcal S\subset\mathcal P(X)$ mit $\sup\{\diam(\mathcal S): S\in \mathcal S\}<\infty$, $1<t<\infty$ und $\emptyset\notin \mathcal S$. Für $T\in\mathcal S$ sei $\mathcal S_T^t \da\{S\in \mathcal S: S\cap T \ne \emptyset,\, \diam(S) \le t \cdot \diam(T)\}$. Dann existiert eine disjunkte Familie $\mathcal T\subset \mathcal S$ mit $\mathcal S = \bigcup_{T\in\mathcal T} \mathcal S_T^t$.
\end{satz}

\begin{beweis}
Definiere
\begin{multline*}
\Omega \da \{\mathcal R\subset \mathcal S: \mathcal R \text{ ist eine disjunkte Familie von Mengen, für alle $S\in\mathcal S$ gilt:}\\
S\cap \bigcup_{R\in\mathcal R}R = \emptyset \text{ oder }
S\in \mathcal S_R^t \text{ für ein } R\in\mathcal R\}.
\end{multline*}
Dann ist $(\Omega,\subset)$ eine geordnete Menge. Nach Voraussetzung gibt es ein $T\in \mathcal S$ mit $\sup\{\diam(S): S\in \mathcal S\} < t \cdot \diam (T)$. Daher ist $\{T\}\in\Omega$.

\textbf{Behauptung:} Jede linear geordnete Teilmenge $\Lambda \subset \Omega$ hat eine obere Schranke.

\textbf{Nachweis:} Die obere Schranke ist $\bigcup_{\mathcal R\in \Lambda}\mathcal R$. Dies ist eine disjunkte Familie: Wenn $R,R'\in \bigcup_{\mathcal R\in\Lambda}\mathcal R$, so ist $R\in \mathcal R$, $R'\in \mathcal R'$ mit $\mathcal R,\mathcal R'\in\Lambda$. Ohne Beschränkung der Allgemeinheit sei $\mathcal R'\supset \mathcal R$, also $R,R'\in \mathcal R'$. Da aber $\mathcal R'$ eine disjunkte Familie ist, ist $R\cap R'=\emptyset$ oder $R=R'$. Sei $S\in \mathcal S$. Ist nun $S\cap(\bigcup_{\mathcal{R}\in\Lambda} \bigcup_{R\in\mathcal R} R) \ne \emptyset$, so ist für ein $\mathcal R'\in\Lambda$ schon $S\cap \bigcup_{R\in\mathcal R'} R \ne \emptyset$. Wegen $\mathcal R'\in\Omega$ ist dann $S\in\mathcal S_{R'}^t$ für ein $R'\in\mathcal R'$ und damit auch $S\in\mathcal S_R^t$ für ein $R\in\bigcup_{\mathcal R\in\Lambda}\bigcup_{R\in\mathcal R}\mathcal S_R^t$. Folglich ist $\bigcup_{\mathcal R\in\Lambda}\mathcal R\in \Omega$.

Aufgrund des Zornschen Lemmas existiert ein maximales Element $\mathcal T\in\Omega$. Sei
\[
\mathcal K\da \{S\in\mathcal S: S\cap \bigcup_{T\in\mathcal T}T = \emptyset\}.
\]
Angenommen, $K\ne \emptyset$. Dann existiert ein $K\in\mathcal K$ mit $\sup\{\diam(S):S\in\mathcal K\}<t\cdot\diam(K)$. Es ist $\mathcal T\cup\{\mathcal K\}$ eine disjunkte Familie. Ist schließlich $S\in\mathcal S$ und $S\cap \bigcup\{T: T\in\mathcal T\cup\{\mathcal K\}\} \ne \emptyset$. Dann gilt $S\cap \bigcup_{T\in\mathcal T}T \ne \emptyset$, und somit $S\in \mathcal S_T^t$ für ein $T\in\mathcal T$, oder $S\cap K\ne\emptyset$ und $S\in\mathcal K$, $\diam(S)<t\cdot \diam(K)$, so dass $S\in\mathcal S_K^t$. In jedem Fall ist $\mathcal T\cup\{\mathcal K\}\in\Omega$ im Widerspruch zur Maximalität von $\mathcal T$.

Folglich ist $\mathcal K=\emptyset$. Für jedes $S\in\mathcal S$ existiert also ein $T\in\mathcal T$ mit $S\cap T\ne\emptyset$. Da $\mathcal T\in\Omega$, folgt $S\in\mathcal S_T^t$ für ein $T\in\mathcal T$, also ist $\mathcal S=\bigcup_{T\in\mathcal T}\mathcal S_T^t$.
\end{beweis}

Dieser Satz ist ein entscheidentes Hilfsmittel im Beweis des folgenden Satzes.

\begin{satz}
\label{satz:2.6}
Seien $\mu\in\mathbb M(X)$, $A\subset X$, $A\ne \emptyset$ sowie $t>1$ und $r>0$. Sei $\mathcal S$ eine abgeschlossene Überdeckung von $A$ mit
\begin{enumerate}
\item $\sup\{\diam(S): S\in\mathcal S\}<\infty$,
\item $\inf\{\diam(S): S\in\mathcal S,\, x\in S\}=0$ für alle $x\in A$,
\item $\mu(\bigcup_{S\in\mathcal S_B^t} S) < r\cdot \mu(B)$ für alle $B\in\mathcal S$.
\end{enumerate}
Dann gibt es zu jeder offenen Menge $U\subset X$ eine abzählbare, disjunkte $\mu$-Überdeckung $\mathcal C$ von $A\cap U$ mit $\mathcal C\subset\mathcal S$ und $\bigcup_{C\in\mathcal C}C\subset U$.
\end{satz}

\begin{beweis}
Sei zunächst $A$ beschränkt. Sei $U\subset X$ offen. Sei ohne Beschränkung der Allgemeinheit $A\cap U\ne \emptyset$ und $U$ beschränkt. Wir betrachten $\mathcal U \da \{S\in\mathcal S: \emptyset \ne S\subset U\}$. Nach Satz \ref{satz:2.5} existiert eine disjunkte Familie $\mathcal C\subset \mathcal U$ mit der Eigenschaft $\mathcal U = \bigcup_{C\in\mathcal C} \mathcal U_C^t \subset \bigcup_{C\in\mathcal C}\mathcal S_C^t$. Wegen $\mu\in\mathbb M(X)$ ist $\mu(U)<\infty$ und wegen (3) ist $\mu(B) >0$ für $B\in\mathcal C$. Da $\bigcup_{C\in\mathcal C} C \subset \mathcal U$ und $\mathcal C$ disjunkt, folgt die Abzählbarkeit von $\mathcal C$.

Weiterhin gilt
\[
\sum_{C\in\mathcal C} \mu(\bigcup_{C'\in\mathcal S_C^t} C') < \sum_{C\in\mathcal C} r\cdot \mu(C) \le r\cdot \mu(U) <\infty.
\]
Somit gibt es zu jedem $\ep>0$ eine endliche Menge $\mathcal D \subset \mathcal C$ mit
\[
\sum_{C\in\mathcal C\setminus\mathcal D} \mu(\bigcup_{C'\in\mathcal S_C^t} C') < \ep.
\]
Sei nun $x\in A\cap U \setminus \bigcup_{D\in\mathcal D} D$. Da $\bigcup_{D\in\mathcal D}D$ abgeschlossen ist und nach (2) gibt es ein $S\in\mathcal S$ mit $S\cap \bigcup_{D\in\mathcal D}D =\emptyset$, $x\in S\subset U$. Insbesondere ist $S\in \mathcal U \subset \bigcup_{C\in\mathcal C}\mathcal S_C^t$. Dann gilt aber $S\in \bigcup_{C\in\mathcal C\setminus\mathcal D} \mathcal S_C^t$. Nun kann man abschätzen:
\begin{align*}
\mu(A\cap U\setminus \bigcup_{C\in\mathcal C}C)
&\le \mu(A\cap U \setminus\bigcup_{C\in\mathcal D} D) \\
&\le \mu(\bigcup_{C\in\mathcal C \setminus\mathcal D} \bigcup_{C'\in\mathcal S_C^t} C') \\
&\le \sum_{C\in\mathcal C\setminus\mathcal D} \mu(\bigcup_{C'\in\mathcal S_C^t} C') < \ep .
\end{align*}
Da $\ep>0$ beliebig ist folgt $\mu(A\cap U\setminus \bigcup_{C\in\mathcal C}C)=0$.

Sei nun $A$ beliebig. Wähle $a\in A$. Induktiv werden beschränkte Mengen $A_j\subset X$, offene Mengen $W_j\subset X$ und endliche disjunkte Folgen $\mathcal C_j\subset\mathcal S$ abgeschlossener Mengen konstruiert. Setze hierzu $W_0\da U$, $A_0\da \emptyset$, $\mathcal C_0\da \emptyset$ und falls $W_{j-1}$, $A_{j-1}$ und $\mathcal C_{j-1}$ für $j>0$ schon definiert sind, so sei $W_j \da W_{j-1} \setminus \bigcup_{C\in\mathcal C_{j-1}}C$, $A_j \da \{x\in A: d(a,x)\le j\}$. Da $A_j$ beschränkt ist und $W_{j-1}\setminus \bigcup_{C\in\mathcal C_{j-1}}C = W_j$ offen, gibt es eine endliche disjunkte Teilfamilie $\mathcal C_j\subset \mathcal S$ mit $\bigcup_{C\in\mathcal C_j} C \subset W_j$ und $\mu(W_j\cap A_j \setminus \bigcup_{C\in\mathcal C_j} C) <\frac1{2^j}$. Die Mengen $W_j = \mathcal U\setminus \bigcup_{i=1}^{j-1} \bigcup_{C\in\mathcal C_i} C$, $j\in\MdN$, sind absteigend, $(A_j)_{j\in\MdN}$ sind aufsteigend, $\mathcal C\da \bigcup_{j\ge 1}\mathcal C_j$ ist nach Konstruktion eine abzählbare disjunkte Mengenfolge in $\mathcal S$. Ferner ist $U\setminus \bigcup_{C\in\mathcal C}C\subset W_i$ für jedes $i\in\MdN$. Für $k\in\MdN$ beliebig gibt also:
\begin{align*}
\mu(U\cap A \setminus\bigcup_{C\in\mathcal C} C)
&= \mu(\bigcup_{j=1}^\infty ( (U\setminus \bigcup_{C\in\mathcal C}C)\cap A_j )) \\
&= \mu(\bigcup_{j=k}^\infty ( (U\setminus \bigcup_{C\in\mathcal C}C)\cap A_j )) \\
&= \mu(\bigcup_{j=k}^\infty (W_{j+1} \cap A_j) )\\
&\le \mu(\bigcup_{j=k}^\infty (W_j \cap A_j \setminus \bigcup_{C\in\mathcal C_j}C)) \\
&\le \sum_{j=k}^\infty \frac1{2^j}.
\end{align*}
Da $k\in\MdN$ beliebig, folgt $\mu(U\cap A \setminus\bigcup_{C\in\mathcal C} C) = 0$.
\end{beweis}

\begin{beispiel}
Sei $V=\{(x,[x,x+h]): x\in [a,b),\, 0<h<b-x\}$. Dann gilt:
\begin{enumerate}
\item $\inf\{\diam([x,x+h]) : 0 < h < b-x\}=0$ für alle $x\in[a,b)$.
\item Seien $Z\subset [a,b)$, $C\subset V$ mit $\inf\{\diam(S): (z,S) \in C\}=0$ für alle $z\in Z$.

Betrachte $C(Z) \da \{S: (z,S)\in C\text{ für ein }z\in Z\}$.

Wir werden Satz \ref{satz:2.6} verwenden und weisen die folgenden Voraussetzungen nach:
\begin{enumerate}
\item $\sup\{\diam(S): S\in C(Z)\} \le b-a < \infty$.
\item $\inf\{\diam(S): S\in C(Z),\, x\in S\} = \inf\{\diam(S) : (z,S) \in C \text{ für ein $z\in Z$}, x\in S\} \le \inf\{\diam(S): (x,S)\in C\} = 0$ für alle $x\in Z$.
\item Sei $B \da [x,x+h] \in C(Z) \ad \mathcal S$. Dann ist $\mathcal S$ eine abgeschlossene Überdeckung von $Z$. Ferner: $[y,y+\bar h]\in\mathcal S_B^t$ impliziert $[y,y+\bar h] \cap [x,x+h] \ne \emptyset$ und $\diam([y,y+\bar h]) \le t\cdot \diam([x,x+h])$, das heißt $\bar h \le h$, also gilt $[y,y+\bar h] \subset [x-th,x+h+th]$ und damit $\lambda(\bigcup_{S\in\mathcal S_B^t}S) \le \lambda([x-th,(1+t)h]) = (1+2t)h = (1+2t)\lambda(B)$.
\end{enumerate}
Nach Satz \ref{satz:2.6} existiert eine abzählbare disjunkte $\lambda$-Überdeckung von $Z$ mit Mengen aus $\mathcal S = C(Z)$. 
\end{enumerate}
Somit ist $V$ eine $\lambda$-Vitali-Relation.
\end{beispiel}

\begin{definition}
Eine metrische abgeschlossene Kugel in $(X,d)$ ist $\mathbb B(x,r) \da  \{z\in X: d(z,x) \le r\}$ für $r>0$.
\end{definition}

\begin{korollar}
Seien $\mu \in \mathbb M(X)$, $A\subset X$, $r>0$, $t>1$. Sei $\mathcal S$ eine Überdeckung von $A$ durch abgeschlossene Kugeln mit
\begin{enumerate}
\item $\sup\{\diam(S): S\in\mathcal S\} <\infty$,
\item $\inf\{\diam(\mathbb B(y,s)): y\in A, x\in\mathbb B(y,s)\in \mathcal S\} = 0$ für alle $x\in A$,
\item für alle $x\in A$ und alle $\mathbb B(x,s)\in\mathcal S$ gilt: $\mu(\mathbb B(x,(1+2t)s)) < r \cdot \mu(\mathbb B(x,s))$.
\end{enumerate}
Dann gibt es zu jeder offenen Menge $U\subset X$ eine abzählbare disjunkte $\mu$-Überdeckung $\mathcal C\subset \mathcal S$ von $U\cap A$, mit $C\subset U$ für alle $C\in\mathcal C$ (wobei alle $C\in\mathcal C$ ihren Mittelpunkt in $A$ haben).
\end{korollar}

\begin{beweis}
Sei $\mathcal S' \da \{\mathbb B(x,s)\in \mathcal S: x\in A\}$. Dann ist $\mathcal S'$ eine abgeschlossene Überdeckung von $A$ wegen (2). Bedingung (1) in Satz \ref{satz:2.6} folgt aus der Voraussetzung (1) hier, die Bedingung (2) in Satz \ref{satz:2.6} folgt aus der Voraussetzung (2) hier. Wir weisen die Bedingung (3) in Satz \ref{satz:2.6} nach. Hierzu sei $\mathbb B(x,s)\in\mathcal S'$, das heißt $x\in A$.

\textbf{Behauptung:} \[
\bigcup_{S\in\mathcal S'^t_{\mathrlap{\mathbb B(x,s)}}} S \subset \mathbb B(x,(1+2t)s)
\]

\textbf{Nachweis:}
Sei $S=\mathbb B(y,\varrho)\in \mathcal S'^t_{\mathbb B(x,s)}$. Dann ist $\mathbb B(y,\varrho) \cap \mathbb B(x,s) \ne \emptyset$ und außerdem gilt $\diam(\mathbb B(y,\varrho)) \le t \cdot \diam(\mathbb B(x,s))$. Sei $\bar z\in \mathbb B(y,\varrho)$ und $z_0\in\mathbb B(y,\varrho)\cap \mathbb B(x,s)$. Dann gilt 
$$d(\bar z, x)\le d(\bar z, z_0) + d(z_0,x)\le \diam(\mathbb B(y,\varrho)) + s \le t \cdot 2s +s = (1+2t)s.
$$ 
Hiermit:
\[
\mu(\bigcup_{S\in\mathcal S'^t_{\mathrlap{\mathbb B(x,s)}}} S )
\le \mu ( \mathbb B(x,(1+2t)s) 
< r\cdot \mu(\mathbb B(x,s)).
\]
Die Aussage des Korollars folgt nun mit Satz \ref{satz:2.6}.
\end{beweis}

\begin{bemerkungen}
\item Die Bedingung (2) ist insbesondere dann erfüllt, wenn $\inf\{\diam(\mathbb B(s,x) ): \mathbb B(x,s)\in\mathcal S\} = 0$ für alle $x\in A$. Falls jetzt $\mu$ eine diametrische Regularitätsbedingung erfüllt, dann ist $V \da \{(x,\mathbb B(x,s)): x\in X,\, s>0\}$ eine $\mu$-Vitali-Relation.

In Satz \ref{satz:2.8} werden wir einen Überdeckungssatz formulieren, mit dessen Hilfe zum Beispiel für $X=\MdR^n$, $d=\|\cdot\|$, gezeigt werden kann dass $V$ für jedes (!) Maß $\mu\in\mathbb M(\MdR^n)$ eine $\mu$-Vitali-Relation ist.

\item Ist $\mu\in\mathbb M(X)$ und gibt es $0<u<v$, $n\in\MdN$ mit
\[
0 < u < \frac{\mu(\mathbb B(x,s))}{s^n} < v
\]
für alle $a\in X$, $s>0$, so folgt 
\begin{align*}
\mu(\mathbb B(x,(2t+1)s))
&\le s^n \cdot v \cdot (2t+1)^n\\
&= s^n \cdot u\cdot \frac vu \cdot (2t+1)^n \\
&< \mu(\mathbb B(x,s)) \cdot \underbrace{\frac vu \cdot (2t+1)^n}_{\ad r} \\
&= r \cdot \mu(\mathbb B(x,s)).
\end{align*}
\end{bemerkungen}

\begin{satz}[Besicovitch]
\label{satz:2.8}
Zu $n\in\MdN$ existiert ein $N_n\in\MdN$ so dass gilt: Ist $\mathcal F$ eine Familie abgeschlossener Kugeln mit positiven Radien in $\MdR^n$ und $\sup\{\diam(B):B\in\mathcal F\}<\infty$ und ist $A$ die Menge der Mittelpunkte dieser Kugeln, dann gibt es $\mathcal G_1,\ldots,\mathcal G_{N_n}\subset \mathcal F$ derart, dass $\mathcal G_i$ eine abzählbare disjunkte Teilfamilie von $\mathcal F$ ist und
\[
A\subset \bigcup_{i=1}^{N_n} \bigcup_{B\in\mathcal G_i}B.
\]
\end{satz}

\begin{beweis} 
Übung bzw. Literatur (Evans \& Gariepy, Mattila, allgemeiner Federer)
\end{beweis}

\begin{korollar}
\label{kor:2.9}
Seien $\mu$ ein Borel-reguläres Maß auf $\MdR^n$ und $\mathcal F$ eine Familie abgeschlossener Bälle mit positiven Radien in $\MdR^n$. Sei $A$ die Menge der Mittelpunkte dieser Bälle. Sei $\mu(A)<\infty$ und $\inf\{r>0: \mathbb B(a,r)\in\mathcal F\}=0$ für alle $a\in A$. Dann gibt es zu jeder offenen Menge $ U \subset\MdR^n$ eine abzählbare disjunkte Teilfamilie $\mathcal G\subset\mathcal F$, so dass $B\subset U$ für alle $B\in \mathcal G$ und $\mu(A\cap U \setminus \bigcup_{B\in\mathcal G}B)=0$, das heißt, $\mathcal G$ ist eine $\mu$-Überdeckung von $A\cap U$.
\end{korollar}

\begin{beweis}
Wähle $\theta \in (1-\frac1{N_n}, 1)$.

\textbf{Behauptung:} Es gibt $M_1\in\MdN$ und disjunkte Bälle $B_1,\ldots,B_{M_1}\in\mathcal F$ mit $B_i \subset  U$ und 
$$\mu(A\cap U\setminus \bigcup_{i=1}^{M_1} B_i) \le \theta \cdot \mu(A\cap U).
$$


\textbf{Nachweis:} Sei $\mathcal F_1 \da \{B\in\mathcal F : \diam(B) \le 1,\, B\subset U\}$. Die Menge der Mittelpunkte von Bällen aus $\mathcal F_1$ ist gerade $A\cap U$. Wegen Satz \ref{satz:2.8} gibt es zu $\mathcal F_1$ Teilfamilien $\mathcal G_1,\ldots,\mathcal G_{N_n}\subset\mathcal F_1$, wobei $\mathcal G_i$ abzählbar viele disjunkte Bälle enthält mit $A\cap U\subset A\cap U \cap \bigcup_{i=1}^{N_n}\bigcup_{B\in\mathcal G_i}B$. Hieraus folgt
\begin{align*}
\mu(A\cap U)
\le \mu(A\cap U \cap \bigcup_{i=1}^{N_n} \bigcup_{B\in\mathcal G_i} B) 
\le \sum_{i=1}^{N_n}\mu(A\cap U \cap \bigcup_{B\in\mathcal G_i} B).
\end{align*}
Es gibt ein $i\in\{1,\ldots,N_n\}$, so dass
\begin{align*}
\mu(A\cap U \cap \bigcup_{B\in\mathcal G_i} B) \ge \frac1{N_n} \mu(A\cap U).
\end{align*}
Da $\mu$ Borel-regulär ist, gibt es $M_1\in\MdN$ und $B_1,\ldots,B_{M_n}\in \mathcal G_i$ mit
\begin{align*}
\mu(A\cap U \cap \bigcup_{i=1}^{M_1} B_i) 
\ge (1-\theta)\mu(A\cap B).
\end{align*}
Nun folgt
\begin{align*}
\infty > \mu(A\cap U)
&= \mu(A\cap U \cap \bigcup_{i=1}^{M_1} B_i) + \mu(A\cap U \setminus \bigcup_{i=1}^{M_1} B_i) \\
&\ge (1-\theta)\mu(A\cap U) + \mu(A\cap U\setminus \bigcup_{i=1}^{M_1} B_i),
\end{align*}
also die Behauptung.

Im Folgenden wenden wir die bewiesene Behauptung wiederholt an. 
Setze  $U_2 \da U\setminus\bigcup_{i=1}^{M_1} B_i$ sowie $\mathcal F_2\da\{B\in\mathcal F: \diam(B)\le 1,\, B\subset U_2\}$. Wie eben findet man $M_2\in\MdN$, $M_2\ge M_1$ und disjunkte abgeschlossene Bälle $B_{M_1+1},\ldots,B_{M_2}\in\mathcal F_2$ mit
\begin{align*}
\mu(A\cap U \setminus \bigcup_{i=1}^{M_2} B_i) 
&= \mu(A\cap  U_2 \setminus\bigcup_{i=M_1+1}^{M_2} B_i) \\
&\le \theta \mu(A\cap U_2) \\
&= \theta \mu(A\cap U \setminus \bigcup_{i=1}^{M_1} B_i)\\
&\le \theta^2 \mu(A\cap U).
\end{align*}
Man erhält so eine Folge disjunkter Bälle in $\mathcal F$, die in $U$ enthalten ist, so dass
\begin{align*}
\mu(A\cap U \setminus\bigcup_{i=1}^{M_k} B_i) \le \theta^k\cdot \underbrace{\mu(A\cap U)}_{<\infty} \to 0 \text{ für } k\to \infty.
\end{align*}
\end{beweis}

\section{Differentiation von Maßen}

\begin{definition}
Seien $\mu\in\mathbb{M} (X)$, $V$ eine $\mu$-Vitali-Relation auf $X$, $x\in X$. Seien $\mathcal{D}\subset \mathcal P(X)$ und $f:\mathcal D\to \MdR$. Dann wird für jede Teilfamilie $C\subset V$ mit $\inf\{\diam(S): (x,S)\in C\}=0$ erklärt:
\begin{align*}
C\text{-}\limsup_x f &\da C\text{-}\limsup_{S\to x} f(S) \da
\limsup_{\ep \downto 0}\{f(S):(x,S)\in C,\, S\in\mathcal D,\, \diam(S)<\ep\}\\
\intertext{und}
C\text{-}\liminf_x f &\da C\text{-}\liminf_{S\to x} f(S) \da
\liminf_{\ep \downto 0}\{f(S):(x,S)\in C,\, S\in\mathcal D,\, \diam(S)<\ep\}.
\end{align*}
Falls $C\text{-}\limsup_x f = C\text{-}\liminf_x f$ gilt, so wird erklärt:
\begin{align*}
C\text{-}\lim_x f \da C\text{-}\lim_{S\to x} f(S)\da C\text{-}\limsup_x f = C\text{-}\liminf_x f.
\end{align*}
Insbesondere: Seien $\mu,\nu\in\mathbb M(X)$ und $V$ eine $\mu$-Vitali-Relation. Dann wird erklärt:
\begin{align*}
\mathbb D(\nu,\mu,V,x) \da V\text{-}\lim_x \frac\nu\mu,
\end{align*}
falls dieser Grenzwert existiert.
\end{definition}

\begin{satz}
\label{satz:2.10}
Seien $\mu,\nu\in\mathbb{M}(X)$. Für $M\subset X$ sei
\begin{align*}
\nu_\mu(M) \da \inf\{\nu(A) : A\in \borel(X),\, \mu(M\setminus A) =0 \}.
\end{align*}
Dann ist $\nu_\mu\in\mathbb{M} (X)$. Es gibt ein $B\in\borel(X)$ mit $\nu_\mu = \nu\MR B$ und $\mu(B^c) = 0$. Ferner ist $\nu_\mu=\nu$ genau dann, wenn jede $\mu$-Nullmenge auch eine $\nu$-Nullmenge ist.
\end{satz}

\begin{beweis}
Sei $M\subset X$. Wir zeigen zunächst, dass es $A\in\borel(X)$ gibt mit $\mu(M\setminus A)=0$ und $\nu_\mu(M) = \nu(A)$. Hierzu sei $\nu_\mu(M)<\infty$ (sonst wähle $A=X$). Zu $k\in\MdN$ existiert $A_k\in\borel(X)$ mit $\mu(M\setminus A_k)=0$ und
\[
0 \le \nu(A_k) - \nu_\mu(M) \le \frac1k.
\]
Setze $A\da\bigcap_{k=1}^\infty A_k \in \borel(X)$. Es gilt:
\begin{align*}
\mu(M\setminus A) = \mu(\bigcup_{k=1}^\infty (M\setminus A_k)) \le \sum_{k=1}^\infty  \mu(M\setminus A_k) = 0
\end{align*}
und damit
\begin{align*}
\nu_\mu(M) \le \nu(A) \le \nu(A_k) \le \nu_\mu(M) + \frac1k \to \nu_\mu(M)
\end{align*}
für $k\to\infty$.

\textbf{Behauptung:} $\nu_\mu$ ist ein äußeres Maß.

Hierzu ist 
\begin{align*}
0 \le \nu_\mu(\emptyset) \le \nu(\emptyset) = 0.
\end{align*}
Seien $M,M_1,M_2,\ldots \subset X$ mit $M\subset \bigcup_{i=1}^\infty M_i$. Zu jedem $i\in\MdN$ gibt es $A_i \in\borel(X)$ mit $\mu(M_i\setminus A_i) = 0$ und $\nu_\mu(M_i) = \nu(A_i)$. Damit ist
\begin{align*}
\mu(M \setminus\bigcup_{i=1}^\infty A_i) 
&\le \mu( (\bigcup_{i=1}^\infty M_i) \setminus (\bigcup_{i=1}^\infty A_i))\\
&\le \mu( \bigcup_{i=1}^\infty (M_i \setminus A_i) )\\
&\le \sum_{i=1}^\infty \mu(M_i\setminus A_i) = 0.
\end{align*}
Folglich ist
\begin{align*}
\nu_\mu(M)
\le \nu(\bigcup_{i=1}^\infty A_i) 
\le \sum_{i=1}^\infty \nu(A_i) = \sum_{i=1}^\infty \nu_\mu(M_i).
\end{align*}

\textbf{Behauptung:} $\borel(X) \subset \A_{\nu_\mu}$.

Sei $A\in\borel(X)$, $M\subset X$. Zu $M$ existiert $C\in\borel(X)$ mit $\mu(M\setminus C) = 0$ und $\nu_\mu(M) = \nu(C)$. Es gilt nun
\begin{align*}
\mu( (M\cap A) \setminus (C\cap A) )
&= \mu( (M\setminus C) \cap A ) 
\le \mu(M\setminus C) = 0 \\
\intertext{und ebenso}
\mu( (M\cap A^c) \setminus (C\cap A^c) )
&= \mu( (M\setminus C) \cap A^c ) 
\le \mu(M\setminus C) = 0.
\end{align*}
Daher ist
\begin{align*}
\nu_\mu(M) = \nu(C) = \nu(C \cap A) + \nu(C\cap A^c) 
\ge \nu_\mu(M\cap A) + \nu_\mu(M \cap A^c).
\end{align*}
Dies zeigt $A\in \A_{\nu_\mu}$.

Sei $A\in\borel(X)$ beschänkt.  Dann existiert ein $C\in\borel(X)$ mit $C\subset A$, $\mu(A\setminus C) = 0$ und $\nu_\mu(A) = \nu(C)$.

\textbf{Behauptung:} Für $M\subset A$ gilt $\nu_\mu(M) = \nu(M\cap C)$.

Fall 1: $M\subset A$, $M\in\borel(X)$. Es gilt 
\begin{align*}
\mu(M\setminus \underbrace{(C\cap M)}_{\in\borel(X)}) &= \mu( (M\cap A)\setminus (C\cap M)) \le \mu(A\setminus C) = 0
\intertext{und ebenso}
\mu((M^c\cap A)\setminus \underbrace{(C\cap M^c)}_{\in\borel(X)}) &\le  \mu(A\setminus C) = 0.
\end{align*}
Daher  ist $\nu_\mu(M) \le \nu(M\cap C)$ und $\nu_\mu(M^c \cap A) \le \nu(M^c\cap C)$ und weiter
\begin{align*}
\nu_\mu(M) + \nu_\mu(M^c\cap A)
&= \nu_\mu(M\cap A) + \nu_\mu(M^c\cap A) \\
&\ge \nu_\mu(A) \\
&= \nu(C) \\
&= \nu(C\cap M) + \nu(C\cap M^c) \tag{wegen $M\in\borel(X)\subset \A_\nu$} \\
&\ge\nu(M\cap C) + \nu_\mu(M^c\cap A) \\
&\ge \nu_\mu(M) + \nu_\mu(M^c\cap A).
\end{align*}
Es besteht also überall Gleichheit, insbesondere ist
\begin{align*}
\infty > \nu(M\cap C ) + \nu( M^c\cap C) = \nu_\mu(M) + \nu_\mu(M^c\cap A)
\end{align*}
Da der erste (bzw. zweite) Summand links größer gleich dem ersten (bzw. zweiten) Summand rechts ist, folgt insbesondere $\nu_\mu(M) = \nu(M\cap C)$.


Fall 2: $M\subset A$ beliebig. Dann existieren $D_1,D_2\in\borel(X)$ mit $M\subset D_i$ und $\nu(M\cap S) = \nu(D_1\cap S)$ und $\mu(M\cap S) = \mu(D_2\cap S)$ für alle $S\in\borel(X)$, da $\borel(X) \subset \mathcal A_\mu,A_\nu$ (vgl. Proposition \ref{prop1.3} (b)). Setze $D\da D_1\cap D_2\cap A\in\borel(X)$. Es gilt $M\subset D\subset A$. Ferner ist
\begin{align*}
\nu(M\cap S) \le \nu(D\cap S) \le \nu(D_1\cap S) = \nu(M\cap S)
\end{align*}
also $\nu(D\cap S) = \nu(M\cap S)$ und ebenso $\mu(D\cap S) = \mu(M\cap S)$. Insbesondere ist dann auch $\mu(M\setminus S) = \mu(D\setminus S)$. Wegen $D\in\borel(X)$, $D\subset A$ ist $\nu_\mu(D) = \nu(C\cap D)$ nach Fall 1. Zusammen erhält man
\begin{align*}
\nu_\mu(M)
&= \inf \{\nu(S) : S\in\borel(X),\, \mu(M\setminus S) = 0\}\\
&= \inf \{\nu(S) : S\in\borel(X),\, \mu(D\setminus S) = 0\}\\
&= \nu_\mu(D) = \nu(C\cap D) = \nu(M\cap C).
\end{align*}

Wir zerlegen nun $X$ in der Form $X=\bigcup_{i\ge 1}A_i$ mit beschränkten disjunkten Borelmengen $A_i$. 
Zu jedem $i\in\mathbb{N}$ existiert $C_i\in \borel(X)$ mit $C_i\subset A_i$, $\mu(A_i\setminus C_i)=0$ und  $\nu_\mu(M)=\nu(M\cap C_i)$ für alle $M\subset A_i$. Setze 
$$B\da \bigcup_{i\ge 1}C_i\in \borel(X).$$
Dann folgt
$$
\mu(B^c)=\mu(X\setminus B)=\mu((\bigcup_{i\ge 1}A_i) \setminus(\bigcup_{i\ge 1}C_i))\le\mu(\bigcup_{i\ge 1}(A_i\setminus C_i))=0.
$$
Es folgt für $M\subset X$, wobei für die zweiten Gleichung $\borel (X)\subset\mathcal{A}_{\nu_\mu}\subset
\mathcal{A}_{(\nu_\mu\MR M)}$ verwendet wird,
\begin{align*}
\nu_\mu(M)&=(\nu_\mu\MR M)(\bigcup_{i\ge 1}A_i)=\sum_{i\ge 1}(\nu_\mu\MR M)(A_i)\\
&=\sum_{i\ge 1}\nu_\mu(\underbrace{M\cap A_i}_{\subset A_i})=\sum_{i\ge 1}\nu(\underbrace{M\cap A_i\cap C_i}_{=M\cap C_i})\\
&=\sum_{i\ge 1}(\nu\MR M)(C_i)=(\nu\MR M)(\bigcup_{i\ge 1} C_i)=(\nu\MR M)(B)\\
&=\nu(M\cap B).
\end{align*}
Dies zeigt $\nu_\mu=\nu\MR B$.

Zu $M\subset X$ existiert $C\in\borel (X)$ mit $M\cap B\subset C$ und $\nu(M\cap B)=\nu(C)$. Damit ist 
$M\subset B^c\cup C\in \borel(X)$ und
$$
\nu_\mu(M)\le \nu_\mu(B^c\cup C)=\nu(B\cap C)\le\nu(C)=\nu(M\cap B)=\nu(M),
$$
das heißt
$$
\nu_\mu(M)=\nu_\mu(B^c\cup C).
$$
Dies schließt den Nachweis ab, dass $\nu_\mu$ Borel-regulär ist. Ist  $M$ beschränkt, 
so ist $\nu_\mu(M)=\nu(M\cap B)<\infty$. 

Ist jede $\mu$-Nullmenge eine $\nu$-Nullmenge, so ist $\nu(B^c)=0$, 
also $\nu=\nu\MR B=\nu_\mu$. Ist schließlich $M\subset X$ und $\mu(M)=0$, so ist 
$\nu_\mu(M)=\nu(\emptyset)=0$. Aus $\nu_\mu=\nu$ folgt damit $\nu(M)=0$, so dass jede $\mu$-Nullmenge  eine $\nu$-Nullmenge.
\end{beweis}

\begin{bemerkung}
Für \(\mu, \nu \in \mathbb{M}(X)\) schreibt man \(\nu \ll \mu\) und sagt \(\nu\) ist absolut stetig bezüglich \(\mu\), falls für alle \(M \subset X\) gilt: 
\[
\mu(M) = 0 \Rightarrow \nu(M) = 0.
\]
Das Maß \(\nu_\mu\) ist absolut stetig bezüglich \(\mu\) und \(\nu_\mu = \nu\MR B\) mit \(\mu(B^c) = 0,$ $B \in \borel(X)\). Also ist
\[
\nu = \nu_\mu + \underbrace{(\nu - \nu_\mu)}_{=: \nu_\mu^\bot}
\]
wobei \(\nu_\mu^\bot \in \mathbb{M}(X)\) wegen \(\nu_\mu^\bot = \nu\MR B^c\), d.h. \(\nu_\mu^\bot \bot \nu_\mu \). 
Letzteres bedeutet, dass es eine Menge $B\subset X$ gibt mit \(\nu_\mu(B^c) = \nu_\mu^\bot(B) = 0\).
\end{bemerkung}

\begin{lemma}
\label{lem2.11}
Seien \(\mu, \nu, \tau \in \mathbb{M}(X), V\) eine \(\mu-\)Vitali-Relation, \(c \in (0,\infty)\) und \[A \subset \{ x \in X : V\text{-}\liminf_x \frac{\nu(S)}{\tau(S)} < c \}.\]
Dann gilt: 
$$
\nu_\mu(A) \leq c \cdot \tau_\mu(A).
$$
\end{lemma}

\begin{beweis}
Sei \(\varepsilon > 0\). Aufgrund des Beweises von Satz \ref{satz:2.10} gibt es zu $A$ ein \(B \in \borel(X)\) mit \(\mu(A \setminus B)=0\) und \(\tau_\mu(A) = \tau(B)\).
Wegen Satz \ref{satz:2.2} gibt es zu $B$ eine offene Menge \(U \subset X\) mit \(B \subset U\) und \(\tau(U \setminus B) \leq \varepsilon\). Also ist \(\mu(A \setminus U) \leq \mu(A \setminus B)=0\) und
\[
\tau(U) = \tau(U \cap B) + \tau(U \setminus B) \leq \tau(B) + \varepsilon = \tau_\mu(A) + \varepsilon.
\]
Sei 
$$
C \da \{ (x,S) \in V : S \subset U , \frac{\nu(S)}{\tau(S)} < c \}.
$$ 
Für \(z\in A\cap U\) gilt: \(V\text{-}\liminf\frac{\nu(S)}{\mu(S)} < c\), also \(\inf\{\diam(S):(z,S)\in C\}=0\) für alle \(z \in A \cap U\). 

Da \(V\) eine \(\mu-\)Vitali-Relation ist, enthält \(C(A\cap U)\) eine abzählbare, disjunkte \(\mu\)-Überdeckung \(\mathcal{S}\) von \(A\cap U\). Aus \(\mu(A \setminus U)=0\), \(\mu(A\cap U \setminus \bigcup_{S\in \mathcal{S}} S) = 0\), \(\nu_\mu \ll \mu\) und \(\nu_\mu \leq \nu\) folgt
\begin{align*}
\nu_\mu(A) &= \nu_\mu(A\cap U \cap \bigcup_{S\in\mathcal{S}}S) \\
&\leq \nu_\mu(\bigcup_{S\in\mathcal{S}} S) \\
&\leq \nu(\bigcup_{S\in\mathcal{S}} S) \\
&\leq \sum_{S\in\mathcal{S}}\underbrace{\nu(S)}_{\leq c\tau(S)}\\
&\leq c\sum_{S\in\mathcal{S}} \tau(S) = c\cdot \tau(\underbrace{\bigcup_{S\in\mathcal{S}}S}_{\subset U})\\
&\leq c\cdot\tau(U) \\
&\leq c\cdot(\tau_\mu(A)+\varepsilon) \\
&= c\cdot\tau_\mu(A)+c\cdot\varepsilon.
\end{align*}
Da \(c<\infty\) und \(\varepsilon>0\) beliebig war, folgt die Behauptung.
\end{beweis}

\begin{bemerkungen}
	\item Sind \(\nu,\tau \ll \mu\), so gilt \(\nu_\mu=\nu, \tau_\mu=\tau\) und in Lemma \ref{lem2.11} lautet die Folgerung \(\nu(A) \leq c\cdot\tau(A)\).
	\item Es gilt \(\mu_\mu=\mu\).
\end{bemerkungen}

\begin{korollar}
\label{kor:2.12}
Seien \(\mu,\nu \in \mathbb{M}(X)\), \(V\) eine \(\mu-\)Vitali-Relation, \(c>0\). Seien ferner
\[
A\subset \{x\in X:\liminf_x\frac{\nu(S)}{\mu(S)}<c\}
\quad\text{und}\quad
B\subset\{x\in X:\limsup_x\frac{\nu(S)}{\mu(S)}>c\}.
\]
Dann gilt:
\[
\nu_\mu(A) \leq c\cdot\mu(A), \quad \nu_\mu(B) \geq c\cdot\mu(B).
\]
\end{korollar}
\begin{beweis}
Wegen Lemma \ref{lem2.11} gilt \(\nu_\mu(A) \leq c\mu_\mu(A) = c\mu(A)\). Ferner gilt:
\[
V\text{-}\liminf_x\frac{\mu(S)}{\nu(S)} = \frac1{V\text{-}\limsup_x\frac{\nu(S)}{\mu(S)}} < \frac1c
\]
und damit aufgrund von Lemma \ref{lem2.11} 
$\mu(B) = \mu_\mu(B) \leq \frac1c \nu_\mu(B)$, so dass $\nu_\mu(B) \geq c\mu(B)$.
\end{beweis}

\begin{lemma}
\label{lem2.13}
Seien \(\mu,\nu \in \mathbb{M}(X)\) und \(V\) eine \(\mu\)-Vitali-Relation. Dann ist \(\mathbb{D}(\nu,\mu,V,\cdot) \in \mathbb{F}_\mu(X,\bar{\mathbb{R}})\) und \(0 \leq \mathbb{D}(\nu,\mu,V,\cdot) < \infty\) \(\mu-\)fast-überall in \(X\).
\end{lemma}
\begin{beweis}
Seien
\begin{align*}
C & \da \{ (x,S) \in V : \mu(S) = 0 \}, \\
P & \da \{ x \in X : \inf\{\diam S : (x,S) \in C\} = 0 \}, \\
Q & \da \{ x \in X : \limsup_x\frac\nu\mu = \infty \}
\end{align*}
und für \(a,b \in \mathbb{R}\) mit \(a < b\) sei
\begin{align*} 
R(a,b) & \da \{ x \in X : V\text{-}\liminf_x\frac\nu\mu<a<b<V\text{-}\limsup_x\frac\nu\mu \}.
\end{align*}

Da \(V\) eine \(\mu\)-Vitali-Relation ist, enthält \(C(P)\) eine abzählbare, disjunkte \(\mu\)-Überdeckung \(\mathcal{S}\) von \(P\), so dass
\[
0 \leq \mu(P) \leq \mu(P\cap\bigcup_{S\in\mathcal{S}}S) \leq \mu(\bigcup_{S\in\mathcal{S}}S) = \sum_{S\in\mathcal{S}}\underbrace{\mu(S)}_{=0} = 0 ,
\]
das heißt \(\mu(P) = 0\).

Seien nun \(a<b,\; a,b\in\mathbb{Q}\) und \(A \subset Q\), \(B \subset R(a,b)\) beschränkt. 
Für ein beliebiges \(c > 0\) gilt:
\[
\infty > \nu_\mu(A) \geq c\cdot\mu(A),
\]
also $\mu(A) = 0$. 
Ebenso gilt
\begin{align*}
b\cdot\mu(B) \leq \nu_\mu(B) \leq a\cdot\mu(B)
\end{align*}
für $a < b$. Wegen $\mu(B)<\infty$ folgt \(\mu(B)=0\).

Es folgt \(\mu(Q) = \mu(R(a,b)) = 0\) und damit die Behauptung.
\end{beweis}

\begin{lemma}
\label{lem2.14}
Sei \(\mu,\nu \in \mathbb{M}(X)\), \(V\) eine \(\mu\)-Vitali-Relation. Dann ist \(\mathbb{D}(\nu,\mu,V,\cdot)\) eine \(\mu\)-messbare Funktion.
\end{lemma}
\begin{beweis}
Seien \(a<b\),
\[
M_1 \da \{ \mathbb{D}(\nu,\mu,V,\cdot) < a \} \quad\text{und}\quad  M_2 \da \{ \mathbb{D}(\nu,\mu,V,\cdot) > b \}.
\]
Seien weiter \(A_i \subset M_i\) beschränkt (\(i=1,2\)).
Es existieren \(B_i \in \borel(X)\) mit \(A_i \subset B_i\), so dass für alle $ C\in\borel(X)$ gilt
\begin{align*}
 \mu(A_i\cap C) = \mu(B_i \cap C) \quad\text{und}\quad \nu_\mu(A_i\cap C) = \nu_\mu(B_i \cap C)\tag{$\ast$}.
\end{align*}
Wegen \(a<b\) gilt mit Korollar \ref{kor:2.12} und \((\ast)\)
\begin{align*}
a \cdot \mu(B_1 \cap B_2)
&= a\cdot\mu(A_1 \cap B_2) \\
&\geq \nu_\mu(A_1 \cap B_2) \\
&= \nu_\mu(B_1 \cap B_2) \\
&= \nu_\mu(B_1 \cap A_2) \\
&\geq b\cdot\mu(B_1 \cap A_2) \\
&= b\cdot\underbrace{\mu(B_1\cap B_2)}_{<\infty}
\end{align*}
und damit $\mu(B_1 \cap B_2) = 0$.

Nun folgt:
\begin{align*}
\mu(A_1 \cap A_2) &= \mu( \underbrace{(A_1 \cup A_2)}_{\supset A_1} \cap B_1 ) + \mu( \underbrace{(A_1\cup A_2)\cap B_1^c}_{=A_2\cap B_1^c} ) \\
&\geq \mu(A_1) + \mu(A_2 \cap B_1^c) \\ &= \mu(A_1) + \mu(A_2 \cap B_1^c \cap B_2) \\
&= \mu(A_1) + \mu(\underbrace{A_2}_{\subset B_2} \cap B_2) \\
&= \mu(A_1) + \mu(A_2).
\end{align*}
Sei \(T \subset X\) beliebig, \(y \in X\). Für $i=1,2$ setze
\[
A_{i,n} \da M_i \cap T \cap B(y,n) .
\]
Es ist \(A_{i,n} \subset M_i\), \(A_{i,n}\nearrow M_i \cap T\) und \(A_{1,n} \cup A_{2,n} \subset T\). Somit 
erhält man
\begin{align*}
\mu(T) &\geq \lim_{n\rightarrow\infty} \mu(A_{1,n} \cup A_{2,n})\\
&\geq \lim_{n\rightarrow\infty} \mu(A_{1,n}) + \lim_{n\rightarrow\infty} \mu(A_{2,n}) \\
&= \mu(M_1 \cap T) + \mu(M_2 \cap T).
\end{align*}
Die Behauptung folgt nun aus Satz \ref{satz:1.8}.
\end{beweis}

\begin{satz}
\label{satz:2.15}
Seien $\nu,\mu\in\mathbb M(X)$ und $V$ eine $\mu$-Vitali-Relation. Dann gilt $\A_\mu \subset \A_{\nu_\mu}$ und für $A\in\A_\mu$ ist
\[
\nu_\mu(A) = \int_A \mathbb D(\nu,\mu,V,x)\mu(dx).
\]
\end{satz}

\begin{beweis}
Sei $A\in\A_\mu$ beschränkt. Dann gibt es eine Borelmenge $B\in\borel(X)$ mit $A\subset B$ und $\mu(A) = \mu(B)$, also $\mu(B\setminus A)=0$.  Dann ist auch $\nu_\mu(B\setminus A) = 0$, das heißt $B\setminus  A \in \mathcal \A_{\nu_\mu}$ und folglich $A = B \setminus (B\setminus A) \in \A_{\nu_\mu}$, da $\borel(X)\subset\A_{\nu_\mu}$ (vgl.
Satz \ref{satz:2.10}). Wegen $A = \bigcup_{i=1}^\infty A_i$ mit beschränkten $A_i \in \A_\mu$ folgt allgemein $A\in\A_{\nu_\mu}$, das heißt $A_\mu\subset\A_{\nu_\mu}$.

Sei nun $Z_0 \da \{ \mathbb D(\nu,\mu,V,\cdot) = 0 \}$, $Z_\infty \da \{ \mathbb D(\nu,\mu,V,\cdot) = \infty \}$. Da aufgrund von Lemma \ref{lem2.13} $\mu(Z_\infty)  = 0$ ist, ist $\nu_\mu(Z_\infty) = 0$ und daher ist
\[
\nu_\mu(Z_\infty) = 0 = \int_{Z_\infty} \mathbb D(\nu,\mu,V,x) \mu(dx).
\]

Ferner ist $Z_0 = \bigcup_{n=1}^\infty Z_n$ mit $Z_n$ beschränkt. Aus Korollar \ref{kor:2.12} erhält man für $\ep>0$ die Abschätzung $\nu_\mu(Z_n) \le \ep \cdot \underbrace{\mu(Z_n)}_{<\infty}$, das heißt $\nu_\mu(Z_n)=0$, also auch $\nu_\mu(Z_0)=0$. Daher ist
\[
\nu_\mu(Z_0) = 0 = \int_{Z_0} \underbrace{\mathbb D(\nu,\mu,V,x)}_{=0} \mu(dx).
\]

Sei schließlich $t\in(1,\infty)$, $n\in\MdZ$ und $P_n^t \da \{t^n \le \mathbb D(\nu,\mu,V,\cdot) < t^{n+1}\}$. Man erhält
\[
A\setminus(Z_0\cup Z_\infty) = \bigcup_{n\in\MdZ} \underbrace{(P_n^t \cap A)}_{\in\A_\mu}
\]
mit disjunkter Vereinigung und daher 
\[
\nu_\mu(A) = \sum_{n\in\MdZ}\nu_\mu(P_n^t \cap A).
\]
Andererseits ist wegen $\mu(Z_\infty)=0$
\begin{align*}
t^{-1} \int_A \mathbb D(\nu,\mu,V,x)\mu(dx)
&= \sum_{n\in\MdZ} t^{-1} \int_{P_n^t\cap A} \underbrace{\mathbb D(\nu,\mu,V,x)}_{<t^{n+1}}\mu(dx) \\
&\le \sum_{n\in\MdZ} t^n \cdot \mu(P_n^t \cap A) \\
&\le \sum_{n\in\MdZ} \nu_\mu(P_n^t \cap A) = \mu_\nu(A) \\
&\le \sum_{n\in\MdZ} t^{n+1} \mu(P_n^t \cap A) \\
&\le \sum_{n\in\MdZ} t \int_{P_n^t \cap A} \mathbb D(\nu,\mu,V,x)\mu(dx) \\
&= t \cdot \int_A \mathbb D(\nu,\mu,V,x)\mu(dx).
\end{align*}
Da $t>1$ beliebig war, folgt die Behauptung.
\end{beweis}

\begin{satz}
\label{satz:2.16}
Sind $\mu,\nu\in\mathbb M(X)$ und $V_1,V_2$ zwei beliebige $\mu$-Vitali-Relationen auf $X$, dann gilt
\[
\mathbb D(\nu,\mu,V_1,\cdot) = \mathbb D(\nu,\mu,V_2,\cdot)
\]
$\mu$-fast-überall.
\end{satz}

\begin{beweis}
Sei $y\in X$ fest. Für $i,j\in\{1,2\}$, $i\ne j$, sei
\[
A_{ij}^n \da \{x\in \mathbb B(y,n): \mathbb D(\nu,\mu,V_i,x) \ge \mathbb D(\nu,\mu,V_j,x) + \frac1n\}.
\]
Hiermit ist
\[
\frac 1n\cdot \mu(A_{ij}^n) \le \int_{A_{ij}^n}(\mathbb D(\nu,\mu,V_i,x) - \mathbb D(\nu,\mu,V_j,x)) \mu(dx) = \nu_\mu(A_{ij}^n) - \nu_\mu(A_{ij}^n) = 0.
\]
Dies zeigt $\mu(A_{ij}^n)=0$ für alle $n\in\MdN$. Die Behauptung folgt nun aus 
\[
\mu(\mathbb D(\nu,\mu,V_1,\cdot) \ne \mathbb D(\nu,\mu,V_2,\cdot)) =  \mu(\bigcup_{n\in\MdN} A_{12}^n \cup \bigcup_{n\in\MdN} A_{21}^n) = 0.
\]
\end{beweis}

\begin{satz}[Lebesguescher Dichtesatz]
\label{satz:2.17}
Seien $\mu\in\mathbb M(X)$, $V$ eine $\mu$-Vitali-Relationen auf $X$ , $f:X\to\bar\MdR$ sei $\mu$-messbar und $\int_A |f|d\mu <\infty$ für jede beschränkte Menge $A\in\A_\mu$. Dann gilt:
\[
V\text-\lim_{S\to x} \frac1{\mu(S)} \cdot \int_S fd\mu = f(x)
\]
für $\mu$-fast-alle $x\in X$.
\end{satz}

\begin{beweis}
Sei zunächst $f\ge 0$. Sei $A\subset X$, $A\in \mathcal{A}_\mu$ und $\bar\nu(A) \da \int_A fd\mu$. Dann ist $\bar\nu$ ein Maß auf der $\sigma$-Algebra $\mathcal{A}_\mu$. Setze
\[
\nu(M) \da \inf\{\bar\nu(A):  A\in \A_\mu,\, M\subset A \}.
\]
Dann ist $\nu$ ein $\A_\mu$-reguläres äußeres Maß (vgl. Satz \ref{satz:1.4}). Ferner ist $\borel(X) \subset \A_\mu \subset A_\nu$. Ist $M\subset X$, so gibt es ein $A\in \A_\mu$ mit $M\subset A$ und $\nu(M)=\nu(A)$. Zu $A\in \A_\mu$ existiert $B\in\borel(X)$ mit $A\subset B$ und $\mu(B\setminus A)=0$ (vgl.\ Satz \ref{satz:2.2} (3)). Daraus folgt
\[
\nu(M) = \nu(A) = \int_Afd\mu = \int_A fd\mu + \int_{B\setminus A} fd\mu = \int_B fd\mu = \nu(B),
\]
da $B\in\borel(X)$ mit $M\subset A \subset B$. Also ist $\nu\in\mathbb M(X)$, denn für ein beschränktes $A\in\A_\mu$ ist $\nu(A) = \int_A fd\mu < \infty$ nach Voraussetzung.

Offenbar ist $\nu \ll \mu$ und daher $\nu_\mu = \nu$. Wegen $A\in\A_\mu$ ist aufgrund von Satz \ref{satz:2.15}
\[
\int_A fd\mu = \nu(A) = \nu_\mu(A) = \int_A \mathbb D(\nu,\mu,V,x)\mu(dx)
\]
für alle $A\in\A_\mu$. Dies zeigt $f = \mathbb D(\nu,\mu,V,\cdot)$ $\mu$-fast-überall, das heißt, für $\mu$-fast-alle $x\in X$ gilt
\[
f(x) = \mathbb D(\nu,\mu,V,x) = V\text-\lim_{S\to x} \frac{\nu(S)}{\mu(S)} =  V\text-\lim_{S\to x} \frac1{\mu(S)} \int_S fd\mu,
\]
wobei $S\in\borel(X)\subset\mathcal{A}_\mu$ benutzt wurde.

Für die allgemeine Aussage wird die Zerlegung $f=f^+-f^-$ verwendet.
\end{beweis}

\begin{definition}
Seien $\mu\in\mathbb M(X)$, $V$ eine $\mu$-Vitali-Relation, $M\subset X$ und $x\in x$. Dann nennt man
\[
V\text-\lim_{S\to x}\frac{\mu(M\cap S)}{\mu(S)} = V\text-\lim_{S\to x} \frac{(\mu\MR M)(S)}{\mu(S)} = \mathbb D(\mu\MR M, \mu, V, x)
\]
die $(\mu,V)$-Dichte von $M$ in $x$, falls der Limes existiert.
\end{definition}

\begin{satz}
\label{satz:2.18}
Seien $\mu\in\mathbb M(X)$, $V$ eine $\mu$-Vitali-Relation, $M\subset X$. Dann existiert die $(\mu,V)$-Dichte 
von $M$ für $\mu$-fast-alle $x\in X$. Setze
\[
P \da \{x\in X: V\text-\lim_{S\to x}\frac{\mu(S\cap M)}{\mu(S)} = 1\} \quad\text{und}\quad
Q \da \{x\in X: V\text-\lim_{S\to x}\frac{\mu(S\cap  M)}{\mu(S)} = 0\}.
\]
Dann gilt $P,Q\in\A_\mu$ und $\mu(M\cap P^c) = \mu(Q\cap M) =0$.

Ferner sind äquivalent:
\begin{enumerate}
\item $M\in \A_\mu$,
\item $\mu(P\cap M^c) = 0$,
\item $\mu(M^c\cap Q^c) = 0$.
\end{enumerate}
\end{satz}

\begin{beweis}
Zu $M\subset X$ existiert $A\in\A_\mu$ mit $M\subset A$ und $\mu(M\cap B) = \mu(A\cap B)$ für alle $B\in\A_\mu$ 
(vgl. Aufgabe 2, Übungsblatt 1). Es gilt $\mu\MR A\in\mathbb M(X)$ und daher ist $\{\mathbb D(\mu\MR A,\mu,V,\cdot)= 1\} \in \A_\mu$. Ferner existiert wegen $S\in\borel(X)$ der Limes
\[
\mathbb D(\mu\MR A, \mu, V, x) = V\text-\lim_{S\to x} \frac{(\mu\MR A)(S)}{\mu(S)} = V\text-\lim_{S\to x}\frac{\mu(M\cap S)}{\mu(S)} 
\]
für $\mu$-fast-alle $x\in X$. Dies zeigt insbesondere $P,Q\in\A_\mu$.

Für $\mu$-fast-alle $x\in X$ gilt wegen Satz \ref{satz:2.17} 
\[
\ind_A(x)  = V\text-\lim_{S\to x} \frac1{\mu(S)} \int_S \ind_A(x) \mu(dx) = V\text-\lim_{S\to x} \frac{\mu(A\cap S)}{\mu(S)} = \mathbb D(\mu\MR A,\mu,V,x).
\]
Sei nun $x\notin P$. Dann ist $x\notin A$ für $\mu$-fast-alle $x\in X$, das heißt $\mu(A\setminus P) = 0$. In  analoger Weise sieht man $\mu(P\setminus A)=0$ ein. Wegen $M\subset A$ ist $\mu(M\setminus P) \le \mu(A\setminus P) =0$. Hieraus folgt $\mu(M\cap P^c)=0$ und damit $M\setminus P\in\A_\mu$. 

Ist $M\in\A_\mu$, so wähle $A=M$, das heißt $\mu(P\setminus M)=0$. 

Ist  dagegen $\mu(P\setminus M)=0$, so ist $P\setminus M\in\A_\mu$ und $M=(M\setminus P) \cup (M\cap P) = (M\setminus P) \cup (P\setminus (P\setminus M)) \in \A_\mu$. 

Die Argumentation für $Q$ verläuft analog.
\end{beweis}

\section{Hausdorffmaße und Hausdorffdimension}

\begin{definition}
Sei \((X,d)\) ein metrischer Raum. Wir definieren für \(\delta > 0\) und \(M \subset X\)
\begin{align*}
& \Omega_\delta(M) \da \{ (A_i)_{i\in\MdN} : A_i \subset X, \diam(A_i) \leq \delta, i\in \MdN, M \subset \bigcup_{i\geq 1} A_i \}, \\
& \tilde\Omega_\delta(M) \da \{ (A_i)_{i\in\MdN} : X \supset A_i \text{ offen}, \diam(A_i) \leq \delta, i\in \MdN, M \subset \bigcup_{i\geq 1} A_i \} \\
\intertext{und}
& \bar\Omega_\delta(M) \da \{ (A_i)_{i\in\MdN} : X \supset A_i \text{ abgeschlossen}, \diam(A_i) \leq \delta, i\in \MdN, M \subset \bigcup_{i\geq 1} A_i \},
\end{align*}
wobei \(\diam(\emptyset) \da 0\).
% Analog \(\tilde\Omega_\delta(M)\) und \(\bar\Omega_\delta(M)\), wobei \(A_i\) offen bzw. abgeschlossen gewählt werden.
\end{definition}

\begin{lemma}
\label{lem:2.19}

Sei \(s\in [0,\infty)\). Für \(\delta>0\) und \(M \subset X\) sei
\begin{align*}
\mu_\delta^s (M) \da \inf\{ \sum_{A \in \mathcal{A}}{(\diam(A))}^s : \mathcal{A} \in \Omega_\delta(M) \}, \\
\tilde\mu_\delta^s (M) \da \inf\{ \sum_{A \in \mathcal{A}}{(\diam(A))}^s : \mathcal{A} \in \tilde\Omega_\delta(M) \} \\
\intertext{und}
\bar\mu_\delta^s (M) \da \inf\{ \sum_{A \in \mathcal{A}}{(\diam(A))}^s : \mathcal{A} \in \bar\Omega_\delta(M) \}.
\end{align*}
Dann gilt:
\begin{enumerate}[(1)]
\item \(\mu_\delta^s\), \(\tilde\mu_\delta^s\), \(\bar\mu_\delta^s\) sind äußere Maße auf \(X\),
\item \(\tilde\mu_\varepsilon^s(M) \leq \bar\mu_\delta^s(M) = \mu_\delta^s(M) \leq \tilde\mu_\delta^s(M)\) für alle \(0 < \delta < \varepsilon\).
\end{enumerate}

\end{lemma}
Beweis: Übung.

\begin{satz}
\label{satz:2.20}

Sei \(s \in [0,\infty)\). Für \(M \subset X\) wird durch
\[
\mu^s(M) \da \sup\{\mu_\delta^s(M) : \delta > 0\}
\]
ein Borel-reguläres, äußeres Maß auf \(X\)  erklärt, das \(s\)-dimensionale Hausdorffmaß auf \((X,d)\). Es gilt:
\begin{enumerate}[(1)]
\item \(\mu^s(M) = \sup\{\bar\mu_\delta^s(M) : \delta > 0\} = \sup\{\tilde\mu_\delta^s(M) : \delta > 0\}\).
\item \(\mu^s(M) = \lim_{\delta \downto 0} \mu_\delta^s(M) = \lim_{\delta\downto0} \bar\mu_\delta^s(M) = \lim_{\delta\downto0} \tilde\mu_\delta^s(M)\).
\end{enumerate}

\end{satz}

\begin{beweis}
Die Aussagen (1), (2) sind klar. Ebenso ist offenbar \(\mu^s(\emptyset) = 0\). Sei \(A \subset \bigcup_{i\geq1} A_i,\;A,A_i\subset X\). Dann ist für \(\delta>0\):
\[
\mu_\delta^s(A) \leq \sum_{i\geq1} \mu_\delta^s(A_i) \leq \sum_{i\geq1} \mu^s(A_i).
\]
Hieraus folgt aber
\[
\mu^s(A) \leq \sum_{i\geq1} \mu^s(A_i).
\]
Wir zeigen \(\borel(X) \subset \mathcal{A}_{\mu^s}\) mit Hilfe von Satz \ref{satz:2.4}. Hierzu seien \(A,B \subset X\) mit $d(A,B)>0$ und o.B.d.A. sei \(\mu(A \cup B) < \infty\). Wir setzen \(\delta \da \frac12 d(A,B) > 0\). Sei \((M_i)_{i\in\MdN} \in \Omega_\delta(M)\). Es folgt
\[
\underbrace{\{ M_i : M_i \cap A \neq \emptyset \}}_{\in \Omega_\delta(A)}  \; \cap \; \underbrace{\{ M_i : M_i \cap B \neq \emptyset \}}_{\in \Omega_\delta(B)} = \emptyset,
\]
und daher
\[
\sum_{i\geq1} {(\diam(M_i))}^s \geq \sum_{\mathclap{\substack{i\geq1\\ M_i\cap A \neq\emptyset}}} {(\diam(M_i))}^s + \sum_{\mathclap{\substack{i\geq1\\ M_i \cap B \neq \emptyset}}} {(\diam(A_i))}^s \geq \mu_\delta^s(A) + \mu_\delta^s(B).
\]
Das zeigt
\[
\mu^s(A \cup B) = \lim_{\delta\downto0}\mu_\delta^s(A\cup B) \geq \lim_{\delta\downto0}\mu_\delta^s(A) + \lim_{\delta\downto0}\mu_\delta^s(B) = \mu^s(A)+\mu^s(B).
\]
Sei jetzt \(M \subset X\) und o.B.d.A. \(\mu^s(M)<\infty\). Zu jedem \(n\in\MdN\) existiert \(\mathcal{C}^n \in \tilde\Omega_{\frac1n}(M)\) mit 
$$
\sum_{C \in \mathcal{C}^n} {(\diam(C))}^s \leq \tilde\mu_{\frac1n}^s(M)+\frac1n.
$$
Setze
\[
B \da \bigcap_{n\in\MdN}\bigcup_{C\in\mathcal{C}^n}C.
\]
Dann ist \(M \subset B \in \borel(X)\) und 
$$
\tilde\mu^s_{\frac1n}(B) \leq \tilde\mu^s_{\frac1n}(\bigcup_{C\in\mathcal C^n}C) \leq \sum_{C\in\mathcal C^n} {(\diam(C))}^s \leq \tilde\mu^s_{\frac1n}(M)+\frac1n,\qquad  n\in\MdN.
$$
Insgesamt erhält man
$$
\mu^s(M) \leq \mu^s(B) = \lim_{n\rightarrow\infty} \tilde\mu^s_{\frac1n}(B) \leq \lim_{n\rightarrow\infty}(\tilde\mu^s_{\frac1n}(M)+\frac1n) = 
\lim_{n\rightarrow\infty} \tilde\mu^s_{\frac1n}(M) + 0 = \mu^s(M),
$$
d.h. \(\mu^s(M) = \mu^s(B)\), was die Borel-Regularität ergibt.
\end{beweis}

\begin{proposition}
\label{prop:2.21}
Zu \(M \subset X\) gibt es genau ein \(s\in[0,\infty]\) mit
\[
\mu^p(M) = \begin{cases} 0, & p>s, \\ \infty, & p<s. \end{cases}
\]
Man nennt diese Zahl \(s\) die Hausdorffdimension von \(M\), \(\dim_H(M)\). Also:
\[
\dim_H(M) \da \inf\{p\geq0 : \mu^p(M)=0\}.
\]

\end{proposition}

\begin{beweis}
Seien \(p,q \in [0,\infty)\) mit \(p<q\). Sei \(\mu^p(M)<\infty\). Dann existiert zu \(\delta>0\) eine Folge \((A_i)_{i\in\MdN} \in \Omega_\delta(M)\) mit \(\sum_{i\geq1} {(\diam(A_i))}^p \leq \mu_\delta^p(M)+1 \leq \mu^p(M)+1\). Hieraus folgt
\begin{align*}
\mu_\delta^q(M) &
\leq \sum_{i\geq1} {(\diam(A_i))}^q \\ &
= \sum_{i\geq1} (\diam(A_i))^p \cdot (\underbrace{\diam(A_i)}_{\leq\delta})^{q-p} \\ &
\leq (\sum_{i\geq1} (\diam(A_i))^p)\cdot \delta^{q-p} \\ &
\leq \delta^{q-p}\cdot (\underbrace{\mu^p(M)+1}_{<\infty}),
\end{align*}
und damit
\[
0 \leq \mu^p(M) = \lim_{\delta\downto0} \mu_\delta^q(M) \leq \lim_{\delta\downto0}(\delta^{q-p})\cdot(\mu^p(M)+1) = 0.
\]
Sei nun \(s \da \inf\{p\geq0 : \mu^p(M) = 0\}\). Ist \(p>s\), so gibt es ein \(q\in(s,p)\) mit \(\mu^q(M) = 0\) und daher \(\mu^p(M)=0\).
\par
Sei jetzt \(p<s\). Wäre \(\mu^p(M)<\infty\), so würde \(\mu^q(M)=0\) für alle \(q>p\) gelten, im Widerspruch zur Definition von \(s\). Also ist \(\mu^p(M)=\infty\) für \(p<s\).
\end{beweis}

\begin{beispiel}
$$
\mu^0(M) = \begin{cases} \#M, & M \text{ endlich}, \\ \infty, & \text{sonst}. \end{cases}
$$
\par
und \(\dim_H(M) = 0\) für \(\#M < \infty\).
\end{beispiel}

\begin{definition}
Seien \((X,d)\) und \((\bar X,\bar d)\) metrische Räume. Eine Abbildung \(f: X \rightarrow \bar X\) heißt Isometrie, falls \(\bar d(f(x),f(y)) = d(x,y)\) für alle $ x,y\in X$. 
\par
Zu \(f:X \rightarrow \bar X\) wird
\[
\Lip(f) \da \sup\left\{\frac{\bar d(f(x),f(y))}{d(x,y)} : x,y \in X, x\neq y\right\}
\]
erklärt. Ist \(\Lip(f)<\infty\), so nennt man \(f\) eine Lipschitzfunktion und \(\Lip(f)\) die Lipschitz-Konstante von \(f\).
\end{definition}

\begin{lemma}
\label{lem:2.22}

Ist \(f: (X,d) \rightarrow (\bar X, \bar d)\) eine Lipschitzfunktion, 
so gilt \(\bar\mu^s(f(M)) \leq \Lip(f)^s \cdot \mu^s(M)\) und \(\dim_H(f(M)) \leq \dim_H(M)\).

\end{lemma}

\begin{beweis}
Sei \((A_i)_{i\in\MdN} \in \Omega_\delta(M)\). Dann gilt \(f(M) \subset \bigcup_{i\geq1} f(A_i)\) und
\begin{align*}
\diam(f(A_i)) & = \sup\{\bar d(f(x),f(y)) : x,y \in A_i\} \\
& \leq \sup \{ \Lip(f) \cdot d(x,y) : x,y \in A_i \} \\
& = \Lip(f) \cdot \diam(A_i) \leq \Lip(f) \cdot \delta,
\end{align*}
d.h. \((f(A_i))_{i\in\MdN} \in \Omega_{\Lip(f)\cdot\delta}(f(M))\) und
$$
\mu_{\Lip(f)\cdot\delta}^s(f(M))  \leq \sum_{i\geq1}(\diam(f(A_i)))^s 
 \leq \Lip(f)^s \cdot \sum_{i\geq1} (\diam(A_i))^s.
$$
Dies zeigt
\[
\mu_{\Lip(f)\cdot\delta}^s(f(M)) \leq \Lip(f)^s\cdot \mu_\delta^s(M).
\]
Aus \(\delta\downto0\) folgt die Behauptung.
\end{beweis}

Sei nun \(K(X) \da \{f:X\rightarrow X: \Lip(f)<1\}\). Für \(m \geq 2\) induziert \(\Psi \da (\psi_1,\dots,\psi_m) \in K(X)^m\) eine Abbildung
\[
\Psi^\ast: \mathcal P(X) \rightarrow \mathcal P(X), \quad M \mapsto \bigcup_{i=1}^m \psi_i(M).
\]
Man erklärt \(\Psi^{\ast k} \da \underbrace{\Psi^\ast \circ \dots \circ \Psi^\ast}_{k \text{ mal}}\) für  $k\in\MdN$. Offenbar gilt \(M \subset M' \Rightarrow \Psi^{\ast k}(M) \subset \Psi^{\ast k}(M')\).
\par
Ist \(A_i \subset X, \; i\in\MdN\), so sei
\[
\limsup_{i\rightarrow\infty} A_i \da \{x \in X : x \text{ liegt in unendlich vielen } A_i\} = \bigcap_{n\in\MdN}\bigcup_{i\geq n} A_i.
\]

\begin{lemma}
\label{lem:2.23}
Seien \(\Psi = (\psi_1,\dots,\psi_m) \in K(X)^m\) und \(\emptyset \neq C \subset X\) kompakt mit \(\Psi^\ast(C) \subset C\). Dann gilt für eine beliebige beschränkte Menge \(M \subset X\) die Inklusion
\[
\limsup_{k\rightarrow\infty} \Psi^{\ast k}(M) \subset C.
\]
\end{lemma}

\begin{beweis}
Sei \(M \subset X\) beschränkt. Sei \(y \in \limsup_{k\rightarrow\infty} \Psi^{\ast k}(M)\), d.h. \(y \in \bigcap_{n\geq1}\bigcup_{k\geq n} \Psi^{\ast k} (M)\). Zu jedem \(n\in\MdN\) existiert also ein \(k\geq n\) mit \(y\in\Psi^{\ast k}(M)\), d.h. \(\exists y_k \in M\) und \(i_1,\dots,i_k \in \{1,\dots,m\}\) mit \(y = \psi_{i_k}\circ\dots\circ\psi_{i_1}(y_k)\). 
\par
Sei \(z \in C\) beliebig. Dann gilt \(\psi_{i_k}\circ\dots\circ\psi_{i_1}(z) \in C\) sowie
\begin{align*}
d(y,C) & \leq d(y,\psi_{i_k}\circ\dots\circ\psi_{i_1}(z)) \\
& \leq \Lip(\psi_{i_k}) \cdot d(\psi_{i_{k-1}}\circ\dots\circ\psi_{i_1}(y_k), \psi_{i_{k-1}}\circ\dots\circ\psi_{i_1}(z)) \\
& \leq \Lip(y_{i_k}) \cdots \Lip(y_{i_1}) \cdot d(y_k, z) \\
& \leq {\underbrace{\max\{\Lip(\psi_i): 1 \leq i \leq m\}}_{\ad r < 1}}^k \cdot d(y_k,z) \\
& \le r^k \cdot \underbrace{\diam(M\cup C)}_{<\infty}.
\end{align*}
Mit \(n\rightarrow\infty \) (und damit \( k\rightarrow\infty\)) folgt \(d(y,C)=0\), d.h. \(y\in C\), da \(C\) abgeschlossen ist.
\end{beweis}

\begin{satz}
\label{satz:2.24}
Sei $(X,d)$ ein vollständiger metrischer Raum und $\Psi=(\psi_1,\ldots,\psi_m)\in K(X)^m$. Dann gibt es genau eine kompakte Menge $\emptyset \ne E\subset X$ mit $\Psi^*(E) = E$. Ist $s\in[0,\infty)$ die eindeutig bestimmte Lösung der Gleichung $\sum_{i=1}^m \Lip(\psi_i)^s = 1$, so gilt $\mu^s(E)<\infty$ und daher $\dim_H(E) \le s$.
\end{satz}

\begin{beweis}
Nach dem Banachschen Fixpunktsatz hat jedes $\psi_i$ einen Fixpunkt $x_i\in X$, das heißt $\psi_i(x_i) = x_i$. Setze $F \da \{x_1,\ldots,x_m\}\subset X$ und
\[
E\da\overline{\bigcup_{k=1}^\infty \Psi^{*k}(F)}.
\]

Zunächst ist $\Psi^{*k}(F)\subset \Psi^{*(k+1)}(F)$ für $k\in \MdN$, da 
\[
\psi_{i_1}\circ \cdots \circ \psi_{i_k}(x_j) = \psi_{i_1}\circ \cdots\circ \psi_{i_k}\circ \psi_j (x_j) \in \Psi^{*(k+1)}(F).
\]
Ferner gilt 
\begin{align*}
\Psi^*(\bigcup_{k=1}^\infty \Psi^{*k}(F))
&= \bigcup_{i=1}^m \psi_i(\bigcup_{k=1}^\infty \Psi^{*k}(F))\\
&= \bigcup_{k=1}^\infty \bigcup_{i=1}^m \psi_i(\Psi^{*k}(F))\\
&= \bigcup_{k=1}^\infty \Psi^*(\Psi^{*k}(F))\\
&= \bigcup_{k=1}^\infty \Psi^{*(k+1)}(F) \\
&= \bigcup_{k=1}^\infty \Psi^{*k}(F).
\end{align*}
Es folgt für $\Psi^*(E)$ wegen der Stetigkeit der $\psi_i$:
\begin{align*}
\Psi^*(E)
& =\Psi^*\left(\overline{\bigcup_{k=1}^\infty \Psi^{*k}(F)}\right)\\
&= \bigcup_{i=1}^m\psi_i\left(\overline{\bigcup_{k=1}^\infty \Psi^{*k}(F)}\right) \\
&\subset \bigcup_{i=1}^m\overline{\psi_i\left(\bigcup_{k=1}^\infty \Psi^{*k}(F)\right)} \\
&\subset \bigcup_{i=1}^m\overline{\Psi^*\left(\bigcup_{k=1}^\infty \Psi^{*k}(F)\right)} \\
&= \overline{\bigcup_{k=1}^\infty \Psi^{*k}(F)} = E.
\end{align*}
Wir zeigen als nächstes, dass $E$ totalbeschränkt ist. Sei hierzu
\[
r\da \max\{\Lip(\psi_i): i\in\{1,\ldots,m\}\} < 1.
\]
Sei $\ep>0$. Dann gibt es ein $k\in \MdN$ mit
\[
\underbrace{\diam(\Psi^*(F))}_{<\infty} \cdot \sum_{i=k}^\infty r^i < \frac\ep2.
\]
Sei $x\in E$. Nach Definition von $E$ gibt es ein $p\in\MdN$, ohne Beschränkung der Allgemeinheit $p\ge k+1$, und ein $y\in\Psi^{*p}(F)$ mit $d(x,y)<\frac\ep2$. Also gibt es Indizes $i_1,\ldots,i_p\in \{1,\ldots,m\}$ und $j\in\{1,\ldots,m\}$ mit $y = \psi_{i_p}\circ\cdots\circ\psi_{i_1}(x_j)$. Wiederholte Anwendung der Dreiecksungleichung ergibt
\begin{align*}
d(x,\Psi^{*k}(F))
&\le d(x,y) + d(y,\Psi^{*k}(F))\\
&\le d(x,y) + \sum _{l=1}^{p-k} d(\psi_{i_p}\circ \cdots \circ \psi_{i_l}, \psi_{i_p} \circ \cdots\circ \psi_{i_{l+1}}(x_j))\\
&\le d(x,y) + \sum_{l=1}^{p-k} r^{p-l} \underbrace{d(\psi_{i_l}(x_j), x_j)}_{\le \diam(\Psi^*(F))}\\
&\le \frac\ep2 + \frac\ep2 = \ep .
\end{align*}
Dies zeigt, dass $E$ total beschränkt ist. Als abgeschlossene Menge in einem vollständigen metrischen Raum 
ist $E$ selbst vollständig, und somit ist $E$ sogar kompakt.

Hiermit zeigen wir nun $E\subset \Psi^*(E)$. Sei $z\in E$. Es existieren dann $z_i \in \bigcup_{k=1}^\infty \Psi^{*k}(F)$ mit $z_i \to z$. Hierbei ist $z_i = \psi_{l(i)}(y)$ mit $y_i\in \bigcup_{k=0}^\infty \Psi^{*k}(F) = \bigcup_{k=1}^\infty \Psi^{*k}(F) \subset E$ und $l(i)\in\{1,\ldots,m\}$. Es existiert eine Teilfolge von $(y_i)_{i\in\MdN}$, ohne Beschränkung der Allgemeinheit die Folge selbst, mit $y_i \to y\in E$. Ferner gibt es ein $l\in\{1,\ldots,m\}$ mit $l(i)=l$ für unendlich viele $i\in\MdN$. Hieraus folgt: $z \leftarrow z_i = \psi_l(y_i) \to \psi_l(y)$. Daher ist $z = \psi_l(y) \in \Psi^*(E)$, das heißt $E\subset \Psi^*(E)$.

Zur Eindeutigkeit: Sei $\emptyset \ne \tilde E\subset X$ kompakt und $\Psi^*(\tilde E) = \tilde E$. Aus Lemma \ref{lem:2.23} folgt:
\[
E = \limsup_{k\to\infty}\Psi^{*k}(E) \subset \tilde E = \limsup_{k\to\infty}\Psi^{*k}(\tilde E) \subset E,
\]
das heißt $E=\tilde E$.

Zur Hausdorffdimension: Ist $\delta >0$, so existiert zunächst $k\in\MdN$ mit $r^k\cdot \diam(E)<\delta$. Wegen $\Psi^{*k}(E)=E$ folgt $\{\psi_{i_k}\circ\cdots\circ\psi_{i_1}(E) : i_1,\ldots,i_k\in\{1,\ldots,m\}\}\in\Omega_\delta(E)$ und damit
\begin{align*}
\mu^s_\delta(E)
&\le \sum_{\mathclap{i_1,\ldots,i_k=1}}^m \diam(\psi_{i_k}\circ\cdots\circ\psi_{i_1}(E))^s \\
&\le \sum_{\mathclap{i_1,\ldots,i_k=1}}^m \Lip(\psi_{i_k})^s\cdots \Lip(\psi_{i_1})^s (\diam(E))^s\\
&= (\diam(E))^s \cdot (\underbrace{\Lip(\psi_1)^s+\cdots+ \Lip(\psi_m)^s}_{=1})^k = \diam(E)^s.
\end{align*}
Dies impliziert
\[
\mu^s(E) = \lim_{\delta\downto0}\mu_\delta^s(E) \le (\diam(E))^s < \infty.
\]
\end{beweis}

\textbf{Zusatz:} Sei $\emptyset \ne C\subset X$ kompakt mit $\Psi^*(C)\subset C$. Dann ist $E\da \bigcap_{k=1}^\infty \Psi^{*k}(C)$.

In der Tat ist
$E = \limsup_{k\to\infty} \Psi^{*k}(E) \subset C$ nach Lemma \ref{lem:2.23}, $\Psi^{*k}(E) = E$, $\Psi^{*(k+1)}(C) \subset \Psi^{*k}(C)$ und damit, wiederum mit Lemma \ref{lem:2.23},
\[
E = \bigcap_{k=1}^\infty \Psi^{*k}(E) \subset \bigcap_{k=1}^\infty \Psi^{*k}(C) = \limsup_{k\to\infty} \Psi^{*k}(C) \subset E.
\]

\begin{satz}[Hutchinson, 1981]
\label{satz:2.25}
Für $\Psi= (\psi_1,\ldots,\psi_m)\in K(\MdR^n)^m$ seien die folgenden Eigenschaften erfüllt:
\begin{enumerate}
\item Für $i=1,\ldots,m$ gibt es $r_i\in(0,1)$ mit $|\psi_i(x) - \psi_i(y)| = r_i|x-y|$ für alle $x,y\in\MdR^n$.
\item Es gibt eine beschränkte, offene Menge $\emptyset \ne U\subset\MdR^n$ derart, dass $\psi_1(U),\ldots,\psi_m(U)$ paarweise disjunkt sind und $\Psi^*(U) \subset U$.
\end{enumerate}
Dann gilt für die durch $\sum_{i=1}^m r_i^s = 1$ festgelegte Zahl $s>0$ und das $\Psi^*$-invariante Kompaktum $E\subset \MdR^n$ die Abschätzung $0<\mu^s(E) < \infty$, das heißt insbesondere $\dim_H(E)= s$. Man nennt $s$ die Ähnlichkeitsdimension von $R$.
\end{satz}

\begin{beweis}
Übung.
\end{beweis}

Konstruktion von Mengen mit Hausdorffdimension $s\in[0,n]$. Für $r\in(0,\frac12]$ und für $\alpha=(\alpha_1,\ldots,\alpha_n)^\top\in\{0,1\}^n$ sei $\psi_\alpha : \MdR^n \to \MdR$ 
erklärt durch 
$$
\psi_\alpha(x) \da r\cdot x + \sum_{k=1}^n (1-r)\alpha_ke_k,
$$ 
wobei $e_1,\ldots,e_n$ die Standardbasis des $\MdR^n$ ist. Nun sei $\Psi=(\psi_\alpha: \alpha\in\{0,1\}^n)$. Setze $U\da (0,1)^n$. Für $x\in U$ ist 
$$
0 < (\psi_\alpha(x))_k = r\cdot x_k + (1-r)\alpha_k < r + (1-r) = 1,
$$ das heißt $\Psi(x) \in U$.

Seien $\alpha,\beta \in\{0,1\}^n$, $\alpha \ne \beta$. Dann gibt es $k\in\{1,\ldots,n\}$ mit $|\alpha_k - \beta_k| = 1$. Seien $x,y\in U$. Wegen $0<r\le \frac12$ gilt
\begin{align*}
|(\psi_\alpha(x))_k - (\psi_\beta(y))_k|
&= |rx_k + (1-r)\alpha_k - ry_k - (1-r)\beta_k|\\
&\ge |(1-r)(\alpha_k - \beta_k)| - |r(x_k-y_k)|\\
&> (1-r) - r \ge 0,
\end{align*}
das heißt $(\psi_\alpha(x))_k \ne (\psi_\beta(x))_k$. Ferner gilt $|\psi_\alpha(x) - \psi_\beta(x)| = r |x-y|$, $x,y\in \MdR^n$.  Nach Satz \ref{satz:2.25} erhält man für die $\Psi^*$-invariante Menge $E$ die Abschätzungen $0<\mu^s(E)<\infty$. 

Genauer: Wegen $r_1=\ldots=r_{2^n}=r\in(0,\frac12]$ ist $1 = \sum_{i=1}^{2^n} r_i^s = 2^n \cdot r^s $ und damit
\[
s = -\frac{n \ln 2}{\ln r}.
\]

\begin{beispiele}
\item Für $n=1$, $r=\frac13$ ist $\dim_H(E_{\frac13}^1) = \frac{\log 2}{\log 3}$.
\item Für $n=2$, $r=\frac14$ ist $\dim_H(E_{\frac14}^2) = 1$.
\end{beispiele}

\section{Hausdorffmaße auf euklidischen Räumen}

\begin{definition}
Das $n$-dimensionale äußere Lebesguemaß $\lambda^n$ auf $\MdR^n$ ist erklärt durch
\begin{multline*}
\lambda^n(M) \da \inf\{\sum_{i=1}^\infty \prod_{j=1}^n (b_i^{(j)} - a_i^{(j)}) : -\infty < a_i^{(j)} < b_i^{(j)} < \infty : \\ i\in\MdN\,,j=1,\ldots,n,\,
M\subset \bigcup_{i=1}^\infty (a_i^{(1)}, b_i^{(1)})\times \cdots \times (a_i^{(n)},b_i^{(n)})\}.
\end{multline*}
\end{definition}

Bei der Definition kann man sich auf Überdeckungsmengen mit Durchmesser kleiner $\ep$ für jedes $\ep>0$ beschränken oder halboffene bzw. abgeschlossene Intervalle verwenden. Wie beim Hausdorffmaß zeigt man dann, dass $\lambda^n$ Borel-regulär ist. Ferner ist $\lambda^n$ das einzige Borel-reguläre und translationsinvariante äußere Maß auf $\MdR^n$, das $[0,1]^n$ den Wert 1 zuordnet. Außerdem gilt
\[
\lambda^n = \underbrace{\lambda^1\otimes \cdots \otimes \lambda^1}_{\text{$n$-mal}},
\]
\begin{align*}
\lambda^n([a_1,b_1]\times \cdots \times [a_n,b_n]) = \prod_{j=1}^n (b_j-a_j)
\end{align*}
und
\begin{align*}
\lambda^n(r\cdot M) = r^n\cdot \lambda^n(M)
\end{align*}
für $r\ge 0$, $M\subset \MdR^n$.

\begin{satz}[Brunn-Minkowski-Ungleichung]
\label{satz:2.26}
Seien $A,B\subset \MdR^n$ nicht leere Mengen. Dann gilt \[
\lambda^n(A+B)^{\frac1n} \ge \lambda^n(A)^{\frac1n} + \lambda^n(B)^{\frac1n},
\]
wobei $A+B \da \{x+y : x\in A,\, y\in B\}$.
\end{satz}

\begin{beispiel}
Sei $B= sA$, $s>0$.
\begin{multline*}
\lambda^n(A + sA)^{\frac1n} = \lambda^n( (1+s)A )^{\frac1n} = (1+s)\lambda^n(A)^{\frac1n} =
\\
\lambda^n(A)^{\frac1n} + (s^n\lambda^n(A))^{\frac1n} = \lambda^n(A)^{\frac1n} + \lambda^n(sA)^{\frac1n} = \lambda^n(A)^{\frac1n} + \lambda^n(B)^{\frac1n}.
\end{multline*}
\end{beispiel}

\begin{beweis}
Seien zunächst $A=I_1\times \cdots \times I_n$, $B=J_1\times\cdots\times J_n$ mit beschränken Intervallen $I_j,J_j\subset\MdR$ mit nichtleerem Inneren. Dann gilt
$A + B = (I_1 + J_1) \times \cdots \times (I_n + J_n)$. Wir setzen $\lambda \da \lambda^1$, $u_k = \frac{\lambda(I_k)}{\lambda(I_k + J_k)}$ und $v_k = \frac{\lambda(J_k)}{\lambda(I_k + J_k)}$ für $k=1,\ldots,n$. Dann folgt
\begin{align*}
\frac{\lambda^n(A)^{\frac1n} + \lambda^n(B)^{\frac1n}}{\lambda^n(A+B)^{\frac1n}}
&= \frac{(\prod_{k=1}^n \lambda(I_k))^{\frac1n} + (\prod_{k=1}^n \lambda(J_k))^{\frac1n}}{\prod_{k=1}^n \lambda(I_k+J_k)^{\frac1n}} \\
&=\prod_{k=1}^n u_k^{\frac1n} + \prod_{k=1}^n v_k^{\frac1n} \\
&= e^{\sum_{k=1}^n \frac1n \ln u_k} + e^{\sum_{k=1}^n \frac1n \ln v_k} \\
&\le e^{\ln(\sum_{k=1}^n \frac1n u_k)} + e^{\ln(\sum_{k=1}^n \frac1n v_k)}\\
&= \frac1n \sum_{k=1}^n \underbrace{(u_k + v_k)}_{=1} = 1.
\end{align*}
Hier wurde verwendet, dass $\ln$ konkav und die Exponentialfunktion monoton wachsend ist. 


Seien nun $\mathcal G, \mathcal F$ endliche Familien von paarweise disjunkten Elementen aus
\[
\{ [a_1,b_1) \times \cdots \times [a_n,b_n) : -\infty < a_i < b_i < \infty \ \forall i=1,\ldots,n\}.
\]
Sei $A \da \bigcup_{P\in\mathcal G}P$, $B\da\bigcup_{Q\in\mathcal F}Q$. Durch vollständige Induktion über $\#\mathcal G + \#\mathcal F$ wird nun gezeigt, dass die Brunn-Minkowski-Ungleichung für solche Mengen $A,B$ gilt. Der Induktionsanfang ist bereits oben gegeben.

Die Brunn-Minkowski-Ungleichung gelte für $\#\mathcal G\ge 1$, $\#\mathcal F\ge 1$ mit $\#\mathcal G + \#\mathcal F \le p$ für ein $p\ge 2$. Sei nun $\#\mathcal G + \#\mathcal F \leq  p+1$ (ohne Beschränkung der Allgemeinheit $\#\mathcal G>1$). Wähle ein $i\in\{1,\ldots,m\}$ und $a\in\MdR$ derart, dass 
\begin{align*}
A_1 \da \{x = (x_1,\dots,x_n)\in A : x_i < a\}, &&
A_2 \da \{x = (x_1,\dots,x_n)\in A : x_i \ge a\}
\end{align*}
jeweils mindestens ein Element aus $\mathcal G$ enthalten. Wähle dann $b\in\MdR$ derart, dass für
\begin{align*}
B_1 \da \{x = (x_1,\dots,x_n)\in A : x_i < b\}, && 
B_2 \da \{x = (x_1,\dots,x_n)\in A : x_i \ge b\}
\end{align*}
gilt:
\begin{align*}
\frac{\lambda^n(B_i)}{\lambda^n(B)} = \frac{\lambda^n(A_i)}{\lambda^n(A)}
\end{align*}
für $i=1,2$.

Sei $\mathcal G_i \da \{P \cap A_i: P\in\mathcal G,\, P\cap A_i\ne \emptyset\}$ und $\mathcal F_i \da \{Q \cap B_i: Q\in\mathcal F,\, Q\cap B_i\ne \emptyset\}$ für $i=1,2$. Dann ist $A_i = \bigcup_{P\in \mathcal G_i}P$, $B_i = \bigcup_{G\in \mathcal F_i}Q$, $\#\mathcal G_i < \#\mathcal G$ und $\#\mathcal F_i \le \#\mathcal F$ für $i=1,2$. Folglich gilt die Induktionsannahme für $(A_i,B_i)$, $i=1,2$. Die Mengen $A_1 + B_1$ und $A_2+B_2$ werden durch die Hyperebene $\{x\in\MdR^n: x_i = a+b\}$ getrennt. Also ist wegen
\[
A+B = (A_1\cup A_2) + (B_1\cup B_2) \supset (A_1 + B_1) \cup (A_2 + B_2)
\]
und der Induktionsannahme
\begin{align*}
\lambda^n(A+B)
&\ge \lambda^n(A_1+B_1) + \lambda^n(A_2+B_2) \\
&\ge \left(\lambda^n(A_1)^{\frac 1n} + \lambda^n(B_1)^{\frac 1n}\right)^n + \left(\lambda^n(A_2)^{\frac 1n} + \lambda^n(B_2)^{\frac 1n}\right)^n \\
&= \left(\left(\frac{\lambda^n(A)}{\lambda^n(B)}\lambda^n(B_1)\right)^{\frac 1n} + \lambda^n(B_1)^{\frac 1n}\right)^n + \left(\left(\frac{\lambda^n(A)}{\lambda^n(B)}\lambda^n(B_2)\right)^{\frac 1n} + \lambda^n(B_2)^{\frac 1n}\right)^n\\
&= \lambda^n(B_1)\left( \frac{\lambda^n(A)^{\frac 1n}}{\lambda^n(B)^{\frac 1n}} + 1 \right)^n +\lambda^n(B_2)\left( \frac{\lambda^n(A)^{\frac 1n}}{\lambda^n(B)^{\frac 1n}} + 1 \right)^n\\
&= \underbrace{(\lambda^n(B_1) + \lambda^n(B_2))}_{=\lambda^n(B)} \cdot \left( \frac{\lambda^n(A)^{\frac 1n} + \lambda^n(B)^{\frac 1n}}{\lambda^n(B)^{\frac 1n}}\right)^n \\
&= \left( \lambda^n(A)^{\frac1n} + \lambda^n(B)^{\frac1n} \right)^n,
\end{align*}
d.h. die behauptete Ungleichung im betrachteten Fall.

Seien schließlich $A,B$ nichtleer und kompakt. Setze für $p\in\MdN$
\begin{align*}
\mathcal C_p &\da \{[0,2^{-p}]^n + 2^{-p}\cdot z : z\in\MdZ^n\}, \\
A_p &\da \bigcup_{\substack{C\in\mathcal C_p \\ C\cap A \ne \emptyset}} C, \\
B_p &\da \bigcup_{\substack{C\in\mathcal C_p \\ C\cap B \ne \emptyset}} C.
\end{align*}
Dann gilt $A_p \downto A$, $B_p \downto B$ für $p\to\infty$. Außerdem ist $A+B$ kompakt und $A_p + B_p \downto A+ B$. Es folgt
\[
\lambda^n(A+B) = \lim_{p\to\infty} \lambda^n(A_p + B_p) 
\ge \lim_{p\to\infty} \left( \lambda^n(A_p)^{\frac 1n} + \lambda_n(B_p)^{\frac1n} \right)^n
=  \left( \lambda^n(A)^{\frac 1n} + \lambda_n(B)^{\frac1n} \right)^n.
\]
Für beliebige beschränkte Mengen folgt die Behauptung nun aus der inneren Regularität 
des Lebesguemaßes (vgl. Satz \ref{satz:2.2}), die allgemeine Aussage erhält man 
mit Hilfe von Proposition \ref{prop1.3} a).
\end{beweis}

\begin{satz}[Isodiametrische Ungleichung]
\label{satz:2.27}
Für jede Menge$A\subset \MdR^n$ gilt
\[
\lambda^n(A)  \le \frac{\lambda^n(B(0,1))}{2^n} \cdot \diam(A)^n.
\]
\end{satz}

\begin{beweis}
Sei ohne Beschränkung der Allgemeinheit $A$ beschränkt und kompakt, da $\diam(A) = \diam(\bar A)$ und $\lambda^n(A) \le \lambda^n(\bar A)$. Setze $A-A \da A + (-A) = \{ x-y : x,y\in A\}$. Aus Satz \ref{satz:2.26} folgt
\begin{align*}
\lambda^n(\frac12(A-A)) = \frac1{2^n} \lambda^n(A + (-A)) \ge \frac 1{2^n} \left(\lambda^n(A)^{\frac1n} + {\underbrace{\lambda^n(-A)}_{=\lambda^n(A)}}^{\frac1n}\right)^n = \lambda^n(A).
\end{align*}
Ist $z\in\frac12(A-A)$, also $z=\frac12(x-y)$ mit $x,y\in A$, so ist $|z| = \frac 12 |x-y| \le \frac12 \diam(A)$. Folglich ist $\frac12(A-A) \subset B(0,\frac12\diam(A))$, also
\begin{align*}
\lambda^n(A) \le \lambda^n(\frac12(A-A)) \le \lambda^n(B(0,\frac12\diam(A))) \le \frac1{2^n} \diam(A)^n \cdot \lambda^n(B(0,1)).
\end{align*}
\end{beweis}

\begin{bemerkung}
\begin{itemize}
\item Ein alternativer Beweis der vorangehenden beiden Ungleichungen kann mit Hilfe 
von Steinersymmetrisierung erfolgen.
\item In den obigen beiden Ungleichungen ist es natürlich, nach dem Gleichheitsfall 
zu fragen. Die Schwierigkeit bei der Beantwortung der Frage hängt dann wesentlich von 
der betrachteten Mengenklasse ab.
\end{itemize}
\end{bemerkung}

\begin{definition}
Für \(\delta > 0, p>0\) und \(M \subseteq \mathbb R^n\) sei
\[
\HM^p_\delta \da \inf\left\{ \sum_{A\in\mathcal A} \alpha(p) 2^{-p} \diam(A)^p \colon \mathcal A \in \Omega_\delta(M) \right\}
\]
und
\[
\HM^p(M) \da \lim_{\varepsilon\downto0} \HM^p_\varepsilon(M).
\]
Notation: Für \(p>0\) sei
\[
\alpha(p) \da \frac{\pi^{\frac{p}2}}{\Gamma(\frac{p}2+1)} .
\]
\end{definition}

Für $p\in\mathbb{N}$ ist dann  \(\alpha(p) = \lambda^p(B^p(0,1))\) mit \(B^p(0,1) \da \{ x \in \mathbb R^p \colon \|x\| \leq 1 \}\).

\begin{lemma}\label{lem:2.28}
Im euklidischen Raum \(\mathbb R^n\) gilt: \(\HM^n \ll \lambda^n\).
\end{lemma}
\begin{beweis}
Ein allgemeines Argument folgt aus Übungsblatt 6, Nr. 1.
\par
Die überdeckenden Mengen \(A=(a_1,b_1) \times \dots \times (a_n,b_n)\) bei der Definition des Maßes \(\lambda^n\) können so gewählt werden, dass \(\diam(A) < \delta\) für ein vorgegebenes \(\delta > 0\) und
\begin{equation}\label{lowb}
 |a_i-b_i| \leq 2 \min\{|b_j-a_j| \colon 1 \leq j \leq n\} = 2|b_{i_0}-a_{i_0}|, \; i_0 \in \{1,\dots,n\}.
\end{equation}
Dann gilt:
\[
\lambda^n(A) = \prod_{i=1}^n(b_i-a_i) \geq (b_{i_0}-a_{i_0})^n
\]
und
\begin{align*}
\diam(A)^n &= \left( \sum_{i=1}^n (b_i-a_i)^2 \right)^{\frac{n}2} \\
&\leq 2^n {\sqrt n}^n |b_{i_0}-a_{i_0}|^n \\
&\leq 2^n n^{\frac{n}2} \lambda^n(A).
\end{align*}
Sei nun \(M \subseteq \mathbb R^n\) mit \(\lambda^n(M)=0\). Sei \(\delta > 0\) fest. Zu \(\varepsilon > 0\) gibt es \(A_i \subseteq \mathbb R^n\), wie oben beschrieben, d.h. \(\diam(A_i) < \delta\) und mit \eqref{lowb}, \(M \subseteq \bigcup_{i=1}^\infty A_i\) und \(\sum_{i=1}^\infty \lambda(A_i) < \varepsilon\). Dann folgt
$$ \sum_{i=1}^\infty \diam(A_i)^n \leq 2^n \cdot n^{\frac{n}2} \cdot \varepsilon 
$$
also
$$ 
\HM^n_\delta(M) \leq \alpha(n) \cdot  n^{\frac{n}2} \cdot \varepsilon. 
$$
Da \(\varepsilon\) beliebig war, ist \(\HM^n_\delta(M)=0\). Folglich ist auch \(\HM^n(M)=0\).
\end{beweis}

\begin{satz}
In \(\mathbb R^n\) gilt: \(\HM^n = \lambda^n = \HM^n_\delta\) für \(\delta > 0\).
\end{satz}
\begin{beweis}
Sei \(M \subseteq \mathbb R^n\) und $\delta>0$. Zu \(\varepsilon > 0\) existiert eine offene Menge \(U\) mit \(M \subseteq U\) und \(\lambda^n(U) \leq \lambda^n(M) + \varepsilon\) (vgl. Satz \ref{satz:2.2} c)).
\par
Sei \(\mathcal S\) die Menge aller Kugeln in \(\mathbb R^n\) mit Radius kleiner als \(\frac{\delta}2\). Wegen Korollar \ref{kor:2.9} und Übung Nr. 1 auf Blatt 4 gibt es eine abzählbare Folge \((B_i)_{i\in\mathbb N}\) in \(\mathcal S\) paarweise disjunkter Kugeln in \(U\) mit \(\lambda^n(M \setminus \bigcup_{i=1}^\infty B_i) = 0\). Mit Lemma \ref{lem:2.28} folgt
\begin{align*}
\HM^n_\delta(M) &\leq \underbrace{\HM^n_\delta(M\setminus \bigcup_{i=1}^\infty B_i)}_{=0} + \HM^n_\delta(\bigcup_{i=1}^\infty B_i) \\
&\leq \sum_{i\geq1} \underbrace{ \alpha(n) \cdot 2^{-n} \diam(B_i)^n }_{= \lambda^n(B_i)} \\
&= \lambda^n(\bigcup_{i\geq1} B_i) \\
&\leq \lambda^n(U) \\
&\leq \lambda^n(M) + \varepsilon
\end{align*}
folglich also
\[
\HM^n_\delta(M) \leq \HM^n(M) \leq \lambda^n(M).
\]
Sei \((A_i)_{i\in\mathbb N} \in \Omega_\delta(M)\). Dann ist mit Satz \ref{satz:2.27}
\[
\lambda^n(M) \leq \lambda^n(\bigcup_{i\geq1} A_i) \leq \sum_{i\geq1} \lambda^n(A_i) \leq \sum_{i\geq1} \alpha(n)\cdot 2^{-n} \diam(A_i)^n.
\]
Wir erhalten
\[
\lambda^n(M) \leq \HM^n_\delta(M) \leq \HM^n(M) \leq \lambda^n(M),
\]
das heißt die Behauptung.
\end{beweis}

\chapter{Lipschitzfunktionen und Rektifizierbarkeit}

\section{Fortsetzbarkeit und Differenzierbarkeit von Lipschitzfunktionen}

\begin{definition}
Seien \((X,d), (\bar X,\bar d)\) metrische Räume, \(f: A\rightarrow \bar X\), \(\emptyset \neq A \subseteq X\). 
Setze
\[
\Lip(f) \da \sup \{ \frac{\bar d(f(x),f(y))}{d(x,y)} \colon x,y \in A, x\neq y\}.
\]
Man nennt \(f\) eine \emph{Lipschitz-Abbildung}, falls \(\Lip(f) < \infty\).
\end{definition}

\begin{satz}
Sei \((X,d)\) metrischer Raum, \(\emptyset \neq A \subseteq X\).
\begin{enumerate}[(1)]
\item Ist \(f: A \rightarrow \mathbb R\) Lipschitzfunktion, so ist
\[
g(x) \da \inf\{f(z)+\Lip(f) \cdot d(x,z) \colon z\in A\} \quad (x\in X)
\]
eine Lipschitzfunktion mit \(\Lip(g) = \Lip(f)\) und \(g|_A=f\).
\item Ist \(f: A \rightarrow \mathbb R^n\) eine Lipschitzfunktion, so gibt es eine Lipschitzfunktion \(h: X \rightarrow \mathbb R^n\) mit \(\Lip(h) \leq \sqrt n \Lip(f)\) und \(h|_A = f\).
\end{enumerate}
\end{satz}

\begin{bemerkung}
\begin{enumerate}[(1)]
\item Johnson, Lindenstrauss und Schechtman '86 zeigen, dass im Allgemeinen nicht \(\Lip(h) = \Lip(f)\) möglich ist. 
Siehe auch Lang '99.
\item Ist \(f: A \subseteq \mathbb R^n \rightarrow \mathbb R^n\) Lipschitzfunktion, so existiert eine Lipschitz-Fortsetzung mit gleicher Lipschitzkonstante (Satz von Kirszbraun '34, Valentine '45).
\end{enumerate}
\end{bemerkung}

\begin{beweis}
\begin{enumerate}[(1)]
\item Sei \(f: A\rightarrow \mathbb R\). Für \(x,y \in X\) ist
\begin{align*}
g(x) &\leq \inf\{f(x)+(d(x,y)+d(y,z))\cdot \Lip(f) \colon z \in A\} \\
&\leq g(y) + \Lip(f)\cdot d(x,y)
\end{align*}
und aus Symmetriegründen also \(|g(x)-g(y)| \leq \Lip(f) \cdot d(x,y)\). Insbesondere ist \(\Lip(g) \leq \Lip(f)\). Für \(x\in A\) gilt
\[
f(x) + \Lip(f)\cdot d(x,x) = f(x) \leq f(z)+ \Lip(f)\cdot  d(x,z)
\]
also \(g(x) \leq f(x) \leq g(x)\), d.h. \(g(x) = f(x)\) für \(x\in A\). Somit ist \(\Lip(g) \geq \Lip(f)\).
\item Wir haben \(f=(f_1,\ldots,f_n)^T, f_i: A \rightarrow \mathbb R, \Lip(f_i) \leq \Lip(f)\). Zu \(i\in \{1,\ldots,n\}\) gibt es nach (1) eine Lipschitz-Fortsetzung \(h_i: X \rightarrow \mathbb R\) von \(f_i\) mit \(\Lip(h_i)=\Lip(f_i)\). Dann ist \(h\da (h_1,\ldots,h_n)^T\) eine Lipschitz-Fortsetzung von \(f\) und \(\Lip(h) \leq \sqrt n \Lip(f)\), denn
\begin{align*}
\|h(x)-h(y)\| &= \left( \sum_{i=1}^n (h_i(x)-h_i(y))^2 \right)^{\frac12} \\
&\leq \left( \sum_{i=1}^n \Lip(h_i)^2 d(x,y)^2 \right)^{\frac12} \\
&\leq \Lip(f) \cdot (n d(x,y)^2)^{\frac12} \\
&= \sqrt n \Lip(f) d(x,y)
\end{align*}
\end{enumerate}
\end{beweis}

\begin{satz}[Kirszbraun, Valentine]
Seien \((\HR_1, \langle\cdot,\cdot\rangle_1)\) und \((\HR_2, \langle\cdot,\cdot\rangle_2)\) Hilberträume, \(\emptyset \neq D \subseteq \HR_1\),
\(f:D\rightarrow \HR_2\) eine Lipschitz-Abbildung. Dann existiert eine Lipschitz-Fortsetzung \(h\) von \(f\) mit \(\Lip(h)=\Lip(f)\) und \(h|_D = f\).
\end{satz}
\begin{beweis}[Reich und Simons '05]
Es genügt, den Fall \(\HR_1 = \HR_2 = \HR\) zu betrachten, sowie \(\Lip(f)=1\).

\emph{Skizze}: \(\HR \da \HR_1 \oplus \HR_2 \da \{ (x,y) \colon x \in \HR_1, y \in \HR_2\}, \langle (x_1,y_1), (x_2,y_2) \rangle \da \langle x_1,x_2\rangle_1 + \langle y_1,y_2 \rangle_2\), \((x_i,y_i) \in \HR_1 \oplus \HR_2\).
Zu \(f:D \subseteq \HR_1 \rightarrow \HR_2\) betrachte \(\tilde f: D \oplus \HR_2 \rightarrow \HR_1 \oplus \HR_2\), \((x,y) \mapsto (0,f(x))\), usw. (Übung)

Sei nun \(f: D \subseteq \HR \rightarrow \HR\) mit \(\Lip(f)=1\). Zu \(f\) gebe es keine echte 1-Lipschitz-Fortsetzung. Seien
\[\HR^2 \da \HR \oplus \HR = \HR \times \HR\]
und \[\chi: \HR^2 \rightarrow (-\infty, \infty], \qquad\chi(x,y) \da \sup\{ \|y-f(z)\|^2 - \|x-z\|^2 \colon z \in D\}.\]

\textbf{Lemma A:}
Es gilt \(\chi \geq 0\) und \(\{\chi=0\} = G(f) \da \{(z,f(z)) \colon z \in D\}\).

\textbf{Beweis des Lemmas:}
Seien \(x,y \in \HR\). Ist \(x \in D\), so ergibt die Wahl \(z \da x \in D\)
\[\chi(x,y) \geq \|y-f(x)\|^2 - \|x-x\|^2 = \|y-f(x)\|^2 \geq 0.\] 
Sei nun \(x \notin D\). Sei \(\tilde f\) die Fortsetzung von \(f\) auf \(D \cup \{x\}\) mit \(\tilde f(x) \da y\). Da \(\tilde f\) keine 1-Lipschitz-Abbildung ist, muss es ein \(z \in D\) geben mit
\[
\|y-f(z)\|^2 = \|\tilde f(x) - \tilde f(z)\|^2 > \|x-z\|^2,
\]
also ist \(\chi(x,y) > 0\). Damit ist \(\chi \geq 0 \) und \(\{\chi = 0\} \subseteq G(f)\) gezeigt.
\par
Sei nun \((x,y) \in G(f)\), d.h. \(x\in D\) und \(y=f(x)\). Also ist für $z\in D$
\[
\|y-f(z)\|^2 = \|f(x)-f(z)\|^2 \leq \|x-z\|^2.
\]
Dies ergibt \(\chi(x,y) \leq 0\) und daher \( \chi(x,y) = 0\), d.h. \(G(f) \subseteq \{\chi=0\}\).

\textbf{Lemma B:}
Definiere die Abbildung \(\varphi: \HR^2 \rightarrow (-\infty, \infty]\),\quad \(\varphi(x,y) \da \frac14 \chi(x+y,x-y) + \langle x,y \rangle\) für \( x,y\in\HR\).
\begin{enumerate}[(1)]
\item Für \(x,y \in \HR\) gilt
\[
4\cdot \varphi(x,y) = \sup\{ \|f(z)\|^2 - \|z\|^2 + 2 \langle x,z-f(z) \rangle + 2 \langle y,z+f(z) \rangle \colon z \in D\}.
\]
\item Für \(z \in D\) gilt
\[
4 \cdot \varphi(\frac{z+f(z)}2, \frac{z-f(z)}2) = \|z\|^2 - \|f(z)\|^2.
\]
\item \(\varphi\) ist eine eigentliche (\(\varphi \not\equiv \infty\)), unterhalbstetige, konvexe Funktion mit \(\varphi^\ast(x,y) \geq \varphi(y,x)\) für \(x,y\in \HR\). Hierbei ist \(\varphi^\ast\) die zu \(\varphi\) konjugierte Funktion, die erklärt ist durch
\[
\varphi^\ast(\zeta) \da \sup\{ \langle \zeta,\xi \rangle - \varphi(\xi) \colon \xi \in \HR^2 \},\qquad  \zeta \in \HR^2.
\]
\end{enumerate}

\textbf{Beweis des Lemmas:}
\begin{enumerate}[(1)]
\item Für \(x,y\in \HR, z \in D\) gilt:
\begin{align*}
&\|x-y-f(z)\|^2 - \|x+y-z\|^2\\
=& -4\langle x,y\rangle - 2\langle x-y, f(z)\rangle + 2\langle x+y, z\rangle + \|f(z)\|^2 - \|z\|^2 \\
=& -4\langle x,y\rangle + 2\langle x,z-f(z)\rangle + 2\langle y,z+f(z)\rangle + \|f(z)\|^2-\|z\|^2.
\end{align*}
Bildung des Supremums über $z\in D$ ergibt die Behauptung.
\item Direkt durch Einsetzen in die Definition erhält man für $z\in D$
\[
4\varphi(\frac{z+f(z)}2, \frac{z-f(z)}2) = \underbrace{\chi(z,f(z))}_{=0} + \langle z+f(z),z-f(z)\rangle = \|z\|^2 - \|f(z)\|^2.
\]
\item Aus der rechten Seite von (1) erkennt man, dass $\varphi$ als Supremum von stetigen, konvexen Funktionen  konvex und unterhalbstetig ist. Aus (2) folgt, dass $\varphi\not\equiv \infty$ gilt. Für $x,y\in\HR$ und $z\in D$ 
gilt
\begin{align*}
\varphi^*(x,y)&\ge \left\langle \left(\frac{z+f(z)}{2}, \frac{z-f(z)}{2}\right),(x,y)\right\rangle -\varphi\left(\frac{z+f(z)}{2}, \frac{z-f(z)}{2}\right)\\
&=\frac{1}{4}\left(2\langle x,z+f(z)\rangle +2\langle y,z-f(z)\rangle+\|f(z)\|^2-\|z\|^2\right),
\end{align*}
wobei zuletzt (2) verwendet wurde. Aufgrund von (1) ist das Supremum über $z\in D$ auf der rechten Seite 
gerade $\varphi(y,x)$. 
\end{enumerate}

\textbf{Lemma C:} Sei $\tilde\HR$ ein Hilbertraum, $h(\zeta) \da \frac12\|\zeta\|^2$, $\zeta\in\tilde\HR$. Sei ferner $\psi : \tilde\HR \to (-\infty,\infty]$ eine eigentliche, unterhalbstetige konvexe Funktion. Gilt $\psi(\zeta) + h(\zeta) \ge 0 $ für alle $\zeta\in\tilde\HR$, dann gibt es ein $v\in\tilde\HR$ mit $\psi^*(v)+ h(v) \le 0$.

\textbf{Beweis des Lemmas:} Nach Voraussetzung gilt $\psi(\zeta) \ge -h(\zeta)$ für $\zeta\in\tilde\HR$. Sei
\[
\epi(\psi) \da \{ (\zeta,t) \in\tilde\HR\times\MdR: \psi(\zeta)\le t\}, \quad 
\widetilde\epi(-h) \da \{ (\zeta,s) \in\tilde\HR\times\MdR: s \le - h(\zeta) \}.
\]
Dann sind $\epi(\psi)$, $\widetilde\epi(-h)$ abgeschlossene, konvexe, nicht leere Mengen, die keine gemeinsamen inneren Punkt haben. Dann gibt es eine „nicht vertikale“ trennende Hyperebene, das heißt es gibt ein $\alpha \in \MdR$ und $v\in\tilde\HR$ mit:
\[
\psi(\zeta) \ge \langle \zeta, v \rangle + \alpha \ge -h(\zeta)
\]
für alle $\zeta\in\tilde\HR$.
Insbesondere 
\begin{align*}
\psi^*(v) &= \sup\{\langle\zeta,v\rangle - \psi(\zeta) : \zeta \in \tilde\HR\} \\
&\le -\alpha \\
&\le \inf\{\langle \zeta,v\rangle + h(\zeta) : \zeta \in \tilde\HR\}\\
&\le \langle -v, v\rangle + \frac12 \|v\|^2\\
&= -\frac12 \|v\|^2 = -h(v),
\end{align*}
also $\psi^*(v)+ h(v) \le 0$.

\textbf{Lemma D}: Hat $f:D\to \HR$ keine echte 1-Lipschitzfortsetzung, so gilt $D=\HR$.

\textbf{Beweis des Lemmas:} Sei $h(x,y) \da \frac12\|(x,y)\|^2$, $(x,y)\in\HR^2$. Es gilt für $x,y\in\HR$:
\begin{align*}
4(\varphi(x,y) + h(x,y)) &= \chi(x+y,x-y) + 4\langle x,y\rangle + 2\|x\|^2 + 2\|y\|^2 \\
&= \chi(x+y,x-y) + 2\|x+y\|^2 \ge 0.
\end{align*}
Wegen Lemma B gilt somit
\[
\varphi^*(y,x) + h(y,x) \ge \varphi(x,y) + h(x,y) \ge0
\]
für alle $x,y\in\HR$. Nach Lemma C gibt es ein $(x_0,y_0)\in\HR^2$ mit
$\varphi(y_0,x_0) + h(y_0,x_0) \le 0$, das heißt $\varphi^*(y_0,x_0) + h(y_0,x_0) = 0 = \varphi(x_0,y_0)+h(x_0,y_0)$ also $\chi(x_0+y_0, x_0-y_0) + 2\|x_0+y_0\|^2 = 0$ und somit $x_0=-y_0$ und $\chi(0,2x_0)= 0$. Dies führt wegen Lemma A auf $(0,2x_0)\in G(f)$, das heißt $0\in D$.

Sei nun $x_1\in\HR$ beliebig. Definiere $f_{x_1} : D-x_1 \to \HR$ durch $f_{x_1}(x) \da f(x+x_1)$. Da $f_{x_1}$ keine echte 1-Lipschitzfortsetzung hat, muss $0\in  = D-x_1$ gelten, also $x_1\in D$ und somit $D=\HR$.

\textbf{Beweis des Satzes:}
Sei $f:D \to \HR$ eine 1-Lipschitzabbildung mit $\emptyset \ne D \subset \HR$. Betrachte
\[
\mathcal S \da \{(E,g) : D\subset E,\, g|_D = f,\, g\text{ ist 1-Lipschitzabbildung}\}.
\]
Durch 
\[
(E,g)\prec(\tilde E, \tilde g) :\iff E\subset \tilde E \text{ und } \tilde g|_E = g
\]
für $(E,g),(\tilde E,\tilde g)\in \mathcal S$ wird auf $\mathcal S$ eine Ordnung eingeführt, in der jede Kette eine obere Schranke besitzt. Nach dem Zornschen Lemma gibt es ein maximales Element in $\mathcal S$, etwa $(E,g)$. Dann ist wegen Lemma D aber $E=\HR$, das heißt $(\HR,g)$ ist die gesuchte 1-Lipschitzfortsetzung von $f$.
\end{beweis}

Eine Abbildung $f:\MdR^m \to \MdR^n$ heißt differenzierbar an der Stelle $x\in\MdR^m$, falls es eine lineare Abbildung $L:\MdR^m\to\MdR^n$ gibt mit
\[
\lim_{y\to x} \frac{f(y)-f(x)- L(y-x)}{\|y-x\|} = 0.
\]
In diesem Fall schreiben wir $Df(x) \da Df_x \da L$. Ferner ist
\[
 D_uf(x)\da Df_x(u) = \lim_{t\to0}\frac{f(x+tu) - f(x)}{t}
\]
für $u\in\MdR^m$.



\begin{satz}[Rademacher]
Jede Lipschitzfunktion $f:\MdR^m \to \MdR^n$ ist $\lambda^m$-fast-überall differenzierbar.
\end{satz}

\begin{beweis}
Sei stets ohne Beschränkung der Allgemeinheit $n=1$.
\begin{enumerate}[{Teil} 1:]
\item $m=1$ und $f$ ist monoton wachsend. Definiere
\[
\nu_f(M) \da \inf\{\sum_{i=1}^\infty (\underbrace{f(b_i) - f(a_i)}_{\ge 0}): a_i < b_i,\, M\subset \bigcup_{i=1}^\infty (a_i,b_i)\}.
\]
Wie im Fall $\lambda^1$ ist $\nu_f\in\mathbb M (\MdR)$. Beachte hierzu $\nu_f(M) \le \Lip(f) \cdot \lambda^1(M)$. Dies zeigt auch $\nu_f \ll \lambda^1$ und daher $(\nu_f)_{\lambda^1} = \nu_f$. Das System
\[
V \da \{ (x,[a,b]): x\in[a,b]\}
\]
ist eine $\lambda^1$-Vitali-Relation. Also ist $\lambda^1$-fast-überall $\mathbb D(\nu_f,\lambda^1,V,\cdot) \in [0,\infty)$  und
\[
\mathbb D(\nu_f,\lambda^1,V,x)
= V\text-\lim_{[a,b]\to x} \frac{\nu_f([a,b])}{\lambda^1([a,b])}
= \lim_{y\to x} \frac{f(y) - f(x) }{y-x} = f'(x)
\]
für $\lambda^1$-fast-alle $x\in\MdR$.

\item $m=1$, $f:\MdR\to\MdR$ nicht notwendigerweise monoton. Für $x\in \MdR$ sei
\[
g(x) \da 
\begin{cases}
\phantom{-}\sup\{\sum_{i=1}^m|f(a_i)-f(a_{i-1})| : 0 = a_0 \le a_1 \le \cdots \le a_m = x, \, m\in\MdN\}, & x\ge 0,\\
-\sup\{\sum_{i=1}^m|f(a_i)-f(a_{i-1})| : x = a_0 \le a_1 \le \cdots \le a_m = 0, \, m\in\MdN\}, & x\le 0.
\end{cases}
\]
Damit ist $g$ monoton wachsend. Für $x>y$ gilt
\begin{align*}
g(x)
&= g(y) + \sup\{\sum_{i=1}^m |f(a_i) - f(a_{i-1})|: y = a_0\le a_1\le \cdots \le a_m = x,\, m\in\MdN\} \\
&\le g(y) + \Lip(f) \cdot \sup\{\sum_{i=1}^m |a_i-a_{i-1}|: y = a_0\le a_1\le \cdots \le a_m = x,\, m\in\MdN\} \\
&= g(y) + \Lip(f) \cdot |y-x|
\end{align*}
also $|g(x) - g(y)| \le \Lip(f) \cdot |x-y|$. Außerdem erhalten wir für $x>y$, dass
\[
g(x)\ge g(y) + |f(x)-f(y)| \ge g(y) + f(x) - f(y),
\]
und somit $(g-f)(x) \ge (g-f)(y)$, das heißt $g-f$ ist ebenfalls monoton wachsend und ebenfalls Lipschitz. Nach Teil 1 sind daher $g$ und $g-f$ $\lambda^1$-fast-überall differenzierbar. Somit auch $f=g-(g-f)$.

\textbf{Nebenbemerkung:} Aus dem Beweis folgt zunächst im Fall einer monotonen Lipschitzfunktion 
\begin{align*}
f(b)-f(a) &= \nu_f([a,b]) = (\nu_f)_{\lambda^1}([a,b])\\
& = \int_{[a,b]} \mathbb D(\nu_f, \lambda^1, V, x)\lambda(dx)\\
& = \int_{[a,b]} f'(x) \lambda(dx).
\end{align*}
Die Gleichung
$$
f(b)-f(a)= \int_{[a,b]} f'(x) \lambda(dx)
$$
erhält man dann sogar für jede Lipschitzfunktion aufgrund der in Teil 2 beschriebenen Zerlegung. 
Dies ist der Hauptsatz der Differenzial- und Integrationsrechnung für Lipschitzfunktionen.

\item $m\ge 1$. Sei $e\in S^{m-1}$. Sei $N_e$ die Menge aller $x\in\MdR^m$, für die $t \mapsto f(x+te)$ in $t=0$ nicht differenzierbar ist. Dann ist $N_e$ eine Borelmenge und $\HM^1(N_e \cap (x+\MdR e)) = 0$ nach Teil 2. Der Satz von Fubini zeigt $\lambda^m(N_e) = 0$ für beliebige $e\in S^{m-1}$.

Für $\lambda^m$-fast-alle $x\in\MdR^m$ existiert dann 
\[
\nabla f(x) \da (D_1 f(x), \ldots, D_mf(x)),
\]
wobei $D_if(x) \da D_{e_i}f(x)$ für die Standardbasis $(e_1,\ldots,e_m)$ des $\MdR^m$. Für festes $e\in S^{m-1}$ existiert $D_ef(x)$ für $\lambda^m$-fast-alle $x\in\MdR^m$. Sei $\varphi \in\mathcal C^\infty_c(\MdR^m)$. Für $t>0$ und $e\in S^{m-1}$ gilt
\[
\int_{\MdR^m} \frac{f(x+t e) - f(x)}{t} \cdot \varphi(x) \lambda^m(dx)
= 
-\int f(x) \cdot \frac{\varphi(x) - \varphi(x-t e)}{t}\varphi(x) \lambda^m(dx).
\]
Wegen $|\varphi| \le \|\varphi\|_\infty \cdot \ind_{\supp(\varphi)}$ und da $f$ Lipschitz-stetig ist, erhält man   mit dem Satz von der majorisierten Konvergenz:
\begin{align*}
\int_{\MdR^m} D_ef(x) \varphi(x) \lambda^m(dx)
&= -\int f(x) D_e\varphi(x) \lambda^m(dx) \\
&= - \sum_{j=1}^m \langle e, e_j\rangle \cdot \int f(x) \cdot D_j \varphi(x)\lambda^m(dx) \\
&= - \sum_{j=1}^m \langle e, e_j\rangle \cdot (-1) \int_{\MdR^m} D_{e_j} f(x) \varphi(x) \lambda^m(dx) \\
&= \int_{\MdR^m} \langle e, \nabla f(x) \rangle \varphi(x) \lambda^m(dx).
\end{align*}
Hieraus liest man $D_ef(x) = \langle e,\nabla f(x) \rangle$ für $\lambda^m$-fast-alle $x\in\MdR^m$ ab, und zwar   
für ein beliebiges (aber festes) $e\in S^{m-1}$.

Sei $E\subset S^{m-1}$ eine abzählbare dichte Teilmenge. Dann ist das Komplement der Menge 
\[
A \da \bigcap_{e\in E}\{x\in\MdR^m : D_ef(x) = \langle e, \nabla f(x) \rangle\}
\]
eine $\lambda^m$-Nullmenge. Wir zeigen, dass $f$ in $x\in A$ differenzierbar ist. Sei also $x\in A$. Für $t>0$ und $e\in S^{m-1}$ sei
\[
\Delta_t(e) \da \frac{f(x+t e)- f(x)}{t} - \langle e,\nabla f(x) \rangle.
\]

Wir zeigen: $\Delta_t(e) \to 0$ für $t\to 0$ gleichmäßig in $e\in S^{m-1}$.

Setze $M\da 2\max\{\Lip(f), \|\nabla f(x)\|, 1\}$. Dann gilt für $e,\bar e\in S^{m-1}$:
\begin{align*}
|\Delta_t (e) - \Delta_t(\bar e)|
\le | \frac{f(x + t e) - f(x + t \bar e)}t | + \|\nabla f(x) \| \cdot \|e-\bar e\|
\le M \cdot \|e-\bar e\|.
\end{align*}
Ferner gilt $\Delta_t(e) \to 0$ für alle $e\in E$, wenn $t \to 0$. 
Sei nun $\ep>0$ beliebig vorgegeben. Dann gibt es eine endliche Teilmenge $\bar E\subset E$ mit $d(e,\bar E) \le \frac\ep{2M}$ für jedes $e\in S^{m-1}$. Bei fest gewählter Menge $\bar{E}$ gibt es ein $\delta>0$, so dass $|\Delta_{t} (\bar e) | \le \frac\ep2$ für alle $\bar e\in\bar E$ und $0<t\le \delta$. Für $e\in S^{m-1}$ und $0<t\le\delta$ gilt dann:
\begin{align*}
|\Delta_{t} (e)|
&\le \min_{\bar e\in\bar E} \{ |\Delta_{t}(e) - \Delta_{t}(\bar e)| + |\Delta_{t}(\bar e)|\} \\
&\le \min_{\bar e\in\bar E} \{ M \cdot \|e-\bar e\| + \frac\ep2\} \\
&\le \frac\ep{2M} \cdot M + \frac\ep2 = \ep.
\end{align*}
Dies zeigt $\Delta_t\to 0$ gleichmäßig für $t\to 0$, und daraus folgt die Behauptung.
\end{enumerate}
\end{beweis}

\section{Die Flächenformel}
Sei $U\subset\MdR^n$ offen und $f:U\to\MdR^n$ ein $C^1$-Diffeomorphismus. Dann ist für eine beliebige 
$\lambda^n$-messbare Funktion $g:\MdR^n\to [0,\infty]$
\begin{equation}
\int_U g\circ f(x)\cdot Jf(x) \lambda^n(dx)=\int_{f(U)}g(y)\lambda^n(dy),\tag{$*$}
\end{equation}
wobei $Jf(x)\da |\det(Df(x))|$.

\textbf{Ansätze für Verallgemeinerungen:}
\begin{itemize}
\item Abbildungen $f:\MdR^m\to\MdR^n$ mit $m,n\in\MdN$ oder sogar allgemeiner $f:M\to N$ mit gewissen $m$-dimensionalen bzw. $n$-dimensionalen Mengen $M,N$.
\item $f$ nicht notwendig injektiv.
\item $f$ nicht notwendig differenzierbar, aber Lipschitz.
\end{itemize}

Ist speziell $A\subset U$ eine Borelmenge und $g(y)\da \ind_{f(A)}(y)$, dann geht $(*)$ über in
\[
\int_A Jf(x)\lambda^n(dx)=\lambda^n(f(A)).
\]

Für eine lineare Abbildung ist dies fast trivial. Umformulierung ergibt
\[
\int_A Jf(x)\lambda^n(dx)=\int_{\MdR^n}\#\left(A\cap f^{-1}\{y\}\right)\lambda^n(dy),
\]
wobei
$$
\#\left(A\cap f^{-1}\{y\}\right)=\begin{cases} 1,&y\in f(A),\\0,&\text{ sonst}, 
\end{cases}
$$
da $f$ injektiv ist.

\begin{beispiel}
Wie sieht es dagegen in der folgenden Situation aus?
Es seien $n=1$, $f:[-1,1]\to\MdR$ mit $f(t) = 1-t^2$. Es ist $f'(t) = -2t$ und damit $Jf(t)=2|t|$.
\[
\int_{[-1,1]}Jf(y) dy = \int_{[-1,1]} 2|t|dt = 2 \cdot d\int_0^1 tdt = 4 \cdot \frac12 \cdot 1 = 2.
\]
Aber es ist $f([-1,1]) = [0,1]$ und $\int_{f(A)}dt = \lambda(f(A)) = 1$. Man beachte jedoch
\[
\int \underbrace{\#( [-1,1] \cap f^{-1}(\{t\}))}_{
=\begin{cases}
2, & t\in(0,1) \\
1, & t\in\{0,1\} \\
0, & \text{sonst}
\end{cases}
}=  2\cdot 1.
\]
\end{beispiel}

\begin{definition}
Eine lineare Abbildung $\varrho: \MdR^m \to \MdR^n$ heißt lineare Isometrie, falls
\[
\langle \varrho(x), \varrho(y) \rangle = \langle x,y \rangle
\]
für alle $x,y\in\MdR^m$.
\end{definition}

\begin{bemerkungen}
\item Eine lineare Isometrie ist injektiv und $\|\varrho(x) - \varrho(y)\| = \|x-y\|$ für alle $x,y\in\MdR^m$.
\item $\HM^t(\varrho(A)) = \HM^t(A)$ für alle $A\subset \MdR^m$ und $t\ge 0$.
\end{bemerkungen}

\begin{lemma}
\label{lem:3.4}
Sei $m\le n$ und $f:\MdR^m \to \MdR^n$ eine lineare Abbildung. Dann gibt es eine symmetrische lineare Abbildung $\sigma: \MdR^m\to\MdR^m$ und eine lineare Isometrie $\varrho:\MdR^m\to\MdR^n$ mit $f =\varrho \circ \sigma$. Ist $(a_1,\ldots,a_m)$ eine Orthonormalbasis von $\MdR^m$, so gilt
\[
\llbracket f \rrbracket \da |\det(\sigma)| = \sqrt{ \det(\langle f(a_i),f(a_j)\rangle)_{i,j=1,\ldots,m}},
\]
und $\HM^m(f(A)) = \llbracket f \rrbracket \cdot \HM^m(A)$ für $A\subset \MdR^m$.
\end{lemma}

\begin{beweis}
Wir definieren eine Hilfsabbildung $\varphi = f^* \circ f : \MdR^m\to\MdR^m$. Sie ist symmetrisch, da 
\[
\langle \varphi(x),y \rangle = \langle f^* \circ f(x),y\rangle = \langle f(x),f(y)\rangle,
\]
und $\varphi$ ist positiv semidefinit. Somit existiert eine Orthonormalbasis $a_1,\ldots,a_m$ aus Eigenvektoren von $\varphi$ mit Eigenwerten $\lambda_1,\ldots,\lambda_m\ge 0$. Wir setzten $b_i \da \frac{1}{\sqrt{\lambda_i}} \cdot f(a_i)$, falls $\lambda_i \ne 0$. Beachte dabei
\begin{align*}
\langle b_i,b_j \rangle 
= \frac1{\sqrt{\lambda_i\lambda_j}} \cdot \langle f(a_i),f(a_j)\rangle 
= \sqrt{\frac{\lambda_i}{\lambda_j}} \cdot \langle a_i,a_j\rangle = \delta_{ij},
\end{align*}
falls $\lambda_i,\lambda_j\ne 0$. Wir ergänzen diese Vektoren zu einer Orthonormalbasis $(b_1,\ldots,b_m)$ des $\MdR^m$.

Wir definieren nun $\sigma(a_i) \da \sqrt{\lambda_i}\cdot a_i$, $i=1,\ldots,m$. $\sigma$ ist symmetrisch, da $(a_1,\ldots,a_m)$ eine Orthonormalbasis ist. Wir definieren weiter $\varrho(a_i) \da b_i$, $i=1\ldots,m$. $\varrho$ ist eine lineare Isometrie. Nun gilt für $\lambda_i \ne 0$
\begin{align*}
\varrho\circ\sigma(a_i) = \varrho(\sqrt{\lambda_i}\cdot a_i) = \sqrt{\lambda_i} \cdot b_i  = \sqrt{\lambda_i} \cdot \frac{1}{\sqrt{\lambda_i}} \cdot f(a_i) = f(a_i),
\end{align*}
und für $\lambda_i = 0$ ist $\varrho\circ\sigma(a_i)=0$ und es gilt
\begin{align*}
\langle f(a_i),f(a_i)\rangle = \langle\varphi(a_i),a_i\rangle = \langle \lambda_i a_i,a_i\rangle = 0
\end{align*}
also $f(a_i) = 0$.

Seien $\varrho$, $\sigma$ und $(a_1,\ldots,a_m)$ wie in den Voraussetzungen des Lemmas gewählt:
\begin{align*}
\langle f(a_i), f(a_j)\rangle = \langle \varrho\circ \sigma(a_i), \varrho\circ\sigma(a_j) \rangle = \langle \sigma(a_i),\sigma(a_j)\rangle.
\end{align*}
Sei $S$ die beschreibende Matrix von $\sigma$ bezüglich $(a_1,\ldots,a_m)$. Dann:
\begin{align*}
(\det(\sigma))^2
&= \det(S^\top)\cdot \det(S) = \det(S^\top\cdot S) = \\
&= \det( (\langle \sigma(a_i), \sigma(a_j)\rangle)_{i,j=1,\ldots,m}) \\
&= \det( (\langle f(a_i), f(a_j)\rangle)_{i,j=1,\ldots,m}).
\end{align*}

Ferner gilt für $A\subset\MdR^m$:
\begin{align*}
\HM^m(f(A)) = \HM^m(\varrho\circ\sigma(A)) = \HM^m(\sigma(A)) = |\det(\sigma)| \cdot \HM^m(A)  = \llbracket f \rrbracket \cdot \HM^m(A).
\end{align*}
\end{beweis}

\begin{bemerkungen}
\item Sei
\[
\Lambda(n,m) \da \{\alpha:\{1,\ldots,m\} \to \{1,\ldots,n\} : \text{$\alpha$ ist streng monoton wachsend}\}
\]
für $m\le n$. Für $\alpha \in\Lambda(n,m)$ sei $p_\alpha: \MdR^n \to\MdR^m$ mit
\[
p_\alpha( (x_1,\ldots,x_n)^\top ) \da (x_{\alpha(1)},\ldots,x_{\alpha(m)})^\top
\]
erklärt. Dann gilt
\[
\llbracket f \rrbracket^2 = \sum_{\alpha\in \Lambda(n,m)}  \det(p_\alpha \circ f)^2.
\]
Dies ist der verallgemeinerte Satz des Pythagoras.
\item Sei $f:\MdR^m \to \MdR^n$ linear mit $f=\varrho\circ\sigma$ und $a_1,\ldots,a_m$ eine Orthonormalbasis. Dann ist
\begin{align*}
\llbracket f \rrbracket 
&= \det(\sigma(a_1),\ldots,\sigma(a_m)) \\
&\le \|\sigma(a_1)\|\cdot \cdots \cdot \|\sigma(a_m)\| \\
&= \|f(a_1)\|\cdot \cdots \cdot \|f(a_m)\|.
\end{align*}
\end{bemerkungen}

\begin{proposition}
\label{prop:3.5}
Seien $m\le n$, $f:\MdR^m\to\MdR^n$ eine Lipschitzabbildung und $A\in \borel(\MdR^m)$. Dann ist die Abbildung
\[
y\mapsto \#(A\cap f^{-1}(\{y\}))
\]
eine $\HM^m$-messbare Abbildung $\MdR^n\to [0,\infty]$ und
\[
\int_{\MdR^n} \#(A\cap f^{-1}(\{y\}))\, \HM^m(dy) \le \Lip(f)^m \cdot \HM^m(A).
\]
\end{proposition}

\begin{bemerkung}
Eine entsprechende Aussage gilt allgemein für $f:(X,d)\to (Y,\bar d)$, wobei $(X,d)$ ein polnischer, das heißt separabler, vollständiger metrischer Raum ist.
\end{bemerkung}

\begin{beweis}
Betrachte Folgen von Zerlegungen für $i\in \MdN$:
\begin{align*}
\mathcal C_i &\da  \{[0,2^{-i})^m + 2^i \cdot z: z \in \MdZ^m\},\\
\mathcal B_i &\da  \{C\cap A : C \in \mathcal C_i\}.
\end{align*}
Dies ist eine abzählbare disjunkte Zerlegung des $\MdR^m$ mit
\[
\sup\{\diam(C): C\in\mathcal C_i\} \to 0
\]
für $i\to \infty$, und ferner ist jedes $C\in\mathcal C_i$ disjunkte Vereinigung von gewissen Mengen 
$\tilde C\in\mathcal C_{i+1}$.

Zu $B\in\mathcal B_i$ existiert eine Folge $(K_j)_{j\in \MdN}$ kompakter Mengen mit $K_j \subset B$ und $\HM^m(B\setminus K_j) < \frac 1j$ und damit $\HM^m(B\setminus\bigcup_{j=1}^\infty K_j) = 0$.
Wegen 
\begin{align*}
f(B) \setminus \bigcup_{j=1}^\infty f(K_j) = f(B) \setminus f(\bigcup_{j=1}^\infty K_j) \subset f(B\setminus \bigcup_{j=1}^\infty K_j)
\end{align*}
folgt
\begin{align*}
0 \le \HM^m(f(B) \setminus \bigcup_{j=1}^\infty f(K_j)) \le \Lip(f)^m \cdot \HM^m(B\setminus \bigcup_{j=1}^\infty K_j) = 0.
\end{align*}

Da $f(K_j) \in \borel(\MdR^n) \subset \A_{\HM^m}$ und $f(B)\setminus \bigcup_{j=1}^\infty f(K_j) \in \A_{\HM^m}$ folgt $f(B) \in \A_{\HM^m}$. Wegen
\begin{align*}
\sum_{B\in\mathcal B_i} \ind_{f(B)} \nearrow \#(A\cap f^{-1}(\{y\}))
\end{align*}
für $i\to \infty$  folgt die Messbarkeitsbehauptung.

Mit dem Satz von der monotonen Konvergenz erhält man schließlich
\begin{align*}
\int_{\MdR^n} \#(A \cap f^{-1}(\{y\}) )\, \HM^m(dy) 
&= \int_{\MdR^n} \lim_{i\to\infty} \sum_{B\in\mathcal B_i} \ind_{f(B)}(y) \HM^m(dy) \\
&= \lim_{i\to\infty} \int_{\MdR^n}\sum_{B\in\mathcal B_i}\ind_{f(B)}(y) \HM^m(dy) \\
&= \lim_{i\to\infty} \sum_{B\in\mathcal B_i} \underbrace{\HM^m(f(B))}_{\le \Lip(f)^m\cdot \HM^m(B)}\\
&\le \Lip(f)^m \cdot \lim_{i\to\infty} \HM^m(\underbrace{\bigcup_{B\in\mathcal B_i} B}_{=A}) \\
&=   \Lip(f)^m \cdot \HM^m(A).
\end{align*}
\end{beweis}

\begin{lemma}
\label{lem:3.6}
Ist $f:\MdR^m\to\MdR^n$ stetig, so ist
\[
\{ x\in\MdR^m: f \text{ ist in $x$ differenzierbar}\} \in\borel(\MdR^m).
\]
\end{lemma}

\begin{beweis}
Übung
\end{beweis}

\begin{definition}
Sei $m\le n$ und $f:\MdR^m\to\MdR^n$ differenzierbar in $x\in\MdR^m$. Dann heißt
\[
Jf(x) \da \llbracket Df_x \rrbracket
\]
die Jakobische (Jakobi-Determinante) von $f$ in $x$.
\end{definition}

Unser Ziel im folgenden ist es, $\{x\in\MdR^m: Jf(x) \ne 0\}$ in Teile zu zerlegen, auf denen $f$ injektiv ist und auf denen $Df_x$ kontrolliert ist.

\begin{lemma}
\label{lem:3.7}
Sei $m\le n$ und $f:\MdR^m\to\MdR^n$ eine stetige Abbildung. Sei $t>1$. Dann existiert eine abzählbare Borel-Überdeckung $\mathcal E$ der Menge
\[
B \da \{x\in\MdR^n: f \text{ ist differenzierbar in $x$, } Jf(x) \ne 0\}
\]
derart, dass für jedes $E\in\mathcal E$ gilt:
\begin{enumerate}
\item $f|_E$ ist injektiv.
\item Es existiert ein Automophismus $\tau_E : \MdR^m \to \MdR^m$ mit 
\begin{enumerate}
\item $\Lip(f|_E \circ \tau_E^{-1}) \le t$ und $\Lip(\tau_E \circ (f|_E)^{-1}) \le t$.
\item $t^{-1} \cdot\|\tau_E(v)\| \le \|Df_b(v)\| \le t \cdot\|\tau_E(v)\|$ für alle $b\in B$ und $v\in \MdR^m$.
\item $t^{-m} \cdot |\det(\tau_E)| \le J(f)|_E \le t^m \cdot |\det(\tau_E)|$.
\end{enumerate}
\end{enumerate}
\end{lemma}

\begin{beweis}
Sei $\ep>0$ mit $t^{-1}+ \ep < 1< t-\ep$. Wähle abzählbare dichte Teilmengen $C\subset \MdR^m$ und $T\subset \GL(m,\MdR)$. Für $c\in C$, $\tau\in T$ und $i\in\MdN$ sei $E(c,\tau,i)$ die Menge aller $b\in B \cap B(c,\frac1i)$, für die gilt:
\begin{enumerate}[\quad(a)]
\item $ (t^{-1} +\ep)\cdot \|\tau(v)\| \le \|Df_b(v)\| \le (t-\ep) \cdot \|\tau(v)\|$ für alle $v\in\MdR^m$ und
\item $\|f(a)- f(b) - Df_b(a-b)\| \le \ep \cdot \|\tau(a-b)\|$ für alle $a\in B(c,\frac1i)$.
\end{enumerate}
Sei $b\in E(c,\tau,i)$ und $a\in B(c,\frac1i)$. Dann gilt 
\begin{align*}
\|f(a) - f(b)\| &\le \|Df_b(a-b)\| + \ep \cdot \|\tau(a-b)\|\\
&\le (t-\ep)\|\tau(a-b)\| +\ep\cdot \|\tau (a-b)\| \\
&= t\cdot \|\tau(a-b)\| = t\cdot \|\tau(a) - \tau(b)\|
\end{align*}
und
\begin{align*}
\|f(a) - f(b)\| &\ge \|Df_b(a-b)\| - \ep \cdot \|\tau(a-b)\| \\
&\ge (t^{-1} + \ep )\|\tau(a-b)\| - \ep \cdot \|\tau(a-b)\| \\
&=  t^{-1} \cdot \|\tau(a)-\tau(b)\|.
\end{align*}
Zusammen erhält man für $a,b\in E(c,\tau,\frac1i)$:
\begin{align*}
t^{-1}\cdot \|\tau(a)-\tau(b)\| \le \|f(a) - f(b)\| \le t\cdot \|\tau(a)-\tau(b)\|.
\end{align*}
Dies zeigt (1). Für $a,b\in E(c,\tau,i)$ gilt also
\begin{align*}
\|f\circ \tau^{-1}(\tau(a)) - f\circ \tau^{-1}(\tau(b))\|\le t \cdot\|\tau(a)-\tau(b)\|,
\end{align*}
das heißt
\begin{align*}
\Lip(\underbrace{f|_{E(c,\tau,i)} \circ \tau^{-1}}_{\mathclap{\text{Abbildung auf }\tau(E(c,\tau,i))}}) \le t
\end{align*}
sowie
\begin{align*}
\|\tau\circ f^{-1}(f(a)) - \tau \circ f^{-1}(f(b))\| \le t \cdot \|f(a)-f(b)\|,
\end{align*}
das heißt
\begin{align*}
\Lip(\underbrace{\tau\circ(f|_{E(c,\tau,i)})^{-1}}_{\mathclap{\text{Abbildung auf }f(E(c,\tau,i))}}) \le t.
\end{align*}

Sei $b\in E(c,\tau,i)$. Sei $(e_1,\ldots,e_m)$ die Standardbasis. Da $Df_b$ injektiv ist, ist in der Zerlegung $Df_b=\rho\circ\sigma$, wobei $\sigma:\MdR^m\to\MdR^m$ symmetrisch und $\varrho:\MdR^m\to\MdR^n$ eine lineare Isometrie ist, die Abbildung $\sigma$ ein Automophismus. Wir schätzen ab
\begin{align*}
Jf(b) &= |\det(\sigma)| = |\det(\sigma\circ\tau^{-1})|\cdot |\det(\tau)|\\
&= |\det(\sigma\circ\tau^{-1}(e_1),\ldots,\sigma\circ\tau^{-1}(e_m))| \cdot |\det(\tau)| \\
&\le \prod_{i=1}^m \|\sigma\circ\tau^{-1}(e_i)\| \cdot |\det(\tau)| \\
&= \prod_{i=1}^m \underbrace{\|Df_b(\tau^{-1}(e_i))\|}_{\le t\cdot \|\tau(\tau^{-1}(e_i))\| = t}\cdot |\det(\tau)| \\
&\le t^m \cdot |\det(\tau)|.
\end{align*}
Analog erhält man:
\begin{align*}
(Jf(b))^{-1} &= |\det(\sigma)|^{-1} = |\det(\sigma^{-1})| = |\det(\tau\circ\sigma^{-1})| \cdot |\det(\tau^{-1})| \\
&\le |\det(\tau \circ \sigma^{-1}(e_1),\ldots,\tau\circ\sigma^{-1}(e_m))| \cdot |\det(\tau)|^{-1} \\
&\le \prod_{i=1}^m \|\tau \circ \sigma^{-1} (e_i)\|\cdot|\det(\tau)|^{-1} \\
&\le \prod_{i=1}^m t \cdot \|Df_b(\sigma^{-1}(e_i))\| \cdot |\det(\tau)|^{-1} \\
&= t^m \cdot |\det(\tau)|^{-1},
\end{align*}
das heißt
\begin{align*}
Jf(b) \ge t^{-1}\cdot |\det(\tau)|.
\end{align*}

Zu zeigen ist noch, dass die Mengen $E(c,\tau,i)$ mit $c\in C, \tau\in T,i\in \MdN$ die Menge $B$ überdecken. Sei hierzu $b\in B$. Dann ist $Df_b = \varrho \circ \sigma$ mit $\varrho, \sigma$ wie oben. Wähle $\delta>0$ so, dass
\begin{align*}
(1-\delta\cdot\|\sigma^{-1}\|)^{-1} < t - \ep \quad\text{und} \quad 1+\delta\cdot\|\sigma^{-1}\| < (t^{-1}+\ep)^{-1}
\end{align*}
und $\tau\in T$ so, dass $\|\sigma-\tau\|< \delta$. Dann gilt:
\begin{align*}
\|\tau\circ\sigma^{-1}\| 
&= \|\id_{\MdR^m}+\tau \circ \sigma^{-1} - \id_{\MdR^m} \|\\
&\le 1 + \| \tau \circ \sigma - \sigma\circ \sigma^{-1}\| \\
&=   1 + \|(\tau -\sigma)\circ \sigma^{-1}\| \\
&\le 1 + \|\tau -\sigma\|\cdot\|\sigma^{-1}\| \\
&<  1+\delta\cdot\|\sigma^{-1}\| \\
&< (t^{-1}+\ep)^{-1}
\end{align*}
und ähnlich folgt
\begin{align*}
\|\sigma\circ\tau^{-1}\| &\le 1 + \|\sigma - \tau\| \cdot \|\tau^{-1}\| \\
&< 1 + \delta\cdot\|\tau^{-1}\| \\
&= 1+\delta\cdot\|\sigma^{-1} \circ \sigma\circ\tau^{-1}\| \\
&\le 1 + \delta\cdot\|\sigma^{-1}\|\cdot \|\sigma\circ\tau^{-1}\| 
\end{align*}
und damit
\begin{align*}
\|\sigma\circ\tau^{-1}\| \le (1-\delta\cdot\|\sigma^{-1}\|)^{-1} < t-\ep.
\end{align*}

Wähle $i\in\MdN$ mit
\begin{align*}
\|f(a)-f(b) - Df_b(a-b)\| \le \ep \cdot \frac{\|a-b\|}{\Lip(\tau^{-1})}
\end{align*}
für alle $a\in B(b,\frac2i)$ und wähle $c\in C$ mit $\|c-b\| < \frac1i$. Für $v\in\MdR^m$ gilt dann
\begin{align*}
(t^{-1} + \ep)\cdot \|\tau(v)\| 
&= (t^{-1} + \ep)\cdot \|\tau\circ\sigma^{-1}(\sigma(v))\| \\
&\le (t^{-1} + \ep)\cdot \|\tau\circ\sigma^{-1}\| \cdot \|\sigma(v)\| \\
&\le \|\sigma(v)\| \\
&= \|Df_b(v)\|
\end{align*}
und
\begin{align*}
\|Df_b(v)\| &= \|\sigma(v)\| = \|\sigma\circ\tau^{-1}(\tau(v))\| \\
&\le \|\sigma\circ\tau^{-1}\| \cdot \|\tau(v)\| \\
&\le (t-\ep)\|\tau(v)\|.
\end{align*}
Dies zeigt Bedingung (a).

Für $a\in B(c,\frac1i) \subset B(b,\frac2i)$ ist 
\begin{align*}
\|f(a) - f(b) - Df_b(a-b)\|
&\le \ep\cdot\frac{\|a-b\|}{\Lip(\tau^{-1})} \\
&= \ep\cdot\frac{\|\tau^{-1}(\tau(a)) - \tau^{-1}(\tau(b))\|}{\Lip(\tau^{-1})} \\
&\le \ep\cdot \|\tau(a-b)\|.
\end{align*}
Dies zeigt Bedingung (b). Somit ist eine Überdeckung gegeben.

Nun kommen wir zur Borel-Messbarkeit der überdeckenden Mengen.

Sei $M$ die Menge aller $x\in \MdR^m$, für die $f$ in $x$ differenzierbar ist. Nach einer 
Übungsaufgabe, ist diese Menge messbar. Sei $\{x_i:i\in \MdN\}$ eine dichte Teilmenge von $\MdR^m$. 
Da $\tau $ und $Df_b(\cdot)$ stetig sind, gilt
\begin{align*}
\{b\in M:\text{ (a) gilt in } b\}=&\bigcap_{j\in\MdN}\{b\in M:(t^{-1}+\ep)\|\tau(x_j)\|\le \|Df_b(x_j)\|\le (t-\ep)\|\tau(x_j)\|\}\\
\in &\mathfrak{B}(\MdR^m)
\end{align*}
und
\begin{align*}
\{b\in M:\text{ (b) gilt in } b\}=&\bigcap_{j\in\MdN\atop x_j\in B(c,1/i)}\{b\in M:
\|f(x_j)-f(b)-Df_b(x_j-b)\|\le\ep\|\tau(x_j-b)\|\}\\
\in &\mathfrak{B}(\MdR^m).
\end{align*}
Zusammen ergibt dies die behauptete Messbarkeitsaussage.
\end{beweis}

\begin{satz}
Sei $m\le n$, $f:\MdR^m\to\MdR^n$ eine Lipschitzabbildung und $A\subset\MdR^m$ eine $\lambda^m$-messbare Menge. Dann gilt
\[
\int_A Jf(x) \lambda^m(dx) = \int_{\MdR^n} \#(A\cap f^{-1}(\{y\}))\,\HM^m(dy).
\]
\end{satz}

\begin{beweis}
Die Abbildung $y\mapsto \#(A\cap f^{-1}(\{y\}))$ ist messbar, falls $A$ eine Borelmenge ist. Zu der
gegebenen $\HM^m$-messbaren Menge $A$ gibt es eine Borelmenge $A'$ mit $A\subset A'$ und $\HM^m(A'\setminus A)=0$. 
Nach Proposition \ref{prop:3.5} ist $y\mapsto \#((A'\setminus A)\cap f^{-1}(\{y\}))$ 
$\HM^m$ fast überall die Nullfunktion und damit $\HM^m$-messbar. Hieraus folgt die Messbarkeit 
der Abbildung $y\mapsto \#(A\cap f^{-1}(\{y\}))$ allgemein. 


Aufgrund des Satzes von Rademacher, Proposition \ref{prop:3.5}, und der Regularität des Lebesguemaßes kann man annehmen, dass $A\in\borel(\MdR^m)$ und $f$ in jedem Punkt von $A$ differenzierbar ist. Ferner kann $\lambda^m(A) < \infty$ angenommen werden.
\begin{enumerate}[{Fall} (a):]
\item $A\subset \{x\in\MdR^m: Jf(x)\ne 0\}$. Sei $t>1$ beliebig. Wähle eine Borel-Überdeckung $\mathcal E$ von $B\da \{x\in\MdR^m: Jf(x) \ne 0\}$ wie in Lemma \ref{lem:3.7}. Sei $\mathcal G$ eine abzählbare Zerlegung von $A$ derart, das für jedes $G\in\mathcal G$ gilt: $G\in \borel(\MdR^m)$ und $G\subset E$ für ein $E\in\mathcal E$ mit  $\tau_E$ wie in Lemma \ref{lem:3.7}. Dann folgt
\begin{align*}
t^{-2m}\cdot\HM^m(f(G)) 
&= t^{-2m} \cdot \HM^m( ( (f|_E)\circ\tau_E^{-1}) (\tau_E(G)) ) \\
&\le t^{-2m} \cdot \underbrace{\Lip( (f|_E)\circ\tau_E^{-1})^m}_{\le t^m} \cdot \HM^m(\tau_E(G))\\
&\le t^{-m} \cdot \HM^m(\tau_E(G))\\
&= t^{-m} \cdot |\det(\tau_E)| \cdot \HM^m(G) \\
&=\int_G t^{-m} |\det(\tau_E)| d\lambda^m \\
&\le \int_G Jfd\lambda^m.
\end{align*}
Die Abschätzungen lassen sich nun in analoger Weise fortsetzen:
\begin{align*}
\int_G Jfd\lambda^m
&\le \int_G t^m |\det(\tau_E)| d\lambda^m \\
&= t^m \cdot |\det(\tau_E)| \cdot \HM^m(G) \\
&= t^m \cdot \HM^m(\tau_E(G)) \\
&= t^m \cdot \HM^m(\tau_E \circ (f|_E)^{-1} (f|_E(G))) \\
&\le t^m \cdot \Lip(\tau_E \circ (f|_E)^{-1})^m \cdot \HM^m(f(G)) \\
&\le t^{2m} \cdot \HM^m(f(G)).
\end{align*}
Da $f|_G$ für alle $G\in\mathcal G$ injektiv ist, erhält man zunächst
\begin{align*}
\sum_{G\in\mathcal G} \HM^m(f(G))
&= \sum_{G\in\mathcal G} \int_{\MdR^n} \ind_{f(G)} d\HM^m \\
&= \sum_{G\in\mathcal G} \int_{\MdR^n} \#(G\cap f^{-1}(\{y\}))\, \HM^m(dy) \\
&= \int_{\MdR^n} \#(A\cap f^{-1}(\{y\}))\, \HM^m(dy)
\end{align*}
und hiermit
\begin{align*}
t^{-2m}\cdot \int_{\MdR^n} \#(A\cap f^{-1}(\{y\}))\, \HM^m(dy) 
&\le \int_A Jfd\lambda^m \\
&\le t^{2m} \cdot \int_{\MdR^n} \#(A\cap f^{-1}(\{y\})) \, \HM^m(dy).
\end{align*}
Dies gilt für jedes $t>1$ und somit folgt die Gleichheit.
\item $A\subset \{x: Jf(x) = 0\}$. Sei $\ep>0$. Dann werden Lipschitzabbildungen $g,p$ erklärt;
\begin{align*}
p: {}&\MdR^n\times\MdR^m \to\MdR^n, & (y,z) &\mapsto y \\
g: {}&\MdR^m\to\MdR^n\times\MdR^m,  & x&\mapsto (f(x),\ep\cdot x).
\end{align*}
Dann ist $f= p\circ g$. Für $x\in A$ gilt $Dg_x(v) = (Df_x(v), \ep\cdot v)$ für $v\in\MdR^m$. Somit ist $Dg_x$ injektiv und daher $Jg(x)\ne 0$ für $x\in A$. Da $Jf(x)=0$ für $x\in A$ ist $q \da \dim(\Kern(Df_x)) \ge 1$.  Sei $(b_1,\ldots,b_q)$ eine Orthonormalbasis von $\Kern(Df_x)$ und $(b_1,\ldots,b_m)$ eine Ergänzung zu einer Orthonormalbasis von $\MdR^m$. Somit gilt
\begin{align*}
\|Dg_x(b_i)\| = \|(0,\ep \cdot b_i)\| = \ep
\end{align*}
für $i=1,\ldots,q$ und
\begin{align*}
\|Dg_x(b_j)\| = \|(Df_x(b_j),\ep\cdot b_j)\| \le \Lip(f) + \ep
\end{align*}
für $j=q+1,\ldots,m$. Nach Fall $(a)$ ist
\begin{align*}
\HM^m(f(A)) &= \HM^m(p\circ g(A))\\
&\le \HM^m(g(A)) = \int_{A} Jgd\lambda^m\\
&\le \ep^q\cdot(\Lip(f)+\ep)^{m-q}\cdot\lambda^m(A).
\end{align*}
Dies zeigt $\HM^m(f(A))=0$ und somit $\#(A\cap f^{-1}(\{y\})) = 0$ für $\HM^m$-fast alle $y\in\MdR^n$. Dies zeigt die Gleichheit auch in diesem Fall.
\end{enumerate}
\end{beweis}

\begin{korollar}
Seien $m\le n$, $f:\MdR^m\to\MdR^n$ eine Lipschitzabbildung und $h:\MdR^m\to\MdR$ eine Funktion, 
deren $\lambda^m$-Integral existiert. Dann gilt
\begin{align*}
\int_{\MdR^m}h(x)Jf(x)\, \lambda^m(dx)&=\int_{\MdR^n}\left(\sum_{x\in f^{-1}(\{y\})}h(x)\right)
\, \HM^m(dy)\\
&=\int_{\MdR^n}\int_{f^{-1}(\{y\})}h(x)\,\HM^0(dx)\, \HM^m(dy).
\end{align*}
\end{korollar}

\begin{beweis}
Übung
\end{beweis}


\emph{Anwendung:}
Sei \(m \leq n\), \(A \subset \MdR^m\), \(f: \MdR^m \to \MdR^n\) sei \(C^1\) und injektiv. Dann gilt:
\[
\HM^m(f(A)) = \int_A Jf(x) \,\HM^m(dx) = \int_A \sqrt{g(x)}\, \HM^m(dx)
\]
mit \(g(x) = \det( (\langle D_i f(x), D_j f(y) \rangle)_{i,j=1,\dots,m} )\).
Ist \(f\) nicht injektiv, so nennt man
\[
\int_{\MdR^n} \#(A \cap f^{-1}(\{y\}))\, \HM^m(dy) \quad \geq \HM^m(f(A))
\]
die Hausdorff-Fläche von  $f|_A$.


\section{Die Koflächenformel}

Sei jetzt \(m \geq n\) und \(f: \MdR^m \to \MdR^n\). Sei ferner \(g: \MdR^m \to [0,\infty]\) eine $\HM^m$-messbare 
Abbildung. Dann besagt die Koflächenformel, dass
\[
\int_{\MdR^m} g(x) Jf(x) \,\HM^m(dx) = \int_{\MdR^n} \int_{f^{-1}(\{y\})} g(x)\, \HM^{m-n}(dx) \,\HM^n(dy)
\]
gilt. Die Jakobische $Jf$ von $f$ wird nachfolgend erklärt. Im Spezialfall \(g(x) = \ind_A(x)\) mit einer $\HM^m$-messbaren Menge \(A \subset \MdR^m\) besagt dies gerade
\[
\int_A Jf(x) \,\HM^m(dx) = \int_{\MdR^n} \HM^{m-n}(A \cap f^{-1}(\{y\})) \,\HM^n(dy).
\]
Aus diesem Spezialfall erhält man umgekehrt die allgemeine Aussage durch die üblichen Routineargumente.
\begin{beispiel}
Für die Abbildung 
\(d: \MdR^m \to [0,\infty),\, x \mapsto \|x\|\) gilt \( d^{-1}(\{r\}) = \{x\in\MdR^m \colon \|x\|=r\}\). 
Die Berechnung der Jakobischen $Jd$ von $d$ erfolgt in den Übungen.
\end{beispiel}

\begin{lemma}\label{lem:3.10}
Sei \(k\geq m\geq n\), \(A\subset\MdR^k\) und \(f: \MdR^k \to \MdR^n\) eine Lipschitz-Abbildung. Dann gilt
\[
\int_{\MdR^n}^* \HM^{m-n}(A \cap f^{-1}(\{y\}))\, \HM^n(dy) \; \leq \; \frac{\alpha(m-n) \alpha(n)}{\alpha(m)} \Lip(f)^m \cdot \HM^m(A).
\]
\end{lemma}
\begin{beweis}
Für \(j \in \MdN\) gilt \(\HM^m_{\frac1j}(A) \leq \HM^m(A) \leq \HM^m(A) + \frac1j\). Dann existiert eine Folge \(\mathcal B_j \in \bar\Omega_{\frac1j}(A)\) mit
\[
\HM^m_{\frac1j}(A) \leq \sum_{B\in\mathcal B_j} \frac{\alpha(m)}{2^m} \diam (B)^m \leq \HM^m(A)+\frac1j.
\]
Für \(B \in \mathcal B_j\) ist \(f(B)\) kompakt, also Borelmenge. Erkläre 
$$g_B: \MdR^n \to \MdR,\qquad  y \mapsto \frac{\alpha(m-n)}{2^{m-n}}\cdot\diam(B)^{m-n}\cdot\ind_{f(B)}(y).
$$ 
Insbesondere ist \(g_B\) eine \(\HM^n\)-messbare Funktion und 
$$
\diam(B\cap f^{-1}(\{y\})) \leq \diam(B) \cdot \ind_{f(B)}(y) \leq \frac1j. 
$$
Es folgt
\[
\HM_{\frac1j}^{m-n}(A\cap f^{-1}(\{y\})) \leq \sum_{B \in \mathcal B_j} \frac{\alpha(m-n)}{2^{m-n}} \diam(B\cap f^{-1}(\{y\}))^{m-n} \leq \sum_{B \in \mathcal B_j} g_B(y).
\]
Mit dem Lemma von Fatou und dem Satz von der monotonen Konvergenz  erhält man nun
\begin{align*}
\int_{\MdR^n}^* \HM^{m-n}(A \cap f^{-1}(\{y\})) \,\HM^n(dy) &= \int_{\MdR^n}^* \lim_{j\to\infty} \HM_{\frac1j}^{m-n}(A\cap f^{-1}(\{y\})) \,\HM^n(dy) \\
&\leq \int_{\MdR^n} \liminf_{j\to\infty} \sum_{B\in\mathcal B_j} g_B(y)\, \HM^n(dy) \\
&\leq \liminf_{j\to\infty} \int_{\MdR^n} \sum_{B\in\mathcal B_j} g_B(y) \,\HM^n(dy) \\
&= \liminf_{j\to\infty} \sum_{B\in\mathcal B_j} \int_{\MdR^n} g_B(y) \,\HM^n(dy) \\
&= \liminf_{j\to\infty} \sum_{B\in\mathcal B_j} \frac{\alpha(m-n)}{2^{m-n}} \diam(B)^{m-n} \cdot \HM^n(\underbrace{f(B)}_{\subset \MdR^n}).
\end{align*}
Mit Hilfe der isodiametrichen Ungleichung ergibt dies
\begin{align*}
\int_{\MdR^n}^* \HM^{m-n}(A \cap f^{-1}(\{y\})) \,\HM^n(dy) 
&\leq \liminf_{j\to\infty} \sum_{B\in\mathcal B_j} \frac{\alpha(m-n)}{2^{m-n}}\diam(B)^{m-n} \cdot \frac{\alpha(n)}{2^n}\diam(f(B))^n \\
&\leq \liminf_{j\to\infty} \frac{\alpha(m-n)\alpha(n)}{\alpha(m)} \Lip(f)^n \cdot \underbrace{\sum_{B\in\mathcal B_j} \frac{\alpha(m)}{2^m}\diam(B)^m}_{\le \HM^m(A) + \frac1j} \\
&\le \frac{\alpha(m-n)\alpha(n)}{\alpha(m)}\Lip(f)^n\HM^m(A).
\end{align*}
\end{beweis}

\begin{lemma}\label{lem:3.11}
Sei \(m\geq n\) und \(f:\MdR^m\to\MdR^n\) eine Lipschitz-Abbildung. Sei ferner \(A \subset \MdR^m\) eine \(\HM^m\)-messbare Menge. Dann gilt:
\begin{enumerate}[(1)]
\item \(A\cap f^{-1}(\{y\})\) ist \(\HM^{m-n}\)-messbar für \(\HM^n\)-fast-alle \(y \in \MdR^n\).
\item \(y\mapsto\HM^{m-n}(A\cap f^{-1}(\{y\}))\) ist \(\HM^n\)-messbar.
\end{enumerate}
\end{lemma}
\begin{beweis}
Sei \(A\) kompakt. Dann ist \(A\cap f^{-1}(\{y\})\) kompakt, also Borelmenge. Sei \(t>0\) beliebig. Für \(j \in \MdN\) sei \(U_j\) die Menge aller \(y\in\MdR^n\), für die es eine endliche, offene Überdeckung \(\mathcal G\) von \(A \cap f^{-1}(\{y\})\) gibt mit \(\diam(G) \leq \frac1j \, (G \in \mathcal G)\) und 
$$
\sum_{G\in \mathcal G} \frac{\alpha(m-n)}{2^{m-n}} \diam(G)^{m-n} \leq t + \frac1j.
$$ 
Dann bestätigt man leicht
\[
\{y\in\MdR^n \colon \HM^{m-n}(A\cap f^{-1}(\{y\})) \leq t\} = \bigcap_{j\in\MdN} U_j.
\]
Wir zeigen, dass \(U_j\) eine offene Menge und damit eine Borelmenge ist. Sei hierzu \(y\in U_j\) und  \(\mathcal G\) eine offene, endliche Überdeckung von \(A \cap f^{-1}(\{y\})\). Da \(A\) kompakt ist, ist \(A \setminus \bigcup_{G \in \mathcal G} G\) kompakt und somit auch \(f(A\setminus \bigcup_{G\in\mathcal G} G)\). Daher ist \(f(A \setminus \bigcup_{G\in\mathcal G} \subset \MdR^n)^c\) offen.
\par
\emph{Behauptung}: Für \(z \in \MdR^n\) gilt: 
$$
z \in f(A \setminus \bigcup_{G\in\mathcal G})^c \Leftrightarrow A \cap f^{-1}(\{z\}) \subseteq \bigcup_{G\in\mathcal G} G.
$$ 
Dies ist leicht einzusehen. 


Eine zweimalige Anwendung der Behauptung zeigt 
\(y\in f(A \cap \bigcup_{G\in\mathcal G} G)^c \subseteq U_j\). Also ist \(U_j\) offen.
\par
Sei \(A \subset \MdR^m\) nun eine beliebige \(\HM^m\)-messbare Menge. Es genügt, \(\HM^m(A)<\infty\) zu betrachten. Zu $A$ gibt es eine aufsteigende Folge kompakter Mengen \( A_i  \subset A\) mit \(\HM^m(A \setminus \bigcup_{i\geq1}A_i)=0\). Lemma \ref{lem:3.10} zeigt 
\[
\HM^{m-n}((A\setminus\bigcup_{i\geq1}A_i)\cap f^{-1}(\{y\})) = 0
\]
für \(\HM^n\)-fast-alle \(y\in\MdR^n\). Hieraus folgt (1) und wegen 
$$
\HM^{m-n}(A\cap f^{-1}(\{y\})) = \lim_{i\to\infty} \HM^{m-n}(A_i\cap f^{-1}(\{y\}))
$$ 
auch (2).
\end{beweis}

\emph{Notation:} Zu \(f \in \L(\MdR^m,\MdR^n)\) ist \(f^* \in \L(\MdR^n,\MdR^m)\) die adjungierte Abbildung, die 
durch die Bedingung \(\langle f(x),y\rangle = \langle x,f^*(y)\rangle\) für \(x\in\MdR^m, y\in\MdR^n\) 
festgelegt ist.

\begin{lemma}\label{lem:3.12}
Sei \(f:\MdR^m\to\MdR^n\) eine lineare Abbildung und \(m\geq n\). Dann gilt:
\begin{enumerate}[(1)]
\item \(f^{**} = f\).
\item \(f\in\L(\MdR^m,\MdR^n),\, g\in\L(\MdR^n,\MdR^p) \; \Rightarrow \; (g\circ f)^* = f^* \circ g^* \in \L(\MdR^p,\MdR^m)\).
\item Ist \(\varrho \in \L(\MdR^n,\MdR^m)\) orthogonal, so gilt \(\varrho^* \circ \varrho = \id_{\MdR^n}\) und \(\Bild(\varrho) = \Kern(\varrho^*)^\perp\).
\item Zu \(f\) existiert \(\sigma \in \L(\MdR^n,\MdR^n)\) symmetrisch und \(\varrho\in \L(\MdR^n,\MdR^m)\) orthogonal mit \(f = \sigma \circ \varrho^*\).
\item Ist \(h \in \L(\MdR^m,\MdR^m)\), so gilt \(\llbracket h \rrbracket = \llbracket h^* \rrbracket = |\det(h)| = |\det(h^*)|\).
\item Setze \(\llbracket f \rrbracket \da \llbracket f^* \rrbracket\). Dann gilt:
	\begin{enumerate}[(a)]
	\item \(\llbracket f \rrbracket = 0\), falls \(\dim(f(\MdR^m)) < n\).
	\item Ist \(\dim(f(\MdR^m)) = n\) und \((b_1,\dots,b_n)\) eine Orthonormalbasis von \(\Kern(f)^\perp\), so gilt:
\[
\llbracket f \rrbracket = |\det(f(b_1),\dots,f(b_n))|.
\]
	\end{enumerate}
\item Ist \((a_1,\dots,a_n)\) eine ONB von \(\MdR^n\), so gilt:
\[
\llbracket f \rrbracket = \sqrt{\det(f\circ f^*)} = \sqrt{
\det(\langle f^*(a_i),f^*(a_j)\rangle_{i,j=1,\dots,n}}).
\]
\end{enumerate}
\end{lemma}
\begin{beweis}
Übung
\end{beweis}

\begin{lemma}\label{lem:3.13}
Sei \(m\leq n\) und seien \(h: \MdR^m\to\MdR^n,\, \tilde h: \MdR^n\to\MdR^m\) Lipschitz-Abbildungen. Sei ferner \(E \subset \{x\in\MdR^m\colon \tilde h \circ h (x) = x \}\) eine \(\HM^m\)-messbare Menge. Dann gibt es eine \(\HM^m\)-messbare Menge \(S_E \subset E\) mit \(\HM^m(E\setminus S_E) = 0\), so dass für \(x \in S_E\) gilt:
\begin{itemize}
\item \(h\) ist in \(x\) differenzierbar.
\item \(\tilde h\) ist in \(h(x)\) differenzierbar.
\item \(D\tilde h_{h(x)} \circ D  h_x = \id_{\MdR^m}\).
\end{itemize}
\end{lemma}
\begin{beweis}
Sei \(Z \da \{ x\in \MdR^m \colon \tilde h\circ h (x) - x = 0 \} = \{ x\in\MdR^m \colon (\tilde h\circ h-\id)(x) = 0 \}\). Dann gilt (nach Übung) für \(\HM^m\)-fast-alle \(x \in Z\), dass \(D(\tilde h \circ h - \id)_x = 0\), d.h. \(D(\tilde h \circ h)_x = \id_{\MdR^m}\). Ist \(E \subset Z\), so gilt dies auch für \(\HM^m\)-fast-alle \(x\in E\). 

Sei \(F \da \{x\in\MdR^m \colon h \text{ ist differenzierbar in } x\}\), \(G \da \{y\in\MdR^n \colon \tilde h \text{ ist differenzierbar in } y\}\) und \(D \da F \cap \{x\in\MdR^m \colon h(x)\in G\}\). Dann gilt:
\[
E\setminus D = (E\setminus F) \cup (E \setminus \{x\in\MdR^m \colon h(x) \in G\}) = (E\setminus F) \cup \{x\in E \colon h(x) \notin G\},
\]
wobei aufgrund des Satzes von Rademacher \(\HM^m(E\setminus F) = 0\) und wegen \(\{x\in E \colon h(x) \notin G\} \subset \tilde h(\MdR^n\setminus G)\) auch 
$$
\HM^m(\{x\in E \colon h(x) \notin G\}) \leq \HM^m(\tilde h(\MdR^n\setminus G)) \leq \Lip(\tilde h)^m \cdot \HM^m(\MdR^n\setminus G) = 0.
$$
Hieraus folgen leicht alle Behauptungen.
\end{beweis}

\begin{lemma}\label{lem:3.14}
Sei \(m\geq n\) und sei \(f: \MdR^m\to\MdR^n\) stetig. Dann existiert eine abzählbare Borelüberdeckung \(\mathcal E\) von \(B \da \{x\in\MdR^m \colon f\text{ differenzierbar in } x \text{ und } Df_x(\MdR^m)=\MdR^n\}\) derart, dass für jedes \(E \in \mathcal E\) eine (lineare) Orthogonalprojektion \(p_E: \MdR^m \to \MdR^{m-n}\) und Lipschitzabbildungen \(h_E: \MdR^m \to \MdR^n \times \MdR^{m-n},\, \tilde h_E:\MdR^n\times \MdR^{m-n} \to \MdR^m\) existieren mit
\[
h_E(x) = (f(x),p_E(x)) \quad \text{und} \quad \tilde h_E \circ h_E(x) = x \quad \text{ für alle } x\in E.
\]
\end{lemma}

\begin{beweis}
Für $\gamma \in \Lambda(m,m-n)$ sei $p_\gamma:\MdR^{m}\to \MdR^{m-n}$ gegeben durch
\[
p_\gamma(x_1,\ldots,x_m) \da (x_{\gamma(1)},\ldots,x_{\gamma(m-n)}).
\]
Ferner sei $h_\gamma: \MdR^m \to \MdR^n \times \MdR^{m-n}$ gegeben durch $h_\gamma(x) \da (f(x),p_\gamma(x))$. Setze
\[A_\gamma \da \{x\in\MdR^m : \text{$h_\gamma$ ist differenzierbar in $x$ und $Dh_\gamma(x)$ ist injektiv}\}.\]
Die Abbildung $f$ ist differenzierbar in $x$ genau dann, wenn $h_\gamma$ in $x$ differenzierbar ist und $\Kern( (Dh_\gamma)_x) = \Kern(Df_x) \cap \Kern(p_\gamma)$. Hieraus folgt leicht
\[
B  = \bigcup_{\mathclap{\gamma\in\Lambda(m,m-n)}} A_\gamma.
\]

Nach Lemma \ref{lem:3.7} gibt es zu jedem $t>1$ eine abzählbare Borel-Überdeckung $\mathcal E_\gamma$ von $A_\gamma$ derart, dass für jedes $E\in\mathcal E_\gamma$ ein Automorphismus $\tau_E : \MdR^m \to \MdR^m$ existiert, so dass gilt:  $(h_\gamma)|_E$ ist injektiv und $\Lip( (h_\gamma)|_E \circ \tau_E^{-1}) \le t$ sowie $\Lip( \tau_E \circ ( (h_\gamma)|_E )^{-1}) \le t$.

Für $x,y\in E$ gilt dann
\begin{align*}
\|h_\gamma(x) - h_\gamma(y) \| 
&= \| (h_\gamma)|_E \circ \tau_E^{-1}( \tau_E^{}(x)) - (h_\gamma)|_E \circ \tau_E^{-1}( \tau_E^{}(y) )\| \\
&\le t \cdot \| \tau_E^{}(x) - \tau_E^{}(y)\| \le t \cdot \Lip(\tau_E^{}) \cdot \|x-y\|
\end{align*}
und für $u,v\in h_\gamma(E)$ gilt
\begin{align*}
\|(h_\gamma|_E)^{-1}(u) - (h_\gamma|_E)^{-1}(v)\| 
&= \| \tau_E^{-1} \circ \tau_E^{} \circ (h_\gamma|_E)^{-1}(u) - \tau_E^{-1} \circ \tau_E^{} \circ (h_\gamma|_E)^{-1}(v)  \| \\
&\le \Lip(\tau_E^{-1}) \cdot t \cdot \|u-v\|.
\end{align*}
Also sind $h_\gamma|_E$ und $(h_\gamma|_E)^{-1}$ Lipschitzabbildungen. Somit existieren Lipschitzfortsetzungen $h$ von $(h_\gamma|_E)$ und $\tilde h$ von $(h_\gamma|_E)^{-1}$ mit $h:\MdR^m \to \MdR^n\times\MdR^{m-n}$ und $\tilde h: \MdR^n \times \MdR^{m-n}\to \MdR^m$, wobei $h|_E = (h_\gamma)|_E$ und $\tilde h|_{h(E)} = (h_\gamma|_E)^{-1}|_{h_\gamma(E)}$. Daher ist $h(x)=(f(x),p_\gamma(x))$ für $x\in E$ und $\tilde h \circ h(x)=x$ für $x\in E$. Hieraus folgt die Behauptung.
\end{beweis}

Der folgende Hilfssatz stellt gewissermaßen eine lokale Version der Koflächenformel dar. 
Die hierbei auftretenden Annahmen sind nach dem vorangehenden Hilfssatz stets erfüllbar. 
Zusammen erhalten wir dann schließlich die allgemeine Aussage.

\begin{lemma}
\label{lem:3.15}
Sei $m>n$, $f:\MdR^m\to\MdR^n$ eine Lipschitzabbildung und $E\subset \MdR^m$ eine Borelmenge derart, dass für alle $x\in E$ die Abbildung $f$ differenzierbar in $x$ ist und $Df_x$ surjektiv. Sei $p:\MdR^m\to\MdR^{m-n}$ eine Orthogonalprojektion und $h:\MdR^m \to \MdR^n \times \MdR^{m-n}$, $\tilde h: \MdR^n\times\MdR^{m-n}\to\MdR^m$ seien Lipschitzabbildungen so dass $\tilde h \circ h(x) = x$ für $x\in E$ und $h(x) = (f(x),p(x))$ für $x\in E$. Dann gilt für alle $A\subset E$ mit $A\in \A_{\HM^m}$
\[
\int_A Jf(x) \,\HM^m(dx) = \int_{\MdR^n} \HM^{m-n}(A \cap f^{-1}(\{y\}))\, \HM^n(dy).
\]
\end{lemma}

\begin{beweis}
Nach Voraussetzung sind $h|_E$ und $\tilde h|_{h(E)}$ injektiv. Für $y\in\MdR^n$ gilt
\begin{align*}
h(E \cap f^{-1}(\{y\}))
&= \{(f(x),p(x)) : x\in E\cap f^{-1}(\{y\})\} \\
&= \{(y,p(x)) : x \in E\cap f^{-1}(\{y\})\} \\
&= \{y\}\times p(E\cap f^{-1}(\{y\})).
\end{align*}

Für $y\in\MdR^n$ setzte $\tilde h_y: \MdR^{m-n}\to\MdR^m$, $\tilde h_y (z) \da \tilde h (y,z)$ für $z\in\MdR^{m-n}$. Da $\tilde h$ auf der Menge $\{y\}\times p(E \cap f^{-1}(\{y\})) \subset h(E)$ injektiv ist, ist $\tilde h_y$ injektiv auf $p(E\cap f^{-1}(\{y\}))$. Ferner gilt
\begin{align*}
\tilde h_y(p(E\cap f^{-1}(\{y\})))
&=\tilde h(\{y\}\times p(E\cap f^{-1}(\{y\}))\\
&= \tilde h (h (E\cap f^{-1}(\{y\})))\\
&= E\cap f^{-1}(\{y\}) \tag{$*$}.
\end{align*}

Nach Lemma $\ref{lem:3.14}$ ist für $\HM^m$-fast-alle $x\in E$ die Abbildung $h$ in $x$ und die Abbildung $\tilde h$ in $h(x)$ differenzierbar und $D\tilde h_{h(x)} = (Dh_x)^{-1}$. Für $\HM^m$-fast-alle $x\in E$ existiert somit die Abbildung $L_x:\MdR^{m-n}\to\MdR^m$ mit $L_x\da D(\tilde h_{f(x)})_{p(x)}$, wobei die Existenz und die erste  Gleichung
$$
L_x= D(\tilde h_{f(x)})_{p(x)} 
= D\tilde h_{(f(x),p(x))} \circ (0_{n,n-m}, \id_{\MdR^{m-n}})=(D h_x)^{-1}\circ (0_{n,n-m}, \id_{\MdR^{m-n}})
$$
aus der Kettenregel  (betrachte hierzu: $z\mapsto (y,z)\mapsto \tilde h(y,z)$) folgt. Wegen 
$Dh_x = (Df_x,p)$ erhält man  
\begin{align*}
L_x(\MdR^{m-n}) = (Dh_x)^{-1}(\{0_{\MdR^n}\} \times \MdR^{m-n}) = \Kern(Df_x).
\end{align*}
Dies zeigt insbesondere, dass $L_x$ maximalen Rang hat. 
Sei nun $(b_1,\ldots,b_m)$ eine Orthonormalbasis von $\MdR^m$, so dass $(b_1,\ldots,b_{m-n})$ eine Orthonormalbasis von $\Kern(Df_x)$ ist, das heißt $(b_{m-n+1},\ldots,b_m)$ ist eine Orthonormalbasis von $\Kern(Df_x)^\bot$. Damit folgt
\begin{align*}
Jh(x) 
&= |\det(Dh_x(b_1),\ldots,Dh_x(b_m))|\\
&= |\det\big( (0,p(b_1)),\ldots,(0,p(b_{m-n})), (Df_x(b_{m-n+1}),p(b_{m-n+1})),\ldots,(Df_x(b_{m}),p(b_{m}))\big)|\\
&= |\det\big( (0,p(b_1)),\ldots,(0,p(b_{m-n})), (Df_x(b_{m-n+1}),0),\ldots,(Df_x(b_{m}),0)\big)|\\
&= |\det\big( p(b_1),\ldots,p(b_{m-n})\big)| \cdot Jf(x) ,
\end{align*}
wobei wir verwendet haben, dass $(0,p(b_1)),\ldots,(0,p(b_{m-n}))$ linear unabhängig sind. 
Für $v\in\MdR^{m-n}$ gilt
\begin{align*}
(0_{n,m-n}, \id_{\MdR^{m-n}})(v) = Dh_x \circ L_x(v) = (Df_x(L_x(v)), p(L_x(v))) 
\end{align*}
und daher
\begin{align*}
p\circ L_x = \id_{\MdR^{m-n}},
\end{align*}
das heißt $\det(p\circ L_x)=1$. Sei nun $\sigma:\MdR^{m-n} \to \MdR^{m-n}$ eine symmetrische und $\varrho:\MdR^{m-n}\to\MdR^m$ eine orthogonale Abbildung, so dass $L_x=\varrho \circ \sigma$. Da $L_x$ 
maximalen Rang hat, ist $\sigma$ ein Automorphismus. Daher ist $\Kern(Df_x) = L_x(\MdR^{m-n}) = \varrho\circ\sigma(\MdR^{m-n}) = \varrho(\MdR^{m-n})$. Da $\varrho$ orthogonal ist, gibt es eine Orthonormalbasis $\bar b_1,\ldots,\bar b_{m-n}$ in $ \MdR^{m-n}$ mit $\varrho(\bar b_i) = b_i$. Somit folgt
\begin{align*}
1 &= \det(p\circ L_x) = \det(p\circ\varrho\circ\sigma) = \det(p\circ \varrho) \cdot \det(\sigma) \\
&= |\det( p\circ\varrho(\bar b_1), \ldots, p\circ\varrho(\bar b_{m-n}))| \cdot \llbracket L_x\rrbracket \\
&= |\det( p(b_1), \ldots, p(b_{m-n}))| \cdot J(\tilde h_{f(x)})(p(x)).
\end{align*}
Dies ergibt schließlich
\begin{align*}
Jf(x) = Jh(x) \cdot J(\tilde h_{f(x)}) (p(x)).
\end{align*}

Da $h$ auf $E$ und $\tilde h$ auf $h(E)$ injektiv sind, folgt für $A\subset E$ mit $A\in\A_{\HM^m}$ durch zweimalige Anwendung der Flächenformel
\begin{align*}
\int_A Jf(x) \, \HM^m(dx) 
&= \int_{\MdR^m} \ind_A(x) \cdot J(\tilde h_{f(x)})(p(x)) \cdot J h(x)\, \HM^m(dx) \\
&= \int_{\MdR^n\times\MdR^{m-n}} \ind_{h(A)} (y,z) \cdot J(\tilde h_y)(z) \, (\HM^n\otimes \HM^{m-n})(d(y,z)) \\
&= \int_{\MdR^n} \int_{p(A\cap f^{-1}(\{y\}))} J(\tilde h_y)(z) \, \HM^{m-n}(dz)\, \HM^n(dy) \\
&= \int_{\MdR^n} \HM^{m-n}(\underbrace{\tilde h_y (p(A\cap f^{-1}(\{y\})))}_{\gleichwegen{(*)} A\cap f^{-1}(\{y\})}) \, \HM^n(dy) \\
&= \int_{\MdR^n} \HM^{m-n}(A\cap f^{-1}(\{y\}))\, \HM^n(dy).
\end{align*}
\end{beweis}

\begin{satz}
\label{satz:3.16}
Seien $m\ge n$ und $f:\MdR^m \to \MdR^n$ eine Lipschitzabbildung. Für jede $\HM^m$-messbare Menge $A\subset\MdR^m$ gilt 
\[
\int_A Jf(x) \,\HM^m(dx) = \int_{\MdR^n} \HM^{m-n} ( A\cap f^{-1}(\{y\})) \, \HM^n(dy).
\]
\end{satz}

\begin{korollar}
\label{kor:3.17}
Seien $m\ge n$, $f:\MdR^m \to \MdR^n$ eine Lipschitzabbildung und $h:\MdR^m\to [0,\infty]$ eine $\HM^m$-messbare Abbildung. Dann gilt
\[
\int h(x) Jf(x) \, \HM^m(dx) =
\int_{\MdR^n} \int_{f^{-1}(\{y\})} h(x) \, \HM^{m-n}(dx) \, \HM^n(dy).
\]
\end{korollar}

\begin{beweis}[von Satz \ref{satz:3.16}]
Ist $\HM^m(A) = 0$, so sind beide Seiten der behaupteten Gleichung Null (Lemma \ref{lem:3.10}). Daher können wir den Beweis wie früher auf den Fall $\HM^m(A)<\infty$, $f$ differenzierbar in allen $x\in A$ sowie $A\in \borel(\MdR^m)$ reduzieren.
\begin{enumerate}[\quad(a)]
\item Sei $Jf(x) > 0$ für alle $x\in A$. Also ist $Df_x(\MdR^m) = \MdR^n$. Dann existiert eine abzählbare Borel-Überdeckung $\mathcal E$ von $A$ derart, dass für jedes $E\in\mathcal E$ eine Orthogonalprojektion $p:\MdR^m\to\MdR^{m-n}$ und Lipschitzabbildungen $h:\MdR^m\to\MdR^n\times\MdR^{m-n}$, $\tilde h:\MdR^n\times\MdR^{m-n}\to \MdR^m$ existieren mit $h(x) = (f(x), p(x))$ und $\tilde h \circ h (x) = x$ für $x\in E$. Wähle eine abzählbare Zerlegung $\mathcal G$ von $A$, so dass für jedes $G\in\mathcal G$ ein $E\in\mathcal E$ existiert mit $G\subset E$. Aus Lemma \ref{lem:3.15} folgt nun
\begin{align*}
\int_A Jf(x) \,\HM^m(dx) 
&= \sum_{G\in\mathcal G} \int_G Jf(x) \, \HM^m(dx) \\
&= \sum_{G\in\mathcal G} \int_{\MdR^n} \HM^{m-n}(G \cap f^{-1}(\{y\})) \, \HM^n(dy) \\
&= \int_{\MdR^n} \HM^{m-n}( A \cap f^{-1}(\{y\}))\, \HM^n(dy).
\end{align*}
\item Sei  $Jf(x) = 0$ für alle $x\in A$, das heißt $\text{Rang}(Df_x(\MdR^m))<n$. Sei $\ep>0$ beliebig 
vorgegeben. Definiere Abbildungen $g:\MdR^m\times \MdR^n\to\MdR^n$ und $p:\MdR^m\times \MdR^n\to\MdR^n$ durch
$$
g(x,y)\da f(x)+\ep y\qquad\text{und}\qquad p(x,y)\da y\qquad \text{ für } x\in\MdR^m, y\in \MdR^n.
$$
Für $x\in A$ und $y\in \MdR^n$ erhält man
$$
Dg_{(x,y)}(v,w)=Df_x(v)+\ep w,\qquad v\in\MdR^m, w\in \MdR^n
$$
und folglich
$$
\| Dg_{(x,y)}(v,w)\|\le \|Df_x(v)\|+\ep \|w\|\le \text{Lip}(f)\|v\|+\ep\|w\|\le (\text{Lip}(f)+\ep)\|(v,w)\|,
$$
also 
$$
\| Dg_{(x,y)}(v,w)\|\le\text{Lip}(f)+\ep.
$$
Ferner ist 
$$
\text{Kern}(Dg_{(x,y)})=\left\{ \left(v,-\frac{1}{\ep}Df_x(v)\right):v\in \MdR^m\right\},
$$
also $\text{dim}(\text{Kern}(Dg_{(x,y)}))=m$ und $\text{dim}(\text{Bild}(Dg_{(x,y)}))=n$. 
Unter (b) gibt es einen Vektor $w\in Df_x(\MdR^m)^\perp$ mit $\|w\|=1$, und wir können hiermit 
$b_1\da \left(\frac{1}{\ep}Df_x^*(w),w\right)\in\MdR^m\times\MdR^n$ definieren. Dann gilt
$$
\|b_1\|^2=\left\langle \frac{1}{\ep}Df_x^*(w),\frac{1}{\ep}Df_x^*(w)\right\rangle+\langle w,w\rangle=
\frac{1}{\ep^2}\langle w,Df_x\circ Df_x^*(w)\rangle+1=1,
$$
wobei
$$ 
\|Df_x\circ Df_x^*(w)\|^2=\langle \underbrace{w}_{\in Df_x(\MdR^m)^\perp},\underbrace{Df_x\circ Df_x^*\circ Df_x\circ Df_x^*(w)}_{\in Df_x(\MdR^m)}\rangle=0,
$$
also $Df_x\circ Df_x^*(w)=0$ verwendet wurde. Wir erhalten $b_1\in \text{Kern}(Dg_{(x,y)})^\perp$, da für alle 
$v\in\MdR^m$ gilt
$$
\left\langle b_1,(v,-\frac{1}{\ep}Df_x(v))\right\rangle=\frac{1}{\ep}\langle Df_x^*(w),v\rangle -\frac{1}{\ep}\langle
w,Df_x(v)\rangle =0,
$$
und ferner
$$
\|Dg_{(x,y)}(b_1)\|=\left\|\frac{1}{\ep} Df_x\circ Df_x^*(w)+\ep w\right\|=\ep.
$$
Wir ergänzen nun $b_1$ zu einer Orthonormalbasis $(b_1,\ldots,b_n)$ von $\text{Kern}(Dg_{(x,y)})^\perp$. 
Hierfür gilt die Abschätzung
$$
Jg(x,y)=|\det(Dg_{(x,y)}(b_1),\ldots, Dg_{(x,y)}(b_n))|=
\prod_{i=1}^n\|Dg_{(x,y)}(b_i)\|\le \ep(\text{Lip}+\ep)^{n-1}.
$$
Die Translationsinvarianz des Hausdorff-Maßes zeigt, dass
$$
\int_{\MdR^n}\HM^{m-n}(A\cap f^{-1}(\{y-\ep z\}))\, \HM^n(dy)
$$
von $\ep$ und $z$ unabhängig ist. Setze $B\da A\times B^n(0,1)$, wobei $B^n(0,1)$ die Einheitskugel im $\MdR^n$ ist. 
Hierbei gilt wegen $\{(x,z):x\in A,f(x)+\ep z=y\}=\{(x,z):x\in A\cap f^{-1}(\{y-\ep z\})\}$ 
$$
B\cap g^{-1}(\{y\})\cap p^{-1}(\{z\})
=\begin{cases}
\{(x,z):x\in A\cap f^{-1}(\{y-\ep z\})\},& \text{falls } z\in B^n(0,1),\\
\emptyset,&\text{sonst}.
\end{cases}
$$
Es bezeichne $\alpha(n)$ das Volumen der $n$-dimensionalen Einheitskugel. 
Mit Hilfe von Lemma \ref{lem:3.10} (mit $k=m+n$) und Anwendung des Resultats aus Teil (a) folgt
\begin{align*}
&\int_{\MdR^n}\HM^{m-n}(A\cap f^{-1}(\{y\}))\,\HM^n(dy)\\
&=\alpha(n)^{-1}\int_{B^n(0,1)}\int_{\MdR^n}\HM^{m-n}(A\cap f^{-1}(\{y-\ep z\}))\,\HM^n(dy)\, \HM^n(dz)\\
&=\alpha(n)^{-1}\int_{\MdR^n}\int_{\MdR^n}\HM^{m-n}(B\cap g^{-1}(\{y\})\cap p^{-1}(\{z\}))\,\HM^n(dz)\, \HM^n(dy)\\
&\le \alpha(n)^{-1}\frac{\alpha(m-n)\alpha(n)}{\alpha(m)}\text{Lip}(p)^n\HM^m(B\cap g^{-1}(\{y\})).
\end{align*}
Wegen $\text{Lip}(p)=1$ und da $Dg_{(x,y)}$ vollen Rang hat, folgt schließlich mit Hilfe von 
Teil (a), der für die Abbildung $g$ anwendbar ist,
\begin{align*}
&\int_{\MdR^n}\HM^{m-n}(A\cap f^{-1}(\{y\}))\,\HM^n(dy)\\
&\le\frac{\alpha(m-n)}{\alpha(m)}\int_{\MdR^n}\HM^m(B\cap g^{-1}(\{y\}))\, \HM^n(dy)\\
&=\frac{\alpha(m-n)}{\alpha(m)}\int_{B}Jg\, d(\HM^m\otimes\HM^n)\\
&\le \frac{\alpha(m-n)\alpha(n)}{\alpha(m)}\cdot\HM^m(A)\cdot \ep\cdot(\text{Lip}(f)+\ep)^{n-1}.
\end{align*}
Da $\ep>0$ beliebig klein gewählt werden kann, ist das Integral Null.
\end{enumerate}
\end{beweis}

\section{Rektifizierbare Mengen}

\begin{motivation}
\begin{itemize}
\item Verallgemeinerung von Mannigfaltigkeit
\item Allgemeine Fassung der Flächenformel und Koflächenformel
\item Trägermengen für rektifizierbare Ströme und rektifizierbare Varifaltigkeiten
\end{itemize}
\end{motivation}

\begin{definition}
\begin{itemize}
\item Eine Menge $M\subset \MdR^p$ heißt $n$-rektifizierbar, falls es eine beschränkte Menge $A\subset\MdR^n$ und eine Lipschitzabbildung $f:A\to\MdR^p$ gibt mit $f(A)=M$. 
\item 
Eine Menge $M\subset \MdR^p$ heißt abzählbar $n$-rektifizierbar, falls es eine Menge $M_0\subset \MdR^p$ mit $\HM^n(M_0)=0$ und Lipschitzabbildungen $f_j:\MdR^n \to \MdR^p$ für $j\in \MdN$ gibt, so dass gilt:
\[
M \subset M_0 \cup \bigcup_{j\in\MdN} f_j(\MdR^n).
\]
\end{itemize}
\end{definition}

Hinweis: Die Terminlogie im Buch von Federer weicht von dieser hier ab.


\begin{beispiele}
\item $\MdR^m$ ist abzählbar $n$-rektifizierbar für $m \le n$.
\item Teilmengen und abzählbare Vereinigungen von abzählbar $n$-rektifizierbaren Mengen sind abzählbar  $n$-rektifizierbare Mengen.
\item Sei $(x_i)_{i\in \MdN} \subset \MdR^2$ abzählbar, dicht. Dann  ist
\[
M \da \bigcup_{i\in\MdN} S^1(x_i,2^{-i})
\]
abzählbar 1-rektifizierbar, $\HM^1(M) <\infty$ und $\overline M = \MdR^2$.
\item Sei $K\subset \MdR^p$ kompakt und konvex. Der Rand $\partial K$ von $K$ ist $(p-1)$-rektifizierbar.
\item Ist $A\subset \MdR^p$ abzählbar $n$-rektifizierbar, so gibt es $B\in \borel(\MdR^p)$ mit $A\subset B$ und $B$ ist abzählbar $n$-rektifizierbar.
\item $D$ sei ein gleichseitiges Dreieck in $\MdR^2$ mit Kantenlänge 1. Eine Kante von $D$ sei parallel zur $x$-Achse. $D_1$ sei die Vereinigung von drei gleichseitigen Dreiecken in $\MdR^2$ mit Kantenlänge $\frac13$ „in den Ecken von $D$“. Definiere $D_i$, $i>1$, analog und 
\[
\hat D \da \bigcap_{i=1}^\infty D_i \subset \MdR^2.
\]
Es ist $\HM^1(\hat D)=1$, aber $\HM^1(\pi_1(\hat D))=0$, wobei $\pi_1$ die Orthogonalprojektion auf die $y$-Achse ist. Seien $\pi_2$, $\pi_3$ Orthogonalprojektionen auf die Geraden, die mit der $y$-Achse den Winkel 120$^\circ$ einschließen. Dann gilt $\HM^1(\pi_i(\hat D)) = 0$, $i=1,2$. Somit ist $\hat D$ nicht abzählbar 1-rektifizierbar.
\end{beispiele}

\begin{satz}[Struktursatz]
Sei $A\subset\MdR^p$ mit $\HM^n(A)<\infty$. Dann gibt es eine disjunkte Zerlegung $A= R \mathop{\dot\cup} N$, so 
dass gilt 
\begin{enumerate}[\quad(a)]
\item $R= A\cap R_0$ mit einer abzählbar $n$-rektifizierbaren Borelmenge $R_0\subset \MdR^p$.
\item Für jede abzählbar $n$-rektifizierbare Menge $F\subset \MdR^p$ gilt $\HM^n(N\cap F)=0$.
\end{enumerate}
\end{satz}

\begin{beweis}
Sei $s\da \sup\{\HM^n(A\cap B): B\in\borel(\MdR^p)\text{, abzählbar $n$-rektifizierbar}\} < \infty$. Es gibt $B_j\in \borel(\MdR^p)$, abzählbar $n$-rektifizierbar mit $\HM^n(A\cap B_j) \ge s-\frac1j$, $j\in \MdN$. 
Dann ist
\[
R_0 \da \bigcup_{j\in\MdN} B_j \in \borel (\MdR^p)
\]
abzählbar $n$-rektifizierbar. Setze $N\da A\setminus R_0$. Sei $F\subset \MdR^p$ abzählbar $n$-rektifizierbar. Es gibt $\tilde F \supset F$ mit $\tilde F\in \borel(\MdR^p)$ und abzählbar $n$-rektifizierbar.

Dann folgt
\begin{align*}
s &\ge \HM^n(A\cap (R_0\cup \tilde F)) \\
&=(\HM^n\MR A) (R_0\cup \tilde F) \\
&=(\HM^n\MR A) (R_0\cup (\tilde F\setminus R_0)) \\
&=(\HM^n\MR A) (R_0) + (\HM^n \MR A) (\tilde F\setminus R_0) \\
&=\underbrace{\HM^n(A\cap R_0)}_{=s} + \HM^n( \underbrace{(A\setminus R_0)}_{=N} \mathop{\cap} \tilde F).
\end{align*}
Also ist $\HM^n(N\cap \tilde F)= 0$. Setze $R\da A\cap R_0$.
\end{beweis}

\begin{definition}
Eine Menge $N\subset \MdR^p$ heißt rein nicht $n$-rektifizierbar, falls $\HM^n(N\cap F)=0$ für jede abzählbar  $n$-rektifizierbare Mengen $F\subset \MdR^p$ gilt.
\end{definition}

\begin{satz}[Charakterisierungssatz]
\begin{enumerate}[\quad(a)]
\item Ist $N\subset \MdR^p$ rein nicht $n$-rektifizierbar, so gilt für fast alle $E\in G(p,n)$:
\[
\HM^n(N\mid E) = 0.
\]
\item  Ist $A\subset \MdR^p$, $A= \bigcup_{i=1}^\infty A_i$, $\HM^n(A_i)<\infty$ und gilt für alle $B\subset A$ mit $\HM^n(B)>0$ stets $\HM^n(B\mid E) > 0$ für eine Menge von $E\in G(p,n)$ positiven Maßes, so ist $A$ abzählbar $n$-rektifizierbar.
\item Ist $A\subset \MdR^p$ abzählbar $n$-rektifizierbar und $0< \HM^n(A) < \infty$, so ist $\HM^n(A\mid E) = 0$ für höchstens ein $E\in G(p,n)$.
\end{enumerate}
\end{satz}
\begin{beweis}
Aussage (a) ist eine aufwändig zu beweisende Aussage. Wir können hier keinen Beweis angeben. 

(b) folgt leicht aus (a) und dem Struktursatz. 

(c) ist ebenfalls leicht zu zeigen.
\end{beweis}

\begin{proposition}
Für $M\subset \MdR^p$ sind äquivalent:
\begin{enumerate}[\quad(a)]
\item $M$ ist abzählbar $n$-rektifizierbar.
\item Es gibt $M_0\subset \MdR^p$ mit $\HM^n(M_0)=0$ und Mengen $A_j\subset \MdR^n$ mit Lipschitzabbildungen $f_j:A_j \to \MdR^p$, so dass 
\[
M\subset M_0 \cup \bigcup_{j=1}^\infty f_j(A_j).
\]
\item Es gibt $N_0\subset \MdR^p$ mit $\HM^n(N_0)=0$ und $n$-dimensionale Untermannigfaltigkeiten $N_j\subset \MdR^p$ der Klasse $C^1$ mit $\HM^n(N_j)<\infty$ und 
\[
M\subset N_0 \cup \bigcup_{j=1}^\infty N_j.
\]
\end{enumerate}
\end{proposition}

\begin{beweis}
\begin{itemize}
\item (a)$\implies$(c): Wir verwenden den folgenden $C^1$-Approximationssatz für Lipschitzabbildungen, der 
unter anderem auf einm Spezialfall eines Satzes von Whitney beruht:

\begin{quotation}
\noindent Sei $f\colon\MdR^n \to \MdR$ Lipschitz und $\ep>0$. Dann existiert eine $C^1$-Abbildung $g\colon\MdR^n\to \MdR$ mit
\[
\HM^n(\{x\in \MdR^n: f(x) \ne g(x) \text{ oder } \nabla f(x) \ne \nabla g(x)\})<\ep.
\]
\end{quotation}

Wiederholte Anwendung dieses Approximationssatz zeigt: Zu $j\in\MdN$ existieren eine Menge $E_j\subset \MdR^p$ und $C^1$-Abbildungen $g_i^{(j)}\colon \MdR^n\to \MdR^p$ mit
\[
f_j(\MdR^n) \subset E_j \cup \bigcup_{i=1}^\infty g_i^{(j)} (\MdR^n),
\]
wobei $\HM^n(E_j)=0$.

Setze $C_{ij}\da \{x\in \MdR^n : \operatorname{rg}(Dg_i^{(j)} (x)) < n \}$. (Ohne Beschränkung der Allgemeinheit ist $p\ge n$). Die Flächenformel ergibt $\HM^n(g_i^{(j)}(C_{ij})) = 0$. Für 
$$N_0 \da \bigcup_{j=1}^\infty E_j \cup \bigcup_{i=1}^\infty g_i^{(j)}(C_{ij})$$ 
ist $\HM^n(N_0)=0$.

Zu $x\in \MdR^n\setminus C_{ij}$ gibt es eine Umgebung $U_{ij}(x)$, so dass $g_j^{(i)}(U_{ij}(x))\subset \MdR^p$ eine $n$-dimensionale Untermannigfaltigkeit der Klasse $C^1$ ist. Da $\MdR^n\setminus C_{ij}$ offen ist, kann es durch abzählbar viele solcher Umgebungen überdeckt werden. (Hierzu schreibt man $\MdR^n\setminus C_{ij}$ 
zunächst als abzählbare Vereinigung von abgeschlossenen und dann von kompakten Teilmengen.)
\end{itemize}
\end{beweis}




\begin{beispiele}
\item $M = \bigcup_{n=1}^\infty \{(x,\frac 1n \cdot x^2) : x\in \MdR\} \subset \MdR^2$ ist eine abzählbar  1-rektifizierbare Menge.
\item $M = \bigcup_{n=1}^\infty ([-e_1, e_1] + \frac 1n \cdot e_2) \subset \MdR^2$ ist eine abzählbar 1-rektifizierbare Menge.
\item $[0,1]\setminus \MdQ $ ist abzählbar 1-rektifizierbar. Diese Menge ist aber nicht \emph{gleich} einer abzählbaren Vereinigung von eindimensionalen Untermannigfaltigkeiten der Klasse $C^1$, da sie überabzählbar viele Zusammenhangskomponenten hat.
\end{beispiele}

\begin{definition}
Sei $M\subset \MdR^p$ eine $\HM^n$-messbare Menge und $(\HM^n\MR M)\in \mathbb M(\MdR^p)$, das heißt $\HM^n(M\cap K) < \infty$ für alle kompakten Mengen $K\subset \MdR^p$. Sei $x\in\MdR^p$ und $P\in G(p,n)$. Dann heißt $P$ \emph{approximativer Tangentialraum} von $M$ in $x$, falls
\[
\HM^n\MR \left(\frac{M-x}{\lambda}\right) \tonach{v} \HM^n\MR P\qquad \text{für $\lambda \downto0$,}
\]
das heißt, für jede Funktion $f\in C_0(\MdR^p)$ gilt
\[
\lim_{\lambda \downto0} \int_{\frac{M-x}\lambda} f(y)\, \HM^n(dy) = \int_P f(y) \,\HM^n(dy).
\]
\end{definition}

\begin{bemerkungen}
\item Man kann umformen:
\[
\lim_{\lambda \downto0} \int_{\frac{M-x}\lambda} f(y)\, \HM^n(dy) =  \lim_{\lambda\downto0} \lambda^{-n}\int_M f(\frac{z-x}\lambda)\,\HM^n(dz).
\]
\item Sei $\eta_{x,\lambda}\colon\MdR^p\to\MdR^p$, $z\mapsto \frac 1\lambda(z-x)$. Dann ist \[
\HM^n\MR \left( \frac{M-x}{\lambda} \right) = (\eta_{x,\lambda})_\# (\lambda^{-n} \cdot (\HM^n\MR M)).
\]
\item Falls der approximative Tangentialraum von $M$ in $x$ existiert, so ist er eindeutig bestimmt und wird mit $T_xM$ bezeichnet. In diesem Fall ist $x\in\overline M$.
\item Ist $N\subset \MdR^p$ eine $n$-dimensionale $C^1$-Untermannigfaltigkeit von $\MdR^p$ und $x_0 \in N$, dann existiert der approximative Tangentialraum und stimmt mit dem differenzialgeometrischen Tangentialraum $\widetilde {T_{x_0}N}$ überein.

\textbf{Ansatz:} $N\cap V = \{x + g(x) \da G(x) : x\in (x_0 + \widetilde{T_{x_0}N}) \cap U\}$ mit $V,U$ Umgebungen von $x_0$, $g:(\widetilde{T_{x_0}N} + x_0) \cap U \to (\widetilde{T_{x_0}N}^\bot)$, wobei $g(x_0)=0$ und \(Dg(x_0)=0\). Weiter folgt für \(f \in C_0(\MdR^p)\), falls \(\lambda>0\) hinreichend klein ist,
\begin{align*}
\lambda^{-n} \cdot \int_N f(\frac{z-x_0}{\lambda}) \HM^n(dz) &= \lambda^{-n} \int_{x_0+T_{x_0}N} f(\frac{x+g(x)-x_0}\lambda) JG(x) \HM^n(dx) \\
&= \lambda^{-n} \int_{T_{x_0}N} f(z+\frac1\lambda g(x_0+\lambda z)) JG(x_0+\lambda z) \lambda^n \HM^n(dz) \\
&\to \int_{T_{x_0}N} f(z) \HM^n(dz)
\end{align*}
für  $ \lambda \downto 0$, 
wobei \(G(x) \da x+g(x)\), \(x\in T_{x_0}N + x_0\).
\end{bemerkungen}

\begin{definition}
Ist \(\mu\) ein äußeres Maß auf \(\MdR^p\), \(A \subset \MdR^p\) und \(x \in \MdR^p\), so nennt man 
\[
\theta^{n*}(\mu,A,x) \da \limsup_{r\downto0} \frac{\mu(A\cap B(x,r))}{\alpha(n)\cdot r^n}
\]
die \(n\)-dimensionale obere Dichte und
\[
\theta^n_*(\mu,A,x) \da \liminf_{r\downto0} \frac{\mu(A\cap B(x,r))}{\alpha(n)\cdot r^n}
\]
die \(n\)-dimensionale untere Dichte von \(A\) in \(x\) bezüglich \(\mu\).
\par
Ist \(\theta^{n*}(\mu,A,x) = \theta^n_*(\mu,A,x)\), so nennt man \(\theta^n(\mu,A,x) \da \theta^n_*(\mu,A,x)\) die Dichte von \(A\) in \(x\) bezüglich \(\mu\).
\end{definition}

\begin{lemma}\label{lem:3.21}
Sei \(\mu\) ein Borelreguläres äußeres Maß auf \(\MdR^p\), \(A \subset \MdR^p\) \(\mu\)-messbar und \(\mu\MR A\in \mathbb{M}(\MdR^p)\). Dann gilt \(\theta^n(\mu,A,x)=0\) für \(\HM^n\)-fast-alle \(x\in\MdR^p\setminus A\).
\end{lemma}
Ohne Beweis (ist kompliziert).

\begin{proposition}\label{prop:3.22}
Ist \(M \subset \MdR^p\) abzählbar \(n\)-rektifizierbar, \(\HM^n\)-messbar und \(\HM^n\MR M \in \mathbb{M}(\MdR^p)\), dann existiert \(T_x M\) für \(\HM^n\)-fast-alle \(x \in M\).
\end{proposition}

\begin{beweis}
Die Menge \(M\) kann in der Form \(M = M_0 \mathop{\dot\cup} {\dot\bigcup}_{j\geq 1} M_j\) mit paarweise disjunkten Mengen \(M_j\), \(M_j\) \(\HM^n\)-messbar und \(M_j \subset N_j\), geschrieben werden, wobei \(N_j\) eine \(n\)-dimensionale \(C^1\)-Untermannigfaltigkeit von \(\MdR^p\) ist mit \(\HM^n(N_j)<\infty\). (\(M_0 \da N_0\), \(M_1 \da M \cap N_1 \setminus M_0\), \(M_2 \da M \cap N_2 \setminus (M_0 \cup M_1)\), \dots). Für \(\HM^n\)-fast-alle \(x \in M_j\) gilt wegen Lemma \ref{lem:3.21}:
\[
\theta^n(\HM^n, M \setminus M_j, x) = \theta^n(\HM^n, N_j \setminus M_j, x) = 0.
\]
Für jedes solches \(x\) existieren die approximativen Tangentialräume \(T_x M\) und \(T_x N_j\). Sei \(f \in C_0(\MdR^p)\) mit \(\supp(f) \subset B(0,R)\) (für ein \(R \in (0,\infty)\)). Dann gilt
\begin{align*}
\left| \lambda^{-n} \cdot \int_{M\setminus M_j} f(\frac{z-x}\lambda) \HM^n(dz) \right| &\leq \lambda^{-n} \cdot \int_{M\setminus M_j} \|f\|_\infty \ind_{B(x,\lambda R)} d\HM^n \\
&\leq \|f\|_\infty \lambda^{-n} \HM^n(M\setminus M_j\cap B(x,\lambda R)) \\
&= \alpha(n) R^n \|f\|_\infty \frac{\HM^n (M\setminus M_j \cap B(x,\lambda R))}{\alpha(n) (\lambda R)^n}.
\end{align*}
Also gilt
\[\lim_{\lambda \downto 0} \lambda^{-n} \int_{M\setminus M_j} f(\frac{z-x}\lambda) \HM^n (dz) = 0.
\] 
Ebenso folgt
\[
\lim_{\lambda\downto0} \lambda^{-n} \int_{N_j\setminus M_j} f(\frac{z-x}\lambda) \HM^n(dz) = 0.
\]
Wegen
\begin{align*}
\int_{T_{x_0}N_j} f(z) \HM^n(dz) &= \lim_{\lambda\downto0} \lambda^{-n}\int_{N_j} f(\frac{z-x}\lambda) \HM^n(dz) \\
&= \lim_{\lambda\downto0} \lambda^{-n} \int_{M_j} f(\frac{z-x}\lambda)\HM^n(dz) \\
&= \lim_{\lambda\downto0} \lambda^{-n} \int_M f(\frac{z-x}\lambda) \HM^n(dz)
\end{align*}
folgt die Behauptung.
\end{beweis}

\begin{definition}
Sei \(M \subset \MdR^p\) eine \(\HM^n\)-messbare Menge. Man nennt eine \(\HM^n\)-messbare  Funktion \(\theta: M \to (0,\infty)\), die 
\[
\int_{M\cap K} \theta d\HM^n < \infty
\]
für alle \(\HM^n\)-messbaren und beschränkten Mengen \(K \subset \HM^n\)  erfüllt, 
eine Gewichtsfunktion zu \(M\).
\end{definition}

\begin{bemerkung}
Sind \(M, \theta\) wie in der Definition, so ist
\[
\HM^n\MR\theta(\bullet) \da \int_\bullet \theta(x) \HM^n(dx) \in \mathbb M(\MdR^p).
\]
\end{bemerkung}

\begin{lemma}\label{lem:3.23}
Sei \(M \subset \MdR^p\) eine \(\HM^n\)-messbare Menge. Dann gilt:
\begin{enumerate}[(a)]
\item Genau dann existiert eine Gewichtsfunktion \(\theta\) zu \(M\), wenn \(M = \bigcup_{j\geq1} M_j\) mit \(\HM^n\)-messbaren Mengen \(M_j \subset \MdR^p\) und \(\HM^n \MR M_j \in \mathbb M(\MdR^p)\).
\item Ist \(M\) abzählbar \(n\)-rektifizierbar, so existiert eine Gewichtsfunktion zu \(M\).
\end{enumerate}
\end{lemma}
\begin{beweis} Übung. \end{beweis}

\begin{definition}
Sei \(M \subset \MdR^p\) eine \( \HM^n\)-messbare Menge und \(\theta: M \to (0,\infty)\) eine Gewichtsfunktion zu \(M\). Dann heißt \(P \in G(p,n)\) \(n\)-dimensionaler \emph{approximativer Tangentialraum} von \(M\) in \(x\) bezüglich \(\theta\), falls
\[
(\eta_{x,\lambda})_\# \left(\lambda^{-n}(\HM^n\MR \theta)\right) \stackrel{v}\to \theta(x) (\HM^n \MR P) \quad \text{für }\lambda\downto0,
\]
das heißt für \(f\in C_0(\MdR^p)\) gelte
\[
\int_{\frac{M-x}\lambda} f(y) \theta(x+\lambda y) \HM^n(dy) = \lambda^{-n} \int_M f(\frac{z-x}\lambda) \theta(z) \HM^n(dz) \xrightarrow{\lambda\downto0} \theta(x) \int_P f(y) \HM^n(dy).
\]
\end{definition}

\begin{bemerkungen}
\begin{enumerate}[(1)]
\item Ist \(\theta\) fest, so ist der approximative Tangentialraum \(P\) von \(M\) in \(x\) bezüglich \(\theta\) eindeutig bestimmt, falls er existiert. In diesem Fall schreiben wir \(T_x^\theta M \da P\).
\item Sei \(x\in M\). Für \(\eta>0\) sei \(M_\eta \da \{y\in M \colon \theta(y) > \eta\}\). Dann hat \(M_\eta\) lokal endliches Hausdorffmaß, das heißt \(\HM^n\MR M_\eta \in \mathbb M(\MdR^p)\). Zu \(x\in M\) existiert stets \(\eta>0\) mit \(x \in M_\eta\). Für \(\HM^n\)-fast-alle \(x\in M_\eta\) gilt:
\[
T_x M_\eta \quad \text{ existiert genau dann, wenn }T_x^\theta M \text{ existiert.}
\]
In diesem Fall gilt $T_x M_\eta = T_x^\theta M$.
\item Sind \(\theta\) und \(\tilde\theta\) Gewichtsfunktionen zu \(M\), so gilt für \(M_\theta \da \{x\in M \colon T_x^\theta M \text{ existiert}\}\), \(M_{\tilde\theta} \da \{x\in M \colon T_x^{\tilde\theta} M 
\text{ existiert}\}\):
\[
\HM^n(M_\theta \mathop\Delta M_{\tilde\theta}) = \HM^n\left( (M_\theta \setminus M_{\tilde\theta}) \cup (M_{\tilde\theta} \setminus M_\theta) \right) = 0,
\]
und für \(\HM^n\)-fast-alle \(x \in M_\theta \cap M_{\tilde\theta}\) ist \(T_x^\theta M = T_x^{\tilde\theta}M\).
\end{enumerate}
\end{bemerkungen}

\begin{satz}\label{satz:3.24}
Sei \(M \subset \MdR^p\) eine \(\HM^n\)-messbare Menge. Dann sind äquivalent:
\begin{enumerate}[(a)]
\item \(M\) ist abzählbar \(n\)-rektifizierbar.
\item \(M\) hat eine Gewichtsfunktion \(\theta\), so dass \(T_x^\theta M\) für \(\HM^n\)-fast-alle \(x \in M\) existiert.
\item Für jede Gewichtsfunktion \(\theta\) von \(M\) gilt: \(T_x^\theta M\) existiert für \(\HM^n\)-fast-alle \(x\in M\).
\end{enumerate}
\end{satz}

\begin{beweis}[Skizze]
(a) \(\Rightarrow\) (c): Zu \(M\) betrachtet man eine Zerlegung \(M = M_0 \mathop{\dot\cup} {\dot\bigcup}_{j\geq1} M_j\) mit \(\HM^n\)-messbaren Mengen \(M_j \subset \MdR^p\), wobei \(\HM^n(M_0)=0\) und \(M_j\subset N_j\), \(N_j\) ist \(n\)-dimensionale \(C^1\)-Unter\-mannig\-faltig\-keit von \(\MdR^p\) mit \(\HM^n(N_j)<\infty\). Sei \(\theta\) eine beliebige Gewichtsfunktion zu \(M\) und \(\mu \da \HM^n \MR \theta\). Betrachte \(x\in M_j\) mit folgenden Eigenschaften:
\begin{enumerate}[(1)]
\item \(\theta^n(\mu, M\setminus M_j, x)=0\),
\item \(\theta^n(\HM^n, N_j \setminus M_j, x) = 0\),
\item \(\theta^n(\mu, M_j \setminus S_\varepsilon, x) = 0\), wobei \(S_\varepsilon \da \{z\in M \colon |\theta(z)-\theta(x)|\leq\varepsilon\},\, \varepsilon>0\),
\item \(\theta^n(\HM^n, M_j\setminus S_\varepsilon, x)=0\).
\end{enumerate}
Mit Hilfe von Lemma \ref{lem:3.21} für (1) und (2) und mit Hilfe des Satzes von Lusin für (3) und (4) folgt, dass \(\HM^n\)-fast-alle \(x\in M_j\) diese Eigenschaften erfüllen. Dann folgert man in mehreren Schritten, dass
\[
\lim_{\lambda\downto0} \lambda^{-n} \cdot \int_M f(\frac{z-x}\lambda) \theta(z) \HM^n(dz) = \theta(x) \int_{T_x N_j} f(z) \HM^n(dz)
\]
für \(f \in C_0(\MdR^p)\). Dies zeigt insbesondere \(T_x^\theta M = T_x N_j\). Die Details werden in den Übungen 
besprochen.
\end{beweis}

\begin{beispiel} Sei
\[M = [-e_1, e_1] \cup \bigcup_{n \in \MdN} \left( [-e_1, e_1] + \frac1n e_2 \right)
\] und setze \(\theta|_{M_i} = \left(\frac12\right)^i\) und \(\theta|_{[-e_1,e_1]} \da \theta(0)\). Man kann nun \(T_0^\theta M = \MdR e_1\) zeigen.  Die Details werden in den Übungen besprochen.
\end{beispiel}

\bigskip

Das Ziel im Folgenden ist es, eine allgemeine Form der Flächen- und Koflächenformel zu finden, sowie eine Definition von Ableitungen von Lipschitz-Abbildungen relativ zu rektifizierbaren Mengen.

Sei $M\subset \MdR^p$ eine abzählbare $m$-rektifizierbare Menge. Dazu sei $M = \dot\bigcup_{j=0} M_j$ eine  disjunkte Zerlegung von $M$ in $\HM^m$-messbare Mengen $M_j\subset\MdR^p$ mit $\HM^m(M_0)=0$, wobei $M_j\subset N_j$ mit $m$-dimensionalen $C^1$-Untermannigfaltigkeiten $N_j$, $j\ge 1$, $\HM^m(N_0)=0$ sowie $\HM^m(N_j)<\infty$.

Sei $U\subset\MdR^p$ offen, $M\subset U$ und $f\colon U\to \MdR$ eine Lipschitzfunktion. Sei $x\in M$. Dann gibt es ein $j\in\MdN_0$ mit $x\in M_j$. Ist $j\ge 1$ und $f|_{N_j}$ differenzierbar in $x$, so erklärt man
\[
\nabla^M f(x) \da \nabla^{N_j} f(x) \da \sum_{k=1}^m  D_{\tau_k}f(x) \cdot \tau_k  \in T_xN_j,
\]
wobei $(\tau_1,\ldots,\tau_m)$ eine Orthonormalbasis von $T_xN_j$ ist.

In diesem Fall existiert eine in $x$ differenzierbare Fortsetzung $\bar f\colon \MdR^p \to \MdR$ von $f|_{N_j}$ und für diese gilt 
\[
\nabla^M f(x) = \sum_{k=1}^m \langle \nabla\bar f(x), \tau_k\rangle \cdot \tau_k,
\]
wobei der Gradient hier im umgebenden Raum $\MdR^p$ gebildet wird.


\begin{bemerkungen}
\item Ist die Zerlegung von $M$ gewählt, so existiert $\nabla^Mf(x)$ für $\HM^m$-fast-alle $x\in M$ und ist eindeutig bestimmt.
\item Tatsächlich ist $\nabla^M f(x)$ von der Wahl der Zerlegung von $M$ unabhängig.

Ist nämlich $x\in M_j\subset N_j$, existiert $\nabla^M f(x) = \nabla^{N_j}f(x)$, das heißt ist $f|_{N_j}$ in $x$ differenzierbar und existiert $T_xM = T_xN_j$, so ist $f|_{(x+T_xM)}$ differenzierbar in $x$ und $\nabla^Mf(x) = \nabla^{N_j} f(x) = \nabla(f|_{(x+T_xM)})$. Für den Nachweis dieser Aussage wird verwendet, dass $f$ Lipschitzfunktion auf $U$ ist.
\end{bemerkungen}

\textbf{Fazit:} Ist $M\subset \MdR^p$ eine abzählbar $m$-rektifizierbare Menge, $U\subset\MdR^p$ offen mit $M\subset U$, $f\colon U\to \MdR$ eine {\em lokale Lipschitzfunktion}, dann existiert $\nabla^Mf(x)$ für $\HM^m$-fast-alle $x\in M$ und die Definition $\nabla^Mf(x) = \nabla^{N_j} f(x)$ ist korrekt. 

In dieser Situation definieren wir:
\begin{align*}
d^M f_x \colon T_xM &\to \MdR \\
\tau &\mapsto d^Mf_x(\tau) \da \langle \nabla^Mf(x), \tau\rangle.
\end{align*}

Ist $f\colon U\to\MdR^q$ lokal Lipschitz, $f=(f^1,\ldots,f^q)^\top$ und existiert $\nabla^Mf^j_x$ für $j=1,\ldots,q$, so sei 
\begin{align*}
d^M f_x \colon T_xM &\to \MdR^q \\
\tau &\mapsto d^Mf_x(\tau) \da \sum_{j=1}^q d^Mf^j_x(\tau) \cdot e_j = \sum_{j=1}^q \langle \nabla^Mf^j(x), \tau\rangle \cdot e_j.
\end{align*}

\begin{definition}[Jakobi-Determinante]
Sei $f\colon U\subset\MdR^p \to \MdR^q$, sei $M$ abzählbar $m$-rektifizierbar und es existiere $\nabla^M f(x)$  in $x\in M$. 

Für $m\le q$ sei
\[
J_M f(x) \da \sqrt{\det \big( (d^Mf_x)^*\circ (d^Mf_x)\big)}
\]
und für $m> q$ sei 
\[
J_M f(x) \da \sqrt{\det \big( (d^Mf_x)\circ (d^Mf_x)^*\big)}.
\]
\end{definition}

\begin{satz}
Sei $M\subset\MdR^p$ eine $\HM^m$-messbare, abzählbar $m$-rektifizierbare Menge. Sei $U\subset \MdR^p$ offen und $M\subset U$ sowie $f\colon U\to\MdR^q$ eine lokale Lipschitzabbildung. Sei schließlich $g\colon M \to [0,\infty]$ eine $\HM^m$-messbare Funktion. 
\begin{enumerate}[\quad(a)]
\item Dann gilt für $m\le q$:
\[
\int_M g(x) \cdot J_Mf(x)\, \HM^m(dx) =
\int_{\MdR^q} \int_{f^{-1}(\{z\})} g(x)\, \HM^0(dx)\, \HM^m(dz).
\]
\item Dann gilt für $m\ge q$:
\[
\int_M g(x) \cdot J_Mf(x)\, \HM^m(dx) =
\int_{\MdR^q} \int_{f^{-1}(\{z\})} g(x)\, \HM^{m-q}(dx)\, \HM^q(dz).
\]
\end{enumerate}
\end{satz}

\pagebreak[1]
Auch diese Fassung lässt sich weiter verallgemeinern:
\begin{satz}
Sei $M\subset\MdR^p$ eine $\HM^m$-messbare, abzählbar $m$-rektifizierbare Menge, und $Z\subset\MdR^q$ eine $\HM^n$-messbare, abzählbar $n$-rektifizierbare Menge. Sei $f\colon M\to Z$ eine lokale Lipschitzabbildung und $g:M\to [0,\infty]$ $\HM^m$-messbar. Dann gilt
\[
\int_M g(x) \cdot J_Mf(x) \, \HM^m(dx) = \int_Z \int_{f^{-1}(\{z\})} g(x)\, \HM^{m-n}(dx)\, \HM^n(dz).
\]
\end{satz}

\chapter{Ströme}

\section{Differentialformen und äußere Ableitung}

\textbf{Ziel:} Integration über orientierte Flächen.

\begin{definition}
Sei $k\in \MdR$ und $V$ ein $n$-dimensionaler reeller Vektorraum.
\begin{enumerate}
\item Eine Abbildung $\Phi\colon V^k\to \MdR$ heißt multilinear, falls $\Phi$ in jeder Komponente linear ist.
\item Eine Abbildung $\Phi\colon V^k\to\MdR$ heißt alternierend, falls $\Phi$ nur das Vorzeichen wechselt, wenn zwei Komponenten vertauscht werden:
\[
\Phi(v_1,\ldots,v_i,\ldots,v_j,\ldots,v_k) = -\Phi(v_1,\ldots,v_j,\ldots,v_i,\ldots,v_k).
\]
\item $\bw^k V \da\{ (\Phi\colon V^k\to \MdR) : \Phi \text{ ist multilinear und alternierend}\}$.
\item $\bw^k V$ wird in kanonischer Weise zu einem Vektorraum. Die Elemente von $\bw^k V$ nennt man $k$-Kovektoren, falls $V=\MdR^n$. Ist $V=(\MdR^n)^*$, so werden die Elemente von $\bw^k (\MdR^n)^* =:  \bw_k\MdR^n$ als $k$-Vektoren bezeichnet.
\end{enumerate}
\end{definition}

\begin{bemerkungen}
\item $\bw^1\MdR^n = (\MdR^n)^*$
\item $\bw_1\MdR^n = ((\MdR^n)^*)^* =\MdR^n$ (mit der üblichen Identifikation).
\item Sei $e_1,\ldots,e_n$ die Standardbasis des $\MdR^n$ und $e_1^*,\ldots,e_n^*$ die Dualbasis. Wir schreiben
\[
\langle e_j^*,e_i\rangle \da e_j^*(e_i)  =\delta_{ij}.
\]
Statt $e_j^*$ wird auch $d{x_j}$ geschrieben.
\item $\bw^n\MdR^n$ sind gerade die Determinantenfunktionen.
\end{bemerkungen}

\begin{definition}
Seien $\eta_1,\ldots,\eta_k\in \bw^1\MdR^n$. Dann wird
\[
\eta_1\wedge \ldots\wedge \eta_k \in \bw^k\MdR^n
\]
durch
\[
(\eta_1\wedge \ldots\wedge \eta_k)(v_1,\ldots,v_k) \da \det\left( \langle \eta_i,v_j\rangle_{i,j=1,\ldots,k}\right)
\]
erklärt. Man beachte hierbei $\langle\eta_i,v_j\rangle\da \eta_i(v_j)$.

\textbf{Alternativ:} Ist $\eta_i = \sum_{j=1}^n \eta_{ij} dx_j$, $\eta_{ij}\in \MdR$, $i=1,\ldots,k$, so kann auch 
\[
(\eta_1\wedge \ldots\wedge \eta_k)(v_1,\ldots,v_k) \da \det\left( (\eta_{ij})\cdot(v_1|\cdots|v_k) \right)
\]
erklären.
\end{definition}

\textbf{Ergänzung:} Mit $\mathcal T^k(V)\da \{ (T\colon V^k\to\MdR) : T \text{ ist $k$-linear}\}$ bezeichnet man die Tensoren der Stufe $k$ über dem Vektorraum $V$. Die Abbildung
\begin{align*}
\mathcal T^k(V) \otimes \mathcal T^l(V) &\to \mathcal T^{k+l}(V) \\
(T,S) &\mapsto T\otimes S
\end{align*}
ist erklärt durch
\[
(T\otimes S)(u_1,\ldots,u_{k+l}) \da T(u_1,\ldots,u_k) \cdot S(u_{k+1},\ldots,u_{k+l}).
\]
Man bezeichnet $T\otimes S$ als das Tensorprodukt von $T$ und $S$. 
Um für $p\in\MdN$ einen $p$-Tensor in einen alternierenden $p$-Tensor zu überführen, 
erklärt man  die Abbildung
\begin{align*}
\operatorname{Alt}\colon \mathcal T^p(V) &\to \bw^p V\\
T & \mapsto \operatorname{Alt}(T),
\end{align*}
wobei
\[
\operatorname{Alt}(T)(v_1,\ldots,v_p) \da \frac{1}{p!} \cdot \sum_{\pi\in S_p} \sgn(\pi) \cdot T(v_{\pi(1)},\ldots,v_{\pi(p)}).
\]
Hier ist $S_p$ die Menge aller Permutationen (Bijektionen) der Menge $\{1,\ldots,p\}$. 
Für $\omega\in \bw^k\MdR^n$ und $\eta\in \bw^l\MdR^n$ sei
\[
\omega \wedge \eta \da \frac{(k+l)!}{k!\cdot l!} \operatorname{Alt}(\omega\otimes \eta) \in \bw^{k+l}\MdR^n.
\]
Man stellt fest, dass dieses „Dachprodukt“ assoziativ ist. Ferner gilt
$$
\eta_1\wedge \ldots\wedge \eta_k=k!\cdot \operatorname{Alt}(\eta_1\otimes\ldots\otimes \eta_k).
$$

In gleicher Weise erklären wir nun auch ein Dachprodukt für $k$-Vektoren.


\begin{definition}
Seien $v_1,\ldots,v_k\in \MdR^n$. Dann wird
\[
v_1\wedge \ldots\wedge v_k \in \bw_k\MdR^n
\]
durch
\[
(v_1\wedge \ldots\wedge v_k)(\eta_1,\ldots,\eta_k) \da \det\left( \langle \eta_i,v_j\rangle_{i,j=1,\ldots,k}\right)
\]
erklärt, wobei $\eta_1,\ldots,\eta_k\in(\MdR^n)^*$. 
\end{definition}


Man kann zeigen, dass
$\eta_1\wedge\ldots\wedge \eta_k\in\bw^k\MdR$ eine Linearform auf $\bw_k\MdR^n$ 
ist, wenn man
$$
(\eta_1\wedge\ldots\wedge \eta_k)(v_1\wedge\ldots\wedge v_k)\da (\eta_1\wedge\ldots\wedge \eta_k)(v_1,\ldots, v_k)
$$
für $v_1,\ldots,v_k\in\MdR^n$ erklärt. Die Wohldefiniertheit ist leicht einzusehen. Im Folgenden schreiben 
wir für $\omega\in\bigwedge^k\MdR^n$ und $\xi\in\bigwedge_k\MdR^n$ 
$$
\langle \omega,\xi\rangle\da \omega(\xi).
$$


In gleicher Weise wird 
$v_1\wedge \ldots\wedge v_k \in \bw_k\MdR^n$ als Linearform auf $\bw^k\MdR^n$ erklärt, 
indem man 
\[
(v_1\wedge\ldots\wedge v_k)(\eta_1\wedge\ldots\wedge \eta_k)\da (v_1\wedge\ldots\wedge v_k)(\eta_1,\ldots, \eta_k)
\]
setzt. Auch hier ist die Wohldefiniertheit leicht zu bestätigen. 

Man kann ferner nachweisen, dass
\[
e_{i_1}^*\wedge\ldots\wedge e_{i_k}^*,\qquad 1\le i_1<\cdots<i_k\le n
\]
eine Basis von $\bw^k\MdR^n$ ist. Ebenso ist 
\[
e_{i_1}\wedge\ldots\wedge e_{i_k},\qquad 1\le i_1<\cdots<i_k\le n
\]
eine Basis von $\bw_k\MdR^n$. Diese Basen sind zueinander dual in Bezug auf obige Deutung von $k$-Kovektoren als Linearformen auf $k$-Vektoren. 

\begin{notation}
Sei
\[
I_k^n \da \{ (i_1,\ldots,i_k) \in \{1,\ldots,n\}^k : 1 \le i_1< \cdots < i_k \le n\}.
\]
Für $I\in I_k^n$ ist $e_I \da e_{i_1}\wedge \ldots \wedge e_{i_k}$ und $dx_I \da dx_{i_1}\wedge \cdots \wedge dx_{i_k}$ und so weiter.
\end{notation}

\begin{bemerkungen}
\item Für $\sigma\in S_k$, $\eta_1,\ldots,\eta_k\in (\MdR^n)^* = \bw^1\MdR^n$ gilt
\[
\eta_{\sigma(1)} \wedge \cdots \wedge \eta_{\sigma(k)} = \sgn(\sigma) \cdot \eta_1 \wedge \cdots\wedge \eta_k.
\]
\item Sei $\Phi \in \bw^k\MdR^n$. Dann ist
$$
\Phi = \sum_{\mathclap{1\le i_1<\cdots < i_k\le n}} \Phi_{(i_1,\ldots,i_k)} dx_{i_1} 
\wedge \cdots \wedge dx_{i_k} 
= \sum_{I\in I_k^n} \Phi_I \cdot dx_I.
$$
Hierbei ist also $\Phi_I = \Phi(e_{i_1}\wedge\cdots\wedge e_{i_k}) = \Phi(e_I)\in \MdR$.
\end{bemerkungen}

\begin{definition}
Sei $W\subset \MdR^n$ offen. Eine Abbildung $\Phi\colon W\to \bw^k\MdR^n$ heißt Differentialform vom Grad $k$ (kurz: $k$-Form). Die $k$-Form $\Phi$ ist von der Klasse $\mathcal C^r$, $r\ge 1$, falls $p\mapsto \Phi(p)(v_1,\ldots,v_k)$ von der Klasse $\mathcal C^r$ ist für jede Wahl von $v_1,\ldots,v_k\in \MdR^n$.
\end{definition}

\begin{bemerkungen}
\item Die $k$-Form $\Phi$ ist von der Klasse $\mathcal C^r$ genau dann, wenn
\[
p\mapsto \Phi_I(p):=\Phi(p)_I = \langle \Phi(p), e_I\rangle = \langle \Phi(p) , e_{i_1}\wedge \cdots \wedge e_{i_k}\rangle
\]
von der Klasse $\mathcal C^r$ ist für alle $I\in I_k^n$.
\item Die $k$-Form $\Phi$ lässt sich schreiben als
\[
p\mapsto \Phi(p) = \sum_{I\in I_k^n} \Phi_I(p) dx_I
\]
mit $\Phi_I(p) \in \MdR$.
\end{bemerkungen}

\begin{definition}[Dachprodukt]
Für $\Phi \in \bw^k\MdR^n$ und $\eta \in \bw^l\MdR^n$ wird $\Phi\wedge \eta \in \bw^{k+l}\MdR^n$ in 
folgender Weise erklärt: 
Ist $\Phi = \sum_{I\in I_k^n} \Phi_Idx_I$ und $\eta = \sum_{J\in I_l^n} \eta_J dx_J$, dann ist
\[
\Phi \wedge \eta \da \sum_{I\in I^n_k,\, J\in I^n_l} \Phi_I\cdot \eta_J \underbrace{dx_I \wedge dx_J}_{\mathclap{dx_{i_1} \wedge \cdots \wedge dx_{i_k} \wedge dx_{j_1} \wedge \cdots \wedge dx_{j_l}}}.
\]
\end{definition}

Eine „invariante Definition“ kann mit Hilfe des $\operatorname{Alt}$-Operators gegeben werden (s.o.).

\begin{bemerkungen}
\item Assoziativgesetz: Für $\Phi\in \bw^k\MdR^n$, $\eta\in\bw^l\MdR^n$ und $\Theta\in \bw^r\MdR^n$ gilt: $$(\Phi\wedge \eta) \wedge \Theta =\Phi \wedge (\eta\wedge \Theta).$$
\item Distributivgesetz: Für $\alpha_1,\alpha_2\in \MdR$, $\Phi_1,\Phi_2 \in \bw^k\MdR^n$ und $\eta\in\bw^l\MdR^n$ gilt: 
$$
(\alpha_1 \Phi_1 + \alpha_2\Phi_2) \wedge \eta = \alpha_1(\Phi_1\wedge \eta) + \alpha_2(\Phi_2\wedge \eta).
$$
\end{bemerkungen}
\pagebreak[2]
\textbf{Ausblick:}
\begin{itemize}
\item Sei $S\subset \MdR^n$ eine $k$-Fläche, das heißt es gibt eine offene Menge $U\subset \MdR^k$ und eine Abbildung $F\colon U\to \MdR^n$ der Klasse $\mathcal C^r$, $r\ge 1$, $F$ ist injektiv, $DF_x$ ist injektiv  und $S= F(U)$.

Die Fläche $S$ wird „orientiert“ durch die Orientierung von $\MdR^k$ und durch $F$.
\item Sei $W\subset \MdR^n$ offen mit $S\subset W$. Sei ferner $\Phi$ eine $k$-Form auf $W$, das heißt $\Phi(p) \in \bw^k\MdR^n$ für $p\in W$.
\item Das Integral von $\Phi$ über $S$ kann erklärt werden durch
\[
\int_S \Phi \da \int_U \left\langle \underbrace{\Phi\circ F(x)}_{\in \bw^k\MdR^n}, \underbrace{\frac{\partial F}{\partial x_1} \wedge \dots \wedge \frac{\partial F}{\partial x_k}}_{\in \bw_k\MdR^n} \right\rangle \lambda^k(dx).
\]
Man zeigt mit Hilfe des Transformationssatz für Gebietsintegrale, dass diese Defintion von der Wahl von $F$ unabhängig ist.
\end{itemize}

\begin{definition}[Äußeres Differential]
Sei $U\subset \MdR^n$ offen und $\Phi: U\to\bw^k\MdR^n$ eine $k$-Form der Klasse $\mathcal C^r$ mit $r\ge 1$.
\begin{enumerate}[\quad(a)]
\item Ist $k=0$, so ist $\Phi = f$ eine Funktion und $d\Phi = df$ ist als 1-Form auf $U$ erklärt durch $df(p)(v) \da D_vf(p)$, das heißt
\[
df = \sum_{i=1}^n \frac{\partial f}{\partial x_i} dx_i.
\]
denn
\[
df(p)(v) = \sum_{i=1}^n \frac{\partial f}{\partial x_i}(p) \cdot dx_i(v) = \sum_{i=1}^n \frac{\partial f}{\partial x_i}(p) \cdot v_i = \langle \nabla f(p),v \rangle = D_vf(p).
\]
\item Ist $k\ge 1$ und $\Phi = f\cdot dx_I$ für ein $I\in I_k^n$, dann sei $d\Phi \da df\wedge dx_I$, das heißt $d\Phi(p) \in \bw^{k+1}\MdR^n$ mit
\[
d\Phi(p) = \underbrace{df(p)}_{\in\bw^1\MdR^n} \wedge \underbrace{dx_I}_{\in \bw^k\MdR^n} \in \bw^{k+1}\MdR^n.
\]
\item Sei $k\ge 1$ und $\Phi = \sum_{I\in I_k^n} \Phi_I dx_I$ allgemein. $d\Phi$ wird durch lineare Fortsetzung erklärt, das heißt
\[
d\Phi \da \sum_{I\in I_k^n} d(\Phi_I dx_I) = \sum_{I\in I_k^n} d\Phi_I \wedge dx_I.
\]
\end{enumerate}
\end{definition}

\begin{bemerkungen}
Es gilt
\[
\langle d\Phi(p), v_1\wedge \cdots \wedge v_{k+1}\rangle = \sum_{i=1}^{k+1} (-1)^{i-1} \langle D_{v_i} \Phi(p), v_1\wedge\cdots \wedge \cancel{v_{i}} \wedge \cdots \wedge v_{k+1}\rangle.
\]
\end{bemerkungen}

\pagebreak[2]
\begin{lemma}
\label{lem:4.1}
Seien $\Phi,\Psi$ jeweils $k$-Formen der Klasse $C^r$, $r\ge 1$, und $\Theta$ eine $l$-Form der Klasse $C^s$, $s\ge 1$. Dann gilt
\begin{enumerate}
\item $d(\Phi + \Psi) = d\Phi + d\Psi$
\item $d(\Phi \wedge \Theta) = d\Phi \wedge \Theta + (-1)^k\cdot \Phi \wedge d\Theta$.
\end{enumerate}
\end{lemma}

\begin{beweis}
\begin{enumerate}
\item[\quad(2)] Seien $\Phi = fdx_I$, $\Theta = gdx_J$. Dann erhält man
\begin{align*}
d(\Phi \wedge \Theta) &= d(f\cdot dx_I \wedge g\cdot dx_J)
= d( (f\cdot g)\cdot dx_I \wedge dx_J )
= d(f\cdot g) \wedge dx_I \wedge dx_J \\
&= (g\cdot df + f \cdot dg) \wedge dx_I \wedge dx_J \\
&= g\cdot df \wedge dx_I \wedge dx_J + f\cdot dg \wedge dx_I \wedge dx_J \\
&= (df \wedge dx_I) \wedge (g\cdot dx_J) + (-1)^k\cdot (f\cdot dx_I ) \wedge (dg\wedge dx_J)\\
&= d\Phi \wedge \Theta + (-1)^k \cdot \Phi \wedge d\Theta.
\end{align*}
\end{enumerate}
\end{beweis}

\begin{lemma}
\label{lem:4.2}
Ist die $k$-Form $\Phi\colon U\to \bw^k\MdR^n$ von der Klasse $\mathcal C^r$, $r\ge 2$, so gilt $dd\Phi = 0$ (als $(k+2)$-Form).
\end{lemma}

\begin{beweis}
Sei ohne Beschränkung der Allgemeinheit $\Phi = f\cdot dx_I$, $I\in I_k^n$. Dann gilt 
\[
d\Phi = df \wedge dx_I = \sum_{i=1}^n \frac{\partial f}{\partial x_i} \cdot dx_i \wedge dx_I 
\]
und ferner
\begin{align*}
d(d\Phi) &= d(\sum_{i=1}^n \frac{\partial f}{\partial x_i} \cdot dx_i \wedge dx_I)\\
&= \sum_{i=1}^n d(\frac{\partial f}{\partial x_i} \cdot dx_i \wedge dx_I)\\
&= \sum_{i=1}^n (\sum_{j=1}^n \frac{\partial^2 f}{\partial x_j \partial x_i} \cdot dx_j \wedge dx_i \wedge dx_I) \\
&= (\sum_{i,j=1}^n \frac{\partial^2 f}{\partial x_j \partial x_i} \cdot dx_j \wedge dx_i) \wedge dx_I \\
&= \sum_{i<j}( \frac{\partial^2 f}{\partial x_j \partial x_i}\cdot dx_j \wedge dx_i + \underbrace{\frac{\partial^2 f}{\partial x_i \partial x_j}}_{\frac{\partial^2 f}{\partial x_j \partial x_i}} \cdot \underbrace{dx_i \wedge dx_j}_{-dx_j \wedge dx_i}) \wedge dx_I = 0.
\end{align*}
\end{beweis}

\begin{bemerkung}
Eine $k$-Form $\Phi$ mit $d\Phi = 0$ heißt geschlossen. Eine $k$-Form $\Phi$, zu der es eine $k-1$-Form $\eta$ gibt mit $d\eta =\Phi$ heißt exakt. Lemma \ref{lem:4.2} besagt, dass jede exakte Form geschlossen ist.

\textbf{Frage:} Gilt auch die Umkehrung? Das heißt: Ist eine geschlossene Form stets exakt? 

Im Allgemeinen gilt dies nicht. Für ein einfach zusammenhängendes Gebiet $U\subset \MdR^n $ ist dies jedoch richtig (Lemma von Poincaré).
\end{bemerkung}


\textbf{Ziele:}
\begin{itemize}
\item Integration von Differetialformen
\item Satz von Stokes (Spezialfall)
\end{itemize}

\medskip

\begin{definition}
Sei $U\subset \MdR^n$ eine $\lambda^n$-messbare Menge und $\omega$ eine stetige $n$-Form auf $U$. Dann wird das Integral von $\omega$ über $U$ erklärt durch
\[
\int_U \omega \da  \int_U \langle \omega(x), e_1\wedge \dots \wedge e_n\rangle \, \lambda^n(dx)
\]
wobei das $U$ im linken Integral als Menge mit Orientierung (durch die rechts verwendete geordnete Standardbasis) zu verstehen ist. So legt man auch fest, dass
\[
\int_{-U} \omega \da  \int_U - \langle \omega(x), e_1\wedge \dots \wedge e_n\rangle \, \lambda^n(dx).
\]
\end{definition}

Niederdimensionale Mengen und Integration:
\begin{beispiel}
Sei $F = \{p\}$ eine 0-dimensionale Menge, sei $\omega$ eine 0-Form, das heißt eine Funktion $\omega \colon U\to \MdR$, $p\in U$. Dann wird erklärt
\[
\int_{F} \omega \da \omega(p) \ad \delta_p(\omega).
\]
\end{beispiel}

\begin{definition}
Sei $n\ge 1$.
\begin{enumerate}
\item Ein $(n-1)$-dimensionaler Quader $F$, der zur $i$-ten Koordinatenachse orthogonal ist, ist von der Form
\[
F = [a_1,b_1] \times \dots \times [a_i,b_i] \times\dots \times [a_n,b_n]
\]
mit $a_i=b_i$ und $a_j < b_j$ für $j\in\{1,\dots,n\}\setminus \{i\}$.
\item Orientierung von $F$ durch den $(n-1)$-Vektor
\[
\hat e_i \da \bigwedge_{j\ne i} e_j \da e_1\wedge \dots \wedge \cancel{e_i} \wedge \dots\wedge e_n.
\]
\item Integration einer $(n-1)$-Form $\omega$ über $F$. Sei $\omega\colon F\to \bw^{n-1}\MdR^n$ stetig (oder $\lambda^{n-1}$-messbar). Dann sei
\[
\int_F \omega \da \int_F \langle \omega(x), \hat e_i\rangle\, \lambda^{n-1}(dx)
\]
und
\[
\int_{-F} \omega \da \int_F -\omega = \int_F -\langle \omega(x), \hat e_i\rangle\, \lambda^{n-1}(dx).
\]
\item Seien $\alpha_k\in \MdR$, $F_k$ orientierte „Seitenflächen“ von $n$-dimensionalen Quadern, $k\in \MdN$. Sei $\sum \alpha_k F_k$ eine formale, endliche Linearkombination. Sei $\omega$ eine $(n-1)$-Form auf $U\subset \MdR^n$, $F_k\subset U$. Dann sei
\[
\int_{\sum \alpha_k F_k} \omega \da \sum \alpha_k\cdot \int_{F_k} \omega.
\]
\end{enumerate}
\end{definition}

\begin{definition}[Orientierter Rand]
Sei $R=[a_1,b_1]\times\dots\times [a_n,b_n]$ ein Quader mit $a_i<b_i$. Dann sei für $1\le i\le n$
\begin{align*}
R_i^+ &\da [a_1,b_1]\times \dots \times \{b_i\} \times \dots \times [a_n,b_n],\\
R_i^- &\da [a_1,b_1]\times \dots \times \{a_i\} \times \dots \times [a_n,b_n]
\end{align*}
und
\[
\partial_o R \da \sum_{i=1}^n (-1)^{i-1} (R_i^+-R_i^-)
\]
sei eine formale Linearkombination von Flächen.
\end{definition}

\begin{satz}
Seien $R=[a_1, b_1] \times \dots \times [a_n,b_n]$, $a_i < b_i$ und $\varphi$ 
eine $(n-1)$-Form der Klasse $C^k$ mit  $k \ge 1$, 
auf einer offenen Menge $U\subset\MdR^n$ mit $R\subset U$. Dann gilt
\[
\int_R d\varphi  = \int_{\partial_o R} \varphi.
\]
\end{satz}

\begin{beweis}
Sei zunächst $n=1$, $\varphi$ eine 0-form, das heißt eine Funktion auf $U$. Es gilt $d\varphi(x) = \varphi'(x)\cdot dx$. Nun gilt
\begin{align*}
\int_{\partial_o R}\varphi = \int_{\partial_o [a_1,b_1]}\varphi = 
\int_{\{b\}-\{a\}} \varphi = \varphi(b)-\varphi(a)
\end{align*}
und
\begin{multline*}
\int_R d\varphi = \int_{[a,b]} \varphi'(x)dx = \int_{[a,b]}\langle \varphi'(x)dx_1, e_1\rangle \, \lambda^1(dx) \\
= \int_a^b \varphi'(x) \,\lambda^1(dx) = \varphi(b)-\varphi(a) = \int_{\partial_oR}\varphi,
\end{multline*}
wobei der Hauptsatz der Differential- und Integralrechnung verwendet wurde. 

Sei nun $n\ge 2$. Dann hat $\varphi$ eine Darstellung der Form
\[
\varphi = \sum_{i=1}^n \varphi_i dx_1\wedge \dots \wedge \cancel{dx_i}\wedge \dots\wedge dx_n.
\]
Es genügt, zu zeigen, dass für $1\le i \le n$ gilt:
\[
\int_R d(\varphi_i \cdot dx_1\wedge \dots \wedge \cancel{dx_i} \wedge \dots\wedge dx_n) = 
\int_{\partial_0 R} \varphi_i \cdot dx_1\wedge \dots \wedge \cancel{dx_i} \wedge \dots\wedge dx_n.
\]

Zunächst ist 
\begin{align*}
d(\varphi_i \cdot dx_1\wedge \dots \wedge \cancel{dx_i} \wedge \dots\wedge dx_n)
&= \sum_{j=1}^n \frac{\partial \varphi_i}{\partial x_j} dx_j \wedge dx_1\wedge \dots \wedge \cancel{dx_i} \wedge \dots\wedge dx_n \\
&= (-1)^{i-1} \frac{\partial \varphi_i}{\partial x_i} dx_1\wedge \dots\wedge dx_n.
\end{align*}
Also
\begin{align*}
\int_R d(\varphi_i \cdot dx_1\wedge \dots \wedge \cancel{dx_i} \wedge \dots\wedge dx_n)
&= (-1)^{i-1} \int_R \frac{\partial \varphi_i}{\partial x_i} dx_1\wedge \dots\wedge dx_n \\
&= (-1)^{i-1} \int_R \langle \frac{\partial \varphi_i}{\partial x_i}(x) dx_1\wedge \dots\wedge dx_n, e_1\wedge\dots\wedge e_n\rangle \, \lambda^n(dx) \\
&= (-1)^{i-1} \int_R \frac{\partial \varphi_i}{\partial x_i}(x)\, \lambda^n(dx) \\
&= (-1)^{i-1} \cdot \left(\int_{R_i^+} \varphi_i\, d\HM^{n-1} - \int_{R_i^-} \varphi_i\, d\HM^{n-1}\right),
\end{align*}
wobei der Satz von Fibini und der Hauptsatz der Differential- und Integralrechnung verwendet wurden. 

Andererseits gilt
\begin{align*}
&\phantom{ {}=} \int_{\partial_o R} \varphi_i \cdot dx_1\wedge \dots \wedge \cancel{dx_i} \wedge \dots\wedge dx_n \\
&= \sum_{j=1}^n (-1)^{j-1}\Big( \int_{R_j^+} \varphi_i(x)\cdot dx_1\wedge \dots \wedge \cancel{dx_i} \wedge \dots\wedge dx_n - \int_{R_j^-} \varphi_i \cdot dx_1\wedge \dots \wedge \cancel{dx_i} \wedge \dots\wedge dx_n \Big) \\
&= \sum_{j=1}^n (-1)^{j-1}\Big( \int_{R_j^+}\varphi_i (x) \underbrace{\langle  dx_1\wedge \dots \wedge \cancel{dx_i} \wedge \dots\wedge dx_n , \hat e_j\rangle}_{=
\begin{cases}
0, &\text{für } j\ne i\\
1, &\text{sonst}
\end{cases}} \, \HM^{n-1}(dx)\\
&\hspace{4cm} - \int_{R_j^-} \varphi_i(x)\langle  dx_1\wedge \dots \wedge \cancel{dx_i} \wedge \dots\wedge dx_n , \hat e_j\rangle \, \HM^{n-1}(dx)\Big) \\
&= (-1)^{i-1} \Big( \int_{R_i^+} \varphi_i \, d\HM^{n-1} - \int_{R_i^+} \varphi_i \, d\HM^{n-1}\Big).
\end{align*}
Dies zeigt die Gleichheit.
\end{beweis}

\medskip

Im vorangehenden Beweis ist die Fallunterscheidung $n=1$ bzw. $n\ge 2$ nicht zwingend erforderlich. Der 
Fall $n=1$ kann dem Fall $n\ge 2$ untergeordnet werden. 

\medskip

\textbf{Spezialfall:} Divergenzsatz/Satz von Gauß-Green.

Sei $U\subset\MdR^n$ offen, $V\colon U\to \MdR^n$ ein Vektorfeld. Setze $V_i(x) \da \langle V(x),e_i\rangle$, $x\in U$. Sei $V$ von der Klasse $\mathcal C^r$, $r\ge 1$. Die Divergenz von $V$ ist
\[
\Div(V)(x) \da \sum_{i=1}^n \frac{\partial V_i}{\partial x_i}(x).
\]

Ist $\varphi$ eine $(n-1)$-Form auf $\MdR^n$ mit einer Darstellung der Form
\[
\varphi = \sum_{i=1}^n \varphi_i\cdot dx_1\wedge \dots \wedge \cancel{dx_i} \wedge \dots\wedge dx_n,
\]
so setzt man
\[
V(x) \da \sum_{i=1}^n (-1)^{i-1}\varphi_i \cdot e_i = 
\begin{pmatrix}
\varphi_1\\
\mathllap{-}\varphi_2\\
\varphi_3\\
\vdots \\
(-1)^{n-1} \varphi_n
\end{pmatrix}.
\]
Dann gilt
\begin{align*}
d\varphi 
&= (\sum_{i=1}^n (-1)^{i-1} \frac{\partial \varphi_i}{\partial x_i}) \cdot dx_1\wedge\dots\wedge d x_n\\
&= (\Div(V)(x)) \cdot dx_1\wedge\dots\wedge d x_n.
\end{align*}
Bezeichnet nun $n$ den äußeren Normaleneinheitsvektor von $R$ in $\partial R$, so folgt:

\begin{korollar}
Für ein $\mathcal C^1$-Vektorfeld auf einer Umgebung von $R$ gilt
\[
\int_R \Div(V)\, d\HM^n = \int_{\partial R} \langle V, n\rangle \, d\HM^{n-1}.
\]
\end{korollar}

\textbf{Zurückholen von Formen:}
Sei $U\subset\MdR^n$ offen, $F\colon U\to \MdR^m$ von der Klasse $\mathcal C^k$ mit $k\ge 1$. Sei $x\in U$ und sei $\varphi$ eine in $F(x)$ erklärte $r$-Form. dann wird eine $r$-Form $(F^\#\varphi)(x)$ erklärt als $r$-Kovektor durch:
\[
\Big(F^\#\varphi\Big)\big(x\big)(v_1,\ldots,v_r) \da \varphi\Big(F(x)\Big)\big(DF_x(v_1),\ldots,DF_x(v_r)\big).
\]
Im Spezialfall $r=0$ ist $\varphi$ eine Funktion und
\[
(F^\#\varphi)(x) = \varphi(F(x)) = (\varphi\circ F)(x).
\]

\begin{bemerkungen}
\item Ist $\varphi$ von der Klasse $\mathcal C^k$, $k\ge 0$, und $F$ von der Klasse $\mathcal C^{k+1}$, so ist $F^\#\varphi$ von der Klasse $\mathcal C^k$.
\item $F^\#\varphi(x)$ kann als lineare Abbildung $\bw_r\MdR^n\to \MdR$ aufgefasst werden.
\item Sei $L\colon V\to W$ linear. Dann wird durch
\begin{align*}
\bw_rL \colon \bw_r V &\to \bw_r W \\
v_1\wedge\dots\wedge v_r &\mapsto L(v_1)\wedge \dots \wedge L(v_r)
\end{align*}
eine lineare Abbildung erklärt.
\item $(F^\# \varphi)(x)(v_1\wedge\dots\wedge v_r) = (\varphi\circ F)(x)(\bw_r DF_x(v_1\wedge \dots\wedge v_r))$.
\end{bemerkungen}

\pagebreak[3]
Wir haben nun vier Operationen für Formen ($\wedge$, $d$, $f^\#$, $+$), für die nun Rechenregeln angegeben werden:
\begin{lemma}
\label{lem:4.5}
Seien $U\subset \MdR^n$, $V\subset\MdR^m$, $W\subset\MdR^l$ offene Mengen. Seien $f\colon U\to V$, $g\colon V\to W$ Abbildungen der Klasse $\mathcal C^r$, $r\ge 1$.
Für $k$-Formen $\varphi$, $\omega$ auf $V$, eine $h$-Form $\eta$  auf $V$ und eine $k$-Form $\zeta$ auf $W$ gelten die folgenden Aussagen:
\begin{enumerate}
\item $f^\#(\omega + \varphi) = f^\# \omega + f^\# \varphi$,
\item $f^\#(\varphi \wedge \eta) = (f^\# \varphi) \wedge (f^\# \eta)$,
\item $d(f^\# \varphi) = f^\#(d\varphi)$,
\item $(g\circ f)^\#\zeta = f^\#(g^\#(\zeta))$.
\end{enumerate}
\end{lemma}

\begin{beweis}
Die Aussagen (1), (2), (4) folgen leicht aus den Definitionen (Übung). Zum Nachweis von (3) sei 
zunächst $k=0$ und daher $\omega$ eine Funktion auf $V$. Dann gilt $f^\#\omega=\omega\circ f$. Für 
$v\in \MdR^n$ gilt
$$
d(f^\#\omega)_x(v)=d(\omega\circ f)_x(v)=D(\omega\circ f)_x(v)=D\omega_{f(x)}(Df_x(v))=d\omega_{f(x)}(Df_x(v))
=f^\#d\omega)_x(v),
$$
und damit die Behauptung im Fall $k=0$. Sei nun $k=1$ und $\omega=d\xi$ mit einer $0$-Form $\xi$ auf $V$. 
Dann gilt
$$
d(f^\#\omega)=d(f^\#d\xi)=d(d(f^\#\xi))=0=f^\#(dd\xi)=f^\#(d\omega).
$$
Die Aussage (3) folgt nun wegen (1) und (2) daraus, dass jede „einfache“ $k$-Form äußeres Produkt einer  $0$-Form und äußeren Ableitungen von $0$-Formen ist. 
\end{beweis}

\begin{definition}
Seien $R$ ein Quader in $\MdR^n$, $U\subset \MdR^n$ offen mit $R\subset U$. Sei $F\colon U\to \MdR^m$ von der Klasse $\mathcal C^k$ mit $k\ge 1$, injektiv und $DF_x$ injektiv für $x\in U$. Dann ist $F(R)$ eine $n$-dimensionale Fläche in $\MdR^m$, die mit $F_\# R$ bezeichnet wird. Formal erklärt man für $\alpha_i \in \MdR$ und Quader $R_i$  in $\MdR^n$
\[
F_\#(\sum_{i}\alpha_i R_i) \da \sum_i \alpha_i F_\# R_i.
\]
Ist $\omega$ eine $n$-Form auf einer Umgebung von $F(R)$ in $\MdR^m$, so sei
\begin{align*}
\int_{F_\# R} \omega &\da \int_R \langle \omega\circ F(x), \underbrace{\frac{\partial F}{\partial x_1}}_{\mathclap{=DF_x(e_1)}}\wedge\dots\wedge  \frac{\partial F}{\partial x_n}\rangle \, \lambda^n(dx)\\
&= \int_R \underbrace{\langle \omega \circ F(x), \bw_n DF_x(e_1\wedge\dots\wedge e_n)\rangle}_{\langle (F^\#\omega)_x, e_1\wedge\dots\wedge e_n\rangle}\, \lambda^n(dx) \\
&= \int_R F^\# \omega
\end{align*}
und analog für formale Linearkombination von Quadern.
\end{definition}

\begin{definition}
Für einen Quader $R$ in $\MdR^n$ (und analog für formale Linearkombination) erklärt man
\[
\partial_o F_\#R \da \sum_{i=1}^n (-1)^{i-1} (F_\#R_i^+ - F_\#R_i^-)= F_\#\partial_0 R.
\]
\end{definition}

\begin{satz}
Sei $R\subset \MdR^n$ ein Quader in $\MdR^n$, $U\subset\MdR^n$ offen mit $R\subset U$, $F\colon U\to\MdR^m$ sei $\mathcal C^k$ mit $k\ge 1$, injektiv und $DF_x$ injektiv für $x\in U$. Sei $\omega$ eine $(n-1)$-Form  auf einer Umgebung von $F(R)$ in $\MdR^m$ von der Klasse $\mathcal C^2$. Dann gilt:
\[
\int_{F_\# R} d\omega = \int_{\partial_o F_\# R} \omega.
\]
\end{satz}

\begin{beweis}
Man erhält
\begin{align*}
\int_{F_\# R} d\omega &= \int_R F^\#(d\omega) 
= \int_R d (F^\#\omega)
= \int_{\partial_oR} F^\# \omega 
= \int_{\F_\#\partial_oR} \omega 
= \int_{\partial_0 F_\#R} \omega.
\end{align*}
\end{beweis}

\section{Grundlagen und Beispiele}

Wir definieren im Folgenden eine Topologie auf Differentialformen und Strömen.

Sei $U\subset \MdR^n$ offen und
\[
\mathcal E^k(U)\da \{ (\varphi\colon U\to \bw^k\MdR^n) : \varphi \text{ ist von der Klasse }\mathcal C^\infty\}.
\]

Definiere zu $i\in\MdN_0$ und $K\subset U$, $K$ kompakt, eine Seminorm $\nu_K^i$ auf $\mathcal E^k(U)$ durch
\[
\nu^i_K(\varphi) \da \sup \{\|D^j\varphi(x)\| : 0 \le j \le i,\, x\in K\}.
\]
Es sei
\[
\mathcal O(\varphi, i, K,\ep) \da \{\psi \in \mathcal E^k(U) : \nu_K^i(\varphi-\psi) < \ep\}
\]
für $i\in\MdN_0$, $K\subset U$ kompakt, $\ep>0$ $\varphi \in \mathcal E^k(U)$. Diese Mengen bilden eine Subbasis einer Topologie $\mathcal O$ auf $\mathcal E^k(U)$. Dann ist $(\mathcal E^k(U),\mathcal O)$ ein topologischer Raum (genauer: ein lokal konvexer, Hausdorffscher, topologischer Vektorraum; vgl. Walter Rudin, Functional Analysis, Seite 7).

\begin{definition}
Es sei
\[
\mathcal E_k(U) \da \{( T\colon \mathcal E^k(U) \to \MdR) : T\text{ ist linear und stetig}\}.
\]
\end{definition}

Zu $\varphi \in \mathcal E^k(U)$, $a,b\in \MdR$, $a<b$ sei
\[
\mathcal O'(\varphi,a,b) \da \{ T\in \mathcal E_k(U) : a < T(\varphi) < b\}.
\]
Auf $\mathcal E_k(U)$ wird die „schwache Topologie“ betrachtet, das heißt $\mathcal O'(\varphi,a,b)$, $a,b\in \MdR$, $a<b$, $\varphi \in \mathcal E^k(U)$ bilden eine Subbasis dieser Topologie.

\begin{definition}[Träger]
Für $\varphi \in \mathcal E^k(U)$ sei
\[
\supp(\varphi) \da U \setminus \bigcup \{ W\subset U: \text{$W$ offen, } \varphi|_W = 0\}.
\]
Für $T\in \mathcal E_k(U)$ sei
\[
\spt(T) \da U \setminus \bigcup \{ W\subset U: \text{$W$ offen, } T(\varphi)=0 \text{ für alle }\varphi \in \mathcal E^k(U) \text{ mit } \supp(\varphi)\subset W\}.
\]
\end{definition}

\begin{lemma}
\label{lem:4.7}
\begin{enumerate}[\quad(a)]
\item Zu $T\in \mathcal E_k(U)$ gibt es $M>0$, $i\in\MdN_0$ und $K\subset U$ kompakt, so dass gilt:
\[
T(\varphi) \le M \cdot \nu_K^i (\varphi)
\]
für alle $\varphi\in\mathcal E^k(U)$.
\item Seien $T_i, T \in\mathcal E_k(U)$, $i\in\mathbb{N}$, und $T_i \stackrel{s}{\rightharpoonup} T$ für $i\to\infty$. Dann existiert $K\subset U$, $K$ kompakt, mit $\spt(T_i), \spt(T) \subset K$ für $i\in\mathbb{N}$.
\end{enumerate}
\end{lemma}

Wir betrachten jetzt Teilmengen von $\mathcal E^k(U)$. Sei hierzu $K\subset U$ kompakt und
\begin{align*}
\mathcal D_K^k(U) &\da \{\varphi\in\mathcal E^k(U): \supp(\varphi) \subset K\} \subset \mathcal E^k(U), \\
\mathcal D^k(U) &\da \bigcup \{\mathcal D_K^k(U) : K\subset U \text{ kompakt}\}.
\end{align*}
Auf $\mathcal D^k(U)$ wird die feinste Topologie betrachtet, für die alle Inklusionsabbildungen 
$$
i_K\colon \mathcal D_K^k(U) \to \mathcal D^k(U)),\qquad \varphi\mapsto\varphi
$$ 
stetig sind. Das heißt, $W\subset \mathcal D^k(U)$ ist offen genau dann, 
wenn $W\cap \mathcal D_K^k(U)$ offen ist in der Spurtopologie von $\mathcal E^k(U)$ auf $\mathcal D_K^k(U)$ 
für alle kompakten Teilmengen $K\subset U$.

\begin{definition}
Es sei
\[
\mathcal D_k(U) \da \{ (T\colon \mathcal D^k(U) \to \MdR) : \text{$T$ linear und stetig}\}.
\]
Auf $\mathcal D_k(U)$ wird durch die Subbasis
\[
\{T\in \mathcal D_k(U) : a < T(\varphi) < b\}
\]
für $a,b\in\MdR$, $a<b$, $\varphi\in\mathcal D^k(U)$ eine Topologie festgelegt.
\end{definition}

\begin{bemerkungenX}
\begin{itemize}
\item Jedes $\varphi\in \mathcal D^k(U)$ hat kompakten Träger.
\item $T\in D_k(U)$ hat im Allgemeinen keinen kompakten Träger.
\item $\mathcal D^k(U) \subset \mathcal E^k(U)$, $\mathcal E_k(U) \subset \mathcal D_k(U)$. Dies folgt etwa aus 
dem nachfolgenden Lemma \ref{lem:4.9}.
\end{itemize}
\end{bemerkungenX}

\begin{lemma}
\label{lem:4.8}
Seien $\varphi_i, \varphi\in \mathcal D^k(U)$, $i\in \MdN$. Es gilt
\[
\varphi_i \to \varphi\text{ für }i\to\infty
\]
genau dann, wenn es eine kompakte Menge $K\subset U$ gibt, so dass $\supp(\varphi_i), \supp(\varphi) \subset K$ für $i\in \MdN$ und für alle $j\in\MdN_0$ gilt $\| D_j(\varphi_i - \varphi)\| \to 0$ für $i\to\infty$.
\end{lemma}

\begin{lemma}
\label{lem:4.9}
Sei $T\colon \mathcal D^k(U) \to\MdR$ linear. Genau dann ist $T\in \mathcal D_k(U)$, wenn es zu jeder kompakten Menge $K\subset U$ ein $i\in\MdN_0$ und $M>0$ gibt mit
\[
T(\varphi) \le M\cdot \nu_K^i(\varphi)  \text{ für alle } \varphi \in \mathcal D_K^k(U).
\]
\end{lemma}

\begin{definition}
\begin{itemize}
\item Die Elemente von $\mathcal D_k(U)$ heißen $k$-dimensionale Ströme auf $U$. ($k=0$: Distributionen).
\item Die Elemente von $\mathcal E_k(U)$ heißen $k$-dimensionale Ströme auf $U$ mit kompakm Träger.
\end{itemize}
\end{definition}

\begin{beispiele}
\item Sei $g:\MdR\to \MdR$ stetig. $S_g \in \mathcal D_0(\MdR)$ wird erklärt durch
\[
S_g(f) \da \int_\MdR g(x)f(x)\, dx\text{ für } f\in \mathcal D^0(\MdR).
\]
Zum Nachweis sei $f\in \mathcal{D}_K^0(\MdR)$, $K\subset U$ kompakt. Wegen
\[
|S_g(f)| = |\int_\MdR g(x) f(x) \, d(x)| \le \int_R |g(x)|\underbrace{|f(g)|}_{\le \nu_k^0(f)} \, dx \le \nu_K^0(f) \cdot \underbrace{\int_K |g(x)|\, dx}_{\ad M}\]
und Lemma \ref{lem:4.9} ist dies ein Strom.
\item Sei $a\in\MdR$, $\delta_a(f) \da f(a)$ für $f\in \mathcal D^0(\MdR)$. Es ist $\delta_a\in\mathcal E_0(\MdR)$.
\item Sei $a\in\MdR$, $T(f) \da f'(a)$ für $f\in\mathcal D^0(\MdR)$. Es ist $T\in \mathcal E_0(\MdR)$.
\item Sei $a< b$, $a,b\in\MdR$. $\llbracket a,b\rrbracket \in \mathcal D_1(\MdR)$ ist gegeben durch
\[
\llbracket a,b\rrbracket(f(x)dx) \da \int_a^b f(x) dx \text{ für } f(x)dx \in \mathcal D^1(\MdR).
\]
\item Sei $M\subset\MdR^n$ eine $k$-dimensionale Untermannigfaltigkeit. Ist $M$ orientierbar, dann gibt es ein stetiges $k$-Vektorfeld $x\mapsto (\xi_1\wedge\dots\wedge \xi_k)_x\in T_xM$, für $x\in M$, mit der Eigenschaft $\|(\xi_1\wedge\dots\wedge\xi_k)_x\|=1$ für alle $x\in M$. (Die Existenz eines solchen stetigen $k$-Vektorfeldes ist äquivalent zur Orientierbarkeit von $M$; vgl. den Anhang, Abschnitt 5.4.) Dann ist $\llbracket M\rrbracket \in \mathcal D_k(\MdR)$ erklärt durch
\[
\llbracket M \rrbracket(\omega) \da \int_M \omega \da \int_M \langle \omega(x), (\xi_1\wedge\dots\wedge \xi_k)_x\rangle \,\HM^k(dx) \text{ für } \omega \in \mathcal D^k(\MdR^n).
\]
\item Sei $\xi\in\mathcal E^{n-k}(U)$, $U\subset \MdR^n$ offen, $0\le k \le n$. Dann sei $T_\xi \in \mathcal D_k(U)$ erklärt durch
\[
T_\xi(\omega) \da \int_U \omega\wedge\xi = \int_U \langle \omega\wedge \xi, e_1\wedge\dots\wedge e_n\rangle  \text{ für }\omega \in \mathcal D^k(U).
\]
\item Sei $T\in \mathcal D_k(U)$, $\psi\in\mathcal E^m(U)$, $m\le k$. Dann ist $T\MR \psi \in \mathcal D_{k-m}(U)$  erklärt durch
\[
(T\MR \psi)(\omega) \da T(\psi \wedge \omega) \text{ für } \omega \in \mathcal D^{k-m}(U).
\]
\end{beispiele}

\begin{definition}[Rand eines Stroms]
Sei $U\subset \MdR^n$ offen und $T\in \mathcal D_k(U)$. Für $k\ge 1$ ist der Rand $\partial T$  von $T $ erklärt durch $\partial T\in \mathcal D_{k-1}(U)$ mit 
\[
\partial T(\omega) \da T(d\omega)
\]
für $\omega\in\mathcal D^{k-1}(U)$.
\end{definition}

\begin{beispiele}
\item $a,b\in \MdR$, $a<b$:
\begin{align*}
\partial(\llbracket a,b\rrbracket) (f) = \llbracket a,b\rrbracket(df) = \llbracket a,b\rrbracket(f'(x)dx) = \int_a^b f'(x) dx = f(b)-f(a) = (\delta_b-\delta_a)(f)
\end{align*}
Also ist $\partial\llbracket a,b\rrbracket = \delta_b - \delta_a$.

\item Sei $g\in \mathcal C^\infty(\MdR)$. Dann ist $T_g\in \mathcal D_1(\MdR)$ erklärt durch
\begin{align*}
T_g(\omega(x)dx) \da \int_\MdR g(x) \omega(x) dx.
\end{align*}
Ferner sei $S_g\in\mathcal D_0(\MdR)$ wie im vorigen Beispiel. Dann folgt
\begin{align*}
\partial T_g(f) = T_g(df) = T_g(f'(x)dx) = \int_\MdR g(x) f'(x) dx = -\int_\MdR g'(x) f(x) dx = -S_{g'}(f) = S_{-g'}(f)
\end{align*}
Also ist $\partial T_g = S_{-g'}$.
\item Sei in (2) nun $g$ nur noch stetig, etwa $g(x) = |x|$. Dann ist $\partial T_g = S_{-\operatorname{sgn}}$.
\item Sei $M$ eine orientierte, kompakte $k$-dimensionale Untermannigfaltigkeit von $\MdR^n$ mit Rand $\partial M$ und $\llbracket M \rrbracket\in\mathcal D_k(\MdR)$ der induzierte Strom. Mit dem Satz von Stokes folgt: Für $\eta \in \mathcal D^{k-1}(U)$ gilt
\begin{align*}
\partial\llbracket M \rrbracket (\eta) = \llbracket M \rrbracket (d\eta) = \int_M d \eta = \int_{\partial M} \eta = \llbracket \partial M\rrbracket(\eta)
\end{align*}
also gilt $\partial\llbracket M \rrbracket = \llbracket \partial M\rrbracket$.
\item Sei $k \ge j + 1$, $\xi\in\mathcal E^{j}(U)$, $T\in\mathcal D_k(U)$ und $\omega \in \mathcal D^{k-j-1}(U)$. Es ist
\begin{align*}
\partial(T\MR \xi)(\omega)
&= T\MR\xi(d\omega) =  T(\xi\wedge d\omega)\\
&= T( (-1)^j d(\xi\wedge \omega) + (-1)^{j-1} d\xi \wedge\omega)\\
&= (-1)^jT(d(\xi\wedge\omega)) + (-1)^{j-1} T(d\xi \wedge \omega) \\
&= (-1)^j\partial T(\xi\wedge \omega) + (-1)^{j-1} (T\MR d\xi)(\omega) \\
&= (-1)^j( (\partial T) \MR \xi)(\omega) + (-1)^{j-1} (T\MR d\xi)(\omega),
\end{align*}
also gilt
\[
\partial(T\MR \xi) = (-1)^j ( (\partial T) \MR \xi) + (-1)^{j-1} (T\MR d\xi)
\]
und somit
\[
(\partial T) \MR \xi = T\MR (d\xi) + (-1)^j \partial (T\MR \xi).
\]
\item $\partial\partial T=0$, für $T\in\mathcal D_k(U)$ mit $k\ge 2$, da $\partial\partial T(\omega) = \partial T(d\omega) = T(dd\omega)=T(0)=0$.
\end{beispiele}

\begin{definition}[Masse von Differentialformen und Strömen]
\begin{itemize}
\item Euklidische Masse von Differentialformen $\omega\in\mathcal D^k(U)$ in $x\in U$:
\[
|\omega(x)| \da \big(\sum_{I\in I_k^n} \omega_I(x)^2\big)^{\frac12}
\]
\item Komasse von $\omega(x)$:
\[
\|\omega(x)\| \da \sup\{\omega(x)(v_1\wedge\dots\wedge v_k) : v_i\in\MdR^n, \|v_i\|\le 1\}
\]
\item Euklidische Masse eines Stromes $T\in\mathcal D_k(U)$:
\[
\underline{\MN}(T) \da \sup\{T(\omega) : |\omega(x)|\le 1 \ \forall x\in U\}
\]
\item Masse von $T$:
\[
{\MN}(T) \da \sup\{T(\omega) : \|\omega(x)\|\le 1 \ \forall x\in U\}
\]
\end{itemize}
\end{definition}

Wegen 
\[
|\omega(x)| \ge \|\omega(x)\| \ge \binom{n}{k}^{-\frac12}\cdot |\omega(x)|
\]
folgt
\[
\binom{n}{k}^{-1/2}  \cdot \MN(T) \le \underline{\MN}(T) \le \cdot \MN(T)
\]
mit einer Konstanten $c$, die nur von $n$ abhängt.

\begin{beispiel}
Ist $T=\llbracket M\rrbracket$, $M$ eine orientierbare, kompakte, $k$-dimensionale Untermannigfaltigkeit von $\MdR^n$, so gilt
\begin{align*}
\MN(T) &= \sup\{ \int_M\omega : \|\omega(x)\|\le 1 \ \forall x\in U\} \\
&= \sup\{ \int_M \underbrace{\langle \omega(x), \xi(x) \rangle}_{\mathclap{\le \|\omega(x)\|\cdot\|\xi(x)\| \le 1}}\, \HM^k(dx) : \|\omega(x)\|\le 1 \ \forall x\in U\} \\ 
&\le \HM^k(M).
Man kann zeigen, dass hier sogar Gleichheit gilt.
\end{align*}
\end{beispiel}

\begin{definition}
Eine Folge $T_i\in\mathcal D_k(U)$ konvergiert in der Massenorm gegen ein $T\in\mathcal D_k(U)$, falls $\MN(T_i - T)\to 0$ für $i\to\infty$.
\end{definition}

\begin{bemerkung}
\begin{itemize}
\item Ist $\MN(0)=0$, so gilt $T=0$.
\item Gilt $T_i\to T$ in der Massenorm, so gilt $T_i \stackrel{s}{\rightharpoonup} T$, denn:

Sei $T_j\to 0$ in der Massenorm für $j\to\infty$, das heißt $\MN(T_j) \to 0$. Für $\omega\in\mathcal D^k(U)$ gilt:
\begin{align*}
|T_j(\omega)| \le \MN(T_j) \cdot \sup_{x\in U}\|\omega(x)\| \to 0
\end{align*}
also $T_j(\omega) \stackrel{s}{\rightharpoonup} 0$ für $j\to\infty$.

Die Umkehrung gilt im Allgemeinen nicht: Seien $T_j = \delta_j\in\mathcal D_0(\MdR)$, $j\in \MdN$. Dann gilt $\mathbf{M}(T_j)=1$, aber $T_j \stackrel s\rightharpoonup 0$, da $T_j(f) \to 0$ für $j\to\infty$ und $f\in\mathcal{D}^0(\MdR)$.
\end{itemize}
\end{bemerkung}

\begin{lemma}
\label{lem:4.10}
Seien $T_j, T \in\mathcal D_k(U)$, $j\in\MdN$ und $T_j \stackrel s\rightharpoonup T$mfür $j\to\infty$. 
Dann gilt
\[
\MN(T) \le \liminf_{j\to\infty} \MN(T_j).
\]
\end{lemma}

\begin{beweis}
Sei $\MN(T)<\infty$. Sei $\ep >0$. Dann existiert $\omega\in\mathcal D^k(U)$ mit $\|\omega(x)\|\le 1$ für alle $x\in U$ und $T(\omega)\ge \MN(T) - \ep$. Daher folgt:
\[
\liminf_{j\to\infty} \MN(T_j) \ge \liminf_{j\to\infty} T_j(\omega) = T(\omega) \ge \MN(T)-\ep
\]
also  $\MN(T) \le \liminf_{j\to\infty} \MN(T_j)$.

Sei $\MN(T)=\infty$. Dann existiert zu $m\in\MdN$ ein $\omega$ wie oben mit $T(\omega) \ge m$. Weiter wie oben.
\end{beweis}

\emph{Beispiel}: Ströme mit minimaler Masse \\
$T = \llbracket B^2 \rrbracket \in \mathcal D_2 (\MdR^2)$, d.h. $T(\omega) = \int_B \langle\omega(x),e_1\wedge e_2\rangle \HM^2(dx)$, $\omega \in \mathcal D^2(\MdR^2)$. 

Frage: Hat $T$ minimale Masse unter allen 2-Strömen $S \in \mathcal D_2(\MdR^2)$ mit $\partial S = \partial T = \llbracket \partial B^2 \rrbracket$.

\begin{lemma}
Sei $T \in \mathcal D_n(U)$, $U \subset \MdR^n$ offen und $\MN(T) < \infty$. Es gebe ein $\Omega = d\varphi \in \mathcal D^n(U)$ mit $\|\Omega\|\leq1$ und $\MN(T) = T(\Omega)$. Dann gilt für alle $S\in\mathcal{D}_n(U)$ mit $\partial S = \partial T$:
$$
	\MN(T) \leq \MN(S).
$$
\end{lemma}

\begin{beweis}
$\MN(T) = T(\Omega) = T(d\varphi) = \partial T(\varphi) = \partial S(\varphi) = S(d\varphi) = S(\Omega) \leq \MN(S)$.
\end{beweis}

\emph{Zurück zum Beispiel}: Sei $\Omega \da dx_1 \wedge dx_2 \in \mathcal D_2(\MdR^2)$. Dann ist $\|\Omega\|=1$ und
$$
	T(\Omega) = \int_{B^2} \underbrace{\langle dx_1 \wedge dx_2, e_1 \wedge e_2 \rangle}_{=1} \HM^2(dx) = \HM^2(B^2) = \MN(T),
$$
da
$$
	T(\omega) = \int_{B^2} \underbrace{\langle \omega, e_1\wedge e_2 \rangle}_{\leq \|\omega\|\cdot\|e_1\wedge e_2\|} d\HM^2.
$$
Ferner gilt für $\varphi(x) = \frac12 (x_1 dx_2 - x_2 dx_1)$ gerade $d\varphi_x = \frac12 \cdot (dx_1 \wedge dx_2 - dx_2 \wedge dx_1) = \Omega$. Mit obigem Lemma folgt, dass $T$ minimierend ist. Zur Berechnung von 
$\partial T$ betrachten wir $\eta = \eta_1 dx_1 + \eta_2 dx_2 \in \mathcal D^1(\MdR^2)$.
\begin{align*}
	\partial T(\eta) = T(d\eta) &= T(\frac{\partial \eta_1}{\partial x_2} dx_2 \wedge dx_1 + \frac{\partial \eta_2}{\partial x_1} dx_1 \wedge dx_2) \\
	&= T((\frac{\partial \eta_2}{\partial x_1} - \frac{\partial \eta_1}{\partial x_2}) dx_1 \wedge dx_2) \\
	&= \int_{B^2} (\frac{\partial \eta_2}{\partial x_1} - \frac{\partial \eta_1}{\partial x_2}) \HM^2(dx) \\
	&= \int_{B^2} div \begin{pmatrix} \eta_2 \\ -\eta_1 \end{pmatrix} \HM^2(dx) \\
	&= \int_{S^1} \underbrace{\langle \begin{pmatrix}\eta_2\\-\eta_1\end{pmatrix}(x),x \rangle}_{= \eta_2(x)\cdot x_1 - \eta_1(x) x_2} \HM^1(dx) \\
	&= \int_{S^1} \langle \eta_1 dx_1 + \eta_2 dx_2, \underbrace{-x_2 e_1 + x_1 e_2}_{\xi \in T_x S^1} \rangle \HM^1(dx) \\
	&= \int_{S^1} \langle \eta, \xi \rangle d\HM^1 \\
	&= \llbracket S^1 \rrbracket (\eta).
\end{align*}


\section{Ströme mit lokalendlicher Masse}

Für Ströme mit lokalendlicher Masse liefert der Rieszsche Darstellungssatz eine „explizite“ Integraldarstellung. Dazu sei
\[
\MN_k(U) \da \{T\in \mathcal D_k(U) : \MN(T)<\infty\}
\]
die Menge der {\em Ströme endlicher Masse} und
\[
\NS_k(U) \da \{T\in\mathcal D_k(U) : \MN(T) < \infty \text{ und } \MN(\partial T)<\infty\}
\]
die Menge der {\em normalen Ströme}.

\begin{beispiel}
$T\in\mathcal D_1(\MdR)$ mit $T(\omega(x)\, dx)\da \omega(0)$. Dann ist $\|\omega(x)\,dx\| = |\omega(x)|$ und daher $\MN(T)=1$. Aber $\partial T(f) = T(df) = T(f'(x)\,dx) = f'(0)$ für $f\in\mathcal D^0(\MdR)$ und somit $\MN(\partial T) = \infty$.
\end{beispiel}

\textbf{Lokalisierung:} Sei $V\subset U\subset\MdR^n$, $V$ und $U$ offen, $T\in\mathcal D_k(U)$. Es sei
\[
\underline\MN_V(T) \da \sup\{T(\omega) : \phantom\|\mathllap|\omega(x)\mathrlap|\phantom\| \le 1 \ \forall x\in U,\, \supp(\omega)\subset V\}
\]
und
\[
\MN_V(T) \da \sup\{T(\omega) : \|\omega(x)\|\le 1 \ \forall x\in U,\, \supp(\omega)\subset V\}\mathrlap{.}
\]
Weiter seien definiert:
\begin{align*}
\underline\MN_{k,\loc}(U) &\da \{T\in \mathcal D_k(U) : \underline\MN_V(T) < \infty\ \forall V\subset U,\, V\text{ offen},\, \bar V \text{ kompakt in }U\}\\
\MN_{k,\loc}(U) &\da \{T\in \mathcal D_k(U) : \MN_V(T) < \infty\ \forall V\subset U,\, V\text{ offen},\, \bar V \text{ kompakt in }U\} \\
\underline\NS_{k,\loc}(U) &\da \{T\in \mathcal D_k(U) : \underline\NS_V(T) < \infty,\, \underline\NS_V(\partial T)<\infty \ \forall V\subset U,\, V\text{ offen},\, \bar V \text{ kompakt in }U\}\\
\NS_{k,\loc}(U) &\da \{T\in \mathcal D_k(U) : \NS_V(T) < \infty,\,\NS_V(\partial T) <\infty\ \forall V\subset U,\, V\text{ offen},\, \bar V \text{ kompakt in }U\}.
\end{align*}


\begin{satz}
Seien $U \subset \MdR^n$ offen und $T_i \in \mathcal D_k(U)$, $i \in \MdN$, mit
$$
	\sup_{i\in\MdN} \MN_V(T_i) < \infty \quad \text{ für alle } V \subset U \text{ offen }, \bar V \text{ kompakt }, \bar V \subset U.
$$
Dann gibt es eine Teilfolge $(T_{n_i})_{i\in\MdN}$ und $T\in\mathcal D_k(U)$ mit $T_{n_i}\stackrel s\rightharpoonup T$ für $i\to \infty$.
\end{satz}

\begin{beweis}[Skizze]
Verwende, dass Ströme stetige, lineare Funktionale auf $\mathcal D^k(U)$ (topologischer Vektorraum) sind. Jetzt kann man lokal den Satz von Banach-Alaoglu anwenden, der die Auswahl einer lokal schwach$^*$ konvergenten Teilfolge 
erlaubt. Diagonalargument.
\end{beweis}

Sei $U \subset \MdR^n$ offen und sei $\mu$ ein borelreguläres Maß auf $U$ mit $\mu(K)<\infty$ für $K\subset \MdR^n$ kompakt (Radonmaß). Sei $\xi: U\to\MdR^m$ eine $\mu$-messbare Abbildung und $\|\xi\| = 1$ $\mu$-fast-überall. Dann wird durch
$$
	L(f) \da \int_U \langle f(x), \xi(x) \rangle \mu(dx), \quad f \in \mathcal C_c(U,\MdR^n)
$$
ein lineares Funktional $L: \mathcal C_c(U,\MdR^m) \to \MdR$ erklärt. Es gilt
$$
	|L(f)| \leq \int_U |\langle f(x),\xi(x)\rangle| \mu(dx) \leq \mu(K) < \infty
$$
falls $f \in \mathcal C_c(U,\MdR^m)$, $\supp(f) \subset K$, $K\subset U$ kompakt, $\|f\|\leq1$. In dieser Situation 
gilt
$$\sup\{L(f) : f\in \mathcal C_c(U,\MdR^m), \|f\|\leq 1, \supp(f)\subset K\} < \infty
$$ für alle $K \subset U$, $K$ kompakt.

\begin{satz}[Riesz]
Sei $L \colon \mathcal C_c(U,\MdR^m)\to\MdR$, $U \subset \MdR^n$ offen, ein lineares Funktional, das 
$$\sup\{L(f): f\in \mathcal C_c(U,\MdR^m), \|f\|\leq 1, \supp(f)\subset K\} < \infty
$$ 
erfüllt. Dann existiert ein Radonmaß $\mu$ auf $U$ und eine $\mu$-messbare Abbildung $\xi: U\to\MdR^m$ mit $\|\xi(x)\| = 1$ für $\mu$-fast-alle $x\in U$ und
$$
	L(f) = \int_U \langle f(x),\xi(x)\rangle \mu(dx), \quad f\in \mathcal C_c(U,\MdR^m).
$$
Ferner gilt für $V\subset U$ offen:
$$
	\mu(V) = \sup\{L(f) : f \in \mathcal C_c(U,\MdR^m),\, \supp(f) \subset V,\, \|f\|\leq 1\}.
$$
\end{satz}

\begin{beweis}
Siehe L. Simon, Lecture Notes of the ANU, Canberra, GMT.
\end{beweis}

Als Folge erhält man für Ströme lokalendlicher Masse.

\begin{satz}
Sei $U \subset \MdR^n$ offen und $T \in \mathcal D_k(U)$. Dann sind äquivalent:
\begin{enumerate}[(1)]
	\item $T \in \MN_{k,\loc}(U)$.
	\item Es gibt ein Radonmaß $\mu_T$ auf $U$ und eine $\mu_T$-messbare Abbildung $\xi: U \to \bw_k \MdR^n$ mit $|\xi(x)|=1$ für $\mu_T$-fast-alle $x\in U$, so dass gilt:
	$$
		T(\omega) = \int \langle\omega(x),\xi(x)\rangle \mu_T(dx), \quad \omega \in \mathcal D^k(U).
	$$
	Hierbei ist für $V \subset U$ offen:
	$$
		\mu_T(V) = \sup\{ T(\omega) : \omega \in \mathcal D^k(U),\, \forall x \in U \colon |\omega(x)| \leq 1,\, \supp(\omega) \subset V \} = \underline\MN_V(T).
	$$
\end{enumerate}
\end{satz}

\begin{bemerkung}
\begin{enumerate}[(1)]
	\item In der Situation des Satzes sagt man, dass $T$ als Integral darstellbar ist.
	\item Auf $\bw^k \MdR^n$ gibt es die euklidische Norm $|\cdot|$, sowie die Komassen-Norm $\|\cdot\|$. Ferner existiert auf $\bw_k \MdR^n$ neben der euklidischen Norm $|\cdot|$ die Masse-Norm $\|\cdot\|$:
	$$
		\|\xi\| \da \sup\{ \langle \omega,\xi \rangle : \omega \in \bw^k \MdR^n,\, \|\omega\| \leq 1 \}
	$$
	\emph{Zusammenhänge:}:
	$$
		|\omega| \geq \|\omega\| \geq {\binom n k}^{-\frac12} \cdot |\omega|,\qquad \omega \in \bw^k\MdR^n,
	$$
$\|\omega\|=|\omega|$ für einen einfachen Kovektor $\omega$. Hierbei nennt man $\omega \in \bw^k\MdR^n$ einfach, falls es $\eta_1,\dots,\eta_k \in \bw^1 \MdR^n$ gibt mit $\omega=\eta_1\wedge\dots\wedge\eta_k$. Sei nun
	$$
		|\xi| \leq \|\xi\| \leq {\binom n k}^{\frac12} |\xi|,\\qquad, \xi \in \bw_k \MdR^n,\,\qquad\text{und } \|\xi\|=|\xi|
	$$
	für $\xi$ einfach. Für die Verknüpfung mit dem äußeren Produkt gilt dann
	$$
		\|\xi \wedge \eta\| \leq \|\xi\| \cdot \|\eta\|,\; \|\varphi \wedge \omega\| \leq \binom{p+q}p \|\varphi\| \|\omega\|,
	$$
	$\xi \in \bw_p \MdR^n$, $\eta \in \bw_q \MdR^n$, $\varphi \in \bw^p \MdR^n$, $\omega \in \bw^q \MdR^n$.
	$$
		|\langle \varphi, \xi \rangle| \leq \|\varphi\| \cdot \|\xi\|
	$$
	für $\varphi \in \bw^p\MdR^n$, $\xi \in \bw_p\MdR^n$.
	\item Im vorangehenden Satz setzt man $\overarrow T(x) \da \frac{\xi(x)}{\|\xi(x)\|}$ und $\|T\| \da \|\xi\| \cdot \mu_T$, wobei 
	$$
	\|T\|(M) = \int \mathbb{1}_M(x) \|\xi\| \mu_T(dx)$$ und erhält so:
	$$
		T(\omega) = \int_U \langle\omega(x),\overarrow T(x)\rangle \, \|T\|(dx)
	$$
	mit $\|\overarrow T\| = 1$, $\|T\|$-fast-überall auf $U$, $\|T\|(V) = \sup\{T(\omega : \omega \in \mathcal D^k(U),\, \|\omega(x)\| \leq 1 \text{ für } x\in U,\, \supp(\omega) \subset V\}$, sowie $\MN_V(T) = \|T\|(V)$.
	\item Ist $T$ durch ein Integral darstellbar, so erklärt man für $A \subset U$, $A$ Borelsch:
	$$
		(T\MR A)(\omega) \da \int_A\langle \omega,\overarrow T \rangle \, d\|T\|
	$$
	oder für eine beschränkte Borelfunktion $f: U \to \MdR$:
	$$
		(T\MR f)(\omega) \da \int_U f\cdot \langle\omega,\overarrow T\rangle \, d\|T\|.
	$$
\end{enumerate}
\end{bemerkung}


\section{Produkt, Push-forward und Homotopieformel}

\textbf{Produkt von Strömen.} Seien $U_1\subset \MdR^{n_1}$, $U_2\subset\MdR^{n_2}$ offen und $S\in \mathcal D_{m_1}(U_1)$, $T\in\mathcal D_{m_2}(U_2)$ Ströme. Im Folgenden sind $x_1,\ldots,x_{n_1}$ Koordinaten von $\MdR^{n_1} \subset \MdR^{n_1+n_2}$ und $y_1,\ldots,y_{n_2}$ sind Koordinaten (bzw. Koordinatenfunktionen) von $\MdR^{n_2} \subset \MdR^{n_1+n_2}$ (mit naheliegenden Identifikationen).

Sei $\omega \in \mathcal D^{m_1+m_2}(U_1\times U_2)$. Dann kann man $\omega$ in der Form
\[
\omega = \sum_{\substack{\alpha,\beta\\ \mathclap{|\alpha| + |\beta| = m_1+m_2}}}  \omega_{\alpha\beta}(x,y) dx_\alpha \wedge dy_\beta
\]
geschrieben werden.

\begin{definition}
Mit obiger Notation setzen wir
\[
S\times T(\omega) \da \sum_{\mathclap{\substack{\alpha,\beta\\ |\alpha|=m_1 \\ |\beta| = m_2}}} S_x( T_y (\omega_{\alpha\beta}(x,y) dy_\beta) dx_\alpha).
\]
\end{definition}

Man kann sich leicht überlegen, dass diese Definition korrekt ist, das heißt etwa, dass das Argument 
von $S$ im Definitionsbereich von $S$ ist und $S\times T\in \mathcal D_{m_1+m_2}(U_1\times U_2)$ wieder ein Strom ist.

\begin{satz}
\label{satz:4.15}Seien $S\in \mathcal D_{m_1}(U_1)$ und $T\in \mathcal D_{m_2}(U_2)$ Ströme.
\begin{enumerate}
\item Seien $p\colon \MdR^{m_1+m_2} \to \MdR^{m_1}$, $(x,y)\mapsto x$ und $q\colon \MdR^{m_1+m_2} \to \MdR^{m_2}$, $(x,y)\mapsto y$ die Projektionsabbildungen. Seien $\varphi \in \mathcal D^k(U_1)$, $\eta \in \mathcal D^{m_1+m_2-k}(U_2)$. Dann gilt:
\[
S \times T (p^\#\varphi \wedge q^\#\eta) = 
\begin{cases}
S(\varphi) \cdot T(\eta), & k=m_1, \\
0, & k\ne m_1
\end{cases}
\]
und für 
\[
\omega = \sum_{\alpha,\beta} \omega_\alpha(x) \omega_\beta(y) dx_\alpha \wedge dx_\beta = \big(\underbrace{\sum_\alpha \omega_\alpha(x) dx_\alpha}_{= \omega_1(x)}\big) \wedge \big( \underbrace{\sum_\beta \omega_\beta(y) dy_\beta}_{=\omega_2(y)}\big) 
\]
gilt
\[
S\times T(\omega) = S(\omega_1) \cdot T(\omega_2).
\]
\item $\spt(S\times T) = \spt(S) \times \spt(T)$.
\item $\partial(S\times T) = \partial S \times T + (-1)^{m_1} S\times \partial T$.
\item Seien $P:\MdR^{n_1} \to \MdR^{n_1+n_2}$, $x\mapsto (x,0)$, $Q:\MdR^{n_2}\to \MdR^{n_1+n_2}$, $y\mapsto (0,y)$. Haben $S$ und $T$ lokalendliche Massen, so auch $S\times T$ und
\[
S\times T(\cdot) = \int \langle \cdot, (\bw_{m_1}P) \overarrow S \wedge (\bw_{m_2}Q) \overarrow T\rangle \, d(\|S\|\otimes \|T\|)
\]
\end{enumerate}
\end{satz}

\begin{beweis}
\begin{enumerate}
\item[(3)] Sei $\omega = \omega_{\alpha\beta}(x,y) dx_\alpha\wedge dy_\beta \in \mathcal D^{m_1+m_2-1}(U_1\times U_2)$. Dann ist
\begin{align*}
d\omega = \sum_{i=1}^{n_1} \frac{\partial \omega_{\alpha\beta}}{\partial x_i} dx_i \wedge dx_\alpha\wedge  dy_\beta + \sum_{j=1}^{n_2}  \frac{\partial \omega_{\alpha\beta}}{\partial y_j} dy_j \wedge dx_\alpha\wedge dy_\beta.
\end{align*}
Hiermit folgt
\begin{align*}
\partial(S\times T)(\omega) &= S\times T(dw) \\
&= S(\sum_{i=1}^{n_1} T(\frac{\partial \omega_{\alpha\beta}}{\partial x_i} dy_\beta) dx_i \wedge dx_\alpha) + (-1)^{|\alpha|} S(T(\sum_{j=1}^{n_2} \frac{\partial \omega_{\alpha\beta}}{\partial y_j} dy_j \wedge dy_\beta) dx_\alpha)\\
&= S(d_x (T(\omega_{\alpha\beta} dy_\beta) dx_\alpha)) + (-1)^{|\alpha|} S(T(d_y (\omega_{\alpha\beta} dy_\beta) dx_\alpha))\\
&= \partial S(T(\omega_{\alpha\beta} dy_\beta)dx_\alpha)) + (-1)^{m_1} S(\partial T(\omega_{\alpha\beta} dy_\beta) dx_\alpha) \\
&= \partial S \times T (\omega_{\alpha\beta}(x,y) dx_\alpha\wedge dy_\beta) + (-1)^{m_1} (S\times \partial T) (\omega_{\alpha\beta}(x,y)d x_\alpha\wedge dy_\beta)\\
&= (\partial S \times T  + (-1)^{m_1} (S\times \partial T)) (\omega_{\alpha\beta}(x,y)d x_\alpha\wedge dy_\beta)\\
&= (\partial S \times T  + (-1)^{m_1} (S\times \partial T)) (\omega).
\end{align*}
\item[(4)] Sei $\omega = \omega_{\alpha\beta}(x,y) dx_\alpha\wedge dy_\beta \in \mathcal D^{m_1+m_2} (U_1\times U_2)$. Die Voraussetzung besagt, dass 
\[
S= \int_{U_1}\langle \cdot, \overarrow S\rangle d\|S\|
\]
und
\[
T=\int_{U_2}\langle \cdot, \overarrow T\rangle d\|T\|.
\]
Es folgt
\begin{align*}
S\times T (\omega) &= S_x(T_y(\omega_{\alpha\beta} dy_\beta) dx_\alpha) \\
&= \int \langle T(\omega_{\alpha\beta}dy_\beta) dx_\alpha, \overarrow S \rangle d\|S\| \\
&= \int T(\omega_{\alpha\beta} dy_\beta) \langle dx_\alpha, \overarrow S \rangle d\|S\| \\
&= \int_{U_1}\int_{U_2} \langle \omega_{\alpha\beta}dy_\beta,\overarrow T\rangle \, d\|T\| \langle dx_\alpha, \overarrow S\rangle \, d\|S\| \\
&= \int_{U_1\times U_2} \langle \omega_{\alpha\beta}(x,y) dy_\beta, \overarrow T(y)\rangle \langle dx_\alpha,\overarrow S(x)\rangle \, d(\|S\| \otimes \|T\|) \\
&= \int_{U_1\times U_2} \langle \omega_{\alpha\beta}(x,y) dx_\alpha\wedge dy_\beta, (\bw_{n_1}P) \overarrow S(x)\wedge (\bw_{n_2}Q)\overarrow T(y)\rangle \, d(\|S\| \otimes \|T\|).
\end{align*}
Fazit: Es gilt insbesondere
$$\|S\times T\| = \|S\|\otimes \|T|,\qquad \overarrow {S\times T} = (\bw_{n_1}P) \overarrow S\wedge (\bw_{n_2}Q)\overarrow T.$$ 
\end{enumerate}
\end{beweis}

\begin{beispiel}
Ist $T$ durch ein Integral darstellbar, so auch $\llbracket 0, 1\rrbracket \times T$ mit $\| \llbracket 0,1\rrbracket \times T\| = \lambda^1_{[0,1]} \otimes \|T\|$ und $\overarrow {\llbracket 0,1\rrbracket \times T} = $„$e_1\wedge \overarrow T$“ (hier wurden die Einbettungsabbildungen weggelassen).
\end{beispiel}

\textbf{Bild eines Stromes.} Seien $U\subset \MdR^n$ offen und $V\subset\MdR^m$ offen. Ferner seien $T\in \mathcal D_k(U)$ und $f\colon U\to V$ eine $\mathcal C^{\infty}$-Abbildung. \textbf{Voraussetzung:} $f|_{\spt(T)}$ sei eigentlich (das heißt, für $K\subset V$, $K$ kompakt sei $f^{-1}(K)\cap \spt(T)\subset U$ stets kompakt).

\begin{beispiel}
Seien $f\colon U\da (0,\infty) \to V\da \MdR$ und $T\da \llbracket 0,b\rrbracket \in \mathcal D_0(U)$ für $b> 0$. Dann ist
\[
f^{-1}([0,b])\cap \spt(T) = (0,b] \cap [0,b] = (0,b] \subset U
\]
nicht kompakt.
\end{beispiel}

\begin{definition}
Seien $f$ und $T$ wie oben. Sei $\omega  \in \mathcal D^k(V)$. Sei $\gamma\in\mathcal D_0(U)$ mit 
\[
\spt(T) \cap \underbrace{\supp(f^\#\omega)}_{\subset f^{-1}(\supp(\omega))} \subset \{\gamma=1\}^o.
\]
Dann setzt man
\[
(f_\# T) (\omega) \da T(\gamma \wedge f^\#\omega).
\]
\end{definition}

\begin{bemerkungen}
\begin{enumerate}
\item $\supp(f^\#\omega) \subset f^{-1}(\supp(\omega))$ und $\supp(\omega)\subset V$ ist kompakt, das heißt $\spt(T) \cap \supp(f^\#\omega) \subset \spt(T) \cap f^{-1}(\supp(\omega))$. Dabei ist $\spt(T) \cap f^{-1}(\supp(\omega))$ kompakt und $\spt(T) \cap \supp(f^\#\omega)$ abgeschlossen und damit auch kompakt.
\item Auf die Einführung von $\gamma$ kann man im Allgemeinen nicht verzichten, da $\spt(f^\#\omega)$ nicht kompakt sein muss.
\item Die Definition von $(f_\#T)(\omega)$ ist von der konkreten Wahl von $\gamma$ unabhängig. Seien nämlich $\gamma_1,\gamma_2$ wie oben. Es ist
\[
\supp((\gamma_1-\gamma_2) \wedge f^\#\omega) \cap \spt(T) = \emptyset.
\]
Daraus folgt mittels einer Zerlegung der Eins
\[
T( (\gamma_1-\gamma_2) \wedge f^\# \omega) = 0.
\]
Dies schließlich ergibt 
\[
T( \gamma_1 \wedge f^\#\omega) = T(\gamma_2 \wedge f^\#\omega).
\]
\item Manchmal geht es auch ohne $\gamma$.
\end{enumerate}
\end{bemerkungen}

\begin{lemma}
\label{lem:4.16}Seien $U\subset \MdR^n$, $V\subset \MdR^m$ offen, $T\in\mathcal D_k(U)$, $f\colon U\to V$ von der Klasse $\mathcal C^\infty$, wobei $f|_{\spt(T)}$ eigentlich ist. Dann gilt
\begin{enumerate}
\item $\spt(f_\# T) \subset f(\spt(T))$
\item $\partial(f_\# T) = f_\#(\partial T)$.
\item Ist $T$ durch ein Integral darstellbar, so gilt das auch für $f_\# T$ und
\[
\| f_\# T\| \le f_\#\big(\|T\| \MR \|(\bw_m Df) \overarrow T\|\big).
\]
Hierbei ist für  eine messbare Menge $\ A\subset V$ die rechte Seite erklärt durch
$$
f_\#\big(\|T\| \MR \|(\bw_m Df) \overarrow T\|\big)(A)=\int_{f^{-1}(A)} \|\bw_m Df_x \overarrow T(x)\|\, \|T\|(dx).
$$
\end{enumerate}
\end{lemma}

\begin{satz}[Homotopieformel]
\label{satz:4.17}
Seien $U\subset\MdR^n$, $V\subset\MdR^m$ offen, seien $f,g\colon U\to V$ von der Klasse $\mathcal C^\infty$ und $h\colon [0,1]\times U\to V$ von der Klasse $\mathcal C^\infty$ mit $h(0,\cdot) = f$ und $h(1,\cdot) = g$. Sei $T\in\mathcal D_k(U)$ und $h|_{[0,1]\times \spt(T)}$ sei eigentlich. Dann gilt
\[
g_\# T - f_\# T = h_\#(\llbracket 0,1\rrbracket \times \partial T) + \partial h_\#(\llbracket 0,1\rrbracket \times T),
\]
wobei  für $k=0$ der erste Term in der Summe entfällt.
\end{satz}

\begin{beweis}
Wegen $\spt(\llbracket 0,1\rrbracket \times T) = [0,1]\times \spt(T)$ und $\spt(\partial T) \subset \spt(T)$ und nach Voraussetzung sind alle Ströme erklärt. Nun gilt:
\begin{align*}
\partial h_\#(\llbracket 0,1\rrbracket \times T) 
&= h_\# \partial (\llbracket 0,1\rrbracket \times T) \\
&= h_\# (\partial \llbracket 0,1\rrbracket\times T + (-1)^1 \llbracket 0,1\rrbracket \times \partial T) \\
&= h_\# ( (\delta_1-\delta_0) \times T  - \llbracket 0,1\rrbracket \times \partial T) \\
&= h_\#  (\delta_1 \times T) - h_\#(\delta_0 \times T) - h_\#(\llbracket 0,1\rrbracket \times \partial T) \\
&= g_\# T - f_\# T - h_\# (\llbracket 0,1\rrbracket \times \partial T).
\end{align*}
Sei für den letzten Schritt $\tau\colon U \to \MdR \times U$, $x\mapsto (0,x)$. Dann ist $\delta_0\times T = \tau_\# T$. In der Tat:
\begin{align*}
\delta_0 \times T(\omega(t, x)dx) = \delta_0(T(\omega(t,x)dx) = T(\omega(0,x) dx)
\end{align*}
und
\begin{align*}
\tau_\# T( \omega(t,x)dx) = T(\gamma \wedge \tau^\# (\omega(t,x)dx)) = T(\gamma \wedge \omega(0,x)dx) = T(\omega(0,x) dx).
\end{align*}
Hiermit folgt
\begin{align*}
h_\# (\delta_0 \times T) = h_\#(\tau_\# T) = (h \circ \tau)_\#T = f_\# T.
\end{align*}
Die hierbei benutzte Eigenschaft $h_\#\circ \tau_\#  = (h \circ \tau)_\#$ ist leicht einzusehen.
\end{beweis}

\begin{beispiel}
Sei $T\in\mathcal D_k(\MdR^n)$ mit $\partial T =0$, $\spt(T)$ kompakt, $k\ge 1$. Nach der Homotopieformel gilt
\[
g_\# T - f_\# T = \partial h_\#(\llbracket 0,1\rrbracket \times T) + 0.
\]
Ist spezieller: $g(x) \da x$, $f(x) \da 0$, so gilt $g_\#T = T$, $f_\#T = 0$ und somit:
\[
T = \partial \underbrace{h_\#(\llbracket 0,1\rrbracket \times T)}_{\in\mathcal D_{k+1}(\MdR^n)}.
\]
\end{beispiel}

\begin{korollar}
Seien die Voraussetzungen wie in Satz \ref{satz:4.17}.
\begin{enumerate}[\quad(a)]
\item Ist $T$ durch ein Integral darstellbar, so gilt mit der affinen Homotopie $h(t,x) = (1-t)\cdot f(x) + t\cdot g(x)$:
\[
\MN(h_\#(\llbracket 0,1\rrbracket \times T)) \le \|T\|\big(|g-f| \cdot \max\{\|Df\|^k, \|Dg\|^k\}\big).
\]
\item Ist $\MN(T)<\infty$ und $h$ wie in (a), so gilt
\[
\MN(h_\#(\llbracket 0,1\rrbracket \times T)) \le \sup_{\mathclap{\spt(T)}} |g-f|\cdot \sup_{\spt(T)} \{\|Df\|^k, \|Dg\|^k\}\cdot \MN(T).
\]
\end{enumerate}
\end{korollar}

\begin{beweis}
Für $\psi\in\mathcal D^{k+1}(V)$ folgt:
\begin{align*}
h_\#(\llbracket 0,1\rrbracket\times T)(\psi)
&= \int_{(0,1)\times U} \langle e_1\wedge \overarrow T, h^\#\psi\rangle \, (\lambda^{1}\otimes \|T\|)(d(t,x)) \\
&= \int_{(0,1)\times U} \langle \psi(h(t,x)),\underbrace{\bw_{k+1}Dh_{(t,x)} (e_1\wedge \overarrow T)}_{\mathclap{=(g(x)-f(x))\wedge \underbrace{\bw_k D_xh_{(t,x)} \overarrow T}_{\mathclap{=(1-t) Df_x(\overarrow T) + t\cdot Dg_x(\overarrow T)}}}}\rangle \, (\lambda^1\otimes\|T\|)(d(t,x)) \\
&= \int_{(0,1)\times U} \|\psi(h(t,x))\| \cdot \big( (1-t) \|Df_x\|^k + t\cdot \|Dg_x\|^k\big) \, (\lambda^1 \otimes \|T\|)(d(t,x)).
\end{align*}
\end{beweis}

\begin{anwendung}
Sei $U$ sternförmig in $\MdR^n$ bezüglich $u\in U$. Betrachte: $g(x) \da x$, $f(x) \da u$, $x\in U$, $h(t,x)\da (1-t)\cdot u + t\cdot x$. Für $\psi \in \mathcal D^k(U)$ ist $f^\#\psi = 0$ $(k\ge 1)$ und $g^\#\psi = \psi$. 
Wir betrachten den speziellen Strom
\[
T(\beta) \da \langle \beta, \eta \rangle, \quad \beta\in \mathcal D^k(U),
\]
wobei $\eta \colon U \to \bw_k\MdR^n$  fest gewählt ist mit kompaktem Träger in $U$ und von der Klasse $\mathcal C^\infty$.

Dann gilt:
\begin{align*}
g_\# T (\psi) &= T(\gamma \wedge g^\#\psi) = T(\gamma \wedge \psi)
= \langle \gamma \wedge \psi, \eta \rangle = \langle \psi, \eta\rangle \\
f_\#T (\psi) &= 0\\
h_\#(\llbracket 0,1\rrbracket\times \partial T) (\psi) &= \llbracket 0,1\rrbracket \times \partial T(h^\#\psi) = \partial T \big( (h^\#\psi)_{\llbracket 0,1\rrbracket} \big ) = T\big(d ((h^\#\psi)_{\llbracket 0,1\rrbracket})  \big) \\
\partial h_\#( \llbracket 0,1\rrbracket\times T)(\psi) &= 
h_\# (\llbracket 0,1\rrbracket\times T)(d\psi) =  \dots
\end{align*}
Ist $d\psi = 0$, so erhält man aus der Homotopieformel für Ströme:
\[
\langle \psi, \eta\rangle = T\big(d( (h^\#\psi)_{\llbracket 0,1\rrbracket})\big) + 0 
\]
und damit
\[
\langle \psi, \eta\rangle = \langle d( (h^\#\psi)_{\llbracket 0,1\rrbracket}), \eta \rangle.
\]
Dies zeigt
\[
\varphi = d( (h^\#\psi)_{\llbracket 0,1\rrbracket} ).
\]
Also ist $\psi$ exakt. Dies ist ein Beweis des Lemmas von Poincaré.
\end{anwendung}

\textbf{Bild eines Stromes unter einer Lipschitzabbildung}. Seien $U\subset\MdR^n$ offen,  $T\in \NS_{k,\loc}(U)$  und sei $f\colon U\to V$ Lipschitz sowie $f|_{\spt(T)}$ eigentlich. Zu $f$ gibt es eine Folge $(f_i)_{i\in\MdN}$ von $\mathcal C^\infty$"~Abbildungen von $U$ nach $V$ mit einer globalen Schranke für die Lipschitzkonstante, wobei $f_i \to f$ gleichmäßig konvergiert. Mit der Homotopieformel sieht man nun
\[
\big|({f_i}_\# T)(\omega) - ({f_j}_\# T)(\omega) \big| \le c\cdot \sup_{\mathclap{f^{-1}(K) \cap \spt(T)}} |f_i - f_j|,
\]
falls $K\subset V$, $K$ kompakt und $\supp(\omega) \subset K^o$. Folglich ist $({f_i}_\#T)(\omega)$ eine Cauchyfolge reeller Zahlen, und es existiert also
\[
(f_\# T)(\omega) \da \lim_{i\to \infty} ({f_i}_\# T) (\omega).
\]
Man kann zeigen:
\begin{itemize}
\item $f_\# T \in \NS_{k,\loc}(V)$.
\item Die Definition ist von der Wahl der Folge unabhängig.
\item $\partial f_\#T = f_\#\partial T$.
\item $\spt(f_\#T) \subset f(\spt(T))$.
\end{itemize}

\section{Rektifizierbare Ströme}

\begin{definition}
\begin{enumerate}[\quad(a)]
\item Sei $U\subset \MdR^n$. Ein Strom $T\in \mathcal D_k(U)$ heißt rektifizierbar, falls es
\begin{itemize}
\item eine $\HM^k$-messbare, abzählbar $k$-rektifizierbare Menge $M\subset\MdR^n$ mit $\HM^k(M\cap K)<\infty$ für $K\subset U$, $K$ kompakt,
\item eine $\HM^k$-messbare Abbildung $\xi\colon M\to\bw_k \MdR^n$ mit $\|\xi\| =1$ $\HM^k$-fast-überall auf $M$ und $\xi = v_1\wedge\dots\wedge v_k$ mit $v_i(x) \in \bw_k T_xM$ für $\HM^k$-fast-alle $x\in M$,
\item eine $\HM^k$-messbare Funktion $\theta\colon M \to [0,\infty]$
\end{itemize}
gibt, so dass gilt
\[
T(\omega) = \int_M \langle \omega , \xi \rangle \theta \, d\HM^k.
\]
\item Ist $\theta$ sogar ganzzahlig, so heißt ein solcher Strom $T$ ganzzahlig rektifizierbar. Die Menge der ganzzahlig rektifizierbaren Ströme wird mit $\mathcal R_k(U)$ bezeichnet.
\item T heißt integraler Strom, falls $T$ und $\partial T$ ganzzahlig rektifizierbare Ströme sind. Die Menge der integralen Ströme wird mit $\mathcal I_k(U)$ bezeichnet.
\end{enumerate}
\end{definition}

\begin{bemerkungen}
\item  Übersicht:
\[
\begin{array}[c]{ccc}
\mathcal I_k(U) & \subset & \mathcal R_k(U) \\
\rotatebox{90}{$\supset$}  & & \rotatebox{90}{$\supset$} \\
\NS_{k,\loc}(U) & \subset & M_{k,\loc}(U)
\end{array}
\]
\item Sei $T\in\mathcal R_k(U)$ und sei $\theta$ auf $M$ integrierbar. Dann ist
\[
\MN(T) = \int_M \theta \, d\HM^k.
\]
\item $\mathcal I_k(U) \subset \mathcal R_k(U) \cap \NS_{k,\loc}(U)$. Gilt „$\supset$“? 
Die positive Antwort wird nachfolgenden als Satz formuliert (Randrektifizierbarkeit).
\end{bemerkungen}

\begin{beispiele}
\item Sei $M\subset \MdR^n$ eine kompakte $k$-dimensionale Mannigfaltigkeit. Dann gilt
\[
\llbracket M \rrbracket \in \mathcal R_k(\MdR^n).
\]
\item Sei $M$ wie in (1) und $\HM^{k-1}(\partial M) < \infty$. Dann gilt
\[
\llbracket M \rrbracket \in \mathcal I_k(\MdR^n),
\]
denn $\partial \llbracket M \rrbracket = \llbracket \partial M \rrbracket$.
\item $\tilde T\in\mathcal D_1(\MdR^2)$ sei definiert durch
\[
\tilde T (\omega_1 dx + \omega_2 dy) \da \int_0^1 \omega_2 (s,0) \, ds.
\]
Dann ist $\tilde T$ nicht 1-rektifizierbar. Aber
\[
T(\omega_1 dx + \omega_2 dy) \da \int_0^1 \omega_1(s,0) ds
\]
ist 1-rektifizierbar.
\item Es ist
\[
T_j \da \sum_{i=1}^j \big\llbracket \{- \frac ij\} \times [0,\frac 1j] \big\rrbracket \in \mathcal R_1(\MdR^2) \cap \NS_1(\MdR^2).
\]
Es gilt $\MN(T_j) = 1$, $\MN(\partial T_j) = 2j$, $T_j\stackrel s \rightharpoonup \tilde T$ für $j\to \infty$. Wir erhalten so eine Folge integraler Ströme, deren schwacher Limes nicht rektifizierbar ist.
\end{beispiele}

\begin{satz}
\begin{enumerate}
\item (Randrektifizierbarkeit)  Sei $T\in\mathcal R_k(U)$ und $\MN_V(\partial T)<\infty$ für alle $V\subset U$ mit $V$ offen, so dass $\bar V$ kompakte Teilmenge von $U$ ist. Dann ist $T\in \mathcal I_k(U)$.
\item (Closure Theorem) Seien $T_j\in\mathcal R_k(U)$, $j\in\MdN$, und
$$\sup_{j\in\MdN}(\MN_V(T_j) + \MN_V(\partial T_j)) < C_V < \infty$$
für alle $V\subset U$ mit $V$ offen, so dass $\bar V$ eine kompakte Teilmenge von $U$ ist. Gilt $T_j \stackrel s \rightharpoonup T\in\mathcal D_k(U)$ für $j\to\infty$, so ist $T\in\mathcal R_k(U)$.
\item (Kompaktheitssatz) Seien $T_j$ wie in (2). Dann gibt es eine Teilfolge $T_{j'}$ von $T_j$ und $T\in\mathcal R_k(U)$ mit $T_{j'} \stackrel s\rightharpoonup T$.
\end{enumerate}
\end{satz}

\begin{beweis}
Idee: Simultaner Beweis von (1) und (2)/(3) durch vollständige Induktion über $k$. Aus (2)/(3) für $k-1$ und dem  Deformationssatz folgt (1) für $k$. Ferner geht das Konzept „Schnitt eines Stroms“ ein.
\end{beweis}

\textbf{Polyedrische Ströme} Sei $\ep>0$. Dann ist $[0,\ep]^n + \ep \cdot z$, $z\in\MdZ^n$ ein kompakter $\ep$-Würfel. Ein $k$-dimensionaler $\ep$-Würfel ist erklärt als das relative Innere einer $k$-dimensionalen  Seite eines solchen $\ep$"~Würfels.

\begin{definition}
Ein $k$-dimensionaler polyedrischer Strom in $\MdR^n$ der Seitenlänge $\ep$ ist ein Strom der Form
\[
P \da \sum_Q a_Q \llbracket Q\rrbracket,
\]
wobei $Q$ ein $k$-dimensionaler $\ep$-Würfel ist. Ein solcher Strom heißt ganzzahlig polyedrischer, falls $a_Q\in \MdZ$.
\end{definition}

\begin{bemerkungen}
\item $\MN(P) = \sum_Q |a_Q| \ep ^k$
\item $\MN(P) \le 2^k \MN(P)$. $\partial P$ ist stets polyedrisch.
\end{bemerkungen}

\begin{satz}[Deformationssatz]
Es gibt eine Konstante $c= c(n)$, so dass für jedes $T\in \NS_k(\MdR^n)$ und jedes $\ep >0$ ein $k$-dimensionaler polyedrischer Strom $P$ existiert und ferner $R\in \NS_{k+1}(\MdR^n)$ und $S\in \MN_k(\MdR^n)$ existieren, so dass gilt
\[
T = P + \partial R + S
\]
mit
\begin{enumerate}
\item $\MN(P) \le c\cdot \MN(T)$, $\MN(\partial P) \le c\cdot \MN(\partial T)$,
\item $\MN(R) \le c\cdot \ep \cdot \MN(T)$, $\MN(S) \le c \cdot \ep \cdot \MN(\partial T)$,
\item $\MN(\partial R) \le c \cdot \big( \MN(T) + \ep \cdot \MN(\partial T)\big)$,
\item $\spt(P), \spt(R) \subset \spt(T)_{\delta(\ep)}$  und $\spt(\partial P), \spt(\partial R) \subset \spt(\partial T)_{\delta(\ep)}$ mit $\delta(\ep)\to 0$ für $\ep \to 0$.
\item Ist $T\in \mathcal R_k(\MdR^n)$, so können $P$, $R$ als rektifizierbare Ströme gewählt werden. Ist $T\in\mathcal I_k(\MdR^n)$, so kann auch $S$ als rektifizierbarer Strom gewählt werden.
\end{enumerate}
\end{satz}

Wir formulieren noch einige Anwendungen:



\begin{satz}[Schwache Polyedrische Approximation] Sei $T\in\mathcal R_k(\MdR^n) \cap \NS_k(\MdR^n)$. Dann gibt es eine Folge $P_i$ von polyedrischen Strömen mit $P_i \stackrel s\rightharpoonup T$, wobei die Massen der $P_i$ uniform beschränkt sind.
\end{satz}

\begin{beweis}
Wähle $\epsilon_i:=1/i$ im Deformationssatz. Dann gibt es Ströme $P_i,R_i,S_i$ mit den im Deformationssatz 
beschriebenen Eigenschaften. Wegen (1) sind die Massen von $P_i$ und $\partial P_i$ uniform beschränkt. 
Aus $\MN(R_i)\le c\cdot \epsilon_i\MN(T)\to 0$ folgt $R_i\stackrel s \rightharpoonup 0$ und daher auch 
$\partial R_i \stackrel s \rightharpoonup 0$. Wegen $\MN(S_i)\le c\cdot \epsilon_i\MN(\partial T)\to 0$ folgt $S_i\stackrel s \rightharpoonup 0$. Insgesamt ist also $\partial R_i+S_i \stackrel s \rightharpoonup 0$ und daher 
$P_i \stackrel s \rightharpoonup T$.
\end{beweis}

Sei $T\in\mathcal D_k(\MdR^n)$, $1\le k\le n-1$ und $\spt(T)$ kompakt.

Gibt es $S\in\mathcal D_{k+1}(\MdR^n)$ mit $\partial S=T$? Notwendige Bedingung: 
$\partial T = 0$. Diese Bedingung ist auch hinreichend, wie wir schon gesehen haben.

{\bf Isoperimetrisches Problem}: Finde eine Massenschranke für die „Füllung S“ von $T$.

\begin{satz}[Isoperimetrische Ungleichung] 
Sei $T\in\mathcal R_k(\MdR^n)$, $\partial T=0$, $\spt(T)$ kompakt. Dann gibt es ein $R\in\mathcal R_{k+1}(\MdR^n)$ mit $\spt(R)$ kompakt, $\partial R=T$ und
\[
\MN(R) \le c\cdot \MN(T)^{\frac {k+1}k}
\]
mit $c = c(n)$.
\end{satz}


\begin{beweis}
Sei o.B.d.A. $\MN(T)\neq 0$. Setze $\epsilon:= (2c\MN(T)^{1/k}$, wobei $c$ wie im 
Deformationssatz gewählt wird. Zu $T$ seien $P,R,S$ wie im Deformationssatz gewählt. 
Wegen (2) und der Voraussetzung folgt $S=0$. Weiterhin gilt
\[
\MN(P)\le c\MN(T)=\frac{1}{2}\epsilon^k<\epsilon^k,
\]
und daher ist $\MN(P)=0$, das heißt $P=0$. Somit ist $P=\partial R$ und
\[
\MN(R)\le c\epsilon\MN(T)=c(2c\MN(T))^{1/k}\MN(T)=c'\MN(T)^{\frac{k+1}{k}}.
\]
Zusammen ergibt dies die Behauptung.
\end{beweis}


Schließlich ergeben die zur Verfügung stehenden Sätze auch einen 
raschen Beweis für eine Lösung des Plateauschen Problems in der Kategorie 
der Ströme. 

\begin{satz}[Plateau Problem]
Sei $T\in\mathcal I_k(\MdR^n)$, $\partial T=0$, $\spt(T)$ kompakt. Dann existiert 
$S\in \mathcal I_{k+1}(\MdR^n)$ mit $\partial S=T$, $\spt(S)$ kompakt und
\[
\MN(S)=\inf\{\MN(R):R\in\mathcal{I}_{k+1}(\MdR^n),\partial R=T\}.
\]
\end{satz} 

\begin{beweis}
Zu $T$ existiert ein $R\in \mathcal R_{k+1}(\MdR^n)$ mit $T=\partial R$ und $\spt(R)$ kompakt 
(vgl. den Beweis der isoperimetrischen Ungleichung). Da $T$ lokalendliche Masse hat, 
gilt dies auch für $\partial R$, so dass der Randrektifizierbarkeitssatz ergibt, dass 
$R\in \mathcal I_{k+1}(\MdR^n)$. Damit ist klar, dass sich das Infimum über eine nichtleere Menge 
erstreckt. Sei $R_i$, $i\in\mathbb{N}$ eine minimierende Folge. Wegen $\partial R_i=T$ 
ist die Voraussetzung des Kompaktheitssatzes erfüllt. Es existiert somit ein 
$S\in \mathcal R_{k+1}(\MdR^n)$ mit $T=\partial R_i\stackrel s \rightharpoonup \partial S$, also $T=\partial S$. 
Eine erneute Anwendung des Randrektifizierbarkeitssatzes zeigt sogar $S\in \mathcal I_{k+1}(\MdR^n)$. 
Wegen der Unterhalbstetigkeit der Masse ist auch $\MN(S)$ gleich dem Infimum. Durch die Projektion $\pi_\#(R_i)$
der Ströme einer minimierenden Folge auf die abgeschlossene, konvexe Hülle von $\spt(T)$ 
erreicht man, dass auch $\spt(S)$ kompakt ist.
\end{beweis}




\chapter{Anhang: Geometrische Integrationstheorie}

\section{Multilineare Algebra}

In diesem Abschnitt sind $V, W$ stets endlich-dimensionale reelle Vektorräume. Für $p 
\in {\mathbb N}$ sei $V^{p} \da  V \times \dots \times V$ das $p$-fache 
kartesische Produkt von $V$. Schließlich sei $V^{*}$ der duale 
Vektorraum der Linearformen auf $V$. Wir werden Tensoren und 
alternierende Tensoren auf elementarem Weg einführen und die Bildung 
von Restklassen sowie die Verwendung universeller Eigenschaften 
vermeiden. 

\bigskip

\noindent {\bf Definition.} Ein $p$-{\em Tensor} über $V$ ist eine 
multilineare Abbildung $T: V^{p} \to {\mathbb R}$. Die Menge ${\cal 
T}^{p}(V)$ aller $p$-Tensoren über $V$ ist mit den Operationen 
\begin{eqnarray*}
(S + T)(v_{1},\dots,v_{p}) & \da  & S(v_{1},\dots,v_{p}) + 
T(v_{1},\dots,v_{p}) \\
(\lambda S)(v_{1},\dots,v_{p}) & \da  & \lambda S(v_{1},\dots,v_{p})
\end{eqnarray*}
für $v_{1},\dots,v_{p} \in V$ 
ein reeller Vektorraum. Wir setzen ${\cal T}^{0}(V) \da  {\mathbb R}$.\\

\noindent
Die aufgestellten Behauptungen gelten offensichtlich. Ferner ist 
$V^{*} = {\cal T}^{1}(V)$.\\

\noindent
Eine wichtige Verknüpfung von Tensoren ist gegeben durch das Tensorprodukt.

\bigskip

\noindent {\bf Definition.} Für $S \in {\cal T}^{p}(V)$, $T \in 
{\cal T}^{q}(V)$ ist das {\em Tensorprodukt} $S \otimes T$ erklärt durch
\[ (S \otimes T)(v_{1},\dots,v_{p},v_{p+1},\dots,v_{p+q}) \da  
S(v_{1},\dots,v_{p})T(v_{p+1},\dots,v_{p+q}) \]
für $v_{1},\dots,v_{p+q} \in V$. Es gilt $S \otimes T \in {\cal 
T}^{p+q}(V)$ (leicht!).

\bigskip

\begin{satz}\label{Satz3.3.1} {Für $S, S_{i} \in {\cal T}^{p}(V)$, 
$T, T_{i} \in {\cal T}^{q}(V)$, $U \in {\cal T}^{r}(V)$ und $\lambda 
\in {\mathbb R}$ gilt}
\begin{enumerate}
\item[{\rm (a)}] $(S_{1} + S_{2}) \otimes T = S_{1} \otimes T + S_{2} \otimes T$,
\item[{\rm (b)}] $S \otimes (T_{1} + T_{2}) = S \otimes T_{1} + S \otimes T_{2}$,
\item[{\rm (c)}] $(\lambda S) \otimes T = S \otimes (\lambda T) = \lambda(S 
\otimes T)$,
\item[{\rm (d)}] $(S \otimes T) \otimes U = S \otimes (T \otimes U)$.
\end{enumerate}
\end{satz}

\noindent
{\em Wegen} (d) {\em können wir
\[ S \otimes T \otimes U \da  (S \otimes T) \otimes U \]
erklären.}\\

\noindent
In der linearen Algebra erklärt man zu einer linearen Abbildung $f: 
V \to W$ die duale lineare Abbildung $f^{*}: W^{*} \to V^{*}$. Dies 
wird jetzt verallgemeinert.

\bigskip

\noindent {\bf Definition.} Sei $f: V \to W$ eine lineare Abbildung. 
Die Abbildung $f^{*}: {\cal T}^{p}(W) \to {\cal T}^{p}(V)$ werde 
definiert durch
\[ (f^{*}T)(v_{1},\dots,v_{p}) \da  T(f(v_{1}),\dots,f(v_{p})) \]
für $v_{1},\dots,v_{p} \in V$ und $T \in {\cal T}^{p}(W)$. Die 
Abbildung $f^{*}$ ist linear (leicht!).\\

\noindent
Man beachte, daß $f^{*}$ eigentlich von $p$ abhängt.\\

\bigskip

\noindent {\bf Definition.} Ein $p$-Tensor $\omega \in {\cal 
T}^{p}(V)$ $(p \ge 2)$ heißt {\em alternierend}, wenn
\[ \omega(v_{1},\dots,v_{i},\dots,v_{j},\dots,v_{p}) = 
-\omega(v_{1},\dots,v_{j},\dots,v_{i},\dots,v_{p}) \quad \mbox{ für } 
i \not= j \]
und $v_{1},\dots,v_{p} \in V$ gilt. Sei $\Omega^{p}(V)$ die Menge 
aller alternierenden $p$-Tensoren; sei ferner $\Omega^{0}(V) \da  
{\mathbb R}$ und $\Omega^{1}(V) \da  {\cal T}^{1}(V)$. Dann ist 
$\Omega^{p}(V)$ ein linearer Unterraum von ${\cal T}^{p}(V)$; seine 
Elemente heißen auch $p$-Kovektoren über $V$.\\

\noindent
{\bf Beispiel.} Inst $n\da \text{dim}(V)$, so ist $\det\in\Omega^n(V)$.\\

\noindent
Sei $S_{p}$ die Gruppe der Permutationen (Bijektionen) von 
$\{1,\dots,p\}$. Sie hat $p!$ Elemente. Mit $(i,j) \in S_{p}$ wird die 
Permutation bezeichnet, die $i$ und $j$ vertauscht und die übrigen 
Elemente fest läßt. Jede solche Permutation heißt {\em 
Transposition}. Jedes $\sigma \in S_{p}$ $(p > 1)$ läßt sich als 
Produkt von Transpositionen schreiben. Ob die Anzahl der dabei 
verwendeten Transpositionen gerade oder ungerade ist, hängt nur von 
$\sigma$ ab. Im geraden Fall nennt man $\sigma$ gerade und setzt $\sgn(\sigma) = 1$, im ungeraden Fall heißt $\sigma$ ungerade und man 
setzt $\sgn (\sigma) = -1$. Die Abbildung sgn ist ein 
Gruppenhomomorphismus von $(S_{p},\circ)$ nach $(\{-1,1\},\cdot)$.

\bigskip

\noindent {\bf Definition.} Für $T \in {\cal T}^{p}(V)$ sei
\[ {\rm Alt}(T)(v_{1},\dots,v_{p}) \da  \frac{1}{p!} \sum_{\sigma \in 
S_{p}} \sgn (\sigma) T(v_{\sigma(1)},\dots,v_{\sigma(p)}) \]
für $v_1,\ldots,v_p\in V$.\\

\bigskip

\begin{satz}\label{Satz3.3.2} {Die Abbildung {\rm Alt}: ${\cal T}^{p}(V) 
\to \Omega^{p}(V)$ ist eine lineare Projektionsabbildung, d.h.}
\begin{enumerate}
\item[{\rm (a)}] $T \in {\cal T}^{p}(V) \Rightarrow {\rm Alt}(T) \in 
\Omega^{p}(V)$;
\item[{\rm (b)}] $\omega \in \Omega^{p}(V) \Rightarrow {\rm Alt}(\omega) = \omega$.
\end{enumerate}
\end{satz}

\bigskip

\noindent {\em Beweis.} Die Linearität ist offensichtlich gegeben.\\

\noindent
(a) Sei $i \not= j$ (fest), $(i,j) \in S_{p}$ die zugehörige 
Transposition und $\sigma' \da  \sigma \circ (i,j)$ für $\sigma \in 
S_{p}$. Dann ist
\begin{eqnarray*} 
&   & {\rm Alt}(T) (v_{1},\dots,v_{j},\dots,v_{i},\dots,v_{p}) \\
& = & \frac{1}{p!} \sum_{\sigma \in S_{p}} \sgn (\sigma) 
T(v_{\sigma(1)},\dots,v_{\sigma(j)},\dots,v_{\sigma(i)},\dots,v_{\sigma(p)}) \\
& = & \frac{1}{p!} \sum_{\sigma \in S_{p}} \sgn (\sigma) 
T(v_{\sigma'(1)},\dots,v_{\sigma'(i)},\dots,v_{\sigma'(j)},
\dots,v_{\sigma'(p)}) \\
& = & \frac{1}{p!} \sum_{\sigma \in S_{p}} - \sgn (\sigma') 
T(v_{\sigma'(1)},\dots,v_{\sigma'(i)},
\dots,v_{\sigma'(j)},\dots,v_{\sigma'(p)}) \\
& = & -{\rm Alt}(T)(v_{1},\dots,v_{i},\dots,v_{j},\dots,v_{p}).
\end{eqnarray*}

\noindent
(b) Für $\omega \in \Omega^{p}(V)$ und $\sigma = (i,j)$ gilt
\[ \omega(v_{\sigma(1)},\dots,v_{\sigma(p)}) = 
-\omega(v_{1},\dots,v_{p}) = \sgn (\sigma) 
\omega(v_{1},\dots,v_{p}). \]
Da jede Permutation Produkt von Transpositionen ist und da sgn 
multiplikativ ist, gilt diese Gleichung für jede Permutation 
$\sigma$. Also ist
\begin{eqnarray*}
{\rm Alt}(\omega)(v_{1},\dots,v_{p}) & = & \frac{1}{p!} \sum_{\sigma 
\in S_{p}} \sgn (\sigma) 
\omega(v_{\sigma(1)},\dots,v_{\sigma(p)}) \\
& = & \frac{1}{p!} \left(\sum_{\sigma \in S_{p}} 1\right) 
\omega(v_{1},\dots,v_{p}) = \omega(v_{1},\dots,v_{p}).
\end{eqnarray*}
\qed\\

\noindent
Jetzt wird die Bildung des Tensorprodukts mit der Alt-Operation 
verknüpft. Man beachte, daß i.a. $\omega \otimes \eta \notin 
\Omega^{p+q}(V)$ für $\omega \in \Omega^{p}(V)$, $\eta \in 
\Omega^{q}(V)$. Daher wendet man noch einmal den Alt-Operator an und 
normiert geeignet, um auf diesem Weg wieder zu einem alternierenden Tensor 
zu gelangen.

\bigskip

\noindent {\bf Definition.} Für $\omega \in \Omega^{p}(V)$, $\eta 
\in \Omega^{q}(V)$ ist das {\em alternierende} (oder {\em äußere}) 
{\em Produkt} $\omega \wedge \eta$ erklärt durch
\[ \omega \wedge \eta \da  \frac{(p+q)!}{p!q!} {\rm Alt}(\omega \otimes 
\eta). \]
Es gilt $\omega \wedge \eta \in \Omega^{p+q}(V)$.\\


\noindent Als nächstes halten wir die wichtigsten Eigenschaften des 
alternierenden Produktes fest.\\



\begin{satz}\label{Satz3.3.3} {Für $\omega, \omega_{i} \in 
\Omega^{p}(V)$, $\eta, \eta_{i} \in \Omega^{q}(V)$, $\lambda \in 
{\mathbb R}$ und für eine lineare Abbildung $f: W \to V$ gilt}
\begin{enumerate}
\item[{\rm (a)}] $(\omega_{1} + \omega_{2}) \wedge \eta = \omega_{1} \wedge 
\eta + \omega_{2} \wedge \eta$,
\item[{\rm (b)}] $\omega \wedge (\eta_{1} + \eta_{2}) = \omega \wedge \eta_{1} + 
\omega \wedge \eta_{2}$, 
\item[{\rm (c)}] $(\lambda \omega) \wedge \eta = \omega \wedge (\lambda \eta) = 
\lambda(\omega \wedge \eta)$,
\item[{\rm (d)}] $\omega \wedge \eta = (-1)^{pq} \eta \wedge \omega$,
\item[{\rm (e)}] $f^{*}(\omega \wedge \eta) = f^{*}(\omega) \wedge f^{*}(\eta)$.
\end{enumerate}
\end{satz}

\bigskip

\noindent {\em Beweis.} Die Aussagen (a), (b), (c) und (e) folgen 
leicht, der Nachweis zu (d) ist eine Übungsaufgabe. \qed\\

\noindent
Für das alternierende Produkt von drei Kovektoren gilt auch das 
Assoziativgesetz. Der Nachweis erfordert eine Vorbereitung.

\bigskip

\begin{lemma}\label{Lemma3.3.4} {Es gilt:}
\begin{enumerate}
\item[{\rm (a)}] {Für $S \in {\cal T}^{p}(V)$, $T \in {\cal T}^{q}(V)$ mit 
${\rm Alt}(S) = 0$ ist}
\[ {\rm Alt}(S \otimes T) = {\rm Alt}(T \otimes S) = 0. \]
\item[{\rm (b)}] {Für $\omega \in \Omega^{p}(V)$, $\eta \in \Omega^{q}(V)$, 
$\theta \in \Omega^{r}(V)$ ist}
\[ {\rm Alt}({\rm Alt}(\omega \otimes \eta) \otimes \theta) = {\rm 
Alt}(\omega \otimes \eta \otimes \theta) = {\rm Alt}(\omega \otimes 
{\rm Alt}(\eta \otimes \theta)). \]
\end{enumerate}
\end{lemma}

\bigskip

\noindent {\em Beweis.} (a) Sei $G \subset S_{p+q}$ die Untergruppe 
aller Permutationen von $\{1,\dots,p+q\}$, die $p+1,\dots,p+q$ fest 
lassen. Für $\tau \in S_{p+q}$ ist dann $\tau G$ eine 
Linksnebenklasse und $LNK \da  \{\tau G: \tau \in S_{p+q}\}$ ist die 
Menge aller Linksnebenklassen von $G$. Setze $v_{\tau(i)} =: 
w_{i}$ für $i = 1,\dots,p+q$ und ein fest gewähltes $\tau \in 
G$. Dann gilt
\begin{eqnarray*}
&   & \sum_{\sigma \in \tau G} \sgn (\sigma) 
S(v_{\sigma(1)},\dots,v_{\sigma(p)}) 
T(v_{\sigma(p+1)},\dots,v_{\sigma(p+q)}) \\
& = & \sum_{\sigma' \in G} \sgn (\tau \circ \sigma') 
S(v_{\tau(\sigma'(1))},\dots,v_{\tau(\sigma'(p))}) 
T(v_{\tau(\sigma'(p+1))},\dots,v_{\tau(\sigma'(p+q))}) \\
& = & \sgn (\tau) \left(\sum_{\sigma' \in G} \sgn (\sigma') 
S(w_{\sigma'(1)},\dots,w_{\sigma'(p)})\right) 
T(w_{p+1},\dots,w_{p+q}) \\
& = & \sgn (\tau) (p! {\rm 
Alt}(S)(w_{1},\dots,w_{p}))T(w_{p+1},\dots,w_{p+q}) \\
& = & 0.
\end{eqnarray*}
Da die Linksnebenklassen von $G$ eine disjunkte Zerlegung von 
$S_{p+q}$ bilden, folgt
\begin{eqnarray*}
&   & (p+q)! {\rm Alt}(S \otimes T)(v_{1},\dots,v_{p+q}) \\
& = & \sum_{\sigma \in S_{p+q}} \sgn (\sigma) 
S(v_{\sigma(1)},\dots,v_{\sigma(p)}) 
T(v_{\sigma(p+1)},\dots,v_{\sigma(p+q)}) \\
& = & \sum_{\tau G \in LNK} \sum_{\sigma \in \tau G} {\rm 
sgn}(\sigma) S(v_{\sigma(1)},\dots,v_{\sigma(p)}) 
T(v_{\sigma(p+1)},\dots,v_{\sigma(p+q)}) \\
& = & 0.
\end{eqnarray*}
Analog folgt ${\rm Alt}(T \otimes S) = 0$.\\

\noindent
(b) Es ist
\begin{eqnarray*}
&   & {\rm Alt}({\rm Alt}(\omega \otimes \eta) \otimes \theta) - {\rm 
Alt}(\omega \otimes \eta \otimes \theta) \\
& = & {\rm Alt}({\rm Alt}(\omega \otimes \eta) \otimes \theta - 
(\omega \otimes \eta) \otimes \theta) \\
& = & {\rm Alt}([{\rm Alt}(\omega \otimes \eta) - \omega \otimes 
\eta] \otimes \theta) \\
& = & 0
\end{eqnarray*}
nach (a), da ${\rm Alt}({\rm Alt}(\omega \otimes \eta) - (\omega 
\otimes \eta)) = 0$ ist. \qed\\

\bigskip

\begin{satz}\label{Satz3.3.5} {Für $\omega \in \Omega^{p}(V)$, 
$\eta \in \Omega^{q}(V)$, $\theta \in \Omega^{r}(V)$ gilt}
\[ (\omega \wedge \eta) \wedge \theta = \omega \wedge (\eta \wedge 
\theta). \]
\end{satz}

\bigskip

\noindent {\em Beweis.} Es gilt
\begin{eqnarray*}
(\omega \wedge \eta) \wedge \theta & = & \frac{(p+q+r)!}{(p+q)!r!} 
{\rm Alt}((\omega \wedge \eta) \otimes \theta) \\
& = & \frac{(p+q+r)!}{(p+q)!r!} {\rm Alt}\left(\frac{(p+q)!}{p!q!} 
{\rm Alt}(\omega \otimes \eta) \otimes \theta\right) \\
& = & \frac{(p+q+r)!}{p!q!r!} {\rm Alt}(\omega \otimes \eta \otimes 
\theta),
\end{eqnarray*}
wobei Lemma \ref{Lemma3.3.4} (b) verwendet wurde. Für die rechte Seite der 
behaupteten Gleichung erhält man denselben Ausdruck. \qed\\

\bigskip

\noindent
In der Folge können wir somit
\[ \omega \wedge \eta \wedge \theta \da  (\omega \wedge \eta) \wedge 
\theta = \omega \wedge(\eta \wedge \theta) \]
setzen und Klammern weglassen. Die im Beweis gewonnene Gleichung
\[ \omega \wedge \eta \wedge \theta = \frac{(p+q+r)!}{p!q!r!} {\rm 
Alt}(\omega \otimes \eta \otimes \theta) \]
überträgt sich sofort auf mehr als drei Faktoren. Insbesondere gilt 
für $\omega_{1},\dots,\omega_{p} \in \Omega^{1}(V) = V^{*}$ die 
Gleichung
\[ {\rm Alt}(\omega_{1} \otimes \dots \otimes \omega_{p}) = 
\frac{1}{p!} \omega_{1} \wedge \dots \wedge \omega_{p}. \]

\bigskip

\noindent {\bf Voraussetzung.} Nachfolgend schreiben wir $n$ für die 
Dimension des Vektorraums $V$.\\

\noindent {\bf Zur Erinnerung.} Sei $(e_{1},\dots,e_{n})$ eine Basis 
von $V$. Definiere $\varphi_{i} \in V^{*}$ durch
\[ \varphi_{i}(e_{j}) \da  \delta_{ij} = \left\{ \begin{array}{ll} 1, & 
\mbox{ für } i = j, \\
0, & \mbox{ für } i \not= j \end{array} \right. \]
für $i, j = 1,\dots,n$ und lineare Erweiterung, d.h.
\[ \varphi_{i}\left(\sum_{j=1}^{n}a_{j}e_{j}\right) = \sum_{j=1}^{n} 
a_{j} \varphi_{i}(e_{j}) = a_{i}. \]
Dann ist $(\varphi_{1},\dots,\varphi_{n})$ eine Basis des Dualraumes 
$V^{*}$. Sie heißt die zu $(e_{1},\dots,e_{n})$ duale Basis. Für $f 
\in V^{*}$ und $v = \sum_{i=1}^{n} a_{i}e_{i} \in V$ gilt
\[
f(v)  =  f\left(\sum_{i=1}^{n}a_{i}e_{i}\right) = \sum_{i=1}^{n} 
a_{i}f(e_{i}) = \sum_{i=1}^{n}f(e_{i}) \varphi_{i}(v) 
 =  \left(\sum_{i=1}^{n} f(e_{i}) \varphi_{i}\right) (v).
\]
Also ist
\[ f = \sum_{i=1}^{n} f(e_{i})\varphi_{i},\]
was man auch einfach durch Einsetzen von $e_1,\ldots,e_n$ bestätigen kann.\\
 
\noindent
Wir verallgemeinern nun diese Vorgehensweise, um eine Basis von 
$\Omega^{p}(V)$ anzugeben.

\bigskip

\begin{satz}\label{Satz3.3.6} {Sei $(e_{1},\dots,e_{n})$ eine Basis 
von V und $(\varphi_{1},\dots,\varphi_{n})$ die hierzu duale Basis von $V^*$. Dann ist 
die Menge
\[ B \da  \{ \varphi_{i_1} \wedge \dots \wedge \varphi_{i_p} : 1 \le 
i_{1} < \dots < i_{p} \le n\} \]
eine Basis von $\Omega^{p}(V)$; sie heißt die Standardbasis von $\Omega^p(V)$  bezüglich 
$(e_{1},\dots,e_{n})$. Insbesondere gilt}
\[ {\rm dim}\, \Omega^{p}(V) = {n \choose p}. \]
\end{satz}

\bigskip

\noindent {\em Beweis.} Sei zunächst $T \in {\cal T}^{p}(V)$. Sei 
$v_{1},\dots,v_{p} \in V$ und
\[ v_{i} = \sum_{j=1}^{n} a_{ij} e_{j}, \qquad i = 1,\dots,p. \]
Dann gilt
\begin{eqnarray*}
T(v_{1},\dots,v_{p}) & = & T\left(\sum_{i_{1}=1}^{n} a_{1i_{1}} 
e_{i_1},\dots,\sum_{i_{p}=1}^{n} a_{pi_{p}} e_{i_p} \right) \\
& = & \sum_{i_{1},\dots,i_{p} = 1}^{n} a_{1i_{1}} \cdots a_{pi_{p}} 
T(e_{i_1},\dots,e_{i_p}).
\end{eqnarray*}
Nun ist
\[
\varphi_{i_1} \otimes \dots \otimes \varphi_{i_p} 
(v_{1},\dots,v_{p})  =  \varphi_{i_1}(v_{1}) \cdots 
\varphi_{i_p}(v_{p}) 
 =  a_{1i_{1}} \cdots a_{pi_{p}},
\]
also
\[ T = \sum_{i_{1},\dots,i_{p} = 1}^{n} T(e_{i_1},\dots,e_{i_p}) 
\varphi_{i_1} \otimes \dots \otimes \varphi_{i_p}. \]
Jetzt sei speziell $T = \omega \in \Omega^{p}(V)$. Dann erhält man
\begin{eqnarray*}
\omega = {\rm Alt}(\omega) & = & \sum_{i_{1},\dots,i_{p} = 1}^{n} 
\omega(e_{i_1},\dots,e_{i_p}) {\rm Alt}(\varphi_{i_1} \otimes \dots 
\otimes \varphi_{i_p}) \\
& = & \frac{1}{p!} \sum_{i_{1},\dots,i_{p} = 1}^{n} 
\omega(e_{i_1},\dots,e_{i_p}) \varphi_{i_1} \wedge \dots \wedge 
\varphi_{i_p}.
\end{eqnarray*}
Für $\sigma\in S_p$ gilt
\[ \varphi_{i_1} \wedge \dots \wedge \varphi_{i_p} = {\rm 
sgn}(\sigma) \varphi_{i_{\sigma(1)}} \wedge \dots \wedge 
\varphi_{i_{\sigma(p)}} \]
sowie
\[ \omega(e_{i_1},\dots,e_{i_p}) = \sgn (\sigma) 
\omega(e_{i_{\sigma(1)}},\dots,e_{i_{\sigma(p)}}). \]
Die zweite Gleichung haben wir schon gezeigt, die erste folgt analog durch 
wiederholte Anwendung von Satz \ref{Satz3.3.3} (d) auf $1$-Kovektoren. Insbesondere 
ist
\[\omega(e_{i_1},\dots,e_{i_p})=0\quad\text{und}\quad \varphi_{i_1} 
\wedge \dots \wedge \varphi_{i_p}=0,\]
falls zwei der Indizes $i_{1},\dots,i_{p}$ gleich sind. Man erhält somit
\[ \omega = \sum_{1 \le j_{1} < \dots < j_{p} \le n} 
\omega(e_{j_1},\dots,e_{j_p}) \varphi_{j_1} \wedge \dots \wedge 
\varphi_{j_p}. \]
Dies zeigt, daß $B$ ein Erzeugendensystem ist. Sei nun eine 
Linearkombination
\[ \sum_{1 \le i_{1} < \dots < i_{p} \le n} a_{i_{1} \dots i_{p}} 
\varphi_{i_1} \wedge \dots \wedge \varphi_{i_p} = 0 \]
gegeben. Sei $1 \le j_{1} < \dots < j_{p} \le n$. Dann ist
\begin{eqnarray*}
\varphi_{i_1} \wedge \dots \wedge 
\varphi_{i_p}(e_{j_1},\dots,e_{j_p}) 
& = & p! {\rm Alt}(\varphi_{i_1} \otimes \dots \otimes 
\varphi_{i_p}) (e_{j_1},\dots,e_{j_p}) \\
& = & \sum_{\sigma \in S_{p}} \sgn (\sigma) 
\varphi_{i_1}(e_{j_{\sigma(1)}}) \cdots \varphi_{i_p} 
(e_{j_{\sigma(p)}}).
\end{eqnarray*}
Hier ist
\begin{eqnarray*}
\varphi_{i_1}(e_{j_{\sigma(1)}}) \cdots 
\varphi_{i_p}(e_{j_{\sigma(p)}}) & = & \left\{ \begin{array}{ll} 1, & 
\mbox{ falls } i_{1} = j_{\sigma(1)},\dots,i_{p} = j_{\sigma(p)} \\
0, & \mbox{ sonst} \end{array} \right. \\
& = & \left\{ \begin{array}{ll} 1, & \mbox{ falls } i_{1} = 
j_{1},\dots,i_{p} = j_{p}, \sigma = id \\ 0, & \mbox{ sonst,} 
\end{array} \right.
\end{eqnarray*}
da die Indizes der Größe nach geordnet sind. Es folgt
\[ 0 = \sum_{1 \le i_{1} < \dots < i_{p} \le n} a_{i_{1}\dots i_{p}} 
\varphi_{i_1} \wedge \dots \wedge 
\varphi_{i_p}(e_{j_1},\dots,e_{j_p}) = a_{j_{1}\dots j_{p}}. \]
Dies zeigt die lineare Unabhängigkeit von $B$. \qed\\

\bigskip

\noindent
Aus Satz \ref{Satz3.3.6} folgt insbesondere
\[ \Omega^{p}(V) = \{0\} \quad \mbox{ für } p > n \]
und
\[ {\rm dim}\, \Omega^{n}(V) = 1. \]

\bigskip

\begin{satz}\label{Satz3.3.7} {Sei $(e_{1},\dots,e_{n})$ eine Basis 
von V, sei $\omega \in \Omega^{n}(V)$ und sei $v_{i} = 
\sum_{j=1}^{n}a_{ij}e_{j}$ für $i = 1,\dots,n$. Dann ist}
\[ \omega(v_{1},\dots,v_{n}) = {\rm det}(a_{ij})_{i,j=1}^{n} 
\omega(e_{1},\dots,e_{n}). \]
\end{satz}

\bigskip

\noindent {\em Beweis.} Für $v_{i} = \sum_{j=1}^{n} a_{ij}e_{j}$, 
$i = 1,\dots,n$ definiere
\[ \eta(v_{1},\dots,v_{n}) \da  {\rm det} (a_{ij})_{i,j=1}^{n}. \]
Dann gilt $\eta \in \Omega^{n}(V)$. Wegen ${\rm dim}\, \Omega^{n}(V) = 
1$ ist $\omega = \lambda \eta$ mit einem $\lambda \in {\mathbb R}$. 
Ferner gilt
\[ \omega(e_{1},\dots,e_{n}) = \lambda \eta(e_{1},\dots,e_{n}) = 
\lambda.\]  
\qed\\

\bigskip

\noindent {\bf Bezeichnung.} Im folgenden verwenden wir gelegentlich 
die Abkürzung
\[ \sum_{1 \le i_{1} < \dots < i_{p} \le n} a_{i_{1}\dots i_{p}} 
\varphi_{i_1} \wedge \dots \wedge \varphi_{i_p} =: \sum_{I \in 
M_{p}^{n}} a_{I} \varphi_{I}. \]
Dabei durchläuft der Multi-Index $I$ die Menge
\[ M_{p}^{n} \da  \{(i_{1},\dots,i_{p}) \in \{1,\dots,n\}^{p}: 1 \le 
i_{1} < \dots < i_{p} \le n\}. \]
Insbesondere sei
\[ \varphi_{I} \da  \varphi_{i_1} \wedge \dots \wedge \varphi_{i_p} \]
für $I = (i_{1},\dots,i_{p})$.



\section{Differentialformen}


In diesem Abschitt betrachten wir Funktionen, 
deren Werte alternierende Tensoren sind.

\bigskip

\noindent {\bf Definition.} Sei $M \subset {\mathbb R}^{n}$ und $p \in {\mathbb N}_{0}$. 
Eine {\em Differentialform}\index{Differentialform} vom Grad $p$ (eine $p$-Form) auf $M$ 
ist eine Abbildung von $M$ in $\Omega^{p}({\mathbb R}^{n})$.

\bigskip

\noindent
Eine Differentialform vom Grad 0 ist also einfach eine reellwertige Funktion. 
Ist $\omega$ eine $p$-Form auf $M$, so ist $\omega(X) \in \Omega^{p}
({\mathbb R}^{n})$ für jedes $X\in\MdR^n$
 ein $p$-Kovektor über ${\mathbb R}^{n}$. Algebraische Operationen werden nun wieder 
 punktweise erklärt.
 
\bigskip

\noindent {\bf Definition.} Seien $\omega_{i},\omega,\eta$ Differentialformen auf $M$ ($\omega_{1},\omega_{2}$ vom selben Grad), sei $f: M \to {\mathbb R}$ eine Funktion. 
Dann werden die Differentialformen $\omega_{1} + \omega_{2}$, $f\omega$ und $\omega \wedge \eta$ erklärt durch
\begin{eqnarray*}
(\omega_{1} + \omega_{2})(X) & \da  & \omega_{1}(X) + \omega_{2}(X), \\
(f\omega) (X) & \da  & f(X)\omega(X) =: (f \wedge \omega)(X), \\
(\omega \wedge \eta)(X) & \da  & \omega(X) \wedge \eta(X). 
\end{eqnarray*}

\bigskip

\noindent {\bf Beispiel.} \begin{itemize}
\item Sei $f: M \to {\mathbb R}$ eine differenzierbare Funktion 
und $M \subset {\mathbb R}^{n}$ offen. Für $X \in M$ ist das Differential 
$Df_{X}: {\mathbb R}^{n} \to {\mathbb R}$ eine lineare Abbildung mit
\[ \lim_{H \to 0} \frac{f(X + H) - f(X) - Df_{X}(H)}{\|H\|} = 0. \]
Für das Differential einer reellwertigen Funktionen schreiben wir künftig 
$df_{X}$ statt $Df_{X}$. Insbesondere gilt also 
$df_{X} \in\Omega^{1}({\mathbb R}^{n}) = {\cal T}^{1}({\mathbb R}^{n})$ für $X \in M$. 
Das Differential $df$ von $f$, d.h. die Abbildung 
\[ df: M \to \Omega^{1}({\mathbb R}^{n}), \qquad X \mapsto df_{X} \]
ist eine Differentialform vom Grad 1 auf $M$.
\item Eine spezielle differenzierbare Abbildung auf $\MdR^n$ ist die Projektionsabbildung
\[ p_{i}: {\mathbb R}^{n} \to {\mathbb R}, \qquad (x_{1},\dots,x_{n}) \mapsto x_{i}, \]
wobei $i\in\{1,\ldots,n\}$. 
Anstelle von $dp_{i}$ schreiben wir prägnanter $dx^{i}$, $i = 1,\dots,n$. Dies sind also sehr spezielle Differentialformen vom Grad 1 mit 
\[(dx^{i})_{X} = p_{i}\quad \text{für } X \in {\mathbb R}^{n}\]
und
\[ (dx^{i})_{X}(E_{j}) = \delta_{ij}, \]
wenn $(E_{1},\dots,E_{n})$ die Standard-Orthonormalbasis des ${\mathbb R}^{n}$ ist.
Insbesondere ist also die Basis $((dx^{1})_{X},\dots,(dx^{n})_{X})$ die zu $(E_{1},\dots,E_{n})$ duale Basis,  
und zwar für alle $X\in\MdR^n$. 
Aus Satz \ref{Satz3.3.6} erhalten wir folglich zu jedem $X \in {\mathbb R}^{n}$ 
eine Basis von $\Omega^{p}({\mathbb R}^{n})$ und können $\omega_{X}$ für 
eine gegebene $p$-Form $\omega$ in dieser Basis ausdrücken.
\end{itemize}

\bigskip

\noindent {\bf Definition.} Sei $\omega$ eine Differentialform vom Grad $p$ auf $M$. 
Dann gibt es Funktionen $a_{i_{1} \dots i_{p}}: M \to {\mathbb R}$ 
für $1 \le i_{1} < \dots < i_{p} \le n$, die {\em Koordinatenfunktionen} von 
$\omega$, mit
\[ \omega_{X} = \sum_{1 \le i_{1} < \dots < i_{p} \le n}
 a_{i_{1}\dots i_{p}}(X)(dx^{i_1})_{X} \wedge \dots \wedge (dx^{i_p})_{X}, \quad X \in M, \]
abgekürzt 
\begin{equation}\label{3.4.1}
\omega = \sum_{1 \le i_{1} < \dots < i_{p} \le n} a_{i_{1} \dots i_{p}} dx^{i_1} 
\wedge \dots \wedge dx^{i_p} = \sum_{I \in M_{p}^{n}} a_{I} dx^{I}.
\end{equation}
Ist $M$ offen, so heißt die Form $\omega$ von der Klasse $C^{r}$ (mit $r \in {\mathbb N}_{0}$), 
wenn ihre Koordinatenfunktionen von der Klasse $C^{r}$ sind.\\ 

\noindent
{\bf Bemerkungen.} 
\begin{itemize}
\item Gilt (\ref{3.4.1}), so kann man die Koordinatenfunktion von $\omega$ ausdrücken in 
der Form
\[ a_{i_{1} \dots i_{p}}(X) = \omega_{X}(E_{i_1},\dots,E_{i_p}). \]
\item Für eine differenzierbare Funktion $f: M \to {\mathbb R}$ erhält man insbesondere
\[ df = \sum_{i=1}^{n} df(E_{i}) dx^{i} = \sum_{i=1}^{n} (\partial_{i}f) dx^{i}. \]
\end{itemize}
Wir hatten für eine lineare Abbildung $L: V \to W$ und $p \in {\mathbb N}$ die duale 
Abbildung $L^{*}: {\cal T}^{p}(W) \to {\cal T}^{p}(V)$ erklärt durch
\[ (L^{*}T)(v_{1},\dots,v_{p}) \da  T(Lv_{1},\dots,Lv_{p}), 
\quad T \in {\cal T}^{p}(W), v_{i} \in V. \]
Für $p = 0$ ist $L^{*}$ sinngemäß als die Identität auf ${\mathbb R}$ erklärt. 
Nun sei $M \subset {\mathbb R}^{n}$ offen und $F: M \to {\mathbb R}^{k}$ eine differenzierbare 
Abbildung. Für jedes $X \in M$ ist dann $DF_{X}: {\mathbb R}^{n} \to {\mathbb R}^{k}$ eine 
lineare Abbildung. Somit ist für $p \in {\mathbb N}$ auch
\[ (DF_{X})^{*}: \Omega^{p}({\mathbb R}^{k}) \to \Omega^{p}({\mathbb R}^{n}) \]
eine lineare Abbildung. Daß $(DF_{X})^{*}(T)$ für $T \in \Omega^{p}({\mathbb R}^{k})$ ein 
alternierender Tensor ist, folgt direkt aus der Definition. 
Wir können auf diese Weise jeder $p$-Form auf $F(M)$ eine $p$-Form auf $M$ zuordnen. 

\bigskip

\noindent {\bf Definition.} Sei $M \subset {\mathbb R}^{n}$ offen und 
$F: M \to {\mathbb R}^{k}$ eine differenzierbare Abbildung. 
Für eine $p$-Form $\omega$ auf $F(M)$ ist die $p$-Form $F^{*}\omega$ auf 
$M$ erklärt durch
\[ (F^{*}\omega)_{X} \da  (DF_{X})^{*}\omega_{F(X)}, \qquad X \in M. \]
Man sagt, $F^{*}\omega$ entstehe aus $\omega$ durch Zurückholen mittels $F$.\\

\noindent
Speziell für $p = 0$ bedeutet dies, daß 
\[ F^{*}\omega = \omega \circ F \]
gilt, da $(DF_{X})^{*}$ für $p=0$ die Identität auf ${\mathbb R}$ ist.\\

\noindent
Wir halten einige Rechenregeln fest.

\bigskip

\begin{satz}\label{Satz3.4.1} {Sei $M \subset {\mathbb R}^{n}$ offen und 
$F: M \to {\mathbb R}^{k}$ eine differenzierbare Abbildung mit 
$F = (f_{1},\dots,f_{k})$. Sei $dy^{i}$ das $i$-te Koordinatendifferential auf 
${\mathbb R}^{k}$. Dann gilt für $p$-Formen $\omega_{i},\omega$, Funktionen $g$ 
und $q$-Formen $\eta$, die jeweils  auf $F(M)$ erklärt sind,}
\begin{enumerate}
\item[{\rm (a)}] $F^{*}dy^{i} = df_{i}$, \qquad $i = 1,\dots,k$,
\item[{\rm (b)}] $F^{*}(\omega_{1} + \omega_{2}) = F^{*}\omega_{1} + F^{*}\omega_{2}$,
\item[{\rm (c)}] $F^{*}(g\omega) = g \circ F\, F^{*}(\omega)$, 
\item[{\rm (d)}] $F^{*}(\omega \wedge \eta) = F^{*}\omega \wedge F^{*}\eta$.
\item[{\rm (e)}] {Ist $N \subset {\mathbb R}^{k}$ offen, $F(M) \subset N$ und $G: N \to {\mathbb R}^{m}$ differenzierbar, dann gilt jede $p$-Form $\omega$ auf $G \circ F(M)$}
\[ (G \circ F)^{*}\omega = F^{*} G^{*}\omega. \]
\end{enumerate}
\end{satz}

\noindent {\em Beweis.} Sei $X \in {\mathbb R}^{n}$, $H \in {\mathbb R}^{n}$ und $(E'_{1},\dots,E'_{k})$ die Standardbasis des ${\mathbb R}^{k}$. Dann gilt
\begin{eqnarray*}
 (F^{*}dy^{i})_{X}(H) &=& (DF_{X})^{*}(dy^{i})_{F(X)}(H)\\
& = & (dy^{i})_{F(X)}(DF_{X}(H)) \\
& = & (dy^{i})_{F(X)}\left(\sum_{j=1}^{k}(Df_{j})_{X}(H)E'_{j}\right) \\
& = & (df_{i})_{X}(H).
\end{eqnarray*}
(b) und (c) gelten nach Definition, (d) folgt aus Satz \ref{Satz3.3.3} (e).\\

\noindent
(e) folgt aus
\begin{eqnarray*}
((G \circ F)^{*}\omega)_{X} & = & (D(G \circ F)_{X})^{*} \omega_{G \circ F(X)} \\
& = & (DG_{F(X)} \circ DF_{X})^{*} \omega_{G(F(X))} \\
& = & (DF_{X})^{*}\left((DG_{F(X)})^{*} \omega_{G(F(X))}\right) \\
& = & (DF_{X})^{*}\left((G^{*}\omega)_{F(X)}\right) \\
& = & \left(F^{*}(G^{*}\omega)\right)_{X} = (F^{*}G^{*}\omega)_{X}.
\end{eqnarray*}
\qed\\

\bigskip

\noindent
Wir kombinieren jetzt obige Rechenregeln und erhalten so
\begin{equation}\label{3.4.2}
F^{*}\left( \sum a_{i_{1} \dots i_{p}} dy^{i_1} \wedge \dots \wedge dy^{i_p}\right)
 = \sum a_{i_{1} \dots i_{p}} \circ F \, df_{i_1} \wedge \dots \wedge df_{i_p} \end{equation}
mit
\begin{equation}\label{3.4.3}
df_{i} = \sum_{j=1}^{n}(\partial_{j}f_{i}) \, dx^{j}.
\end{equation}
Man kann (\ref{3.4.3}) in (\ref{3.4.2}) einsetzen, um eine Darstellung bezüglich der Standardbasis zu erhalten. Wir betrachten explizit nur den folgenden Spezialfall.

\bigskip

\begin{satz}\label{Satz3.4.2} {Sei $M \subset {\mathbb R}^{n}$ offen 
und $F: M \to {\mathbb R}^{n}$ differenzierbar. Dann ist für eine Funktion 
$a: F(M) \to {\mathbb R}$} 
\[ F^{*}(adx^{1} \wedge \dots \wedge dx^{n}) = a \circ F {\rm det}\, 
(JF) \, dx^{1} \wedge \dots \wedge dx^{n}. \]
\end{satz}

\bigskip

\noindent {\em Beweis.} Wir erhalten
\begin{eqnarray*}
F^{*}(adx^{1} \wedge \dots \wedge dx^{n}) & = & a \circ F \, df_{1} 
\wedge \dots \wedge df_{n} \\
& = & a \circ F \left(\sum_{i_1}(\partial_{i_1}f_{1}) dx^{i_1} \wedge 
\dots \wedge \sum_{i_n}(\partial_{i_n} f_{n}) dx^{i_n}\right) \\
& = & a \circ F \sum_{i_{1},\dots,i_{n} = 1}^{n} \partial_{i_1} 
f_{1} \cdots \partial_{i_n} f_{n}\, dx^{i_1} \wedge \dots \wedge 
dx^{i_n} \\
& = & a \circ F \sum_{\sigma \in S_{n}} \partial_{\sigma(1)} f_{1} 
\cdots \partial_{\sigma(n)} f_{n} \, dx^{\sigma(1)} \wedge \dots 
\wedge dx^{\sigma(n)} \\
& = & a \circ F \sum_{\sigma \in S_{n}} \sgn (\sigma) 
\partial_{\sigma(1)} f_{1} \cdots \partial_{\sigma(n)} f_{n} \, 
dx^{1} \wedge \dots \wedge dx^{n} \\
& = & a \circ F {\rm det}(\partial_{i}f_{j})_{i,j=1}^{n} \, dx^{1} 
\wedge \dots \wedge dx^{n}.
\end{eqnarray*}
Alternativ kann man mit Satz \ref{Satz3.3.7} argumentieren:
\begin{eqnarray*}
&   & F^{*}(dx^{1} \wedge \dots \wedge dx^{n})_{X}(E_{1},\dots,E_{n}) 
\\
& = & (dx^{1})_{X} \wedge \dots \wedge 
(dx^{n})_{X}(DF_{X}(E_{1}),\dots,DF_{X}(E_{n})) \\
& = & (dx^{1})_{X} \wedge \dots \wedge (dx^{n})_{X} 
\left(\sum_{i=1}^{n}(\partial_{1} f_{i})_{X} 
E_{i},\dots,\sum_{i=1}^{n}(\partial_{n}f_{i})_{X}E_{i}\right) \\
& = & {\rm det}(\partial_{i}f_{j}(X))_{i,j=1}^{n} (dx^{1} \wedge 
\dots \wedge dx^{n})_{X}(E_{1},\dots,E_{n}),
\end{eqnarray*}
d.h.
\[ F^{*}(dx^{1} \wedge \dots \wedge dx^{n}) = {\rm 
det}(\partial_{i}f_{j})_{i,j=1}^{n} \, dx^{1} \wedge \dots \wedge 
dx^{n}. \]
\qed\\

\bigskip

\noindent
Von grundlegender Bedeutung ist der folgende Differentiationsprozeß 
für Differentialformen.

\bigskip

\noindent {\bf Definition.} Sei $M \subset {\mathbb R}^{n}$ offen und
\[ \omega = \sum_{1 \le i_{1} < \dots < i_{p} \le n} a_{i_{1} \dots 
i_{p}} dx^{i_1} \wedge \dots \wedge dx^{i_p} = \sum_{I \in M_{p}^{n}} 
a_{I} dx^{I}  \]
eine Differentialform vom Grad $p$ der Klasse $C^{1}$ auf $M$. Die 
{\em äußere Ableitung} oder das {\em äußere Differential} \index{{ä}ußere Ableitung} 
\index{{ä}ußeres Differential}  von $\omega$ ist 
die $(p+1)$-Form
\begin{eqnarray*}
d\omega & \da  & \sum_{1 \le i_{1} < \dots < i_{p} \le n} da_{i_{1} 
\dots i_{p}} \wedge dx^{i_1} \wedge \dots \wedge dx^{i_p} \\
& = & \sum_{I \in M_{p}^{n}} da_{I} \wedge dx^{I} \\
& = & \sum_{1 \le i_{1} < \dots < i_{p} \le n} \sum_{j=1}^{n} 
(\partial_{j} a_{i_{1} \dots i_{p}}) dx^{j} \wedge dx^{i_1} \wedge 
\dots \wedge dx^{i_p}.
\end{eqnarray*}
Die angegebene Definition ist konsistent mit dem Fall $p = 0$, wo 
für $f: M \to {\mathbb R}$ von der Klasse $C^{1}$ gilt
\[ df = \sum_{j=1}^{n}(\partial_{j}f) \, dx^{j}. \]
Der folgende Satz faßt die wichtigsten Rechenregeln für die äußere 
Differentiation von Differentialformen zusammen. 

\bigskip

\begin{satz}\label{Satz3.4.3} {Die $p$-Formen $\omega,\omega_{i}$  
und eine $q$-Form $\eta$ seien auf der offenen Menge $M\subset\MdR^n$ jeweils 
von der Klasse $C^{1}$. Dann gilt}
\begin{enumerate}
\item[{\rm (a)}] $d(\omega_{1} + \omega_{2}) = d\omega_{1} + d\omega_{2}$,
\item[{\rm (b)}] $d(\omega \wedge \eta) = d\omega \wedge \eta + (-1)^{p} \omega 
\wedge d\eta$,
\item[{\rm (c)}] $dd\omega = 0$, {\em falls $\omega$ von der Klasse $C^{2}$ ist.}
\item[{\rm (d)}] {\em Ist $M\subset\MdR^n$ offen, $F: M \to {\mathbb R}^{k}$ eine Abbildung von der Klasse 
$C^{2}$ und $\omega$ eine $p$-Form der Klasse $C^{1}$ auf einer offenen 
Obermenge von $F(M)$, so gilt $F^{*}d\omega = dF^{*}\omega$.}
\end{enumerate}
\end{satz}

\bigskip

\noindent {\em Beweis.} (a) ist klar. Zusammen mit (b) im Fall einer 
0-Form $\omega$ zeigt dies die Linearität der äußeren Ableitung.\\

\noindent
(b) Sei $f: M \to {\mathbb R}$ differenzierbar. Ist $I \in 
M_{p}^{n}$, $J \in M_{q}^{n}$, so gilt
\begin{equation}\label{3.4.4}
d(f\, dx^{I} \wedge dx^{J}) = df \wedge dx^{I} \wedge dx^{J}.
\end{equation}
Sind die Indexmengen zu $I, J$ nicht disjunkt, dann sind beide Seiten 
Null. Sonst gibt es $m \in {\mathbb N}$ und Indizes $1 \le k_{1} < 
\dots < k_{p+q} \le n$ mit
\[ dx^{I} \wedge dx^{J} = (-1)^{m} dx^{k_1} \wedge \dots \wedge 
dx^{k_{p+q}}, \]
so daß die Behauptung nach Definition gilt.\\

\noindent
Nun sei
\[ \omega = \sum_{I \in M_{p}^{n}} a_{I} dx^{I}, \quad \eta = \sum_{J 
\in M_{q}^{n}} b_{J} \, dx^{J}, \]
also
\[ d(\omega \wedge \eta) = \sum_{I,J} d\left(a_{I} b_{J} \, dx^{I} \wedge 
dx^{J}\right). \]
Mit (\ref{3.4.4}) und der gewöhnlichen Produktregel folgert man
\begin{eqnarray*}
 d \left( a_{I} b_{J} \, dx^{I} \wedge dx^{J}\right)& = &
d(a_{I}b_{J}) \wedge dx^{I} \wedge dx^{J} \\
& = & (b_{J} da_{I} + a_{I} db_{J}) \wedge dx^{I} \wedge dx^{J} \\
& = & da_{I} \wedge dx^{I} \wedge (b_{J} dx^{J}) + db_{J} \wedge 
(a_{I} dx^{I}) \wedge dx^{J} \\
& = & da_{I} \wedge dx^{I} \wedge b_{J} dx^{J} + (-1)^{1 \cdot p} 
a_{I} dx^{I} \wedge db_{J} \wedge dx^{J}.
\end{eqnarray*}
Summation liefert daher
\begin{eqnarray*}
 d(\omega \wedge \eta) 
& = & \left(\sum_{I} da_{I} \wedge dx^{I}\right) \wedge 
\left(\sum_{J} b_{J} dx^{J}\right) + (-1)^{p} \left(\sum_{I} a_{I} 
dx^{I}\right) \wedge \left(\sum_{J} db_{J} \wedge dx^{J}\right) \\
& = & d\omega \wedge \eta + (-1)^{p} \omega \wedge d\eta.
\end{eqnarray*}



\noindent
(c) Für eine Funktion $f$ der Klasse $C^{2}$ gilt zunächst
\[ df = \sum_{i=1}^{n} (\partial_{i}f) dx^{i}, \]
und weiter
\begin{eqnarray*}
d(df) & = & \sum_{i=1}^{n} d(\partial_{i}f) \wedge dx^{i} \\
& = & \sum_{i=1}^{n} \left(\sum_{j=1}^{n} \partial_{j} 
(\partial_{i}f) \wedge dx^{j}\right) \wedge dx^{i} \\
& = & \sum_{i < j} (\partial_{i} \partial_{j} - \partial_{j} 
\partial_{i}) \, dx^{i} \wedge dx^{j} \\
& = & 0.
\end{eqnarray*}
Wir erhalten also für $\omega = \sum_{I} a_{I} dx^{I}$
\begin{eqnarray*}
d\omega & = & \sum_{I} da_{I} \wedge dx^{I}, \\
dd\omega & = & \sum_{I} \left( dd a_{I} \wedge dx^{I}+(-1) da_{I} 
\wedge d 1\cdot dx^{I}\right) = 0.
\end{eqnarray*}

\noindent
(d) Seien $F$ und $\omega$ wie beschrieben mit $F = 
(f_{1},\dots,f_{k})$. Ist $\omega = g$ eine Funktion, d.h. $p = 0$, so 
gilt
\begin{eqnarray*}
F^{*}d\omega & = & F^{*}dg = F^{*} 
\left(\sum_{j=1}^{k}(\partial_{j}g) dx^{j}\right) \\
& = & \sum_{j=1}^{k} (\partial_{j}g) \circ F \, df_{j} \\
& = & \sum_{j=1}^{k} (\partial_{j}g) \circ F 
\sum_{i=1}^{n}(\partial_{i}f_{j})\, dx^{i} \\
& = & \sum_{i=1}^{n} \left(\sum_{j=1}^{k} (\partial_{j}g) \circ F 
\partial_{i} f_{j}\right) \, dx^{i} \\
& = & \sum_{i=1}^{n} \partial_{i}(g \circ F) \, dx^{i} \\
& = & d(g \circ F) = d(F^{*}g) = d(F^{*}\omega). 
\end{eqnarray*}
Sei schließlich $r \in {\mathbb N}_{0}$ und die Behauptung schon 
bewiesen für alle $r$-Formen. Um die Gleichung für alle 
$(r+1)$-Formen zu beweisen, die auf einer offenen Umgebung von $F(M)$ erklärt sind,
 genügt es, spezieller $(r+1)$-Formen der 
Form $\omega \wedge dy^{i}$ zu betrachten, wobei $\omega$ eine 
$r$-Form auf einer offenen Umgebung von $F(M)$ und $dy^i$ ein Koordinatendifferential  
im $\MdR^k$ ist. Es gilt
\begin{eqnarray*} 
 F^{*}(d(\omega \wedge dy^{i})) &=& F^{*}(d\omega \wedge dy^{i} + 
(-1)^{r} \omega \wedge ddy^{i}) \\
& = & F^{*}(d\omega \wedge dy^{i}) = F^{*} d\omega \wedge 
F^{*}dy^{i} \\
& = & dF^{*}\omega \wedge dF^{*} y^{i},
\end{eqnarray*}
wobei die Induktionsannahme  verwendet 
wurde. Mit Hilfe von Teil (c) folgt
\begin{eqnarray*}
 dF^{*}(\omega \wedge dy^{i}) &=& d(F^{*} \omega \wedge 
F^{*}dy^{i}) \\
& = & dF^{*} \omega \wedge F^{*} dy^{i} + (-1)^{r} F^{*} \omega 
\wedge dF^{*} dy^{i} \\
& = & dF^{*} \omega \wedge F^{*} dy^{i} + (-1)^{r} F^{*}\omega \wedge 
dd f_i \\
& = & dF^{*} \omega \wedge F^{*} dy^{i}.
\end{eqnarray*}
Ein Vergleich ergibt die Behauptung im betrachteten Fall. \qed\\

\bigskip

\noindent
Mit Hilfe des Differentialformen-Kalküls lassen sich die Operationen 
der klassischen Vektoranalysis übersichtlich zusammenfassen.

\bigskip

\noindent {\bf Beispiel.} Sei $M \subset {\mathbb R}^{3}$ offen und 
$V: M \to {\mathbb R}^{3}$ ein Vektorfeld der Klasse $C^{2}$. Wir 
definieren hierzu
\begin{eqnarray*}
(V) & \da  & v_{1} dx^{1} + v_{2} dx^{2} + v_{3}dx^{3} \\
((V)) & \da  & v_{1} dx^{2} \wedge dx^{3} + v_{2} dx^{3} \wedge 
dx^{1} + v_{3}dx^{1} \wedge dx^{2}.
\end{eqnarray*}
Dann bestätigt man leicht
\[ d(V) = (({\rm rot} V)), \quad d((V)) = {\rm div}(V)  dx^{1} 
\wedge dx^{2} \wedge dx^{3}. \]
Für $f: M \to {\mathbb R}$ von der Klasse $C^{2}$ gilt
\[ df = (\nabla f). \]
Hiermit erhält man
\[ 0 = dd(V) = d(({\rm rot} V)) = ({\rm div\, rot} (V)) dx^{1} 
\wedge dx^{2} \wedge dx^{3}, \]
also ${\rm div\, rot} (V) = 0$. Außerdem gilt
\[ 0 = ddf = d(\nabla f) = (({\rm rot} \nabla  
f)), \]
folglich ${\rm rot} \nabla f = 0$.\\

\noindent
Allgemein ist klar, daß wenn $\omega$ eine $p$-Form ist mit 
$\omega = d\eta$ für eine $(p-1)$-Form $\eta$ der Klasse $C^{2}$, 
dann gilt $d\omega = 0$. Dies wurde oben ausgenutzt.\\

\noindent
{\bf Definition.}  Man nennt eine 
Differentialform $\omega$ der Klasse $C^{1}$ 
{\em geschlossen}\index{geschlossene Differentialform}, falls 
$d \omega = 0$ gilt. Ferner heißt eine $p$-Form $\omega$ der 
Klasse $C^{0}$ {\em exakt}\index{exakte Differentialform}, 
falls es eine $(p-1)$-Form $\eta$ der Klasse 
$C^{1}$ gibt mit $\omega = d\eta$ (hier ist $p\ge 1$).\\


\noindent Bei der Definition einer geschlossenen Differentialform wird man in der Regel
an die Situation  $p \ge 1$ denken, auch wenn $p=0$ zugelassen ist. Der Fall $p = 
0$ ist nämlich trivial, da $d\omega = 0$ für eine 0-Form $\omega$ 
(d.h. eine Funktion) impliziert, daß $\omega$  auf jeder Zusammenhangskomponente konstant ist. In 
der somit eingeführten Terminologie können wir nun sagen, daß jede exakte 
Differentialform geschlossen ist.\\

\noindent
{\bf 
Gilt hiervon die Umkehrung, d.h. ist eine geschlossene 
Differentialform stets exakt?}\\

\noindent
Wir wissen schon (etwa für $p = 1$, $\omega = (V)$), daß dies im 
allgemeinen nicht richtig ist. Unter geeigneten Voraussetzungen an den 
zugrundeliegenden Definitionsbereich (etwa für sternförmige Mengen) 
gilt dagegen der folgende Satz.\\

\bigskip

\begin{satz}[Poincar\'{e}]\label{Satz3.4.4}  {Sei $M \subset {\mathbb 
R}^{n}$ ein sternförmiges Gebiet und $p\ge 1$. Dann ist jede geschlossene p-Form 
der Klasse $C^{1}$ auf M exakt.}
\end{satz}

\bigskip

\noindent {\em Beweis.} O.B.d.A. sei $M$ sternförmig bezüglich $0$. 
Wir schreiben die $p$-Form $\omega$ auf $M$ in der Form
\[ \omega = \sum_{1 \le i_{1} < \dots < i_{p} \le n} a_{i_{1} \dots 
i_{p}} \, dx^{i_1} \wedge \dots \wedge dx^{i_p} \]
und definieren eine $(p-1)$-Form $I\omega$ auf $M$ durch
\[ (I\omega)_{X} \da  \sum_{1 \le i_{1} < \dots < i_{p} \le n} 
\left(\int_{0}^{1} a_{i_{1} \dots i_{p}} (tX) t^{p-1}\, dt \right) 
\sum_{k=1}^{p} (-1)^{k-1} x_{i_k} dx^{i_{1}} \wedge \dots \wedge 
\widehat{dx^{i_k}} \wedge \dots \wedge dx^{i_p}. \]
Dabei ist $X = (x_{1},\dots,x_{n})$ gesetzt, $\widehat{dx^{i_k}}$ 
bedeutet, daß $dx^{i_k}$ weggelassen werden soll. Nun ist $\omega$ 
eine $p$-Form der Klasse $C^{1}$ auf $M$, für die wir $d(I\omega)$ 
und $I(d\omega)$ berechnen. Als Ergebnis dieser Rechnung werden wir
\begin{equation}\label{3.4.5}
d(I\omega) + I(d\omega) = \omega
\end{equation}
erhalten. Wegen $d\omega = 0$, folgt die gewünschte Behauptung 
$d(I\omega) = \omega$, d.h. die Exaktheit von $\omega$.\\

\noindent
Einerseits gilt
\begin{eqnarray*}
d(I\omega)_{X} & = & p \sum_{i_{1} < \dots < i_{p}} 
\left(\int_{0}^{1} a_{i_{1} \dots i_{p}} (tX) t^{p-1} \, dt \right) 
dx^{i_{1}} \wedge \dots \wedge dx^{i_p} \\
&   & + \sum_{i_{1} < \dots < i_{p}} \sum_{k=1}^{p} (-1)^{k-1} 
\sum_{j=1}^{n} \left(\int_{0}^{1} \partial_{j} a_{i_{1} \dots i_{p}} 
(tX) t^{p}\, dt \right) x_{i_k} \\
&   & \enspace dx^{j} \wedge dx^{i_1} \wedge \dots \wedge 
\widehat{dx^{i_k}} \wedge \dots \wedge dx^{i_p}.
\end{eqnarray*}
Andererseits erhält man aus
\[ d\omega = \sum_{i_{1} < \dots < i_{p}} \sum_{j=1}^{n} \partial_{j} 
a_{i_{1} \dots i_{p}} dx^{j} \wedge dx^{i_1} \wedge \dots \wedge 
dx^{i_p} \]
mit Hilfe einer elementaren, aber umständlich aufzuschreibenden 
Zusatzüberlegung ($j$ muß in $i_{1} < \dots < i_{k}$ eingereiht 
werden, bevor $I$ gebildet wird)
\begin{eqnarray*}
I(d\omega)_{X} & = & \sum_{i_{1} < \dots < i_{p}} \sum_{j=1}^{n} 
\left(\int_{0}^{1} \partial_{j} a_{i_{1} \dots i_{p}} (tX) t^{p}\, dt 
\right) x_{j} dx^{i_1} \wedge \dots \wedge dx^{i_p} \\
&   & - \sum_{i_{1} < \dots < i_{p}} \sum_{j=1}^{n} 
\left(\int_{0}^{1} \partial_{j} a_{i_{1} \dots i_{p}} (tX) t^{p}\, 
dt \right) \sum_{k=1}^{p} (-1)^{k-1} x_{i_k} \\
&   & \enspace dx^{j} \wedge dx^{i_1} \wedge \dots \wedge 
\widehat{dx^{i_k}} \wedge \dots \wedge dx^{i_p}.
\end{eqnarray*}
Addition ergibt nun
\begin{eqnarray*} 
 d(I\omega)_{X} + I(d\omega)_{X} 
& = & \sum_{i_{1} < \dots < i_{p}} \left[ p \left(\int_{0}^{1} 
a_{i_{1} \dots i_{p}} (tX) t^{p-1}\, dt \right) + \sum_{j=1}^{n} 
\left( \int_{0}^{1} \partial_{j} a_{i_{1} \dots i_{p}} (tX) t^{p} 
x_{j}\, dt \right) \right] \\
&   & dx^{i_1} \wedge \dots \wedge dx^{i_p} \\
& = & \sum_{i_{1} < \dots < i_{p}} \left(\int_{0}^{1} \frac{d}{dt} 
\left[a_{i_{1} \dots i_{p}} (tX) t^{p}\right] dt \right) dx^{i_1} 
\wedge \dots \wedge dx^{i_p} \\
& = & \sum_{i_{1} < \dots < i_{p}} a_{i_{1} \dots i_{p}} dx^{i_1} 
\wedge \dots \wedge dx^{i_p} \\
& = & \omega_{X}.
\end{eqnarray*}
Dies zeigt gerade (\ref{3.4.5}). \\

\noindent
\begin{color}{green}
Wer die besagten umständlichen Zusatzüberlegungen zur Bildung von $I(d\omega)_X$ sehen möchte, 
lese einfach weiter:\end{color}\\

\noindent
Wir halten $i_{1},\dots,i_{p}$ und $j$ fest, wobei es 
genügt, den Fall $j \notin \{i_{1},\dots,i_{p}\}$ zu betrachten. Wir 
erklären $j_{1} < \dots < j_{p+1}$ durch $\{j_{1},\dots,j_{p+1}\} = 
\{j,i_{1},\dots,i_{p}\}$. Steht jetzt $j$ an $m$-ter Stelle in 
$(j_{1},\dots,j_{p+1})$, so gilt
\begin{eqnarray*}
&   & \partial_{j} a_{i_{1} \dots i_{p}} \, dx^{j} \wedge dx^{i_1} 
\wedge \dots \wedge dx^{i_p} \\
&   & \qquad = \partial_{j} a_{i_{1} \dots i_{p}} (-1)^{m-1}\, 
dx^{j_1} \wedge \dots \wedge dx^{j_{p+1}}.
\end{eqnarray*}
Wir erhalten also
\begin{eqnarray*}
&   & I(\partial_{j} a_{i_{1}\dots i_{p}} \, dx^{j} \wedge dx^{i_1} 
\wedge \dots \wedge dx^{i_p}) \\
& = & (-1)^{m} \left(\int_{0}^{1} \partial_{j}a_{i_{1}\dots i_{p}}(tX) 
t^{p}\, dt\right) \sum_{k=1}^{p+1} (-1)^{k-1} x_{j_k} \, dx^{j_1} 
\wedge \dots \wedge \widehat{dx^{j_k}} \wedge \dots \wedge 
dx^{j_{p+1}}.
\end{eqnarray*}
Nun werden drei Fälle unterschieden:\\

\noindent
(a) $j_{k} = j$. Dann ist $m = k$ und
\[ (-1)^{m-1}(-1)^{k-1} x_{j_k} \, dx^{j_1} \wedge \dots \wedge 
\widehat{dx^{j_k}} \wedge \dots \wedge dx^{j_{p+1}} 
 =  x_{j} \, dx^{i_1} \wedge \dots \wedge dx^{i_p}.
\]

\noindent
(b) $j < j_{k}$. Dann gilt mit $j = j_{m} < j_{k}$:
\begin{eqnarray*}
&   & (-1)^{m-1}(-1)^{k-1} x_{j_k} \, dx^{j_1} \wedge \dots \wedge 
\widehat{dx^{j_k}} \wedge \dots \wedge dx^{j_{p+1}} \\
& = & (-1)^{m+k} x_{j_k} (-1)^{m-1}\, dx^{j} \wedge dx^{j_1} \wedge 
\dots \wedge \widehat{dx^{j_m}} \wedge \dots \wedge 
\widehat{dx^{j_k}} \wedge \dots \wedge dx^{j_{p+1}} \\
& = & (-1)^{k-1} x_{i_{k-1}} \, dx^{j} \wedge dx^{i_1} \wedge \dots 
\wedge \widehat{dx^{i_{k-1}}} \wedge \dots \wedge dx^{i_p} \\
& = & -(-1)^{k-2} x_{i_{k-2}} \, dx^{j} \wedge dx^{i_1} \wedge \dots 
\wedge dx^{i_{k-1}} \wedge \dots \wedge dx^{i_p}.
\end{eqnarray*}

\noindent
(c) $j > j_{k}$. Dann gilt mit $j = j_{m} > j_{k}$:
\begin{eqnarray*}
&   & (-1)^{m-1}(-1)^{k-1} x_{j_k} \, dx^{j_1} \wedge \dots \wedge 
\widehat{dx^{j_k}} \wedge \dots \wedge dx^{j_{p+1}} \\
& = & (-1)^{m+k} x_{i_k} \, dx^{i_1} \wedge \dots \wedge 
\widehat{dx^{i_k}} \wedge \dots \wedge dx^{j} \wedge \dots \wedge 
dx^{i_p} \\
& = & (-1)^{m+k} (-1)^{m-2} x_{i_k} \, dx^{j} \wedge dx^{i_1} \wedge 
\dots \wedge \widehat{dx^{i_k}} \wedge \dots \wedge dx^{i_p} \\
& = & -(-1)^{k-1} x_{i_k} \, dx^{j} \wedge dx^{i_1} \wedge \dots 
\wedge \widehat{dx^{i_k}} \wedge \dots \wedge dx^{i_p}.
\end{eqnarray*}
Insgesamt folgt daher
\begin{eqnarray*}
&   & I(\partial_{j} a_{i_{1}\dots i_{p}} \, dx^{j} \wedge dx^{i_1} 
\wedge \dots \wedge dx^{i_p}) \\
& = & x_{j} \, dx^{i_1} \wedge \dots \wedge dx^{i_p} - 
\sum_{k=1}^{p} (-1)^{k-1} x_{i_k} \, dx^{j} \wedge dx^{i_1} \wedge 
\dots \wedge \widehat{dx^{i_k}} \wedge \dots \wedge dx^{i_p},
\end{eqnarray*}
so daß die gewünschte Behauptung durch Summation folgt.\qed\\




\section{Integration von Differentialformen}




In den Abschnitten 3.1 und 3.2 wurden Integrale über geometrische 
Objekte wie (parametrisierte) Kurven und Flächen erklärt. Man kann 
diese Integrale als Integrale geeigneter 1- bzw. 2-Formen auffassen. 
Unser Ziel ist es, Integrale von $p$-Formen auf $p$-dimensionalen 
Untermannigfaltigkeiten im ${\mathbb R}^{n}$ zu erklären. Zunächst 
betrachten wir allerdings Integrale von speziellen parametrisierten 
Mengen.\\


\noindent
Sei $p \in {\mathbb N}$ und $[0,1]^{p}$ der $p$-dimensionale 
abgeschlossene Einheitswürfel. Speziell setzen wir $[0,1]^{0} \da  \{0\}$.

\bigskip

\noindent {\bf Definition.} Sei $\omega$ eine stetige $p$-Form auf 
$[0,1]^{p}$, also
\[ \omega = f \, dx^{1} \wedge \dots \wedge dx^{p} \]
mit einer stetigen Funktion $f$ auf $[0,1]^{p}$. Dann sei
\[ \int_{[0,1]^{p}} \omega \da  \int_{[0,1]^{p}} f = \int_{0}^{1} \dots 
\int_{0}^{1} f(x_{1},\dots,x_{p}) \, dx_{1} \dots dx_{p}. \]
In einem zweiten Schritt soll das Integral über allgemeine Mengen 
im ${\mathbb R}^{n}$ erklärt werden.

\bigskip

\noindent {\bf Definition.} Sei $M \subset {\mathbb R}^{n}$. Ein 
{\em singulärer Würfel}\index{singul{ä}rer W{ü}rfel} in $M$ 
ist eine Abbildung $c: [0,1]^{p} \to M$ 
der Klasse $C^{2}$.

\bigskip

\noindent
Die Differenzierbarkeitsvoraussetzung soll bedeuten, daß $c$ 
Einschränkung einer $C^{2}$-Funktion mit offenem Definitionsbereich 
ist. Die Voraussetzung der zweimaligen Differenzierbarkeit ist zunächst eigentlich 
zu stark, einmalige stetige Differenzierbarkeit wäre ausreichend. Die stärkere 
Voraussetzung wurde im Hinblick auf den Satz von Stokes getroffen. 

\bigskip

\noindent {\bf Beispiele.} 
\begin{itemize}
\item{} Ein singulärer 0-Würfel $c: \{0\} \to M$.
\item{} Ein singulärer 1-Würfel $c: [0,1] \to M$ ist eine parametrisierte Kurve 
der Klasse $C^{2}$, die nicht regulär zu sein braucht.
\item{} Ein singulärer 2-Würfel $c: [0,1]^{2} \to M$ ist i.a. nicht 
regulär, d.h. keine Fläche. Insbesondere ist $c$ i.a. nicht lokal 
injektiv. Man kann $c$ auch konstant wählen. Dies begründet die 
Bezeichnungsweise „singulär“.
\end{itemize}

\bigskip

\noindent {\bf Definition.} Sei $M \subset {\mathbb R}^{n}$ offen, 
$\omega$ eine stetige $p$-Form auf $M$ und $c$ ein singulärer 
$p$-Würfel in $M$. Das {\em Integral von $\omega$ über den singulären 
$p$-Würfel $c$} ist definiert durch
\[ \int_{c} \omega \da  \int_{[0,1]^{p}} c^{*}\omega, \quad p \ge 1 \]
und
\[ \int_{c} \omega \da  \omega(c(0)), \quad p = 0. \]
Sind nun  $M, c, \omega$ wie oben und ist $\tau: [0,1]^{p} \to [0,1]^{p}$ 
eine bijektive Abbildung der Klasse $C^{2}$ mit $\det J \tau 
\ge 0$, so gilt
\begin{eqnarray*}
\int_{c} \omega & = & \int_{[0,1]^{p}} c^{*} \omega = \int_{[0,1]^{p}} 
(c^{*}\omega)_{Y} \, dY \\
& = & \int_{[0,1]^{p}} (c^{*}\omega)_{\tau(X)} {\rm det}\, J\tau(X)\, 
dX \\
& = & \int_{[0,1]^{p}} \tau^{*}(c^{*}\omega)_{X}\, dX \\
& = & \int_{[0,1]^{p}} (c \circ \tau)^{*} \omega \\
& = & \int_{c \circ \tau} \omega,
\end{eqnarray*}
d.h. es liegt eine spezielle Form der Parametrisierungs-Invarianz vor.

\bigskip

\noindent {\bf Beispiele.} $\bullet$ Sei $n \ge 2$ und $p = 1$. Wir schreiben 
$F: [0,1] \to {\mathbb R}^{n}$ für einen singulären 1-Würfel der 
Klasse $C^{2}$ und $\omega = v_{1}dx^{1} + \dots + v_{n}dx^{n}$ für 
eine stetige 1-Form auf ${\mathbb R}^{n}$. Dann gilt
\[ \int_{F} \omega = \int_{[0,1]} F^{*}\omega. \]
Sei $F = (f_{1},\dots,f_{n})$ und $dt^{1}$ das Koordinatendifferential 
im ${\mathbb R}^{1}$. Mit $V \da  (v_{1},\dots,v_{n})$ erhält man für 
die 1-Form $F^{*}\omega$ auf $[0,1]$:
\begin{eqnarray*}
F^{*}\omega & = & F^{*}(v_{1}dx^{1} + \dots + v_{n}dx^{n}) \\
& = & v_{1} \circ F \, df_{1} + \dots + v_{n} \circ F\, df_{n} \\
& = & v_{1} \circ F f'_{1} \, dt^{1} + \dots + v_{n} \circ F f'_{n}\, 
dt^{1} \\
& = & \langle V \circ F, F' \rangle \, dt^{1}.
\end{eqnarray*}
Es folgt
\[ \int_{F} \omega = \int_{[0,1]} \langle V \circ F, F' \rangle \, 
dt^{1} = \int_{0}^{1} \langle V \circ F(t), F'(t) \rangle \, dt .\]
Dies stellt den Zusammenhang mit Abschnitt 3.1 und Analysis II her.\\

\noindent $\bullet$ Sei $n = 3$, $p = 2$. Sei $F : [0,1]^{2} \to 
{\mathbb R}^{3}$ ein singulärer 2-Würfel im ${\mathbb R}^{3}$ und 
\[\omega = v_{1} dx^{2} \wedge dx^{3} + v_{2} dx^{3} \wedge dx^{1} + 
v_{3} dx^{1} \wedge dx^{2}\]
eine stetige 2-Form im ${\mathbb R}^{3}$. 
Nach Definition ist
\[ \int_{F} \omega = \int_{[0,1]^{2}} F^{*}\omega. \]
Mit $F = (f_{1},f_{2},f_{3})$ und $du^{1}, du^{2}$ als 
Koordinatendifferentialen im ${\mathbb R}^{2}$ erhält man für die 
2-Form $F^{*}\omega$ auf $[0,1]^{2}$:
\begin{eqnarray*}
F^{*}\omega & = & (v_{1} \circ F) df_{2} \wedge df_{3} + (v_{2} \circ 
F) df_{3} \wedge df_{1} + (v_{3} \circ F)\, df_{1} \wedge df_{2} \\
& = & (v_{1} \circ F)[\partial_{1} f_{2} du^{1} + \partial_{2} f_{2} 
du^{2}] \wedge [\partial_{1} f_{3} du^{1} + \partial_{2} f_{3} 
du^{2}] + \dots \\
& = & (v_{1} \circ F) (\partial_{1} f_{2} \cdot \partial_{2} f_{3} - 
\partial_{1} f_{3} \cdot \partial_{2}f_{2}) du^{1} \wedge du^{2} + 
\dots \\
& = & \left[ v_{1} \circ F 
\frac{\partial(f_{2},f_{3})}{\partial(u_{1},u_{2})} + v_{2} \circ F 
\frac{\partial(f_{3},f_{1})}{\partial(u_{1},u_{2})} + v_{3} \circ F 
\frac{\partial(f_{1},f_{2})}{\partial(u_{1},u_{2})} \right] \, 
du^{1} \wedge du^{2}.
\end{eqnarray*}
Es folgt
\begin{eqnarray*}
\int_{F} \omega & = & \int_{0}^{1} \int_{0}^{1} \left(v_{1} \circ F 
\frac{\partial(f_{2},f_{3})}{\partial(u_{1},u_{2})} + v_{2} \circ F 
\frac{\partial(f_{3},f_{1})}{\partial(u_{1},u_{2})} + v_{3} \circ F 
\frac{\partial(f_{1},f_{2})}{\partial(u_{1},u_{2})} \right) \, 
du_{1}\, du_{2} \\
& = & \int \langle V, N \rangle \, d{\cal O},
\end{eqnarray*}
wobei die letzte Gleichung nur dann gilt, falls $F$ regulär ist.\\



\noindent
In Abschnitt 3.2 hatten wir schon angedeutet, daß sich der Satz von 
Stokes in der Form
\[ \int_{M} \, d\omega = \int_{\partial M} \omega \]
schreiben läßt. Nun haben wir aber zunächst noch nicht das Integral 
einer Differentialform über eine Untermannigfaltigkeit $M$ zur 
Verfügung, stattdessen steht das Integral über singuläre Würfel zur Verfügung. 
Wir versuchen 
daher, für solche singulären Würfel einen Randbegriff zu 
erklären, so daß schließlich
\[ \int_{c}\, d\omega = \int_{\partial c} \omega \]
gültig wird. Es ist naheliegend, daß man dafür den Rand des 
Parameterbereichs verwenden wird. Andererseits ist aber auch eine 
geeignete Orientierung festzulegen. Der auf diesem Weg schließlich 
zustandekommende Satz von Stokes wird später beim Beweis der 
 geometrischen Fassung des Satzes wieder verwendet werden.

\bigskip

\noindent {\bf Beispiel.} Sei $\omega = fdx^{1} + gdx^{2}$ eine stetig 
differenzierbare 1-Form auf $[0,1]^{2}$. Man erhält
\[ d\omega = df \wedge dx^{1} + dg \wedge dx^{2} = (\partial_{1} g - 
\partial_{2} f) \, dx^{1} \wedge dx^{2}. \]
Wir betrachten den singulären 2-Würfel $c = I^{2}: [0,1]^{2} \to 
{\mathbb R}^{2}$, $X \mapsto X$. Dann gilt
\begin{eqnarray*}
\int_{I^{2}} \, d\omega & = & \int_{[0,1]^{2}} (I^{2})^{*} \, 
d\omega = \int_{[0,1]^{2}} \, d\omega \\
& = & \int_{[0,1]^{2}} (\partial_{1}g - \partial_{2} f) \, dx^{1} 
\wedge dx^{2} \\
& = & \int_{0}^{1} \int_{0}^{1} (\partial_{1}g(x,y) - 
\partial_{2}f(x,y))\, dx\, dy \\
& = & \int_{0}^{1}\left(\int_{0}^{1} \partial_{1}g(x,y)\, dx \right)\, 
dy - \int_{0}^{1} \left(\int_{0}^{1} \partial_{2}f(x,y)\, dy \right) 
\, dx \\
& = & \int_{0}^{1}(g(1,y) - g(0,y))\, dy - \int_{0}^{1}(f(x,1) - 
f(x,0))\, dx .
\end{eqnarray*}
Die so erhaltenen Integrale lassen sich als Integrale der 1-Form 
$\omega$ über singuläre 1-Würfel darstellen. Wir definieren zu diesem Zweck
\begin{eqnarray*}
&   & I_{1,0}^{2}: [0,1] \to {\mathbb R}^{2}, \quad t \mapsto (0,t), \\
&   & I_{1,1}^{2}: [0,1] \to {\mathbb R}^{2}, \quad t \mapsto (1,t), \\
&   & I_{2,0}^{2}: [0,1] \to {\mathbb R}^{2}, \quad t \mapsto (t,0), \\
&   & I_{2,1}^{2}: [0,1] \to {\mathbb R}^{2}, \quad t \mapsto (t,1).
\end{eqnarray*}
Dann gilt etwa
\[ \int_{I_{2,1}^{2}} \omega = \int_{[0,1]} (I_{2,1}^{2})^{*} 
\omega = \int_{0}^{1}f(t,1)\, dt \]
oder
\[ \int_{I_{1,0}^{2}} \omega = \int_{[0,1]} (I_{1,0}^{2})^{*} 
\omega = \int_{0}^{1} g(0,t)\, dt \]
u.s.w., d.h.
\[ \int_{I^{2}} \, d\omega = - \left(\int_{I_{1,0}^{2}} \omega - 
\int_{I_{1,1}^{2}} \omega\right) + \left(\int_{I_{2,0}^{2}} \omega - 
\int_{I_{2,1}^{2}} \omega \right). \]
Dieses und weitere Beispiele legen es nahe, den Rand $\partial I^{2}$ 
als formale Linearkombination von singulären 1-Würfeln zu schreiben, d.h.
\[ \partial I^{2} \da  - (I_{1,0}^{2} - I_{1,1}^{2}) + (I_{2,0}^{2} - 
I_{2,1}^2), \]
und das Integral über eine solche Linearkombination als die 
entsprechende Linearkombination der Integrale der 1-Würfel zu erklären. \\

\noindent
Anstatt von „formalen Linearkombinationen“ zu sprechen, sollte man 
besser den Begriff einer {\em frei erzeugten abelschen Gruppe} verwenden. 
Sei $A$ eine nichtleere Menge und
\[ {\cal F}(A) \da  \{f: A \to {\mathbb Z}: f(x) \not= 0 \mbox{ für 
höchstens endlich viele } x \in A.\} \]
Für $f, g \in {\cal F}(A)$ sei
\[ (f \pm g)(x) \da  f(x) \pm g(x), \quad x \in A. \]
Man nennt ${\cal F}(A)$ die von $A$ erzeugte freie abelsche Gruppe. 
Für $x \in A$ sei $\overline{x} \in {\cal F}(A)$ erklärt als
\[ \overline{x}(y) \da  \left\{ \begin{array}{ll} 1, & \mbox{ für } 
y = x, \\ 0, & \mbox{ sonst.} \end{array} \right. \]
Somit gilt für $f \in {\cal F}(A)$ gerade
\[ f = \sum_{x \in A} f(x) \overline{x}. \]
Statt $\overline{x}$ schreibt man auch einfach $x$, d.h. die Elemente 
von ${\cal F}(A)$ sind von der Form
\[ \sum_{x \in A} n_{x} x \]
mit $n_{x} \in {\mathbb Z}$, wobei nur endlich viele dieser Zahlen 
$\not= 0$ sind. Die Elemente von ${\cal F}(A)$ kann man daher auch 
als formale Linearkombinationen mit ganzzahligen Koeffizienten der 
Elemente von $A$ auffassen.

\bigskip

\noindent {\bf Definition.} Sei $M \subset {\mathbb R}^{n}$ und $p 
\in {\mathbb N}_{0}$. Sei ${\cal K}_{p}(M)$ die von der Menge der 
singulären $p$-Würfel in $M$ erzeugte freie abelsche Gruppe. Die 
Elemente von ${\cal K}_{p}(M)$ (endliche formale Linearkombinationen 
von singulären $p$-Würfeln in $M$ mit ganzzahligen Koeffizienten) 
heißen {\em $p$-Ketten}\index{Ketten} in $M$.

\bigskip

\noindent {\bf Definition.} Das {\em Integral der stetigen $p$-Form 
$\omega$ auf $M$ über die $p$-Kette} $c = \sum a_{i}c_{i}$ mit $a_{i} 
\in {\mathbb Z}$ und singulären $p$-Würfeln $c_{i}$ in $M$ wird erklärt 
durch
\[ \int_{c} \omega = \sum a_{i} \int_{c_{i}} \omega. \]

\bigskip

\noindent {\bf Randbildung.}\\



\noindent {\bf Definition.} \begin{itemize}
\item Der singuläre $p$-Würfel $I^{p}: 
[0,1]^{p} \to {\mathbb R}^{p}$ sei erklärt durch $I^{p}(X) \da  X$ 
für $X \in [0,1]^{p}$. 
\item Die singulären $(p-1)$-Würfel $I_{i,0}^{p}: 
[0,1]^{p-1} \to {\mathbb R}^{p}$ und $I_{i,1}^{p}: [0,1]^{p-1} \to 
{\mathbb R}^{p}$ für $p \ge 2$ und $i \in \{1,\dots,p\}$ sind 
erklärt durch
\begin{eqnarray*}
&   & I_{i,0}^{p}(X) \da  (x_{1},\dots,x_{i-1},0,x_{i},\dots,x_{p-1}), 
\\
&   & I_{i,1}^{p}(X) \da  (x_{1},\dots,x_{i-1},1,x_{i},\dots,x_{p-1}) 
\end{eqnarray*}
für $X = (x_{1},\dots,x_{p-1}) \in [0,1]^{p-1}$. 
\item Setze 
$I_{1,0}^{1}(0) \da  0$ und $I_{1,1}^{1}(0) = 1$. 
\item Der {\em Rand des 
singulären $p$-Würfels}\index{Rand} $c$ ist erklärt durch
\[ \partial c \da  \sum_{i=1}^{p} (-1)^{i} \left[ c \circ I_{i,0}^{p} - 
c \circ I_{i,1}^{p}\right]. \]
\item Der Rand der $p$-Kette $\sum a_{i}c_{i}$ ist erklärt durch
\[ \partial\left(\sum a_{i}c_{i}\right) \da  \sum a_{i}\partial c_{i}. \]
\end{itemize}
Man kann zeigen, daß stets $\partial \partial c = 0$ für Ketten $c$ 
gilt. Wir werden dies nicht benötigen. Eine analoge Aussage gilt 
für Untermannigfaltigkeiten $M$ des ${\mathbb R}^{n}$, d.h. hier 
gilt $\partial \partial M = \emptyset$.

\bigskip

\begin{satz}[Stokes für Ketten]\label{Satz3.5.1}  {Sei $M \subset 
{\mathbb R}^{n}$ offen, $p \in {\mathbb N}$, $\omega$ eine 
$(p-1)$-Form der Klasse $C^{1}$ auf $M$ und $c$ eine $p$-Kette in 
$M$. Dann gilt}
\[ \int_{c} \, d\omega = \int_{\partial c} \omega. \]
\end{satz}

\bigskip

\noindent {\em Beweis.} Sei $p = 1$, also $\omega = f$ eine Funktion 
der Klasse $C^{1}$ auf $M$. Sei $c$ ein singulärer 1-Würfel in $M$. 
Dann ist
\[ df = f_{1} dx^{1} + \dots + f_{n}dx^{n} \]
und
\[
\int_{c} df  =  \int_{0}^{1} \langle \nabla f \circ c(t), 
c'(t) \rangle \, dt = \int_{0}^{1}(f \circ c)'(t)\, dt 
 =  f \circ c(1) - f \circ c(0).
\]
Andererseits gilt
\begin{eqnarray*}
\int_{\partial c} f & = & (-1) \int_{c \circ I_{1,0}^{1}} f + 
\int_{c \circ I_{1,1}^{1}} f 
 =  -f(c \circ I_{1,0}^{1}(0)) + f(c \circ I_{1,1}^{1}(0)) \\
& = & -f(c(0)) + f(c(1)). 
\end{eqnarray*}
Dies zeigt die behauptete Gleichheit für singuläre 1-Würfel, die 
Ausdehnung auf 1-Ketten folgt wegen Linearität.\\

\noindent
Sei jetzt $p \ge 2$. Wir betrachten zunächst den Fall $n = p$ und 
$c = I^{p}$ sowie
\begin{equation}\label{3.4.5a}
\omega = f dx^{1} \wedge \dots \wedge \widehat{dx^{i}} \wedge \dots 
\wedge dx^{p}
\end{equation}
für $i \in \{1,\dots,p\}$. Dann gilt
\[ d\omega = (-1)^{i-1} \partial_{i}f \, dx^{1} \wedge \dots \wedge 
dx^{p} \]
und
\begin{eqnarray*}
\int_{I^{p}} \, d\omega &= &\int_{[0,1]^{p}} \, d\omega \\
&=& 
(-1)^{i-1} \int_{[0,1]^{p}} \partial_{i} f \\
& = & (-1)^{i-1} \int_{0}^{1} \dots \int_{0}^{1} \left(\int_{0}^{1} 
\partial_{i}f (x_{1},\dots,x_{p}) \, dx_{i}\right)\, dx_{1} \dots 
\widehat{dx_{i}} \dots dx_{p} \\
& = & (-1)^{i-1} \int_{0}^{1} \dots \int_{0}^{1} 
[f(x_{1},\dots,x_{i-1},1,x_{i+1},\dots,x_{p}) \\
&&\qquad\qquad\qquad\qquad\quad -f(x_{1},\dots,x_{i-1},0,x_{i+1},\dots,x_{p})]  dx_{1} \dots 
\widehat{dx_{i}} \dots dx_{p}.
\end{eqnarray*}
Andererseits gilt
\[ \int_{\partial I^{p}} \omega = \sum_{j=1}^{p} (-1)^{j} \left[ 
\int_{I_{j,0}^{p}} \omega - \int_{I_{j,1}^{p}} \omega \right]. \]
Wir formen die Integrale in den Klammern um. Zunächst ist
\[ \int_{I_{j,0}^{p}} \omega = \int_{[0,1]^{p-1}} 
\left(I_{j,0}^{p}\right)^* \omega. \]
Mit $I_{j,0}^{p} = (g_{1},\dots,g_{p})$ folgt
\[ \left(I_{j,0}^{p}\right)^{*} \omega = f \circ I_{j,0}^{p} \, 
dg_{1} \wedge \dots \wedge \widehat{dg_{i}} \wedge \dots \wedge 
dg_{p}. \]
Seien $du^{1},\dots,du^{p-1}$ die Koordinatendifferentiale von 
${\mathbb R}^{p-1}$. Dann gilt nach Definition von $I_{j,0}^{p}$ 
offenbar
\[ dg_{k} = \left\{ \begin{array}{ll}
du^{k}, & k = 1,\dots,j-1, \\ 0, & k = j, \\ du^{k-1}, & k = 
j+1,\dots,p. \end{array} \right. \]
An der Stelle $U = (u_{1},\dots,u_{p-1})$ erhält man also
\[ \left[\left(I_{j,0}^{p}\right)^{*} \omega \right]_{U} = \left\{ 
\begin{array}{ll}
0,& j \not= i, \\ f(u_{1},\dots,u_{i-1},0,u_{i},\dots,u_{p-1}) \, 
du^{1} \wedge \dots \wedge du^{p},& j = i. \end{array} \right. \]
Es folgt
\[ \int_{I_{j,0}^{p}} \omega = \left\{ \begin{array}{ll}
0,& j \not= i, \\ \int_{0}^{1} \dots \int_{0}^{1} 
f(u_{1},\dots,u_{i-1},0,u_{i},\dots,u_{p-1}) \, du_{1} \dots du_{p-1},&
j = i. \end{array} \right. \]
Mit einem ähnlichen Ausdruck für $\int_{I_{j,1}^{p}}$ erhält man 
schließlich
\begin{eqnarray*}
\int_{\partial I^{p}} \omega & = & (-1)^{i} \int_{0}^{1} \dots 
\int_{0}^{1} \left[ f(u_{1},\dots,u_{i-1},0,u_{i},\dots,u_{p-1}) - 
\right. \\
&   & \qquad \left. 
f(u_{1},\dots,u_{i-1},1,u_{i},\dots,u_{p-1})\right] \, du_{1} \dots 
du_{p-1}. 
\end{eqnarray*}
Es folgt also
\begin{equation}\label{3.4.6a}
\int_{I^{p}} \, d\omega = \int_{\partial I^{p}} \omega.
\end{equation}
Da jede $(p-1)$-Form auf $[0,1]^{p}$ Summe von $(p-1)$-Formen der 
Gestalt (\ref{3.4.5a}) ist, gilt (\ref{3.4.6a}) für jede $(p-1)$-Form 
$\omega$ der Klasse $C^{1}$ auf $[0,1]^{p}$.\\

\noindent
Sei schließlich $c$ ein beliebiger singulärer $p$-Würfel in $M 
\subset {\mathbb R}^{n}$ und $\omega$ eine stetig differenzierbare 
$(p-1)$-Form auf $M$. Dann gilt zunächst wegen (\ref{3.4.6a})
\[
\int_{c} \, d\omega  =  \int_{[0,1]^{p}} c^{*} \, d\omega = 
\int_{I^{p}} c^{*} \, d\omega = \int_{I^{p}} \, dc^{*} \omega 
 =  \int_{\partial I^{p}} c^{*} \omega.
\]
Ferner ist
\[
\int_{c \circ I_{i,0}^{p}} \omega = \int_{[0,1]^{p-1}} (c \circ 
I_{i,0}^{p})^{*} \omega = \int_{[0,1]^{p-1}} (I_{i,0}^{p})^{*} 
(c^{*} \omega) =\int_{I_{i,0}^{p}} c^{*} \omega,
\]
und analog für $I_{i,1}^{p}$. Es folgt also
\begin{eqnarray*}
\int_{\partial c} \omega & = & \sum_{i=1}^{p} (-1)^{i} \left[ 
\int_{c \circ I_{i,0}^{p}} \omega - \int_{c \circ I_{i,1}^{p}} \omega 
\right] \\
& = & \sum_{i=1}^{p}(-1)^{i} \left[ \int_{I_{i,0}^{p}} c^{*} \omega - 
\int_{I_{i,1}^{p}} c^{*} \omega \right] \\
& = & \int_{\partial I^{p}} c^{*} \omega \\
& = & \int_{c} \, d\omega.
\end{eqnarray*}
Ist schließlich $c = \sum a_{i} c_{i}$ eine $p$-Kette in $M$, so 
folgt
\[ \int_{c} \, d\omega = \sum a_{i} \int_{c_{i}} \, d\omega = \sum 
a_{i} \int_{\partial c_{i}} \omega = \int_{\partial c} \omega. \]
\qed\\

\bigskip

\noindent
Wir betrachten zur Illustration ein abschließendes Beispiel, das in 
den Übungen besprochen werden kann. 

\bigskip

\noindent {\bf Beispiel.} Sei $c$ der singuläre 2-Würfel im 
${\mathbb R}^{3}$ mit
\[ c: [0,1]^{2} \to {\mathbb R}^{3}, \quad (u,v) \mapsto (u^{2} - 
v^{3}, u, -v^{2}) \]
und $\omega$ die 2-Form auf ${\mathbb R}^{3}$ mit
\[ \omega_{X} = 4(dx^{1})_{X} \wedge (dx^{3})_{X} - 
x_{1}(dx^{2})_{X} \wedge (dx^{3})_{X} \]
sowie $\eta$ die 1-Form auf ${\mathbb R}^{3}$ mit
\[ \eta_{X} = 2(dx^{1})_{X} - x_{1}^{2}(dx^{3})_{X}. \]
Zunächst gilt mit $c = (c_{1},c_{2},c_{3})$
\begin{eqnarray*}
\int_{c} \omega & = & \int_{[0,1]^{2}} c^{*} \omega = 
\int_{[0,1]^{2}} (4dc_{1} \wedge dc_{3} - c_{1}^{2} dc_{2} \wedge 
dc_{3}) \\
& = & \int_{[0,1]^{2}} \left[4 \frac{\partial(c_{1},c_{3})}{\partial 
(u,v)} - (u^{2} - v^{3}) \frac{\partial(c_{2},c_{3})}{\partial(u,v)} 
\right] \, du\, dv \\
& = & \int_{0}^{1} \int_{0}^{1} [-16uv + 2v(u^{2} - v^{3})] \, du\, 
dv \\
& = & -\frac{61}{15}.
\end{eqnarray*}
Mit $d \eta = -2x_{1} dx^{1} \wedge dx^{3}$ gilt analog
\begin{eqnarray*}
\int_{c} \, d\eta & = & \int_{[0,1]^{2}} - 2(u^{2} - v^{3}) 
\frac{\partial(c_{1},c_{3})}{\partial (u,v)} \, du\, dv \\
& = & -2 \int_{0}^{1} \int_{0}^{1} (-4uv)(u^{2} - v^{3})\, du\, dv \\
& = & \frac{1}{5}.
\end{eqnarray*}
Andererseits erhält man wegen
\[ \partial c = c \circ I_{1,1}^{2} - c \circ I_{1,0}^{2} + c \circ 
I_{2,0}^{2} - c \circ I_{2,1}^{2} \]
und
\begin{eqnarray*}
&   & c \circ I_{1,1}^{2} : [0,1] \to {\mathbb R}^{3}, \quad t 
\mapsto (1-t^{3}, 1, -t^{2}), \\
&   & c \circ I_{1,0}^{2} : [0,1] \to {\mathbb R}^{3}, \quad t 
\mapsto (-t^{3},0, -t^{2}), \\
&   & c \circ I_{2,0}^{2} : [0,1] \to {\mathbb R}^{3}, \quad t 
\mapsto (t^{2}, t,0), \\
&   & c \circ I_{2,1}^{2} : [0,1] \to {\mathbb R}^{3}, \quad t 
\mapsto (t^{2} - 1,t,-1) \\
\end{eqnarray*}
auch
\begin{eqnarray*}
\int_{\partial c} \eta & = & \int_{c \circ I_{1,1}^{2}} \eta - 
\int_{c \circ I_{1,0}^{2}} \eta + \int_{c \circ I_{2,0}^{2}} \eta - 
\int_{c \circ I_{2,1}^{2}} \eta \\
& = & \int_{0}^{1} [2(-3t^{2}) - (1-t^{3})^{2}(-2t)] \, dt 
 - \int_{0}^{1} [2(-3t^{2}) - t^{6}(-2t)] \, dt \\
&   & + \int_{0}^{1} 2 \cdot 2t\, dt - \int_{0}^{1} 2 \cdot 2t\, dt \\
& = & \frac{1}{5},
\end{eqnarray*}
was den Satz von Stokes exemplarisch bestätigt.




\section{Differenzierbare Untermannigfaltigkeiten}


In diesem Abschnitt werden niederdimensionale glatte geometrische Teilmengen 
des ${\mathbb R}^{n}$ untersucht, über die schließlich integriert 
werden soll. Die dabei betrachteten $k$-dimensionalen 
Untermannigfaltigkeiten des ${\mathbb R}^{n}$ können zugleich als 
Einstieg in die Theorie der abstrakten Mannigfaltigkeiten verstanden werden. 
Dies ist dann etwa Gegenstand einer Vorlesung über Differentialgeometrie.\\

\noindent
Um die Darstellung möglichst einfach zu halten, nehmen wir an, daß 
alle Abbildungen von der Klasse $C^{\infty}$ sind, falls nichts 
anderes vorausgesetzt wird. Ferner sei für $k \in \{0,\dots,n\}$
\begin{eqnarray*}
{\mathbb R}^{n}_{k} & \da  & \{X \in {\mathbb R}^{n}: x_{k+1} = \dots = 
x_{n} = 0\}, \\
H_{k}^{n} & \da  & \{X \in {\mathbb R}^{n}: x_{1} \ge 0, x_{k+1} = 
\dots = x_{n} = 0\}, \\
\partial H^{n}_k & \da  & \{X \in {\mathbb R}^{n}:x_{1} = 0, x_{k+1} = 
\dots = x_{n} = 0\}.
\end{eqnarray*}
Grob gesprochen werden durch die nachfolgende Definition Mengen 
erklärt, die lokal nach Anwendung einer geeigneten differenzierbaren 
Transformation wie ${\mathbb R}_{k}^{n}$ oder $H_{k}^{n}$ aussehen.

\bigskip

\noindent {\bf Definition.} Sei $M \subset {\mathbb R}^{n}$ und $k 
\in {\mathbb N}_{0}$. Die Menge $M$ heißt $k$-{\em dimensionale 
Mannigfaltigkeit im}\index{Mannigfaltigkeit} ${\mathbb R}^{n}$, 
wenn für jedes $X \in M$ 
gilt: Es gibt eine offene Umgebung $U\subset \MdR^n$ von $X$, eine offene Menge $V 
\subset {\mathbb R}^{n}$ und einen Diffeomorphismus $h: U \to V$ mit
\begin{equation}
 h(U \cap M) = V \cap {\mathbb R}_{k}^{n}\tag{a}
\end{equation}
oder
\begin{equation}
h(U \cap M) = V \cap H_{k}^{n}\quad\text{und}\quad h(X)\in\partial H^n_k.\tag{b}
\end{equation}
Gilt (a), so heißt $X$ {\em innerer Punkt} von $M$; gilt (b), so 
heißt $X$ {\em Randpunkt} von $M$.

\bigskip

\noindent {\bf Bemerkungen.}\begin{itemize}
\item Die Begriffe innerer Punkt und Randpunkt 
beziehen sich nicht auf die Topologie des ${\mathbb R}^{n}$. 
\item Ein Punkt $X 
\in M$ kann nicht zugleich innerer Punkt und Randpunkt sein. Sonst 
gäbe es offene Umgebungen $U_{1}, U_{2}\subset\MdR^n$ von $X$,  offene Mengen
 $V_{1}, V_{2} \subset {\mathbb 
R}^{n}$ und Diffeomorphismen
\[ h_{1} : U_{1} \to V_{1} \quad \mbox{ und } \quad h_{2}: U_{2} \to 
V_{2} \]
mit
\[ h_{1}(U_{1} \cap M) = V_{1} \cap {\mathbb R}_{k}^{n}, \quad 
 h_{2}(U_{2} \cap M) = V_{2} \cap H_{k}^{n} \quad \mbox{ 
und }\quad h_2(X)\in\partial H^n_k.\]
Die Abbildung
\[ h_{2} \circ h_{1}^{-1}: h_{1} (U_{1} \cap U_{2}) \to h_{2}(U_{1} 
\cap U_{2}) \]
ist ein Diffeomorphismus, der also offene Mengen auf offene Mengen 
abbildet. Insbesondere ist auch $h_{2} \circ h_{1}^{-1} \mid 
h_{1}(U_{1} \cap U_{2}) \cap {\mathbb R}_{k}^{n}$ eine 
differenzierbare Abbildung in $\MdR^n_k$ mit nicht verschwindender 
Funktionaldeterminante und 
\begin{eqnarray*}
h_{2} \circ h_{1}^{-1}(h_{1}(U_{1} \cap U_{2}) \cap {\mathbb 
R}_{k}^{n}) & = & h_{2} \circ h_{1}^{-1}(V_{1} \cap {\mathbb 
R}_{k}^{n} \cap h_{1}(U_{1} \cap U_{2})) \\
& = & h_{2}(U_{1} \cap M \cap U_{2}) \\
& \subset & V_{2} \cap H_{k}^{n} \subset {\mathbb R}_{k}^{n}.
\end{eqnarray*}
Also wäre $h_{2} \circ h_{1}^{-1} \mid h_{1}(U_{1} \cap U_{2}) 
\cap {\mathbb R}_{k}^{n}$ ein Diffeomorphismus zwischen offenen 
Teilmengen des ${\mathbb R}_{k}^{n}$. Andererseits wäre $h_{2} \circ 
h_{1}^{-1}(h_{1}(X))$ kein innerer Punkt 
 der Bildmenge dieser Abbildung bezüglich ${\mathbb 
R}_{k}^{n}$ als dem umgebenden Raum, ein Widerspruch.
\end{itemize}

\bigskip

\noindent {\bf Definition.} Ist $M$ eine $k$-dimensionale 
Mannigfaltigkeit im ${\mathbb R}^{n}$, so wird die Menge der 
{\em Randpunkte}\index{Randpunkt} von $M$ mit $\partial M$ bezeichnet. Ist $\partial M = 
\emptyset$, so heißt $M$ {\em unberandet}\index{unberandet}. Ist $\partial M = \emptyset$ 
und $M$ kompakt, so heißt $M$ {\em geschlossen}\index{geschlossen}.

\bigskip

\noindent {\bf Beispiele.} \begin{itemize}
\item $0$-dimensionale Mannigfaltigkeiten, d.h. diskrete Punktmengen.
\item Ein Ball $B^{n}$ ist eine $n$-dimensionale Mannigfaltigkeit mit $\partial 
B^{n} = S^{n-1}$.
\item Eine Sphäre $S^{n-1}$ ist eine kompakte $(n-1)$-dimensionale Mannigfaltigkeit
 mit $\partial S^{n-1} = \emptyset$, d.h. $S^{n-1}$ ist geschlossen.
\item Der Volltorus im $\MdR^3$ (Rettungsring) ist eine kompakte $3$-dimensionale Mannigfaltigkeit, 
deren Rand gerade der $2$-Torus $T^2$ ist. Dieser erfüllt wieder $\partial T^2=\emptyset$.
\item Eine Hemisphäre $S^{n-1}_{+}\da S^{n-1}\cap E_n^\perp$ ist eine 
$(n-1)$-dimensionale Mannigfaltigkeit, und es gilt  $\partial S_{+}^{n-1} = 
S^{n-2}$.
\item Eine $2$-Mannigfaltigkeit ist beispielsweise  
 $\{(u,v,4 - u^{2} - v^{2}): (u,v) \in 
{\mathbb R}^{2} \cap B^{2}(0,2)\}$. Allgemeiner ist jeder Graph einer 
glatten Funktion $f:\MdR^{n-1}\to\MdR$ eine $(n-1)$-dimensionale Mannigfaltigkeit in $\MdR^n$.
\item {\em Möbiusband}. %\index{M{ö}biusband}
„Ein Zeigervektor $a(\varphi)$ beschreibe 
eine Kreislinie, um welche sich wiederum mit halber Geschwindigkeit 
ein zweiter Zeiger windet, dessen Mitte auf der Kreislinie und 
dessen Anfangs- und Endpunkt auf dem Rand des Möbiusbands liegt“.

Sei
\[ a(\varphi) \da  (\cos \varphi, \sin \varphi, 0), \qquad \varphi 
\in [-\pi,\pi] \]
und
\[ b_{\varphi}(\psi) \da  \cos (\psi) a(\varphi)  + \sin( \psi) e_{3} , 
\qquad \psi \in \left[ -\frac{\pi}{2}, \frac{\pi}{2}\right]. \]
Sei $R > 1$ und $\phi: [-\pi, \pi] \times (-1,1) \to {\mathbb R}^{3}$ 
mit
\begin{eqnarray*}
\phi(\varphi,t) & \da  & Ra(\varphi) + t b_{\varphi} 
\left(\frac{\varphi}{2}\right) \\
& = & R \left(\begin{array}{c} \cos \varphi \\ \sin \varphi \\ 0 
\end{array} \right) + t \left( \begin{array}{c} \cos 
\frac{\varphi}{2} \cos \varphi \\ \cos \frac{\varphi}{2} \sin \varphi 
\\ \sin \frac{\varphi}{2} \end{array} \right).
\end{eqnarray*}
Die Menge $\text{Bild}(\phi)$ ist eine 2-dimensionale Mannigfaltigkeit im 
${\mathbb R}^{3}$ (Möbiusband), die nicht orientierbar ist (vgl.\ später).
\end{itemize}

\noindent
Die erforderlichen Nachweise zu den vorangehenden Beispielen werden durch die nachfolgenden 
drei Sätze erheblich vereinfacht.

\bigskip

\begin{satz}\label{Satz3.6.1} {Ist $M$ eine k-dimensionale 
Mannigfaltigkeit im ${\mathbb R}^{n}$ und $k \ge 1$, so ist $\partial 
M$ eine $(k-1)$-dimensionale unberandete Mannigfaltigkeit.}
\end{satz}



\bigskip

\noindent {\em Beweis.} 
Sei $M$ eine $k$-dimensionale Mannigfaltigkeit in ${\mathbb R}^{n}$ 
$(k \ge 1)$. Sei $X \in \partial M$. Dann gibt es eine offene 
Umgebung $U$ von $X$, eine offene Menge $V \subset {\mathbb R}^{n}$ 
und einen Diffeomorphismus $h: U \to V$ mit $h(U \cap M) = V \cap 
H_{k}^{n}$ und $h(X) \in \partial H_{k}^{n}$. 

\bigskip

\noindent {\bf Behauptung.} $h(U \cap \partial M) = V \cap \partial 
H_{k}^{n}$.

\bigskip

\noindent {\bf Nachweis.} Sei $Y \in U \cap \partial M$. Wäre $h(Y) 
\notin \partial H_{k}^{n}$, so gäbe es eine offene Umgebung $V' 
\subset V$ von $h(Y)$ mit $V' \cap {\mathbb R}_{k}^{n} \subset 
H_{k}^{n} \setminus \partial H_{k}^{n}$. Die Menge $U' \da  h^{-1}(V')$ 
ist eine offene Umgebung von $Y$ mit $h(U' \cap M) = V' \cap {\mathbb 
R}_{k}^{n}$. Also ist $Y \in M \setminus \partial M$, ein Widerspruch. 
Dies zeigt $h(U \cap \partial M) \subset V \cap \partial H_{k}^{n}$.\\

\noindent
Sei $Z \in V \cap \partial H_{k}^{n}$. Dann gibt es ein $Y \in U \cap M$ mit $h(Y) = 
Z$. Nun muß aber $Y \in \partial M$ gelten, da sonst $h(Y) = Z 
\notin \partial H_{k}^{n}$ wäre. Dies zeigt $V \cap \partial 
H_{k}^{n} \subset h(U \cap \partial M)$.\\

\noindent
Sei $L: {\mathbb R}^{n} \to {\mathbb R}^{n}$ ein (linearer) 
Automorphismus mit $L(\partial H_{k}^{n}) = {\mathbb R}_{k-1}^{n}$. 
Die Abbildung $L \circ h: U \to L(V)$ ist ein Diffeomorphismus der 
offenen Umgebung $U$ von $X$ auf die offene Menge $L(V) \subset 
{\mathbb R}^{n}$ mit $L \circ h(U \cap \partial M) = L(V) \cap 
{\mathbb R}_{k-1}^{n}$. Da $X \in \partial M$ beliebig war, ist 
$\partial M$ eine $(k-1)$-dimensionale unberandete Mannigfaltigkeit 
des ${\mathbb R}^{n}$. \qed\\

\bigskip

\noindent
Wir zeigen jetzt, daß sich Mannigfaltigkeiten lokal als parametrisierte 
Flächen beschreiben lassen. Hierzu setzen wir speziell
\[H^k\da \{X\in \MdR^k:x_1\ge 0\}=H^k_k.\]

\bigskip

\begin{satz}\label{Satz3.6.2} {Sei M eine k-dimensionale 
Mannigfaltigkeit in ${\mathbb R}^{n}$. Dann gibt es zu jedem $X \in M$ 
eine offene Umgebung U von X, eine offene Menge $W \subset {\mathbb 
R}^{k}$, falls $X \notin \partial M$, bzw. eine relativ offene Menge 
$W \subset H^{k}$, falls $X \in \partial M$, und eine injektive, 
differenzierbare Abbildung $F: W \to {\mathbb R}^{n}$ mit folgenden 
Eigenschaften:}
\begin{enumerate}
\item[{\rm (a)}] $F(W) = M \cap U$ {\em und $X = F(Z)$ mit $Z\in\partial H^k$, falls $X 
\in \partial M$},
\item[{\rm (b)}] $F^{-1}: F(W) \to W$ {\em ist stetig} bezüglich der Spurtopologie, 
\item[{\rm (c)}] $DF_{Z}$ {\em ist vom Rang $k$ für alle $Z \in W$.}
\end{enumerate}
\end{satz}

\bigskip

\noindent
Jede solche Abbildung $F$ wird als ein {\em Koordinatensystem}\index{Koordinatensystem} der 
Mannigfaltigkeit um den Punkt $X$ bezeichnet.

\bigskip

\noindent {\em Beweis.} Sei $X \in M$ und $X \notin \partial M$. Dann 
gibt es eine offene Menge $U \subset {\mathbb R}^{n}$ um $X$, eine 
offene Menge $V \subset {\mathbb R}^{n}$ und einen Diffeomorphismus 
$h: U \to V$ mit $h(U \cap M) = V \cap {\mathbb R}_{k}^{n}$. Setze
\[ W \da  \{Z \in {\mathbb R}^{k}: (z_{1},\dots,z_{k},0,\dots,0) \in 
V\} \]
und definiere eine injektive Abbildung $F: W \to {\mathbb R}^{n}$ durch
\[ F(Z) \da  h^{-1}(z_{1},\dots,z_{k},0,\dots,0),\qquad Z=(z_1,\ldots,z_k). \]
Nach Konstruktion gilt $F(W) = U \cap M$. Mit $h$ ist auch $F^{-1}=h\vert U\cap M$ 
stetig. Definiere $H: U \to {\mathbb R}^{k}$ durch
\[ H(Y) \da  (h_{1}(Y),\dots,h_{k}(Y)) \]
mit $h = (h_{1},\dots,h_{n})$. Für $Z \in W$ gilt $H(F(Z)) = Z$ und 
damit $DH_{F(Z)} \circ DF_{Z} = \mbox{\rm id}_{{\mathbb R}^{k}}$, d.h. ${\rm 
rang}\, DF_{Z} = k$.\\

\noindent
Im Fall $X \in \partial M$ ist die Argumentation dieselbe. \qed\\

\bigskip

\noindent
Man kann Satz \ref{Satz3.6.2} auch umkehren.

\bigskip

\begin{satz}\label{Satz3.6.3} {Sei $M \subset {\mathbb R}^{n}$. Zu 
jedem $X \in M$ gebe es eine offene Umgebung $U\subset\MdR^n$ von $X$, eine offene 
Menge $W \subset {\mathbb R}^{k}$ oder eine relativ offene Menge $W 
\subset H^{k}$ und eine injektive differenzierbare Abbildung $F: W 
\to {\mathbb R}^{n}$ mit (a), (b), (c) wie in Satz 3.6.3. Dann ist $M$ 
eine $k$-dimensionale Mannigfaltigkeit im ${\mathbb R}^{n}$.}
\end{satz}

\bigskip

\noindent {\em Beweis.} Wir betrachten wieder nur den Fall eines 
Punktes $X \in M$, zu dem eine offene Menge $W \subset {\mathbb 
R}^{k}$ und eine Abbildung $F: W \to {\mathbb R}^{n}$ mit den 
behaupteten Eigenschaften existiert. Sei $X = F(Z)$. Aus dem Satz von 
der Umkehrabbildung folgert man mit (c) die Existenz einer offenen Umgebung 
$W_{1}$ von $Z$ in $W$, einer offenen Umgebung $W_{2}$ von 0 in 
${\mathbb R}^{n-k}$, einer offenen Umgebung $U_{1}$ von $F(Z)$ in 
${\mathbb R}^{n}$ und eines Diffeomorphismus $g: W_{1} \times W_{2} 
\to U_{1}$ mit $g(Y,0) = F(Y)$ für $Y \in W_{1}$. Sei $h: U_{1} \to 
W_{1} \times W_{2}$ die Umkehrabbildung von $g$. Da $F^{-1}: F(W) \to 
W$ stetig ist, gibt es eine offene Menge $U_{2} \subset {\mathbb 
R}^{n}$ mit $F(W_{1}) = U_{2} \cap F(W)$. Setze $U_{0}\da  U \cap U_{1} 
\cap U_{2}$ und $V \da  g^{-1}(U_{0})$. Dann gilt wegen (a)
\begin{eqnarray*}
h(U_{0} \cap M) & = & h(U \cap U_{1} \cap U_{2} \cap M) \\
& = & h(F(W) \cap U_{0} \cap U_{2}) \\
& = & h(F(W_{1}) \cap U_{0}) \\
& = & g^{-1}(F(W_{1}) \cap U_{0}) \\
& = & g^{-1}(g(W_{1} \times \{0\}) \cap U_{0}) \\
& = & V \cap (W_{1} \times \{0\}) \\
& = & V \cap {\mathbb R}_{k}^{n}.
\end{eqnarray*}
Zuletzt wurde benutzt, daß 
\[V=g^{-1}(U_0)=h(U_0)\subset h(U_1)= W_1\times W_2\]
gilt.
\qed\\

\bigskip

\begin{satz}\label{Satz3.6.4} {Sei M eine k-dimensionale 
Mannigfaltigkeit. Seien $F_{1} : W_{1} \to {\mathbb R}^{n}$ und 
$F_{2}: W_{2} \to {\mathbb R}^{n}$ Koordinatensysteme um einen Punkt 
von $M$. Dann ist die Abbildung
\[ F_{2}^{-1} \circ F_{1}: F_{1}^{-1}(F_{2}(W_{2})) \to {\mathbb 
R}^{k} \]
differenzierbar und vom Rang $k$.}
\end{satz}

\bigskip

\noindent {\em Beweis.} Sei $Y \in F_{1}^{-1}(F_{2}(W_{2}))$ und $X = 
F_{1}(Y)$. Dann gibt es in einer Umgebung von $X$ einen 
Diffeomorphismus $H$ mit
\[ F_{2}^{-1}(F_{1}(Z)) = H(F_{1}(Z)) \quad \mbox{ für } Z \in 
F_{1}^{-1}(F_{2}(W_{2}')), \]
wobei $W_{2}'$ eine passende Umgebung von $F_{2}^{-1}(X)$ ist. Mit 
$H, F_{1}$ ist also auch $F_{2}^{-1} \circ F_{1}$ differenzierbar (im 
Definitionsbereich).  Ferner ist 
$\text{rang}(D(F_{2}^{-1} \circ F_{1})_Y)=k$, da $\text{rang}(DH_{F_1(Y)})=n$ 
und $\text{rang}((DF_1)_{Y})=k$ gilt. 
\qed\\



\noindent Wir erklären nun zunächst den Tangentialraum an $M$ in 
einem Punkt $X \in M$ und schließlich Differentialformen auf $M$.

\bigskip

\noindent {\bf Definition.} Sei $M$ eine $k$-dimensionale 
Mannigfaltigkeit in ${\mathbb R}^{n}$, $X \in M$ und $F: W \to 
{\mathbb R}^{n}$ ein Koordinatensystem um $X$. Dann wird der 
$k$-dimensionale Unterraum
\[ TM_{X} \da  DF_{F^{-1}(X)} ({\mathbb R}^{k}) \subset {\mathbb R}^{n} 
\]
als der {\em Tangentialraum}\index{Tangentialraum} von $M$ in $X$ bezeichnet.\\

\noindent
Damit diese Terminologie gerechtfertigt ist, sollte $TM_{X}$ von der 
Wahl des Koordinatensystems unabhängig sein. 
Zum Nachweis sei also $\overline{F}: \overline{W} \to {\mathbb R}^{n}$ ein weiteres 
Koordinatensystem um $X$. Nach Satz \ref{Satz3.6.4} ist
\[ D(F^{-1} \circ \overline{F})_{\overline{F}^{-1}(X)}: {\mathbb 
R}^{k} \to {\mathbb R}^{k} \]
bijektiv. Wegen $F \circ (F^{-1} \circ \overline{F}) = \overline{F}$ 
(auf geeignetem Definitionsbereich) folgt so
\[
D\overline{F}_{\overline{F}^{-1}(X)} ({\mathbb R}^{k})  =  
DF_{F^{-1}(X)} \circ D(F^{-1} \circ 
\overline{F})_{\overline{F}^{-1}(X)} ({\mathbb R}^{k}) 
 =  DF_{F^{-1}(X)}({\mathbb R}^{k}),
\]
was zu zeigen war.



\bigskip

\noindent {\bf Definition.} Sei $M$ eine Mannigfaltigkeit in ${\mathbb 
R}^{n}$, sei $p \in {\mathbb N}_{0}$. Eine {\em Differentialform vom Grad}\index{Differentialform}  
$p$ (kurz: $p$-Form) {\em auf} $M$ ist eine Abbildung, die jedem $X \in M$ 
ein Element von $\Omega^{p}(TM_{X})$ zuordnet.

\bigskip

\noindent {\bf Beispiel.} Sei $A \subset {\mathbb R}^{n}$ 
offen, $M\subset A$ eine $k$-dimensionale 
Mannigfaltigkeit im ${\mathbb R}^{n}$ und $\omega$ eine $p$-Form auf $A$. Für 
$X \in A$ und insbesondere für $X \in M$ ist $\omega_{X} \in 
\Omega^{p}({\mathbb R}^{n})$, d.h.
\[ \omega_{X}: \underbrace{{\mathbb R}^{n} \times \dots \times 
{\mathbb R}^{n}}_{p-\text{mal}} \to {\mathbb R} \]
ist eine alternierende multilineare Abbildung. Einschränkung von 
$\omega_{X}$, $X \in M$, auf das $p$-fache Produkt $TM_{X} \times \dots \times TM_{X}$ 
ergibt eine Differentialform vom Grad $p$ auf $M$.\\

\noindent
Sei nun $\omega$ eine $p$-Form auf der $k$-dimensionalen 
Mannigfaltigkeit $M$, sei $X \in M$ und $F: W \to {\mathbb R}^{n}$ 
ein Koordinatensystem um $X$ sowie $Z \in W$ mit $F(Z) = X$. Sei 
ferner $\overline{F}: \overline{W} \to {\mathbb R}^{n}$ ein weiteres 
Koordinatensystem um $X$ und $\overline{Z} \in \overline{W}$ mit 
$\overline{F}(\overline{Z}) = X$. Wegen $DF_{Z}: {\mathbb R}^{k} \to 
TM_{X}$ ist
\[ (DF_{Z})^{*}: \Omega^{p}(TM_{X}) \to \Omega^{p}({\mathbb R}^{k}) \]
erklärt, und wir setzen
\[ (F^{*}\omega)_{Z} \da  (DF_{Z})^{*} \omega_{X}. \]
Auf diese Weise wird eine $p$-Form $F^{*}\omega$ auf $W$ erklärt. 
Nach Satz \ref{Satz3.6.4} ist die Abbildung $G \da  F^{-1} \circ \overline{F}$ in 
einer Umgebung von $\overline{Z}$ ein Diffeomorphismus. Aus 
$\overline{F} = F \circ G$ folgt daher
\begin{eqnarray*}
(\overline{F}^{*}\omega)_{\overline{Z}} & = & 
(D\overline{F}_{\overline{Z}})^{*} \omega_{X} = (DF_{Z} \circ 
DG_{\overline{Z}})^{*} \omega_{X} \\
& = & (DG_{\overline{Z}})^{*} \left((DF_{Z})^{*} \omega_{X}\right) = 
(DG_{\overline{Z}})^{*} (F^{*} \omega)_{Z} \\
& = & G^{*} (F^{*} \omega)_{\overline{Z}},
\end{eqnarray*}
also $\overline{F}^{*}\omega = G^{*}(F^{*} \omega)$ im gemeinsamen 
Definitionsbereich. Mit $F^{*} \omega$ ist also auch 
$\overline{F}^{*} \omega$ differenzierbar.

\bigskip

\noindent {\bf Definition.} Die $p$-Form $\omega$ auf der 
Mannigfaltigkeit $M$ heißt {\em differenzierbar}, wenn für jedes 
Koordinatensystem $F$ von $M$ die Form $F^{*} \omega$ differenzierbar 
ist.\\

\noindent
Wir überlegen uns jetzt, daß man auch für $p$-Formen auf $M$ eine 
äußere Ableitung erklären kann.

\bigskip

\begin{satz}\label{Satz3.6.5.} {Sei M eine k-dimensionale 
Mannigfaltigkeit in ${\mathbb R}^{n}$ und $\omega$ eine 
differenzierbare $p$-Form auf $M$. Dann gibt es eine eindeutig bestimmte 
$(p+1)$-Form $d\omega$ auf $M$, das äußere Differential von $\omega$, so daß
\[ F^{*}(d\omega) = d(F^{*}\omega) \]
für jedes Koordinatensystem $F$ der Mannigfaltigkeit $M$ gilt.}
\end{satz}

\bigskip

\noindent {\em Beweis.} Sei $F: W \to {\mathbb R}^{n}$ ein 
Koordinatensystem um $X \in M$ und $Z \in W$ mit $F(Z) = X$. Zu 
beliebigen Vektoren $Y_{1},\dots,Y_{p+1} \in TM_{X}$ gibt es eindeutig 
bestimmte Vektoren $V_{1},\dots,V_{p+1} \in {\mathbb R}^{k}$ mit 
$DF_{Z}(V_{i}) = Y_{i}$ für $i = 1,\dots,p+1$. Falls $d\omega$ 
tatsächlich existiert mit der geforderten Eigenschaft, so folgt
\begin{eqnarray*}
(d\omega)_{X}(Y_{1},\dots,Y_{p+1}) & = & 
(d\omega)_{X}(DF_{Z}(V_{1}),\dots,DF_{Z}(V_{p+1})) \\
& = & \left[ (DF_{Z})^{*} \, d\omega_{X}\right] (V_{1},\dots,V_{p+1}) 
\\
& = & (F^{*}\, d\omega)_{Z}(V_{1},\dots,V_{p+1}) \\
& = & d(F^{*}\omega)_{Z} (V_{1},\dots,V_{p+1}). 
\end{eqnarray*}
Dies zeigt, daß $d\omega$ eindeutig bestimmt ist durch die 
geforderte Eigenschaft. Wir definieren nun
\[ (d\omega)_{X}(Y_{1},\dots,Y_{p+1}) \da  
d(F^{*}\omega)_{Z}(V_{1},\dots,V_{p+1}). \]
Wir  rechnen nun nach, daß diese Definition nicht von der Wahl 
des Koordinatensystems abhängt und die gewünschte Eigenschaft 
besitzt.\\


\noindent {\bf Wahlunabhängigkeit}: Seien $F: W \to M$, 
$\overline{F}: \overline{W} \to M$ Koordinatensysteme von $M$ um $X$ 
mit $F(Z) = \overline{F}(\overline{Z}) = X$. Sei $G \da  F^{-1} \circ 
\overline{F}$ in einer geeigneten Umgebung von $\overline{Z}$, d.h. 
$G(\overline{Z}) = Z$. Schließlich seien $V_{i},\overline{V}_{i} 
\in {\mathbb R}^{k}$ mit
\[ DF_{Z}(V_{i}) = Y_{i} = 
D\overline{F}_{\overline{Z}}(\overline{V}_{i}), \quad i = 
1,\dots,p+1. \]
Wegen $\overline{F} = F \circ G$ gilt 
$D\overline{F}_{\overline{Z}}(\overline{V}_{i}) = 
DF_{Z}(DG_{\overline{Z}}(\overline{V}_{i}))$ und damit 
$DG_{\overline{Z}}(\overline{V}_{i}) = V_{i}$. Nun folgt
\begin{eqnarray*}
d(\overline{F}^{*}\omega)_{\overline{Z}}(\overline{V}_{1},\dots,
\overline{V}_{p+1})  & = & d((F \circ 
G)^{*}\omega)_{\overline{Z}}(\overline{V}_{1},\dots,\overline{V}_{p+1}) 
\\
& = & 
d(G^{*}(F^{*}\omega))_{\overline{Z}}(\overline{V}_{1},\dots,
\overline{V}_{p+1}) \\
& = & (G^{*}d(F^{*}\omega))_{\overline{Z}}(\overline{V}_{1},\dots,
\overline{V}_{p+1}) \\
& = & 
d(F^{*}\omega)_{Z}(DG_{\overline{Z}}(\overline{V}_{1}),\dots,DG_{\overline{Z}}
(\overline{V}_{p+1})) \\
& = & d(F^{*}\omega)_{Z} (V_{1},\dots,V_{p+1}).
\end{eqnarray*}

\bigskip

\noindent {\bf Geforderte Eigenschaft:} Wir verwenden obige Notation, 
die Definition von $F^{*}(d\omega)$ und schließlich die Definition 
von $d\omega$, um so zu erhalten
\begin{eqnarray*}
F^{*}(d\omega)_{Z}(V_{1},\dots,V_{p+1}) & = & 
(d\omega)_{X}(DF_{Z}(V_{1}),\dots,DF_{Z}(V_{p+1})) \\
& = & d(F^{*}\omega)_{Z}(V_{1},\dots,V_{p+1}),
\end{eqnarray*}
also in der Tat $d(F^{*}\omega) = F^{*}(d\omega)$.
\qed\\

\bigskip

\noindent {\bf Orientierung.} \\

\noindent
In Abschnitt 3.2 hatten wir gesehen, daß die klassischen Sätze der 
Vektoranalysis eine geeignete Orientierung der betrachteten Kurven und 
Flächen voraussetzen. Der Begriff einer Orientierung
 ist jetzt auf Mannigfaltigkeiten und deren 
Ränder zu übertragen.\\

\noindent
Sei $V$ ein $n$-dimensionaler reeller Vektorraum mit Basen 
$(e_{1},\dots,e_{n})$ und 
$(\overline{e}_{1},\dots,\overline{e}_{n})$.  
 Ist $\overline{e}_{i} = \sum_{j=1}^{n} 
\alpha_{ji}e_{j}$ für $i = 1,\dots,n$, so setzen wir
\[ \Delta \da  {\rm det}\, (\alpha_{ij})_{i,j=1}^{n}. \]
Man nennt $(e_{1},\dots,e_{n})$ und 
$(\overline{e}_{1},\dots,\overline{e}_{n})$ äquivalent, falls 
$\Delta > 0$ gilt. Hierdurch ist eine Äquivalenzrelation auf den 
geordneten Basen von $V$ gegeben mit genau zwei Äquivalenzklassen. 
Die Äquivalenzklasse zu $(e_{1},\dots,e_{n})$ wird mit
\[ [e_{1},\dots,e_{n}] \]
bezeichnet, die davon verschiedene Äquivalenzklasse wird mit 
$-[e_{1},\dots,e_{n}]$ bezeichnet. Eine Orientierung auf $V$ ist eine 
Äquivalenzklasse geordneter Basen.\\

\bigskip

\noindent {\bf Definition.} Sei $M$ eine $k$-dimensionale 
Mannigfaltigkeit in ${\mathbb R}^{n}$. Eine 
{\em Orientierung}\index{Orientierung} auf $M$ ist 
eine Abbildung $o$, die jedem $X \in M$ eine Orientierung 
$o_{X}$ des Tangentialraumes $TM_{X}$ zuordnet, so daß gilt: Für 
jedes Koordinatensystem $F: W \to {\mathbb R}^{n}$ von $M$ mit 
zusammenhängendem $W$ gilt
\begin{eqnarray*}
\mbox{ entweder } &   & [DF_{Z}(E_{1}),\dots,DF_{Z}(E_{k})] = 
o_{F(Z)} \quad \, \mbox{ für } Z \in W \\
\mbox{ oder } &   & [DF_{Z}(E_{1}),\dots,DF_{Z}(E_{k})] = -o_{F(Z)} 
\quad \mbox{ für } Z \in W.
\end{eqnarray*}
Im ersten Fall heißt $F$ {\em orientierungstreu}, im zweiten Fall 
{\em orientierungsumkehrend}. Ist $o$ eine Orientierung auf $M$, so 
heißt $(M, o)$ orientierte Mannigfaltigkeit.

\bigskip

\noindent {\bf Beispiele.} \begin{itemize}
\item Der $\MdR^n$ als orientierte $n$-dimensionale Mannigfaltigkeit, offene 
Teilmengen davon. Diskussion des Zusammenhangs der Menge $W$ in der vorangehenden 
Definition.
\item Reguläre Kurven; zumindest lokal: parametrisierte Flächen.
\item Wir hatten schon erwähnt, daß das 
Möbiusband im ${\mathbb R}^{3}$ ein Beispiel für eine 
Mannigfaltigkeit ist, auf der keine Orientierung existiert. Für 
eine $(n-1)$-dimensionale Mannigfaltigkeit im ${\mathbb R}^{n}$ gibt 
es genau dann eine Orientierung (wie man zeigen kann), wenn $M$ ein stetiges 
Einheitsnormalenvektorfeld besitzt. Dies kann man verwenden, um die 
Nichtorientierbarkeit des Möbiusbandes einzusehen.
\end{itemize}

\bigskip

\noindent {\bf Bemerkung.} Die Orientierbarkeit einer 
$k$-dimensionalen Mannigfaltigkeit $M$ ist äquivalent zur Existenz 
einer nirgends verschwindenden stetigen $k$-Form auf $M$ oder auch zur Existenz 
eines {\em orientierten Atlas}. Ein {\em Atlas} ist hierbei eine Familie von 
Koordinatensystemen von $M$, deren Bilder $M$ überdecken. Ein Atlas heißt orientiert, 
wenn je zwei seiner Elemente gleichorientiert sind. 

\bigskip


\noindent
Bei der folgenden Aussage ist die Argumentation im Beweis ebenso wichtig 
wie die Aussage selbst.\\


\begin{lemma}\label{Lemma3.6.6} {Sei $(M,o)$ eine k-dimensionale 
orientierte Mannigfaltigkeit im ${\mathbb R}^{n}$, seien $F, 
\overline{F}$ zwei orientierungstreue Koordinatensysteme um X, sei 
$F(Z) = X = \overline{F}(\overline{Z})$. Dann ist ${\rm det}\, 
D(\overline{F}^{-1} \circ F)_{Z} > 0$.}
\end{lemma}

\bigskip

\noindent {\em Beweis.} Nach Voraussetzung gilt
\[ [DF_{Z}(E_{1}),\dots,DF_{Z}(E_{k})] = o_{X} = 
[D\overline{F}_{\overline{Z}}(E_{1}),
\dots,D\overline{F}_{\overline{Z}}(E_{k})]. \]
Zwei äquivalente Basen bleiben nach Anwendung eines beliebigen 
Isomorphismus äquivalent. Also folgt durch Anwendung des Isomorphismus  
$(D\overline{F}_{\overline{Z}})^{-1}:
TM_X\to \MdR^k$
\[ \left[D(\overline{F}^{-1} \circ F)_{Z} 
(E_{1}),\dots,D(\overline{F}^{-1} \circ F)_{Z}(E_{k})\right] = 
[E_{1},\dots,E_{k}], \]
was die Behauptung zeigt. \qed\\

\bigskip

\noindent
Aufgrund von Satz \ref{Satz3.6.1} ist bekannt, daß der Rand $\partial M$ einer 
$k$-dimensionalen Mannigfaltigkeit $M$ von ${\mathbb R}^{n}$ eine 
$(k-1)$-dimensionale Mannigfaltigkeit ist. Ist $(M,o)$ orientiert, so 
kann man auch auf $\partial M$ eine (induzierte) Orientierung 
einführen.

\bigskip

\begin{satz}\label{Satz3.6.7} {Sei $(M,o)$ eine k-dimensionale 
orientierte Mannigfaltigkeit in ${\mathbb R}^{n}$. Dann gibt es eine 
Orientierung $o'$ auf $\partial M$, so daß gilt: Für $X \in 
\partial M$, jedes Koordinatensystem $F: W \to {\mathbb R}^{n}$ von 
$M$ um $X$ und für alle Vektoren $Y_{1},\dots,Y_{k-1} \in T\partial 
M_{X}$ mit
\[ [DF_{Z}(-E_{1}),Y_{1},\dots,Y_{k-1}] = o_{X},\qquad  Z = F^{-1}(X)\]
gilt
\[ [Y_{1},\dots,Y_{k-1}] = o'_{X}. \]
Man nennt $o'$ die von $o$ induzierte Orientierung.}
\end{satz}

\bigskip




\noindent {\em Beweis.}  Sei $X \in \partial M$. Es gibt 
ein Koordinatensystem $F$ von $M$ um $X$, so daß für die Basis 
$(DF_{Z}(-E_{1}), DF_{Z}(E_{2}),\dots,DF_{Z}(E_{k}))$ von $TM_{X}$ 
gilt
\[ o_{X} = [DF_{Z}(- E_{1}), DF_{Z}(E_{2}),\dots,DF_{Z}(E_{k})]. \]
Dann ist zunächst $F(0,\cdot)$ ein Koordinatensystem von $\partial M$ 
um $X$ und $(DF_{Z}(E_{2}),\dots,DF_{Z}(E_{k}))$ ist eine Basis von 
$T\partial M_{X}$. Wir definieren
\[ o'_{X} \da  [DF_{Z}(E_{2}),\dots,DF_{Z}(E_{k})]. \]

\bigskip

\noindent {\bf Wohldefiniertheit:} Sei $\overline{F}$ ein weiteres 
Koordinatensystem von $M$ um $X$ mit $\overline{F}(\overline{Z}) = X$ 
und
\[ o_{X} = [D\overline{F}_{\overline{Z}}(-E_{1}), 
D\overline{F}_{\overline{Z}}(E_{2}),\dots,D\overline{F}_{\overline{Z}}(E_{k})]. 
\]
Wie im Beweis zu Lemma \ref{Lemma3.6.6} folgt ${\rm det}(D(\overline{F}^{-1} 
\circ F)_{Z}) > 0$. Ferner gilt
\[
\langle E_{1},D(\overline{F}^{-1} \circ F)_{Z}(E_{1})\rangle  =  
\lim_{t \downarrow 0} \frac{1}{t} \langle \overline{F}^{-1} \circ 
F(Z + tE_{1}), E_{1} \rangle  \ge  0.
\]
Wegen
\[ D(\overline{F}^{-1} \circ F)_{Z} (E_{i}) \in {\rm 
lin}\{E_{2},\dots,E_{k}\} \]
für $i = 2,\dots,k$, folgt schließlich mit $E_1^\perp=\text{lin}\{E_2,\ldots,E_k\}$
\[ {\rm det}(D(\overline{F}^{-1} \circ F)_{Z} \mid E_{1}^{\perp}) > 
0 \]
und damit
\[ [DF_{Z}(E_{2}),\dots,DF_{Z}(E_{k})] = 
[D\overline{F}_{\overline{Z}}(E_{2}),\dots,D\overline{F}_{\overline{Z}}(E_{k})]. 
\]
Ist nun $Y_{1},\dots,Y_{k-1} \in T\partial M_{X}$ und
\[ o_{X} = [DF_{Z}(-E_{1}),Y_{1},\dots,Y_{k-1}], \]
wobei $F$ wie in der Definition von $o_{X}$ gewählt sei, so gilt
\[ Y_{i} = \sum_{j=2}^{k} \alpha_{ji} DF_{Z} (E_{j}), \quad i = 
1,\dots,k-1, \]
mit ${\rm det}(\alpha_{ji}) > 0$. 
Daher folgt
\[ o'_{X} = [DF_{Z}(E_{2}),\dots,DF_{Z}(E_{k})] = 
[Y_{1},\dots,Y_{k-1}]. \]
Wir zeigen nun noch, daß $o'$ eine Orientierung von $\partial M$ 
ist. Sei hierzu $G: W \to {\mathbb R}^{n}$ eine Karte von $\partial 
M$ mit zusammenhängender Menge $W$. Sei $Z_{0} \in W$ und $X_{0} \da  
G(Z_{0})$. Sei $F$ eine Karte von $M$ um $X_{0}$, die wie in der 
Definition von $o'$ orientiert sei. Sei $i: {\mathbb R}^{k-1} \to 
{\mathbb R}^{k}$, $i(x) \da  (0,x)$. Ferner setze $\tilde{F} \da  F \circ 
i$  auf einen passenden Definitionsbereich. Dann gilt zunächst lokal 
um $Z_{0}$:
\begin{eqnarray*}
&   & [DG_{Z}(E_{1}),\dots,DG_{Z}(E_{k-1})]  \\
& = & \left[D(\tilde{F} \circ(\tilde{F}^{-1} \circ 
G))_{Z}(E_{1}),\dots,D(\tilde{F} \circ(\tilde{F}^{-1} \circ 
G))_{Z}(E_{k-1})\right] \\
& = & \left[D\tilde{F}_{\tilde{F}^{-1}(X)} (D(\tilde{F}^{-1} \circ 
G)_{Z}(E_{1})),\dots,D\tilde{F}_{\tilde{F}^{-1}(X)} 
(D(\tilde{F}^{-1} \circ G)_{Z}(E_{k-1})) \right] \\
& = & \sgn ({\rm det}(D(\tilde{F}^{-1} \circ 
G)_{Z}))[D\tilde{F}_{\tilde{F}^{-1}(X)}(E_{1}),\dots,
D\tilde{F}_{\tilde{F}^{-1}(X)}(E_{k-1})] \\
& = & f(z) \cdot [DF_{Z}(E_{2}),\dots,DF_{Z}(E_{k-1})] \\
& = & f(z) o'_{G(z)}, 
\end{eqnarray*}
wobei
\[ f(z) \da  \sgn ({\rm det}(D(\tilde{F} \circ G)_{Z})). \]
Die bewiesene Gleichung zeigt, daß $f$ unabhängig von der lokalen 
Darstellung ist und $f(z) \in \{-1,1\}$. Ferner ist $f$ aufgrund der 
lokalen Darstellung stetig. Dies zeigt die Konstanz von $f$ auf der 
zusammenhängenden Menge $W$. \qed\\









\bigskip

\noindent {\bf Bemerkung.} Ist $(M,o)$ eine orientierte 
Mannigfaltigkeit, so läßt man gelegentlich die Angabe des Symbols $o$ weg, 
wenn dieses aus dem Zusammenhang klar ist.\\

\bigskip

\noindent {\bf Zerlegung der Eins.} \\

\noindent
Häufig liegt in geometrischen, topologischen oder analytischen 
Fragestellungen die folgende Situation vor. Ein Objekt läßt sich 
lokal gut beschreiben bzw. definieren, und man will diese lokalen 
Beschreibungen global „glatt“ zusammensetzen. Ein nützliches 
Hilfsmittel hierfür ist die nachfolgend beschriebene Konstruktion, 
die zu einer Menge von nichtnegativen Funktionen führt, die sich an jeder Stelle zu 
Eins aufsummieren. Die Vorgehensweise hier ist spezieller als 
üblicherweise bei der Untersuchung {\em parakompakter Räume} in der 
Topologie. Allerdings können die zerlegenden Funktionen  
zusätzlich differenzierbar gewählt werden. 

\bigskip

\begin{lemma}\label{Lemma3.6.8} {Sei $U \subset {\mathbb R}^{n}$ offen 
und $C \subset U$ kompakt. Dann gibt es eine kompakte Menge $D \subset 
U$ mit $C \subset D^{0}$.}
\end{lemma}

\bigskip

\noindent {\em Beweis.} Zu jedem $X \in C$ gibt es ein $a_{X} > 0$ mit
\[ W(X, a_{X}) \da  \{Y \in {\mathbb R}^{n}: \|Y - X\|_{\max} \le 
a_{X}\} \subset U. \]
Das System $\{W(X,a_{X})^{0}: X \in C\}$ ist eine offene Überdeckung 
von $C$, enthält also eine endliche Teilüberdeckung 
$\{W(X_{i},a_{X_i}): i = 1,\dots,m\}$. Die Menge
\[ D \da  \bigcup_{i=1}^{m} W(X_{i},a_{X_i}) \]
leistet das Gewünschte. \qed\\

\bigskip

\begin{lemma}\label{Lemma3.6.9} {Sei $U \subset {\mathbb R}^{n}$ 
offen und $C \subset U$ kompakt. Dann gibt es eine kompakte Menge $D 
\subset U$ mit $C \subset D^{0}$ und eine differenzierbare Funktion 
$\varphi: {\mathbb R}^{n} \to [0,1]$ mit
\[ \varphi(X) = \left\{ \begin{array}{ll}
1, & \mbox{ für } X \in C, \\ 0, & \mbox{ für } X \in {\mathbb 
R}^{n} \setminus D. \end{array} \right. \]}
\end{lemma}

\bigskip

\noindent {\em Beweis.} (1) Die Funktion $f: {\mathbb R} \to {\mathbb 
R}$ ist erklärt durch
\[ f(x) \da  \left\{ \begin{array}{cl}
{\rm exp}\{-(x-1)^{-2}\} {\rm exp}\{-(x+1)^{-2}\}, & x \in (-1,1), \\
0, & \mbox{sonst.} \end{array} \right. \]
Dann ist $f > 0$ auf $(-1,1)$ und $f = 0$ auf ${\mathbb R} 
\setminus (-1,1)$. Ferner ist $f$ von der Klasse $C^{\infty}$.\\

\noindent
(2) Für $Z \in {\mathbb R}^{n}$ und $a > 0$ sei $g: {\mathbb R}^{n} 
\to {\mathbb R}$ erklärt durch
\[ g(X) \da  \prod_{j=1}^{n} f\left(\frac{x_{j}-z_{j}}{a}\right). \]
Dann ist $g$ von der Klasse $C^{\infty}$, $g > 0$ auf $W(Z,a)^{0}$ 
und $g = 0$ auf ${\mathbb R}^{n} \setminus W(Z,a)^{0}$.\\

\noindent
(3) Seien $\{W(X_{i},a_{X_i}): i = 1,\dots,m\}$ und $D$ wie im Beweis 
von Lemma \ref{Lemma3.6.8} konstruiert. Zu $i \in \{1,\dots,m\}$ sei $g_{i}$ die 
Funktion, die zu $X_{i}, a_{X_i}$ so erklärt ist wie oben $g$ zu $Z, 
a$. Definiere $\psi: {\mathbb R}^{n} \to {\mathbb R}$ durch
\[ \psi(X) \da  \sum_{i=1}^{m} g_{i}(X). \]
Dann ist $\psi$ von der Klasse $C^{\infty}$. Zu $X \in C$ gibt es ein 
$j \in \{1,\dots,m\}$ mit $X \in W(X_{j},a_{X_j})^{0}$ und somit 
$\psi(X) \ge g_{j}(X) > 0$. Für $X \in {\mathbb R}^{n} \setminus D$ 
gilt $X \notin W(X_{i},a_{X_i})$, also $g_{i}(X) = 0$, für $i = 
1,\dots,m$, d.h. $\psi(X) = 0$. Da $\psi$ stetig und $C$ kompakt ist, 
gibt es ein $\epsilon > 0$ mit $\psi(X) \ge \epsilon > 0$ für alle 
$X \in C$.\\

\noindent
(4) Sei $\tilde{f} : {\mathbb R} \to {\mathbb R}$ eine Funktion der 
Klasse $C^{\infty}$ mit $\tilde{f}(x) > 0$ für $x \in (0,\epsilon)$ 
und $f(X) = 0$ sonst (siehe (1)). Setze
\[ h(x) \da  \int_{0}^{x} f / \int_{0}^{\epsilon} f, \qquad x \in 
{\mathbb R}. \]
Dann ist $h: {\mathbb R} \to [0,1]$ differenzierbar, und es gilt 
$h(x) = 0$ für $x \le 0$ und $h(x) = 1$ für $x \ge \epsilon$.\\ 

\noindent
(5) Die Funktion $\varphi \da  h \circ \psi: {\mathbb R}^{n} \to [0,1]$ 
ist von der Klasse $C^{\infty}$, $\varphi(X) = 1$ für $X \in C$ und 
$\varphi(X) = 0$ für $X \in {\mathbb R}^{n} \setminus D$. Dies zeigt 
sämtliche Behauptungen. \qed\\

\bigskip

\noindent
Das im folgenden Satz konstruierte Funktionensystem 
$\{\varphi_{1},\dots,\varphi_{m}\}$ nennt man eine der Überdeckung 
$\{U_{1},\dots,U_{m}\}$ untergeordnete {\em Zerlegung der Eins}\index{Zerlegung der Eins}.\\

\bigskip

\begin{satz}\label{Satz3.6.10} {Sei $M \subset {\mathbb R}^{n}$ 
kompakt und $\{U_{1},\dots,U_{m}\}$ eine offene Überdeckung von M. 
Dann gibt es differenzierbare reelle Funktionen 
$\varphi_{1},\dots,\varphi_{m}$ auf ${\mathbb R}^{n}$ mit den 
folgenden Eigenschaften:}
\begin{enumerate}
\item[{\rm (a)}] $0 \le \varphi_{i} \le 1$ {\em für} $i = 1,\dots,m$,
\item[{\rm (b)}] $\varphi_{1}(X) + \dots + \varphi_{m}(X) = 1$ {\em für alle} $X 
\in M$,
\item[{\rm (c)}] {es gibt eine kompakte Menge $A_{i} \subset U_{i}$ mit 
$\varphi_{i}(X) = 0$ für $X \in {\mathbb R}^{n} \setminus A_{i}$ 
für $i = 1,\dots,m$.}
\end{enumerate}
\end{satz}

\bigskip

\noindent {\em Beweis.} Zunächst werden kompakte Mengen $D_{i} 
\subset U_{i}$ konstruiert, so daß $M \subset D_{1}^{0} \cup \dots 
\cup D_{m}^{0}$ gilt. Hierzu zeigen wir allgemeiner, um vollständige 
Induktion verwenden zu können: Für $k \in \{0,\dots,m\}$ gibt es 
kompakte Mengen $D_{i} \subset U_{i}$ $(i = 1,\dots,k)$ mit
\[ M \subset D_{1}^{0} \cup \dots \cup D_{k}^{0} \cup U_{k+1} \cup 
\dots \cup U_{m}. \]
Beweis durch vollständige Induktion über $k$. Für $k = 0$ ist 
nichts zu zeigen. Seien $D_{1},\dots,D_{k}$ schon konstruiert. Dann 
ist 
\[ C_{k+1} \da  M \setminus \left(D_{1}^{0} \cup \dots \cup D_{k}^{0} 
\cup U_{k+2} \cup \dots \cup U_{m}\right) \]
eine kompakte Teilmenge von $U_{k+1}$. Nach Lemma \ref{Lemma3.6.8} gibt es eine 
kompakte Teilmenge $D_{k+1} \subset U_{k+1}$ mit $C_{k+1} \subset 
D_{k+1}^{0}$, also
\[ M \subset D_{1}^{0} \cup \dots \cup D_{k}^{0} \cup D_{k+1}^{0} \cup 
U_{k+2} \cup \dots \cup U_{m}. \]
Dies beendet den Induktionsschluß.\\

\noindent
Insbesondere gilt mit obigen Mengen $D_{i}$, $i = 1,\dots,m$:
\[ M \subset D_{1}^{0} \cup \dots \cup D_{m}^{0} =: U \]
Nach Lemma \ref{Lemma3.6.9} gibt es eine kompakte Menge $A_{i} \subset U_{i}$ 
mit $D_{i} \subset A_{i}^{0}$ und eine Funktion $\psi_{i}: {\mathbb 
R}^{n} \to [0,1]$ der Klasse $C^{\infty}$ mit $\psi_{i}(X) = 1$ für 
$X \in D_{i}$ und $\psi_{i}(X) = 0$ für $X \in {\mathbb R}^{n} 
\setminus A_{i}$, $i = 1,\dots,m$.\\

\noindent
Zu $X \in U$ gibt es ein $j \in \{1,\dots,m\}$ mit $X \in D_{j}^{0}$, 
d.h. $\psi_{j}(X) > 0$. Somit gilt $\psi_{1}(X) + \dots + 
\psi_{m}(X) > 0$. Nach Lemma \ref{Lemma3.6.9} gibt es eine kompakte Menge $A 
\subset U$ mit $M \subset A^{0}$ und eine differenzierbare Funktion 
$f: {\mathbb R}^{n} \to [0,1]$ mit $f(X) = 1$ für $X \in M$ und 
$f(X) = 0$ für $X \in {\mathbb R}^{n} \setminus A$. Setze
\[ \varphi_{i}(X) \da  \left\{ \begin{array}{ll}
f(X) \cdot \frac{\psi_{i}(X)}{\psi_{1}(X) + \dots + \psi_{m}(X)}, & X 
\in U, \\ 0, & X \in {\mathbb R}^{n} \setminus U. \end{array} 
\right. \]
Die Funktionen $\varphi_{1},\dots,\varphi_{m}$ leisten das 
Gewünschte. \qed\\




\section{Der Satz von Stokes}
Bevor wir den Satz von Stokes formulieren können, muß die 
 Integration von Differentialformen über Mannigfaltigkeiten erklärt 
 werden. Alle Differentialformen seien als stetig vorausgesetzt. 
 Ferner sei $M$ stets eine $k$-dimensionale orientierte Mannigfaltigkeit. 
 Für eine auf einer offenen Menge $\tilde{M} \subset {\mathbb R}^{n}$ erklärte 
 Differentialform $\omega$ vom Grad $p$ und einen singulären 
 $p$-Würfel in $\tilde{M}$ hatten wir 
\[ \int_{c} \omega \da  \int_{[0,1]^{p}} c^{*} \omega \]
definiert. Wir verallgemeinern dies jetzt auf den Fall einer Mannigfaltigkeit 
$M \subset {\mathbb R}^{n}$.

\bigskip

\noindent {\bf Definition.} Für eine $p$-Form $\omega$ auf $M$ und einen 
singulären $p$-Würfel $c$ in $M$ sei
\[ \int_{c} \omega \da  \int_{[0,1]^{p}} c^{*} \omega. \]

\bigskip

\noindent
Im folgenden werden wir tatsächlich ausschließlich $p$-Formen über $p$-Mannigfaltigkeiten 
integrieren.\\

\noindent {\bf Definition.} Ein singulärer $k$-Würfel $c: [0,1]^{k} \to M$ 
heißt {\em orientierungstreu}, wenn es ein orientierungstreues Koordinatensystem 
$F: W \to {\mathbb R}^{n}$ in $M$ gibt mit $c = F \mid [0,1]^{k}$. In diesem Fall nennen 
wir $c$ einen {\em orientierungstreuen $k$-Würfel}.\\

\noindent
Der folgende Hilfssatz zeigt an, daß das Integral einer $k$-Form auf $M$ 
über einen singulären Würfel „im wesentlichen“ von der Wahl des 
singulären Würfels unabhängig ist.

\bigskip

\begin{lemma}\label{Lemma3.7.1} {Seien $c_{1},c_{2}: [0,1]^{k} \to M$ 
orientierungstreue  $k$-Würfel in $M$. Sei $\omega$ eine $k$-Form auf $M$ 
mit $\omega_{X} = 0$ für $X \notin c_{1}([0,1]^{k}) \cap c_{2}([0,1]^{k}) =: B$. 
Dann ist}
\[ \int_{c_1} \omega = \int_{c_2} \omega. \]
\end{lemma}


\noindent {\em Beweis.} Sei $c_{2}^{*}\omega =: f \, dx^{1} \wedge \dots \wedge dx^{k}$ 
und $G \da  c_{2}^{-1} \circ c_{1}$ auf $c_{1}^{-1}(B)$. Dann gilt auf $c_{1}^{-1}(B)$:
\begin{eqnarray*}
c_{1}^{*}\omega & = & [c_{2} \circ (c_{2}^{-1} \circ c_{1})]^{*} \omega = 
G^{*} c_{2}^{*} \omega \\
& = & G^{*}(f \, dx^{1} \wedge \dots \wedge dx^{k}) \\
& = & f \circ G\, {\rm det}\, JG \, dx^{1} \wedge \dots \wedge dx^{k} \\
& = & f \circ G\, |{\rm det}\, JG| \, dx^{1} \wedge \dots \wedge dx^{k},
\end{eqnarray*}
wobei zuletzt benutzt wurde, daß ${\rm det}\, JG > 0$ gilt wegen Lemma \ref{Lemma3.6.6}. 
Mit dem Transformationssatz für Gebietsintegrale, angewandt mit dem Diffeomorphismus 
$G:c_1^{-1}(B)\to c_2^{-1}(B)$ erhält man nun
\[
\int_{c_{2}^{-1}(B)} c_{2}^{*} \omega  =  \int_{c_{2}^{-1}(B)} f 
 =  \int_{c_{1}^{-1}(B)} f \circ G\, |{\rm det}\, JG| 
 =  \int_{c_{1}^{-1}(B)} c_{1}^{*} \omega.
\]
Da $\omega$ außerhalb von $B$ ohnehin verschwindet, folgt
\[
\int_{c_1} \omega  =  \int_{[0,1]^{k}} c_{1}^{*} \omega = \int_{c_{1}^{-1}(B)} 
c_{1}^{*} \omega = \int_{c_{2}^{-1}(B)} c_{2}^{*} \omega \\
 =  \int_{[0,1]^{k}} c_{2}^{*} \omega = \int_{c_{2}} \omega,
\]
was zu zeigen war. \qed\\

\bigskip

\noindent
Lemma \ref{Lemma3.7.1} stellt gerade  die für die folgende Definition 
erforderliche Wohldefiniertheitsaussage bereit.

\bigskip

\noindent {\bf Definition.} Sei $c$ ein orientierungstreuer 
$k$-Würfel in $M$ und $\omega$ eine $k$-Form auf $M$ mit $\omega_{X} = 0$ 
für $X \notin c([0,1]^{k})$. Dann sei
\[ \int_{M} \omega \da  \int_{c} \omega. \]

\bigskip

\noindent
Der singuläre $k$-Würfel $c$ in $M$ heißt {\em orientierungsumkehrend}, wenn es 
ein orientierungsumkehrendes Koordinatensystem $F: W \to {\mathbb R}^{n}$ 
gibt mit $c = F \mid [0,1]^{k}$. Ist dies der Fall und ist 
$\omega$ eine $k$-Form auf $M$, die außerhalb $c([0,1]^{k})$ verschwindet, 
so ist
\[ \int_{c} \omega = - \int_{M} \omega. \]
Zum Nachweis sei\[ \tilde{F} : [0,1]^{k} \to {\mathbb R}^{n}, \quad 
\tilde{F}(x_{1},\dots,x_{n}) \da  F(x_{1},\dots,x_{k-1},1 - x_{k}) \]
und $\tilde{c} \da  \tilde{F} \mid [0,1]^{k}$. Dann ist $\tilde{F}$ und 
somit $\tilde{c}$ orientierungstreu, d.h.
\[ \int_{M} \omega = \int_{\tilde{c}} \omega. \]
Andererseits folgt wie im Beweis Lemma \ref{Lemma3.7.1}, daß
\[ \int_{\tilde{c}} \omega = -\int_{c} \omega, \]
da nun $|\det JG|=-\det JG$ gilt.

\bigskip

\noindent {\bf Definition.} Sei $c$ ein singulärer $k$-Würfel in $M$. 
Dann heißt $c$ {\em normal}, wenn $c$ orientierungstreu ist und wenn
\[ c([0,1]^{k}) \cap \partial M = \emptyset \qquad\mbox{oder}\qquad c([0,1]^{k}) \cap \partial M = c \circ I_{1,0}^{k}([0,1]^{k-1}) \]
gilt.

\bigskip

\begin{lemma}\label{Lemma3.7.2} {Sei M kompakt. Dann gibt es eine offene 
Überdeckung $\{U_{1},\dots,U_{m}\}$ von M derart, 
daß zu jedem $i \in \{1,\dots,m\}$ ein normaler k-Würfel 
$c_{i}$ existiert mit}\[ U_{i} \cap M \subset c_{i}([0,1]^{k}). \]
\end{lemma}

\bigskip

\noindent {\em Beweis.} Zu $X \in M$ existiert ein Koordinatensystem 
$F: W \to {\mathbb R}^{n}$, das o.B.d.A. so gewählt werden kann, 
daß $F$ orientierungstreu ist, $[0,1]^{k} \subset W$ und
\[ X = \left\{ \begin{array}{ll}  F\left(\frac{1}{2},\dots,\frac{1}{2} \right), 
\quad & \mbox{ falls } X \notin \partial M, \\ \\
F\left(0,\frac{1}{2},\dots,\frac{1}{2}\right), \quad & 
\mbox{ falls } X \in \partial M \end{array} \right. 
\]
erfüllt. Die Einschränkung $c \da  F \mid [0,1]^{k}$ ist dann ein normaler 
Würfel. Da $F^{-1}$ stetig ist, gibt es eine offene Umgebung $U_{X}$ 
von $X$ mit $F^{-1}(U_{X} \cap M) \subset [0,1]^{k}$, also mit 
$U_{X} \cap M \subset c([0,1]^{k})$. Das System $\{U_{X}: X \in M\}$ 
ist eine offene Überdeckung von $M$, enthält also wegen der Kompaktheit von $M$ eine 
endliche Teilüberdeckung von $M$. \qed\\

\noindent
Nun sind alle Vorbereitungen abgeschlossen, um die Definition des Integrals einer $k$-Form 
über eine kompakte $k$-dimensionale orientierte Mannigfaltigkeit geben zu können.\\

\noindent
{\bf  Definition.} Sei $M$ kompakt und $\omega$ eine $k$-Form auf $M$. Sei $\{U_1,\ldots,U_m\}$ 
eine Überdeckung von $M$ wie in Lemma \ref{Lemma3.7.2} und hierzu $\{\varphi_1,\ldots,\varphi_m\}$ 
eine der Überdeckung untergeordnete Zerlegung der Eins. Dann definiert man
\[ \int_{M} \omega \da  \sum_{i=1}^{m} \int_{M} \varphi_{i}\omega. \]

\bigskip

\noindent {\bf Nachweis.} Zunächst ist festzustellen, daß das Integral
\[ \int_{M} \varphi_{i}\omega \]
schon definiert ist. Zu $U_{i}$ existiert ja ein normaler $k$-Würfel $c_{i}$ mit 
$U_{i} \cap M \subset c_{i}([0,1]^{k})$. Die Funktion $\varphi_{i}$ und 
damit die Form $\varphi_{i}\omega$ verschwindet außerhalb von $U_{i}$, 
also erst recht außerhalb $c_{i}([0,1]^{k})$. Ist schließlich 
$V_{1},\dots,V_{r}$ eine weitere offene Überdeckung von $M$ wie in Lemma \ref{Lemma3.7.2} und $\{\psi_{1},\dots,\psi_{r}\}$ eine zugehörige Zerlegung der Eins, so ist wegen der Linearität 
des Integrals
\[
\sum_{i} \int_{M} \varphi_{i} \omega  =  \sum_{i} \sum_{j} \int_{M} \varphi_{i} 
\psi_{j} \omega = \sum_{j} \sum_{i} \int_{M} \varphi_{i} \psi_{j}\omega  =  
\sum_{j} \int_{M} \psi_{j}\omega,
\]
was die Wohldefiniertheit zeigt.\\

\noindent
Wir kommen schließlich zum Satz von Stokes.

\bigskip

\begin{satz}\label{Satz3.7.3} {Sei $(M,o)$ eine kompakte orientierte 
k-dimensionale Mannigfaltigkeit und $\omega$ eine $(k-1)$-Form der Klasse $C^1$ auf M. 
Der Rand $\partial M$ sei mit der induzierten Orientierung versehen. 
Dann gilt}
\[ \int_{M} \, d\omega = \int_{\partial M} \omega. \]
\end{satz}

\bigskip

\noindent {\em Beweis.} Sei zunächst $c$ ein normaler $k$-Würfel in $M$, und 
$\omega$ verschwinde außerhalb von $c([0,1]^{k})$. Dann gilt
\[
\int_{M} \, d\omega  =  \int_{c}  d\omega = \int_{[0,1]^{k}} c^{*}  d\omega 
= \int_{[0,1]^{k}} dc^{*} \omega 
 =  \int_{I^{k}}  dc^{*} \omega = \int_{\partial I^{k}} c^{*} \omega 
= \int_{\partial c} \omega, 
\]
wobei für die vorletzte Gleichheit der Satz von Stokes für Ketten verwendet wurde.

\bigskip

\noindent {\bf Fall 1:} $c([0,1]^{k}) \cap \partial M = \emptyset$.\\


\noindent
Dann gilt $\omega_{X} = 0$ für $X \in c \circ 
I_{i,\alpha}^{k}([0,1]^{k-1})$ für $i = 1,\dots,k$ und $\alpha = 
0,1$, da $c$ Einschränkung eines Koordinatensystems ist und somit $X$ 
Häufungspunkt von Punkten $Y \in M$ mit $\omega_{Y} = 0$. Es folgt 
daher
\[ \int_{\partial c} \omega = 0 = \int_{\partial M} \omega, \]
wobei auch für das Integral auf der rechten Seite verwendet wird, daß
 $\omega$ außerhalb $c([0,1]^{k})$ verschwindet.

\bigskip

\noindent {\bf Fall 2:} $c([0,1]^{k}) \cap \partial M = c \circ 
I_{1,0}^{k}([0,1]^{k-1})$.\\

\noindent
Dann gilt
\[ \int_{\partial c} \omega = \sum_{i=1}^{k} (-1)^{i} \left(\int_{c 
\circ I_{i,0}^{k}} \omega - \int_{c \circ I_{i,1}^{k}} \omega 
\right) = -\int_{c \circ I_{1,0}^{k}} \omega. \]
Daß die übrigen Integrale verschwinden, sieht man wie in Fall 1. 
Da $c$ ein normaler $k$-Würfel ist, gibt es ein orientierungstreues 
Koordinatensystem $F: W \to {\mathbb R}^{n}$ mit $c = F \mid 
[0,1]^{k}$. Da Fall 2 vorliegt, ist $W$ eine relativ offene Teilmenge 
von $H^{k}$. Definiere
\[ W' \da  \{(x_{1},\dots,x_{k-1}) \in {\mathbb R}^{k-1}: 
(0,x_{1},\dots,x_{k-1}) \in W\} \]
und $F'(x_{1},\dots,x_{k-1}) \da  F(0,x_{1},\dots,x_{k-1})$ mit 
$(x_{1},\dots,x_{k-1}) \in W'$. Dann ist $F'$ ein Koordinatensystem 
von $\partial M$ mit $F' \mid [0,1]^{k-1} = c \circ I_{1,0}^{k}$. 
Für $Z \in W$ mit $z_{1} = 0$ gilt
\[ [DF_{Z} (E_{1}),\dots,DF_{Z}(E_{k})] = o_{F(Z)}, \]
da $F$ orientierungstreu ist. Nach Definition der induzierten 
Orientierung $o'$ von $\partial M$ gilt
\[ o'_{F(Z)} = -[DF_{Z}(E_{2}),\dots,DF_{Z}(E_{k})]. \]
Hieraus folgt, daß $F'$ orientierungsumkehrend ist. Also erhält man
\[ -\int_{c \circ I_{1,0}^{k}} \omega = \int_{\partial M} \omega. \]
In beiden Fällen ist also
\[ \int_{\partial c} \omega = \int_{\partial M} \omega. \]

\bigskip

\noindent
Sei schließlich $\omega$ eine beliebige $(k-1)$-Form der Klasse 
$C^{1}$ auf $M$. Wähle eine offene Überdeckung 
$\{U_{1},\dots,U_{m}\}$ von $M$ wie in Lemma \ref{Lemma3.7.2} mit zugehörigen 
normalen $k$-Würfeln $c_{1},\dots,c_{m}$. Sei schließlich 
$\{\varphi_{1},\dots,\varphi_{m}\}$ eine untergeordnete Zerlegung der 
Eins.\\

\noindent
Für jedes $i \in \{1,\dots,m\}$ verschwindet die Form 
$\varphi_{i}\omega$ außerhalb $c_{i}([0,1]^{k})$. Nach dem schon 
Bewiesenen gilt damit
\[ \int_{M} \, d(\varphi_{i}\omega) = \int_{\partial M} 
\varphi_{i}\omega. \]
Wegen $\sum_{i=1}^{m} \varphi_{i} = 1$ auf $M$ ist mit naheliegenden 
Rechenregeln für Differentialformen auf Mannigfaltigkeiten
\begin{eqnarray*}
\sum_{i=1}^{m}\, d(\varphi_{i}\omega) & = & 
\sum_{i=1}^{m}(d\varphi_{i} \wedge \omega + \varphi_{i}\, d\omega) 
 =  d \left(\sum_{i=1}^{m} \varphi_{i}\right) \wedge \omega + 
\left(\sum_{i=1}^{m} \varphi_{i}\right)\, d\omega 
 =  \left(\sum_{i=1}^{m} \varphi_{i}\right) \, d\omega \\
&=& d\omega.
\end{eqnarray*}
Hiermit folgert man aufgrund der Linearität des Integrals
\[
\int_{M} \, d\omega  =  \int_{M}\sum_{i=1}^{m} d(\varphi_{i} 
\omega) = \sum_{i=1}^{m} \int_{M}\, d(\varphi_{i}\omega) 
=  \sum_{i=1}^{m} \int_{\partial M} \varphi_{i}\omega = 
\int_{\partial M} \omega, 
\]
was zu zeigen war. \qed\\

\bigskip

\noindent
Als eine abschließende Anwendung zeigen wir einen Satz über die 
„Unmöglichkeit einer Retraktion“. Hieraus kann man mit 
Zusatzüberlegungen den Brouwerschen Fixpunktsatz herleiten.

\bigskip

\begin{satz}\label{Satz3.7.4} {Sei $M \subset {\mathbb R}^{n}$ eine 
kompakte zusammenhängende $n$-dimensionale Mannigfaltigkeit. Dann 
gibt es keine Abbildung $F: M \to \partial M$ der Klasse $C^{2}$, die 
den Rand punktweise festläßt.}
\end{satz}

\bigskip

\noindent {\em Beweis.} Wir nehmen indirekt an, $F$ wäre doch so 
eine Abbildung. Dann erklären wir die Abbildung
 $G(x,t) \da  x + t(F(x) - x)$ für $x \in M$ und 
$t \in [0,1]$ sowie $G = :(g_{1},\dots,g_{n})$. Die Funktionen $g_{i} : 
M \to {\mathbb R}$ sind 0-Formen auf $M$, $dg_{i}$ ist deren äußeres 
Differential. Da $M$ als offene Teilmenge von $\MdR^n$ in kanonischer Weise 
orientierbar ist, kann die $n$-Form $\omega = dg_{1} \wedge \dots 
\wedge dg_{n}$ auf $M$ integriert werden. Wir definieren also
\[ \varphi(t) \da  \int_{M} dg_{1} \wedge \dots \wedge dg_{n} = 
\int_{M} [dx^{1} + t(df_{1} - dx^{1})] \wedge \dots \wedge [dx^{n} + 
t(df_{1} - dx^{n})] \]
Differentiation nach $t$ und Anwendung des Satzes von 
Stokes ergibt
\begin{eqnarray*} 
\varphi'(t) & = & \int_{M} \sum_{i=1}^{n} dg_{1} \wedge \dots \wedge 
(df_{i} - dx^{i}) \wedge \dots \wedge dg_{n} \\
& = & \sum_{i=1}^{n} (-1)^{i-1} \int_{M} (df_{i} - dx^{i}) \wedge 
dg_{1} \wedge \dots \wedge \widehat{dg}_{i} \wedge \dots \wedge 
dg_{n} \\
& = & \sum_{i=1}^{n} (-1)^{i-1} \int_{M} d[(f_{i}-x^{i}) dg_{1} 
\wedge \dots \wedge \widehat{dg}_{i} \wedge \dots \wedge dg_{n}] \\
& = & \sum_{i=1}^{n} (-1)^{i-1} \int_{\partial M} 
\underbrace{(f_{i}-x^{i})}_{= 0} dg_{1} \wedge \dots \wedge 
\widehat{dg}_{i} \wedge \dots \wedge dg_{n} \\
& = & 0,
\end{eqnarray*}
wobei $F(x)=x$ für $x\in\partial M$ verwendet wurde. 
Nun gilt aber
\[ \varphi(0) = \int_{M} \, dx^{1} \wedge \dots \wedge dx^{n} = 
\lambda_{n}(M) > 0 \]
und
\begin{eqnarray*}
\varphi(1) & = & \int_{M} \, df_{1} \wedge \dots \wedge df_{n} = 
\int_{M} {\rm det}\, JF \, dx^{1} \wedge \dots \wedge dx^{n} \\
& = & \int_{M} {\rm det}\, JF(x) \lambda_{n}(dx) = 0,
\end{eqnarray*}
da ${\rm det}\, JF(x) = 0$ für alle $x \in M$ gelten muß. Hier wird 
verwendet, daß $F(M) \subset \partial M$ gilt, $\partial M$ eine 
$(n-1)$-dimensionale Mannigfaltigkeit ohne innere Punkte im ${\mathbb 
R}^{n}$ ist (und der Satz von der Umkehrabbildung). Da  $\varphi' = 0$ 
auf $[0,1]$ gilt, muß andererseits 
$\varphi$ konstant sein, ein Widerspruch. \qed\\

\bigskip

\noindent
Als Folgerung erhalten wir für $B \da  \{X \in {\mathbb R}^{n}: \|X\| 
\le 1\}$:

\bigskip

\begin{satz}\label{Satz3.7.5} {Jede Abbildung $F: B \to B$ der 
Klasse $C^{2}$ hat einen Fixpunkt.}
\end{satz}

\bigskip

\noindent {\em Beweis.} Hätte $F$ keinen Fixpunkt, d.h. würde $F(X) \not= 
X$ für alle $X\in B$ gelten, so könnte man eine Abbildung definieren durch
\[ f: B \to \partial B, \quad f(x) = x + \lambda(x)(x - F(x)) \]
mit $\lambda(x) \ge 0$ so, daß $f(x) \in \partial B$ gilt. Wegen
\[ \langle x + \lambda(x)(x-F(x)), x + \lambda(x)(x - F(x)) \rangle = 
1 \]
erhält man
\[ \lambda(x) = \frac{-\langle x, x - F(x)\rangle + \sqrt{\langle x, 
x - F(x) \rangle^{2} + (1-\|x\|^{2}) \|x - F(x)\|^{2}}}{\|x - 
F(x)\|^{2}}. \]
Die Abbildung $f$ ist dann eine Retraktion von $B$ auf $\partial B$ 
der Klasse $C^{2}$, im  Widerspruch zu Satz \ref{Satz3.7.4}. \qed\\

\noindent
Mit Hilfe eines Approximationsarguments (Weierstraßscher Approximationssatz) 
sieht man ein, daß es auch 
keine {\em stetige} fixpunktfreie Abbildung von $B$ auf sich geben kann. 
Damit erhält man schließlich den folgenden Satz als einfache Konsequenz.

\bigskip

\begin{satz}[Brouwerscher Fixpunktsatz]\label{Satz3.7.6}  {Ist $A 
\subset {\mathbb R}^{n}$ homöomorph zu einer $n$-dimensionalen 
abgeschlossenen Kugel, so hat jede stetige Abbildung von $A$ in sich 
einen Fixpunkt.}
\end{satz}



\end{document}
