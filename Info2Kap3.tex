\documentclass[a4paper]{scrartcl}
\newcounter{chapter}
\setcounter{chapter}{2}
\usepackage{info}
\usepackage{makeidx}
\usepackage{clrscode}
\usepackage[all]{xy}

\title{Objektorientierte Programmierung}
\author{Felix Brandt, Lars Volker}
% Wer nennenswerte Änderungen macht, schreibt sich bei \author dazu

%\setlength{\parskip}{0.5cm}
%\setlength{\parindent}{0cm}
\begin{document}


\maketitle

Ein Wert kann erst übergeben werden, wenn eine konkrete Instanz vorliegt.

\paragraph{Wie kommt man an den Verkaufspreis?}
Die Klasse sendet ein Botschaft an die konkrete Instanz.

    \begin{xy}
    		\centering
        \entrymodifiers={++<20pt>[-][F-]}
        \xymatrix {
            \txt{Klasse} \ar@/^5ex/[rr]^{\txt{Anfrage Preis}}&
            *{}&
            \txt{Instanz}\ar@/^5ex/[ll]^{\txt{Antwort 2 Mio.}} \\
        }
		\end{xy}


\paragraph{Was ist ein Objekt?}
Ein Objekt (\begriff{Object}) ist ein Gegenstand des Interesses einer Beobachtung, Untersuchung oder Messung.
Jedes Objekt hat eine eindeutige Objekt-Identität.

Der Zustand (\begriff{state}) eines Objektes wird durch seine Attribute bzw. Daten und Verbindung zu anderen
Objekten bestimmt.

Das Verhalten (\begriff{behavior}) eines Objektes wird durch eine Menge von Operationen beschrieben.

Synonyme für den Begriff des Objekts sind Instanz oder Exemplar.

\lectureof{22.06.2005}
% Lars
\section{Klassen}
\subsection{Definition} Eine \begriff{Klasse} ist eine Gemeinsamkeit von Objekten mit den selben Eigenschaften (Attributen), dem selben Verhalten (Operationen bzw. Methoden) und den selben Beziehungen zu anderen Objekten (Vererbung).
\subsection{Gleichheit und Identität von Objekten} Zwei Objekte sind \emph{gleich}, wenn sie die gleichen Attributswerte besitzen aber unterschiedliche Objektidentitäten.\\
\paragraph{Beispiel: } Zwei Firmen haben eine gemeinsame Tochterfirma.
\paragraph{Analog für Gleichheit: } Zwei Firmen besitzen jeweils eine Tochterfirma gleichen Namens. Die Objekte sind somit gleich, aber nicht identisch.
\section{Objektdiagramm}
\begin{xy}
	\centering
	\entrymodifiers={++<20pt>[-][F-]}
	\xymatrix {
		\txt{:Klasse1} \ar@{-}[d]^{\txt{Verbindung}}\\
		\txt{:Klasse2} \ar@{-}[d]^{\txt{Verbindung}}\\
		\txt{:Klasse3} \\ 
	}
\end{xy}
\paragraph{Bemerkung: } Was die Verbindung bedeutet bleibt offen.
\section{Verwendung von Klassen}
$\ra$ Schablone \\
Ein Objekt kennt seine Klasse, aber die Klasse \grqq weiß \glqq nicht welche Objekt von ihr abgeleitet werden.


\end{document} 	
