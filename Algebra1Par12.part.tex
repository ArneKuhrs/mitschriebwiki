\section{Beispiele und Konstruktionen}

\begin{enum}
    \item[(1)] Sei $M$ eine Menge. $\\M^M \defeqr \{ f:M \ra M$ Abbildung $\}$
    ist mit der Verknüpfung $\cd$ ein Monoid. $(M^M)^X = \{f: M \ra M$ bijektiv
    $\} \defeql$ Perm$(M) = S_M$. \newline
    insbesondere: $M=\{1,\dots ,n\}: S_{ \{1, \dots, n\} } = S_n$ Ist $(M,\cd)$
    ein \bla, so ist $End(M) \defeqr \{f \in M^M :f\;$ Hom.$\}$ ein Untermonoid
    von $M^M$ und \newline
    $Aut(M) \defeqr$ Perm$(M) \cap$ End$(M)$ Untergruppe von Perm$(M)$

    \item[(2a)] Sei $X$ Menge, $M$ ein(e) \bla. Dann ist $M^X = \{f:X \ra M$
    Abbildung $\}$ mit der Verknüpfung $(f\cd g)(x) = f(x)\cd g(x)$ ein(e) \bla
    
    \item[(2b)] Ist $(M,\cd)$ Halbgruppe, $(H,+)$ kommutative Halbgruppe, so
    ist Hom$(M,H) \defeqr \{f \in H^M: f\;$ Homomorphismus$\}$ eine kommutative
    Unterhalbgruppe von $H^M$. \newline\textbf{denn}: Sind $f,g: M \ra H$
    Homomorphismen, so ist $\forall x,y \in M$: \newline
    $(f+g)(x\cd y) = f(x \cd y) + g(x \cd y) = f(x) + f(y) + g(x) + g(y) = f(x)
    + g(x) + f(y) + g(y) = (f + g)(x) + (f + g)(y)$

    \item[(3)] Sei $I$ eine Indexmenge. Für jedes $i \in I$ sei $(M_i, \cd)$
    ein(e) \bla.
    \begin{enum}
        \item $\displaystyle \prod_{i \in I} M_i$ ist mit komponentenweiser
        Verknüpfung ein(e) \bla.
        
        \item Sind $M_i$ Monoide, so ist \[ \bigoplus_{i \in I} M_i \defeqr \{ 
        (x_i)_{i \in I} \in \prod_{i \in I} M_i, x_i = e_i \mbox{ ffa.} i\}\]
        ein Monoid.
    \end{enum}
\end{enum}

\begin{DefBem}
\mbox{}
\begin{enum}
\item $\prod$ heißt \emp{direktes Produkt} \newline
$\bigoplus$ heißt \emp{direkte Summe}
\item Ist $I$ endlich, so ist $\prod M_i \cong \bigoplus M_i$
\item Sei $M$ ein(e) \bla und für jedes $i \in I: g_i:M \ra
M_i$ ein Homomorphismus. Dann gibt es genau einen Homomorphismus
$\displaystyle G:M \ra \prod_{i \in I} M_i$, so daß $g_i = pr_i
\circ G$, wobei $\displaystyle pr_i: \prod_{j \in I} M_j \ra M_i$
Projektion.
\newline

%\[\begindc{\commdiag} \obj(1,1){$M$}
%                      \obj(5,1){$M_i$}
%                      \obj(3,3){$\pi M_j$}
%                      \mor{$M$}{$M_i$}{$g_i$}
%                      \mor{$M$}{$\pi M_j$}{$\exists!\;G$}[1,1]
%                      \mor{$\pi M_j$}{$M_i$}{pr$_i$}
%\enddc\]

\sbew{0.9}{Setze $G(m) \defeqr (m_j)_{j \in I}$ mit $m_j =
g_j(m)$ für $m \in M$. $G$ ist Homomorphismus. $\chk\;\\$ $G$ ist
eindeutig, da $pr_i(G(m)) = g_i(m)$ sein muss.}
\item Ist $(M,+)$ ein kommutatives Monoid, und für jedes $i \in I\; f_i:M_i
\ra M$ ein Homomorphismus, so gibt es genau einen Homomorphismus \[
F: \bigoplus_{j\in I} M_j \ra M \mbox{, so dass für jedes } i \in I: f_i = F
\circ \nu_i \mbox{, wobei } \nu_i:M_i \ra \bigoplus_{j\in I} M_j\] \[m \mapsto
(m_j)_{j \in I}\mbox{, wobei } m_j = \left\{
\begin{array}{rl}
            m & i=j \\
            e_j & \mbox{sonst}
          \end{array}\right.\]

%\[\begindc{\commdiag} \obj(1,3){$M_i$}
%                      \obj(3,3){$M$}
%                      \obj(2,1){$\oplus_{j\in I} M_j$}
%                      \mor{$M_i$}{$M$}{$f_i$}
%                      \mor{$M_i$}{$\oplus_{j\in I} M_j$}{$\nu_i$}
%                      \mor{$M$}{$\oplus_{j\in I} M_j$}{$\exists!\;F$}[1,1]
%\enddc\]

\sbew{0.9}{Setze $F((m_j)_{j \in I}) = \displaystyle \sum_{j \in I} f_j(m_j)$
\newline Brauche: $F((e,\dots, e,m_i,e,\dots,e)) = F(\nu_i(m_i)) \overset{!}{=}
f_i(m_i)$
\newline $\Rightarrow F((e,\dots,e,m_i,e,\dots,e,m_j,e,\dots,e)) = f_i(m_i) +
f_j(m_j) = F((e,\dots,e,m_i,e,\dots,e)) + F((e,\dots,e,m_j,e,\dots,e))$}

\end{enum}
\begin{enum}
\item[(4)] Sei $S$ eine Menge (''Alphabet'') $F^a(S) \defeqr
\displaystyle \bigcup^{\infty}_{n=1} S^n$ ist Halbgruppe mit Verknüpfung
''Nebeneinanderschreiben'' $\underset{\in S^n}{(x_1,\dots,x_n)}\cd
\underset{\in S^m}{(y_1,\dots,y_m)} \defeqr \underset{\in
S^{n+m}}{(x_1, \dots, x_n, y_1, \dots, y_m)}$ $F^a(S)$ heißt ''Worthalbgruppe'' über
$S$.
\newline Variante: $S^0$ = ''leeres Wort'' $\varepsilon$ oder $()$ $\widehat =$
leeres Tupel
\end{enum}
\end{DefBem}

\begin{Bem} 
Ist $(H,\cd)$ Halbgruppe, $f:S\ra H$ eine
Abbildung, so gibt es genau einen Homomorphismus $F:F^a(S) \ra H$
mit $F|S = f$
\newline \sbew{1.0}{Setze $F(x_1, \dots, x_n) \defeqr f(x_1)\cd f(x_2)
\cd \; \dots \; \cd f(x_n)$ \newline $\Rightarrow F((x_1)\cdot (x_2) \cd \dots \cd
(x_n)) = F(x_1) \cd F(x_2) \cd \dots \cd F(x_n)$}

\begin{enum} 
\item[(5)] Sei $(M,\cd)$ ein Monoid. Für $x \in M$ ist $\varphi_x: \mathbb{N}_0
\to M$, $n \mapsto x^n$ ein Homomorphismus.\newline
Ist $(G,\cd)$ Gruppe, $x \in G$, so ist $\varphi_x: \mathbb{Z} \to G$, $n \mapsto x^n$
ein Gruppenhomomorphismus.
\end{enum}
\end{Bem}


\begin{BemDef}
\label{1.9}
\mbox{}
\begin{enum} 

\item Sei $G$ Gruppe, $\langle x \rangle \defeqr Bild(\varphi_x)$ heißt die von
x erzeugte \emp{zyklische Untergruppe}

\item $| \langle x \rangle |$ heißt \emp{Ordnung} von $x$

\item $| G |$ heißt \emp{Ordnung} von $G$ (falls $| G |$ endlich)
\end{enum}


\begin{enum}
\item[(6)] Sei $G$ Gruppe, für $g \in G$ sei $\tau_g:G \ra G$, $x\mapsto g \cd 
x$ (''Linksmultiplikation'')\newline
$\tau_g(e)=g \Rightarrow$ kein Gruppenhomomorphismus, $\tau_e = id_G$
\end{enum}
\end{BemDef}

\begin{Bem}[Satz von Cayley]
Für jede Gruppe $G$ ist die Abbildung:
\[ \begin{array}{lcc}
    \tau    &   :   &   G \ra \mbox{Perm}(G) \\
            &       &   g \mapsto \tau_g
    \end{array} \]
ein injektiver Gruppenhomomorphismus. \newline\bew{}{ \item[(1)]
$\tau_g \in$ Perm$(G) : \tau_g$ ist bijektiv mit Umkehrabbildung
$\tau_{g^{-1}}$
\item[(2)] $\tau$ ist Homomorphismus: $\tau(g_1 g_2) = \tau(g_1)
\circ \tau(g_2)$, denn: $\forall x \in G: \tau(g_1 \circ g_2)(x) = (g_1 g_2)x =
g_1(g_2 x) = \tau_{g_1}(\tau_{g_2}(x)) = (\tau_{g_1}
\circ \tau_{g_2})(x)$ \item[(3)] Kern$(\tau)$ = $\{e\}$, denn ist
$\tau(g) = id_g$, so ist $\forall x \in G: \tau_g(x) = gx = x$, also $g = e$}
\end{Bem}

\begin{DefBem}[7]
Sei $G$ Gruppe, $g \in G$
\begin{enum}
\item Die Abbildung $c_g:G \ra G, x \mapsto gxg^{-1}$ ist ein
\emp{Automorphismus}, sie heißt \emp{Konjugation} mit $g$.
\newline \sbew{0.9}{$c_g$ ist Homomorphismus: $c_g(x_1 x_2) = g(x_1
x_2)g^{-1} \\ c_g(x_1) c_g(x_2) = (g x_1 g^{-1})(g x_1 g^{-1}) = ...
= \chk \\
c_g$ ist bijektiv: Die Umkehrabbildung ist $c_{g^{-1}}$ }
\item Die Abbildung $c:G \ra$ Aut$(G), g \mapsto c_g$ ist ein
Gruppenhomomorphismus. \newline \sbew{0.9}{$\forall x \in G: c(g_1 g_2)(x) = (g_1
g_2)x(g_1 g_2)^{-1} = g_1(g_2 x g_2^{-1})g_1^{-1} = (c(g_1) \circ
c(g_2))(x)$ }

\item $Z(G)\defeqr$ Kern$(c)$ heißt \emp{Zentrum} von $G$. Es ist $Z(G) =
\{ g \in G:gx=xg \;\forall x \in G\}$

\item Die Elemente von Bild$(c) \defeql$ Aut$_i(G)$ heißen \emp{innere
Automorphismen} von $G$.

\item Eine Untergruppe $N \subseteq G$ heißt \emp{Normalteiler} in
$G$, wenn $\forall g \in G: c_g(N) \subseteq N$.
Äquivalent: $\forall g \in G, x \in N: g x g^{-1} \in N$

\item Ist $f:G \ra G'$ Gruppenhomomorphismus, so ist Kern$(f)$
Normalteiler in G. \newline \sbew{0.9}{\newline Sei $x \in$
Kern$(f), g \in G$. Dann ist $f(g x g^{-1}) = f(g)
\underset{e'}{\underbrace{f(x)}} f(g)^{-1} = e'$.}

\item Aut$_i(G)$ ist Normalteiler in Aut$(G)$
\newline
\sbew{0.9}{\newline Sei $\varphi \in \mbox{Aut}(G), g\in G :
\mbox{z.z.: } \varphi \cd c_g \cd \varphi^{-1} \in \mbox{Aut}_i(g).
\\ \mbox{Es ist } (\varphi \cd c_g \cd \varphi^{-1})(x) =
\varphi(c_g(\varphi^{-1}(x))) = \varphi(g \cd \varphi^{-1}(x)\cd
g^{-1}) = \varphi(g) \cd \varphi(\varphi^{-1}(x)) \cd
\varphi(g^{-1}) = \varphi(g) \cd x \cd \varphi(g)^{-1} =
c_{\varphi(g)}(x) \Rightarrow \varphi \circ c_g \circ \varphi^{-1} =
c_{\varphi(g)} \in \mbox{Aut}_i(G)$}
\end{enum}
\end{DefBem}

\begin{DefBem}[8]
\label{1.12}
Sei $G$ Gruppe, $H \subseteq
G$ Untergruppe. \begin{enum}
\item Für $g \in G$ heißt $g \cd H = \{g\cd h : g \in H\} =
\tau_g(H)$ \emp{Linksnebenklasse} von $g$ bzgl. $H$ und $H \cd g=
\{h \cd g : h \in H \}$ \emp{Rechtsnebenklasse}


\item Für $g_1$, $g_2 \in G$ gilt: $g_1 H \cap g_2 H \neq
\emptyset \lra g_1 H = g_2 H$ \newline \sbew{0.9}{Sei $ y = g_1
h_1 = g_2 h_2 \in g_1 H \cap g_2 H$ und $h1, h2 \in H \Rightarrow g_1 = g_2
h_2 h_1^{-1} \in g_2 H \Rightarrow g_1 H \subseteq g_2 H$, die Umkehrung
folgt analog.}
\item $H$ ist genau dann Normalteiler, wenn $\forall g\in G: g\cd H = H \cd g
$ \newline \sbew{0.9}{$gH = Hg \lra H =
gHg^{-1}$}

\item Alle Nebenklassen von $G$ bzgl. $H$ sind gleichmächtig.
\newline
\sbew{0.9}{$\tau_g: \underset{e\cd H}{\underbrace{H}} \ra g\cd H, h
\mapsto g\cd h$ ist bijektiv.}
\item Die Anzahl der Linksnebenklassen bzgl. $H$ ist gleich der
Anzahl der Rechtsnebenklassen. Sie heißt \emp{Index} $[G:H]$ von $H$
in $G$. \newline \sbew{0.9}{Die Zuordnung \[\begin{array}{lcl}
    \{\mbox{Linksnebenklasse}\} & \ra & \{\mbox{Rechtsnebenklasse}\} \\
    g\cd H                      & \mapsto & H \cd g^{-1}
    \end{array}\]
ist \emp{wohldefiniert} und bijektiv.
\newline
\textbf{Wohldefiniertheit:} ist $g_1 H = g_2 H$, also $g_2 = g_1 h$
für ein $h \in H \Rightarrow Hg_2^{-1} = H(g_1 h)^{-1} = H\cd
h^{-1}g_1^{-1} = Hg_1^{-1}$}


\item
\emp{Satz von Lagrange:} Ist $G$ endlich, so ist
\[[G:H] = \frac{|G|}{|H|}\] \sbew{0.9}{$G$ ist disjunkte Vereinigung
der $[G:H]$ Linksnebenklassen bzgl. $H$. Diese haben alle $|H|$
Elemente.} \label{\thesubsection \theenumiii}
\end{enum}
\end{DefBem}