% Dies ist die Datei makros.tex in der Fassung vom 4. Juli 1997,
% die alle in der Aufgabensammlung enthaltenen Makros und Umgebungen
% enthaelt.
%
% Die Namen aller nur fuer die Aufgabensammlung verwendeten Makros
% werden gross geschrieben.
%
% In den TeX-Blaettern werden die Aufgaben mit dem Makro ABINPUT
% eingelesen, dem der Dateiname mit und ohne geschweifte Klammern
% folgen kann. Der Titel eines Blattes wird mit der Umgebung
% ABTITEL gekennzeichnet, damit es vom Programm zur Auswahl der
% Aufgaben gefunden wird.
%
% Die Quelltexte der Aufgaben duerfen keine INPUT-Anweisungen,
% etwa zum Einlesen von Graphiken, enthalten, da die zugehoerigen
% Dateien vom Programm ABSELECT, das die Auswahl der Aufgaben
% erledigt, nicht gefunden werden.
%
%*********************************************************************************************
%
% Um bei der Verwendung von KOMA-Script (srcbook) bei enumerate (2. Ebene)
% die Doppelklammer im Label zu haben [(a) statt a)] (RR, 2001)
%
\renewcommand*\labelenumii{(\theenumii)}
%
%*********************************************************************************************
%
%% Kopf- und Fusszeilen auf den Blaettern
% Die Makros KOPFTEXT und FUSSTEXT legen Kopf- und Fusszeile jedes
% TeX-Blattes fest. Durch die Leerdefinitionen von KopfU, KopfG usw.
% zu Beginn des Uebersetzungsvorganges ist eine einheitliche Definition
% von Kopf- und Fusszeilen im gesamten zum Ausdruck benoetigten
% TeX-Dokument moeglich.
%---------------------------------------------------------------------------------------------
\newcommand{\KopfU}{}
\newcommand{\KopfG}{}
\newcommand{\FussU}{}
\newcommand{\FussG}{}
\newcommand{\KOPFTEXT}[1]{%
\renewcommand{\KopfU}{\parbox{\textwidth}%
{\sf\small #1 \hfill\thepage\\%
\rule[2ex]{\textwidth}{0.2mm}}}
\renewcommand{\KopfG}{\parbox{\textwidth}%
{\sf\small\thepage\hfill #1\\%
\rule[2ex]{\textwidth}{0.2mm}}}}
% Fusszeile
\newcommand{\FUSSTEXT}[1]{%
\renewcommand{\FussU}{\parbox{\textwidth}%
{\sf\small\rule[-0.9ex]{\textwidth}{0.2mm}\\%
#1 \\%
\rule[2ex]{\textwidth}{0.2mm}}}
\renewcommand{\FussG}{\parbox{\textwidth}%
{\sf\small\rule[-0.9ex]{\textwidth}{0.2mm}\\%
\hfill #1 \\%
\rule[2ex]{\textwidth}{0.2mm}}}
}
%*********************************************************************************************
%
%% ABTITEL
% Diese Umgebung kennzeichnet den Titel eines TeX-Blattes und muss auf
% jedem Blatt vorhanden sein, da das Programm zur Aufgabenauswahl nach
% ABTITEL sucht.
% Der Schalter PRIVATE erlaubt es, beim Gesamtausdruck fuer die
% TeX-Blaetter numerierte Ueberschriften zu erzeugen.
%---------------------------------------------------------------------------------------------
\newif\ifPRIVATE
\PRIVATEfalse
\newenvironment{ABTITEL}{\ifPRIVATE\begin{center}%
\setbox0=\vbox\bgroup\else\begin{center}\bf \large\fi}%
{\ifPRIVATE\egroup\fi\end{center}}%
%*********************************************************************************************
%
%% ABINPUT, FILE
% Das Makro ABINPUT ersetzt die TeX-eigene INPUT-Anweisung und dient
% zum Einlesen des Quelltextes einer Aufgabe. Das Makro uebernimmt ausserdem
% die Funktion des item-Befehls, zeigt also auch die Aufgabennummer auf dem
% jeweiligen Blatt an. Dies hat zur Folge, dass die AUFGABEN-Umgebung ein
% enumerate enthalten muss.
%
% Der Schalter FILE bestimmt, ob die Dateinamen der Aufgaben
% ausgegeben werden. Der Dateiname ist ggf. Teil der Itemmarkierung.
%---------------------------------------------------------------------------------------------
\newif\ifFILE
\def\ABINPUT #1 {%
\ifFILE{\item[\hspace{-2em}{\footnotesize\sf\fbox{#1}}\hspace{2em}%
\addtocounter{enumi}{1}\labelenumi]\input{#1}}%
\else{\item\input{#1}\ }\fi}
%*********************************************************************************************
%
%% AUFGABEN
% Eine Umgebung fuer Aufgaben, zunaechst im Wesentlichen eine Verkleidung
% fuer enumerate, kann aber spaeter weitere Stilmerkmale aufnehmen.
% Gibt zur Zeit den Namen des TeX-Blattes rechts unten auf der letzten
% Seite aus, falls die entsprechenden Anweisungen nicht auskommentiert
% sind.
%---------------------------------------------------------------------------------------------
\newenvironment{AUFGABEN}{\begin{enumerate}}
                         {\end{enumerate}}
                          %\vfill
                          %\hfill\tiny\jobname}
%*********************************************************************************************
%
%% ANG, ANGABE  (RR 2001)
% ANGABE ist eine Boolsche Variable, um zu entscheiden, ob Angaben
% gedruckt werden. ANG ist eine Umgebung fuer Angaben,
% im Wesentlichen, um diese auf Wunsch ein- und ausblenden zu koennen.
%---------------------------------------------------------------------------------------------
\newif\ifANGABE
\newcommand{\ANG}[1]{
    \ifANGABE{#1}\else{}\fi
  }
%*********************************************************************************************
%
%% LSG, LOESUNG
% LOESUNG ist eine Boolsche Variable, um zu entscheiden, ob Loesungen
% gedruckt werden. LSG ist eine Umgebung fuer Loesungen,
% im Wesentlichen, um diese auf Wunsch ein- und ausblenden zu koennen.
% Gelegentlich kann der Loesungstext auch methodische und didaktische
% Hinweise enthalten.
%---------------------------------------------------------------------------------------------
%%\newif\ifLOESUNG
%%\newdimen\mydim
%%\newenvironment{LSG}{\ifLOESUNG{{\it L"osung:}\quad}\else{}\fi%
%%\mydim=\hsize%
%%\advance\mydim by -30mm%
%%\setbox0=\vbox\bgroup\hsize=\mydim\begin{minipage}[t]{0.98\hsize}
%%\begin{small}}%
%%{\end{small}\end{minipage}\egroup\ifLOESUNG{\box0}\else{}\fi}
%
% Diese neue Definition des LSG-Makros erlaubt Seitenumbrueche in der
% Loesung und Loesungen, die laenger als eine Seite sind. 
% Aus typographischen Gruenden wurde der Schriftstil \small
% statt \it gewaehlt. (RR 1998, 2001)
%
\newif\ifLOESUNG
\newcommand{\LSG}[1]%
  { \ifANGABE
      { \ifLOESUNG\begingroup
        \begin{small}
        \begin{list}{\it L�sung:\hfill}%
                { \setlength{\itemindent}{0mm}
                  \setlength{\labelwidth}{14mm}
                  \setlength{\leftmargin}{0mm}
                }
        \item #1
        \end{list}\end{small}\endgroup\else{}\fi
      }
    \else
      { \ifLOESUNG{
        \begin{small}%
        #1
        \end{small}%
        }\else{}\fi%
      }\fi
  }
%%
%
%*********************************************************************************************
%
%% BILD
% Das folgende Makro dient der Einbindung von PCX-Grafiken.
% Diese muessen alle in einem Verzeichnis direkt unterhalb des
% jeweiligen Verzeichnisses fuer die Jahrgangsstufe
% liegen.
% Auch bei den Grafiken muss die Trennung nach Jahrgangsstufen
% beibehalten werden, da jede Jahrgangsstufe gesondert per FTP
% aktualisiert werden koennen muss.
% Da PCX-Bilder recht umfangreich sind und zudem nicht vom WWW-Server
% nicht ohne weiteres uebertragen werden koennen, muss ihre Verwendung
% auf das unbedingt notwendige Mass beschraenkt bleiben.
%---------------------------------------------------------------------------------------------
\newcommand{\BILD}[2]{%
\special{em: graph c:/ab/#1/grf/#2}}
%*********************************************************************************************
% 
%% Einbindung von EPS-Graphiken (RR 1999)
%
\newcommand{\EPSBASIS}[3]{
    \ifthenelse{\equal{#2}{}}
         {\ifthenelse{\equal{#3}{}}
                {}
                {\includegraphics[height=#3]{#1}}
         }
         {\ifthenelse{\equal{#3}{}}
                {\includegraphics[width=#2]{#1}}
                {\includegraphics[width=#2,height=#3]{#1}}
         }
  } 
%
%
%
\newcommand{\EPS}[4]{ \raisebox{2.5mm}{\parbox[t]{#2}{
                          \vspace{0pt}
                          #4
                          \EPSBASIS{#1}{#2}{#3}
                          }}
                        }
%
\newcommand{\EPSB}[4]{ \setlength{\fboxsep}{0pt}
                       \raisebox{2.5mm}{\parbox[t]{#2}{
                          \vspace{0pt}
                          #4
                          \fbox{\EPSBASIS{#1}{#2}{#3}}
                        }}
                       \setlength{\fboxsep}{3pt}
                      }
%
\newcommand{\EPSR}[6]{\parbox[t]{#1}{%\vspace{0pt}
                          #6}\hfill
                          \raisebox{2.5mm}{\parbox[t]{#3}{
                          \vspace{0pt}
                          #5
                          \EPSBASIS{#2}{#3}{#4}
                          }}
                        }
%
\newcommand{\EPSC}[4]{\begin{center}
                            \raisebox{2.5mm}{\parbox[t]{#2}{
                            \vspace{0pt}
                            #4
                            \EPSBASIS{#1}{#2}{#3}
                            }}
                            \end{center}
                           }
%
\newcommand{\EPSRB}[6]{\parbox[t]{#1}{%\vspace{0pt}
                          #6}\hfill
                          \raisebox{2.5mm}{\parbox[t]{#3}{
                          \setlength{\fboxsep}{0pt}
                          \vspace{0pt}
                          #5
                          \fbox{\EPSBASIS{#2}{#3}{#4}}
                          \setlength{\fboxsep}{3pt}
                          }}
                        }
%
\newcommand{\EPSCB}[4]{\begin{center}
                            \raisebox{2.5mm}{\parbox[t]{#2}{
                            \setlength{\fboxsep}{0pt}
                            \vspace{0pt}
                            #4
                            \fbox{\EPSBASIS{#1}{#2}{#3}}
                            \setlength{\fboxsep}{3pt}
                            }}
                            \end{center}
                           }
%
\newcommand{\PSF}[2]{\psfrag{#1}{\scriptsize #2}}
\newcommand{\PSFM}[2]{\psfrag{#1}[][]{\scriptsize #2}}
\newcommand{\PSFROT}[3]{\psfrag{#2}[c][c][1][#1]{\scriptsize #3}}
%
%% AUFZAEHLUNG, TEILAUFGABE, TAG
% Automatisches Durchzaehlen nebeneinander angeordneter Teilaufgaben
% Beispiel \begin{AUFZAEHLUNG}{l}
%          \TEILAUFGABE{......}
%          \end{AUFZAEHLUNG}
%---------------------------------------------------------------------------------------------
\newcounter{TAG}
\newenvironment{AUFZAEHLUNG}[1]{\setcounter{TAG}{1}%
\begin{tabular}{#1}}{\end{tabular}\hspace{-2em}}
%
\newcommand{\ZAEHLE}{\alph{TAG})}
\newcommand{\TEILAUFGABE}[1]{\ZAEHLE\hspace{1em}#1\hspace{1em}%
\addtocounter{TAG}{1}}
%*********************************************************************************************
%
%% PUNKTE
% Das folgende Makro dient dazu, am Rand einer Schulaufgabe die
% Bewertungseinheiten bei jeder Teilaufgabe zu notieren.
%---------------------------------------------------------------------------------------------
\newcommand{\PUNKTE}[1]{\mbox{}\marginpar{\hspace*{1em}\fbox{#1}}}
%*********************************************************************************************
%
%% TIMMS
% Das folgende Makro dient dazu, am Rand der Aufgaben einen Hinweis
% auf die Art der Aufgabe ("neue Aufgabenkultur") zu plazieren.
%---------------------------------------------------------------------------------------------
\newcommand{\TIMMS}[1]{\mbox{}\marginpar{\hspace*{3em}\fbox{#1}}}
%*********************************************************************************************
%
%% EGY
% Aufgaben f"ur das Europ"aische Gymnasium 
% veraltet
%---------------------------------------------------------------------------------------------
%\newcommand{\EGY}{\mbox{}\marginpar[]{\fbox{\small{EGY}}}}
\newcommand{\EGY}{}
%*********************************************************************************************
%
%% D, T, SC, SSC
% Einschalten von Displaystil bzw. Textstil beim Mathematiksatz
%---------------------------------------------------------------------------------------------
\newcommand{\D}{\displaystyle}
\newcommand{\T}{\textstyle}
\newcommand{\SC}{\scriptstyle}
\newcommand{\SSC}{\scriptscriptstyle}
%*********************************************************************************************
%
%% GF
% Anfuehrungszeichen fuer Zitate
%---------------------------------------------------------------------------------------------
\newcommand{\GF}[1]{\glqq{}#1\grqq{}}
%*********************************************************************************************
%
%% VPFEIL
% Zeichen fuer Vektorpfeil, nur im Mathmode verwenden
%---------------------------------------------------------------------------------------------
%\newcommand{\VPFEIL}[1]{\stackrel{\D\longrightarrow}{#1}}
\newcommand{\VPFEIL}[1]{\overrightarrow{#1}}
\newcommand{\VPFEILRM}[1]{\overrightarrow{\mathrm{#1}}}
%*********************************************************************************************
%
%% RVEKTOR, EVEKTOR
% Raeumliche und ebene Spaltenvektoren
\newcommand{\RVEKTOR}[4]{\left(\begin{array}{#1}\negthickspace#2\negthickspace\\\negthickspace#3\negthickspace\\\negthickspace#4\negthickspace\end{array}\right)}
\newcommand{\EVEKTOR}[3]{\left(\begin{array}{#1}\negthickspace#2\negthickspace\\\negthickspace#3\negthickspace\end{array}\right)}
%*********************************************************************************************
%%
%*********************************************************************************************
%
%% EPUNKT, RPUNKT
% Ebene und raeumliche Punktkoordinaten
\newcommand{\EPUNKT}[3]{
  {\rm #1}\left(\,#2\;\vline\;#3\,\right)}
\newcommand{\RPUNKT}[4]{
  {\rm #1}\left(\,#2\;\vline\;#3\;\vline\;#4\,\right)}
%
%*********************************************************************************************
%
%% POLAR
% Polarform komplexer Zahlen in der Form (r|phi)_p
% Verwendung: \POLAR{r}{phi}
\newcommand{\POLAR}[2]{
  \left(\,#1\;\vline\;#2\,\right)_{\mathrm{p}}}
%
%*********************************************************************************************
%
%% POLARE
% Polarform komplexer Zahlen in der Form r E(phi)
% Verwendung: \POLARE{r}{phi}
\newcommand{\POLARE}[2]{
  #1\cdot\mathrm{E}\left(#2\right)}
%
%*********************************************************************************************
%
%% KREIS
% Kreis in der Form k(M;r) im Mathemodus
% Verwendung: \KREIS{M}{r}
\newcommand{\KREIS}[2]{
  \text{k}(\text{#1;}#2)}
\newcommand{\KREISi}[2]{
  \text{k}_{\text{i}}(\text{#1;}#2)}
\newcommand{\KREISa}[2]{
  \text{k}_{\text{a}}(\text{#1;}#2)}
%
%*********************************************************************************************
%
%% DANN (daraus folgt), GENAUDANN
% Das logische Zeichen "daraus folgt" und "genau dann, wenn"
\newcommand{\DANN}{\Longrightarrow}
\newcommand{\GENAUDANN}{\Longleftrightarrow}
%
%*********************************************************************************************
%
%% OHNE (Mengenzeichen)
\newcommand{\OHNE}{\mathrel{\setminus}}
%
%*********************************************************************************************
%
%% SATZB{Zeile1\\Zeile2\\...}
% Ein gerahmter und zentrierter Satz, wobei sich die Rahmenbreite der
% Textbreite anpasst. SATZB{} stellt eine mehrzeilige math-Umgebung
% bereit, die Zeilen sind zentriert. 
% Normaler Text wird mit \text{} eingegeben.
\newcommand{\SATZB}[1]{\begin{displaymath}\boxed{\begin{gathered}
                       #1\end{gathered}}\end{displaymath}}
%
%*********************************************************************************************
%
%% RFRAC{z}{n}
% Bruch in Roman (fuer Einheiten)
\newcommand{\RFRAC}[2]{\frac{\rm #1}{\rm #2}}
%
%*********************************************************************************************
%
%% ggT, kgV
% im mathem. Modus in Roman gesetzt
\newcommand{\ggT}{\mathrm{ggT}}
\newcommand{\kgV}{\mathrm{kgV}}
%
%*********************************************************************************************
%
%% \RMd
% im mathem. Modus in Roman gesetztes d (Differentiale)
\newcommand{\RMd}{\mathrm{d}}
%
%*********************************************************************************************
%
%% EURO und MEUR (EURO im Mathemodus), Ct fuer Cent
\DeclareFontFamily{OT1}{mvs}{}
\DeclareFontShape{OT1}{mvs}{m}{n}{<-> fmvr8x}{}
\def\mvs{\usefont{OT1}{mvs}{m}{n}}
\def\mvchr{\mvs\char}
\def\textmvs#1{{\mvs #1}}
\def\EURhv{{\mvchr99}}\def\EURcr{{\mvchr100}}\def\EURtm{{\mvchr101}}
\def\EUR{{\mvchr164}}
\newcommand{\MEUR}{\mbox{\EUR}}
\newcommand{\Ct}{\ensuremath{\mbox{Ct}}}
%
%*********************************************************************************************
%
%% Mmu
%   aufrechtes mu im Mathemodus
%\newcommand{\Mmu}{\mbox{\textmu}}
\newcommand{\Mmu}{\mu}
%
%*********************************************************************************************
%
%% GC
%  Grad Celsius (im Mathemodus)
\newcommand{\GC}{\,^{\circ}\mathrm{C}}
%
%*********************************************************************************************
%

%
% Neben ein Befehlswort kommt eine \fbox bis Zeilenende
% Beispiel: \FBEFEHL{Beweise:\quad}{Sind ...}
%
\newlength{\LBefehl}
\newlength{\BBefehl}
\newcommand{\FBEFEHL}[2]{\settowidth{\LBefehl}{#1}\setlength{\BBefehl}{\linewidth}%
\addtolength{\BBefehl}{-\LBefehl}%\addtolength{\BBefehl}{-7mm}%
#1\fbox{\parbox[t]{\BBefehl}{#2}}\\}  
% das Gleiche ohne \fbox
\newcommand{\BEFEHL}[2]{\settowidth{\LBefehl}{#1}\setlength{\BBefehl}{\linewidth}%
\addtolength{\BBefehl}{-\LBefehl}%\addtolength{\BBefehl}{-2mm}%
#1\parbox[t]{\BBefehl}{#2}\\} 
%
%*********************************************************************************************
%
\newcommand{\relsqrt}{\sqrt{1-\beta^{2}}}
\newcommand{\relgamma}{\frac{1}{\sqrt{1-\beta^{2}}}}
%
%*********************************************************************************************
%
% Zeichen in einem Kreis
%\newcommand{\IMKREIS}[1]{ \unitlength 1mm 
%                          \begin{picture}(3,3)
%                          \put(1,1){\circle{4}}
%                          \put(1,1){\makebox(0,0){#1}}
%                          \end{picture}
%}
\def\IMKREIS#1{ ${%
	\psset{unit=1pt}
	\setbox0=\hbox{#1}\relax
	\dimen0=\the\wd0\dimen1=\the\ht0
	\divide\dimen0 by 2\divide\dimen1 by 2
	\ifdim\dimen0>\dimen1\dimen2=\dimen0\else\dimen2=\dimen1\fi
	\advance\dimen2 by +5pt
	\rput(\dimen0, \dimen1){\pscircle{\dimen2}}
	\box0%
}$ }
%*********************************************************************************************
%
% plus - nicht minus, nicht plus - minus
\newcommand{\PNM}{\raisebox{-1mm}{$\SC\,\stackrel{+}{(-)}\,$}}
\newcommand{\NPM}{\raisebox{-1mm}{$\SC\,\stackrel{(+)}{-}\,$}}

%
%% \BOX{Breite}{Hoehe}{Zeichen}
%  Zeichen in Box einer festen Breite und Hoehe
\newcommand{\BOX}[3]{\framebox[#1]{\rule[-1.5mm]{0mm}{#2}#3}}
%
%*********************************************************************************************
%% NN, BB, ZZ, QQ, RR, CC, GG, LL, DD, WW
% Makros zur Bezeichnung von Zahlenmengen
%---------------------------------------------------------------------------------------------
\newcommand{\NN}{\mathds{N}}
\newcommand{\BB}{\mathds{B}}
\newcommand{\CC}{\mathds{C}}
\newcommand{\ZZ}{\mathds{Z}}
\newcommand{\QQ}{\mathds{Q}}
\newcommand{\PP}{\mathds{P}}
\newcommand{\RR}{\mathds{R}}
\newcommand{\GG}{G}
\newcommand{\DD}{D}
\newcommand{\LL}{L}
\newcommand{\WW}{W}
%\def\NN{{\sf I\!N}} % natuerliche Zahlen
%
%\def\BB{{\sf I\!B}} % Brueche
%
%\def\ZZ{% ganze Zahlen
%{\mathchoice {\hbox{$\sf\textstyle Z\kern-0.4em Z$}}
%{\hbox{$\sf\textstyle Z\kern-0.4em Z$}}
%{\hbox{$\sf\scriptstyle Z\kern-0.3em Z$}}
%{\hbox{$\sf\scriptscriptstyle Z\kern-0.2em Z$}}}}
%
%\def\QQ{% rationale Zahlen
%{\mathchoice {\setbox0=\hbox{$\displaystyle\sf Q$}\hbox{\raise
%0.15\ht0\hbox to0pt{\kern0.3\wd0\vrule height0.8\ht0\hss}\box0}}
%{\setbox0=\hbox{$\textstyle\sf Q$}\hbox{\raise
%0.15\ht0\hbox to0pt{\kern0.3\wd0\vrule height0.8\ht0\hss}\box0}}
%{\setbox0=\hbox{$\scriptstyle\sf Q$}\hbox{\raise
%0.15\ht0\hbox to0pt{\kern0.3\wd0\vrule height0.7\ht0\hss}\box0}}
%{\setbox0=\hbox{$\scriptscriptstyle\sf Q$}\hbox{\raise
%0.15\ht0\hbox to0pt{\kern0.3\wd0\vrule height0.7\ht0\hss}\box0}}}}
%
%\def\RR{{\sf I\!R}} % reelle Zahlen
%
%\def\CC{% komplexe Zahlen
%{\mathchoice {\setbox0=\hbox{$\displaystyle\sf C$}\hbox{\hbox
%to0pt{\kern0.4\wd0\vrule height0.9\ht0\hss}\box0}}
%{\setbox0=\hbox{$\textstyle\sf C$}\hbox{\hbox
%to0pt{\kern0.4\wd0\vrule height0.9\ht0\hss}\box0}}
%{\setbox0=\hbox{$\scriptstyle\sf C$}\hbox{\hbox
%to0pt{\kern0.4\wd0\vrule height0.9\ht0\hss}\box0}}
%{\setbox0=\hbox{$\scriptscriptstyle\sf C$}\hbox{\hbox
%to0pt{\kern0.4\wd0\vrule height0.9\ht0\hss}\box0}}}}
%
%\def\GG{% Grundmenge
%{\mathchoice {\setbox0=\hbox{$\displaystyle\sf G$}\hbox{\raise
%0.03\ht0\hbox to0pt{\kern0.4\wd0\vrule height0.9\ht0\hss}\box0}}
%{\setbox0=\hbox{$\textstyle\sf G$}\hbox{\raise
%0.03\ht0\hbox to0pt{\kern0.4\wd0\vrule height0.9\ht0\hss}\box0}}
%{\setbox0=\hbox{$\scriptstyle\sf G$}\hbox{\raise
%0.03\ht0\hbox to0pt{\kern0.4\wd0\vrule height0.87\ht0\hss}\box0}}
%{\setbox0=\hbox{$\scriptscriptstyle\sf G$}\hbox{\raise
%0.03\ht0\hbox to0pt{\kern0.4\wd0\vrule height0.87\ht0\hss}\box0}}}}
%
%\def\LL{{\sf I\!L}}% Loesungsmenge
%
%\def\DD{{\sf I\!D}}% Definitionsmenge, ich schlage ein serifenloses D vor
%
%\def\WW{
%{\sf W\hspace{-0.825em}W}}% Wertemenge, ich schlage ein serifenloses W vor
%{\mathchoice {\displaystyle\sf W\hspace{-0.825em}W}
%{\textstyle\sf W\hspace{-0.825em}W}
%{\scriptstyle\sf W\hspace{-0.7em}W}
%{\scriptscriptstyle\sf W\hspace{-0.7em}W}}}
%
% Die Zeichen fuer "Winkel" und "entspricht" werden umdefiniert
%
\renewcommand{\angle}{<\hspace{-0.6em})\hspace{0.3em}}
\renewcommand{\equiv}{\mathrel{\widehat{=}}}
%*********************************************************************************************
%*********************************************************************************************
% Realschulmakros
\renewcommand{\leq}{\leqq}
\renewcommand{\geq}{\geqq}
\renewcommand{\subseteq}{\subseteqq}
\renewcommand{\supseteq}{\supseteqq}
\newcommand{\LQ}{\leqq}
\newcommand{\GQ}{\geqq}
%*********************************************************************************************
% Die nachfolgenden Makros wurden bisher (4. Juli 1996) nicht verwendet.
% Sie verbleiben in der Makrosammlung, um in Zukunft die Erstellung
% von PICTEX-Graphiken zu erleichtern.
%*********************************************************************************************
%*********************************************************************************************
%% XPOS, YPOS, XNEU, YNEU, STRECKE, SIZE, DRAW, MOVE, DRAWTO, MOVETO,
%% POLARDRAW, POLARMOVE, BOGEN, BEZEICHNE, INIT
%
% Die folgenden Makros dienen der einfacheren Erstellung von Zeichnungen
% in PICTEX. Sie werden zum Teil auch von einem speziellen Logo-Programm
% verwendet, mit dem sich Igelgrafiken in PICTEX-Bilder umsetzen lassen.
% Naehere Informationen dazu finden sich in der Beschreibung dieses Logo-
% Programms mit dem Namen TEXUTILS, das sich im Unterverzeichnis
% UTILTIES des Verzeichnisses AB der Aufgabenbank befindet.
%---------------------------------------------------------------------------------------------
%
\newlength\XPOS% aktuelle x-Koordinate des Zeichenstiftes
\newlength\YPOS% aktuelle y-Koordinate ...
\newlength\XNEU% x-Koordinate nach Ausfuehrung der jeweiligen Bewegung
\newlength\YNEU% y-Koordinate nach ...
\newlength\STRECKE% Laenge der Zeichenstrecke bei POLAR...
\newlength\SIZE% legt die Groesse der jeweiligen Zeichnung fest
%
\newcommand{\DRAW}[2]{% Zeichnen relativ zur alten Position
\advance\XNEU by #1%
\advance\YNEU by #2%
\plot {\XPOS} {\YPOS} {\XNEU} {\YNEU} /
\XPOS\XNEU\YPOS\YNEU}
%
\newcommand{\MOVE}[2]{% Bewegen relativ zur alten Position
\advance\XNEU by #1%
\advance\YNEU by #2%
\XPOS\XNEU\YPOS\YNEU}
%
\newcommand{\DRAWTO}[2]{% Zeichnen von alter auf neue -absolute- Position
\XNEU#1%
\YNEU#2%
\plot {\XPOS} {\YPOS} {\XNEU} {\YNEU} /
\XPOS\XNEU\YPOS\YNEU}
%
\newcommand{\MOVETO}[2]{% Bewegen von alter auf neue -absolute- Position
\XNEU#1%
\YNEU#2%
\XPOS\XNEU\YPOS\YNEU}
%
\newcommand{\POLARDRAW}[3]{% Zeichnet relativ zur alten Position
% cos und sin des Winkels zur Nordrichtung und die Streckenlaenge
% sind anzugeben
\STRECKE#3%
\advance\XNEU by #1\STRECKE%
\advance\YNEU by #2\STRECKE%
\plot {\XPOS} {\YPOS} {\XNEU} {\YNEU} /
\XPOS\XNEU\YPOS\YNEU}
%
\newcommand{\POLARMOVE}[3]{% Bewegt den Zeichenstift relativ zur
% alten Position.
% cos und sin des Winkels zur Nordrichtung und die Streckenlaenge
% sind anzugeben
\STRECKE#3%
\advance\XNEU by #1\STRECKE%
\advance\YNEU by #2\STRECKE%
\XPOS\XNEU\YPOS\YNEU}
%
\newcommand{\BOGEN}[5]{% Zeichnen eines Bogens, Winkel, Ausgangspunkt
% und Zentrum sind anzugeben
\circulararc #1 degrees from #2 #3 center at #4 #5 }
%
\newcommand{\BEZEICHNE}[3]{% Beschriften einer Zeichnung, Text und
% Versatz sind anzugeben
\put {$#1$}[tl] <#2,#3> at {\XPOS} {\YPOS} }
%
\newcommand{\INIT}{% Initialisieren der Grafikroutinen
\XPOS0pt\YPOS0pt\XNEU0pt\YNEU0pt\SIZE0.3mm}
%*********************************************************************************************
%*********************************************************************************************
