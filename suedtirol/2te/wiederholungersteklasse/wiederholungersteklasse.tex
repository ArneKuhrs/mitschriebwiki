\chapter{�bungen zur Wiederholung 1. Klasse}

\ba
\bn


\item Berechne:

\bt{ll} a) $\D{\bigl(\frac{xy^4 z^5}{(xy^2)^{-1}z}\bigr)^2= }$
\qquad \qquad & b) $(a^2 b^3 c^4)^3 \cdot (a^{-4} b^{-3}
c^{-2})^3=$ \et


\item Berechne:

\bt{lll} a) $(x-5)^2= $ &  b) $(7a+2b)(-a+3b)-(2a-2b)^2=$ \\
          c) $\D{(\frac{1}{7}c-\frac{3}{4}b)^2=}$ &  d) $\D{(4x+\frac{1}{2}y)^2-5(4xy+y)(4xy-y)=}$\\
          e) $\D{(\frac{5}{6}u^2 v^3+\frac{4}{7}z^6)(\frac{5}{6}u^2 v^3-\frac{4}{7}z^6)=} $
          & f) $(2s-2)^3=$ \\
          g) $(10x^5 y^7 -11)(10x^5 y^7 +11)=$ & h) $[(x^2 +1)(x^2 -1)]^2=$ \\ f) $(4x^2+2y^3)^3=$ & \et

\item Berechne:

\bt{lll} a) $\D{\frac{5a^2 -5ab}{(a-b)^2}=}$ &
         b) $\D{\frac{x-y}{x+y}+\frac{x+y}{x-y}=}$ &
         c) $\D{\frac{x+6}{(x-3)^2}+\frac{x}{x^2-9}+\frac{2}{x+3}=}$
         \\[0.2cm]
         d) $\D{\frac{2x-50x^3}{25x^2 -10x+1}=}$ &
         e) $\D{\frac{b}{b^2 -a^2}-\frac{a}{b^2 -a^2}=}$ &
         f) $\D{\frac{a^2}{a-b}-\frac{4ab^3}{(a^2-b^2)(a+b)}-\frac{b^2(a-b)}{(a+b)^2}=}$
         \\[0.2cm]
         g) $\D{\frac{x^2 -8x+16}{3x^2 -48}=}$ &
         h) $\D{\frac{xy}{xy-y^2}(x^2-xy)=}$ &
         i) $\D{\frac{35z}{49z^2-4}-1+\frac{14z-1}{14z+4}=}$
         \\[0.2cm]
         j) $\D{\frac{a^2+1}{a^2-1}:\frac{a+1}{a-1}=}$ &
         k) $\D{\frac{5y-5x}{3y-2x}:\frac{x^2-y^2}{4x^2-9y^2}=}$ &
         l) $\D{\frac{x^2+4y^2}{x^2-4y^2}:(\frac{x}{x-2y}-\frac{2y}{x+2y})=}$ \et

\item Bestimme den Definitionsbereich und berechne die
L�sungsmenge:

\bt{ll}  a) $\D{\frac{6x-3}{x}=5}$ &
         b) $\D{\frac{4}{x-2}-\frac{3}{x-1}=\frac{1}{x}}$ \\[0.2cm]
         c) $\D{\frac{5x}{x-2}-\frac{x}{x+2}=4}$ &
         d) $\D{\frac{x}{x^2-6x+9}-\frac{1}{x^2-3x}=\frac{1}{x}}$ \\[0.2cm]
         e) $\D{\frac{x}{x-4}-\frac{x}{x+4}=0}$ &
         f) $\D{\frac{x+2}{2-x}+\frac{x-2}{2+x}=\frac{4}{4-x^2}}$ \\[0.2cm]
         g) $\D{\frac{x+3}{2x-4}=\frac{x+9}{2x}}$ &
         h) $\D{\frac{x-1}{(x+1)^2}=\frac{1}{x-1}-\frac{2}{x^2-1}}$  \et

\item Bestimme Definitionsbereich und L�sungsmenge:

\bt{llll} a) $5(x-10)\leq \frac{10x+5}{7}+38 \qquad$ &
          b) $\frac{x}{x-3}<0 \qquad$ &
          c) $\frac{2x-3}{4-x}>2 \qquad$ &
          d)$\frac{4x+1}{x-5}\leq 7 $
\et

\item Berechne die Funktionsvorschrift der beiden Geraden $g$ und
$h$, die durch den Punkt $P(4,-3)$ gehen und parallel bzw.
senkrecht zur Geraden $j$ sind, welche durch die Punkte $P_1
(2,-1)$ und $P_2 (-2,5)$ erzeugt wird. Berechne weiters die
Nullstellen aller drei Geraden und �berpr�fe, ob
$(-\frac{2}{3},3)$ auf einer dieser Geraden liegt. Zeichne alle
drei Geraden.

\item 60000 \euro \quad sollen unter drei Preistr�gern derart verteilt
werden, dass auf den zweiten Preis $\frac{2}{3}$ des ersten
Preises und auf den dritten die H�lfte des zweiten Preises
entfallen. Welche Betr�ge entfallen auf die drei Preise?

\item Wie viel kg einer $22\%-$igen Salzl�sung sind zu 6 kg einer
$15\%$-igen Salzl�sung hinzuzuf�gen, um eine $19\%$-ige Salzl�sung
zu erhalten?

\item L�se die folgenden Gleichungssysteme:

$(a)\;%
\begin{array}{c}
\frac{x+2}{3}-\frac{3y-5}{4}=\frac{y+3}{6} \\
\frac{2x+13}{7}=\frac{3x-2}{5}-\frac{y-10}{7}%
\end{array}
$ 

$ (b)\;%
\begin{array}{c}
\frac{4}{x}-\frac{5}{y}=0 \\
-\frac{2}{x}+\frac{1}{y}=4%
\end{array}
$

$ (c)\;%
\begin{array}{c}
2x+2y+2z=1 \\
5x+6y=2 \\
-3x-4y-3z=-1%
\end{array}%
$
\en
\ea