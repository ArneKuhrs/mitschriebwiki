\usepackage[ansinew]{inputenc}
\usepackage{amsfonts}
\usepackage{amsmath}
\usepackage{amssymb}
\usepackage{amsthm}
\usepackage[ngerman]{babel}
%\usepackage[T1]{fontenc}
%\usepackage{graphicx}

\usepackage[final]{pst-pdf}
\usepackage{pst-all}
\usepackage{bm}
\date{2007-12-13}


\usepackage{bm}


\usepackage{color}
\usepackage{longtable}
\usepackage{hyperref}
\usepackage{multicol}
\usepackage{framed}



\usepackage{bbm} % f�r Zahlenmengen

\usepackage{enumitem}
\usepackage[top=4.5cm,left=3cm,headsep=1cm]{geometry}
\usepackage{chngpage}

\usepackage{hyperref}

 % \pagestyle{scrheadings}

\definecolor{dg}{gray}{0.55}
\definecolor{lg}{gray}{0.85}

 \sloppy
%Inhaltsverzeichnis% 
\setcounter{secnumdepth}{3} \setcounter{tocdepth}{1}


\newcommand{\lsg}[1]{\ifthenelse{\boolean{loesung}}{{\cbstart #1 \cbend}}{}}
%\newcommand{\lsglsg}{\ifthenelse{\boolean{loesung}}{\rule{0mm}{0mm}\color{red}}{}}
\newcommand{\lsglsg}{\ifthenelse{\boolean{loesung}}{\rule{0mm}{0mm}}{}}
\newcommand{\bbB}{\mathbb{B}}
\newcommand{\bbN}{\mathbb{N}}
\newcommand{\bbP}{\mathbb{P}}
\newcommand{\bbQ}{\mathbb{Q}}
\newcommand{\bbR}{\mathbb{R}}
\newcommand{\bbZ}{\mathbb{Z}}
\newcommand{\ld}{\operatorname{ld}}
\newcommand{\ltrue}{\text{\em true}}
\newcommand{\lfalse}{\text{\em false}}
\renewcommand{\epsilon}{\varepsilon}
%\setlength{\changebarsep}{1em}

 \sloppy
%Inhaltsverzeichnis% 

%%Eurosymbol zeichnen:
 \newcommand\euro{{\sffamily C%
    \makebox[0pt][l]{\kern-.70em\mbox{--}}%
    \makebox[0pt][l]{\kern-.68em\raisebox{.25ex}{--}}}}
\renewcommand{\baselinestretch}{1.1}



\theoremstyle{plain}
\newtheorem{Satz}{Satz}%[chapter]
\newtheorem*{Merke}{Merke}
\newtheorem{HS}{Hilfssatz}%[chapter]
\renewcommand{\proofname}{Beweis}
\theoremstyle{definition}
\newtheorem{Def}{Definition}%[chapter]
\newtheorem{Bsp}{Beispiel}%[chapter]
\newtheorem{Auf}{Aufgabe}%[chapter]
\newtheorem*{Folg}{Folgerung}
\newtheorem*{Bem}{Bemerkung}
\newtheorem*{Mer}{Merke}



\definecolor{dunkelgelb}{rgb}{1,0.94,0.05}
\definecolor{orange}{rgb}{1,0.52,0}

\newenvironment{fshaded}{%
\def\FrameCommand{\fcolorbox{framecolor}{shadecolor}}%
\MakeFramed {\FrameRestore}}%
{\endMakeFramed}


%Formatierung Satz - Rahmenfarbe und Hintergrundfarbe
\newenvironment{fsatz}[1][]{\definecolor{shadecolor}{rgb}{.973,.513,.27}%
\definecolor{framecolor}{rgb}{1,0.25,0}%
\begin{fshaded}\begin{Satz}[#1]}{\end{Satz}\end{fshaded}}
%Formatierung Definition - Rahmenfarbe und Hintergrundfarbe
\newenvironment{fdef}[1][]{\definecolor{shadecolor}{rgb}{1,1,.29}%
\definecolor{framecolor}{rgb}{1,1,0}%
\begin{fshaded}\begin{Def}[#1]}{\end{Def}\end{fshaded}}
%Formatierung Beispiele - Rahmenfarbe und Hintergrundfarbe
\newenvironment{fbsp}[1][]{\definecolor{shadecolor}{rgb}{.57,1,.28}%
\definecolor{framecolor}{rgb}{.25, .63, 0}%
\begin{fshaded}\begin{Bsp}[#1]}{\end{Bsp}\end{fshaded}}
%Formatierung Aufgaben - Rahmenfarbe und Hintergrundfarbe
\newenvironment{fauf}[1][]{\definecolor{shadecolor}{rgb}{.58,.788,1}%
\definecolor{framecolor}{rgb}{.13,.25,.9}%
\begin{fshaded}\begin{Auf}[#1]}{\end{Auf}\end{fshaded}}


\newcommand{\bb}{\begin{fbsp}}
\newcommand{\eb}{\end{fbsp}}
\newcommand{\ba}{\begin{fauf}}
\newcommand{\ea}{\end{fauf}}
\newcommand{\bn}{\begin{enumerate}}
\newcommand{\en}{\end{enumerate}}
\newcommand{\bi}{\begin{itemize}}
\newcommand{\ei}{\end{itemize}}
\newcommand{\bd}{\begin{fdef}}
\newcommand{\ed}{\end{fdef}}
\newcommand{\bs}{\begin{fsatz}}
\newcommand{\es}{\end{fsatz}}
\newcommand{\bme}{\begin{Mer}}
\newcommand{\eme}{\end{Mer}}
\newcommand{\bt}{\begin{tabular}}
\newcommand{\et}{\end{tabular}}



\newcommand{\beq}{\begin{equination}}
\newcommand{\eeq}{\end{equination}}

\definecolor{blau}{rgb}{0.09, 0.1, 0.64}
\definecolor{hellblaugrau}{rgb}{0.82, 0.906, 0.871}
\newcommand{\mygrid}{\psgrid[gridlabels=0pt, subgriddiv=2, gridwidth=0.5pt, subgridwidth=0.5pt, gridcolor=hellblaugrau, subgridcolor=hellblaugrau]}

\usepackage{scrpage2}        % Seitenkopf gestalten%
\special{papersize=210mm,297mm}         % DVI das DIN A4 Format mitteilen %
\renewcommand{\baselinestretch}{1}        % Zeilenabstand: 1.4-zeilig %
\parskip1.5ex                   % Absatzabstand %
\ohead{\pagemark}
\ihead{\headmark}
\cfoot{Realgymnasium Albert Einstein - Meran}
%\ofoot{\includegraphics[width=1.5cm]{einstein.jpg}}
\ifoot{}
\setheadsepline{.04pt}
\setfootsepline{.04pt}

\usepackage{scrpage2}        % Seitenkopf gestalten%
\special{papersize=210mm,297mm}         % DVI das DIN A4 Format mitteilen %
\renewcommand{\baselinestretch}{1}        % Zeilenabstand: 1.4-zeilig %
\parskip1.5ex                   % Absatzabstand %


\newcommand{\zitat}[2]{ \begin{quote}       % Zitate am Anfang jedes Kapitels
            \textit{%
            \begin{center}%
            {\small#1}\\%
            \end{center}%
            \begin{flushright}%
            {\footnotesize#2}%
            \end{flushright}}%
            \end{quote}}

 
%\usepackage{rotating}

\newcommand{\mybox}[4]{%
    \ifthenelse{\isodd{#4}}
        {% ungerade
            \begin{tabular}{p{3mm}ll}%
            \colorbox{dg}{\parbox[c][#1]{5mm}{\begin{turn}{90}{\Large \sf%
            \textbf{#3}}\end{turn}}}&%
                \colorbox{lg}{\parbox[c][#1]{0.97\textwidth}{\sf#2}}\\%
            \end{tabular}%
        }{%gerade
            \begin{tabular}{l@{}l}%
            \colorbox{lg}{\parbox[c][#1]{0.97\textwidth}{\sf#2}}&%
            \colorbox{dg}{\parbox[c][#1]{5mm}{\begin{turn}{-90}{\Large \sf%
            \textbf{#3}}\end{turn}}}
            \end{tabular}%
        }
    }



\topmargin-1.4cm \oddsidemargin  0.7cm 
\evensidemargin -0.7cm 
\textwidth15.5cm \textheight25cm


\pagestyle{scrheadings}
% Schriftart
\setkomafont{pagenumber}{\normalfont\normalcolor\sffamily}
\setkomafont{pagefoot}{\normalfont\normalcolor\sffamily}
\setkomafont{pagehead}{\normalfont\normalcolor\sffamily}

\renewcommand{\familydefault}{\sfdefault}


\usepackage{framed, color, pifont}
\usepackage{color}
\usepackage{pst-eucl}
\usepackage{multicol}
%\usepackage{sty/secnum}
\SpecialCoor
\usepackage{pstricks,pstricks-add,pst-math,pst-xkey}
\usepackage{pst-pdf}
\usepackage{pstricks-add}