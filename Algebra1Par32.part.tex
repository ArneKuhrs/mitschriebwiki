\section{Algebraischer Abschluß}

\begin{PropDef}[Kronecker]
Sei $K$ Körper, $f\in K[X]$

\begin{enum}
\item Es gibt eine endliche Körpererweiterung $L/K$, so daß $f$ in
$L$ eine Nullstelle hat.
\newline
\sbew{0.9}{\OE $f$ irreduzibel. Setze $L \defeqr K[X]/(f)$. $L$ ist
Körper, da $(f)$ maximales Ideal ist. $\alpha = \bar X$ = Klasse von $X$ in
$L$ ist Nullstelle von $f$.
}
\item Es gibt eine endliche Körpererweiterung $L/K$, so daß $f$
über $L$ in Linearfaktoren zerfällt.
\newline
\sbew{0.9}{Induktion über $n =$ deg$(f)$:
\begin{description}
\item[$n=1$] $\chk$
\item[$n\geq 1$] $L_1$ wie in $(a)$. Dann ist $f(X) = (X - \alpha)
\cd f_1(X)$ in $L_1[X]$, deg$(f_1) = n-1$. Also gibt es $L_2/L_1$,
so daß $f_1(X) =\displaystyle \prod_{i=1}^{n-1} (X-\alpha_i)$ mit $\alpha_i \in
L_2$. Dabei ist $L_2/L_1$ endlich, $L_1/K$ endlich, also $L_2/K$
endlich.
\end{description}}

\item $L/K$ heißt \emp{Zerfällungskörper} von $f$, wenn $f$ über $L$
in Linearfaktoren zerfällt, und $L$ über $K$ von den Nullstellen von
$f$ erzeugt wird.
\item Für jedes $f \in K[X]$ gibt es einen Zerfällungskörper $Z(f)$.
\item Ist $f$ irreduzibel und $n =$ deg$(f)$, so ist $[Z(f) : K]
\leq n!$
\newline
\sbew{0.9}{
In (a) ist $[L:K] = n =$ deg$(f)$ und $f = (X-\alpha) \cd f_1$
mit deg$(f_1) = n-1$. Mit Induktion folgt die Behauptung. }
\end{enum}


\bsp{\begin{enumerate}
\item[(1)] $f \in K[X]$ irreduzibel vom Grad 2. Dann ist $L=K[X]/(f)$
der Zerfällungskörper von $f$. $f(X) = (X - \alpha)(X-\beta)$,
$\alpha,\beta \in L$. Ist $f(X) = X^2 + pX + q$, so ist $\alpha +
\beta = -p$

\item[(2)] $f(X) = X^3 - 2 \in \mathbb{Q}[X]$.
Sei $\alpha = \sqrt[3]{2} \in \mathbb{R}$ Nullstelle von $f$. In
$\mathbb{Q}(\alpha)$ liegt keine weiter Nullstelle von $f$, da
$\mathbb{Q}(\alpha) \subset \mathbb{R}$
\[X^3 - 2 = (X-\alpha)\underset{\mbox{\scriptsize{irreduzibel über}
}\mathbb{Q}(\alpha)}{\underbrace{(X^2 + \alpha X + \alpha^2)}} \Ra
[Z(f) : \mathbb{Q}] = 6\]

\item[(3)] $K = \mathbb{Q}$, $p$ Primzahl, $f(X) = X^p - 1 = (X
-1)\underset{f_1}{\underbrace{(X^{p-1} + X^{p-1} +\dots + X + 1)}}$ $\\f_1$ irreduzibel
(siehe \ref{Bsp 2.27}). $\\\\L \defeqr \mathbb{Q}[X]/(f_1) \defeql
\mathbb{Q}(\zeta_p)$; $(\zeta_p^k)^p = \zeta_p^{pk} = 1$;
$k=1,\dots,p-1$

$\Ra \mathbb{Q}(\zeta_p) = Z(f)$ \end{enumerate} }
\end{PropDef}

\begin{DefBem}
\label{3.7}
Sei $K$ ein Körper.
\begin{enum}
\item $K$ heißt \emp{algebraisch abgeschlossen}, wenn jedes
nichtkonstante Polynom $f \in K[X]$ in $K$ eine Nullstelle hat.
\item Die folgenden Aussagen sind äquivalent:

\begin{enumerate}
\item[(i)] $K$ ist algebraisch abgeschlossen
\item[(ii)] Jedes $f \in K[X]$ zerfällt über $K$ in Linearfaktoren
\item[(iii)] $K$ besitzt keine echte algebraische
Körpererweiterung.
\end{enumerate}
\end{enum}

\sbew{1.0}{
\begin{description}
\item[(i)$\Ra$(iii)] Angenommen $L/K$ algebraisch, $\alpha \in L
\setminus K$. Dann sei $f_\alpha \in K[X]$ das Minimalpolynom
von $\alpha$; $f_\alpha$ ist irreduzibel und hat nach Voraussetzung
Nullstelle in $K \Ra$ deg$(f) = 1 \;\blitzb$
\item[(iii)$\Ra$(ii)] $Z(f) = K$
\end{description}
}
\end{DefBem}

\begin{Satz}
Zu jedem Körper $K$ gibt es eine algebraische Körpererweiterung
$\bar K/K$, so daß $\bar K$ algebraisch abgeschlossen ist. $\bar K$
heißt \emp{algebraischer Abschluß} von $K$.

\sbew{1.0}{

\textbf{Hauptschritt}: Es gibt algebraische Körpererweiterung
$K'/K$, so daß jedes nichtkonstante $f\in K[X]$ in $K'$ eine
Nullstelle hat.

\textbf{Dann}: sei $K'' \defeqr (K')'$ und weiter $K^i \defeqr
(K^{i-1})',\;i\geq 3$; Es ist $K^i \subset K^{i+1}$.

$L\defeqr \displaystyle \bigcup_{i\geq 1} K^i$. Es gilt:
\begin{enumerate}
\item[(i)] $L$ ist Körper: $a+b \in L$ für $a \in K^i, b\in K^j$,
da \OE: $i \leq j \Ra a$ auch in $K^j$

\item[(ii)] $L$ ist algebraisch über $K$: jedes $\alpha \in L$ liegt
in einem $K^i,\;K^i$ ist algebraisch über $K$.

\item[(iii)] $L$ ist algebraisch abgeschlossen.
\newline \textbf{denn}: Sei $f \in L[X],\;f = \displaystyle \sum_{i=0}^n c_i
X^i,\; c_i \in L$. Also gibt es $j$ mit $c_i \in K^j$ für
$i=0,\dots,n \Ra f$ hat Nullstelle in $(K^j)' = K^{j+1} \subset L
\Ra$ Behauptung \end{enumerate}

\textbf{Bew.(Hautpschritt)}: Für jedes $f \in K[X] \setminus K$ sei $X_f$
ein Symbol. $\mathcal{X} \defeqr \{X_f : f \in K[X] \setminus K\}$, $R
\defeqr K[\mathcal{X}]$, $I$ sei das von allen $f(X_f)$ in $R$
erzeugte Ideal.

Sei $\mathfrak{m} \subset R$ ein maximales Ideal mit $I \subset
\mathfrak{m}$, $K'\defeqr R/\mathfrak{m}$, $K'$ ist Körper, $K'/K$
ist algebraisch,
\newline \textbf{denn}: $K'$ wird über $K$ erzeugt von den $X_f \in
\mathcal{X}$ und $f(X_f) = 0$ in $K'$, weil $f(X_f) \in I \subset
\mathfrak{m}$. $f$ hat in $K'$ die Nullstellen (Klasse von) $X_f$. }

\sbew{1.0}{noch zu zeigen:
\begin{enumerate}
\item[(1)] $I \neq R$
\item[(2)] Es gibt maximales Ideal $\mathfrak{m}$ mit $I \subset
\mathfrak{m}$
\end{enumerate}
\textbf{Bew.}:\begin{enumerate}
\item[(1)] Angenommen $I = R$, also $1 \in I$. Dann gibt es $n \geq
1, f_1,\dots,f_n \in K[X] \setminus K$ und $g_1,\dots,g_n \in R$ mit $1 =
\ds \sum_{i=1}^n g_i f_i (X_{f_i})$. Sei $L/K$ Körpererweiterung, in
der jedes $f_i, i=1,\dots,n$ Nullstelle $\alpha_i$ hat (z.B. der
Zerfällungskörper von $f_1\cd\dots\cd f_n$).

Setze nun $\alpha_i$ für $X_{f_i}$ ein ($i=1,\dots,n$) (und $42$ für
alle anderen $X_f$). Dann ist $1 = \ds\sum_{i=1}^n g_i (\alpha_1,
\dots, \alpha_n, 42, \dots) \cd
\underset{=0}{\underbrace{f_i(\alpha_i)}} = 0 \blitzb$

\item[(2)] Proposition: Sei $R$ kommutativer Ring mit $1$, $I\subset
R$ echtes Ideal. Dann gibt es ein maximales Ideal $\mathfrak{m}$ in
$R$ mit $I \subseteq \mathfrak{m}$.

\textbf{Lemma von Zorn}: Sei $M\neq \emptyset$ geordnet. Hat jede
totalgeordnete Teilmenge von $M$ eine obere Schranke, so hat $M$ ein
maximales Element. (dh. ein $x\in M$, so daß aus $y \in M, x\leq y$
folgt: $x = y$)

Sei $M$ die Menge der echten Ideale in $R$, die $I$ enthalten. $I
\in M$, also $M \neq \emptyset$; Sei $N \subset M$ totalgeordnet.

\textbf{Beh.}: $\wt{J}\defeqr \ds\bigcup_{J\in N}\; J$ ist Element von
$M$. (und damit auch obere Schranke für $N$) \textbf{denn}: $I
\subseteq \wt{J}$ Ideal: $x_1,x_2 \in \wt{J} \Ra x_1 \in J_1, x_2
\in J_2$; \OE $J_2 \subseteq J_1 \Ra x_2 \in J_1 \Ra x_1 \pm x_2 \in
J_1 \subset J$;

genauso $r x_1 \in J_1$ für $r \in R$. $1 \not\in \wt{J}$, da  sonst $1
\in J$ für ein $J \in N$.
\end{enumerate}}
\end{Satz}