\documentclass{article}
\newcounter{chapter}
\usepackage{ana}


\author{Joachim Breitner, Pascal Maillard und Wenzel Jakob}
\title{Differenzierbarkeit}
\setcounter{chapter}{21}

\setlength{\parindent}{0pt}
\setlength{\parskip}{2ex}

\begin{document}
\maketitle

In diesem Paragraphen seien stets: $I\subseteq\MdR$ ein Intervall und $f:I\to\MdR$ eine Funktion.

\begin{definition}
\begin{liste}
\item $f$ heißt in $x_0\in I$ \begriff*{differenzierbar}\indexlabel{Differenzierbarkeit} (db) genau dann, wenn der $\lim_{x\to x_0}\frac{f(x) - f(x_0) }{x-x_0}$ existiert und $\in\MdR$ ist. ($\equizu \exists \lim_{h\to0}\frac{f(x_0+h) - f(x_0)}{h}$ und ist $\in\MdR$). In diesem Fall heißt $f'(x_0) = \lim_{x\to x_0}\frac{f(x) - f(x_0) }{x-x_0}$ die \begriff{Ableitung} von $f$ in $x_0$.

\item $f$ heißt auf $I$ differenzierbar genau dann, wenn $f$ in jedem $x\in I$ differenzierbar ist. In diesem Fall wird durch $x\mapsto f'(x)$ eine Funktion $f': I\to\MdR$ definiert, die \begriff*{Ableitung von $f$ auf $I$}.
\end{liste}
\end{definition}

\begin{beispiele}
\item Sei $c\in\MdR$ und $f(x)=c\ \forall x\in I$. $f$ ist differenzierbar auf $I$, $f'(x)=0 \ \forall x \in I$.
\item Sei $I=\MdR $, $n\in\MdN$ und $f(x)=x^n$. Seien $x, x_0 \in\MdR$, $x_0 \ne x$. $\frac{f(x)-f(x_0)}{x-x_0} = \frac{x^n-x_0^n}{x-x_0} \gleichnach{§1} x^{n-1}+x_0^{n-2}x+x^{n-3}x^2+\cdots x_0x^{n-2} + x^{n-1} \to nx_0^{n-1}\ (x\to x_0)$. $f$ ist also differenzierbar auf $\MdR$ und $f'(x) = nx^{n-1} \ \forall x\in\MdR$. Kurz: $(x^n)' = nx^{n-1}$ auf $\MdR$.
\item $I=\MdR$, $f(x)=|x|$, $x_0 = 0$. $x\ne0: \frac{f(x)-f(x_0)}{x-x_0} = \frac{|x|}{x} = \begin{cases}1&x>0\\-1&x<0\end{cases} \folgt$ $f$ ist in $x_0=0$ nicht differenzierbar. (Beachte: $f$ ist stetig in $x_0$)
\item $I=\MdR$, $f(x)=e^x$. 17.3 $\folgt \lim_{h\to0}\frac{e^{x_0+h} - e^{x_0}}{h} = e^{x_0} \ \forall x_0 \in\MdR$. Kurz: $(e^x)' = e^x$.
\end{beispiele}

\begin{satz}[Differenzierbarkeit und Stetigkeit]
Ist $f$ differenzierbar in $x_0\in I$, so ist $f$ stetig in $x_0$
\end{satz}

\begin{beweis}
$f(x)-f(x_0) = \frac{f(x)-f(x_0)}{x-x_0}(x-x_0) \towegen{x\to x_0} f'(x_0)\cdot 0 = 0 \ (x\to x_0) \folgt \lim_{x\to x_0}f(x) = f(x_0)$
\end{beweis}

\begin{satz}[Ableitungsregeln]
$g:I\to\MdR$ sei eine weitere Funktion, $f$ und $g$ ableitbar in $x_0 \in I$.
\begin{liste}
\item Für $\alpha$, $\beta \in \MdR$ ist $\alpha f + \beta g$ differenzierbar in $x_0$ und $$(\alpha f+ \beta g)'(x_0) = \alpha f'(x_0) + \beta g'(x_0)$$
\item $fg$ ist differenzierbar in $x_0$ und $$(fg)'(x_0) = f'(x_0)g(x_0)+f(x_0)g'(x_0)$$
\item Es sei $g(x) \ne 0 \ \forall x\in I$. $\frac{f}{g}$ differenzierbar in $x_0$ und $$(\frac{f}{g})'=\frac{f'(x_0)g(x_0) - f(x_0)g'(x_0)}{g(x_0)^2}$$
\end{liste}
\end{satz}

\begin{beweis}
\begin{liste}
\item Klar. Für (2) und (3) beachte: $f(x)\to f(x_0), g(x)\to g(x_0)\ (x\to x_0)$ (wegen 21.1)
\item $\frac{f(x)g(x)-f(x_0)g(x_0)}{x-x_0} = \frac{f(x) -f(x_0)}{x-x_0}g(x) + \frac{g(x)-g(x_0)}{x-x_0}f(x_0) \to f'(x_0)g(x_0) + g'(x_0)f(x_0) \ (x\to x_0)$
\item $h:=\frac{f}{g}: \frac{h(x)-h(x_0)}{x-x_0}=\frac{1}{g(x)g(x_0)}\left(\frac{f(x)-f(x_0)}{x-x_0}g(x_0) - \frac{g(x)-g(x_0)}{x\to x_0}f(x_0)\right) \to \frac{1}{g(x_0)^2}(f'(x_0)g(x_0) - g'(x_0)f(x_0)) \ (x\to x_0)$.
\end{liste}
\end{beweis}

\begin{beispiele}
\item $f(x) = e^{-x} = \frac{1}{e^x}$, $f'(X) = \frac{-e^x}{(e^x)^2} = -\frac{1}{e^x} = -e^{-x} \ \forall x\in\MdR$
\item $(\cosh x)' = (\frac{1}{2}(e^x+e^{-x}))' = \frac{1}{2}(e^x-e^{-x}) = \sinh x$ auf \MdR.\\
 $(\sinh x)' = (\frac{1}{2}(e^x-e^{-x}))' = \frac{1}{2}(e^x+e^{-x}) = \cosh x$ auf \MdR.
\end{beispiele}

\begin{satz}[Kettenregel]
Sei $J\subseteq\MdR$ ein Intervall, $g:J\to \MdR$ eine Funktionen und $g(J)\subseteq I$. Weiter sei $g$ differenzierbar in $x_0 \in J$ und $f$ differenzierbar in $y_0 := g(x_0)$.
Dann ist $f\circ g: J\to \MdR$ differenzierbar in $x_0$ und $(f\circ g)' (x_0) = f'(g(x_0))g'(x_0)$
\end{satz}

\begin{beweis}
$h(y) = \begin{cases}\frac{f(y)-f(y_0)}{y-y_0} &, y \in I\backslash\{y_0\} \\ f'(y_0) &, y=y_0\end{cases}$ ist differenzierbar in $y_0 \folgt h(y) \to f'(y) = f'(g(x)) \ (y\to y_0)$. 21.1 $\folgt g(x) \to g(x_0) = y_0 \ (x\to x_0) \folgt h(g(x)) \to f'(g(x_0))$ Es ist $f(y) - f(y_0) = h(y)(y-y_0) \ \forall y\in I \folgt \frac{f(g(x)) - f(g(x_0))}{x-x_0} = h(g(x))\frac{g(x)-g(x_0)}{x-x_0} \to f'(g(x))g'(x_0)\ (x\to x_0)$
\end{beweis}

\begin{beispiele}
\item Sei $I=\MdR$, $a>0$, $a^x = e^{x\log a} = f(g(x))$ mit $f(x)= e^x$, $g(x)= x\log a \folgt (a^x)' = f'(g(x))g'(x) = e^{x\log a} \log a = a^x \log a $ auf $\MdR$
\item $I= [0,\infty), f(x)=x^2, f'(x)=2x, f'(0)=0$\\
 $f^{-1}(x) = \sqrt{x} \ (x\in [0,\infty))$. \\
 Es gilt: $x= f(f^{-1}(x)) (*) \ \forall x\ge 0$ Annahme: $f^{-1}$ ist differenzierbar in $x_0 = 0 \folgtnach{21.3, \ensuremath{(*), x_0 =0}} 1 = \underbrace{f'(f^{-1}(0))}_{0}\cdot(f^{-1})'(0) = 0$ Widerspruch!\\ Das heißt $f^{-1}(x_0)$ ist in $x_0 = 0$ nicht differenzierbar.
\end{beispiele}

\begin{satz}[Ableitung der Umkehrfunktion]
$f\in C(I)$ sei streng monoton, $f$ differenzierbar in $x_0\in I$ und $f'(x_0)\ne 0$. Dann ist $f^{-1}: f(I) \to I$ differenzierbar in $y_0 := f(x_0)$ und $(f^{-1})'(y_0) = \frac{1}{f'(x_0)}$
\end{satz}

\begin{beweis}
Sei $(y_n)$ eine Folge in $f(I)\backslash\{y_0\}$ und $y_n \to y_0$ und $\alpha_n= \frac{f^{-1}(y_n) - f^{-1}(y_0)}{y_n-y_0}$. Zu zeigen: $\alpha_n \to \frac{1}{f'(x_0)} \ (n\to\infty)$
$x_n := f^{-1}(y_n) \folgt y_n = f(x_n)$, $x_n \in I$, $\forall n\in\MdN \folgt \alpha_n = \frac{x_n-x_0}{f(x_n)-f(x_0)} \to \frac{1}{f'(x_0)} \ (n\to\infty)$
\end{beweis}

\begin{beispiele}
\item $I=\MdR$, $f(x) = e^x$, $f^{-1}(y) = \log y\ (y>0)$. Sei $y>0$, also $y=e^x\ (x\in\MdR) \folgt (f^{-1})(y) = \frac{1}{f'(x)} = \frac{1}{e^x} = \frac{1}{y}$. Kurz: $(\log x)' = \frac{1}{x}$ auf $(0,\infty)$.
\item Sei $\alpha \in\MdR$ und $f(x) = x^\alpha \ (x>0)$, dann: $f(x) = e^{\alpha \log x} \folgt f'(x) = e^{\alpha \log x}\cdot (\alpha \log x)' = x^\alpha\cdot\frac{\alpha}{x} = \alpha x^{\alpha-1}$. Kurz: $(x^\alpha)' = \alpha x^{\alpha-1}$ auf $(0,\infty)$
\item Für $\alpha = \frac{1}{2}$ liefert Beispiel (2): $(\sqrt{x})' = \frac{1}{2\sqrt{x}}$ auf $(0,\infty)$
\end{beispiele}

\begin{definition}
Zu $\emptyset \ne M \subseteq \MdR$ und $x_0\in M$. $x_0$ heißt ein \begriff{innerer Punkt} von $M$ genau dann, wenn es ein $\delta>0$ gibt, so dass $U_\delta(x_0) \subseteq M$.
\end{definition}

\begin{beispiele}
\item $M$ ist offen genau dann, wenn jedes $x\in M$ ein innerer Punkt von $M$ ist.
\item Sei $a<b$, $M\in\{ [a,b], (a,b), [a,b), (a,b] \}$. $x_0 \in M$ ist innerer Punkt von $M$ genau dann, wenn $x_o\in(a,b)$
\item $\MdQ$ hat keine inneren Punkte
\end{beispiele}

\begin{definition}
Sei $\phi \ne D \subseteq \MdR,\ g: D \to \MdR$ und $x_0 \in D,\ g$ hat in $x_0$ ein \begriff*{relatives Maximum}\indexlabel{relatives!Maximum}\indexlabel{Maximum!relatives} $:\equizu\ \exists \delta>0: g(x)\leq g(x_0)\ \forall x \in D \cap U_\delta(x_0).$

Sei $\phi \ne D \subseteq \MdR,\ g: D \to \MdR$ und $x_0 \in D,\ g$ hat in $x_0$ ein \begriff*{relatives Minimum}\indexlabel{relatives!Minimum}\indexlabel{Minimum!relatives} $:\equizu\ \exists \delta>0: g(x)\geq g(x_0)\ \forall x \in D \cap U_\delta(x_0).$

Ein \begriff*{relatives Extremum}\indexlabel{relatives!Extremum}\indexlabel{Extremum!relatives} ist ein relatives Maximum oder Minimum.
\end{definition}

\begin{satz}[Erste Ableitung am relativen Extremum]
$f$ sei differenzierbar in $x_0 \in I$, $f$ habe in $x_0$ ein relatives Extremum und $x_0$ sei ein innerer Punkt von $I$. Dann gilt: $f'(x_0) = 0$.
\end{satz}

\begin{beweis}
$f$ habe in $x_0$ ein relatives Maximum. Dann existiert $\delta > 0: U_\delta(x_0) \subseteq I$ und $f(x) \leq f(x_0)\ \forall x \in U_\delta(x_0).$

\begin{tabbing}
Sei $x \in U_\delta(x_0)$ und $x\ $\=$< x_0 \folgt \frac{f(x)-f(x_0)}{x-x_0}\ $\=$\geq 0 \folgt f'(x_0)\quad $\=$ (x \to {x_0}-)$\\
\>$>$ \>$\le$ \>$(x \to {x_0}+)$
\end{tabbing}

Also: $f'(x_0) = 0.$
\end{beweis}

\begin{bemerkungen}
\begin{liste}
\item Die Voraussetzung "`$x_0$ ist ein innerer Punkt von $I$"' ist wesentlich. Beispiel: $f(x) = x,\ x \in [0,1],\ x_0 = 0$ oder $x_0 = 1$.
\item Ist $f$ differenzierbar in $x_0$ und $f'(x_0) = 0$, so muss $f$ in $x_0$ \emph{kein} relatives Extremum haben. Beispiel: $f(x) = x^3,\ x_0 = 0$.
\end{liste}
\end{bemerkungen}

\begin{satz}[Mittelwertsatz der Differenzialrechnung]
Sei $I = [a,b]\ (a<b),\ f,g \in C(I)$ und $f$ und $g$ seien differenzierbar auf $(a,b)$. Weiter sei $g'(x) \ne 0\ \forall x \in (a,b)$.

\begin{liste}
\item \textbf{Satz von Rolle}: Es sei $f(a) = f(b)$. Dann existiert $\xi \in (a,b):$
$$f'(\xi) = 0.$$
\item \textbf{Mittelwertsatz (MWS) der Differenzialrechnung}:
$$\exists \xi \in (a,b): \frac{f(b)-f(a)}{b-a} = f'(\xi).$$
\item \textbf{Erweiteter Mittelwertsatz}: Es ist $g(b) \ne g(a)$ und $\exists \xi \in (a,b):$
$$\frac{f(b)-f(a)}{g(b)-g(a)} = \frac{f'(\xi)}{g'(\xi)}.$$
\end{liste}
\end{satz}

\begin{beweise}
\item 18.3 $\folgt \exists s,t \in  [a,b]: f(s) \le f(x) \le f(t)\ \forall x \in [a,b].$

\underline{Fall 1}: $s,t \in \{a,b\} \folgt f$ ist auf $I$ konstant $\folgt f' = 0$ auf $I \folgt$ Beh.

\underline{Fall 2}: $s \in (a,b)$ oder $t \in (a,b)$. Etwa: $s \in (a,b) \folgt s$ ist ein innerer Punkt von $I$ und $f$ hat in $s$ ein Minimum. 21.5 $\folgt f'(s) = 0$.
\item folgt aus (3) mit $g(x) = x$.
\item $h(x) := (f(b)-f(a))g(x) - (g(b)-g(a))f(x)\ (x \in I)$. Dann gilt: $h \in C(I),\ h$ ist differenzierbar auf $(a,b)$.

$h(a) = h(b) \overset{\text{(1)}}{\folgt} \exists \xi \in (a,b): 0 = h'(\xi) = (f(b)-f(a))g'(\xi) - (g(b)-g(a))f'(\xi)$

$$\folgt (f(b)-f(a))g'(\xi) = (g(b)-g(a))f'(\xi).$$

Aus (1) folgt: $g(a) \ne g(b)$ (sonst existierte $x_0 \in (a,b)$ mit $g'(x_0) = 0$).

$$\folgt \frac{f(b)-f(a)}{g(b)-g(a)} = \frac{f'(\xi)}{g'(\xi)}.$$
$ $
\end{beweise}

\begin{folgerungen}

$f,g: I \to \MdR$ seien differenzierbar auf $I$.
\begin{liste}
\item \begin{tabbing}
Ist $f' $\=$= 0$ auf $I \folgt f$ ist auf $I$ \=konstant\\
\>$\geq$ \>wachsend\\
\>$\leq$ \>fallend\\
\>$>$    \>streng wachsend\\
\>$<$    \>streng fallend\\
\end{tabbing}

\item Ist $f' = g'$ auf $I \folgt \exists c \in \MdR: f=g+c$ auf $I$.
\end{liste}
\end{folgerungen}

\begin{beweise}
\item Seien $x_1,x_2 \in I$ und $x_1<x_2$. 21.6 (2) $\folgt \exists \xi \in (x_1,x_2): f(x_2)-f(x_1) = f'(\xi)(x_2-x_1) \folgt$ Beh.
\item $h := f-g \folgt h' = 0$ auf $I \overset{(1)}{\folgt}$ Beh.
\end{beweise}

\begin{beispiele}
\item Es existiert genau ein $x_0 \in \MdR: e^{-x_0} = x_0$.
	\begin{beweis}
	$f(x) := e^{-x}-x\ (x \in \MdR)\quad f(0) = 1 > 0,\ f(1) = \frac{1}{e}-1 < 0$. 18.2 $\folgt \exists x_0 \in (0,1): f(x_0) = 0$, also: $e^{-x_0} = x_0$.

	$f'(x) = -e^{-x}-1 < 0\ \forall x \in \MdR \overset{\text{21.7}}{\folgt} f$ ist streng fallend $\folgt f$ hat genau eine Nullstelle, nämlich $x_0$. $\folgt$ Beh.
	\end{beweis}
\item Ist $f: \MdR \to \MdR$ differenzierbar auf $\MdR$ und $f' = f$ auf $\MdR \folgt \exists c \in \MdR: f(x) = ce^x\ (x \in \MdR)$.
	\begin{beweis}
	$h(x) := \frac{f(x)}{e^x} \folgt h'(x) = \frac{f'(x)e^x - e^xf(x)}{(e^x)^2} = 0\ \forall x \in \MdR \overset{\text{21.7}}{\folgt} \exists c \in \MdR: h(x) = c\ \forall x \in \MdR \folgt$ Beh.
	\end{beweis}
\end{beispiele}

\newcommand{\dlim}[2]{\displaystyle{\lim_{#1}{#2}}}

\begin{satz}[Die Regeln von de l'Hospital]
$f,g:(a,b) \to \MdR$ seien auf $(a,b)$ differenzierbar und es sei $g'(x) \ne 0\ \forall x \in (a,b)$ ($a=-\infty$ oder $b=\infty$ zugelassen). Weiter existiere $L := \dlim{\underset{x \to b}{x \to a}}{\frac{f'(x)}{g'(x)}}$ ($L=\pm\infty$ zugelassen) und es gelte
\begin{enumerate}
\item[(I)] $\dlim{\underset{x \to b}{x \to a}}{f(x)} = \dlim{\underset{x \to b}{x \to a}}{g(x)} = 0$ \emph{oder}
\item[(II)] $\dlim{\underset{x \to b}{x \to a}}{g(x)} = \pm\infty$.
\end{enumerate}

Dann gilt: $\dlim{\underset{x \to b}{x \to a}}{\frac{f(x)}{g(x)}} = L$.
\end{satz}

\begin{beweis}
Nur unter der Voraussetzung (I) und nur für $x \to a$.

\underline{Fall 1}: $a \in \MdR$. $f(a) := g(a) := 0 \overset{\text{(I)}}{\underset{\text{21.1}}{\folgt}} f,g \in C[a,b)$.

Sei $x \in (a,b)$. 21.6 (3) $\folgt \exists \xi = \xi(x) \in (a,x): \frac{f(x)}{g(x)} = \frac{f(x)-f(a)}{g(x)-g(a)} = \frac{f'(x)}{g'(x)} \to L$ (für $x \to a$, da dann auch $\xi \to a$).

\underline{Fall 2}: $a = -\infty$. Substituiere $x = \frac{1}{t}$, also $t = \frac{1}{x}$ ($x \to a = -\infty \equizu t \to 0-$).

$\varphi(t) := f(\frac{1}{t}) = f(x),\ \psi(t) := g(\frac{1}{t}) = g(x)$. z.z.: $\frac{\varphi(t)}{\psi(t)} \to L$ ($t \to 0-$)

$\varphi'(t) = f'(\frac{1}{t})(\frac{1}{-t^2}) = f'(x)(-x^2)$\\
$\psi'(t) = g'(x)(-x^2)$

$\folgt \frac{\varphi'(t)}{\psi'(t)} = \frac{f'(x)}{g'(x)} \to L$ ($t \to 0-$) $\overset{\text{Fall 1}}{\folgt} \frac{\varphi(t)}{\psi(t)} \to L$ ($t \to 0-$).
\end{beweis}

\begin{beispiele}
\item $a,b > 0: \dlim{x \to 0}{\frac{a^x-b^x}{x}} = \dlim{x \to 0}{\frac{a^x\log{a} - b^x\log{b}}{1}} = \log{a} - \log{b}$
\item $\dlim{x \to \infty}{\frac{\log{x}}{x}} = \dlim{x \to \infty}{\frac{\frac{1}{x}}{1}} = 0$
\item $\dlim{x \to \infty}{x^{\frac{1}{x}}} = \dlim{x \to \infty}{e^{\frac{\log{x}}{x}}} = e^0 = 1$
\item $\dlim{z \to 0}{\frac{\log{(1+tz)}}{z}} = \dlim{z \to 0}{\frac{\frac{1}{1+tz} \cdot t}{1}} = t$ ($t \in \MdR$)
\item Für $t \in \MdR: \dlim{x \to \infty}{(1+\frac{t}{x})^x} = e^t$ (insbesondere $\dlim{n \to \infty}{(1+\frac{t}{n})^n} = e^t,\ n \in \MdN$)
\begin{beweis}
$\varphi(x) := (1+\frac{t}{x})^x.$\\
$\lim_{x \to \infty}{\log{\varphi(x)}} = \lim_{x \to \infty}{x\log{(1+\frac{t}{x})}} \overset{z = \frac{1}{x}}{=} \lim_{x \to \infty}{\frac{\log{(1+tz)}}{z}} = t$

$\folgt \varphi(x) \to e^t$ ($x \to \infty$).
\end{beweis}
\end{beispiele}

\begin{satz}[Ableitung von Potenzreihen]
Sei $\reihenull{a_n(x-x_0)^n}$ eine Potenzreihe mit Konvergenzradius $r>0, I:=(x_0-r, x_0+r), (I=\MdR$, falls $r=\infty)$ und $f(x):=\reihenull{a_n(x-x_0)^n} (x \in I)$
\begin{liste}
\item Die Potenzreihe $\reihenull{na_n(x-x_0)^{n-1}}$ hat den Konvergenzradius $r$.
\item $f$ ist auf $I$ differenzierbar und $f'(x):=\reihenull{na_n(x-x_0)^{n-1}}\ \forall x \in I$, also $(\reihenull{a_n(x-x_0)^n}) = \reihenull{(a_n(x-x_0)^n)'}$
\end{liste}
\end{satz}

\begin{beweise}
\item $\limsup \sqrt[n]{|na_n|}=\limsup \sqrt[n]{n}\sqrt[n]{|a_n|}=\limsup \sqrt[n]{|a_n|} \folgt$ Behauptung.
\item Sp"ater
\end{beweise}

\begin{beispiele}
\item $(\sin x)' = \reihenull{((-1)^n\frac{x^{2n+1}}{(2n+1)!}})'=\reihenull{\frac{x^{2n}}{(2n)!}}=\cos x$ auf $\MdR$.
\item $(\cos x)' = -\sin x$
\end{beispiele}

\begin{satz}[Eigenschaften trigonometrischer Funktionen]
\begin{liste}
\item $\forall x \in \MdR: \cos^2 x + \sin^2 x = 1$, $|\cos x| \le 1$, $|\sin x| \le 1$, $|sin x| \le |x|$
\item Additionstheoreme: $\ \forall x,y \in \MdR: \sin(x+y) = \sin x \cos y + \cos x \sin y$, $\cos(x+y) = \cos x \cos y - \sin x \sin y$
\item $\sin x > x - \frac{x^3}{3!} > 0 \ \forall x \in (0, 2)$; insbesondere: $\sin 1 > \frac{5}{6}$.
\item $\exists \xi_0 \in (0, 2)$ mit $\cos \xi_0=0$ und $\cos x \ne 0\ \forall x \in [0, \xi_0), \pi:=2\xi_0$ (Pi). Also: $\pi \in (0, 4)\ (\pi \approx 3,14..), \cos\frac{\pi}{2}=0, \cos x \ne 0 \ \forall x \in [0, \frac{\pi}{2}]$.
\item $\sin\frac{\pi}{2}=1$
\item \begin{tabbing}
$\sin(-x)=-\sin x, $\ \ \ \ \=$\cos(-x)=\cos x$\\
$\sin(x+\frac{\pi}{2})=\cos x, $\>$\cos(x+\frac{\pi}{2})=-\sin x$\\
$\sin(x+\pi)=-\sin x, $\>$\cos(x+\pi)=-\cos x$\\
$\sin(x+2\pi)=\sin x, $\>$\cos(x+2\pi)=\cos x$
\end{tabbing}
\item F"ur $x \in [0, \pi]: \cos x = 0 \equizu x=\frac{\pi}{2}$
\item $\sin x=0 \equizu \exists k \in \MdZ: x = k\pi$. \\
	$\cos x=0 \equizu \exists k \in \MdZ: x = k\pi+\frac{\pi}{2}$.
\end{liste}
\end{satz}

\begin{beweise}
\item $f(x):=\cos^2 x + \sin^2 x -1$. $f'(x)=2\cos x(-\sin x)+2\sin x\cos x=0$. 21.7 $\folgt f$ ist auf $\MdR$ konstant. $f(0)=0$ $|\cos x|=\sqrt{\cos^2 x} \le \sqrt{\cos^2 x + \sin^2 x}=1$, ObdA $x\ne0$. $\sin x=\sin x-\sin 0 \gleichnach{MWS} \underbrace{|\cos \xi|}_{\le 1}|x| \le |x|$
\item Sei $y \in \MdR$ und $f(x):=(\sin(x+y)-\sin x \cos y - \cos x \sin y)^2 + (cos(x+y)-\cos x \cos y + \sin x \sin y)^2$. Klar: $f(0)=0$. Nachrechnen: $f'=0$ auf $\MdR$. 21.7 $\folgt f\equiv 0$ auf $\MdR$.
\item F"ur $x\in(0,2): \sin x = \underbrace{(x-\frac{x^3}{3!})}_{>0}+\underbrace{(\frac{x^5}{5!}-\frac{x^7}{7!})}_{>0}+\cdots \folgt$ Behauptung.
\item $cos 0 = 1> 0$. $\cos 2 = \cos(1+1) = \cos^2 1 - \sin^2 1=\cos^2 1 + \sin^2 1 - 2\sin^2 1 = 1-2\sin^2 1 \overset{\text{(3)}}{<} 1-2\frac{25}{36}<0$. 18.2 $\folgt \exists \xi_0 \in (0, 2): \cos \xi_0=0$, In $(0,2)$: $(\cos x)'=-\sin x \overset{\text{(3)}}{<} 0 \folgt \cos x$ ist in $(0,2)$ streng monoton fallend $\folgt \cos x \ne 0 \ \forall x \in [0, \xi_0)$
\item $\sin^2\frac{\pi}{2}=1-\cos^2\frac{\pi}{2}=1\folgt\sin\frac{\pi}{2}=\pm 1$. (3) $\folgt\sin\frac{\pi}{2}>0\folgt\sin\frac{\pi}{2}=1$.
\item Die erste Behauptung mit kann mit Potenzreihen, der Rest mit den Additionstheoremen bewiesen werden.
\item \glqq$\Leftarrow$\grqq: klar, \glqq$\Rightarrow$\grqq: Sei $x\in[0, \pi]$ und $\cos x = 0 \folgtnach{(4)} x\ge\frac{\pi}{2}, y:=\pi-x, y \in [0, \frac{\pi}{2}]$ und $\cos y=\cos(x+\pi)\gleichnach{(6)}-\cos(-x)=-\cos(x)\folgtnach{(4)}y\le\frac{\pi}{2}, x=\frac{\pi}{2}$.
\item In den gr. "Ubungen
\end{beweise}

\begin{wichtigedefinition}[Tangens]
$$\tan x:=\frac{\sin x}{\cos x}\ \text{f"ur}\ x \in \MdR\ \backslash\ \{k\pi+\frac{\pi}{2}\ |\ k \in \MdZ\}.$$ $I:=(-\frac{\pi}{2}, \frac{\pi}{2})$; $f(x):=\tan x\ (x\in I)$. Dann: $f\in C(I)$. $
\displaystyle\lim_{x\to\frac{\pi}{2}}{f(x)}=\infty$, $\displaystyle\lim_{x\to-\frac{\pi}{2}}{f(x)}=-\infty$, $f'(x)=\frac{\cos^2 x + \sin^2 x}{\cos^2 x}=\frac{1}{\cos^2 x}=1+\frac{\sin^2 x}{\cos^2 x}=1+\tan^2 x>0$ auf $I \folgt f$ ist auf $I$ streng monoton wachsend $\folgt \exists f^{-1}: \MdR \to I$, $\arctan x:=f^{-1}(x) (x \in \MdR)$ \begriff{Arcustangens}. Sei $y=\tan x\ (x \in I)$. $(f^{-1})'(y)=\frac{1}{f'(x)}=\frac{1}{1+\tan^2 x}=\frac{1}{1+y^2}$. Also: $(\arctan x)'=\frac{1}{1+x^2}$ auf $\MdR$.
\end{wichtigedefinition}


\begin{definition}
Sei $I \subseteq \MdR$ ein Intervall; $f:I\to\MdR$ eine Funktion und $x_0\in I$. $f$ wird in einer Umgebung von $x_0$ durch eine Potenzreihe dargestellt $:\equizu \exists \delta>0$ und $\exists$ eine Potenzreihe $\reihenull{a_n(x-x_0)^n}$ mit Konvergenzradius $\ge \delta$ und $f(x)=\reihenull{a_n(x-x_0)^n}\ \forall x \in I \cap U_{\delta}(x_0)$.
\end{definition}

\begin{beispiele}
\item $I=(-\infty, 1), f(x)=\frac{1}{1-x}.$ Bekannt: $\reihenull{x^n}=\frac{1}{1-x}$ f"ur $x \in (-1, 1)$. Also: $f(x)=\reihenull{x^n}$ f"ur $x \in (-1, 1)$
\item $I=\MdR, f(x)=\frac{1}{1+x^2}=\frac{1}{1-(-x^2)}=\reihenull{(-x^2)^n}=\reihenull{(-1)^nx^{2n}}\ (x\in(-1, 1))$
\item $I=(-1, \infty), f(x)=\log(1+x).$ Behauptung: (*) \\
\fbox{$\log(1+x)=\reihenull{(-1)^n\frac{x^{n+1}}{n+1}}\ (x\in (-1, 1))$}
\begin{beweis}
$g(x):=\reihenull{(-1)^n\frac{x^{n+1}}{n+1}}\ (x \in (-1, 1))$ 21.9 $\folgt g$ ist auf $(-1, 1)$ differenzierbar und $g'(x)=\reihenull{(-1)^nx^n}=\frac{1}{1-(-x)}=\frac{1}{1+x}=f'(x)\ \forall x \in (-1, 1)$. 21.7 $\folgt \exists c \in \MdR: f(x)=g(x)+c\ \forall x \in (-1, 1) \folgtwegen{x=0}0=f(0)=g(0)+c=c\folgt f(x)=g(x)\ \forall x \in (-1, 1) \folgt$ Behauptung.
In den gr. "Ubungen wird gezeigt (Abelscher Grenzwert-Satz): $(*)$ gilt noch f"ur $x=1$. Also: $\log 2=\reihenull{(-1)^n\frac{(-1)^{n+1}}{n}}=1-\frac{1}{2}+\frac{1}{3}-\frac{1}{4}+\cdots$
\end{beweis}
\end{beispiele}


\end{document}
