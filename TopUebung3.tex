\documentclass{article}
\usepackage[utf8]{inputenc}
\usepackage{mathrsfs}
\usepackage{stmaryrd}

\usepackage{mathe}

\title{3. Topologie-Übung}
\author{Joachim Breitner}

\begin{document}
\maketitle
\section*{1. Aufgabe}

Sei $(X, d)$ ein beschränkter metrischer Raum, $\mathcal C_b(x) (X) \da \{\text{beschränkte reellwertige Fnktionen auf $X$}\}$, $|f|_\infty \da \sup \{ |f(x)| \mid x \in X\}$, Metrik $d_\infty(f,g) = |f-g|_\infty$ auf $\mathcal C_b(X)$.

\paragraph{Behauptung:} Für jedes $a\in X$ ist die Funktion $f_a: X\to \MdR$, $x\mapsto d(a,x)$ stetig und beschränkt.

Seien $x\in x$, $f_a(x) \da r \in \MdR$, $\ep > 0$. Dann gilt für alle $y\in B_{\frac\ep2}(x)$ dass $d(a,y) \le d(a,x) + d(x,y) \le d(a,x) + \frac\ep2$ sowie dass $d(a,x) \le d(a,y) + d(y,x) \le d(a,y) + \frac\ep2$. Also ist $d(x,a)-\frac\ep2 \le d(y,a) \le d(x,a) + \frac\ep2$ und $f_a(y)= d(a,y) \in B_\ep(f_a(x))$.

Mit $\delta = \frac\ep2$ gilt also: $f_a(B_\delta(x)) \subseteq B_\ep(f_a(x)) = B_\ep(r)$, also ist $f_a$ stetig.

$f_a$ ist beschränkt, da $X$ beschränkt ist (klar nach Definition von $f_a$).\hfill$\blacksquare$

\paragraph{Behauptung:} $\varphi : X\to \mathcal C_b(X)$, $x\mapsto f_X$ ist abstandserhaltend bezüglich $d$ und $d_\infty$.

Zu zeigen ist: $\forall x,y\in X: d_\infty(f_x,f_y) = d(x,y)$. Einerseits gilt: $d_\infty(f_x,f_y) = \sup_{x\in X} | f_x(t) - f_y(t) | = \sup _{x\in X} |d(x,t) - d(y,t)| \le d(x,y)$ wegen der Dreiecksungleichung für $d$. Andererseits gilt: $d_\infty(f_x, f_y) = \sup_{x\in X} |f_x(t) - f_y(t)| \ge |f_x(y) - f_y(y)| = |d(x,y) - 0| = d(x,y)$. Insgesamt gilt also: $d_\infty(f_x,f_y) = d(x,y)$ \hfill$\blacksquare$

\section*{2. Aufgabe}

$(X,d)$ metrischer Raum, $f:\mathbb R_{\ge0} \to \mathbb R_{\ge0}$ monoton wachsend, nicht konstant, konkav mit $f(0)=0$.

\paragraph{Behauptung:} $f\circ d$ ist Metrik auf $X$.

\begin{itemize}
\item Symmetrie: \checkmark
\item $f(d(x,x)) = f(0) = 0$ für alle $x\in X$.
\item $f(d(x,y)) = 0 \iff x=y$;

Da $f$ monoton wachsend, nicht konstant und konkav ist, ist $0$ die einzige Nullstelle von $f$: Es gibt ein $\tilde x \in X : f(\tilde x) > 0$, da $f$ nicht konstant ist. Wäre $x\ne 0$ eine weitere Nullstelle von $f$, so gelte $x<\tilde x$, da $f$ monoton ist. Dann: $0 = f(x) = f(\frac x {\tilde x}\cdot \tilde x)\ge \frac x {\tilde x} f(\tilde x) > 0$, $\lightning$.

\item $\Delta$-Ungleichung. Zu zeigen: $f(a) + f(b) \ge f(a + b)$, denn dann gilt $\forall x,y,z\in X: f(d(x,y)) + f(d(y,z)) \ge f(d(x,y) + d(y,z)) \ge f(d(x,z))$

Es ist: $f(a) = f(\frac a {a+b}(a+b)) = f(\frac a {a+b} (a+b) + 0 ) \ge \frac a{a+b} f(a+b) + (1- \frac a{a+b}) f(0) = \frac a{a+b} f(a+b)$. Ebenso ist: $f(b) \ge \frac b{a+b} f(a+b)$. Addiert man diese Ungleichungen, erhält man $f(a) + f(b) \ge \frac a{a+b}f(a+b) + \frac b{a+b}f(a+b) = f(a+b)$.\hfill$\blacksquare$
\end{itemize}

\paragraph{Behauptung:} Ist $f$ streng monoton, dann definieren $f$ und $f\circ d$ die selbe Topologie auf $X$.

$f$ streng Monoton wachsend, also gilt $\forall \ep>0: d(x,y) <\ep \iff f(d(x,y)) < f(\ep)$. Also ist: $B_\ep^d(x) = \{y\in X\mid d(x,y) <\ep \} = \{y\in X\mid f(d(x,y))< f(\ep)\} = B_{f(x)}^{f\circ d}(x)$. Also ist jeder offene Ball bezüglich $d$ ist ein offener Ball bezüglich $f\circ g$ und umgekehrt.

\emph{Das ist falsch!} Gegenbeispiel: $X\in R$, $d$ der euklidische Abstand, $f:\mathbb R_{\ge0} \to \mathbb R_{\ge0}$, $0 \mapsto 0$, $x\mapsto 1+x$ für $x>0$. Dann ist $B_{\frac 12}^{f\circ d}(x) = \{x\}$

Der Beweis funktioniert mit der zusätzlichen Annahme $\lim_{x\to 0} f(x) = 0$.

\paragraph{Behauptung} Auch wenn $f$ streng monoton ist könnte $X$ bezüglch $f\circ d$ vollständig sein und bezüglich $d$ nicht.

Beispiel: Sei $(X,d)$ ein nicht vollständiger Raum und $f$ wie im letzten Gegenbeispiel. Dann sind Cauchyfolgen gerade die, die konstant wird, also konvergieren sie.

\section*{3. Aufgabe}

Sei $X=\MdR^2$ versehen mit der SNCF-Metrik:
\[
d(x,y) =
\begin{cases}
|x-y|, &x = \lambda y,\, y\in \MdR \\
|x|+|y|, &\text{sonst.}
\end{cases}
\]

Einfach zu zeigen: $d$ ist eine Metrik.



\section*{4. Aufgabe}
\end{document}
