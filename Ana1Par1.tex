\documentclass{article}
\newcounter{chapter}
\setcounter{chapter}{1}
\usepackage{ana}

\title{Reelle Zahlen}
\author{Joachim Breitner}
% Wer nennenswerte �nderungen macht, schreibt euch bei \author dazu

\begin{document}
\maketitle

Die \begriff{Reellen Zahlen} sind eine Erfindung des menschlichen Geistes, sie haben von Natur aus keine Eigenschaften. Wie Schachfiguren haben sie nur eine Bedeutung im Rahmen der Regeln. Diese Regeln hei�en hier Axiome, das sind Forderungen, die wir an etwas stellen, und aus denen wir dann weitere Erkenntnisse erlangen.

Die Grundmenge der Analysis ist $\MdR$, die Menge der reellen Zahlen: Diese Menge f�hren wir axiomatisch ein, durch die folgenden 15 Axiome.

In $\MdR$ sind zwei Verkn�pfungen \glqq +\grqq und \glqq $\cdot$\grqq gegeben, die jedem Paar $a,b \in \MdR$ genau ein $ a+b \in \MdR$ und genau ein $ ab := a \cdot b \in \MdR$ zuordnen.

\indexlabel{K�rperaxiome}\begin{axiom}[K"orperaxiome]
\begin{liste}
\item[(A1)] $a+(b+c) = (a+b)+c \ \forall a,b,c \in \MdR$
\item[(A2)] $a(bc) = (ab)c \ \forall a,b,c \in \MdR$
\item[(A3)] $a+b = b+a \ \forall a,b \in \MdR$
\item[(A4)] $ab = ba \ \forall a,b \in \MdR$
\item[(A5)] $\exists 0 \in \MdR: a + 0 = a \ \forall a \in \MdR$
\item[(A6)] $\exists 1 \in \MdR\setminus\{0\}: a \cdot 1 = a \ \forall a \in \MdR$
\item[(A7)] $\forall a \in \MdR\ \exists -a \in \MdR: a + (-a) = 0 $
\item[(A8)] $\forall a \in \MdR \setminus \{0\}\ \exists a^{-1} \in \MdR: a a^{-1} = 1 $
\item[(A9)] $a(b+c) = ab + ac \ \forall a,b,c \in \MdR$
\end{liste}
\end{axiom}

Dabei nennt man \textbf{A1} und \textbf{A2} \begriff{Assoziativgesetze}, \textbf{A3} und \textbf{A4} \begriff{Kommutativgesetze} und \textbf{A9} \begriff{Distributivgesetz}, 

Alle Regeln der Grundrechenarten lassen sich aus \textbf{(A1)} bis \textbf{(A9)} herleiten. diese Regeln seien von nun an bekannt.

\begin{beispiele}
\item \textbf{Behauptung:} Es gibt genau ein $0 \in \MdR$ mit $a+0 = a \ \forall a\in \MdR$.

\textbf{Beweis:} Die Existenz folgt direkt aus \textbf{(A5)}. Der Beweis der Eindeutigkeit: Es sei $\tilde 0 \in \MdR$ mit $a+0 = a \ \forall a \in \MdR$. Daraus folgt $0 + \tilde 0 = 0 \Rightarrow 0 = 0 + \tilde 0 = \tilde 0 + 0 = \tilde 0$, also $0 = \tilde 0$. \textit{Aufgabe: Beweise die Eindeutigkeit von 1, $-a$...)}

\item \textbf{Behauptung:} $a \cdot 0 = 0 \ \forall a \in \MdR$

\textbf{Beweis:} Sei $a\in\MdR$ und $b := a \cdot 0$. Dann $b = a(0+0) = a \cdot 0 + a \cdot 0 = b$. Aus \textbf{(A7)} folgt dann $0 = b + (-b) = (b+b)+(-b) = b + (b+ (-b)) = b + 0  =b$.

\item

\textbf{Behauptung:} Aus $ab= 0$ folgt $a = 0$ oder $b=0$. \textit{Beweis zur �bung}

\end{beispiele}

\begin{schreibweisen}
F�r $a,b \in\MdR: a - b := a+ (-b)$; ist $b \neq 0: \frac{a}{b} := ab^{-1}$.
\end{schreibweisen}

\indexlabel{Anordnungsaxiome}\begin{axiom}[Anordnungsaxiome]
In \MdR\ ist eine Relation \glqq$\le$\grqq\ gegeben. Es sollen gelten:
\begin{liste}
\item[(A10)] f�r $a,b\in\MdR$ gilt $a\le b$ oder $b \le a$.
\item[(A11)] aus $a \le b$ und $b \le a $ folgt $a = b$.
\item[(A12)] aus $a \le b$ und $b \le c $ folgt $a \le c$.
\item[(A13)] aus $a \le b$ und $c \in \MdR$ folgt $a+c \le b+c$.
\item[(A14)] aus $a \le b$ und $0 \le c$ folgt $ac \le bc$.
\end{liste}
\end{axiom}

\textbf{Alle} Regeln f�r Ungleichungen lassen sich aus \textbf{(A1)} bis \textbf{(A14)} herleiten. Diese Regeln seinen von nun an bekannt.

\begin{schreibweisen}
\begin{liste}
\item $a < b :\Leftrightarrow a \le b $ und $a \ne b$
\item $a > b :\Leftrightarrow b < a$
\item $a \ge b :\Leftrightarrow b \le a$
\end{liste}
\end{schreibweisen}

\begin{definition}[Betrag]
F�r $a \in \MdR$ hei�t $ |a| := 
\begin{cases}
 a & \mbox{falls } a \ge 0 \\
-a & \mbox{falls } a < 0
\end{cases} $. 

$|a|$ wird der \begriff{Betrag} von a genannt und entspricht dem \glqq Abstand\grqq\ von $a$ und $0$. $|a-b|$ entspricht dem \glqq Abstand\grqq\ von $a$ und $b$.
\end{definition}

\indexlabel{Betragss�tze}\begin{satz}[Betragss�tze]
\begin{liste}
\item $|a| \ge 0 \ \forall a \in \MdR; |a|= 0 \Leftrightarrow a = 0$
\item $|a-b| = |b-a| \ \forall a,b \in \MdR$
\item $|ab|  = |a| \cdot |b| \ \forall a,b, \in \MdR$
\item $\pm a \le |a|$
\item $|a+b| \le |a|+|b| \ \forall a,b \in \MdR$
\item $||a| - |b|| \le |a - b| \ \forall a,b \in \MdR$
\end{liste}
\end{satz}

\begin{beweise}
\setcounter{enumi}{4}
\item Fall 1: $a+b \ge 0 \Leftrightarrow |a+b| = a+b \le |a| + |b|$ \\
Fall 2: $a+b  <  0 \Leftrightarrow |a+b| = -(a+b) = - a + (-b) \le |a| + |b|$ \\
\item $|a| = |(a-b) + b| \le |a-b| + |b| \Rightarrow |a| - |b| \le |a-b|$, analog $|b|-|a| \le |b-a| = |a-b|$.
\end{beweise}

\indexlabel{Intervall}\begin{definition}[Intervall]
Seien $a,b \in \MdR$, $a<b$:

\begin{liste}
\item $(a,b) := \{ x \in \MdR: a < x < b\} $: \begriff{offenes Intervall}
\item $[a,b] := \{ x \in \MdR: a \le x \le b\} $: \begriff{abgeschlossenes Intervall}
\item $(a,b] := \{ x \in \MdR: a < x \le b\} $: \begriff{halboffenes Intervall}
\item $[a, \infty) := \{ x \in \MdR: a \le x \}$
\end{liste}

Entsprechend: $[a,b), (-\infty, a], (a, \infty), (-\infty, a), (-\infty, \infty) := \MdR$.
\end{definition}

\begin{definition}[Beschr�nkte Menge]

Es sei $\emptyset \ne M \subseteq \MdR$. $M$ hei�t nach oben (\textit{unten}) beschr�nkt genau dann, wenn es ein $\gamma \in \MdR$, so dass alle $x \in M$ kleiner gleich \alt{gr��er gleich} $\gamma$ sind. In diesem Fall hei�t $\gamma$ \begriff{obere Schranke} (OS) \alt{\begriff{untere Schranke} (US)} von $M$.

Ist $\gamma$ eine OS \alt{US} von $M$ und gilt $\gamma \le \tilde\gamma$ ($\gamma \ge \tilde\gamma$) f�r jede weitere OS \alt{US} $\tilde\gamma$ von $M$, so hei�t $\gamma$ das \begriff{Supremum} \alt{\begriff{Infimum}} von $M$ und man schreibt $\gamma = \sup{M}$ \alt{$\gamma = \inf{M}$}.

Ist $\gamma = \sup{M} \in M$ ($\gamma = \inf{M} \in M$), so hei�t $\gamma$ das \begriff{Maximum} \alt{\begriff{Minimum}} von $M$: $\gamma = \max{M}$ \alt{$\gamma = \min{M}$}.
\end{definition}

\begin{beispiele}
\item aus $M = (1,2)$ folgt: $2 = \sup{M}$, $M$ hat kein Maximum
\item aus $M = (1,2]$ folgt: $2 = \sup{M} = \max{M}$
\item aus $M = [3,\infty)$ folgt: $M$ ist nicht nach oben beschr�nkt, $3 = \inf{M}$
\end{beispiele}

\indexlabel{Vollst�ndigkeitsaxiom}\begin{axiom}[Vollst"andigkeitsaxiom]
\begin{description}
\item[(A15)] Ist $\emptyset \ne M \subseteq \MdR$ und ist $M$ nach oben beschr�nkt, so existiert $\sup{M}$.
\end{description}
\end{axiom}

\textbf{Anmerkung:} $M = \{x\in\MdQ: x > 0, x^2 < 2\}$ hat kein Supremum in $\MdQ$, also sind die rationalen Zahlen keine Menge, die unsere Anforderungen an die reellen Zahlen erf�llt.

\begin{satz}[Vollst�ndigkeit von \MdR\ bez�glich dem Infimum]
Sei $ \emptyset \ne M \subseteq \MdR$ und sei $M$ nach unten beschr�nkt, dann existiert $\inf{M}$
\end{satz}

\begin{beweis} Sei $\tilde M := \{ -x : x\in M\}$. Sei $\gamma$ eine untere Schranke von $M$. d.h. $\gamma \le x \ \forall x \in M \folgt -x \le -\gamma \ \forall x \in M \folgt \tilde M $ ist nach oben beschr�nkt, $-\gamma$ ist eine obere Schranke von $\tilde M$. \textbf{(A15)} $\folgt \exists s := \sup{\tilde M} \folgt s \le - \gamma$. $-x \le s \ \forall x \in M \folgt -s \le x \ \forall x \in M \folgt -s $ ist eine untere Schranke von $M$. Aus $s \le -\gamma \folgt \gamma \le -s$, daher ist $-s = \inf{M}$.
\end{beweis}

\begin{satz}[Existenz des Supremum]
Sei $\emptyset \ne M \subseteq \MdR$, $M$ sei nach oben beschr�nkt, $\gamma$ sei eine obere Schranke von $M$.
\[ \gamma = \sup{M} \equizu \ \forall\varepsilon > 0 \ \exists x \in M: x > \gamma - \varepsilon \]
\end{satz}

\begin{beweis} \glqq$\folgt$\grqq: Sei $\gamma = sup{M}$ und $\varepsilon > 0 \folgt \gamma - \varepsilon$ ist keine obere Schranke von $M \folgt \ \exists x\in M: x > \gamma - \varepsilon$. \\
 \glqq$\Leftarrow$\grqq: \textbf{(A15)} $\folgt \ \exists s = \sup{M}$. Annahme: $\gamma \ne s \folgt s < \gamma \folgt \varepsilon = \gamma - s > 0$. Laut Vorausetzung gilt: $\exists x \in M: x > \gamma - \varepsilon = \gamma - (\gamma - s) = s$, Widerspruch zu $x \le s$.
\end{beweis}

Analog gilt: Sei $\emptyset \ne M \subseteq \MdR$, $M$ sei nach unten beschr�nkt, $\gamma$ sei eine untere Schranke von $M$.
\[ \gamma = \inf{M} \equizu \ \forall\varepsilon > 0 \ \exists x \in M: x < \gamma + \varepsilon \]

\begin{definition}[Beschr"anktheit von Mengen]
Sei $\emptyset \ne M \subseteq \MdR$. $M$ hei�t \begriff{beschr�nkt}: $\equizu$ $M$ ist nach oben und nach unten beschr�nkt $\equizu \ \exists c > 0: |x| \le c \ \forall x\in M$. \textit{Beweis als �bung}
\end{definition}
\end{document}
