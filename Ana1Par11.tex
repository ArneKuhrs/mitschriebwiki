\documentclass{article}
\newcounter{chapter}
\setcounter{chapter}{11}
\usepackage{ana}
\title{Unendliche Reihen}
\author{Pascal Maillard}

\begin{document}
\maketitle

\begin{definition}
Sei $(a_n)$ eine Folge in $\MdR$. Die Folge $(s_n)$ mit $s_n := a_1+a_2+\ldots+a_n\quad(n\in\MdN)$ heißt \begriff{(unendliche) Reihe} und wird mit $\areihe$ bezeichnet (oder mit $a_1+a_2+\ldots+a_n$).\\
$s_n$ heißt die \textbf{n-te \begriff{Teilsumme}} von $\areihe$ und $a_n$ heißt \textbf{n-tes \begriff{Reihenglied}} von $\areihe$.\\
$\areihe$ heißt \begriff{konvergent} \alt{\begriff{divergent}} $:\equizu (s_n)$ konvergiert \alt{divergiert}.\\
Ist $\areihe$ konvergent, so heißt $\lim_{n\to\infty}{s_n}$ der \begriff{Reihenwert} oder die \begriff{Reihensumme} und wird mit $\areihe$ bezeichnet (Im Konvergenzfall hat also das Symbol $\areihe$ \emph{zwei} Bedeutungen).
\end{definition}

\begin{bemerkung}
\begin{liste}
\item $\displaystyle{\areihe = \sum_{i=1}^{\infty}{a_i} = \sum_{k=1}^{\infty}{a_k}}$
\item Sei $p\in\MdZ$ und $(a_n)_{n \ge p}$ eine Folge. Dann definiert man entsprechend $s_n := a_p+a_{p+1}+\ldots+a_n\quad(n \ge p)$ und $\sum_{n=p}^{\infty}{a_n}$. Meist gilt: $p=1$ oder $p=0$.
\end{liste}
\end{bemerkung}

\begin{beispiele}
\item Die \begriff{harmonische Reihe} $\displaystyle{\reihe{\frac{1}{n}}}$ :\\
$a_n = \frac{1}{n}, s_n = 1+\frac{1}{2}+\frac{1}{3}+\ldots+\frac{1}{n} \folgtnach{10.2} (s_n) \text{ divergiert.}\\
\text{Also: }\displaystyle\reihe{\frac{1}{n}}\text{ divergiert.}$

\item Die \begriff{geometrische Reihe} $\displaystyle{\reihenull{x^n}}\quad(x\in\MdR)$ :\\
$\folgtnach{7.3} (s_n)\text{ konvergiert} \equizu |x| < 1 \text{. In diesem Fall: } s_n \to \frac{1}{1-x}\quad(n\to\infty).\\
\text{Also: } \displaystyle\reihenull{x^n} \text{ ist konvergent } \equizu |x| < 1 \text{. In diesem Fall: } \displaystyle\reihenull{x^n} = \frac{1}{1-x}$

\item $\reihenull{\frac{1}{n!}}$. §7 $\folgt \reihenull{\frac{1}{n!}}$ ist konvergent und $\reihenull{\frac{1}{n!}} = e$.

\item $\reihe{\frac{1}{n(n+1)}}$, $a_n=\frac{1}{n(n+1)} = \frac{1}{n} - \frac{1}{n+1} \folgt s_n = (1-\frac{1}{2})+(\frac{1}{2} - \frac{1}{3})+\ldots+(\frac{1}{n-1}-\frac{1}{n}) + (\frac{1}n - \frac{1}{n+1}) = 1- \frac{1}{n+1} \to 1 \ (n\to\infty)$. $\reihe{\frac{1}{n(n+1)}}$ ist konvergent, $\reihe{\frac{1}{n(n+1)}} = 1$
\item $\MdQ=\{a_1, a_2, \cdots\}$ Sei $\ep>0$.\\ $I_n:=(a_n-\frac{\ep}{2^{n+1}}, a_n+\frac{\ep}{2^{n+1}})$. $a_n \in I_n \ \forall n\in\MdN \folgt \MdQ \subseteq \displaystyle \bigcup_{n=1}^{\infty}I_n$. \\
L"ange von $I_n:=|I_n|=\reihe{|I_n|}=\reihe{\frac{\ep}{2^n}};\ s_n=\frac{\ep}{2}+\frac{\ep}{2^2}+\cdots+\frac{\ep}{2^n}=\frac{\ep}{2}(1+\frac{1}{2}+\cdots+(\frac{1}{2})^{n-1})=\frac{\ep}{2}(\frac{1-(\frac{1}{2})^n}{1-\frac{1}{2}})\to\ep\ (n\to\infty)$
(\emph{Unendliche geometrische Reihe}). D.h. $\reihe{|I_n|}$ ist konvergent und $\reihe{|I_n|}=\ep$. Die Rationalen Zahlen k"onnen so mit abz"ahlbaren Intervallen "uberdeckt werden, dass die Summe der Intervalle beliebig klein ist.
\end{beispiele}

\begin{satz}[Cauchy- und Monotoniekriterium sowie Nullfolgeneigenschaft]
$(a_n)$ sei eine Folge in $\MdR$ und $s_n := a_1+a_2+\ldots+a_n$.
\begin{liste}
\item Cauchy-Kriterium:
$\areihe$ konvergiert $\equizu \forall \ep>0\ \exists n_0 := n_0(\ep)\in\MdN:$\\
$\qquad|\underbrace{\sum_{k=m+1}^n{a_k}}_{=s_n-s_m}| < \ep\ \forall n>m \ge n_0.$

\item Monotoniekriterium: Sind alle $a_n \ge 0$ und ist $(s_n)$ beschränkt, so folgt daraus: $\areihe$ konvergiert.

\item $\areihe$ sei konvergent. Dann:
 \begin{liste}
 \item $a_n \to 0\quad(n\to\infty)$
 \item Für $\nu\in\MdN$ ist $\sum_{n=\nu+1}^{\infty}{a_n} = a_{\nu+1}+a_{\nu+2}+\ldots$ konvergent und für $r_\nu := \sum_{n=\nu+1}^{\infty}{a_n}$ gilt: $r_\nu \to 0\quad(\nu \to \infty)$
 \end{liste}
\end{liste}
\end{satz}

\begin{beweise}
\item Wende Cauchy-Kriterium (10.1) auf $(s_n)$ an.

\item $s_{n+1} = a_1+a_2+\ldots+a_n+a_{n+1} = s_n+a_{n+1} \ge s_n \folgt s_n\text{ ist monoton wachsend } \overset{\text{Vor.}}{\underset{\text{6.3}}{\folgt}} (s_n)$ konvergiert.

\item Sei $s := \lim s_n$, also $\areihe = s.$
 \begin{liste}
 \item $s_n-s_{n-1} = a_n \folgt a_n \to s-s = 0\quad(n\to\infty)$
 \item Für $n \ge \nu+1: \sigma_n := a_{\nu+1}+a_{\nu+2}+\ldots+a_n = s_n-(a_1+\ldots+a_\nu) = s_n-s_\nu\\
 \folgt \sigma_n \to s-s_\nu\quad(n\to\infty)\\
 \folgt \sum_{n=\nu+1}^{\infty}{a_n}\text{ konvergiert und } r_\nu = s-s_\nu\\
 \folgt r_\nu \to 0\quad(\nu\to\infty)$
 \end{liste}
\end{beweise}

\begin{satz}[Rechenregeln bei Reihen]
Seien $\areihe$ und $\reihe{b_n}$ konvergent. Weiter seien $\alpha,\beta \in \MdR$. Dann ist $\reihe{(\alpha a_n+\beta b_n)}$ konvergent und $\reihe{(\alpha a_n+\beta b_n)} = \alpha\areihe+\beta\reihe{b_n}$.
\end{satz}

\begin{beweis}
klar.
\end{beweis}

\begin{definition}
Die Reihe $\areihe$ heißt \begriff{absolut konvergent} $:\equizu \reihe{|a_n|}$ ist konvergent. 
\end{definition}

\begin{satz}[Dreiecksungleichung für Reihen]
Ist $\areihe$ absolut konvergent, so ist $\areihe$ konvergent und
$$|\areihe| \le \reihe{|a_n|}$$
\ 
\end{satz}

\begin{beweis}
Sei $\ep > 0$. Aus der Voraussetzung und Satz 11.1(1) folgt:\\
$\displaystyle{\exists n_0\in\MdN: \sum_{k=m+1}^{n}{|a_k|} < \ep\ \forall n>m \ge n_0}$\\
$\displaystyle{\folgt |\sum_{k=m+1}^{n}{a_k}| \le \sum_{k=m+1}^{n}{|a_k|} < \ep\ \forall n>m \ge n_0}$\\
$\displaystyle{\folgtnach{11.1(1)} \areihe \text{ ist konvergent.}}$

$\displaystyle{s_n := a_1+a_2+\ldots+a_n;\quad\sigma_n := |a_1|+|a_2|+\ldots+|a_n| \folgt |s_n|\le\sigma_n}$\\
$\displaystyle{\folgtwegen{n\to\infty} |\areihe| \le \reihe{|a_n|}}.$
\end{beweis}

\begin{beispiel}
Die \begriff{alternierende Harmonische Reihe} $\reihe{(-1)^{n+1} \frac{1}n}$.\\
Hier: $a_n = (-1)^{n+1}\frac{1}{n}$. $|a_n| = \frac{1}{n} \folgt \reihe{a_n}$ konvergiert nicht absolut.\\
\textbf{Behauptung:} $\reihe{a_n}$ ist konvergent. (Später: $\reihe{a_n} = \log 2$)\\
\textbf{Beweis:} $s_n = a_1 + a_2 + \ldots + a_n$. $s_{2n+2} = s_{2n} + a_{2n+1} + a_{2n+1} = s_n + \underbrace{\frac{1}{2n+1} - \frac{1}{2n+2}}_{>0} \folgt (s_{2n})$ ist monoton wachsend. Analog: $(s_{2n-1})$ ist monoton fallend. 
$s_{2n} = s_{2n-1} + a_{2n} = s_{2n-1} - \frac{1}{2n}$ $(*)$\\
Dann gilt $s_2 \le s_4 \le \ldots \le 2_{2n} \gleichwegen{(*)} s_{2n-1} - \frac{1}{2n} < s_{2n-1} \le \ldots \le s_3 \le s_1 \folgt (s_{2n})$ und $(s_{2n-1})$ sind beschränkt. 6.3 $\folgt$ $(s_{2n})$ und $(s_{2n-1})$ sind konvergent. Aus $(*)$ folgt dann $\lim s_{2n} = \lim s_{2n-1}$. A16 $\folgt$ $(s_n)$ hat genau einen Häufungswert. 9.3 $\folgt$ $(s_n)$ ist konvergent.
\end{beispiel}

\end{document}
