\documentclass{article}
\newcounter{chapter}
\setcounter{chapter}{13}
\usepackage{ana}
\def\gdw{\equizu}
\def\Arg{\text{Arg}}
\def\MdD{\mathbb{D}}
\def\Log{\text{Log}}
\def\Tr{\text{Tr}}
\def\abnC{\ensuremath{[a,b]\to\MdC}}
\def\wegint{\ensuremath{\int\limits_\gamma}}
\def\iint{\ensuremath{\int\limits}}
\def\ie{\rm i}

\title{Isolierte Singularit�ten}
\author{Bernhard Konrad} % Wer nennenswerte �nderungen macht, schreibt euch bei \author dazu

\begin{document}
\maketitle

Vereinbarung: In diesem Paragraphen sei stets $D \subseteq \MdC$ offen, $z_0 \in D, \dot{D} := D \backslash \{z_0\}$ und $f \in H(\dot{D})$.\\
$z_0$ hei"st dann eine \begriff{isolierte Singularit"at} von f.

\begin{definition}
$z_0$ hei"st eine \begriff{hebbare Singularit"at} von f $:\Leftrightarrow \, \exists h\in H(D): h=f$ auf $\dot{D}$. I.d. Fall ist $h$ eindeutig bestimmt und wir sagen kurz: $f \in H(D)$.
\end{definition}

\begin{beispiel}
$D=\MdC, z_0=0$
\[
f(z) = \frac{\sin z}{z} = \frac1z\left(z - \frac{z^3}{3!} + \frac{z^5}{5!} - \cdots + \cdots \right) = \underbrace{1 - \frac{z^2}{3!} + \frac{z^4}{5!} - \cdots + \cdots}_{=:h(z)}
\]
Dann: $h \in H(\MdC). h=f$ auf $\MdC \backslash\{0\}. f$ hat also in $0$ eine hebbare Singularit"at.
\end{beispiel}

%satz 13.1
\begin{satz}[Riemannscher Hebbarkeitssatz]
$f$ hat in $z_0$ eine hebbare Singularit"at $\Leftrightarrow \, \exists \delta > 0: U_{\delta}(z_0) \subseteq D$ und $f$ ist auf $U_{\delta}(z_0)$ beschr"ankt.
\end{satz}

\begin{beweis}
\item[$\Rightarrow:$] klar
\item[$\Leftarrow:$] $M:=sup_{z \in U_{\delta}(z_0)} |f(z)|$. Def: $g: D \rightarrow \MdC$ durch:
\[
g(z) := \begin{cases} (z-z_0)^2f(z) &, z \in \dot{D} \\
0 &, z=z_0 \end{cases}
\]
F"ur $z \in \dot{U}_{\delta}(z_0): \left| \frac{g(z)-g(z_0)}{z-z_0} \right| = \left| \frac{g(z)}{z-z_0} \right| = |f(z)(z-z_0)| \leq M|z-z_0|$\\
$\Rightarrow g$ ist komplex db in $z_0$, also $g \in H(D)$ und $g'(z_0) = 0$.\\
Fall 1: $g = 0$ auf $D$. Dann: $f=0$ auf $\dot{D}$\\
Fall 2: $g \not= 0$ auf $D$. Es ist $g(z_0) = g'(z_0) = 0.$ 11.8 $\Rightarrow \, \exists h \in H(D): g(z) = (z-z_0)^2 h(z) \, \forall z \in D.$ Dann: $h=f$ auf $\dot{D}$.
\end{beweis}

%satz 13.2
\begin{satz}
$z_0$ ist ein \begriff{Pol} von $f : \Leftrightarrow \,\exists m \in \MdN, \, \exists g \in H(D)$ mit:
\[
f(z) = \frac{g(z)}{(z-z_0)^m} \, \forall z \in \dot{D} \mbox{ und } g(z_0) \not= 0.
\]
I. d. Fall ist $m$ eindeutig bestimmt und hei"st die \begriff{Ordnung des Pols} $z_0$ von $f$
\end{satz}

\begin{beweis}
Seien $m,l \in \MdN, g,h \in H(D), g(z_0) \not= 0 \not= h(z_0)$ und $\frac{g(z)}{(z-z_0)^m} = f(z) = \frac{h(z)}{(z-z_0)^l} \, \forall z \in \dot{D}.$\\
Annahme: $m > l$, also $m-l \geq 1. h(z_0) \not= 0. \, \exists \delta > 0: U_{\delta}(z_0) \subseteq D$ und $h(z) \not= 0 \, \forall z \in U_{\delta}(z_0).$ F"ur $z \in \dot{U}_{\delta}(z_0): \frac{g(z)}{h(z)} = (z-z_0)^{m-l} \stackrel{z \rightarrow z_0}{\Rightarrow} g(z_0) = 0$. Wid! Also: $m \leq l$. Analog: $l \leq m$.
\end{beweis}

%satz 13.3
\begin{satz}
Hat $f$ in $z_0$ einen Pol, so gilt: $|f(z)| \rightarrow \infty \, (z \rightarrow z_0)$
\end{satz}

\begin{beweis}
Folgt aus 13.2
\end{beweis}

\begin{beispiele}
\begin{liste}
\item[(1)] $f(z) = \frac1z. f$ hat im Nullpunkt einen einfachen Pol.
\item[(2)] $f(z) = \frac{e^z}{z^{17}}. f$ hat in $0$ einen Pol der Ordnung 17.
\end{liste}
\end{beispiele}

\begin{definition}
$z_0$ hei"st eine \begriff{wesentliche Singularit"at} von $f :\Leftrightarrow z_0$ ist nicht hebbar und kein Pol von $f$.
\end{definition}

\begin{beispiel}
$f(z) = e^{\frac1z} \, (D = \MdC, z_0=0)$\\
$z_n := \frac1n, f(z_n) = e^n \rightarrow \infty (n \rightarrow \infty), z_n \rightarrow 0.$ 13.1 $\Rightarrow 0$ ist nicht hebbar.\\
$w_n := \frac{i}n = - \frac1{in}. |f(w_n)| = | e^{-in} | = 1 \, \forall n \in \MdN, w_n \rightarrow 0.$ 13.3 $\Rightarrow z_0 = 0$ ist kein Pol von $f$. $f$ hat also in $z_0 = 0$ eine wesentliche Singularit"at.
\end{beispiel}

%satz 13.4
\begin{satz}[Satz von Casorati-Weierstra"s]
$f$ habe in $z_0$ eine wesentliche Singularit"at und es sei $\delta > 0$ so, dass $U_{\delta}(z_0) \subseteq D$. Dann:
\[
\overline{f(\dot{U}_{\delta}(z_0))} = \MdC
\]
d.h. ist $b \in \MdC$ und $\varepsilon > 0$, so existiert ein $z \in \dot{U}_{\delta}(z_0): |f(z) - b| < \varepsilon$.
\end{satz}

\begin{beweis}
Sei $b \in \MdC$ und $\varepsilon > 0.$ Ann: $|f(z) - b | \geq \varepsilon \, \forall z \in \dot{U}_{\delta}(z_0). g:= \frac1{f-b}.$ Dann: $g \in H(\dot{U}_{\delta}(z_0))$ und $|g| \leq \frac1{\varepsilon}$ auf $\dot{U}_{\delta}(z_0)$ 13.1 $\Rightarrow g$ hat in $z_0$ eine hebbare Singularit"at. Kurz: $g \in H (U_{\delta}(z_0))$\\
Fall 1: $g(z_0) \not= 0.$ O.B.d.A: $g(z) \not= 0 \, \forall z \in U_{\delta}(z_0). f = \frac1g+b$ auf $\dot{U}_{\delta}(z_0) \Rightarrow f$ hat in $z_0$ eine hebbare Singularit"at.\\
Fall 2: $g(z_0) = 0.$ 11.8 $\Rightarrow \, \exists m \in \MdN, \varphi \in H(U_{\delta}(z_0)): g(z) = (z-z_0)^m \varphi(z) \, \forall z \in U_{\delta}(z_0)$ und $\varphi(z_0) \not= 0$. O.B.d.A: $\varphi(z) \not= 0 \, \forall z \in U_{\delta}(z_0).$ Def: $\Psi: D \rightarrow \MdC$ durch:
\[
\Psi(z) = \begin{cases}
\frac1{\varphi(z)} &, z \in U_{\delta}(z_0)\\
(z-z_0)^m(f(z)-b) &,z \in \dot{D}
\end{cases}
\]
$\Psi$ ist wohldefiniert: F"ur $z \in \dot{U}_{\delta}(z_0): \frac1{\varphi(z)} = \frac{(z-z_0)^m}{g(z)}(f(z)-b)$.
Dann: $\Psi \in H(D)$ und $\Psi(z_0) = \frac1{\varphi(z_0)} \not= 0.$\\
$g(z) := \Psi(z) + b(z-z_0)^m \, (z \in D).$ Klar: $g \in H(D)$\\
$g(z_0) = \Psi(z_0) \not= 0$.
Weiter: $\frac{g(z)}{(z-z_0)^m} = \frac{\Phi(z)}{(z-z_0)^m}+b = f(z) - b + b = f(z) \, \forall z \in \dot{D} \stackrel{13.2}{\Rightarrow} f$ hat in $z_0$ einen Pol. Wid!
\end{beweis}

%satz 13.5
\begin{satz}[Klassifikation]
Die isolierte Singularit"at $z_0$ von $f$ ist
\begin{liste}
\item hebbar $\Leftrightarrow \, \exists \delta > 0 : U_{\delta}(z_0) \subseteq D$ und $f$ ist auf $U_{\delta}(z_0)$ beschr"ankt.
\item ein Pol von f $\Leftrightarrow |f(z)| \rightarrow \infty (z \rightarrow z_0)$
\item wesentlich $\Leftrightarrow \, \forall \delta > 0$ mit $U_{\delta}(z_0) \subseteq D$ gilt: $\overline{f(\dot{U}_{\delta}(z_0))} = \MdC$
\end{liste}
\end{satz}

\begin{beweis}
\begin{liste}
\item 13.1
\item $\Rightarrow:$ 13.3\\
$\Leftarrow:$ Vor und 13.1 $\Rightarrow z_0$ nicht hebbar. Vor und 13.4 $\Rightarrow z_0$ nicht wesentlich
\item $\Rightarrow:$ 13.4\\
$\Leftarrow:$ Vor und 13.1 $\Rightarrow z_0$ ist nicht hebbar. Vor und 13.3 $\Rightarrow z_0$ ist kein Pol!
\end{liste}
\end{beweis}

\begin{beispiele}
\begin{liste}
\item $f(z) = e^{\frac1z}.$ "Ubung: $f(\dot{U}_{\delta}(0)) = \MdC \backslash \{0\} \ \forall \delta > 0.$
\item $f(z) = \sin \frac1z$. "Ubung: $f(\dot{U}_{\delta}(0)) = \MdC \ \forall \delta > 0.$
\end{liste}
\end{beispiele}

\end{document}
