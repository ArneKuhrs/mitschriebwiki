\documentclass[a4paper, 10pt]{report}

\usepackage{GuGStyle}

\begin{document}

\setcounter{chapter}{1}

\section{Graphen}

\begin{Def}
  Ein \emp{Graph}\index{Graph} $\Gamma$ besteht aus 2 Mengen $E = E(\Gamma)$ (\emp{Ecken}\index{Ecke}) und $K =
  K(\Gamma)$ (\emp{orientierte Kanten}\index{Kante!orientierte}) sowie einer Abbildung $I: K \to E \times
  E, k \mapsto (i(k), t(k))$ und $-: K \to K, k \mapsto \bar{k}$ mit folgenden
  Eigenschaften:
  % I = Inzidenzabbildung
  \begin{enumerate}
    \item $k \not= \bar{k} \; \forall k \in K$
    \item $k = \bar{\bar{k}} \; \forall k \in K$
    \item $i(k) = t(\bar{k}) \; \forall k \in K$
  \end{enumerate}
  (Dann gilt auch $t(\bar{k}) = i(\bar{\bar{k}}) = i(k)$)\\
  Ein Paar $\{k, \bar{k} \} \textrm{ mit } k \in K$ heißt \emp{geometrische Kante}\index{Kante!geometrische}.
\end{Def}

\begin{nnBsp}
\fboxsep20pt
% Einzelner Knoten
\fbox{\xygraph{
!{<0cm,0cm>;<1cm,0cm>:<0cm,1cm>::}
!{(0,0) }*+{\bullet}
}}
% 2 Knoten mit Kante
\fbox{\xygraph{
!{<0cm,0cm>;<1cm,0cm>:<0cm,1cm>::}
!{(0,0) }*+{\bullet}="a"
!{(0,-1) }*+{\bullet}="b"
"a"-"b"^{k}
}}
% mehrere Kanten zwischen 2 Knoten
\fbox{\xygraph{
!{<0cm,0cm>;<1cm,0cm>:<0cm,1cm>::}
!{(0,-1) }*+{\bullet}="a"
!{(2,-1) }*+{\bullet}="b"
"a"-"b"
"a"-@/^1cm/"b"
"a"-@/_1cm/"b"
}}
% Schleifen
\fbox{\xygraph{
!{<0cm,0cm>;<1cm,0cm>:<0cm,1cm>::}
!{(0,-1) }*+{\bullet}="a"
!{(2,-1) }*+{\bullet}="b"
!{(1,-2) }*+{\bullet}="c"
"a"-"b"
"a" -@(lu,ld) "a"
"b" -@(ld,rd) "b"
"b" -@(lu,ru) "b"
"a"-"c"
}}
% Pfad
\fbox{\xygraph{
!{<0cm,0cm>;<1mm,0cm>:<0cm,1cm>::}
!{(3,0) }*+{}="a3"
!{(4,0) }*+{}="a4"
!{(5,0) }*+{}="a5"
!{(6,0) }*+{}="a6"
!{(7,0) }*+{}="a7"
!{(8,0) }*+{}="a8"
!{(9,0) }*+{}="a9"
!{(10,0) }*+{\bullet}="b"
!{(20,0) }*+{\bullet}="c"
!{(30,0) }*+{\bullet}="d"
!{(33,0) }*+{}="e3"
!{(34,0) }*+{}="e4"
!{(35,0) }*+{}="e5"
!{(36,0) }*+{}="e6"
!{(37,0) }*+{}="e7"
"a3"-"a4"
"a5"-"a6"
"a7"-"b"
"b"-"c"
"c"-"d"
"d"-"e3"
"e4"-"e5"
"e6"-"e7"
}}
% Stern
\fbox{\xygraph{
!{<0cm,0cm>;<1cm,0cm>:<0cm,1cm>::}
!{(1,-1) }*+{\bullet}="a"
!{(0,0) }*+{\bullet}="b"
!{(1,0) }*+{\bullet}="c"
!{(2,0) }*+{\bullet}="d"
!{(0,-1) }*+{\bullet}="e"
!{(2,-1) }*+{\bullet}="f"
!{(0,-2) }*+{\bullet}="g"
!{(1,-2) }*+{\bullet}="h"
!{(2,-2) }*+{\bullet}="i"
"a"-"b"
"a"-"c"
"a"-"d"
"a"-"e"
"a"-"f"
"a"-"g"
"a"-"h"
"a"-"i"
}}
\end{nnBsp}

\begin{nnBem}
  $\Gamma$ kann ``topologisiert'' werden.\\
  Geometrische Kanten werden mit $[0,1]$ identifiziert.\\
  Möglich sogar	als allgemeiner Teilraum von $\R^3$, im Allgemeinen aber nicht
  von $\R^2$:
% Netzwerk, dass im R^2 nicht planar ist
\fbox{\xygraph{
!{<0cm,0cm>;<1cm,0cm>:<0cm,1cm>::}
!{(0,0) }*+{\bullet}="a"
!{(1,0) }*+{\bullet}="b"
!{(2,0) }*+{\bullet}="c"
!{(0,1) }*+{\bullet}="d"
!{(1,1) }*+{\bullet}="e"
!{(2,1) }*+{\bullet}="f"
"a"-"d"
"a"-"e"
"a"-"f"
"b"-"d"
"b"-"e"
"b"-"f"
"c"-"d"
"c"-"e"
"c"-"f"
}}
\end{nnBem}

\begin{DefBem}
\begin{enumerate}
  \item Seien $\Gamma \textrm{ und } \Gamma'$ Graphen.
  Ein \emp{Morphismus}\index{Morphismus} $f: \Gamma \to \Gamma'$ ist ein Paar $f = (f_E, f_K)$ von
  Abbildungen $f_E: E(\Gamma) \to E(\Gamma'), \; f_K: K(\Gamma) \to K(\Gamma')$
  mit $I'(f_K(k)) = (f_E(i(k)), f_E(t(k)))  = (f_E \times f_E)(I(k)) \forall k \in K(\Gamma)$ und $f_k(\bar{k}) = \overline{f_k(k)} \; \forall k \in
  K(\Gamma)$.
  \item $f$ heißt \emp{Isomorphismus}\index{Isomorphismus}, wenn es einen Morphismus $g: \Gamma' \to \Gamma$
  gibt mit $f \circ g = \id_{\Gamma'} \textrm{ und } g \circ f = \id_{\Gamma}$.
  \item $f$ ist Isomorphismus $\Leftrightarrow f_E \textrm{ und } f_K$ sind
  bijektiv.
  \item Ein Isomorphismus $f: \Gamma \to \Gamma$ heißt \emp{Automorphismus}\index{Automorphismus}.
\end{enumerate}
\end{DefBem}

% TODO Beweis zu c) ist Übungsaufgabe

\begin{DefBem}
\begin{enumerate}
  \item Ein \emp{Weg}\index{Weg} (der Länge $n \ge 0$) in $\Gamma$ ist eine Folge $w = (k_1,
  \ldots, k_n)$ von Kanten $k_i \in K \textrm{ mit } t(k_i) = i(k_{i+1}) \mbox{
  für }i=1, \ldots, n-1$.\\
  $i(w) \defeqr i(k_1), \; t(w) \defeqr t(k_n)$ heißt Anfangs-, bzw. Endpunkt 
  von $w$.\\ Ist $n=0$, so wird $i(w) = t(w) \in E(\Gamma)$ definiert. (Für jede Ecke ein Weg der Länge 0.)
  \item Sei $P_n = $
  \fbox{\xygraph{
  !{<0cm,0cm>;<1mm,0cm>:<0cm,1cm>::}
  !{(0,0) }*+{\bullet_{0}}="a"
  !{(10,0) }*+{\bullet_{1}}="b"
  !{(14,0) }*+{}="c4"
  !{(15,0) }*+{}="c5"
  !{(16,0) }*+{}="c6"
  !{(17,0) }*+{}="c7"
  !{(18,0) }*+{}="c8"
  !{(19,0) }*+{}="c9"
  !{(25,0) }*+{\bullet_{n-1}}="d"
  !{(37,0) }*+{\bullet_{n}}="e"
  "a"-"b"
  "b"-"c4"
  "c5"-"c6"
  "c7"-"c8"
  "c9"-"d"
  "d"-"e"
  }}
  (fester Graph mit $n+1$ Ecken und $2n$ Kanten).
  Dann ist jeder Weg der Länge $n$ in $\Gamma$ das Bild von $P_n$ unter einem
  Morphismus $P_n \to \Gamma$.
  \item $\Gamma$ heißt \emp{zusammenhängend}\index{Graph!zusammenhängender}, wenn es für alle $x,y \in E(\Gamma)$
  einen Weg $w$ in $\Gamma$ von $x$ nach $y$ gibt mit $i(w) = x \textrm{ und }t(w) = y$.
  \item Ein Weg heißt \emp{stachelfrei}\index{Weg!stachelfreier} (``without backtracking''), wenn $k_{i+1} \not=
  \overline{k_i}$ für $i=1,\ldots, n-1$.
  \item Gibt es in $\Gamma$ einen Weg von $x$ nach $y$, so gibt es auch einen
  stachelfreien Weg von $x$ nach $y$.
  \begin{Bew}
  Sei $w = (k_1, \ldots, k_n)$ und $k_{i+1} = \overline{k_i} \Rightarrow
  i(k_{i+2}) = t(k_{i+1}) = i(\overline{k_i}) = t(k_{i-1}) \Rightarrow w' =
  (k_1, \ldots, k_{i-1}, k_{i+2}, \ldots, k_n)$ ist Weg mit $i(w') = i(w), \;
  t(w') = t(w)$.
  \end{Bew}
\end{enumerate}  
\end{DefBem}

\begin{DefBem}
\begin{enumerate}
  \item Ein Weg $w \textrm{ in } \Gamma$ heißt \emp{geschlossen}\index{Weg!geschlossener}, wenn $t(w) = i(w)$.
  \item $w$ heißt \emp{einfach}\index{Weg!einfacher}, wenn $i(k_i) \not= i(k_j)$ für $i \not= j$.
  \item Einfache Wege sind stachelfrei \fbox{\xygraph{
  !{<0cm,0cm>;<1cm,0cm>:<0cm,1cm>::}
  !{(0,0) }*+{\bullet}="a"
  !{(2,0) }*+{\bullet}="b"
  "a":"b"
  "b":"a"
  }}\\
  (für Wege der Länge 2 ist das eine neue Definition!).
  \item Ein einfacher geschlossener Weg der Länge $n \ge 1$ heißt \emp{Kreis}\index{Kreis}.\\
  Ein Kreis der Länge 1 heißt \emp{Schleife}\index{Schleife} (loop).\\
  \fboxsep20pt
  % Schleife
  \fbox{\xygraph{
  !{<0cm,0cm>;<1cm,0cm>:<0cm,1cm>::}
  !{(1,0) }*+{\bullet}="a"
  "a"-@(ld,rd) "a"
  }}
  % Kreis mit 2 Knoten
  \fbox{\xygraph{
  !{<0cm,0cm>;<1cm,0cm>:<0cm,0.5cm>::}
  !{(0,-1) }*+{\bullet}="a"
  !{(2,-1) }*+{\bullet}="b"
  "a"-@/^0.5cm/"b"
  "a"-@/_0.5cm/"b"
  }}
\end{enumerate}
\end{DefBem}

\begin{DefBem}
\begin{enumerate}
  \item Ein Paar $k_1 \not= k_2$ von Kanten in $\Gamma$ heißt \emp{Doppelkante}\index{Kante!Doppel}, wenn
  $i(k_1) = i(k_2)$ und $t(k_1) = t(k_2)$ ist.
  % Doppelkante
  \fbox{\xygraph{
  !{<0cm,0cm>;<1cm,0cm>:<0cm,0.5cm>::}
  !{(0,-1) }*+{\bullet}="a"
  !{(2,-1) }*+{\bullet}="b"
  "a":@/^0.5cm/"b" _{k_1}
  "a":@/_0.5cm/"b" ^{k_2}
  }}
  \item Ein Graph heißt \emp{kombinatorisch}\index{Graph!kombinatorischer}, wenn er keine Schleifen und keine
  Doppelkanten enthält (also keine Kreise der Länge $\le 2$).
  \item Ein Graph ist genau dann kombinatorisch, wenn er als topologischer Raum
  ein Simplizialkomplex ist.\\
  ($n$-Simplex $=$ konvexe Hülle der Einheitsvektoren im $\R^{n+1}$)
\end{enumerate}
\end{DefBem}

\begin{DefBem}
Sei $\Gamma$ ein zusammenhängender Graph.\\
Für $x,y \in E(\Gamma)$ sei
$$d(x,y) \defeqr \min\{n: \textrm{ es gibt einen Weg der Länge } n \textrm{ von 
} x \textrm{ nach } y \}$$
$d$ ist eine Metrik auf $\Gamma$ (eigentlich auf $E(\Gamma)$).\\
$d(\Gamma) \defeqr \sup\{d(x,y): x,y \in E(\Gamma) \}$ heißt \emp{Durchmesser}\index{Durchmesser} von $\Gamma$.
\end{DefBem}

\end{document}