\documentclass[a4paper,twoside,DIV15,BCOR12mm]{scrbook}
\usepackage{mathe}
\usepackage{saetze-schmoeger}

\author{A. R.}
\title{Übung zur Linearen Algebra II}

% \usepackage[T1]{fontenc}
% \usepackage[latin1]{\inputenc}
% \usepackage{a4wide}
% \usepackage{ngerman}
% \usepackage{amssymb}
% 
\usepackage{enumerate}
% \usepackage{latexsym}
% \usepackage[fleqn,intlimits]{amsmath}
% \usepackage{stmaryrd}
% \usepackage{amsthm}
% \usepackage{color}
% \usepackage{makeidx}
% 
\newcommand{\enue}{\ \begin{enumerate}[(1)]}
\newcommand{\enur}{\ \begin{enumerate}[(i)]}
\newcommand{\enua}{\ \begin{enumerate}[a)]}
\newcommand{\une}{\end{enumerate}}


\renewcommand{\K}{\ensuremath{\mathbb{K}}}
\renewcommand{\F}{\ensuremath{\mathbb{F}}}

\newcommand{\LA}{\Leftarrow}
\newcommand{\RA}{\Rightarrow}
\newcommand{\longra}{\longrightarrow}
\newcommand{\longRA}{\Longrightarrow}
\newcommand{\LRA}{\Leftrightarrow}
\newcommand{\LLRA}{\Longleftrightarrow}
\newcommand{\x}{\cdot}
\newcommand{\error}{\lightning}
\newcommand{\vect}{\begin{pmatrix}}
\newcommand{\tcev}{\end{pmatrix}}
\newcommand{\trans}{^\top}
\providecommand{\matr}[1]{\begin{pmatrix}#1\end{pmatrix}}
\providecommand{\det}[1]{\begin{vmatrix}#1\end{vmatrix}}

\newenvironment{bew}{\pagebreak[2]\textbf{Beweis: }}{\qed}
\renewcommand{\qed}{\hspace*{\fill} \ensuremath{\square}}

\sloppy
\frenchspacing
\parskip=10pt
\parindent=0pt
\setcounter{tocdepth}{1}

\setcounter{section}{-1}

\begin{document}
\maketitle
\tableofcontents
\newpage


\section{Übung 0, 01.11.2004}

\subsection{Aufgabe 2}

\enua
\item
Über allen Gipfeln ist Ruh\\
Über einem Gifpel ist keine Ruh

\item
Es gibt einen Hund der M"ohren fri"st\\
Alle Hunde fressen keine M"ohren

\item
Es gibt einen Topf auf den alle Deckel passen\\
F"ur alle T"opfe gibt es einen Deckel der nicht passt

\item$\forall x \in\MdR\exists y \in \MdZ: [y \le x \wedge (\forall z \in \MdZ: z \le y)]\LRA\exists x \in\MdR\forall y\in \MdZ:\neg[y \le x \wedge (\forall z \in \MdZ: z \le y)]$\par
Es gilt: $\neg[A\wedge B]\LRA\neg A\vee\neg B$\\
$\RA\quad\exists x\in\MdR\forall y\in\MdZ:[y>x\vee(\exists z\in\MdZ:z>y)]$
\une


\subsection {Aufgabe 3}
$a,b \in \MdN$\\
$a\mid b \in \MdN:\LRA\exists c\in\MdN: a\x c=b$\\
$p \in \MdN\setminus\{1\}:\LRA $1 und p sind die einzigen Teiler von p\par
Satz: Zu jeder nat"urlichen Zahl $n\neq 1$ gibt es eindeutig bestimmte Primzahlen $p_1,...,p_k$ und $l_1,...,l_k \in\MdN$, so dass $n={p_1}^{l_1}\x...\x {p_k}^{l_k}$.

\enua
\item Die Anzahl aller Primzahlen ist unendlich\\
Widerspruchsbew.: Ann.: Es gibt nur endl. viele Primzahlen
\[\left.\begin{array}{rcc}
  n:=p_1\x...\x p_k+1\\
  Satz \RA P_l\x m=n
\end{array}\right\} p_l*m = p_1\x...\x p_k1+1 \RA p_L\x (m-p_1\x...\x p_l \x...\x p_k)=1\error\]
Es gibt also doch unendlich viele Primzahlen

\item
Widerspruchsbew.: Ann.: $\sqrt{2} \in\MdQ$, d.h. $\sqrt{2}=\frac{m}{n}$ mit $m,n\in\MdN$. Wir d"urfen annehmen, das es keine Primzahl gibt, die sowohl m als auch n teil.
\[\sqrt{2}=\frac{m}{n}\RA 2=\frac{m^2}{n^2} \RA 2n^2=m^2\]\[ \RA 2\tilde{m}=m \RA 2n^2=4\tilde{m}^2 \RA n^2=2\x\tilde{m}^2 \RA 2\mid n \error\]
Also $\sqrt{2} \not\in\MdQ$
\une

\subsection{Aufgabe 5}
z.Z.: $A \setminus (B\setminus C) = (A \setminus B) \cup (A \cap C)$\\
\begin{bew}\begin{align*}
&x \in A \setminus (B \setminus C)\\
\LRA& x\in A, x \not\in B\setminus C\\
\LRA& x\in A, \neg(x \in B \setminus C)\\
\LRA& x\in A, \neg(x \in B, x \not\in C)\\
\LRA& (x \in A, x \not\in B) \vee (x \in A, x \in C)\\
\LRA& x \in A\setminus B \cup (A \cap C)
\end{align*}\end{bew}


\section{Übung 1, 08.11.2004}
\subsection{Aufgabe 1}

\enua
\item[d)]
M,N Mengen, $f: M\to N$ Abb., $A,B\subset M$ \\
z.Z. $f(A) \setminus f(B) \subset f(A \setminus B)$\\
\begin{bew}
F"ur $f(A)\setminus f(B) = \emptyset$ gilt die Behauptung.\\
Sei daher im folgenden $f(A)\setminus f(B) \neq \emptyset$\\
%FIXME: Müsste man mal schön 
\begin{eqnarray*}\mbox{Sei }y \in f(A) \setminus f(B) & \LRA & \exists x\in A: f(x)=y \wedge\forall x'\neq y\\
& \RA & \exists x\in A \setminus B: f(x)=y\\
& \LRA & y\in f(A\setminus B)\\
\mbox{Also gilt: }f(A)\setminus f(B) \subset f(A\setminus B)
\end{eqnarray*}
\end{bew}
\une

\subsection{Aufgabe 3}
\enua
\item[b)]
A endliche Menge, $|A| =n$\\
Gesucht: Eine Abb. von $\mathcal{P}(A)$ nach $\{0,1\}^A$.\\
Wie definieren wir $f$ ?
\begin{eqnarray*}
f: \mathcal{P}(A)\to & \{0,1\}^A & \\
M\to & h_M: & A\to\{0,1\}\\
& & x\to \left\{\begin{array}{rcc}1, x\in M\\0, x\not\in M\end{array}\right.
\end{eqnarray*}
\begin{eqnarray*}
f^{-1}:\{0,1\}\to & \mathcal{P}(A)\\
h \to & h^{-1}(\{1\})
\end{eqnarray*}
Was man "`leicht"' sieht: $f^{-1}\circ f = id_{f(A)}$ und $f\circ f^{-1}=id_{\{0,1\}^A}$.
\une 

\section{Übung 1, 08.11.2004}

\subsection {Aufgabe 1}
\enua
\item[b)]
$A\neq\emptyset$ eine Menge, $(G,\circ)$ eine Gruppe.\\
z.Z.: $(G^A,\ast)$ ist Gruppe.\\
\begin{bew}$G^A\neq\emptyset$, da $A\neq\emptyset$
\enue
  \item $\ast$ is assoziativ\\
    Sei $x\in A, f,g,h \in G^A$\\
    $((f\ast g)\ast h)(x)=(f\ast g)(x)\circ h(x)=((f(x)\circ g(x))\circ h(x)=f(x)\circ g(x)\circ h(x)=(f\ast(g\ast h))(x)$\par
    Da $x\in A$ bel. gilt $(f\ast g)\ast h=f\ast(g\ast h)$
  \item neutrales Element\\
    Wir def. $f: A\to G,\; x \mapsto e$. Sei $x \in A$ bel., dann gilt:\\
    $f(n(x))=f(x)$. Also ist $n$ das neutrale Element in $G^A$.
\une
\end{bew}
\une

\subsection {Aufgabe 3}
%FIXME: Muss man seine eigene Handschrift eigentlich lösen können, bei mir steht nen Delta mit nem Strich drunter
%       soll das so ? Dann ersetzte man die beiden Fragezeichen entsprechend.
Im Hinterkopf: $A=\MdN_0$; $\circ$ ?? $+$ und $(B,\ast)=(\MdZ,+)$
\enua
\item
z.Z.: $\sim$ ist ÄR\\
\begin{bew}
\enue
  \item $\sim$ ist reflexiv: $x_1\circ y_1=x_1\circ y_2 \RA (x_1,y_1)\sim(x_1,y_1)$
  \item $\sim$ ist symmetrisch da "`$=$"' symmetrisch ist
  \item $\sim$ ist transitiv:\\ $(x_1,y_1)\sim(x_2,y_2)$ und $(x_2,y_2)\sim(x_3,y_3)$\\
  Es gilt also: $x_1\circ y_2 = x_2 \circ y_2$ und $x_2 \circ y_1 = x_3 \circ y_2
  \RA x_1\circ y_2\circ x_2\circ y_3=x_2\circ y_1\circ x_3\circ y_2 \RA x_1\circ y_3=y_1\circ x_3$
\une
\end{bew}

\item
  Seien $(x_1,y_1)\sim(x_2,y_2)$ und $(x_2,y_2)\sim(x_2,y_2)$\\
  z.Z.: $(x_1\circ x_2, y_1\circ y_2)\sim(x_1'\circ x_2', y_1'\circ y_2')$ (dann wohldefiniert.)\\
  $x_1\circ x_2\circ y_1'\circ y_2'=x_1'\circ y_1\circ x_2'\circ y_2=x_1'\circ x_2'\circ y_1\circ y_2$

\item
$A\times A_{/\sim}\neq0$\\
$[(x_1,y_1)]_\sim\times[(x_2,y_2)]_\sim=[(x_1\circ x_2,y_1\circ y_2)]_\sim$
\enue
  \item $\ast$ ist assoziativ weil $\circ$ assoziativ ist
  \item $[(e,e)]_\sim$ ist das neutrale Element bzgl. $\ast$
  \item $\ast$ ist kommutativ\\
  Sei $x_1,y_1 \in A, [(y_1,x_1)]_\sim$ ist invers zu $[(x_1,y_1)]_\sim$, denn\\ $[(x_1,y_1)]_\sim\ast[(y_1,x_1)]_\sim=[(x_1\circ y_1,x_1\circ y_1)]_\sim=[(e,e)]_\sim$, denn $x_1\circ y_1\circ e=e\circ x_1\circ y_1$
\une

\item
  $f: A\to B,\; x \mapsto [(x,e)]_\sim$ ist die gesuchte Abb.
\une


\section{Übung 2, 15.11.2004}
\subsection{Aufgabe 1}
$(A,\circ)$ endliche Gruppe, $e$ neutr. Element; $x\in A$ fest
\enua
\item
z.Z.: Es gibt ein kleinstes $k\in\MdN$ mit $x^k=e$.\\
\begin{bew}
$B:=\{x,x^2,...,x^n,x^{n+1}\}\subset A$ mit $|A|=n$
\begin{align*}
&\RA |B|\le n\\
&\RA\exists i,j\in \{1,...,n+1\}: i<j\quad\mbox{und}\quad x^i=x^j=x^i\circ x^{j-1}\\
&\RA x^{j-1}=e\quad\mbox{und}\quad 1\le j-1\le n
\end{align*}
Damit ist $M:=\{m\in N|x^m=e\}\neq\emptyset$. Da $(j-i)\in M$ ex. au"serdem $k=\mbox{min }M$.
\end{bew}

\item
z.Z.: $B:=\{x,x^2,...,x^n\}$ ist eine Untergruppe von A\\
\begin{bew}\begin{itemize}
\item $B\neq\emptyset$, da $x\in B$
\item Seien $y,z\in B$. Dann $\exists i,j\in\{1,...,k\}$ mit $y=x^i$ und $z=x^j$\\
  $y\circ z^{-1}=x^i\circ x^{k-j}=x^{i+k-j}=
  \left\{
  \begin{array}{rcc}
    x^{k+i-j}\mbox{, falls }i\le j\\
    x^{i-j}\mbox{, falls }i>j
  \end{array}\right.$\\
  In beiden F"allen $y\circ z^{-1}\in B$.
\item Seien $y,z\in B$. Dann $\exists i,j\in \{1,...,k\}, y=x^i, z=x^j$\\
  $y\circ z=x^i\circ x^j=x^{i+j}=x^{j+i}=z\circ y$\par
\end{itemize}\end{bew}\\
z.Z.: Ann.: $|B|<k$. Dann $\exists i,j\in \{1,...,k\}$ mit $i<j$ und $x^i=x^j$
\[\RA x^{j-1}=e\quad\mbox{und}\quad 1\le j-1<k\error\]
\[\RA |B|=k\]
\une

\subsection{Aufgabe 3}
\enua
\item
z.Z.: Die Menge $\{Ba|a\in A\}$ (mit $Ba=\{b\circ a|b\in B\}$) bildet eine Partition von A.\par
\begin{bew}F"ur alle $a\in A$ gilt $a\in Ba$, da $e\in B$ und somit $e\circ a\in Ba$ ist.\\
Damit gilt: $Ba\neq\emptyset$ f"ur alle $a\in A$ und $\bigcap\limits_{a\in A}{Ba}=a$.\\
Sei $a,a'\in A$ und $Ba\cap Ba'\neq\emptyset$. Dann ex. $Ba\cap Ba'$ und es gibt $b,b'\in B$ mit $x=b\circ a$ und $x=b'\circ a' \RA a=\underbrace{b^{-1}\circ b'}_{\in B}\circ\;a'$ und $a'=\underbrace{b'^{-1}\circ b}{\in B}\circ\;a$\\
$\RA a\in Ba'$ und $a'\in Ba$\\
$\RA Ba\subset Ba'\RA Ba=Ba'$.
\end{bew}

\item
z.Z.: $|B|$ teilt $|A|$\\
\begin{bew}\enue
\item Wir zeigen: $|Ba|=|B|$ f"ur alle $a\in A$\par
$h:B\to Ba, h\mapsto b\circ a$ ist bijektiv denn $h^{-1}:Ba\to B, x\mapsto x\circ a^{-1}$ ist ihre Umkehrabbildung.\\
$\RA |Ba|=|B|$ f"ur alle $a\in A$
\item z.Z.: $\exists m\in\MdN: m|B|=|A|$\\
Wir zeigen aus a), dass $A=\bigcup\limits_{a\in A}{Ba}$\\
Wir definieren $m:=\{Ba|a\in A\}$ (die Anzahl der verschiedenen Nebenklassen)\\
Dann gilt $|A|=m|B|$.
\une\end{bew}

\item
z.Z. $\forall a\in A: a^{|A|}=e$\\
\begin{bew}Sei $k$ die Ordnung von $a$\\
$B:=\{a,a^2,...,a^k\}$\\
Wir wissen aus Aufgabe 1: $B$ ist Untergruppe von $A$. Dann ex. wegen b) ein $m\in\MdN$ mit $|A|=mk$.\\
Somit: $a^{|A|}=a^{mk}=a^{k^n}=e^m=e$
\end{bew}

\item
z.Z.: $|A|\ge Z: |A|$ ist Primzahl $\LRA\{e\},\quad A$ sind die einzigen Untergruppen von A\\
\begin{bew}\\
"`$\RA$"' Wegen b) gilt für jede Untergrupp B, dass $|B|\text{teilt}|A|$.\\
"`$\LA$"' Wegen $|A|\ge2$ gibt es $x\in A\setminus\{e\}$. Also ist die Ordnung $k$ von $x$ echt gr"o"s"er 1.\\
$\{e\}\subsetneqq\{a,a^k\}=A$ (nach Vor.) und $k=|A|$\\
Falls: $|A|$ keine Primzahl ist, so ex. $m\neq1\neq n$ mit $|A|=mn=e$.\\
$\{a^n,a^{2n},a^{3n},...,a^{nm}\}$ ist Untergruppe von $A$ und $1|\{a^n,...,a^{n^m}\}=m<k$ ist Widerspruch zur Vor.\\
Also gilt: $|A|$ ist Primzahl
\end{bew}
\une

\subsection{Aufgabe 2}
\enua
\item[b)]
\begin{align*}
\pi=&\matr{1&2&3&4&5&6&7&8\\7&3&5&4&8&6&2&1}\\
\LRA\tau^{(1,8)}\circ\pi=&\matr{1&2&3&4&5&6&7&8\\7&3&5&4&1&6&2&8}\\
\LRA\tau^{(2,7)}\circ\tau^{(1,8)}\circ\pi=&\matr{1&2&3&4&5&6&7&8\\2&3&5&4&1&6&7&8}\\
...\\
\LRA id=&\tau^{(1,2)}\circ\tau^{(1,3)}\circ\tau^{(1,5)}\circ\tau^{(2,7)}\circ\tau^{(1,8)}\circ\pi\\
\LRA\pi=&\tau^{(1,8)}\circ\tau^{(2,7)}\circ\tau^{(1,5)}\circ\tau^{(1,3)}\circ\tau^{(1,2)}
\end{align*}
\une

\section {Übung 3, 22.11.2004}
\subsection{Aufgabe 1}
\enua
\item
Sei $f:S_n\to\MdZ_m$ ein (Gruppen-)Homomorphismus\\
z.Z.: $\forall\pi\in\S_n: f(\pi)+f(\pi)=[0]_\sim$\\
\begin{bew} Sei $\tau\in S_n$ eine Transposition. Dann gild $\tau\circ\tau=id$\\
Also: $[0]_\sim=f(id)=f(\tau\circ\tau)=f(\tau)+f(\tau)$\\
Sei: $\pi\in S_n$ bel.\\
Dann existieren Transposition $\tau_1,...,\tau_k\in S_n$ mit $\pi=\tau_1\circ...circ\tau_k$\par
Also: $f(\pi)+f(\pi)+f(\tau_1)+f(\tau_2)+...+f(\tau_k)+f(\pi_1)+f(\pi_2)+...$\\
$f(\tau_1)+f(\tau_1)+f(\tau_2)+f(\tau_2)+...=[0]_\sim$
\end{bew}

\item
Sei $f$ surjektiv. Dann ex. $\pi\in S_n$ mit $f(\pi)=[1]_\sim$. Nach a) $[0]_\sim=f(\pi)+f(\pi)=[1]_\sim+[1]_\sim=[2]_\sim$.\\
Also teilt m 2. Somit ist m=2.
\une

\subsection{Aufgabe 2}
\enua
\item
\begin{eqnarray*}
\mbox{Sei }n\in\MdN\mbox{. }3\mbox{ teilt } n \LRA&[n]_\sim =[0]_\sim\quad & \mbox{in}\MdZ_3\\
\LRA&[\sum\limits_{i=0}^n{a_i\x10^i}]_\sim=[0]_\sim & \mbox{in}\MdZ_3\\
\LRA&\sum\limits_{i=0}^n{[a_i\x10^i]_\sim}=[0]_\sim & \mbox{in}\MdZ_3\\
\LRA&\sum\limits_{i=0}^n{[a_i]_\sim\x[10^i]_\sim}=[0]_\sim & \mbox{in}\MdZ_3\\
\LRA&\sum\limits_{i=0}^n{[a_i]_\sim\x[1]^i_\sim}=[0]_\sim & \mbox{in}\MdZ_3\\
\LRA&\sum\limits_{i=0}^n{a_i}=[0]_\sim & \mbox{in}\MdZ_3\\
\LRA&\mbox{3 teilt }\sum\limits_{i=0}^n{a_i}
\end{eqnarray*}
\item
Analog ($[10]_\sim=[-1]_\sim$ in $\MdZ_{11}$)
\une

\subsection{Aufgabe 3}
\enua
\item
  (Hier fehlen noch ein paar Angaben f"ur die Menge der Einsen in den Klammern)
  Die Charakteristik eines K"orpers $\K$ ist 0, wenn f"ur alle $n\in\MdN$ gilt: $\underbrace{1+...+1}_{n\mbox{-mal}}\neq0$\quad, 0,1 neutr. El.\par
  $(\MdR,+,\x)$ ist ein K"orper mit Char 0, also kann 0 nicht die Char von $\K$ sein.\par
  Sei also $m\in\MdN$, die Char von $\K$\\par
  Wir nehmen: $m$ ist keine Primzahl\\
  Wir wissen:
  \enur
    \item (m-mal)$1+...+1=0$
    \item m ist die kleinste nat. Zahl mit dieser Eigenschaft
    \item $\exists k,l\in\MdN: k>1, l>1$ und $m=k\x l$. $(k<m, l<m)$
  \une
  Aus (i) ergibt sich $(1+...+1)+(1+...+1)+...+(1+...+1)=0$ ($l\x k$-mal 1)\\
  $\LRA (1+...+1)\x(1+...+1)=0$ ($l\x1\x k\x1$)\\
  Wir haben $x=1+...+1\neq0$ (l-mal) und $y=1+...+1\neq0$ (k-mal) gefunden mit $x\x y=0$. Dies ist ein Widerspruch zur Nullteilerfreiheit von $\K$. $\RA$ m mu"s Primzahl sein.
\item
  (Hier fehlen noch ein paar Angaben f"ur die Menge der Einsen in den Klammern)
  Sei p eine Primzahl und p die Char von $\K$. Wir def.: \[f:\F_p\to\K, k\mapsto\left\{\begin{array}{ll}
  \overline0&,k=0\\
  1+...+1&, k\in\{1,...,p-1\}
  \end{array}\right.\]
  Sei $k,k'\in\F_p, k+k'=r$ und $k\x k'=r'$ mit $r,r'\in \{0,...,p-1\}$
  \[f(k)+f(k')=(\overline1+...+\overline1)+(\overline1+...+\overline1)=(\overline1+...+\overline1)+(\overline1+...+\overline1)=(\overline1+...+\overline1)=f(r)=f(k+k')\]
  Analog $f(k)\x f(k')=f(k\x k')$
\une
\subsection{Aufgabe 4}
\enua
\item
\enue
\item $\frac{z_1-z_2-2}{z_1+z_2+3i}=\frac35\sqrt2$; $\frac12(\frac{z_2}{z_3}+\frac{\overline{z_3}}{z_3})=\frac17+i\x0$
\item $\overline{z_1+z_3\x(z_3-z_2)}=(i3-3\sqrt3)+i(12+9\sqrt3)$
\une
\item
\enue
\item
  $z+\overline z = z\x\overline z\LRA za = a^2 + b^2 \LRA (a-1)^1+b^2=1$\par
  Kreislinie eines Kreises um (1,0) mit Radius 1.
\item
  $Re(iz)=-b$, $0<Re(iz)<1$\par Zeichnung...
\item
  $|z-z_i|<3\LRA a^2+(b-2)^2<9$; $3<|z|\LRA 9<a^2+b^2$\par
  Der Teil des Kreises um (0,2) mit Radius 3, der nicht im Inneren des Kreises um (0,0) mit Radius 3 liegt.
\une
\une

\section {Übung 4, 29.11.2004}
Can somebody please tell where the heck I've been ? Oh, wait, now I remember, good old Bremen
\section {Übung 5, 06.12.2004}
\subsection {Aufgabe 1}
Geg: $A\in\MdR^{n\times n}$ mit $a_{ij}=0$, falls $i\ge j$\\
\enua
\item z.Z.: $A^n=0$\\
\begin{bew}$A^k=((a_{ij}^{(k)}))$ Beh.: $a_{ij}^{(k)}=0$, falls $i\ge j-k+1$\\
(Vollst"andige Induktion)\par
I.A.: $A^n=A=((a_{ij}))$. Nach Vor.: $a_{ij}=0$ für $i\ge j-1+1$\\
I.V.: F"ur $A^k=((a_{ij}))$ gilt $a_{ij}^{(k)}=0$ f"ur $i\ge j-k+1$\\
I.S.: z.Z. $A^{k+1}=((a_{ij}^{(k+1)}))$ erf"ullt
\[a_{ij}^{(k+1)}=0\mbox{ f"ur }i\ge(k+1)+1=j-k\]
\textbf{Bew.:} Sei $i_0,j_0\in\{1,...,n\}$ mit $i_0\ge j_0-k=i_0\ge (j_0-1)+k+1$\par
$A^{k+1}=A^K\x A.$ Also gilt:\par
$a_{i_0j_0}^{k+1}=\sum\limits_{l=1}^n{a_{i_0l}a_{lj_0}}$\par
$\underbrace{\underbrace{a_{i_o1}^{(k)}\x a_{1j_0}}_0
+...
+\overbrace{a_{i_0j_0-1}^{(k)}}^{=0}}_{\text{0 nach I.V.}}\x\underbrace{a_{(j_0-1)j_0}
+a_{i_0j_0}^{(k)}\x a_{i_0j_0}^{(k)}\x \overbrace{a_{j_0j_0}}^{=0}
+...
+a_{i_0n}^{(k)}\x \underbrace{a_{nj_0}}_0}_{\text{nach Vor. aus A}}
=0$\par
Damit gilt f"ur $k=n: A^n=((d_{ij}^{(n)}))a_{ij}^{(n)}=0$, falls $i\ge \underbrace{j-n+1}_{\text{Gilt f"ur alle}i,j<\{1,...,n\}}$. Also gilt: $A^n=0$
\end{bew}
\item Wo ist Teil b ?, bzw. was war a ?
\item $P: \MdR\to\MdR, t\mapsto t^3$. Dazu geh"ort dann: $\varphi: \MdR^{n\times n}\to\MdR^{n\times n}, A\mapsto A^3$\par
P ist injektiv aber $A=\begin{pmatrix}
0 & 0 & \cdots & 1\\
\vdots & \ddots & \ddots & 0\\
\vdots & \vdots & \ddots & \vdots\\
0 & 0 & \cdots & 0
\end{pmatrix}$ erf"ullt: $\varphi_n(A)=A^3=0$. Damit ist $\varphi_n$ nicht injektiv.
\une
\subsection{Aufgabe 2}
Gibbet nicht
\subsection{Aufgabe 3}
\enua
\item z.Z.: $P(\pi\circ\sigma)=P(\pi)\x P(\sigma)$\par
\begin{bew}\begin{eqnarray*}
P(\pi)P(\sigma)=((P_{ij}))\\
= P_{ij}=\delta_{i\pi(1)}\x\delta_1\sigma(j)+\delta_{i\pi(2)}\x\delta_2\sigma(j)+...+\delta_{i\pi(n)}\x\delta_n\sigma(j)\\
= \left\{\begin{array}{ll}1, \begin{array}{cl}
&\exists k\in\{1,...,n\}: \delta_{i\pi(k)}=1\text{ und }\delta_{k\pi(j)}=1\\
\LRA&\exists k\in\{1,...,n\}:\pi(k)=i\text{ und }k=\sigma(j)\\
\LRA&\exists k\in\{1,...,n\}: k=\pi^{-1}(i)\text{ und }k=\sigma(j)\\
\LRA&\pi^{-1}=\sigma(j)\\
\LRA& i=\pi(\sigma(j))
\end{array}\\
0, \text{sonst}\end{array}\right.\\
= \delta_{i\pi(\sigma(j))}\end{eqnarray*}
Also gilt: $P(\pi\circ\sigma)=P(\pi)\x P(\sigma)$
\end{bew}
\item z.Z.: $P(\pi^{-1})=P(\pi)\trans$\par
\begin{bew} $P(\pi^{-1})=((\delta_{i\pi}^{-1}(j)$
\[\delta\pi^{-1}(j)=\left\{\begin{array}l \text{1, falls }i=\pi^{-1}(j)\LRA\pi(i)=j\\
\text{0, sonst}\end{array}\right.\]
$P(t)\trans=((\delta_{ij}))\text{ und }P(\pi)=((a_{ij}=((\delta_i\pi(j)))$\par
$\delta_{ij}=a_{ji}=\delta_{j\pi(i)}\delta_{\pi(i)j}$\par
Also gilt: $P(\pi^{-1})=P(\pi)$
\end{bew}
\item\enue\item z.Z.: $P(\pi)\in GL(n,\K)$ f"ur alle $\pi\in S_n$:
\[E_n=P(id_{\{1,...,n\}})=P(\pi^{-1}\circ\pi)=P(\pi^{-1})\x P(\pi)=P(\pi)\x P(\pi^{-1})\]
\item P ist Gruppenhomomorphismus wege a)
\item P injektiv $\LRA$ Kern $P=\{id_{\{1,...,n\}}\}$\par
z.Z.: $P(\pi)=E_n\RA\pi=id_{\{1,...n\}}$\par
\begin{bew} Sei $P(\pi)=E_n=((\delta_{ij}))$
\begin{eqnarray*}
\RA & \forall i,j\in\{1,...,n\}: \delta_{i\pi(j)}=\delta_{ij}\\
\RA & \forall i,j\in\{1,...,n\}: \pi(j)=j\\
\RA & \pi = id_{\{1,...,n\}}
\end{eqnarray*}
\end{bew}
\une
\une
\subsection{Aufgabe 4}
\enua
\item $G=\{A\in GL(4,\MdR)| A\trans JA=J\}$\quad z.Z.: G ist Gruppe\par
\begin{bew}$G\subset GL(4,\MdR)$, d.h. wir k"onnen das UGK anwenden.
\enue
\item $E_4\trans JE_4=J$, als ist $E_4\in G$ und $G\ne\emptyset$
\item Seien $A,B\in G: (A\x B^{-1})\trans J(AB)=B^{-1\top}A\trans JAB^{-1}=(B^{-1})\trans JB^{-1}$. Nach Vors.: $B\trans JB=J\LRA J={(B\trans)}^{-1}JB^{-1}={(B^{-1})}\trans JB^{-1}=AB^{-1}\in G$
\une
Damit ist G eine Gruppe
\end{bew}
\une
\section {Übung 6, 13.12.2004}
\subsection{Aufgabe 1}
\enua
\item Grad $q=:m\quad m=\text{Grad }q=\text{Grad }r_1>\text{Grad }r_2>...>\text{Grad }r_n>\text{Grad }r_{n+1}>\text{Grad }r_{n+2}$\par
Falls kein $n\in\MdN$ ex. mit $k_n=0$, dann gilt: $k_{n+1}$ ex., Grad $k_{n+1}\ge0$\par
Also gibt es $m+2$ verschiedene Elemente in der Menge $\{0,...,m\}$.$\error$
\item Der Eukl.-Algo liefert
\begin{eqnarray*}
r_0&=s_1r_1+r2\\
r_1&=s_2r_2+r3\\
r_2&=s_3r_3+r4\\
&...\\
r_{n-2}&=s_{n-1}r_{n-1}+r_n\\
r_{n-1}&=s_nr_n+0\\
\end{eqnarray*}
Wir zeigen: $r_n$ teilt $r_{n-k}$ f"ur alle $k=0,...,n$\par
\begin{bew}\par I.A.: $r_n$ teilt $r_n=r_{n-0}$; $r_n$ teilt $r_{n-1}$ wg. der letzten Gleichung\\
I.V.: $r_n$ teilt $r_{n-(k-1}$ und $r_n$ teilt $r_{n-i}$\\
I.S.: z.Z.: $r_n$ teilt $r_{n-(k-1}$\par
Wir wissen: $r_{n-(k+1)}=s_{n-k}\x r_{n-k}+r_{n-(k-1)}$\par
Nach I.V.: $\exists l,m\in\K[x]\quad k_{r-k}=l\x r_n$ und $r_{n-(k-1)}=n\x r_n$\par
Damit: $r_{n\x(k-1)}=s_{n-k}lr_n+m\x kn=(s_{n-k}l+m)r_n$, d.h. $r_n$ teilt $r_{n-(k+1)}$\par
Also ist $r_n$ ein Teiler von $r_0=p, r_1=q$
\end{bew}
\item
Ist d ein Teiler von p und q, so teilt d auch $r_k$ f"ur alle $k\in\{0,...,n\}$\par
\begin{bew}\\
IA: d teilt $r_0$, d teilt $r_1$ nach Vor.\\
IV: d teilt $r_{k-1}$ und d teil $r_2$\\
IS: $r_{k-1}=s_kr_k+r_{k+1}$\par
Nach IV: $\exists l,m\in\K[x]: r_{k-1}=d$ und $r_m=md$\\
Damit $r_{k+1}=(l-s_km)d$, d teilt also $r_{k+1}$\\
Insbesondere teilt d also k
\end{bew}
\une
\subsection{Aufgabe 2}
$\underbrace{(x^4+3x^3+2x^2-3)}_{r_0}=\underbrace{(x^3-x)}_{r_1}\underbrace{(x+3)}_{s_1}+\underbrace{(3x^2+3x-3)}_{r_2}$\\
$\underbrace{(x^3-x)}_{r_1}=\underbrace{(3x^2+3x-3)}_{r_2}\underbrace{(\frac13x-\frac13)}_{s_2}+\underbrace{(x-1)}_{r_3}$\\
$\underbrace{(3x^2+3x-3)}_{r_2}=\underbrace{(x-1)}_{r_3}\underbrace{(3x+6)}_{s_3}+\underbrace{3}_{r_4}$\par
$1=\frac13((s_2-s_3)+1)r_0+(-s_1s_2s_3-s_1-s_3)r_1\\
\begin{array}{ll}d.h.:& r=\frac13s_2s_3+1=\frac13x^2+\frac13x-\frac13\\
& s=-\frac13s_1s_2s_3-\frac13s_1-\frac13s_3\\
& = -\frac13x^3-\frac43x^2-\frac53x-1
\end{array}$
\subsection{Aufgabe 3}
$
\matr{0&1&1&2&\vrule&0\\2&2&1&2&\vrule&1\\2&0&1&1&\vrule&1\\1&2&2&0&\vrule&2}\leadsto
\matr{0&1&1&2&\vrule&0\\0&-2&-3&2&\vrule&-3\\0&-4&-3&1&\vrule&-3\\1&2&2&0&\vrule&2}\leadsto
\matr{0&1&1&2&\vrule&0\\0&0&-1&6&\vrule&-3\\0&0&1&9&\vrule&-3\\1&2&2&0&\vrule&2}\leadsto
\matr{1&2&2&0&\vrule&2\\0&1&1&2&\vrule&0\\0&0&1&9&\vrule&-3\\0&0&0&15&\vrule&-6}\leadsto
\matr{1&2&0&-18&\vrule&8\\0&1&0&-7&\vrule&3\\0&0&1&9&\vrule&-3\\0&0&0&15&\vrule&-6}\leadsto
\matr{1&0&0&-4&\vrule&2\\0&1&0&-7&\vrule&3\\0&0&1&9&\vrule&-3\\0&0&0&15&\vrule&-6}
$\par
\begin{tabular}{lll}$\K=\MdR:$ & $\K=\F_3$ & $\K=\F_5$\\
$\matr{1&0&0&0&\vrule&\frac25\\0&1&0&0&\vrule&\frac15\\0&0&1&0&\vrule&\frac35\\0&0&0&1&\vrule&-\frac25}$ &
$\matr{1&0&0&2&\vrule&2\\0&1&0&2&\vrule&0\\0&0&1&0&\vrule&0\\0&0&0&0&\vrule&3}$ &
$\matr{1&0&0&1&\vrule&2\\0&1&0&3&\vrule&3\\0&0&1&4&\vrule&3\\0&0&0&0&\vrule&4}$\\
& $L=\{\matr{2\\0\\0\\0}+r\matr{1\\0\\0\\1}| r\in\F_3\}$ & $L=\emptyset$
\end{tabular}
\subsection {Aufgabe 4}
$A\in\MdR^{m\times n}, b\in\MdR^m$\par
Wir betrachten: (i) $Ax=b$\quad(ii)$A\trans Ax=A\trans b$\\
Zu zeigen: (i) l"osbar $\RA$ (ii) l"osbar und die L"osungsmengen sind gleich
\begin{bew}\par
$\begin{array}{lll}\text{(i)}&\LRA&\exists x_0\in\MdR^n: Ax_0=b\\
&\RA&\exists x_0\in\MdR^n: (A\trans A)\x x_0=A\trans b\LRA\text{(ii) l"osbar}\end{array}$\par
Und: Da $x_0$ sowohl (i) als auch (ii) l"ost, reicht es z.z., dass die L"osungsmengen der homogenen LGS'e gleich sind.\par
Sei $x_1$ eine L"osung von $Ax=0$, d.h. es gilt $Ax_1=0 \RA A\trans x_1=A\trans\x0=0$\\
Also ist $x_1$ L"osung von $A\trans Ax=0$\par
Sei $x_2$ eine L"osung von $\underbrace{A\trans Ax=0}_{(*)}$, d.h. $A\trans Ax_2=0$\\
\[\begin{array}{ll}Ax=\matr{y_1\\y_2}=y\text{ (*) } \RA & x\trans A\trans Ax=x\trans=0\\
\LRA & (Ax)\trans(Ax)=0\LRA y\trans y=0\LRA y_1^2+...+y_n^2=0\\
\LRA & y_1=y_2=...=0\\
\LRA & y=0\\
\LRA & Ax_2=0\\
\end{array}\]
Also ist $x_2$ L"osung von $Ax=0\RA$ L"osungsmengen gleich
\end{bew}


\section {Übung 7, 17.12.2004}

\section {Übung 8, 3.01.2005}

\section {Übung 9, 10.01.2005}

\section {Übung 10, 17.01.2005}
\subsection {Aufgabe 1}
\enua
\item
Sei $x=(x_1, ..., x_n)\trans\in\MdR^n$. Dann ist $\phi(x)=Ax=(a_1, ..., a_n)\x(x_1, ..., x_n)\trans=x_1\x a_1+...+x_n\x a_n$.
Also ist Bild $\phi = \{\phi(x)|x\in\MdR^n\}=\{x_1\x a_1+...+x_n\x a_n|x_1,...,x_n\in\MdR\}=[a_1,...,a_n]$.
\item
$A=\matr{1 & 2 & -1 & -1 & 0\\
-2 & 1 & -2 & -2 & -3\\
8 & 1 & 4 & 4 & 7\\
7 & 4 & 1 & 1 & 5}\leadsto
\matr{1 & 0 \frac35 & \frac35 & 0\\
0 & 1 & -\frac45 & -\frac45 & 0\\
0 & 0 & 0 & 0 & 1\\
0 & 0 & 0 & 0 & 0}$\\
Aus der NF lesen wir ab:
\[\{\vect3\\-4\\-5\\0\\0\tcev,\vect3\\-4\\0\\-5\\0\tcev\}\text{ ist Basis von Kern}\phi\RA\text{dim Kern}\phi=2\]
Teil a)$\RA\{\vect1\\-2\\5\\4\tcev,\vect2\\1\\1\\4\tcev,\vect0\\-3\\7\\5\tcev\}$ ist Basis von Bild $\phi\RA$ dim Bild $\phi=3$
\une
\subsection{Aufgabe 2}
\enua
\item
Nach Tutorium: Die Spalten von $A\x B$ sind genau $[a_1,...,a_n]\RA Rg(AB)\le Rg(A)$.\\
Andererseits: $B\x x=0\RA A\x B\x x=0\RA p-Rg(A\x B)\ge p-Rg B \LRA Rg A\x Rg B\le Rg B$.
\item
Sei $B=(b_1\vert ...\vert b_n)$ und $U$ der Lösungsraum von $A\x x=0.$\\
$A\x B=0\RA b_1,...,b_n$ sind Lösungen von $A\x x=0\RA b_1, ..., b_n\in U\RA \text{dim }[b_1,...,b_n]=Rg B\le\text{ dim }U$.\\
Die Dimension des Lösungsraums erfüllt dim $U=n-Rg(A)\RA Rg(B)\le n-Rg(A)\RA Beh.$
\une

\section {Übung 11, 17.01.2005}
\subsection {Aufgabe 1}
\enua
\item Seien\\
$\begin{array}{ll}
 & \{x_1,...,x_r\}\text{ eine Basis von }U_1\cap U_2\\
 & \{x_1,...,x_r,x_{r+1},...,x_k\}\text{ eine Basis von }u_2\\
 & \{x_1,...,x_r,x_{r+1}^{'},...,x_k^{'}\}\text{ eine Basis von }U_2\\
\text{ also } & \{x_1,...,x_r,x_{r+1},...,x_k,x_{r+1}^{'},...,x_k^{'}\}\text{ eine Basis von }U_1+U_2
\end{array}$\\
Dann ist durch
\[\{x_1,...,x_r,x_{r+1},...,x_k, x_{r+1}^{'}+x_{r+1}, ..., x_k^{'}+x_k\}\]
und
\[\{x_1,...,x_r,x_{r+1}^{'},...,x_k^{'}, x_{r+1}+x_{r+1}^{'}, ..., x_k+x_k^{'}\}\]
eine Basis von $U_1+U_2$ gegeben.\par
Dann gilt: $U_1\oplus\tilde{W}.=U_1+U_2=U_2\oplus\tilde{W}$.\\
Nun ergänzen wir eine der Basen von $U_1+U_2$ durch Hinzufügen von $y_1,...,y_2$ zu einer Basis von $V$ und setzen
\[W:=[x_{r+1}+x_{r+1}^{'},...,x_k+x_k^{'}]\]
Dann erfüllt W die Behauptung.
\item
Gilt dim $U_1\le$dim $U_2$, so ex. ein Vektorraum U mit der Eigenschaft dim$(U_1\oplus U)=$dim $U_2$.
Nach a) ex. nun ein UVR $W_2$ mit $V=(U_1\oplus U)\oplus W_2=U_2\oplus W_2$.\\
Setzen wir nun $W_1:=U\oplus W_2$ so folgt die Beh.
\une
\subsection {Aufgabe 2}
Zunächst bestimmen wir vereinfachte Baesn von $U_1$ und $U_2$
\[\matr{3&1&4&0\\2&2&4&0}\leadsto\matr{1&0&1&0\\0&1&1&0}\]
\[\matr{0&4&3&1\\0&2&1&1}\leadsto\matr{0&0&1&-1\\0&1&0&1}\]
Also ist $\{\vect 1\\0\\1\\0\tcev, \vect 0\\1\\1\\0\tcev\}$ eine Basis von $U_1$ und $\{\vect 0\\0\\1\\-1\tcev, \vect 0\\1\\0\\1\tcev\}$ eine Basis von $U_2$.\par
Schreiben wir die Basisvektoren in die Spalten einer Matrix und wenden den Gauß-Algorithmus an, so erhalten wir:
\[\matr{1&0&0&0\\0&1&0&1\\1&1&1&0\\0&0&-1&1}\leadsto\matr{1&0&0&0\\0&1&0&1\\0&0&1&-1\\0&0&0&0}\]
Die ersten drei Spalten sind l.u., d.h.
\[\{\vect 1\\0\\1\\0\tcev, \vect 0\\1\\1\\0\tcev, \vect 0\\1\\0\\1\tcev\}\]
ist eine Basis von $U_1+U_2$ und dim$(U_1+U_2)=3$.\par
Aus der dritten Zeile lesen wir ab:
\[a_1\x\vect0\\0\\1\\-1\tcev+a_2\vect0\\1\\0\\1\tcev\in U_1\LRA a_1-a_2=0\LRA a_1=a_2\]
Also ist $\{\vect0\\0\\1\\-1\tcev+\vect0\\1\\0\\1\tcev\}=\{\vect0\\1\\1\\0\tcev\}$ eine Basis von $U_1\cap U_2$ und dim$(U_1\cap U_2)=1$.

\subsection {Aufgabe 3}
\enua
\item
Ist dim $U_1=\infty$ für ein $i=\{1,...,k\}$, so steht auf der rechten Seite unendlich und die Gleichung gilt..\\
Andernfalls betrachten wir Basen $B_i$ von $U_i$ für $i=1,..,k$. Der Vektor $x\in U_1+...+U_k$ lässt sich als Linearkombination von Vektorn aus $B_1\cup...\cup B_k$ schreiben, also ist $B_1\cup...\cup B_k$ ein Erzeugendensystem.\\
$\RA$ dim$(U_1+...+U_k)\le|B_1\cup...\cup B_k|\le|B_1|+...+|B_k|=\text{dim }U_1+...+\text{dim }U_k$.\par
Für $i=1,...,k$ sein $B_i$ eine Basis von $U_i$.
\item
$RA$:\\
Jedes $x\in U_1+...+U_k$ hat eine eindeutige Darstellung $x=u_1+...+u_k$ mit $u_i\in U_i$ für $i=1,...,k$.\\
Insbesondere sind also die Vektoren in $B_1\cup...\cup B_k$ l.u.\\
$\RA$ dim$(U_1+...+U_k)\ge$dim $U_1+...+$dim $U_k \overset{a)}{\RA}$ Beh.
$LA$:\\
Nach a) kann Gleichheit nur dann gelten, wenn $B_1\cup...\cup B_k$ eine Basis von $U_1+...+U_k$ist und $B_i\cap b_j=\emptyset$ für $i\ne j$ ist. Bezüglich der Basis ist jedes $x\in U_1+...+u_k$ eindeutig als Linearkombination darstellbar. Also ist auch die Darstellung $x=u_1+...+u_k$ mit $u_i\in U_i(i=1,...,k)$ eindeutig.
\item
$V=R^2, U_1=[\vect1\\0\tcev], U_2=[\vect0\\1\tcev], U_3=[\vect1\\1\tcev]$ erfüllen alle Forderungen.
\une


\section {Übung 12, 31.01.2005}
\section {Übung 13, 07.02.2005}
\section {Übung 14, 14.02.2005}
\subsection {Aufgabe 3}
\enua
\item
\[\matr{0&1&0&1\\&-1&2&0&2\\-1&1&0&2\\-1&1&-1&2}\x\matr{x_1\\x_2\\x_3\\x_4}=\matr{x_2+x_4\\-x_1+2x_2+2x_4\\-x_1+x_2+2_x4\\-x_1+x_2-x_3+2x_4}\]
\[y_1-y_2+y_3-y_4=x_2+x_4+x_1-2_x2-2x_4-x_1+x_2+2x_4+x_1-x_2+x_3-2x_4=x_1-x_2+x_3-x_4\]
\item
\enur
\item 1-1+0-0=0
\item 0+1-1+0=0
\item 0-0+1-1=0
\une
\item
\enur
\item dim $U=3$
\item W $\phi$-invariant bedeutet $\phi(W)\in W$
\une
$\RA$ dim $W=1$\par
Wir suchen also $x\in V$ mit: $\phi(x)=ax$ für ein $a\in\K$\\
a)$\RA\underbrace{x_1-x_2+x_3-x_4}_{=0}=y_1-y_2+y_3-y_4=a(\underbrace{x_1-x_2+x_3-x_4}_{=0})\RA a=1$\par
Also muß gelten: $\phi(x)=x$
\[A_\phi\x\hat{x}=\hat{x}\LRA(A_\phi-E_n)\hat{x}=0\]
Löse LGS:
\[\matr{-1&1&0&1\\-1&1&0&2\\-1&1&-1&2\\-1&1&-2&2}\leadsto\matr{-1&1&0&1\\0&0&0&1\\0&0&-1&1\\0&0&-2&1}\leadsto\matr{1&-1&0&0\\0&0&1&0\\0&0&0&1\\0&0&0&0}\]
also: $\hat{x}=\matr{1\\1\\0\\0}$, also $x=U_1$.\\
Widerspruch zu $U\oplus[x]=V$.
\une
\subsection {Aufgabe 4}
\enua
\item
Es gilt: Bild$(\psi\circ\phi)\subset$ Bild $\psi$\\
$\psi\circ\phi$ surjektiv $\LRA$ Bild $\psi\circ\phi=W\RA$ Bild $(\psi\circ\phi)=$ Bild $\psi\RA$ Beh.
\item
Ann.: Bild $\phi\cap$ Kern $\psi\neq\{0\}$\\
$\RA\exists x\in U, x\neq 0_U$: $\phi(x)\in$ Kern $\psi$, $\phi(x)\neq O_V$.\\
$\RA\exists x\in U, x\neq 0$: $\psi(\phi(x))=0_W$\\
$\RA$ Kern$(\psi\circ\phi)\neq\{0_W\}$\\
$\psi\circ\phi$ nicht injektiv.$\error$\par
Also gilt: Bild $\phi$ $\oplus$ Kern $\psi$ ist direkt.\par
z.Z.: Bild $\phi$ $\oplus$ Kern $\psi=V$\\
Sei $v\in V$. Dann ex. $u\in U$ mit $(\psi\circ\phi)(u)=\psi(v)$.\par
Sei $v_1:=v-\phi(u)$ und $v_2:=\phi(u)$.\\
Dann gilt:
\enur
\item $\psi(v_1)=\psi(v-\phi(u))=\psi(v)-(\psi\circ\phi)(u)=0\RA v_1\in$ Kern $\phi$
\item $v_2=\phi(u)\in$ Bild $\phi$
\item $v=v_1+v_2$
\une
$\RA v\in$ Bild $\phi$ $\oplus$ Kern $\psi$.
\une


\section {Übung 16, 18.04.2005}
\subsection {Aufgabe 1}
\enua
\item
$det(A-XE_4)=\begin{vmatrix}
2-x & 2 & 1 & -1\\
3 & 3-x & 1 & 1\\
3 & 4 & -x & 1\\
-3 & -2 & -1 & -x
\end{vmatrix}  =
\begin{vmatrix}
2-x & 2 & 1 & -1\\
3 & 3-x & 1 & 1\\
3 & 4 & -x & 1\\
-1-x & 0 & 0 & -1-x
\end{vmatrix}$\par$=
\begin{vmatrix}
3-x & 2 & 1 & -1\\
2 & 3-x & 1 & 1\\
2 & 4 & -x & 1\\
0 & 0 & 0 & -1-x
\end{vmatrix}=
(-1-x)\x\begin{vmatrix}
3-x & 2 & 1\\
2 & 3-x & 1\\
2 & 4 & -x\\
\end{vmatrix}$\par$=
(-1-x)\x\begin{vmatrix}
3-x & 2 & 1\\
-1+x & 1-x & 0\\
2 & 4 & -x\\
\end{vmatrix}=
(-1-x)\x\begin{vmatrix}
5-x & 2 & 1\\
0 & 1-x & 0\\
0 & 4 & -x\\
\end{vmatrix}$\par$=
(-1-x)(1-x)\x\begin{vmatrix}
5-x & 1\\
6 & -x\\
\end{vmatrix}=
(-1-x)(1-x)\x\begin{vmatrix}
6-x & 1\\
6-x & -x\\
\end{vmatrix}$\par$=
(-1-x)(1-x)\x\begin{vmatrix}
0-x & 1\\
0 & 1-x\\
\end{vmatrix}=(-1-x)^2(1-x)(6-x)$\par
Eigenraum zum EW $-1$: $0\neq x\in\MdR^4$ ist EV  zum EW $-1\LRA$
\[\begin{array}{ll}
 & Ax=-x\\
\LRA & Ax+x=0\\
\LRA & (A+E_4)x=0\\
\LRA & 0\neq x\in\text{Kern}(A)+E_4
\end{array}\]
$\matr{3 & 2 & 1 & -1\\
3 & 4 & 1 & 1\\
3 & 4 & 1 & 1\\
-3 & -2 & -1 & 1
}\leadsto
\matr{
1 & 0 & \frac13 & -1\\
0 & 1 & 0 & 1\\
0 & 0 & 0 & 0\\
0 & 0 & 0 & 0}$, d.h. $E_{-1}=[\matr{1\\0\\-3\\0},\matr{1\\-1\\0\\1}]$\par
ebenso: $E_1=[\matr{-3\\2\\2\\-3}]$, $E_6=[\matr{-1\\1\\1\\1}]$.
\item
Offensichtlich $B=[\vect1\\0\\-3\\0\tcev,\vect1\\-1\\0\\1\tcev\vect-3\\2\\2\\-3\tcev\vect-1\\1\\1\\1\tcev]$ ist Basis von $\MdR^4$.\par
Definieren wir eine lineare Abbildung $\Phi:\MdR^4\to\MdR^4$ durch $\Phi:\MdR^4\to\MdR^4, x\mapsto Ax$ so ist die Abbildung von $\Phi$ bzgl. der Std.-Basis.\par
Bzgl. B hat $\Phi$ die Abb.
$A_\Phi=
\matr{1 & 1 & -3 & -1\\
0 & -1 & 2 & 1\\
-3 & 0 & 2 & 1\\
0 & 1 & 3 & 1}
$ dann gilt: $A_\Phi=S^{-1}AS$\par
Nebenrechnung:\\
\fbox{$\Phi\vect1\\0\\-3\\0\tcev=-1\vect1\\0\\-3\\0\tcev,\widehat{\Phi\vect1\\0\\-3\\0\tcev}=A_\Phi\vect1\\0\\0\\0\tcev=\vect-1\\0\\0\\0\tcev$}\par
$S=\matr{1 & 0 & \frac13 & -1\\
0 & 1 & 0 & 1\\
0 & 0 & 0 & 0\\
0 & 0 & 0 & 0}
$, dann gilt: $A_\Phi=S^{-1}AS$
\item
c ist EW von A mit EV $x\neq0\LRA Ax=cx\LRA(A-cE)x=0$
\une
\subsection {Aufgabe 2}
Seien $A,B\in\K^{n\times n}$ mit $AB-E_n$ regul"ar.\\
Ann.: $BA-E_n$ nicht regul"ar.
\[\begin{array}{ll}
\RA & \exists x\in\K^n,x\neq0: (BA-E_n)x=0\\
\LRA & \exists0\neq x\in\K^n: BAx=x\\
\LRA & \exists0\neq x\in\K^n: (AB)Ax=Ax
\end{array}\]
$Ax\neq0$, sonst: $x=B(AX)=B\times0=0\error$\par
Damit gilt $(AB-E_n)(Ax)=0$. Also ist $AB-E_n$ nicht regul"ar. Insgesamt $BA-E_n$ ist regul"ar.
\subsection {Übungsaufgabe 2}
Es sei $(G,\circ)$ eine Gruppe mit neutralem Element e. Weiter sei $M={x\in G|x\circ x=e}.$\par
\enua\item
Zeigen Sie: ist G kommutativ so ist M eine Untergruppe von G.\par
G: Gruppe: G Menge und $\circ:G\times G\to G$ mit folgendenen Eigeschaften:
\enur
\item $a\circ(b\circ c)=(a\circ b)\circ c, a,b\in G$
\item $\exists e\in G:\forall a\in G: a\circ e=a=e\circ a$
\item $\forall a\in G\exists a^{-1}\in G: a\circ a^{-1}=e=a^{-1}\circ a$
\une
G heisst abelsch falls zus"atzlich gilt:
\enur
\item (iv) $\forall a,b\in G a\circ b=b\circ a$
\une
Sei $M\subset G$: M heisst Untergruppe von G, falls:
\[(M,\circ)\text{eine Gruppe ist}\]
Untergruppenkriterium:
\[\begin{array}{llcl}
  M\subset G\text{ ist Untergruppe } & \LRA & \text{(i)} & M\neq\emptyset\\
  & & \text{(ii)} & x,y\in M: x^{-1}\circ y\in M\\
  & \LRA & \text{(i) } & M\neq\emptyset\\
  & & \text{(ii)} & x,y\in M: x\circ y\in M\\
  & & \text{(iii)} & x\in M: x^{-1}\in M\\
\end{array}
\]
\begin{bew}
Zu zeigen:
\enue\item $M\neq\emptyset$, gilt da, $e\circ e=e$ ist, ist $e\in M$
\item
$\begin{array}{ll}\forall x\in M: x^{-1}\in M\text{, denn }x\in M & \LRA x\circ x=e\\
& \LRA (x^{-1}\circ x)\circ =x^{-1}\circ e\\
& \LRA x=x^{-1}\\
& \LRA x^{-1}\circ x\circ x^{-1}\circ x^{-1}=e\\
& \LRA e=x^{-1}\circ x^{-1}\in M
\end{array}$
\item $\forall x,y\in M: x\circ y\in M$, denn:
\[(x\circ y)\circ(x\circ y)=\overset e{(x\circ x)}\circ\overset e{(x\circ\x)}=e\circ e=e\]
\une
Also ist M eine Untergruppe von G.
\end{bew}
\item
Sei $n\ge 3$ und $G=S_n$, dann ist M keine Untergruppe von G\par
\[S_n:={\pi{1,...,n}\to{1,...,n}: \pi bijektiv}\]
Verkettung von Abbildungen:
\[\tau_1=\matr{1 & 2 & 3 & 4 & ... & n\\3 & 2 & 1 & 4 & ... & n}\in M\]
\[\tau_2=\matr{1 & 2 & 3 & 4 & ... & n\\2 & 1 & 3 & 4 & ... & n}\in M\]
$(\tau_1\circ\tau_2)\circ(\tau_1\circ\tau_2)=3\text{, also }\neq1\text{, d.h. }\tau_1\circ\tau_2\notin m$. Also M keine Untergruppe.
\une
\section {Übung 17, 25.04.2005}
\subsection {Aufgabe 1}
\enua
\item
$\Phi: V\to V$ End., $\Phi$ ist diag., V ist n-dim $\MdR-VR$.\par
z.z.: Es existiert $\Psi: V\to V$ End., mit $\Psi^3=\Psi\circ\Psi\circ\Psi=\Phi$.\par
\begin{bew}$\Phi$ ist diagonalisierbar, d.h. es existiert eine Abbilduntsmatrix von $\Phi$ der Form \[A_\Phi=\matr{c_1 & & 0\\ & \ddots & \\ 0 & & c_n}\]
Definieren wir $A_\Psi:=\matr{\sqrt[3]{c_1} & & 0\\ & \ddots & \\ 0 & & \sqrt[3]{c_n}}$, so gilt: $A_\Psi\circ A_\Psi\circ A_\Psi=A_\Phi$.\par
Nach Vorlesung existiert genau ein lineare Abbildung $\Psi:V\to V$ mit $A_\Psi$ und es gilt: $\Psi^3=\Phi$.
\end{bew}
\item
Man rechnet nach: $c_1=-8$ und $c_2=8$ sind die Ew. von A.\par
Der Eigenraum zum Eigenwert $c_1=-8$ ist $E_{-8}=[\matr{1\\2\\1}].$\\
Der Eigenraum zum Eigenwert $c_1=8$ ist $E_8=[\matr{2\\-5\\0}\matr{4\\0\\-5}].$\par
Dann gilt(vgl. Aufgabe 1, Blatt 1):
\[\underbrace{\matr{8&0&0\\0&8&0\\&0&0&-8}}_{B^3}=
\underbrace{\matr{2&4&1\\-5&0&2\\0&-5&-1}^{-1}}_S\x A\x
\underbrace{\matr{2&4&1\\-5&0&2\\0&-5&-1}}_S\]
\[\LRA\underbrace{\matr{2&0&0\\0&2&0\\&0&0&-2}}_{\tilde B}=S^{-1}AS\LRA S\tilde{B}S^{-1}=A\LRA
(S\tilde BS^{-1})(S\tilde BS^{-1})(S\tilde BS^{-1})=A\]
w"ahlen wir:
\[B:=S\tilde BS^{-1}=\matr{-2&-\frac85&-\frac{16}5\\-8&-\frac65&-\frac{32}5\\4&\frac85&\frac{26}5}\text{, so gilt: }B^3=A\]
\une
\subsection {Aufgabe 2}
$A\in\MdR^{n\times n}$ mit Rang $A=1$.\\
v sei eine Spalte von A mit $v\neq0$
\enua
\item z.Z.: $\exists w\in\MdR^n$ mit $A=vw\trans$ und $w$ ist eindeutig bestimmt.
Da $v\neq 0$ ist, ex. $a_1,...,a_n\in\MdR$ mit $A=(a_1v|...|a_nv)$\\
Wenn wir $w=\matr{a1\\\vdots\\a_n}$ setzen, so gilt $A=vw\trans$.\par
Sei $\tilde w=\matr{\tilde{a_1}\\\vdots\\\tilde{a_n}}\in\MdR^n$ mit $A=v\x\tilde w\trans$. F"ur $i=1,...,n$ gilt:
$a_1v=\tilde {a_1}v\overset{v=0}{\RA}a_i=\tilde{a_i}$ f"ur $i=1,...,n$.
\item
$A=((a_{ij}))$, dann ist Spur$(A)=\sum\limits_{i=1}^n{a_{ij}}$. In unserem Fall:\\
Spur$(A)=\sum\limits_{i=1}^n{a_iv_i}$, wobei $v=(v_1,...,v_n)\trans$.\par
z.Z.: $Av=Spur(A)v$\par
\begin{bew}\par
$Av=(v\x w\trans)=v\underbrace{(w\trans v)}_{\in\MdR}=(w\trans v)\x v=\text{Spur}(A)v$\par
Weil $v\neq 0$ ist also Spur A EW von A.
\end{bew}
\une
\section {Übung 18, 06.06.2005}
\subsection {Aufgabe 1}
\enua
\item
$<A,B>=Spur(A\trans B)\overset{(*)}=\sum\limits_{i=1}^n{\sum\limits_{j=1}^n{a_{ji}b_{ji}}}$\quad\quad (**)
\begin{itemize}
\item $Spur \left(((c_{ij})\right) )=\sum\limits_{i=1}^nc_{ii}$
\item $A\trans B=((c_{ij}))$, dann gilt $c_{ii}=\sum\limits_{j=1}^n{a_{ji}b_{ji}}$ wobei $A=((a_{ij}))$ und $B=((b_{ij}))$ sein soll.
\end{itemize}
Aus (**) ergibt sich die Symmetrie von $<\x,\x>$ ebenso wie die Linearit"at im ersten Argument direkt.\par
F"ur $<A,A>\overset{(**)}=\sum\limits_{i=1}^n{\sum\limits_{j=1}^n{a_{ji}^2\ge0}}$ und
\begin{eqnarray*}
<A,A>=0  \LRA & & a_{ji}^2=0\quad\text{, f"ur alle }i,j\in\{1,n\}\\
\LRA & & a_{ji}=0\quad\text{, f"ur alle }i,j\in\{1,n\}\\
\LRA & & A=0
\end{eqnarray*}
\item
\une
$||Ax||^2=\sum\limits_{i=1}^n({\underbrace{\sum\limits_{j=1}^n{a_{ij}x_j}}_{\text{j-te Komp. des Vektors Ax}})^2}\overset{\text{CSU}}\le\sum\limits_{i=1}^n{\sum\limits_{j=1}^n{(a_{ij})^2}}\x(\sum\limits_{i=1}^n{x_i^2})=<A,A>\x||x||^2$\par$\overset{\text{Def}}=||A||^2||x||^2\RA||Ax||\le||A||\x||x||$
\subsection {Aufgabe 2}
Seien $||\x||_1$ und $||\x||_2$ die durch $<\x,\x>_1$ bzw. $<\x,\x>_2$ induzierten Normen.\\
Weiter seien $x,y\in V$ mit $||x||_1=||y||_1$.\par
\begin{bew}\\
Dann gilt: $<x+y,x-y>_1=<x,x>_1+\overbrace{<y,x>_1-<x,y>_1}^{=0}-<y,y>_1=||x||_1^2-||y||_1^2=0$\par
Daraus folgt: $0=<x+y,x-y>_2=||x||_2^2-||y||_2^2$, also $||x||_2-||y||_2$\par
Sei $x_0\in V$ mit $||x_0||_1=1$ und $x\in V$
\[||x||_1=||x||_1\x||x_0||_1=||\quad||x_1||_1\x x_0\quad||_1\\
\RA||x||_2=||\quad||x||_1\x x_0\quad||_2=||x||_1\x \underbrace{||x_0||_2}_{=c}\]
Nun gilt aber auch f"ur $x,y\in V$:
\[<x,y>_2\overset{\text{Vorl.}}=\frac14(||x+y||_2^2-||x-y||_2^2)=\frac14\x(c'||x+y||_1^2-c'||x-y||_1^2)=c'<x,y>_1\]
\end{bew}
\subsection {Übungsaufgabe 1}
Es seien $G,G'$ Gruppen und $\Phi:G\to G'$ und $\Psi:G\to G'$ Gruppenhomomorphismen. Zeigen Sie:\par
Ist $H\subsetneq G$ eine (echte) Untergruppe von G und gilt:
\[\Phi(a)=\Psi(a)\text{ f"ur alle }a\in G\setminus H\]
so ist $\Phi=\Psi$.
\begin{bew} Sei $a\in G\setminus H, b\in H.$\par
Is $a\circ b\in G\setminus H$ ? Ja, denn:
\[\left.\begin{array}{c}a\circ b\in H\\b^{-1}\in H\end{array}\right\}\underbrace{(a\circ b)}_{\in H}\underbrace{\circ b^{-1}}_{\in H}=a\circ(b\circ b^{-1})=a\in H\error\]
Also gilt: $\Phi(a\circ b)\overset{\text{Vor}}=\Psi(a\circ b)=\Psi(a)\circ'\Psi(b)$\par
Insgesamt: 
\[\Psi(a)\circ'\Phi(b)=\Psi(a)\circ'\Psi(b)\]
\[\RA\Psi(a)^{-1}\circ\Psi(a)\circ\Phi(b)=\Psi(a)^{-1}\circ\Psi(b)\LRA\Phi(b)=\Psi(b)\]
Also gilt: $\Phi=\Psi$
\subsection {Übungsaufgabe 2}
Geben Sie alle Gruppenhomomorphismen von $(\MdZ_6,+)$ nach $(\MdZ_7,+)$ an.
\[\MdZ_m:=\{[z]_\sim: z\in\MdZ\}\]
\[\forall z_1,z_2\in\MdZ: z_1\sim z_2\LRA z_1-z_2=k\x m\text{ f"ur ein }k\in\MdZ\]
$[z_1]_\sim+[z_2]_\sim:=[z_1+z_2]_\sim$\quad($+$ ist wohldefiniert)\par
Sei $z_1'\in[z_1]_\sim$ und $z_2'\in[z_2]\sim$, $z_1'=z_1+k\x m$ und $z_2'=z_2+l\x m$
\[z_1'+z_2']_\sim = [z_1+km+z_2+lm]_\sim=[z_1+z_2+(k+l)\x m]_\sim=[z_1+z_2]_\sim\]
F"ur alle $m\in\MdN$ ist $\MdZ_m$ eine Gruppe
\[\MdZ_6:=\{[z]_6:z\in\MdZ\}\text{ und }\MdZ_7:=\{[z]_7:z\in\MdZ\}\]
F"ur alle $m\in\MdN$ gilt: $(\MdZ_m,+)$ ist zyklisch.
\[[k]_m=\underbrace{[1]_m+...+[1]_m}_{k-mal}\]
D.h.: F"ur jede Gruppe $G'$ gilt: $\Phi:\MdZ_m\to G'$ ist eindeutig durch $\Phi(1)$ festgelegt.
\[\Phi([0]_m)=e_{G'}\]
Sei $\Phi:\MdZ_6\to\MdZ_6$ Gruppenhomomorphismus\par
Dann muss gelten: $\Phi([0]_6)=[0]_7$\quad(Eigenschaften eines Grp.-hom)
\[[3]_6+[3]_6=[6]_6=[0]_6\text{, d.h. }[3]_6\text{ ist selbstinvers}\]
\[\Psi(x^{-1}=\Psi(x)^{-1}\text{, d.h. }x=x^{-1}\text{, so ist }\Psi(x)^{-1}=\Psi(x)\]
also: selbstinverse Elemente werden auf selbstinverse Elemente abgebildet.\par
Also gilt: $\Phi([3]_\sim)=[0]_\sim$\par
\[\Phi([1]_6)=[k]_7\text{ f"ur ein }k\in\{0,...,6\}\]
\[\Phi([3]_6)=\Phi([1]_6+[1]_6+[1]_6)=\Phi([1]_6)+\Phi([1]_6)+\Phi([1]_6)=[k]_7+[k]_7+[k]_7=[3k]_7=[3]_7\x[k]_7=[0]_7\]
\[\overset{\MdZ_7\text{ ist K"orper}}\RA k=0\text{, also }\Phi([1]_6)=[0]_7\]
Also ist:  $\Phi:\MdZ_6\to\MdZ_7, [k]_6\mapsto[0]_7$. Es gibt also nur einen Grp.-homo. von $\MdZ_6$ nach $\MdZ_7$.
\end{bew}

% \appendix
% \chapter{Satz um Satz (hüpft der Has)}
% \listtheorems{satz,wichtigedefinition}
% 
% \renewcommand{\indexname}{Stichwortverzeichnis}
% \addcontentsline{toc}{chapter}{Stichwortverzeichnis}
% \printindex


\end{document}
