\documentclass[a4paper,oneside]{scrbook}

% Deutsche Sprache
%\usepackage{ngerman}
\usepackage[ngerman]{babel}

% Verschiebt \sections auf die naechste seite falls sie zu tief sind. Muss vor
% hyperref kommen.
\usepackage[nobottomtitles]{titlesec}

% Schicke Schrift
\usepackage[utf8]{inputenc}
\usepackage[T1]{fontenc}
\usepackage{lmodern}
\usepackage{tikz}
\usetikzlibrary{matrix,arrows}

% schmaler rand
\usepackage{geometry}
\geometry{a4paper,tmargin=2cm,lmargin=2cm,rmargin=2cm}
\setlength\parskip{\smallskipamount}
\setlength\parindent{0pt}
\tolerance=900

% Vokabelliste erzeugen
\usepackage{index}
\newindex{default}{idx}{ind}{Vokabeln}

% links
\usepackage{color}
\usepackage[hyperfigures=true,pdftex]{hyperref}

\definecolor{rltred}{rgb}{0.75,0,0}
\definecolor{rltgreen}{rgb}{0,0.5,0}
\definecolor{rltblue}{rgb}{0,0,0.75}

\hypersetup{
  pdftitle={Algebraische Geometrie II Prof. Herrlich},
  pdfsubject={Algebraische Geometrie II},
  pdfkeywords={Algebraische Geometrie Herrlich},
  pdfproducer={pdfLaTeX},
  pdfpagemode={UseOutlines},
  colorlinks=true,
  bookmarksopen=true,
  bookmarksnumbered=true,
  urlcolor=rltblue,
  filecolor=rltgreen,
  linkcolor=rltblue,
  backref=true,
  pagebackref=true,
%  pdfpagemode=None
}

% Mathe-Pakete
\usepackage{amssymb}
\usepackage{amsmath}
\usepackage{amsfonts}
\usepackage{stmaryrd}

% Verschiedene items in enumerate Umgebungen
\usepackage{enumerate}

% Für Diagramme
%\usepackage[arrow,matrix,curve]{xy}

\usepackage{etex}
\usepackage{pictex}
\usepackage{graphicx}

% Theorem-Umgebung
\usepackage[hyperref,amsmath,thmmarks,thref]{ntheorem}

% keine kursiv schrift in theorems
\theorembodyfont{}


% Theoreme definieren
\theoremstyle{break}
    \newtheorem{Satz}{Satz}
    \newtheorem{SatzDef}[Satz]{Satz + Definition}
    \newtheorem{Def}{Definition}[section]
    \newtheorem{DefBem}[Def]{Definition + Bemerkung}
    \newtheorem{ErinnDefBem}[Def]{Erinnerung / Definition + Bemerkung}
    \newtheorem{ErinnDef}[Def]{Erinnerung / Definition}
    \newtheorem{DefSatz}[Def]{Definition + Satz}
    \newtheorem{Bem}[Def]{Bemerkung}
    \newtheorem{BemDef}[Def]{Bemerkung + Definition}
    \newtheorem{Prop}[Def]{Proposition}
    \newtheorem{PropDef}[Def]{Proposition + Definition}
    \newtheorem{Folg}[Def]{Folgerung}
    \newtheorem{Bsp}[Def]{Beispiele}
    \newtheorem{DefProp}[Def]{Definition + Proposition}
    \newtheorem{anBew}[Def]{Beweis}
    \newtheorem{Kor}[Def]{Korollar}
    \newtheorem{Lemma}{Lemma}
\theoremstyle{nonumberbreak}
    \newtheorem{nnBem}{Bemerkung}
    \newtheorem{nnBsp}{Beispiele}
    \newtheorem{nnSatz}{Satz}
    \newtheorem{nnFolg}{Folgerung}
    \newtheorem{Beo}{Beobachtung}
    \newtheorem{Eri}{Erinnerung}
    \newtheorem{Beh}{Behauptung}
\theoremstyle{nonumberplain}
\theoremsymbol{\ensuremath{\Box}}
    \newtheorem{Bew}{Beweis}

\theoremstyle{break}
\theoremsymbol{}
\theoremseparator{}
\theoremprework{\medskip\hrule}
\theorempostwork{\hrule\medskip}
\theorembodyfont{\small}
\newtheorem{anmerkung}{Anmerkung}


%Aufzählungstypen
\newenvironment{aaufz}
             {\renewcommand{\labelenumi}{\alph{enumi})}              
             \renewcommand{\labelenumii}{\alph{enumii})}
                \begin{enumerate}}
               {\end{enumerate}}

\newenvironment{iaufz}            
							{\renewcommand{\labelenumi}{(\roman{enumi})}            
							\renewcommand{\labelenumii}{(\roman{enumii})}
                \begin{enumerate}}
               {\end{enumerate}}
               
\newenvironment{1aufz}             
							{\renewcommand{\labelenumi}{\arabic{enumi}.)}               
							\renewcommand{\labelenumii}{\arabic{enumii}.)}
                \begin{enumerate}}
               {\end{enumerate}}


%Selbstdefinierter Schnickschnack
\newcommand{\emp}[1]{\textbf{\emph{#1}}}
\newcommand{\begriff}[1]{{\index{#1}}\emp{#1}}
\newcommand{\defeqr}[0]{\mathrel{\mathop:}=}
\newcommand{\defeql}[0]{=\mathrel{\mathop:}}

\newcommand{\folgtnach}[1]{\ensuremath{\DOTSB\;\xRightarrow{\text{#1}}\;}}
\newcommand{\folgtwegen}[1]{\ensuremath{\DOTSB\;\stackrel{#1}{\Rightarrow}\;}}
\newcommand{\equizunach}[1]{\ensuremath{\DOTSB\;\xLeftrightarrow{\text{#1}}\;}}
\newcommand{\equizuwegen}[1]{\ensuremath{\DOTSB\;\xLeftrightarrow{#1}\;}}
\newcommand{\tonach}[1]{\ensuremath{\DOTSB\;\xrightarrow{\text{#1}}\;}}
\newcommand{\towegen}[1]{\ensuremath{\DOTSB\;\xrightarrow{#1}\;}}
\newcommand{\tomit}[1]{\ensuremath{\DOTSB\;\xrightarrow{#1}\;}}
\newcommand{\gleichnach}[1]{\ensuremath{\DOTSB\;\stackrel{\text{#1}}{=}\;}}
\newcommand{\gleichwegen}[1]{\ensuremath{\DOTSB\;\stackrel{#1}{=}\;}}

\newcommand{\Abb}[5]{\ensuremath{#1:\begin{array}{ccc} #2 & \longrightarrow & #3 \\ #4 & \longmapsto & #5 \end{array}}}


\newcommand{\myref}[2]{%
\hyperref[#2]{#1~\ref*{#2}}%
}

\newcommand{\Ob}{% objects
	\ensuremath{\operatorname{Ob}}%
}

\newcommand{\Mor}{% morphisms
	\ensuremath{\operatorname{Mor}}%
}

\newcommand{\Off}{% open subseteqs
	\ensuremath{\operatorname{Off}}%
}

\newcommand{\Cat}[1]{% category
	\ensuremath{\underline{#1}}%
}

\newcommand{\mono}{% monomorphism
	\ensuremath{\hookrightarrow}%
}

\newcommand{\epi}{% epimorphism
	\ensuremath{\twoheadrightarrow}%
}

\renewcommand{\emptyset}{% empty set
	\ensuremath{\varnothing}%
}

\newcommand{\RR}{% real numbers
	\ensuremath{\mathbb{R}}%
}

\newcommand{\CC}{% complex numbers
	\ensuremath{\mathbb{C}}%
}

\newcommand{\NN}{% natural numbers
	\ensuremath{\mathbb{N}}%
}

\newcommand{\ZZ}{% integers
	\ensuremath{\mathbb{Z}}%
}

\renewcommand{\phi}{% use the nice phi
	\ensuremath{\varphi}%
}

\renewcommand{\theta}{% use the nice theta
	\ensuremath{\vartheta}%
}

\newcommand{\directlim}{% more intuitive name
	\ensuremath{\varinjlim}%
}

\renewcommand{\ker}{% kern of a morphism
	\ensuremath{\operatorname{Kern}}%
}

\newcommand{\img}{% image of a morphism
	\ensuremath{\operatorname{Bild}}%
}

\newcommand{\define}{% nice Def symbol
	\ensuremath{\mathrel{\mathop:}=}%
}

\newcommand{\todo}[1]{% TODO macro
	\marginpar{ {\Large \textbf{TODO}}\\#1}%
}

\newcommand{\Spec}{% Spectrum of a ring
	\ensuremath{\operatorname{Spec}}%
}

\newcommand{\Proj}{% Projective spectrum of a ring
	\ensuremath{\operatorname{Proj}}%
}

\newcommand{\HomSheaf}{% Hom-Sheaf
	\ensuremath{\operatorname{\mathcal H\emph{om}}}%
}

\newcommand{\closure}[1]{% closure of a set U subseteq X
	\ensuremath{\overline{#1}}%
}

\newcommand{\rad}[1]{% radical of an ideal
	\ensuremath{\sqrt{#1}}%
}

\DeclareMathOperator{\Aut}{Aut}
\DeclareMathOperator{\Hom}{Hom}
\DeclareMathOperator{\Quot}{Quot}

\DeclareMathOperator{\Kern}{Kern}
\DeclareMathOperator{\Bild}{Bild}
\DeclareMathOperator{\Cl}{Cl}
\DeclareMathOperator{\Rat}{Rat}
\DeclareMathOperator{\id}{id}
\DeclareMathOperator{\Rg}{Rg}
\DeclareMathOperator{\height}{ht}
\DeclareMathOperator{\trdeg}{trdeg}
\DeclareMathOperator{\Div}{Div}
\DeclareMathOperator{\ord}{ord}


\newcommand{\FakRaum}[2]{
  \raisebox{0.7ex}{\ensuremath{#1}}
  \ensuremath{\mkern-3mu}\big/\ensuremath{\mkern-3mu}
  \raisebox{-0.6ex}{\ensuremath{#2}}} 

\renewcommand{\labelenumi}{\theenumi}    
\renewcommand{\theenumi}{(\alph{enumi})}

\newcommand{\ilim}{\mathop{\varprojlim}\limits}

\renewcommand{\OE}{O\!\!E~}
\newcommand{\nsubset}{\subset\!\!\!\!\!/~}

% Weniger Abstand nach der Ueberschrift des Inhaltsverzeichnisses
\makeatletter
\let\@my@starttoc\@starttoc
\renewcommand*{\@starttoc}[1]{%
  \addvspace{-1.5em}%
  \@my@starttoc{#1}%
}
\makeatother

% Einige Anstrengungen, um den § vor die Section-Nummer zu stellen
% \renewcommand{\thesection} allein führt zu einem Konflikt mit ntheorem-hyper
\makeatletter
\def\@seccntformat#1{\@ifundefined{#1@cntformat}%
{\csname the#1\endcsname\quad}% default
{\csname #1@cntformat\endcsname}% individual control
}
\def\section@cntformat{§\@arabic\c@section\quad}
\makeatother
\setcounter{secnumdepth}{-1}
\title{Algebraische Geometrie II -- Sommersemester 2009\\ Prof. Dr. F. Herrlich}
\author{Die Mitarbeiter von \url{http://mitschriebwiki.nomeata.de/}}

\begin{document}
\maketitle

% Inhaltsverzeichnis
\pdfbookmark[1]{Inhaltsverzeichnis}{contents}
\setlength\parskip{0.6pt}
\tableofcontents

% Liste der benannten Saetze
\section*{Benannte Sätze}
\pdfbookmark[1]{Benannte Sätze}{contents}

\theoremlisttype{optname}
\listtheorems{Satz,SatzDef,Def,DefBem,BemDef,Prop,PropDef,Bsp,DefProp}

\setlength\parskip{\smallskipamount}

\chapter{Vorwort}
\setcounter{secnumdepth}{2}
\section*{Über dieses Skriptum}
Dies ist ein Mitschrieb der Vorlesung \glqq Algebraische Geometrie II\grqq\ von Prof. Dr. F. Herrlich im
Sommersemester 09 an der Universität Karlsruhe.
Die Mitschriebe der Vorlesung werden mit ausdrücklicher Genehmigung von Prof. Dr. F. Herrlich hier veröffentlicht,
Prof. Dr. F. Herrlich ist für  den Inhalt nicht verantwortlich.
\section*{Wer}
Getippt wurde das Skriptum soweit von \dots.

\section*{Wo}
Alle Kapitel inklusive \LaTeX-Quellen können unter \url{http://mitschriebwiki.nomeata.de} abgerufen werden.
Dort ist ein von Joachim Breitner programmiertes \emph{Wiki}, basierend auf \url{http://latexki.nomeata.de} installiert. 
Das heißt, jeder kann Fehler nachbessern und sich an der Entwicklung
beteiligen. Auf Wunsch ist auch ein Zugang über \emph{Subversion} möglich.

\chapter{Schemata}

%% -- CHAPTER 1: SHEAVES ----------------------------------------------------------------------------------------------------------
\section{Garben}

%%% - Definition: presheaf -------------------------------------------------------------------------------------------------------
\begin{Def}[Prägarbe]
	\label{def:presheaf}
	Sei $X$ ein topologischer Raum, $\Off(X)$ die Menge der offenenen Teilmengen von $X$ und $\mathcal{C}$ eine Kategorie.
	Eine \emph{Prägarbe} auf $X$ mit Werten in $\mathcal{C}$ ist ein kontravarianter Funktor
	\[ \mathcal{F}\colon \Cat{\Off}(X) \to \mathcal{C} \]
	wobei $\Cat{\Off}(X)$ die Kategorie mit den Objekten $\Off(X)$ und den Morphismen 
	\[
	 \Mor(U,U') = \begin{cases}  i\colon U \mono U' & \text{falls } U \subseteq U'\\%
	\emptyset & \text{sonst}%
	\end{cases}
	\]
	ist.
	Für $U \subseteq U'$ heißt $\rho_{U}^{U'} = \mathcal{F}( U \mono U' )$ \emph{Restriktionsmorphismus}.
	Ist $U \subseteq U'$ und $f \in \mathcal{F}(U')$, so schreibt man statt $\rho_{U}^{U'}(f)$ auch $f\restriction U$.
\end{Def}
%%% - End Definition: presheaf ---------------------------------------------------------------------------------------------------



%%% - Definition: sheaf ----------------------------------------------------------------------------------------------------------
\begin{Def}[Garbe]
	\label{def:sheaf}
	Eine Prägarbe $\mathcal{F}$ auf $X$ heißt \emph{Garbe}, falls folgende Bedingung erfüllt ist:

	Ist $U \subseteq X$ offen, $(U_i)_{i\in I}$ eine offene Überdeckung von $U$ und ist für jedes $i\in I$ ein $s_i \in \mathcal{F}(U_i)$ gegeben, so dass $s_i\restriction U_i \cap U_j = s_j \restriction U_i \cap U_j$ für alle $i,j \in I$, dann gibt es genau ein $s \in \mathcal{F}(U)$, so dass für alle $i\in I$ gilt: $s\restriction U_i = s_i$.
\end{Def}
%%% - End Definition: sheaf ------------------------------------------------------------------------------------------------------



%%% - Example: sheaves/presheaves -------------------------------------------------------------------------------------------------
\begin{Bsp}
	\begin{enumerate}
		\item Sei $X$ quasi-projektive Varietät, $\mathcal{O}_X(U)$ der Ring der regulären Funktionen auf $U$,
		dann ist $\mathcal{O}_X$ Garbe auf $X$.
		\item Sei $X$ ein topologischer Raum, $\mathcal{C}(U)$ die Menge der stetigen Funktionen $f\colon X \to \RR$.
		$\mathcal{C}$ ist Garbe von Ringen auf $X$.
		Ist $X$ eine differenzierbare Mannigfaltigkeit, dann sind auch $\mathcal{C}^{\infty}(U)$ und $\mathcal{C}^{k}(U)$ Garben von Ringen auf $X$.
		\item Sei $X$ ein topologischer Raum, $G$ eine (abelsche) Gruppe. Definiere $\mathcal{G}(U) = G$ für jedes offene $U \subseteq X$ 
		und wähle als Restriktionsmorphismen $\rho_{U}^{U'} = \id_G$ für alle $U\subseteq U'$.

		$\mathcal{G}$ ist offenbar Prägarbe, muss aber nicht zwingend Garbe sein. Gibt es in $X$ disjunkte offene Mengen $U_1,U_2$, dann ist $U = U_1 \cup U_2$ offen und $\{ U_1, U_2 \}$ ist eine Überdeckung von $U$. Jedoch gibt es für $g_1 \in \mathcal{G}(U_1), g_2 \in \mathcal{G}(U_2)$ mit $g_1 \neq g_2$ kein $g \in \mathcal{G}(U)$, so dass $g\restriction U_1 = g_1$ und $g\restriction U_2 = g_2$.

		$\mathcal{G}$ kann zur Garbe gemacht werden, indem man $\mathcal{G}(U) = G^{ \text{ \#Zsh.-komp. von }U}$ setzt.
	\end{enumerate}
\end{Bsp}
%%% - End Example: sheaves/presheaves ---------------------------------------------------------------------------------------------



%%% - Remark: F(0) ----------------------------------------------------------------------------------------------------------------
\begin{Bem}\label{rem:sheaf_0}
	Ist $\mathcal{F}$ Garbe von abelschen Gruppen auf $X$, so ist $\mathcal{F}(\emptyset) = 0$.
\end{Bem}
\begin{Bew}
	Sei $G = \mathcal{F}(\emptyset)$. Offenbar kann $\emptyset$ durch eine leere Überdeckung von offenen Teilmengen überdeckt werden.
	Für jedes $g \in G$ und jedes $i\in I$ gilt also $g\restriction U_i = g_i$. Da $\mathcal{F}$ eine Garbe ist, kann es also nur
	ein $g \in G$ geben und somit ist $G = 0$.
\end{Bew}
%%% - End Remark: F(0) ------------------------------------------------------------------------------------------------------------




%%% - Definition: (pre-) sheaf morphisms -----------------------------------------------------------------------------------------
\begin{Def}[Morphismen von Prägarben]
	\label{def:sheaf_morphism}
	Sei $X$ ein topologischer Raum und $\mathcal{F}, \mathcal{G}$ Prägarben auf $X$ mit Werten in $\mathcal{C}$.
	Ein Morphismus $\phi\colon \mathcal{F} \to \mathcal{G}$ ist eine natürliche Transformation von $\mathcal{F}$ nach $\mathcal{G}$, d.h.
	für jedes offene $U \subseteq X$ ist ein Morphismus $\phi_U\colon \mathcal{F}(U) \to \mathcal{G}(U)$ gegeben, so dass folgendes Diagramm für alle $U,U'$ mit $U \subseteq U'$ kommutiert:
	\begin{center}
	\begin{tikzpicture}
		\matrix (m) [matrix of math nodes, row sep=3em,
		column sep=2.5em, text height=1.5ex, text depth=0.25ex]
		{
			\mathcal{F}(U')	&	\mathcal{F}(U)	\\
			\mathcal{G}(U')	&	\mathcal{G}(U)	\\
		};
		\path[->, font=\scriptsize]
		(m-1-1) edge node[auto] {$\rho_{U}^{U'}$}	(m-1-2)
				edge node[auto] {$\phi_{U'}$}		(m-2-1)
		(m-2-1) edge node[auto] {$\rho_{U}^{U'}$}	(m-2-2)
		(m-1-2) edge node[auto] {$\phi_{U}$}		(m-2-2);
	\end{tikzpicture}
	\end{center}

\end{Def}
%%% - End Definition (pre-) sheaf morphisms --------------------------------------------------------------------------------------



Im Folgenden ist mit einer Garbe auf $X$ immer eine Garbe von abelschen Gruppen gemeint.



%%% - Definition: stalk / germ ----------------------------------------------------------------------------------------------------
\begin{Def}[Halm und Keim]
	\label{def:stalk_germ}
	Sei $X$ ein topologischer Raum, $x\in X$ und $\mathcal{F}$ eine Prägarbe auf $X$.
	\begin{enumerate}[(a)]
		\item \[\mathcal{F}_{x} = \directlim_{x\in U \in \Off(X)} \mathcal{F}(U)\] 
		heißt \emph{Halm} von $\mathcal{F}$ in $x$. Dabei ist
		\[ \directlim \mathcal{F}(U) = \left\{ (U,f)\ |\ U \in \Off(X), x \in U, f \in \mathcal{F}(U) \right\} \big/ \sim \]
		mit $(U,f) \sim (U',f') :\Leftrightarrow $ es gibt eine offene Menge $U'' \subseteq U \cap U'$, so dass $x \in U''$ und $f \restriction U'' = f'\restriction U''$.
		\item Für eine offene Menge $U \subseteq X$ mit $x\in U$ sei
		\[ \mathcal{F}(U) \to \mathcal{F}_{x},\ f \mapsto [ (U,f) ]_{\sim} =: f_x \]
		der natürliche Morphismus. $f_x$ heißt \emph{Keim} von $f$ in $x$.
	\end{enumerate}
\end{Def}
%%% - End Definition: stalk / germ ------------------------------------------------------------------------------------------------



%%% Remark: local 0 <-> global 0 --------------------------------------------------------------------------------------------------
\begin{Bem}\label{bem:local=0_iff_global=0}
	Sei $\mathcal{F}$ eine Garbe auf $X$, $U \subseteq X$ eine offene Teilmenge und $f\in \mathcal{F}(U)$. Dann gilt:
	\[ f = 0 \Leftrightarrow f_x = 0 \text{ für alle } x \in U \]
\end{Bem}
\begin{Bew}
	\begin{description}
		\item["`$\Rightarrow$"':] Ist $f = 0$, dann ist offenbar $f_x = 0$ für alle $x \in U$.
		\item["`$\Leftarrow$"':] Sei $f_x = 0$ für alle $x \in U$. Dann gibt es für jedes $x\in U$ eine offene Umgebung $U_x$ von $x$, so dass $( U_x,0) \in f_x$ und damit insbesondere $(U_x,0 ) \sim (U,f)$. Die $U_x$ überdecken $U$ und daher gibt es genau ein $g \in \mathcal{F}(U)$ mit $g \restriction U_x = 0$ für jedes $x\in X$ $\Rightarrow 0 = g = f$.
	\end{description}
\end{Bew}
%%% End Remark: local 0 <-> global 0 -----------------------------------------------------------------------------------------------



%%% Example: local 0 <-> global 0 ONLY for sheaves ---------------------------------------------------------------------------------
Das folgende Beispiel zeigt, dass die Aussage aus Bemerkung~\ref{bem:local=0_iff_global=0} für Prägarben nicht unbedingt gilt.
\begin{Bsp}
	Sei $X$ ein topologischer Raum, so dass jedes $x \in X$ eine offene Umgebung $U \neq X$ besitzt.
	\[ \mathcal{F}(U) = \begin{cases}%
	\ZZ	& U = X\\%
	0	& \text{sonst}%
	\end{cases}
	\]
	mit den natürlichen Restriktionsmorphismen ist eine Prägarbe von abelschen Gruppen auf $X$.
	Für alle $x\in X$ ist $\mathcal{F}_x = 0$, also ist auch für jedes $f \in \mathcal{F}(X)$ und jedes $x\in X$ $f_x = 0$ -- auch wenn $f \neq 0$.
\end{Bsp}
%%% End Example: local 0 <-> global 0 ONLY for sheaves -----------------------------------------------------------------------------



%%% Remark: sheaf morphism -> stalk morphism ----------------------------------------------------------------------------------------
\begin{Bem}
	\label{rem:sheaf_morphism:induces_stalk_morphism}
	Jeder Morphismus $\phi\colon \mathcal{F} \to \mathcal{G}$ von Prägarben induziert für jedes $x\in X$ einen natürlichen Morphismus $\phi_x\colon \mathcal{F}_x \to \mathcal{G}_x$. 
\end{Bem}
\begin{Bew}
	Sei $x\in X$. Definiere
	\[ \phi_x\colon \mathcal{F}_x \to \mathcal{G}_x,\ [ (U,f) ]_{\sim} \mapsto [ (U,\phi_{U}(f) ]_{\sim} \]
	Für $(U,f) \sim (U',f')$ ist $f\restriction U'' = f'\restriction U''$ für ein geeignetes $U''$ und daher
	\[ \phi_{U'}( f' ) \restriction U'' = \phi_{U''}( f' \restriction U'' ) = \phi_{U''}( f \restriction U'' ) = \phi_{U}( f ) \restriction U'' \]
	Somit ist auch $(U , \phi_{U}(f)) \sim (U', \phi_{U'}(f'))$ und $\phi_x$ ist wohldefiniert.
\end{Bew}
%%% End Remark: sheaf morphism -> stalk morphism -------------------------------------------------------------------------------------



%%% Remark: sheaf morphism epi/mono/iso <-> stalk morphism epi/mono/ist --------------------------------------------------------------
\begin{Bem}\label{bem:sheaf_morphism:epi_mono_iso_stalk_morphism}
	Seien $\mathcal{F}, \mathcal{G}$ Garben abelscher Gruppen, $\phi\colon \mathcal{F} \to \mathcal{G}$ ein Morphismus. Dann gilt:\\
	{%
	\centering
		\begin{tabular}{clcl}
		(a)	& $\forall U \in \Off(X)\colon\phi_U$ ist injektiv		& $\Longleftrightarrow$ & $\forall x \in X\colon \phi_x$ ist injektiv.\\
		(b) & $\forall U \in \Off(X)\colon\phi_U$ ist surjektiv		& $\Longrightarrow$		& $\forall x \in X\colon \phi_x$ ist surjektiv.\\
		(c) & $\forall U \in \Off(X)\colon\phi_U$ ist Isomorphismus	& $\Longleftrightarrow$	& $\forall x \in X\colon \phi_x$ ist Isomorphismus.
		\end{tabular}
	}
\end{Bem}
\begin{Bew}
	\begin{enumerate}[(a)]
	\item
		\begin{description}
			\item[ "`$\Rightarrow$"':] Seien $x \in X$ und  $f_x \in \mathcal{F}_x$ mit $\phi_x(f_x) = 0$.
				Dann ist $[ (U, \phi_U(f) ) ]_{\sim} = 0$ für einen Repräsentanten $(U,f)$ von $f_x$.
				Ohne Einschränkung ist $\phi_U(f) = 0$ und nach Vorraussetzung somit auch $f = 0$ $\Rightarrow f_x = 0$.
			\item["`$\Leftarrow$"':] Seien $U \in \Off(X)$ und $f \in \mathcal{F}(U)$ mit $\phi_{U}(f) = 0$.
				Für alle $x\in U$ ist dann $\phi_x(f_x) = 0$ und somit auch $f_x = 0$.
				Nach Bemerkung~\ref{bem:local=0_iff_global=0} ist $f = 0$.
		\end{description}

	\item Sei $g_x \in \mathcal{G}_x$ für ein $x \in X$ und sei $(U,g)$ ein Repräsentant von $g_x$.
		Nach Vorraussetzung gibt es ein $f \in \mathcal{F}(U)$, so dass $\phi_U(f) = g$. Insgesamt ist dann $\phi_x(f_x) = g_x$.
	\item 
		\begin{description}
			\item[ "`$\Rightarrow$"':] Folgt aus (a) und (b).
			\item[ "`$\Leftarrow$"':] Nach (a) ist $\phi_{U}$ injektiv und es bleibt nur zu zeigen, dass $\phi_{U}$ surjektiv ist.
				Sei also $ g\in \mathcal{G}(U)$.
				Für jedes $x \in U$ sei $f_x = \phi_x^{-1}(g_x)$ und $(U^{(x)}, f^{(x)} )$ ein Repräsentant von $f_x$.
				Offenbar ist $\left(  U^{(x)} \right)_{x \in U}$ eine offene Überdeckung von $U$.
				Weiterhin kann man die $U^{(x)}$ klein genug wählen, so dass $\phi_{U^{(x)}}( f^{(x)} ) = g \restriction U^{(x)}$.
				Dann ist $f^{(x)} = \phi_{U^{(x)}}^{-1}( g \restriction U^{(x)} )$ und für alle $x,x' \in U$ gilt:
				\[ f^{(x)}\restriction U^{(x)} \cap U^{(x')} = \phi_{U^{(x)} \cap U^{(x')}}^{-1}( g \restriction U^{(x)} \cap U^{(x')} ) = f^{(x')} \restriction U^{(x)} \cap U^{(x')} \]
				Da $\mathcal{F}$ Garbe ist, gibt es genau ein $f \in \mathcal{F}(U)$ mit $f \restriction U^{(x)} = f^{(x)}$ für alle $x \in U$. Offenbar ist dann $\phi_U(f) \restriction U^{(x)} = g \restriction U^{(x)}$ für jedes $x \in U$ und somit auch $\phi_U(f) = g$.
		\end{description}
	\end{enumerate}
\end{Bew}
%%% End Remark: sheaf morphism epi/mono/iso <-> stalk morphism epi/mono/ist ----------------------------------------------------------



%%% Example: All stalk morphisms epi, but sheaf morphism not epi ---------------------------------------------------------------------
Das folgende Beispiel zeigt, dass die Aussage (b) aus Bemerkung~\ref{bem:sheaf_morphism:epi_mono_iso_stalk_morphism} keine Äquivalenz ist.
\begin{Bsp}
	Sei $X = \CC \setminus \{0\}$ und $\mathcal{F}$ die Garbe der invertierbaren, holomorphen Funktionen.
	Weiter sei $\phi\colon \mathcal{F} \to \mathcal{F}$ durch $f \mapsto f^2$ gegeben. Dann ist $\phi_x$ für jedes $x \in X$ surjektiv, $\phi_X$ hingegen nicht.
\end{Bsp}
%%% End Example: All stalk morphisms epi, but sheaf morphism not epi -----------------------------------------------------------------



%%% Definition: Associated Sheaf -----------------------------------------------------------------------------------------------------
\begin{BemDef}[Assoziierte Garbe]\label{def:associated_sheaf}
	Sei $X$ ein topologischer Raum, $\mathcal{F}$ eine Prägarbe von abelschen Gruppen auf $X$.
	\begin{enumerate}[(a)]
		\item Es gibt genau eine Garbe $\mathcal{F}^+$ auf $X$ und einen Morphismus $\theta\colon \mathcal{F} \to \mathcal{F}^+$, so dass
			$\theta_x\colon \mathcal{F}_x \to \mathcal{F}^+_x$ für jedes $x \in X$ ein Isomorphismus ist.
		\item $\mathcal{F}^+$ heißt die zu $\mathcal{F}$ \emph{assoziierte Garbe}.
		\item Zu jeder Garbe $\mathcal{G}$ auf $X$ und jedem Morphismus $\phi\colon \mathcal{F} \to \mathcal{G}$ von Prägarben gibt es genau einen Morphismus $\phi^+\colon \mathcal{F}^+ \to \mathcal{G}$, so dass folgendes Diagramm kommutiert:
		\begin{center}
		\begin{tikzpicture}
			\matrix (m) [matrix of math nodes, row sep=3em, column sep=2.5em, text height=1.5ex, text depth=0.25ex]
			{ 
				\mathcal{F}	&	& \mathcal{F}^{+}	\\
					&	\mathcal{G}	&				\\
			};
			\path[->,font=\scriptsize]
			(m-1-1) edge node[auto] {$ \theta $}	(m-1-3)	
					edge node[auto] {$ \phi $}		(m-2-2)
			(m-1-3) edge node[auto] {$\exists!\ \phi^+$ } (m-2-2);
		\end{tikzpicture}
		\end{center}
	\end{enumerate}
\end{BemDef}
\begin{Bew}
\begin{enumerate}[(a)]
	\item
	Für jede offene Menge $U \subseteq X$ sei
	\begin{align*}
	 \mathcal{F}^+(U) = \bigg\{ s\colon U \to \bigcup^{\centerdot}_{x \in U} \mathcal{F}_x\ \Big|\
			&\forall x \in U \text{ ist }  s(x) \in \mathcal{F}_x \text{ und } \exists \text{ Umgebung } U_x \text{ von } x\\
			&\text{ und ein } f \in \mathcal{F}(U_x) \text{ mit } s(y) = f_y \text{ für jedes } y \in U_x 
			& \bigg\}
	\end{align*}
	\end{enumerate}
	Dann ist $\mathcal{F}^+$ zusammen mit den offensichtlichen Restriktionen Garbe auf $X$.
	Weiter ist $\theta: \mathcal{F} \to \mathcal{F}^+,\ \theta_U(f) = ( x \mapsto f_x )$ ein Morphismus und $\theta_x$ ist Isomorphismus für
	jedes $x \in X$. Die Eindeutigkeit von $\mathcal{F}^+$ und $\theta$ folgt aus (c).
	\item[(c)]\todo{Beweis klarer machen!}
	Sei $\phi\colon \mathcal{F} \to \mathcal{G}$ ein Morphismus von Prägarben.
	Ist $s \in \mathcal{F}^+(U)$, dann ist $(U_x)_{x \in U}$ eine offene Überdeckung von $U$.
	Für $x,x' \in U$ gibt es $f^{(x)} \in \mathcal{F}(U_x)$ und $f^{(x')} \in \mathcal{F}(U_{x'})$, so dass 
	$s(y) = f^{(z)}_{y}$ für jedes $y \in U_z$ und $z \in \{x,x'\}$.
	Daher ist $f^{(x)}_{y} = f^{(x')}_{y}$ für jedes $y \in U_{x} \cap U_{x'}$ und für jedes $y \in U_{x}\cap U_{x'}$ gibt es eine Umgebung $U'$ von $y$, so dass
	$f^{(x)} \restriction U' = f^{(x')}\restriction U'$.
	Weil die $U'$ eine Überdeckung von $U_{x}\cap U_{x'}$ sind, ist $f^{(x)} \restriction U_{x} \cap U_{x'} = f^{(x')} \restriction U_{x} \cap U_{x'}$ und insbesondere
	$\phi_{U_x}( f^{(x)} )\restriction U_{x}\cap U_{x'} = \phi_{U_{x'}}( f^{(x')} ) \restriction U_{x} \cap U_{x'}$.
	Da $\mathcal{G}$ eine Garbe ist, gibt es ein eindeutig bestimmtes $g \in \mathcal{G}(U)$, so dass $g \restriction U_x = \phi_{U_x}( f^{(x)} )$.
	Definiert man nun $\phi^+(s) = g$, dann ist offenbar $\phi = \phi^+ \circ \theta$ und $\phi^+$ ist eindeutig.
\end{Bew}
%%% End Definition: Associated Sheaf -------------------------------------------------------------------------------------------------



%%% Definition: Kern/Image of Sheaf morphisms ----------------------------------------------------------------------------------------
\begin{BemDef}[Kern, Bild, Mono- und Epimorphismen]
	\label{def:sheaf_morphism:kern_image}
	Sei $\phi\colon \mathcal{F} \to \mathcal{G}$ ein Morphismus von Garben auf $X$.
	\begin{enumerate}[(a)]
		\item $\ker(\phi)$ mit $\ker(\phi)(U) = \ker(\phi_U)$ ist Garbe.
		\item $\img(\phi)$ sei die zu $U \mapsto \img(\phi_U)$ assoziierte Garbe.
		\item $\phi$ heißt \emph{Monomorphismus}, falls $\ker(\phi) = 0$.
		\item $\phi$ heißt \emph{Epimorphismus}, falls $\img(\phi) = \mathcal{G}$.
	\end{enumerate}
\end{BemDef}
%%% End Definition Kern/Image of Sheaf morphisms -------------------------------------------------------------------------------------



%%% Definition: Quotient Sheaf -------------------------------------------------------------------------------------------------------
\begin{Def}[Quotientengarbe]
	\label{def:quotient_sheaf}
	Seien $\mathcal{G} \mono \mathcal{F}$ Garben von abelschen Gruppen auf $X$. Die zur Prägarbe $U \mapsto \mathcal{F}(U) \big/ \mathcal{G}(U)$ assoziierte Garbe heißt \emph{Quotientengarbe} von $\mathcal{F}$ nach $\mathcal{G}$.
\end{Def}
%%% End Definition Quotient Sheaf ----------------------------------------------------------------------------------------------------



%%% Example: Quotient Sheaf ----------------------------------------------------------------------------------------------------------
\begin{Bsp}
	Sei $\mathcal{F} = \mathcal{C}_x$ die Garbe der stetigen Funktionen von $S^1$ nach $\RR$ und $\mathcal{G}$ die konstante Garbe zu $\ZZ$ auf $S^1$.
	In Abbildung~\ref{fig:quotient_sheave} ist eine Überlagerung von $S^1$ durch $U_1,U_2$ zu sehen, so dass $U_1 \cap U_2 = D_1 \overset{\centerdot}{\cup} D_2$ für zwei offene Mengen $D_1,D_2$.
	\begin{figure}
		\centering
		\begin{tikzpicture}
			\draw (0,0) circle (2cm);
			\draw[red, very thick] (-10:2.1cm) arc(-10:190:2.1cm);
			\draw[blue, very thick] (-190:2.2cm) arc(-190:10:2.2cm);
			\draw[very thick] (10:2.3cm) arc(10:-10:2.3cm);
			\draw[very thick] (170:2.3cm) arc(170:190:2.3cm);
			\path	node at (0,-2.5cm) {$U_2$}
					node at (0,2.5cm) {$U_1$}
					node at (0,0) {$S^1$}
					node at (2.6cm,0) {$D_2$}
					node at (-2.6cm,0) {$D_1$};
		\end{tikzpicture}
		\caption{Überlagerung von $S^1$ durch $U_1$ (rot) und $U_2$ (blau)}
		\label{fig:quotient_sheave}
        \end{figure}

	Seien nun $0 = f_1 \in \mathcal{F}(U_1)$ und $f_2 \in \mathcal{F}(U_2)$ mit $f_2 \restriction D_1 = 0$ und $f_2 \restriction D_2 = 1$.
	Dann ist $f_2 - f_1 \in \mathcal{G}( U_1 \cap U_2 )$ und daher $\bar{f_1} = \bar{f_2}$ in $\mathcal{F}\big/\mathcal{G}(S^1)$.
\end{Bsp}
%%% End Example: Quotient Sheaf ------------------------------------------------------------------------------------------------------



%%% Definition: Direct and inverse image Sheaf  --------------------------------------------------------------------------------------
\begin{BemDef}[Direkte und inverse Bildgarbe]
	\label{def:image_sheaf}
	Sei $f: X \to Y$ stetig.
	\begin{enumerate}[(a)]
		\item Sei $\mathcal{F}$ Garbe auf $X$, dann ist die Prägarbe $U \mapsto \mathcal{F}( f^{-1}(U) )$ auf $Y$ eine Garbe.
		Sie heißt die \emph{direkte Bildgarbe} und wird mit $f_{*}\mathcal{F}$ bezeichnet.
		\item Sei $\mathcal{G}$ Garbe auf $Y$, dann heißt die zur Prägarbe 
		\[ U \mapsto \directlim_{\substack{V \subseteq Y\text{ offen}\\f(U) \subseteq V}} \mathcal{G}(V) \]
		assoziierte Garbe $f^{-1}\mathcal{G}$ \emph{inverse Bildgarbe} zu $\mathcal{G}$.
		\item $f_{*}$ und $f^{-1}$ sind kovariante Funktoren
		\item $f^{-1}$ ist linksadjungiert zu $f_{*}$, d.h. es gibt natürliche Bijektionen
		\[ \Hom(f^{-1}\mathcal{G}, \mathcal{F}) \to \Hom( \mathcal{G}, f_{*}\mathcal{F} ) \]
	\end{enumerate}
\end{BemDef}
\begin{Bew}
	\begin{enumerate}[(a)]
		\item Da $\mathcal{F}$ Garbe auf $X$ und $f^{-1}(U)$ offen ist für jedes $U \subseteq Y$, ist $f_{*}\mathcal{F}$ Garbe auf $Y$.
		\item[(c)] Offensichtlich.
		\item[(d)] Es sollen natürliche Bijektionen 
		$ \Hom(f^{-1}\mathcal{G}, \mathcal{F}) \to \Hom( \mathcal{G}, f_{*}\mathcal{F} ) $
		konstruiert werden.

		\minisec{Der Weg von $\Hom( f^{-1}\mathcal{G}, \mathcal{F} )$ nach $\Hom( \mathcal{G}, f_{*}\mathcal{F} )$}
		Jedes $\alpha \in \Hom( f^{-1}\mathcal{G}, \mathcal{F} )$ induziert einen Morphismus
		$ f_{*}(\alpha)\colon f_{*}f^{-1}\mathcal{G} \to f_{*}\mathcal{F}$.\\
		$f_{*}(\alpha)$ kann fortgesetzt werden zu einem Morphismus
		\[ \mathcal{G} \xrightarrow{\psi_{\mathcal{G}}} f_{*}f^{-1} \mathcal{G} \xrightarrow{f_{*}(\alpha)} f_{*}\mathcal{F} \]
		Dazu ist folgende Konstruktion nötig:
		Die universelle Eigenschaft des direkten Limes liefert einen natürlichen Morphismus
		\[ \mathcal{G}(V) \to \directlim_{ f(f^{-1}(V)) \subseteq W \subseteq V } \mathcal{G}(W) 
							= \directlim_{f(f^{-1}(V)) \subseteq W } \mathcal{G}(W)
		\]
		Dabei beruht die Gleichheit der direkten Limetes darauf, dass $f(f^{-1}(V)) \subseteq V$ und somit ohne Einschränkung jedes $W \supset f(f^{-1}(V))$ mit $V$ geschnitten werden kann.
		Nun ist $f^{-1} \mathcal{G}$ die zu 
		\[ V \mapsto \directlim_{f(V) \subseteq W} \mathcal{G}( W ) \]
		assoziierte Garbe und man erhält
		$ \psi_{\mathcal{G}}(V)\colon \mathcal{G}(V) \to f^{-1} \mathcal{G}( f^{-1}(V) ) = f_{*} f^{-1} \mathcal{G}(V) $

		\minisec{Der Weg von $\Hom( \mathcal{G}, f_{*}\mathcal{F} )$ nach $\Hom( f^{-1}\mathcal{G}, \mathcal{F} )$}
		Jedes $\beta \in \Hom( \mathcal{G}, f_{*}\mathcal{F} )$ induziert einen Morphismus
		$ f^{-1}(\beta)\colon f^{-1} \mathcal{G} \to f^{-1}f_{*}\mathcal{F} $\\
		Auch $f^{-1}(\beta)$ lässt sich fortsetzen zu
		\[ 
			f^{-1}\mathcal{G} \xrightarrow{f^{-1}(\beta)} f^{-1}f_{*}\mathcal{F} \xrightarrow{\phi_{\mathcal{F}}} \mathcal{F} 
		\]
		$f^{-1}f_{*}\mathcal{F}$ ist die zu 
		\[ 
				U \mapsto \directlim_{f(U) \subseteq V } f_{*}\mathcal{F}(V) 
					= \directlim_{f(U) \subseteq V } \mathcal{F}(f^{-1}(V))
		\]
		assoziierte Garbe und daher reicht es für jedes $U$ einen Morphismus 
		\[ 
			\chi_{\mathcal{F}}(U)\colon \directlim_{f(U) \subseteq V} \mathcal{F}(f^{-1}(V)) \to \mathcal{F}(U) 
		\]
		zu konstruieren. Für jedes $V$ mit $f(U) \subseteq V$ ist $U \subseteq f^{-1}(V)$, also gibt es Restriktionsmorphismen
		$\mathcal{F}(f^{-1}(V)) \to \mathcal{F}(U)$. Die universelle Eigenschaft des direkten Limes liefert nun einen eindeutigen
		Morphismus $\chi_{\mathcal{F}}(U)$ der wiederum $\phi_{\mathcal{F}}(U)$ induziert.
		\minisec{Die beiden Konstruktionen sind zueinander invers}
		Das ist so.
	\end{enumerate}
\end{Bew}
%%% Definition: Direct and inverse image Sheaf  --------------------------------------------------------------------------------------




%%% Remark: Sheaves as left-exact functors --------------------------------------------------------------------------------------------
\begin{Bem}
	Sei $X$ topologischer Raum, $U \subseteq X$ offen. Dann ist $\mathcal{F} \mapsto \mathcal{F}(U)$ linksexakter, kovarianter Funktor.
\end{Bem}
\begin{Bew}
	\todo{Beweis klarer machen}
	Ist $\phi\colon \mathcal{F} \to \mathcal{G}$ Morphismus, so ist $\phi_{U}\colon \mathcal{F}(U) \to \mathcal{G}(U)$ der zugehörige Morphismus.
	Sei nun 
	\[
		0 \rightarrow \mathcal{F}' \xrightarrow{\phi} \mathcal{F} \xrightarrow{\psi} \mathcal{F}'' \rightarrow 0 
	\]
	eine kurze exakte Sequenz von Garben. Nach Definition ist $\mathcal{F} \mapsto \mathcal{F}(U)$ linksexakt, falls
	\[
		0 \rightarrow \mathcal{F}'(U) \xrightarrow{\phi_{U}} \mathcal{F}(U) \xrightarrow{\psi_{U}} \mathcal{F}''(U)
	\]
	exakt ist.

	Nach Definition~\ref{def:sheaf_morphism:kern_image} und Bemerkung~\ref{bem:sheaf_morphism:epi_mono_iso_stalk_morphism} ist
	\[
		0 \rightarrow \mathcal{F}'_x \rightarrow \mathcal{F}_x \rightarrow \mathcal{F}''_x \rightarrow 0 
	\]
	exakt für jedes $x\in X$.
	Wiederum nach Bemerkung \ref{bem:sheaf_morphism:epi_mono_iso_stalk_morphism} ist
	\[
		0 \rightarrow \mathcal{F}'(U) \rightarrow \mathcal{F}(U) \rightarrow \mathcal{F}''(U)
	\]
	exakt.
\end{Bew}
%%% End Remark: Sheaves as left-exact functors -----------------------------------------------------------------------------------------

%% -- CHAPTER 2: Affine Schemes ----------------------------------------------------------------------------------------------------------
\section{Affine Schemata}


%%% Definition: Spectrum, Zariski-Topology -----------------------------------------------------------------------------------------------
\begin{BemDef}[Spektrum, Zariski-Topologie und Verschwindungsideal]

	\label{def:spectrum}
	\label{def:zariski_topology}

	Sei $R$ ein Ring.
	\begin{enumerate}[(a)]
		\item $\Spec R = \left\{ \mathfrak{p} \subseteq R\ |\ \mathfrak{p} \text{ Primideal}\right\}$ heißt \emph{Spektrum} von $R$.
		\item Für $I \subseteq R$ sei $V\left(I\right) = \left\{ \mathfrak{p} \in \Spec R\ |\ I \subseteq \mathfrak{p} \right\}$. 
		
		Es gilt $V\left( I \right) = V( ( I ) )$.
		\item $\left\{ V\left(I\right)\ |\  I \text{ ist Ideal in } R\right\}$ sind abgeschlossene Mengen einer Topologie auf $\Spec R$, der \emph{Zariski-Topologie}.
		\item Für $Z \subseteq \Spec R$ sei $I\left(Z\right) = \bigcap_{\mathfrak{p} \in Z} \mathfrak{p}$ das \emph{Verschwindungsideal} von $Z$.
	\end{enumerate}
\end{BemDef}
%%% End Definition: Spectrum, Zariski-Topology --------------------------------------------------------------------------------------------



%%% Anmerkung: subset, I, V
\begin{anmerkung}
	\begin{enumerate}[(a)]
		\item Ist $A \subseteq B \subseteq \Spec R$, dann ist $I\left(A\right) \supseteq I(B)$.
		\item Ist $I \subseteq J \subseteq R$, dann ist $V\left(I\right) \supseteq V(J)$.
	\end{enumerate}
	\begin{Bew}
		\begin{enumerate}[(a)]
			\item Ist $A \subseteq B$, dann ist
				\[ 
					I\left(A\right) = \bigcap_{\mathfrak{p}\in A} \mathfrak{p} 
					\overset{A \subseteq B}{\supseteq} 
					\bigcap_{\mathfrak{p} \in B} \mathfrak{p} = I\left(B\right)
				\]
			\item Ist $I \subseteq J$, dann ist
				\[
					V\left(I\right) = \left\{ \mathfrak{p}\ |\ I \subseteq \mathfrak{p} \right\} 
					\overset{I \subseteq J}{\supseteq}
					\left\{ \mathfrak{p}\ |\ J \subseteq \mathfrak{p} \right\}
					= V\left(J\right)
				\]
		\end{enumerate}
	\end{Bew}
\end{anmerkung}
%%% End Anmerkung






%%% Remark: 'Inverse' I\left(V(I\right)), V(I(V)) ----------------------------------------------------------------------------------------------------
\begin{Bem}
	\begin{enumerate}[(a)]
		\item $V\left( I\left(Z\right) \right) = \closure{Z}$
		\item $I\left( V\left(I\right) \right) = \rad{I}$
	\end{enumerate}
\end{Bem}
\begin{Bew}
	\begin{enumerate}[(a)]
		\item 
			\begin{description}
				\item["`$\supset$"':] Nach Definition ist $V\left( I\left(Z\right) \right)$ abgeschlossen und daher gilt $\closure{Z} \subseteq V\left(I\left(Z\right)\right)$.
				\item["`$\subseteq$"':] 
					Nach Definition ist
					\[
					\closure{Z} = \bigcap_{\substack{I \text{ Ideal}\\
						Z \subseteq V\left(I\right)}} V(I)
					\]
					Aus $Z \subseteq V\left(I\right)$ folgt $I \subseteq \mathfrak{p}$ für alle $\mathfrak{p} \in Z$. Somit ist
					\[
					I \subseteq \bigcap_{\mathfrak{p} \in Z} \mathfrak{p} = I\left(Z\right)
					\]
					und deshalb $V\left(I\right) \supset V(I(Z))$.
			\end{description}
		\item
			\[ I\left(V\left(I\right)\right)	= \bigcap_{\mathfrak{p} \in V\left(I\right)} \mathfrak{p} 
						= \bigcap_{ \substack{ \mathfrak{p} \text{ Primideal}\\
							I \subseteq \mathfrak{p}}} \mathfrak{p}
						= \rad{I}
			\]
	\end{enumerate}
\end{Bew}
%%% End remark: 'Inverse' I\left(V(I\right)), V(I(V)) ------------------------------------------------------------------------------------------------




%%% Anmerkung Ideal Kalkül und ``Varietäten''
\begin{anmerkung}
	\begin{enumerate}[(a)]
		\item Sind $\left(I_j\right)_{j \in J}$ Ideale, dann ist
			\[
				\bigcap_{j\in J} V\left(I_j\right) = V\left( \sum_{j\in J} I_j \right)
			\]
		\item Sind $I_1, I_2$ Ideale, dann ist
			\[
				V\left(I_1\right) \cup V\left(I_2\right) = V\left( I_1 \cdot I_2 \right) = V\left( I_1 \cap I_2 \right)
			\]
	\end{enumerate}
	\begin{Bew}
		\begin{enumerate}[(a)]
			\item 
				\begin{description}
					\item[``$\subseteq$'':]
						Ist $\mathfrak{p} \in \bigcap V\left(I_j\right)$, dann ist $I_j \subseteq \mathfrak{p}$ für jedes $j \in J$.
						Also ist auch $\sum I_j \subseteq \mathfrak{p}$ und somit $\mathfrak{p} \in V\left( \sum I_j \right)$.
					\item[``$\supseteq$'':]
						Ist $\mathfrak{p} \in V\left( \sum I_j \right)$, dann ist $I_j \subseteq \sum I_j \subseteq \mathfrak{p}$ für jedes $j \in J$ und
						somit ist $\mathfrak{p} \in \bigcap V\left(I_j\right)$.
				\end{description}
			\item
				\begin{description}
					\item[``$\subseteq$'':]
						Ist $\mathfrak{p} \in V\left(I_1\right) \cup V(I_2)$, dann ist $I_1 \subseteq \mathfrak{p}$ oder $I_2 \subseteq \mathfrak{p}$.
						Auf jeden Fall ist aber $I_1 \cdot I_2 \subseteq \mathfrak{p}$ und $I_1 \cap I_2 \subseteq \mathfrak{p}$ und somit
						$V\left(I_1\right) \cup V(I_1) \subseteq V\left(I_1 \cdot I_2\right)$ und 
						$V(I_1) \cup V\left(I_2\right) \subseteq V\left( I_1 \cap I_2 \right)$.
					\item[``$\supseteq$'':]
						Es gilt: $I_1 \cdot I_2 \subseteq I_1 \cap I_2$ und somit 
						$V\left(I_1 \cdot I_2\right) \supseteq V\left(I_1 \cap I_2\right)$. Also
						genügt es zu zeigen, dass $V\left(I_1 \cdot I_2 \right) \subseteq V\left(I_1\right) \cup V\left(I_2\right)$.

						Ist also $\mathfrak{p} \in V\left(I_1 \cdot I_2\right)$, dann ist $I_1 \cdot I_2 \subseteq \mathfrak{p}$.
						Angenommen $I_2 \nsubseteq \mathfrak{p}$. Dann gibt es ein $a \in I_2$, so dass $a \notin \mathfrak{p}$.
						Nach Vorraussetzung ist aber $aI_1 \subseteq \mathfrak{p}$ und somit ist auch $I_1 \subseteq \mathfrak{p}$,
						insbesondere also $\mathfrak{p} \in V\left(I_1\right)$.
				\end{description}
		\end{enumerate}
	\end{Bew}
\end{anmerkung}
%%% End Anmerkung Kalkül der Ideal <-> ``Varietäten''




%%% Remark: V irreducible iff I\left(V\right) prime --------------------------------------------------------------------------------------------------
\begin{Bem}\label{bem:irreducible-prime_ideal}
	Sei $\emptyset \neq V \subseteq \Spec R$ abgeschlossen. $I\left(V\right)$ ist ein Primideal, genau dann wenn $V$ irreduzibel ist.
\end{Bem}
\begin{Bew}
	\begin{description}
		\item["`$\Leftarrow$"':]
			Sei $V \subseteq \Spec R$ abgeschlossen, dann gibt es ein Ideal $I \subseteq R$, so dass $V = V\left(I\right)$.
			Seien nun $a,b \in R$ mit $ab \in I\left(V\right)$.
			Nach Definition ist
			\[ I\left(V\right) = \bigcap_{I \subseteq \mathfrak{p} } \mathfrak{p} \]
			und daher ist für jedes Primideal $\mathfrak{p}$ mit $I \subseteq \mathfrak{p}$ offenbar $a \in \mathfrak{p}$ oder $b\in \mathfrak{p}$.
			Definiere nun $V_a = \left\{ \mathfrak{p}\ |\ \mathfrak{p} \text{ Primideal mit } I \subseteq \mathfrak{p} \text{ und } a \in \mathfrak{p} \right\}$ und $V_b$ analog.
			Offenbar ist $V = V_a \cup V_b$ und $V_a, V_b$ sind abgeschlossen.
			Da $V$ irreduzibel ist, kann man ohne Einschränkung $V_a = V$ annehmen. Dann ist aber offenbar auch $a \in I\left(V\right)$.
		\item["`$\Rightarrow$"':]
			Sei $V \subseteq \Spec R$ abgeschlossen, so dass $I(V)$ Primideal ist und 
			seien $V_1 = V\left(I_1\right)$ und $V_2 = V\left(I_2\right)$ abgeschlossene Mengen mit $V = V_1 \cup V_2$.
			Ohne Einschränkung sind $I_1,I_2$ Radikalideale, da 
			$V( \rad{I_i} ) = V( I( V_i ) ) = \closure{V_i} = V_i$ für $i \in \{1,2\}$

			Dann ist 
			$ V = V(I_1) \cup V(I_2) = V( I_1 \cap I_2 ) $.
			Da $I_1, I_2$ Radikalideale sind, ist $I_1 \cap I_2$ Radikalideal und daher ist 
			$I_1 \cap I_2 =\rad{I_1 \cap I_2} = I(V)$ ein Primideal.

			Ist nun $I_2 \nsubseteq I_1$, dann gibt es ein $b \in I_2 \setminus I_1$.
			Für jedes $a \in I_1$ ist $ab \in I_1 \cap I_2$ und daher $a \in I_1 \cap I_2$ oder $b \in I_1 \cap I_2$.
			Da $b$ aber aus $I_2 \setminus I_1$ gewählt war, muss $a \in I_1 \cap I_2$ und somit 
			$I_1 \subseteq I_1 \cap I_2$ sein.

			Somit ist aber $V_1 = V(I_1) \supseteq V(I_1 \cap I_2) = V$ 
			$\quad\Rightarrow V_1 = V$
	\end{description}
\end{Bew}
%%% End remark: V irreducible iff I\left(V\right) prime ----------------------------------------------------------------------------------------------



%%% Proposition: Ring-Morphism -> continouus f --------------------------------------------------------------------------------------------
\begin{Prop}
	Jeder Morphismus $\alpha\colon R \to R'$ von Ringen induziert durch $f_{ \alpha} \left(\mathfrak{p} \right) = \alpha^{-1}( \mathfrak{p} )$
	eine stetige Abbildung $f_{\alpha}\colon \Spec R' \to \Spec R$.
\end{Prop}
\begin{Bew}
	$\alpha^{-1}\left( \mathfrak{p} \right)$ ist Primideal. Ist $V(I) \subseteq \Spec R$ abgeschlossen, dann ist
	$f_{\alpha}^{-1}\left( V\left(I\right) \right) = V\left( \alpha\left( I \right) \right)$
\end{Bew}
%%% End Proposition Ring Morphism -> continouus f ----------------------------------------------------------------------------------------



%%% Remark: affine Varieties/Spectrum ----------------------------------------------------------------------------------------------------
\begin{Bem}
	Sei $k$ algebraisch abgeschlossen, $V \subseteq \mathbb{A}^{n}\left(k\right)$ affine Varietät.
	Dann ist 
	\[ m\colon V \to \Spec k[V],\ x \mapsto m_x \] 
	stetig und injektiv.
\end{Bem}
\begin{Bew}
	Die maximalen Ideale in $k[V]$ entsprechen bijektiv den Punkten in $V$. Also ist $m$ injektiv.
	Sei nun $V\left(I\right) \subseteq \Spec k[V]$ abgeschlossen, dann ist
	\begin{align*}
		m^{-1}\left( V(I\right)) &= \left\{ x \in V\ |\ m_x \in V\left(I\right) \right\}	\\
					&= \left\{ x \in V\ |\ I \subseteq m_x \right\} \\
					&= \left\{ x \in V\ |\ f\left(x\right) = 0 \text{ für alle } x\in I \right\} \\
					&= V\left(I\right) \text{ im Sinne von affinen Varietäten}
	\end{align*}
\end{Bew}
%%% End remark: affine Varieties/Spectrum -----------------------------------------------------------------------------------------------



%%% Definition: Generic Point -----------------------------------------------------------------------------------------------------------
\begin{BemDef}[Generischer Punkt]
	\begin{enumerate}[(a)]
		\item Ein Punkt $x$ in einem topologischen Raum $X$ heißt \emph{generisch}, falls $\closure{ \left\{x\right\} } = X$.
		\item Jede abgeschlossene, irreduzible Teilmenge von $\Spec R$ ($R$ ein Ring) besitzt genau einen generischen Punkt.
		\item Die maximalen, irreduziblen Teilmengen von $\Spec R$ entsprechen bijektiv den minimalen Primidealen in $R$.
	\end{enumerate}
\end{BemDef}
\begin{Bew}
	\begin{enumerate}[(a)]
		\item[(b)] Sei $V = V\left(I\right) \subseteq \Spec R$ abgeschlossen und irreduzibel.
		Nach Bemerkung~\ref{bem:irreducible-prime_ideal} ist $I\left(V\right) = \rad{I}$ ein Primideal.
		Es ist $I \subseteq \rad{I}$ und somit auch $\rad{I} \in V$. 
		Ist $W = V\left(J\right)$ eine abgeschlossene Menge mit $\rad{I} \in W$, dann ist $J \subseteq \rad{I}$ und für jedes
		Primideal $\mathfrak{p}$ mit $I \subseteq \mathfrak{p}$ ist $J \subseteq \rad{I} \subseteq \mathfrak{p}$
		$\Rightarrow \closure{ \left\{ \rad{I} \right\} } = V$.
		\item[(c)] Folgt aus Bemerkung~\ref{bem:irreducible-prime_ideal}.
	\end{enumerate}
\end{Bew}
%%% End Definition: Generic Point ------------------------------------------------------------------------------------------------------



%%% Remark: Basis of Z-Topology --------------------------------------------------------------------------------------------------------
\begin{BemDef}
	Für jedes $f \in R$ ist $D\left(f\right) = \Spec R \setminus V(f) = \left\{ \mathfrak{p} \in \Spec R\ |\ f \notin \mathfrak{p} \right\}$
	offen in $\Spec R$.
	$\left\{ D\left(f\right)\ |\ f \in R \right\}$ ist eine Basis der Zariski-Topologie auf $\Spec R$.
\end{BemDef}
\begin{Bew}
	Sei $U \subseteq \Spec R$ offen und $\mathfrak{p} \in U$.
	$V = \Spec R \setminus U$ ist abgeschlossen, also $V = V\left(I\right)$ für ein Ideal $I \subseteq R$.
	Für jedes $f \in I$ gilt $V\left(I\right) \subseteq V\left(f\right)$, also $D\left(f\right) \subseteq U$.
	Nun ist $\mathfrak{p} \in U = \left\{ \mathfrak{q}\ |\ I \nsubseteq \mathfrak{q} \right\}$, also gibt es ein $f \in I$, so dass $f \notin \mathfrak{p}$ und somit ist $\mathfrak{p} \in D\left(f\right)$.
\end{Bew}
%%% End remark: Basis of Z-Topology ---------------------------------------------------------------------------------------------------



%%% Anmerkung: Ohne Einschränkung
\begin{anmerkung}
	Ist $\left(U_i\right)_{i \in I}$ eine offene Überdeckung von $A \subseteq X$, dann kann man ohne Einschränkung annehmen, dass
	$U_i = D\left(f_i\right)$ für geeignete $f_i \in R$.
	\begin{Bew}
		Die $D\left(f\right)$ mit $f \in R$ bilden eine Basis der Topologie, also ist jedes $U_i$ Vereinigung von $D\left(f\right)$'s, woraus die Behauptung folgt.
	\end{Bew}
\end{anmerkung}
%%% End Anmerkung: Ohne Einschränkung



%%% Remark: Spec is quasi-compact -----------------------------------------------------------------------------------------------------
\begin{Bem}
	$\Spec R$ ist quasi-kompakt.
\end{Bem}
\begin{Bew}
	Sei $\left(U_i\right)_{i\in I}$ offene Überdeckung von $\Spec R$. 
	Ohne Einschränkung sei $U_i = D\left(f_i\right)$ für geeignetes $f_i \in R$.
	Dann gilt:
	\begin{align*}
		\bigcup_{i \in I} D\left(f_i\right) = \Spec R	& \Leftrightarrow \bigcap_{i\in I} V\left(f_i\right) = \emptyset \\
			& \Leftrightarrow \left( \sum_{i\in I} \left(f_i\right) \right) = R
	\end{align*}
	und daher gilt für geeignete $a_j$ und eine endliche Menge $J\subseteq I$:
	\[ 1 = \sum_{j \in J} a_j f_j \]
	bzw.
	\[ \bigcup_{j \in J} D\left(f_j\right) = \Spec R \]
\end{Bew}
%%% End Remark: Spec is quasi-compact ----------------------------------------------------------------------------------------------------



%%% Remark: D\left(f\right) is quasi-compact ----------------------------------------------------------------------------------------------------
\begin{Bem}
	Für jedes $f \in R$ ist $D\left(f\right) \subseteq \Spec R$ quasi-kompakt bzgl. der induzierten Topologie.
\end{Bem}
\begin{Bew}
	Sei $\left(U_i\right)_{i \in I}$ offene Überdeckung von $\Spec R$.
	Ohne Einschränkung sei $U_i = D\left(f_i\right) \cap D\left(f\right)$ für geeignetes $f_i \in R$.
	Dann gilt:
	\begin{align*}
		\bigcup_{i \in I} \left( D\left(f_i\right) \cap D\left(f\right) \right) = D\left(f\right) 
			& \Leftrightarrow \bigcup_{i \in I} D\left(f_i\right) \supseteq D\left(f\right) \\
			& \Leftrightarrow \bigcap_{i \in I} V\left(f_i\right) \subseteq V\left(f\right)\\
			& \Leftrightarrow \sum_{i \in I} \left(f_i\right) \supseteq \left(f\right)
	\end{align*}
	und daher gilt für geeignete $a_j$ und eine endliche Menge $J \subseteq I$:
	\[ f = \sum_{j \in J} a_j f_j \]
	bzw.
	\[ \bigcup_{j \in J} D\left(f_j\right) \supseteq  D\left(f\right) \]
\end{Bew}
%%% End Remark: D\left(f\right) is quasi-compact ------------------------------------------------------------------------------------------------



%%% Example: Spec R does not suffice ----------------------------------------------------------------------------------------------------
\begin{Bsp}
	Dieses Beispiel soll zeigen, dass $\Spec R$ alleine nicht ausreichend ist und so die folgende Definition motivieren.
	Seien $k$ ein Körper und $R = k[\varepsilon] \big/ \left(\varepsilon^2\right)$.
	Dann ist $\Spec R = \left\{ \left( \varepsilon \right) \right\}$ und
	\[ \alpha\colon R \to k,\ \varepsilon \mapsto 0 \]
	ist ein $k$-Algebra-Homomorphismus. $\alpha$ induziert eine stetige Abbildung $f_\alpha$. 
	Aus offensichtlichen Gründen ist $f_{\alpha}\colon \Spec k \to \Spec R$ sogar ein Homöomorphismus.

	Fazit: $\Spec R$ besitzt zu wenig Information über $R$.
\end{Bsp}
%%% End Example: Spec R does not suffice ------------------------------------------------------------------------------------------------



%%% Definition: Affine Scheme -----------------------------------------------------------------------------------------------------------
\begin{BemDef}[Strukturgarbe und affines Schema]
	\label{defbem:structure_germ}
	Sei $R$ ein Ring, $X = \Spec R$.
	\begin{enumerate}[(a)]
		\item Für $U \subseteq X$ offen sei
		\begin{align*}
			\mathcal{O}_X\left(U\right) = 
				\bigg\{ s\colon U \to \bigcup_{ \mathfrak{p} \in U }^{\centerdot} R_{\mathfrak{p}}\ \bigg|\
					& \text{Für alle } \mathfrak{p} \in U \text{ ist } s\left(\mathfrak{p}\right) \in R_{\mathfrak{p}}\\
					& \text{und es gibt eine Umgebung } U_{\mathfrak{p}} \text{ von } \mathfrak{p}
					\text{ sowie } f,g \in R\\ 
					& \text{so dass für alle } \mathfrak{q} \in U_{\mathfrak{p}}\colon g \notin \mathfrak{q} 
					\text{ und } s\left(\mathfrak{q}\right) = \frac{f}{g}
					& \bigg\}
		\end{align*}
		\item $\mathcal{O}_X$ ist eine Garbe von Ringen auf $X$. Sie heißt \emph{Strukturgarbe} von $\Spec R$.
		\item $\left(X, \mathcal{O}_X\right)$ heißt \emph{affines Schema}.
	\end{enumerate}
\end{BemDef}
%%% End Definition: Affine Scheme -------------------------------------------------------------------------------------------------------



%%% Proposition: Scheme does what it should ---------------------------------------------------------------------------------------------
\begin{Prop}
  \label{prop:2.10}
	Sei $\left(X = \Spec R, \mathcal{O}_X\right)$ ein affines Schema. Dann gilt:
	\begin{enumerate}[(a)]
		\item Für jedes $\mathfrak{p} \in X$ ist $\mathcal{O}_{X,\mathfrak{p}} \cong R_{\mathfrak{p}}$
		\item Für jedes $f \in R$ ist $\mathcal{O}_{X}\left(D(f\right)) \cong R_{f}$
	\end{enumerate}
\end{Prop}
\begin{Bew}
	\begin{enumerate}[(a)]
		\item
			Definiere $\psi\colon \mathcal{O}_{X,\mathfrak{p}} \to R_{\mathfrak{p}}$ durch $[\left(U,s\right)]_{\sim} \mapsto s( \mathfrak{p} )$.

			\minisec{$\psi$ ist wohldefinierter Ringhomomorphismus}
			
			\minisec{$\psi$ ist surjektiv}
				Sei $\frac{a}{f} \in R_{\mathfrak{p}}$ mit $a \in R, f \in R\setminus \mathfrak{p}$.
				Es ist $\mathfrak{p} \in U$ für $U = D\left(f\right)$.
				Für ein $\mathfrak{q} \in U$ definiere $s\left( \mathfrak{q} \right) = \frac{a}{f} \in R_{\mathfrak{q}}$.

				$\Rightarrow \psi\left( [ \left(U,s\right) ]_{\sim} \right) = \frac{a}{f}$, 
				wobei $[ \left(U,s\right) ]_{\sim} \in \mathcal{O}_{X,\mathfrak{p}}$.

			\minisec{$\psi$ ist injektiv}
				Sei $[\left(U,s\right)]_{\sim} \in \mathcal{O}_{X,\mathfrak{p}}$ mit $\psi( [(U,s)]_{\sim} ) = 0$, also 
				$s \left( \mathfrak{p} \right) = 0$ in $R_{\mathfrak{p}}$.
				Ohne Einschränkung gilt $s\left(\mathfrak{q}\right) = \frac{a}{f}$ für alle $\mathfrak{q} \in U$ und geeignete $a \in R, f \in R \setminus \bigcup_{\mathfrak{q} \in U}\mathfrak{q}$.\\
				$s\left(\mathfrak{p}\right) = 0$ in $R_{\mathfrak{p}}$ bedeutet, dass es ein $h \in R \setminus \mathfrak{p}$ mit
				$ h a = 0$ gibt.
				$U' = U \cap D\left(h\right)$ ist eine offene Umgebung von $\mathfrak{p}$ mit $h \notin \mathfrak{q}$ für alle $\mathfrak{q} \in U'$.
				Also ist $\frac{a}{f} = 0$ in $R_{\mathfrak{q}}$ für alle $\mathfrak{q} \in U'$.

				$\Rightarrow \left(U,s\right) \sim (U',s) \sim 0$
		\item
			Definiere $\phi\colon R_{f} \to \mathcal{O}_{X}\left(D(f\right))$ durch $\frac{a}{f^n} \mapsto ( \mathfrak{p} \mapsto \frac{a}{f^n} )$.

			\minisec{$\phi$ ist wohldefinierter Ringhomomorphismus}

			\minisec{$\phi$ ist injektiv}
				Sei $\phi\left( \frac{a}{f^n} \right) = 0$.
				Dann ist für jedes $\mathfrak{p} \in D\left(f\right)$ offenbar $\frac{a}{f^n} = 0$ in $R_{\mathfrak{p}}$.
				Also gibt es $h_{\mathfrak{p}} \in R \setminus \mathfrak{p}$, so dass $h_{\mathfrak{p}} a = 0$.
				Sei nun $\mathfrak{a} = \left\{ r \in R\ |\ r\cdot a = 0 \right\}$ der \emph{Annihilator} von $a$.
				$\mathfrak{a}$ ist ein Ideal und $\mathfrak{a} \nsubseteq \mathfrak{p}$ für alle $\mathfrak{p} \in D\left(f\right)$,
				da alle $h_{\mathfrak{p}} \in \mathfrak{a}$.
				Somit ist $V\left(\mathfrak{a}\right) \cap D(f) = \emptyset$, also $V(\mathfrak{a}) \subseteq V(f)$.
				Dann ist aber $f \in I\left(V(f\right)) \subseteq I(V(\mathfrak{a})) = \rad{\mathfrak{a}}$.
				Also gibt es ein $n\in \NN$, so dass $f^n \in \mathfrak{a}$, also ist $\frac{a}{f^n} = 0$ in $R_f$.

			\minisec{$\phi$ ist surjektiv}
				Sei $s \in \mathcal{O}_{X}\left(D(f\right))$.
				Für jedes $\mathfrak{p} \in D\left(f\right)$ gibt es eine Umgebung $U_{\mathfrak{p}}$ von $\mathfrak{p}$ und 
				$a,h \in R$, so dass für alle $\mathfrak{q} \in U_{\mathfrak{p}}$ gilt:
				$h \notin \mathfrak{q}$ und 
				$s\left(\mathfrak{q}\right) = a/h$.

				$D\left(f\right)$ ist quasi-kompakt und die $U_{\mathfrak{p}}$ überdecken $D(f)$, also muss es endlich viele 
				$\mathfrak{p}_1,\dots,\mathfrak{p}_n$ geben, so dass $U_i = U_{\mathfrak{p}_i}$ für $i \in \{1,\dots,n\}$
				eine Überdeckung von $D(f)$ ist.
				Seien $a_1,\dots,a_n,h_1,\dots,h_n \in R$, so dass für alle $\mathfrak{q} \in U_i$ gilt:
				$h_i \notin \mathfrak{q}$ und $s\left(\mathfrak{q}\right) = a_i/h_i$.
				Ohne Einschränkung kann man $U_i = D\left(h_i\right)$ annehmen und erhält
				\[
					V\left(f\right) \supseteq \Spec R \setminus \bigcup_{i = 1}^{n} D(h_i) = \bigcap_{i = 1}^{n} V(h_i)
				\]
				Insbesondere gilt dann
				\[
					f \in I\left(V(f\right) ) \subseteq I\left( \bigcap_{i = 1}^{n} V(h_i) \right) 
						= I(V( h_1,\dots,h_n ) ) = \rad{ (h_1,\dots,h_n) }
				\]
				Somit gibt es ein $n \in \NN$ und $b_i \in R$, so dass $f^n = \sum_{i = 1}^{n} b_i h_i$.
				Wählt man nun $a = \sum_{i=1}^{n} b_i a_i$, dann gilt:
				\[
					a_j f^m = \sum_{i=1}^{n} b_i a_j h_i \overset{\text{Einschub}}{=} \sum_{i=1}^{n} b_i a_i h_j = a h_j
				\]
				und somit $a_j/h_j = a/f^m \Longrightarrow \phi\left( a/f^m \right) = s$
{
	\medskip\hrule
				\minisec{Einschub}
				{ \small
				Ohne Einschränkung gilt $ a_i h_j = a_j h_i$ in $R$ \\
				Auf $U_i \cap U_j$ gilt $\frac{a_i}{h_i} = \frac{a_j}{h_j}$, also gibt es ein $y_{i,j} \in R$, so dass
					$y_{i,j} \notin \mathfrak{q}$ für jedes $\mathfrak{q} \in U_i \cup U_j$.
				\[
					y_{i,j} a_i h_j = y_{i,j} a_j h_i
				\]
				Wählt man nun 
				\[
					a_i' = a_i\prod_{j} y_{i,j} \text{ und }h_i' = h_i \prod_{j} y_{i,j}
				\]
				dann ist offenbar $a_i'/h_i' = a_i/h_i$ und
				\[
					a_i' h_j'	= y_{i,j} a_i h_j \prod_{ k \neq j} y_{i,k} \prod_{k } y_{j,k} 
								= y_{i,j} a_j h_i \prod_{ k \neq j} y_{i,k} \prod_{k } y_{j,k} 
								= a_j' h_i'
				\]
				}
	\hrule \medskip
}
	\end{enumerate}
\end{Bew}

\begin{Bsp}
 Sei R ein diskreter Bewertungsring. Dann gilt:
	\begin{enumerate}[(a)]
		\item $\Spec R = \{(0), \mathfrak m\}$
		\item offene Mengen sind: $\emptyset, \Spec R, \{(0)\}$
		\item $\mathcal{O}_{X}\left( \{(0)\} \right) = R_{(0)}=Quot(R)=:K$
		\item $\{(0)\}=D(f)$ für $0 \neq f \notin \mathfrak m$
		\item $\mathcal{O}_{\Spec R}\left( \Spec R \right)= \mathcal{O}_{\Spec R,\mathfrak m}=R_\mathfrak{m} = R$
		\item $\mathcal{O}_{\Spec R}\left( \{(0)\}\right)= \mathcal{O}_{\Spec R,(0)}=K$
	\end{enumerate}
\end{Bsp}

\section{Die Kategorie der Schemata}

\begin{Def}
	\begin{enumerate}[(a)]
		\item Ein \emph{geringter Raum} ist ein Paar $\left( X, \mathcal{O}_X \right)$ mit einem topologischen Raum $X$ und einer Garbe von Ringen $\mathcal{O}_X$ auf $X$.
		\item Ein geringter Raum $\left( X, \mathcal{O}_X \right)$ heißt \emph{lokal geringt}, wenn $\mathcal{O}_{X,x}$ für jedes $x \in X$ ein lokaler Ring ist.
	\end{enumerate}
\end{Def}

\begin{Bsp}
 Für $X=\Spec R$ und $\mathcal{O}_X =\mathcal{O}_{\Spec R}$ die Strukturgarbe aus \ref{defbem:structure_germ} ist $\left( \Spec R,\mathcal{O}_{\Spec R} \right)$ lokal geringter Raum.
\end{Bsp}

\begin{Def}
	\begin{enumerate}[(a)]
		\item Ein Morphismus zwischen lokal geringten Räumen $\left( X, \mathcal{O}_X \right)$ und $\left( Y, \mathcal{O}_Y \right)$ ist ein Paar $\left( f, f^\sharp \right)$, wobei $f : X \rightarrow Y$ eine stetige Abbildung und $f^\sharp: \mathcal{O}_Y \rightarrow f_* \mathcal{O}_X$ ein Homomorphismus von Garben auf $X$ ist.
		\item Sind $\left( X, \mathcal{O}_X \right)$ und $\left( Y, \mathcal{O}_Y \right)$ lokal geringte Räume, so ist ein Morphismus $\left( f, f^\sharp \right) :\left( X, \mathcal{O}_X \right) \rightarrow \left( Y, \mathcal{O}_Y \right) $ ein Morphismus von lokal geringten Räumen, wenn für jedes $x \in X$ gilt: Die induzierte Abbildung $f^\sharp_x: \mathcal{O}_{Y,f(x)} \rightarrow  \mathcal{O}_{X,x}$ ist ein lokaler Homomorphismus (das heißt $f^\sharp_x \left( \mathfrak{m}_{f(x)}\right) \subseteq \mathfrak{m}_x$).
	\end{enumerate}
\end{Def}

\begin{Bsp}
	Sei $R$ ein lokaler nullteilerfreier Ring, $K=Quot(R)$ und $i: R \hookrightarrow K$ sei kein lokaler Homomorphismus. Aber $i$ induziert einen Morphismus lokal geringter Räume zwischen $X=\Spec K$ und $Y= \Spec R$ durch $f: X \rightarrow Y, (0) \mapsto (0)$ und $f^\sharp:\mathcal{O}_{\Spec R} \rightarrow f_* \mathcal{O}_{\Spec K} $ gegeben durch i. Es gilt für alle offenen $U \neq \emptyset$: $f_* \mathcal{O}_{\Spec K} \left( U \right) = \mathcal{O}_{\Spec K} \left( f^{-1}(U) \right) = \mathcal{O}_{\Spec K} \left( (0) \right) = K$ und $\mathcal{O}_{\Spec R}(U)=R'$ für $R \subseteq R' \subseteq K$
\end{Bsp}

\begin{Prop}
\label{prop:3.3}
 Die Kategorie der affinen Schemata ist äquivalent zur Kategorie der Ringe.
\end{Prop}
\begin{Bew}
	Für Objekte ist dies klar, denn $\mathcal{O}_{\Spec R}\left(\Spec R\right) = R$. \\
	Ist $\left( f, f^\sharp \right): \left( \Spec R,\mathcal{O}_{\Spec R} \right)  \rightarrow \left( \Spec R',\mathcal{O}_{\Spec R'} \right)$ ein Morphismus affiner Schemata, so ist $f^\sharp: R' =\mathcal{O}_{\Spec R'}\left(\Spec R'\right) \rightarrow  f_* \mathcal{O}_{\Spec R}\left(\Spec R'\right)=\mathcal{O}_{\Spec R}\left(f^{-1} (\Spec R)\right) = R$ ein Ringhomomorphismus $R' \rightarrow R$. \\
	Sei umgekehrt $\alpha : R' \rightarrow R$ ein Ringhomomorphismus. Dann wird durch $\alpha$ induziert: 
	\begin{itemize}
		\item $f_\alpha : \Spec R' \rightarrow R, \mathfrak{p} \mapsto \alpha^{-1}(\mathfrak{p})$  (stetig)
		\item $f_\alpha^\sharp: \mathcal{O}_{\Spec R} \rightarrow (f_\alpha)_* \mathcal{O}_{\Spec R'}$ induziert durch $\frac{a}{b} \mapsto \frac{\alpha(a)}{\alpha(b)}$, $a,b \in R, b \notin \dots$ 
	\end{itemize}
	Auf den Halmen induziert $f_\alpha^\sharp$ die Abbildung $\alpha':= (f_\alpha^\sharp)_{\mathfrak{p}'}: R_{f^{-1}(\mathfrak{p})} = \mathcal{O}_{\Spec R, f_\alpha(\mathfrak{p}')} \rightarrow  \mathcal{O}_{\Spec R', \mathfrak{p}'}= R'_{\mathfrak{p}'}$
	Es ist $\alpha' (\alpha'^{-1} (\mathfrak{p}') ) \subseteq \mathfrak{p}'$
\end{Bew}

\begin{Bem}
	Ist $\left( X, \mathcal{O}_X \right)$ Schema und $U \subseteq X$ offen, so ist $\left( U, \mathcal{O}_X \restriction U \right)$ auch ein Schema, offenes Unterschema genannt.
\end{Bem}
\begin{Bew}
 Sei $(U_i)_{i \in I}$ Überdeckung von $X$ durch affine Schemata. Dann ist $(U \cap U_i)_{i \in I}$ offene Überdeckung von $U$. (Achtung: i. A. ist $(U \cap U_i)$ kein affines Schema) Aber $(U \cap U_i)$ ist Vereinigung von $D(f_{ij})$ für geeignete $f_{ij} \in R_i$. Es gilt $D(f_{ij})$ ist affines Schema und $\mathcal{O}_{\Spec R} \restriction D(f_{ij}) \cong \mathcal{O}_{\Spec R_{f_{ij}}}$
\end{Bew}


\begin{Bem}
	Aus zwei Schemata kann man durch Verkleben längs isomorpher Unterschemata ein neues Schema erhalten.
	Genauer: Seien $X_1,X_2$ Schemata $\emptyset \neq U_i \subseteq X_i$ offene Unterschemata und $\varphi: \left( U_1, \mathcal{O}_{X_1} \restriction U_1 \right) \rightarrow \left( U_2, \mathcal{O}_{X_2} \restriction U_2 \right)$ ein Isomorphismus von Schemata. Sei $\sim$ die Äquivalenzrelation, die durch $x \sim \varphi (x)$ erzeugt wird. Dann ist $X=(U_1 \dot\cup U_2)_\sim $ topologischer Raum versehen mit der Quotiententopologie. Für $U \subseteq X$ offen sei $\mathcal{O}_X(U):=\left\lbrace  (s_1,s_2) \in \mathcal{O}_X(U^1) \times \mathcal{O}_X(U^2) | s_1 \restriction U^1 \cap \varphi^{-1}(U^2) = \varphi^\sharp_{\varphi(U^1) \cap U^2}(s_2 \restriction \varphi(U^1) \cap U^2) \right\rbrace$ wobei $U^1=(U \cap X_1), U^2=(U \cap X_2)$.
\end{Bem}

\begin{Bsp}
 Sei $X_1 = X_2 = \mathbb{A}^1_k:=\Spec k[T]$ und 
 $U_1=U_2=\mathbb{A}^1 \setminus \{0\} = \Spec k[T] \setminus \{(T)\}$ sowie $\varphi_1 : U_1 \rightarrow U_2, \varphi_1 = \id$ und $\varphi_1: U_1 \rightarrow U_2, \varphi_2(T) = \frac{1}{T} $.\\ BILDER EINFÜGEN WENN DIE JEMAND MITGESCHRIEBEN HAT
\end{Bsp}

\begin{Prop}
\label{prop:3.7}
	Sei $\left(X, \mathcal{O}_X \right)$ ein Schema und $R$ ein Ring. Dann ist die Zuordnung $Mor(X, \Spec R) \rightarrow Hom(R, \mathcal{O}_X(X)), (\varphi, \varphi^\sharp) \mapsto \varphi^\sharp_{\Spec R}$ bijektiv.
\end{Prop}
\begin{Bew}
	Definiere Umkehrabbildung: Sei $\alpha : R \rightarrow \mathcal{O}_X(X)$ ein Ringhomomorphismus. Für $x \in X$ sei $\mathcal{O}_{X,x}$ der Halm und $\mathfrak{m}_x$ das maximale Ideal in $\mathcal{O}_{X,x}$.Weiter sei $\alpha_x: R \stackrel{\alpha}\rightarrow \mathcal{O}_{X}(X) \rightarrow \mathcal{O}_{X,x}$. Setze $\varphi_\alpha(x):=\alpha^{-1}_x(\mathfrak{m}_x)$. Es gilt $\varphi_\alpha : X \rightarrow \Spec R $ ist stetig (Übung). Der Garbenhomomorphismus $\varphi_\alpha^\sharp : \mathcal{O}_{\Spec R} \rightarrow \mathcal{O}_{X,x}$ wird definiert durch $\frac{a}{b} \mapsto \frac{\alpha(a)}{\alpha(b)}$.
\end{Bew}

\begin{Def}
	Sei $S$ ein Schema.
	\begin{enumerate}[(a)]
		\item Ein $S$-Schema ist ein Schema $(X,\mathcal{O}_X)$ zusammen mit einem Morphismus $\varphi: X \rightarrow S$.
		\item Ein Morphismus von $S$-Schemata $(X,\varphi)$ und $(Y,\psi)$ ist ein Schema-Morphismus $f: X  \rightarrow Y$ mit $\varphi = \psi \circ f$.
\begin{center}
	\begin{tikzpicture}
		\matrix (m) [matrix of math nodes, row sep=3em,
		column sep=2.5em, text height=1.5ex, text depth=0.25ex]
		{
			X	&	Y	\\
			S	&	\\
		};
		\path[->, font=\scriptsize]
		(m-1-1) edge node[auto] {$f$}	(m-1-2)
				edge node[auto,swap] {$\varphi$}		(m-2-1)
		(m-1-2) edge node[auto] {$\psi$}		(m-2-1);
	\end{tikzpicture}
	\end{center}

	\end{enumerate}
\end{Def}

\begin{Prop}
	\label{faithful_functor_k-variety_k-scheme}
	Sei $k$ algebraisch abgeschlossener Körper. Die Zuordnung $V \rightarrow \Spec k[V]$ ($V$ affine Varietät über $k$) induziert einen volltreuen Funktor $t$ von der Kategorie der $k$-Varietäten in die Kategorie der $k$-Schemata.
\end{Prop}
\begin{Bew}
	$V \mapsto k[V]$ ist Äquivalenz von Kategorien (Algebraische Geometrie I Satz???). $k[V] \mapsto \Spec k[V]$ ist Äquivalenz von Kategorien. Das heißt, wie haben eine Äquivalenz von Kategorien $k$-Algebren $\rightarrow$ affine $k$-Varietäten $\rightarrow$ affine Schemata. Die Behauptung folgt durch Verkleben.
\end{Bew}

\section{Projektive Schemata}

\begin{DefBem}
	Sei $S = \bigoplus_{d \geq 0} S_d$ graduierter Ring, $S^+:=\bigoplus_{d > 0} S_d$
	\begin{enumerate}[(a)]
		\item  $\Proj S := \left\lbrace \mathfrak{p} \subseteq \Proj S: \mathfrak{p} \text{ homogenes Primideal }, S^+ \nsubseteq \mathfrak{p} \right\rbrace $ heißt \emph{homogenes Spektrum} von $S$.
		\item Für ein homogenes Ideal $\mathfrak a$ in $S$ sei $V(\mathfrak{a}) := \{\mathfrak{p} \in \Proj S : \mathfrak{a \subseteq p} \}$. Die $V(\mathfrak{a})$ bilden die abgeschlossenen Teilmengen einer Topologie auf $\Proj S$.
		\item Für homogenes $f \in S$ sei $D_+(f):=\{\mathfrak{p} \in \Proj S : f \in \mathfrak p\} = \Proj S \setminus V(f)$. Die $D_+(f)$ bilden eine Basis der Zariski-Topologie auf $\Proj S$.
		\item Für $\mathfrak{p} \in \Proj S$ sei $S_{(\mathfrak{p})}:= \left\lbrace \frac{a}{b} \in S_{\mathfrak{p}}:a,b \text{ homogen vom gleichen Grad}  \right\rbrace $. $S_{(\mathfrak{p})}$ ist lokaler Ring mit maximalem Ideal $\mathfrak{p}S_{(\mathfrak{p})} := \left\lbrace \frac{a}{b} \in S_{(\mathfrak{p})} : a \in \mathfrak{p} \right\rbrace $
		\item  Für $U \subseteq \Proj S$ offen sei 
			\begin{align*}
			\mathcal{O}_{\Proj S}\left(U\right) = 
				\bigg\{ s\colon U \to \bigcup_{ \mathfrak{p} \in U }^{\centerdot} S_{\mathfrak{p}}\ \bigg|\
					& \text{Für alle } \mathfrak{p} \in U \text{ ist } s\left(\mathfrak{p}\right) \in S_{(\mathfrak{p})}\\
					& \text{und es gibt eine Umgebung } U_{(\mathfrak{p})} \text{ von } \mathfrak{p}
					\text{ sowie } a,b \in S \\ 
					& \text{homogen vom gleichen Grad, so dass für alle } \mathfrak{q} \in U_{(\mathfrak{p})}\colon b \notin \mathfrak{q} \\
					& \text{ und } s\left(\mathfrak{q}\right) = \frac{a}{b}
					& \bigg\}
		\end{align*}
		\item $(\Proj S, \mathcal{O}_{\Proj S})$ ist lokal geringter Raum mit $\mathcal{O}_{\Proj S, \mathfrak{p}} = S_{(\mathfrak p)}$
		\item \begin{align*} (\Proj S, \mathcal{O}_{\Proj S})  \text{ ist Schema, wobei } \Proj S = \bigg\{\bigcup^{\centerdot}_{ f  \in S, f \text{ homogen} }D_+(f) \bigg\} \text{ und }D_+(f) \cong \Proj S_{(f)}.
		       \end{align*}

	\end{enumerate}
\end{DefBem}
\begin{Bew}
 Sei $S$ graduierter Ring. $\Proj{S} = \left\lbrace \mathfrak{p} \text{ homogenes Primideal, } S_+ \not\subset \mathfrak{p} \right\rbrace$ und \\
$S_{(\mathfrak{p})}=\left\lbrace \frac{a}{b}: a,b \text{ homogen vom gleichen Grad, } b \notin \mathfrak{p} \right\rbrace $ \\
sowie $S_{(f)}=\left\lbrace \frac{a}{f^n}: a \text{ homogen vom Grad } n \cdot~ deg(f) \right\rbrace $
\begin{enumerate}[(g)]
 \item $(\Proj S, \mathcal{O}_{\Proj S})$ ist Schema. Genauer $D_+(f) \cong \Spec{\mathfrak{p} S_{(f)}}$.  
\end{enumerate}

\end{Bew}

\begin{DefBem}
	\begin{enumerate}[(a)]
		\item Ein Schema $(X,\mathcal{O}_X)$ heißt projektiv, wenn es einen graduierten Ring $S$ gibt, so dass $(X,\mathcal{O}_X) \cong (\Proj S, \mathcal{O}_{\Proj S})$  gilt.
		\item Ist $R$ ein Ring, so heißt $\mathbb{P}^n_R = \Proj {R[X_0,\dots X_n]}$ der n-dimensionale projektive Raum über R.
		\item Sei $k$ ein Körper und $X=\mathbb{P}^1_k$. Dann ist $\mathcal{O}_X(X)=k$.
	\end{enumerate}
\end{DefBem}
\begin{Bew}
	\begin{enumerate}[(c)]
		\item
			\begin{align*}
			 X = \bigcup_{i=1}^n D_+(X_i), \mathcal{O}_X(X_i)=k[\frac{X_0}{X_i},\dots,\frac{X_{i-1}}{X_i}, \frac{X_{i+1}}{X_i}, \dots \frac{X_n}{X_i} ] \Rightarrow \mathcal{O}_X(X) = \bigcap_{i=1}^n \mathcal{O}_X(X_i)=k
			\end{align*}
 
	\end{enumerate}
\end{Bew}

\begin{Bem}
	Sei $k$ algebraisch abgeschlossener Körper, $V/k$ eine projektive Varietät und $S=k[V]$ ein homogener Koordinatenring von V. Dann ist $t(V) \cong \Proj S$ ($t$ wie in \ref{faithful_functor_k-variety_k-scheme}).
\end{Bem}
\begin{Bew}
	Für homogenes $f \in S_+$ ist $D_+(f) \cong \Spec{S_{(f)}}$. Außerdem wissen wir aus der Algebraischen Geometrie 1, dass $\mathcal{O}_V(D(f))=S_{(f)} \Rightarrow D_+(f)=t(D(f))$.
	Die Behauptung folgt durch Verkleben.
\end{Bew}

\begin{DefBem}
\label{defbem:4.4}
Sei $X$ ein Schema und $x \in X$:
\begin{enumerate}[(a)]
	\item $\kappa(x):=\mathcal{O}_{X,x}/m_x$ heißt Restklassenkörper von $X$ in $x$.
	\item Sei $f: X \rightarrow Y$ ein Morphismus von Schemata und $y=f(x)$, dann induziert $f$ einen Körperhomomorphismus $\kappa(y) \hookrightarrow \kappa(x)$.
	\item Sei $k$ ein Körper. Genau dann gibt es einen Morphismus $\iota: \Spec k \rightarrow X$ mit $\iota(0) = x$, wenn $\kappa(x)$ isomorph zu einem Teilkörper von $k$ ist.
	\item $x$ (beziehungsweise genauer $\iota$) heißt $k$-wertiger Punkt von $X$.
\end{enumerate}
\end{DefBem}
\begin{Bew}
	\begin{enumerate}
		\item[(b)] $f$ induziert $f^\sharp_x: \mathcal{O}_{Y,y} \rightarrow \mathcal{O}_{X,x}$ mit $f^\sharp_x(m_y) \subseteq m_x$. Die Behauptung folgt aus dem Homomorphiesatz.
		\item[(c)] Sei $U = \Spec R$ affine Umgebung von $x$: $\iota$ ist äquivalent zu dem Ringhomomorphismus $\alpha: R \rightarrow k $ mit $\alpha(m_x)=(0)$ $\Leftrightarrow$ $\alpha$ faktorisiert über $\kappa(x)$.
% $R \rightarrow \kappa(x) \rightarrow k$
	\end{enumerate}
\end{Bew}

\section{Faserprodukte}

Sei $S$ ein Schema und $X,Y$ $S$-Schemata. Dann heißt das Produkt über $X$ und $Y$ in der Kategorie der $S$-Schemata Faserprodukt von $X$ und $Y$, geschrieben $X \times_S Y$.

\begin{Bem}
	Das Faserprodukt $X \times_S Y$ ist ein $S$-Schema zusammen mit $S$-Morphismen $pr_X: X \times_S Y \rightarrow X$ und $pr_Y: X \times_S Y \rightarrow Y$, so dass für jedes $S$-Schema $Z$ und alle $S$-Schemamorphismen $f: Z \rightarrow X, g: Z \rightarrow Y$ genau ein $S$-Schemamorphismus $h: Z \rightarrow X \times_S Y$ existiert mit $f=pr_X \circ h, g = pr_Y \circ h$.

\begin{center}
\begin{tikzpicture}
			\matrix (m) [matrix of math nodes, row sep=3em, column sep=2.5em, text height=1.5ex, text depth=0.25ex]
			{ 
				& Z & \\
				&	X \times_S Y	&	\\
				X & & Y \\
				& S & \\
			};
			\path[->,font=\scriptsize]
			(m-1-2) edge[densely dotted] node[auto] {$ \exists h $}	(m-2-2)	
					edge[bend right=30] node[auto,swap] {$ f $}		(m-3-1)
					edge[bend left=30] node[auto] {$ g $}		(m-3-3)
			(m-2-2) edge node[auto,swap] {$pr_X$} (m-3-1)
					edge node[auto] {$pr_Y$} (m-3-3)
			(m-3-1) edge node[auto,swap] {} (m-4-2)
			(m-3-3) edge node[auto,swap] {} (m-4-2);
			
		\end{tikzpicture}
\end{center}

\end{Bem}

\begin{Satz}
\label{satz:1}
  Das Faserprodukt $X\times_S Y$ existiert für alle $S$-Schemata $X,Y$.
\end{Satz}

\begin{Bew}
  Seien zunächst $X,Y$ und $Z$ affin: $X=\Spec A,Y=\Spec B$ und $S=\Spec R$. Nach Voraussetzung sind $A$ und $B$ $R$-Algebren.
  Die UAE des Tensorprodukts $A\otimes_R B$ besagt: \\
  $\Spec(A\otimes_R B)$ erfüllt die UAE des Faserprodukts für jedes affine Schema $Z$. \\
  Noch zu zeigen: die UAE ist auch für beliebige $Z$ erfüllt. 
  Nach Proposition \ref{prop:3.7} entspricht $f:Z\to X$ einem $R$-Algebrenhomomorphismus $\varphi_1:A\to\mathcal O_Z(Z)$,
  ebenso gehört zu $g$ ein $\varphi_2:B\to\mathcal O_Z(Z)$. $\varphi_1$ und $\varphi_2$ induzieren einen $R$-Algebrenhomomorphismus
  $\varphi:A\otimes_R B\to\mathcal O_Z(Z)$. Nach Proposition \ref{prop:3.7} induziert $\varphi$ einen Schemamorphismus $h:Z\to\Spec(A\otimes_R B)$. \\
  Für den allgemeinen Fall sei $S_i$ eine affine Überdeckung von $S$
  \begin{align*}
    S=\bigcup S_i \text{, mit } S_i=\Spec R_i
  \end{align*}
  Seien $X_i=p_X^{-1}(S_i), Y_i=p_Y^{-1}$ auch affin überdeckt:
  \begin{align*}
     X_i&=\bigcup X_{ij}\text{, mit } X_{ij}=\Spec A_{ij} \\
      Y_i&=\bigcup Y_{ik}\text{, mit } Y_{ij}=\Spec B_{ik}
  \end{align*}
  Nach dem affinen Fall oben existieren die Faserprodukte $X_{ij}\times_{S_i} Y_{ik}$ für alle $i,j,k$.
  \begin{Beh}[1]
    Sei $T$ ein Schema, $V,W$ $T$-Schemata, $(V_l)$ offene Überdeckung von $V$, dann gilt:
    \begin{align*}
      \text{Existiert $V_l\times_T W$ für jedes $l$, so existiert auch $V\times_T W$}
    \end{align*}
    
  \end{Beh}
  Wende diese Behauptung an auf 
  \begin{align*}
    T=S_i,~ V=X_i,~ V_l=X_{il},~ W=Y_{ik}
  \end{align*}
  womit $X_i\times_{S_i} Y_{ik}$ für alle $i,k$ existiert. Damit lässt sich die Behauptung auf
  \begin{align*}
    T=S_i,~ V_l=Y_{il},~ V=Y_i,~ W=X_i
  \end{align*}
  anwenden. Dies zeigt die Existenz von $X_i\times_{S_i}Y_i$ für alle $i$.
  \begin{Beh}[2]
    Für jedes $i$ gilt
    \begin{align*}
      X_i\times_{S_i}Y_i\cong X_i\times_S Y
    \end{align*}
  \end{Beh}
  Daraus folgt der Satz aus Behauptung (1) mit
  \begin{align*}
    T=S,~ V=X,~ V_l=X_l,~ W=Y
  \end{align*}
\end{Bew}
\begin{Bew}[Behauptung (1)]
  Idee: Verklebe die $V_l\times_T W$! \\
  Für Indizes $l,m$ seien
  \begin{align*}
    U_{lm}&\defeqr pr_l^{-1}(V_l\cap V_m)\subseteq V_l\times_T W \\
    \text{ und } U_{ml}&\defeqr pr_m^{-1}(V_l\cap V_m)\subseteq V_m\times_T W
  \end{align*}
  Es gilt: $U_{lm}=(V_l\cap V_m)\times_T W$, weil in der Situation
  \begin{center}
    \begin{tikzpicture}
      \matrix (m) [matrix of math nodes, row sep=3em, column sep=2.5em, text height=1.5ex, text depth=0.25ex]
      { 
        & Z & \\
        &	V_l \times_T W	&	\\
        V_l\cap V_m & V_l  & W \\
        & T & \\
      };
      \path[->,font=\scriptsize]
      (m-1-2) edge[densely dotted] node[auto] {$ \exists h $}	(m-2-2)	
       edge[bend right=30] node[auto,swap] {$ f $}		(m-3-1)
       edge[bend left=30] node[auto] {$ g $}		(m-3-3)
      (m-2-2) edge node[auto,swap] {$pr_l$} (m-3-2)
       edge node[auto,swap] {} (m-3-3)
      (m-3-1) edge node[auto,swap] {} (m-4-2)
      (m-3-2) edge node[auto,swap] {} (m-4-2)
      (m-3-3) edge node[auto,swap] {} (m-4-2);
      \path[right hook->] 
      (m-3-1) edge node[auto,swap] {} (m-3-2);
      
    \end{tikzpicture}
  \end{center}
  gilt:
  \begin{align*}
    h(z)\subseteq pr_l^{-1}(f(z))\subseteq pr_l^{-1}(V_l\cap V_m)=U_{lm}
  \end{align*}
  Also ist $U_{lm}$ Faserprodukt von $V_l\cap V_m$ und $W$. Genauso: $U_{ml}$ ist Faserprodukt von $V_l\cap V_m$ und $W$.
  Die UAE liefert einen eindeutigen Isomorphismus $U_{lm}\to U_{ml}$. Verklebe die $V_l\times_T W$ längs der $U_{lm}$ zu einem
  Schema $\tilde{V}$. \\ 
  Noch zu zeigen: $\tilde{V}$ erfüllt die UAE von $V\times_T W$.
  Seien $Z$ ein $T$-Schema, $f:Z\to V$ und $g:Z\to W$ $T$-Morphismen. Sei $Z_l\defeqr f^{-1}(V_l)$. \\
  Nach Voraussetzung existiert für jedes $l$ genau ein $h_l:Z_l\to V_l\times_T W\hookrightarrow \tilde{V}$, mit $\dots$ \\
  Die $h_l$ bestimmen einen eindeutigen Morphismus $h:Z\to \tilde{V}$.
\end{Bew}
\begin{Bew}[Behauptung (2)]
  \emph{Der Beweis war Übungsaufgabe} \\
  Zu zeigen:\\
  $X_i\times_{S_i}Y_i$ ist ein Faserprodukt von $X_i$ und $Y$ über $S_i$. \\
  Sei $Z$ ein $S$-Schema mit $S$-Morphismen $f:Z\to X_i$ und $g:Z\to Y$.
  Weil
  \begin{center}
    \begin{tikzpicture}
      \matrix (m) [matrix of math nodes, row sep=3em, column sep=2.5em, text height=1.5ex, text depth=0.25ex]
      { 
        & Z & \\
        & &	\\
        X_i & & Y \\
        & S & \\
      };
      \path[->,font=\scriptsize]
      (m-1-2) 	edge[bend right=30] node[auto,swap] {$ f $}		(m-3-1)
       edge[bend left=30] node[auto] {$ g $}		(m-3-3)
      (m-3-1) edge node[auto,swap] {$p_i$} (m-4-2)
      (m-3-3) edge node[auto] {$p$} (m-4-2);
      
    \end{tikzpicture}
  \end{center}
  kommutiert, gilt:
  \begin{align*}
    p_i(X_i)\subseteq S_i \Rightarrow (p\circ g)(Y)\subseteq S_i \Rightarrow  \Bild g\subseteq Y_i
  \end{align*}
  Damit faktorisiert das eindeutige $h$ der UAE vom Faserprodukt $X_i\times_{S_i}Y_i$ $f$ und $g$. Also
  ist $X_i\times_{S_i}Y_i$ Faserprodukt von $X_i$ und $Y$ über $S$.
\end{Bew}

\begin{Bem}
  \label{bem:5.2}
  Seien $X,Y$ $S$-Schemata. Dann ist die Abbildung
  \begin{align*}
    \Abb{F}{X\times_S Y}{\{(x,y)\in X\times Y:p_X(x)=p_Y(y)\}}{z}{(pr_X(z),pr_Y(z))}
  \end{align*}
  stetig und surjektiv.
\end{Bem}

\begin{Bew}
  Stetig: Klar. \\
  surjektiv: \\
  Seien $x\in X, y\in Y$ mit $p_X(x)=p_Y(y)\defeql s\in S$. Seien weiter $\kappa\defeqr\kappa(s),\kappa(x),\kappa(y)$ die Restklassenkörper.
  Dann ist $\kappa\subseteq\kappa(x), \kappa\subseteq\kappa(y)$. \\
  Sei $K/k$ eine Körpererweiterung mit $\kappa(x)\subseteq K, \kappa(y)\subseteq K$ und $Z\defeqr\Spec K$.
  Nach \ref{defbem:4.4} gibt es einen Morphismen $f:Z\to X$ und $g:Z\to Y$ mit $f(0)=x,g(0)=y$. $f$ und $g$ sind $S$-Morphismen.
  Also gibt es ein $h:Z\to X\times_S Y$ mit $pr_X(h(0))=x$ und $pr_Y(h(0))=y$. Daraus folgt: $F(h(0))=(x,y)$.
\end{Bew}

\begin{DefBem}
  \label{defbem:5.3}
  \begin{enumerate}
  \item Für $y\in Y$ heißt $X_y\defeqr f^{-1}(y)=X\times_Y\Spec(\kappa(y))$ \emph{Faser} von $f$ über $y$.
  \item $pr_X:X_y\to X$ ist injektiv, das heißt
    \begin{align*}
      X_y\to \{x\in X:f(x)=y\}
    \end{align*}
    ist bijektiv.
  \item Ist $y$ ein abgeschlossener Punkt, so ist $X_y$ abgeschlossen in $X$. 
  \end{enumerate}
\end{DefBem}

\begin{Bew}
  \begin{enumerate}
  \item[(c)] Klar.
  \item[(b)] Für $z_1,z_2\in X_y$ mit $pr_X(z_1)=pr_X(z_2)\defeql x$ gilt $f(x)=y$.\\
    Seien $Z\defeqr\Spec\kappa(x)$ und $\varphi:Z\to X$ mit $\varphi(0)=x$. Sei weiter $\psi:Z\to\Spec\kappa(y)$
    der von $f$ induzierte Morphismus. \\
    Nach \ref{defbem:4.4} (b) gibt es Morphismen $h_i:Z\to X_y$
    mit $h_i(0)=z_i$ für $i\in\{1,2\}$. \\
    Es ist $pr_X\circ h_i=\varphi$, woraus mit der UAE des Faserprodukts $X_y$ folgt: $h_1=h_2$, also $z_1=z_2$.
  \end{enumerate}
\end{Bew}

\begin{nnBsp}
  Sei
  \begin{align*}
    \Abb{f}{\mathbb A_k^1}{\mathbb A_k^1}{x}{x^2}
  \end{align*}
  Dann ist $f^{-1}(0)=\Spec(k[X]\otimes_{k[X]}k)\cong\Spec(\FakRaum{k[X]}{(X^2)})$.
\end{nnBsp}

\begin{DefBem}
  \label{defbem:5.4}
  Sei $g:S^\prime\to S$ ein Morphismus.
  \begin{enumerate}
  \item Ist $f:X\to S$ ein $S$-Schema, so ist $X^\prime\defeqr X\times_S S^\prime$ ein $S^\prime$-Schema mit
    $f^\prime:X^\prime\to S^\prime$ und $f^\prime=pr_{S^\prime}$. \\
    \begin{center}
      \begin{tikzpicture}
        \matrix (m) [matrix of math nodes, row sep=3em, column sep=2.5em, text height=1.5ex, text depth=0.25ex]
        { 
          X^\prime & X \\
          S^\prime & S \\
        };
        \path[->,font=\scriptsize]
        (m-1-1) edge node[auto,swap] {$g^\prime$} (m-1-2)
        (m-1-1) edge node[auto,swap] {$f^\prime$} (m-2-1)
        (m-1-2) edge node[auto,swap] {$f$} (m-2-2)
        (m-2-1) edge node[auto,swap] {$g$} (m-2-2);
      \end{tikzpicture}
    \end{center}
    $X^\prime$ heißt das durch \emph{Basiswechsel} $g$ aus $X$ hervorgegangene Schema.
  \item Basiswechsel ist ein kovarianter Funktor $\underline{S-Sch}\to \underline{S-Sch}$.
  \item Basiswechsel ist transitiv:
    \begin{align*}
      X^{\prime\prime}=(X\times_S S^\prime)\times_{S^\prime} S^{\prime\prime}\cong X\times_S S^{\prime\prime}
    \end{align*}
  \end{enumerate}
\end{DefBem}

\begin{Def}
  \label{def:5.5}
  Ein Schema $(X,\mathcal O_X)$ heißt \emph{lokal noethersch}, wenn es eine offene Überdeckung von $X$
  durch affine Schemata $U_i=\Spec R_i$ gibt, sodass jedes $R_i$ noetherscher Ring ist. \\
  $(X,\mathcal O_X)$ heißt \emph{noethersch}, wenn es eine endliche solche Überdeckung gibt.
\end{Def}

\begin{Prop}
  \label{prop:5.6}
  \begin{enumerate}
  \item Ein affines Schema $X=\Spec R$ ist genau dann noethersch, wenn $R$ noethersch ist.
  \item Ein Schema $(X,\mathcal O_X)$ ist genau dann lokal noethersch, wenn für jedes offene affine 
    $U=\Spec R$ gilt: $R$ ist noethersch.
  \end{enumerate}
\end{Prop}

\begin{Bew}
  \begin{enumerate}
  \item[(a)] ``$\Leftarrow$'' Klar. \\
  ``$\Rightarrow$'' folgt aus (b) ``$\Rightarrow$''.
  \item[(b)] ``$\Leftarrow$'' Klar. \\
    ``$\Rightarrow$'': \\
    Sei $U=\Spec R\subseteq X$ offen und $(U_i)_{i\in\mathcal I}$ eine offene Überdeckung von $X$ mit $U_i=\Spec R_i$, $R_i$ noethersch.
    Dann folgt: $U_i\cap U$ ist offen in $U_i$, also $U_i\cap U=\bigcup D(f_{ij})$ für geeignete $f_{ij}\in R_i$. \\
    Nach Proposition \ref{prop:2.10} (b) ist $D(f_{ij})=\Spec R_{ij}$ mit $R_{ij}=(R_i)_{f_{ij}}$. Damit sind die $R_{ij}$ auch noethersch.
    $D(f_{ij})$ ist auch offen in $U$, wird also überdeckt von $D(g_{ijk})$ mit $g_{ijk}\in R$. $D(f_{ij})\hookrightarrow U$ induziert,
    vermöge Einschränkungen, einen Schemamorphismus $\Spec R\to\Spec R_{ij}$ und damit auch einen Ringhomomorphismus $\varphi_{ij}:R\to R_{ij}$.
    % ich (felix), habe hier noch eine Erklärung für R_{g_{ijk}}\cong (R_{ij})_{\varphi(g_{ijk})} eingebaut (weil ich das so nicht einsehen konnte)
    Es gilt $ R_{g_{ijk}}\cong (R_{ij})_{\varphi(g_{ijk})}$, weil die Einschränkung hier die Identität ist. $(R_{ij})_{\varphi(g_{ijk})}$ ist noethersch,
    also auch $ R_{g_{ijk}}$. \\
    Dies liefert eine Überdeckung $U=\bigcup_{i\in \mathcal I}D(h_i)$, wobei für jedes $i\in\mathcal I$ gilt: $R_{h_i}$ ist noethersch.
    Wegen $\bigcup D(h_i)=U$, gilt $\sum_{i\in\mathcal I}(h_i)=R$ und damit
    \begin{align*}
      1=\sum^n_{i=1}a_i h_i\text{, mit } a_i\in R
    \end{align*}
    Sei nun $I_1\subseteq I_2\subseteq\dots$ eine aufsteigende Kette von Idealen in $R$.
    Für $i=1,\dots,n$ wird 
    \begin{align*}
      \varphi_i(I_1)\cdot R_{h_i}\subseteq\varphi_i(I_2)\cdot R_{h_i}\subseteq\dots
    \end{align*}
    stationär (wobei $\varphi_i:R\to R_{h_i}$ der natürliche Homomorphismus $a\mapsto \frac{a}{1}$ sei).
    Es genügt also zu zeigen:
    \begin{Beh}
      Für jedes Ideal $I$ in $R$ gilt:
      \begin{align*}
        I=\bigcap_{i=1}^n\varphi_i^{-1}(\varphi_i(I)\cdot R_{h_i})
      \end{align*}
    \end{Beh}
  \end{enumerate}
\end{Bew}

\begin{Bew}[der Behauptung]
  \begin{enumerate}
  \item[``$\subseteq$''] Klar.
  \item[``$\supseteq$''] Sei $b\in\bigcap_{i=1}^n\varphi_i^{-1}(\varphi_i(I)\cdot R_{h_i})$, 
    dann gibt es für jedes $i$ ein $b_i\in I$ und $k_i\in\mathbb N$ mit
    \begin{align*}
      \frac{b}{1}=\frac{b_i}{h_i^{k_i}}\text{ in } R_{h_i}
    \end{align*}
    Also existiert $m_i\geq 0$ mit $h_i^{m_i}(bh_i^{k_i}-b_i)=0$ in $R$
    \begin{align*}
      \Rightarrow h_i^{k_i+m_i}b=h_i^{m_i}b_i\in I
    \end{align*}
    Die $h_i^{k_i+m_i}$ erzeugen $R$, denn: \\
    Sei $\mathfrak J=(h_1^{k_1+m_1},\dots,h_n^{k_n+m_n})$, dann ist nach Definition der $h_i$ $\sqrt\mathfrak J=R$, also $\mathfrak J=R$. \\
    $\Rightarrow$ es existieren $a_i$, sodass $\sum a_ih_i^{k_i+m_i}=1$. \\
    $\Rightarrow$ $b\in I$.
  \end{enumerate}
\end{Bew}

\chapter{Morphismen von Schemata}
\setcounter{section}{5}
\section{Einbettungen}
\begin{Def}
  \label{def:6.1}
  Sei $i:(Y,\mathcal O)\to(X,\mathcal O_X)$ ein Morphismus von Schemata.
  \begin{enumerate}
  \item $i$ heißt \emph{offene Einbettung}, wenn $i$ ein Isomorphismus auf ein offenes Unterschema von $X$ ist.
  \item $i$ heißt \emph{abgeschlossene Einbettung}, wenn $i$ ein Homöomorphismus auf eine abgeschlossene Teilmenge $Z\defeqr i(Y)$ von $X$ ist und
    $i^\sharp:\mathcal O_X\to i_\ast\mathcal O_Y$ surjektiv ist.
    $(Z, i_\ast\mathcal O)$ heißt dann \emph{abgeschlossenes Unterschema} von $X$.
  \end{enumerate}
\end{Def}

\begin{Bsp}
  \begin{enumerate}
  \item Sei $X=\Spec R$ affin. Die abgeschlossenen Teilmengen von $X$ sind die $V(I)$, ($I\subseteq R$ Ideal).
    $V(I)$ wird zum abgeschlossenen Unterschema durch die Schemastruktur als $\Spec (\FakRaum{R}{I})$.
    Die abgeschlossene Einbettung $\Spec (\FakRaum{R}{I})\to\Spec R$ wird induziert von der Restklassenabbildung $R\to\FakRaum{R}{I}$.
    \textbf{Warnung:} $V(I)=V(I^2)$, aber $\FakRaum{R}{I}=\FakRaum{R}{I^2}$ gilt im Allgemeinen \textbf{nicht}!
  \item Seien $k$ ein Körper, $R=k[X,Y]$ und $I=(X^2,XY)\subsetneq(X)$. Es gilt $V(I)=V(X)$ ($y$-Achse). \\
    In $V(I)$ ist außerhalb von $0=(0,0)=V(X,Y)$, also auf 
    \begin{align*}
      D(Y)=\Spec\left(\FakRaum{k[X,Y]}{I}\right)_Y=\Spec(k[Y]_Y)
    \end{align*}
    das abgeschlossene Unterschema $V(I)$, also $\Spec(\FakRaum{R}{I})$, isomorph zu $\Spec(\FakRaum{R}{(X)})$. ? \\
    Aber: $\mathcal O_{\Spec(\FakRaum{R}{I}),0}$ enthält ein nilpotentes Element, nämlich $X$.
  \end{enumerate}
\end{Bsp}

\begin{ErinnDef}
  \label{erinndef:6.3}
  (Übungblatt 3, Aufgabe 1) \\
  Ein Schema $(X,\mathcal O_X)$ heißt reduziert, wenn für jedes $x\in X$ der Halm $\mathcal O_{X,x}$ ein reduzierter Ring ist. \\
  Äquivalent: Für jedes offene $U\subseteq X$ ist $\mathcal O_X(U)$ ein reduzierter Ring.
\end{ErinnDef}



\begin{Prop}
  \label{prop:6.4}
  Zu jedem Schema $(X,\mathcal O_X)$ gibt es ein eindeutiges abgeschlossenes Unterschema $X_{red}$ von $X$, das folgende UAE erfüllt: \\
  Ist $f:Y\to X$ ein Morphismus von einem reduzierten Schema $Y$, so gibt es genau einen Morphismus $\tilde{f}:Y\to X_{red}$ mit 
  \begin{center}
    \begin{tikzpicture}
      \matrix (m) [matrix of math nodes, row sep=3em, column sep=2.5em, text height=1.5ex, text depth=0.25ex]
      { 
        Y & X \\
         & X_{red} \\
      };
      \path[->,font=\scriptsize]
      (m-1-1) edge node[auto] {$f$} (m-1-2)
      (m-1-1) edge node[auto,swap] {$\tilde{f}$} (m-2-2)
      (m-2-2) edge node[auto,swap] {$i$} (m-1-2);
    \end{tikzpicture}
  \end{center}
  $f=i\circ\tilde{f}$. Dabei ist $X=X_{red}$ (gleich als topologische Räume).
\end{Prop}

\begin{Bew}
  \begin{enumerate}
  \item[(1)]   Sei $X=\Spec R$ affin. Setze $X_{red}\defeqr\Spec(\FakRaum{R}{\sqrt{(0)}})$, dann ist $X_{red}$ 
    ein reduziertes abgeschlossenes Unterschema. \\
    UAE: Sei $Y$ reduziert, $f:Y\to X$ ein Morphismus mit zugehörigem Ringhomomorphismus $\alpha_f:R\to\mathcal O_Y(Y)$. \\
    Zu zeigen: $\sqrt{(0)}\subseteq\Kern(\alpha_f)$ \\
    Sei also $a\in\sqrt{(0)}$, also $a^n=0$ für $n\geq 1$. Daraus folgt: $(\alpha_f(a))^n=0$. Und weil $Y$ reduziert ist: $\alpha_f(a)=0$.
  \item[(2)] Allgemeiner Fall: \\
    Benutze:
    \begin{align*}
      \left( \FakRaum{R}{\sqrt{(0)}}\right)_f\cong\FakRaum{R_f}{\sqrt{(0)}}
    \end{align*}
  \end{enumerate}
\end{Bew}

\begin{Folg}
  \label{folg:6.5}
  Zu jedem abgeschlossenen Unterschema $Z$ von $X$ gibt es ein eindeutig bestimmtes reduziertes Unterschema $Z_{red}$
  (die ``reduzierte induzierte Struktur'').
\end{Folg}

\section{Separierte Morphismen}
\begin{Def}
  \label{def:7.1}
  \begin{enumerate}
  \item Ein Morphismus $f:X\to S$ von Schemata heißt \emph{separiert},  wenn der ``Diagonalmorphismus'' $\Delta_f:X\to X\times_S X$ 
    eine abgeschlossene Einbettung ist.
    \begin{center}
      \begin{tikzpicture}
        \matrix (m) [matrix of math nodes, row sep=3em, column sep=2.5em, text height=1.5ex, text depth=0.25ex]
        { 
          & X & \\
          &	X \times_S X	&	\\
          X & & X \\
          & S & \\
        };
        \path[->,font=\scriptsize]
        (m-1-2) edge[densely dotted] node[auto] {$ \Delta_f $}	(m-2-2)	
        edge[bend right=30] node[auto,swap] {$ \id $}		(m-3-1)
        edge[bend left=30] node[auto] {$ \id $}		(m-3-3)
        (m-2-2) edge node[auto,swap] {} (m-3-1)
        edge node[auto] {} (m-3-3)
        (m-3-1) edge node[auto,swap] {$f$} (m-4-2)
        (m-3-3) edge node[auto,swap] {$f$} (m-4-2);
        
      \end{tikzpicture}
    \end{center}
  \item $X$ heißt \emph{separiert}, wenn $X\to \Spec \mathbb Z$ separiert ist.
  \end{enumerate}
\end{Def}

\begin{nnBsp}
  Sei $X$ die affine Gerade mit doppeltem Nullpunkt. $X$ ist nicht separiert (über $k$): \\
  Seien also $S=\Spec k, U=\mathbb A_k^1\backslash\{(0,0)\}=\Spec(k[X]_X)$ und $X$ die Verklebung von $\mathbb A_k^1$ mit sich selbst längs $U$. 
  Es ist
  \begin{align*}
    & U\times_S U=\mathbb A_k^2-\text{``Achsenkreuz''} \\
    & \Delta=\Delta_f(X)=\{(u,u):u\in U\}\cup\{(0_1,0_1),(0_2,0_2)\}
  \end{align*}
  Es gilt
  \begin{align*}
    \bar{\Delta}=\Delta\cup\{(0_1,0_2),(0_2,0_1)\}
  \end{align*}
  denn: jede Umgebung von $(0_1,0_2)$ enthält Punkte von $\Delta$!
\end{nnBsp}

\begin{Bem}
  \label{bem:7.2}
  Jeder Morphismus von affinen Schemata ist separiert.
\end{Bem}

\begin{Bew}
  Sei $X=\Spec B, Y=\Spec A, f:X\to Y, \alpha:A\to B$, $\alpha$ der Ringhomomorphismus zu $f$.
  Dann ist $X\times_YX=\Spec(B\otimes_A B)$. $\Delta$ wird induziert von
  \begin{align*}
    \Abb{\mu}{B\otimes_A B}{B}{b_1\otimes b_2}{b_1\cdot b_2}
  \end{align*}
  $\mu$ ist surjektiv, also ist $\Delta$ abgeschlossen. (Das ist so, weil ein surjektiver Ringhomomorphismus
  Primideale auf Primideale abbildet und deswegen alle Primideale, die
  \begin{align*}
    \bigcap_{\mathfrak p\text{ Primideal }}\mu^{-1}(\mathfrak p)
  \end{align*}
  enthalten, schon Urbilder von Primidealen waren.)
\end{Bew}

\begin{Bem}
  \label{bem:7.3}
  Seien $f,g:X\to Y$ Morphismen von $S$-Schemata. Ist $Y$ über $S$ separiert, so ist
  \begin{align*}
    E(f,g)\defeqr\{x\in X:f(x)=g(x)\}
  \end{align*}
  abgeschlossen in $X$.
\end{Bem}

\begin{Bew}
  Sei $h:X\to Y\times_S Y$ der von $f$ und $g$ induzierte Morphismus.
  \begin{center}
    \begin{tikzpicture}
      \matrix (m) [matrix of math nodes, row sep=3em, column sep=2.5em, text height=1.5ex, text depth=0.25ex]
      { 
        & X & \\
        &	Y \times_S Y	&	\\
        Y & & Y \\
        & S & \\
      };
      \path[->,font=\scriptsize]
      (m-1-2) edge[densely dotted] node[auto] {$h$}	(m-2-2)	
      edge[bend right=30] node[auto,swap] {$f$}		(m-3-1)
      edge[bend left=30] node[auto] {$g$}		(m-3-3)
      (m-2-2) edge node[auto,swap] {} (m-3-1)
      edge node[auto] {} (m-3-3)
      (m-3-1) edge node[auto,swap] {$p$} (m-4-2)
      (m-3-3) edge node[auto,swap] {} (m-4-2);
      
    \end{tikzpicture}
  \end{center}
  Dann ist $E(f,g)=h^{-1}(\Delta)$, ($\Delta=\Delta_p(Y)$). Also ist $E(f,g)$ abgeschlossen.
\end{Bew}

\begin{Prop}
  \label{prop:7.4}
  Seien $(X,\mathcal O_X)$ ein Schema, $R$ ein diskreter Bewertungsring, $K=\Quot(R)$, $T=\Spec R$. 
  Dann gibt es eine natürliche Bijektion
  \begin{align*}
    \Hom(T,X)\longrightarrow \{ & (x_0,x_1,i):x_0,x_1\in X\text{ mit } x_0\in\overline{\{x_1\}}, i:\kappa(x_1)\to K\text{ Körperhomomorphismus} \\
    & \text{ mit } i(\mathcal O_{Z,x_0})\subseteq R\text{ und } i(m_{Z,x_0})=m_R\cap i(\mathcal O_{Z,x_0})\},
  \end{align*}
  wobei $Z=\overline{\{x_1\}}_{red}$ sei. Dann ist $\mathcal O_{Z,x_1}=\kappa(x_1)=\FakRaum{\mathcal O_{X,x_1}}{m_{x_1}}$.
\end{Prop}

\begin{Bew}
  Für $f:T\to X$ sei $x_0\defeqr f(m_R), x_1=f(0), i=f_{x_1}^\sharp$. Da $T$ reduziert ist, ``ist'' $f$ ein Morphismus nach $Z$:
  \begin{center}
    \begin{tikzpicture}
      \matrix (m) [matrix of math nodes, row sep=3em, column sep=2.5em, text height=1.5ex, text depth=0.25ex]
      { 
        T & X \\
        & Z\\
      };
      \path[->,font=\scriptsize]
      (m-1-1) edge node[auto] {$f$}	(m-1-2)	
      edge node[auto,swap] {$f$}	(m-2-2);
      \path[right hook->,font=\scriptsize]
      (m-2-2) edge node[auto,swap] {} (m-1-2);
    \end{tikzpicture}
  \end{center}
  $f^\sharp$ induziert also einen Morphismus
  \begin{align*}
    \mathcal O_{Z,x_0}\longrightarrow\mathcal O_{T,m}=R
  \end{align*}
  mit $f^\sharp(m_{Z,x_0})\subseteq m$. \\
  Umgekehrt induziert jedes $i:\mathcal O_{Z,x_0}\hookrightarrow R$ einen Morphismus
  \begin{align*}
    \Spec R=T\to\Spec(\mathcal O_{Z,x_0})\to Z\to X
  \end{align*}
\end{Bew}

\begin{Satz}
  \label{satz:2}
  Sei $f:X\to Y$ ein Morphismus noetherscher Schemata. $f$ ist genau dann separiert, wenn es zu jedem ``Bewertungsdiagramm''

   \begin{center}
    \begin{tikzpicture}
      \matrix (m) [matrix of math nodes, row sep=3em, column sep=2.5em, text height=1.5ex, text depth=0.25ex]
      { 
        U & X\\
        T & Y\\
      };
      \path[->,font=\scriptsize]
      (m-1-1) edge node[auto] {}	(m-1-2)	
      edge node[auto,swap] {}	(m-2-1)
      (m-1-2) edge node[auto] {$f$} (m-2-2)
      (m-2-1) edge node[auto,swap] {} (m-2-2)
      (m-2-1) edge[densely dotted] node[auto,swap] {$h$} (m-1-2);
    \end{tikzpicture}
  \end{center}
  ($T=\Spec R$, $R$ diskreter Bewertungsring, $U=\Spec K$, $K=\Quot R$) \\
  höchstens einen Morphismus $h:T\to X$ gibt, der das Diagramm kommutativ macht. 
\end{Satz}

\begin{nnBsp}
  Seien $X$ die affine Gerade mit doppeltem Nullpunkt, $Y=\Spec k$ für einen Körper $k$, $R=k[X]_{(X)}$, $K=k(X)$.
  Sei weiter $X^\prime=\Spec k[X]$, dann existiert ein Morphismus, der das Bewertungsdiagramm kommutativ macht:
   \begin{center}
    \begin{tikzpicture}
      \matrix (m) [matrix of math nodes, row sep=3em, column sep=2.5em, text height=1.5ex, text depth=0.25ex]
      { 
        K & k[X]\\
        k[X]_{(X)} & k\\
      };
      \path[->,font=\scriptsize]
      (m-2-2) edge node[auto] {}	(m-2-1)	
      edge node[auto,swap] {}	(m-1-2)
      (m-1-2) edge node[auto] {} (m-1-1)
      (m-2-1) edge node[auto,swap] {} (m-1-1)
      (m-1-2) edge[densely dotted] node[auto,swap] {$h^\sharp$} (m-2-1);
    \end{tikzpicture}
  \end{center}
  Also gibt es für beide offenen Teile von $X$, die gleich $\mathbb A_k^1$ sind, je eine Fortsetzung.
\end{nnBsp}

\begin{Bew}
  \begin{enumerate}
  \item[``$\Rightarrow$'']
    Sei ein Bewertungsdiagramm (mit den üblichen Bezeichnungen) gegeben. Zwei $h_1,h_2:T\to X$ Fortsetzungen von
    $h_0:U\to X$, induzieren einen Morphismus $h$:
    \begin{center}
      \begin{tikzpicture}
        \matrix (m) [matrix of math nodes, row sep=3em, column sep=2.5em, text height=1.5ex, text depth=0.25ex]
        { 
          & T & \\
          &	X \times_Y X	&	\\
          X & & X \\
          & Y & \\
        };
        \path[->,font=\scriptsize]
        (m-1-2) edge[densely dotted] node[auto] {$h$}	(m-2-2)	
        edge[bend right=30] node[auto,swap] {$h_1$}		(m-3-1)
        edge[bend left=30] node[auto] {$h_2$}		(m-3-3)
        (m-2-2) edge node[auto,swap] {} (m-3-1)
        edge node[auto] {} (m-3-3)
        (m-3-1) edge node[auto,swap] {$f$} (m-4-2)
        (m-3-3) edge node[auto] {$f$} (m-4-2);
      \end{tikzpicture}
    \end{center}
    Es ist $h_1(0)=h_0(0)=h_2(0)$
    \begin{align*}
      \Rightarrow & h(0)\in\Delta=\Delta_f(X) \Rightarrow h(m)\in\overline{\{h(0)\}}\subseteq\Delta \\
      \Rightarrow & h_1(m)=h_2(m)
    \end{align*}
  \item[``$\Leftarrow$''] Nach Übungsblatt 6, Aufgabe 1 genügt es zu zeigen: $\Delta$ ist abgeschlossen in $X\times_Y X$.
    \begin{Beh}[1]
      Ist für jedes $x_1\in\Delta$ auch $\overline{\{x_1\}}\subseteq\Delta$, so ist $\Delta$ abgeschlossen.
    \end{Beh}
    Seien also $x_1\in\Delta, x_0\in\overline{\{x_1\}}, Z\defeqr\overline{\{x_1\}}_{red}, \mathcal O\defeqr\mathcal O_{Z,x_0}, K=\mathcal O_{Z,y_1}=\kappa(x_1)$
    \begin{Beh}[2]
      Es gibt einen diskreten Bewertungsring $R\subseteq K$, der $\mathcal O$ dominiert, das heißt $\mathcal O\subseteq R$ 
      und $m_{\mathcal O}=m_R\cap\mathcal O$.
    \end{Beh}
    Dann gibt es nach Proposition \ref{prop:7.4} einen Morphismus $h:T=\Spec R\to X\times_Y X$ mit $h(0)=x_1$ und $h(m)=x_0$.
    Für $h_i=pr_i\circ h$, $i=1,2$, ist $f\circ h_1=f\circ h_2$, $h_i:T\to X$. \\
    Da $x_1\in \Delta$, ist $h_1(0)=h_2(0)$. Mit $h_0\defeqr h\vert U$ folgt: $h_1=h_2\Rightarrow h(m)\in\Delta$.
  \end{enumerate}
\end{Bew}

\begin{Bew}[2]
  $m=m_{\mathcal O}$ ist endlich erzeugt, etwa $m=(x_1,\dots,x_n)$. Sei $\mathcal O^\prime=\mathcal O[\frac{X_2}{X_1},\dots,\frac{X_n}{X_1}]$
  und $I=X_1\cdot\mathcal O^\prime$. (\OE $I\neq\mathcal O^\prime$) \\
  Krullscherr Hauptidealsatz: es gibt ein Primideal $\mathfrak p\subseteq\mathcal O^\prime$ der Höhe 1 mit $I\subseteq\mathfrak p$
  (Eisenbud Theorem 10.1) \\
  $\mathcal O^\prime_{\mathfrak p}$ ist ein noetherscher lokaler Ring der Dimension 1. Sei $\tilde{\mathcal O}$ der ganze Abschluss von
  $\mathcal O^\prime_{\mathfrak p}$ in $K$. \\
  $\Rightarrow\tilde{\mathcal O}$ ist normal, $\dim\tilde{\mathcal O}=1$, \OE$\tilde{\mathcal O}$ lokal, $\tilde{\mathcal O}$ ist noethersch
  (Satz von Krull-Akizuki, Eisenbud Theorem 11.13) \\
  $\Rightarrow\tilde{\mathcal O}$ ist diskreter Bewertungsring. Es gilt: \\
  $m_{\tilde{\mathcal O}}\cap\mathcal O\subseteq m_{\mathcal O}$: Klar. \\
  $m_{\tilde{\mathcal O}}\cap\mathcal O\supseteq m_{\mathcal O}$, weil $X_1,\dots,X_n\in I$.
\end{Bew}
Behauptung 1 ist (für $f=\Delta$) ein Spezialfall von 
\begin{Prop}
  \label{prop:7.5}
  Sei $f:W\to X$ Morphismus noetherscher Schemata. Dann gilt: 
  \begin{align*}
     & f(W) \text{ ist abgeschlossen }  \\
     \Leftrightarrow & f(W) \text{ ist abgeschlossen unter Spezialisierung: Für } x_1\in f(W)\text{ und } x_0\in\overline{\{x_1\}}\text{ ist } x_0\in f(W).
  \end{align*}
\end{Prop}

\begin{Bew}
  \begin{enumerate}
  \item[``$\Rightarrow$''] Klar.
  \item[``$\Leftarrow$''] Sei $Y=\overline{f(W)}$ (als abgeschlossenes Unterschema mit reduzierter Struktur) \\
    Sei $y\in Y$; zu zeigen: $y\in f(W)$. \\
    \OE $Y=\Spec A$ affin, sei $B=\mathcal O_W(W)$. $f$ wird also induziert von $\alpha:A\to B$ und $\alpha$ ist injektiv, weil $f$ dominant ist
    (AG I, Proposition 6.8 (b) ). Sei $y^\prime\subseteq y$ ein minimales Primideal, dann gilt $y\in\overline{\{y^\prime\}}$.
    Also genügt es zu zeigen: $y^\prime\in f(W)$ (das ist die Voraussetzung) \\
    Es gilt $f^{-1}(y^\prime)=\Spec(\underbrace{B\otimes_A\kappa(y^\prime)}_{\defeql R})$.  Zu zeigen: $R\neq\{0\}$ \\
    Es ist $\kappa(y^\prime)=\FakRaum{A_{y^\prime}}{y^\prime A_{y^\prime}}$ und $A_{y^\prime}$ ist ein Körper, weil $A$ reduziert ist. ?
    Damit gilt: $R=B\otimes_A A_{y^\prime}$. Weiter gilt: $A\subseteq B\Rightarrow A\otimes_A A_{y^\prime}\subseteq B\otimes_A A_{y^\prime}=R$.
    Und $A_{y^\prime}$ ist ein flacher $A$-Modul, weil er eine Lokalisierung ist.
  \end{enumerate}
\end{Bew}

\begin{nnBsp}
  $A=\FakRaum{k[X,Y]}{(X\cdot Y)}$, $y^\prime=(X)\Rightarrow A_{y^\prime}=k(Y)$.
\end{nnBsp}

\begin{Folg}
  \label{folg:7.6}
  Für noethersche Schemata gilt:
  \begin{enumerate}
  \item Affine und abgeschlossene Einbettungen sind separiert.
  \item Die Komposition separierter Morphismen ist separiert.
  \item ``separiert'' ist stabil unter Basiswechsel.
  \item $g\circ f$ separiert $\Rightarrow$ $f$ separiert.
  \item ``separiert'' ist lokal bezüglich der Basis, das heißt: \\ $f:X\to Y$ separiert $\Leftrightarrow$ es existiert 
    eine offene Überdeckung $(U_i)$ von $Y$, sodass 
    \begin{align*}
      f\vert f^{-1}(U_i):f^{-1}(U_i)\to U_i \text{ separiert }
    \end{align*}
  \end{enumerate}
\end{Folg}

\begin{Bew}
  Übung!
\end{Bew}

\section{Eigentliche Morphismen}

\begin{Def}
  \label{def:8.1}
  Sei $f:X\to Y$ ein Morphismus von Schemata.
  \begin{enumerate}
  \item $f$ heißt \emph{lokal von endlichem Typ}, wenn es eine offene Überdeckung $(U_i)_{i\in\mathcal I}$, mit $U_i=\Spec A_i$, von $Y$ gibt
    und für jedes $i\in\mathcal I$ eine offene Überdeckung $(U_{ij})_{j\in\mathcal J_i}$, mit $U_{ij}=\Spec B_{ij}$, von $f^{-1}(U_i)$ existiert, 
    sodass für alle $i,j$ $B_{ij}$ vermöge $f^\sharp$ zu einer endlich erzeugten $A_i$-Algebra wird.
  \item $f$ heißt \emph{von endlichem Typ}, wenn in (a) alle $J_i$ endlich gewählt werden können.
  \item $f$ heißt \emph{endlich}, wenn in (a) jedes $J_i$ einelementig gewählt werden kann (also $f^{-1}(U_i)\defeql\Spec B_i$)
    und $B_i$ ein endlich erzeugter $A_i$-\emph{Modul} ist.
  \end{enumerate}
\end{Def}

\begin{Bem}
  \label{bem:8.2}
  In Definition \ref{def:8.1} kann ``es gibt eine offene affine Überdeckung'' ersetzt werden durch ``für jedes offene affine $U\subseteq Y$ gilt''.
\end{Bem}

\begin{Bew}
 (a) Übungsblatt 5, Aufgabe 2, (b) und (c) analog.
\end{Bew}

\begin{Bem}
  \label{bem:8.3}
  Ist $f:X\to Y$ endlich, so ist $f^{-1}(y)$ für jedes $y\in Y$ eine endliche Menge.
\end{Bem}

\begin{Bew}
  Sei \OE $Y=\Spec A$ affin. Dann ist auch $X=\Spec B$ affin. Es ist $f^{-1}(y)=\Spec(B\otimes_A\kappa(y))$.
  $B\otimes_A\kappa(y)$ ist eine $\kappa(y)$-Algebra und, da $B$ ein endlich erzeugter $A$-Modul ist, ist $B\otimes_A\kappa(y)$ ein 
  endlich dimensionaler $\kappa(y)$-Vektorraum. Es ist $\dim(B\otimes_A\kappa(y))=0$ (?), also $\Spec (B\otimes_A\kappa(y))$ endlich.
\end{Bew}

\begin{Def}
  \label{def:8.4}
  Ein Morphismus $f:X\to Y$ heißt \emph{eigentlich}, wenn er von endlichem Typ, separiert und universell abgeschlossen ist,
  das heißt für jeden Basiswechsel 
  \begin{center}
    \begin{tikzpicture}
      \matrix (m) [matrix of math nodes, row sep=3em, column sep=2.5em, text height=1.5ex, text depth=0.25ex]
      { 
        X\times_Y Y^\prime & X \\
        Y^\prime & Y\\
      };
      \path[->,font=\scriptsize]
      (m-1-1) edge node[auto,swap] {$f^\prime$} (m-2-1)
      edge node[auto] {} (m-1-2)
      (m-2-1) edge node[auto,swap] {} (m-2-2)
      (m-1-2) edge node[auto,swap] {$f$} (m-2-2);
    \end{tikzpicture}
  \end{center}
  ist $f^\prime$ abgeschlossen.
\end{Def}

\begin{nnBsp}
  $f:\mathbb A_k^1\to\Spec k$ ist abgeschlossen. Basiswechsel:
  \begin{center}
    \begin{tikzpicture}
      \matrix (m) [matrix of math nodes, row sep=3em, column sep=2.5em, text height=1.5ex, text depth=0.25ex]
      { 
        \mathbb A_k^2 & \mathbb A_k^1 \\
        \mathbb A_k^1 & \Spec k\\
      };
      \path[->,font=\scriptsize]
      (m-1-1) edge node[auto,swap] {$f^\prime$} (m-2-1)
      edge node[auto] {} (m-1-2)
      (m-2-1) edge node[auto,swap] {} (m-2-2)
      (m-1-2) edge node[auto,swap] {$f$} (m-2-2);
    \end{tikzpicture}
  \end{center}
  $f^\prime$ ist nicht abgeschlossen, denn: \\
  $V=V(XY-1)$ ist abgeschlossene Teilmenge von $\mathbb A_k^2$, aber $f^\prime(V)=\mathbb A_k^1-\{0\}$ ist nicht abgeschlossen.
\end{nnBsp}

\begin{Satz}
  \label{satz:3}
  Seien $X,Y$ noethersche Schemata, $f:X\to Y$ ein Morphismus von endlichem Typ. $f$ ist genau dann eigentlich, wenn es zu jedem Bewertungsdidagramm
  \begin{center}
    \begin{tikzpicture}
      \matrix (m) [matrix of math nodes, row sep=3em, column sep=2.5em, text height=1.5ex, text depth=0.25ex]
      { 
        U & X\\
        T & Y\\
      };
      \path[->,font=\scriptsize]
      (m-1-1) edge node[auto] {$\varphi$} (m-1-2)	
      edge node[auto,swap] {}	(m-2-1)
      (m-1-2) edge node[auto] {$f$} (m-2-2)
      (m-2-1) edge node[auto,swap] {} (m-2-2)
      (m-2-1) edge[densely dotted] node[auto,swap] {$h$} (m-1-2);
    \end{tikzpicture}
  \end{center}
  genau eine Fortsetzung $h$ gibt.
\end{Satz}

%HIER FEHLT NOCH WAS

\chapter{Kohomologie von Garben}
\setcounter{section}{8}
\section{$\mathcal O_X$-Modulgarben}
\begin{Def}
  Sei $(X,\mathcal O_X)$ ein lokal geringter Raum, $\mathcal F$ eine Garbe von abelschen Gruppen auf $X$. 
  $\mathcal F$ heißt \begriff{$\mathcal O_X$-Modulgarbe}, wenn gilt:
  \begin{iaufz}
  \item Für jedes offene $U\subseteq X$ ist $\mathcal F(U)$ ein $\mathcal O_X$-Modul
  \item Für $U^\prime\subseteq U\subseteq X$ offen ist $\mathcal F(U)\to\mathcal F(U^\prime)$
    ein $\mathcal O_X(U)$-Modulhomomorphismus, wobei $\mathcal F(U^\prime)$ durch den Ringhomomorphismus
     $\mathcal O_X(U)\to\mathcal O_X(U^\prime)$ zum $\mathcal O_X(U)$-Modul wird.
  \end{iaufz}
\end{Def}

\begin{Bem}
  \label{bem:9.2}
  Die $\mathcal O_X$-Modulgarben bilden mit den $\mathcal O_X$-linearen Abbildungen eine Kategorie \Cat{\mathcal O_X-Mod}.
\end{Bem}

\begin{nnBsp}
  Sei $X$ eine nichtsinguläre Kurve über einem algebraisch abgeschlossenen Körper $k$
  und $D=\sum_{P\in X}n_PP$ ein Divisor auf $X$. \\
  Für offenes $U\subseteq X$ sei
  \begin{align*}
    \mathcal L(D)(U)\defeqr & \{f\in k(X):\operatorname{div} f\vert U+D\vert U \geq 0\} \\
    =& \{f\in k(X):\forall P\in U:\ord_P(f)+n_P\geq 0\}
  \end{align*}
  $\mathcal L(D)$ ist eine $\mathcal O_X$-Modulgarbe, denn $\operatorname{div}(f\cdot g)=\operatorname{div}(f)+\operatorname{div}(g)$.
\end{nnBsp}

\begin{DefBem}
  \label{defbem:9.3}
  Seien $\mathcal F,\mathcal G$ $\mathcal O_X$-Modulgarben.
  \begin{enumerate}
  \item $\mathcal F\otimes_{\mathcal O_X}\mathcal G$ sei die zu $U\mapsto \mathcal F(U)\otimes_{\mathcal O_X(U)}\mathcal G(U)$ assoziierte Garbe.
  \item Für offenes $U\subseteq X$ sei 
    \begin{align*}
      \HomSheaf(\mathcal F,\mathcal G)(U)\defeqr\Hom_{\mathcal O_X\vert U}(\mathcal F\vert U,\mathcal G\vert U)
    \end{align*}
  \end{enumerate}
  $\mathcal F\otimes\mathcal G$ und $\HomSheaf(\mathcal F,\mathcal G)$ sind $\mathcal O_X$-Modulgarben.
\end{DefBem}

\begin{DefBem}
  \label{defbem:9.4}
  Sei $f:X\to Y$ ein Morphismus von lokalgeringten Räumen.
  \begin{enumerate}
  \item Für jede $\mathcal O_X$-Modulgarbe $\mathcal F$ ist $f_\ast\mathcal F$ eine $\mathcal O_Y$-Modulgarbe auf $Y$.
  \item Für jede $\mathcal O_Y$-Modulgarbe $\mathcal G$ ist $f^{-1}\mathcal G$ eine $f^{-1}\mathcal O_Y$-Modulgarbe und
    \begin{align*}
      f^\ast\mathcal G\defeqr f^{-1}\mathcal G\otimes_{f^{-1} \mathcal O_Y}\mathcal O_X
    \end{align*}
    eine $\mathcal O_X$-Modulgarbe.
  \end{enumerate}
\end{DefBem}

\begin{Bew}
  \begin{enumerate}
  \item Für offenes $U\subseteq Y$ ist $f_\ast \mathcal F(U)=\mathcal F(f^{-1}(U))$ ein $\mathcal O_X(f^{-1}(U))$-Modul.
    $f^\sharp_U$ ist ein Ringhomomorphismus $\mathcal O_Y(U)\to\mathcal O_X(f^{-1}(U))$. Dadurch wird $f_\ast\mathcal F(U)$
    zu einem $\mathcal O_Y(U)$-Modul.
  \item Den Garbenhomomorphismus $f^{-1}\mathcal O_Y\to\mathcal O_X$ erhält man aus $f^\sharp:\mathcal O_Y\to f_\ast\mathcal O_X$
    \begin{align*}
      f^{-1}(f^\sharp):f^{-1}\mathcal O_Y\to f^{-1}f_\ast\mathcal O_X\to\mathcal O_X
    \end{align*}
    den hintere Morphismus liefert \ref{def:image_sheaf} (d).
  \end{enumerate}
\end{Bew}

\section{Quasikohärente Garben}
\begin{DefBem}
  \label{defbem:10.1}
  Sei $X=\Spec R$ ein affines Schema, $M$ ein $R$-Modul. Für offenes $U\subseteq X$ sei
  \begin{align*}
    \widetilde{M}(U)\defeqr\{s:U\to\bigcup_{\mathfrak p\in U}M_{\mathfrak p}:~ &\text{für jedes $\mathfrak p\in U$ gibt es eine Umgebung $U_{\mathfrak p}$} \\
    &\text{und Elemente $m_{\mathfrak p}\in M$, $f_{\mathfrak p}\in R-\mathfrak p$, sodass } \\
    & \text{für alle $\mathfrak q\in U_{\mathfrak p}$ gilt: $s(\mathfrak q)=\frac{m_{\mathfrak p}}{f_{\mathfrak p}}\in M_{\mathfrak p}$}\}
  \end{align*}
  wobei $M_{\mathfrak p}=M\otimes_RR_{\mathfrak p}$ ist.
\end{DefBem}

\begin{Prop}
  \label{prop:10.2}
  Seien $X=\Spec R, M,\widetilde{M}$ wie in 10.1.
  \begin{enumerate}
  \item Für jedes $\mathfrak p\in X$ ist $\widetilde{M}_{\mathfrak p}\cong M_{\mathfrak p}$.
  \item Für jedes $f\in R$ ist $\widetilde{M}(D(f))\cong M_f$ (insbesondere $\widetilde{M}(X)\cong M$).
  \end{enumerate}
\end{Prop}

\begin{Bew}
  Wie für $\mathcal O_X$.
\end{Bew}

\begin{Bem}
  \label{bem:10.3}
  $M\mapsto \widetilde{M}$ ist ein exakter, volltreuer Funktor $\Cat{R-Mod}\to\Cat{\mathcal O_X-Mod}$, denn: \\
  Lokalisieren ist exakt, da $R_{\mathfrak p}$ flacher $R$-Modul ist (was Tensorieren exakt macht)
\end{Bem}

\begin{Bem}
  \label{bem:10.4}
  \begin{enumerate}
  \item $\widetilde{M\otimes_RN}\cong\widetilde{M}\otimes_{\mathcal O_X}\widetilde{N}$
  \item $\widetilde{\bigotimes M_i}\cong\bigotimes\widetilde{M_i}$
  \end{enumerate}
\end{Bem}

\begin{Bew}
  \begin{enumerate}
  \item $(M\otimes_RN)\otimes_RR_{\mathfrak p}\cong (M\otimes_RR_{\mathfrak p})\otimes_{R_{\mathfrak p}}(N\otimes_RR_{\mathfrak p})$
  \end{enumerate}
\end{Bew}

\begin{Bem}
  \label{bem:10.5}
  Sei $f:X\to Y$ ein Morphismus, $X=\Spec R, Y=\Spec R^\prime, \alpha: R^\prime\to R$ der zugehörige Ringhomomorphismus.
  \begin{enumerate}
  \item Für jeden $R$-Modul $M$ ist $f_\ast\widetilde{M}\cong \widetilde{_\alpha M}$. ($_\alpha M$ sei $M$ aufgefasst als $R^\prime$-Modul über $\alpha$)
  \item Für jeden $R^\prime$-Modul $N$ ist $f^\ast\widetilde{N}=\widetilde{N\otimes_{R^\prime}R}$
  \end{enumerate}
\end{Bem}

\begin{Bew}
  \begin{enumerate}
  \item
    \begin{align*}
      f_\ast\widetilde{M}(U)=&\widetilde{M}(f^{-1}(U))\text{ als $\mathcal O_Y(U)$-Modul} \\
      =& _\alpha \widetilde{M}(U) 
    \end{align*}
  \item $f^\ast\widetilde{N}(X)=(f^{-1}\widetilde{N}\otimes_{f^{-1}\mathcal O_Y}\mathcal O_X)(X)=N\otimes_{R^\prime}R$
  \end{enumerate}
\end{Bew}

\begin{Def}
  \label{def:10.6}
  Sei $(X,\mathcal O_X)$ ein Schema, $\mathcal F$ eine $\mathcal O_X$-Modulgarbe.
  \begin{enumerate}
  \item $\mathcal F$ heißt \begriff{quasi-kohärent}, wenn es eine offene affine Überdeckung $(U_i=\Spec R_i)_i$
    con $X$ und $R_i$-Moduln $M_i$ gibt, sodass 
    \begin{align*}
      \mathcal F\vert U_i\cong\widetilde{M_i}
    \end{align*}
    für alle $i$ gilt.
  \item $\mathcal F$ heißt \begriff{kohärent}, wenn in (a) jedes $M_i$ endlich erzeugbarer $R_i$-Modul ist.
  \end{enumerate}
\end{Def}

\begin{Prop}
  \label{prop:10.7}
  Eine $\mathcal O_X$-Modulgarbe $\mathcal F$ auf einem Schema $X$ ist genau dann quasi-kohärent, wenn für jedes offene affine 
  $U=\Spec R\subseteq X$ ein $R$-Modul $M$ existiert mit $\mathcal F\vert U\cong \widetilde{M}$.
\end{Prop}

\begin{Bew}
  \begin{enumerate}
  \item[1. Schritt] Sei $X$ \OE affin, denn: \\
    Sei $U=\Spec R\subseteq X$ offen und affin, $(U_i=\Spec R_i)$ die gegebene Überdeckung von $X$.
    $(U\cap U_i)$ ist eine offene Überdeckung von $U$. Überdecke $U\cap U_i$ durch $D(f_{ij}), f_{ij}\in R_i$.
    Dann gilt:
    \begin{align*}
      \mathcal F\vert D(f_{ij})=(\mathcal F\vert U)\vert D(f_{ij})=\widetilde{M_i}\vert D(f_{ij}) =\widetilde{(M_i)_{f_{ij}}}
    \end{align*}
  \end{enumerate}
\end{Bew}


\appendix

\def\indexspace{\par\medskip}
\printindex[default][\phantomsection\addcontentsline{toc}{chapter}{Vokabeln}\vspace{-1.2em}]

\end{document}
