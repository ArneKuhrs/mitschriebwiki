\documentclass[a4paper,DIV15,BCOR12mm]{article}
\newcounter{chapter}
\setcounter{chapter}{1}
\usepackage{ztheo}

\author{Die Mitarbeiter von \url{http://mitschriebwiki.nomeata.de/}}
\title{Primzerlegung}
\begin{document}
\maketitle

\section[Einführung und Motivation]{Faszination Primzahlen: Primzahlsatz (o.Bew.), gelöste
und ungelöste Probleme über Primzahlen}

\begin{satz}[Euklid, ca. 300 v. Chr.]
\[ \#\MdP = \infty \]
\end{satz}

\begin{bemerkung}
Analysis:
\begin{align*}
\sum_{n\in\MdN} \frac{1}{n} &= \infty \\
\sum_{n\in\MdN} \frac{1}{n^2} &< \infty \\
\intertext{Euler:} \sum_{p\in\MdP} \frac{1}{p}  &= \infty
\end{align*}
\end{bemerkung}

\begin{definition}
$p\in\MdP$ heiße Zwillingsprimzahl $\equizu p, p+2 \in \MdP$

$\{p,p+2\}$ heißt Primzahlzwilling
\end{definition}

\textbf{Frage:} Gibt es unendlich viele Primzahlzwillinge? Kein
Mensch hat eine Idee, wie das zu zeigen ist.

\begin{satz}[Primzahlzwillingsatz von Viggo Brun, ca. 1915]
\[ \sum_{p \text{ Zwillingsprimzahl}} \left(\frac1p + \frac 1 {p+2}\right) < \infty \]
\end{satz}

\textbf{Pierre de Fermat} (1601 -- 1665) schreibt auf den Rand
seines Exemplars von Arithmetica des Diophant: \glqq Die Gleichung
$x^n+y^n=z^n$ (mit $n\in\MdN$, $n>2$) hat keine Lösung mit $x,y,z
\in \MdNp$\grqq. \textbf{Heute}: Fermat hat recht. (Wiles 1995/96)

Fermat schrieb auch: Die Zahlen $F_n = 2^{(2^n)}+1$ sind prim. Die
Aussage ist ok für $n=0,1,2,3,4$. \textbf{Euler} konnte zeigen, dass
$F_5  = 4294967297 = 641 \cdot 6700417$. Noch 2000 ist unbekannt, ob
$F_{24}$ prim ist.

Möglichkeiten:
\begin{enumerate}
\item Kein $F_n$ mit $n>24$ ist prim.
\item Nur endlich viele $F_n$ sind prim.
\item $\#\{F_n|F_n\in \MdP\} = \infty$
\item $\#\{F_n|F_n\notin \MdP\} = \infty$
\end{enumerate}
Niemand weiß oder vermutet, was richtig ist, keine Beweisideen!

\begin{definition}
$M_p = 2^p - 1 \text{ heißt $p$-te Mersenne-Zahl}$
\end{definition}

\begin{satz}
$M_p$ ist höchsten dann prim, wenn $p\in\MdP$
\end{satz}
\begin{beweis}
Übungsaufgabe
\end{beweis}

Die größte bekannte Primzahl ist seit längerem eine
Mersenne-Primzahl, da es gute Tests gibt, z.B. Lucas/Lehmer,
verbessert von Grandall. Heute: $M_p\in\MdP$ für $p=3021327$,
$M_p>10^{2000000}$.

Eine weitere Frage an Primzahlen ist die nach der Verteilung von
$\MdP$ in $\MdN$. Bei dieser Frage spielt die Analysis eine Rolle.
\begin{satz}[Elementarer Primzahlsatz]
Sei $\Pi(x) = \#\{p\in\MdP|p\le x\}$ ($x\in\MdR$). Dann gilt:
\[ \Pi(x) \sim \frac{x}{\log x} \text{ (fast asymptotisch gleich)} \]
\end{satz}
Der Satz wurde 1792 von Gauß vermutete und 1896 von Hadamard und von
de la Vaille-Poussin nach Vorarbeiten von Riemann bewiesen

\begin{folgerung}
Sei $p_n$ die $n$-te Primzahl der Größe nach ($p_1=2, p_2=3,
p_3=5,\ldots$). Dann gilt:
\[ p_n \sim n\cdot \log n \quad (n\to\infty) \]
\end{folgerung}
\begin{beweis}
$p_n = x \implies n=\Pi(x)$
\begin{align*}
\lim_{n\to\infty} \frac{n\cdot \log n}{p_n} &= \lim_{n\to\infty} \frac{\Pi(x) \log \Pi(x)}{x}\\
&= \lim_{n\to\infty} \frac{\Pi(x)}{x/\log x} \cdot \frac{x}{\log x} \cdot \frac{\log \Pi(x)}{x} \\
&= \lim_{n\to\infty} \frac {\log  \Pi(x)}{\log x}\\
&= \lim_{n\to\infty} \frac1{\log x} \cdot \log \frac{\Pi(x)}{x/\log x} x/\log x\\
&= \lim_{n\to\infty} \frac{1}{\log x} \left( \log \frac{\Pi(x)}{x/\log x} + (\log x - \log \log x) \right) \\
&= 1 - \lim_{x\to\infty} \frac{\log(\log x)}{\log x}\\
&= 1 - \lim_{t\to\infty} \frac{\log t}t\\
&= 1 - \lim_{n\to\infty}\frac n {e^n} = 1
\end{align*}
\end{beweis}

\begin{folgerung}
$\forall \ep > 0 \ \exists N\in\MdN \ \forall x\ge N \ \exists
p\in\MdP$:
\[ x \le p \le x(1+\ep) \]
\end{folgerung}

\textbf{Riemann} (1826--66): \glqq Über die Anzahl der Primzahlen
unter einer gegebenen Größe\grqq{}  stellt Zusammenhang mit Riemanns
$\zeta$-Funktion her.
\[ \zeta(s) = \sum_{n\in\MdNp} \frac{1}{n^s}, s\in\MdC \]
$\zeta(s)$ konvergiert für $\Re s>1$ und hat eindeutige Fortsetzung
zum analytischer Funktion $\MdC\setminus1\to\MdC$ mit Pol in $s=1$.
Man kann zeigen: Primzahlsatz $\equizu$ $\zeta$ hat keine Nullstelle
mit $\Re \ge 1$.

\textbf{Vermutung}: Alle nichtreellen Nullstellen von $\zeta$ liegen
auf $\frac12+i\MdR$. Gauß vermutet: Besser als $x/\log x$
approximiert
\[ \text{li}(x) = \int_2^x \frac {du}{\log u} \quad \text{(Integrallogarithmus).} \]
Man will möglichst gute Abschätzung des Restglieds $R(x) = |\Pi(x) -
\text{li}(x)|$.

\textbf{Fakt}: Je größer die nullstellenfreien Gebiete von $\zeta$,
desto bessere Restgliedabschätzung möglich. Demnach: Beste
Restgliedabschätzung möglich, wenn Riemanns Vermutung stimmt.
\[ R(x) \le \text{Const} \cdot x^{\frac 12} \log x \]
\textbf{Fakt 2:} Von der Qualität der Restgliedabschätzung hängen in
der Informatik viele Aussagen über die theoretische Effektivität von
numerischen Algorithmen ab.

%2. Vorlesung!!!
\section{Elementare Teilbarkeitslehre in integren Ringen}
In dieser Vorlesung gilt die Vereinbarung, dass ein Ring
definitionsgemäß genau ein Einselement $1_R$ besitzt.

\begin{definition}
Ein Ring $R$ heißt \emph{integer}, wenn gilt:
\begin{enumerate}
\item $R$ ist kommutativ.
\item $\forall a,b \in R:\ ab=0 \iff a=0 \vee b=0$.
\end{enumerate}
\end{definition}
\begin{beispiel}
    Jeder Unterring eines Körpers ist integer.
\end{beispiel}
\begin{definition}
    Die Menge
    $$
    R^{\times}:=\{a \in R|\ \exists x \in R:\ ax=1=xa\}
    $$
    heißt \emph{Einheitengruppe} $R^{\times}$ des (allgemeinen)
    Ringes $R$.\\
    Leicht zu sehen ist, dass $R^{\times}$ eine Gruppe ist, $x$ ist
    das eindeutig bestimmte Inverse $a^{-1}$ von $a$.
\end{definition}
\begin{beispiel}
    $\MdZ^{\times}=\{\pm 1\}$ (klar!)\\
    $\MdZ^{n \times n}$ ist der Ring der ganzzahligen $n\times
    n$-Matrizen, $GL(\MdZ)=(\MdZ^{n \times n})^{\times}$.
    Beispielsweise für $n=2$:\\
    $$
    A=\left(
        \begin{array}{cc}
          2 & 5 \\
          1 & 3 \\
        \end{array}
      \right),\ A^{-1}=\left(
        \begin{array}{cc}
          3 & -5 \\
          -1 & 2 \\
        \end{array}
      \right),\ A\,A^{-1}=I=A^{-1}\,A \Rightarrow A \in GL_2(\MdZ).
    $$\\
    $R=K[X]$ ist der Ring der Polynome in $X$ über dem Körper $K$.
    $R^{\times}=\{\alpha \in K^{\times}=K\backslash \{0\}\}$ (Konstante,
    von $0$ verschiedene Polynome)\\
    $\MdZ, K[X]$ sind integere Ringe.
\end{beispiel}
Ab jetzt sei $R$ ein integerer Ring, $a,b,c,d,x,y,u,v,w \in R$.

\textbf{Problem}: Gleichung $ax=b$ mit der Variablen $x$.
Beispielsweise ist $3x=5$ in $R=\MdZ$ nicht lösbar, $3x=6$ hingegen
schon.

\begin{definition}
    \[ a|b \iff \exists x\in R:\ ax=b \]
    Sprechweise: $a$ teilt $b$,
    $b$ ist Vielfaches von $a$, $a$ ist Teiler von $b$.

    $\neg a|b \iff a \not |\ b$ ($a$ teilt nicht $b$).
\end{definition}
\begin{beispiel}
    $R=\MdZ$: $3 \not|\ 5,\ 3|0,\ \pm 3,\ \pm 6 \dotsc$\\
    $R=K[X]$: $(X-1)|(X^2-1)$.\\
    In jedem $R:\ \forall a \in R: 1|a \text{ (denn $a=a\cdot 1$ )}\
    \wedge\
    a|0$ (denn $0=0\cdot a$).
\end{beispiel}
\begin{satz}[Elementare Teilbarkeitseigenschaften]
\begin{enumerate}
    \item $|$ ist mit $\cdot$ verträglich:\\
          $a|b\ \wedge\ c|d \Rightarrow ac|bd$.
    \item $|$ ist mit Linearkombinationen verträglich:\\
          $a|b\ \wedge\ a|c \Rightarrow\ \forall x,y\in R:\
          a|xb+yc.$
    \item $|$ ist eine transitive und reflexive Relation und für
    $a\neq 0$ gilt:\\
    $ a|b\ \wedge\ b|a \iff \exists e \in R^{\times}:\ a=eb$.
\end{enumerate}
\end{satz}
\begin{beweis}
    Treppenbeweis $\copyright$ Dr. Rehm.
\end{beweis}
\begin{bemerkung}
$(2)$ hat einen häufigen Spezialfall: $a|b\ \wedge\ a|c \Rightarrow
a|b\pm c$.\\
\textbf{Anwendungsbeispiel}: $a|b^2\ \wedge\ a|b^2+1 \Rightarrow
a|\underbrace{b^2+1-b^2}_{=1}$.\\
\textbf{Folgerung}: $e\in R^{\times}:\ a|b \iff ea|b \iff a|eb$.\\
\textbf{Grund}: $b=xa=(xe^{-1})ea$.\\
Merke: Einheitsfaktoren ändern Teilbarkeit nicht!\\
\textbf{Folge 2}: $R$ ist disjunkte Vereinigung aller Mengen
$R^{\times}a=\{ea|e\in R^{\times}\}$.\\
\textbf{Grund}: $u\in R^{\times}a\cap R^{\times}b\iff u|a\ \wedge\
a|u\ \wedge\ u|b\ \wedge b|u$, also
$R^{\times}a=R^{\times}u(=R^{\times}b$, $eu\in R^{\times}a
\Rightarrow R^{\times}u\subset R^{\times}a$, genauso zeigt man
$R^{\times}a \subset R^{\times}u$.
\end{bemerkung}
\begin{definition}[Normierung]Auswahl je eines festen $a_{nor}$ in
$R^{\times}a$. Man wählt immer $e_{nor}=1,\ 0_{nor}=0$.\\
Standard-Normierung: $R=\MdZ,\ R^{\times}a=\{\pm a\},\
a_{nor}=\max\{R^{\times}a\}=|a|$.\\
$R=K[X],\ 0\neq f=\alpha_0+\alpha_1 X + \dotsb + \alpha_n X^n$ mit
$\alpha_n \neq 0$. Dann ist $f_{nor}=\frac{1}{\alpha_n}f$.
\end{definition}
Klar ist: Jedes $a\in R$ hat die trivialen Teiler $e\in R^{\times}$
und $ea,  e \in R^{\times}$. Nichttriviale Teiler heißen auch echte
Teiler.
\begin{beispiel}
 $R=\MdZ$, triviale Teiler von $6$ sind $\pm 1,\ \pm 6$. Echte
 Teiler sind $\pm 2,\ \pm 3$.
\end{beispiel}
\begin{definition}
    \begin{enumerate}
        \item $a\in R$ heißt unzerlegbar oder irreduzibel, falls
            $a\neq 0,\ a \notin R^{\times}$ und $a$ hat nur triviale Teiler.
        \item $R=\MdZ$. $p\in\MdZ$ heißt Primzahl $\iff$ p normiert
            und irreduzibel.
        \item $R=K[X]$. $f \in R$ heißt Primpolynom $\iff$ $f$
            irreduzibel.
    \end{enumerate}
\end{definition}

\subsubsection*{Größter gemeinsamer Teiler und kleinstes Gemeinsames
Vielfaches}

\begin{definition}
    $d$ heißt ein größter gemeinsamer Teiler von
    $a_1,a_2,\dotsc,a_n$ $:\iff$
    \begin{enumerate}
        \item $d|a_1\ \wedge\ d|a_2\ \wedge\ \dotsb \ \wedge\ d|a_n$
        (d ist gemeinsamer Teiler)
        \item $u|a_1\ \wedge\ u|a_2\ \wedge\ \dotsb \ \wedge\ u|a_n \Rightarrow u|d$
    \end{enumerate}
\end{definition}
\begin{bemerkung}
    \begin{enumerate}
        \item Bei $R=\MdZ$ ist ein bezüglich $\leq$ größter
        gemeinsamer Teiler ein normierter ggT.
        \item Eindeutigkeit des ggT: Ist $d$ ein ggT von
        $a_1,a_2,\dotsc,a_n$, so ist auch $d_{nor}$ ein ggT und $d_{nor}$
        ist durch $a_1,a_2,\dotsc,a_n$ eindeutig bestimmt:
        $d=d_{nor}=\ggt(a_1,a_2,\dotsc,a_n)$\\
        \textbf{Grund}: $e \in R^{\times}$ spielt bei Teilbarkeit keine Rolle, und
        $d_{nor}=ed$ für ein $e \in R^{\times}$. Sind $d,d'$ ggTs von $a_1,a_2,\dotsc,a_n$
        $\Rightarrow d|d'\ \wedge\ d'|d \iff d'=ed\text{, da
        normiert}\Rightarrow d=d'$.
    \end{enumerate}
\end{bemerkung}
Der kgV wird analog zum ggT unter Umkehrung aller
Teilbarkeitsrelationen definiert:

\begin{definition}
    $k$ heißt ein kgV von $a_1,a_2,\dotsc,a_n$ $:\iff$
        \begin{enumerate}
        \item $a_1|k\ \wedge\ a_2|k\ \wedge\ \dotsb \ \wedge\ a_n|k$
        (k ist gemeinsames Vielfaches)
        \item $a_1|u\ \wedge\ a_2|u\ \wedge\ \dotsb \ \wedge\ a_n|u \Rightarrow k|u$
    \end{enumerate}
\end{definition}
Die Eindeutigkeitsaussage des ggT gilt für den kgV ebenfalls.

\begin{satz}[Euklids Primzahlsatz]
    Für $R=\MdZ$ gilt:
    \[\#\MdP=\infty\]
\end{satz}
\begin{beweis}
    Es seien $p_j,\ j=1,2,\dotsc,n$ paarweise verschiedene Primzahlen.
    Betrachte $1+\prod p_i>0$.\\
    \textbf{Aussage}: Ist $a\in \MdN,\ a > 1$, so ist $\min\{d\in \MdN:
    d|a\}$ eine Primzahl und das Minimum existiert wegen $a|a$.
    Benutzt, dass jede Teilmenge der natürlichen Zahlen eine
    kleinste Zahl enthält $\Rightarrow$ Behauptung, da ein echter Teiler
    kleiner wäre  $\Rightarrow \exists p \in R:\ p|1+\prod p_j$.\\
    Wäre $p=p_j$ für ein $j\in\{1,2,\dotsc,n\}$, so $p|\prod p_j$.
    $p|\underbrace{1+\prod p_j-1(\prod p_j)}_{=1} \iff p|1
    \Rightarrow p\in \MdZ^{\times} \Rightarrow$ Widerspruch.
\end{beweis}
\section{Primzerlegung in Euklidischen Ringen, Faktorielle Ringe}
In diesem Abschnitt sei $R$ integerer Ring, $a,b,c,d,\dotsc \in
R$.\\
\textbf{Sprechweise}: $a=qb+r$. Man sagt $r$ ist der Rest bei
Division von $a$ durch $b$, $q$ ist der Quotient (Division mit
Rest).\\
\textbf{Mathematischer Wunsch}: Rest $r$ soll im geeigneten Sinn
kleiner sein als der Divisor $b$. Man benötigt dafür eine
Größenfunktion $gr:\ R\mapsto \MdN$.
\begin{definition}
    Ein Ring $R$, beziehungsweise ein Paar $(R,gr)$ heißt euklidisch
    $:\iff$
    \begin{enumerate}
        \item $R$ ist integer\\
        \item Man hat Division mit Rest, das heißt:\\ $\forall a,b \in
        R,\ b\neq 0,\ \exists q,r\in R: a=qb+r$, wobei $r=0$ oder
        $gr(r)<gr(b)$.
    \end{enumerate}
\end{definition}

Es ist $(\MdZ, |\cdot|)$ ein euklidischer Ring.
%Dritte Vorlesung!!!
\begin{beweis}
O.b.d.A: $b>0$, da $|b| = gr(b) = gr(-b)$.

$q=\lfloor \frac a b \rfloor$ ist geeignet: $0\le \frac a b - q < 1
|\cdot b \Rightarrow 0 \le a - qb  = r < b \Rightarrow gr(r) = |r| =
r<b = |b| =gr(b)$
\end{beweis}

Viele Porgrammiersprachen, etwa MAPLE, bieten einen modulo-Operator:\\
\texttt{r := (a mod b) }$= a - \lfloor \frac a {|b|} \rfloor$.

Im $K[X]$ ist die Division mit Rest möglich bezüglich $gr(f) :=
\grad f = n$, ($f\ne 0$).

Der Ring $R = \MdZ + \MdZ i \subset \MdC$, also $R =
\{x+iy|x,y\in\MdZ\}$ heißt "`Ring der ganzen Gaußschen Zahlen"'. $R$
ist euklidisch mit $gr(x,iy) = |x+iy| = \sqrt{x^2+y^2}$. Die Idee
für die Division mit Rest ist: Suche einen Gitterpunkt nahe $\frac a
b$. (siehe Übung)

\begin{lemma}
$R$ integer, $a= qb+r$, $a,b,q,r\in R$. Dann gilt
\[ \ggt(a,b) = \ggt(b,r)\,, \]
und falls eine Seite existiert, so auch die andere.
\end{lemma}

\begin{beweis}
Sind $u,v \in R$, so kann Existenz und $\ggt(u,v)$ abgelesen werden
an
\[ T(u,v) = \{d\in R \big| d|u \wedge d|v \}\,,\]
der Menge der gemeinsamen Teiler. Es ist aber $T(a,b) = T(b,r)$:\\
\glqq$\subseteq$\grqq: $d|a \wedge d|b \implies d|r$ (Linearkombination)\\
\glqq$\supseteq$\grqq: $d|r \wedge d|b \implies d|a$
(Linearkombination)
\end{beweis}

Euklids glänzende Idee ist nun: Bei der Division mit Rest
verkleinert der Übergang von $(a,b)$ zu $(b,r)$ das Problem. Sein
Algorithmus ist wie folgt:

\lstset{morecomment=[l]{\#}, texcl=True, commentstyle=\textrm}
\begin{lstlisting}
ggT := proc(a,b);      # Prozedur, die \texttt{ggT} $ = \ggt(a,b)$
aus $a$,$b$ berechnet if b = 0
  then normiere(a)     # es ist immer $\ggt(a,0) = a_\text{nor}$
  else ggT(b, a mod b) # terminiert wegen $gr(a \mod b) < gr(b)$
fi
\end{lstlisting}

Idee: $r$ ist Linearkombination von $a$ und $a,b$. Die Hoffnung
dabei ist: Auch $d:= \ggt(a,b)$ lässt sich linear kombinieren.

\begin{satz}[Satz der Linearkombination des ggT]\label{satz:LinKom}
Sei $R$ ein euklidischer Ring. Dann existiert $d=\ggt(a,b)$ für alle
$a,b\in R$ und ist als $R$-Linearkombination von $a,b,$ darstellbar:
\[ \exists x,y\in R: d= \ggt(a,b) = xa + yb \]
\end{satz}

\begin{beweis}
\begin{itemize}
\item[I] Falls $b=0$ ("`Induktionsanfang"') gilt $d= a_\text{nor}= e\cdot a + 0 \cdot b$ mit geeignetem $e\in R^\times$
\item[II] Falls $b\ne 0$: Division mit Rest $a  = qb + r$ \\
Falls $r=0$ ist $d=b_\text{nor}$, fertig!\\
Falls $r\ne0$, so gilt $\ggt(a,b) = \ggt(b,r) = d$ und $gr(r)<gr(b)$

Induktionshypothese: $\exists x_0,y_0\in R$: $d=x_0b + y_0 r = x_0 b + (a-qb)y_0  = y_0 a + (x_0-qy_0)b = xa+yb$\\
Induktionsschritt geleistet.
\end{itemize}
\end{beweis}


Die Idee ist, dass ein Ring \emph{faktoriell} heißt, wenn man in ihm
eine eindeutige Primzerlegung, wie aus $\MdZ$ bekannt, hat. Ein Ziel
der Vorlesung ist die Feststellung, dass euklidische Ringe
faktoriell sind (Euler-Faktoriell-Satz).

\begin{definition}
Ein Ring $R$ heißt faktoriell (älter: "`ZPE-Ring"') wenn gilt:
\begin{itemize}
\item[(i)] $R$ ist integer
\item[(ii)] Es gibt eine Menge $P\subseteq R$, bezüglich der jedes $a\in R$ mit $a\ne 0$ eine "`eindeutige Primzerlegung"' hat, also:

$\exists e(a)\in\MdR^\times \ \exists v_p(a) \in \MdN$, mit nur
endlich vielen $v_p(a)\ne 0$ mit
\[a = e(a) \cdot \prod_{p\in P} p^{v_p(a)} \text{ "`Primzerlegung von $a$"'}\]
Eindeutigkeit heißt: Durch $a$ sind $e(a)$ und alle $v_p(a)$
eindeutig bestimmt.
\end{itemize}
\end{definition}

Der Fall $R=\MdZ$ ist aus der Schule bekannt, und wird nicht
bewiesen. Ein Beispiel ist $-100 = -1 \cdot 2^2 \cdot 5^2$, also
$e(-100)=-1$, $v_2(-100)=v_5(-100)=2$ und $\forall p\in P, p\ne 2,
p\ne5: v_p(-100)=0$

Im Fall $R=K$, wobei $K$ ein Körper ist, gilt
$R^\times=\MdK\setminus\{0\}$ und $ P=\emptyset$.

Ist $R$ faktoriell, so ist die Standardnormierung \[a_\text{nor} =
\prod_{p\in P} p^{v_p(a)}\,.\]

\begin{bemerkung}
$P$ besteht aus unzerlegbaren Elementen. Hätte man nämlich $p=uv$
mit echten Teilern $u,v$, so gilt $u,v\notin R^\times$, also
$\forall p_1,p_2 \in P$: $v_{p_1}>0,v_{p_2}>0$. Nun haben wir zwei
Primzerlegungen, da $v_p(p) = 1$, $\forall q\in P, q\ne p, v_q(p)=0$
und damit $p=1 \cdot p^1 = 1\cdot p_1^1 \cdot p_2^1$
\end{bemerkung}

Ein Zweck der Primfaktorzelegung ist, dass die Multiplikation in $R$
auf die $R^\times$ und die Addition in $\MdN$ zurückgeführt werden
kann. Denn mit $a=e(a) \cdot \prod_{p\in P} p^{v_p(a)}$, $b=e(b)
\cdot \prod_{p\in P} p^{v_p(b)}$ gilt:
\begin{align*}
ab &= e(a) \cdot e(b) \cdot \prod_{p\in P} p^{v_p(a) + v_p(b)} \\
   &= e(ab)\cdot \prod_{p\in P} p^{v_p(ab)}
\end{align*}
Aus der Eindeutigkeit folgt nun: $e(ab) = e(a) \cdot e(b)$ und
$v_p(ab) = v_p(a)+ v_p(b)$. $v_p(a)$ heißt "`additiver $p$-Wert
von $a$"'. $v_p$ heißt (additive) $p$-adische Bewertung von $R$.

Ein weiterer Zweck liegt in der Rückführung der Teilbarkeit auf
$\le$ in $\MdN$: Für $a,b\ne 0$ gilt \[b|a \iff \ \forall p\in P:
v_p(b) \le v_p(a) \] Begründung: $nb=a \implies v_p(b) \le v_p(b) + \underbrace{v_p(n)}_{\ge 0} = v_p(a)$

Eine Folgerung davon ist, dass $\forall p\in P$ gilt:
$v_p(\ggt(a,b)) = \min\{v_p(a), v_p(b)\}$ und allgemeiner:
$v_p(\ggt(a_1,\ldots,a_n)) = \min\{v_p(a_1),\ldots,v_p(a_n)\}$.
(Damit das auch bei $a = 0$ Sinn macht, kann man $v_p(0) = \infty$
definieren, was auch üblich ist.) Ebenso gilt: $\forall p\in P$:
$v_p(\kgv(a,b)) = \max\{v_p(a), v_p(b)\}$.

Allerdings ist zur Bestimmung von $\kgv(a,b)$ folgener Algorithmus
besser als der Weg über die Primfaktorzelegung:
\begin{enumerate}
\item Berechne $\ggt(a,b)$ mit Euklids Algorithmus
\item Verwende: Sind $a,b$ normiert, so gilt:
\[ \ggt(a,b) \cdot \kgv(a,b) = ab \]
\end{enumerate}
Begründung: $\min\{v_p(a), v_p(b)\} + \max\{v_p(a), v_p(b)\} =
v_p(a) + v_p(b)$ und $ab = \prod_{p\in P} p^{v_p(a) + v_p(b)}$

Anwendungsbeispiel: Ist $m,n\in\MdN_+$, so gilt $\ggt(a^m,b^n)=1
\iff \ggt(a,b)=1$

Zusammenfassung: Für alle $a,b\in R$, $a,b\ne 0$ gilt:
\begin{itemize}
\item $v_p(ab) = v_p(a) + v_p(b)$
\item $a\in R^\times \iff \forall p\in P: v_p(a) = 0$
\item $v_p(a+b) \ge \min\{v_p(a),v_p(b)\}$
\item $v_p(\ggt(a,b)) = \min\{v_p(a),v_p(b)\}$
\end{itemize}

Noch zu zeigen: $v_p(a+b) \ge \min(v_p(a),v_p(b))$.\\
O.B.d.A: $v_p(a) \le v_p(b)$, also $\min(v_p(a),v_p(b)) = v_p(a)$.
$a = p^{v_p(a)} \cdot a_0$, $b = p^{v_p(b)}b_0$ mit $a_0, b_0 \in \MdR$.\\
$a+b = p^{v_p(a)}(a_0 + p^{v_p(b)-v_p(a)}b_0)$ $\Rightarrow
p^{v_p(a)} | a+b \Rightarrow v_p(p^{v_p(a)}) = v_p(a) \le v_p(a+b)$

\begin{bemerkung}
Ist $R$ (integrer Rang) enthalten in einem Körper, so ist $K = \{\frac{a}{b} = x | a, b \in R, b \not=0\}$ ein Körper.\\
Man kann $v_p$ auf $K$ ausdehnen: $v_p(x) = v_p(a) - v_p(b)$ ($x \ne
0$) Ist $R$ faktoriell, so hat man die "`Primzerlegung"' von $x =
\frac{a}{b}:$
\[ x = e(x) \cdot \prod_{p \in P} p^{v_p(x)} \]
mit $e(x) \in R^\times, v_p(x) \in \MdZ$. Nur endlich viele $v_p(x)$
sind $\not=$ 0.\\$x \in R \Leftrightarrow v_p(x) \ge 0\ (\forall p
\in P)$. Die Rechenregeln 1-4 gelten auch auf $K$ (siehe $R$ [Beweis
leicht]).
\end{bemerkung}

\begin{beispiel}
$v_7(\frac{7}{25}) = 1, v_5(\frac{7}{25}) = -2, v_p(\frac{7}{25}) =
0$ sonst
\end{beispiel}

\begin{lemma}
\label{lemma1} Sei $R$ euklidisch, dann gibt es eine
"`Größenfunktion"' $gr: R \to \mathbb{N}$ für die (zusätzlich) gilt:
\begin{itemize}
\item Ist $e \in R^\times, a \in R, a\not= 0: gr(ea) = gr(a)$
\item Ist $b$ ein \emph{echter} Teiler von $a \not= 0$, so ist $gr(b) < gr(a)$
\end{itemize}
\end{lemma}

\begin{beweis}
\textbf{Idee:} Ist $gr$ die gegebene Größenfunktion, so erfüllt
$$gr^*(a) = \min\{gr(ea) | e \in R^\times\}$$
die beiden Punkte des Lemmas. (Beweis wird auf die Homepage
gestellt!)
\end{beweis}

Für $R = \MdZ$ und $R = K[X]$ sind beide ohnehin richtig.\\
(z.B. $\MdZ, gr(a) = |a|, b$ echter Teiler. $a = bu, u \in
\MdZ^\times = \{\pm 1 \} \Rightarrow |a| > 1 \Rightarrow gr(a) = |a| =
|b||u|, gr(b) = |b| = \frac{|a|}{|u|} < |a| = gr(a)$. Ähnlich in
$K[x]$)

\begin{lemma}
\label{lemma2} $R$ sei euklidisch, $p \in R$ irreduzibel, $a,b \in
R$. Dann gilt:
\[p | ab \implies p | a \text{ oder } p|b\]
\end{lemma}

\begin{beweis}
O.B.d.A.: $p$ normiert, die normierten Teiler von $p$ sind $1$ und $p$.\\
\underline{Annahme:} $p \nmid a \wedge p \nmid b$\\
Falls $p \nmid a \Rightarrow \ggt(p,a) = 1$ \\
(anderenfalls $\ggt(p,a) = p$, damit $p | a$, Widerspruch!). \\
$p \nmid b \Rightarrow \ggt(p,b) = 1$. \\
Nach dem Linearkombinations-Satz: \\
$$\exists x_0, y_0, x_1, y_1 \in R: 1 = x_0p + y_0a = x_1p + y_1b$$
$$1 = 1 \cdot 1 = \underbrace{(...)}_{\in R}p + y_0y_1ab$$
$p | ab \Rightarrow p | 1 \Rightarrow p \in R^\times$, also nicht
irreduzibel, Widerspruch!
\end{beweis}

\begin{beweis}
Des Euler-Faktoriell-Satzes: $R$ euklidisch $\Rightarrow R$ faktoriell.\\
$P = \{p_{\text{nor}} | p \text{ irreduzibel}\}$ (z.B. $P = \MdP$ für $R = \MdZ$).\\
\textbf{Existenz} der Primzerlegung für $a\in R$ ($a \ne 0$)
\begin{itemize}
\item[I] Fall: $a \in R^\times$, Primzerlegung $a = e(a), \forall p \in P$: $v_p(a) = 0$
\item[II] Fall: $a$ irreduzibel $\Rightarrow p = a_{\text{nor}} \in P$, $a = ea_{\text{nor}} = ep, e \in R^\times, e(a) := e, v_p(a) = \begin{cases}1 & q = p\\0 & q \not= p\end{cases}$
\end{itemize}
Allgemeiner Fall wird durch Induktion nach $gr(a)$ bewiesen.\\
Es ist nur noch $a \in R, a \not= 0, a \not\in R^\times, a$ nicht
unzerlegbar zu betrachten $\Rightarrow a = u \cdot v$ mit $u,v$
echte Teiler. Induktions-Hypothese mit Hilfe des Lemma \ref{lemma1}
$\Rightarrow gr(u) < gr(a) \wedge gr(v) < gr(a)$, also haben $u,v$
Primzerlegung $\Rightarrow$ (Durch Ausmultiplizieren) $a$ hat
Primzerlegung: $e(a) = e(u) \cdot e(v) \in R^\times$, $v_p(a) =
v_p(u) + v_p(v)$

\textbf{Eindeutigkeit:}
$a = e(a) \cdot \prod p^{v_p(a)} = e'(a) \cdot \prod p^{v_p'(a)}$ seien zwei Primzerlegungen.\\
Zu zeigen: $e(a) = e'(a)$, $\forall p \in P: v_p(a) = v_p'(a)$\\
Induktion nach $n =: \sum_{p \in P}(v_p(a) + v_p'(a)) \in \MdN$\\
Induktionsanfang: $n=0 \Rightarrow \forall p: v_p(a) = 0 = v_p'(a) \Rightarrow e(a) = e'(a)$\\
Induktionsschritt: $n > 0 \Rightarrow \exists p: v_p(a) > 0 \vee v_p'(a) > 0$, O.B.d.A.: $v_p(a) > 0 \Rightarrow p|a = e'(a) \prod_{q\in P}q^{v_q'(a)}$\\
Aus Lemma \ref{lemma2} leicht induktiv: $p|a_1 \cdot ... \cdot a_n \Rightarrow \exists j: p | a_j \Rightarrow \underbrace{p|e'(a)}_{\text{geht nicht}} \vee \exists q\in\MdP: p|q^{v_q'(a)} \Rightarrow p|q$\\
$\Rightarrow p$ ist normierter Teiler von $q \Rightarrow p = q$ ($p = 1$ geht nicht) $\Rightarrow p | p^{v_p'(a)} \Rightarrow v_p'(a) > 0$\\
$\tilde{a} = e(a) p^{v_p(a)-1} \prod_{q \not= p}p^{v_p(a)} = e'(a)p^{v_p'(a)-1} \prod_{q \not= p}q^{v_p'(a)}$\\
Zwei Primzerlegungen von $\tilde{a}$ mit $n-2$ statt $n$.
Induktionshypothese anwendbar auf $\tilde{a} \Rightarrow e(a) =
e'(a), \forall q\not= p: v_p(a) = v_q'(a)$. $v_p(a) -1 = v_p'(a) -1
\Rightarrow$ Induktionsschritt geleistet.
\end{beweis}

Primzerlegung hat viele Anwendungen, z.B.: $\ggt(a,b) = 1
\Rightarrow \ggt(a^n, b^m) = 1$

\begin{satz}[Irrationalitätskriterium]
Sei $\alpha \in \MdC$ eine Nullstellen von $f = X^m + \gamma_1
X^{m-1} + ... + \gamma_{m-1}X + \gamma_m \in \MdZ[X]$ (d.h.
$\gamma_1,...,\gamma_m \in \MdZ$) Ist dann $\alpha \notin \MdZ$, so
$\alpha \notin \MdQ$.
\end{satz}

\begin{beweis}
Annahme $\alpha \in \MdQ$, $\alpha = \frac{z}{n}, z \in \MdZ, n \in \MdN_+$, $\ggt(z, n) = 1$\\
$0 = f(\frac{z}{n}) = \frac{z^m}{n^m} + \gamma_1\frac{z^{m-1}}{n^{m-1}} + ... + \gamma_{m-1}\frac{z}{n} + \gamma_m$, multiplizieren mit $n^m \Rightarrow$\\
$0 = z^m + n\underbrace{(...)}_{\in \MdZ} \Rightarrow n|z^m \Rightarrow n|\ggt(z^m, n) = 1$, da $\ggt(z, n) = 1$ (s.o.)\\
$n|1 \Rightarrow \alpha = \frac{z}{n} = z\in \MdZ$.
\end{beweis}

\textbf{Anwendung:} z.B. auf $f = X^k - a, a \in \MdZ (k > 1)$. Ist
$a$ keine $k$-te Potenz in $\MdZ, \alpha$ eine Nullstelle von $f$ in
$\MdC$ (sozusagen $\alpha = \sqrt[k]{a}$), so ist $\alpha$
irrational.

[$\alpha \in \MdZ: a = \alpha^k$ ist $k$-te Potenz in $\MdZ$] Tritt
zum Beispiel ein, wenn $\exists p \in \MdP: k \nmid v_p(a)$ (denn $a
= z^k \Rightarrow v_p(a) = k \cdot v_p(z)$. Etwa $\sqrt[k]{q}, q \in
\MdP$ ist immer irrational, z.B. $\sqrt{2}$.

\paragraph{Die erste Grundlagenkrise der Mathematik}
Die Pythagoräer glaubten, alle Naturwissenschaften seien durch $\MdN$ "`mathematisierbar"'. Zum Beispiel wurde Folgendes als selbstverständlich betrachtet:\\
Man kann kleinen Einheitsmaßstab $e$ (verdeutlicht durch einen gezeichneten Streckenstab mit kleinen Einheiten) wählen, so dass die Strecke $a$ und die Strecke $b$ in der Form $a = n \cdot e, b = m \cdot e$ ist, mit $n, m \in \MdN \Leftrightarrow \frac{b}{a} \in \MdQ$.\\
Modern ist die Aussage $\frac{b}{a} = \sqrt{2} \Rightarrow $ Seite und Diagonale erfüllen nicht dem Glauben.\\
Der Glaube besagt: Nur natürliche und rationale Zahlen sind Zahlen.
$\Rightarrow$ Die Länge einer Strecke ist keine Zahl.

Der Dozent glaubt, dies hat die Griechen daran gehindert "`reelle Zahlen"' zu erfinden, d.h. mit Längen von Strecken wie in einem Körper zu rechnen (wirkt über 1000 Jahre, relle Zahlen exakt erst seit ca. 1800 exakt erklärt!).\\
Heute bekannt: Die Proportionenlehre von Eudoxos von Knidos ist
logisch äquivalent zu der Konstruktion der rellen Zahlen.

\end{document}
