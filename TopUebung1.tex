\documentclass{article}
\usepackage[utf8]{inputenc}
\usepackage{amsmath}
\usepackage{amsfonts}
\usepackage{amsthm}
\usepackage{mathrsfs}
\title{1. Topologie Übung}
\author{Ferdinand Szekeresch}
\date{24. Oktober 2007}
\begin{document}
\maketitle

\textbf{Faserprodukt von Mengen}\\
\textbf{Definition}\\
Seien $A,B,S$ Mengen, $f_A : A \rightarrow S, f_B: B \rightarrow S$ Weiter sei $F$ eine Menge mit Abb $\pi_A : F \rightarrow B$ mit $f_A \circ \pi_A = f_B \circ \pi_B$ \\
F heißt Faserprodukt, von $A$ und $B$ über $S$ (Symbol: $F = A\times_sB$), wenn für jede Menge M und jedes Paar von Abbildungen $g_a,g_b$ von $M$ nach $A$ bzw. $B$ genau eine Abbildung $h: M \rightarrow F$, so dass $g_a = \pi_A \circ h, g_b = \pi_B \circ h$. \\
%\end{definition}
\textbf{Behauptung}\\
Zwischen zwei Faserprodukten von $A$ und $B$ über $S$ gibt es genau eine "sinnvolle" Bijektion.\\
\textbf{Beweis}\\
Seien $F,F'$ Faserprodukte. Nach Definition des Faserprodukts:\\
$\exists ! h: F' \rightarrow F \pi_A' = \pi_A\circ h, \pi_B' = \pi_B\circ h$\\
$\exists ! h': F \rightarrow F' \pi_A = \pi_A'\circ h, \pi_B = \pi_B'\circ h$\\
$\Rightarrow h\circ h'$ ist Abbildung von $F$ nach $F$ mit $\pi_A = \pi_A\circ (h\circ h')$, $\pi_B = \pi_B\circ (h\circ h')$
$id_F$ ist aber auch eine Abbildung mit dieser Eigenschaft.\\
$\stackrel{Def. Faserprodukt}{\Rightarrow} h \circ h' = id_F$. Genauso: $h' \circ h = id_F$.\\
\textbf{Bemerkung}\\
Zu $A,B,S,f_A,f_B$ wie oben existiert immer ein Faserprodukt.\\
\textbf{Denn}\\
Definiere $F:=\{(a,b)\in A\times B | f_A(a) = f_B(b)\}$.\\
Zu $M$ wie oben definiere $h:M\rightarrow F$, $m \mapsto\big(g_A(m),g_B(m)\big)$.\\
\textbf{Bemerkung}\\
Es gilt: $F = \bigcup\limits_{s\in S}\big(f_A^{-1}(s)\times f_B^{-1}(s)\big)$

\textbf{Beispiel eines metrischen Raums: Die Hasudorff - Metrik}\\
$M=\mathbb{R}^2, d$ sei der euklidische Abstand.\\
\textbf{Ziel}\\
Messe den Abstand zwischen Teilmengen von $M$.\\
\textbf{Definition}\\
Sei $x\in M, S\subseteq M$. Definiere $d(x,S) := \inf\{d(x,y)|y\in S\}$.\\
Seien $S,S' \subseteq M$. Definiere $d(S',S) := \sup\{d(x,S)|x\in S'\}$.\\
Das definiert keine Metrik auf $\mathscr{P}(m)$, denn im Allgemeinen ist $d(S,S')\ne d(S',S)$!\\
Definiere $H(M) := \{S\subseteq M | S$ beschränkt und abgeschlossen$\}$.\\
Definiere nun $h: H(M) \times H(M) \rightarrow \mathbb{R}_{\geq 0}, h(S,S') := \max\{d(S,S'),d(S',S)\}$.\\
\textbf{Satz}\\
$h$ ist eine Metrik auf $H(M)$.\\
\textbf{Beweis}\\
Sei $S\in H(M). h(S,S)=0$ (da $d(S;S) = 0$). Seien nun $S,S'\in H(M)$ mit $h(S,S') = 0 \Rightarrow d(S,S')=0, d(S',S)=0$.\\
$\Rightarrow S\subseteq S'$ und $S' \subseteq S$. Denn: $d(x,S) = 0 \Rightarrow x\in S$ oder $x$ ist Häufungspunkt von $S$.\\
$\big(x\notin S \Rightarrow\forall n\in\mathbb{N}\exists x_n\in S:d(x,x_n)<\frac{1}{n}\big)$\\
$\Rightarrow S=S'$.\\
Symmetrie: klar.
Dreiecksungleichung gilt auch, denn:\\
\begin{enumerate}
\item $\forall S\in H(M),x,y\in M : d(x,S)\leq d(x,y)+d(y;S)+\epsilon$\\
$\Rightarrow d(x,S) \leq d(x,y') \leq d(x,y)+d(y,S)+\epsilon$.
\item $\forall S,S' \in H(M),x\in M: d(x,S)\leq d(x,S')+d(S',S)$.\\
Denn: Sei $y'\in S'$ mit $d(x,y') \leq d(x,S')+\epsilon+d(S',S)$.\end{enumerate}
$\Rightarrow \forall x\in S_1 : d(x,S_3) \leq d(x,S_2) + d(S_2,S_3) \Rightarrow$ Beh.\\
Über wenig weitere Umformungen erhält man das Gewünschte, leider geht mir jetzt der Akku aus.
\end{document}
