\section{Freie Gruppen}

\begin{DefBem}
    Sei $F$ eine Gruppe und $X \subseteq F$
    \begin{enum}
        \item $F$ hei�t \empind{freie Gruppe mit Basis $\mathbf{X}$}{freie Gruppe},
        wenn jedes $y
        \in F$ eine eindeutige Darstellung $y = x_1^{\varepsilon_1} \cd
        \dots \cd x_n^{\varepsilon_n}$, in der
        \begin{itemize}
            \item $n \geq 0$ ($n=0$ ist das ''leere Wort'', es ist das neutrale
            Element in $F$)

            \item $x_i \in X$ f�r $i=1,\dots,n$

            \item $\varepsilon_i \in \{+1,-1\}$ f�r $i=1,\dots,n$

            \item $x_{i+1}^{\varepsilon_{i+1}} \neq x_i^{-\varepsilon_i}$ f�r $i
             =1,\dots,n-1$
        \end{itemize}

        \item Ist $F$ frei mit Basis $X$, so gilt f�r jedes $x \in X$:  $x^{-1}
        \not \in X$ und ord$(x) = \infty$.

        \item $(\mathbb{Z},+)$ ist frei mit Basis $\{1\}$ oder Basis $\{-1\}$

        \item Ist $F$ frei mit Basis $X$ und $|X| \geq 2$, so ist $F$ nicht 
        abelsch. \newline
        
        \sbew{0.9}{
            Seien $x_1, x_2 \in X: x_1 \neq x_2 \Ra x_1 x_2 x_1^{-1} x_2^{-1}
            \neq e \Ra x_1 x_2 \neq x_2 x_1$
        }
    \end{enum}
\end{DefBem}

\begin{Satz}
\label{Satz 4}
    \mbox{}
    
    \begin{enum}
        \item Zu jeder Menge $X$ gibt es eine freie Gruppe $F(X)$ mit Basis $X$.

        \item Zu jeder Gruppe $G$ und jeder Abbildung $f: X \ra G$ gibt es genau
        einen Gruppenhomomorphismus $\phi: F(X) \ra G$ mit $\phi(x) = f(x)$ f�r
        alle $x \in X$.
        
        \item Jede Gruppe ''ist'' Faktorgruppe einer freien Gruppe.
        
        \item $F(X) \cong F(Y) \Leftrightarrow |X| = |Y|$
                                
      \end{enum}
\bew{}{  
\item Sei $X^{\pm} = X \times \{\pm 1\}$ und $i: X^{\pm} \ra
X^{\pm}$ die Abbildung: $i(x, \varepsilon) = (x, -\varepsilon)$.
$i$ ist bijektiv und $i^2 = id$. \newline Schreibweise: $(x,1)
\defeql x\;,\;(x,-1) \defeql x^{-1} \Ra i(x) = x^{-1}\;,\; i(x^{-1}) = x$
\newline Ein Element $g = (x_1,\dots,x_n) \in F^a(X^{\pm})$ (freie
Worthalbgruppe) hei�t \emp{reduziert}, wenn $x_{\nu +1} \not= i(x_{\nu})$
f�r $\nu=1,\dots,n-1$. Sei $F(X)$ die Menge der reduzierten W�rter
in $F^a(X^{\pm})$
\newline \textbf{Def}.: Zwei W�rter in $F^a(X^{\pm})$ hei�en
\emp{�quivalent}, wenn sie durch endliches Einf�gen oder Streichen
von W�rtern der Form $(x,i(x)), x\in X^{\pm}$ auseinander
hervorgehen.
\newline \textbf{Bsp}.: $x_1 \sim x_1 x_2 x_2^{-1} \sim x_1 x_2
x_3^{-1} x_3 x_2^{-1}$
\newline \textbf{Beh}.: In jeder �quivalenzklasse gibt es genau
ein reduziertes Wort. Dann definiere Produkt auf $F(X) :
(x_1,\dots,x_n) \star (y_1,\dots,y_n)$ sei \emp{das} reduzierte Wort
in der �quivalenzklasse von $(x_1,\dots,x_n,y_1,\dots,y_m)$. Dieses
Produkt ist \textbf{assoziativ}: F�r $x,y,z \in F(X)$ ist $(xy)z$
das eindeutig bestimmte reduzierte Wort in der Klasse von
$(x_1,\dots,x_n,y_1,\dots,y_m,z_1,\dots,z_l)$ Das gleiche gilt f�r
$x(yz)$.
\newline neutrales Element: $e=()$ \newline
inverses Eement zu $(x_1,\dots,x_n)$ ist $(i(x_n), i(x_{n-1}),
\dots, i(x_1)) \Ra F(X)$ ist Gruppe. $F(X)$ ist frei mit Basis $X$
nach Konstruktion. \newline \textbf{Bew. der Beh}.: In jeder Klasse
gibt es ein reduziertes Wort: \textbf{ja}! \newline
\textbf{Eindeutigkeit}: Seien $x,y$ reduziert und �quivalent. Dann
gibt es ein Wort $w$, aus dem sowohl $x$ als auch $y$ durch
Streichen hervorgehen. Zu zeigen also: Jede Reihenfolge von
Streichen f�hrt zum selben reduzierten Wort. \newline Induktion �ber die L�nge
$l(w)$:\newline $l(w) = 0\; \chk$ \newline $l(w) = 1\; \chk$ \newline Sei $l(w) \geq 2$; Ist
$w$ reduziert, so ...\newline Enth�lt $w$ genau ein Paar
$(x_{\nu},i(x_{\nu}))$, so mu� dies als erstes gestrichen werden. Es
entsteht $w'$ mit $l(w') = l(w) - 2 \overset{\textbf{IV}}{\Ra}$ Beh.
Enth�lt $w$ Paare $(x_{\nu},i(x_{\nu}))$ und $(x_{\mu},
i(x_{\mu}))$, so gibt es zwei F�lle: (Sei oBdA $\mu > \nu$)\newline
$\mu = \nu +1$: $x_{\nu} i(x_{\nu}) x_{\nu}$ Dann f�hren beide
Streichungen zum selben Wort. \newline $\mu \geq \nu+2$: Streichen
beider Paare, erhalte $w''$ mit $l(w'') = l(w) - 4
\overset{\textbf{IV}}{\Ra}$ Beh.
}
\bew{}{
% TODO m�sste es hier nicht eigentlich \Varphi hei�en, denn in der Aussage gibt
% es kein \varphi oder andersrum weil sp�ter in Bezug hierauf von \varphi die
% Rede ist
\item[(b)] $\varphi(x_1,\dots,x_n) \defeqr
\tilde{f}(x_1)\cd\tilde{f}(x_2)\cd\dots \cd \tilde{f}(x_n)$ mit
\[\tilde{f}(x_i) = \left\{\begin{array}{ll} f(x_i) & x_i \in X \\
f(x_i^{-1})^{-1}    &   x_i \in X^- = \{(x,-1) \in
X^{\pm}\}\end{array}\right.\]
\item[(c)] Sei $S \subseteq G$ ein Erzeugendensystem (d.h. die einzige
Untergruppe $H$ von $G$ mit $S \subseteq H$ ist $G$ selbst). Sei
$F(S)$ die freie Gruppe mit Basis $S$, $f:S \ra G$ die Identit�t und
$\varphi:F(S) \ra G$ der Homomorphismus aus (b). $\varphi$ ist
surjektiv, weil $\varphi(F(S))$ Untergruppe ist, die $S$ enth�lt.
Also ist nach Homomorphiesatz $G \cong F(S)/\mbox{Kern}(\varphi)$
\item[(d)] ''$\Leftarrow$'' Sei $f:X \ra Y$ bijektive Abbildung. Dazu gibt
es Gruppenhomomorphismus $\varphi_f: F(X) \ra F(Y)$, sowie
$\varphi_f^{-1} F(Y) \ra F(X)\\$ $\varphi_f \circ
\varphi_{f^{-1}}\mid Y = id_Y,\; \varphi_f \circ \varphi_{f^{-1}}
\mid X = id_X\\$ $id_{F(Y)}\mid Y = id_Y \overset{\mbox{Eindt. in
(b)}}{\Ra} \varphi_f \circ \varphi_{f^{-1}} = id_{F(Y)}$, ebenso
$\varphi_f \circ \varphi_{f^{-1}} = id_{F(X)}$
\newline ''$\Ra$'' Sei $|X| \neq |Y|$. Die Anzahl der
Gruppenhomomorphismen von $F(X)$ in $\mathbb{Z}/2\mathbb{Z}$ ist
gleich der Anzahl der Abbildungen von $X$ nach
$\mathbb{Z}/2\mathbb{Z}$. (wegen (b))\newline Diese ist $|2^x| =
2^{|x|} = |(\mathbb{Z}/2\mathbb{Z})^x| = 2^{|x|}$}
\end{Satz}