\documentclass[a4paper, 10pt]{report}

% Deutsche Sprache
\usepackage{ngerman}
\usepackage[latin1]{inputenc}
\usepackage[T1]{fontenc}

\usepackage{color}
\usepackage{makeidx}
\usepackage[linktocpage,colorlinks=true,
                        urlcolor=rltblue,
                     filecolor=rltgreen,
                     linkcolor=rltblue,
                      pdftitle={Algebra II Prof. Herrlich},
                    pdfsubject={Algebra II},
                   pdfkeywords={Algebra II Herrlich A2},
                   pdfproducer={pdfLaTeX},
                   pagebackref,
                   pdfpagemode=None,
                 bookmarksopen=true]{hyperref}

\definecolor{rltred}{rgb}{0.75,0,0}
\definecolor{rltgreen}{rgb}{0,0.5,0}
\definecolor{rltblue}{rgb}{0,0,0.75}

% Mathe-Pakete
\usepackage{amssymb}
\usepackage{amsmath}
\usepackage{amsfonts}
\usepackage{stmaryrd}
\usepackage{eufrak}
%F�r Diagramme
\usepackage[arrow, graph, matrix, curve]{xy}

% Theorem-Umgebung
\usepackage[hyperref,amsmath,thmmarks,thref]{ntheorem}

% Theoreme definieren
\theoremstyle{break}
    \newtheorem{Satz}{Satz}
    \newtheorem{SatzDef}[Satz]{Satz + Definition}
    \newtheorem{Def}{Definition}[chapter]
    \newtheorem{DefBem}[Def]{Definition + Bemerkung}
    \newtheorem{Bem}[Def]{Bemerkung}
    \newtheorem{BemDef}[Def]{Bemerkung + Definition}
    \newtheorem{Prop}[Def]{Proposition}
    \newtheorem{PropDef}[Def]{Proposition + Definition}
    \newtheorem{Folg}[Def]{Folgerung}
    \newtheorem{Bsp}[Def]{Beispiele}
    \newtheorem{DefProp}[Def]{Definition + Proposition}
    \newtheorem{anBew}[Def]{Beweis}
\theoremstyle{nonumberbreak}
    \newtheorem{Bew}{Beweis}
    \newtheorem{nnBem}{Bemerkung}
    \newtheorem{nnBsp}{Beispiele}
    \newtheorem{nnSatz}{Satz}
    \newtheorem{nnFolg}{Folgerung}
    \newtheorem{Beo}{Beobachtung}
    \newtheorem{Eri}{Erinnerung}
\theoremstyle{nonumberplain}
    \theoremsymbol{\ensuremath{\Box}}
    \newtheorem{proof}{Beweis}

\newcommand{\emp}[1]{\textbf{\emph{#1}}}
\newcommand{\defeqr}[0]{\mathrel{\mathop:}=}
\newcommand{\defeql}[0]{=\mathrel{\mathop:}}
\newcommand{\Aut}[0]{\mbox{Aut}}
\newcommand{\Hom}[0]{\mbox{Hom}}
\newcommand{\Quot}[0]{\mbox{Quot}}

\newcommand{\id}[0]{\textrm{id}}

\renewcommand{\labelenumi}{\theenumi}    
\renewcommand{\theenumi}{(\alph{enumi})}
    
% Einige Anstrengungen, um den � vor die Section-Nummer zu stellen
% \renewcommand{\thesection} allein f�hrt zu einem Konflikt mit ntheorem-hyper
\makeatletter
\def\@seccntformat#1{\@ifundefined{#1@cntformat}%
{\csname the#1\endcsname\quad}% default
{\csname #1@cntformat\endcsname}% individual control
}
\def\section@cntformat{�\@arabic\c@section\quad}

\def\R{\ensuremath{\mathbb{R}}}

\setlength{\fboxrule}{0pt}
\setlength{\fboxsep}{0pt}

\makeatother

\title{Gruppen und Graphen - Wintersemester 06/07\\ Prof. Dr. F. Herrlich}
\author{Die Mitarbeiter von \url{http://mitschriebwiki.nomeata.de/}}

\makeindex

\begin{document}

\maketitle

% Inhaltsverzeichnis
\addcontentsline{toc}{chapter}{Inhaltsverzeichnis}
\tableofcontents

\documentclass[a4paper, 10pt]{report}

\usepackage{GuGStyle}

\begin{document}

\setcounter{chapter}{1}

\section{Graphen}

\begin{Def}
  Ein \emp{Graph}\index{Graph} $\Gamma$ besteht aus 2 Mengen $E = E(\Gamma)$ (\emp{Ecken}\index{Ecke}) und $K =
  K(\Gamma)$ (\emp{orientierte Kanten}\index{Kante!orientierte}) sowie einer Abbildung $I: K \to E \times
  E, k \mapsto (i(k), t(k))$ und $-: K \to K, k \mapsto \bar{k}$ mit folgenden
  Eigenschaften:
  % I = Inzidenzabbildung
  \begin{enumerate}
    \item $k \not= \bar{k} \; \forall k \in K$
    \item $k = \bar{\bar{k}} \; \forall k \in K$
    \item $i(k) = t(\bar{k}) \; \forall k \in K$
  \end{enumerate}
  (Dann gilt auch $t(\bar{k}) = i(\bar{\bar{k}}) = i(k)$)\\
  Ein Paar $\{k, \bar{k} \} \textrm{ mit } k \in K$ heißt \emp{geometrische Kante}\index{Kante!geometrische}.
\end{Def}

\begin{nnBsp}
\fboxsep20pt
% Einzelner Knoten
\fbox{\xygraph{
!{<0cm,0cm>;<1cm,0cm>:<0cm,1cm>::}
!{(0,0) }*+{\bullet}
}}
% 2 Knoten mit Kante
\fbox{\xygraph{
!{<0cm,0cm>;<1cm,0cm>:<0cm,1cm>::}
!{(0,0) }*+{\bullet}="a"
!{(0,-1) }*+{\bullet}="b"
"a"-"b"^{k}
}}
% mehrere Kanten zwischen 2 Knoten
\fbox{\xygraph{
!{<0cm,0cm>;<1cm,0cm>:<0cm,1cm>::}
!{(0,-1) }*+{\bullet}="a"
!{(2,-1) }*+{\bullet}="b"
"a"-"b"
"a"-@/^1cm/"b"
"a"-@/_1cm/"b"
}}
% Schleifen
\fbox{\xygraph{
!{<0cm,0cm>;<1cm,0cm>:<0cm,1cm>::}
!{(0,-1) }*+{\bullet}="a"
!{(2,-1) }*+{\bullet}="b"
!{(1,-2) }*+{\bullet}="c"
"a"-"b"
"a" -@(lu,ld) "a"
"b" -@(ld,rd) "b"
"b" -@(lu,ru) "b"
"a"-"c"
}}
% Pfad
\fbox{\xygraph{
!{<0cm,0cm>;<1mm,0cm>:<0cm,1cm>::}
!{(3,0) }*+{}="a3"
!{(4,0) }*+{}="a4"
!{(5,0) }*+{}="a5"
!{(6,0) }*+{}="a6"
!{(7,0) }*+{}="a7"
!{(8,0) }*+{}="a8"
!{(9,0) }*+{}="a9"
!{(10,0) }*+{\bullet}="b"
!{(20,0) }*+{\bullet}="c"
!{(30,0) }*+{\bullet}="d"
!{(33,0) }*+{}="e3"
!{(34,0) }*+{}="e4"
!{(35,0) }*+{}="e5"
!{(36,0) }*+{}="e6"
!{(37,0) }*+{}="e7"
"a3"-"a4"
"a5"-"a6"
"a7"-"b"
"b"-"c"
"c"-"d"
"d"-"e3"
"e4"-"e5"
"e6"-"e7"
}}
% Stern
\fbox{\xygraph{
!{<0cm,0cm>;<1cm,0cm>:<0cm,1cm>::}
!{(1,-1) }*+{\bullet}="a"
!{(0,0) }*+{\bullet}="b"
!{(1,0) }*+{\bullet}="c"
!{(2,0) }*+{\bullet}="d"
!{(0,-1) }*+{\bullet}="e"
!{(2,-1) }*+{\bullet}="f"
!{(0,-2) }*+{\bullet}="g"
!{(1,-2) }*+{\bullet}="h"
!{(2,-2) }*+{\bullet}="i"
"a"-"b"
"a"-"c"
"a"-"d"
"a"-"e"
"a"-"f"
"a"-"g"
"a"-"h"
"a"-"i"
}}
\end{nnBsp}

\begin{nnBem}
  $\Gamma$ kann ``topologisiert'' werden.\\
  Geometrische Kanten werden mit $[0,1]$ identifiziert.\\
  Möglich sogar	als allgemeiner Teilraum von $\R^3$, im Allgemeinen aber nicht
  von $\R^2$:
% Netzwerk, dass im R^2 nicht planar ist
\fbox{\xygraph{
!{<0cm,0cm>;<1cm,0cm>:<0cm,1cm>::}
!{(0,0) }*+{\bullet}="a"
!{(1,0) }*+{\bullet}="b"
!{(2,0) }*+{\bullet}="c"
!{(0,1) }*+{\bullet}="d"
!{(1,1) }*+{\bullet}="e"
!{(2,1) }*+{\bullet}="f"
"a"-"d"
"a"-"e"
"a"-"f"
"b"-"d"
"b"-"e"
"b"-"f"
"c"-"d"
"c"-"e"
"c"-"f"
}}
\end{nnBem}

\begin{DefBem}
\begin{enumerate}
  \item Seien $\Gamma \textrm{ und } \Gamma'$ Graphen.
  Ein \emp{Morphismus}\index{Morphismus} $f: \Gamma \to \Gamma'$ ist ein Paar $f = (f_E, f_K)$ von
  Abbildungen $f_E: E(\Gamma) \to E(\Gamma'), \; f_K: K(\Gamma) \to K(\Gamma')$
  mit $I'(f_K(k)) = (f_E(i(k)), f_E(t(k)))  = (f_E \times f_E)(I(k)) \forall k \in K(\Gamma)$ und $f_k(\bar{k}) = \overline{f_k(k)} \; \forall k \in
  K(\Gamma)$.
  \item $f$ heißt \emp{Isomorphismus}\index{Isomorphismus}, wenn es einen Morphismus $g: \Gamma' \to \Gamma$
  gibt mit $f \circ g = \id_{\Gamma'} \textrm{ und } g \circ f = \id_{\Gamma}$.
  \item $f$ ist Isomorphismus $\Leftrightarrow f_E \textrm{ und } f_K$ sind
  bijektiv.
  \item Ein Isomorphismus $f: \Gamma \to \Gamma$ heißt \emp{Automorphismus}\index{Automorphismus}.
\end{enumerate}
\end{DefBem}

% TODO Beweis zu c) ist Übungsaufgabe

\begin{DefBem}
\begin{enumerate}
  \item Ein \emp{Weg}\index{Weg} (der Länge $n \ge 0$) in $\Gamma$ ist eine Folge $w = (k_1,
  \ldots, k_n)$ von Kanten $k_i \in K \textrm{ mit } t(k_i) = i(k_{i+1}) \mbox{
  für }i=1, \ldots, n-1$.\\
  $i(w) \defeqr i(k_1), \; t(w) \defeqr t(k_n)$ heißt Anfangs-, bzw. Endpunkt 
  von $w$.\\ Ist $n=0$, so wird $i(w) = t(w) \in E(\Gamma)$ definiert. (Für jede Ecke ein Weg der Länge 0.)
  \item Sei $P_n = $
  \fbox{\xygraph{
  !{<0cm,0cm>;<1mm,0cm>:<0cm,1cm>::}
  !{(0,0) }*+{\bullet_{0}}="a"
  !{(10,0) }*+{\bullet_{1}}="b"
  !{(14,0) }*+{}="c4"
  !{(15,0) }*+{}="c5"
  !{(16,0) }*+{}="c6"
  !{(17,0) }*+{}="c7"
  !{(18,0) }*+{}="c8"
  !{(19,0) }*+{}="c9"
  !{(25,0) }*+{\bullet_{n-1}}="d"
  !{(37,0) }*+{\bullet_{n}}="e"
  "a"-"b"
  "b"-"c4"
  "c5"-"c6"
  "c7"-"c8"
  "c9"-"d"
  "d"-"e"
  }}
  (fester Graph mit $n+1$ Ecken und $2n$ Kanten).
  Dann ist jeder Weg der Länge $n$ in $\Gamma$ das Bild von $P_n$ unter einem
  Morphismus $P_n \to \Gamma$.
  \item $\Gamma$ heißt \emp{zusammenhängend}\index{Graph!zusammenhängender}, wenn es für alle $x,y \in E(\Gamma)$
  einen Weg $w$ in $\Gamma$ von $x$ nach $y$ gibt mit $i(w) = x \textrm{ und }t(w) = y$.
  \item Ein Weg heißt \emp{stachelfrei}\index{Weg!stachelfreier} (``without backtracking''), wenn $k_{i+1} \not=
  \overline{k_i}$ für $i=1,\ldots, n-1$.
  \item Gibt es in $\Gamma$ einen Weg von $x$ nach $y$, so gibt es auch einen
  stachelfreien Weg von $x$ nach $y$.
  \begin{Bew}
  Sei $w = (k_1, \ldots, k_n)$ und $k_{i+1} = \overline{k_i} \Rightarrow
  i(k_{i+2}) = t(k_{i+1}) = i(\overline{k_i}) = t(k_{i-1}) \Rightarrow w' =
  (k_1, \ldots, k_{i-1}, k_{i+2}, \ldots, k_n)$ ist Weg mit $i(w') = i(w), \;
  t(w') = t(w)$.
  \end{Bew}
\end{enumerate}  
\end{DefBem}

\begin{DefBem}
\begin{enumerate}
  \item Ein Weg $w \textrm{ in } \Gamma$ heißt \emp{geschlossen}\index{Weg!geschlossener}, wenn $t(w) = i(w)$.
  \item $w$ heißt \emp{einfach}\index{Weg!einfacher}, wenn $i(k_i) \not= i(k_j)$ für $i \not= j$.
  \item Einfache Wege sind stachelfrei \fbox{\xygraph{
  !{<0cm,0cm>;<1cm,0cm>:<0cm,1cm>::}
  !{(0,0) }*+{\bullet}="a"
  !{(2,0) }*+{\bullet}="b"
  "a":"b"
  "b":"a"
  }}\\
  (für Wege der Länge 2 ist das eine neue Definition!).
  \item Ein einfacher geschlossener Weg der Länge $n \ge 1$ heißt \emp{Kreis}\index{Kreis}.\\
  Ein Kreis der Länge 1 heißt \emp{Schleife}\index{Schleife} (loop).\\
  \fboxsep20pt
  % Schleife
  \fbox{\xygraph{
  !{<0cm,0cm>;<1cm,0cm>:<0cm,1cm>::}
  !{(1,0) }*+{\bullet}="a"
  "a"-@(ld,rd) "a"
  }}
  % Kreis mit 2 Knoten
  \fbox{\xygraph{
  !{<0cm,0cm>;<1cm,0cm>:<0cm,0.5cm>::}
  !{(0,-1) }*+{\bullet}="a"
  !{(2,-1) }*+{\bullet}="b"
  "a"-@/^0.5cm/"b"
  "a"-@/_0.5cm/"b"
  }}
\end{enumerate}
\end{DefBem}

\begin{DefBem}
\begin{enumerate}
  \item Ein Paar $k_1 \not= k_2$ von Kanten in $\Gamma$ heißt \emp{Doppelkante}\index{Kante!Doppel}, wenn
  $i(k_1) = i(k_2)$ und $t(k_1) = t(k_2)$ ist.
  % Doppelkante
  \fbox{\xygraph{
  !{<0cm,0cm>;<1cm,0cm>:<0cm,0.5cm>::}
  !{(0,-1) }*+{\bullet}="a"
  !{(2,-1) }*+{\bullet}="b"
  "a":@/^0.5cm/"b" _{k_1}
  "a":@/_0.5cm/"b" ^{k_2}
  }}
  \item Ein Graph heißt \emp{kombinatorisch}\index{Graph!kombinatorischer}, wenn er keine Schleifen und keine
  Doppelkanten enthält (also keine Kreise der Länge $\le 2$).
  \item Ein Graph ist genau dann kombinatorisch, wenn er als topologischer Raum
  ein Simplizialkomplex ist.\\
  ($n$-Simplex $=$ konvexe Hülle der Einheitsvektoren im $\R^{n+1}$)
\end{enumerate}
\end{DefBem}

\begin{DefBem}
Sei $\Gamma$ ein zusammenhängender Graph.\\
Für $x,y \in E(\Gamma)$ sei
$$d(x,y) \defeqr \min\{n: \textrm{ es gibt einen Weg der Länge } n \textrm{ von 
} x \textrm{ nach } y \}$$
$d$ ist eine Metrik auf $\Gamma$ (eigentlich auf $E(\Gamma)$).\\
$d(\Gamma) \defeqr \sup\{d(x,y): x,y \in E(\Gamma) \}$ heißt \emp{Durchmesser}\index{Durchmesser} von $\Gamma$.
\end{DefBem}

\end{document}
\documentclass[a4paper, 10pt]{report}

\usepackage{GuGStyle}

\setcounter{chapter}{1}
\setcounter{section}{1}
\setcounter{Satz}{0}

\begin{document}

\section{Bäume}

\begin{Def}
Ein \emp{Baum}\index{Baum} ist ein zusammenhängender Graph ohne Kreise (der Länge $\ge 1$)
\fboxsep20pt
% Knoten
\fbox{\xygraph{
!{<0cm,0cm>;<1cm,0cm>:<0cm,1cm>::}
!{(0,0) }*+{\bullet}="a"
}}
% Baum mit 2 Knoten
\fbox{\xygraph{
!{<0cm,0cm>;<1cm,0cm>:<0cm,1cm>::}
!{(0,0) }*+{\bullet}="a"
!{(1,0) }*+{\bullet}="b"
"a"-"b"
}}
% Baum mit 3 Knoten
\fbox{\xygraph{
!{<0cm,0cm>;<1cm,0cm>:<0cm,1cm>::}
!{(0,0) }*+{\bullet}="a"
!{(1,0) }*+{\bullet}="b"
!{(2,0) }*+{\bullet}="c"
"a"-"b"
"b"-"c"
}}
% Baum (Stern)
\fbox{\xygraph{
!{<0cm,0cm>;<1cm,0cm>:<0cm,1cm>::}
!{(1,-1) }*+{\bullet}="a"
!{(1,0) }*+{\bullet}="b"
!{(0,-2) }*+{\bullet}="c"
!{(2,-2) }*+{\bullet}="d"
"a"-"b"
"a"-"c"
"a"-"d"
}}
\end{Def}

\begin{Prop}
Ein Graph ist genau dann ein Baum, wenn es zu je 2 Ecken $x,y \in E(\Gamma)$
genau einen stachelfreien Weg von $x$ nach $y$ in $\Gamma$ gibt.
\end{Prop}
\begin{Bew}
\begin{itemize}
  \item[$\Rightarrow:$] Seien $x,y \in E(\Gamma), \; w = (k_1, \ldots, k_n)
  \textrm{ und } w' = (k_1', \ldots, k_n')$ stachelfreie Wege von $x$ nach $y$.\\
  Ist $k_n \not= k_m'$, so ist $\tilde{w} = (k_1, \ldots, k_n, k_m', \ldots,
  k_n')$ ein stachelfreier geschlossener Weg, enthält also einen Kreis
  $\Rightarrow$ Widerspruch.\\
  
  % TODO da oben noch ein Bildchen

  $\Rightarrow k_n = k_m'$. Induktion über $n$ ergibt Behauptung.
\end{itemize}

\end{Bew}

% 2006-10-26

\begin{DefBem}
Sei $\Gamma$ ein Graph, $x \in E(\Gamma)$
\begin{enumerate}
  \item Sei $K_x \defeqr \{k \in K(\Gamma): i(k)=x\}$,\\
  $v(x) \defeqr \#K_x$ heißt \emp{Ordnung}\index{Ecke!Ordnung} (oder Valenz,
  Index, $\ldots$) von $x$.\\

  \fbox{ \xygraph{
  !{<0cm,0cm>;<1cm,0cm>:<0cm,1cm>::}
  !{(0,0) }*+{\bullet_{x}}="a"
  !{(1,0) }*+{\bullet}="b"
  "a":@/^0.5cm/"b" ^{k_1}
  "a":@/_0.5cm/"b" _{k_2}
  }}
  $\Rightarrow v(x)=2$

  \item $x$ heißt Endpunkt, wenn $v(x) = 1$.
  \item $\Gamma \setminus x$ sei der Graph mit $E(\Gamma \setminus x) =
  E(\Gamma) \setminus \{x\}$ und $K(\Gamma \setminus x) = K(\Gamma) \setminus
  (K_x \cup \overline{K_x})$\\
  $\Gamma \setminus x$ ist ein Teilgraph von $\Gamma$ (Entfernen des ``Sterns''
  um $x$)\\
  
  
  $\Gamma=$
  \fbox{ \xygraph{
  !{<0cm,0cm>;<1cm,0cm>:<0cm,1cm>::}
  !{(0,0) }*+{\bullet_{x}}="a"
  !{(1,0) }*+{\bullet}="b"
  !{(2,1) }*+{\bullet}="c"
  !{(2,-1) }*+{\bullet}="d"
  "a":"b"
  "b":"c"
  "b":"d"
  "c":"a"
  }}
  $\Rightarrow \Gamma \setminus x = $
  \fbox{ \xygraph{
  !{<0cm,0cm>;<1cm,0cm>:<0cm,1cm>::}
  !{(1,0) }*+{\bullet}="b"
  !{(2,1) }*+{\bullet}="c"
  !{(2,-1) }*+{\bullet}="d"
  "b":"c"
  "b":"d"
  }}
  

  \item Ist $x$ ein Endpunkt, so gilt:
  \begin{enumerate}
    \item[(1)] $\Gamma$ zusammenhängend $\Leftrightarrow \Gamma \setminus x$
    zusammenhängend.
    \item[(2)] Jeder Kreis von $\Gamma$ ist in $\Gamma \setminus x$ enthalten.
    \item[(3)] $\Gamma$ ist Baum $\Leftrightarrow \Gamma \setminus x$ ist Baum.
  \end{enumerate}
  \begin{Bew}
  (1) und (2) sind offensichtlich, (3) folgt daraus.
  \end{Bew}
\end{enumerate}
\end{DefBem}


\begin{Prop}
\begin{enumerate}
  \item Sei $f: \Gamma \to \Gamma'$ ein Isomorphismus von Graphen.\\
  Dann gilt für alle $x \in E(\Gamma): v(x) = v(f_E(x))$.
  \item Sei $\Gamma$ ein Graph, $\textrm{Endpunkte}(\Gamma) \defeqr ...$,
  $\Gamma' \defeqr \Gamma \setminus \textrm{Endpunkte}(\Gamma)$.\\
  Jeder Automorphismus von $\Gamma$ induziert einen Automorphismus von
  $\Gamma'$.
  \item Ist $\Gamma$ ein Baum von endlichem Durchmesser $n$, so gibt es falls
  $n$ gerade/ungerade eine Ecke $x \in E(\Gamma)$/geometrische Kante $\kappa =
  (k, \bar{k})$ mit $f(x) = x$/$f(\kappa) = \kappa$ für jeden Automorphismus $f$
  von $\Gamma$.
  
  % TODO Die Fallunterscheidung überarbeiten
  
\end{enumerate}
\end{Prop}

\begin{Bew}
\begin{enumerate}
  \item \label{Bew2.4a}
  $f_K$ induziert Bijektion $K_x \to K'_{f_E(x)}$
  \item folgt aus \ref{Bew2.4a}
  \item
  \begin{itemize}
    \item[n$=$0]
    \fbox{\xygraph{
    !{<0cm,0cm>;<1mm,0cm>:<0cm,1cm>::}
    !{(0,0) }*+{\bullet}="a"
    }}
    
    \item[n$=$1]
    \fbox{\xygraph{
    !{<0cm,0cm>;<1mm,0cm>:<0cm,1cm>::}
    !{(0,0) }*+{\bullet}="a"
    !{(10,0) }*+{\bullet}="b"
    "a"-"b"
    }}
    
    \item[n$=$2]
    \fbox{\xygraph{
    !{<0cm,0cm>;<1mm,0cm>:<0cm,1cm>::}
    !{(0,0) }*+{\bullet}="a"
    !{(10,0) }*+{\bullet}="b"
    !{(20,0) }*+{\bullet}="c"
    "a"-"b"
    "b"-"c"
    }}\\
    \begin{Beh}
      $\Gamma'$ ist Baum vom Durchmesser $n-2$.
    \end{Beh}

    Daraus folgt mit Induktion über $n$ die Aussage der Proposition.
    
    \begin{Bew}
      Sei $w = (k_1, \ldots, k_m)$ stachelfreier Weg in $\Gamma'$, $x=i(w), \; y
      = t(w)$.
      Dann ist $m = d(x,y)$.\\
      In $\Gamma$ gilt: $v(x) \ge 2, \; v(y) \ge 2$, also gibt es Kanten $k_1',
      \; k_m'$ mit $i(k_1')=x, \; k_1' \not= k_1$ und $i(k_m') = y, \; k_m'
      \not= \overline{k_m}$.\\
      
      \fbox{\xygraph{
      !{<0cm,0cm>;<1mm,0cm>:<0cm,1cm>::}
      !{(1,1) }*+{\bullet}="s"
      !{(11,0) }*+{\bullet_{x}}="x"
      !{(21,0) }*+{\bullet}="a"
      !{(22,0) }*+{}="a2"
      !{(23,0) }*+{}="a3"
      !{(24,0) }*+{}="a4"
      !{(25,0) }*+{}="a5"
      !{(26,0) }*+{}="a6"
      !{(27,0) }*+{}="a7"
      !{(28,0) }*+{}="a8"
      !{(29,0) }*+{}="a9"
      !{(30,0) }*+{\bullet}="b"
      !{(40,0) }*+{\bullet_{y}}="c"
      !{(50,1) }*+{\bullet}="d"
      "x":"s" _{k_1'}
      "x":"a" ^{k_1}
      "a"-"a4"
      "a5"-"a6"
      "a7"-"b"
      "b":"c" ^{k_m}
      "c":"d" _{k_m'}
      }}\\
      Dann ist $w' = (\overline{k_1'}, k_1, \ldots, k_m, k_m')$ stachelfreier
      Weg in $\Gamma$.\\
      $\Rightarrow m+2 \le n \Rightarrow m \le n-2$.\\
      Sei umgekehrt $w = (k_1, \ldots, k_n)$ stachelfreier weg in $\Gamma$.\\
      Für $i=2, \ldots, n$ ist $v(i(k_i)) \ge 2$\\
      $\Rightarrow (k_2, \ldots, k_{n-1})$ ist stachelfreier Weg in $\Gamma'$\\
      $\Rightarrow d(\Gamma') \ge 2$.
    \end{Bew}
  \end{itemize}
\end{enumerate}
\end{Bew}

\begin{nnBsp}

\fbox{\xygraph{
!{<0cm,0cm>;<1mm,0cm>:<0cm,1cm>::}
!{(3,0) }*+{}="a3"
!{(4,0) }*+{}="a4"
!{(5,0) }*+{}="a5"
!{(6,0) }*+{}="a6"
!{(7,0) }*+{}="a7"
!{(8,0) }*+{}="a8"
!{(9,0) }*+{}="a9"
!{(10,0) }*+{\bullet}="b"
!{(20,0) }*+{\bullet}="c"
!{(30,0) }*+{\bullet}="d"
!{(33,0) }*+{}="e3"
!{(34,0) }*+{}="e4"
!{(35,0) }*+{}="e5"
!{(36,0) }*+{}="e6"
!{(37,0) }*+{}="e7"
"a3"-"a4"
"a5"-"a6"
"a7"-"b"
"b"-"c"
"c"-"d"
"d"-"e3"
"e4"-"e5"
"e6"-"e7"
}}
ist Baum.
Translation ist Automorphismus ohne Fixpunkt und Fixkante.
\end{nnBsp}


\begin{Folg}
\label{2.15}
Jeder endliche Baum besteht aus
\fbox{\xygraph{
!{<0cm,0cm>;<1mm,0cm>:<0cm,1cm>::}
!{(0,0) }*+{\bullet}="a"
}}
durch wiederholtes Anhängen von ``Endpunkten''.
\end{Folg}


\begin{DefBem}
Sei $\Gamma$ ein Graph.
\begin{enumerate}
  \item Ein Teilbaum $T \subseteq \Gamma$ heißt
  \emp{aufspannend}\index{Baum!auspannender} (oder \emp{Gerüst}\index{Gerüst}),
  wenn $E(T)=E(\Gamma)$.
  \item Jeder zusammenhängende Graph hat einen aufspannenden Teilbaum.
  \begin{Bew}
  Sei $T \subset \Gamma$ Teilbaum.\\
  Ist $E(T) \not= E(\Gamma)$, so gibt es eine Kante $k \in K(\Gamma)$ mit $i(k)
  \in E(\Gamma)$ und $t(k) \not\in E(\Gamma)$.\\
  Dann ist $T' = T \cup \{k, \bar{k}, t(k)\}$ Teilbaum von $\Gamma$.\\
  Ist $T$ endlich, so erhalten wir mit Induktion Teilbaum $T$ mit $E(T) =
  E(\Gamma)$.\\
  Falls nicht betrachte die Menge $\mathcal{T}$ der Teilbäume von $\Gamma$.\\
  $\mathcal{T} \not= \emptyset$ \chk\\ 
  Ist $T_1 \subseteq T_2 \subseteq \ldots$ Kette von Teilbäumen, so ist
  $\bigcup_i T_i \in \mathcal{T}$ \chk\\
  Zorn sagt: $\mathcal{T}$ enthält maximales Element $T$.\\
  Angenommen $E(T) \not= E(\Gamma)$. Dann gibt es nach obiger Konstruktion
  Teilbaum $T'$ mit $T \subsetneq T'$ \WSpr zu $T$ maximal
  \end{Bew}
\end{enumerate}
\end{DefBem}


\begin{BemDef}
\label{2.7}
Sei $\Gamma$ ein endlicher zusammenhängender Graph, $e(\Gamma) \defeqr
\#E(\Gamma), \; k(\Gamma) \defeqr \frac{1}{2}\#K(\Gamma)$.
Dann gilt:
\begin{enumerate}
  \item $g(\Gamma) \defeqr k(\Gamma) - e(\Gamma) + 1 \ge 0$.
  \item \label{2.7b}
  $g(\Gamma) = 0 \Leftrightarrow \Gamma$ ist Baum.
  \item $g(\Gamma)$ heißt \emp{Geschlecht}\index{Geschlecht} von $\Gamma$ (oder
  zyklomatische Zahl, Betti-Zahl).
\end{enumerate}
\end{BemDef}

\begin{Bew}
Ist $\Gamma$ ein Baum, so ist $g(\Gamma) = 0$ wegen Folgerung \ref{2.15}.\\
Ist $\Gamma$ kein Baum, so sei $T$ Gerüst von $\Gamma$, also $e(T) = e(\Gamma),
\; k(\Gamma) > k(T) \Rightarrow$ Behauptung
\end{Bew}


\begin{DefBem}
Sei $\Gamma$ ein Graph, $Z$ ein zusammenhängender Teilgraph.\\
$\Gamma/Z$ sei der folgende Graph:\\
$E(\Gamma/Z) = (E(\Gamma) \setminus E(Z)) \cup \{Z\}$\\
$K(\Gamma/Z) = K(\Gamma) \setminus K(Z)$\\
$I_Z(k) = (i_Z(k), t_Z(k))$ mit 
$i_Z = \begin{cases}i(k), & i(k) \not\in E(Z)\\ Z, & i(k) \in E(Z)
\end{cases}$ und 
$t_Z = \begin{cases}t(k), & t(k) \not\in E(Z)\\ Z, & t(k) \in E(Z)
\end{cases}$ für $k \in K(\Gamma/Z)$.
\end{DefBem}


\begin{nnBsp} 
$Z$ ist Gerüst von $\Gamma$.\\
\fbox{ \xygraph{
!{<0cm,0cm>;<1cm,0cm>:<0cm,1cm>::}
!{(0,0) }*+[blue]{\bullet_{a}}="a"
!{(1,1) }*+[blue]{\bullet_{b}}="b"
!{(2.5,0.5) }*+[blue]{\bullet_{c}}="c"
!{(3,-1)}*+[blue]{\bullet_{d}}="d"
"a"-@[blue]"b" "a"-"d"
"b"-@[blue]"c"
"b"-@[blue]"d"
} }
$\Rightarrow$
\fbox{ \xygraph{
!{<0cm,0cm>;<1cm,0cm>:<0cm,1cm>::}
!{(0,0) }*+[blue]{\bullet_{z}}="a"
"a"-@(ld,rd)"a"
} }
\end{nnBsp}

\begin{Bem}
\label{2.9}
Ist $\Gamma$ endlicher zusammenhängender Graph, $Z$ Teilgraph, so gilt:
\begin{enumerate}
  \item $g(\Gamma/Z) \le g(\Gamma)$.
  \item Falls $Z$ Teilbaum ist, so ist $g(\Gamma/Z) = g(\Gamma)$.
\end{enumerate}
\end{Bem}


\begin{Prop}
Sei $\Gamma$ ein zusammenhängender Graph. Z ein Teilgraph, dessen
Zusammenhangskomponenten Bäume sind. Dann gilt:\\
$\Gamma/Z$ ist Baum $\Leftrightarrow \Gamma$ ist Baum.
\end{Prop}

\begin{Bew}
Ist $\Gamma$ endlich, so folgt die Behauptung aus Bemerkung \ref{2.9} und
Bemerkung \ref*{2.7}\ref{2.7b}.\\
Ist $\Gamma$ unendlich, so sei $\Gamma'$ endlicher zusammenhängender Teilgraph
von $\Gamma$.\\
Dann ist $\Gamma' \cap Z$ ein ``Wald''.
Also ist $\Gamma'$ Baum $\Leftrightarrow \Gamma'/(\Gamma' \cap Z)$ Baum.\\
Schöpfe $\Gamma$ aus durch endliche Teilgraphen.
\end{Bew}

\end{document}


\appendix
\renewcommand{\indexname}{Stichwortverzeichnis}
\addcontentsline{toc}{chapter}{Stichwortverzeichnis}
\printindex

\end{document}
