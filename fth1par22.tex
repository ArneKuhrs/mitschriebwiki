\documentclass{article}
\newcounter{chapter}
\setcounter{chapter}{22}
\usepackage{ana}
\def\gdw{\equizu}
\def\Arg{\text{Arg}}
\def\MdD{\mathbb{D}}
\def\Log{\text{Log}}
\def\Tr{\text{Tr}}
\def\Ext{\text{Ext}(\ensuremath{\gamma)}}
\def\abnC{\ensuremath{[a,b]\to\MdC}}
\def\wegint{\ensuremath{\int\limits_\gamma}}
\def\iint{\ensuremath{\int\limits}}
\def\ie{\rm i}

\title{Cauchyscher Integralsatz (Homologieversionen)}
\author{Christian Schulz} % Wer nennenswerte �nderungen macht, schreibt euch bei \author dazu
%�22 Cauchyscher Integralsatz (Homologieversionen)
\begin{document}
\maketitle 
In diesem Paragraphen sei $G \subseteq \MdC$ stets ein \underline{Gebiet}.
\begin{definition}
Sei $\gamma$ ein geschlossener Weg in $\MdC$. 
\begin{liste}
\item Int($\gamma$) $:= \{ z \in \MdC \backslash \text{Tr}(\gamma): n(\gamma,z)
\neq 0\}$ (''Inneres'' von $\gamma$)\\
Ext($\gamma$) $:= \{ z \in \MdC \backslash \text{Tr}(\gamma): n(\gamma,z)
= 0\}$ (''�u�eres'' von $\gamma$)
\item Sei Tr($\gamma$) $ \subseteq G$. $\gamma$ hei�t in $G$
\begriff{nullhomolog} $:\equizu$ $n(\gamma,z) = 0$ $\forall z \in \MdC \backslash G$ \\
($\equizu$ Int$(\gamma) \subseteq G$)
\end{liste}
\end{definition}

\begin{beispiele}
\begin{liste}
\item Jeder geschlossene Weg in $\MdC$ ist in $\MdC$ nullhomolog.
\item $G := \MdC \backslash \{0\}$, $\gamma(t) = e^{\ie t}$ $(t\in [0,2\pi])$,
$n(\gamma,0) = 1 \neq 0;$ $\gamma$ ist in $G$ nicht nullhomolog.
\end{liste}
\end{beispiele}

\begin{satz}
Sei $\gamma$ ein geschlossener Weg mit Tr($\gamma$) $\subseteq G$.
\begin{liste}
\item Ist $\gamma$ nullhomotop in $G$ $\Rightarrow$ $\gamma$ ist nullhomolog in
$G$. 
\item Ist $G$ einfach zusammenh�ngend, so ist $\gamma$ in $G$ nullhomolog.
\end{liste}
\end{satz}
\begin{beweis}
\begin{liste}
\item Sei $z_0 \in \MdC \backslash G$. Dann ist $f(z) = \frac{1}{z-z_0}$
holomorph auf $G$. \\ $\stackrel{\text{21.2}}{\Rightarrow} \underbrace{\wegint
f(z) dz}_{= 2 \pi \ie \text{ } n(\gamma, z_0)}=0$ $\Rightarrow$ $n(\gamma, z_0) =
0$
\item folgt aus (1)
\end{liste}
\end{beweis}
\begin{satz}
Sei $f \in H(G)$ und $\gamma$ sei ein geschlossener Weg mit Tr($\gamma$)
$\subseteq G$ . \\
$\varphi: G \times G \to \MdC$ sei definiert durch: \\ \\
\centerline{$\varphi(w,z) := $
$\begin{cases}
\frac{f(w)-f(z)}{w-z}& , w \neq z \\
f'(z)  & , w = z
\end{cases}$} \\
\begin{liste}
\item $\varphi$ ist stetig.
\item F�r $z \in G$ (fest) hat $w \mapsto \varphi(w, z)$ in $z$ eine hebbare
Singularit�t; $w \mapsto \varphi(w, z)$ ist also holomorph auf $G$ \\
F�r $w \in G$ (fest) hat $z \mapsto \varphi(w, z)$ in $w$ eine hebbare
Singularit�t; $z \mapsto \varphi(w, z)$ ist also holomorph auf $G$
\item $h(z) := \wegint \varphi(w,z) dw$ $(z\in G)$. Ist $\gamma$ nullhomolog in
$G$, so ist $h \equiv 0$ auf G.
\end{liste} 
\end{satz}
\begin{beweis}
\begin{liste}
\item 11.9
\item 13.1
\item (A) Es ist $h \in C(G)$. Sei $z_0 \in G$ und $(z_n)$ eine Folge in G mit
$z_n \to z_0$. $g_n(w) := \varphi(w, z_n)$, $g(w) := \varphi(w, z_0)$ $(w \in
G)$. Sei $\Gamma$ der st�ckweise glatte Ersatzweg f�r $\gamma$ (wie in �20). \\
�bung: $(g_n)$ konvergiert auf $\Gamma$ gleichm��ig gegen $g$. \\
$\stackrel{\text{8.4}}{\Rightarrow} \int_{\Gamma}$ $g_n(w) dw$ $\to
\int_{\Gamma}$ $g(w) dw = \int_{\Gamma}$ $\varphi(w,z_0) dw = \wegint
\varphi(w,z_0) dw = h(z_0)$ \\
Also: $h(z_n) \to h(z_0)$ \\ \\
(B) Es ist $h \in H(G)$. Sei $\Delta \subseteq G$ ein Dreieck. Wegen 9.7 gen�gt
es zu zeigen: $\int_{\partial \Delta} h(z) dz = 0$ \\ 
9.1 und (2) $\Rightarrow$ $\int_{\partial \Delta} \varphi(w,z) dz = 0$ $\forall
w \in G$ \\ $\Rightarrow$ $\int_{\partial \Delta} h(z) dz = \int_{\partial
\Delta} (\wegint \varphi(w,z) dw) dz$ $\stackrel{\text{Fubini}}{=}  \wegint ( \underbrace{\int_{\partial
\Delta} \varphi(w,z) dz}_{= 0}) dw = 0$ \\ \\
(C) $\MdC = \underbrace{\text{Int}(\gamma)}_{\subseteq G} \cup \text{Ext}(\gamma)
\cup \underbrace{\text{Tr}(\gamma)}_{\subseteq G}$ $= G \cup \text{Ext}(\gamma)$
\\ Sei $z_0 \in \text{Ext}(\gamma)$. Sei $C$ die Komponente von $\MdC \backslash
\Tr(\gamma)$: $z_0 \in C$. \\ $\stackrel{16.2}{\Rightarrow}$ $n(\gamma, z) =
n(\gamma, z_0) = 0$ $\forall z \in C$ \\ $\Rightarrow$ $C \subseteq
\text{Ext}(\gamma)$. $\stackrel{16.1/2}{\Rightarrow}$ $C$ ist offen.
$\Rightarrow$ $\exists \delta > 0 :$ $U_\delta(z_0) \subseteq C \subseteq
\text{Ext}(\gamma)$. \\Also ist $\text{Ext}(\gamma)$ offen. [Analog: $\text{Int}(\gamma)$ offen]\\
$g(z) := \wegint \frac{f(w)}{w-z} dw$ $(z \not \in \Tr(\gamma))$
$\stackrel{9.5}{\Rightarrow}$ $g \in H(\MdC \backslash \Tr(\gamma))$,
insbesondere gilt $g \in H(\text{Ext}(\gamma))$. \\
Sei $z \in G \cap \Ext$: $h(z) = \wegint \frac{f(w)-f(z)}{w-z} dw = \wegint
\frac{f(w)}{w-z} dw -  f(z) \wegint \frac{1}{w-z} dw$ $= g(z) - f(z) 2 \pi \ie$
$\underbrace{n(\gamma, z)}_{=0} = g(z)$. Also: $ h = g$ auf $G \cap \Ext$. Dann
ist \\
$F(z) = \begin{cases}
  h(z) &, z \in G \\
        g(z)&, z \in \Ext
	\end{cases}
$
eine ganze Funktion.\\ �bung: $F(z) \to 0$ $(|z| \to \infty)$. 10.2 $\Rightarrow$
$F \equiv 0$ $\Rightarrow$ $h \equiv 0$
\end{liste}
\end{beweis}

\begin{satz} [Allgemeine Cauchysche Integralformel]
Sei $\gamma$ ein geschlossener Weg mit $\Tr(\gamma) \subseteq G$  und $\gamma$
sei nullhomolog in $G$. Dann:
\\  
\centerline{ $n(\gamma,z) f(z) = \frac{1}{2 \pi \ie} \wegint \frac{f(w)}{w-z}
dw$ $\forall f \in H(G)$ $\forall z \in G \backslash \Tr(\gamma)$ }
\end{satz}
\begin{beweis}
Sei $f \in H(G)$ und $z \in G \backslash \Tr(\gamma)$.
$\stackrel{22.2(3)}{\Rightarrow} $ $0 = \frac{1}{2 \pi \ie} \wegint
\frac{f(w)-f(z)}{w-z} dw = \frac{1}{2 \pi \ie} \wegint \frac{f(w)}{w-z} dw -
f(z)  \underbrace{\frac{1}{2 \pi \ie} \wegint \frac{dw}{w-z}}_{=n(\gamma,z)}$
\end{beweis}
\begin{satz} [CIS, Homolgieversion I]
Sei $\gamma$ ein gechlossener Weg mit $\Tr(\gamma) \subseteq G$.\\ Dann: \\ 
\centerline{$\wegint f(z) dz = 0$ $\forall f \in H(G)$ $\equizu$ $\gamma$ ist in
$G$ nullhomolog}
\end{satz}
\begin{beweis}
$''\Rightarrow''$: Sei $z_0 \in \MdC \backslash G$; $f(z) := \frac{1}{2 \pi \ie}
\frac{1}{z-z_0} $ $\Rightarrow$ $f \in H(G)$
$\stackrel{\text{Vor.}}{\Rightarrow} \underbrace{\wegint f(z) dz}_{= n(\gamma, z_0
)} = 0$ \\
$''\Leftarrow''$: Sei $f \in H(G)$ und $z_0 \in G \backslash \Tr(\gamma)$;
$g(z) = (z-z_0)f(z)$; $g \in H(G)$. \\ Wende 22.3 auf $g$ an : \\
\centerline{$n(\gamma,z_0) \underbrace{g(z_0)}_{=0} = \frac{1}{2 \pi \ie} \wegint
\underbrace{\frac{g(w)}{w-z_0}}_{=f(w)} dw$ $\Rightarrow \wegint f(w) dw = 0.$} 
\end{beweis}
\begin{satz}
$G$ ist einfach zusammenh�ngend $\equizu$ jeder geschlossene Weg $\gamma$ mit
$\Tr(\gamma) \subseteq G$ ist in $G$ nullhomolog.
\end{satz}
\begin{beweis}
$''\Rightarrow''$ 22.1(2) \\
$''\Leftarrow''$ Sei $\gamma$ ein geschlossener Weg mit $\Tr(\gamma) \subseteq
G$ und $f \in H(G)$ \\
Vorraussetzungen $\Rightarrow$ $\gamma$ ist in $G$ nullhomolg. 22.4
$\Rightarrow$ $\wegint f(z) dz = 0$. 21.5 $\Rightarrow$ $G$ ist einfach zusammenh�ngend.
\end{beweis}
\begin{definition}
Seien $\gamma_1$ und $\gamma_2$ geschlossene Wege mit $\Tr(\gamma_1),
\Tr(\gamma_2) \subseteq G.$ $\gamma_1, \gamma_2$ hei�en \begriff{in $G$ homolog}
$:\equizu$ $n(\gamma_1,z) = n(\gamma_2, z)$ $\forall z \in \MdC \backslash G.$
\end{definition}
\begin{satz} [CIS, Homologieversion II]
$\gamma_1, \gamma_2$ seien wie in obiger Definition und in $G$ homolog. \\
Dann: \\ 
\centerline{$\int_{\gamma_1} f(z) dz = \int_{\gamma_2} f(z) dz$ $\forall f \in
H(G)$}
\end{satz}
\begin{beweis}
Sei $f \in H(G)$ und $z_j :=$ Anfangspunkt von $\gamma_j$ ($j=1,2$). \\
$\stackrel{\text{3.4}}{\Rightarrow}$ $\exists$ Weg $\gamma: [0,1] \to \MdC$:
$\Tr(\gamma) \subseteq G$, $\gamma(0) = z_1$, $\gamma(1) = z_2$ \\
$\Gamma := \gamma_1 \oplus \gamma \oplus \gamma_2^- \oplus \gamma^-$.
$\Gamma$ ist ein geschlossener Weg mit $\Tr(\gamma) \subseteq G$ \\
Sei $z_0 \in \MdC \backslash G$: $n(\Gamma, z_0) = n(\gamma_1, z_0)+ n(\gamma,
z_0) - n(\gamma_2,z_0) - n(\gamma,z_0) = 0$ \\
D.h.: $\Gamma$ ist in $G$ nullhomolog. 22.4 $\Rightarrow$ $0 = \int_\Gamma f(z)
dz = \int_{\gamma_1}+\int_{\gamma}-\int_{\gamma_2}-\int_{\gamma} = \int_{\gamma_1}-\int_{\gamma_2}$
\end{beweis}
\end{document} 
