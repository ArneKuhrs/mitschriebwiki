\chapter{Algebraische Körpererweiterungen}

\section{Grundbegriffe}

\begin{Def}
Sei $L$ ein Körper, $K \subset L$
Teilkörper.
\begin{enum}
\item Dann heißt $L$ Körpererweiterung von $K$. Schreibweise: $L/K$
Körpererweiterung.

\item $[L:K] =$ dim$_K$ $L$ heißt \emp{Grad} von $L$ über $K$

\item $L/K$ heißt \emp{endlich}, wenn $[L:K] < \infty$

\item $\alpha \in L$ heißt \emp{algebraisch} über $K$, wenn es ein
$0\neq f \in K[X]$ gibt mit $f(\alpha) = 0$

\item $\alpha \in L$ heißt \emp{transzendent} über $K$, wenn $\alpha$
nicht algebraisch über $K$ ist.

\item $L/K$ heißt \emp{algebraische Körpererweiterung}, wenn jedes
$\alpha \in L$ algebraisch über $K$ ist.
\end{enum}

\bsp{
\begin{enumerate}
\item[(1)] Für $a \in \mathbb{Q}$ und $n \geq 2$ ist $\sqrt[n]{a}$
algebraisch über $\mathbb{Q}$, da Nullstelle von $X^n - a$
\newline Summe und Produkt von solchen Wurzeln sind auch algebraisch
über $\mathbb{Q}$
\newline z.B.: $\sqrt{2} + \sqrt{3}$ ist Nullstelle von $X^4 - 10 X^2 + 1$

\item[(2)] Sei $L = K(X) =$Quot$(K[X])$. Dann ist $X$ transzendent
über $K$. Das gleiche gilt für jedes $f \in K(X) \setminus K$

\item[(3)] In $\mathbb{R}$ gibt es sehr viele über $\mathbb{Q}$
transzendente Elemente. Da $\mathbb{Q}$ abzählbar ist, ist auch
$\mathbb{Q}[X]$ abzählbar, da jedes $f \in \mathbb{Q}[X]$ endlich viele
Nullstellen hat. Das heißt, es gibt nur abzählbar viele Elemente in
$\mathbb{R}$, die algebraisch über $\mathbb{Q}$ sind. $\mathbb{R}$
ist aber nicht abzählbar.
\end{enumerate} }
\end{Def}

\begin{DefBem}
Sei $L/K$ Körpererweiterung,
$\alpha \in L$, $\\\varphi_\alpha: K[X] \ra L,\;f\mapsto f(\alpha)$
Einsetzunghomomorphismus.
\begin{enum}
\item Kern$(\varphi_\alpha)$ ist Primideal in $K[X]$
\newline
\sbew{0.9}{Kern$(\varphi_\alpha)$ ist Ideal, da $\varphi_\alpha$
Homomorphismus ist. Seien nun $f,g \in K[X]$ mit $f g \in$
Kern$(\varphi_\alpha) \Ra
(fg)(\alpha) = f(\alpha)g(\alpha) = 0 \overset{L \scriptsize\mbox{ Körper}}{\Ra} f(\alpha)
= 0$ oder $g(\alpha) =0$}

\item $\alpha$ algebraisch $\lra$ Kern$(\varphi_\alpha) \neq \{0\}$

\item Ist $\alpha$ algebraisch über $K$, so gibt es ein eindeutig
bestimmtes, irreduzibles und normiertes Polynom $f_\alpha \in K[X]$
mit $f_\alpha(\alpha) = 0$ und Kern$(\varphi_\alpha) = (f_\alpha)$.
\newline $f_\alpha$ heißt \emp{Minimalpolynom} von $\alpha$.
\newline
\sbew{0.9}{$K[X]$ ist Hauptidealring $\Ra \exists
\widetilde{f_\alpha}$ mit Kern$(\varphi_\alpha) =
(\widetilde{f_\alpha})$. Wegen (a) ist $\widetilde{f_\alpha}$
irreduzibel, eindeutig bis auf Einheit in $K[X]$, also ein Element
aus $K^x \Ra \exists! \lambda \in K^x$, so daß $\lambda
\widetilde{f_\alpha} = f_\alpha$ normiert ist.}

\item $K[\alpha] \defeqr$ Bild$(\varphi_\alpha) = \{f(\alpha):\;f
\in K[X]\} \subset L$ ist der kleinste Unterring von $L$, der $K$
und $\alpha$ enthält.

\item $\alpha$ ist transzendent $\lra$ $K[\alpha] \cong K[X]$
\newline
\sbew{0.9}{
    $\alpha$ ist transzendent $\Ra$ Kern$(\varphi_\alpha) = \{ 0 \} \Ra \varphi_\alpha$
    injektiv }

\item Ist $\alpha$ algebraisch über $K$, so ist $K[\alpha]$ ein
Körper und $[K[\alpha]:K] =$ deg$(f_\alpha)$

\sbew{0.9}{Nach Homomorphiesatz ist $K[\alpha] \cong
K[X]/$Kern$(\varphi_\alpha)$.
\newline Kern$(\varphi_\alpha)$ ist maximales Ideal, da Primideal
$\neq (0)$ in $K[X]$ (siehe Bew. Satz \ref{Satz 9}, Beh.2) $\Ra K[\alpha]$ ist
Körper.
\newline $f_\alpha(\alpha) = 0$, also $\alpha^n + c_{n-1} \alpha^{n-1} + \dots
+ c_1 \alpha + c_0 = 0$ mit $c_i \in K,\; c_0 \neq 0$ (da $f_\alpha$
irreduzibel), $\alpha(\alpha^{n-1} + \dots + c_1) = -c_0$. Ebenso:
$1,\alpha,\alpha^2,\dots,\alpha^{n-1}$ ist $K$-Basis von $K[\alpha]$.
}
\end{enum}
\end{DefBem}

\begin{Def}
Sei $L/K$ Körpererweiterung.
\begin{enum}
\item Für $A \subset L$ sei $K(A)$ der kleinste Teilkörper von $L$,
der $A$ und $K$ umfaßt; $K(A)$ heißt der \empind{von $\mathbf{A}$
erzeugte Teilkörper}{von A erzeugte Teilkörper} von $L$. Es ist
\[ K(A) = \left\{
\frac{f(\alpha_1,\dots,\alpha_n)}{g(\alpha_1,\dots,\alpha_n)} : n
\geq 1, \alpha_i \in A, f, g \in K[X_1,\dots,X_n], g \neq 0 \right\} \]

\item $L/K$ heißt \emp{einfach}, wenn es $\alpha \in L$ gibt mit $L
= K(\alpha)$

\item $L/K$ heißt \emp{endlich erzeugt}, wenn es eine endliche Menge
$\{\alpha_1,\dots,\alpha_n\} \subset L$ gibt mit $L =
K(\alpha_1,\dots,\alpha_n)$
\end{enum}
\end{Def}

\begin{Bem}
\label{3.4}
Für eine Körpererweiterung $L/K$ sind
äquivalent:
\begin{enumerate}
\item[(i)] $L/K$ ist endlich.
\item[(ii)] $L/K$ ist endlich erzeugt und algebraisch.
\item[(iii)] $L$ wird von endlich vielen über $K$ algebraischen
Elementen erzeugt.
\end{enumerate}

\sbew{1.0}{
\begin{description}
\item[(i) $\Ra$ (ii)] Sei $[L:K] = n$, $\alpha \in L \Ra 1, \alpha,
\alpha^2, \dots, \alpha^n$ sind $K$-linear abhängig $\Ra \exists c_i
\in K$, nicht alle $0$, mit $\ds \sum_{i=0}^n c_i \alpha^i = 0 \Ra
f(\alpha) = 0$ für $\ds f= \sum_{i=0}^n c_i X^i \in K[X]$
\item[(ii) $\Ra$ (iii)] $\chk$
\item[(iii) $\Ra$ (i)] Induktion über die Anzahl $n$ der Erzeuger:
\newline $n=1$: 3.2(f), \\$n>1$: auch 3.2(f)
\end{description}
}
\end{Bem}

\begin{Bem}
\label{3.5}
Seien $K\subset L \subset M$ Körper.
\begin{enum}
\item Sind $M/L$ und $L/K$ algebraisch, so auch $M/K$
\newline
\sbew{0.9}{ Sei $\alpha \in M$, $f_\alpha = \displaystyle \sum_{i=0}^n c_i X^i
\in L[X]$ mit $f_\alpha(\alpha) = 0$. Dann ist $\alpha$ algebraisch über
$K(c_0,\dots,c_n) \defeql L' \subset L, L'$ ist endlich erzeugt über 
$K \overset{\ref{3.4}}{\Ra} L'/K$ endlich.
Außerdem ist $L'(\alpha)/L'$ endlich. $\overset{(b)}{\Ra} L'(\alpha)/K$
endlich $\Ra \alpha$ algebraisch über $K$.}

\item Sind $M/L$ und $L/K$ endlich, so auch $M/K$ und es gilt $[M:K]
= [M:L]\cd[L:K]$

\sbew{0.9}{Sei $b_1,\dots,b_m$ $K$-Basis von $L$ und
$e_1,\dots,e_n$ $L$-Basis von $M \Ra B = \{e_i
b_j:\;i=1,\dots,n;j=1,\dots,m\}$ ist $K$-Basis von $M$.
\newline \textbf{denn}: $B$ erzeugt $M$: Sei $\alpha \in M$, $\ds
\alpha = \sum_{i=1}^n \lambda_i e_i$ mit $\lambda_i \in L$, $\ds
\lambda_i = \sum_{j=1}^m \mu_j b_j$ einsetzen $\Ra$ Behauptung.

$B$ linear unabhängig:

Ist $\sum \mu_{ij} e_i b_j = 0$, so ist für jedes feste $i$ :
$\displaystyle \sum_{j=1}^n \mu_{ij} b_j = 0$, da $e_i$ über $L$ linear unabhängig
sind. Da die $b_j$ linear unabhängig sind, sind die $\mu_{ij} = 0$
\newline \textbf{Notation}: $L/K$ Körpererweiterung, $\alpha \in L$,
$K[\alpha] = $ Bild$(\varphi_\alpha) = \dots\\$ $K(\alpha)=$
Quot$(K[\alpha]) = K[\alpha]$, falls $\alpha$ algebraisch.}
\end{enum}

\bsp{$\cos \frac{2\pi}{n}$ ist für jedes $n \in \mathbb{Z} \setminus
\{0\}$ algebraisch über $\mathbb{Q}$.

\textbf{denn}: \[\cos \frac{2\pi}{n} = \Re\left(e^{\frac{2\pi
i}{n}}\right) = \frac{1}{2}\left(e^{\frac{2\pi i}{n}} +
\overline{e^{\frac{2\pi i}{n}}}\right) = \frac{1}{2}\left(e^{\frac{2\pi i}{n}}
+ e^{-\frac{2\pi i}{n}}\right)\]

$e^{\frac{2\pi\imath}{n}}$ ist Nullstelle von $X^n - 1$, also
algebraisch (über $\mathbb{Q}$) $\Ra K =
\mathbb{Q}\left(e^{\frac{2\pi\imath}{n}}\right)$ ist endliche
Körpererweiterung von $\mathbb{Q}$, $\cos \frac{2\pi}{n} \in K
\overset{3.4(i)\Ra(ii)}{\Ra} \cos \frac{2\pi}{n}$ ist algebraisch.

$\mathbb{Q} \subset \mathbb{Q}\left(\cos \frac{2\pi}{n}\right)
\subsetneq K\;(n\geq3)$}
\end{Bem}