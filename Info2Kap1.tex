\documentclass[a4paper]{scrartcl}
\newcounter{chapter}
\setcounter{chapter}{1}
\usepackage{info}
\usepackage{makeidx}
\usepackage{clrscode}
\usepackage[all]{xy}

\title{Prädikatenlogik}
\author{Joachim Breitner, Felix Brandt, Lars Volker}
% Wer nennenswerte Änderungen macht, schreibt sich bei \author dazu

%\setlength{\parskip}{0.5cm}
%\setlength{\parindent}{0cm}
\begin{document}

\maketitle

\section{Syntax prädikatenlogischer Formeln}

\subsection{Grundsymbole}

\lectureof{11.04.2005}Die Syntax definiert eine Menge von Grundsymbolen. Diese lassen sich weiter aufteilen:

\subsubsection{logische Symbole}
\begin{tabular}{lll}
$\vee$ & Disjunktion & (\glqq oder\grqq) \\
$\wedge$ & Konjunktion & (\glqq und\grqq) \\
$\neg$ & Negation & (\glqq nicht\grqq)  \\
$\exists$ & Existenz & (\glqq es existiert\grqq) \\
$\forall$ & Allquantor & (\glqq für all\grqq) \\
$\dann$ & Implikation & (\glqq dann\grqq) \\
$\wenn$ & (umgekehrte) Implikation & (\glqq wenn\grqq) \\
$\equi$ & Äquivalenz & (\glqq genau dann wenn \grqq) \\
$( )$ & Hilfssymbol
\end{tabular}

\subsubsection{frei wählbare, nicht-logische Symbole}
\begin{itemize}
\item Symbole, die für ein Individuum (Objekt) stehen
  \begin{itemize}
  \item Konstanten, z. B. $a, b, c,$ John, 0,1,2, \ldots
  \item Variablen, z. B. $x, y, z,$ Name, Organisation, \ldots
 \end{itemize}
\item Funktionssymbole, z. B. $f, g, h,$ mother\_of, $\sin$, +, -, \ldots
\item Prädikatensymbole, z. B. $p, q, r,$ father, $=, <, \le$, \ldots
\end{itemize}

Die freie Wählbarkeit wird durch Konventionen eingeschränkt, zum Beispiel in \produkt{Prolog}: Variablen beginnen mit Großbuchstaben, Konstanten sowie Funktions- und Prädikatensymbole mit Kleinbuchstaben.

\subsection{Terme}
\begin{itemize}
\item Konstanten
\item Variablen
\item strukturierte Terme der Form $\varphi(\tau_1,\tau_2,\ldots,\tau_n)$, wobei $\varphi$ ein $n$-stelliges Funktionssymbol und $\tau_1,\tau_2,\ldots,\tau_n$ Terme sind.
\end{itemize}

Konvention nach Schöning: $f$ ist ein Funktionssymbol mit Stelligkeit $k$ und $\tau_i$ ($i=1\ldots k$) sind Terme, dann ist auch $f^k(\tau_1,\tau_2,\ldots,\tau_k)$ ein Term.

\subsection{Atome}
Auch atomare Formeln oder Primformeln genannt. Atome stehen für die Beziehung zwischen Individuen. Sie haben die Form $\pi(\tau_1,\tau_2,\ldots,\tau_n)$ wobei $\pi$ ein $n$-stelliges Prädikatensymbol ist und $\tau_i$ Terme sind.

\subsection{Formeln}
Formeln erlauben eine logische Verknüpfung von Atomen und Termen. Im folgenden sind $\Phi$ und $\Psi$ Formeln und $x$ eine Variable.

\begin{itemize}
\item Atome
\item $(\Phi \vee \Psi)$
\item $(\Phi \wedge \Psi)$
\item $(\neg \Phi)$
\item $(\exists x \Phi)$ (\glqq Es gibt ein $x$, so dass $\Phi$ gilt\grqq)
\item $(\forall x \Phi)$ (\glqq Für alle $x$ gilt $\Phi$\grqq)
\end{itemize}

Die folgenden Formeln sind nur Abkürzungen:
\begin{itemize}
\item $(\Phi \dann \Psi)$ Abkürzung für $(\neg \Phi \vee \Psi)$.
 (\glqq $\Phi$ impliziert $\Psi$\grqq), (\glqq wenn $\Phi$, dann $\Psi$\grqq).
\item $(\Phi \wenn \Psi)$ Abkürzung für $(\Phi \vee \neg \Psi)$. (\glqq $\Phi$ wenn $\Psi$\grqq), (\glqq $\Phi$ falls $\Psi$\grqq) 
\item $(\Phi \equi \Psi)$ Abkürzung für $((\Phi \wenn \Psi)\wedge(\Phi\dann\Psi))$. (\glqq $\Phi$ genau, dann wenn $\Psi$\grqq)
\end{itemize}

Ferner gelten folgende Konventionen:
\begin{itemize}
\item $\exists$ und $\forall$ binden stärker als $\neg$
\item $\neg$ bindet stärker als $\wedge$
\item $\wedge$ bindet stärker als $\vee$
\item $\vee$ bindet stärker als $\wenn$, $\dann$, $\equi$
\item \glqq unnötige\grqq Klammern werden weggelassen
\end{itemize}

Dies ist die Syntax der Prädikatenlogik \emph{erster Stufe} (First Level Order). Erste Stufe bedeutet, dass Variablen nur für Individuen stehen können, nicht aber für Funktionen oder Prädikate.

\paragraph{Beispiele} für Formeln
\begin{enumerate}
\item  loves(romeo,juliet)
\item $\forall x$ ancestor(adam,$x$)
\item $\forall x$ (person($x$) $\dann$ ancestor(adam,$x$))
\item $\forall x$ (is\_mother($x$) $\dann$ $\exists y$ (is\_child($y$) $\wedge$ mother($x$,$y$)))
\item $\forall$ sub $\forall$ super (subset(sub,super) $\wenn$ $\forall x$(member($x$,sub) $\dann$ member($x$,super)))
\end{enumerate}

\paragraph{Definition} einige weitere Begriffe.
\begin{itemize}
\item Die Menge der Variablen eines Terms $\tau$, vars($\tau$) kann induktiv wie folgt definiert werden:
 \begin{itemize}
 \item Für eine Konstante $c$ gilt: vars($c$):=$\{\}$
 \item Für eine Variable $x$ gilt: vars($x$):=$\{x\}$
 \item Für einen Term $\varphi(\tau_1, \tau_2, \ldots, \tau_n)$ gilt: \\vars($\varphi(\tau_1, \tau_2, \ldots, \tau_n)$) := vars($\tau_1$) $\cup$ vars($\tau_2$) $\cup \ldots \cup$ vars($\tau_n$)
 \end{itemize}
\item Eine Variable, die nicht durch einen Quantor gebunden ist, heißt \begriff{frei} (free). Die Menge der freien Variablen einer Formel $\Phi$, free($\Phi$), ist induktiv wie folgt definiert:
 \begin{itemize}
 \item Für ein Atom $\pi(\tau_1,\tau_2,\ldots,\tau_n)$ gilt: \\
free($\pi(\tau_1,\tau_2,\ldots,\tau_n)$) := vars($\tau_1$) $\cup$ vars($\tau_2$) $\cup \ldots \cup$ vars($\tau_n$)
 \item free($\neg\Phi$) := free($\Phi$)
 \item free($\Phi\vee\Psi$) := free($\Phi\wedge\Psi$) := free($\Phi$) $\cup$ free($\Psi$)
 \item free($\exists x\Phi$) := free($\forall x\Phi$) := free($\Phi$)$\backslash \{x\}$
 \end{itemize}
\item In $\forall x \Phi$ bzw. $\exists x\Phi$ heißt die Variable $x$ durch den jeweiligen Quantor gebunden (bound).
\item Eine Formel mit freien Variablen heißt offen (open).
\item Eine Formel ohne freie Variablen heißt geschlossen (closed).
\end{itemize}


\section{Semantik prädikatenlogischer Formeln }

\subsection{Definition}
(exzerpt aus ``Logik für Informatiker'', Seite 52 ff)
Eine Struktur ist ein Paar $A(U_A,I_A)$, wobei $U_A$ eine beliebige nicht-leere Menge ist, das \begriff{Universum} (auch Grundmenge oder Grundbereich) genannt wird. Ferner ist $I_A$ eine Abbildung, die
\begin{itemize}
\item jedem $k$-stelligen Prädikatensymbol $P$ ein $k$-stelliges Prädikat über $U_A$ zuordnet,
\item jedem $k$-stelligen Funktionssymbol $F$ eine $k$-stellige Funktion über $U_A$ zuordnet sowie
\item jeder Variablen $x$ ein Element der Grundmenge $U_A$ zuordnet.
\end{itemize}
Wir schreiben abkürzend $P^A$ statt $I_A(P)$, $F_A$ statt $I_A(F)$ und $x^A$ statt $I_A(x)$.

\paragraph{Das Herbrand-Universum} ist die Menge aller aus den Objekt- und Funktionssymbolen einer Formelmenge $\Delta$ konstruierbaren Objekte.

Sei $F$ eine Formel und $A=(U_A,I_A)$ eine Struktur. $A$ heißt zu $F$ passend, falls $I_A$ für alle in $F$ vorkommenden Symbole und freien Variablen definiert ist.


\paragraph{Beispiel:} $$F = \forall x P(x,f(x)) \wedge Q(g(a,z))$$ ist eine Formel. Eine zu $F$ passende Struktur ist dann Beispielsweise:
\begin{eqnarray*}
U_A &=& \{1,2,3,\ldots\} \\
I_A(P) &=& \{ (m,n)\ |\ m,n \in U_A \text{ und } m<n\} \\
I_A(Q) &=& \{ n\in U_A\ |\ n \text{ ist Primzahl }\} \\
I_A(f) &=& f^A(n) = n+1 \\
I_A(g) &=& g^A(m,n) = m+n \\
I_A(a) &=& 2 \\
I_A(z) &=& 3
\end{eqnarray*}

Beachte: $x$ kommt gebunden vor.

\subsection{Interpretation}

\lectureof{13.04.2005}Die Interpretation einer prädikatenlogischen Formel ergibt sich, wenn man jeder verwendeten Konstanten 
ein bekanntes Objekt eines Diskursbereichs $D$ zuordnet und jedes verwendete Prädikatensymbol als Beziehungstyp zwischen 
zwei oder mehreren Objekten ansieht. Die Abbildung der Formel auf $D$ wird als beabsichtigte Interpretation 
(\begriff{intended interpretation}) bezeichnet.

\def\J{\mathbb{J}}
\def\R{\mathbb{R}}

\paragraph{Beispiel:} Einfacher Diskursbereich
$$D := \{\J,\R\}$$

Eine Interpretation $I$ einer prädikatenlogischen Sprache ist ein Paar $\langle D,I\rangle$ wobei $I$ eine Abbildung der
nicht-logischen Symbole auf (nicht-leeres) $D$ ist, wobei $I$

\begin{itemize}
\item jeder Konstanten der Sprache ein Element von $D$ zuordnet
\item jedem $n$-stelligen Funktionssymbol eine Abbildung $D^n \mapsto D$ zuordnet
\item jedem $n$-stelligen Prädikatensymbol eine Abbildung $D^n \mapsto \{F,T\}$ zuordnet
\end{itemize}

Für Grundterme der Form $\varphi (\tau_1, \ldots, \tau_n)$ wird die Interpretation wie folgt berechnet:
$$I (\varphi (\tau_1, \ldots, \tau_n) := I(\varphi)(I(\tau_1) \ldots I(\tau_n))$$
Jeder Term $\tau_i$ wird demzufolge durch Anwendung der Abbildung $I$ auf ein Element $I(\tau_i) \in D$ abgebildet.
Auch $\varphi$ wird durch $I$ abgebildet auf $I(\varphi)$.

Für Grundatome $\pi(\tau_1, \ldots, \tau_n)$ wird $I$ wie folgt berechnet:
$$I(\pi(\tau_1, \ldots, \tau_n)) := I(\pi)(I(\tau_1, \ldots, \tau_n))$$

\paragraph{Beispiel:} Interpretation von Elementen der Prädikatenlogik

\begin{tabular}{@{}ll}
Konstanten: & $I(\text{juliet}) = \J$                            \\
            & $I(\text{romeo}) = \R$                             \\
Funktionen: & $I(\text{girlfriend\_of}) = \{\R \mapsto \J\}$       \\
Prädikate:  & $I(\text{woman}) = \{ \J \mapsto T, \R \mapsto F\}$ \\
            & $I(\text{loves}) = \{\langle\J, \J\rangle \mapsto F, \langle\J,\R\rangle \mapsto T, \langle\R,\J\rangle \mapsto T, \langle\R,\R\rangle \mapsto F \}$ \\\\
Grundterme:&
$I(\text{girlfriend\_of}(\text{romeo})) = \{\R \mapsto \J\} (\R)=\J$
\\
Grundatome:&
$I(\text{woman}(\text{juliet})) = \{\J \mapsto T, \R \mapsto F\} (\J) = T$\\
&$I(\text{woman}(\text{romeo})) = \{\J \mapsto T, \R \mapsto F\} (\R) = F$
\end{tabular}%\bigskip

\subsection{Definition (Fortsetzung)}
(Exzerpt aus "`Logik für Informatiker"', Schöning , Seite 53 ff.)\\
Sei $F$ eine Formel und $A$ eine zu $F$ passende Struktur. Für jeden Term $t$, den man aus den Bestandteilen von $F$ bilden kann, definieren wir nun den Wert von $t$ in der Struktur $A$, den wir mit $A(t)$ bezeichnen.

\begin{enumerate}
\item Falls $t$ eine Variable ist (also $t=x$), so ist $A(t)=x^A$ .
\item Falls $t$ die Form $t=f(t_1, \ldots, t_k)$ hat, wobei $t_1, \ldots, t_k$ Terme sind und $f$ ein k-stelliges Funktionssymbol ist, so ist $A(t)=f^A(A(t_1), \ldots, A(t_k))$
\end{enumerate}

Auf analoge Weise definieren wir den (Wahrheits-) Wert der Formel $F$ wobei wir ebenfalls die Bezeichnung $A(F)$ verwenden.

\begin{enumerate}
\item Falls $F$ die Form $F=P(t_1, \ldots, t_k)$ mit Termen $t_1, \ldots, t_k$, so ist \[A(F)=
	\begin{cases}
	1, & \text{falls } ((A(t_1), \ldots, A(t_k)) \in P^A) \\
	0, & \text{sonst }
	\end{cases}\]
\item Falls "`$F = \neg G$"' hat, so ist \[ A(F)=
	\begin{cases}
	1, & \text{falls $A(G)=0$} \\
	0, & \text{sonst}
	\end{cases}\]
\item Falls "`$F=(G \wedge H)$"' so ist \[ A(F)=
	\begin{cases}
	1, & \text{falls $A(G)=1$ und $A(H)=1$} \\
	0, & \text{sonst}
	\end{cases}\]
\item Falls "`$F=(G \vee H)$"' so ist \[ A(F)=
	\begin{cases}
	1, & \text{falls $A(G)=1$ oder $A(H)=1$} \\
	0, & \text{sonst}
	\end{cases}\]
\item Falls "`$F=\forall xG$"' so ist \[ A(F)=
	\begin{cases}
	1, & \text{falls für alle $d \in U_A$ gilt $A_{[x/d]}(G)=1$} \\
	0, & \text{sonst}
	\end{cases}\]
\item Falls "`$F=\exists xG$"' so ist \[ A(F)=
	\begin{cases}
	1, & \text{falls es ein $d \in U_A$ gibt, mit $A_{[x/d]}(G)=1$}\\
	0, & \text{sonst}
	\end{cases}\]
\end{enumerate}

Hierbei bedeute $A_{[x/d]}$ diejenige Struktur A', die überall mit A identisch ist, bis auf die Definition von $x^A$. Hier ist $x^{A'}=d$ wobei $d \in U_A = U_{A'}$

\paragraph{Bemerkungen}
\begin{enumerate}
\item $F$ ist gültig $\Gdw$ $\neg F$ unerfüllbar ist.\\ \textbf{Beweis:} es gilt $F$ ist gültig 
	\\$\Gdw$ jede zu $F$ passende Belegung ist ein Modell für $F$ 
	\\$\Gdw$ jede zu $F$ und damit auch zu $\neg F$ passende Belegung ist kein Modell für $\neg F$ 
	\\$\Gdw$ $\neg F$ besitzt kein Modell 
	\\$\Gdw$ $\neg F$ ist unerfüllbar.
\item Nicht jede mathematische Aussage kann im Rahmen der Prädikatenlogik formuliert werden. Erst wenn auch Quantoren über Prädikaten und Funktionssymbolen erlaubt werden ist dies möglich. Dies ist dann die Prädikatenlogik der zweiten Stufe. Die obige ist die Prädikatenlogik der ersten Stufe.
\end{enumerate}

\paragraph{Beispiel} Interpretation eines Atoms mit einer Variablen 

Die Notation für eine Variablenbelegung (\begriff{valuation}) ist $[x \mapsto c] \ (c \in D)$.

Atom mit Variable
	$$\text{Für die Variablenbelegung $\mu=[x \mapsto \J] \text{ ist } I_\mu(\text{woman}(x))=T$ }$$
	$$\text{Für die Variablenbelegung $\nu=[x \mapsto \R] \text{ ist } I_\mu(\text{woman}(x))=F$ }$$

Seien $\Phi$ und $\Psi$ Formeln, deren Interpretationen $I_\mu(\Phi)$ und $I_\mu(\Psi)$ unter der Variablenbelegung $\mu$ bekannt sind, dann folgt daraus, dass eine Interpretation für "`komplexe"' Formeln exisitiert und induktiv wie folgt bestimmt werden kann:
\begin{eqnarray*}
I_\mu(\Phi \vee \Psi) := T & \Gdw & I_\mu(\Phi)=T \vee I_\mu(\Psi)=T \\
I_\mu(\Phi \wedge \Psi) := T & \Gdw & I_\mu(\Phi)=T \wedge I_\mu(\Psi)=T \\
I_\mu(\neg \Phi) := T & \Gdw & I_\mu(\Phi)=F \\
I_\mu(\Phi \rightarrow \Psi) := T & \Gdw & I_\mu(\Phi)=F \vee I_\mu(\Psi)=T \\
I_\mu(\Phi \leftarrow \Psi) := T & \Gdw & I_\mu(\Phi)=T \vee I_\mu(\Psi)=F \\
I_\mu(\Phi \leftrightarrow \Psi) := T & \Gdw & (I_\mu(\Phi)=T \vee I_\mu(\Psi)=T) \wedge (I_\mu(\Phi)=F \vee I_\mu(\Psi)=F) \\
I_\mu(\exists x \Phi) := T & \Gdw & \text{ein $c \in D$ existiert, so dass $I_{\mu[x \dann c]}(\Phi) = T$} \\
I_\mu(\forall x \Phi) := T & \Gdw & \text{für alle $c \in D$ gilt $I_{\mu[x \dann c]}(\Phi) = T$}
\end{eqnarray*}

\paragraph{Bemerkung:} Eine Formel kann auf verschiedene Arten interpretiert werden.

\paragraph{Beispiel:} verschiedene Interpretationen einer Formel

$$\text{Sei }\Phi = \forall x(g(x,f,t) \dann l(t,x))$$

Die Interpretation $I$ sei wie folgt defininiert:

\begin{itemize}
\item $I(f)$ := "`(Katzen-)Futter"'
\item $I(t)$ := "`Katze namens Tom"'
\item $I(g)$ := "`Die Abbildung, die einem Tripel $\langle X,Y,Z\rangle$ $T$ zuordnet, falls Objekt $X$ dem Objekt $Z$ das Objekt $Y$ gibt und $F$ sonst"'
\item $I(l)$ := "`Die Abbildung, die einem Paar $\langle X,Y \rangle$ T zuordnet, falls das Objekt $X$ das Objekt $Y$ liebt und $F$ sonst"'
\end{itemize}

Die Aussage "`Falls $X$ der Katze namens Tom Katzenfutter gibt, dann liebt Tom dieses $X$"' kann sehr wohl für jede Variablenbelegung von $X$ aus einem bestimmten Diskursbereich als wahr betrachtet werden. Damit wäre die Formel $\Phi$ bezüglich der Interpretation $I$ erfüllt, das heißt $I(\Phi) = T$.

\begin{figure}[htb]
	\centering
	\begin{tabular}{|c|c|c|c|c|c|}
		\hline
		$A$	&	$B$	&	$A \vee B$ & $A \wedge B$ & $A \dann B$ ($\equiv\neg A \vee B$) &
		$A \equi B$ ($\equiv (A\wedge B) \vee (\neg A \wedge \neg B)$) \\
		\hline
		W	&W	&W	&W	&W	&W	\\
		W	&F	&W	&F	&F	&F	\\
		F	&W	&W	&F	&W	&F	\\
		F	&F	&F	&F	&W	&W	\\
		\hline
	\end{tabular}
	\caption{Wahrheitswerte der Aussagen}
\end{figure}

%begin 18. April 2005 (Lars)

\section{Normalformen}

\lectureof{18.04.2005}
\subsection{Definition (Äquivalenzbegriff)}

Zwei prädikatenlogische Formeln $F$ und $G$ sind äquivalent, falls für alle sowohl zu $F$ als auch zu $G$ passenden Strukturen $A$ gilt: $$A(F) = A(G)$$

\paragraph{Beispiel (de Morgansches Gesetz)} \[\neg (F \wedge G) \equiv \neg F \vee \neg G\]

\subsection{Satz}

Seien $F$ und $G$ beliebige Formeln. Dann gilt:
\begin{enumerate}
\item
\begin{eqnarray*}
	\neg \forall xF & \equiv & \exists x\neg F \\
	\neg \exists xF & \equiv & \forall x\neg F
\end{eqnarray*}
\item Falls $x$ in $G$ nicht frei vorkommt, gilt:
\begin{eqnarray*}
	(\forall xF\wedge G) & \equiv & \forall x(F\wedge G) \hfill ~(*) \\
	(\forall xF\vee G) & \equiv & \forall x(F\vee G) \\
	(\exists xF\wedge G) & \equiv & \exists x(F \wedge G) \\
	(\exists xF\vee G) & \equiv & \exists x(F \vee G)
\end{eqnarray*}
\item
\begin{eqnarray*}
	(\forall xF \wedge \forall xG) & \equiv & \forall x(F \wedge G) \\
	(\exists xF \vee \exists xG) & \equiv & \exists x(F \vee G)
\end{eqnarray*}
\item
\begin{eqnarray*}
	\forall x\forall yF & \equiv & \forall y\forall xF \\
	\exists x\exists yF & \equiv & \exists y\exists xF
\end{eqnarray*}
\end{enumerate}

\paragraph{Beweis} (*) (exemplarisch)

Sei $A=(U_A,I_A)$ eine zu den beiden Seiten der zu beweisenden Äquivalenz passende Struktur. Es gilt:
\begin{eqnarray*}
	& &A(\forall xF \wedge G) = 1 \\
	\Gdw & & A(\forall xF) = 1 \text{ und } A(G) = 1 \\
	\Gdw & & \text{ für alle } d\in U_A \text{ gilt: } \\
	& & A_{[x/d]}(F) = 1 \text{ und } A(G) = 1 \\
	\Gdw & & \text{ für alle } d\in U_A \text{ gilt: } \\
	& & A_{[x/d]}(F) = 1 \text{ und } A_{[x/d]}(G) = 1\\
	& & \text{\ (\textbf{Bem.:} da $x$ in $G$ nicht frei vorkommt, ist nämlich $A(G)=A_{[x/d]}(G)$)} \\
	\Gdw & & \text{ für alle } d\in U_A \text{ gilt: } \\
	& & A_{[x/d]}(F\wedge G) = 1 \\
	\Gdw & & A(\forall x(F\wedge G)) = 1
\end{eqnarray*}

\subsection{Definition (NNF, KNF, DNF)}

\begin{itemize}
\item Eine Formel ist in \emph{Negationsnormalform}, wenn sie außer $\neg$, $\wedge$, $\vee$ keine Junktoren enthält und wenn in ihr $\neg$ nur vor atomaren Teilformen auftritt (NNF).
\item Eine Formel ist in \emph{konjunktiver} (bzw. \emph{disjunktiver}) \emph{Normalform}, wenn sie eine Konjunktion (bzw. Disjunktion) von Disjunktionen (bzw. von Konjunktionen) von Literalen ist.
\end{itemize}

\subsection{Definition (Substitution in Formeln)}

Sei $F$ eine Formel, $x$ eine Variable und $t$ ein Term. Dann bezeichnet $F_{[x/t]}$ diejenige Formel, die man aus $F$ erhält, indem man jedes freie Vorkommen der Variablen $x$ in $F$ durch den Term $t$ ersetzt. $[x/t]$ beschreibt eine \emph{Substitution}. Substitutionen (oder Folgen von Substitutionen) behandeln wir auch als selbstständige Objekte, z. B. $sub=[x/t_1][y/t_2]$ (wobei $t_1$ auch $y$ enthalten darf).

\subsection{Lemma (gebundene Umbenennung)}

Sei $F = QxG$ eine Formel mit $Q\in \{\exists, \forall\}$. Es sei $y$ eine Variable, die in $G$ nicht vorkommt. Dann gilt:
$$F\equiv QyG_{[x/y]}$$

\subsection{Lemma}

Zu jeder Formel $F$ gibt es eine äquivalente Formel $G$ in \emph{bereinigter} Form.\\ Hierbei heißt die Formel bereinigt, sofern es keine Variable gibt, die in der Formel sowohl gebunden als auch frei vorkommt, und sofern hinter allen vorkommenden Quantoren verschiedene Variablen stehen.

\paragraph{Beispiel}

$$F:=\forall x\exists y(P(x,y)\wedge Q(x,a))$$
$$G:=\exists u(\forall vP(u,v)\vee Q(v,v))$$
$F$: $x$, $y$ sind gebunden, $a$ ist frei $\rightarrow$ $F$ ist bereinigt. \\
$G$: alle Vorkommen der Variablen $u$ sind gebunden, allerdings tritt $v$ sowohl gebunden als auch frei auf $\rightarrow$ $G$ ist nicht in bereinigter Form.

\subsection{Definition (Pränexform)}

Eine Formel heißt pränex oder in Pränexform, falls sie die Form $$Q_1y_1Q_2y_2\ldots Q_ny_nF$$ hat, wobei $Q_i\in\{\exists, \forall\}$, $n\geq0$, $y_i$ Variablen sind. Es kommt ferner kein Quantor in $F$ vor.

\subsection{Satz}

Für jede Formel $F$ gibt es eine äquivalente (und bereinigte) Formel $G$ in Pränex"-form.

\paragraph{Beweis} (Induktion über den Formelaufbau)

\begin{description}
\item[Induktionsanfang:] $F$ ist atomare Formel. Dann liegt $F$ bereits in der ge"-wünsch"-ten Form vor. Wähle also $G=F$.
\item[Induktionsschritt:] Wir betrachten wieder die verschiedenen Fälle.
	\begin{enumerate}
		\item $F$ hat die Form $\neg F_1$ und $G_1=Q_1y_1Q_2y_2\ldots Q_ny_nG'$ sei die nach Induktionsvoraussetzung existierende
					zu $F_1$ äquivalente Formel. Dann gilt:
					$$F\equiv \quer{Q}_1 y_1 \quer{Q}_2 y_2 \ldots \quer{Q}_n y_n \neg G'$$
					wobei $\quer{Q}_i=\exists$, falls $Q_i=\forall$ und $\quer{Q}_i=\forall$ falls $Q_i=\exists$. Diese Formel hat die gewünschte Form.
		\item $F$ hat die Form $(F_1 \circ F_2)$ mit $\circ\in\{\wedge,\vee\}$. Dann gibt es zu $F_1$ und $F_2$ äquivalente Formeln 
					$G_1$ und $G_2$ in 	bereinigter Pränexform. Durch gebundenes Umbenennen können wir die gebundenen Variablen von $G_1$ 
					und $G_2$ disjunkt machen. Dann hat:
					\begin{itemize}
						\item $G_1$ die Form $Q_1y_1Q_2y_2\ldots Q_ky_kG_1'$
						\item $G_2$ die Form $Q_1'z_1Q_2'z_2\ldots Q_l'z_lG_2'$
					\end{itemize}
					mit $Q_i,Q_j'\in\{\exists,\forall\}$. Daraus folgt, dass $F$ zu 
						$$Q_1y_1Q_2y_2\ldots Q_ky_kQ_1'z_1\ldots Q_l'z_l(G_1'\circ G_2')$$
					äquivalent ist.
		\item $F$ hat die Form $QxF_1$, $Q\in\{\exists,\forall\}$. Die nach Induktionsvorraussetzung existierende bereinigte Pränexform habe die
					Bauart $Q_1y_1Q_2y_2\ldots Q_ky_kF_1'$. Durch gebundenes Umbenennen kann die Variable $x$ verschieden gemacht werden von 
					$y_1,\ldots ,y_k$. Dann ist $F$ zu $QxQ_1y_1Q_2y_2\ldots Q_ky_kF_1'$ äquivalent.
	\end{enumerate}
\end{description}

\subsection{Definition (Skolemform)}

Für jede Formel $F$ in BPF (bereinigte Pränexform) definieren wir ihre \emph{Skolem"-form(-el)} als das Resultat der Anwendung folgenden Algorithmus' auf $F$: 
 
\begin{codebox}
\Procname{\proc{Skolemform-Algorithmus}}
\li \While $F$ enthält einen Existenzquantor
\li     \Do
            $F$ habe die Form $F=\forall y_1\forall y_2\ldots\forall y_n\exists zG$ für eine Formel $G$ in BPF
\zi         und $n\geq 0$ (der Allquantorblock kann auch leer sein);
\li         Sei $f$ ein neues, bisher in $F$ nicht vorkommendes $n$-stelliges
\zi         Funktionssymbol;
\li         $F:=\forall y_1\forall y_2\ldots\forall y_nG_{[z/f(y_1,y_2,\ldots ,y_n)]}$; (d.h. der Existenzquantor in $F$ wird
\zi         gestrichen und jedes Vorkommen der Variable $z$ in $G$ durch 
\zi         $f(y_1,y_2,\ldots ,y_n)$ ersetzt.
    \End
\end{codebox}

\subsection{Satz}

Für jede Formel $F$ in BPF gilt: $F$ ist erfüllbar $\Gdw$ die Skolemform ist erfüllbar.
\paragraph{Beweis} siehe Schöning, U.: "`Logik für Informatiker"'  

\subsection{Definition (Herbrand-Universum)}

Das Herbrand-Universum $D(F)$ einer geschlossenen Formel $F$ in Skolemform ist die Menge aller variablenfreien Terme, die aus den Bestandteilen von $F$ gebildet werden können. Falls keine Konstante vorkommt, wählen wir eine beliebige Konstante, z. B. $a$ und bilden dann die variablenfreien Terme.\\
$D(F)$ wird wie folgt induktiv definiert:
\begin{enumerate}
\item Alle in $F$ vorkommenden Konstanten sind in $D(F)$.\\
Falls $F$ keine Konstanten enthält, so ist $a$ in $D(F)$.
\item Für jedes in $F$ vorkommene $n$-stellige Funktionssymbol $f$ und Terme $t_1,t_2,\ldots ,t_n$ in $D(F)$ ist der Term $f(t_1,t_2,\ldots ,t_n)$ in $D(F)$.
\end{enumerate}

\paragraph{Beispiel}

\begin{eqnarray*}
	F & = &\forall x\forall y\forall zR(x,f(y),g(z,x)) \\
	G & = & \forall x\forall yQ(c,f(x),h(y,b))
\end{eqnarray*}
Für F liegt der Spezialfall aus 1. vor (kein Vorkommen einer Konstanten).
\begin{eqnarray*}
D(F) & = & \{a,f(a),g(a,a),f(g(a,a)),f(f(a)),\\
& & g(a,f(a)),g(f(a),a),g(f(a)),\ldots\}\\
D(G) & = & \{c,b,f(c),f(b),h(c,c),h(c,b),h(b,c),\\
& & h(b,b),f(f(c)),f(f(b)),f(h(c,c)),f(h(c,b)),\ldots\}
\end{eqnarray*}


\subsection{Definition (Herbrand-Expansion)}
\lectureof{20.04.2005} Sei $F=\forall y_1 \forall y_2 \ldots \forall y_n F^*$ eine Aussage in Skolemform,
dann ist $E(F)$ die Herbrand-Expansion von $F$ definiert als
$$E(F)=\{F^*_{[y_1/t_1][y_2/t_2]\ldots[y_n/t_n]} | t_1,t_2, \ldots, t_n \in D(F) \}$$

Anmerkung, nicht im Skript: Es werden also alle gebundenen Variablen von $F$ in der Matrix $F*$ von $F$ durch beliebige Terme aus dem Herbrand-Universum $D(F)$ ersetzt.

\subsection{Satz (Gödel-Herbrand-Skolem)}
Für jede Aussage $F$ in Skolemform gilt:
$F$ ist erfüllbar genau dann, wenn die Formelmenge $E(F)$ (im aussagenlogischen Sinn) erfüllt ist.

\paragraph{Beispiel}
$$F=\forall x \forall y \forall z P(x,f(y,y), g(z,x))$$
Die einfachsten Formeln in $E(F)$ sind die folgenden:
\begin{eqnarray*}
	P(a,f(a,a), g(a,a)) & \text{ mit } & [x/a], \ldots, [z/a]\\
	P(f(a), f(a,a), g(a,f(a)))& \text{ mit } &[x/f(a)], [y/a], [z/a]
\end{eqnarray*}

\section{Prolog und Prädikatenlogik}

\subsection{Logikprogrammierung}
Prolog wurde um 1970 von Alain Colmerauer und seinen Mitarbeitern mit dem Ziel entwickelt, die Programmierung von Computern mit den Mitteln "`der Logik"' zu ermöglichen.

\subsection{Pure Prolog}
Das sogenannte Pure Prolog oder Database-Prolog entspricht einer Teilmenge der Sprachdefinition eines praktischen Prolog-Entwicklungssystems und enthält keine extra- oder metalogischen Komponenten wie:
\begin{itemize}
\item Cut, Type-Checking
\item Arithmetische Operationen
\item Datenbasismanipulation zur Laufzeit
\end{itemize}

\subsection{Prolog und Logik}
Pure Prolog-Programme entsprechen den Ausdrücken der Hornklausellogik, die eine Teilmenge der Prädikatenlogik 1. Stufe ist.
Das Beweisverfahren Resolution ermöglicht Inferenzen aufgrund von Prologprogrammen oder Hornklauseln.

\paragraph{Bemerkung}Zur Transformation von prädikatenlogischen Sachverhalten nach Prolog wird der Weg über die Hornklauseln genutzt.

$$\text{Prädikatenlogik 1. Stufe} \stackrel{\text{KNF}}{\Longrightarrow} \text{Hornklauseln} \Longrightarrow \text{Prolog}$$

\subsection{Prädikatenlogik 1. Stufe}
\paragraph{Inventar der Syntax:}
\begin{itemize}
\item Individuenkonstanten $a$, $b$, $c$, $\ldots$
\item Individuenvariablen $x$, $y$, $z$, $\ldots$
\item Prädikate $P(t_1, \ldots, t_n),\ t_i \in$ TERM
\item Quantoren $\forall$, $\exists$
\item Junktoren $\neg, \vee, \wedge, \dann, \gdw$
\end{itemize}

\subsection{Formeln der Prädikatenlogik}
\begin{itemize}
\item Wenn $P$ ein $n$-stelliges Prädikat ist und $t_1, \ldots, t_n$ Terme sind, dann ist $P(t_1, \ldots, t_n)$ ein Literal.
\item Literale sind Formeln.
\item Wenn $\Phi$ und $\Psi$ Formeln sind, dann sind auch $\neg\Phi$, $\Phi\vee\Psi$, $\Phi\wedge\Psi$, $\Phi\dann\Psi$, $\Phi\gdw\Psi$ Formeln.
\item Wenn $\Phi$ eine Formel ist und $x$ eine Variable, dann sind auch $\forall x\Phi$, $\exists x\Phi$ Formeln.
\end{itemize}

\subsection{Klauseln}
\begin{itemize}
\item Wenn $P$ ein $n$-stelliges Prädikat ist und $t_1, \ldots, t_n$ Terme sind, dann ist $P(t_1, \ldots, t_n)$ ein Literal.
\item Literale sind Klauseln.
\item Wenn $\Phi$ ein Literal ist, dann sind auch $\neg\Phi$, eine Klausel.
\item Wenn $\Phi$ und $\Psi$ Klauseln sind, dann ist auch $\Phi \vee \Psi$ eine Klausel.
\end{itemize}

\subsection{Hornklauseln}

Hornklauseln sind Klauseln, die genau ein nicht-negiertes Literal und beliebig viele negierte Literale enthalten.

\paragraph{Beispiele}
\begin{itemize}
\item $\text{vater}(x,y) \vee \neg\text{elternteil}(x,y) \vee \neg\text{männlich}(x)$ 
\item $\text{sterblich}(sokrates)$
\end{itemize}

Anmerkung, nicht in der Vorlesung: Hornklauseln lassen sich als Implikationen darstellen. Die beiden Beispielen sind zu den folgenden äquivalent:

\begin{itemize}
\item $(\text{elternteil}(x,y) \wedge \text{Männlich}(x)) \Rightarrow \text{vater}(x,y)$
\item $wahr \Rightarrow \text{sterblich}(sokrates)$
\end{itemize}

\subsection{Konjunktive Normalform}
Eine Formel ist in konjunktiver Normalform, wenn sie eine Konjunktion von Klauseln repräsentiert.
$$K_1 \wedge \ldots \wedge K_n,\ K_i \in \text{KLAUSEL}$$
Formeln der Prädikatenlogik können durch Anwendung logischer Äquivalenzregeln in die konjunktive Normalform gebracht werden.

\subsection{Logische Äquivalenzregeln}
\subsubsection{Kommutativität}
\begin{eqnarray*}
 P \wedge Q & \Gdw & Q \wedge P \\
 P \vee Q & \Gdw & Q \vee P 
\end{eqnarray*}

\subsubsection{Assoziativität}
\begin{eqnarray*}
	(P \wedge Q) \wedge R & \Gdw & P \wedge (Q \wedge R) \\
	(P \vee Q) \vee R & \Gdw & P \vee (Q \vee R)
\end{eqnarray*}

\subsubsection{De Morgan}
\begin{eqnarray*}
	\neg(P \vee Q)   & \Gdw & \neg P \wedge \neg Q \\
	\neg(P \wedge Q) & \Gdw & \neg P \vee \neg Q 
\end{eqnarray*}

\subsubsection{Konditional- \& Bikonditionalgesetz}
\begin{eqnarray*}
	P \dann Q & \Gdw & \neg P \vee Q\\
	P \gdw Q  & \Gdw & (P \dann Q) \wedge (Q \dann P)
\end{eqnarray*}

\subsubsection{Idempotenz}
\begin{eqnarray*}
	P \vee P   & \Gdw & P \\
	P \wedge P & \Gdw & P
\end{eqnarray*}

\subsubsection{Identität}
\begin{eqnarray*}
	P \vee 0   & \Gdw & P \\
	P \vee 1   & \Gdw & 1 \\
	P \wedge 0 & \Gdw & 0 \\
	P \wedge 1 & \Gdw & P
\end{eqnarray*}

\subsubsection{Komplementarität}
\begin{eqnarray*}
	P \vee \neg P   & \Gdw & 1\ \text{(Tautologie, allgemeingültig)} \\
	P \wedge \neg P & \Gdw & 0\ \text{(Kontradiktion, Inkonsistenz)} \\
	\neg\neg P      & \Gdw & P\ \text{(Doppelte Negation)} 
\end{eqnarray*}

\subsection{Quantorengesetze}
\subsubsection{Quantoren-Negation}
\begin{eqnarray*}
	\neg \forall x \Phi      & \Gdw & \exists x \neg \Phi \\
	\forall x \Phi           & \Gdw & \neg \exists x \neg \Phi \\
	\neg \forall x \neg \Phi & \Gdw & \exists x \Phi \\
	\forall x \neg \Phi      & \Gdw & \neg \exists x \Phi
\end{eqnarray*}

\subsubsection{Quantoren-Distribution}
\begin{eqnarray*}
	\forall x (\Phi \wedge \Psi)       & \Gdw   & \forall x \Phi \wedge \forall x \Psi \\
	\exists x (\Phi \vee \Psi)         & \Gdw   & \exists x \Phi \vee \exists x \Psi \\
	\forall x \Phi \vee \forall x \Psi & \folgt & \forall x (\Phi \vee \Psi) \\
	\exists x (\Phi \wedge \Psi)       & \folgt & \exists x \Phi \wedge \exists x \Psi
\end{eqnarray*}

\subsubsection{Quantoren-Dependenz}
\begin{eqnarray*}
	\forall x \forall y \Phi & \Gdw   & \forall y \forall x \Phi \\
	\exists x \exists y \Phi & \Gdw   & \exists y \exists x \Phi \\
	\exists x \forall y \Phi & \folgt & \forall y \exists x \Phi 
\end{eqnarray*}

\subsubsection{Quantoren-Bewegung}
\begin{eqnarray*}
	\Phi \dann \forall x \Psi & \Gdw & \forall x (\Phi \dann \Psi) \\
	\Phi \dann \exists x \Psi & \Gdw & \exists x (\Phi \dann \Psi) \\
	(\forall x \Phi) \dann \Psi & \Gdw & \exists x (\Phi \dann \Psi) \\
	(\exists x \Phi) \dann \Psi & \Gdw & \forall x (\Phi \dann \Psi) 
\end{eqnarray*}

\subsection{Pränex-Normalform}
Eine prädikatenlogische Formel befindet sich in PNF, wenn alle Quantoren am Anfang der Formel stehen.
$$\begin{array}{c}
	(\exists x F(x)) \dann (\forall y G(y)) \\
	\Downarrow\\
	\forall y \forall x (F(x) \dann G(y))
\end{array}$$

\subsection{Skolemisierung}
Existenzquantoren können eleminiert werden, indem existenzquantifizierte Variablen durch Skolemkonstanten substituiert werden.
Liegt ein Existenzquantor im Skopus von Allquantoren, werden die Skolemkonstanten mit den jeweiligen allquantifizierten Variablen durch Parametrisierung in Abhängigkeit gebracht.

\paragraph{Beispiele}
\begin{itemize}
\item $\exists y \forall x ((\text{man}(x) \wedge (\text{woman}(y)) \dann \text{loves}(x,y))
		\\\Longrightarrow\ \forall x ((\text{man}(x) \wedge \text{woman}(G)) \dann \text{loves}(x, G))$\\
\item $\forall x (\text{man}(x) \dann \exists y (\text{woman}(y) \wedge \text{loves}(x,y)))\\
		\Longrightarrow\ \forall x (\text{man}(x) \dann (\text{woman} (G(x)) \wedge \text{loves}(x,G(x))))$
\end{itemize}

\section{Aussagenlogische Resolution}
Sie dient zum Nachweis der Unerfüllbarkeit einer Formel. Viele Aufgaben können auf einen Test auf Unerfüllbarkeit reduziert werden:
\begin{itemize}
\item $F$ ist Tautologie gdw. $\neg F$ ist unerfüllbar
\item Folgt $G$ aus $F_1,F_2,\dots, F_n$ ? \\
      d.h.: Ist $F_1 \wedge F_2 \wedge \dots \wedge F_n \dann G$ eine Tautologie? \\
      d.h.: Ist $F_1 \wedge F_2 \wedge \dots \wedge F_n \wedge \neg G$ unerfüllbar? 
\end{itemize}

Die aussagenlogische Resolution ist eine Umformungsregel und geht von der \begriff{Klauselschreibweise} der KNF Formel aus. 
Die Formel $F$ liegt in KNF vor, wenn gilt:
	$$F=(L_{1,1}\vee\dots\vee L_{1,n})\wedge\dots\wedge(L_{m,1} \vee\dots\vee L_{m,n})$$
wobei $L_{i,j}$ Literale sind, also 
	$$L_{i,j}\in \{A_1,A_2,\dots\}\cup\{\neg A_1,\neg A_2,\dots\}$$
Die Klauselschreibweise von $F$ ist dann: 
	$$F=\{\{L_{1,1},\dots ,L_{1,n})\},\dots,\{(L_{m,1},\dots , L_{m,n}\}\}$$


\subsection{Definition (Resolvent)}
\begin{itemize}
\item Es seien $K_1$ und $K_2$ Klauseln mit $A\in K_1$ und $\neg A \in K_2$, 
    wobei $A$ eine atomare Formel ist. Dann heißt 
    $$ R=(K_1 \setminus \{A\}) \cup (K_2 \setminus \{\neg A\}) $$ der \begriff{Resolvent} von 
    $K_1$ und $K_2$ (bezüglich $A$).
\item Diagramm: \\
    \xymatrix{
        K_1 \ar@{-}[dr] &   & K_2 \ar@{-}[dl] \\
            & R &
    }

\item Falls $K_1 =\{A\}$ und $K_2=\{\neg A\}$, dann ist $R= \emptyset$
\item Eine Klauselmenge, die $\emptyset$ enthält, ist unerfüllbar. 
\item Beispiel:\\
    \xymatrix {
        \{A_3, \neg A_1, A_2\} \ar@{-}[dr] &                        & \{A_1, A_2, \neg A_3\} \ar@{-}[dl] \\
                                           & \{A_3, A_2, \neg A_3\} &
    }
\end{itemize}


\subsection{Definition (Res($F$))}
Für eine Klauselmenge $F$ sei
	$$ \mbox{Res}(F) = F\cup\{R\ |\ R \mbox{ ist Resolvent zweier Klauseln in $F$}\}.$$
Weiter sei
\begin{itemize}
\item $\mbox{Res}^0(F) = F$
\item $\mbox{Res}^{n+1}(F) = \mbox{Res}(\mbox{Res}^n(F))$
\item $\mbox{Res}^{\ast}(F) = \bigcup_{n\geq 0} \mbox{Res}^n$
\end{itemize} 

\paragraph{Resolutionssatz der Aussagenlogik:} Eine Klauselmenge $F$ ist genau dann
unerfüllbar, wenn $\emptyset \in \mbox{Res}^\ast (F)$


\paragraph{Beispiel}
%    \begin{xy}
    % In der folgenden XYPic Matrix sind die Abstände zwischen Zeilen auf .5pc
    % und zwischen Spalen auf .2pc gesetzt. 
    \xymatrix@C=.2pc@R=.5pc{
        \{\{A_1, A_2\}, \ar@{-}[dr] &         & \ar@{-}[dl] \{\neg A_1, A_2\}, &           & \{A_1, \neg A_2\},  \ar@{-}[dr] &              & \ar@{-}[dl] \{\neg A_1, \neg A_2\}\} \\
                            & \{A_2\} \ar@{-}[drr] &                           &           &                          & \ar@{-}[dll] \{\neg A_2\} & \\
                                    &         &                    & \emptyset &                    &              &
    }
%    \end{xy}
	Die Formel ist unerfüllbar!
                        
\subsection{Resolution in der Prädikatenlogik}
Es gibt zwei Varianten
\begin{enumerate}
\item Zurückführen auf die aussagenlogische Resolution (Grundresolution)
\item Direkte Resolution (Unifikation)
\end{enumerate} 

\paragraph{Grundresolution} 
\begin{itemize}
\item Stelle die Matrix der Formel $F$ in KNF dar. Dann sind die Formeln in $E(F)$ ebenfalls in KNF. 
\item Da die Formeln in $E(F)$ variablenfrei sind kann die Resolution der Aussagenlogik verwendet werden.
\end{itemize}

\paragraph{Bemerkung} Dieser Algorithmus kann nicht terminieren, falls die Formel erfüllbar ist!

\paragraph{Beispiel}
	$$F = \forall x (P(x) \wedge \neg P(f(x))) $$
Matrix von $F$ ist schon in KNF:
	$$F^\ast = P(x) \wedge P(f(x))$$
als Klauselmenge: 
\begin{eqnarray*}
	F^\ast & = & \{\{P(x)\},\{\neg P(f(x))\}\} \\
	D(F)   & = & \{a,f(a),f(f(a)),\dots\}      \\
	E(F)   & = & \{P(a)\wedge P(f(a)),P(f(a))\wedge P(f(f(a))),\dots\} 
\end{eqnarray*}
als Klauselmenge
	$$E(F) = \{\{P(a)\},\{\neg P(f(a))\},\{P(f(a))\},\{\neg P(f(f(a))),\dots\}\}$$
wegen

 \xymatrix@C=.2pc@R=.5pc{
    \{\{\neg P(f(a))\},  \ar@{-}[dr] &           & \ar@{-}[dl] \{P(f(a))\}\} \\
                                     & \emptyset &
 }

ist F unerfüllbar.

\paragraph{Verbesserung}
Bei diesem Verfahren werden i. A. zu viele Klauseln generiert, die für die Herleitung von $\emptyset$ nicht gebraucht werden, da die gleiche Substitution auf die ganze Klauselmenge $F^\ast$ angewendet wird.

Im vorherigen Beispiel genügt es aber für die Klausel $\{P(f(x))\}$ die Substitution $[x/f(a)]$ und für die Klausel $\{\neg P(f(x))\}$ die Substitution $[x/a]$ durchzuführen um $\emptyset$ herleiten zu können. 

\subsection{Allgemeinster Unifikator}
\begin{itemize}
\item Es sei $L=\{L_1,L_2,\dots,L_k\}$ eine Menge von prädikatenlogischen Literalen und sub eine Substitution. Wir schreiben 
  $L\mbox{sub}$ für die Menge $\{L_1\mbox{sub},\dots, L_k\mbox{sub}\}$.
\item sub ist ein \emph{Unifikator} für $L$, falls $|L\mbox{sub}| = 1$, dann heißt $L$ \emph{unifizierbar} 
\item Ein Unifikator sub von $L$ heißt \emph{allgemeinster Unifikator} von $L$, wenn es für jeden Unifikator sub' von $L$ eine 
  Substitution $s$ gibt mit sub' $= \mbox{sub}s$
\end{itemize}


\section{Prädikatenlogische Resolution}
Es seien $K_1$,$K_2$ prädikatenlogische Klauseln mit 
\begin{enumerate}
\item Es gibt Substitutionen $s_1$ und $s_2$, so dass $K_1s_1$ und $K_2s_2$ keine gemeinsamen Variablen enthalten.
\item Es gibt Literale $L_1,\dots,L_m \in K_1s_1 \ (m\geq 1)$ und $L_1^\prime,\dots,L_n^\prime \in K_2s_2 \ (n\geq 1)$, so dass
  $L = \{\neg L_1,\dots,\neg L_m,L_1^\prime,\dots,L_n^\prime\}$ unifizierbar ist. Es sei sub ein allgemeinster Unifikator von $L$. 
\end{enumerate}

Dann heißt die Klausel
$$ R:= ((K_1 s_1 \setminus \{L_1,\dots,L_m\})\cup
        (K_2 s_2 \setminus \{L_1^\prime,\dots,L_n^\prime\} ))sub $$
ein \begriff{prädikatenlogischer Resolvent} von $K_1$ und $K_2$. 

\end{document}
