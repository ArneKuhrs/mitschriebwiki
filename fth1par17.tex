\documentclass{article}
\newcounter{chapter}
\setcounter{chapter}{17}
\usepackage{ana}
\def\gdw{\equizu}
\def\Arg{\text{Arg}}
\def\MdD{\mathbb{D}}
\def\Log{\text{Log}}
\def\Tr{\text{Tr}}
\def\abnC{\ensuremath{[a,b]\to\MdC}}
\def\wegint{\ensuremath{\int\limits_\gamma}}
\def\iint{\ensuremath{\int\limits}}
\def\ie{\rm i}

\title{Der Residuensatz und Folgerungen}
\author{Christian Schulz, Florian Mickler} % Wenn ihr nennenswerte \"Anderungen macht, schreibt euch bei \author dazu

\begin{document}
\maketitle 

%17.1
\begin{satz}[Residuensatz]
  $G$ sei ein Elementargebiet, es seien $z_1, \ldots, z_k \in G (z_j \neq z_l$ f\"ur
   $j \neq l)$ und es sei $f \in H(G \backslash \{z_1, \ldots, z_k\})$. Jedes $z_j$
  ist also eine isolierte Singularit\"at von $f$. Weiter sei $\gamma$ ein
  geschlossener, st\"uckweise glatter Weg mit Tr$(\gamma) \subseteq G \backslash
  \{z_1, \ldots, z_k\}$. \\
  Dann: \\ \\
  \centerline{$\frac{1}{2 \pi \ie}$ 
  $ \int\limits_{\gamma} f(z) dz $
  $= \sum\limits_{j=1}^k n(\gamma, z_j) Res(f, z_j)$ }
\end{satz}
\begin{beweis}
  $\forall j \in \{1 \ldots k\}$ existiert ein $R_j > 0$: 
  $\overline{U_{R_j}(z_j)} \subseteq G$  
  und $\overline{U_{R_j}(z_j)} \cap \overline{U_{R_l}(z_l)} = \emptyset ( j \neq l)$.
  Sei $j \in \{1..k\}$. \\
  $\stackrel{\text{14.4}}{\Rightarrow} $ $f$ hat auf $U_{R_j}(z_j)$ die 
  Laurententwicklung \\ \\ \centerline{$f(z) = \sum\limits_{n=0}^{\infty}
  a_n^{(j)}(z-z_j)^n + 
  \underbrace{\sum\limits_{n=1}^{\infty}
  a_{-n}^{(j)}(z-z_j)^{-n}}_{\varphi_j(z)}$}, wobei $\varphi_j \in H(\MdC
  \backslash \{z_j \})$ \\ 
  Definiere $g \in H(G \backslash \{z_1,\ldots, z_k\})$ durch $g(z) = f(z) -
  \sum\limits_{j=1}^{k} \varphi_j(z)$. \\
  Dann hat $g$ in $z_j$ eine hebbare Singularit\"at$(j = 1\ldots k)$. Also $g \in
  H(G)$. $G$ ist ein Elementargebiet $\Rightarrow$ $g$ hat eine Stammfunktion auf
  $G$ $\stackrel{\text{8.6}}{\Rightarrow}$ $\int\limits_{\gamma} g(z)  dz = 0$
  $\Rightarrow$ $\wegint f(z) dz = \sum\limits_{j=1}^k \wegint \varphi_j(z) dz$.\\
  Noch zu zeigen: $\wegint  \varphi_j(z) dz = 2 \pi \ie $ $ n(\gamma, z_j)
  a_{-1}^{(j)}$ $(j = 1\ldots k)$. \\
  Die Reihe f\"ur$\varphi_j$ konvergiert lokal gleichm\"a{\ss}ig(14.3). \\
  $\stackrel{\text{8.4}}{\Rightarrow} \wegint \varphi_j(z) dz = \sum\limits_{n=1}^{\infty}
  a_{-n}^{(j)} \wegint (z-z_j)^{-n} dz$. Sei $n \in \{2,3,4, \ldots\}$. Die
  Funktion $\frac{1}{(z-z_j)^n}$ hat auf $G \backslash \{z_j\}$ die Stammfunktion
  $\frac{(z-z_j)^{-n+1}}{-n+1}$ \\ $\stackrel{8.6}{\Rightarrow}$ $\wegint
  (z-z_j)^{-n} dz = 0$ $\forall n \in \{2,3,4,\ldots\}$ \\ $\Rightarrow$ $\wegint
  \varphi_j dz = a_{-1}^{(j)} \wegint \frac{1}{(z-z_j)} dz = a_{-1}^{(j)}$
  $n(\gamma, z_j) 2 \pi \ie$
\end{beweis}

%17.2 Folgerung
\begin{folgerung} 
  $G \subseteq \MdC$ sei ein Elementargebiet, es sei $f \in H(G)$ und $\gamma$ sei
  ein geschlossener, st\"uckweise glatter Weg mit $Tr(\gamma) \subseteq G$.\\
  Dann: 
  \begin{liste}
    \item \begriff{Cauchyscher Integralsatz f\"ur Elementargebiete} \\ \\
    \centerline{$\wegint f(z) dz = 0$}
    \item \begriff{Cauchysche Integralformel} \\ \\
    \centerline{$n(\gamma, z) f(z) = \frac{1}{2 \pi \ie} \wegint \frac{f(w)}{w-z}
    dw$ $\forall z \in G \backslash \text{Tr}(\gamma)$}
  \end{liste}
\end{folgerung}

\begin{beweis}
  \begin{liste}
    \item Alle $z_j$ in 17.1 sind hebbare Singularit\"aten.
    $\stackrel{\text{14.4}}{\Rightarrow}$ $Res(f(z_j)) = 0$ $\Rightarrow$
    Behauptung. 
    \item Sei $z_0 \in G \backslash \text{Tr}(\gamma)$. $g \in H(G \backslash
    \{z_0\})$ sei definiert durch $g(w) := \frac{f(w)}{w-z_0}$. Sei $r > 0$, so dass
    $U_r(z_0) \subseteq G$\\ $\stackrel{10.4}{\Rightarrow}$ $f(w) = a_0+a_1(w-z_0) +
    \ldots$ $\forall w \in U_r(z_0)$ \\ $\Rightarrow $ $g(w) = \frac{a_0}{w-z_0} +
    a_1 + a_2(w-z_0) + \ldots $ $\forall w \in \dot{U}_r(z_0)$ \\
    $\Rightarrow $ $Res(g, z_0) = a_0 = f(z_0)$ \\ 
    $\Rightarrow$ $\frac{1}{2 \pi \ie} \wegint \frac{f(w)}{w-z} dw = \frac{1}{2 \pi \ie
    } \wegint g(w) dw \stackrel{\text{17.1}}{=} n(\gamma, z_0) f(z_0)$
  \end{liste}
\end{beweis}

\underline{F\"ur die Berechnung von Residuen an Polstellen}

%17.3 Lemma
\begin{satz}
  Sei $D \subseteq \MdC$ offen, $z_0 \in D$, $f \in H(D \backslash \{z_0\})$ und
  $f$ habe in $z_0$ einen Pol der Ordnung $m \geq 1$. \\
  Es existiert also (siehe 13.2) ein $g \in H(D)$ mit: \\
  \centerline{$f(z) = \frac{g(z)}{(z-z_0)^m }$ $\forall z \in D \backslash \{z_0\}$} und
  $g(z_0) \neq 0.$ Dann: \\
  \begin{liste}
    \item Res$(f,z_0) = \frac{g^{(m-1)}(z_0)}{(m-1)!}$
    \item Ist $m=1$, so ist Res$(f, z_0) = \lim\limits_{z \to z_0} (z-z_0) f(z)$
  \end{liste}
\end{satz}
\begin{beweis}
  \begin{liste}
    \item Sei $r > 0$ so, dass $U_r(z_0) \subseteq D$. \\ 
    $\stackrel{10.4}{\Rightarrow}$ $g(z) = b_0 + b_1(z-z_0) + \ldots + b_m(z-z_0)^m
    + \ldots$ $\forall z \in U_r(z_0)$ \\
    $\Rightarrow$ $f(z) = \frac{b_0}{(z-z_0)^m} + \ldots + \frac{b_{m-1}}{(z-z_0)} + b_m +
    b_{m+1}(z-z_0) + \ldots$ $\forall z \in \dot{U}_r(z_0)$ $\Rightarrow$ Res$(f, z_0) =
    b_{m-1} \stackrel{\text{10.4}}{=} \frac{g^{(m-1)}(z_0)}{(m-1)!}$
    \item Aus (1) folgt: Res$(f, z_0) = g(z_0) = \lim\limits_{z\to z_0} g(z) =
    \lim\limits_{z\to z_0} (z-z_0) f(z)$
  \end{liste}
\end{beweis}
\begin{beispiel}
  \begin{liste}
    \item 
    $f(z) = \frac{1}{(z-i)(z+1)}$ hat in $z = i$ und in $z = -1$ jeweils einen
    Pol der Ordnung 1. Also: Res$(f, i) = \lim\limits_{z\to i} (z-i) f(z) =
    \frac{1}{i+1} = \frac{1}{2}- \ie \frac{1}{2}$; Res$(f, -1) = -\frac{1}{2}+ \ie
    \frac{1}{2}$ 
    \item $f(z) = \frac{1}{(z-i)^3 z}$ hat in $z = i$ einen Pol der Ordnung 3 und in
    $z = 0$ eine Pol der Ordnung 1. Hier ist $g(z) = \frac{1}{z}.$ $g'(z) = -
     \frac{1}{z^2}, g''(z) = \frac{2}{z^3}$ $\Rightarrow$ Res$(f,i) = \frac{2}{i^3
    2!} = i$
  \end{liste}
\end{beispiel}

%17.4 Satz
\begin{satz} [Das Argumentenprinzip]
  $G \subseteq \MdC$ sei ein ElementarGebiet, es sei $f \in M(G)$, $f$ habe in $G$ genau die
  Pole $b_1, \ldots, b_m$ (je der Pole sei so oft aufgef\"uhrt, wie seine Ordnung
  angibt), $f$ habe in $G$ genau die Nullstellen $a_1, \ldots , a_n$ (jede
  Nullstelle sei so oft aufgef\"uhrt, wie ihre Ordnung angibt) und $\gamma$ sei ein
  st\"uckweise glatter und geschlossener Weg mit Tr$(\gamma) \subseteq G \backslash
  \{b_1, \ldots, b_m, a_1, \ldots, a_n\}$. Dann: \\ \\
  \centerline{$\frac{1}{2 \pi \ie} \wegint \frac{f'(z)}{f(z)} dz =
  \sum\limits_{j=1}^{n} n(\gamma, a_j) - \sum\limits_{j=1}^{m} n(\gamma, b_j)$}
\end{satz}

\begin{bemerkung}
  \begin{liste}
    \item in 17.4 ist $\{b_1,\ldots, b_m\} = \emptyset$ oder $\{a_1,\ldots, a_n\} =
    \emptyset$ zugelassen. I.d.Fall: $ \sum\limits_{j=1}^{m} n(\gamma, b_j) = 0$ oder
    $\sum\limits_{j=1}^{n} n(\gamma, a_j) = 0$
    \item $n(\gamma, a_j) =  n(\gamma, b_k)$ $(j=1,\dots,n, k = 1,\dots,m)$. Dann: \\
    $\frac{1}{2 \pi \ie} \wegint \frac{f'(z)}{f(z)} dz = $ Anzahl der Nullstellen
    von $f$ - Anzahl der Polstellen von $f$ (jeweils gez\"ahlt mit Vielfachheiten!)
  \end{liste}
\end{bemerkung}

\begin{beispiel}
  $f(z) = \frac{z}{(z-i)^2}$ $n = 1, a_n=0, m = 2, b_1 = b_2 = i; \gamma(t) = 2
  e^{it}$ $t \in [0,2 \pi]$. $\frac{1}{2 \pi \ie} \wegint \frac{f'(z)}{f(z)} dz = 1
  -2 = -1 $
\end{beispiel}
\begin{beweis} (von 17.4)
  Sei $\beta_1, \ldots, \beta_p$ die paarweise verschiedenen Pole von $f$ $( p
  \leq m)$ und $\alpha_1, \ldots , \alpha_q$ die paarweise verschiedenen
  Nullstellen $(q \leq n)$. \\
  $h := \frac{f'}{f}$. \\ Dann: $h \in H( G \backslash \{\alpha_1, \ldots ,
  \alpha_q,\beta_1, \ldots, \beta_p\})$. \\ Dann: \\
  $\frac{1}{2 \pi \ie} \wegint \frac{f'(z)}{f(z)} dz = \frac{1}{2 \pi \ie}
  \wegint h(z) dz$ $\stackrel{\text{17.1}}{=} \sum\limits_{j=1}^{q} n(\gamma,
  \alpha_j) \text{Res}(n, \alpha_j) + \sum\limits_{j=1}^{p} n(\gamma,
  \beta_j) \text{Res}(n,\beta_j)$. \\
  Sei $\alpha \in \{\alpha_1, \ldots , \alpha_q\}$, $\beta \in \{\beta_1, \ldots,
  \beta_p\}$, $\nu = $ Ordnung der Nullstelle von $\alpha$ von $f$ und $\mu = $
  Ordnung der Polstelle $\beta$ von $f$. \\ \\
  Zu zeigen: Res$(h, \alpha) = \nu$ und Res$(h, \beta) = -\mu$. \\
  $\stackrel{11.8}{=}$ $\exists \delta > 0: U_{\delta}(\alpha) \subseteq G$,
  $\exists \varphi \in H(U_{\delta}(\alpha))$ und $f(z) =
  (z-\alpha)^{\nu}\varphi(z)$ $\forall z \in U_{\delta}(\alpha)$ und $\varphi(z)
  \neq 0$ $\forall z \in U_{\delta}(\alpha)$. \\ \\
  Dann: $f'(z)=\nu (z-\alpha)^{\nu -1}\varphi(z) + (z-\alpha)^{\nu}\varphi'(z)$
  $\forall z \in U_{\delta}(\alpha)$\\ $\Rightarrow$ $h(z) = \frac{f'(z)}{f(z)} =
  \frac{\nu}{z - \alpha}+ \underbrace{\frac{\varphi'(z)}{\varphi(z)}}_{\text{holomorph
  auf} \ U_{\delta}(\alpha)} $ $\forall z \in U_{\delta}(\alpha)$ $\Rightarrow$
  Res$(h, \alpha) = -\nu$. \\
  Analog: Res$(h, \beta) = \mu$ (statt 11.8 nimmt man 13.2)
\end{beweis}

%17.5 Folgerungen
\begin{folgerungen}
  Sei $G \subseteq \MdC$ ein Gebiet, $z_0 \in G$, $r > 0$, $\overline{U_r(z_0)}
  \subseteq G$, $\gamma(t) = z_0 + r e^{\ie t}$ $(t \in [0, 2 \pi])$ und $f,g \in
  H(G)$. Sei $N_f := $ Anzahl der Nullstellen von $f$ in $U_r(z_0)$ (gez\"ahlt mit
  Vielfachheiten!).  
  \begin{liste}
    \item Ist $f(z) \neq 0$ $\forall z \in$ Tr$(\gamma)$ $\Rightarrow$
     $N_f = \frac{1}{2 \pi \ie} \wegint \frac{f'(z)}{f(z)} dz $ 
    \item \textbf{Satz von Rouch\'{e}} \\
     Gilt (*) $|g(z)- f(z)| < |f(z)|$ $\forall z \in \text{Tr}(\gamma)$, so gilt 
     $N_f = N_g$
  \end{liste} 
\end{folgerungen}
\begin{beweis}
  \begin{liste}
    \item $\exists R > r: \overline{U_r(z_0)} \subseteq \overline{U_R(z_0)}
    \subseteq G$. Also: $\overline{U_r(z_0)} \subseteq U_R(z_0)$. $U_R(z_0)$ ist ein
    Elementargebiet. Seien $a_1, \ldots, a_n$ die Nullstellen von $f$ in $U_R(z_0)$.
    (gez\"ahlt mit Vielfachheiten).\\
    $\stackrel{\text{17.4}}{\Rightarrow}$ $\frac{1}{2 \pi \ie} \wegint
    \frac{f'(z)}{f(z)} dz = \sum\limits_{j=1}^n \underbrace{n(\gamma, a_j)}_{
    \stackrel{\text{16.2}}{=}
    \begin{cases} 
    	1 & , a_j \in U_r(z_0) \\ 
	0 & ,a_j \not\in U_r(z_0)
    \end{cases}
    }$
    \item F\"ur $s \in [0,1]: h_s := f + s(g-f) \in H(G)$; $N(s) := N_{h_s}$. Aus (*)
    folgt $h_s(z) \neq 0$ $ \forall s \in [0,1]$ $\forall z \in Tr(\gamma)$. \\
    Aus (1): $N(s)= \frac{1}{2 \pi \ie} \wegint \frac{h_s'(z)}{h_s(z)} dz =  
    \frac{1}{2 \pi \ie} \wegint \frac{f'(z)+s(g'(z)-f'(z))}{f(z) + s(g(z)-f(z))} dz$
    \\ $\Rightarrow$ die Funtion $s \mapsto N(s)$ ist stetig. Wegen $N(s) \subseteq
    \MdN_0$ $\forall s \in [0,1]$: $N(s)$ ist konstant. Also $N_f = N(0) = N(1) = N_g$
  \end{liste}
\end{beweis}

%17.6
\begin{satz} [Satz  von Hurwitz]
  $G \subseteq \MdC$ sei ein Gebiet. $(f_n)$ sei eine Folge in $H(G)$ und $(f_n)$
  konvergiert auf G lokal gleichm\"a{\ss}ig gegen eine Funktion $f: G \to \MdC$.
  ($\stackrel{\text{10.5}}{\Rightarrow}$ $f \in H(G)$). \\ Dann:
  \begin{liste}
    \item Ist $Z(f_n) = \emptyset$ $\forall n \in \MdN$ $\Rightarrow$ $Z(f) =
    \emptyset$ oder $f \equiv 0$
    \item Sind alle $f_n$ auf $G$ injektiv $\Rightarrow$ $f$ ist auf $G$ injektiv
    oder $f$ ist auf $G$ konstant.
  \end{liste}
\end{satz}

\begin{beweis}
  \begin{liste}
    \item Sei $f \not \equiv 0$ auf $G$; $z_0 \in G$, $r > 0$ so, dass
    $\overline{U_r(z_0)} \subseteq G$ und $f(z) \neq 0$ $\forall z \in
    \overline{U_r(z_0)} \backslash \{z_0\} $. \\
    $\gamma(t) = z_0 + re^{\ie t}$ $(t \in [0, 2 \pi])$. $(f_n)$ konvergiert auf
    Tr$(\gamma)$ gleichm\"a{\ss}ig gegen $f$. $\stackrel{\text{10.5}}{\Rightarrow}$
    $(f_n')$ konvergiert auf Tr$(\gamma)$ gleichm\"a{\ss}ig gegen $f'$. \\
     \"Ubung: $(\frac{1}{f_n})$ konvergiert auf Tr$(\gamma)$ gleichm\"a{\ss}ig gegen
    $\frac{1}{f}$. \\
    Fazit: $(\frac{f_n'}{f_n})$ konvergiert auf Tr$(\gamma)$ gleichm\"a{\ss}ig gegen
    $(\frac{f'}{f})$. \\ \\
    \centerline{$\underbrace{\frac{1}{2 \pi \ie} \wegint \frac{f_n'}{f_n} dz}_{\stackrel{\text{17.5}}{N_{f_n}=0}} \to 
    \underbrace{\frac{1}{2 \pi \ie} \wegint \frac{f'}{f}
    dz}_{\stackrel{\text{17.5}}{N_{f}}} $} \\ Also: $N_f = 0$. Somit: $f(z_0) \neq
    0$ 
    \item Sei $z_0 \in G$. $g_n = f_n-f_n(z_0)$, $g := f - f(z_0)$. $\widetilde{G}:=G\backslash \{z_0\}$. Dann:\\
    $(g_n)$ konvergiert auf $\widetilde{G}$ lokal gleichm\"a{\ss}ig gegen g. $g_n(z)\neq 0 \forall z \in \widetilde{G}$\\
    (1) $\Rightarrow$ $g \equiv 0$ oder $g(z) \neq 0$ $\forall z \in
    \widetilde{G}$ $\Rightarrow$ $f$ ist auf $G$ konstant oder $f(z) \neq f(z_0)$
    $\forall z  \in G \backslash \{z_0\}$
   \end{liste}
\end{beweis}

\underline{Berechnung von Integralen}
%17.7 Satz
\begin{satz}
  Sei $R(x,y)= R(x+iy) = R(z)$ eine rationale Funktion ohne Pole auf $\partial\mathbb{D}$. Weiter sei 
  $R_1(z)=\frac{1}{iz}R(\frac{z+\frac{1}{z}}{2},\frac{z-\frac{1}{z}}{2i})$ und $M:=\{z\in\mathbb{D}:z$ ist ein Pol von $R_1\}$ (endlich)\\
  Dann:
  $$ \int\limits_0^{2\pi}R(\cos{t},\sin{t})dt=2\pi i \sum\limits_{z\in M} Res(R_1,z)$$
\end{satz}
\begin{beweis}

  $\iint_0^{2\pi}R(\cos{t},\sin{t})dt = \iint_0^{2\pi}\frac{1}{ie^{it}} R(\frac{e^{it}+e^{-it}}{2},\frac{e^{it}-e^{-it}}{2i}) ie^{it}dt$\\
  $= \wegint R_1(z)dz$, wobei $\gamma(t)=ie^{it}$ $(t\in[0,2\pi])$.\\
  Also: $\iint_0^{2\pi}R(\cos{t},\sin{t})dt = \wegint R_1(z)dz \stackrel{17.1}{=}2\pi i \sum\limits_{\text{z Pol von $R_1$}} \underbrace{n(\gamma, z)}_{
    =
     \begin{cases} 
       1 & , z \in M \\ 
       0 & ,z \not\in M
     \end{cases}
  }  Res(R_1,z)$.
\end{beweis}
%17.8 Satz
\begin{satz}
  $Z$ und $N$ seien Polynome. $R:=\frac{Z}{N}$ habe auf $\MdR$ keine Pole und es gelte (*) grad $N$ $\ge$ grad $Z+2 (\folgt \iint_\MdR R(x)dx$ konvergiert absolut). 
  Weiter sei $M:=\{z\in\MdC: Im z >0$, $z$ ist Pol von R\}. Dann:\\
  $\iint_{-\infty}^\infty R(x) dx = 2\pi i \sum\limits_{z\in M} Res(R,z)$
\end{satz}
\begin{beweis}
  (*) $\folgt \exists m \ge 0 \exists c>0: |R(z)| \le \frac{m}{|z|^2} \forall z\in\MdC$ mit $|z|>c$. (**)\\
  Sei $\delta > c$ so gross, dass alle Pole von R in $U_\delta(0)$ liegen.\\
  $\gamma_1(t) := t$ $(t\in[-\delta,\delta])$; $\gamma_2(t):=\delta e^{it}$ $(t\in[0,\pi])$ $\gamma := \gamma_1 \oplus \gamma_2$.\\
  $\wegint R(z)dz = \iint_{\gamma_1}R(z)dz+\iint_{\gamma_2}R(z)dz.$ \\
  $\iint_{\gamma_1}R(z)dz = \iint_{-\delta}^\delta \to \iint_{-\infty}^\infty R(t)dt$ $(\delta\to\infty)$.\\
  Sei $z\in\Tr(\gamma_2)$. Dann: $|z| = \delta>0$, also nach (**): $|R(z)| \le \frac{m}{|z|^2} = \frac{m}{\delta^2} \folgt |\iint_{\gamma_2}R(z)dz| \le \frac{m}{|z|^2}L(\gamma_2)\le\frac{m\pi\delta}{\delta^2}=\frac{m\pi}{\delta}$\\
  $\folgt \iint_{\gamma_2}R(z)dz \to 0$ $(\delta\to\infty)$. Dann: $\wegint R(z)dz \to \iint_{-\infty}^\infty R(x) dx$ $(\delta \to \infty)$. 17.1 $\folgt \wegint R(z)dz = 2\pi i 
  \sum\limits_{\text{z Pol von R}} \underbrace{n(\gamma, z)}_{
      =
       \begin{cases} 
       1 & , z \in M \\ 
       0 & ,z \not\in M
       \end{cases}
  }  Res(R,z)$.
.				
\end{beweis}

\end{document} 
