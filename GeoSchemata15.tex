\documentclass[11pt]{book}
\usepackage{amssymb}
\usepackage[utf8]{inputenc}
\usepackage[T1]{fontenc}
\usepackage{german}
\usepackage[german]{babel}
\usepackage{geometry}
\usepackage{paralist}
\usepackage[explicit]{titlesec}
\usepackage{amsmath}
\usepackage{setspace}
\usepackage{polynom}
\usepackage{stmaryrd}
\usepackage[arrow, matrix, curve]{xy}
\usepackage{float}
\usepackage{titletoc}
\usepackage{pgfplots}
\usepackage{graphicx} 
\usepackage{wrapfig}
\usepackage{mathabx}
\usepackage{ifthen}
\usepackage{caption}
\usepackage{subcaption}

% Meta-Daten fuer Latexki
\usepackage{latexki}
\lecturer{Prof. Dr. Frank Herrlich}
\semester{Wintersemester 2015/2016}
\scriptstate{complete}


\usepackage[amsmath,thmmarks]{ntheorem}


\newtheorem{theorem}{Satz}[section]
\newtheorem{lemma}[theorem]{Lemma}
\newtheorem{proposition}[theorem]{Proposition}
\newtheorem{corollary}[theorem]{Korollar}
\newtheorem{motivation}[theorem]{Motivation}
\newtheorem{definition}[theorem]{Definition}
\newtheorem{remark}[theorem]{Bemerkung}
\newtheorem{example}[theorem]{Beispiel}
\newtheorem*{exampleo}[theorem]{Beispiel}
\newtheorem{bemdefini}[theorem]{Bemerkung + Definition}
\newtheorem{definibem}[theorem]{Definition + Bemerkung}
\newtheorem{erinndefini}[theorem]{Erinnerung + Definiton}
\newtheorem{erinner}[theorem]{Erinnerung}
\newtheorem{propdefini}[theorem]{Proposition + Definiton}
\newtheorem{definiprop}[theorem]{Definiton + Proposition}
\newtheorem{folg}[theorem]{Folgerung}
\newtheorem{erinnbemer}[theorem]{Erinnerung + Bemerkung}

\newenvironment{proof}[1][\it Beweis.]{\begin{trivlist}
\item[\hskip \labelsep {#1}]}{\end{trivlist}}

\theoremstyle{nonumberbreak} 
\theoremseparator{:} 
\theoremindent0.5cm 
\theoremheaderfont{\scshape} 
\theorembodyfont{\normalfont} 
\theoremsymbol{\ensuremath{_\bigboxvoid}} 
\RequirePackage{amssymb} 
\qedsymbol{\ensuremath{_\bigboxvoid}}

\newenvironment{defin}[1][]{\ifthenelse{\equal{#1}{}}{\definition}{\definition[#1]}\rm}{\enddefinition}
\newenvironment{motiv}[1][]{\ifthenelse{\equal{#1}{}}{\motivation}{\motivation[#1]}\rm}{\endmotivation}
\newenvironment{pr}[1][]{\ifthenelse{\equal{#1}{}}{\proof}{\proof[#1]}\rm}{\endproof}
\newenvironment{ex}[1][]{\ifthenelse{\equal{#1}{}}{\example}{\example[#1]}\rm}{\endexample}
\newenvironment{exo}[1][]{\ifthenelse{\equal{#1}{}}{\exampleo}{\exampleo[#1]}\rm}{\endexampleo}
\newenvironment{bemdefin}[1][]{\ifthenelse{\equal{#1}{}}{\bemdefini}{\bemdefini[#1]}\rm}{\endbemdefini}
\newenvironment{erinndefin}[1][]{\ifthenelse{\equal{#1}{}}{\erinndefini}{\erinndefini[#1]}\rm}{\enderinndefini}
\newenvironment{erinnbem}[1][]{\ifthenelse{\equal{#1}{}}{\erinnbemer}{\erinnbemer[#1]}\rm}{\enderinnbemer}
\newenvironment{propdefin}[1][]{\ifthenelse{\equal{#1}{}}{\propdefini}{\propdefini[#1]}\rm}{\endpropdefini}
\newenvironment{definprop}[1][]{\ifthenelse{\equal{#1}{}}{\definiprop}{\definiprop[#1]}\rm}{\enddefiniprop}
\newenvironment{er}[1][]{\ifthenelse{\equal{#1}{}}{\erinner}{\erinner[#1]}\rm}{\enderinner}
\newenvironment{definbem}[1][]{\ifthenelse{\equal{#1}{}}{\definibem}{\definibem[#1]}\rm}{\enddefinibem}

\newcommand{\QED}{\hskip\hsize\hskip-\marginparwidth\hskip-\marginparsep\qedsymbol} 

\newcommand{\spec}{\mathrm{Spec} \hspace{1pt} }
\newcommand{\p}{\mathfrak{p}}
\newcommand{\q}{\mathfrak{q}}
\newcommand{\K}{k}
\newcommand{\LL}{\mathcal{L}}
\newcommand{\ideal}{\subseteq}
\newcommand{\ringe}{\underline{\mathrm{Ringe}}}
\newcommand{\topraum}{\underline{\mathrm{Top}}}
\newcommand{\affsch}{\underline{\mathrm{Aff. Sch.}}}
\newcommand{\Hom}{\mathrm{Hom} }
\newcommand{\sets}{\underline{\mathrm{Sets}}}
\newcommand{\Oxmod}{\underline{\mathcal{O}_X\textrm{-}\mathrm{Mod}}}
\newcommand{\Rmod}{\underline{R\textrm{-}\mathrm{Mod}}}
\newcommand{\AbX}{\underline{\mathcal{A}b}(X)}
\newcommand{\Ab}{\underline{\mathrm{Ab}}}
\newcommand{\kvr}{\underline{k\textrm{-}\mathrm{VR}}}
\newcommand{\grp}{\underline{\mathrm{Grp}}}
\newcommand{\bild}{\mathrm{Bild}\ }
\newcommand{\kernel}{\mathrm{Kern}\ }
\newcommand{\Proj}{\mathrm{Proj} \hspace{1.5pt}}




\titlecontents{chapter}[2em]{\addvspace{2pc}\bfseries}{\contentslabel{1.7em}}{}{\titlerule*[1.5pc]{}\contentspage}
\titlecontents{section}[4.5em]{}{\contentslabel{2.5em}}{}{\titlerule*[0.5pc]{.}\contentspage}

\titleformat{\subsection}{\normalfont\normalsize\bfseries}{}{0em}{#1 \thesubsection}
\titleformat{\section}{\normalfont\Large\bfseries}{}{0em}{\thesection  #1}
\renewcommand*\thechapter{\Roman{chapter}\quad}
\titlespacing{\chapter}{0pt}{*5}{*1.5}
\titlespacing{\section}{0pt}{*5}{20pt}
\titlespacing{\subsection}{0pt}{16pt}{0pt}
\geometry{a4paper, top=30mm, left=25mm, right=25mm, bottom=30mm, headsep=10mm, footskip=15mm}
\renewcommand{\labelenumi}{(\roman{enumi})}
\renewcommand{\labelenumii}{(\arabic{enumii})}
\setlength{\parindent}{0pt}
\newcommand{\slant}[2]{{\raisebox{.1em}{$#1$}\left/\raisebox{-.1em}{$#2$}\right.}}
\newcommand{\bigslant}[2]{{\raisebox{.2em}{$#1$}\left/\raisebox{-.2em}{$#2$}\right.}}
\usepackage{graphicx}
\newcommand\tabrotate[1]{\rotatebox{90}{#1\hspace{\tabcolsep}}}
\newcommand\verschiebung[1][-.75\normalbaselineskip]{\hspace{#1}}
\makeatletter
\renewcommand*{\env@matrix}[1][*\c@MaxMatrixCols c]{
  \hskip -\arraycolsep
  \let\@ifnextchar\new@ifnextchar
  \array{#1}}
\makeatother
\renewcommand{\chaptermark}[1]{ 
  \markboth{ 
     \MakeUppercase{\thechapter #1} 
  }{} 
} 
\renewcommand{\sectionmark}[1]{ 
  \markright{ 
     \MakeUppercase{\thesection#1} 
  } 
}

\usepackage{hyperref}



\begin{document}
\begin{titlepage}

\textrm{ }\\[64pt]

\begin{center}
{\fontsize{40}{40} \selectfont \textbf{Geometrie der Schemata}}
\end{center}
\textrm{ } \\[36pt]
\begin{center} \large{\textrm{gelesen von Prof. Dr. Frank Herrlich im Wintersemester 2015/16 am KIT}} \end{center}
\textrm{ } \\[320pt]
\begin{center} \large{\textit{Geschrieben in } \LaTeX \textit{ von Arthur Martirosian, arthur.martirosian.93@gmail.com}}\end{center}
\textrm{ }\\[24pt]
\begin{center} \large{\today} \end{center}

\end{titlepage}
\thispagestyle{empty}



\begin{spacing}{1.6}
\setcounter{tocdepth}{1}
\tableofcontents
\thispagestyle{empty}
\end{spacing}
\newpage



\begin{spacing}{1.4}
\thispagestyle{empty}


\chapter{Schemata} %KAPITEL I
\setlength\abovedisplayshortskip{0pt}
\setlength\belowdisplayshortskip{10pt}
\setlength\abovedisplayskip{10pt}
\setlength\belowdisplayskip{10pt}


\renewcommand*\thesection{§ \arabic{section}\quad}
\section{Affine Schemata} %PARAGRAPH 1
\renewcommand*\thesection{\arabic{section}}
\thispagestyle{empty}

Dieses Semester wollen wir das bereits in der algebraischen Geometrie Gelernte auf ein anderes, verallgemeinertes Konzept übertragen, das bereits mit schwächeren Voraussetzungen Gutes leistet.
Wir werden auf einen algebraisch abgeschlossenen Körper verzichten und uns lediglich auf Ringe und ihre Lokalisierungen beschränken. Ebenso werden wir, statt Punkte mit maximalen Idealen zu identifizieren, Primideale heranziehen. Wir erinnern uns an die Basiskonstruktionen:\\
Sei $I \subseteq k[X_1, \ldots, X_n]$ ein Ideal. Dann heißt $V=V(I)= \{x \in k^n \ \vert \ f(x)=0 \textrm{ für alle } f \in I\}$ \textit{affine Varietät}. Der zugehörige \textit{affine Koordinatenring} ist definiert als $k[V]:= \slant{k[X_1, \ldots, X_n]}{I}$.	Der Hilbertsche Nullstellensatz erlaubt folgende Identifizierung:
\setlength{\abovedisplayskip}{5.5pt}
\setlength{\belowdisplayskip}{5.5pt}
\begin{alignat*}{5}
\textrm{Punkte in }V \qquad &\longleftrightarrow && \qquad \textrm{maximale Ideale in } k[V] \\
x \qquad &\longleftrightarrow && \qquad \mathfrak{m}_x = \{f \in k[V] \ \vert \ f(x)=0 \}
\end{alignat*}
Die \textit{Zariski-Topologie} auf dem $k^n$ erklärt eine Teilmenge $V\subseteq k^n$ als abgeschlossen, falls es ein Ideal $I \subseteq k[X_1, \ldots, X_n]$ gibt mit $V=V(I)$. Das Verschwindungsideal von $V$ ist
\setlength{\abovedisplayskip}{5.5pt}
\setlength{\belowdisplayskip}{5.5pt}
\begin{alignat*}{5}
I(V)\ \ &=&& \ \ \{ f \in k[X_1, \ldots, X_n] \ \vert \ f(x)=0 \textrm{ für alle } x \in V \} \\
&=&& \ \ \{f \in k[X_1, \ldots, X_n] \ \vert \ f \in \mathfrak{m}_x \textrm{ für alle } x \in V \} \\
&=&& \ \ \bigcap_{x \in V} \mathfrak{m}_x 
\end{alignat*}

\begin{defin}   %%Definition 1.1
Sei $R$ ein Ring (kommutativ mit Eins stets vorausgesetzt). 
\begin{compactenum}
\item Das \textit{Spektrum} von $R$ ist 
$$\spec R := \{ \mathfrak{p} \subset R \ \vert \ \mathfrak{p} \textrm{ ist Primideal} \}.$$
\item Für $I \subseteq R$ heißt
$$V(I) :=\{ \mathfrak{p} \in \spec R \ \vert \ I \subseteq \mathfrak{p} \}$$ \textit{Verschwindungsmenge} von $I$.
\item Für $V \subseteq \spec R$ heißt 
$$I(V)= \bigcap_{\mathfrak{p} \in V } \mathfrak{p}$$
\textit{Verschwindungsideal} von $V$.
\end{compactenum}
\end{defin}


\begin{ex}   %%Beispiel
Sei $R=\mathbb{Z}$. Es gilt $\spec \mathbb{Z}= \{ ( 0 ) \} \cup \{ p \mathbb{Z} \ \vert \ p \textrm{ ist Primzahl} \}$. Für das von $6$ erzeugte Ideal ist die Verschwindungsmenge demnach
$$V(6) := V(( 6 ) ) = \{( 2 ), ( 3 ) \}.$$
Beachte: Die Primideale $( 2 ), ( 3 )$ werden nun als Punkte aufgefasst!
Für $I=(6,7)$ und $V=\{(2),(3),(7)\}$ gilt
$$V(I)  = V( ( 6,7) ) = \emptyset$$
$$I(V)=I(\{(2),(3),(7)\})= (2) \cap (3) \cap (7)= ( 42).$$
Weiter halten wir fest:
$$I( \spec R) \ = \ \bigcap_{\mathfrak{p} \in \textrm{spec}R} \mathfrak{p} \ =\ \sqrt{( 0 )},$$
wobei $\sqrt{( 0 )}$ das \textit{Nilradikal} bezeichnet.
\end{ex}


\begin{remark}  %%Bemerkung 1.2
Die $V(I)$, $I\subseteq R$, bilden die abgeschlossenen Mengen einer Topologie auf $\spec R$, die \textit{Zariski-Topologie}.
\begin{pr}
Wir rechnen die Axiome einer Topologie nach:
\begin{compactenum}
\item Es gilt $V(R)= \emptyset$.
\item Weiter ist $V\left( \sqrt{ ( 0 ) }\right) = \spec R$.
\item Für Ideale $I_i, \subseteq R$ für $i \in J$ gilt
$$\bigcap_{i \in J} V(I_i) \ = \ \bigcap_{i \in J} \{ \p \in \spec R \ \vert \ I_i \subseteq \p \} \ = \ \{ \p \in \spec R \ \vert \ I_i \subseteq \p \textrm{ für alle } i \in J \} \ = \ V \left( \sum_{i \in J} I_i \right).$$
\item Seien $I_1, I_2 \ideal R$. Dann ist
$$V(I_1) \cup V(I_2) \ = \ \{ \p \in \spec R \ \vert \ I_1 \subseteq \p \textrm{ oder } I_2 \subseteq \p \}  \ \overset{(*)}{=} \ \{ \p \in \spec R \ \vert \ I_1 \cap I_2 \subseteq \p \} \ = \ V(I_1 \cap I_2).$$ 
Zu $(*)$: Sei $I_1 \nsubseteq \p$. Dann wähle $f \in I_1 \setminus \p$. Dann ist $fg \in \p$ für alle $g \in I_2$, also $g \in \p$ für alle $g \in I_2$, also $I_2 \subseteq \p$. $\hfill \Box$
\end{compactenum}
\end{pr}
\end{remark}

\begin{proposition} %%Proposition 1.3
Es ergeben sich folgende Identitäten:
\begin{compactenum}
\item Für jede Teilmenge $V \subseteq \spec R$ gilt $V\left(I\left(V\right)\right) = \overline{V}$.
\item Für jedes Ideal $I \subseteq R$ gilt $I\left(V\left(I\right)\right)= \sqrt{I}$.
\end{compactenum}
\begin{pr}
Übung.
\end{pr}

\begin{ex} Sei $R=\mathbb{R}[X]$, $I=( X^2+1)$. Dann ist $I$ ein Primideal in $\mathbb{R}[X]$. Per Definition folgt damit $V(I)= \{I\}$, also $I(V(I))= I(\{I\}) = I$. Die Konstruktionen der algebraischen Geometrie liefern diese Behauptung nicht!
\end{ex}

\begin{ex}
Betrachte den Ring $R:=k[X]_{( X )} = \mathcal{O}_{\mathbb{A}^1(k)}$. Es gilt
$$\spec R = \{ ( 0 ), ( X ) = \mathfrak{m} \}.$$
Allgemeiner lässt sich sagen: Ist $S$ ein diskreter Bewertungsring, so ist $\spec S = \{( 0 ), \mathfrak{m} \}$. Die Abgeschlossenen Teilmengen von $\spec R$ sind gerade $\emptyset, \spec R, V(\mathfrak{m}) = \{\mathfrak{m}\}, V(( 0 )) = \spec R$. Die Zariski-Topologie ist also gegeben durch $\mathcal{Z}=\{ \emptyset,\ \spec R,\ \{\mathfrak{m} \} \}$. Weiter ist $\overline{\{ ( 0 ) \} } = \spec R$.
\end{ex}

\begin{bemdefin}   %%Bem. + Defi 1.4
\begin{compactenum}
\item Sind $\p, \q \ideal R$ Primideale mit $\p \subseteq \q$, so ist $\q \in \overline{ \{\p\} }$.
\item Ist $R$ nullteilerfrei, so ist $\overline{\{ ( 0 ) \}} = \spec R$.
\item Ein Punkt $x \in X$ eines topologischen Raums mit $\overline{\{ x \}} = X$ heißt \textit{generischer Punkt}.
\item Eine Teilmenge $Y\subseteq \spec R$ enthält einen generischen Punkt genau dann, wenn $Y$ irreduzibel ist.
\item Die maximalen irreduziblen Teilmengen von $\spec R$ sind die minimalen Primideale in $R$.
\end{compactenum}
\begin{pr}
\begin{compactenum}
\item Ist $I \ideal R$ mit $\p \in V(I)$, so gilt $I \subseteq \p \subseteq \q$, also auch $\q \in V(I)$. Damit ist 
$$\q \in \bigcap_{I \subseteq \p} V(I) = \bigcap_{A \supseteq V(\{\p\})=\p} A =\overline{ \{ \p \} }.$$
\item Folgt direkt aus (i) und $( 0 ) \in \spec R$.
\item[(iv)] Sei zunächst $V$ irreduzibel. Ohne Einschränkung gelte $V=V(I)$ für ein Ideal $I \ideal R$. Nach Voraussetzung ist dann $I= \p$ für ein $\p \in \spec R$. Damit ist $\p$ ein generischer Punkt.\\
Sei nun $\spec R=:V=V(I_1) \cup V(I_2)$ mit $V(I_1) \neq V \neq V(I_2)$. Wäre $\p$ ein generischer Punkt, so wäre ohne Einschränkung $\p \in V(I_1)$, also $V(I_1) \supseteq \overline{ \{ \p \} } = \spec R = V$, ein Widerspruch zur Reduzibilität von $V$.

\end{compactenum}
\end{pr}

\end{bemdefin}




\begin{remark}   %%Bem. 1.5

Seien $R,R'$ Ringe, $\alpha: R \longrightarrow R'$ ein Ringhomomorphismus. Dann ist die von $\alpha$ induzierte Abbildung
$$f_{\alpha}: \spec \it{R'} \longrightarrow \spec \it{R}, \qquad \p \mapsto \alpha ^{-1}(\p)$$
stetig. Wir erhalten einen kontravarianten Funktor
 $$ \ringe \longrightarrow \topraum, \qquad \{R, \alpha\} \dashrightarrow \{ \spec R, f_{\alpha} \}.$$
\begin{pr}
Sei $V(I) \subseteq \spec R$ abgeschlossen. Es gilt
\setlength{\abovedisplayskip}{5.5pt}
\setlength{\belowdisplayskip}{5.5pt}
\begin{alignat*}{5}
f_{\alpha}^{-1}\left(V(I)\right) \ \ &=&& \ \ \{ \p \in \spec R' \ \vert \ f_{\alpha}(\p) = \alpha^{-1}(\p) \in V(I) \} \\
&=&& \ \ \{ \p \in \spec R' \ \vert \ \alpha^{-1}(\p) \supseteq I \}\\
&=&& \ \ \{ \p \in \spec R' \ \vert \ \p \supseteq \alpha(I) \}\\
&=&& \ \ V\left( \alpha(I) \right),
\end{alignat*}
womit die Stetigkeit von $f_{\alpha}$ folgt. $\hfill \Box$
\end{pr}
\end{remark}
\end{proposition}

\begin{remark}  %Bem. 1.6
Sei $k$ algebraisch abgeschlossen, $V \subseteq \mathbb{A}^n(\K)$ affine Varietät. Dann ist 
$$\mu: V \longrightarrow \spec \left( \K[V]\right), \qquad \it{x} \mapsto \mathfrak{m}_{\it{x}}$$
injektiv und stetig.
\begin{pr}
Injektivität ist klar, denn verschiedene Punkte haben verschiedene maximale Ideale. Ist nun $I \ideal \K[V]$ ein Ideal, so ist
$$\mu^{-1}\left(V(I)\right) = \mu^{-1}\left( \{ \mathfrak{m} \ideal \K[V] \ \vert \ I \subseteq \mathfrak{m} \} \right) = V(I) \subseteq V,$$
womit die Stetigigkeit folgt. $\hfill \Box$
\end{pr}

\end{remark}

\begin{remark}   %%Bem. 1.7
Sei $R$ ein Ring, $f\in R$. Definiere
$$D(f):= \spec \it{R} \setminus V(f) = \{ \p \in \spec \it{R} \ \vert \ f \notin \p \}.$$
Dann bilden die $D(f)$ eine Basis der Zariski-Topologie auf $\spec \it{R}$.
\begin{pr}
Sei $U \subseteq \spec R$ offen, $\p \in U$. Gesucht: $f$ mit $\p \in D(f) \subseteq U$. Sei $V=V(I)= \spec R \setminus U$ für ein Ideal $I \ideal R$. Da $\p \notin V(I)$, ist $I \nsubset \p$. Sei also $f \in I \setminus \p$. Dann gilt $\p \in D(f)$. Außerdem ist $f \in \q$ für alle $\q \in \spec R$ mit $I \subseteq \q$, also $V(f) \supseteq V(I)$. Damit folgt insgesamt 
$$D(f) \subseteq \spec R \setminus V(I) = U,$$
die Behauptung. $\hfill \Box$
\end{pr}
\end{remark}

\begin{proposition}    %Prop. 1.8
Für jeden Ring $R$ ist $\spec R$ quasikompakt.
\begin{pr}
Sei $\{U_i\}_{i \in J}$ eine offene Überdeckung von $X:= \spec R$. Ohne Einschränkung gelte $U_i = D(f_i)$ für ein $f_i \in R$ für alle $i \in J$. Nach Voraussetzung ist 
$$\bigcup_{i \in J} U_i \ = \ \bigcup_{i \in J} D(f_i) \ = \ X, \qquad \bigcap_{i \in J} V(f_i) \ = \ V \left( \sum_{i \in J} ( f_i ) \right) = \emptyset,$$
also 
$$R = I(\emptyset) = I \left( V \left( \sum_{i \in J } ( f_i ) \right) \right) = \sqrt{ \sum_{i \in J} ( f_i )}.$$
Wir finden also $i_1, \ldots, i_n \in J$ sowie $\lambda_1, \ldots, \lambda_n \in R$ mit 
$$1 = \sum_{k=1}^n \lambda_k f_{i_k}.$$
Damit ist 
$$X= \bigcup_{k=1}^n D\left(f_{i_k}\right)$$
und $X$ ist quasikompakt. $\hfill \Box$


\end{pr}
\end{proposition}

\newcommand{\bigcupdot}{\overset{\bf{.}}{\bigcup}}

\begin{erinndefin}    %% erinn + Defi 1.9
Sei $R$ ein Ring, $\p \in \spec R=:X$. 
\begin{compactenum}
\item Wir definieren durch 
$$R_{\p} := \left\{ \frac{f}{g} \ \bigg \vert \ f \in R, g \in R \setminus \p \right\}$$
die \textit{Lokalisierung} von $R$ nach $\p$, wobei wir $\frac{f}{g}$ statt $(f,g)_{\sim}$ schreiben. Es gilt
$$(f,g) \sim (f', g')\ \Longleftrightarrow \ h\left(fg'-f'g\right) = 0 \textrm{ für ein } h \in R\setminus \p$$
\item Sei $U \subseteq X$ offen. Eine Abbildung 
$$s: U \longrightarrow \bigcup_{ \p \in U}^{.} R_{\p}, \qquad \q \mapsto s(\q) \in R_{\q}$$
heißt \textit{regulär} in $\p \in U$, falls es eine Umgebung $U_0 \subseteq U$ von $\p$ und $f,g \in R$ gibt sodass für alle $\q \in U_0$ gilt $g \notin \q$ und $s(\q) = \frac{f}{g}.$
\item Die Menge
$$\mathcal{O}_X(U) := \left \{ s: U \longrightarrow \bigcup_{\p \in U}^{.} R_{\p} \ \bigg\vert \ s \textrm{ ist regulär in allen }\p \in U \right\}$$
heißt \textit{Ring der regulären Funktionen auf }$U$.
\item Die Zuordnung $U \mapsto \mathcal{O}_X(U)$ ist eine Garbe von Ringen auf $X=\spec R$.
\item Das Paar $(X, \mathcal{O}_X)$ heißt \textit{affines Schema}.
\item Für $x=\p \in X$ heißt
$$\mathcal{O}_{X,x} = \bigslant{\{ (U,s) \ \vert \ x \in U\subseteq X \textrm{ offen }, s \in \mathcal{O}_{X}(U) \} }{\sim}$$
\textit{Halm der Strukturgarbe in }$x$, wobei 
$$(U,s) \sim (U',s') \ \Longleftrightarrow \ \textrm{ Es gibt } U'' \subseteq U \cap U' \textrm{ mit } x \in U'' \textrm{ und } s \vert_{U''}= s'\vert_{U''}.$$
$\mathcal{O}_{X,x}$ ist ein lokaler Ring.
\end{compactenum}
\end{erinndefin}

\begin{proposition}   %% Prop. 1.10
Sei $R$ ein Ring, $X= \spec\it{R}$ sowie $(X,\mathcal{O}_X)$ das dazugehörige affine Schema.
\begin{compactenum}
\item Für jedes $x=\p \in X$ ist $\mathcal{O}_{X,x} \cong R_{\p}$.
\item Für jedes $f \in R$ ist $\mathcal{O}_X\left(D(f) \right) \cong R_f$.
\end{compactenum}
\begin{pr}
\begin{compactenum}
\item Definiere
$$\psi: \mathcal{O}_{X,x} \longrightarrow R_{\p}, \qquad [(U,s)]_{\sim} \mapsto s(\p).$$
Dann ist $\psi$ wohldefinierter Ringhomomorphismus, denn für $(U,s) \sim (U',s')$ gilt $s(\p)=s'(\p)$.\\
\textit{injektiv.} Sei $[(U,s)] \in \mathcal{O}_{X,x}$ mit $\psi\left([(U,s)]\right) = 0$, also $s(\p)=0$. Ohne Einschränkung sei für alle $\q \in U$
$$s(\q)= \frac{a}{f} \in R_{\q} \qquad \textrm{ für geeignete } a,f \in R, f \notin \q.$$
Da $s(\p)=0$, gibt es also $h \in R \setminus \p$ mit $h \cdot a = 0$ in $R$. Damit ist aber $\frac{a}{f}=0$ in $R_{\q}$ für alle $\q \in D(h) \cap U \ni x$, also gerade $s(\q)=0$ für alle $\q \in D(H) \cap U$. Es folgt
$$[(U,s)] = 0 \quad \textrm{ in } \mathcal{O}_{X,x},$$
wie gewünscht.\\
\textit{surjektiv.} Sei $\frac{a}{f} \in R_{\p}$ für $a \in R, f \in R\setminus \p$. Sei $U:= D(f)$, also $x \in U$. Setze für alle $\q \in U$
$$s(\q) := \frac{a}{f}  \ \in R_{\q}.$$
Dann gilt $s \in \mathcal{O}_X(U)$ sowie $s(\p)= \frac{a}{f} \in R_{\p}$, also 
$$\psi\left([(u,s)]\right) = \frac{a}{f},$$
womit die Surjektivität von $\psi$ und damit die Behauptung folgt.

\item Definiere nun 
$$\phi: R_f \longrightarrow \mathcal{O}_X\left( D(f)\right), \qquad \frac{a}{f^n} \mapsto \left[ \left(D(f), \frac{a}{f^n} \right) \right].$$
\textit{injektiv.} Sei $\phi\left( \frac{a}{f^n}\right) =0$. Dann gibt es für jeden Punkt $\p \in D(f)$ ein $h_{\p} \in R \setminus \p$ mit $h_{\p} \cdot a = 0$ in $R$. Sei 
$$\mathfrak{a}:= \textrm{Ann}(a) = \{ r \in R \ \vert \ r \cdot a = 0 \} \ideal R$$
das Annulatorideal von $a$. Es gilt $\mathfrak{a} \nsubset \p$ für alle $\p \in D(f)$, also $V(\mathfrak{a}) \subseteq V(f)$.  Dann ist aber 
$$I(V(f)) \subseteq I(V(\mathfrak{a})) = \sqrt{\mathfrak{a}},$$
also $f \in \sqrt{\mathfrak{a}}$. Damit gibt es ein $m \in \mathbb{N}$ mit $f^m \in \mathfrak{a}$, also gerade $a f^m =0$. Damit gilt $$\frac{a}{f^n} = 0 \quad \textrm{ in } R_f,$$
was zu zeigen war.\\
\textit{surjektiv.}
 Sei $s \in \mathcal{O}_X\left(D(f)\right)$, d.h. für jedes $x= \p \in D(f)$ gibt es ein $U_x \subseteq D(f)$ sowie $a_{\p} \in R, h_{\p} \in R \setminus \p$ mit
 $$s(\q) = \frac{a_{\p}}{h_{\p}} \qquad \textrm{ für alle } \q \in U_x.$$
Da $D(f)$ quasikompakt ist, sei ohne Einschränkung 
$$D(f) = \bigcup_{i=1}^r U_i, \qquad U_i = D(h_i)$$
sowie für alle $i \in \{1, \ldots, r\}$
$$s= \frac{a_i}{h_i} \qquad \textrm{ auf } D(h_i).$$
Auf $D(h_i) \cap D(h_j)$ ist $\frac{a_i}{h_i} = \frac{a_j}{h_j}$, das heißt wir finden $n \in \mathbb{N}$ mit
$$(h_i h_j)^n \left( a_ih_j - a_j h_i \right) = 0 \qquad \textrm{ in } R.$$
Damit gilt
$$h_j^{n+1} h_i^n a_i - h_i^{n+1}h_j^n a_j ) =0.$$
Setze
$$\tilde{h}_j := h_j^{n+1}, \quad \tilde{h}_i := h_i^{n+1}, \quad \tilde{a}_i := a_i h_i^n, \quad \tilde{a_j} := a_j h_j^n$$
Dann wird die Gleichung zu 
$$\tilde{h}_j \tilde{a}_i - \tilde{h}_i \tilde{a}_j = 0 \quad \Longleftrightarrow \quad s = \frac{\tilde{a}_i}{\tilde{h}_i} = \frac{\tilde{a}_j}{\tilde{h}_j}.$$
Nun ist
\setlength{\abovedisplayskip}{5.5pt}
\setlength{\belowdisplayskip}{5.5pt}
\begin{alignat*}{5}
D(f) = \bigcup_{i=1}^r D(\tilde{h}_i) \quad &\Longrightarrow && \quad f^m \in \sum_{i=1}^m ( \tilde{h}_i ) \ \textrm{ für ein } m \geqslant 1 \\
& \Longrightarrow && \quad f^m = \sum_{i=1}^m b_i \tilde{h}_i \ \textrm{ für geeignete } b_i \in R
\end{alignat*}
Setze $a:= \sum_{i=1}^r b_i \tilde{a}_i$. Dann gilt für $1 \leqslant j \leqslant m$
$$a \tilde{h}_j = \sum_{i=1}^r b_i \tilde{h}_j \tilde{a}_i = \sum_{i=1}^r b_i \tilde{h}_i \tilde{a}_j = \tilde{a}_j f^m,$$
also $$s= \frac{a}{f^m}$$
auf ganz $D(f)$, womit die Behauptung gezeigt ist. $\hfill \Box$
\end{compactenum}

\end{pr}


\end{proposition}


\begin{proposition}    %% Prop. 1.11
Sei $\phi: R \longrightarrow R'$ ein Ringhomomorphismus, $U \subseteq X = \spec\it{R}$ offen, $X' = \spec\it{R'}$ sowie 
$$f_{\phi}: X' \longrightarrow X, \qquad \q \mapsto \phi^{-1}(\q)$$
die von $\phi$ induzierte stetige Abbildung, d.h $U'= f_{\phi}^{-1}(U)$ ist offen in $X'$. Dann induziert $\phi$ einen Ringhomomorphismus
$$\phi_U: \mathcal{O}_X(U) \longrightarrow \mathcal{O}_{X'}(U'), \qquad s \mapsto \phi_U(s) = \phi_{\p} \left( s \left( f_{\phi}(\q) \right) \right) \in R'_{\q},$$
wobei $\phi_{\p}$ gegeben ist durch 
$$\phi_{\p}: R_{\p} \longrightarrow R_{\phi(\p)}, \qquad \frac{g}{h} \mapsto \frac{\phi(g)}{\phi(h)}.$$
\end{proposition}


\renewcommand*\thesection{§ \arabic{section}\quad}
\section{Garben} %PARAGRAPH 2
\renewcommand*\thesection{\arabic{section}}

\begin{defin}   %%Definition 2.1
 Sei $X$ ein topologischer Raum, $\mathcal{C}$ eine Kategorie. Eine \textit{Prägarbe} $\mathcal{F}$ auf $X$ mit Werten in $\mathcal{C}$ ist eine Zuordnung $U \mapsto \mathcal{F}(U)$, die jeder offenen Teilmenge $U \subseteq X$ ein Objekt $\mathcal{F}(U)$ in $\mathcal{C}$ sowie für je zwei offene $U\subseteq U' \subseteq X$ einen Morphismus $\rho^{U'}_U: \mathcal{F}(U') \longrightarrow \mathcal{F}(U)$ in $\mathcal{C}$ zuordnet, sodass gilt
\begin{compactenum}
\item Für alle $U \subseteq X$ offen gilt $\rho^{U}_U = \textrm{id}_{\mathcal{F}(U)}$.
\item Für alle $U \subseteq U' \subseteq U'' \subseteq X$ offen gilt $\rho^{U''}_U = \rho^{U'}_{U} \circ \rho^{U''}_{U'}$.
\end{compactenum}
\end{defin}

\begin{remark}  %%bemerkung 2.2
Sei $\underline{\text{Off}}(X)$ die Kategorie der offenen Teilmengen von $X$ mit den Inklusionen als Morphismen, also 
$$\mathrm{Mor}_{\underline{\mathrm{Off}}(X)}(U,V) = \begin{cases} \ \{ \iota: U \hookrightarrow V \}, & \ \textrm{ falls }U \subseteq V \\ \ \emptyset, & \ \textrm{ sonst.} \end{cases} $$
Dann ist eine Prägarbe $\mathcal{F}$ auf $X$ mit Werten in $\mathcal{C}$ ein kontravarianter Funktor $\mathcal{F}: \underline{\text{Off}}(X) \longrightarrow \mathcal{C}$.
\end{remark}

\begin{defin}   %%Definition 2.3
Eine Prägarbe $\mathcal{F}$ auf einem topologischen Raum $X$ mit Werten in $\mathcal{C}$ heißt \textit{Garbe}, falls folgendes gilt: Für jedes offene $U \subseteq X$, jede offene Überdeckung $\{U_i\}_{i \in I}$ von $U$ und jede Familie $s_i \in \mathcal{F}(U_i)$ mit $\rho^{U_i}_{U_i \cap U_j}(s_i) = \rho^{U_j}_{U_i \cap U_j} (s_j)$ für alle $i,j \in I$ gibt es genau ein $s \in \mathcal{F}(U)$ mit $\rho^{U}_{U_i}(s) = s_i$ für alle $i \in I$.
Anders ausgedrückt: Eine Prägarbe $\mathcal{F}$ ist eine Garbe, falls für eine beliebige offene Teilmenge $U \subseteq X$ und eine offene Überdeckung $\{U_i \}_{i \in I}$ von $U$ in der Sequenz
$$\mathcal{F}(U) \ \  \overset{p_0}{\longrightarrow} \ \  \prod_{i \in I} \mathcal{F}(U_i) \ \ \underset{p_2}{\overset{p_1}{\rightrightarrows}} \ \ \prod_{i,j \in I} \mathcal{F}(U_i \cap U_j).$$
$\mathcal{F}(U)$ der \textit{Equalizier} von $p_1$ und $p_2$ sowie $p_0$ injektiv ist. Die erste Forderung sichert uns, dass lokale Daten, welche auf den Schnitten der $U_i$ übereinstimmen, von globalen Schnitten herkommen, während die Forderung der Injektivität von $p_0$ die Eindeutigkeit impliziert.
\end{defin}

\begin{ex}   %%Beispiel 
\begin{compactenum}
\item Für eine quasiprojektive Varietät $V$ ist $\mathcal{O}_V$ eine Garbe auf $V$.
\item Für ein affines Schema $X=\spec R$ ist $\mathcal{O}_X$ eine Garbe auf $X$.
\item Die Zuordnung $U \mapsto \mathcal{C}(U) := \{f: U \longrightarrow \mathbb{R} \ \vert \ f \textrm{ ist stetig } \}$ ist eine Garbe auf jedem topologischen Raum $X$.
\item Sei $\mathcal{O}_X$ wie in (i) und (ii). Setze
$$\mathcal{O}_X^{\times}(U) := \{ f \in \mathcal{O}_X(U) \ \vert \ f \textrm{ ist Einheit } \}.$$
Dann ist $\mathcal{O}_X^{\times}$ ein Garbe von abelschen Gruppen (betrachte die Abbildung $f \mapsto \exp(f)$).
\item Sei $X$ ein topologischer Raum, $G$ eine Gruppe. Setze $\mathcal{G}(U) := G$ für alle $U \subseteq X$ offen sowie $\rho^{U}_{U'} = \textrm{id}_G$ für alle $U' \subseteq U \subseteq X$ offen. Dann ist $\mathcal{G}$ Prägarbe, aber keine Garbe! Betrachte hierfür $U= U_1 \overset{.}{\cup} U_2$.
\end{compactenum}
\end{ex}

\begin{remark} %%Bemerkung 2.4
Ist $\mathcal{F}$ Garbe von abelschen Gruppen auf $X$, so ist $\mathcal{F}( \emptyset) = \{0 \}$.
\begin{pr}
Überdecke $\emptyset$ durch $\{U_i\}_{i \in \emptyset}$. Für die (konsistente!), leere Familie der $\{s_i \in \mathcal{F}(U_i) \}_{i \in \emptyset}$ gibt es genau ein $s \in \mathcal{F}(\emptyset)$ mit $\rho^{U_i}_{U_i \cap U_j}(s_i) = \rho^{U_j}_{U_i \cap U_j}(s_j)$ für alle $i,j \in \emptyset$. Damnit muss $\vert \mathcal{F}(\emptyset) \vert =1$ gelten. $\hfill \Box$
\end{pr}
\end{remark}

\newcommand{\offx}{\underline{\mathrm{Off}}(X)}

\begin{defin}  %%Definition 2.5
Sei $X$ toplogischer Raum, $\mathcal{F}, \mathcal{G}$ Prägarben auf $X$. Ein \textit{Morphismus} $\phi: \mathcal{F} \longrightarrow \mathcal{G}$ ist eine natürliche Transformation zwischen $\mathcal{F}$ und $\mathcal{G}$, d.h. für jedes $U \in \textrm{Ob} \left( \offx \right)$ ist ein Morphismus $\phi_U: \mathcal{F}(U) \longrightarrow \mathcal{G}(U)$ gegeben, sodass das folgende Diagramm kommutiert:
$$
\begin{xy}
\xymatrix{
\mathcal{F}(U') \ar[rrr]^{\phi_{U'}} \ar[dd]^{\rho^{U'}_{U}} &&& \mathcal{G}(U') \ar[dd]^{\rho^{U'}_{U}} \\
&&& \\
\mathcal{F}(U) \ar[rrr]^{\phi_U} &&& \mathcal{G}(U)
}
\end{xy}
$$

\end{defin}

\begin{bemdefin} %%Def. + Bem. 2.6
Sei $\mathcal{F}$ eine Prägarbe von abelschen Gruppen auf $X$, $x \in X$.
\begin{compactenum}
\item Wir definieren den \textit{Halm von} $\mathcal{F}$ \textit{in} $x$ durch
$$\mathcal{F}_x := \slant{\{ (U,f) \ \vert \ x \in U \subseteq X, s \in \mathcal{F}(U) \}}{\sim}$$
wobei
$$(U,f) \sim (U',f') \quad \Longleftrightarrow \quad \textrm{Es gibt } U'' \subseteq U \cap U', \textrm{ sodass } \rho^{U}_{U''}(f) = f \vert_{U''} = f' \vert_{U''} = \rho^{U'}_{U''} (f').$$
$\mathcal{F}_x$ ist eine abelsche Gruppe für alle $x \in X$.
\item Für jedes $x \in U \subseteq X$ offen ist 
$$p_x^{U}: \mathcal{F}(U) \longrightarrow \mathcal{F}_x, \qquad f \mapsto [(U,f)] =: f_x$$
ein Gruppenhomomorphismus. $f_x$ heißt \textit{Keim von} $f$ in $x$.
\item $\mathcal{F}_x$ ist der Kolimes des filtrierten Diagramms der $\mathcal{F}(U), x\in U$, das heißt: Ist $G$ abelsche Gruppe und $\phi_U: \mathcal{F}(U) \longrightarrow G$ Gruppenhomomorphismus für jede offene Teilmenge $U\subseteq X$ mit $x \in U$, sodass $\phi_{U'} = \phi_U \circ\rho_{U}^{U'}$ für alle $x \in U \subseteq U' \subseteq X$ offen, so gibt es genau einen Homomorphismus $\phi_x: \mathcal{F}_x \longrightarrow G$, sodass $\phi_U = \phi_x \circ p_x^{U}$ für alle $ x \in U$.
\item Sei $\psi: \mathcal{F} \longrightarrow \mathcal{G}$ Homomorphismus von Prägarben abelscher Gruppen auf $X$. Dann induziert $\psi$ für jedes $x \in X$ einen Gruppenhomomorphismus $\psi_x: \mathcal{F}_x \longrightarrow \mathcal{G}_x$.
\end{compactenum}

\end{bemdefin}

\begin{remark} %& Bemerkung 2.7
Sei $\mathcal{F}$ eine Garbe abelscher Gruppen auf, $U \subseteq X$ offen, $s \in \mathcal{F}(U)$. Dann gilt
$$s=0 \quad \Longleftrightarrow \quad s_x =0 \quad \textrm{ für alle } x \in U.$$
\begin{pr}
Wir zeigen die Notwendigkeit. Es gebe also zu jedem $x \in U$ eine Umgebung $U_x$ mit $s\vert_{U_x}=0$, also $\rho^{U}_{U_x}(s)=0$. Die $\{U_x\}_{x \in U}$ bilden eine offene Überdeckung von $U$, die $\{s \vert_{U_x}\}_{x \in U}$ sind eine konsistente  Familie von Schnitten. Nach Definition 1.3 gibt es genau ein $t \in \mathcal{F}(U)$ mit $t \vert_{U_x} = s \vert_{U_x}$ für alle $x$. Da $0 \in \mathcal{F}(U)$ diese Eigenschaft besitzt, folgt aus der Eindeutigkeit von $s$ bereits $s=0$.
\end{pr}
\end{remark}

\newcommand{\F}{\mathcal{F}}
\newcommand{\G}{\mathcal{G}}

\begin{proposition}
Seien $\mathcal{F}, \mathcal{G}$ Garben abelscher Gruppen auf $X$, $\phi: \mathcal{F} \longrightarrow \mathcal{G}$ Garbenmorphismus. Dann gilt
\begin{compactenum}
\item 
$\phi_U$ ist injektiv für alle $U \subseteq X$ offen genau dann, wenn $\phi_x$ injektiv ist für alle $x \in X$.
\item Ist $\phi_U$ surjektiv für alle $U \subseteq X$ offen, so ist $\phi_x$ surjektiv für alle $x \in X$.
\item $\phi_U$ ist bijektiv für alle $U \subseteq X$ offen genau dann, wenn $\phi_x$ bijektiv ist für alle $x \in X$.
\end{compactenum}
\begin{pr}
Wir zeigen die Notwendigkeit von (i) und (iii).
\begin{compactenum}

\item Sei $U \subseteq X$ offen, $s\in \mathcal{F}(U)$ mit $\phi_U(s) = 0$.
Dann gilt $\phi_x(s_x) =0$ für alle $x\in U$, also $s_x = 0$.
Damit folgt bereits $s = 0$.

\item[(iii)] Sei $U \subseteq X$ offen, $g \in \mathcal{G}(U)$. Nach Voraussetzung gibt es dann für jedes $x \in U$ ein $s_x \in \F_x $ mit $\phi_x(s_x) = g$.
Also finden wir einen Repräsentanten $\left(U_x, s^{(x)}\right)$ von $s_x$ mit $\phi_{U_x}(s^{(x)})=g\vert_{U_x}$.\\
Die $\{U_x\}_{x \in U}$ bilden eine offene Überdeckung von $U$, (i) liefert also Injektivität von $\phi_{U_x}$. Damit ist $\{s^{(x)}\}_{x\in U}$ eine konsistente Familie, es gibt also $s\in \F(U)$ mit $ s\vert_{U_x} = s^{(x)}$ für alle $x \in U$. Wegen 
$$\phi_U(s)\vert_{U_x} = \phi_{U_x}(s^{(x)})= g\vert _{U_x} \qquad \textrm { für alle } x \in U$$ gilt $\phi_U (s) = g$. $\hfill\Box$

\end{compactenum}

\end{pr}
\end{proposition}

\begin{ex}
Das Beispiel soll den fehlenden Pfeil in der vorangegangenen Proposition füllen. Betrachte den topologischen Raum $X= \mathbb{C}$, die Garbe $\mathcal{O}$ der holomorphen Funktionen auf $\mathbb{C}$ sowie die Garbe $\mathcal{O}^{\times}$ der invertierbaren holomorphen FUnktionen auf $\mathbb{C}$. Betrachte nun die Abbildung
$\phi: \mathcal{O} \longrightarrow \mathcal{O}^{\times}$, welche für offene Mengen $U \subseteq \mathbb{C}$ wie folgt definiert sei:
$$\qquad \phi_U: \mathcal{O}(U) \longrightarrow \mathcal{O}^{\times}(U), \qquad f \mapsto \mathrm{exp}(2\pi i f).$$
Dann ist $\phi$ ein Garbenmorphismus und die auf den Halmen induzierten Homomorphismen $\phi_z$ für $z \in \mathbb{C}$ sind surjektiv (denn lokal ist eine Funktion ohne Nullstellen invertierbar), auf $U:= \mathbb{C}\setminus \{0\}$ ist $\phi_U$ aber nicht surjektiv!
\end{ex}


\begin{propdefini} %%Propdefin 2.9

Sei $X$ ein topologischer Raum, $\mathcal{F}$ eine Prägarbe auf $X$.

\begin{compactenum}
\item Es gibt eine Garbe $\mathcal{F}^{+}$ auf $X$ und einen Morphismus $\theta: \mathcal{F} \longrightarrow \mathcal{F}^{+}$ von Prägarben, sodass $\theta_x: \mathcal{F}_x \longrightarrow \mathcal{F}_x^{+}$ für jedes $x \in X$ ein Isomorphismus ist.
\item $\mathcal{F}^+$ und $\theta$ aus (i) sind eindeutig bis auf Isomorphie. $\mathcal{F}^+$ heißt \textit{die zu} $\mathcal{F}$ \textit{assoziierte Garbe}.
\item Wir haben folgende universelle Abbildungseigenschaft: Zu jeder Garbe $\mathcal{G}$ auf $X$ und jeden Morphismus $\phi: \mathcal{F} \longrightarrow \mathcal{G}$ von Prägarben gibt es einen eindeutigen Morphismus $\phi^+: \F^+ \longrightarrow \G$, sodass gilt $\phi = \phi^+ \circ \theta$.

\end{compactenum}

\begin{pr} 
\begin{compactenum}
\item Für $U \subseteq X$ sei
$$\mathcal{F}^+(U) := \left\{ s:U \longrightarrow \bigcup_{x \in U}^{.} \mathcal{F}_x, s(x) \in\mathcal{F}_x \ \big\vert \ \forall x \in U \textrm{ ex. } U_x \ni x \textrm{ und } f \in \mathcal{F}(U_x) \textrm{ mit } s \vert _{U_x} = f \right\}.$$
Dann ist $\mathcal{F}^{+}$ offensichtlich Garbe. Definiere weiter $\theta$ für $U \subseteq X$ offen mit $x \in U$ durch
$$\theta_{U}: \F(U) \longrightarrow \G(U), \qquad f(x) \mapsto f_x.$$
Dann ist $\theta$ Morphismus von Prägarben und $\theta_x$ Isomorphismus für alle $x \in X$.
\item Folgt aus (iii).
\item Für $U \subseteq X$ sei $\phi_U^+: \mathcal{F}^+(U) \longrightarrow \G(U)$ wie folgt definiert: Für $s \in \mathcal{F}^+(U)$ und $x \in U$ seien $U_x$ und $f^{(x)} \in \mathcal{F}(U_x)$ wie in der Definition von $\F^+(U)$. Dann ist $\phi_{U_x}\left( f^{(x)} \right) \in \G(U_x)$. Weiter bilden die $\{U_x\}_{x \in U}$ eine offene Überdeckung von $U$ und es gilt
$$\phi_{U_x} \left( f^{(x)} \right) \vert _{U_x \cap U_y} \ = \ \phi_{U_y}\left(f^{(y)}\right) \vert_{U_x \cap U_y}.$$
Da $\G$ Garbe ist, existiert ein eindeutiges 
$$H:= \phi_U^+(s) \in \G(U) \textrm{ mit }h \vert_{U_x} = \phi_{U_x} \left(f^{(x)}\right) \textrm{ für alle } x \in X,$$ was zu zeigen war. $\hfill \Box$
\end{compactenum}
\end{pr}

\end{propdefini}

\begin{bemdefini} %% Bem.+ Def. 2.10
Sei $\phi: \F \longrightarrow \G$ Morphismus von Garben abelscher Gruppen auf $X$.
\begin{compactenum}
\item Die Prägarbe $U \mapsto \ker \phi_U$ ist eine Garbe, genannt $\rm{Kern}\ \phi$.
\item $\phi$ heißt \textit{Monomorphismus} oder \textit{injektiv}, falls $\ker \phi = 0$.
\item $\rm{Bild} \ \phi$ sei die zu $U \mapsto \rm{im } \phi_U$ assoziierte Garbe.
\item $\phi$ heißt \textit{Epimorphismus} oder \textit{surjektiv}, falls $\rm{Bild} \phi = \G$.
\item $\phi$ ist \textit{Epimorphismus} genau dann, wenn $\phi_x$ surjektiv ist für alle $x \in X$.

\end{compactenum}

\begin{pr}

\begin{compactenum}

\item Sei $U \subseteq X$ offen, $\{U_i\}_{i \in I}$ ist offene Überdeckung von $U$, $s_i \in \ker \phi_{U_i}$ eine konsistente Familie. Da $\F$ Garbe ist, existiert ein eindeutiges $s \in \F(U)$ mit $s \vert_{U_i}= s_i$. Für jedes $x \in X$ sei $i \in I$ mit $x \in U_i$, dann ist $\phi_x(s_x) = \left( \phi_{U_i}(s_i) \right)_x = 0$, also ist $\left( \phi_U(s) \right)_x = \phi_x(s_x) = 0$ für alle $x \in X$, nach 2.8 also $\phi_U(s)=0$. 

\item[(v)] Es gilt
$$\textrm{Bild} \phi = \G \quad \Longleftrightarrow \quad \textrm{Bild} \phi_x = \G_x \textrm{ für alle }x \in X \quad \Longleftrightarrow \quad \phi_x \textrm{ surjektiv für alle } x \in X,$$
es folgt also unmittelbar die Behauptung. $\hfill \Box$

\end{compactenum}

\end{pr}
\end{bemdefini}

\begin{defin}  %%Def. 2.11
Seien $\G \subseteq \F$ Garben abelscher Gruppen auf $X$. Dann ist $U \mapsto \slant{\F(U)}{\G(U)}$ eine Prägarbe (i.A. keine Garbe!). Die dazu assoziierte Garbe $\slant{\F}{\G}$ heißt \textit{Quotientengarbe}.
\end{defin}

  %%Beispiel
\begin{ex} 
Sei $X= \mathbb{S}^1$, $\F = \mathcal{C}_X = \{f: X \longrightarrow \mathbb{R} \ \vert \ f \textrm{ ist stetig } \}$ und $\G$ die konstante Garbe $\G= \mathbb{Z}$. Wähle
$$U_1:= \left\{ \left( \cos u, \sin u\right) \ \big\vert \ u \in \left[- \frac{\pi}{4}, \frac{5 \pi}{4} \right] \right\}, \qquad U_2:= \left\{ \left( \cos u, \sin u\right) \ \big\vert \ u \in \left[\frac{3\pi}{4}, \frac{ \pi}{4} \right] \right\}.$$
Dann ist $U_1 \cap U_2= D_1 \overset{.}{\cup} D_2$, es gilt also $\G(U_1 \cap U_2) = \mathbb{Z}^2$.\\
Sei nun $f_1 \in \F(U_1)$ mit $f_1 \vert_{D_1} = 0$ und $f_1 \vert_{D_2} = 1$ sowie $f_2 \in \F(U_2)$ mit $f_2 = 0$. Dann gilt $f_1 \vert_{U_1 \cap U_2} \in \G ( U_1 \cap U_2)$. Daraus folgt $\overline{f}_1 = \overline{f}_2 = 0$ in $\slant{\F(U_1 \cap U_2)}{\G(U_1 \cap U_2)}$.
Damit bilden $\overline{f}_1 \in \slant{\F(U_1)}{\G(U_1)}$ und $\overline{f}_2 \in \slant{\F(U_2)}{\G(U_2)}$ eine konsistente Familie in der Prägarbe $U \mapsto \slant{\F(U)}{\G(U)}$ bezüglich der Überdeckung $X=U_1 \cup U_1$. 
Allerdings gibt es kein $f \in \slant{\F(X)}{\G(X)}= \slant{\F(X)}{\mathbb{Z}}$ mit $f \vert_{U_1} = \overline{f}_1$ und $f \vert_{U_2} = \overline{f}_2 = 0$, mit anderen Worten, $\slant{\F}{\G}$ ist keine Garbe.
\end{ex}

\newcommand{\abgrup}{\underline{\mathrm{Ab}}}
\newcommand{\garben}{\underline{\mathrm{Sh}}}



\begin{proposition}  %%Prop. 2.12
Sei $X$ topologischer Raum, $\underline{\rm{Sh}}(X)$ die Kategorie der Garben abelscher Gruppen auf $X$.
\begin{compactenum}
\item Für jedes $x \in X$ ist die Zuordnung $\mathcal{F} \longrightarrow \F_x$ ein kovarianter, exakter Funktor
$$\Phi_x: \garben(X) \longrightarrow \abgrup, \qquad \left( \phi: \F \longrightarrow \G\right) \mapsto \left( \phi_x: \F_x \longrightarrow \G_x\right).$$
\item Für jedes $U \in \underline{\rm{Off}}(X)$ ist die Zuordnung $\F \longrightarrow \F(U)$ ein kovarianter, linksexakter Funktor
$$\Phi_U: \garben(X) \longrightarrow \abgrup, \qquad \left( \phi: \F \longrightarrow \G \right) \mapsto \left( \phi_U: \F(U) \longrightarrow \G(U) \right).$$
\end{compactenum}
\begin{pr}
Sei 
$$0 \longrightarrow \F' \overset{\phi}{\longrightarrow} \F \overset{\psi}{\longrightarrow} \F'' \longrightarrow 0$$
eine kurze exakte Sequenz in $\underline{\textrm{Sh}}(X)$, d.h. $\phi$ ist Monomorphismus, $\psi$ ist Epimorphismus und es gilt $\textrm{Bild} \phi = \textrm{Kern} \psi$. 
\begin{compactenum}
\item Wir erhalten eine kurze Sequenz in $\underline{\textrm{Ab}}$:
$$0 \longrightarrow \F_x' \overset{\phi_x}{\longrightarrow} \F_x \overset{\psi_x}{\longrightarrow} \F_x'' \longrightarrow 0.$$
Nach 2.9 und 2.11 ist diese exakt.
\item Wir erhalten eine Sequenz 
$$0 \longrightarrow \F'(U) \overset{\phi_U}{\longrightarrow} \F(U) \overset{\psi_U}{\longrightarrow} \F''(U).$$
Diese ist exakt, da $\phi_U$ nach 2.11 injektiv ist. Außerdem gilt $\textrm{Bild} \phi_U = \textrm{Kern} \psi_U$, also 
$$\textrm{Bild} \phi(U) \overset{(*)}{=} \textrm{Bild} \phi_U = \textrm{Kern} \phi_U = \textrm{Kern} \phi(U),$$
wobei $(*)$ aus der Injektivität von $\phi$ folgt.  $\hfill \Box$
\end{compactenum}
\end{pr}

\end{proposition}


\begin{bemdefin}   %%Bem + Def. 2.13

Sei $f: X \longrightarrow Y$ stetige Abbildung topologischer Räume.
\begin{compactenum}
\item Sei $\mathcal{F}$ Garbe abelscher Gruppen auf $X$. Dann ist die Prägarbe $V \mapsto \mathcal{F}\left(f^{-1}(V)\right)$ auf $Y$ eine Garbe $f_{*}\mathcal{F}$. Sie heißt \textit{direkte Bildgarbe} von $\mathcal{F}$.
\item Sei $\mathcal{G}$ Garbe abelscher Gruppen auf $Y$. $f^{-1}\mathcal{G}$ sei die zur Prägarbe $$U \mapsto \underset{f(U) \subseteq V \subseteq Y \textrm{offen}}{\underrightarrow{\textrm{lim}}} \mathcal{G}(V)$$ assoziierte Garbe. Sie heißt \textit{Urbildgarbe} von $\mathcal{G}$.
\item Wir haben kovariante Funktoren $f_{*}: \underline{\textrm{Sh}}(X) \longrightarrow \underline{\textrm{Sh}}(Y)$ und $f^{-1}: \underline{\textrm{Sh}}(Y) \longrightarrow \underline{\textrm{Sh}}(X)$.
\end{compactenum}

\begin{pr}
\begin{compactenum}
\item Sei $V \subseteq Y$ offen, $\{V_i\}_{i \in I}$ offene Überdeckung von $V$ sowie $\{s_i\}_{i \in I} \subseteq \{\mathcal{F}(V_i)\}_{i \in I}$ konsistente Familie. Dann ist $\{f^{-1}(V_i)\}_{i \in I}$ offene Überdeckung von $f^{-1}(V)$ und $s_i \in \mathcal{F}\left(f^{-1}(V_i)\right)$ eine konsistente Familie. Da $\mathcal{F}$ Garbe ist, existiert ein eindeutiges $s \in \mathcal{F}\left(f^{-1}(V)\right)=f_{*}\mathcal{F}(V)$ mit $s \vert_{f^{-1}(V)}=s_i$ für alle $i \in I$.
\item[(iii)] Es gilt:
\begin{compactenum}
\item[(a)] Sei $\phi: \mathcal{F} \longrightarrow \mathcal{F}'$ Garbenmorphismus in $\underline{\textrm{Sh}}(X)$. Setze
\setlength{\abovedisplayskip}{5.5pt}
\setlength{\belowdisplayskip}{5.5pt}
\begin{alignat*}{5}
\phi_{*}: f_{*}\mathcal{F}\ & \longrightarrow&& \ f_{*}\mathcal{F}' \\
V\ & \mapsto&& \ \left(\phi_*\right)_V: \left( \mathcal{F}\left(f^{-1}(V)\right) = f_{*}\mathcal{F}(V) \longrightarrow f_{*}\mathcal{F}'(V) = \mathcal{F}'\left( f^{-1}(V)\right)  \right) 
\end{alignat*}
\item[(b)] Sei $\phi: \mathcal{G} \longrightarrow \mathcal{G}'$ Garbenmorphismus in $\underline{\textrm{Sh}}(Y)$. Setze
\setlength{\abovedisplayskip}{5.5pt}
\setlength{\belowdisplayskip}{5.5pt}
\begin{alignat*}{5}
f^{-1}\phi: f^{-1}\mathcal{G}\ & \longrightarrow&& \ f^{-1}\mathcal{G}&&&&\\
U \ & \mapsto&& \ \left(f^{-1}\phi\right)_U : f^{-1} \mathcal{G}(U)\ &&=&& \ \underset{f(U) \subseteq V \subseteq Y \textrm{ offen}}{\underrightarrow{\textrm{lim}}} \mathcal{G}(V) \\
&&&&&=&& \ \underset{f(U) \subseteq V \subseteq Y \textrm{ offen}}{\underrightarrow{\textrm{lim}}} \mathcal{G}(V) = f^{-1}\mathcal{G}'(U),
\end{alignat*}
wobei die letzte Gleichung aus der UAE des Kolimes folgt. Insgesamt folgt dann die Behauptung. $\hfill \Box$

\end{compactenum}
\end{compactenum}

\end{pr}

\end{bemdefin}




\begin{proposition}  %%Prop. 2.14
Für jede stetige Abbildung $f: X \longrightarrow Y$ von topologischer Räumen sind die Funktoren
$$f_{*}: \garben(X) \longrightarrow \garben(Y),  \qquad f^{-1}: \garben(Y) \longrightarrow \garben(X)$$
zueinander adjungiert. Genauer: $f^{-1}$ ist linksadjungiert zu $f_*$, bzw. $f_*$ ist rechtsadjungiert zu $f^{-1}$, d.h. die Funktoren $\Hom_{\underline{\rm{Sh}}(X)}\left( f^{-1}(\cdot), \cdot\right)$ und $\Hom_{\garben(Y)}\left(\cdot, f_{*}(\cdot)\right)$ sind isomorph als Funktoren von $\garben(Y) \times \garben(X)$ nach $\sets$. Das wiederum heißt: Für alle Garben $\mathcal{F} \in \garben(X)$ und $\mathcal{G} \in \garben(Y)$ gibt es eine Bijektion
$$\alpha_{\F, \G}: \Hom\left(f^{-1}\mathcal{G}, \F \right) \longrightarrow \Hom\left( \G, f_* \F\right),$$
sodass für alle Paare von Garbenmorphismen $\phi: \F \longrightarrow \F'$ und $\psi: \G \longrightarrow \G'$ das folgende Diagramm kommutiert:

$$
\begin{xy}
\xymatrix{
\Hom\left(f^{-1}\mathcal{G}', \mathcal{F} \right) \ar[rrr]^{\alpha_{\F, \G'}} \ar[dd]^{} &&& \Hom \left(\mathcal{G}', f_*\mathcal{F}\right) \ar[dd]^{} \\
&&& \\
\Hom\left(f^{-1}\mathcal{G}, \mathcal{F}'\right) \ar[rrr]^{\alpha_{\F, \G}} &&& \Hom\left(\G, f_+\F\right)
}
\end{xy}
$$

\begin{pr}
Zur Definition von $\alpha_{\F, \G}$ konstruieren wir einen Garbenmorphismus $\sigma_{\G}: \G \longrightarrow f_* f^{-1} \G$. Ist dann $\phi: f^{-1} \G \longrightarrow \F$, so sei 
$$\alpha_{\F, \G} = f_*(\phi) \circ \sigma_{\G}$$
die Komposition dieser zwei Morphismen. Zur Definition von $\sigma_{\G}$.: Für $V \subseteq Y$ ofen sei
\setlength{\abovedisplayskip}{5.5pt}
\setlength{\belowdisplayskip}{5.5pt}
\begin{alignat*}{5}
f_* f^{-1} \G(V) := f^{-1} \G \left(f^{-1}(V) \right) \ \ &=&& \ \ \underset{f(f^{-1}(V)) \subseteq W \subseteq Y \textrm{ offen}}{\underrightarrow{\textrm{lim}}} \mathcal{G}(W)
&=&& \ \ \underset{f(f^{-1}(V)) \subseteq W \subseteq V \textrm{ offen}}{\underrightarrow{\textrm{lim}}} \mathcal{G}(W)
\end{alignat*}

Für jedes solche $W$ gibt es eine Restriktionsabbildung $\G(V) \longrightarrow \G(W)$. Diese induzieren die Abbildung
$$\left(\sigma_{\G}\right)_V: \G(V) \longrightarrow \underrightarrow{\textrm{lim}}\  \G(W)$$
Übung: Nachrechnen, dass das Diagramm kommutiert.
Hinweis: Man muss für jeden Morphismus $\psi: \G \longrightarrow \G'$ zeigen, dass das folgende Diagramm kommutiert:

$$
\begin{xy}
\xymatrix{
& \G' \ar[rd]^{\sigma_{\G'}} & \\ 
\G \ar[rr] \ar[rd]_{\sigma_{\G}} \ar[ru]^{\psi} && f_{*} f^{-1} \G' \\ 
& f_* f^{-1} \G \ar[ru] & 
}
\end{xy}
$$
Es bleibt nun zu zeigen: $\alpha_{\F, \G}$ ist bijektiv. Konstruiere die Umkehrabbildung
$$\beta_{\F, \G}: \Hom\left( \G, f_{*} \F \right) \longrightarrow \Hom\left( f^{-1} \G, \F \right)$$
als Komposition von 
$$f^{-1}: \Hom\left(\G, f_* \F \right) \longrightarrow \Hom\left(f^{-1}\G, f^{-1} f_* \F \right)$$
und
$$\rho_{\F}: f^{-1} f_* \F \longrightarrow \F.$$
Betrachte dazu für $U\subseteq X$ offen
$$f^{-1}f_* \F(U) =  \underset{f(U) \subseteq V \subseteq Y \textrm{ offen}}{\underrightarrow{\textrm{lim}}} f_* \F(V) \ =\  \underset{f(U) \subseteq V \textrm{ offen}}{\underrightarrow{\textrm{lim}}} \F \left( f^{-1}(V) \right) .$$
Da $U \subseteq f^{-1} \left(f(U) \right) \subseteq f^{-1}(V)$ für alle $V$ mit $f(U) \subseteq V$, gibt es für jedes solche $V$ Restriktionsabbildungen
$\F\left(f^{-1}(V)\right) \longrightarrow \F(U)$. Mit der UAE des Kolimes erhalten wir
$$\left(\rho_{\F}\right)_U: \underset{f(U) \subseteq V \textrm{ offen}}{\underrightarrow{\textrm{lim}}} \F \left( f^{-1}(V) \right) \longrightarrow \F.$$
Übung: Zeige, dass $\alpha_{\F, \G}$ und $\beta_{\F, \G}$invers zueinander sind. $\hfill \Box$

\end{pr}

\end{proposition}



\renewcommand*\thesection{§ \arabic{section}\quad}
\section{Schemata} %PARAGRAPH 3
\renewcommand*\thesection{\arabic{section}}

\begin{defin}  %%defin. 3.1
Sei $X$ toplogischer Raum, $\mathcal{O}_X$ Garbe von Ringen auf $X$.
\begin{compactenum}
\item Das Paar $\left(X, \mathcal{O}_X\right)$ heißt \textit{geringter Raum}, $\mathcal{O}_X$ heißt \textit{Strukturgarbe}.
\item Ist für jedes $x \in X$ der Halm $\mathcal{O}_{X,x}$ ein lokaler Ring, so heißt $\left(X, \mathcal{O}_X\right)$ \textit{lokal geringter Raum}.
\end{compactenum}
\end{defin}

\begin{ex}
\begin{compactenum}
\item Ist $R$ ein Ring, $X= \spec R,$ und $\mathcal{O}_X$ die Garbe der regulären Funktionen auf $X$, so ist $\left(X, \mathcal{O}_X \right)$ ein geringter Raum.
\item Ist $X$ ein toplogischer Raum und $\mathcal{C}_X$ die Garbe der stetigen $\mathbb{R}$-wertigen Funktionen auf $X$, dann ist $\left(X, \mathcal{C}_X\right)$ ein lokal geringter Raum.
\end{compactenum}
\end{ex}

\begin{defin} %%Def. 3.2

Seien $\left(X, \mathcal{O}_X \right), \left(Y, \mathcal{O}_Y \right)$ geringte Räume.
\begin{compactenum}
\item Ein \textit{Morphismus geringter Räume} $f: \left(X, \mathcal{O}_X\right) \longrightarrow \left(Y, \mathcal{O}_Y\right)$ ist ein Paar $\left( f^{\textrm{top}}, f^{\#} \right)$ bestehend aus einer stetigen Abbildung $f^{\textrm{top}}: X \longrightarrow Y$ und einem Garbenmorphismus $f^{\#}: \mathcal{O}_Y \longrightarrow f_* \mathcal{O}_X$.
\item Sind $\left(X, \mathcal{O}_X\right), \left(Y, \mathcal{O}_Y\right)$ lokal geringte Räume, so verlangen wir für einen \textit{Morphismus lokal geringter Räume} zusätzlich, dass für jedes $x \in X$ der von $f^\#$ induzierte Gruppenhomomorphismus $f_x^{\#}: \mathcal{O}_{Y, f(x)} \longrightarrow \mathcal{O}_{X,x}$ lokal ist, also gilt
$$\left(f_x^{\#}\right)^{-1} \left(\mathfrak{m}_{f(x)}\right) = \mathfrak{m}_x.$$
\end{compactenum}
\end{defin}


\begin{ex}
Sei $R$ nullteilerfreier, lokaler Ring, $\mathfrak{m}_R \neq ( 0 )$ sowie $k:= \textrm{Quot}(R)$ und $\iota: R \longrightarrow k$. Dann ist $\iota$ kein lokaler Homomorphismus, denn es gilt:
$$\iota(\mathfrak{m}_R) = \mathfrak{m}_R \nsupseteq ( 0 )  = \mathfrak{m}_{k}.$$
Die zugehörige Abbildung der Spektren ist
$$f_{\iota}: \spec k \longrightarrow \spec R, ( 0 ) \mapsto ( 0 ).$$
Diese induziert auf den Halmen
$$\mathcal{O}_{\spec R, ( 0 ) } = k, \qquad \mathcal{O}_{\spec k, (0) } = k$$
die Identität $\textrm{id}_{k}$, das heißt $(f_{\iota}, \iota)$ ist lokaler Homomorphismus.
\end{ex}



\begin{proposition} %%Proposition 3.3
Seien $R,S$ Ringe. Dann entsprechen die Morphismen der lokal geringten Räume $\left(\spec S, \mathcal{O}_{\spec S}\right) \longrightarrow \left( \spec R, \mathcal{O}_{\spec R}\right)$ bijektiv den Ringhomomorphismen $R \longrightarrow S$. Die Zuordnung $R \mapsto \spec R$ ist also eine \textit{Antiäquivalenz} von Kategorien $\ringe \longrightarrow \affsch$.

\begin{pr}
Ist $f: \spec S \longrightarrow \spec R$ Morphismus lokal geringter Räume, so ist 
$$f^\#_{\spec R}: R=\mathcal{O}_{\spec R}\left( \spec R\right) \longrightarrow \mathcal{O}_{\spec S} \left( f^{-1}(\spec R) \right) = f_{*} \mathcal{O}_{\spec S} \left( \spec R \right)  = S$$
Ringhomomorphismus. Sei nun $\alpha: R \longrightarrow S$ Ringhomomorphismus. Dann ist 
$$f_{\alpha}: \spec S \longrightarrow \spec R, \qquad \q \mapsto \alpha^{-1}(\q)$$
stetig. Für jedes offene $U \subseteq \spec R$ induziert $\alpha$ nach 1.11 einen Ringhomomorphismus
$$\alpha_U: \left(f^\#_{\alpha}\right)_U: \mathcal{O}_{\spec R}(U) \longrightarrow \mathcal{O}_{\spec S}\left(f^{-1}(U)\right).$$
Ist $U = D(g)$ für ein $g \in R$, so ist 
$$\mathcal{O}_{\spec R}(U) = \mathcal{O}_{\spec R}\left(D(g)\right) = R_g$$
und 
$$\alpha_U\left(\frac{a}{g^n}\right) = \frac{\alpha(a)}{\alpha(g)^n} \in S_{\alpha(g)} = \mathcal{O}_{\spec S}\left(D(\alpha(g))\right) = \mathcal{O}_{\spec S} \left( f_{\alpha}^{-1}(D(g))\right) = f_{\alpha *} \mathcal{O}_{\spec S}\left(D(g)\right),$$

das heißt $\alpha_U$ induziert den gewünschten Garbenmorphismus. Auf den Halmen induziert $\alpha$ folgende Abbildung:\\
Sei $\q \in \spec S$, $\p:= f_{\alpha}(\q) = \alpha^{-1}(\q) \in \spec R$. Die maximalen Ideale in $\mathcal{O}_{\spec R, \p}$ bzw. $\mathcal{O}_{\spec S, \q}$ sind $\p R_{\p}$ bzw. $\q R_{\q}$. Für $a= \frac{b}{g} \in \mathfrak{m}_{\p}$, also $b \in \p$ und $g \in R \setminus \p$, ist dann 
$$\left(f_{\alpha}^\#\right)_{\p} (a) = \alpha_{\p} (a) =\frac{\alpha(b)}{\alpha(g)} \in \mathfrak{m}_{\q},$$
denn aus $b \in \p= \alpha^{-1}(\q)$ folgt $\alpha(b) \in \q$ und aus $g \notin \p= \alpha^{-1}(\q)$ folgt $\alpha(g) \notin \q$. Insgesamt liegt also ein Morphismus lokal geringter Räume vor. $\hfill \Box$
\end{pr}
\end{proposition}


\begin{defin} %%Definiiton 3.4
Ein lokal geringter Raum $(X, \mathcal{O}_X)$ heißt \textit{Schema}, falls es eine offene Überdeckung $\{U_i\}_{i \in I}$ von $X$ gibt, sodass $(U_i, \mathcal{O}_X(U_i))$ für jedes $i \in I$ als lokal geringter Raum isomorph zu einem affinen Schema $\left(\spec R, \mathcal{O}_{\spec R}\right)$ ist.
\end{defin}

\begin{ex}   %Beispiel

\begin{compactenum}
\item Ist $V$ affine Variteät über $k$, so ist $t(V)= \spec \K [V]$ affines Schema.
\item Sei $V$ quasiaffine Varietät. Überdecke $V$ durch affine Varietäten $V_i$ für $1 \leqslant i \leqslant r$. Klebe die $t(V_i)$ zum topologischen Raum $t(V)$ zusammen. $\mathcal{O}_{t(V)}$ sei dann die Garbe, die auf jedem $V_i$ mit $\mathcal{O}_{V_i}$ übereinstimmt.
\end{compactenum}
\end{ex}


\begin{proposition}  %%Proposition 3.5

Seien $\left(X, \mathcal{O}_X\right), \left(Y, \mathcal{O}_Y\right)$ Schemata, $U \subseteq X$, $V \subseteq Y$ offen und 
$$f: \left(U, \mathcal{O}_X \vert_U \right) \longrightarrow \left(V, \mathcal{O}_Y \vert_V \right)$$
ein Isomorphismus von Schemata. Sei $Z:= \slant{(X \overset{.}{\cup} Y)}{\sim}$ mit 
$$u \sim f(u) \quad \Longleftrightarrow \quad u \in U$$
die \rm{Verklebung von} $X$ \rm{und} $Y$ \rm{längs} \it $f$. Dann gibt es genau eine Garbe $\mathcal{O}_Z$ auf $Z$ mit 
$$\mathcal{O}_Z \vert_X = \mathcal{O}_X, \quad \mathcal{O}_Z \vert_Y = \mathcal{O}_Y.$$
Insbesondere ist $(Z, \mathcal{O}_Z)$ damit ein Schema.

\begin{pr}
Es ist $W \subseteq Z$ offen genau dann, wenn $W \cap X$ und $W \cap Y$ offen sind. Also bilden die offenen Teilmengen von $X$ und $Y$ eine Basis der Topologie auf $Z$. Nach Aufgabe 1 auf dem 2. Übungsblatt gibt es also genau eine Garbe $\mathcal{O}_Z$ auf $Z$, die auf $X$ bzw. $Y$ mit $\mathcal{O}_X$ bzw. $\mathcal{O}_Y$ übereinstmmt. $\hfill \Box$
\end{pr}
\end{proposition}

\begin{remark}

Sei $(X, \mathcal{O}_X)$ Schema, $U \subseteq X$ offen. Dann ist auch $(U, \mathcal{O}_X \vert_U)$ ein Schema. Es heißt \textit{offenes Unterschema} von $(X, \mathcal{O}_X)$. 
\begin{pr}
Ist $\{U_i\}_{i \in I}$ offene Überdeckung von $X$ durch affine Schemata, so ist $\{U \cap U_i\}_{i \in I}$ offene Überdeckung von $U$. Überdecke nun $U \cap U_i$ durch offene Mengen der Form $\{D(f_{ij})\}_{j \in J}$ mit $f_{ij} \in R$, wobei $U_i = \spec R_i$ und $D(f_{ij}) = \spec (R_i)_{f_{ij}}$. Insgesamt ist $U$ damit durch affine Schemata überdeckt.
\end{pr}
\end{remark}

\begin{ex}
Sei $X=Y= \mathbb{A}^1(k)= \spec \K[T]$, $U=V= \mathbb{A}^1(k) \setminus \{ 0 \} = \spec k[T]_T = D(T)$. Setze $f= \textrm{id}: U \longrightarrow V$. Sei $Z$ die Verklebung von $U$ und $V$ längs $f$. Dann ist $Z= U \cup \{0_X, 0_Y\}$, wobei
$$\iota_X: X \longrightarrow Z, \quad \iota_Y: Y \longrightarrow Z, \qquad 0_X= \iota_X(0),\quad 0_Y = \iota_Y(0).$$
Es gilt:
\begin{compactenum}
\item $Z$ ist irreduzibel.
\item Für jedes offene $W\subseteq Z$ mit $0_X \in W, 0_Y \in W$ und jedes $f \in \mathcal{O}_Z(W)$ ist $f(0_X) = f(0_Y)$.
\end{compactenum}
\end{ex}


\renewcommand*\thesection{§ \arabic{section}\quad}
\section{Abgeschlossene Unterschemata} %PARAGRAPH 4
\renewcommand*\thesection{\arabic{section}}


\begin{er}  %Erinnerung 4.1
Sei $X= \spec R$, $R$ ein Ring. Dann gilt
$$V \subseteq X \textrm{ ist abgeschlossen } \quad \Longleftrightarrow \quad V = V(I) = \{ \p \in \spec R \ \vert \ I \subseteq \p \}$$
für ein Ideal $I \ideal R$. Dabei gilt
$$V(I_1) = V(I_2) \quad \Longleftrightarrow \quad \sqrt{I_1} = \sqrt{I_2}$$
\end{er}


\begin{bemdefin}  %%bemerkung + Definition 4.2
Sei $R$ ein Ring.
\begin{compactenum}
\item Für jedes Ideal $I$ ist die Abbildung
$$\pi: V(I) \longrightarrow \slant{\spec R}{I}, \qquad \p \mapsto \p \mod I$$
ein Homoöomorphismus.
\item Für jedes Ideal $I \ideal R$ heißt das affine Schema $\left(V(I), \mathcal{O}_{\spec \slant{R}{I}}\right) = \left( \spec \slant{R}{I}, \mathcal{O}_{\spec \slant{R}{I}} \right)$ \textit{abgeschlossene Unterschema} von $\spec R$.
\end{compactenum}
Die Ideale in $R$ entsprechen also bijektiv den abgeschlossenen Unterschemata von $\spec R$.
\end{bemdefin}

\begin{definbem}   %%Definition + Bemerkung 4.3
Sei $R$ ein Ring, $X= \spec R$, $I \ideal R$ ein Ideal in $R$, setze $Z= \spec \slant{R}{I}$. Sei weiter $j: Z \longrightarrow X$ der von der Restklassenabbildung $R \longrightarrow \slant{R}{I}$ induzierte Morphismus affiner Schemata.
\begin{compactenum}
\item Für $U \subseteq X$ sei 
$$\mathcal{I}(U) := I \cdot \mathcal{O}_X(U) = \rho^X_U(I) \cdot \mathcal{O}_X(U) \subseteq \mathcal{O}_X(U).$$
Dann ist $\mathcal{I}(U)$ Ideal in $\mathcal{O}_X(U)$ für alle $U \subseteq X$ offen. $\mathcal{I}$ ist Garbe abelscher Gruppen und heißt \textit{Idealgarbe} auf $X$.
\item Es gilt
$$j_{*} \mathcal{O}_Z = \slant{\mathcal{O}_X}{\mathcal{I}}.$$
\end{compactenum}

\begin{pr}
\begin{compactenum}
\item[(ii)] Für $U \subseteq X$ offen ist $\mathcal{I}(U)$ offenbar der Kern der surjektiven Abbildung 
$$j_{U}^\#: \mathcal{O}_X(U) \longrightarrow j_{*} \mathcal{O}_Z(U) = \mathcal{O}_Z\left( j^{-1}(U)\right)  =  \slant{\mathcal{O}_X(U)}{I \cdot \mathcal{O}_X(U)} = \slant{\mathcal{O}_X(U)}{\mathcal{I}(U)},$$
woraus die Behauptung folgt. $\hfill \Box$
\end{compactenum}
\end{pr}
\end{definbem}



\begin{defin}     %%Definition 4.4
Sei $(X, \mathcal{O}_X)$ ein Schema.
\begin{compactenum}
\item Eine Garbe $\mathcal{I}$ abelscher Gruppen auf $X$ heißt \textit{Idealgarbe}, wenn für jedes $U\subseteq X$ offen gilt: $\mathcal{I}(U)$ ist Ideal in $\mathcal{O}_X(U)$. Insbesondere ist $\mathcal{I}$ also Untergarbe von $\mathcal{O}_X$.
\item Ist $X=\spec R$ affines Schema, so heißt eine Idealgarbe $\mathcal{I}$ \textit{quasikohärent}, wenn es ein Ideal $I \ideal R$ gibt mit $\mathcal{I}(U) = I \cdot \mathcal{O}_X(U)$ für alle $U \subseteq X$ offen.
\item Für ein allgemeines Schema $X$ heißt eine Idealgarbe $\mathcal{I}$ \textit{quasikohärent}, wenn für jedes affine offene Unterschema $U \subseteq X$ die Einschränkung $\mathcal{I}\vert_U$ quasikohärent ist.
\end{compactenum}
\end{defin}

\begin{proposition}    %%Proposition 4.5
Eine Idealgarbe auf einem Schema $X$ ist genau dann quasikohärent, wenn es eine offene Überdeckung $\{U_i\}_{i \in I}$ durch affine Schemata gibt mit $U_i = \spec R_i$, sodass $\mathcal{I}\vert_{U_i}$ für jedes $i \in I$ quasikohärent ist.
\end{proposition}

\begin{pr}
Ist $\mathcal{I}$ Idealgarbe auf $X$, so folgt unmittelbar die Behauptung. \\
Sei nun $U = \spec R \subseteq X$ affines Schema in $X$, $U$ offen in $X$. Nach Voraussetzung gibt es affine Unterschemata $\{U_i\}_{i \in J}$ mit 
$$\bigcup_{i \in J} U_i = U \quad \textrm{ und } \quad \mathcal{I}(U_i) = I_i \cdot \mathcal{O}_X(U_i)$$
Sei dabei ohne Einschränkung $U_i = D(f_i)$ für ein $f_i \in R$. Da $\spec R$ quasikompakt ist, genügt $J= \{1, \ldots, n \}$. Dann ist aber $1 \in ( f_1, \ldots f_n )$, also schreibe
$$1 = \sum_{i=1}^n \mu_i f_i, \qquad \mu_i \in R \textrm{ geeignet für } i \in \{1, \ldots, n \}.$$
zu zeigen: Es gibt ein Ideal $I \ideal R$ mit $I \cdot R_{f_i} = I_i \ideal R_{f_i}$ ( Beachte: $\mathcal{O}_{\spec R} \left( D(f_i)\right) = R_{f_i}$).
\begin{compactenum}
\item[\textbf{Beh. (a)}] Ist $\alpha_i: R \longrightarrow R_{f_i}$, $a \mapsto \frac{a}{1}$, so ist 
$$I= \alpha^{-1}(I_i),$$
also insbesondere nicht von $i$ abhängig.
\item[\textbf{Bew. (a)}] Aus $D(f_i) \cap D(f_j) = D(f_i f_j)$ folgt  $I_i R_{f_if_j} = I_j R_{f_{ij}}$. Sei nun also $\frac{a}{f_i^k} \in I_i$. Für jedes $j \in \{1, \ldots, n \}$ gibt es $b_j \in R$, $k_j \in \mathbb{N}_0$ mit  $\frac{a}{f_i^k} = \frac{b_j}{f_j^{k_j}}$. Ohne Einschränkung gelte $k=k_j$ für alle $j$ (erweitere, bis die Potenzen im Nenner gleich sind). Dann gilt $a f_j^k = b f_i^k$. Wir erhalten
$$a \sum_{j=1}^n \tilde{\mu}_j f_j^k = \sum_{j=1}^n b_j \mu_j f_i^k,$$
also
$$a = \left( \sum_{j=1}^n b_j \mu_j \right) f_i^k \ \in ( f_i^k ), $$
also $\frac{a}{f_i^k} \in R$, was zu zeigen war. $\hfill \Box$

\end{compactenum}

\end{pr}

\newcommand{\OO}{\mathcal{O}}

\begin{definbem}   %%Definition + Bemerkung 4.6
Sei $(X, \mathcal{O}_X)$ ein Schema.
\begin{compactenum}
\item Ein \textit{abgeschlossenes Unterschema} von $X$ ist ein Schema $(Y, \OO_Y)$, wobei $Y \subseteq X$ abgeschlossen und $\OO_Y = \slant{\OO_X}{\mathcal{I}}$ für eine quasikohärente Idealgarbe $\mathcal{I}$ auf $X$.
\item Ist $(Y, \OO_Y)$ abgeschlossenes Unterschema, $U= \spec R \subseteq X$ offenes affines Unterschema von $X$, so ist $U \cap Y$ das durch das Ideal $\mathcal{I}(U) \ideal R$ definierte abgeschlossene Unterschema von $U$.
\end{compactenum}
\end{definbem}

\begin{definbem}  %%Definition + Bemerkung 4.7
\begin{compactenum}

\item Sei $R$ ein Ring, $X = \spec R$, $N_R := \sqrt{( 0 ) }$ das Nilradikal in $R$. Dann ist $\slant{R}{N_R}$ reduziert, $X_{\textrm{red}} = \spec \left( \slant{R}{N_R}\right)$ heißt das \textit{zu $X$ assoziierte reduzierte Schema}.
\item Der von $\pi: R \longrightarrow \slant{R}{N_R}$ induzierte Schemamorphismus ist ein Homoömorphismus. Er mach $X_{\textrm{red}}$ zu einem abgeschlossenen Unterschema vom $X$.
\item Sei $(X, \OO_X)$ ein Schema, $\mathcal{N}_X$ sei die Idealgarbe, für die $\mathcal{N}_X(U)$ das Nilradikal in $\mathcal{O}_X(U)$ ist. Dann ist $\mathcal{N}_X$ quasikohärent. $X_{\textrm{red}}= \left(X, \slant{\OO_X}{\mathcal{N}_X}\right)$ heißt das \textit{zu $X$ assoziierte reduzierte Schema}.
\item $(X, \OO_X)$ heißt \textit{reduziert}, falls $\mathcal{N}_X = ( 0 )$.
\end{compactenum}

\end{definbem}



\renewcommand*\thesection{§ \arabic{section}\quad}
\section{Faserprodukte} %PARAGRAPH 5
\renewcommand*\thesection{\arabic{section}}

\begin{defin}  %%Definition 5.1
Sei $\mathcal{C}$ eine Kategorie, $X,Y,S$ Objekte sowie $f: X \longrightarrow$, $g: Y \longrightarrow S$ Morphismen in $\mathcal{C}$. 
\begin{compactenum}
\item Dann heißt der Limes des Diagramms
$$
\begin{xy}
\xymatrix{
X \ar[rrd]^{f} && \\
&& S \\
Y \ar[rru]_{g} &&
}
\end{xy}
$$

 \textit{Faserprodukt} von $X$ und $Y$, schreibe $X \times_S Y$:
\item Explizit bedeutet das:
\begin{compactenum}
\item Es gibt Morphismen $p_X: X \times_S Y \longrightarrow X, p_Y: X \times_S Y \longrightarrow Y$ mit $f \circ p_X = g \circ p_Y$.
\item \textit{Universelle Abbildungseigenschaft:} Für jedes Objekt $Z$ und alle Morphismen $\phi: Z \longrightarrow X$ und $\psi: Z \longrightarrow Y$ mit $f \circ \phi = g \circ \psi$ gibt es genau einen Morphismus $h: Z \longrightarrow X \times_S Y$ mit $\phi = p_X \circ h$ und $\psi = p_Y \circ h$.
\end{compactenum}

$$
\begin{xy}
\xymatrix{
 && && X \ar[rrd]^{f} && \\
 Z \ar@{-->}[rr]^{\exists !h} \ar @/^/[rrrru]^{\phi} \ar @/_/[rrrrd]^{\psi} && X \times_S Y \ar[rru]_{p_X} \ar[rrd]^{p_Y} &&&& S \\
 && && Y \ar[rru]_{g} && 
}
\end{xy}
$$

\item Existiert für jedes Diagramm wie in (i) ein Limes in $\mathcal{C}$, so sagt man, dass in $\mathcal{C}$ Faserprodukte existieren.


\end{compactenum}
\end{defin}

\newcommand{\la}{\longrightarrow}

\begin{remark}  %%Bemerkung 5.2
\begin{compactenum}
\item In der Kategorie der Mengen existieren Faserprodukte.
\item In der Kategorie der topologischen Räume existieren Faserprodukte.
\end{compactenum}
\begin{pr}
Der Beweis erfolgt in beiden Fällen analog - wir zeigen exemplarisch (i). Sind $X,Y,S$ Mengen und $f: X \longrightarrow S$, $g:Y \longrightarrow S$ Abbildungen, so ist das Faserprodukt gegeben durch 
$$X \times_S Y = \{ (x,y) \in X \times Y \ \vert \ f(x) = g(y) \}$$
Wir haben die Projektionen
$$p_X: X \times_S Y \longrightarrow X, \quad (x,y) \mapsto x, \qquad p_Y: X \times_S Y \longrightarrow Y, \quad (x,y) \mapsto y.$$
Ist nun $Z$ eine weitere Menge und $\phi: Z \la X$, $\psi: Z \la Y$ Abbildungen mit $f \circ \phi = g \circ \psi$ gegeben, so erhalten wir eine Abbildung
$$h: Z \longrightarrow X \times_S Y, \qquad z \mapsto \left( \phi(z), \psi(z) \right),$$
welche offenbar auch eindeutig ist.

\end{pr}
\end{remark}



\begin{remark}  %%Bemerkung 5.3
In $\sets$ und $\topraum$ gilt:
\begin{compactenum}
\item Ist $S = \{s\}$; so ist $X \times_S Y = X \times Y$.
\item Es gilt 
$$X \times_S Y = \bigcup_{s \in S} f^{-1}(\{s\}) \times g^{-1}(\{s\}).$$
\item Sind $X \subseteq S$ und $Y \subseteq S$, $f,g$ die Inklusionen, so ist $X \times_S Y = X \cap Y$.
\item Ist $Y \subseteq S$, $g$ die Inklusion, so ist $X \times_S Y = f^{-1}(Y)$.
\item Ist $X=Y$, so ist $X \times_S Y$ der sogenannte \textit{Equalizer} von $f$ und $g$.
\end{compactenum}

\end{remark}

\begin{remark}
In jeder Kategorie gilt:
\begin{compactenum}
\item Das Faserprodukt ist eindeutig bis auf Isomorphie.
\item Für jedes $f: X \longrightarrow S$ gilt $X \times_S S = X$.
\item Für Morphismen $f: X \longrightarrow S$, $g: T \longrightarrow S$, $h: Y \la T$ in $\mathcal{C}$ gilt
$$\left( X \times_S T \right) \times_T Y = X \times_S Y$$
mit $$p_T: X \times_S T \longrightarrow T, \qquad \left( g \circ h \right): Y \longrightarrow S.$$

\end{compactenum}

\begin{pr}
\begin{compactenum}
\item[(ii)] Zu zeigen ist: $X$ erfüllt die UAE von $X \times_S S$. Haben $p_X = \textrm{id}: X \la X$, $o_S=f: X \la S$. Sei nun $Z$ beliebiges Objekt und $\phi: Z \la X$, $\psi: Z \la S$ mit $f \circ \phi = \textrm{id} \circ \psi = \psi$. Dann ist $h= \phi$ eindeutig und es gilt $\textrm{id} \circ \phi = \phi$ und $f \circ \phi = \psi$.

\item[(iii)] Zu zeigen: $X \times_S Y$ erfüllt die UAE von $\left( X \times_S T \right) \times_T Y$. Dazu benötigen wir $\tilde{h}: X \times_S Y \longrightarrow X \times_S T$. Betrachte also das folgende Diagramm:

$$
\begin{xy}
\xymatrix{
&&&& Y \ar[rrd]^{h} &&& \\
Z \ar@/^/[rrrru]^{\beta} \ar[rr] ^{\exists ! \pi} \ar@/_/[rrrrd]_{\alpha} && X \times_S Y \ar[rru]_{p_Y} \ar[rrd]^{\tilde{h}} \ar[rrrr]^{\phi=p_X} &&&& T \ar[rrd]^{g} \\
&&&& X \times_S T \ar[rru]^{p_T} \ar[rrd]_{\tilde{p}_X} &&&& S \\
&&&&&&X \ar[rru]_{f} && 
}
\end{xy}
$$
$\tilde{h}$ erhalten wir über die $UAE$ von $X \times_S T$, denn es gilt $f \circ \phi = \left(g \circ h \right) \circ p_Y$.
Seien nun $Z$ sowie Morphismen 
$$\alpha: Z \la X \times_S T, \qquad \beta: Z \la Y$$
gegeben, sodass $p_T \circ \alpha = h \circ \beta$. Beachte: Für $ X \times_S Y$ gilt
$$\tilde{p}_X \circ \tilde{h}: X \times_S Y \la X, \qquad p_Y: X \times_S Y \la Y$$
und 
$$f: X \la S, \qquad \left( g \circ h \right): Y \la S.$$
Die UAE davon liefert also eine eindeutige Abbiildung
$$\pi: Z \la X \times_S Y$$
mit 
$$ f \circ ( \tilde{p}_X \circ \tilde{h} ) \circ \pi = g \circ p_Y \circ \pi, \qquad \textrm{ also } \ \ \tilde{h} \circ \pi = p_Y \circ \pi,$$
woraus die Behauptung folgt. $\hfill \Box$


\end{compactenum}
\end{pr}

\end{remark}


\begin{lemma}  %%Lemma 5.5
Seien $X,Y,S$ Schemata, $f: X \la S$, $g: Y \la S$ Morphismen von Schemata sowie $\{S_i\}_{i \in I}, $ eine Überdeckung von $S$ und setze $X_i:= f^{-1}(S_i)$ und $Y_i := g^{-1}(S_i)$. 
\begin{compactenum}
\item Existiert $X_i \times_S Y$ für alle $i \in I$, so existiert bereits $X \times_S Y$.
\item Für jedes $i \in I$ gilt $X_i \times_{S_i} Y_i = X_i \times_S Y$. 
\end{compactenum}

\begin{pr}
\begin{compactenum}
\item Verklebe die $X_i \times_S Y$ längs der 
$$Z_{ij} := p_{X_i}^{-1}\left( X_i \cap X_j \right) \subseteq X_i \times_S Y.$$
Dafür benötigen wir einen Isomorphismus $\phi: Z_{ij} \la Z_{ji}$. Zeige dazu:
$$Z_{ij} = \left(X_i \cap X_j \right) \times_S Y.$$
Betrachte hierfür

$$
\begin{xy}
\xymatrix{
&&& X_i \cap X_j \ \ \subseteq \ \ X_i  \ar[rrdd]{f} && \\
&&&&& \\
Z\ar@{-->}[rr]^{\exists ! h} \ar@/^/[rrruu]^{\phi} \ar@/_/[rrrdd]_{\psi} && Z_{ij} \ar[ruu]_{p_{X_i}} \ \ \subseteq \ \ X_i \times_S Y \ar[rdd]^{p_Y} &&& T \\
&&&&& \\
&&& W \ar[rruu]^{g} &&
}
\end{xy}
$$
Wir zeigen nun: $h(Z) \subseteq Z_{ij}$. Dies gilt jedoch, da $\phi(Z) \subseteq X_i \cap X_j$, $p_{X_i} \left( \phi(Z) \right) \subseteq Z_{ij}$, also 
$$h(Z) = p_{X_i}^{-1} \left( \phi(Z) \right) \subseteq Z_{ij}.$$
Sei nun also $\tilde{X}$ die Verklebung der $X_i \times_S Y$ längs der $Z_{ij}$. Es bleibt noch zu zeigen: $\tilde{X} = X \times_S Y$. Verklebe nun die $p_{X_i}: X_i \times_S Y \la X_i \subseteq X$ zu Morphismen $p_X: \tilde{X} \la X$. Betrachte
$$
\begin{xy}
\xymatrix{
&&&& X  \ar[rrd]^{f}&& \\
Z \ar@/^/[rrrru]^{\phi} \ar@/_/[rrrrd]_{\psi} \ar@{-->}[rr]^{\exists ! h} && \tilde{X} \ar[rru]_{p_X} \ar[rrd]^{p_Y} &&&& T \\
&&&& Y \ar[rru]_{g} && 
}
\end{xy}
$$
Für jedes $i \in I$ sei $Z_i := \phi^{-1}(X_i)$ Erhalte dadurch 
$$h_i: Z_i \la X_i \times_S Y$$
mit $h_i \vert_{Z_i \cap Z_j} = h_j \vert_{Z_i \cap Z_j}$ für alle $i,j \in I$. Damit verkleben sich die $h_i$ zu einem Morphismus
$$h: Z \la \tilde{X},$$
was den Beweis beendet.
\item Zu zeigen ist: $X_i \times_{S_i} Y_i$ erfüllt die UAE von $X_i \times_S Y$- Betrachte als das Diagramm
$$
\begin{xy}
\xymatrix{
&&&& X_i \ar[rrd]^{f \vert_{X_i}} && \\
Z \ar@/^/[rrrru]^{\phi} \ar@/_/[rrrrd]_{\psi} \ar@{-->}[rr]^{h} && X_i \times_{S_i} Y_i \ar[rrd]^{p_{Y_i}} \ar[rru]_{p_{X_i}} &&&& S \\
&&&& Y \ar[rru]_{g} && 
}
\end{xy}
$$

Es gilt $f \circ \phi = g \circ \psi$, also 
$\left(g \circ \psi\right)(Z) \subseteq S_i$ und damit $\psi(Z) \subseteq g^{-1}(S_i) = Y_i,$
das heißt es gibt $h: Z \la X_i \times_{S_i} Y_i$. Damit folgt
$$X_i \times_{S_i} Y_i = X_i \times_S Y,$$
die Behauptung. $\hfill \Box$


\end{compactenum}

\end{pr}
\end{lemma}

\begin{theorem} %%Satz 5.5
In der Kategorie der Schemata existieren Faserprodukte.
\begin{pr}
\begin{compactenum}
\item[\textbf{Fall (1)}] Zeige die Behauptung für den affinen Fall $X= \spec A, Y = \spec B, S = \spec R$, wobei $A$ und $B$ mit den Morphismen $f: X\la S$ und $g: Y \la S$ als $R$-Algebren aufgefasst werden . Wir brauchen den Kolimes von $A$ und $B$ als $R$-Algebren in der Kategorie der Ringe. Wir behaupten, dass das Tensorprodukt $A \otimes_R B$ diese Eigenschaft besitzt. Erinnerung: $A \otimes_R B$ wird zur Algebra durch 
$$(a_1 \otimes b_1)(a_2 \otimes b_2) = a_1a_2 \otimes b_1 b_2.$$
Setze nun 
$$\sigma_A: A \la A \otimes_R B, \qquad a \mapsto a \otimes 1, \qquad \sigma_B: B \la A \otimes_R B, \qquad b \mapsto 1 \otimes b.$$
$\sigma_A$ und $\sigma_B$ sind Homomorphismen von $R$-Algebren. Die UAE für $A \otimes_R B$ ist: Ist $\Phi: A \times B \la C$ bilineare Abbildung in eine $R$-Algebra $C$, so existiert eine eindeutige $R$-lineare Abbildung $\gamma: A \otimes_R B \la C$ sodass das folgende Diagramm kommutiert:

$$
\begin{xy}
\xymatrix{
A \times B \ar[rrd]_{\Phi} \ar[rrrr]^{\otimes} &&&& A \otimes_R B \ar@{-->}[lld]_{\exists ! \gamma} \\
&& C &&}
\end{xy}
$$

Betrachte nun das Diagramm

$$
\begin{xy}
\xymatrix{
&&&& B \ar[lld]^{\sigma_B} \ar@/_/[lllld]_{\beta} && \\
C && A \otimes_R B \ar@{-->}[ll]_{\gamma} &&&& R \ar[llu] \ar[lld] \\
&&&& A \ar[llu]_{\sigma_A} \ar@//[llllu]^{\alpha}
}
\end{xy}
$$
mit $\Phi: A \times B \la C, \qquad (a,b) \mapsto \left(\alpha(a), \beta(b)\right)$. Es lässt sich leicht nachrechnen, dass $\gamma$ ein Ringhomomorphismus ist. Damit erhalten wir 
$$\spec \left( A \otimes_R B \right) = \spec A \times_S \spec B = X \times_S Y$$
in der Kategorie der affinen Schemata. Ist $Z$ nun einbeliebiges Schema $\phi: Z \la X$, $\psi: Z \la Y$ mit
$$\alpha = \phi_Z^\# : \mathcal{O}_{\spec A} (\spec A) = A \la \mathcal{O}_Z(Z) \phi_{*} \mathcal{O}_Z (\spec A)$$
$$\beta= \psi_Z^\#: B \la \mathcal{O}_Z(Z).$$
Dann induzieren $\alpha, \beta$ die Abbildung 
$$\gamma: A\otimes_R B \la \mathcal{O}_Z(Z)$$
und in der Übung haben wir gesehen, dass wir ein 
$$h: Z \la \spec \left(A \otimes_R B \right)$$
finden.
\item[\textbf{Fall (2)}] Seien nun $X,Y,S$ beliebig. Überdecke $S$ durch affine Schemata $S_i = \spec R_i, i \in I$. Seien weiter Morphismen $f: X \la S$, $g: Y \la S$ gegeben und setze 
$X_i := f^{-1}(S_i), Y_i := g^{-1}(S_i)$. Überdecke nun nochmals die $X_i$ durch offene affine Schemata $X_{ij} = \spec A_{ij}, j \in J$ sowie die $Y_i$ durch $Y_{ik} = \spec B-{ik}, k \in K$. Nach Schritt (i) existiert $X_{ij} \times_{S_i} Y_{ik}$ für alle $(i,j,k) \in I \times J \times K$.\\
Wir wenden nun Lemma 5.5(i) an auf
\begin{compactenum}
\item $T= S_i, V=X_i, V_j = X_{ij}, W= Y_{ik}$ für feste $k \in K$. Dies liefert die Existenz von $X_i \times_{S_i} Y_{ik}$ für alle $i \in I$.
\item $T=S_i, V=Y_i, V_j = Y_{ij}, W=X_i$. Die Symmetrie des Faserprodukts liefert dann die Existenz von $X_i \times_{S_i} Y_i$ für alle $i \in I$.
\end{compactenum}
Wenden wir nun darauf Lemma 5.5(ii) an, existiert $X_i \times_S Y$ für alle $i \in I$ und nochmals mit (i) erhalten wir die Existenz von $X \times_S Y$, was gerade die Behauptung war. $\hfill \Box$

\end{compactenum}

\end{pr}
\end{theorem}


\begin{defin}  %%definiiton 5.7 (5.6)
\begin{compactenum}
\item Sei $f: X \la S$ Mophismus von Schemata. Dann heißt $X$ ein $S-Schema$, schreibe $X/S$.
\item Sei $X/S$ ein $S$-Schema, $\phi: T \la S$ ein Morphismus von Schemata. Dann heißt das $T$-Schema
$$X \times_S T \overset{p_T}{\la} T$$
\textit{durch Basiswechsel aus} $X/S$ \textit{hervorgegangenes Schema}. Das dadurch entstehende kommutierende Diagramm
$$
\begin{xy}
\xymatrix{
 X \times_S T \ar[rr]^{p_X} \ar[d]_{p_T}&& X \ar[d]^{f} \\
 T \ar[rr]_{\phi} && S
}
\end{xy}
$$
auch \textit{Basiswechseldiagramm} oder \textit{cartesisches Diagramm}. Beachte: Die UAE des Faserprodukts stellt sicher, dass $X \times_S T$ minimal mit der Eigenschaft ist, das Diagramm mit $X,S,T$ kommutativ zu machen.

\end{compactenum}

\end{defin}

\begin{ex}
Sei $f \in \mathbb{Z}[X,Y]$. Die Ringhomomorphismen $\mathbb{Z} \rightarrow \mathbb{Q} \rightarrow \mathbb{R} \rightarrow \mathbb{C}$ sowie $\mathbb{Z} \rightarrow \mathbb{F}_p$ ergeben folgende Diagramme mit $X:= \spec \left( \slant{\mathbb{Z}[X,Y]}{( f)} \right)$ und der Schreibweise
$$X \times_{\spec \mathbb{Z}} \spec \ \mathbb{C} = X \times_{\mathbb{Z}} \mathbb{C}$$

$$
\begin{xy}
\xymatrix{
X \times_{\mathbb{Z}} \mathbb{C} \ \ar[rr] \ar[dd]&&\ X \times_{\mathbb{Z}} \mathbb{R} \ \ar[rr] \ar[dd]&&\ X \ \times_{\mathbb{Z}} \mathbb{Q}\ \ar[rr]\ar[dd] &&\ X \ar[dd] && \ X \times_{\mathbb{Z}} \mathbb{F}_p \ar[ll] \ar[dd]\\
&&&&&&&& \\
\spec \mathbb{C} \ar[rr]&& \ \spec \mathbb{R} \ \ar[rr] &&\ \spec \mathbb{Q} \ \ar[rr]&&\ \spec \mathbb{Z} \ &&\ \spec \mathbb{F}_p\ \ar[ll]
}
\end{xy}
$$
\textrm{ } \\
Eine Fragestellung könnte dann lauten: Gibt es für ein vorgegebenes $f \in \mathbb{Z}[X,Y]$ ein Schema, sodass wir durch Basiswechsel wieder zum ursprünglichen Faserprodukt zurückkehren?

\end{ex}



\renewcommand*\thesection{§ \arabic{section}\quad}
\section{Punkte} %PARAGRAPH 6
\renewcommand*\thesection{\arabic{section}}

\begin{definbem}
Sei $X$ ein Schema, $x \in X$.
\begin{compactenum}
\item Wir definieren den \textit{Restklassenkörper} von $X$ in $x$ als
$$\kappa(x) := \slant{\mathcal{O}_{X,x}}{\mathfrak{m}_x}.$$
\item Sei $f: X \la Y$ Morphismus von Schemata, $y = f(x) \in Y$. Dann induziert $f$ einen Homomorphismus $\kappa(y) \la \kappa(x)$. Damit wird $\kappa(x)/\kappa(y)$ zur Körpererweiterung.
\item Für einen Körper $k$ gibt es genau dann einen Morphismus $\iota: \spec k \la X, \iota(0) = x$, falls $\kappa(x)$ isomorph zu einem Teilkörper von $k$ ist.
\item Ein Punkt wie in (iii) heißt $k$-wertig.
\end{compactenum}
\begin{pr}
\begin{compactenum}
\item[(ii)] $f$ induziert $f_x^\#: \mathcal{O}_{Y,y} \la \mathcal{O}_{X,x}$. $f_x^\#$ ist lokaler Homomorphismus, das heißt es gilt $f_x^\#(\mathfrak{m}_y) \subseteq \mathfrak{m}_x$. Der Homomorphiesatz liefert die Behauptung.
\item[(iii)] 
Sei $U= \spec R$ eine affine Umgebung von $x$, $\p \in \spec R$ das zu $x$ zugehörige Primideal in $R$. Angenommen, eine solche Abbildung $\iota$ existiere. Dann gibt es $\alpha: R \la k$ mit $\kernel \alpha = \p$. Wegen $\mathcal{O}_{X,x} \cong R_{\p}$ gilt $\kappa(x)= \slant{R_{\p}}{\p R_{\p}}$ wird durch Restklassenbildung ein Homomorphismus $\kappa(x) \la k$ induziert.\\
Ist nun umgekehrt $\kappa(x)$ ein Teilkörper von $k$, so wird durch $R\la R_{\p} \la \slant{R_{\p}}{\p R_{\p}} \la k$ der gewünschte Morphismus induziert. $\hfill \Box$
\end{compactenum}
\end{pr}
\end{definbem}


\begin{folg} %%Folgerung 6.2
Seien $X,Y$ $S$-Schemata. Dann ist die Abbildung
$$\Phi: X \times_S Y \la \{ (x,y) \in X \times Y \ \vert \ f(x) = g(y) \}, \qquad z \mapsto \left(p_X(z), p_Y(z) \right)$$
surjektiv.

\begin{pr}
Seien $x \in X, y \in Y$ mit $f(x) = g(y)=:s \in S$. Nach 6.1 (iii) ist $\kappa(s) \subseteq \kappa(x)$ und $\kappa(s) \subseteq \kappa(y)$. Sei $\K$ das Kompositum von $\kappa(s)$ und $\kappa(y)$, also $\kappa(s) \subseteq \K$. Wieder nach 6.1(iii) gibt es Morphismen
$$\phi: Z := \spec \K \la X, \quad  0 \mapsto x, \qquad \psi: Z \la Y, \quad 0 \mapsto y.$$
Die Komposition $f \circ \phi = g \circ \psi$ ist der von $\kappa(s) \subseteq \K$ induzierte Morphismus. Damit gibt es einen eindeutigen Morphismus $h: Z \la X \times_S Y$ mit $p_X \circ h = \phi$ und $p_Y \circ h = \psi$. Für $z:= h(0)$ gilt dann somit
$$p_X(z) = x, \qquad p_Y(z)=y,$$
wie gewünscht. $\hfill \Box$
\end{pr}

\end{folg}


\begin{ex}
Die Abbildung $\Phi$ ist im Allgemeinen nicht injektiv. Betrachte hierfür den Basiswechsel
$$
\begin{xy}
\xymatrix{
\mathbb{A}^1_{\mathbb{C}} \ar[rr]^{\Phi} \ar[d] && \mathbb{A}^1_{\mathbb{R}} \ar[d] \\
\spec \mathbb{C} \ar[rr] && \spec \mathbb{R} 
}
\end{xy}
$$
mit $\mathbb{A}^1_{k} := \spec k[X]$ sowie $\mathbb{A}^1_{\mathbb{C}} = \mathbb{A}^1_{\mathbb{R}} \times_{\spec \mathbb{R}} \spec \mathbb{C}$. Dann ist $\Phi$ nicht injektiv.
\end{ex}

\begin{ex}
Sei $p$ Primzahl, $X= \spec\ \mathbb{Z}$. Dann gilt
$$\kappa(p) = \slant{\mathbb{Z}_{(p)}}{p \mathbb{Z}_{( p )}} = \mathbb{F}_p, \qquad \kappa(0) = \mathbb{Q}.$$
\end{ex}


\begin{definbem} %%Definition + Bemerkung 6.3

Sei $f: X \la Y$ Morphismus, $y \in Y$.
\begin{compactenum}
\item $X_y := f^{-1}(y) := X \times_Y \spec \kappa(y)$ heißt \textit{Faser} von $f$ über $y$.
\item Die Projektion
$$p_X: X_y \la X, \qquad p_X(X_y) = \{x \in X \ \vert \ f(x) =y \}$$
ist injektiv.
\item Ist $y$ abgeschlossen, so ist $X_y$ abgeschlossenes Unterschema von $X$.
\end{compactenum}
\begin{pr}[Beweis von (ii)]
Seien $x_1, x_2 \in X_y$ mit $p_X(x_1) = p_X(x_2) = x$. Dann ist insbesondere $f(x) = y$. Sei $Z= \spec \kappa(x)$ und $\iota: Z \la X$ der Morphismus mit $\iota(0) =x$. Weiter sei $\psi: Z \la \spec \kappa(y)$ der von $f_x^{\#}$ induzierte Morphismus. Wegen 6.1(ii) ist $\kappa(y) \subseteq \kappa(x) \subseteq \kappa(x_i)$ für $i\in \{1,2\}$. Nach 6.1(iii) gibt es Morphismen
$$h_i: Z \la X_y, \qquad h_i(0) = x_i.$$
Diese erfüllen $p_X \circ h_i = \iota$ und $p_{\spec \kappa(y)} \circ h_i = \psi$. Die UAE des Faserprodukts liefert $h_1 = h_2$ und damit $x_1 = x_2$. $\hfill \Box$

\end{pr}
\end{definbem}

\newcommand{\Sch}{\underline{\rm{Sch}} \it}
\newcommand{\Fun}{\underline{\rm{Funk}} \it}
\newcommand{\Mor}{\rm{Mor} \it}
\newcommand{\Sets}{\underline{\rm{Sets}} \it}

\begin{definbem}   %%definition + Bemerkung

Sei $X$ ein Schema, $T$ ein weiteres Schema.
\begin{compactenum}
\item Ein $T$-\textit{wertiger Punkt} von $X$ ist ein Morphismus $f: T \la X$.
\item Der Funktor
$$h_X: \Sch \la \Sets, \qquad T \mapsto \Mor_{\Sch}(T,X)$$
ist kontravariant. $h_x$ heißt \textit{Punktfunktor}.
\item Die $h_X$ definieren durch
$$h: \Sch \la \Fun\left( \Sch^{\rm{op}\it} \la \Sets\right), \qquad X \mapsto h_X$$
einen kovarianten Funktor.
\end{compactenum}
\end{definbem}


\begin{ex}   %%Beispiel 
Sei $\K$ ein Körper, $T=\spec \slant{\K[\epsilon]}{\epsilon^2}$ und $X = \mathbb{A}^2_{\K} = \spec \K[X,Y]$. Ein $T$-wertiger Punkt von $\mathbb{A}^2_{\K}$ ist ein Morphismus
$$\psi: \spec \slant{\K[\epsilon]}{\epsilon^2} \la \mathbb{A}^2_k.$$
Dieser entpricht bijektiv einem Ringhomomorphismus $\alpha: \K[X,Y] \la \slant{\K[\epsilon]}{\epsilon^2}$. Sei $\alpha$ surjektiv, sowie $\alpha^{-1}\left(( \epsilon) \right) = ( X,Y)$. Dann gibt es $a,b \in \K$ mit 
$$\alpha(X) = b \cdot \epsilon, \qquad \alpha(Y) = a \cdot \epsilon.$$
Dann gilt $\alpha(aX-bY) = ab\epsilon - ab \epsilon = 0$, also $aX-bY \in \ker \alpha$. Damit gilt
$$\ker \alpha = ( X,Y)^2 + ( aX - bY ).$$
Geometrische Interpretation: $\alpha$ bestimmt einen Punkt $P \in \mathbb{A}^2_{\K}$ und eine Gerade durch den Punkt, also einer "Richtung" bzw. einem Element aus $T_P\left(\mathbb{A}^2_{\K}\right)$.


\end{ex}

\newcommand{\quot}{\text{Quot} }
\newcommand{\homset}{\text{Hom} }

\begin{ex}   %%Beispiel 
Sei $R$ diskreter Bewertungsring, $T = \spec R$ und $\K = \quot(R)$. Sei $X$ ein weiteres Schema. Dann gilt 
$$\homset (T,X) = \left\{ \left(x_0, x_1, \iota\right) \ \vert \ x_0 \in \overline{\{x_1\}}, \iota: \kappa(x_1) \la \K \textrm{ Hom. }, \iota\left( \mathcal{O}_{\overline{\{x_1\}}_{\rm{red}\it, x_0}} \right) \subseteq R \textrm{ und } \iota(\mathfrak{m}_{x_0}) \subseteq \mathfrak{m} \right\},$$
denn:
\begin{compactenum}
\item["$\subseteq$"] Setze $f: T \la X$, $x_0:= f(\mathfrak{m}), x_1:= f(0)$ und $\iota:= f_{x_0}^\#$. Da $T$ reduziert ist, faktorisiert $f$ nach Übungsaufgabe 5.3 über $\overline{\{x_1\}}_{\rm{red}\it}=:Z$. Weiter induziert $f: T \la Z$ einen Ringhomomorphismus auf dem Halm $f_{x_0}^\#: \mathcal{O}_{Z,x_0} \la \mathcal{O}_{T,\mathfrak{m}} = R$ mit $f_{x_0}^\#\left(\mathfrak{m}_{x_0}\right) \subseteq \mathfrak{m}$.
\item["$\supseteq$"] Sei $\iota: \mathcal{O}_{Z,x_0} \la R$ Ringhomomorphismus. Dann induziert $\iota$ einen Morphismus $$f: T \la \spec \mathcal{O}_{Z,x_0} \overset{\psi}{\la} Z \la X,$$wobei $\psi$ durch die Abbildung $\spec R \la \mathcal{O}_{Z,x_0}, \p \mapsto \frac{\p}{1}$ induziert wird.

\end{compactenum}



\end{ex}





\newcommand{\Ocal}{\mathcal{O}}
\newcommand{\comp}{\begin{compactenum}}
\newcommand{\cend}{\end{compactenum}}



\renewcommand*\thesection{§ \arabic{section}\quad}
\section{Endlichkeitseigenschaften} %PARAGRAPH 7
\renewcommand*\thesection{\arabic{section}}


\begin{defin}   %%definition 7.1
Sei $(X, \Ocal_X)$ ein Schema.
\comp
\item $X$ heißt \textit{lokal noethersch}, wenn es eine offene Überdeckung $\{U_i\}_{i \in I}$ von $X$ durch affine Schemata $U_i =\spec R_i$ gibt, sodass für jedes $i \in I$ $R_i$ noetersch ist.
\item $X$ heißt noethersch, falls es die Überdeckung in (i) endlich gewählt werden kann.
\cend

\end{defin}


\begin{lemma} %% Lemma
Sei $R$ Ring, $g_1, \ldots, g_r \in R$, $I \ideal R$ sowie $\phi_i: R \la R_{g_i}$ die natürlichen Homomorphismen für $I \in \{1, \ldots, r\}$. Dann gilt
$$I = \bigcap_{i=1}^r \ \phi_i^{-1} \left( \phi_i(I) \cdot R_{g_i} \right).$$

\begin{pr}
\comp
\item["$\subseteq$"] Klar.
\item["$\supseteq$"] Sei $b \in \bigcap_{i=1}^r \phi_i^{-1}\left(\phi_i(I) \cdot R_{g_i}\right)$. Dann gilt: Für alle $i \in \{1, \ldots, r\}$ lässt sich $\phi_i(b)$ schreiben als
$$\frac{b}{1} \overset{!}{=} \phi_i(b) = \frac{a_i}{g_i^{n_i}} \ \in R_{g_i}, \qquad a_i \in I, n_i \in \mathbb{N}_0,$$
also $$\frac{b}{1} - \frac{a_i}{g_i^{n_i}} = 0 \quad \Longleftrightarrow \qquad g_i^{m_i} \cdot \left(b \cdot g_i^{n_i} - a_i \cdot 1\right) = 0 \qquad \textrm{ für ein }m_i \in \mathbb{N}_0,$$
also 
$$g_i^{m_i + n_i} \cdot b = g_i^{m_i} a_i \ \in I \qquad \textrm{ für alle } i \in \{1, \ldots, r \}.$$
Ohne Einschränkung gelte $m_i + n_i = N$ für alle $i\in \{1,\ldots, r \}$. Nun gilt 
$$\bigcup_{i=1}^r D(g_i) = R \quad \Longrightarrow \quad \bigcap_{i=1}^r V(g_i) = \emptyset,$$
die $g_1, \ldots, g_r$ erzeugen also $R$ (als Modul). Schreibe
$$1 = \sum_{i=1}^r c_i g_i \qquad \textrm{ mit geeigneten } c_i \in R.$$
Da die $g_i$ $R$ erzeugen, tun es die $g_1^N, \ldots, g_r^N$ ebenfalls, denn es gilt
$$1^M = 1 = \left( \sum_{i=1}^r c_i g_i \right)^M$$
und falls $M$ groß genug ist, kommt in jedem Monom ein $g_i$ mit Exponent $\geqslant N$ vor. Insgesamt erhalten wir damit
$$b= b \cdot 1 = \sum_{i=1}^r c_i g_i^N b \ \ \in I,$$
was zu zeigen war. $\hfill \Box$


\cend

\end{pr}

\end{lemma}


\begin{proposition}   %%Prop 7.2
\comp
\item Ein affines Schema $X=\spec R$ ist noethersch genau dann, wenn $R$ noethersch ist.
\item Ein Schema ist genau dann noethersch, wenn für jedes offene affine Unterschema $U= \spec R$ der Ring $R$ noethersch ist.
\cend
  



\begin{pr}
\comp
\item Die nichttriviale Implikation folgt aus (ii).
\item Sei also $X$ noethersch. Zeige: Jedes affine Unterschema $U = \spec R$ ist noethersch. Nach Voraussetzung gibt es $U_i = \spec R_i$ offen in $X$, $i \in I$, sodass $\bigcup_{i \in I} \ U_i = X$ und $R_i$ noethersch. Sei nun $U = \spec R \subseteq X$ offen. Zu zeigen: $R$ ist noethersch. Überdecke die $U \cap U_i$ durch Mengen $\{D(f_{ij})\}_{j \in J}$ mit $f_{ij} \in R_i$. Es gilt bekanntlich $D(f_{ij}) = \spec \left(R_i\right)_{f_{ij}}$. Die Lokalisierung eines noetherschen Rings ist noethersch (Algebra), also sind die $R_{ij} := R_{f_{ij}}$ noethersch für alle $i,j \in I,J$. Überdecke nun eine weiteres Mal die $D(f_{ij})$ durch offene Mengen $D(g_{ijk})$ mit geeigneten Elementen $g_{ijk} \in R$. Wegen $D(f_{ij}) \subseteq U$ wird also $\phi_{ij}: R \la \left(R_i\right)_{f_{ij}}$ induziert. Weiter ist 
$$R_{g_{ijk}} = \left(R_{ij}\right)_{\phi_{ij}(g_{ijk})}$$
als Lokalisierung eines noetehrschen Rings wieder noethersch. Da $U$ quasikompakt ist, genügen endlich viele der $D(g_{ijk})$, um $U$ zu überdecken - bezeichne diese als $g_1, \ldots, g_r$. Wir haben nun also 
$$\spec R = \bigcup_{i=1}^r \spec R_{g_i}$$
mit Noetherschen Ringen $R_g$. Sei nun
$$I_1 \subseteq I_2 \subseteq I_3 \subseteq \ldots$$
aufsteigende Kette von Idealen in $R$. Seien $\phi_i: R \la R_{g_i}, \ a \mapsto \frac{a}{1}$ die natürlichen Homomorphismen in die Lokalisierungen der $g_i$. Wir erhalten eine stationär werdende Kette von Idealen 
$$\phi_i (I_1) R_{g_i} \subseteq \phi_i(I_2) R_{g_i} \subseteq \phi_i(I_3) R_{g_i} \subseteq \ldots $$
in $R_{g_i}$. Mit Lemma 7.2 folgt die Behauptung. $\hfill \Box$

\cend
\end{pr}

\end{proposition}



\begin{definprop}
Sei $f: X \la Y$ Morphismus von Schemata.
\comp
\item $f$ heißt \textit{lokal von endlichem Typ}, wenn es eine offene affine Überdeckung $\{U_i = \spec A_i\}_{i \in I}$ von $Y$ durch affine Schemata und für jedes $i \in I$ eine affine Überdeckung $\{U_{ij}= \spec B_{ij} \}_{j \in J}$ von $f^{-1}(U_i)$ gibt, sodass $B_{ij}$ endlich erzeugte $A_i$-Algebra ist für alle $i \in I, j \in J$.
\item $f$ heißt \textit{von endlichem Typ}, wenn in (i) $J_i$ für jedes $i \in I$ endlich gewählt werden kann.
\item Ist $f$ lokal von endlichem Typ, so gibt es für jedes offene affine Unterschema $U = \spec A \subseteq Y$ eine affine Überdeckung $\{U_{i} = \spec B_{i}\}_{i \in I}$ von $f^{-1}(U)$, sodass $B_{i}$ endlich erzeugte $A$-Algebra ist.
\cend

\begin{pr}[Beweis von (iii)] Wie in 7.2, nur einfacher. Beachte: Ist $B$ endlich erzeugte $A$-Algebra, so auch $B_f$ für jedes $f \in B$.
\end{pr}

\end{definprop}  %%Definition ++ Pop 7.3

\begin{ex}
\comp
\item Morphismen affiner / quasiprojektiver Varietäten über $\K$ sind von endlichem Typ.
\item Für jede quasiprojektive Varietät ist der Strukturmorphismus $V \la \spec \K$ von endlichem Typ.
\item Sei $X$ von der Gestalt
$$
\begin{xy}
\xymatrix{
\ar@{-}[rrdd] &\ar@{-}[rrdd]&\ar@{-}[rrdd]&\ar@{-}[rrdd]&\ar@{-}[rrdd]&\ar@{-}[rrdd]&& \\
&&&&&&& \\
\ar@{-}[rruu] &\ar@{-}[rruu]&\ar@{-}[rruu]&\ar@{-}[rruu]&\ar@{-}[rruu]&\ar@{-}[rruu]&&
}
\end{xy}
$$
das heißt, $X$ besteht aus einer unendlichen Kette projektiver Geraden über $\K$. Der Strukturmorphismus $X \la \spec \K$ ist lokal von endlichem Typ, aber nicht von endlichem Typ.
\item Der Morphismus
$\spec \K[X_1, X_2, X_3, \ldots ] \la \spec \K$
ist nicht von lokal endlichem Typ.
\item Der Morphismus $\spec \mathbb{C} \la \spec \mathbb{Q}$ ist nicht lokal von endlichem Typ.
\cend
\end{ex}



\begin{defin} %%Definition 7.4

Ein Morphismus $f: X \la Y$ vom Schemata heißt \textit{endlich}, falls es eine offene affine Überdeckung $\{U_i = \spec A_i\}_{i \in I}$ von $Y$ gibt, sodass $f^{-1}(U_i) = \spec B_i$ für jedes $i \in I$ affin und $B_i$ endlich erezugter $A_i$-Modul ist.
\end{defin}


\begin{ex} %%Beispiel
Sei $S/R$ ganze Ringerweiterung, die als $R$-Algebra endlich erzeugt ist. Dann ist $\spec S \la \spec R$ endlich.
\end{ex}

\begin{proposition}
Ist $f: X\la Y$ endlicher Morphismus, so ist $f^{-1}(y)$ endlich für jedes $y \in Y$.
\begin{pr}
Sei $U = \spec A \subseteq Y$ offene Umgebung von $y$, sodass $f^{-1}(U)=\spec B$ affin und $B$ endlich erzeugter $A$-Modul ist. Dann ist 
$$f^{-1}(y) = X_y = \spec B \times_{\spec A} \spec \kappa(y) = \spec \left( B \otimes_A \kappa(y)\right)$$
das Spektrum eines endlichdimensionalen $\kappa(y)$-Vektorraums und hat damit nur endlich viele Primideale, denn: Als ganze Ringerweiterung von $\kappa(y)$ gilt für die Krulldimension
$$0 = \dim \kappa(y) = \dim \left( B \otimes_A \kappa(y) \right),$$
also sind alle Primideale minimal. Als endlich erzeugte $\kappa(y)$-Algebra besitzt $B \otimes_A \kappa(y)$ jedoch nur endlich viele minimale Primideale, woraus die Behauptung folgt. $\hfill \Box$
\end{pr}

\end{proposition}



\begin{remark}
Abgeschlossene Einbettungen sind endliche Morphismen. Dabei entsprechen abgeschlossene Einbettungen $Y$ von $X$ den abgeschlossenen Unterschemata, das heißt es gilt $Y \subseteq X$ abgeschlossen sowie $\iota_{*} \mathcal{O}_Y = \slant{\Ocal_X}{\mathcal{I}}$ für eine quasikohärente Idealgarbe $\mathcal{I}$ auf $X$.
\end{remark}


\newcommand{\A}{\mathbb{A}}
\newcommand{\Pro}{\mathbb{P}}


\renewcommand*\thesection{§ \arabic{section}\quad}
\section{Eigentliche Morphismen} %PARAGRAPH 8
\renewcommand*\thesection{\arabic{section}}


\begin{defin}    %%Definiiton 8.1
Ein Morphismus $f: X \la Y$ von Schemata heißt \textit{universell abgeschlossen}, wenn für jeden Morphismus $g: Y' \la Y$ der induzierte Morphismus $f': X \times_Y Y' \la Y'$ abgeschlossen ist.
\end{defin}

\begin{ex}   %%Beispiel 
\begin{compactenum}
\item Betrachte die Projektion $p: \A^2_k \la \A^1_k, (x,y) \mapsto x$. $p$ ist nicht abgeschlossen, denn: $V=V(XY-1) \subseteq \A^2$ ist abgeschlossen, aber $p(V) = \A^1 \setminus \{0\}$ nicht.
\item Die Abbildung $p_1: \A^1_k \la \spec k$ ist abgeschlossen, aber nicht universell abgeschlossen, denn für den Basiswechsel
$$
\begin{xy}
\xymatrix{
\A^1_k \times_{\spec k} \A^1_k = \A^2_k \ar[rr] \ar[d]_{p} && \A^1_k \ar[d]^{p_1} \\
\A^1_k \ar[rr]_{p_1} && \spec k 
}
\end{xy}
$$
ist $p$ wie oben, also nicht abgeschlossen.

\item Für die entsprechende Konstellation mi $\Pro^1_k$ erhalten wir
$$
\begin{xy}
\xymatrix{
\A^1_k \times_{\spec k} \Pro^1_k \ar[rr] \ar[d]_{p} && \Pro^1_k  \ar[d]\\
\A^1_k \ar[rr] && \spec k
}
\end{xy}
$$
Behauptung: $p$ ist abgeschlossen, denn: Sei $V \subseteq \A^1_k \times \Pro^1_k$ abgeschlossen. Ist $\dim V=0$, so ist $V$ endlich, also $p(V)$ endlich und damit abgeschlossen in $\mathbb{A}^1_k$. Ist $\dim V=1$, so ist $V=V(f)$ für ein $f \in k[X,Y_0, Y_1]$, das homogen von Grad $d\geqslant 0$ in den $Y_0, Y_1$ ist. Ist $d=0$, so ist $p(V)$ endlich (Nullstellenmenge sind vertikale Gerade), ist $d \geqslant 1$, so ist $p(V) = \A^1_k$, denn: Sei $x_0 \in \A^1_k$. Gesucht: $(y_0:y_1) \in \Pro^1$, sodass $f(x_0, y_0, y_1) = 0$. Da $f(x_0, Y_0, Y_1)$ homogen von Grad $d$ ist, zerfällt es in Linearfaktoren und besitzt also Nullstellen.
\end{compactenum}
\end{ex}

\begin{erinner}   %%Erinnerung 8.2
Ein topologischer Raum $X$ ist genau dann hausdorffsch, wenn die Diagonale $\Delta X := \{ (x,x) \in X \times X \}$ abgeschlossen in $X \times X$ ist.
\end{erinner}

\begin{defin}   %Definition 8.3
Sei $f: X \la S$ Morphismus von Schemata.
\begin{compactenum}
\item Der von $\textrm{id}_X: X \la X$ induzierte Morphismus $\Delta = \Delta_f: X \la X \times_S X$ heißt \textit{Diagonalmorphismus}.
\item $f$ heißt separiert, wenn $\Delta_f$ abgeschlossene Einbettung ist. Man sagt auch: $X$ ist eigentlich über $Y$. 
\item $X$ heißt weiter eigentlich, falls $X$ eigentlich über $\spec \mathbb{Z}$ ist.
\end{compactenum}
\end{defin}

\begin{ex} %%Beispiel
Sei $X$ die affine Gerade mit doppeltem Nullpunkt sowie $f: X \la \spec k$ der Strukturmorphismus. Dann ist $\Delta_f(X)$ nicht abgeschlossen in $X \times_k X$, denn in $\Delta_f(X)$ liegen nur $(0_1, 0_1)$ und $(0_2, 0_2)$, in $\overline{\Delta_f(X)}$ aber auch noch $(0_1, 0_2)$ und $(0_2, 0_1)$.
\end{ex}

\begin{remark}   %%Bemerkung 8.4
Jeder Morphismus affiner Schemata ist separiert.
\begin{pr}
Sei $f: X= \spec B \la \spec A =Y$ ein Morphismus induziert vom Ringhomomorphismus $\alpha: A \la B$. Dann ist $ X \times_Y X= \spec \left( B \otimes_A B\right)$ und $\Delta_f: X \la X \times_Y X$ wird induziert von 
$$\mu: B \otimes_A B \la B, \qquad b_1 \otimes b_2 \mapsto b_1b_2.$$
$\Delta_f$ ist abgeschlossene Einbettung, denn $\mu$ ist surjektiv und damit $B= \slant{B \otimes_A B}{\ker \mu}$, wobei $\ker \mu$ erzeugt wird von den $1 \otimes b- b \otimes 1$ für $b \in B$. $\hfill \Box$
\end{pr}

\end{remark}


\begin{remark}    %%Bemerkung 8.5
Offene und abgeschlossene Einbettungen sind separiert.
\begin{pr}
Ist $\iota: U \la X$ offene (bzw. abgeschlossene) Einbettung, so ist $U \times_X U=U$ und damit $\Delta_{\iota} = \textrm{id}_U$, also ist $\Delta_{\iota}$ separiert.
\end{pr}
\end{remark}


\begin{defin} %%Deinfiiton 8.6
Ein Morphismus $f: X \la Y$ von Schemata heißt \textit{eigentlich}, falls $f$ von endlichem Typ, separiert und universell abgeschlossen ist. Man sagt auch: $X$ ist eigentlich über $Y$. $X$ heißt weiter eigentlich, falls $X$ eigentlich über $\spec \mathbb{Z}$ ist.
\end{defin}

\begin{defin}  %%Definition 8.7
Sei $R$ diskreter Bewertungsring, $k = \textrm{Quot}R$, $U= \spec k$ und $T= \spec R$. Weiter sei $f: X \la Y$ ein Morphismus von Schemata. Dann heißt ein kommutatives Diagramm der Form 
$$
\begin{xy}
\xymatrix{
U \ar[rr]^{h_0} \ar[d] && X \ar[d]^{f} \\
T \ar[rr]_{h_1} && Y 
}
\end{xy}
$$
\textit{Bewertungsdiagramm} für $f$.
\end{defin}

\begin{theorem} %%Satz 8.8
Seien $X,Y$ noethersche Schemata, $f: X \la Y$ von endlichem Typ. Dann gilt:
\begin{compactenum}
\item $f$ ist genau dann separiert, wenn es zu jedem Bewertungsdiagramm für $f$ wie in 8.9 \rm{höchstens} \it einen Morphismus $h:T \la X$ gibt, sodass $f \circ h = h_1$.
\item $f$ ist genau dann eigentlich, wenn es zu jedem Bewertungsdiagramm für $f$ wie in 8.9 \rm{genau} \it einen Morphismus $h: T \la X$ gibt, sodass $f \circ h = h_1$.
\end{compactenum}

\begin{pr}
Es werden nur jeweils die Implikation "$\Rightarrow$" gezeigt - die Rückrichtung lässt sich beispielsweise in Hartshorne II.4.3 bzw. II.4.7 nachlesen.
\begin{compactenum}
\item Sei nun ein Bewertungsdiagramm von $X$ über $Y$ sowie zwei Fortsetzungen $h, h': T \la X$ gegeben.

$$
\begin{xy}
\xymatrix{
U \ar[dd]_{i} \ar[rr]^{h_0} && X \ar[dd]^f \\ &&\\T \ar[rr]_{h_1} \ar@<2pt>@{-->}[rruu]^h \ar@<-2pt>@{-->}[rruu]_{h'} && Y
}
\end{xy}
\qquad\qquad
\begin{xy}
\xymatrix{
&& T \ar[d]^{\tilde{h}} \ar@/^1cm/[rrdd]^{h'} \ar@/_1cm/[lldd]_{h} && \\ && X \times_Y X \ar[rrd]_{p_2} \ar[lld]^{p_1} && \\ X \ar[rrd]_{f} &&&& X \ar[lld]^{f} \\ &&Y &&
}
\end{xy}
$$
Wir müssen zeigen, dass $h=h'$. Betrachte die UAE des Faserprodukts. Dann induzieren die Morphismen $h, h'$ einen Morphisus $\tilde{h}: T \longrightarrow X \times_Y X$. Für den offenen Punkt $t_1 \in T$ gilt wegen $h \circ i = h_0 = h' \circ i$ demnach $h(t_1) = h_0(t_1) = h'(t_1) =: x_1$, das heißt, $\tilde{h}(t_1) \in \Delta_f(X)$. Da $\Delta_f(X)$ nach Voraussetzung abgeschlossen ist (denn $f$ ist separiert) und $t_0 \in \overline{\{t_1\}}$ ist $\tilde{h}(t_0) \in \overline{\{\tilde{h}(t_1)\}} \subseteq \Delta_f(X)$, also ist 
$$h(t_0) = p_1( \tilde{h}(t_0)) = p_2 ( \tilde{h}(t_0)) = h'(t_0).$$
Dann gilt nach Beispiel 6.8 bereits $h=h'$, denn die Abbildung $\iota: \kappa(x_1) \la k$ ist durch $h_0$ bereits festgelegt.
\item Sei nun $f$ eigentlich und das folgende Bewertungsdiagramm gegeben. Wir finden Ein Faserproduktdiagramm darin:

$$
\begin{xy}
\xymatrix{
U \ar[rrr]^{h_0} \ar[dd]_{i} \ar[rd]_{\phi} &&& X \ar[dd]^{f} \\ & T \times_Y X \ar[ld]_{p_T} \ar[rru]^{p_X} && \\ T \ar[rrr]_{h_1} \ar@{-->}@/_0.6cm/[rrruu]_{h} &&& Y
}
\end{xy}
$$
Nach der UAE des Faserprodukts gibts es $\phi: U \la T \times_Y X$ mit $ p_T\circ \phi = i$ und $p_X \circ \phi = h_0$. $i$ ist dominant, $f'=p_T$ damit auch und somit surjektiv, da $p_T$ abgeschlossen ist. Sei nun $z_1= \phi(t_1) \in T \times_Y X$. Dan gilt $f'(z_1) = i(t_1) = t_1$. Weiter sei $Z:= \overline{\{z_1\}}$ mit der induzierten reduzierten Struktur. Dann ist auch $f' \vert_Z$ surjektiv, es gibt also $z_0 \in \overline{\{z_1\}}$ mit $f'(z_0)=t_0$. Wir behaupten nun, dass es einen Morphismus $h: T \la X$ gibt, sodass $x_0 := h(t_0)= p_X(z_0)$ und $h(t_1)= h_0(t_1)=p_X(z_1)=:x_1$. Nach Konstruktion ist $p_X(z_0) \in \overline{\{h_0(t_1)\}}$. Wir brauchen also einen Homomorphismus $\iota: \kappa(h_0(t_1)) \la \kappa(t_1) = k$ mit $\iota \left( \mathcal{O}_{\overline{\{x_1\}}_{\textrm{red}}, x_0}\right) \subseteq R$ und $\iota\left(\mathfrak{m}_{x_0}\right) \subseteq \mathfrak{m}$. Vermöge $p_X$ ist $\mathcal{O}_{\overline{\{x_1\}}_{\textrm{red}}, x_0} \subseteq \mathcal{O}_{Z,z_0}$ und $\mathfrak{m}_{x_0} \subseteq \mathfrak{m}_{z_0}$ mit dem von $f'$ induzierten lokalen Homomorphismus $R \hookrightarrow \mathcal{O}_{Z, z_0}$ und $k \hookrightarrow \kappa(z_1)$. Andererseits induziert $\phi$ einen Homomorphismus $\kappa(z_1) \hookrightarrow k$, das heißt wir erhalten $k= \kappa(z_1)$. Import aus der Algebra: Diskrete Bewertungsringe in $k$ sind maximal bezüglich Dominanz (d.h. es gilt $R \subseteq R', \mathfrak{m}_R = \mathfrak{m}_{R'} \cap R$), das heißt wir erhalten wie gewünscht $R = \mathcal{O}_{Z, z_0}$. $\hfill \Box$

\end{compactenum}
\end{pr}

\end{theorem}

\begin{ex}  %%Beispiel 8.9

\begin{compactenum}
\item Sei $X= \mathbb{A}^1_k$, $Y = \spec k$, und $f: X \la Y$ induziert von $k \la k[T]$. Sei $K= \textrm{Quot} k[T] = k(T)$. Es ist 
$$\nu: k(T)^{\times} \la \mathbb{Z}, \qquad \frac{f}{g} \mapsto \deg g - \deg f$$
die Nullstellenordnung vom Punkt $P= \infty$. $\nu$ ist diskrete Bewertung auf $k(T)$. Der Bewertungsring ist gegeben durch 
$$R= \left\{ \frac{f}{g}\in k(T) \ \vert \ \deg f \leqslant \deg g \right\}.$$
Wir erhalten aus dem entsprechenden Bewertungsdiagramm das folgende Diagramm von $k$-Algebren:
$$
\begin{xy}
\xymatrix{
\spec K \ar[rr] \ar[d] && \mathbb{A}^1_k \ar[d]  \\ \spec R \ar[rr] && \spec k
}
\end{xy}
\qquad \qquad
\begin{xy}
\xymatrix{
k(T) && k[T] \ar@{->}[ll] \ar@{-->}[lld]^{\alpha} \\ R \ar@{->}[u] && k \ar[ll] \ar[u]
}
\end{xy}
$$
Es ist klar: Den Ringhomomorphimus $\alpha$ gibt es nicht, denn $k[T] \nsubset R$, $f$ ist also nicht eigentlich.
\item Sei $X$ die affine Gerade mit zwei Nullpunkten also $X= \mathbb{A}^1_k \setminus \{0\} \cup \{0_1, 0_2\}$. $X$ ist irreduzibel mit Funktionenkörper $k(X)= k(T)$. Sei $R= k[T]_{(T)} = \mathcal{O}_{\mathbb{A}^1_k, 0} = \mathcal{O}_{X, 0_i}$ für $i\in \{1,2 \}$. Das zugehörige Bewertungsdiagramm ist
 $$
 \begin{xy}
\xymatrix{
\spec k(T) \ar[dd]_{i} \ar[rr] && X \ar[dd]^f \\ &&\\ \spec k[T]_{(T)} \ar[rr] \ar@<2pt>@{-->}[rruu]^{h_1} \ar@<-2pt>@{-->}[rruu]_{h_2} && \spec k
}
\end{xy}
 $$

Für $i \in \{1, 2\}$ induziert der Morphismus $\mathcal{O}_{X, 0_i} \la R$ einen Morphismus $h_i: \spec R \la X$, der das Diagramm kommutativ macht. Weiter ist $h_1 \neq h_2$, da $h_1(\mathfrak{m}) = 0_1 \neq 0_2 = h_2(\mathfrak{m})$ und $f$ damit nicht separiert.

\end{compactenum}
\end{ex}


\begin{folg}   %%Folgerung 8.10
Für Morphismen noetherscher Schemata gilt:
\begin{compactenum}
\item Die Komposition separierter Morphismen ist separiert.
\item Die Komposition eigentlicher Morphismen ist eigentlich.
\item Separiertheit ist stabil unter Basiswechsel.
\item Eigentlichkeit ist stabil unter Basiswechsel.
\item Ist $g \circ f$ separiert, so ist $f$ separiert.
\item Ist $g \circ f$ eigentlich und $g$ separiert, so ist $f$ eigentlich.
\end{compactenum}
\begin{pr}
\begin{compactenum}
\item Sei ein Bewertungsdiagramm von $g \circ f$ gegeben, dabei seien $f$ und $g$ separiert.
$$
\begin{xy}
\xymatrix{
X \ar[rr]^{f} && Y \ar[rr]^{g} && Z \\ &&&& \\ U \ar[uu]^{h_0} \ar@{-->}[rruu]^{f \circ h_0} \ar[rrrr]_{i} &&&& T \ar[uu]_{h_1} \ar[lluu]_{\tilde{h}} \ar@<2pt>[lllluu]^{h} \ar@<-2pt>[lllluu]_{h'}
}
\end{xy}
$$


 Angenommen es gibt $h,h': T \la X$ mit $h \circ i = h_0 = h' \circ i$ und $g \circ f \circ h = h_1 = g\circ f \circ h'$. Aus der Separiertheit von $g$ folgt: $f \circ h= f \circ h' = \tilde{h}$. Da $\tilde{h}$ mit $i$ und $h_0$ ein Bewertungsdiagramm für $f$ ergibt, ist $h = h'$. 
 \item Genau so wie (i)
 \end{compactenum}
 Die übrigen Aussagen ergeben sich auf ähnliche Weise und werden hier nicht aufgeführt.


\end{pr}

\end{folg}


\begin{remark}   %%Bemerkung 8.11
Abgeschlossene Einbettungen sind eigentlich.
\begin{pr}
Sei $f: V \longrightarrow X$ eine abgeschlossene Einbettung. Da $f$ lokal (also affin) einer Restklassenbildung mit einem Ideal entspricht, ist $f$ von endlichem Typ. Nach 8.7 ist $f$ separiert, bleibt also zu zeigen, dass $f$ universell abgeschlossen ist. Für jeden Basiswechsel mit $Y$ ist aber $V \times_X Y = f^{-1}(V)$ und die hervorgehende Abbildung ist ebenfalls abgeschlossen. $\hfill \Box$
\end{pr}
\end{remark}

\begin{remark}
Projektive Morphismen sind eigentlich.
\begin{pr}
Seien $X,Y$ projektive Varietäten, $\phi:X \la Y$ Morphismus, $h: X \la \mathbb{P}_k^n$ die zugehörige abgeschlossene Einbettung, $s_X, s_Y$ die zugehlörigen Strukturmorphismen. Betrachte die Diagramme
$$
\begin{xy}
\xymatrix{
\mathbb{P}_k^n \ar[dd]_{g} \ar[rr] && \mathbb{P}_{\mathbb{Z}}^n \ar[dd]^{f} \\ && \\ \spec k \ar[rr] && \spec \mathbb{Z}
}
\end{xy}
\qquad \qquad
\begin{xy}
\xymatrix{
X \ar[rr] \ar[rdd]_{s_X} \ar[rr]^{h} && \mathbb{P}_{k}^n \ar[ldd]^{g} \\ && \\ & \spec k & 
}
\end{xy}
\qquad \qquad
\begin{xy}
\xymatrix{
X \ar[rr]^{\phi} \ar[rdd]_{s_X} && Y \ar[ldd]^{s_Y} \\ && \\ & \spec k &
}
\end{xy}
$$
Dann ist $f$ eigentlich nach Übungsaufgabe, $g$ also nach 8.12 (iv) auch. $h$ ist eigentlich nach 8.13, die Strukturmorphismen als Verkettung eigentlicher Morphismen dann nach 8.12(ii) ebenfalls. Damit ist $s_Y \circ \phi$ eigentlich und $s_Y$ separiert (da eigentlich), folglich ist $\phi$ eigentlich. $\hfill \Box$


\end{pr}
\end{remark}



















\chapter{Garben und Divisoren} %KAPITEL II
\setlength\abovedisplayshortskip{0pt}
\setlength\belowdisplayshortskip{10pt}
\setlength\abovedisplayskip{10pt}
\setlength\belowdisplayskip{10pt}

\setcounter{section}{8}
\renewcommand*\thesection{§ \arabic{section}\quad}
\section{$\mathcal{O}_X$-Modulgarben} %PARAGRAPH 9
\renewcommand*\thesection{\arabic{section}}
\thispagestyle{empty}





\begin{defin}    %%Defin 9.1
Sei $(X, \mathcal{O}_X)$ lokal geringter Raum, $\mathcal{F}$ Garbe von abelschen Gruppen auf $X$. $\mathcal{F}$ heißt $\mathcal{O}_X$-Modulgarbe auf $X$, falls
\begin{compactenum}
\item Für jede offene Teilmenge $U \subseteq X$ die abelsche Gruppe $\mathcal{F}(U)$ ein $\mathcal{O}_X(U)$-Modul ist.
\item Für alle offenen Teilmengen $U' \subseteq U \subseteq X$ der Gruppenhomomorphismus $\rho^{U}_{U'}: \mathcal{F}(U) \la \mathcal{F}(U')$ ein $\mathcal{O}_X(U)$-Modulhomomorphismus ist. Dabei wird $\mathcal{F}(U')$ vermöge $_{\mathcal{O}_X} \rho^{U}_{U'}$ als $\mathcal{O}_X(U)$-Modul aufgefasst.
\end{compactenum}

\end{defin}

\newcommand{\divrm}{\textrm{div}}

\begin{definbem}     %%Definbem 9.2
\begin{compactenum}
\item Ein \textit{Homomorphismus von} $\mathcal{O}_X$\textit{-Modulgarben} $\mathcal{F}$ und $\mathcal{G}$ ist ein Garbenmorphismus $\phi: \mathcal{F} \la \mathcal{G}$, sodass für jede offene Teilmenge $U \subseteq X$ der Gruppenhomomorphismus $\phi_U: \mathcal{F}(U) \la \mathcal{G}(U)$ ein $\mathcal{O}_X(U)$-Modulhomomorphismus ist. Man sagt, $\phi$ ist $\mathcal{O}_X$\textit{-linearer Garbenmorphismus}.
\item Die $\mathcal{O}_X$-Modulgarben bilden zusammen mit den $\mathcal{O}_X$-linearen Garbenmorphismen eine Kategorie $\underline{\mathcal{O}_X\rm{-Mod}}$.
\end{compactenum}
\end{definbem}


\begin{ex}    %%Beispiel 9.2
Sei $X$ eine nichtsinguläre, projektive Kurve über einem algebraisch abgeschlossenen Körper $k$. 
\begin{compactenum}
\item Sei $D:= \sum_{P \in X} n_P P$ ein Divisor auf $X$, das heißt es ist $n_P \in \mathbb{Z}$, $n_P \neq 0$ nur für endlich viele $P \in X$. Für $U \subseteq X$ offen sei 
$$\mathcal{L}(D)(U) := \left\{ f \in k(X) \ \vert \ \textrm{div} \left(f\vert_U\right) + D\vert_U \geqslant 0 \right\} \cup \{0\}.$$
Dann ist $\mathcal{L}(D)$ eine $\mathcal{O}_X$-Modulgarbe, denn: für $g \in \mathcal{O}_X(U)$ ist $\textrm{div}g \geqslant 0$ und damit
$$\textrm{div} \left(fg \vert_U\right) + D\vert_U = \textrm{div} \left(f\vert_U\right) + \textrm{div}\left(g\vert_U\right) + D\vert_U = \textrm{div}\left(f\vert_U\right) + D\vert_U \geqslant 0.$$
Weiter ist der globale Schnitt
$$\mathcal{L}(D)(X) = \left\{f \in k(X) \ \vert \ \textrm{div}f + D \geqslant 0 \right\} \cup \{0\} = L(D)$$
gerade der Riemann-Roch-Raum für $D$. Dieser ist demnach ein $\mathcal{O}_X(X)$-Modul, also ein $k$-Vektorraum. Dieses Resultat hatten wir vergangener Semester bereits gesehen. Betrachte nun eine kleine Umgebung $U \subseteq X$ von $P \in X$, das heißt es gilt $n_Q = 0$ für alle $Q \in U \setminus \{P\}$. Sei $t_P$ ein Erzeuger des zu $P$ zugehörigen maximalen Ideals $\mathfrak{m}_P \subset \mathcal{O}_{X,P}$. Wähle nun $U$ so, dass $\divrm \ t_P \vert_{U \setminus \{P\}} =0$, das heißt $t_P \in \mathcal{O}_X(U)$. Dann ist $t_P^{-n_P} \in \mathcal{L}(D)(U)$ und $t_P^{-n_p}$ erzeugt $\mathcal{L}(D)(U)$ als $\mathcal{O}_X(U)$-Modul, denn: Ist $g \in \mathcal{O}_X(U)$, so ist $$\divrm \left(t_P^{-n_P} g \vert_U \right) = \divrm\left(t_P^{-n_P} \vert_U\right) + \divrm \left(g \vert_U\right) \geqslant \divrm \left(t_P^{-n_P} \vert_U\right) = -n_P P,$$
also
$$\divrm \left(t_P^{-n_P} g \vert_U\right) + D\vert_U \geqslant -n_PP + n_pP \geqslant 0$$
und damit $t_P^{-n_P} g \in \mathcal{L}(D)(U)$. Ist umgekehrt $g \in \mathcal{L}(D)(U)$, also $$\divrm \left(g \vert_U\right) + D\vert_U = \divrm\left(g \vert_U\right) + n_P P \geqslant 0,$$
so gilt 
$$\divrm\left(t_P^{n_P} g \vert_U\right) = n_P P + \divrm \left(g \vert_U\right) \geqslant 0,$$
also $t_P^{n_P} g \in \mathcal{O}_X(U)$ und damit $g= t_P^{-n_P} \left( t_P^{n_P} g \right)$.


\item Sei $\Omega= \Omega_{k(X)/k}$ der $k(X)$-Vektorraum der Kählerdifferentiale von $k(X)/k$. Die Elemente von $\Omega$ heißen rationale Differentiale auf $X$. Ohne Einschränkung gelte $X=V(f)$ mit einem irreduziblen Polynom $f \in k[X,Y]$. Dann ist $k(X) = \textrm{Quot} \slant{k[X,Y]}{(f)}$. Damit wird $\Omega$ erzeugt von den Elementen $\textrm{d}g$ für $g \in k(X)$, wobei $\textrm{d}$ die universelle Derivation bezeichne. Da $\textrm{d}(X^2) = 2X \textrm{d}X$und induktiv $\textrm{d}(X^n) = n! X \textrm{d}X$, genügen die linearen Terme. $\textrm{d}f=0$ ergibt also eine lineare Gleichung zwischen $\textrm{d}X$ und $\textrm{d}Y$ und wir erhalten $\dim_{k(X)} \Omega =1$. Wir wollen uns daraus nun eine $\mathcal{O}_X$-Modulgarbe basteln. Für $ \omega \in \Omega_{k(X)/k}$ sei
$$\textrm{div} \omega = \sum_{P \in X} \mathrm{ord}_P \omega P$$
folgendermaßen definiert: Für $P \in X$ sei $t_P$ Uniformisierende, also Erzeuger vom maximalen Ideal $\mathfrak{m}_P$. Dann gilt $\textrm{d}t_P(P)=0$ aber $\textrm{d}t_P \neq 0$, also bildet $\{\textrm{d}t_P\}$ eine Basis von $\Omega_{k(X)/k}$. Schreibe also $\omega= f_P \textrm{d}t_P$ für ein $f_P \in k(X)$. Setze nun $\textrm{ord}_P(\omega) = \textrm{ord}_P(f_P)$. Beachte: $t_P - t_P(Q)$ ist Uniformisierende für $Q$ auf einer offenen (und dichten) Teilmenge von $X$ und $\textrm{d}(t_P-t_P(Q)) = \text{d}t_P$. Damit ist $\textrm{div} \omega$ wohldefiniert. Setze nun 
$$\Omega_X(U) := \left\{ \omega \in \Omega_{k(X)/k} \ \vert \ \text{div} \omega \vert_U \geqslant 0 \right\} \cup \{0\}.$$
$\Omega_X(U)$ ist für jedes $U \subseteq X$ offen ein $\mathcal{O}_X(U)$-Modul, also ist $\Omega_X$ eine $\mathcal{O}_X$-Modulgarbe. Die Elemente in $\Omega_X(U)$ heißen \textit{reguläre Differentiale} auf $U$. Bachte: Mit der Notation aus (i) gilt $\Omega_X \cong \mathcal{L}(\text{div} \omega_0)$ für $\omega_0 \in \Omega_{k(X)/k} \setminus \{0\}$, denn: $\omega_0$ ist eine Basis von $\Omega_{k(X)/k}$ und es gilt
\setlength{\abovedisplayskip}{5.5pt}
\setlength{\belowdisplayskip}{5.5pt}
\begin{alignat*}{5}
\mathcal{L}(\text{div} \omega_0)(U) \ \ &=&& \ \ \left\{ f \in k(X) \ \vert \ \left(\text{div} f + \text{div} \omega_0\right) \vert_U \geqslant 0 \right\} \cup \{0\} \\
&=&& \ \ \left\{ f \in k(X) \ \vert \ \text{div}(f \omega_0)\vert_U \geqslant 0 \right\} \cup \{0\} \\
&=&& \ \ \left\{ w \in \Omega_{k(X)/k} \ \vert \ \text{div}(\omega) \vert_U \geqslant 0 \right\} \cup\{0\} \\
&=&& \ \ \Omega_X(U).
\end{alignat*}
$\text{div} \omega_0$ heißt auch \textit{kanonischer Divisor}. Erinnern wir uns nun an den Satz von Riemann-Roch aus der algebraischen Geometrie, welcher besagt:
$$\dim L(D) - \dim L(K-D) = \deg D +1 - g,$$
wobei $g$ das Geschlecht der Kurve und $K$ einen kanonsichen Divisor bezeichne, so erhalten wir mit $D=0$:
$$1- \dim L(K) = 1-g \quad \Longleftrightarrow \quad \dim L(K) = g$$
und mit $D=K$
$$\dim L(K)-1 = \dim L(K) - \dim L(0) = \deg K +1 - g,$$
zusammen also
$\deg K = 2g-2$. Da sist praktisch! Betrachte wir uns beispielsweise den Punkt $\infty = (1:0) \in \Pro^1$ , das Differential $\omega = \mathrm{d}X$ und die Uniformisierende $t_{\infty} = \frac{1}{X}$, so gilt 
$$ \text{d}X =\omega= f_p \text{d}\left( \frac{1}{X}\right) = -f_P \frac{1}{X^2} \text{d}X,$$
also $f_P= -X^2$ und $\text{ord}_P \text{d}X = \text{ord}_P X^2 = -2$, was mit unserer oben gefunden Formel und $g=0$ für $\Pro^1$ übereinstimmt. Eine weitere Anwendung ist natürlich auch die Bestimmung des Geschlechts einer Kurve mithilfe der obigen Formel.

\end{compactenum}
\end{ex}


\begin{definbem}    %%definbem 9.4

\begin{compactenum}
\item Sei $X = \spec R$ ein affines Schema und $M$ ein $R$-Modul. Dann gibt es genau eine Garbe $\tilde{M}$ auf $X$, sodass $\tilde{M}(D(f)) = M_f = M \otimes_R R_f$ für jedes $f \in R$. $\tilde{M}$ wird so zur $\mathcal{O}_X$-Modulgarbe. Weiterhin ist für jedes $\p \in \spec R$ der Halm gegeben durch 
$$\tilde{M}_{\p} = M_{\p} = M \otimes_R R_{\p}.$$
\item Eine $\mathcal{O}_X$-Modulgarbe $\mathcal{F}$ heißt \textit{quasikohärent} auf $\spec R$, falls es einen $R$-Modul $M$ gibt mit $\mathcal{F} \cong \tilde{M}$, also $\mathcal{F}(D(f)) \cong M_f$ als $R_f$-Moduln.
\item Sei nun $(X, \mathcal{O}_X)$ ein allgemeines Schema. Eine $\mathcal{O}_X$-Modulgarbe $\mathcal{F}$ auf $X$ heißt \textit{quasikohörent}, falls es eine offene Überdeckung von $X$ durch affine Unterschemata $\{U_i= \spec R_i\}_{i \in I}$ gibt, sodass die Einschränkung $\mathcal{F}\vert_{U_i}$ quasikohärent ist für jedes $i \in I$, also $\mathcal{F}\vert_{U_i} \cong \tilde{M}_i$ für einen $R_i$-Modul $M_i$.

\item Eine quasikohärente $\mathcal{O}_X$-Modulgarbe auf $X$ heißt \textit{kohärent}, falls $X$ noethersch ist und die $R_i$-Moduln $M_i$ aus (iii) allesamt endlich erzeugt sind.
\end{compactenum}

\end{definbem}


\begin{proposition}     %%Proposition 9.5
Sei $(X, \mathcal{O}_X)$ ein Schema. Eine $\mathcal{O}_X$-Modulgarbe $\mathcal{F}$ auf $X$ ist quasikohärent genau dann, wenn für jedes offene, affine Unterschema $U \subseteq X$ die Einschränkung $\mathcal{F}\vert_U$ quasikohärent ist.
\begin{pr}
Wie zum Beispiel 4.5 oder 7.3.

\end{pr}

\end{proposition}

\begin{remark}     %%Bemerkung 9.6
Sei $X= \spec R$ ein affines Schema. Dann ist die Zuordnung 
$$\Rmod \longrightarrow \Oxmod, \qquad M \mapsto \tilde{M}$$
 ein volltreuer, exakter Funktor, dessen Bild die quasikohärenten $\mathcal{O}_X$-Modulgarben sind.  
 \begin{pr}
 Sei $0 \la M' \la M \la M'' \la 0$ exakte Sequenz von $R$-Moduln. Zu zeigen: Für jedes $\p \in \spec R$ ist die lokalisierte Sequenz 
 $$0 \la M'_{\p} \la M_{\p} \la M''_{\p} \la 0$$
 ebenfalls exakt (als Sequenz von $R_{\p}$-Moduln): Übung. Man sagt, $R_{\p}$ ist "flacher" $R$-Modul.
 
 \end{pr}

\end{remark}


\begin{definbem}    %%Definbem 9.7
\begin{compactenum}
\item Sei $(X, \mathcal{O}_X)$ ein lokal geringter Raum, $\mathcal{F}, \mathcal{G}$ zwei $\mathcal{O}_X$-Modulgarben. dann ist die zur Prägarbe $U \mapsto \mathcal{F}(U) \otimes_{\mathcal{O}_X(U)} \mathcal{G}(U)$ assoziierte Garbe $\mathcal{F} \otimes_{\mathcal{O}_X} \mathcal{G}$ eine $\mathcal{O}_X$-Modulgarbe. \item Ist $X= \spec R$ affin, so gilt für $R$-Moduln $M,N$ 
$$\widetilde{M \otimes_R N} = \tilde{M} \otimes_{\tilde{R}} \tilde{N} = \tilde{M} \otimes_{\mathcal{O}_X} \tilde{N}.$$
\item Sind für $i \in I$ $R$-Moduln $M_i$ gegeben, so ist
$$\widetilde{\bigoplus _{i \in I} M_i} = \bigoplus_{i \in I} \tilde{M}_i.$$
\end{compactenum}

\end{definbem}


\begin{bemdefin}   %%Bemdefin 9.8
Sei $f: X \la Y$ Morphismus lokal geringter Räume.
\begin{compactenum}
\item Für jede $\mathcal{O}_X$-Modulgarbe $\mathcal{F}$ auf $X$ ist $f_*\mathcal{F}$ eine $\mathcal{O}_Y$-Modulgarbe auf $Y$.
\item Für jede $\mathcal{O}_Y$-Modulgarbe $\mathcal{G}$ auf $Y$ ist $f^{-1} \mathcal{G}$ eine $f^{-1} \mathcal{O}_Y$-Modulgarbe und die zur Prägarbe $U \mapsto f^{-1}\mathcal{G}(U) \otimes_{f^{-1}\mathcal{O}_Y(U)} \mathcal{O}_X(U)$ assoziierte Garbe $f^*\mathcal{G} := f^{-1} \mathcal{G} \otimes_{f^{-1}\mathcal{O}_Y} \mathcal{O}_X$ eine $\mathcal{O}_X$-Modulgarbe. $f^*\mathcal{G}$ heißt \textit{Pullback} von $\mathcal{G}$ unter $f$.
\end{compactenum}
\begin{pr}
\begin{compactenum}
\item Für $U \subseteq Y$ offen ist $f_*\mathcal{F}(U)= \mathcal{F}\left(f^{-1}(U)\right)$ ein $\mathcal{O}_X\left(f^{-1}(U)\right)$-Modul. Der Garbenmorphismus $f^\#: \mathcal{O}_Y \la f_*\mathcal{O}_X$ induziert einen Ringhomomorphismus $f_U^\#: \mathcal{O}_Y(U) \la f_*\mathcal{O}_X(U) = \mathcal{O}_X\left(f^{-1}(U)\right)$, welcher die gewünschte $\mathcal{O}_Y(U)$-Modulstruktur liefert.
\item Zur Wohldefiniertheit brauchen wir noch einen Morphismus $f^{-1}\mathcal{O}_Y \la \mathcal{O}_X$. Der Garbenmorphismus $f^\#: \mathcal{O}_Y \la f_*\mathcal{O}_X$ liefert den Morphismus $f^{-1} \mathcal{O}_Y \la f^{-1}f_*\mathcal{O}_X$ und Proposition 2.16 liefert $f^{-1} f_* \mathcal{O}_X \la \mathcal{O}_X$, was zusammen die Behauptung liefert. $\hfill \Box$

\end{compactenum}


\end{pr}

\end{bemdefin}


\begin{remark}   %Bemerkung 9.9
Seien $X= \spec R$, $Y= \spec S$ affine Schemata, $f: X \la Y$ Morphismus mit zugehörigem Ringhomomorphismus $\alpha: S \la R$.
\begin{compactenum}
\item Für jeden $R$-Modul $M$ gilt: $f_* \tilde{M} = \widetilde{_{\alpha}M}$, wobei $_{\alpha}M$ die von $\alpha$ als $S$-Modul aufgefasste abelsche Gruppe $M$ bezeichne.
\item Für jeden $S$-Modul $N$ gilt $f^* \tilde{N} = \widetilde{N \otimes_S R}$.
\end{compactenum}
\begin{pr}
\begin{compactenum}
\item Für $U \subseteq Y$ offen ist $f_*\tilde{M}(U) = \tilde{M}(f^{-1}(U))$. Das wird durch $f_U^\#$ (von $\alpha$ induziert) zum $\mathcal{O}_Y(U)$-Modul. Für $U=D(g)$ gilt wegen $f^{-1}(D(g)) = D(g \circ f) = D(\alpha(g))$:
$$f_*\tilde{M}(U)=\tilde{M}(f^{-1}(U)) = M\left(D(\alpha(g))\right) = M_{\alpha(g)}= \widetilde{_{\alpha}M}_g = \widetilde{_{\alpha}M}(U).$$
\item Für die globalen Schnitte gilt
$$f^* \tilde{N}(X) = \left( f^{-1} \tilde{N} \otimes_{f^{-1}\mathcal{O}_Y} \mathcal{O}_X\right)(X) = N \otimes_S R$$
und für $U \subseteq X$ offen
\setlength{\abovedisplayskip}{5.5pt}
\setlength{\belowdisplayskip}{5.5pt}
\begin{alignat*}{5}
f^*\tilde{N}(U) \ \ &=&& \ \ \left(f^{-1}\tilde{N} \otimes_{f^{-1}\mathcal{O}_Y} \mathcal{O}_X\right)(U) \\
&=&& \ \ \left( N \otimes_S f^{-1}\mathcal{O}_Y(U) \right) \otimes_{f^{-1}\mathcal{O}_Y(U)} \mathcal{O}_X(U) \\
&=&& \ \ N \otimes_S \mathcal{O}_X(U)\\
&=&& \ \ \left(\widetilde{N \otimes_S R} \right) (U),
\end{alignat*}
also gerade die Behauptung. $\hfill \Box$

\end{compactenum}
\end{pr}
\end{remark}



\begin{proposition} %Proposition 9.10



Sei $f: X \la Y$ Morphismus von Schemata.
\begin{compactenum}
\item Ist $\mathcal{G}$ eine quasikohärente $\mathcal{O}_Y$-Modulgarbe auf $Y$, so ist $f^{*} \mathcal{G}$ eine quasikohärente $\mathcal{O}_X$-Modulgarbe auf $X$.
\item Sind $X,Y$ noethersch und $\mathcal{G}$ zusätzlich kohärent, so ist auch $f^*\mathcal{G}$ kohärent.
\item Ist $X$ noethersch und $\mathcal{F}$ eine quasikohärente $\mathcal{O}_X$-Modulgarbe auf $X$, so ist $f_* \mathcal{F}$ eine quasikohärente $\mathcal{O}_Y$-Modulgarbe auf $Y$.
\end{compactenum}

\begin{pr}
\begin{compactenum}
\item Die Eigenschaft quasikohärent zu sein ist eine lokale Eigenschaft, ohne Einschränkung sei also $X$ und damit auch $Y$ affin. Dann folgt die Aussage mit 9.9 aus $\mathcal{G} = \tilde{N}$ für einen $S$-Modul $N$ und $f^* \mathcal{G} = \widetilde{N \otimes_S R}$.
\item Ist $N$ als $S$-Modul erzeugt von den $n_1, \ldots, n_r$, so ist $N \otimes_S R$ erzeugt von den $n_1 \otimes 1, \ldots, n_r \otimes 1$, also insbesondere endlich erzeugt als $R$-Modul.
\item Ohne Einschränkung sei $Y$ affin. Da $X$ noethersch ist, gibt es eine endliche Überdeckung $X= \bigcup_{i=1}^r U_i$, ohne Einschränkung sei $U_i \cap U_j$ affin für alle $i,j$. $\mathcal{F}$ ist eine Garbe, die Sequenz
$$0 \la \mathcal{F} \overset{\alpha}{\la} \bigoplus_{i=1}^r \mathcal{F}\vert_{U_i} \overset{\beta}{\la} \bigoplus_{i < j} \mathcal{F} \vert_{U_i \cap U_j}$$
mit $\alpha(m) = \left( m \vert_{U_i}\right)_i$ und $\beta\left( (m_i)_i\right) = \left( m_i \vert_{U_i \cap U_j} - m_j \vert_{U_i \cap U_j} \right)_{i < j}$
ist also exakt. Der Funktor $f_*$ ist linksexakt, denn: Ist $0 \la \mathcal{F}' \la \mathcal{F} \la \mathcal{F}'' \la 0$ exakt, so ist die Sequenz $$0 \la f_* \mathcal{F}'(U) = \mathcal{F}\left(f^{-1}(U)\right) \la \mathcal{F}\left(f^{-1}(U)\right) \la \mathcal{F}'' \left(f^{-1}(U)\right)$$ ebenfalls exakt (2.9 und 2.14). Da nun $\bigoplus_{i=1}^r f_* \mathcal{F} \vert_{U_i}$ und $\bigoplus_{i<j} f_* \mathcal{F} \vert_{U_i \cap U_j}$ quasikohärent sind, ist $f_* \mathcal{F}$ als Kern eines Homomorphismus quasikohärenter Garben ebenfalls quasikohärent, was zu zeigen war. $\hfill \Box$



\end{compactenum}
\end{pr}

\end{proposition}






\renewcommand*\thesection{§ \arabic{section}\quad}
\section{Lokal freie Garben} %PARAGRAPH 10
\renewcommand*\thesection{\arabic{section}}



\begin{ex} %%Beispiel 10.1
Sei $X$ nichtsinguläre, projektive Kurve über einem algebraisch abgeschlossenen Körper $k$, $D= \sum_{P \in X} n_P P$ ein Divisor auf $X$ sowie $\mathcal{L}(D)$ die zu $D$ assoziierte $\mathcal{O}_X$-Modulgarbe
$$\mathcal{L}(D)(U) = \left\{ f \in k(X) \ \vert \ \left( \text{div} f + D \right)\vert_U \geqslant 0 \right\} \cup \{0\}.$$
Erinnerung: Ist $U$ "klein", so ist $\mathcal{L}(D)(U) = t_U \mathcal{O}_X(U)$. Außerdem gilt für den Halm in jeden Punkt $P \in X$ :
$$\mathcal{L}(D)_P = \left\{ f \in k(X) \ \vert \ \text{ord}_P(f) \geqslant -n_P \right\} \cup \{0\} = t_P^{-n_P} \mathcal{O}_{X,P}.$$
\end{ex}





\begin{remark}
Ist $X$ wie in Beispiel 10.1, so gilt für Divisoren $D,D'$ auf $X$
$$\mathcal{L}(D) \otimes_{\mathcal{O}_X} \mathcal{L}(D') \cong \mathcal{L}(D+D').$$
\begin{pr}
Für $U \subseteq X$ offen ist 
$$\psi: \mathcal{L}(D)(U) \times \mathcal{L}(D')(U) \la \mathcal{L}(D+D')(U), \qquad (f,g) \mapsto fg$$
eine wohldefinierte, bilineare Abbildung von $\mathcal{O}_X(U)$-Moduln, denn es gilt 
$$\left( \text{div} (fg) + (D+D') \right) \vert_U = \left( \text{div} f + D \right) \vert_U + \left( \text{div} g + D' \right) \vert_U \geqslant 0 + 0 = 0.$$
Damit induziert $\phi$ also eine $\mathcal{O}_X(U)$-lineare Abbildung
$$\phi_U: \left(\mathcal{L}(D) \otimes_{\mathcal{O}_X} \mathcal{L}(D') \right)(U) \la \mathcal{L}(D+D')(U).$$
Diese Abbildungen verkleben sich zu einem Garbenmorphismus 
$$\phi:\mathcal{L}(D) \otimes_{\mathcal{O}_X} \mathcal{L}(D') \la \mathcal{L}(D+D').$$
Nach Beispiel 10.1 haben wir in jedem Punkt eine Isomorphismus der Halme
$$\phi_P: t_P^{-n_P} \mathcal{O}_{X,P} \otimes_{\mathcal{O}_{X,P}} t_P^{-n_P'} \mathcal{O}_{X,P} \la t_P^{-n_P-n_P'} \mathcal{O}_{X,P},$$
$\phi$ ist also Isomorphismus. Beachte: Es gilt $\mathcal{L}(D) \otimes_{\mathcal{O}_X} \mathcal{L}(-D) \cong \mathcal{O}_X$. $\hfill \Box$
\end{pr}
\end{remark}


\newcommand{\lokring}{(X, \mathcal{O}_X)}
\newcommand{\oxg}{\mathcal{O}_X}


\begin{defin}   %%Deinfiiton 10.2
Sei $\lokring$ lokal geringter Raum, $\mathcal{F}$ eine $\oxg$-Modulgarbe, $n \in \mathbb{N}$.
\begin{compactenum}
\item $\mathcal{F}$ heißt \textit{frei von Rang} $n$, falls gilt $\mathcal{F} \cong \oxg^n = \bigoplus_{i=1}^n \oxg$.
\item $\mathcal{F}$ heißt \textit{lokal frei von Rang} $n$, wenn es eine offene Überdeckung $\{U_i\}_{i\in I}$ von $X$ gibt, sodass für alle $i \in I$ gilt $\mathcal{F}\vert_{U_i} \cong \left( \oxg\vert_{U_i}\right)^n$.
\end{compactenum}

\end{defin}


\begin{remark}   %%Bemerkung 10.3
Aus Freiheit folgt sicherlich lokale Freiheit, die Umkehrung ist im Allgemeinen allerdings nicht der Fall. Betrachte hierfür $X= \Pro^1_k$ für einen algebraisch abgeschlossenen Körper $k$ und für den Divisor $D=1 \cdot P$ auf $X$ für ein $P \in X$ die Garbe $\mathcal{L}(D)$ auf $X$. In Beispiel 10.1 haben wir bereits gesehen, dass $\mathcal{L}(D)$ lokal frei von Rang $1$ ist. Betrachte nun die globalen Schnitte von $\mathcal{L}(D)$ und der Strukturgarbe. Es gilt 
$$\mathcal{O}_{\Pro^1_k} \left( \Pro^1_k\right) = k$$
und 
\setlength{\abovedisplayskip}{5.5pt}
\setlength{\belowdisplayskip}{5.5pt}
\begin{alignat*}{5}
\mathcal{L}(D)(\Pro^1_k) \ \ &=&& \ \ \left\{ f \in k(X) \ \vert \ \mathrm{div} f + P \geqslant 0 \right\} \\
&=&& \ \ \left\{ f \in k(X) \ \vert \ \mathrm{ord}_Qf \geqslant 0 \text{ für alle Q } \in \Pro^1 \setminus \{P\} \right\} \oplus \left\{ f \in k(X) \ \vert \ \mathrm{ord}_P f \geqslant -1 \right\}\\
&=&& \ \ \left\{ f \in k(X) \ \vert \ f \in \mathcal{O}_{\Pro^1_k}\left( \Pro^1_k \setminus \{P\} \right) \right\} \oplus \left\{ f \in k(X) \ \vert \ \mathrm{ord}_P f \geqslant -1 \right\}\\
&=&& \ \ k \oplus \frac{1}{X-X_P} k,
\end{alignat*}
womit die Garben nicht isomorph sein können.

\end{remark}


\begin{remark} %%Bemerkung 10.4

Ist $\lokring$ ein Schema, so ist jede lokalfreie Garbe auf $X$ quasikohärent. Ist $X$ weiterhin noethersch, so ist jede lokalfreie Garbe sogar kohärent.
\begin{pr} Sei $\F$ eine lokal freie Garbe von Rang $n$ sowie $\{U_i=\spec R_i\}_{i \in I}$ eine offene, ohne Einschränkung affine Überdeckung von $X$ derart, dass $\mathcal{F} \vert_{U_i} = (\mathcal{O}_X \vert_{U_i})^n$ für alle $i \in I$. Dann gilt $\mathcal{F}\vert_{U_i} = \tilde{R_i}^n$. Ist zudem $X$ noethersch, so ist $R_i$ noethersch für alle $i \in I$, $\F$ also kohärent. $\hfill \Box$
\end{pr}
\end{remark}


\begin{remark}   %%Bemerkung 10.5
Jede lokalfreie Garbe $\mathcal{L}$ von Rang $1$ auf einer nichtsingulären, projektiven Kurve über einem algebraisch abgeschlossenen Körper $k$, die sich einbetten lässt in die konstante Garbe des Funktionenkörpers, ist isomorph zu einer Garbe $\mathcal{L}(D)$ für einen Divisor $D$ auf $X$.
\begin{pr}
Sei $\{U_i\}$, $i \in \{1, \ldots, n \}$ offene Überdeckung von $X$ mit $\mathcal{L}\vert_{U_i} \cong t_i \mathcal{O}_X \vert_{U_i}$ mit Erzeugern $t_i \in k(X) \cap \mathcal{L}(U_i)$. Es ist $t_i \mathcal{O}_X \vert_{U_i \cap U_j} = t_j \mathcal{O}_{U_i \cap U_j}$, das heißt es gilt $\frac{t_i}{t_j} \in \mathcal{O}_X( U_i \cap U_j)^{\times}$. Damit gilt 
$$\mathrm{div\ } t_i\vert_{U_i \cap U_j} = \mathrm{div\ } \frac{t_jt_i}{t_j} \bigg\vert_{U_i \cap U_j} = \mathrm{div\ } \frac{t_i}{t_j} \bigg\vert_{U_i \cap U_j} + \mathrm{div\ } t_j \vert_{U_i \cap U_j} = \mathrm{div\ }t_j \vert_{U_i \cap U_j},$$
das heißt, wir können einen wohldefinierten Divisor $D$ auf $X$ durch $D\vert_{U_i} = \mathrm{div\ } \frac{1}{t_i} \bigg\vert_{U_i}$ definieren. Dann gilt $\mathcal{L} \cong \mathcal{L}(D)$, denn für die Schnitte erhalten wir
\setlength{\abovedisplayskip}{5.5pt}
\setlength{\belowdisplayskip}{5.5pt}
\begin{alignat*}{5}
\mathcal{L}(U_i) \ \ &=&& \ \ \left\{t_i f \ \vert \ f \in \mathcal{O}_X(U_i) \right\} = \left\{ t_i f \ \vert \ \mathrm{div\ }f\vert_{U_i} \geqslant 0\right\}\\
&=&& \ \ \left\{ f \in \mathcal{O}_X(U_i) \ \vert \ \mathrm{div\ }\frac{f}{t_i} \bigg\vert_{U_i} \geqslant 0 \right\}\\
&=&& \ \ \left\{ f \in \mathcal{O}_X(U_i) \ \vert \ \mathrm{div\ }f  \vert_{U_i} \geqslant \mathrm{div\ }t_i \vert_{U_i} \right\}\\
&=&& \ \ \left\{ f \in k(X) \ \vert \ \left( \mathrm{div\ } f - \mathrm{div\ } t_i \right) \vert_{U_i} \geqslant 0 \right\}\\
&=&& \ \ \left\{f \in k(X) \ \vert \ \left( \mathrm{div\ } f +\mathrm{div\ } \frac{1}{t_i} \right) \bigg\vert_{U_i} \geqslant0\right\}\\
&=&& \ \ \left\{ f \in k(X) \ \vert \ \left( \mathrm{div\ } + D\right) \vert_{U_i} \geqslant 0 \right\}\\
&=&& \ \ \mathcal{L}(D)(U_i),
\end{alignat*}
woraus die Behauptung folgt. $\hfill \Box$

\end{pr}
\end{remark}


\begin{proposition}
Sei $\lokring$ lokal geringter Raum, $\mathcal{F}$ lokal freie Garbe  von Rang $n$ auf $X$. Dann gilt:
\begin{compactenum}
\item Ist $\mathcal{G}$ lokal frei von Rang $m$, so ist $\mathcal{F} \otimes_{\oxg} \mathcal{G}$ lokal frei von Rang $mn$.
\item Ist $f:Y \la X$ Morphismus von lokal geringten Räumen, so ist $f^*\mathcal{F}$ lokal freie Garbe von Rang $n$ auf $Y$.
\end{compactenum}
\textit{Warnung}: Es gibt keine entsprechende Aussage für $f_*$.
\begin{pr}
\begin{compactenum}
\item Wähle eine ausreichend feine Überdeckung $\{U_i\}_{ \in I}$ von $X$, sodass 
$$\mathcal{F} \vert_{U_i} \cong \left( \mathcal{O}_X \vert_{U_i} \right)^n, \qquad \mathcal{G}\vert_{U_i} \left( \oxg\vert_{U_i} \right)^m$$
und wähle Basen $t_{i1}, \ldots, t_{in}$ von $\mathcal{F}\vert_{U_i}$ und $s_{i1}, \ldots, s_{im}$ von $\mathcal{G}\vert_{U_i}$ als $\mathcal{O}_X(U_i)$-Moduln. Dann bilden die $t_{i1} \otimes s_{i1}, \ldots, t_{i1} \otimes s_{im}, \ldots, t_{in} \otimes s_{im}$ eine Basis von $\mathcal{F} \otimes_{\mathcal{O}_X} \mathcal{G}$, woraus die Behauptung folgt.
\item Sei $U \subseteq X$ offen mit $\mathcal{F} \vert_U \cong \left( \oxg\vert_U\right)^n$. Dann ist
\setlength{\abovedisplayskip}{5.5pt}
\setlength{\belowdisplayskip}{5.5pt}
\begin{alignat*}{5}
f^*\mathcal{F} \vert_{f^{-1}(U)} \ \ &=&& \ \ \left( f^{-1}\mathcal{F} \otimes_{f^{-1}\mathcal{O}_X} \mathcal{O}_Y\right)\bigg\vert_{f^{-1}(U)}\\
&=&& \ \ (f^{-1}\mathcal{F}) \vert_{f^{-1}(U)} \otimes_{f^{-1}\mathcal{O}_X \vert_{f^{-1}(U)}} \mathcal{O}_Y \vert_{f^{-1}(U)} \\
&=&& \ \ f^{-1}\left(\mathcal{F}\vert_U\right) \otimes_{f^{-1}\mathcal{O}_X \vert_{f^{-1}(U)}} \mathcal{O}_Y \vert_{f^{-1}(U)} \\
&=&& \ \ f^{-1} \left(\oxg \vert_U\right)^n \otimes_{f^{-1}\mathcal{O}_X \vert_{f^{-1}(U)}} \mathcal{O}_Y \vert_{f^{-1}(U)} \\
&=&& \ \ \left( f^{-1} \mathcal{O}_X\right)^n \vert_{f^{-1}(U)} \otimes_{f^{-1}\mathcal{O}_X \vert_{f^{-1}(U)}} \mathcal{O}_Y \vert_{f^{-1}(U)} \\
&=&& \ \ \left( f^{-1} \mathcal{O}_X \big\vert_{f^{-1}(U)} \otimes_{f^{-1}\mathcal{O}_X \vert_{f^{-1}(U)}} \mathcal{O}_Y \vert_{f^{-1}(U)} \right)^n \\
&=&& \ \ \left( f^{-1} \mathcal{O}_X  \otimes_{f^{-1}\mathcal{O}_X } \mathcal{O}_Y   \big\vert_{f^{-1}(U)}\right)^n\\
&=&& \ \ \left( \mathcal{O}_Y\vert_{f^{-1}(U)}\right)^n
\end{alignat*}
Ist nun $\{U_i\}_{i\in I}$ eine offene Überdeckung von $X$, so ist auch $\{f^{-1}(U)\}$ eine offene Überdeckung für $Y$ und damit ist $f^*\mathcal{F}$ lokal frei von Rang $n$ wie gewünscht. $\hfill \Box$

\end{compactenum}
\end{pr}

\end{proposition}



\begin{definprop}
Sei $\lokring$ lokal geringter Raum sowie $\mathcal{F}$, $\mathcal{G}$ $\mathcal{O}_X$-Modulgarben auf $X$. Für $U \subseteq X$ offen sei
$$\boldsymbol{\mathrm{Hom}}_{\mathcal{O}_X}(\mathcal{F}, \mathcal{G})(U) := \mathrm{Hom}_{\mathcal{O}_X\vert_U}\left( \mathcal{F}\vert_U, \mathcal{G}\vert_U \right).$$
Dann ist $\boldsymbol{\mathrm{Hom}}_{\mathcal{O}_X}(\mathcal{F}, \mathcal{G})$ eine $\mathcal{O}_X$-Modulgarbe.
\begin{pr}
Sei $U \subseteq X$ offen. Klar: $\boldsymbol{\mathrm{Hom}}_{\mathcal{O}_X}(\mathcal{F}, \mathcal{G})(U)$ ist abelsche Gruppe. Wir brauchen also nur noch eine $\mathcal{O}_X(U)$-Modulstruktur. Für $\alpha \in \mathcal{O}_X(U)$ und $\phi \in \mathrm{Hom}_{\mathcal{O}_X\vert_U} \left( \mathcal{F}\vert_U, \mathcal{G}\vert_U\right)$. Definiere $\alpha \phi$ wie folgt: Für jede offene Teilmenge $V \subseteq U$ sei
$$(\alpha \phi)_V: \mathcal{F}(V) \la \mathcal{G}(V), \qquad (\alpha\phi)_V(s) := \alpha\vert_V \phi_V(s).$$
Die $(\alpha\phi)_V$ ergeben den gewünschten Garbenmorphismus $\alpha\phi: \mathcal{F}\vert_U \la \mathcal{G}\vert_U$, womit also \linebreak $\boldsymbol{\mathrm{Hom}}_{\mathcal{O}_X}(\mathcal{F}, \mathcal{G})(U)$ zum $\mathcal{O}_X(U)$-Modul wird. Es bleibt noch zu zeigen: die Zuordnung $U \mapsto \mathrm{Hom}_{\mathcal{O}_X\vert_U}\left( \mathcal{F}\vert_U, \mathcal{G}\vert_U \right)$ ist eine Garbe. Übung! $\hfill \Box$


\end{pr}


\end{definprop}


\newcommand{\homg}{\boldsymbol{\mathrm{Hom}}_{\mathcal{O}_X}(\mathcal{F}, \mathcal{O}_X)}

\begin{definprop}

Sei $\lokring$ lokal geringter Raum, $\mathcal{F}$ lokal freie Garbe von Rang $n$ auf $X$.
\begin{compactenum}
\item $\mathcal{F}^* := \boldsymbol{\mathrm{Hom}}_{\mathcal{O}_X}(\mathcal{F}, \mathcal{O}_X)$ ist lokal frei von Rang $n$.
\item $\mathcal{F}^*$ heißt \textit{die zu} $\mathcal{F}$ \textit{duale Garbe}.
\item Für jede $\mathcal{O}_X$-Modulgarbe $\mathcal{G}$ auf $X$ gilt 
$$\boldsymbol{\mathrm{Hom}}_{\mathcal{O}_X}(\mathcal{F}, \mathcal{G}) \cong \mathcal{F}^* \otimes_{\mathcal{O}_X} \mathcal{G}.$$
\end{compactenum}
\begin{pr}
\begin{compactenum}
\item Es genügt zu zeigen: Ist $\{U_i\}_{i \in I}$ offene Überdeckung von $X$, so ist $\homg \vert_{U_i}$ frei. Es sei also ohne Einschränkung $\mathcal{F} \cong \mathcal{O}_X^n$. Dann ist zu zeigen: 
$$\homg  = \boldsymbol{\mathrm{Hom}}_{\mathcal{O}_X}\left(\mathcal{O}_X^n, \mathcal{O}_X\right) \overset{!}{\cong} \mathcal{O}_X^n.$$
Aus der linearen Algebra wissen wir, dass für einen Körper $k$ gilt: $\mathrm{Hom}_k(k^n,k) \cong k^n$. Der zugehörige Isomorphismus ist $l \mapsto \left(l(e_1), \ldots, l(e_n)\right)$, wobei $\{e_1, \ldots, e_n\}$ eine Basis des $k^n$ ist. Dieselbe Aussage kann auf freie Moduln übertragen werden, woraus die Behauptung folgt.
\item[(iii)] Die entsprechende Aussage aus der linearen Algebra für $k$-Vektorräume $V$ und $W$ lautet
$$\mathrm{Hom}_k(V,W) \cong V^* \otimes_k W,$$
denn: Betrachte die bilineare Abbildung 
$$\psi: V^* \times W \la \mathrm{Hom}_k(V,W), \qquad (l,w) \mapsto (\alpha: V \la W, v \mapsto l(v) w).$$
Es induziert $\psi$ eine Abbildung $\phi: V^* \otimes W \la \mathrm{Hom}_k(V,W)$. Wir müssen zeigen: $\phi$ ist bijektiv. Surjektivität sehen wir wie folgt ein: Sind Basen $\{b_1, \ldots, b_n\}$ für $V$ und $\{c_1, \ldots, c_m\}$ für $W$ gegeben, so wird $\mathrm{Hom}_k(V,W)$ erzeugt von den $f_{ij}$ für $1 \leqslant i \leqslant n$ und $1 \leqslant j \leqslant m$, wobei $f_{ij}$ gegeben ist durch $f_{ij}(b_k) = \delta_{ik}c_j$. Ist $\{b_1^*, \ldots, b_n^*\}$ die zu $\{b_1, \ldots b_n\}$ duale Basis von $V^*$, so erhalten wir die Darstellung
$f_{ij}= \psi(b_i^*, c_j)$, das heißt, $\phi$ ist surjektiv. Nun gilt 
$$\dim V^* \otimes_k W = \dim V^* \dim W = \dim V \dim W = nm = \dim \mathrm{Hom}_k(V,W),$$
also ist $\phi$ auch injektiv und damit bereits Isomorphismus und die Behauptung folgt. $\hfill \Box$


\end{compactenum}
\end{pr}
\end{definprop}



\begin{definprop}
Sei $\lokring$ lokal gringter Raum.
\begin{compactenum}
\item Für jede lokal freie $\oxg$-Modulgarbe von Rang $1$ gilt $\mathcal{L} \otimes_{\oxg} \mathcal{L}^{*} \cong \oxg$.
\item Eine $\oxg$-Modulgarbe $\mathcal{L}$ heißt \textit{invertierbar}, falls es eine $\mathcal{O}_X$-Modulgarbe $\mathcal{L}'$ gibt, sodass $\mathcal{L} \otimes_{\oxg} \mathcal{L}' \cong \oxg$.
\item Ist $\lokring$ noethersches Schema, so ist jede invertierbare $\mathcal{O}_X$-Modulgarbe auf $X$ lokal frei von Rang $1$.
\item Die Isomorphieklassen der invertierbaren $\mathcal{O}_X$-Modulgarben auf $X$ bilden eine abelsche Gruppe, die sogenannte \textit{Picard-Gruppe} $\mathrm{Pic}(X)$.

\end{compactenum}
\begin{pr}
\begin{compactenum}
\item[(i)] Nach 10.9 gilt $\mathcal{L} \otimes_{\mathcal{O}_X} \mathcal{L}^* \cong \boldsymbol{\mathrm{Hom}}_{\mathcal{O}_X}(\mathcal{L}, \mathcal{L})$. Definiere nun
$$\phi: \mathcal{O}_X \la \boldsymbol{\mathrm{Hom}}_{\mathcal{O}_X}(\mathcal{L}, \mathcal{L}), \qquad 1 \mapsto \mathrm{id}_{\mathcal{L}}.$$
Dann ist $\phi$ ein wohldefinierter, injektiver Morphismus von $\mathcal{O}_X$-Modulgarben. Für den Beweis genügt es nun, die Surjetkivität von $\phi$ nachzuweisen. Dies zeigen wir halmweise. Sei $x \in X$ und betrachte $\phi_x$. Sei $\alpha \in \mathrm{Hom}_{\mathcal{O}_{X,x}}(\mathcal{L}_x, \mathcal{L}_x)=\left(\boldsymbol{\mathrm{Hom}}_{\mathcal{O}_X}(\mathcal{F}, \mathcal{F}) \right)_x$, also $\alpha=(U,s)$, wobei ohne Einschränkung $U$ klein genug ist, sodass $\mathcal{L}\vert_U \cong \mathcal{O}_X\vert_U$. Dann gilt $s= \phi_x(s(1))$, also ist $\phi_x$ und damit $\phi$ surjektiv.
\item[(iii)] Übung.
\item[(v)] Die Strukturgarbe $\oxg$ ist neutral bezüglich des Tensorprodukt (welches auch assoziativ und kommutativ ist) und inverses Element folgt aus (i). $\hfill \Box$
\end{compactenum}
\end{pr}

\end{definprop}




\begin{ex}
Sei $X$ differenzierbare Mannigfaltigkeit, $\mathcal{O}_X$ die Garbe der $C^{\infty}$-Funktionen auf $X$. Dann ist $\lokring$ lokal geringter Raum. Sei $E$ eine weitere differenzierbare Mannigfaltigkeit und $p: E \la X$ differenzierbare Abbildung. Das Paar $(E,p)$ heißt \textit{Vektorbündel von Rang $n$ über $X$}, falls es eine offene Überdeckung $\{U_i\}_{i \in I}$ von $X$ und für jedes $i \in I$ einen Diffeomorphismus $\phi_i: p^{-1}(U_i) \la U_i \times \mathbb{R}^n$ gibt, sodass das Diagramm
$$
\begin{xy}
\xymatrix{
E \supseteq p^{-1}(U_i) \ar[rrrr]^{\phi_i}  \ar[rrd]_{p} &&&& U_i \times \mathbb{R}^n \ar[lld]^{\pi_1} \\
&& U_i&&
}
\end{xy}
$$
kommutiert, also $p = \pi_1 \circ \phi_i$, und gilt
$$\phi_{ij} := \phi_i \circ \phi_j^{-1}: \left(U_i \cap U_j\right) \times \mathbb{R}^n \la (U_i \cap U_j) \mathbb{R}^n$$
faserweise linear ist für alle $i,j \in I$, nach Wahl eine Basis des $\mathbb{R}^n$ also durch eine Matrix $A=(A_{ij}) \in \mathrm{GL}_n\left(\mathcal{O}_X(U_i \cap U_j)\right)$ dargestellt wird. Im folgenden wollen wir zeigen, dass die Vektorbündel von Rang $n$ auf $X$ gerade den lokal freien $\oxg$-Modulgarben von Rang $n$ auf $X$ entsprechen. Sei dazu zunächst $(E,p)$ ein Vektorbündel von Rang $n$ auf $X$ wie oben definiert und $\mathcal{E}$ die Garbe der Schnitte in $E$ auf $X$, für offene Teilmengen $U \subseteq X$ gilt also 
$$\mathcal{E}(U) = \left\{s: U \la E \ \vert \ s \textrm{ ist differenzierbare Abbildung mit } p \circ s = \mathrm{id}_U \right\}.$$
Dann ist $\mathcal{E}$ lokal frei von Rang $n$, denn für $i \in I$ gilt
$$\mathcal{E}(U_i) = \left\{ s: U_i \la \mathbb{R}^n \ \vert \ s \textrm{ ist differenzierbar } \right\} = \mathcal{O}_X(U_i)^n.$$
Dasselbe erhalten wir für Einschränkungen auf beliebige offene $V \subseteq U_i$.\\
Sei nun umgekehrt $\mathcal{E}$ lokal freie $\mathcal{O}_X$-Modulgabe von Rang $n$ auf $X$. Dann ist für jedes $x \in X$ der Halm $\mathcal{E}_x$ ein freier $\mathcal{O}_{X,x}$-Modul von Rang $n$. Weiter gilt $\slant{\mathcal{O}_{X,x}}{\mathfrak{m}_x} \cong \mathbb{R}$ sowie $\slant{\mathcal{E}_x}{\mathfrak{m}_x \mathcal{E}_x} \cong \mathbb{R}^n$. Ist nun $U_i\subseteq X$ offen mit $\mathcal{E}\vert_{U_i} \cong \left( \oxg \vert_{U_i}\right)^n$ via $\phi_i: \mathcal{E}\vert_{U_i} \la \left( \oxg\vert_{U_i}\right)^n$, so ist $\phi_i \phi_j^{-1}$ ein $\mathcal{O}_X\vert_{U_i \cap U_j}$-Modulgarbenisomorphsmus von $\left( \mathcal{O}_X \vert_{U_i \cap U_j} \right)^n$ auf sich selbst, also ein Element $A_{ij} \in \mathrm{GL}_n\left(\mathcal{O}_X(U_i \cap U_j)\right)$. In jedem $x \in U_i \cap U_j$ induziert $A_{ij}$ einen Vektorraumisomorphismus von $\slant{\mathcal{E}_x}{\mathfrak{m}_x\mathcal{E}_x} \cong \mathbb{R}^n$. Verklebt man nun die $U_i \times \mathbb{R}^n$ mithilfe der $A_{ij}$ zum Vektorbündel $E$, so erhält man die gewünschte Aussage.


\end{ex}


\begin{definprop}

Sei $\lokring$ ein Schema, $p: E \la X$ Morphismus von Schemata.
\begin{compactenum}
\item $(E,p)$ heißt \textit{geometrisches Vektorbündel von Rang $n$ über $X$}, falls es eine offene Überdeckung $\{U_i\}_{i \in I}$ von $X$ und für jedes $i \in I$ Isomorphismen $\phi_i: p^{-1}(U_i) \la \mathbb{A}^n_{U_i} = U_i \times_{\spec \mathbb{Z}} \mathbb{A}^n_{\mathbb{Z}}$ gibt, sodass für alle $i,j \in I$ und jedes affine offene Unterschema $\spec R=U \subseteq U_i \cap U_j$ von $X$ die Abbildungen $\phi_i \circ \phi_j^{-1}$ $R$-lineare Automorphismen sind, also von linearen Automorphismen von $R[X_1, \ldots, X_n]$ induziert werden.
\item Die Isomorphieklassen von geometrischen Vektorbündeln von Rang $n$ entsprechen bijektiv den Isomorphieklassen von lokal freien $\mathcal{O}_X$-Modulgarben von Rang $n$ auf $X$.
\end{compactenum}
\begin{pr}
Wie in 10.11 $\hfill \Box$
\end{pr}
\end{definprop}


\renewcommand*\thesection{§ \arabic{section}\quad}
\section{Divisoren und invertierbare Garben} %PARAGRAPH 11
\renewcommand*\thesection{\arabic{section}}

Erinnerung: Ist $X$ nichtsinguläre, projektive Kurve über einem algebraisch abgeschlossenen Körper $k$ und $D= \sum_{P \in X} n_P P$ ein Divisor auf $X$, so wird durch 
$$\mathcal{L}(D)(U) = \left\{ f \in k(X) \ \vert \ \mathrm{ord}_Pf+ n_P >0  \textrm{ für alle } p \in U\right\}$$
eine lokal freie Garbe von Rang $1$ auf $X$ definiert.


\begin{ex}
Betrachte den Newtonknoten $X= V(Y^2-X^3-X^2) \subseteq \Pro^2_k$ in der projektiven Ebene und definere einen Divisor durch $D= 1 \cdot P_0$, wobei $P_0$ den singulären Punkt der Kurve bezeichne. Können wir nun auch die Garbe $\mathcal{L}(D)$ definieren? Betrachte das maximale Ideal im Punkt $P_0$: Es wird erzeugt von den Restklassen von $X,Y$, bezeichne sie mit $x,y$. Es gilt $y^2=x^2(x-1)$, es kann aber auf keinen Erzeuger verzichtet werden, $\mathfrak{m}_{P_0}$ ist also kein Hauptideal. Wie kann man dann $\mathrm{ord}_{P_0} f$ bestimmen? Ist $\mathrm{ord}_{P_0} x=2$? Betrachte den Faktorring $\slant{\mathcal{O}_{X,P_0}}{(f)}$ und setze $\mathrm{ord}_{P_0} := \dim_k \slant{\mathcal{O}_{X,P_0}}{(f)}$ und erhalte beispielsweise $\mathrm{ord}_{P_0} x = 2$ (denn in $\slant{\mathcal{O}_{X,P_0}}{(x)}$ sind $1$, $y$ linear unabhängig). Wir können die Garbe $\mathcal{L}(D)$ als konstruieren, für den Halm in $P_0$ gilt aber
$$\mathcal{L}(D)_{P_0} = \left\{f \in k(X) \ \vert \ \mathrm{ord}_{P_0} f \geqslant 1 \right\} = \mathfrak{m}_P,$$
weswegen dieser nicht frei von Rang $1$ ist und $\mathcal{L}(D)$ damit nicht lokal frei von Rang $1$, also nicht invertierbar ist.

\end{ex}



\begin{defin}
Ein Schema $\lokring$ heißt \textit{integer}, falls es reduziert und irreduzibel ist.
\end{defin}


\begin{definbem}
Sei $\lokring$ ein Schema.
\begin{compactenum}
\item Ein \textit{Primdivisor} auf $X$ ist ein integres, abgeschlossenes Unterschema der Kodimension $1$.
\item Die von den Primdivisoren auf $X$ erzeugte frei abelsche Gruppe $\mathrm{Div}(X)$ heißt \textit{Gruppe der Weil-Divisoren}. Die Elemente von $\mathrm{Div}(X)$ heißen \textit{Weil-Divisoren}.
\item Ist $X$ eine Kurve über einem algebraisch abgeschlossenen Körper $k$, so sind die Primdivisoren gerade die abgeschlossenen Punkte in $X$ und die Weil-Divisoren von der Form $D=\sum_{P \in X} n_P P$ für gewisse $n_P \in \mathbb{Z}$.
\item Sei $W$ ein Primdivisor auf $X$, $\gamma_W$ der generische Punkt von $W$.
\end{compactenum}
\end{definbem}


































\chapter{Kohomologie von Garben} %KAPITEL III
\setlength\abovedisplayshortskip{0pt}
\setlength\belowdisplayshortskip{10pt}
\setlength\abovedisplayskip{10pt}
\setlength\belowdisplayskip{10pt}

\newcommand{\C}{\mathcal{C}}
\newcommand{\coker}{\mathrm{coker}}

\setcounter{section}{11}
\renewcommand*\thesection{§ \arabic{section}\quad}
\section{Garbenkohomologie als abgeleiteter Funktor} %PARAGRAPH 12
\renewcommand*\thesection{\arabic{section}}
\thispagestyle{empty}


\begin{er}   %er 12.0
Sei $\mathcal{C}$ eine Kategorie mit Nullobjekten, $A,B \in \mathrm{Ob}(\C)$ und $f: A \la B$ ein Morphismus.
\begin{compactenum}
\item Der \textit{Kern} von $f$ ist das Paar $(\ker f,  \iota)$ mit $\iota: \ker f \la A$ und $f \circ \iota = 0$, sodass für jedes Paar $(C, \tilde{\iota})$ mit $C \in \mathrm{Ob}(\C)$ und $\tilde{\iota}: C \la A$ mit $f \circ \tilde{\iota} = 0$ der Morphismus $\tilde{\iota}$ eindeutig über $\ker f$ faktorisiert, es also einen eindeutigen Morphismus $h: C \la \ker f$ gibt, sodass das folgende Diagramm kommutiert:
$$
\begin{xy}
\xymatrix{
\ker f \ar[rr]^{\iota} && A \ar[rr]^f && B \\
& C \ar[lu]^{h} \ar[ru]_{\tilde{\iota}} & &&
}
\end{xy}
$$

\item der \textit{Kokern} von $f$ ist das Paar $(\coker f, \pi)$ mit $\pi: B \la \coker f$ und $\pi \circ f=0$, sodass für jedes Paar $(C, \tilde{\pi})$ mit $C \in \mathrm{Ob}(\C)$ und $\tilde{\pi}: B \la C$ mit $\tilde{\pi} \circ f=0$ der Morphismus $\tilde{\pi}$ eindeutig über $\coker f$ faktorisiert, es also einen eindeutigen Morphismus $h: \coker f \la C$ gibt, sodass das folgende Diagramm kommutiert:
$$
\begin{xy}
\xymatrix{
A \ar[rr]^f && B \ar[rr]^{\pi} \ar[rd]_{\tilde{\pi}} && \coker f \ar[ld]^{h} \\ &&& C & 
}
\end{xy}
$$

\end{compactenum}
\end{er}



\begin{defin}  %%defin 12.
Eine Kategorie $\mathcal{C}$ heißt \textit{abelsch}, falls folgende Eigenschaften erfüllt sind:
\begin{compactenum}
\item Es bildet $\Hom_{\C}(A,B)$ eine abelsche Gruppe bezüglich der Addition $"+"$ für alle $A,B \in \mathrm{Ob}(\C)$.
\item Für Homomorphismen gelten die Distributivgesetze bezüglich $"+"$ und $"\circ"$, das heißt es gilt
$$(f+g) \circ h = f\circ h + g \circ h, \qquad e \circ (f +g) = e \circ f + e \circ g$$
für alle $h \in \mathrm{Mor}_{\mathcal{C}}(A,B)$, $f,g \in \mathrm{Mor}_{\mathcal{C}}(B,C)$, $e \in \mathrm{Mor}_{\mathcal{C}}(C,D)$.
\item Endliche direkte Summen, Kerne, Kokerne existieren.
\item Jeder Monomorphismus ist der Kern seines Kokerns.
\item Jeder Epimorphismus ist der Kokern seines Kerns.

\end{compactenum}
\end{defin}


\begin{ex}
Beispiele für abelsche Kategorien sind $\abgrup, \kvr, \ringe, \Rmod, \Oxmod$. Nicht abelsch dagegen sind beispielsweise die Kategorien $\grp, \sets$.

\end{ex}

\begin{defin} %%Defin 12.2
Sei $\C$ eine abelsche Kategorie.
\begin{compactenum}
\item Ein \textit{Komplex} in $\C$ ist eine Sequenz
$$C^{\bullet} := \qquad \ldots \la \ C^{i-1}\ \overset{d^{i-1}}{\la}\ C^{i} \ \overset{d^{i}}{\la} \ C^{i+1} \ \la \ldots $$
von Morphismen in $\C$, sodass gilt $d^{i} \circ d^{i-1} = 0$ für alle $i \in \mathbb{Z}$. 

\item Für einen Komplex $C^{\bullet}$ in $\C$ heißt
$$H^{i}(C^{\bullet}) := \slant{\kernel d^{i}}{\bild d^{i-1}}$$
das $i$-te Kohomologieobjekt von $C^{\bullet}$. 
\end{compactenum}

\end{defin}

\begin{proposition} %% Proposition 12.3
Sei $\C$ eine abelsche Kategorie.
\begin{compactenum}
\item Die Komplexe in $\C$ bilden eine Kategorie $\C^{\bullet}$ mit Morphismen
$$
\begin{xy}
\xymatrix{
C^{\bullet} \ar[d]_{\alpha} && \ldots \ar[rr] && C^{i-1} \ar[d]_{\alpha_{i-1}} \ar[rr]^{d^{i-1}} && C^{i} \ar[d]_{\alpha_{i}} \ar[rr]^{d^{i}} && C^{i+1} \ar[rr] \ar[d]_{\alpha_{i+1}} && \ldots \\
D^{\bullet} && \ldots \ar[rr] && D^{i-1} \ar[rr]_{d'^{i-1}} && D^{i} \ar[rr]_{d'^{i}} && D^{i+1} \ar[rr] && \ldots
}
\end{xy}
$$
\item $H^{i}$ ist ein kovarianter, linksexakter Funktor $\C^{\bullet} \la \C$.
\item Zu jeder kurzen exakten Sequenz $0 \la C'^{\bullet} \overset{\alpha}{\la} C^{\bullet} \overset{\beta}{\la} C''^{\bullet} \la 0$ von Komplexen in $\C^{\bullet}$ gibt es eine lange exakte \textit{Kohomologiesequenz}
$$\ldots \la H^{i}(C'^{\bullet}) \la H^{i}(C^{\bullet}) \la H^{i}(C''^{\bullet}) \overset{\delta^{i}}{\la} H^{i+1}(C'^{\bullet}) \la H^{i+1}(C^{\bullet}) \la \ldots.$$

\end{compactenum}

\begin{pr}
\begin{compactenum}
\item[(ii)] Sei $\alpha: C^{\bullet} \la D^{\bullet}$ Morphismus von Komplexen in $\C^{\bullet}$ wie in (i), wir haben also für alle $i \in \mathbb{Z}$ Morphismen $\alpha_i: C^{i} \la D^{i}$ gegeben. Wir suchen nun 
$$\tilde{\alpha}_i: H^{i}(C^{\bullet}) = \slant{\kernel d^{i}}{\bild d^{i-1}}  \la \slant{\kernel d'^{i}}{\bild d'^{i-1}} = H^{i}(D^{\bullet})$$
Für $x \in \kernel d^{i}$ ist 
$$0 = \alpha_{i+1} \circ d^{i} = d'^{i} \circ \alpha_i,$$
also $\alpha_i \vert_{\kernel d^{i}} \in \Hom \left( \kernel d^{i}, \kernel d'^{i}\right)$ und damit induziert $\alpha_i$ durch Restklassenbildung die Abbildung $\tilde{\alpha}_i: \kernel d^{i} \la H^{i}(D^{\bullet})$. Wir müssen noch zeigen: $\bild d^{i-1} \subseteq \kernel \tilde{\alpha}_i$. Sei also $x = d^{i-1}(y)$ für ein $y \in C^{i-1}$. Dann gilt 
$$\alpha_i(x) = \alpha_i(d^{i-1}(y)) = d'^{i-1}(\alpha_{i-1}(y)) \ \in \bild (d'^{i-1}),$$
also $\tilde{\alpha}_i(x) = 0$ in $H^{i}(D^{\bullet})$.

\item[(iii)] Wir haben folgende Situation:
$$
\begin{xy}
\xymatrix{
& 0 \ar[d] & 0 \ar[d] & 0 \ar[d] & 0 \ar[d] & \\
\ldots \ar[r] & \ C'^{i-1}\  \ar[d] \ar[r] & \ C'^{i}\  \ar[d]_{\alpha_i} \ar[r]^{d'^{i}} &  \ C'^{i+1} \ \ar[d]_{\alpha_{i+1}} \ar[r]^{d'^{i+1}} & \ \ C'^{i+2}\ \ \ar[d]_{\alpha_{i+2}} \ar[r] &\ \ \ldots \\
\ldots \ar[r] &\ \ C^{i-1}\ \ \ar[d] \ar[r] & \ \ C^{i}\ \  \ar[d]_{\beta_i} \ar[r]^{d^{i}} & \ \ C^{i+1}\ \ \ar[d]_{\beta_{i+1}} \ar[r]^{d^{i+1}} & \ \ C^{i+2}\ \  \ar[d]_{\beta_{i+2}} \ar[r] & \ \ \ldots \\
\ldots \ar[r] & \ \ C''^{i-1} \ \ \ar[d] \ar[r] &\ \  C''^{i}\ \  \ar[d] \ar[r]^{d''^{i}} & \ \ C''^{i+1} \ \ \ar[d] \ar[r]^{d''^{i+1}} & \ \ C''^{i+2} \ \ \ar[d] \ar[r]& \ \ \ldots \\
& 0 & 0 & 0 & 0 &
}
\end{xy}
$$
Wir brauchen eine Abbildung $H^{i}(C''^{\bullet}) \la H^{i+1}(C'^{\bullet})$. Sei $x \in \kernel d''^{i} \subseteq C''^{i}$. Da $\beta_i$ surjektiv ist, können wir ein Urbild $y \in C'^{i}$ mit $\beta_i(y)=x$ wählen. Dann gilt
$$0= d''^{i}(\beta_i(y)) = \beta_{i+1}(d^{i}(y)),$$
also $d^{i}(y) \in \kernel \beta_{i+1} = \bild \alpha_{i+1}$. Wegen letzterem können wir schreiben $d^{i}(y) =\alpha_{i+1}(z)$ mit eindeutigem $z \in C'^{i+1}$, $\alpha_{i+1}$ Monomorphismus ist. Damit gilt in unserer Rechnung nun $x \mapsto f_y(x):=\alpha_{i+1}^{-1}(d^{i}(y)) \in C'^{i+1}$ mit einem $y \in \beta_i^{-1}(x)$. Es gilt sogar $f_y(x) \in \kernel d'^{i+1}$, denn es ist 
$$\alpha_{i+2}(d'^{i+1}(f_y(x))) = d^{i+1}( d^{i}(y)) = d^{i+1}(d^{i}(y)) = 0$$
und da $\alpha_{i+1}$ ein Monomorphismus ist, muss bereits gelten $d'^{i+1}(f_y(x))=0$, das Gewünschte. Es bleibt zu zeigen, dass für eine andere Wahl $\tilde{y}$ von $y$ gilt $f_y(x)- f_{\tilde{y}}(x) \in \bild d'^{i}$ - dann ist die Abbildung
$$\tilde{\delta}^{i}: \kernel d''^{i} \la H^{i}(C'^{\bullet}), \qquad x \mapsto f_y(x) \quad \textrm{ für ein } y \in \beta_i^{-1}(x)$$
wohldefiniert. Sei also $\tilde{y} \in \beta_i^{-1}(x)$ beliebig und sei analog $\tilde{z} = \alpha_{i+1}^{-1}(d^{i}(\tilde{y}))$. Es gilt $\beta_i(y) = \beta_i(\tilde{y})$, also $y - \tilde{y} \in \kernel \beta_i = \bild \alpha_i$, etwa $y-\tilde{y} = \alpha_i(w)$. Dann ist
$$f_y(x) - f_{\tilde{y}}(x) = \alpha_{i+1}^{-1}(d^{i}(y - \tilde{y})) = d'^{i}(\alpha_i^{-1}(y-\tilde{y})) = d'^{i}(w) \ \in \bild d'^{i},$$
was zu zeigen war, womit $\tilde{\delta}^{i}$ wohldefiniert ist. Schließlich ist noch zu zeigen, dass $\tilde{\delta}^{i}$ über $H^{i}(C''^{\bullet})$ faktorisiert. Sei also $x \in \bild d''^{i-1} \subseteq \kernel d''^{i}$, etwa $x= d''^{i-1}(v)$ für ein $v \in C''^{i-1}$. Da $\beta_{i-1}$ Epimorphismus ist, gilt $v= \beta_{i-1}(\tilde{v})$ für ein $\tilde{v} \in C^{i-1}$, also
$$d''^{i-1}(\beta_{i-1}(\tilde{v})) = d''^{i-1}(v) = x = \beta_i(y) = \beta_i(d^{-1}(\tilde{w}))$$
und wir erhalten $d^{i}(y) = d^{i}(d^{i-1}(\tilde{w})) = 0$. Schließlich folgt 
$$\tilde{\delta}_i(x) = \alpha_{i+1}^{-1}(d^{i}(y)) = \alpha_{i+1}(0) = 0,$$
da $\alpha_{i+1}$ injektiv ist, was zu zeigen war. $\hfill \Box$
\end{compactenum}
\end{pr}


\end{proposition}

Sei nun $\lokring$ ein Schema. Ziel soll es sein, für jede Garbe $\F$ von abelschen Gruppen auf $X$ und jedes $i \geqslant 0$ eine abelsche Gruppe $H^{i}(X, \mathcal{F})$ mit folgenden Eigenschaften zu definieren:
\begin{compactenum}
\item Es gilt $H^{0}(X, \mathcal{F}) = \Gamma(X, \mathcal{F}) := \mathcal{F}(X)$.
\item Ist $0 \la \F' \la \F \la \F'' \la 0$ eine kurze exakte Sequenz von Garben, so gibt es eine lange exakte Kohomologiesequenz
$$0 \la H^{0}(X, \F') \la H^{0}(X, \F) \la H^{0}(X, \F'') \la H^{1}(X, \F') \la H^{1}(X, \F)\la \ldots$$

\end{compactenum}


\begin{proposition}    %%Proposition 12.4
Sei $H^{i}(X, \cdot)$ mit (i) und (ii) gegeben, $0 \la \F \la \G_0 \la \G_1 \la \ldots$ eine exakte Sequenz von Garben auf $X$ (eine sogenannte \textit{Auflösung} von $\F$), sodass $H^{i}(X, \G_j)=0$ für alle $j \geqslant 0$ und $i \geqslant 1$ (ein solche Garbe $\G_j$ heißt \textit{azyklisch}). Dann ist 
$$H^{i}(X, \F) = H^{i}\left(\Gamma(X, \G^{\bullet})\right).$$
\begin{pr}
Durch Induktion über $i$.
\begin{compactenum}
\item[$i = 0$] Da der globale Schnittfunktor $\Gamma(X, \cdot)$ linksexakt ist, ist die Sequenz der globalen Schnitte
$$0 \la \Gamma(X, \F) \overset{\alpha}{\la} \Gamma(X, \G_0) \overset{d^{0}}{\la} \Gamma(X, \G_1) \la \ldots$$
exakt. Dann gilt aber bereits
$$H^{0}(X, \F) = \Gamma(X, \F) = \kernel d^{0} = \bild \alpha = H^{0}(\Gamma(X, \G^{\bullet})).$$

\item[$i=1$] Die Auflösung von $\F$ zerlegt sich in exakte Sequenzen 
$$(1) \qquad 0 \la \F \la \G_0 \la \slant{\G_0}{\F} \la 0$$
$$(2) \qquad 0 \la \slant{\G_0}{\F} \la \G_1 \la \G_2 \la \ldots$$
Nach Voraussetzung gibt es zu (1) eine lange exakte Sequenz
$$0 \la H^{0}(X, \F) \la H^{0}(X, \G_0) \la H^{0}\left(X, \slant{\G_0}{\F}\right) \la H^{1}(X, \F) \la H^{1}(X, \G_0) = 0,$$
also gilt, da $H^{0}\left(X, \slant{\G_0}{\F}\right) \la H^{1}(X, \F)$ Epimophismus ist
\setlength{\abovedisplayskip}{5.5pt}
\setlength{\belowdisplayskip}{5.5pt}
\begin{alignat*}{5}
H^{1}(X, \F) \ &=&& \ \slant{H^{0}\left(X, \slant{\G_0}{\F}\right)}{\left(\kernel\left( H^{0}\left(X, \slant{\G_0}{\F}\right) \la H^{1}(X, \F)\right)\right)}\\
&=&& \ \slant{H^{0}\left(X, \slant{\G_0}{\F}\right)}{\left(\bild\left( H^{0}(X, \G_0) \la H^{0}\left(X, \slant{\G_0}{\F}\right)\right)\right)}.
\end{alignat*}
Aus (2) folgt, dass die Sequenz
$$0 \la H^{0}\left(X, \slant{\G_0}{\F}\right) \la H^{0}(X, \G_1) \la H^{0}(X, \G_2) \longrightarrow \ldots$$
exakt ist, wir erhalten also
\setlength{\abovedisplayskip}{5.5pt}
\setlength{\belowdisplayskip}{5.5pt}
\begin{alignat*}{5}
H^{0}\left(X, \slant{\G_0}{\F}\right) \ &=&& \ \slant{\bild \left( H^{0}\left(X, \slant{\G_0}{\F}\right) \la H^{0}(X, \G_1) \right)}{\left( \kernel \left(0 \la H^{0}\left(X, \slant{\G_0}{\F}\right)\right) \right)} \\
&=&& \ \kernel \left( H^{0}(X, \G_1) \la H^{0}(X, \G_2) \right)
\end{alignat*}
und schließlich
\setlength{\abovedisplayskip}{5.5pt}
\setlength{\belowdisplayskip}{5.5pt}
\begin{alignat*}{5}
H^{1}(X, \F) \ &=&& \ \slant{ \kernel \left( H^{0}(X, \G_1) \la H^{0}(X, \G_2) \right)}{\left( \bild \left( H^{0}(X, \G_0) \la H^{0}(X, \G_1) \right) \right)} \\
&=&& \ H^{1}\left( \Gamma(X, \G^{\bullet}) \right).
\end{alignat*}
\item[$i>1$] Folgt analog. $\hfill \Box$



\end{compactenum}
\end{pr}
\end{proposition}


\begin{er}
Seien $\C$, $\mathcal{D}$ Kategorien und $0 \la C' \la C \la C'' \la 0$ eine exakte Sequenz in $\C$.
\begin{compactenum}
\item Ein kovarianter Funktor $F: \C \la \mathcal{D}$ heißt \textit{linksexakt} (bzw. \textit{rechtsexakt}), falls die Sequenzen
$$0 \la F(C') \la F(C) \la F(C'') \qquad \textrm{bzw.} \qquad F(C') \la F(C) \la F(C'') \la 0$$
in $\mathcal{D}$ exakt sind.
\item Ein kontravarianter Funktor $F: \C \la \mathcal{D}$ heißt \textit{linksexakt} (bzw. \textit{rechtsexakt}), falls die Sequenzen
$$0 \la F(C'') \la F(C) \la F(C') \qquad \textrm{bzw.} \qquad F(C'') \la F(C) \la F(C') \la 0$$
in $\mathcal{D}$ exakt sind.
\item Eine Funktor $F: \C \la \mathcal{D}$ heißt \textit{exakt}, falls er links- und rechtsexakt ist.
\end{compactenum}
\end{er}



\begin{definbem}

Sei $\C$ eine abelsche Kategorie.
\begin{compactenum}
\item Ein Objekt $I$ in $\C$ heißt \textit{injektiv}, falls $\Hom_{\C}(\cdot, I)$ exakt ist.
\item Ein Objekt $P$ in $\C$ heißt \textit{projektiv}, falls $\Hom_{\C}(P, \cdot)$ exakt ist.
\end{compactenum}
Betrachte nun das folgende Diagramm mit Objekten in $\C$.
$$
\begin{xy}
\xymatrix{
0 \ar[rr] && C' \ar[rr]^{\iota} \ar[d]_{\phi} && C \ar@{-->}[lld]_{\tilde{\phi}} \ar[rr]^{\pi} && C'' \ar[rr] && 0 \\
&& I && && P \ar[u]_{\psi} \ar@{-->}[llu]_{\tilde{\psi}} &&
}
\end{xy}
$$
Dann gilt:
\begin{compactenum}
\item[(iii)] $I$ ist genau dann injektiv, falls für jedes solche linksexakte Diagramm ein $\tilde{\phi} \in \Hom_{\C}(C,I)$ existiert, sodass gilt $\tilde{\phi} \circ \iota = \phi$.
\item[(iv)] $P$ ist genau dann projektiv, falls für jedes solche rechtsexakte Diegramm ein $\tilde{\psi} \in \Hom_{\C}(P,C)$ existiert, sodass gilt $\pi \circ \tilde{\psi} = \psi$.
\end{compactenum}


\end{definbem}

\newcommand{\QZ}{\slant{\mathbb{Q}}{\mathbb{Z}}}




\begin{ex}
$\slant{\mathbb{Q}}{\mathbb{Z}}$ ist eine injektive abelsche Gruppe.
\begin{pr}
Sei $A' \subseteq A$ abelsche Gruppen und betrachte das Diagramm
$$
\begin{xy}
\xymatrix{
0 \ar[rr] && A' \ar[rr] \ar[d]_{\phi} && A \\
&& \slant{\mathbb{Q}}{\mathbb{Z}} && 
}
\end{xy}
$$
mit einem Gruppenhomomorphismus $\phi: A' \la \slant{\mathbb{Q}}{\mathbb{Z}}$. Nach Bemerkung 2.8(iii) müssen wir für die Injektivität $\phi$ auf $A$ fortsetzen. Für $a \in A$ sei
$$\tilde{\phi}_a(a) := \begin{cases}\ 0, & \ \textrm{ falls } n \cdot a \notin A' \textrm{ für alle } n \in \mathbb{N} \\ \ \frac{1}{n} \phi(na), & \ \textrm{ falls } n= \mathrm{min} \{k \ \vert \ k a \in A' \}. \end{cases}$$
\end{pr}
Dann ist $$\tilde{\phi}_a: \langle A', a\rangle \la \QZ, \qquad a' + k \cdot a \mapsto \phi(a') + k \cdot  \tilde{\phi}_a(n  a)$$ wohldefiniert und ein Homomorphismus, denn es gilt für $k a \in A'$ mit $k=k_0 n$
$$\tilde{\phi}_a(ka) = \tilde{\phi}_a(k_0na) = k_0 \phi(na) = k_0 n \tilde{\phi}_a(a) = k \tilde{\phi}_a(a).$$
Damit haben wir bereits eine Fortsetzung von $\phi$ auf $\langle A', a\rangle$ für alle $a \in A$. Für $a \in A$ sei nun 
$$\Phi:= \left\{ (\overline{A}, \overline{\phi} \ \vert \ A \subseteq \tilde{A} \leqslant A', \ \overline{\phi}: \overline{A} \la \QZ \textrm{ mit } \overline{\phi} \vert_{A'} = \phi \right\}.$$
Dann ist $\Phi$ nichtleer und durch "$\leqslant$" geordnet, enthält nach Zorns Lemma also ein maximales Element $(A_{\textrm{max}}, \phi_{\textrm{max}})$. Wäre $A_{\textrm{max}} \neq A$, so wähle $\overline{a} \in A \setminus \overline{A}$ und verfahre wir oben und führe diesen Fall zum Widerspruch. Damit folgt die Behauptung. $\hfill \Box$
\end{ex}

\begin{lemma}
Sei $\C$ eine Kategorie, $I$ eine beliebige Indexmenge und $I_i$ injektive Objekte in $\C$ für alle $i \in I$. Dann ist auch das direkte Produkt $I:= \prod_{i \in I} I_i$ injektiv.
\begin{pr}
Sei ein Diagramm 
$$
\begin{xy}
\xymatrix{
0 \ar[rr] && C' \ar[d]_{\phi} \ar[rr]^{\iota} && C \\ && I && 
}
\end{xy}
$$
gegeben. Da $I_i$ injektiv ist für jedes $i \in I$ erhalten wir ein Diagramm
$$
\begin{xy}
\xymatrix{
0 \ar[rr] && C' \ar[d]_{\phi_i} \ar[rr]^{\iota} && C \ar[lld]_{\tilde{\phi}_i} \\ && I_i && 
}
\end{xy}
$$
und die $\tilde{\phi}_i$ setzen sich nach der UAE des direkten Produkts zu einem Homomorphismus $\tilde{\phi}: C \la I$ mit der gewünschten Kommutativität zusammen, was zu zeigen war. $\hfill \Box$
\end{pr}
\end{lemma}


\begin{proposition}
Jede abelsche Gruppe kann in eine injektive abelsche Gruppe eingebettet werden.
\begin{pr}
Sei $A$ abelsche Gruppe, $a \in A \setminus \{0\}$. Definiere
$$\phi_a: \langle a \rangle \la \QZ, \qquad a \mapsto c_a \neq 0,$$
wobei $c_a \in \QZ$ beliebig gewählt ist, mit der Eigenschaft $\mathrm{ord}(c_a) \vert \mathrm{ord}(a)$, falls $\mathrm{ord}(a) < \infty$. Da $\QZ$ injektiv ist, lässt sich $\phi_a$ fortsetzen zu $\phi_a: A \la \QZ$. Die $\phi_a$ für $a \in A$ definieren einen Homomorphismus 
$$\phi: A\la \prod_{a \in A\setminus \{0\}} \QZ, \qquad g \mapsto (\phi_a(g))_{a \in A\setminus \{0\}}.$$
$\phi$ ist injektiv, denn für alle $a \in A\setminus \{0\}$ gilt $(\phi(a))_a = \phi_a(a) = c_a \neq 0$, also $\phi(a) \neq 0$. Damit ist $\phi$ injektiver Homomorphismus in eine nach Lemma 12.10 injektive Gruppe. $\hfill \Box$
\end{pr}

\end{proposition}


\begin{proposition}
In den Kategorien $\Rmod, \Oxmod$ und in der Kategorie der Garben abelscher Gruppen $\AbX$ auf einem Schema $X$ lässt sich jedes Objekt in ein injektives Objekt einbetten. Man sagt: Es gibt in diesen Kategorieren genügend viele injektive Objekte.
\begin{pr}
Für $\Rmod$ siehe dazu in Hilton-Stammbach, I.Prop.8, für $\AbX$ und $\Oxmod$ siehe Hartshorne III.2.2 sowie III.2.3.
\end{pr}

\end{proposition}


\begin{remark}
Ist $\C$ eine abelsche Kategorie mit genügend vielen injektiven Objekten, so besitzt jedes Objekt eine injektive Auflösung, also eine Auflösung mit injektiven Objekten.
\begin{pr}
Sei $C$ ein Objekt in $\C$. Beweise die Behauptung durch Induktion über $i$.
\begin{compactenum}
\item[$i=0$] $I^{0}$ existiert nach Voraussetzung.
\item[$i \geqslant 1$] Sei $0 \la C \la I^{0} \la \ldots \overset{d^{i-1}}{\la} I^{i}$ exakt mit injektiven Objekten $I^{j}$ in $\C$. Sei $I^{i+1}$ ein injektives Objekt mit 
$$\slant{I^{i}}{d^{-1}(I^{i-1})} \subseteq I^{i+1}$$
(das existiert, da es genügend viele injektive Objekte in $\C$ gibt). Dann gilt für die Abbildung $d^{i}: I^{i} \la I^{i+1}$ gerade $\kernel d^{i} = d^{i-1}(I^{i-1})=\mathrm{Bild} \hspace{1.0pt} d^{i-1}$, die Sequenz 
$$0 \la C \la I^{0} \la \ldots \overset{d^{i-1}}{\la} I^{i} \overset{d^{i}}{\la} I^{i+1}$$
ist also exakt, was zu zeigen war. $\hfill \Box$

\end{compactenum}
\end{pr}

\end{remark}

\newcommand{\I}{\mathcal{I}}

\begin{defin}

Sei $\lokring$ ein Schema, $\F$ eine Garbe abelscher Gruppen aus $X$ und 
$$0 \la \F \la \I^{0} \la \I^{1} \la \ldots $$
eine injektive Auflösung von $\F$. Dann heißt für $i \geqslant 0$
$$H^{i}(X, \F) := H^{i}(\Gamma(X, \I^{\bullet})$$
die $i$-te Kohomologiegruppe von $\F$.
\end{defin}

\begin{remark}
\begin{compactenum}
\item Es gilt $H^{0}(X, \F) = \Gamma(X, \F) = \F(X)$ für jedes $\F \in \AbX$. 
\item Es gilt $H^{i}(X, \I)=0$ für jede injektive Garbe $\I \in \AbX$ und $i \geqslant 1$. 
\end{compactenum}
\begin{pr}
\begin{compactenum}
\item Siehe 12.4.
\item Wähle eine Auflösung $0 \la \I \la \I \la 0$ von $\I$. Dann folgt die Behauptung. $\hfill \Box$
\end{compactenum}
\end{pr}
\end{remark}


\begin{theorem}
Sei $\lokring$ ein Schema.
\begin{compactenum}
\item Für $\F \in \AbX$ ist $H^{i}(X, \F)$ nicht von der gewählten injektiven Auflösung abhängig.
\item $H^{i}(X, \cdot): \AbX \la \Ab$ ist ein Funktor.
\item Ist $0 \la \F' \la \F \la \F'' \la 0$ eine kurze exakte Sequenz von Garben abelscher Gruppen, so gibt es eine lange exakte Kohomologiesequenz
$$0 \la H^{0}(X, \F') \la H^{0}(X, \F) \la H^{0}(X, \F'') \la H^{1}(X, \F') \la H^{1}(X, \F) \la  \ldots$$


\end{compactenum}
\begin{pr}
\begin{compactenum}
\item[(ii)] Sein $\F, \G$ Garben abelscher Gruppen mit injektiven Auflösungen 
$$0 \la \F \la \I^{\bullet}, \qquad 0 \la \G \la \mathcal{J}^{\bullet}$$
sowie $\phi: \F \la \G$ ein Garbenmorphismus. Definiere $\phi^{i}: \I^{i} \la \mathcal{J}^{i}$ wie folgt:

$$
\begin{xy}
\xymatrix{
0 \ar[rr] && \F \ar[d]_{\phi} \ar[rr]^{\epsilon} \ar@{-->}[rrd]^{{\epsilon} \circ \phi} && \mathcal{I}^{0} \ar[d]_{\phi^0} \ar[rr]^{d^{0}} && \I^{1} \ar[d]_{\phi1} \ar[rr] && \ldots \\
0 \ar[rr] && \G \ar[rr]^{\tilde{\epsilon}} && \mathcal{J}^{0} \ar[rr]^{\tilde{d}^{0}} && \mathcal{J}^{1} \ar[rr] && \ldots
}
\end{xy}
$$
$\phi^{0}$ sei die Fortsetzung von $\tilde{\epsilon} \circ \phi$ auf $\I^{0}$ ($\mathcal{J}^{0}$ ist injektiv). Zur Definition von $\phi^{1}$ brauchen wir, dass $\tilde{d}^{0} \circ \phi^{0}$ über $\slant{\I^{0}}{\kernel d^{0}}$ faktorisiert mit $\kernel d^{0} = \bild \epsilon = \F$. Es ist aber $\F \subseteq \kernel \tilde{d}^{0} \circ \phi^{0}$, da $\phi^{0}(\F) \subseteq \tilde{\epsilon}(\G) = \kernel \tilde{d}^{0}$. Die $\phi^{i}$ induzieren einen Morphismus $\Gamma(X, \I^{0}) \la \Gamma(X, \mathcal{J}^{0})$, wir erhalten also einen Morphismus von Komplexen $\mu: \Gamma(X, \mathcal{I}^{\bullet}) \la \Gamma(X, \mathcal{J}^{\bullet})$. Nach 12.5 induzieren diese Homomorphismen
$$\overline{\phi}_{i}: H^{i}(X, \F) = H^{i}(\Gamma(X, \I^{\bullet}) \la H^{i}(\Gamma(X, \mathcal{J}^{\bullet}) = H^{i}(X, \G).$$
Es bleibt noch zu zeigen, dass $\overline{\phi}_i$ nicht von der Wahl der $\phi^{i}$ abhängt. Seien also $\phi^{i}, \tilde{\phi}^{i}$ Fortsetzungen, ohne Einschränkung gelte $\phi=0$, $\tilde{\phi}^{i} =0$. Zu zeigen ist: Es gilt ebenfalls $\phi^{i}=0$, das heißt, $\phi^{\bullet}$ induziert die Nullabbildung auf $H^{\bullet}(\Gamma(X, \mathcal{I}^{\bullet}))$. 
\begin{compactenum}
\item[\textbf{Beh. (a)}] Für $i \geqslant 1$ gibt es Homomorphismen $h^{i}: \I^{i} \la \mathcal{J}^{i-1}$ mit
$$\phi^{i} = \tilde{d}^{i-1} \circ h^{i} + h^{i+1} \circ d^{i}, \qquad \phi^{0} = h^{1} \circ d^{0}.$$\end{compactenum}
Wenden wir nun den Schnittfunktor $\Gamma(X, \cdot)$ auf das Diagramm
$$
\begin{xy}
\xymatrix{
&& \I^{i} \ar[lld]_{h^{i}} \ar[d]^{\phi^{i}} \ar[rr]^{d^{i}} && \I^{i+1} \ar[lld]^{h^{i+1}} \\ \mathcal{J}^{i-1} \ar[rr]_{\tilde{d}^{i-1}} && \mathcal{J}^{i} &&
}
\end{xy}
$$
an, so erhalten wir Homomorphismen 
$$h^{i}: \Gamma(X, \I^{i}) \la \Gamma(X, \mathcal{J}^{i-1})$$
mit derselben Eigenschaft (wobei alle Homomorphismen ihren Namen behalten haben). Sei nun $x \in H^{i}(X, \mathcal{F}) = H^{i}(\Gamma(X, \I^{\bullet}))$, also $x \in \kernel \left( d^{i}: \Gamma(X, \I^{\bullet}) \la \Gamma(X, \I^{i+1}) \right)$ Dann gilt 
$$\phi^{i}(x) = \tilde{d}^{i-1}(h^{i}(x)) + h^{i+1}(d^{i}(x)) = \tilde{d}^{i-1}(h^{i}(x)) \in \bild \tilde{d}^{i-1}$$
und damit $\phi^{i}(x) =0$ in $H^{i}(X, \G) = \slant{(\kernel \tilde{d}^{i})}{(\bild \tilde{d}^{i-1})}$, was zu zeigen war.
\begin{compactenum}
\item[\textbf{Bew. (a)}] Betrachte das Diagramm
$$
\begin{xy}
\xymatrix{
0 \ar[rr] && \F \ar[dd]_{\phi=0} \ar[rr] && \I^{0} \ar[dd]_{\phi^{0}} \ar[rr]^{d^{0}} \ar[rd] && \I^{1}  \ar@/^1.1cm/[lldd]^{h^{1}}\\
&&&&& \slant{\I^{0}}{\F} \ar[ru] \ar[ld]_{\overline{\phi}^{0}} & \\
&& \G \ar[rr] && \mathcal{J}^{0} &&
}
\end{xy}
$$
Wegen $\phi=0$ ist $\F= \kernel d^{0} \subseteq \kernel \phi^{0}$, $\phi^{0}$ faktorisiert also über $\slant{\I^{0}}{\F}$. Da $\mathcal{J}^{0}$ injektiv ist und $\slant{\I^{0}}{\F} \hookrightarrow \I^{1}$, existiert die Fortsetzung $h^{1}: \I^{1} \la \mathcal{J}^{0}$ von $\overline{\phi}^{0}$ nach $\I^{1}$ und es gilt $\phi^{0} = h^{1} \circ d^{0}$. Betrachte nun für $i \geqslant 1$ das Diagramm
$$
\begin{xy}
\xymatrix{
\I^{i-2} \ar[rr]^{d^{i-2}} \ar[dd]_{\phi^{i-2}} && \I^{i-1} \ar[lldd]_{h^{i-1}} \ar[dd]_{\phi^{i-1}} \ar[rr]^{d^{i-1}} && \I^{i} \ar[lldd]^{h^{i}}\\
&&&& \\
\mathcal{J}^{i-2} \ar[rr]^{\tilde{d}^{i-2}} && \mathcal{J}^{i-1} &&
}
\end{xy}
$$
Setze $\tilde{\phi}^{i-1}= \phi^{i-1} - \tilde{d}^{i-2} \circ h^{i-1}: \I^{i-1} \la \mathcal{J}^{i-1}$. Es gilt $\kernel d^{i-1} = \bild d^{i-2}$. Für $x \in \kernel d^{i-1} = \bild d^{i-2}$, $x= d^{i-1}(y)$ gilt dann $\kernel d^{1} \subseteq \subseteq \tilde{\phi}^{1}$, wir erhalten also eine Fortsetzung $h^{i}: \I^{i} \la \mathcal{J}^{i-1}$ von $\tilde{\phi}^{1}$ mit 
$$h^{i} \circ d^{i-1} = \tilde{\phi}^{i-1} = \phi^{i-1} - \tilde{d}^{i-1} \circ h^{i-1},$$
was zu zeigen war.
\end{compactenum}
\item[(i)] Folgt aus (ii) mit $\G = \F$ und $\phi = \mathrm{id}$. 
\item[(iii)] Sei $0 \la \F' \la \F \la \F'' \la 0$ eine kurze exakte Sequenz in $\AbX$. Wähle injektive Auflösungen 
$$0 \la \F' \la \I'^{\bullet}, \qquad 0 \la \F'' \la \I''^{\bullet}$$
von $\F'$ und $\F''$. betrachte nun das folgende Diagramm:

$$
\begin{xy}
\xymatrix{
& 0 \ar[d] & 0 \ar[d] & 0 \ar[d] & \\
0 \ \ar[r] & \ \F' \ \ar[r]^{\epsilon'} \ar[d]_{\iota} & \ \I'^{0} \ \ar[r] \ar[d]_{\iota^{1}} & \ \I'^{1} \ \ar[d] \ar[r] & \ldots \\
0 \ \ar[r] & \ \F \ \ar[r]_{\epsilon} \ar@{-->}[ru]^{\psi} \ar[d]_{\pi} &\ \I'^{0} \oplus \I''^{0}\ \ar[d] \ar[r] &\ \I'^{1} \oplus \I''^{1} \ \ar[d] \ar[r] & \ldots \\
0\ \ar[r] & \ \F'' \ \ar[d] \ar[r]^{\epsilon''} & \ \I''^{0} \ \ar[d] \ar[r] & \ \I''^{1} \ar[r] \ar[d] & \ldots \\
&0&0&0&
}
\end{xy}
$$
Da $\I'^{0}$ injektiv ist und $\F' \subseteq \F$, gibt es $\psi: \F \la \I'^{0}$. $\psi$ und $\epsilon'' \circ \pi$ induzieren gemeinsam $\epsilon: \F \la \I'^{0} \oplus \I''^{0}$. $\epsilon$ ist injektiv, denn: Gilt $(\epsilon'' \circ \pi)(x) =0$, so ist $\pi(x)=0$, also $x \in \kernel \pi = \bild \iota = \F'$, also 
$$\epsilon(x) = \psi(\iota(x')) = \psi(x)$$
und damit, da $\epsilon'$ injektiv ist
$$\epsilon(x) = 0 \quad \Longleftrightarrow \quad \iota^{1}(\psi(x)) = 0 \quad \Longleftrightarrow \quad \psi(x) = \epsilon(x) = 0 \quad \Longleftrightarrow \quad x=0.$$
Per Induktion erhalten wir auf diese Weise eine "injektive Auflösung der kurzen exakten Sequenz". Wir wenden nun den globalen Schnittfunktor $\Gamma(X, \cdot)$ auf $\I'^{\bullet}, \I^{\bullet}, \I''^{\bullet}$ an und erhalten eine kurze exakte Sequenz von Komplexen in $\Ab$(Beachte: Das geht nun wegen der geeigneten Wahl von $\I^{\bullet}$ gut). Dazu gibt es nach 12.5 (iii) eine lange exakte Kohomologiesequenz. $\hfill \Box$


\end{compactenum}
\end{pr}
\end{theorem}

\begin{remark}
Folgende Verallgemeinerungen des Vorgehens in diesem Abschnitt sind möglich:
\begin{compactenum}
\item Es genügt, dass $X$ ein topologischer Raum ist; die Schemastruktur ist nicht nötig.
\item Alles geht genauso für Objekte in einer beliebigen abelschen Kategorie mit genüend vielen injektiven Objekten $\mathcal{A}$ (statt Garben abelscher Gruppen auf $X$) und einem kovarianten, linksexakten Funktor $F: \mathcal{A} \la \mathcal{B}$ (statt dem globalen Schnittfunktor). Genauer heißt dies: Für $A \in \mathrm{Ob}(\mathcal{A})$ wähle eine injektive Auflösung $0 \la A \la I^{\bullet}$. Dann ist $F(I^{\bullet})$ ein Komplex in $\mathcal{B}^{\bullet}$. Definiere
$$R^{i}F(A) := H^{i}(F(I^{\bullet})).$$
Die $R^{i}F$ sind Funktoren $\mathcal{A} \la \mathcal{B}$. Sie heißen (rechts) abgeleitete Funktoren von $F$. Insbesondere gilt $R^{0}F=F$. Eine Auswahl an Kategorien mit genügend vielen Injektiven haben wir bereits kennengelernt. Weitere linksexakte Funktoren sind beispielsweise
\begin{compactenum}
\item Die Hom-Funktoren. Die $R^{i}\Hom$ heißen auch $\mathrm{Ext}$ und $\mathrm{Tor}$.
\item Für Schemata $X,Y$ und einem Schemamorphismus $f: X \la Y$ ist $f_{*}$ linsexakt und kovariant. Die $R^{i}f_{*}$heißen auch höhere direkte Bildgarben.
\end{compactenum}

\end{compactenum}
\end{remark}







\renewcommand*\thesection{§ \arabic{section}\quad}
\section{\v{C}ech-Kohomologie} %PARAGRAPH 13
\renewcommand*\thesection{\arabic{section}}

\newcommand{\cechk}{C^{k}( \mathfrak{U}, \F)}
\newcommand{\cechkk}{C^{k+1}( \mathfrak{U}, \F)}

\begin{definbem}     %%Definition + Bemerkung 13.1
Sei $X$ ein topologischer Raum, $\F$ eine Garbe in $\AbX$ sowie $\mathfrak{U} = \{U_i\}_{i \in \mathbb{N}}$ eine offene Überdeckung von $X$.
\begin{compactenum}
\item Für $k \geqslant 0$ ist
$$C^{k}( \mathfrak{U}, \F) := \prod_{i_0 < \ldots < i_k} \F\left(U_{i_0} \cap \ldots \cap U_{i_k}\right)$$
eine abelsche Gruppe.
\item Für $k \geqslant 0$ ist
\setlength{\abovedisplayskip}{5.5pt}
\setlength{\belowdisplayskip}{5.5pt}
\begin{alignat*}{5}
d^{k}: \cechk \ &\la&& \ \cechkk \\
 \qquad \left(s_{i_0\ldots i_k}\right)_{i_0 < \ldots < i_k} \ &\mapsto&&\ \left( \sum_{\nu =0}^{k+1} (-1)^{\nu} s_{i_0 \ldots i_{\nu-1} i_{\nu+1} \ldots i_{k+1}} \big\vert_{U_{i_0} \cap \ldots \cap U_{i_{k+1}}} \right)_{i_0 < \ldots < i_{k+1}}
 \end{alignat*}
 ein  Gruppenhomomorphismus.
 \item Es gilt $d^{k+1} \circ d^{k} =0$ für alle $k \geqslant 0$, d.h. $C^{\bullet}(\mathfrak{U}, \F)$ ist ein Komplex.
 \item Die Gruppe
 $$\check{H}^{k}(\mathfrak{U}, \F) := H^{k}(C^{\bullet}(\mathfrak{U}, \F))$$
 heißt $k$-te \v{C}echkohomologiegruppe von $\F$ bezüglich der Überdeckung $\mathfrak{U}$. 
 \item Es gilt $\check{H}^{0}(\mathfrak{U}, \F) = \F(X)$. 
\end{compactenum}
\begin{pr}
\begin{compactenum}
\item[(iii)] Durch Induktion über $k$:
\begin{compactenum}
\item[$k=0:$] Sei $(s_i)_{i \in \mathbb{N}} \in \prod_{i \in \mathbb{N}} \F(U_i)= C^{0}(\mathfrak{U}, \F)$. Dann gilt
$$d^{0}((s_i)_{i \in \mathbb{N}}) = (s_j \vert_{U_i \cap U_j} - s_i \vert_{U_i \cap U_j})_{i <j}$$
und damit mit $U_{ijl} := U_i \cap U_j \cap U_l$
$$d^{1} \left(d^{0}((s_i)_{i \in \mathbb{N}})\right) = \left((s_l-s_j) \vert_{U_{ijl}} - (s_l-s_i)\vert_{U_{ijl}} + (s_j - s_i)\vert_{U_{ijl}} \right)_{i<j<l} = 0.$$

\item[$k\geqslant 1:$] Sei nun $(s_{i_0 \ldots i_k})_{i_0 < \ldots <i_k} \in C^{k}(\mathfrak{U}, \F)$. Dann gilt mit $\tilde{U}_{k} := U_{i_0} \cap \ldots \cap U_{i_k}$  
 $$(d^{k+1} \circ d^{k})\left((s_{i_0 \ldots i_k})_{i_0 < \ldots <i_k}\right) = d^{k+1} \left(\left( \sum_{\nu =0}^{k+1} (-1)^{\nu} s_{i_0 \ldots i_{\nu-1} i_{\nu+1} \ldots i_{k+1}} \big\vert_{\tilde{U}_{k+1}} \right)_{i_0 < \ldots < i_{k+1}}\right)$$
 Die Vorzeichen kürzen sich weg und es bleibt $d^{k+1} \circ d^k = 0$.


\end{compactenum}
\item[(v)] Es gilt $\check{H}^{0}(\mathfrak{U}, \F) = \kernel d^{0}$. Weiter ist 
$$C^{0}(\mathfrak{U}, \F) = \prod_{i \in \mathbb{N}} \F(U_i).$$
Sei nun $(s_i)_{i \in \mathbb{N}} \in \kernel d^{0}$. Dann gilt 
$$d^{0}((s_i)_{i \in \mathbb{N}}) = (s_j-s_i \vert_{U_i \cap U_j})_{i < j} \overset{!}{=} 0.$$
Da $\F$ eine Garbe ist, existiert ein eindeutiger globaler Schnitt $s \in \F(X)$ mit $s_i = s \vert_{U_i}$, also $\check{H}^{0}(\mathfrak{U}, \mathcal{F}) = \kernel d^{0} \subseteq \mathcal{F}(U)$. Ist hingegen $s \in \mathcal{F}(U)$, so gilt selbstverständlich $(s\vert_{U_i})\vert_{U_i \cap U_j} - (s\vert_{U_j})\vert_{U_i \cap U_j} = 0$, also $s \in \kernel d^{0}$. $\hfill \Box$
\end{compactenum}
\end{pr}
\end{definbem}


\begin{ex}   %%Beispiel 13.2

Sei $X= \mathbb{S}^{1}$ sowie $\F= \mathbb{Z}$ die zur konstanten Prägarbe $\F(U)=\mathbb{Z}$ assoziierte Garbe auf $X$. Sei durch $\mathfrak{U}:= \{U_1, U_2\}$ mit 
$$U_1:= \left\{ \left( \cos u, \sin u\right) \ \big\vert \ u \in \left[- \frac{\pi}{4}, \frac{5 \pi}{4} \right] \right\}, \qquad U_2:= \left\{ \left( \cos u, \sin u\right) \ \big\vert \ u \in \left[\frac{3\pi}{4}, \frac{ \pi}{4} \right] \right\}.$$
eine offene Überdeckung von $X$ gegeben. Dann hat $U_1 \cap U_2 = D_1 \overset{.}{\cup} D_2$ zwei Zusammenhangskomponenten. Für den \v{C}echkomplex erhalten wir 
$$C^{0}(\mathfrak{U}, \mathbb{Z}) = \F(U_1) \times \F(U_2) \cong \mathbb{Z}^2,$$
$$C^{1}(\mathfrak{U}, \mathbb{Z}) = \F(U_1 \cap U_2) \cong \mathbb{Z}^2,$$
$$C^{k}(\mathfrak{U}, \mathbb{Z}) = 0, \qquad \textrm{ für } k \geqslant 0.$$
Für $d^{0}$ gilt 
$$d^{0}: \mathbb{Z}^2 = C^{0}(\mathfrak{U}, \mathbb{Z}) \la C^{1}(\mathfrak{U}, \mathbb{Z})= \mathbb{Z}^2, \qquad (a,b) \mapsto (b-a, b-a) $$
und damit $\bild(d^{0}) = \{(a,a) \in \mathbb{Z}^2 \} = \Delta \mathbb{Z}$. Die \v{C}echkohomologiegruppen ergeben sich dann zu 
$$\check{H}^{0}(\mathfrak{U}, \mathbb{Z}) = \mathbb{Z}(X) = \mathbb{Z}, $$
$$\check{H}^{1}(\mathfrak{U}, \mathbb{Z}) = \slant{\kernel d^{1}}{\bild d^{0}} = \slant{\mathbb{Z}^2}{\Delta \mathbb{Z}} \cong \mathbb{Z},$$
$$\check{H}^{k}(\mathfrak{U}, \mathbb{Z}) = 0 \qquad \textrm{ für } k \geqslant 0.$$

\end{ex}



\begin{definprop}   %%Definition + Proposition 13.3
Sei $X$ ein topologischer Raum, $\F \in \AbX$ sowie $\mathfrak{U}=\{U_i\}_{i \in \mathbb{N}}$ eine offene Überdeckung von $X$. 
\begin{compactenum}
\item Für $k \geqslant 0$ sei 
$$\mathcal{C}^{k}(\mathfrak{U}, \F) = \prod_{i_0< \ldots < i_k} \left( \iota_{i_0 \ldots i_k} \right)_{*} \mathcal{F}\vert_{U_{i_0} \cap \ldots U_{i_k}},$$
wobei $\iota_{i_0 \ldots i_k}: U_{i_0} \cap \ldots \cap U_{i_k} \hookrightarrow X$ die Inklusion ist. Für eine offene Teilmenge $U \subseteq X$ ist also
$$\Gamma\left(U, \mathcal{C}^{k}(\mathfrak{U}, \F)\right)=\mathcal{C}^{k}(\mathfrak{U}, \F)(U)= \prod_{i_0< \ldots i_k} \F(U \cap U_{i_0} \cap \ldots \cap U_{i_k}).$$
Insbesondere gilt für die globalen Schnitte
$$\Gamma\left(X, \mathcal{C}^{k}(\mathfrak{U}, \F) \right) = C^{k}(\mathfrak{U}, \F).$$
\item Definiere für $k \geqslant 0$ Garbenmorphismen 
$$d^{k}: \mathcal{C}^{k}(\mathfrak{U}, \F) \la \mathcal{C}^{k+1}(\mathfrak{U}, \F)$$
wie in 13.1(ii) und erhalte dadurch ebenfalls einen Komplex $\mathcal{C}^{\bullet}(\mathfrak{U}, \mathcal{F})$.
\item Definiere einen weiteren Garbenmorphismus $\epsilon: \F \la \mathcal{C}^{0}(\mathfrak{U}, \F)$ wie folgt: Für jede offene Teilmenge $U\subseteq X$ sei $\epsilon_U$ der Gruppenhomomorphismus
\setlength{\abovedisplayskip}{5.5pt}
\setlength{\belowdisplayskip}{5.5pt}
\begin{alignat*}{5}
\epsilon_U: \F(U) \ \ &&\la& \ \ \Gamma\left(U, \mathcal{C}^{0}(\mathfrak{U}, \F) \right) = \prod_{i \in \mathbb{N}} \F(U \cap U_i) \\
s \ \ &&\mapsto & \ \ \left(s\vert_{U \cap U_i} \right)_{i \in \mathbb{N}}.
\end{alignat*}
Dann ist $\epsilon$ wegen der Garbeneigenschaft ein Monomorphismus.

\item Die Sequenz
$$0 \la \F \overset{\epsilon}{\la} \mathcal{C}^{0}(\mathfrak{U}, \F) \overset{d^{0}}{\la} \mathcal{C}^{1}(\mathfrak{U}, \F) \overset{d^{1}}{\la} \mathcal{C}^{2}(\mathfrak{U}, \F) \overset{d^{2}}{\la} \ldots$$
ist exakt, also eine Auflösung von $\F$. 
\end{compactenum}

\begin{pr}
\begin{compactenum}
\item[(iv)] Zeige zunächst die Exaktheit bei $\mathcal{C}^{0}(\mathfrak{U}, \F)$. Sei $U \subseteq X$ offen, $(s_i)_{i \in \mathbb{N}} \in \Gamma\left(U,\mathcal{C}^{0}(\mathfrak{U}, \F)\right)$. Dann gilt $(s_i)_{i \in \mathbb{N}} \in \kernel d^{0}$ genau dann, wenn gilt $s_i \vert_{U \cap U_i \cap U_j} = s_j\vert_{U \cap U_i \cap U_j}$ für alle $i,j \in \mathbb{N}$. Da $\F$ eine Garbe ist, gibt es einen eindeutigen Schnitt $s \in \F(U)$ mit $s_i = s\vert_{U \cap U_i}$ für alle $i \in \mathbb{N}$, das heißt es gilt $(s_i)_{i \in \mathbb{N}} = \epsilon_U(s)$ und damit im Bild von $\epsilon$, also $\kernel d^{0} = \bild \epsilon$, was die Exaktheit an dieser Stelle bedeutet.
Sei nun $k \geqslant 1$ beliebig. Es genügt, die Exaktheit halmweise zu zeigen. Sei also $x \in X$, $j \in \mathbb{N}$ mit $x \in U_j$.
\begin{compactenum}
\item[\textbf{Beh. (a)}] Es gibt einen Gruppenhomomorphismus 
$$h_x^{k}: \mathcal{C}^{k}_x(\mathfrak{U}, \F) \la \mathcal{C}^{k-1}_x(\mathfrak{U}, \F)$$
mit $d_x^{k-1} \circ h_x^{k} + h_x^{k+1} \circ d_x^{k} = \mathrm{id}$.
\end{compactenum}
Dann gilt für $\overline{s} \in \kernel d_x^{k}$
$$\overline{s} = \mathrm{id}(\overline{s}) = d_x^{k-1}(h_x^{k}(\overline{s})) + h_x^{k+1}(d_x^{k}(\overline{s})) = d_x^{k-1}(h_x^{k}(\overline{s})) \in \bild d_x^{k-1},$$
also $\kernel d_x^{k} \subseteq \bild d_x^{k-1}$. Wegen $d_x^{k} \circ d_x^{k-1} =0$ folgt dann bereits $\kernel d_x^{k} = \bild d_x^{k-1}$, das heißt, der Komplex ist exakt. Es bleibt noch die Behauptung zu zeigen.
\begin{compactenum}
\item[\textbf{Bew. (a)}] Sei $\overline{s} \in \mathcal{C}^{k}_x(\mathfrak{U}, \F)$, das heißt es ist $\overline{s} = [(V,s)]$ mit $V \subseteq U_j$ und $s=(s_{{i_0} \ldots {i_k}})_{i_0 < \ldots <i_k} \in \F(V \cap U_{i_0} \cap \ldots \cap U_{i_k})$. Sei weiter
$$t_{i_0 \ldots i_{k-1}} := \begin{cases} \ 0, & \ \textrm{ falls } j \in \{i_0, \ldots i_{k-1}\} \\ \ (-1)^{\nu} s_{i_0, \ldots i_{\nu-1}, j, i_{\nu}, \ldots i_{k-1}}, & \ \textrm{ falls } i_{\nu-1} < j < i_{\nu}. \end{cases}$$
Setze nun 
$$h_x^{k}(\overline{s}) := \left[(V, (t_{i_0, \ldots, i_{k-1}})_{i_0 < \ldots i_{k-1}}\right]$$
und zeige, dass der so definierte Homomorphismus $h_x^{k}$ die gewünschte Eigenschaft erfüllt (Übung). $\hfill \Box$
\end{compactenum}

\end{compactenum}
\end{pr}

\end{definprop}


\begin{folg}
Sei $X$ ein topologischer Raum, $\F \in \AbX$ eine Garbe abelscher Gruppen auf $X$ sowie $\mathfrak{U}=\{U_i\}_{i \in \mathbb{N}}$ eine offene Überdeckung von $X$. Dann gibt es für jedes $k \geqslant 0$ einen natürlichen Homomorphismus
$$\check{H}^{k}(\mathfrak{U}, \F) \la H^{k}(X, \F).$$
\begin{pr}
Sei $0 \la \F \la \mathcal{I}^{\bullet}$ eine injektive Auflösung von $\F$. 
$$
\begin{xy}
\xymatrix{
0\ \ar[rr] && \ \F \ \ar[d]_{\mathrm{id}} \ar[rr] && \ \mathcal{C}^{\bullet}(\mathfrak{U}, \F) \ \ar[d]_{\phi^{\bullet}} \\ 0 \ \ar[rr] && \ \F \ \ar[rr] && \ \mathcal{I}^{\bullet}
}
\end{xy}
$$

Nach Übungsaufgabe 10.2 gibt es einen Morphismus von Komplexen $\phi^{\bullet}: \mathcal{C}^{\bullet}(\mathfrak{U}, \F) \la \mathcal{I}^{\bullet}$, welcher auf $\F$ die Identität induziert. Anwenden des globalen Schnittfunktors liefert die gewünschten Gruppenhomomorphismen.


\end{pr}

\end{folg}







\renewcommand*\thesection{§ \arabic{section}\quad}
\section{Kohomologie quasikohärenter Garben} %PARAGRAPH 14
\renewcommand*\thesection{\arabic{section}}






\begin{defin}    %%Defin 14.1
Sei $X$ ein topologischer Raum, $\F$ eine Garbe abelscher Gruppen auf $X$. $\F$ heißt \textit{welk}, falls für alle offenen Teilmengen $U \subseteq V \subseteq X$ die Restriktionsabbildung $\rho^{U}_{V}: \F(U) \la \F(V)$ surjektiv ist.

\end{defin}


\begin{ex} %%% Beispiel
Sei $X$ ein topologischer Raum.
\begin{compactenum}
\item Sei $x \in X$ sowie $A$ eine abelsche Gruppe. Dann ist die Wolkenkratzergarbe
$$x_*(A)(U):= \begin{cases} \ A, & \ \textrm{ falls } x \in U \\ \ 0, & \ \textrm{ sonst. } \end{cases}$$
welk.
\item Ist $X$ irreduzibel, so ist jede Konstante Garbe auf $X$ welk.
\end{compactenum}
\end{ex}


\begin{proposition}
Sei $0 \la \F' \overset{\alpha}{\la} \F \overset{\beta}{\la} \F'' \la 0$ eine kurze exakte Sequenz in $\AbX$.
\begin{compactenum}
\item Ist $\F'$ welk, so ist für jede offene Teilmenge $U \subseteq X$ die Sequenz
$$0 \la \F'(U) \overset{\alpha_U}{\la} \F(U) \overset{\beta_U}{\la} \F''(U) \la 0$$
exakt, das heißt $\beta_U$ ist surjektiv.
\item Sind $\F$ und $\F'$ welk, so ist auch $\F''$ welk.

\end{compactenum}

\begin{pr}
\begin{compactenum}
\item Sei $s'' \in \F''(U)$ und zeige: Es gibt ein $s\in \F(U)$ mit $\beta_U(s)=s''$. Da $\beta$ ein Epimorphismus ist, gibt es eine offene Überdeckung $\{U_i\}_{i \in I}$ von $U$ und $s_i \in \F(U_i)$ mit $\beta_{U_i}(s_i) = s''\vert_{U_i}$. Definiere nun 
$$\Phi:= \{ (V,s) \ \vert \ V \subseteq U \textrm{ offen, } s \in \F(V) \textrm{ mit } \left(\beta_U\right)\vert_V=\beta_V(s) = s''\vert_V \}$$
Wegen $(U_i, s_i) \in \Phi$ ist $\Phi$ nichtleer. Außerdem ist $\Phi$ durch 
$$(V,s) \leqslant (V',s') :\Longleftrightarrow V \subseteq VÄ' \textrm{ und } s'\vert_V = s$$
halbgeordnet und für jede aufsteigende Kette $(V_1,s_1) \leqslant (V_2, s_2) \leqslant \ldots$ in $\Phi$ ist durch 
$$V:= \bigcup_{i=1}^{\infty} V_i$$
sowie der Verklebung der $s_i$ (Garbeneigenschaft) eine obere Schranke gegeben. Zorns Lemma liefert also die Existenz eines maximalen Elements $(U_0, s_0) \in \Phi$. 
\begin{compactenum}
\item[\textbf{Beh. (a)}] Es gilt $U_0 = U$.
\item[\textbf{Bew. (a)}] Angenommen es gelte $U_0 \subsetneq U$. Dann wähle $x \in U \setminus U_0$ sowie $(V,s_1) \in \Phi$ mit $x \in V$. Dann gilt 
$$s_1 \vert_{U_0 \cap V} - s_0 \vert_{U_0 \cap V} \in \kernel \beta_{U_0 \cap V} = \bild \alpha_{U_0\cap V},$$
das heißt es gibt ein $s' \in \F(U_0 \cap V)$ mit 
$$\alpha(s') \vert_{U_0 \cap V} = (s_1 - s_0) \vert_{U_0 \cap V}.$$
Da $\F'$ welk ist, gilt sogar $s' \in \F(V)$. Damit stimmen $s_1 - \alpha(s')$ und $s_0$ auf $U_0 \cap V$ überein, es gibt also $s \in \F(U_0 \cap V)$ mit $s\vert_{U_0} = s_0$, $s\vert_V = s_1 - \alpha(s')$ und $\beta_{U_0 \cap V}(s) = s''\vert_{U_0 \cup V}$. Damit ist $ (U_0, s_0) <(U_0 \cup V, s) \in \Phi$, ein Widerspruch zur Maximalität von $(U_0, s_0)$.

\end{compactenum}
Damit gilt $\beta_U(s) = s''$ und $\beta_U$ ist surjektiv, was zu zeigen war.


\item Seien $\tilde{U} \subseteq U$ offen in $X$ und $\tilde{s}'' \in \F''(\tilde{U})$. Nach (i) gibt es $\tilde{s} \in \F(\tilde{U})$ mit $\beta_{\tilde{U}}(\tilde{s}) = \tilde{s}''$. Da $\F$ welk ist, gibt es $s \in \F(U)$ mit $_{\F}\rho^{U}_{\tilde{U}}(s) = \tilde{s}$. Dann gilt für $s'':= \beta_U(s) \in \F''(U)$:
$$_{\F''}\rho^{U}_{\tilde{U}}(s'') = \beta_{\tilde{U}}(_{\F}\rho^{U}_{\tilde{U}}(s)) = \beta_{\tilde{U}}(\tilde{s}) = \tilde{s}'',$$
das heißt $\rho^{U}_{\tilde{U}}$ ist surjektiv, was zu zeigen war. $\hfill \Box$
\end{compactenum}


\end{pr}

\end{proposition}



\begin{defin}  %%definition
Sei $X$ ein topologischer Raum, $U \subseteq X$ offen und $\F$ eine Garbe abelscher Gruppen auf $U$. Ist $\iota: U \hookrightarrow X$ die Inklusion, so ist die \textit{durch Null fortgesetzte Garbe} $\iota_{!}(\F)$ die zur Prägarbe
$$V \mapsto \begin{cases} \ \F(V), & \ \textrm{ falls } V \subseteq U \\ \ 0 & \ \textrm{ sonst } \end{cases}$$
assozierte Garbe auf $X$.
\end{defin}


\begin{proposition}   %%Proposition 14.3
Ein $\lokring$ ein lokal geringter Raum, $\I$ eine injektive $\mathcal{O}_X$-Modulgarbe auf $X$. Dann ist $\I$ welk.

\begin{pr}
Seien $U' \subseteq U \subseteq X$ offen in $X$. Zu zeigen: Die Restriktionsabbildung $\rho^{U}_{U'}: \I(U) \la \I(U')$ ist surjektiv. Es gilt 
$$\I(U) \cong \Hom_{\mathcal{O}_X}\left( \mathcal{O}_X \vert_U, \mathcal{I}\vert_U \right),$$
denn: Ein Garbenmorphismus $\phi: \mathcal{O}_X \la \I$ wird eindeutig durch $\phi(1)$ bestimmt. Fasse nun $\mathcal{O}_X\vert_{U'}$ als Untergarbe von $\mathcal{O}_X\vert_U$ auf. Sei dazu $\mathcal{O}_U := \iota_{!}\mathcal{O}_X \vert_U$; dann gilt $\mathcal{O}_{U'} \subseteq \mathcal{O}_U$. Da $\I$ injektiv ist, gilt
$$\Hom_{\mathcal{O}_X} (\mathcal{O}_{U'}, \I) \cong \Hom_{\mathcal{O}_X}(\mathcal{O}_U, \I).$$
Insbesondere ist also 
$$\rho^{U}_{U'}: \I(U) = \Hom_{\mathcal{O}_X}(\mathcal{O}_U, \I) \la \Hom_{\mathcal{O}_X}(\mathcal{O}_{U'}, \I ) = \I(U')$$
surjektiv, was zu zeigen war. $\hfill \Box$


\end{pr}

\end{proposition}



\begin{proposition} %%Proposition 14.4
Sei $\lokring$ ein geringter Raum, $\F$ eine welke $\mathcal{O}_X$-Modulgarbe auf $X$. Dann Ist $\F$ azyklisch, das heißt es gilt 
$$H^{i}(X,\F) = 0$$
für alle $i \geqslant 1$. 

\begin{pr}
Sei $\I$ eine injektive $\mathcal{O}_X$-Modulgarbe auf $X$ mit $\F \subseteq \I$ und setze $\G:= \slant{\I}{\F}$. Wir erhalten damit eine kurze exakte Sequenz
$$0 \la \F \la \I \la \G \la 0.$$
Nach 14.5 ist $\I$ welk, nach 14.3 ist also auch $\G$ welk. Die lange exakte Kohomologiesequenz ist 
$$0 \rightarrow \F(X) \overset{\alpha_0}{\rightarrow} \I(X) \overset{\beta_0}{\rightarrow} \G(X) \overset{\delta^{0}}{\rightarrow} H^{1}(X, \F) \overset{\alpha_1}{\rightarrow} H^{1}(X, \I) \overset{\beta_1}{\rightarrow} H^{1}(X, \G) \overset{\delta^{1}}{\rightarrow} H^{2}(X, \F) \overset{\alpha_2}{\rightarrow} \ldots$$
Nach 14.2 (i) ist $\beta_0$ surjektiv, also $\kernel \delta^{0} = \bild \beta_0 = \G(X)$. Damit ist $\kernel \alpha_1 = \bild \delta^{0} = 0$, $\alpha_1$ ist also injektiv. Da $\I$ injektiv ist, also $H^{1}(X, \I) = 0$ für $i \geqslant 1$. Dann folgt aber bereits $H^{1}(X, \F) = 0$. Mit gleichem Argument gilt auch $H^{1}(X, \G) = 0$. Iterativ folgt damit die Behauptung. $\hfill \Box$

\end{pr}

\end{proposition}


\begin{proposition}   %%Proposition 14.5
Sei $X= \spec R$ ein affines noethersches Schema, $I$ ein injektiver $R$-Modul. Dann ist $\tilde{I}$ welk.
\begin{pr}
Da aus der Surjektivität von $\rho^{X}_{U'} = \rho^{U}_{U'} \circ \rho^{X}_{U}$ für $U' \subseteq U \subseteq X$ bereits die Surjektivität von $\rho^{U}_{U'}$ folgt, genügt es zu zeigen, dass für alle offenen Teilmengen $U \subseteq X$ die Restriktionsabbildung 
$$\rho^{X}_{U}: \tilde{I}(X) = I \la \tilde{I}(U)$$
surjektiv ist. Sei dazu zunächst $U=D(f)$ für ein $f\in R$. Dann ist 
$$\tilde{I}(U) = \tilde{I}(D(f)) = I_f = I\otimes_R R_f.$$
Sei also $\frac{b}{f^n} \in I_f$ mit $b \in I$ und $n \in \mathbb{N}_0$ und zeige, dass es ein $a \in I$ gibt mit 
$$\rho^{X}_{D(f)}(a) = \frac{a}{1} = \frac{b}{f^n}$$
in $I_f$, also
$f^{m} \left( f^{n} a - b \right) = 0 $
für ein $m \in \mathbb{N}_0$. Für jedes $m \in \mathbb{N}_0$ sei nun die $R$-lineare Abbildung 
$$\phi_m: R \la \left( f^{m+n}\right), \qquad 1 \mapsto f^{m+n}$$
gegeben. Dann gilt für den Kern
$$\kernel \phi_m = \mathrm{Ann}(f^{m+n}) = \{ r \in R \ \vert \ r f^{m+n} = 0 \} \subset R.$$
Außerdem gilt $\mathrm{Ann}(f^{k}) \subseteq \mathrm{Ann}(f^{k+1})$ für alle $k \in \mathbb{N}_0$. Da $R$ noethersch ist, gibt es ein $m \in \mathbb{N}_0$, sodass
$$\mathrm{Ann}(f^m) = \mathrm{Ann}(f^{m+1}) = \ldots = \mathrm{Ann}(f^{m+n}) = \ldots, $$
das heißt es gilt $\kernel \phi_m = \mathrm{Ann}(f^m)$. Mit dem Homomorpiesatz ist 
$$\slant{R}{\mathrm{Ann}(f^m)} \cong (f^{m+n})$$
als $R$-Moduln. Sei nun durch
$$\psi: R \la I, \qquad 1 \mapsto f^{m}b$$
eine weitere $R$-lineare Abbildung definiert. Dann gilt $\mathrm{Ann}(f^m) \subseteq \kernel \psi$, $\psi$ induziert also 
$$\overline{\psi}: \slant{R}{\mathrm{Ann}(f^m)} \cong \left(f^{m+n}\right) \la I.$$
$I$ ist injektiv, wir können $\overline{\psi}$ also fortsetzen zu $\tilde{\overline{\psi}}: R \la I$. Setze nun $a:= \tilde{\overline{\psi}}$. Dann gilt
$$f^{m} b = \psi(1) = \overline{\psi}(f^{m+n}) = \tilde{\overline{\psi}}(f^{m+n} \cdot 1) = \tilde{\overline{\psi}}(f^{n+m}) \tilde{\overline{\psi}} = f^{m+n} a,$$
woraus also die Behauptung für den fall $U= D(f)$ folgt. Sei nun $U$ beliebig. Sei $f \in R$ mit $D(f) \subseteq U$ und $t \in \tilde{I}(U)$. Dann gibt es nach dem Spezialfall ein $s \in I$ mit $s \vert_{D(f)} = t\vert_{D(f)}$. Damit gilt 
$$\mathrm{Supp}(s-t) \subseteq \mathrm{Supp}(\tilde{I}) \cap \mathrm{Supp}(U \setminus D(f)).$$
Zeige nun die Behauptung durch Induktion über $\dim \mathrm{Supp}(\tilde{I})=:n$: Für $n=0$ ist $\tilde{I}$ eine Wolkenkratzergarbe (bzw. eine Summe von Wolkenkratzergarben) und damit welk.
Den Fall $n\geqslant 1$ wollen wir nicht diskutieren - allerdings sei an dieser Stelle ein algebraischer Import bemerkt: Der $R$-Modul $J \subseteq I$, der von den Elementen mit Träger in $\overline{\mathrm{Supp}(\tilde{I})} \setminus D(f)$ ist injektiv. $\hfill \Box$
\end{pr}

\end{proposition}


\begin{folg}  %%Folgerung 14.6
Sei $X= \spec R$ ein noethersches, affines Schema und $\F$ eine quasikohärente Garbe auf $X$. Dann gilt für alle $i \geqslant 1$ 
$$H^{i}(X, \F) = 0.$$

\begin{pr}
Sei also $\F= \tilde{F}$ für den $R$-Modul $F= \mathcal{F}(X)$ sowie 
$$0 \la F \la I^{\bullet}$$
eine injektive Auflösung von $F$ in $\Rmod$. Dann ist die Sequenz
$$0 \la \tilde{F} \la \tilde{I}^{\bullet}$$
ebenfalls exakt, wir haben also eine Auflösung von $\F$ durch welke, also azyklische Garben. Damit gilt nach 12.6 und wegen der Exaktheit
$$H^{i}(X, \F) = H^{i}(\Gamma(X, \tilde{I}^{\bullet})) =H^{i}(I^{\bullet})= 0$$
für $i \geqslant 1$, was zu zeigen war. $\hfill \Box$

\end{pr}
\end{folg}


\begin{folg} %%Folgerung 14.7


Sei $\lokring$ ein noethersches Schema. Dann lässt sich jede quasikohärente Garbe auf $X$ in eine welke, quasikohärente Garbe einbetten.
\begin{pr}


Sei $\F$ quasikohärent und $X= \bigcup_{i=1}^n U_i$ eine offene Überdeckung von $X$ durch affine, noethersche Schemata $U_i = \spec R_i$. Nach Voraussetzung gibt es $R_i$-Moduln $M_i$ mit $\F \vert_{U_i} = \tilde{M}_i$. Bette nun $M_i$ in injektive $R_i$-Moduln $I_i$ für jedes $1 \leqslant i \leqslant n$ ein. Nach 14.5 ist die dazu gehörige quasikohärente Garbe $\tilde{I}_i$ welk. Man sieht leicht, dass auch das direkte Bild ${{\iota_i}_*}(\tilde{I}_i)$ für die Inklusion $\iota_i: U_i \hookrightarrow X$ welk ist. Setze nun
$$\rho_i : \tilde{M}_i = \F \vert_{U_i} \hookrightarrow \tilde{I}_i$$
und 
$$\rho: \F \hookrightarrow \bigoplus_{i=1}^n \iota_{i_{*}}(\tilde{I}_i)$$
Dann ist $\rho$ eine Einbettung von $\F$ in eine injektive, quasikohärente Garbe, woraus die Behauptung folgt. $\hfill \Box$
\end{pr}
\end{folg}


\begin{proposition}    %%Proposition 14.8
Sei $X$ ein topologischer Raum, $\F \in \AbX$ welk und $\mathfrak{U}$ eine offene Überdeckung von $X$. Dann gilt für alle $i \geqslant 1$
$$\check{H}^{i}(\mathfrak{U}, \F) = 0.$$

\begin{pr}
Wir zeigen zunächst, dass $\mathcal{C}^{k}(\mathfrak{U}, \mathcal{F})$ welk ist für alle $k \geqslant 0$, falls $\F$ welk ist. Ist $\F$ welk, so ist auch die Einschränkung $\F \vert_{U_{i_0} \cap \ldots \cap U_{i_k}}$ welk für alle $i_0 <\ldots < i_k$. Weiter ist für stetige Abbildungen auch das direkte Bild $f_* \F$ welk, es ist also $(\iota_{i_0 \ldots i_k})_* \F \vert_{U_{i_0} \cap \ldots \cap U_{i_k}}$ welk. Damit ist auch das direkte Produkt 
$$\mathcal{C}^{k}(\mathfrak{U}, \mathcal{F}) = \prod_{i_0 < \ldots < i_k} (\iota_{i_0 \ldots i_k})_* \F \vert_{U_{i_0} \cap \ldots \cap U_{i_k}}$$
welk. Wählen wir nun mit 
$$0 \la \F \la \mathcal{C}^{\bullet}(\mathfrak{U}, \F)$$
eine Auflösung von $\F$, so gilt nach 14.6
$$\check{H}^i(\mathfrak{U}, \F) = H^{i}(C^{\bullet}(\mathfrak{U}, \F))=H^{i}\left( \Gamma\left(X, \mathcal{C}^{\bullet}\left( \mathfrak{U}, \F\right)\right)\right) = H^{i}(X, \mathcal{F}) = 0$$
für alle $i \geqslant 1$, was zu zeigen war. $\hfill \Box$
\end{pr}


\end{proposition}


\begin{lemma}
Sei $X$ noethersches separiertes Schema und $\mathfrak{U}=\{U_i\}_{i \in \mathbb{N}}$ eine offene, affine Überdeckung von $X$. Dann gibt es auch für die \v{C}echkohomologie eine lange exakte Sequenz.
\begin{pr}
Sei also eine kurze exakte Sequenz $0 \la \F \la \G \la \mathcal{H} \la 0$ quasikohärenter Garben gegeben. Da $X$ separiert ist, ist $U_{i_0} \cap \ldots \cap U_{i_k}$ affin für alle $i_0 < \ldots < i_k$. Nach 14.6 gilt also 
$$H^{i}\left(U_{i_0} \cap \ldots \cap U_{i_k}, \F \vert_{U_{i_0} \cap \ldots \cap U_{i_k}} \right) = 0$$
für alle $i \geqslant 1$, das heißt die Sequenz 
$$0 \la \F \left( U_{i_0} \cap \ldots \cap U_{i_k}\right) \la \G\left( U_{i_0} \cap \ldots \cap U_{i_k}\right) \la \mathcal{H}\left( U_{i_0} \cap \ldots \cap U_{i_k}\right) \la 0$$
ist exakt. Produktbildung liefert exakte Sequenzen


$$ 0 \la  \prod_{i \in \mathbb{N}} \mathcal{F}(U_i) \la \prod_{i \in \mathbb{N}} \mathcal{G}(U_i) \la \prod_{i \in \mathbb{N}} \mathcal{H}(U_i) \la 0$$
$$ 0 \la  \prod_{i<j \in} \mathcal{F}(U_i \cap U_j) \la \prod_{i <j} \mathcal{G}(U_i \cap U_j) \la \prod_{i <j} \mathcal{H}(U_i \cap U_j) \la 0$$
$$0 \la \prod_{i_0 < \ldots < i_k} \F(U_{i_0} \cap \ldots \cap U_{i_k}) = C^{k}(\mathfrak{U}, \F) \la C^{k}(\mathfrak{U}, \G) \la C^{k}(\mathfrak{U}, \mathcal{H}) \la 0,$$

also eine exakte Sequenz von Komplexen. Dazu gibt es aber eine lange exakte Sequenz, woraus die Behauptung folgt. $\hfill \Box$
\end{pr}
\end{lemma}


\begin{theorem}    %%Satz 14.9
Sei $\lokring$ ein noethersches, separiertes Schema, $\F$ eine quasikohärente $\mathcal{O}_X$-Modulgarbe auf $X$ und $\mathfrak{U}$ eine offene, affine Überdeckung von $X$. Dann gilt für alle $i \geqslant 1$
$$H^{i}(X, \F) \cong \check{H}^{i}(\mathfrak{U}, \F).$$

\begin{pr}
Wir beweisen die Aussage durch Induktion über $i$.
\begin{compactenum}
\item[$i=0:$] Klar, denn es gilt $H^{0}(\F, X) = \F(X) = \Gamma(X, \F) = \check{H}^{0}(\mathfrak{U}, \F)$.
\item[$i \geqslant 0:$] Bette $\F$ in eine welke, quasikohärente Garbe $\G$ ein und setze $\mathcal{H}:= \slant{\G}{\F}$. Dann ist auch $\mathcal{H}$ quasikohärent und die Sequenz 
$$0 \la \F \la \G \la \mathcal{H} \la 0$$
ist exakt. Damit gibt es eine lange, exakte Kohomologiesequenz
$$0 \rightarrow \F(X) \rightarrow \G(X) \rightarrow \mathcal{H}(X) \rightarrow H^{1}(X, \F) \rightarrow H^{1}(X, \G) \rightarrow H^{1}(X, \mathcal{H}) \rightarrow H^{2}(X, \F) \rightarrow  \ldots $$
und wegen $H^{i}(X, \G)= 0$ für alle $i \geqslant 1$ wir diese zu
$$0 \rightarrow \F(X) \rightarrow \G(X) \rightarrow \mathcal{H}(X) \rightarrow H^{1}(X, \F) \rightarrow 0 \rightarrow H^{1}(X, \mathcal{H}) \rightarrow H^{2}(X, \F) \rightarrow 0 \rightarrow  \ldots $$
da diese exakt ist, folgt daraus
$$H^{i}(X, \mathcal{H}) \cong H^{i+1}(X, \F)$$
für alle $i \geqslant 1$. Nach Lemma 14.11 gibt es für die \v{C}echkohomologie ebenfalls eine lange exakte Sequenz
$$0 \rightarrow \F(X) \rightarrow \G(X) \rightarrow \mathcal{H}(X) \rightarrow \check{H}^{1}(\mathfrak{U}, \F) \rightarrow \check{H}^{1}(\mathfrak{U}, \G) \rightarrow \check{H}^{1}(\mathfrak{U}, \mathcal{H}) \rightarrow \check{H}^{2}(\mathfrak{U}, \F) \rightarrow  \ldots $$
welche zu 
$$0 \rightarrow \F(X) \rightarrow \G(X) \rightarrow \mathcal{H}(X) \rightarrow \check{H}^{1}(\mathfrak{U}, \F) \rightarrow 0 \rightarrow \check{H}^{1}(\mathfrak{U}, \mathcal{H}) \rightarrow \check{H}^{2}(\mathfrak{U}, \F) \rightarrow 0 \rightarrow  \ldots $$
wird. Auf gleiche Weise erhalten wir $\check{H}^{i}(\mathfrak{U}, \mathcal{H}) \cong \check{H}^{i+1}(\mathfrak{U}, \F)$. Damit gilt 
$$\check{H}^{1}(\mathfrak{U}, \F) = \slant{\check{H}^{0}(\mathfrak{U}, \mathcal{H})}{\check{H}^{0}(\mathfrak{U}, \G)} \cong \slant{H^{0}(X, \mathcal{H})}{H^{0}(X, \G)} \cong H^{1}(X, \F)$$
und 
$$\check{H}^2(\mathfrak{U}, \F) \cong \check{H}^{1}(\mathfrak{U}, \mathcal{H}) \cong H^{1}(X, \mathcal{H}) \cong H^{2}(X, \F).$$
Iterativ folgt damit die Behauptung. $\hfill \Box$

\end{compactenum}

\end{pr}

\end{theorem}


\begin{ex}
Sei $X=\mathbb{A}^{2}_k \setminus \{(0,0)\}$ und $\mathfrak{U}=\{U_1, U_2\}$ mit $U_1 = D(x), U_2 =D(y)$ eine offene Überdeckung von $X$ sowie $\F = \mathcal{O}_X = \mathcal{O}_{\mathbb{A}^2_k} \vert_{\mathbb{A}^2_k \setminus \{(0,0)\}}$ die Strukturgarbe. Dann ist 
$$\check{C}^{0}(\mathfrak{U}, \mathcal{O}_X) = \mathcal{O}_X(U_1) \oplus \mathcal{O}_X(U_2) = k[X,Y]_X \oplus k[X,Y]_Y$$
$$\check{C}^{1}(\mathfrak{U}, \mathcal{O}_X) = \mathcal{O}_X(U_1 \cap U_2) = \mathcal{O}_X(D(XY)) = k[X,Y]_{XY}.$$
Weiter ist 
$$d^{0}: \check{C}^{0}(\mathfrak{U}, \mathcal{O}_X) \la \check{C}^{1}(\mathfrak{U}, \mathcal{O}_X), \qquad \left( \frac{f}{X^{i}}, \frac{g}{Y^{j}} \right) \mapsto \frac{g}{Y^{j}} - \frac{f}{X^{i}}$$
Damit ist 
$$H^{1}(X, \mathcal{O}_X) = \check{H}^{1}(\mathfrak{U}, \mathcal{O}_X) = \slant{\check{C}^{1}(\mathfrak{U}, \mathcal{O}_X)}{d^{0}(\check{C}^{1}(\mathfrak{U}, \mathcal{O}_X))}$$
der von den $\frac{1}{X^{i}Y^{j}}$ für $i,j \geqslant 1$ erzeugte unendlichdimensionale $k$-Vektorraum.
\end{ex}









\renewcommand*\thesection{§ \arabic{section}\quad}
\section{Kohomologie auf projektiven Schemata} %PARAGRAPH 15
\renewcommand*\thesection{\arabic{section}}


\begin{erinnbem}    %%Erinnerung + Bemerkung 15.1
Sei $S= \bigoplus_{d=0}^{\infty}S_d$ ein graduierter, kommutativer Ring mit Eins. Definiere
$$\Proj S :=  \left\{ \p \in \spec S \ \vert \ \p \textrm{ ist homogen mit } S_+ \nsubset \p \right\},$$
wobei $S_+:= \bigoplus_{d=1}^{\infty} S_d$ das irrelevante Ideal von $S$ ist. Die Menge $\Proj S$ wird \textit{homogenes Spektrum von} $S$ genannt. Für ein homogenes Ideal $I \subset S$ setzen wir
$$V(I) := \left\{ \p \in \Proj S \ \vert \ \p \supseteq I \right\}.$$
Wir erhalten damit auf $\Proj S$ eine Topologie, die Zariski-Topologie, indem wir $V(I)$ als abgeschlossen definieren. Für homogenes Elemente $f \in S$ bilden die Mengen 
$$D_+(f) := \left\{ \p \in \Proj S \ \vert \ f \notin \p \right\}$$
eine Basis der Topologie. Die Strukturgarbe auf $\Proj S$ erhalten wir durch Fortsetzen von 
$$\mathcal{O}_{\Proj S}\left(D_+(f)\right) := S_{(f)} := \left\{ \frac{s}{f^n} \ \vert \ s \in S,\ n \in \mathbb{N}, \ \deg s = n \deg f \right\} $$
zu einer Garbe auf $\Proj S$, wobei $f \in S$ homogen ist. Damit wird $(\Proj S, \mathcal{O}_{\Proj S})$ zu einem lokal geringten Raum und wir schreiben $\Proj S$ statt $(\Proj S, \mathcal{O}_{\Proj S})$. Mit
$$\left( D_+(f), \mathcal{O}_{\Proj S} \vert_{D_+(f)} \right) \cong \left( \spec S_{(f)}, \mathcal{O}_{\spec S_{(f)}} \right)$$
erhalten wir eine affine Überdeckung von $\Proj S$, wodurch $\Proj S$ also zum Schema wird. Ist $\p \in \Proj S$, so ist der lokale Ring in $\p$ gegeben durch 
$$\mathcal{O}_{\Proj S, \p} = S_{(\p)} = \left\{ \frac{s}{r} \ \vert \ s,r \in S \textrm{ homogen, } r \notin \p, \ \deg r = \deg s \right\}.$$
Sei nun $M= \bigoplus_{d \in \mathbb{Z}} M_d$ ein graduierter $S$-Modul.
\begin{compactenum}
\item $M$ bestimmt eine Garbe $\tilde{M}$ auf $\Proj S$ durch 
$$\tilde{M}(D_+(f)) = M_{(f)} = \left\{ \frac{m}{f^n} \ \vert \ m \in M \textrm{ homogen }, \ \deg m = n \deg f \right\}.$$
Für jedes homogene $f \in S$. Insbesondere ist 
$$\Gamma(X, \tilde{M}) = \tilde{M}(X) = M_0.$$
\item Für die Halme gilt $\tilde{M}_x = M_{(\p)}$ für $x=\p \in \Proj S$.

\end{compactenum}
\end{erinnbem}


\begin{remark}   %%Bemerkung 15.2
Sei $R$ ein noetherscher Ring, $S=R[X_0, \ldots, X_n]$, $S= \Proj S =: \Pro^n_R$. Ist $\F$ eine kohärente $\mathcal{O}_X$ Modulgarbe auf $X$, so gibt es einen graduierten, endlich erzeugten $S$-Modul $M$, sodass gilt $\F=\tilde{M}$.
\begin{pr}
Siehe Übungsaufgabe 9.3. $\hfill \Box$
\end{pr}

\end{remark}

\begin{definbem}   %%deifnition + Bemerkung 15.3
Sei $S=\bigoplus_{d=0}^{\infty}$ ein graduierter, kommutativer Ring mit Eins und $X= \Proj S$.
\begin{compactenum}
\item Es gilt $\tilde{S}=\mathcal{O}_X$.
\item Für $n \in \mathbb{Z}$ sei $S(n)= \bigoplus_{d=0}^{\infty} S(n)_d$ der graduierte $S$-Modul mit getwisteter Graduierung 
$$S(n)_d= S_{d+n}.$$
\item $\mathcal{O}_X(n) := \widetilde{S(n)}$ heißt die $n$-\textit{fach getwistete Strukturgarbe}.
\item Ist $S=R[X_0, \ldots, X_n]$, so ist
$$H^{0}(X, \mathcal{O}_X(1)) = \Gamma(X, \mathcal{O}_X(1)) = S(1)_0 = S_1$$
der freie $R$-Modul mit Basis $X_0, \ldots, X_n$. Allgemeiner ist 
$$H^{0}(X, \mathcal{O}_X(d)) = \Gamma(X, \mathcal{O}_X(d)) = S(d)_0 = S_d$$
der freie $R$-Modul erzeugt von den Monomen von Grad $d$, das heißt die 
$$\left\{X_0^{d_0} \cdots X_n^{d_n} \ \bigg\vert \ \sum_{i=0}^n d_i = d \right\}$$
bilden eines Basis. Insbesondere gilt damit für alle $d <0$
$$H^{0}(X, \mathcal{O}_X(d)) = 0.$$

\end{compactenum}

\end{definbem}


\begin{remark}
Sei $R$ ein noetherscher Ring und $X \subseteq \Pro^n_R$ ein abgeschlossenes Unterschema sowie $j: X \hookrightarrow \Pro^n_R$ die Einbettung. Ist $\F$ eine Garbe abelscher Gruppen auf $X$, so gilt 
$$H^{i}(X, \F) \cong H^{i}(\Pro^n_R, j_*\F)$$
für alle $i \geqslant 0$.
\begin{pr}
Ist $0 \la \F \la \mathcal{J}^{\bullet}$ eine welke Auflösung. Dann ist $0 \la j_*\F \la j_* \mathcal{J}^{\bullet}$ eine welke Auflösung von $j_* \F$. Aus $\Gamma(X, \mathcal{J}^k) = \Gamma(\Pro^n_R, j_*\mathcal{J}^k)$ folgt dann $H^{i}(X, \F) = H^{i}(\Pro^n_R, j_*\F)$, was zu zeigen war. $\hfill \Box$
\end{pr}

\end{remark}


\begin{theorem}   %%Satz 15.4
Sei $R$ ein noetherscher Ring, $n \geqslant 1$, $S=R[X_0, \ldots, X_n]$ und $(X, \mathcal{O}_X):= (\Pro^n_R, \mathcal{O}_{\Pro^n_R})$.
\begin{compactenum}
\item Es gilt für die $n$-te Kohomologiegruppe der Garbe $\mathcal{O}_X({-n-1})$
$$H^{n}(X, \mathcal{O}_X(-n-1)) \cong R.$$
\item Für jedes $d \in \mathbb{Z}$ gibt es eine natürliche, bilineare Abbildung
$$\beta: H^{0}(X, \mathcal{O}_X(d)) \times H^{n}(X, \mathcal{O}_X(-d-n-1)) \la R.$$
Diese ist eine nicht ausgeartete Paarung zwischen freien $R$-Moduln von endlichem Rang.
\item Für alle $i \notin \{0,n\}$ und alle $d \in \mathbb{Z}$ ist
$$H^{i}(X, \mathcal{O}_X(d)) = 0.$$

\end{compactenum}

\begin{pr}
\begin{compactenum}
\item Sei $U_i=D(X_i)$ und $\mathfrak{U}= \{U_0, \ldots, U_n\}$ die kanonische Überdeckung von $X$ durch affine, offene Teilmengen. Nach 14.9 gilt dann 
$$H^{i}(X, \mathcal{O}_X(d)) = \check{H}^{i}(\mathfrak{U}, \mathcal{O}_X(d))$$
für alle $i \in \mathbb{N}_0$ und $d\in \mathbb{Z}$. Damit folgt sofort
$$H^{i}(X, \mathcal{O}_X(d)) = 0$$
für alle $i \geqslant n+1$ und $d \in \mathbb{Z}$. Wir betrachten nun den \v{C}ech-Komplex an der $n$-ten Stelle:

$$\ldots \overset{d^{n-2}}{\rightarrow} \bigoplus_{i=0}^n \mathcal{O}_X(d)(U_0 \cap \ldots \cap U_{i-1} \cap U_{i+1} \cap \ldots \cap U_n) \ \overset{d^{n-1}}{\longrightarrow} \ \mathcal{O}_X(d)(U_0 \cap \ldots \cap U_n) \ \overset{d^{n}}{\rightarrow} \ 0 \ \rightarrow \ldots $$
Es gilt 
$$\check{H}^n(\mathfrak{U}, \mathcal{O}_X(d)) = \slant{\kernel d^{n}}{\bild d^{n-1}}.$$
Es ist zum einen 
\setlength{\abovedisplayskip}{5.5pt}
\setlength{\belowdisplayskip}{5.5pt}
\begin{alignat*}{5}
\kernel d^{n} \ \ &=&& \ \ \mathcal{O}_X(d)(U_0 \cap \ldots \cap U_n)\\
& =&& \ \ \mathcal{O}_X(d)(D(X_0 \cdots X_n))\\
&=&& \ \ R[X_0, \ldots, X_n]_{(X_0 \cdots X_n)}\\
&=&& \ \ \left\{ \frac{f}{X_0^{d_0} \cdots X_n^{d_n}} \ \bigg\vert  \ f \in S] \textrm{ homogen, } \deg f = d+ \sum_{i=0}^n d_i \right\}
\end{alignat*}
der freie $R$-Modul mit Basis 
$$\left\{ X_0^{d_0} \cdots X_n^{d_n} \ \bigg\vert \ d_i \in \mathbb{Z}, \ \sum_{i=0}^n d_i = d \right\}$$
sowie 
\setlength{\abovedisplayskip}{5.5pt}
\setlength{\belowdisplayskip}{5.5pt}
\begin{alignat*}{5}
\left(\check{C}^{n-1}(\mathfrak{U}, \mathcal{O}_X(d))\right)_i \ \ &=&& \ \ \mathcal{O}_X(d)(U_0 \cap \ldots \cap U_{i-1} \cap U_{i+1} \cap \ldots \cap U_n) \\
&=&& \ \ \mathcal{O}_X(d)(D(X_0 \cdots X_{i-1}X_{i+1} \cdots X_n))\\
&=&& \ \ \left\{ \frac{f}{X_0^{d_0} \cdots X_{i-1}^{d_{i-1}} X_{i+1}^{d_{i+1}} \cdots X_n^{d_n}} \ \bigg\vert \ f \in S \textrm{ homogen}, \ \deg f = d + \sum_{i=0}^n d_i \right\}
\end{alignat*}
der freie $R$-Modul mit Basis 
$$\left\{ X_0^{d_0}\cdots X_n^{d_n} \ \bigg \vert \ d_j \in \mathbb{Z}, \ \sum_{j=0}^{n} d_j = d, \ d_j \geqslant 0 \right\}.$$
Damit sehen wir ein:
$$X_0^{d_0} \cdots X_n^{d_n} \in \bild d^{n-1} \quad \Longleftrightarrow \quad d_i \geqslant 0 \textrm{ für ein } 0 \leqslant i \leqslant n.$$
Daraus folgt: Ist $d \geqslant -n$, so ist $d^{n-1}$ surjektiv, es gilt also 
$$H^{n}(X, \mathcal{O}_X(d)) = 0.$$
Ist $d=-n-1$, so folgt $d_i=-1$ für alle $0 \leqslant i \leqslant n$, es ist also 
$$H^{n}(X, \mathcal{O}_X(-n-1)) = \left\langle \frac{1}{X_0 \cdots X_n} \right\rangle_{R-\mathrm{Mod}}$$
ein freier $R$-Modul von Rang $1$, woraus die Behauptung folgt.

\item Ist $d<0$, so haben wir oben gesehen, dass $d^{n-1}$ surjektiv ist, denn aus $$\sum_{j=0}^n d_j = -d -n-1 > -n-1$$ folgt $d_j \geqslant 0$ für ein $0 \leqslant j \leqslant n$ und damit $X_0^{d_0} \cdots X_n^{d_n} \in \bild d^{n-1}$. Damit ergibt sich 
$$H^{n}(X, \mathcal{O}_X(-d-n-1)) = 0.$$
Sei nun also $d \geqslant 0$. Dann wird $H^{0}(X, \mathcal{O}_X(d))$ als $R$-Modul frei erzeugt von den $X_0^{\nu_0} \cdots X_n^{\nu_n}$ mit $\sum_{j=0}^n \nu_j =d$ und $\nu_j \geqslant 0$ für alle $0 \leqslant \nu \leqslant n$ (das sind die gewöhnlichen Monome von Grad $d$). Wie oben bereits gesehen, wird $H^{n}(X, \mathcal{O}_X(-d-n-1))$ als $R$-Modul frei erzeugt von den $X_0^{\mu_0} \cdots X_n^{\mu_n}$ mit $\sum_{j=0}^n \mu_j = -d-n-1$ und $\mu_j <0$ für alle $0 \leqslant j \leqslant n$. Definiere nun die Abbildung 

\setlength{\abovedisplayskip}{5.5pt}
\setlength{\belowdisplayskip}{5.5pt}
\begin{alignat*}{5}
\beta:H^{0}(X, \mathcal{O}_X(d)) \times H^{n}(X, \mathcal{O}_X(-d-n-1)) \ \ && \la& \ \ H^{n}(X, \mathcal{O}_X(-n-1)) \cong R \\
\left(X_0^{\nu_0} \cdots X_n^{\nu_n}, \ X_0^{\mu_0} \cdots X_n^{\nu_n} \right)\ \  && \mapsto  & \ \ X_0^{\mu_0 + \nu_0} \cdots X_n^{\mu_n + \nu_n} 
\end{alignat*}
$\beta$ ist wohldefiniert, denn die Summe der Exponenten ist gerade $-n-1$. Sicherlich ist $\beta$ bilinear. Es bleibt noch zu zeigen, dass $\beta$ nicht ausgeartet ist, für jede Wahl von $(\nu_0, \ldots, \nu_n)$ und $(\mu_0, \ldots, \mu_n) \in \mathbb{Z}^n$ mit $\sum_{j=0}^n \nu_j =d$ und $\nu_j \geqslant 0$ für alle $0 \leqslant j \leqslant n$ bzw. $\sum_{j=0}^n \mu_j = -d-n-1$ und $\mu_j <0$ für alle $0 \leqslant j \leqslant n$ also die Abbildungen 
$$\beta_1: H^{0}(X, \mathcal{O}_X(d)) \la H^{n}(X, \mathcal{O}_X(-n-1)), \qquad s \mapsto \beta\left(s, X_0^{\mu_0} \cdots X_n^{\mu_n} \right)$$
$$\beta_2: H^{n}(X, \mathcal{O}_X(-d-n-1)) \la H^{n}(X, \mathcal{O}_X(-n-1)), \qquad s \mapsto \beta\left(X_0^{\nu_0} \cdots X_n^{\nu_n}, s \right)$$
nicht die Nullabbildungen sind. Das folgt aus 
$$\beta_1\left(X_0^{-\mu_0-1} \cdots X^{-\mu_n-1} \right) = \beta\left(X_0^{-\mu_0-1} \cdots X^{-\mu_n-1},\ X_0^{\mu_0} \cdots X^{\mu_n} \right) = \frac{1}{X_0 \cdots X_n} \neq 0,$$
$$\beta_2\left(X_0^{-\nu_0-1} \cdots X^{-\nu_n-1} \right) = \beta\left( X_0^{\nu_0} \cdots X^{\nu_n},\ X_0^{-\nu_0-1} \cdots X^{-\nu_n-1} \right) = \frac{1}{X_0 \cdots X_n} \neq 0.$$


\item Aus den Überlegungen in (i) wissen wir, dass $H^{i}(X, \mathcal{O}_X(d)) = 0$ für alle $i \geqslant n+1$, wir nehmen also $0 < i \leqslant n-1$ an. Man überlegt sich leicht, dass jedes Element in $H^{i}(X, \mathcal{O}_X(d))$ von einem Tupel von Linearkombinationen von Monomen der Form 
$X_{j_0}^{\nu_0} \cdots X_{j_i}^{\nu_i}$ mit $ \sum_{l=0}^{i} \nu_l = d$ und $\nu_l <0$ für alle $0 \leqslant l \leqslant i$
repräsentiert wird.
\begin{compactenum}
\item[\textbf{Beh. (a)}] Für jedes $k \in \{0, \ldots, n \}$ ist die Multiplikation mit $X_k$
$$\mu_i: H^{i}(X, \mathcal{O}_X(d)) \la H^{i}(X, \mathcal{O}_X(d+1)), \qquad s \mapsto X_k s$$
ein Isomorphismus.
\end{compactenum}
Dann folgt: Ist $X_{j_0}^{\nu_0} \cdots X_{j_i}^{\nu_i} \in H^{i}(X, \mathcal{O}_X(d))$, so multiplizieren wir mit $X_{j_0}^{-\nu_0}$ und erhalten 
$$X_{j_0}^{-\nu_0} \left( X_{j_0}^{\nu_0} \cdots X_{j_i}^{\nu_i} \right) = X_{j_1}^{\nu_1} \cdots X_{j_i}^{\nu_i} = 0$$
in $H^{i}(X, \mathcal{O}_X(d-\nu_0))$. Mit Behauptung (a) folgt dann, dass bereits $X_{j_0}^{\nu_0} \cdots X_{j_i}^{\nu_i}=0$ in $H^{i}(X, \mathcal{O}_X(d))$ gilt, also $H^{i}(X, \mathcal{O}_X(d))=0$, was zu zeigen war. Wir müssen nun noch die Behauptung (a) zeigen.

\begin{compactenum}
\item[\textbf{Bew. (a)}] Sei ohne Einschränkung $k=n$. Wir haben eine exakte Sequenz von graduierten $S$-Moduln
$$0 \la S(d) \overset{\cdot X_n}{\la} S(d+1) \la \slant{S(d+1)}{X_n S(d)} \la 0 \quad (*)$$
mit $$\slant{S(d+1)}{X_n S(d)} \cong \left( \slant{S}{X_n S}\right)(d+1) \cong R[X_0, \ldots, X_{n-1}] (d+1).$$
Diese liefert eine exakte Sequenz
$$0 \la \mathcal{O}_X(d) \la \mathcal{O}_X(d+1) \la \slant{\mathcal{O}_X(d+1)}{X_n \mathcal{O}_X(d)} \la 0 \quad (**)$$
von Garben. Ist $j: Z:=V(X_n) \cong \Pro^{n-1}_R \hookrightarrow \Pro^n_R=X$ die Einbettung, so erhalten wir 
$$\slant{\mathcal{O}_X(d+1)}{X_n \mathcal{O}_X(d)} \cong j_* \mathcal{O}_{Z}(d+1).$$
Wir zeigen die Behauptung nun durch Induktion über $n$. Ist $n=1$, so ist wegen $0 < i \leqslant 0$ ist nichts zu zeigen, sei also $n \geqslant 2$. Nach Bemerkung 15.4 gilt $H^{i}(\Pro^{n-1}_R, \F) = H^{i}(X, j_*\F)$ für jede Garbe $\F$ auf $Z$. Wir betrachten nun die lange exakte Kohomologiesequenz zu (**):
\setlength{\abovedisplayskip}{5.5pt}
\setlength{\belowdisplayskip}{5.5pt}
\begin{alignat*}{5}
\ldots \rightarrow H^{i-1}(Z, \mathcal{O}_{Z}(d+1)) \ &\rightarrow&& \  H^{i}(X, \mathcal{O}_X(d)) \overset{\mu_i}{\rightarrow} H^{i}(X, \mathcal{O}_X(d+1))\\
& \rightarrow&& \ H^{i}(Z, \mathcal{O}_{Z}(d+1)) \rightarrow \ldots \quad (***)
\end{alignat*}
Nach Induktionsvoraussetzung ist $H^{i}(Z, \mathcal{O}_{Z}(d+1)) = 0$ für $0 \leqslant i \leqslant n-2$, also 
$$H^{i-1}(Z, \mathcal{O}_{Z}(d+1)) = 0 = H^{i}(Z, \mathcal{O}_{Z}(d+1))$$
für $1 \leqslant i \leqslant n$. Dann wird die Sequenz zu
$$\ldots \rightarrow 0 \rightarrow H^{i}(X, \mathcal{O}_X(d))\overset{\mu_i}{\rightarrow} H^{i}(X, \mathcal{O}_X(d+1)) \rightarrow 0 \rightarrow \ldots $$
und da sie noch exakt ist folgt 
$$H^{i}(X, \mathcal{O}_X(d)) \overset{\mu_i}{\cong} H^{i}(X, \mathcal{O}_X(d+1)) $$
für alle $1 \leqslant i \leqslant n-2$. Es bleiben noch die Fälle $i \in \{1, n-1\}$ zu betrachten.
\begin{compactenum}
\item[$i=1:$] Wir betrachten (***) an der richtigen Stelle:
\setlength{\abovedisplayskip}{5.5pt}
\setlength{\belowdisplayskip}{5.5pt}
\begin{alignat*}{5}
\ldots \rightarrow H^{0}(Z, \mathcal{O}_{Z}(d+1)) \ &\overset{\alpha}{\rightarrow} && \  H^{1}(X, \mathcal{O}_X(d)) \overset{\mu_1}{\rightarrow} H^{1}(X, \mathcal{O}_X(d+1))\\
& \rightarrow&& \ H^{1}(Z, \mathcal{O}_{Z}(d+1)) \rightarrow \ldots
\end{alignat*}
dies entspricht aber gerade der Sequenz (*), also ist $\alpha$ die Nullabbildung und $\mu_1$ somit injektiv. Für $n \geqslant 3$ ist außerdem $H^{i}(Z, \mathcal{O}_{Z}(d+1)) =0$, woraus die Surjektivität von $\mu_1$ für $n \geqslant 3$ folgt. Für $n=2$ betrachte den Fall $i=n-1$.

\item[$i=n-1:$] Wir betrachten nun also 
\setlength{\abovedisplayskip}{5.5pt}
\setlength{\belowdisplayskip}{5.5pt}
\begin{alignat*}{5}
\ldots \rightarrow H^{n-2}(Z, \mathcal{O}_{Z}(d+1)) \ &\overset{\alpha}{\rightarrow} && \  H^{n-1}(X, \mathcal{O}_X(d)) \overset{\mu_{n-1}}{\rightarrow} H^{n-1}(X, \mathcal{O}_X(d+1))\\
& \rightarrow&& \ H^{n-1}(Z, \mathcal{O}_{Z}(d+1)) \rightarrow \ldots
\end{alignat*}

Für $n\geqslant 3$ ist $H^{n-2}(Z, \mathcal{O}_{Z}(d+1))=0$, also $\mu_{n-1}$ injektiv. Für $n=2$ betrachte den Fall $i=1$. Zeige nun noch die Surjektivität von $\mu_{n-1}$.

Wir haben die Sequenz
\setlength{\abovedisplayskip}{5.5pt}
\setlength{\belowdisplayskip}{5.5pt}
\begin{alignat*}{5}
\ldots \rightarrow H^{n-1}(X, \mathcal{O}_X(d)) \ \ &\overset{\mu_{n-1}}{\longrightarrow}&& \ \ H^{n-1}(X, \mathcal{O}_X(d+1)) \\
&\overset{\alpha_{n-1}}{\longrightarrow}&& \ \ H^{n-1}(Z, \mathcal{O}_{Z}(d+1))\\
& \overset{\delta^{n-1}}{\longrightarrow}&& \ \ H^{n}(X, \mathcal{O}_X(d)) \\
& \overset{\mu_{n}}{\longrightarrow}&&\ \ H^{n}(X, \mathcal{O}_X(d+1)) \ \ \longrightarrow \ldots 
\end{alignat*}
$\mu_{n-1}$ ist surjektiv genau dann, wenn $\alpha_{n-1}$ die Nullabbildung ist, was wiederum äquivalent dazu ist, dass $\delta^{n-1}$ injektiv ist. Dies wiederum ist genau dann der Fall, wenn 
$$\mathrm{Rang}\left( \bild \delta^{n-1}\right) = \mathrm{Rang } \left( H^{n-1}(Z, \mathcal{O}_Z(d+1))\right)$$
(als freie $R$-Moduln). Dies können wir zeigen: Es ist $H^{n-1}(Z, \mathcal{O}_Z(d+1))$ der freie $R$-Modul mit Basis 
$$ \left\{ X_0^{\nu_0} \cdots X_{n-1}^{\nu_{n-1}} \ \bigg \vert \ \sum_{j=0}^{n-1} \nu_j = d+1, \ \nu_j <0 \textrm{ für alle } 0 \leqslant j \leqslant n-1 \right\}.$$
Weiter gilt wegen der Exaktheit der Sequenz $\bild \delta^{n-1} = \kernel \mu_n$ und $\kernel \mu_n$ ist der frei erzeugte $R$-Modul mit Basis
$$\left\{ X_0^{\mu_0} \cdots X_{n-1}^{\mu_{n-1}} X_n^{-1} \ \bigg\vert \ \left(\sum_{j=0}^{n-1}\mu_j \right) -1 = d, \ \mu_j < 0 \textrm{ für alle } 0 \leqslant j \leqslant n-1 \right\}.$$
Damit folgt offenbar $\kernel \mu_n = H^{n-1}(Z, \mathcal{O}_Z(d+1))$, und rückwärts lesen liefert die Surjkektivität von $\mu_{n-1}$, was zu zeigen war. $\hfill \Box$



\end{compactenum}



\end{compactenum}

\end{compactenum}


\end{pr}


\end{theorem}


\begin{proposition}   %%Proposition 15.6
 Sei $R$ ein noetherscher Ring, $X \subseteq \Pro^n_R$ ein projektives Schema und $\F$ eine kohärente $\mathcal{O}_X$-Modulgarbe auf $X$. Dann gibt es ein $d_0 \in \mathbb{Z}$, sodass für alle $d \geqslant d_0$ die getwistete Garbe $\F(d) := \F \otimes_{\mathcal{O}_X} \mathcal{O}_X(d)$ von globalen Schnitten erzeugt wird, es also $s_1, \ldots, s_r \in \Gamma( X, \mathcal{F}(d))$ gibt, sodass für alle $x \in X$ die Keime $(s_1)_x, \ldots, (s_r)_x$ den Halm $\F(d)_x$ als $\mathcal{O}_{X,x}$-Modul erzeugen.
 
 \begin{pr}
Ohne Einschränkung sei $X= \Pro^n_R$. Überdecke $\Pro^n_R$ durch $D(X_i)$ für $0 \leqslant i \leqslant n$. Nach Voraussetzung gibt es endlich erzeugte Moduln $M_i= \F(U_i)$ über $\mathcal{O}_X(U_i) \cong R\left[ \frac{X_0}{X_i}, \ldots, \frac{X_n}{X_i} \right]$, sodass gilt $\mathcal{F}\vert_{U_i} = \tilde{M_i}$. Sei nun $i$ fest und $s_{i_1}, \ldots, s_{i_{k_i}}$ Erzeuger von $M_i$ als $R\left[ \frac{X_0}{X_i}, \ldots, \frac{X_n}{X_i} \right]$-Modul. Dann wird für jedes $x \in \Pro^n_R$ mit $x \in U_i$ der Halm $\F_x= (\tilde{M_i})_x$ erzeugt von den Keimen $(s_{i_1})_x, \ldots, (s_{i_{k_i}})_x$. Ziel soll es sein, die $s_{i_{j}}$ auf geeignete Weise zu globalen Schnitten fortzusetzen. Dazu zeigen wir, dass sich $t_{i_j}:= s_{i_j} X_i^{d_i} \in \tilde{M_i} \otimes_{\mathcal{O}_X} \mathcal{O}_X(d)$ zu einem globalen Schnitt in $\F(d_i)$ fortsetzt. Sei $ 0 \leqslant j \leqslant n$ mit $j\neq i$ und $\nu \in \{1, \ldots, k_i\}$. Dann ist $$s_{i_{\nu}} \vert_{U_i \cap U_j} \in \F(U_i \cap U_j) = (M_j)_{\frac{X_j}{X_i}} = \left\{ \left(\frac{X_i}{X_j}\right)^p \cdot m_j \ \vert \ m_j \in M_j, \ p \in \mathbb{N}_0 \right\}.$$
Dann gibt es ein $d_{j_{\nu}} \in \mathbb{N}_0$ mit $s_{i_{\nu}} X_i^{d_{j_{\nu}}} \in M_j$, also $s_{i_{\nu}} X_i^{d_{j_{\nu}}} \in \F(d_{j_{\nu}})(U_i \cap U_j)$ und damit aber bereits $s_{i_{\nu}} X_i^{d_{j_{\nu}}} \in \F(d_{j_{\nu}})(U_i \cup U_j)$. Für
$$d_i := \max_{j\in \{0, \ldots, n \}} \max _{\nu \in \{1, \ldots k_i\}} d_{j_{\nu}}$$
sind alle $s_{i_{\nu}}X_i^{d_i}$ globale Schnitte. Verfahren so für alle $0 \leqslant i \leqslant n$, es folgt dann also die Behauptung. $\hfill \Box$
 
 \end{pr}

\end{proposition}





\begin{theorem}   %%Satz 15.7
 Sei $R$ ein noetherscher Ring, $X \subseteq \Pro^{n}_R$ ein abgeschlossenes Unterschema. Dann ist  $H^{i}(X, \F)$ für jede kohärente Garbe $\F$ auf $X$ ein endlich erzeugter $R$-Modul für alle $i \geqslant 0$. 
 
 \begin{pr}
 Ohne Einschränkung gelte $X= \Pro^{n}_R$, denn für die abgeschlossene Einbettung $j:X \hookrightarrow \Pro^{n}_R$ ist $j_*\F$ eine kohärente $\mathcal{O}_X$-Modulgarbe auf $\Pro^n_R$ und nach Bemerkung 15.4 $H^{i}(X, \F) = H^{i}(\Pro^{n}_R, j_* \F)$ für alle $i \geqslant 0$. 
 
 \begin{compactenum}
 \item[\textbf{Beh. (a)}] Es gibt einen Epimorphismus von Garben 
 $$\Phi: \bigoplus_{i=1}^r \mathcal{O}_X(d_i) \la \F.$$
 
 \end{compactenum}
 
Dann gibt es mit $\mathcal{K} = \kernel \Phi$ eine kurze Exakte Sequenz 

$$0 \la \mathcal{K} \la \bigoplus_{i=1}^r \mathcal{O}_X(d_i) \la \F \la 0.$$
Beachte hierbei, dass $\mathcal{K}$ ebenfalls kohärent ist, denn der Kern eines Homomorphismus von $R$-Moduln ist ein $R$-Untermodul, womit $\mathcal{K}$ quasikohärent ist. Da $R$ noethersch ist, ist jeder $R$-Untermodul eines endlich erzeugten $R$-Moduls endlich erzeugt, $\mathcal{K}$ ist also kohärent. Weiter sieht man (ähnlich wie im Beweis von Satz 12.16 (iii) ), dass 
$$H^{i}\left( X, \bigoplus_{j=1}^r \mathcal{O}_X(d_j)\right) = \bigoplus_{j=1}^r H^{i}(X, \mathcal{O}_X(d_j)).$$ Aus der langen exakten Kohomologiesequenz
$$0 \rightarrow \mathcal{K}(X) \rightarrow \bigoplus_{j=1}^r \Gamma(X, \mathcal{O}_X(d_j)) \rightarrow \F(X) \rightarrow H^{1}(X, \mathcal{K}) \rightarrow \bigoplus_{j=1}^r H^{1}(X, \mathcal{O}_X(d_j)) \rightarrow H^{1}(X, \F) \rightarrow \ldots $$
 
 folgt nun die Behauptung, denn für alle $i \geqslant 0$ ist $\bigoplus_{j=1}^r H^{i}(X, \mathcal{O}_X(d_j))$ nach Satz 15.5 endlich erzeugt, damit auch $H^{1}(X, \mathcal{K})$ und also auch $H^{1}(X, \F)$. Iterativ folgt der Satz. Es bleibt noch die Behauptung zu zeigen.
 
 \begin{compactenum}
 \item[\textbf{Bew. (a)}] Nach Proposition 15.6 gibt es einen Epimorphismus 
 $$\Psi: \bigoplus_{i=1}^r \mathcal{O}_X \la \F(d) = \F \otimes_{\mathcal{O}_X} \mathcal{O}_X(d), \qquad e_i \mapsto s_i.$$
 Tensorieren mit $\mathcal{O}_X(-d)$ liefert (wegen der Rechtsexaktheit des Tensorprodukts) einen Epimorphismus
 $$\Phi: \bigoplus_{i=1}^r \mathcal{O}_X(-d)= \left(\bigoplus_{i=1}^r \mathcal{O}_X\right) \otimes_{\mathcal{O}_X} \mathcal{O}_X(-d) \la \F(d) \otimes_{\mathcal{O}_X} \mathcal{O}_X(-d) = \F \otimes_{\mathcal{O}_X} \mathcal{O}_X \cong \F,$$
 was zu zeigen war.   $\hfill \Box$
 
 \end{compactenum}
 
 
 

 \end{pr}

\end{theorem}











\renewcommand*\thesection{§ \arabic{section}\quad}
\section{Der Satz von Riemann-Roch für Kurven} %PARAGRAPH 16
\renewcommand*\thesection{\arabic{section}}





Sei $X$ eine nichtsinguläre, projektive Kurve über einem algebraisch abgeschlossenen Körper $k$. Dann haben wir bereits gesehen, dass jeder Divisor $D= \sum_{P \in X} n_p P$ auf $X$ eine invertierbare Garbe $\mathcal{L}(D)$ via
$$\mathcal{L}(D)(U) = \left\{ f \in k(X)^{\times} \ \vert \ \left( \mathrm{div} \hspace{1pt} f + D \right)\vert_U \geqslant 0 \right\} \cup \{0\}.$$
Für die globalen Schnitte gilt mit der Notation aus der algebraischen Geometrie
$$L(D):= \Gamma(X, \mathcal{L}(D))= H^{0}(X, \mathcal{L}(D))$$
sowie 
$$l(D):= \dim_k L(D) = \dim_k H^{0}(X, \mathcal{L}(D)).$$
Sei weiter $\Omega_X$ die Garbe der regulären Differentiale auf $X$
$$\Omega_X(U)= \left\{ \omega \in \Omega_{k(X)/k} \ \vert \ \mathrm{div}\hspace{1pt} \omega\vert_U \geqslant 0 \right\},$$
wobei $\mathrm{ord}_P \omega = \mathrm{ord}_P f$, falls $\omega = f \mathrm{d}t_P$ für ein uniformisierendes Element $t_P \in \mathcal{O}_{X,P}$ und damit $\mathrm{div}\hspace{1pt} \omega\vert_U := \sum_{P \in U} \mathrm{ord}_P \omega P$. Für $\omega_0 \in \Omega_{k(X)/k} \setminus \{0\}$ heißt $K:=K_{\omega_0} = \mathrm{div} \hspace{1pt} \omega_0$ \textit{kanonischer Divisor}. Dass $K$ dabei unabhängig von der Wahl von $\omega_0$ ist, sieht man leicht ein: Ist $0 \neq \omega_1  \in \Omega_{k(X)/k}$ ein weiteres Differential, so ist $\omega_1 = f \hspace{1pt} \omega_0$ für ein $f \in k(X)^{\times}$, also $\mathrm{div} \hspace{1pt} \omega_1 = \mathrm{div}\hspace{1pt} \omega_0 + \mathrm{div} \hspace{1pt} f$, also $\mathcal{L}( \mathrm{div}\hspace{1pt} \omega_1) \cong \mathcal{L}(K) = \Omega_X\cong \mathcal{L}(\mathrm{div}\hspace{1pt} \omega_0)$.



\begin{theorem}[Satz von Riemann-Roch aus der algebraischen Geometrie]
Sei $X$ eine nichtsinguläre, projektive Kurve über einem algebraisch abgeschlossenen Körper $k$ von Geschlecht $g$. Dann gilt für jeden Divisor $D$ auf $X$
$$l(D) - l(K-D) = \deg D + 1 -g,$$
also
$$\dim_k H^{0}(X, \mathcal{L}(D)) - \dim_k H^{0}(X, \Omega_X \otimes_{\mathcal{O}_X} \mathcal{L}(D)^{-1}) = \deg D +1 -g.$$
\begin{pr}
Siehe zum Beispiel Hartshorne IV.1.3. $\hfill \Box$
\end{pr}

\end{theorem}


\begin{defin}
Sei $\lokring$ ein noethersches Schema und $\F$ eine kohärente Garbe auf $X$. Dann definieren wir die \textit{Eulercharakteristik} von $\F$ auf $X$ als
$$\chi(\mathcal{F}) := \sum_{k=0}^{\infty} (-1)^k \dim_k H^{k}(X, \F).$$

\end{defin}

\begin{theorem}
Sei $X$ eine nichtsinguläre, projektive Kurve über einem algebraisch abgeschlossenen Körper $k$ und $g= \dim_k H^{1}(X, \mathcal{O}_X)$. Für jeden Divisor $D$ auf $X$ gilt
$$\dim_k H^{0}(X, \mathcal{L}(D)) - \dim_k H^{1}(X, \mathcal{L}(D)) = \deg D +1 - g.$$

\begin{pr}
Mit $D=0$ erhalten wir mit 16.1 und $\mathcal{L}(0)= \mathcal{O}_X$ die Behauptung sofort. Ein beliebiger Divisor $D$ ist nun eine endliche Summe und geht damit aus endlich viele Additionen von Punkten zum Divisor $D=0$ hervor. Wir zeigen die Behauptung also durch Induktion über $n=\sum_{P \in X} \vert n_p \vert$, wobei der Induktionsanfang bereits geleistet ist. Sei nun $D$ ein Divisor und $P \in X$ beliebige. Wir zeigen, dass
$$\dim_k H^{0}(X, \mathcal{L}(D)) - \dim_k H^{1}(X, \mathcal{L}(D)) = \deg D +1 - g$$
genau dann gilt, wenn 
$$\dim_k H^{0}(X, \mathcal{L}(D+P)) - \dim_k H^{1}(C, \mathcal{L}(D+P)) = \deg( D+P) +1 - g$$
erfüllt ist. Nach Definition gilt $\mathcal{L}(D) \subseteq \mathcal{L}(D+P)$. Wir erhalten damit eine kurze exakte Sequenz 
$$0 \la \mathcal{L}(D) \la \mathcal{L}(D+P) \la \slant{\mathcal{L}(D+P)}{\mathcal{L}(D)} \la 0.$$
Betrachte die Halme der Faktorgarbe. Wegen $\left( \slant{\mathcal{L}(D+P)}{\mathcal{L}(D)}\right)_Q = \slant{\mathcal{L}(D+P)_Q}{\mathcal{L}(D)_Q}$ erhalten wir für $Q\neq P$
$$\left( \slant{\mathcal{L}(D+P)}{\mathcal{L}(D)}\right)_Q =0,$$
 $\slant{\mathcal{L}(D+P)}{\mathcal{L}(D)}$ ist also eine Wolkenkratzergarbe. Ist nun $f \in k(X)$, so gilt $f_P \in \mathcal{L}(D+P)$ genau dann, wenn $\mathrm{ord}_P f \geqslant -n_P -1$ und $f_P \in \mathcal{L}(D)$ genau dann, wenn $\mathrm{ord}_Pf \geqslant -n_P$. Insgesamt erhalten wir dadurch
 $$H^{0}\left(X, \slant{\mathcal{L}(D+P)}{\mathcal{L}(D)} \right) \cong k, \qquad H^{i}\left(X, \slant{\mathcal{L}(D+P)}{\mathcal{L}(D)} \right) =0 \ \ \textrm{für }i \geqslant 0$$
 Übungsaufgabe 13.3 liefert 
 $$\chi(\mathcal{L}(D+P)) = \chi(\mathcal{L}(D)) + \chi \left( \slant{\mathcal{L}(D+P)}{\mathcal{L}(D)}\right) = \chi(\mathcal{L}(D)) + 1,$$
 woraus zusammen mit $\deg D+P = \deg D + 1$ die Behauptung folgt. $\hfill \Box$


\end{pr}


\end{theorem}





\begin{folg}
Vergleicht man die Sätze 16.1 und 16.2, so folgt 
$$\dim_k H^{1}(X, \mathcal{L}(D)) = \dim_k H^{0}(X, \Omega_X \otimes_{\mathcal{O}_X} \mathcal{L}(D)^{-1}).$$
Dieser Zusammenhang wird als \textit{Serre-Dualität} bezeichnet. Wir werden später noch sehen, dass $H^{1}(X, \mathcal{L}(D))$ kanonisch isomorph zu $H^{0}(X, \Omega_X \otimes_{\mathcal{O}_X} \mathcal{L}(D)^{-1})$ ist.
\end{folg}

\begin{remark}

Mit $P\neq Q$ erhalten wir durch $\{X\setminus \{P\}, X\setminus \{Q\} \}$ eine affine Überdeckung durch zwei offene Mengen. Da $\mathcal{L}(D)$ quasikohärent ist, gilt also 
$H^{k}(X, \mathcal{L}(D)) = 0$
für $k \geqslant 2$. Der linke Term von Satz 16.2 ist also gerade die \textit{Eulercharakteristik}
$$\chi(\mathcal{L}(D)) := \sum_{k=0}^{\infty} (-1)^k \dim_k H^{k}(X, \mathcal{L}(D))$$
von $\mathcal{L}(D)$. 

\end{remark}


\begin{ex}
Nach 16.4 folgt für $D=K$
$$\dim_k H^{1}(X, \LL(K))= \dim_k H^{1}(X, \Omega_X) = \dim_k H^{0}(X, \Omega_X \otimes_{\mathcal{O}_X} \Omega_X^{-1}) = \dim_k H^{0}(X, \mathcal{O}_X) = 1.$$
Was genau aber ist $H^{1}(X, \LL(K))$? Wir berechnen die Dimension im Folgenden per Hand:
Seien $P_1 \neq P_2 \in X$ Punkte und wähle eine affine Überdeckung $\mathfrak{U}=\{U_1, U_2\}$ von $X$, wobei $U_i = X \setminus \{P_i\}$ für $i \in\{1,2\}$. Dann gilt 
$$H^{1}(X, \Omega_X) = \check{H}^{1}(\mathfrak{U}, \Omega_X) = \slant{\check{C}^{1}(\mathfrak{U}, \Omega_X)}{\bild d^{0}} = \slant{\Omega_X(U_1 \cap U_2)}{(\Omega_X(U_1) - \Omega_X(U_2))}.$$
Wir behaupten nun:
\begin{compactenum}
\item Es gibt kein $\omega \in \Omega_{k(X)/k}$ mit $\mathrm{ord}_{P_1} \omega = -1$ und $\mathrm{ord}_P\omega = 0$ für alle $P \in X \setminus \{P_1\}$. 
\item Es gibt für alle $n \geqslant 2$ ein $\omega_n \in \Omega_{k(X)/k}$ mit $\mathrm{ord}_{P_1} \omega_n = -n$ und $\mathrm{ord}_P \omega_n = 0$ für alle $P \in X \setminus \{P_1\}$. 
\item Es gibt ein $\omega_0 \in \Omega_{k(X)/k}$ mit $\mathrm{ord}_{P_1} \omega_0 = \mathrm{ord}_{P_2}\omega_0 = -1$ und $\mathrm{ord}_P \omega_0 = 0$ für alle $P \in X \setminus \{P_1, P_2\}$.
\end{compactenum}
Dann folgt aus den Behauptungen: $\dim_k \left( \slant{\Omega_X(U_1 \cap U_2)}{\Omega_X(U_1) - \Omega_X(U_2)} \right) = 1$, denn: (folgt). Wir zeigen noch die Behauptungen.
\begin{pr}
\begin{compactenum}
\item  Ein solches $\omega$ wäre in $H^{0}(X, \Omega_X(P_1))$, wobei wir 
$\Omega_X(P_1) := \Omega_X \otimes_{\mathcal{O}_X} \LL(P_1)$ setzen. Nach Riemann-Roch ist 
$$\dim_k H^{0}(X, \Omega_X(P_1)) = \dim_k H^{0}(X, \LL(P_1)) - \deg (-P_1) -1 + g = g,$$
denn es gibt keine Funktionen auf $X$ mit mindestens einer Nullstelle und ohne Polstellen. Wegen $\dim_k H^{0}(X, \mathcal{O}_X)=g$ folgt 
$$\omega \in H^{0}(X, \Omega_X(P_1)) \quad \Longleftrightarrow \quad \omega \in H^{0}(X; \mathcal{O}_X),$$
also $\omega \notin H^{0}(X, \Omega_X(P_1))$, ein Widerspruch.
\item Wir verfahren völlig analog. Riemann-Roch liefert
$$\dim_k H^{0}(X, \Omega_X(nP_1)) = \dim_k H^{0}(X, \LL(-nP_1)) - \deg nP_1 -1 + g = n-1 + g.$$
Für $n \geqslant 2$ ist also 
$$\dim_k H^{0}(X, \Omega_X(nP_1)) = \dim_k H^{0}(X, \Omega_X((n-1)P_1)) + 1,$$
woraus die Existenz des gewünschten Differentials folgt.
\item Ebenso erhalten wir
$$\dim_k H^{0}(X, \Omega_X(P_1+P_2)) = \dim_k H^{0}(X, \LL(-P_1-P_2)) - \deg (-P_1-P_2) - 1 + g = 1 + g.$$
Damit existiert ein $\omega_0 \in H^{0}(X; \Omega_X(P_1+P_2)) \setminus H^{0}(X, \mathcal{O}_X)$ und wegen (i) gilt bereits 
$\mathrm{ord}_{P_1} \omega_0 + \mathrm{ord}_{P_2} \omega_0 = -1$. $\hfill \Box$

\end{compactenum}
\end{pr}


\end{ex}


\begin{ex}
Werden wir nun konkreter und betrachten $X= \Pro^1_k$. Wir wollen nun zu gegebenen Punkte Differentiale finden, welche den Behauptungen aus Beispiel 13.6 nach existieren. Betrachte $\frac{\mathrm{d}z}{z} \in \Omega_{k(X)(k}$. Dann gilt 
$$\mathrm{ord}_0 \left(\frac{\mathrm{d}z}{z}\right) = -1, \qquad \mathrm{ord}_{\infty} \left(\frac{\mathrm{d}z}{z}\right) = -1.$$
Um zweiteres einzusehen, müssen wir das Differential eins uniformiserendes Elements des zugehörigen maximalen Ideals nehmen. Beachte an dieser Stelle, dass $\mathrm{d}z$ uniformisierend in jedem Punkt außer $\infty$ ist. Es gilt 
$$\frac{\mathrm{d}z}{z} = - \frac{z^2}{z} \mathrm{d}\left( \frac{1}{z}\right) = -z \mathrm{d}\left( \frac{1}{z}\right) = - \frac{1}{\frac{1}{z}} \mathrm{d}\left(\frac{1}{z}\right),$$
also gerade 
$$\mathrm{ord}_{\infty} \left(\frac{\mathrm{d}z}{z}\right) = -1.$$
Wählen wir nun allgemeine Punkte $a,b \in \Pro^{1}_k \setminus \{\infty\}$, so ist für
$$\omega_{a,b} = \frac{\mathrm{d}{z}}{(z-a)(z-b)}$$
offensichtlich $\mathrm{ord}_a \omega_{a,b} = -1 = \mathrm{ord}_b \omega_{a,b}$ und $\mathrm{ord}_P \omega_{a,b} = 0$ für alle $P \in \Pro^{1}_k \setminus \{a,b, \infty\}$. Was ist mit dem Punkt $\infty$? Wir schreiben wieder 
$$\omega_{a,b} = \frac{\mathrm{d}z}{(z-a)(z-b)} = - \frac{z^2}{(z-a)(z-b)} \mathrm{d}\left(\frac{1}{z}\right),$$
also auch $\mathrm{ord}_{\infty} \omega_{a,b}=0$. 

\end{ex}



\begin{definbem}
Sei $X$ eine nichtsinguläre, projektive Kurve über einem algebraisch abgeschlossenen Körper $k$, $\omega \in \Omega_{k(X)/k} \setminus \{0\}$ ein Differential sowie $P \in X$. Sei weiter $t_P$ ein uniformisierendes Element in $P$ und $f \in k(X)^{\times}$, sodass $\omega = f \mathrm{d}t_P$ und es gelte $\mathrm{ord}_P \omega = -n$ für ein $n \in \mathbb{N}$.
\begin{compactenum}
\item $f$ besitzt eine eindeutige Darstellung $f=a_{-n}t_P^{-n} + a_{-n-1} t_P^{-n-1} + \ldots + a_{-1} t_P^{-1} + f_0$ mit $f_0 \in \mathcal{O}_{X,P}$ und Körperelementen $a_i \in k$.
\item Wir definieren das \textit{Residuum} von $\omega$ in $P$ als 
$$\mathrm{res}_P \omega := a_{-1}.$$
Gilt $\mathrm{ord}_P \omega \geqslant 0$, so setzen wir $\mathrm{res}_P \omega :=0$.
\item Das Residuum hängt nicht von der Wahl von $t_P$ ab.
\end{compactenum}
\begin{pr}
\begin{compactenum}
\item Klar.
\item[(iii)] Wir zeigen die Aussage lediglich für $\mathrm{char}k \neq 2$.
\begin{compactenum}

\item[\textbf{Fall (a)}] Es gilt $\omega = \frac{\mathrm{d}t_P}{t_P}$, also $f= \frac{1}{t_P}$ und damit $\mathrm{res}_P \omega = 1$. Sei nun $z$ eine weitere Uniformisierende in $P$, es gelte also $z = t_Pu$ mit einer Einheit $u \in \mathcal{O}_{X,P}^{\times}$. Dann gilt $u=1 + t_P h$ mit $h \in \mathcal{O}_{X,P}$ und damit $z= t_P ( 1 + t_P h)$. Dann gilt 
$$\mathrm{d}z = \mathrm{d}t_P + \mathrm{d}(t_P^2h) = \mathrm{d}t_P + t_P^2 \mathrm{d}h + 2t_P \mathrm{d}h = \mathrm{d}t_P + t_P^2 h' \mathrm{d}t_P + 2 t_P h' \mathrm{d}t_P' =  \mathrm{d}t ( 1 + 2h't_P + t_P^2h).$$
Wir erhalten damit 
$$\frac{\mathrm{d}t_P}{t_P} = \frac{\mathrm{d}z}{1 + 2h't_P + t_P^2 h'} \frac{1 + t_Ph'}{z} = \frac{\mathrm{d}z}{z} \frac{1 + t_Ph}{1 + 2h't_P + t_P^2h'}.$$
Wegen $t_P(P)=0$ gilt $$\frac{1+t_Ph'}{1 + 2h't_P + t_P^2h'} \in \mathcal{O}_{X,P}^{\times}$$ und damit
$$\mathrm{res}_P \frac{\mathrm{d}z}{z} = 1 = \mathrm{res}_P \frac{\mathrm{d}t_P}{t_P},$$
was zu zeigen war.
\item[\textbf{Fall (b)}] Man kann leicht die Identität
$$\frac{\mathrm{d}t_P}{t_P^n} = - \frac{1}{n-1} \mathrm{d} \left(\frac{1}{t_P^{n-1}}\right)$$
für beliebiges $n \in \mathbb{N}$ verifizieren. Ist $z$ nun ein beliebiges uniformisierendes Element in $P$, so schreibe
$$\frac{1}{t_P^{n-1}} = \frac{1}{z^{n-1}} + b_{n-2} \frac{1}{z^{n-2}} + \ldots + b_1 \frac{1}{z} + c_0$$
mit $c_0 \in \mathcal{O}_{X,P}^{\times}$. Dann gilt 
$$\mathrm{d} \left( \frac{1}{t_P^{n-1}}\right) = \sum_{i=1}^{n-1} b_i \mathrm{d} \left(\frac{1}{z^{i}} \right) + \mathrm{d}c_0 = - \sum_{i=1}^{n-1} b_i i \frac{\mathrm{d}z}{z^{i+1}} + c_0' \mathrm{d}z.$$
In dieser Darstellung fehlt der $\frac{1}{z}$-Term, das heißt das Residuum bleibt unverändert. $\hfill \Box$

\end{compactenum}

\end{compactenum}
\end{pr}


\end{definbem}

\begin{remark}
Ist $X= \Pro^{1}_k$ und $\omega = \frac{\mathrm{d}z}{z}$, so gilt $\mathrm{ord}_0 \omega = 1$ und $\mathrm{ord}_{\infty} \omega =1$. Allgemeiner haben wir gesehen:
$$\mathrm{res}_a \left( \frac{\mathrm{d}z}{z-a} - \frac{\mathrm{d}z}{z-b} \right) = 1, \qquad \mathrm{res}_b \left( \frac{\mathrm{d}z}{z-a} - \frac{\mathrm{d}z}{z-b} \right) = -1.$$
\end{remark}




\begin{folg}
Die Konstruktion aus 16.8 liefert einen wohldefinierten Isomorphismus
$$\mathrm{Res}: H^{1}(X, \Omega_X) \la k$$
wie folgt: Für $\omega \in H^{1}(X, \Omega_X)$ wähle $P_1, P_2 \in X$ und $\omega_0 \in \Omega_X(X \setminus \{P_1, P_2\})$ mit $\mathrm{ord}_{P_i} \omega_0 = -1$ für $i=1,2$. Setze dann $\mathrm{Res}(\omega) := \mathrm{res}_{P_1} \omega_0$.

\begin{pr}[\textit{Beweisskizze.}]  Sei zunächst $X= \Pro^{1}_k$. Dann ist $\mathrm{res}_{P_2} \omega_0 = - \mathrm{res}_{P_1} \omega_0$ nach der vorangegangenen Bemerkung. Vertauschen von $P_1$ und $P_2$ liefert $-\omega_0$. Weiter zeigt das Beispiel, dass $\mathrm{res}_{P_1} \omega_0$ unabhängig von der Wahl von $P_1 \neq P_2$ ist. Diese Beobachtungen liefern die Wohldefiniertheit von $\mathrm{Res}$ im Fall $X= \Pro^{1}_k$. Ist nun $X$ beliebig, wähle $f \in k(X)$ mit $\mathrm{ord}_{P_i} \mathrm{d}f = 0$ für $i=1,2$, es soll also $f: X \la \Pro^{1}_k$ unverzeigt in den $P_i$ sein. Dann induziert $f$ eine \textit{Spurabbildung} $\mathrm{tr}: f_{*} \Omega_X \la \Omega_{\Pro^{1}_k}$ mit $\mathrm{res}_{P_1} \omega_0 = \mathrm{res}_{f(P_1)}( \mathrm{tr}(\omega_0))$. Damit lässt sich dieser Fall auf $X= \Pro^{1}_k$ zurückführen.

\end{pr}
\end{folg}

\begin{proposition}[\textrm{\textit{Serre-Dualität}}]
Sei $X$ eine nichtsinguläre, projektive Kurve über einem algebraisch abgeschlossenen Körper $k$. Dann ist für jeden Divisor $D$ auf $X$ die Abbildung
$$\Phi: H^{0}(X, \mathcal{L}(D)) \times H^{1}(X, \Omega_X \otimes_{\mathcal{O}_X} \mathcal{L}(D)^{-1}) \la H^{1}(X, \Omega_X) \overset{\mathrm{Res}}{\cong} k, \qquad  (f,\omega) \mapsto f \omega$$
eine nichtausgeartete Bilinearform.

\end{proposition}








\end{spacing}

\end{document}

