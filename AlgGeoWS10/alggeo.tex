\documentclass[a4paper,12pt]{scrbook}

\usepackage{fontspec}
\usepackage{polyglossia}
\setdefaultlanguage{german}

\setmainfont[Mapping=tex-text,Numbers=OldStyle,Ligatures=Rare]{Linux Libertine O}
\setsansfont{Linux Biolinum O}

\usepackage{csquotes}
\usepackage{hyperref,amsmath,amsthm,amsfonts,amssymb,bbm,enumitem,aliascnt}

\usepackage{tikz}
\usetikzlibrary{matrix,arrows,calc}

\deffootnote[1em]{1em}{1em}{\textsuperscript{\thefootnotemark}\ }
\setlength{\parindent}{0pt}
\setlength{\parskip}{3pt}

\newtheorem{thmcnt}{Dummy}[section]
\newtheorem{satzcnt}{SatzDummy}
\newcommand{\definetheorem}[3]{
  \newaliascnt{#1}{#3}
  \newtheorem{#1}[#1]{#2}
  \aliascntresetthe{#1}
  \newtheorem*{#1*}{#2}
  \expandafter\def\csname #1autorefname\endcsname{\sc#2}
}

\newtheoremstyle{blah}{10pt}{3pt}{}{}{\sc}{:}{ }{}
\newtheoremstyle{foo}{10pt}{3pt}{\it}{}{\sc}{:}{ }{}
\newtheoremstyle{stz}{10pt}{3pt}{\it}{}{\bfseries}{:}{ }{}

\theoremstyle{blah}
% coole gabi-style umgebungen:
\newtheorem*{ziel}{Ziel}
\newtheorem*{motivation}{Motivation}
\newtheorem*{zeige}{Zeige}
%\newtheorem*{erinnerung}{Erinnerung}
\newtheorem*{q}{Frage}
%\newtheorem*{bsp*}{Beispiel}
%\newtheorem{dfn}{Definition}[section]
%\newtheorem{bsp}[dfn]{Beispiel}
\definetheorem{dfn}{Definition}{thmcnt}
\definetheorem{bsp}{Beispiel}{thmcnt}
\definetheorem{lem}{Lemma}{thmcnt}
\definetheorem{prop}{Proposition}{thmcnt}
\definetheorem{kor}{Korollar}{thmcnt}
%\theoremstyle{foo}
%\newtheorem{bem}[dfn]{Bemerkung}
%\newtheorem{db}[dfn]{Definition/Bemerkung}
\definetheorem{bem}{Bemerkung}{thmcnt}
\definetheorem{db}{Definition/Bemerkung}{thmcnt}
\definetheorem{de}{Definition/Erinnerung}{thmcnt}
\theoremstyle{stz}
%\newtheorem{satz}{Satz}
\definetheorem{satz}{Satz}{satzcnt}

% Warnungen :)
\usepackage{manfnt,environ}
\NewEnviron{w}{\begin{center}\begin{tikzpicture}%
\node[right,text width=0.85\textwidth,rounded corners,fill=orange,inner sep=1ex]{%
  \hbox{\hspace{1em}\makebox[\width][t]{\raisebox{1.3ex}{\lhdbend}}\hspace{2em}%
  \begin{minipage}[c]{.85\textwidth}\smallskip\BODY\medskip\end{minipage}%
}};\end{tikzpicture}\end{center}\par}

\usepackage{textcomp}
\newcommand{\sect}[1]{\begin{tikzpicture}%
\node[right,text width=\textwidth,rounded corners,fill=blue!15,inner sep=5pt]{%
  \hbox{%
  \begin{minipage}[c]{.96\textwidth}\section{#1}\end{minipage}%
  \hfill\makebox[\width][t]{\raisebox{-5pt}{\raisebox{2pt}{\textleaf}}}%
}};\end{tikzpicture}}

%\AtBeginEnvironment{ziel}{\begin{addmargin}[1cm]{0cm}\hspace*{-1cm}}
%\AtEndEnvironment{ziel}{\end{addmargin}}

\makeatletter
\renewcommand{\proofname}{Beweis}
\renewenvironment{proof}[1][\proofname]{\par
  \pushQED{\qed}%
  \normalfont \topsep6\p@\@plus6\p@\relax
  \trivlist
  \item[\hskip\labelsep
        \itshape
    #1\@addpunct{:}]\ignorespaces
}{%
  \popQED\endtrivlist\@endpefalse
}
\makeatother

\DeclareMathAlphabet{\mathpzc}{OT1}{pzc}{m}{it}

\def\C{\mathbb{C}}
\def\A{\mathbb{A}}
\def\V{\mathfrak{V}}
\def\I{\mathfrak{I}}
\newcommand{\D}{\mathpzc{D}}
\def\B{\mathpzc{B}}
\newcommand{\J}{\mathpzc{J}}
\newcommand{\AffVar}{\mathpzc{AffVar}}
\newcommand{\Grp}{\mathpzc{Grp}}
\newcommand{\Cat}{\mathpzc{Cat}}
\newcommand{\Set}{\mathpzc{Set}}
\newcommand{\Top}{\mathpzc{Top}}
\newcommand{\Ob}{\mathpzc{Ob}}
\newcommand{\Ar}{\mathpzc{Ar}}
\newcommand{\F}{\mathpzc{F}}
\def\G{\mathpzc{G}}
\renewcommand{\H}{\mathpzc{H}}
\renewcommand{\L}{\mathpzc{L}}
\def\T{\mathpzc{T}}
\def\M{\mathpzc{M}}
\newcommand{\Hom}{\mathrm{Hom}}
\newcommand{\Abb}{\mathrm{Abb}}
\newcommand{\Func}{\mathrm{Func}}
\newcommand{\Nat}{\mathrm{Nat}}
\newcommand{\id}{\mathrm{id}}
\newcommand{\dom}{\operatorname{dom}}
\newcommand{\cod}{\operatorname{cod}}
\newcommand{\Cone}{\mathrm{Cone}}
\newcommand{\op}{\mathrm{op}}
\newcommand{\Eq}{\mathrm{Eq}}
\newcommand{\ra}{\longrightarrow}
\renewcommand{\phi}{\varphi}
\newcommand{\ep}{\varepsilon}
\newcommand{\da}{:=}
\newcommand{\leer}{\ensuremath{\varnothing}}
\newcommand{\Quot}{\mathrm{Quot}}
\newcommand{\card}[1]{|#1|}

\newcommand{\set}[1]{\ensuremath{\mathbb{#1}}}
\newcommand{\Q}{\set{Q}}
\newcommand{\N}{\set{N}}
\newcommand{\R}{\set{R}}
\newcommand{\Z}{\set{Z}}

\newcommand{\smapsto}{\mapsto}
\renewcommand{\mapsto}{\longmapsto}
\newcommand{\oldbar}[1]{\bar{#1}}
\def\Bar#1{\ensuremath\overline{#1}}

\newcommand{\Quotient}[2]{
  \raisebox{0.7ex}{\ensuremath{#1}}
  \ensuremath{\mkern-3mu}\big/\ensuremath{\mkern-3mu}
  \raisebox{-0.6ex}{\ensuremath{#2}}}

%Listenformate:
\setenumerate[1]{leftmargin=*,labelindent=\parindent,label=(\alph*)}
%seperate listen im Beweise
\newlist{prooflist}{enumerate}{2}
\setlist[prooflist]{leftmargin=*,labelindent=\parindent,label=(\arabic*)}
\setlist[prooflist,2]{label=(\roman*)}

%Polynomringe
\newcommand{\polyx}[1][n]{\ensuremath%
  [X_{1},\dotsc,X_{#1}]}

% sectionformat:
\renewcommand{\thesection}{\arabic{section}}
\renewcommand{\thechapter}{\Roman{chapter}}

\begin{document}

% Titelseite kommt irgedwann noch....

% 18.10.10
%\setcounter{chapter}{-1}
\chapter*{Motivation}

\begin{ziel}
Untersuche Nullstellenmengen von Polynomen: Für eine Menge von Polynomen
\[p_{1},\dotsc,p_{r}\in k\polyx\]
über einem Körper $k$ möchte man die Menge der Nullstellen
\[\{x=(x_{1},\dotsc,x_{n})\mid p_{i}(x)=0\text{ für alle }i\}\]
analysieren.
\end{ziel}

\begin{bsp*}
\begin{enumerate}
\item Betrachte $ax^{2}+by^{2}=1\iff x^{2}+y^{2}-1=0$ über $k=\R$. Das liefert eine Ellipse, für $a=b=1$ einen Kreis.
%Bild...
\item Betrachte $x^{2}+y^{2}=z^{2}$.
%Bild...
\item Betrachte (b) mit $x=1$: Dann ist $1+y^{2}=z^{2}\iff 1=z^{2}-y^{2}$, also eine Hyperbel.
%Bild
\item Bei linearen Gleichungen sehen wir mit Hilfe der linearen Algebra, das wir affine Unterräume erhalten.
\item Die Lösungsmengen sind abhängig vom Körper, z.B. sehen wir für $k=\Z/3\Z$ hat das Polynom $X^{3}-X$ als Lösungsmenge ganz $k$.
\end{enumerate}
\end{bsp*}

% Rest kommt noch ;)

\chapter{Die Kategorie der affinen Varietäten}
\sect{Affine Varietäten und Verschwindungsideale}

Sei $k$ stets ein Körper.

\begin{dfn}
$V\subseteq k^{n}$ heißt \emph{affine Varietät}, wenn
$\F\subseteq k\polyx$
mit \[V=\V(\F):=\{x=(x_{1},\dotsc,x_{n})\in k^{n}\mid f(x)=0\;\forall f\in\F\}\] existiert.
\end{dfn}

\begin{bsp}\label{bsp1.2}
\begin{enumerate}
\item $k^{n}=\V(\{0\})$
\item $\leer=\V(\{1\})$
\end{enumerate}
\end{bsp}

\begin{q}Wie eindeutig ist das $\F$? Zum Beispiel liefern Produkte und Summen von Polynomen keine neuen Nullstellen:
\[V(x^{2}+y^{2}+z^{2}-1) \supseteq V(x^{2}+y^{2}+z^{2}-1, z-\frac{1}{2}) = V(x^{2}+y^{2}+z^{2}-1, z-\frac{1}{2}, x^{2}+y^{2}+z^{2}-1 + z-\frac{1}{2}).\]
\end{q}

\begin{bem} Seien $\F_{1},\F_{2}\subseteq k\polyx$. Dann gilt:
\begin{enumerate}
\item\label{1.1.3a} $\F_{1}\subseteq\F_{2}\implies \V(\F_{1})\supseteq \V(\F_{2})$
\item\label{1.1.3b} Sei $(\F)$ das von $\F$ erzeugte Ideal. Dann gilt: $\V(\F) = \V((\F))$.
\item\label{1.1.3c} Sei
\[\sqrt{(\F)}:=\{p\in k\polyx\mid\exists\, d\geq 1\text{ mit }p^{d}\in(\F)\}\]
das \emph{Radikalideal} von $(\F)$ ($\sqrt{(\F)}$ ist Ideal!). Dann gilt: $\V(\sqrt{(\F)})=\V(\F)$.
\item\label{1.1.3d} Zu jeder affinen Varietät $V\subseteq k^{n}$ gibt es endlich viele Polynome $f_{1},\dotsc,f_{r}$ mit \[V=\V(\{f_{1},\dotsc,f_{r}\})=:\V(f_{1},\dotsc,f_{r}).\]
\end{enumerate}
\end{bem}

\begin{proof}
\ref{1.1.3a} und \ref{1.1.3b} folgen direkt aus der Definition.
\begin{enumerate}%\addtocounter{enumi}{2}
\item[\ref{1.1.3c}] Offenbar ist $\sqrt{(\F)}\supseteq(\F)$, also gilt nach \ref{1.1.3a} und \ref{1.1.3b}: $\V(\sqrt{(\F)})\subseteq \V((\F))=\V(\F)$.

Für die andere Richtung: Sei $x=(x_{1},\dotsc,x_{n})\in \V(\F)$ und $f\in\sqrt{(\F)}$. Nach Definition existiert $d\in\N$, so dass $f^{d}\in(\F)$. Also gilt $f^{d}(x)=0$ und damit $f(x)=0$, demnach ist $x\in \V(\sqrt{(\F)})$.
\item[\ref{1.1.3d}] Nach \ref{1.1.3b} gilt $\V(\F)=\V((\F))$ und nach Hilberts Basissatz wird $(\F)$ von endlich vielen Elementen erzeugt.\qedhere
\end{enumerate}
\end{proof}

\begin{db}
Sei $V\subseteq k^{n}$. Wir nennen
\[\I(V):=\{f\in k\polyx\mid f(x)=0\;\forall x\in V\}\]
das \emph{Verschwindungsideal von $V$}. Es ist ein Ideal.
\end{db}

\begin{bem} Es gilt:
\begin{enumerate}
\item\label{1.1.5a} Seien $V_{1},V_{2}\subseteq k^{n}$ mit $V_{1}\subseteq V_{2}$. Dann ist $\I(V_{1})\supseteq \I(V_{2})$.
\item\label{1.1.5b} Sei $V\subseteq k^{n}$. Dann ist $\I(V)$ ein Radikalideal.
\item\label{1.1.5c} Sei $V$ eine affine Varietät. Dann gilt $V=\V(\I(V))$.

Es gilt sogar: $\I(V)$ ist das größte Ideal mit dieser Eigenschaft, d.h. für $J\subseteq k\polyx$ mit $\V(J)=V$ folgt schon $J\subseteq \I(V)$.
\item\label{1.1.5d} Seien $V_{1},V_{2}$ affine Varietäten. Dann gilt:
\[V_{1}=V_{2}\iff \I(V_{1})=\I(V_{2})\text{ und } V_{1}\subseteq V_{2}\iff \I(V_{1})\supseteq \I(V_{2}).\]
\end{enumerate}
\end{bem}

\begin{proof} Die Aussagen \ref{1.1.5a} und \ref{1.1.5b} folgen sofort aus der Definition.
\begin{enumerate}
\item[\ref{1.1.5c}] Sei zuerst $x\in V$. Dann gilt für alle $f\in \I(V)$: $f(x)=0$, also gilt $x\in \V(\I(V))$.

Sei nun $V$ eine affine Varietät. Dann gilt $V=\V(\F)$ für eine geeignete Menge $\F$. Es gilt $\F\subseteq \I(V)$, also ist
\[V=\V(\F)\supseteq \V(\I(V)).\]

Der Rest der Behauptung folgt aus der Definition des Verschwindungsideals.
\item[\ref{1.1.5d}] Die eine Richtung ist klar, bzw. folgt aus \ref{1.1.5a}. Für die andere Richtung überlegt man sich, dass nach \ref{1.1.5c} $V_{1}=\V(\I(V_{1}))=\V(\I(V_{2}))=V_{2}$ gilt.  Die Aussage für die Inklusionen folgt analog.\qedhere
\end{enumerate}
\end{proof}

\begin{q}
Wir haben gesehen, dass die Zuordnung
\[V\mapsto \I(V)\]
injektiv ist. Ist sie auch surjektiv?
\end{q}

\sect{Zariski-Topologie}

\begin{db}\label{1.2.1}Sei $n\in\N$. Die affinen Varietäten im $k^{n}$ bilden die abgeschlossenen Mengen einer Topologie auf dem $k^{n}$. Diese heißt \emph{Zariski-Topologie}.\end{db}
\begin{proof}\begin{prooflist}
\item $\leer$ und $k^{n}$ sind affine Varietäten nach \autoref{bsp1.2}.
\item Seien $V_{1},V_{2}$ affine Varietäten, d.h. $V_{1}=\V(I_{1})$ und $V_{2}=\V(I_{2})$, wobei $I_{1},I_{2}$ Ideale in $k\polyx$ sind. Dann gilt:
\[V_{1}\cap V_{2}=\V(I_{1}\cup I_{2}) = \V(I_{1}+I_{2}).\]
Das gleiche Argument funktioniert für beliebige Familien $V_{\lambda}$ mit $\lambda\in\Lambda$ und $\Lambda$ Indexmenge, d.h. $\displaystyle\bigcap_{\lambda\in\Lambda}V_{\lambda}$ ist wieder eine affine Varietät.
% 20.10.2010
\item Seien $V_{1},V_{2}$ affine Varietäten mit $I_{1}=\I(V_{1})$ und $I_{2}=\I(V_{2})$.
\begin{zeige} $\V(I_{1}\cdot I_{2})\subseteq V_{1}\cup V_{2}\subseteq \V(I_{1}\cap I_{2})\subseteq \V(I_{1}\cdot I_{2})$.
\begin{prooflist}
\item $V_{1}\cup V_{2}\subseteq \V(I_{1}\cap I_{2})$ ist nach Definition klar.
\item Sei $x\in \V(I_{1}\cdot I_{2})$. Angenommen $x\notin V_{1}$. Dann existiert $g\in I_{1}$ mit $g(x)\neq 0$. Sei $f\in I_{2}$. Dann ist $f\cdot g\in I_{1}\cdot I_{2}$, also ist $f\cdot g(x) = f(x)\cdot g(x) = 0$ und da $g(x)\neq 0$ ist $f(x)=0$, also $x\in \V(I_{2})=V_{2}$.
\end{prooflist}\end{zeige}
\end{prooflist}\end{proof}

\begin{dfn}\label{1.2.2}
Schreibe $\A^{n}(k)$ für $k^{n}$ mit der Zariski-Topologie.
\end{dfn}

\begin{bem}\label{1.2.3}
Sei $M\subseteq\A^{n}(k)$. Der Abschluss $\Bar{M}$ von $M$ bezüglich der Zariski-Topologie ist $\Bar{M}=\V(\I(M))$.
\end{bem}
\begin{proof}
$\Bar{M}\subseteq \V(\I(M))$ folgt sofort aus der Definition.

Für die andere Inklusion überlegt man sich: Sei $M\subseteq\V(J)$ eine abgeschlossene Obermenge von $M$. Dann ist $\I(M)\supseteq\I(\V(J))$ und damit
\[\V(\I(M))\subseteq\V(\I(\V(J)))=\V(J).\qedhere\]
\end{proof}

\begin{bsp}\label{1.2.4}
Sei $n=1$. Dann gilt: $X\subseteq\A^{1}(k)$ ist genau dann abgeschlossen, wenn $X$ eine endliche Teilmenge oder ganz $\A^{1}(k)$ ist.
\end{bsp}

\begin{w}\enquote{Kleine Umgebungen} sind riesig groß!\end{w}

\begin{bem}\label{1.2.5}
\begin{enumerate}
\item\label{1.2.5a} Wenn $k$ endlich ist, entspricht die Zariski-Topologie auf $\A^{1}(k)$ der diskreten Topologie.
\item\label{1.2.5b} Wenn $k$ unendlich ist, ist die Zariski-Topologie nicht hausdorffsch.
\end{enumerate}\end{bem}
\begin{proof}\begin{enumerate}
\item[\ref{1.2.5a}] Punkte sind abgeschlossen, denn sei $x=(x_{1},\dotsc,x_{k})\in k^{n}$, dann ist
\[{x}=\V(X_{1}-x_{1},\dotsc,X_{n}-x_{n}).\]
Damit sind auch endliche Vereinigungen von Punkten abgeschlossen und somit schon \emph{alle} Teilmengen von $\A^{1}(k)$.
\item[\ref{1.2.5b}] {\sc Erinnerung:} Ein topologischer Raum $X$ heißt hausdorffsch, wenn für alle $x,y\in X$ offene Umgebungen $U_{x}\ni x$ und $U_{y}\ni y$ mit $U_{x}\cap U_{y}=\leer$ existieren.

Für $n=1$ folgt die Behauptung also aus \autoref{1.2.4}.

Den Fall $n\geq 2$ führen wir zurück auf den Fall $n=1$:

Seien $x,y\in\A^{n}(k)$, $U_{x},U_{y}$ offene Umgebungen von $x$ bzw. $y$. Wir setzen
\[V_{1}:=\A^{n}(k)\setminus U_{x} = \V(I_{1})\text{ und }V_{2}:=\A^{n}(k)\setminus U_{y}=\V(I_{2})\]
für entsprechende Ideale $I_{1}$ und $I_{2}$. Ohne Einschränkung gebe es Polynome $f\in I_{1}$ und $g\in I_{2}$, in denen $X_{1}$ als Variable vorkommt (sonst

\begin{w}Hilfe!\end{w}

Sei $W:=\V(X_{2},\dotsc,X_{n})$. Dann besteht $V_{1}\cap W$ aus nur endlich vielen Punkten, da diese Nullstellen von $f(X_{1},0,\dotsc,0)$ sein müssen. Gleiches gilt für $V_{2}\cap W$. Also schneiden sich $U_{x}$ und $U_{y}$ sogar schon in $W$.\qedhere
\end{enumerate}\end{proof}

\begin{de}\label{1.2.6} Seien $X,X_{1},X_{2}$ topologische Räume.
\begin{enumerate}
\item Sei $Y\subseteq X$. Definiere auf $Y$ die \emph{Spurtopologie} durch
\[U\subseteq Y\text{ offen}\: :\Longleftrightarrow\:\exists\;V\subseteq X\text{ offen mit }U=V\cap Y.\]
\item Sei $X\times Y$ das kartesische Produkt (als Mengen) und seien
\begin{align*}p_{1}\colon X_{1}\times X_{2}\rightarrow X_{1},\quad(x_{1},x_{2})\mapsto x_{1},\\p_{2}\colon X_{1}\times X_{2}\rightarrow X_{2},\quad(x_{1},x_{2})\mapsto x_{2},\end{align*}
die zugehörigen Projektionen. Die \emph{Produkttopologie} auf $X_{1}\times X_{2}$ ist die gröbste Topologie (d.h. möglichst wenig offene Mengen), so dass $p_{1}$ und $p_{2}$ stetig sind.

D.h. $U\subseteq X_{1}\times X_{2}$ ist genau dann offen, wenn $U$ beliebige Vereinigung endlicher Schnitte von Urbildern offener Mengen in $X_{1}$ bzw. $X_{2}$ unter $p_{1}$ bzw. $p_{2}$ ist.

\item $X$ heißt \emph{reduzibel}, wenn es abgeschlossene echte Teilmengen $A,B\subsetneq X$ mit $X=A\cup B$ gibt.

\item $X$ heißt \emph{irreduzibel}, wenn $X$ nicht reduzibel ist.

\item Eine maximale irreduzible Teilmenge von $X$ heißt \emph{irreduzible Komponente}.
\end{enumerate}\end{de}

\begin{bsp}\label{1.2.7}
Sei $X$ hausdorffsch und $M\subseteq X$. Dann gilt:
\[M\text{ ist irreduzibel (bzgl. Spurtopologie)}\iff \card{M}\leq 1,\]
d.h. $M$ ist einelementig oder leer.

\emph{Denn}: Liegen $x\neq y$ in $M$, so finden wir (offene) Umgebungen $U_{x}$ und $U_{y}$ mit leerem Schnitt, können also
\[M=(M\setminus U_{x})\cup(M\setminus U_{y})\]
schreiben und sehen so, dass $M$ reduzibel ist. $\leer$ ist nie irreduzibel.
\end{bsp}

\begin{bsp}\label{1.2.8} Sei $k$ ein Körper mit unendlich vielen Elementen.
\begin{enumerate}
\item\label{1.2.8a} $\A^{1}(k)$ ist irreduzibel, da echte abgeschlossene Teilmengen endlich sind.
\item $\V(X\cdot Y) = \V(X)\cup\V(Y)$ ist reduzibel mit irreduziblen Komponenten $\V(X)$ und $\V(Y)$.
\end{enumerate}\end{bsp}
\ref{1.2.8a} und \autoref{1.2.8a}

\begin{bem}\label{1.2.9}
Sei $V\subseteq\A^{n}(k)$ eine affine Varietät. Dann gilt:
\[V\text{ ist irreduzibel}\iff\I(V)\text{ ist ein Primideal}.\]
\end{bem}
\begin{proof}
Sei zuerst $V$ irreduzibel. Seien $f,g\in k\polyx$ mit $f\cdot g\in\I(V)$. 

Angenommen $f\notin\I(V)$, dann gilt auch $\V(f)\not\supseteq\V(\I(V))=V$.

Andererseits gilt nach \autoref{1.2.1}: $\V(f)\cup\V(g)=\V(f\cdot g)\supseteq V$, also
\[V=(V\cap \V(f))\cup(V\cap\V(g)),\]
wobei $V\cap\V(f)$ und $V\cap\V(g)$ in $V$ abgeschlossen sind. Außerdem ist $V\neq V\cap\V(f)$, also gilt, da $V$ irreduzibel ist, $V=V\cap\V(g)$. Daraus folgt $V\subseteq\V(g)$ und damit ist $g\in\I(V)$ und $\I(V)$ ist somit ein Primideal.

Sei nun $\I(V)$ ein Primideal. Seien $V_{1},V_{2}$ Varietäten mit $V=V_{1}\cup V_{2}$ und $I_{1}:=\I(V_{1})$, $I_{2}:=\I(V_{2})$.

Angenommen $V\neq V_{1}$, d.h. $V\supsetneq V_{1}$, dann ist auch $\I(V)\subsetneq\I(V_{1})=I_{1}$.

Andererseits ist $V=V_{1}\cup V_{2}=\V(I_{1}\cdot I_{2})$, also ist $I_{1}\cdot I_{2}\subseteq\I(V)$. Das impliziert aber $I_{2}\subseteq\I(V)$, da $\I(V)$ ein Primideal ist und $I_{1}\not\subseteq\I(V)$. Daher gilt $V_{2}=\V(I_{2})\supseteq\V(\I(V))=V$, also ist schon $V=V_{2}$ und damit ist $V$ irreduzibel.
\end{proof}

\begin{prop}\label{1.2.10} Sei $X$ ein topologischer Raum. Dann ist jede irreduzible Teilmenge in einer irreduziblen Komponente enthalten.\end{prop}
% 25.10.2010
\begin{proof} Verwende das Lemma von Zorn:

{\sc Erinnerung:} Hat in einer halbgeordneten Menge $\M$ jede Kette (d.h. totalgeordnete Teilmenge) eine obere Schranke, dann hat $\M$ mindestens ein maximales Element.

Seien also $X'\subseteq X$ eine irreduzible Teilmenge, $\M:=\{Y\subseteq X\mid Y\text{ irreduzibel, }Y\supseteq X'\}$ und $\{Y_{\lambda}\}_{\lambda\in\Lambda}$ eine Familie aus $\M$, die totalgeordnet ist. Sei $\displaystyle Y:=\bigcup_{\lambda\in\Lambda}Y_{\lambda}$.

{\sc Zeige:} $Y\in\M$, d.h. $Y$ ist irreduzibel.

Wir nehmen an, es gäbe abgeschlossene Mengen $A,B\subsetneq X$, so dass $Y\cap A$ und $Y\cap B$ echte Teilmengen von $Y$ sind, für die $Y=(Y\cap A)\cup(Y\cap B)$ gilt. Insbesondere gilt dann:
\[(X\setminus A)\cap Y\neq\leer\neq (X\setminus B)\cap Y.\]
Folglich existieren ein $\lambda_{1}$ mit $(X\setminus A)\cap Y_{\lambda_{1}}\neq\leer$ und ein $\lambda_{2}$ mit $(X\setminus B)\cap Y_{\lambda_{2}}\neq\leer$. Da die $Y_{\lambda_{i}}$ Teil einer Kette sind, können wir ohne Einschränkung $Y_{\lambda_{1}}\subseteq Y_{\lambda_{2}}$ annehmen. Damit ist aber auch $(X\setminus A)\cap Y_{\lambda_{2}}\neq\leer$ und wir finden eine echte Zerlegung
\[Y_{\lambda_{2}}=(Y_{\lambda_{2}}\cap A)\cup(Y_{\lambda_{2}}\cap B),\]
was im Widerspruch zur Irreduzibilität von $Y_{\lambda_{2}}$ steht. Folglich hat jede Kette eine obere Schranke und nach dem Lemma von Zorn hat $\M$ somit ein maximales Element.
\end{proof}

\begin{satz}\label{satz1} Sei $V$ eine affine Varietät. Dann gilt:
\begin{enumerate}
\item\label{s1a} $V$ ist eine endliche Vereinigung von irreduziblen affinen Varietäten.
\item\label{s1b} $V$ hat nur endlich viele irreduzible Komponenten $V_{1},\dotsc,V_{r}$. Insbesondere ist die Zerlegung
\[V=V_{1}\cup\dotsm\cup V_{r}\]
eindeutig.\end{enumerate}\end{satz}
\begin{proof}\begin{enumerate}
\item[\ref{s1a}] Seien 
\begin{align*}&\B:=\{V\subseteq k^{n}\mid V\text{ ist affine Varietät und erfüllt \emph{nicht} \ref{s1a}}\},\\
&\J:=\{\I(V)\subseteq k\polyx\mid V\in\B\}.\end{align*}
Wir nehmen an, $\B$ wäre nicht leer. Dann ist auch $\J$ nicht leer. Da $k\polyx$ noethersch ist, finden wir in $\J$ ein maximales Element $I_{0}=\I(V_{0})$. Damit ist $V_{0}$ ein minimales Element in $\B$. Dann ist $V_{0}$ aber \emph{nicht} irreduzibel, also existieren affine Varietäten $V_{1},V_{2}\subsetneq V$ mit $V=V_{1}\cup V_{2}$. Insbesondere sind diese aber nicht in $\B$, da $V_{0}$ minimal gewählt war, lassen sich also als Vereinigung endlich vieler irreduzibler Varietäten schreiben. Dann geht das aber auch für $V_{0}$ und das ist ein Widerspruch, da $V_{0}\in\B$.
\item[\ref{s1b}] Mit Hilfe von \autoref{1.2.10} und dem \ref{s1a}-Teil sehen wir, dass wir \[V=V_{1}\cup\dotsm\cup V_{r}\] schreiben können, wobei die $V_{i}$ irreduzible Komponenten sind. Wir zeigen noch die Eindeutigkeit dieser Zerlegung: Sei $W$ eine irreduzible Komponente von $V$. Wir schreiben
\[W=(W\cap V_{1})\cup\dotsm\cup(W\cap V_{r})\]
und sehen, da $W$ irreduzibel ist, dass es ein $i$ mit $W=W\cap V_{i}$ gibt. Also gilt $W\subseteq V_{i}$ und damit schon $W=V_{i}$, da $W$ als irreduzible \emph{Komponente} eine \emph{maximale} irreduzible Teilmenge von $V$ ist.\qedhere
\end{enumerate}\end{proof}

\sect{Der Hilbertsche Nullstellensatz}

\begin{motivation}
  Bisher haben wir die Mengen
  \[ \mathpzc{V}_n = \{ V\subseteq k^n \mid V \text{ affine Varietät }\} \quad\text{und}\quad
     \mathpzc{J}_n = \{ I\subseteq k\polyx \mid I \text{ Radikalideal }\} \]
  und zwischen ihnen die Abbildungen
  \begin{align*}
    \V\colon& \mathpzc{J}_n \ra \mathpzc{V}_n, \quad I\mapsto\V(I)\\
    \I\colon& \mathpzc{V}_n \ra \mathpzc{J}_m, \quad V\mapsto\I(V)
  \end{align*}
  betrachtet. Wir haben gesehen, dass $\V\circ\I=\id$ gilt. Gilt auch $\I\circ\V=\id$?
\end{motivation}

\begin{bsp*}
  Wenn $k$ nicht algebraisch abgeschlossen ist, muss das nicht gelten: sei $k=\R$ und $I=(X^2+1)$. Dann ist $\V(I)=\leer$, aber
  $\I(\V(I)) = k\polyx$.
\end{bsp*}

Wir werden sehen, dass $\I\circ\V=\id$ gilt, falls $k$ algebraisch abgeschlossen ist. Die einzige \enquote{Obstruktion} ist,
dass $\V(I)=\leer$ gilt, obwohl $I\neq k\polyx$.

\begin{satz}[Hilbertscher Nullstellensatz]\label{satz2}\label{HNS}
  \begin{enumerate}
  \item {\bf Algebraische Form:}\label{satz2a} Sei $m$ ein maximales Ideal in $k\polyx$. Dann ist $\Quotient{k\polyx}{m}$ eine
    endliche algebraische Körpererweiterung von $k$.
  \item {\bf Schwacher Hilbertscher Nullstellensatz:}\label{satz2b} Ist $k$ algebraisch abgeschlossen und $I\subsetneq k\polyx$
    ein echtes Ideal, dann ist $\V(I)\neq\leer$.
  \item {\bf Starker Hilbertscher Nullstellensatz:}\label{satz2c} Ist $k$ algebraisch abgeschlossen und $I\subseteq k\polyx$ ein
    Ideal, dann ist $\I(\V(I))=\sqrt(I)$.
  \end{enumerate}
\end{satz}

Die Aussage wird in mehreren Schritten im Rest dieses Abschnitts bewiesen.

Wir bezeichnen $x_1=\Bar{X}_1,\dotsc,x_n=\Bar{X}_n$. Ohne Einschränkung können wir nach einer eventuellen Umsortierung
der Variablen annehmen, dass $x_1,\dotsc,x_k$ algebraisch unabhängig über $k$ sind und $x_{k+1},\dotsc,x_n$ algebraisch über
$k(x_1,\dotsc,x_k)$. Also:
\[ k \subseteq S = k(x_1,\dotsc,x_k) \subseteq L=\Quotient{k\polyx}{m}, \]
dabei ist $k(x_1,\dotsc,x_k)\cong\Quot(k\polyx[k])$ und $L$ ist endlich erzeugt als $S$-Modul.

\begin{lem}\label{1.3.1}
  Seien $R$, $S$, $T$ Ringe mit $R\subseteq S\subseteq T$, sodass gilt:
  \begin{itemize}
  \item $R$ ist noethersch
  \item $T$ ist endlich erzeugt als $R$-Algebra; seien $x_1,\dotsc,x_n$ solche Erzeuger
  \item $T$ ist endlich erzeugt als $S$-Modul; seien $w_1,\dotsc,w_m$ solche Erzeuger
  \end{itemize}
  Dann ist $S$ als $R$-Algebra endlich erzeugt.
\end{lem}
\begin{proof}
  Wir schreiben 
  \[ x_i=\displaystyle\sum_{j=1}^m a_{ij}w_j \text{ mit } a_{ij}\in S, \qquad
     w_iw_j=\displaystyle\sum_{k=1}^m b_{ijk}w_k \text { mit } b_{ijk}\in S. \]
  Sei $S_0$ die $R$-Unteralgebra von $S$, die von allen $a_{ij}$ und $b_{ijk}$ erzeugt wird. Nach dem Hilbertschen Basissatz ist
  $S_0$ ein noetherscher Ring. $T$ wird von den $w_i$ auch als $S_0$-Modul erzeugt und ist noethersch als $S_0$-Modul, da er ein
  endlich erzeugter Modul über einem noetherschen Ring ist. $S$ ist ein $S_0$-Untermodul von $T$. Damit ist $S$ endlich erzeugt
  als $S_0$-Modul, also auch als $R$-Algebra.
\end{proof}

Insbesondere ist in der Situation im Beweis des Hilbertschen Nullstellensatzes (\autoref{HNS}) $k(x_1,\dotsc,x_k)$ als
$k$-Algebra endlich erzeugt.

\begin{lem}\label{1.3.2}
  Es sei $k$ ein Körper. Dann ist $k(X_1,\dotsc,X_r)$ nicht endlich erzeugt als $k$-Algebra für $r\ge1$.
\end{lem}
\begin{proof}
  Wir nehmen an, wir hätten endlich viele Erzeuger $h_1,\dotsc,h_k$ und schreiben $h_i=\frac{f_i}{g_i}$ mit $f_i,g_i\in
  k\polyx[r]$. Dann wählen wir ein Primpolynom $p$, das keines der $g_i$ teilt und schreiben
  \[ \frac1p = \sum_{i=1}^N a_i h_{i_1}\dotsm h_{i_{l_i}} \quad (a_i\in k,\ i,j\in\{1,\dotsc,k\}). \]
  Sei $H=g_1\dotsm g_k$. Multipliziert man obige Gleichung mit $H$ durch, bekommt man $\frac{H}{p}\in k\polyx[r]$, also
  ist $p$ ein Teiler von $H$. Aber $p$ war teilerfremd zu allen $g_i$ gewählt. Das ist ein Widerspruch.
\end{proof}

\begin{prop}\label{1.3.3}
  Die algebraische Version des Hilbertschen Nullstellensatzes stimmt.
\end{prop}
\begin{proof}
  Das folgt nun direkt aus \autoref{1.3.1} und \autoref{1.3.2}.
\end{proof}

%27.10.10

Um die Implikation $I\subsetneq k\polyx \implies \V(I)\neq\leer$ zu zeigen, betrachten wir für einen Punkt $p=(x_1,\dotsc,x_n)\in
k^n$ den Einsetzungshomomorphismus \[ \phi_p\colon k\polyx \ra k, \quad f\mapsto f(p) \] (das ist ein
$k$-Algebrenhomomorphismus). Dieser steigt genau dann ab auf $A=\Quotient{k\polyx}{I}$, wenn $p\in\V(I)$ gilt. Außerdem ist jeder
$k$-Algebrenhomomorphismus $\phi\colon A\ra k$ von dieser Art: wähle $p=(\phi(\Bar{X_1}),\dotsc,\phi(\Bar{X_n}))$. Ist $I=m$ ein
maximales Ideal, dann ist $A=\Quotient{k\polyx}{m}$ eine endliche algebraische Körpererweiterung von $k$. In diesem Fall
existiert genau dann ein $k$-Algebrenhomomorphismus $A\ra k$, wenn $A=k$ gilt.

\begin{lem}\label{1.3.4}
  Aus der algebraischen Version \ref{satz2a} folgt der schwache Hilbertsche Nullstellensatz \ref{satz2b}.
\end{lem}
\begin{proof}
  Sei $m$ ein maximales Ideal mit $I\subseteq m$. Da $k$ algebraisch abgeschlossen ist, ist nach \ref{satz2a}
  $L=\Quotient{k\polyx}{m}\cong k$. Sei $\phi\colon L\ra K$ ein Isomorphismus und $p=(\phi(\Bar{X_1}),\dotsc,\phi(\Bar{X_n}))\in
  k^n$. Für $f\in m$ gilt dann $f(p)=\phi(f(\Bar{X}_1,\dotsc,\Bar{X}_n))=\phi(\Bar{f(X_1,\dotsc,X_n)}) = \phi(0)=0$. Also ist
  $p\in\V(I)$ und damit $\V(I)\neq\leer$.
\end{proof}

\begin{lem}[Schluss von Rabinowitsch]
  Aus \ref{satz2b} folgt \ref{satz2c}.
\end{lem}
\begin{proof}
Zu zeigen ist $\I(\V(I))=\sqrt{I}$. Die Inklusion \enquote{$\supseteq$} ist klar. Für die andere Inklusion nehmen wir uns ein
$g\in\I(\V(I))$ und zeigen: es gibt ein $d\ge1$ mit $g^d\in I$.

Wir betrachten $k^{n+1}$ und definieren $J=(I,gX_{n+1}-1)$. Dann ist $\V(J)=\leer$, denn wäre
$p=(x_1,\dotsc,x_{n+1})\in\V(J)$, dann wäre $(x_1,\dotsc,x_n)\in\V(I)$, also $g(x_1,\dotsc,x_n)=0$, aber damit wäre
$(gX_{n+1}-1)(p)=-1\neq0$, was ein Widerspruch ist.

Nach \ref{satz2b} muss dann also $J=k\polyx[n+1]$ gelten. Also gibt es $b_1,\dotsc,b_{n+1}\in k\polyx[n+1]$, sodass
\[ 1=b_1f_1 + \dotsm + b_nf_n + b_{n+1}(gX_{n+1}-1) \] gilt. Wir verwenden den $k$-Algebrenhomomorphismus
\[ \phi\colon k\polyx[n+1] \ra k(X_1,\dotsc,X_n),\quad X_i\mapsto X_i \text{ für } i\in\{1,\dotsc,n\},\ X_{n+1}\mapsto\frac1g \]
und bekommen
\[ 1 = \phi(1) = \phi(b_1)f_1 + \dotsm + \phi(b_n)f_n + 0. \]
Nun schreiben wir $\phi(b_i)=\frac{\tilde{b_i}}{g}a_i$ für $\tilde{b_i}\in k\polyx$. Durchmultiplizieren mit $g^d$ für genügend
großes $d$ ergibt dann $g^d\in I$.
\end{proof}

\begin{kor}\label{1.3.6}
  Ist $k$ algebraisch abgeschlossen, dann entsprechen die affinen Varietäten in $\A^n(k)$ bijektiv den Radikalidealen in
  $k\polyx$ via $V\mapsto\I(V)$.
\end{kor}

\sect{Morphismen zwischen affinen Varietäten}

\begin{ziel}In diesem Abschnitt sollen Morphismen definiert werden. Die Idee dabei ist, Abbildungen zu betrachten, die von Polynomen herkommen.
\end{ziel}
\begin{dfn}
\begin{enumerate}
\item\label{1.4.1a} Seien $V\subseteq k^{n}, W\subseteq k^{m}$ affine Varietäten. Eine Abbildung $f:V\mapsto W$ heißt Morphismus, wenn es Polynome $f_1,\dotsc,f_m \in \polyx$ mit 
\[f(p)=(f_1(p),\dotsc,f_m(p)) \in k^{m}\]
gibt.
\item\label{1.4.1b} Ein Morphismus $f:V\mapsto W$ heißt Isomorphismus, wenn es einen Morphismus $g:W\mapsto V$ mit $g\circ f=\id_V, f\circ g=\id_W$ gibt.
\item\label{1.4.1c} Gibt es einen Isomorphismus zwischen $V$ und $W$, so heißen $V$ und $W$ isomorph.
\end{enumerate}
\end{dfn}
\begin{bem}\label{1.4.2} Die affinen Varietäten über $k$ bilden zusammen mit den Morphismen eine Kategorie: $\AffVar_k$.
Dabei sind die Objekte gerade die affinen Varietäten und die Morphismen zwischen zwei affinen Varietäten sind wie in \ref{1.4.1a} 
gegeben. Man beachte, dass die Verkettung von Polynomen wieder ein Polynom ist, d.h. die Verkettung zweier Morphismen ist auch wieder ein Morphismus.
\end{bem}

\begin{bsp}
\begin{enumerate}
\item Projektoren bzw. Einbettungen 
\end{enumerate}
\end{bsp}

\end{document}
