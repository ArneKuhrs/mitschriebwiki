\documentclass[a4paper,12pt]{scrbook}

\usepackage{fontspec}
\usepackage{polyglossia}
\setdefaultlanguage{german}

\setmainfont[Mapping=tex-text,Numbers=OldStyle,Ligatures=Rare]{Linux Libertine}
\setsansfont{Linux Biolinum}

\usepackage{csquotes}
\usepackage{hyperref,amsmath,amsthm,amsfonts,bbm,enumitem}

\usepackage{tikz}
\usetikzlibrary{matrix,arrows,calc}

\deffootnote[1em]{1em}{1em}{\textsuperscript{\thefootnotemark}\ }
\setlength{\parindent}{0pt}
\setlength{\parskip}{3pt}

\newtheoremstyle{blah}{10pt}{3pt}{}{}{\sc}{:}{ }{}
\newtheoremstyle{foo}{10pt}{3pt}{\it}{}{\sc}{:}{ }{}
\newtheoremstyle{stz}{10pt}{3pt}{\it}{}{\bfseries}{:}{ }{}

\theoremstyle{blah}
% coole gabi-style umgebungen:
\newtheorem*{ziel}{Ziel}
\newtheorem*{zeige}{Zeige}
\newtheorem*{q}{Frage}
\newtheorem*{bsp*}{Beispiel}
\newtheorem{dfn}{Definition}[section]
\newtheorem{bsp}[dfn]{Beispiel}
%\theoremstyle{foo}
\newtheorem{bem}[dfn]{Bemerkung}
\newtheorem{db}[dfn]{Definition/Bemerkung}
\theoremstyle{stz}
\newtheorem{satz}{Satz}

% Warnungen :)
\usepackage{manfnt,environ}
\NewEnviron{w}{\begin{tikzpicture}%
\node[right,text width=\textwidth,rounded corners,fill=orange,inner sep=1ex]{%
  \hbox{\hspace{1em}\makebox[\width][t]{\raisebox{1.3ex}{\lhdbend}}\hspace{2em}%
  \begin{minipage}[c]{.85\textwidth}\smallskip\BODY\medskip\end{minipage}%
}};\end{tikzpicture}\par}

%\AtBeginEnvironment{ziel}{\begin{addmargin}[1cm]{0cm}\hspace*{-1cm}}
%\AtEndEnvironment{ziel}{\end{addmargin}}

\makeatletter
\renewcommand{\proofname}{Beweis}
\renewenvironment{proof}[1][\proofname]{\par
  \pushQED{\qed}%
  \normalfont \topsep6\p@\@plus6\p@\relax
  \trivlist
  \item[\hskip\labelsep
        \itshape
    #1\@addpunct{:}]\ignorespaces
}{%
  \popQED\endtrivlist\@endpefalse
}
\makeatother

\DeclareMathAlphabet{\mathpzc}{OT1}{pzc}{m}{it}

\renewcommand{\C}{\mathbb{C}}
\newcommand{\D}{\mathpzc{D}}
\newcommand{\Grp}{\mathpzc{Grp}}
\newcommand{\Cat}{\mathpzc{Cat}}
\newcommand{\Set}{\mathpzc{Set}}
\newcommand{\Top}{\mathpzc{Top}}
\newcommand{\Ob}{\mathpzc{Ob}}
\newcommand{\Ar}{\mathpzc{Ar}}
\newcommand{\F}{\mathpzc{F}}
\renewcommand{\G}{\mathpzc{G}}
\renewcommand{\H}{\mathpzc{H}}
\renewcommand{\L}{\mathpzc{L}}
\renewcommand{\T}{\mathpzc{T}}
\newcommand{\Hom}{\mathrm{Hom}}
\newcommand{\Abb}{\mathrm{Abb}}
\newcommand{\Func}{\mathrm{Func}}
\newcommand{\Nat}{\mathrm{Nat}}
\newcommand{\id}{\mathrm{id}}
\newcommand{\dom}{\operatorname{dom}}
\newcommand{\cod}{\operatorname{cod}}
\newcommand{\Cone}{\mathrm{Cone}}
\newcommand{\op}{\mathrm{op}}
\newcommand{\Eq}{\mathrm{Eq}}
\newcommand{\ra}{\longrightarrow}
\renewcommand{\phi}{\varphi}
\newcommand{\ep}{\varepsilon}
\newcommand{\da}{:=}
\newcommand{\leer}{\ensuremath{\emptyset}}

\newcommand{\set}[1]{\ensuremath{\mathbb{#1}}}
\newcommand{\Q}{\set{Q}}
\newcommand{\N}{\set{N}}
\newcommand{\R}{\set{R}}
\newcommand{\Z}{\set{Z}}

\newcommand{\smapsto}{\mapsto}
\renewcommand{\mapsto}{\longmapsto}

%Listenformate:
\setenumerate[1]{leftmargin=*,labelindent=\parindent,label=(\alph*)}
%seperate listen im Beweise
\newlist{prooflist}{enumerate}{2}
\setlist[prooflist]{leftmargin=*,labelindent=\parindent,label=(\arabic*)}
\setlist[prooflist,2]{label=(\roman*)}

%Polynomringe
\newcommand{\polyx}{\ensuremath%
  [X_{1},\dotsc,X_{n}]}

% sectionformat:
\renewcommand{\thesection}{\arabic{section}}
\renewcommand{\thechapter}{\Roman{chapter}}

\begin{document}

% Titelseite kommt irgedwann noch....

% 18.10.10
%\setcounter{chapter}{-1}
\chapter*{Motivation}

Hallo.
\begin{ziel}
Untersuche Nullstellenmengen von Polynomen: Für eine Menge von Polynomen
\[p_{1},\dotsc,p_{r}\in k\polyx\]
über einem Körper $k$ möchte man die Menge der Nullstellen
\[\{x=(x_{1},\dotsc,x_{n})\mid p_{i}(x)=0\text{ für alle }i\}\]
analysieren.
\end{ziel}

\begin{bsp*}
\begin{enumerate}
\item Betrachte $ax^{2}+by^{2}=1\iff x^{2}+y^{2}-1=0$ über $k=\R$. Das liefert eine Ellipse, für $a=b=1$ einen Kreis.
%Bild...
\item Betrachte $x^{2}+y^{2}=z^{2}$.
%Bild...
\item Betrachte (b) mit $x=1$: Dann ist $1+y^{2}=z^{2}\iff 1=z^{2}-y^{2}$, also eine Hyperbel.
%Bild
\item Bei linearen Gleichungen sehen wir mit Hilfe der linearen Algebra, das wir affine Unterräume erhalten.
\item Die Lösungsmengen sind abhängig vom Körper, z.B. sehen wir für $k=\Z/3\Z$ hat das Polynom $X^{3}-X$ als Lösungsmenge ganz $k$.
\end{enumerate}
\end{bsp*}

% Rest kommt noch ;)

\chapter{Die Kategorie der affinen Varietäten}
\section{Affine Varietäten und Verschwindungsideale}

Sei $k$ stets ein Körper.

\begin{dfn}
$V\subseteq k^{n}$ heißt \emph{affine Varietät}, wenn
$\F\subseteq k\polyx$
mit \[V=V(\F):=\{x=(x_{1},\dotsc,x_{n})\in k^{n}\mid f(x)=0\;\forall f\in\F\}\] existiert.
\end{dfn}

\begin{bsp}\label{bsp1.2}
\begin{enumerate}
\item $k^{n}=V(\{0\})$
\item $\leer=V(\{1\})$
\end{enumerate}
\end{bsp}

\begin{q}Wie eindeutig ist das $\F$? Zum Beispiel liefern Produkte und Summen von Polynomen keine neuen Nullstellen:
\[V(x^{2}+y^{2}+z^{2}-1) \supseteq V(x^{2}+y^{2}+z^{2}-1, z-\frac{1}{2}) = V(x^{2}+y^{2}+z^{2}-1, z-\frac{1}{2}, x^{2}+y^{2}+z^{2}-1 + z-\frac{1}{2}).\]
\end{q}

\begin{bem} Seien $\F_{1},\F_{2}\subseteq k\polyx$. Dann gilt:
\begin{enumerate}
\item\label{1.1.3a} $\F_{1}\subseteq\F_{2}\implies V(\F_{1})\supseteq V(\F_{2})$
\item\label{1.1.3b} Sei $(\F)$ das von $\F$ erzeugte Ideal. Dann gilt: $V(\F) = V((\F))$.
\item\label{1.1.3c} Sei
\[\sqrt{(\F)}:=\{p\in k\polyx\mid\exists\, d\geq 1\text{ mit }p^{d}\in(\F)\}\]
das \emph{Radikalideal} von $(\F)$ ($\sqrt{(\F)}$ ist Ideal!). Dann gilt: $V(\sqrt{(\F)})=V(\F)$.
\item\label{1.1.3d} Zu jeder affinen Varietät $V\subseteq k^{n}$ gibt es endlich viele Polynome $f_{1},\dotsc,f_{r}$ mit \[V=V(\{f_{1},\dotsc,f_{r}\})=:V(f_{1},\dotsc,f_{r}).\]
\end{enumerate}
\end{bem}

\begin{proof}
\ref{1.1.3a} und \ref{1.1.3b} folgen direkt aus der Definition.
\begin{enumerate}%\addtocounter{enumi}{2}
\item[\ref{1.1.3c}] Offenbar ist $\sqrt{(\F)}\supseteq(\F)$, also gilt nach \ref{1.1.3a} und \ref{1.1.3b}: $V(\sqrt{(\F)})\subseteq V((\F))=V(\F)$.

Für die andere Richtung: Sei $x=(x_{1},\dotsc,x_{n})\in V(\F)$ und $f\in\sqrt{(\F)}$. Nach Definition existiert $d\in\N$, so dass $f^{d}\in(\F)$. Also gilt $f^{d}(x)=0$ und damit $f(x)=0$, demnach ist $x\in V(\sqrt{(\F)})$.
\item[\ref{1.1.3d}] Nach \ref{1.1.3b} gilt $V(\F)=V((\F))$ und nach Hilberts Basissatz wird $(\F)$ von endlich vielen Elementen erzeugt.\qedhere
\end{enumerate}
\end{proof}

\begin{db}
Sei $V\subseteq k^{n}$. Wir nennen
\[I(V):=\{f\in k\polyx\mid f(x)=0\;\forall x\in V\}\]
das \emph{Verschwindungsideal von $V$}. Es ist ein Ideal.
\end{db}

\begin{bem} Es gilt:
\begin{enumerate}
\item\label{1.1.5a} Seien $V_{1},V_{2}\subseteq k^{n}$ mit $V_{1}\subseteq V_{2}$. Dann ist $I(V_{1})\supseteq I(V_{2})$.
\item\label{1.1.5b} Sei $V\subseteq k^{n}$. Dann ist $I(V)$ ein Radikalideal.
\item\label{1.1.5c} Sei $V$ eine affine Varietät. Dann gilt $V=V(I(V))$.

Es gilt sogar: $I(V)$ ist das größte Ideal mit dieser Eigenschaft, d.h. für $J\subseteq k\polyx$ mit $V(J)=V$ folgt schon $J\subseteq I(V)$.
\item\label{1.1.5d} Seien $V_{1},V_{2}$ affine Varietäten. Dann gilt:
\[V_{1}=V_{2}\iff I(V_{1})=I(V_{2})\text{ und } V_{1}\subseteq V_{2}\iff I(V_{1})\supseteq I(V_{2}).\]
\end{enumerate}
\end{bem}

\begin{proof} Die Aussagen \ref{1.1.5a} und \ref{1.1.5b} folgen sofort aus der Definition.
\begin{enumerate}
\item[\ref{1.1.5c}] Sei zuerst $x\in V$. Dann gilt für alle $f\in I(V)$: $f(x)=0$, also gilt $x\in V(I(V))$.

Sei nun $V$ eine affine Varietät. Dann gilt $V=V(\F)$ für eine geeignete Menge $\F$. Es gilt $\F\subseteq I(V)$, also ist
\[V=V(\F)\supseteq V(I(V)).\]

Der Rest der Behauptung folgt aus der Definition des Verschwindungsideals.
\item[\ref{1.1.5d}] Die eine Richtung ist klar, bzw. folgt aus \ref{1.1.5a}. Für die andere Richtung überlegt man sich, dass nach \ref{1.1.5c} $V_{1}=V(I(V_{1}))=V(I(V_{2}))=V_{2}$ gilt.  Die Aussage für die Inklusionen folgt analog.\qedhere
\end{enumerate}
\end{proof}

\begin{q}
Wir haben gesehen, dass die Zuordnung
\[V\mapsto I(V)\]
injektiv ist. Ist sie auch surjektiv?
\end{q}

\section{Zariski-Topologie}

\begin{db}Sei $n\in\N$. Die affinen Varietäten im $k^{n}$ bilden die abgeschlossenen Mengen einer Topologie auf dem $k^{n}$. Diese heißt \emph{Zariski-Topologie}.\end{db}
\begin{proof}\begin{prooflist}
\item $\leer$ und $k^{n}$ sind affine Varietäten nach \autoref{bsp1.2}.
\item Seien $V_{1},V_{2}$ affine Varietäten, d.h. $V_{1}=V(I_{1})$ und $V_{2}=V(I_{2})$, wobei $I_{1},I_{2}$ Ideale in $k\polyx$ sind. Dann gilt:
\[V_{1}\cap V_{2}=V(I_{1}\cup I_{2}) = V(I_{1}+I_{2}).\]
Das gleiche Argument funktioniert für beliebige Familien $V_{\lambda}$ mit $\lambda\in\Lambda$ und $\Lambda$ Indexmenge, d.h. $\displaystyle\bigcap_{\lambda\in\Lambda}V_{\lambda}$ ist wieder eine affine Varietät.
\item Seien $V_{1},V_{2}$ affine Varietäten mit $I_{1}=I(V_{1})$ und $I_{2}=I(V_{2})$.
\begin{zeige} $V(I_{1}\cdot I_{2})\subseteq V_{1}\cup V_{2}\subseteq V(I_{1}\cap I_{2})\subseteq V(I_{1}\cdot I_{2})$.
\begin{prooflist}
\item $V_{1}\cup V_{2}\subseteq V(I_{1}\cap I_{2})$ ist nach Definition klar.
\item Sei $x\in V(I_{1}\cdot I_{2})$. Angenommen $x\notin V_{1}$. Dann existiert $g\in I_{1}$ mit $g(x)\neq 0$. Sei $f\in I_{2}$. Dann ist $f\cdot g\in I_{1}\cdot I_{2}$, also ist $f\cdot g(x) = f(x)\cdot g(x) = 0$ und da $g(x)\neq 0$ ist $f(x)=0$, also $x\in V(I_{2})=V_{2}$.
\end{prooflist}\end{zeige}
\end{prooflist}\end{proof}

\begin{w}hilfe!\end{w}

\begin{w}Sei $x\in V(I_{1}\cdot I_{2})$. Angenommen $x\notin V_{1}$. Dann existiert $g\in I_{1}$ mit $g(x)\neq 0$. Sei $f\in I_{2}$. Dann ist $f\cdot g\in I_{1}\cdot I_{2}$, also ist $f\cdot g(x) = f(x)\cdot g(x) = 0$ und da $g(x)\neq 0$ ist $f(x)=0$, also $x\in V(I_{2})=V_{2}$.\end{w}

\end{document}