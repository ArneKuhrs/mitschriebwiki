\section{Graduierte Ringe und Moduln}

\begin{DefBem}
\label{2.13}

\begin{enumerate}

\item Ein Ring $S$ zusammen mit einer Zerlegung $S=\bigoplus_{i\geq 0}S_i$
in abelsche Gruppen $S_i$ hei\ss t \emp{graduierter Ring}\index{Ring!graduierter}, wenn f\"ur alle $i, j\in \NN$
$S_i\cdot S_j\subseteq S_{i+j}$

\item Ist $S=\bigoplus_{i\geq 0} S_i$ graduierter Ring, so hei\ss en die Elemente
von $S_i$ \emp{homogen}\index{Ring!homogene Elemente} vom Grad $i$.
F\"ur $f=\sum_{i=0}^{\infty} f_i$ hei\ss en die $f_i$ die homogenen Komponenten von $f$.

\item Ist $S=\bigoplus_{i\geq 0} S_i$ graduierter Ring, so ist $S_0$ Unterring mit $1\in S_0$.
\end{enumerate}

\end{DefBem}

\begin{Bew}
\begin{enumerate}
\item[(c)] $S_0\cdot S_0\subseteq S_{0+0}=S_0$\\
Sei $1=\sum_{i\geq 0}e_i$ mit $e_i\in S_i$. Sei $f\in S_n$ mit $n\geq 1, f\neq 0$.
$\Rightarrow f=f\cdot 1 = \sum_{i\geq 1}fe_i$ mit $f\cdot e_i\in S_{n+i}$.
Da $f$ nur auf eine Weise als Summe von homogenen Elementen geschrieben werden
kann, ist $e_i=0$ f\"ur $i\geq 0$ und $e_0=1$.

\end{enumerate}
\end{Bew}

\begin{DefBem}
\label{2.14}
Sei $S=\bigoplus_{i\geq 0} S_i$ graduierter Ring.
\begin{enumerate}
\item Ein Ideal $I\subseteq S$ hei\ss t homogen, wenn es von homogenen Elementen erzeugt wird.
\item\label{2.14b} Ein Ideal $I\subseteq S$ ist genau dann homogen, wenn
f\"ur jedes $f\in I$, $f=\sum_{i\geq 0} f_i$ ($f_i\in S_i$) gilt: $f_i\in I$.

\item Sei $I\subseteq S$ homogenes Ideal, erzeugt von homogenen Elementen $(h_\nu)_{\nu\in J}$.
Dann hat jedes homogene $f\in I$ eine Darstellung $f=\sum_{\nu}g_\nu h_\nu$ mit $g_\nu$ 
homogen von Grad $\deg f - \deg{h_\nu}$.

\item Ist $I$ homogenes Ideal in $S$, so ist $S/I$ graduierter Ring mit
$(S/I)_i=S_i/(I\cap S_i)$.

\end{enumerate}
\end{DefBem}

\begin{Bew}
\begin{enumerate}
\item[(b)] \glqq$\Leftarrow$\grqq: \chk\\
\glqq$\Rightarrow$\grqq: Sei $(h_{\nu})_{\nu\in J}$ homogenes Erzeugendensystem von $I$,
$f\in I$. Dann gibt es $g_\nu\in S$ mit $f=\sum_{\nu}g_\nu h_\nu$.
Sei $g_\nu=\sum_{i\geq 0}g_{\nu_i}$ Zerlegung in homogene Komponenten
$\Rightarrow f=\sum_{\nu, i} g_{\nu_i}h_\nu
\Rightarrow f_i=\sum_\nu g_{\nu,i-\deg{h_\nu}}h_\nu$ 
(mit $g_{\nu,j}=0$ f\"ur $j< 0$) $\Rightarrow f_i\in I$.

\item[(d)] $\varphi:S=\bigoplus_{i\geq 0}S_i\to \bigoplus_i S_i/(I\cap S_i)$ ist
surjektiver Ringhomomorphismus. $\K{\varphi}$ wird erzeugt von $I\cap S_i$, $i\geq 0$.
Da $I$ homogen, ist $\K{\varphi}=I$. Aus dem Homomorphiesatz folgt dann:
$S/I\cong \bigoplus_{i\geq 0}S_i/(I\cap S_i)$.
\end{enumerate}
\end{Bew}

\begin{nnBsp}
\begin{enumerate}

\item[(1)] $S=k[X,Y]$, $I=(Y-X^2)$ ist \emph{nicht} homogen.
$S/I\cong k[X]$, $\bigoplus_i S_i/(I\cap S_i)=\bigoplus_i S_i=S$

\item[(2)] Sei $S_{+}\defeqr\bigoplus_{i> 0}S_i$ ist homogenes Ideal.
Ist $S_0$ K\"orper, so ist $S_{+}$ das einzige maximale homogene Ideal.

\item[(3)] $S=k[X,Y]$, $\deg{X}=1$, $\deg{Y}=2$. 
Dann ist $I=(Y-X^2)$ homogenes Ideal!

\end{enumerate}
\end{nnBsp}

\begin{DefBem}
\label{2.15}
F\"ur einen graduierten Ring $S=\bigoplus_{i\geq 0} S_i$ sind \"aquivalent:
\begin{enumerate}
\item[(i)] $S$ noethersch
\item[(ii)] $S_0$ ist noethersch und $S_{+}$ endlich erzeugbares Ideal.
\item[(iii)] $S_0$ ist noethersch und $S$ ist endlich erzeugbare $S_0$-Algebra.
\end{enumerate}
\end{DefBem}

\begin{Bew}
\glqq$(i)\Rightarrow (ii)$\grqq: $S_0\cong S/S_{+}$; $S_{+}$ endlich erzeugbar, da $S$ noethersch.
$S_0$ also noethersch.

\glqq$(iii)\Rightarrow (i)$\grqq: $S\cong \underbrace{S_0[X_1, \ldots, X_n]}_
{\text{noethersch nach Satz 1}}/I$ f\"ur ein $n\geq 0$
und ein Ideal $I\subset S_0[X_1, \ldots, X_n]$. $S$ ist also noethersch.

\glqq$(ii)\Rightarrow (iii)$\grqq: Sei $f_1, \ldots, f_\nu$ homogenes Erzeugersystem
von $S_+$ und  
\[
S'\defeqr S_0[f_1, \ldots, f_\nu]\subset S
\]
die von den $f_i$ erzeugte
$S_0$-Unteralgebra von $S$.\\
\emph{Beh.:} S'=S\\
Zeige dazu: $S_i\subset S'$ f\"ur alle $i$.

Beweis der Behauptung induktiv \"uber $i$:
$i=0$: \chk\\
$i> 0$: $g\in S_i\stackrel{2.14(c)}{\Rightarrow}g=\sum_{\nu=1}^{r} g_{\nu}f_\nu$ mit
$g_\nu\in S_{i-\deg{f_\nu}}$\\
$f_\nu\in S_{+} \Rightarrow \deg{f_\nu}> 0 \Rightarrow i-\deg{f_\nu} < i$.
Aus der Induktionsvoraussetzung folgt dann: $g_\nu\in S'$, also ist $g\in S'$
\end{Bew}

\begin{DefBem}
\label{2.16} Sei $S=\bigoplus_{i\geq 0}S_i$ graduierter Ring.
\begin{enumerate}

\item Ein \emp{graduierter}\index{R-Modul!graduierter} $S$-Modul ist ein $S$-Modul $M$ zusammen mit
einer Zerlegung $M=\bigoplus_{i\in\ZZ} M_i$ in abelsche Gruppen $M_i$,
sodass f\"ur alle $i\in\NN, j\in\ZZ$  gilt:
$S_i\cdot M_j\subseteq M_{i+j}$.

\item Eine $S$-lineare Abbildung $\varphi:M\to M'$ zwischen graduierten $S$-Moduln
hei\ss t \emp{graderhaltend}\index{Abbildung!graderhaltende}, wenn $\varphi(M_i)\subseteq
M_i'$ f\"ur alle $i\in \ZZ$.

\item Ein Ideal $I\subseteq S$ ist homogen $\Leftrightarrow I$ ist als $S$-Modul
graduiert (mit der geerbten Graduierung).

\item $\varphi: M \to M'$ vom Grad $d \Leftrightarrow \varphi(M_i) \subseteq
      M_{i+d}$ für alle $i$.\\
      Kern$(\varphi)$ ein graduierter Untermodul.
      
\item Ist $I \subseteq S$ homogenes Ideal, so ist $\varphi: S \to S/I =
\bigoplus_{i \ge
0} S_i/(I \cap S_i)$ graderhaltend. $\K{\varphi}$ ein graduierter Untermodul.
\end{enumerate}
\end{DefBem}

\begin{nnBsp}
Sei $M$ graduierter $S$-Modul (z.B.: $M=S$). Für $d\in\ZZ$ sei
$M(l)$ der $S$-Modul $M$ mit der Graduierung $(M(l))_i\defeqr M_{l+i}$
(insbes.: $(M(l))_0=M_l$)
$S_j(M(l))_i=S_j\cdot M_{l+i}\subseteq M_{j+l+i}=(M(l))_{i+j}$

$M(l)$ heißt ($l$-facher) \emp{Twist}\index{Twist} von $M$.
\end{nnBsp}

\begin{Bew}
  \begin{enumerate}
    \item[(e)] Sei $\varphi: M \to M'$ lineare Abbildung von $S$-Moduln vom Grad
    $d$. Sei $x \in \K{\varphi}, x = \sum_{i \in \ZZ} x_i
    \Rightarrow 0 = \varphi(x) = \sum_{i \in \ZZ} \underset{\in M'_{i +
    d}}{\underbrace{\varphi(x_i)}}$ ist Zerlegung in homogene Komponenten
    $\Rightarrow \varphi(x_i) = 0 \; \forall i \Rightarrow x_i \in \K{\varphi} \; \forall i \Rightarrow \K{\varphi}$ ist graduiert
    (nach Bemerkung \ref{2.14}\ref{2.14b}).
  \end{enumerate}
\end{Bew}

\begin{Beo}
  Ist $\varphi: M \to M'$ vom Grad $d$, so ist $\varphi: M \to M'(d)$
  graderhaltend. Dabei ist $M'(d) = M'$ als $S$-Modul, aber $(M'(d))_i = M'_{d
  +i}$. Genauso ist $\varphi: M(-d) \to M'$ graderhaltend.
\end{Beo}

\begin{nnBsp}
  $M = S (=k[X_1, \dots, X_n]), f \in S$ homogen vom Grad $d \Rightarrow
  \varphi_f: S \to S, g \mapsto f \cdot g$ ist linear vom Grad $d$.
\end{nnBsp}

\begin{Prop}
  Sei $S = k[X_1, \dots, X_n], k$ ein Körper, $S = \bigoplus_{d = 0}^{\infty} S_d$.\\
  $\dim[k]{S_d^{(n)}} = \binom{n+d-1}{d} = \frac{1}{(n-1)!} \cdot (n+d-1)
  \cdot ... 
  \cdot (d+1)$. Das ist ein Polynom vom Grad $n-1$ in $d$ (mit Leitkoeffizient
  $\frac{1}{(n-1)!}$).
\end{Prop}

\begin{Bew}
  Induktion über $n$:\\
  $n=1,2$: klar\\
  $n>2$:  Induktion über $d$:\\
  $d=0,1$: klar\\
  $d>1$: $\dim[k]{S_d^{(n)}}$ ist die Anzahl der Monome vom Grad $d$ in $X_1, \dots,
  X_n$. In $S_d^{(n)}$ gibt es $\dim[k]{S_{d-1}^{(n)}}$ Monome in denen $X_n$ vorkommt und $\dim[k]{S_d^{(n-1)}}$ Monome in denen $X_n$ nicht
  vorkommt $\overset{IV}{\Rightarrow}
  \dim[k]{S_d^{(n)}} = \binom{n+d-2}{d} +
  \binom{n+d-2}{d-1} = \frac{(n+d-2)!}{(d-1)! (n-2)!} (\frac{1}{d} + \frac{1}{n-1}) =
  \frac{(n+d-2)!}{(d-1)! (n-2)!} \frac{n+d-1}{d(n-1)} = \frac{(n+d-1)!}{d!
  (n-1)!} = \binom{n+d-1}{d}$.
\end{Bew}

\begin{Satz}
\label{Satz6}
  Sei $k$ ein Körper, $S=k[X_1, \dots, X_n], M$ ein endlich erzeugbarer
  graduierter $S$-Modul. Dann gibt es ein Polynom $P_M \in \QQ[T]$ vom
  Grad $\le n -1 $ und ein $d_0 \in \NN$, sodass $P_M(d) = \dim[k]{M_d}$ für alle $d \ge d_0$. $P_M$ heißt das
  \emp{Hilbert-Polynom}\index{Hilbert!-Polynom} von $M$.
\end{Satz}

\begin{Bew}
  Induktion über $n$:\\
  $n=0$: $M$ ist endlich dimensionaler $k$-Vektorraum, also $M_d=0$ für alle $d
  \gg 0, P_M = 0$ tut's.\\
  $n\geq 1$: Sei $\varphi: M \to M$ die $S$-lineare Abbildung $x \mapsto X_n x,
  \varphi$ ist vom Grad $1$, Kern$(\varphi)$ ist also graduierter Untermodul,
  ebenso ist Bild$(\varphi)$ graduierter Untermodul, also auch $M/X_n M$.
  Dann ist
  \[
  0 \to \underset{\defeqr\K \varphi}{\underbrace{K}} \to M(-1) \overset{\varphi}{\to}
  M \to M/X_n M \to 0
  \]
  exakte Sequenz von graderhaltenden Homomorphismen zwischen graduierten\\
  endlich erzeugbaren $S$-Moduln.\\
  Beachte: $M$ ist noetherscher Modul, da $S$ noethersch und $M$ endlich
  erzeugbar, also ist $K$ auch endlich erzeugbar.\\
  Alle $M_d, K_d,(M/X_n M)_d$ sind endlich dimensionale $k$-Vektorräume
  $\Rightarrow$ für jedes $d \in \ZZ$ gilt: $\dim[k]{K_d} -
  \dim[k]{M(-1)_d} + \dim[k]{M_d} - \dim[k]{(M/X_n M)_d} = 0$
  bzw. $\dim[k]{M_d} - \dim[k]{M_{d-1}} = \dim[k]{(M/X_n M)_d}
  - \dim[k]{K_d}$\\
  \textbf{Beh.:} $M/X_n M$ und $K$ sind (in natürlicher Weise) $k[X_1, \dots,
  X_n]$-Moduln.\\
  \textbf{Bew.:} klar für $M/X_n M$.\\
  für $K$: Seien $y_1, \dots, y_r$ Erzeuger von $K$ als $S$-Modul. Sei $y =
  \sum_{i = 1}^r f_i y_i \in K, f_i \in S$.
  Dann ist \OE\ $f_i \in k[X_1, \dots, X_{n-1}]$, da $X_n \cdot y_i = 0$ für
  alle $i$. Nach IV gibt es $\tilde{P}(d) \in \QQ[T]$ mit $\deg{\tilde{P}}
  \le n-2$ und $\tilde{P}(d) = \dim[k]{(M/X_n M)_d} - \dim[k]{K_d} =
  \dim[k]{M_d} - \dim[k]{M_{d-1}} \defeql H(d) - H(d-1)$.\\
  Sei $\binom{T}{k} \defeqr \frac{1}{k!} T (T-1) \dots (T-k+1) \in
  \QQ[T], \deg{\binom{T}{k}} = k$.
  Schreibe $\tilde{P} = \sum_{k = 0}^{n-1} c_k \binom{T}{k}$. Es gilt
  $\binom{T+1}{k+1} - \binom{T}{k+1} = \binom{T}{k}$. Setze
  \[
  P_1(T) \defeqr \sum_{k=0}^{n-2} c_k \binom{T}{k+1},\ \deg{P_1} \le n-1
  \]
  und $P_1(d+1)-P_1(d) = \tilde{P}(d)$. $P_M \defeqr P_1 +c$, sodass $P_M(d_0) =
  \dim[k]{M_{d_0}}$.
\end{Bew}

\begin{Def}
\label{2.18}
  Sei $S$ endlich erzeugte graduierte $k$-Algebra, $S_0 = k,\; M$ endlich
  erzeugbarer graduierter $S$-Modul. Dann heißt die formale Potenzreihe $$H_M(t)
  \defeqr \sum_{i=0}^{\infty} (\dim[k]{M_i}) t^i$$
  \emp{Hilbert-Reihe}\index{Hilbert!-Reihe} zu $M$.
\end{Def}

\begin{nnBsp}
  \begin{enumerate}
    \item[1.)] $M = S = k[X] \Rightarrow \dim{M_i} = 1 \text{ für alle }i
               \Rightarrow H_M(t) = \sum_{i=0}^{\infty}t^i = \frac{1}{1-t}
               \defeqr (1-t)^{-1}$.
    \item[2.)] $M = S = k[X_1, \dots, X_n]$\\
               \textbf{Beh.:} $H_M(t) = \frac{1}{(1-t)^n}$\\
               \textbf{Bew.:} $\frac{1}{(1-t)^n} = \left( \sum_{i=0}^{\infty}
               t^i\right)^n = \sum_{i=0}^{\infty} c_i t^i$ mit $c_i =
               \#$Partitionen von $i$ durch höchstens $n$ Summanden = Anzahl der
               Monome vom Grad $i$ in $X_1, \dots, X_n$.
    \item[3.)] $M = S = k[Y] (\cong k[X^d]), \deg{Y} = d > 0 \Rightarrow
               H_M(t) = \sum_{i = 0}^{\infty} t^{d \cdot i} = \frac{1}{1-t^d}$\\
               dim $M_i = \begin{cases}1: & d \mid i\\ 0: & \text{sonst} \end{cases}$
  \end{enumerate}
\end{nnBsp}

\begin{nnSatz}
  Seien $S, M$ wie in Def \ref{2.18}.\\
  $f_1, \dots, f_r$ homogene Erzeuger von $S$ als $k$-Algebra, $d_i \defeqr
  \deg{f_i}$.\\
  Dann gibt es ein Polynom $F(t) \in \ZZ[t]$, sodass gilt:
  $$ H_M(t) = \frac{F(t)}{(1-t^{d_1}) \cdot (1-t^{d_2}) \cdot ... \cdot 
  (1-t^{d_r})}$$
\end{nnSatz}

\begin{Bew}
  Induktion über $r$:\\
  $r=0$: $S = S_0 = k \Rightarrow \dim[k]{M_i} = 0$ für $i \gg 0
  \Rightarrow F(t) \defeqr H_M(t)$ ist Polynom in $\ZZ[t]$.\\
  $r>0$: Multiplikation mit $f_r$ gibt exakte Sequenz von graderhaltenden
  $S$-Modul-Homomorphismen:
  $$ 0 \to K \to M \overset{\cdot f_r}{\to} M(d_r) \to (M/f_r M)(d_r) \to 0$$
  Wie in Beweis von Satz \ref{Satz6} sind $K$ und $Q \defeqr (M/f_r M)$ Moduln
  über $S' \defeqr k[f_1, \dots, f_{r-1}] \subset S \Rightarrow$ für jedes $i
  \ge 0$ ist
  \[
   -\dim{M_i} + \dim{M_{i + d_r}} = \dim{Q_{i+d_r}}-
  \dim{K_i}.
  \]
  Es folgt
  \begin{equation*}
  \begin{split}
  \sum_{i=0}^{\infty} \dim{M_{i+d_r}}t^{i+d_r} - t^{d_r}\sum_{i=0}^{\infty}
  \dim{M_i}t^i \\
   = \sum_{i=0}^{\infty} \dim{Q_{i+d_r}}t^{i+d_r} - t^{d_r}\sum_{i=0}^{\infty} \dim{K_i}t^i
  \end{split}
  \end{equation*}
  \[
  \Rightarrow H_M(t) - \sum_{i=0}^{d_r -1} \dim{M_i}t^i - t^{d_r} H_M(t)
  = H_Q(t) - \sum_{i=0}^{d_r - 1} \dim{Q_i}t^i - t^{d_r} H_K(t)
  \]
  Nach Induktionsvoraussetzung gibt es $F_1(t), F_2(t) \in \ZZ[t]$ mit
  \[
  (1-t^{d_r})H_M(t)= \frac{F_1(t)}{\displaystyle\prod_{i=0}^{r-1}(1-t^{d_i})}-
  \frac{t^{d_r} F_2(t)}{\displaystyle \prod_{i=0}^{r-1}(1-t^{d_i})} +
  \underset{G(t)}{\underbrace{\sum_{i=1}^{d_r-1} \dim{M_i}t^i -
  \sum_{i=0}^{d_r - 1} \dim{Q_i}t^i}}
  \]
  $\Rightarrow$ Behauptung mit $F(t) = F_1(t) - t^{d_r}F_2(t) + G(t)\cdot \displaystyle\prod_{i=1}^{r-1}(1-t^{d_i})$
\end{Bew}
