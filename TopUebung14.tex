\documentclass{article}
\usepackage[utf8]{inputenc}
\usepackage{mathrsfs}
\usepackage{stmaryrd}

\usepackage{mathe}
\usepackage{enumerate}
\usepackage{amscd}

\title{14. Topologie-Übung}
\author{Joachim Breitner}
\date{6. Februar 2008}

\begin{document}
\maketitle

\section*{Aufgabe 1}

Seien $X, Y$ wegzusammenhängend, lokal wegzusammenhängend, hausdorff’sch, und $Y$ kompakt.

\paragraph{Behauptung:} Ein lokaler Homöomorphismus $f: Y\to X$ ist Überlagerung.

Sei $x\in X$. Es git: $f^{-1}(x)$ endlich.

Denn wäre $f^{-1}(x)$ unendlich dann hätte $f^{-1}(x)$ einen Häufungspunkt $h$ im kompakten Hausdorffraum $Y$. $h$ hat keine Umgebung $V$, so dass $f|_V$ ein Homöomorphismus ist, denn in jeder Umgebung von $h$ liegt ein Element von $f^{-1}(x)$, im Widerspruch zur Injektivität von $f|_V$.

Also gilt: $f^{-1}(x)=\{y_1,\ldots,y_n\}$. Wähle für jedes $y_i$ eine Umgebung $U_i$, so dass alle $U_i$ disjunkt und wegzusammenhängend sind und $f|_{U_i}$ Homöomorphismen auf eine offene Teilmenge $U$ von $X$ sind.

Setze $V\da \bigcap_{i=1}^n f(U_i)$, dann enhält $f^{-1}(V)$ $n$ „Kopien“ von $V$, in jedem $U_i$ eine.

Noch zu zeigen: Es gibt eine Umgebung $\tilde V$ von $x$, $\tilde V\subseteq V$, so dass für alle $\tilde x\in \tilde V$ gilt: $\#f^{-1}(\tilde x) = \#f^{-1}(x)=n$.

Angenommen, in jeder Umgebung $\tilde V$ von $x$ gibt es ein $\tilde x$, so dass $\#f^{-1}(\tilde x) > n$, dann gibt es eine Folge $(x_k)\in V$, mit $\#f^{-1}(x_k)>n$. $f^{-1}(\tilde x)>n \implies \exists\tilde y\in f^{-1}(\tilde x): \tilde y \notin V_1\cap\ldots\cap V_n$. $Y$ ist kompakt, also hat $(x_k)$ einen Häufungspunkt, o.B.d.A: $\lim_{k\to \infty} x_k$ existiert, also $\lim_{k\to \infty}x_k = x$, Wid!.

\section*{Aufgabe 2}

Seien $p_1:Y_1\to X_1$, $p_2:Y_2\to X_2$ Überlagerungen.

\paragraph{Behauptung:} $p_1\times p_2: Y_1\times Y_2\to X_1\times X_2$ ist auch eine Überlagerung.

Sei $(x_1,x_2)\in X_1\times X_2$. Zu zeigen: Es gibt eine Umgebung $U$ von $(x_1,x_2)$, so dass $p^{-1}(U)$ disjunkte Vereinigung von offenen Mengen in $Y_1\times Y_2$ ist die alle homöomorph sind zu $U$.

$p_1$ ist Überlagerung, das heißt es gibt eine offene Umgebung $U_1$ von $x_1$, so dass $f^{-1}(U_1)=\bigcup_{x\in I}\tilde V_i$, $\tilde V_i$ homöomorph zu $U_i$ für $i\in I$.
$p_2$ ist auch Überlagerung, das heißt es gibt eine offene Umgebung $U_2$ von $x_2$, so dass $f^{-1}(U_2)=\bigcup_{x\in J}\tilde W_i$, $\tilde W_i$ homöomorph zu $U_i$ für $i\in J$.

Dann gilt:
\[(p_1\times p_2)^{-1}(U_1\times U_2) = f^{-1}(U_1) \times f^{-1}(U_2) = \bigcup _{i\in I} \tilde V_i \times \bigcup_{j_\in J} \tilde W_j = \bigcup_{\mathclap{(i,j)\in I\times J}} (\tilde V_i \times \tilde W_j)\]
da die $\tilde V_i$, $\tilde W_j$ disjunkt und offen sind, sind die $\tilde V_i \times \tilde W_j$ disjunkt und offen, und $p_1\times p_2$ ist ein Homöomorphismus von $\tilde V_i \times \tilde W_j$ nach $U_1\times U_2$. Also ist $p_1\times p_2$ eine Überlagerung.

\section*{Aufgabe 3}

Sei $p:Y\to X$ eine Überlagerung.

\paragraph{Behauptung:} Ist $X$ ein Hausdorffraum, so auch $Y$.

Seien $y_1,y_2\in Y$, $y_1\ne y_2$.

1. Fall: $p(y_1) = p(y_2) = x$, dann „Überlagerungsumgebungen“ von $y_1$, $Y_2$ sind offen und disjunkt, trennen also $y_1$ und $y_2$.

2. Fall: $p(y_1) \ne p(y_2)$. Man findet offene disjunkte Umgebungen, die $p(y_1)$ und $p(y_2)$ trennen. Deren Urbilder sind disjunkte offene Umgebungen, die $y_1$ und $y_2$ trennen.

\paragraph{Behauptung:} Ist $X$ kompakt und $p^{-1}(x)$ für jedes $x\in X$ endlich, so ist auch $Y$ kompakt.

Sei $(U_i)_{i\in I}$ eine offene Überdeckung von $Y$. Zu zeigen: Man kann daraus eine endliche Teilüberdeckung wählen. Für jedes $x\in X$ ist $p^{-1}(x)$ endlich. Das heißt: Es gibt eine endliche Teilmenge $J\subseteq I$, sodass $f^-1{x}$ von $(U_j)_{j\in J}$ überdeckt wird.

Setze $O_x \da \bigcup_{j\in J} U_j$. Zu $O_x$ existiert eine offene Teilmenge $V_x$ von $x$, so dass $x\in V_x$ und $f^{-1}(V_x)\subseteq O_x$.

Denn $p$ ist Überlagerung, also gibt es eine Umgebung $V$ von $x$, so dass $p^{-1}(V) = \bigcup_{i=1}^n U_i$, $U_i$ disjunkt und homöomorph zu $V$. Definiere $V_x \da p(\bigcup_{i=1}^n U_i \cap O_x)$.

Klar: $\{V_x, x\in x\}$ ist offene Überdeckung von $X$ und $X$ ist kompakt, also gibt es $x_1,\ldots,x_m$ aus $X$, so dass $V_{x_1},\ldots V_{x_n}$ schon $X$ überdecken.

Die zu $V_{x_i}$ „gehörigen“ $U_j, j\in J_{x_i}$, für $i=1,\ldots,n$ bilden eine Teilüberdeckung von $(U_i)_{i\in I}$, denn für $y\in Y$, $x=p(y)\in V_{x_i}$, so ist $y\in f^{-1}(V_{x_i} \subseteq O_x = \bigcup_{j\in J_{x_i}} U_j$, also ist $y\in U_j$ für so ein $j$.

\end{document}
