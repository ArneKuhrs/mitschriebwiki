\documentclass{article}
\newcounter{chapter}
\setcounter{chapter}{6}
\usepackage{ana}

\title{Differentialgleichungen: Grundbegriffe}
\author{Pascal Maillard}
% Wer nennenswerte �nderungen macht, schreibt sich bei \author dazu

\begin{document}
\maketitle

In diesem Paragraphen sei $I$ stets ein Intervall in $\MdR$.

\paragraph{Erinnerung:}
Sei $p\in\MdN$ und $y:I\to \MdR^p,\ y=(y_1,\ldots,y_p).\ y$ hei�t auf $I$ k-mal (stetig) db auf $I \equizu y_j$ ist auf $I$ k-mal (stetig) db $(j=1,\ldots,p).$

In diesem Fall gilt: $$y^{(j)} = (y_1^{(j)},\ldots,y_p^{(j)})\quad(j=0,\ldots,k)$$

\begin{definition}
\indexlabel{Differentialgleichung}
\indexlabel{Differentialgleichung!gew�hnliche}
Seien $n,p\in\MdN$ und $D\subseteq\MdR\times\underbrace{\MdR^p\times\ldots\times\MdR^p}_{n+1\text{ Faktoren}}$ und $F:D\to\MdR^p$ eine Funktion.

Eine Gleichung der Form $$(i)\quad F(x,y,y',\ldots,y^{(n)})=0$$ hei�t eine \textbf{(gew�hnliche) Differentialgleichung (Dgl) $n$-ter Ordnung}.

\indexlabel{Differentialgleichung!L�sung einer gew�hnlichen}
\indexlabel{L�sung einer gew�hnlichen Differentialgleichung}

Eine Funktion $y:I\to\MdR^p$ hei�t eine \textbf{L�sung} von $(i)$, gdw. gilt:
\begin{itemize}
\item $y$ ist auf $I$ $n$-mal db,
\item $\forall x\in I: (x,y(x),y'(x),\ldots,y^{(n)}(x))\in D$ und
\item $\forall x\in I: F(x,y(x),y'(x),\ldots,y^{(n)}(x)) = 0.$
\end{itemize}
\end{definition}

\begin{beispiele}
\item $n=p=1,\ F(x,y,z) = y^2+z^2-1,\ D=\MdR^3$.

Dgl: $y^2+y'^2-1=0.$

$y:\MdR\to\MdR,\ y(x)=1$ ist eine L�sung,\\
$\bar{y}:\MdR\to\MdR,\ \bar{y}(x)=\sin x$ ist eine weitere L�sung.

\item $n=p=1,\ F(x,y,z)=z+\frac{y}{x},\ D=\{(x,y,z)\in\MdR^3:x\ne0\}$.

Dgl: $y'+\frac{y}{x}=0.$

$y:(0,\infty)\to\MdR,\ y(x)=\frac{1}{x}$ ist eine L�sung,\\
$\bar{y}:(-\infty,0)\to\MdR,\ \bar{y}(x)=\frac{17}{x}$ ist eine weitere L�sung.

\item $n=1,p=2.$ Mit $y=(y_1,y_2):$
$$y'=\begin{pmatrix}y_1'\\ y_2'\end{pmatrix} = \begin{pmatrix}-y_2\\ y_1\end{pmatrix}$$

$y:\MdR\to\MdR^2,\ y(x)=(\cos x,\ \sin x)$ ist eine L�sung.
\end{beispiele}

\begin{definition}
\indexlabel{explizite Differentialgleichung}
\indexlabel{Differentialgleichung!explizite}
Seien $n,p \in\MdN,\ D\subseteq\MdR\times\underbrace{\MdR^p\times\ldots\times\MdR^p}_{n\text{ Faktoren}}$ und $f:D\to\MdR^p$.

Eine Gleichung der Form $$(ii)\quad y^{(n)} = f(x,y,y',\ldots,y^{(n-1)})$$ hei�t \textbf{explizite Differentialgleichung $n$-ter Ordnung}.

\indexlabel{Anfangswertproblem}
\indexlabel{AWP}

Ist $(x_0,y_0,y_1,\ldots,y_{n-1})\in D$ (fest), so hei�t das Gleichungssystem
$$(iii)\quad\begin{cases}y^{(n)} = f(x,y,y',\ldots,y^{(n-1)})\\ y(x_0)=y_0,\ y'(x_0)=y_1,\ldots,\ y^{(n-1)}(x_0)=y_{n-1}\end{cases}$$

ein \textbf{Anfangswertproblem (AWP)}

\indexlabel{Differentialgleichung!L�sung einer expliziten}
\indexlabel{L�sung einer expliziten Differentialgleichung}

$y:I\to\MdR^p$ hei�t eine \textbf{L�sung} von $(ii)$, gdw. gilt:
\begin{itemize}
\item $y$ ist auf $I$ $n$-mal db,
\item $\forall x\in I: (x,y(x),y'(x),\ldots,y^{(n-1)}(x))\in D$ und
\item $\forall x\in I: y^{(n)}(x) = f(x,y(x),y'(x),\ldots,y^{(n-1)}(x)).$
\end{itemize}

\indexlabel{Anfangswertproblem!L�sung eines}
\indexlabel{L�sung eines Anfangswertproblems}

$y:I\to\MdR^p$ hei�t eine \textbf{L�sung} von $(iii)$, gdw. gilt:
\begin{itemize}
\item $y$ ist eine L�sung von $(ii)$,
\item $x_0 \in I$ und
\item $y^{(j)}(x_0) = y_j\ (j=0,\ldots,n-1)$
\end{itemize}

\indexlabel{Anfangswertproblem!eindeutig l�sbares}
\indexlabel{endeutig l�sbares Anfangswertproblem}
Das AWP $(iii)$ hei�t eine \textbf{eindeutig l�sbar}, gdw. gilt:
\begin{itemize}
\item $(iii)$ hat eine L�sung und
\item f�r je zwei L�sungen $y_1:I_1\to\MdR^p,\ y_2:I_2\to\MdR^p$ von $(iii)\ (I_1,I_2$ Intervalle in $\MdR)$ gilt: $y_1\equiv y_2$ auf $I_1 \cap I_2$
\end{itemize}
\end{definition}

\begin{beispiele}
\item $$\text{AWP: }\begin{cases}y'=2\sqrt{|y|}\\ y(0)=0\end{cases}\ (n=1,\ p=1)$$

$y:\MdR\to\MdR,\ y(x)=0$ ist eine L�sung des AWPs,\\
$\bar{y}:[0,\infty)\to\MdR,\ \bar{y}(x)=x^2$ ist eine weitere L�sung.

\item $$\text{AWP: }\begin{cases}y'=2y\\ y(0)=1\end{cases}\ (n=1,\ p=1)$$
$y:\MdR\to\MdR,\ y(x)=e^{2x}$ ist eine L�sung des AWPs.

Sei $\bar{y}:I\to\MdR$ eine L�sung des AWPs. Wir definieren $$g(x) := \frac{\bar{y}(x)}{e^{2x}}\ (x\in I)$$.

Nachrechnen: $g'(x)=0\ \forall x\in I \folgt \exists c\in\MdR:g(x)=c\ \forall x\in I \folgt \bar{y}(x) = ce^{2x}\ (x\in I).$

$1=\bar{y}(0)=c \folgt \bar{y}(x)=e^{2x}\ \forall x\in I.$

Das AWP ist also eindeutig l�sbar.
\end{beispiele}

\end{document}
