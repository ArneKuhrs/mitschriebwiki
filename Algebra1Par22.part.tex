\section{Polynomringe}

\begin{DefBem}
Sei $R$ ein kommutativer Ring
mit Eins, $R \neq \{0\}$.
\begin{enum}
\item Ein \emp{Polynom} über $R$ ist eine (endliche) Folge\medskip\newline $f=(a_0,
a_1,\dots, a_n,a_{n+1},\dots)$ mit $n \in \mathbb{N}$ und $a_i \in
R, a_i = 0$ für fast alle $i$.
\newline symbolisierte Schreibweise: $\ds f=\sum_{i=0}^n a_i X^i$
\newline($n$ so groß, daß $a_i = 0$ für $i > n$)
\item $R[X] = \{f = (a_0,\dots,a_n,\dots)\;: f$ Polynom über $R\}$
ist kommutativer Ring mit Eins mit den Verknüpfungen
\[\begin{array}{lclcl}
(a_0,a_1,\dots)& + &(b_0,b_1,\dots)& = &(a_0+b_0,a_1+b_1,\dots) \\
(a_0,a_1,\dots)& \cd &(b_0, b_1,\dots)& = &(c_0, c_1,\dots) \\
&&\mbox{mit } c_i & =& \sum_{k=0}^i a_k b_{i-k}\end{array}\]
\item $R \ra R[X]$, $a \mapsto (a,0,\dots)$ ist injektiver
Ringhomomorphismus
\item Für $n \geq 2$ heißt $R[X_1,\dots,X_n] =
(R[X_1,\dots,X_{n-1}])(X_n)$ \newline\emp{Polynomring in
$\mathbf{\mathit{n}}$ Variablen über $\mathbf{\mathit{R}}$}
\end{enum}
\end{DefBem}

\begin{Prop}
Sei $R$ kommutativer Ring mit Eins.
\begin{enum}
\item Zu jedem $x \in R$ gibt es genau einen Ringhomomorphismus.
$\varphi_x: R[X] \ra R$ mit $\varphi_x|R = id_R$ und $\varphi_x(X) =
x$. Es ist $\ds \varphi_x(a_0, a_1, \dots) = \sum_{i\geq 0} a_i x^i$
\item Zu jedem Homomorphismus $\alpha: R \ra R'$ von Ringen mit Eins
und jedem $y \in R'$ gibt es genau einen Ringhomomorphismus
$\varphi_y:R[X]\ra R'$, $\varphi_y|R = \alpha$ und $\varphi_y(X) =
y$
\end{enum}
\bew{}{
\item ist (b) für $R' = R$ und $\alpha = id_R$
\item $\ds \varphi_y(a_0,a_1,\dots) \defeqr \sum_{i\geq 0} \alpha(a_i)y^i$
ist die einzig mögliche Definition eines Ringhomomorphismus, weil
$\ds \varphi_y(a_0,a_1,\dots) = \varphi_y(\sum_{i=0}^n a_i X^i) = \sum_{i=0}^n
\varphi_y(a_i) \varphi_y(X)^i$ sein muß. }
\end{Prop}

\begin{Folg}
Die Zuordnung $R \mapsto R[X]$ ist ein
kovarianter Funktor: Ringe mit Eins $\ra$ Ringe mit Eins.
\newline \sbew{1.0}{Ist $\alpha: R \ra R'$ Ringhomomorphismus,
so sei $\Psi: R[X] \ra R'[X]$ der Homomorphismus, der durch
$\alpha:R \ra R' \underset{2.8(c)}{\hookrightarrow} R'[X]$ und $X
\mapsto X$ bestimmt ist.}
\end{Folg}

\begin{DefBem}
\begin{enum}
\item Für $f=(a_0,a_1,\dots) \in R[X] (f \neq 0)$ sei Grad($f) \defeqr
\max\{i: a_i \neq 0\} = $deg($f$).
\item Für $f,g$ ist Grad($f+g$) $\leq \max($Grad($f$), Grad($g$))
\item Für $f,g$ ist Grad($f\cd g$) $\begin{array}{cc} \leq & \mbox{Grad}(f) +
\mbox{Grad}(g) \\
= & \mbox{, falls } R \mbox{ nullteilerfrei} \end{array}$
\end{enum}
\end{DefBem}

\begin{Folg}
Ist $R$ Integritätsbereich, so ist $R[X]$
ebenfalls Integritätsbereich und $R[X]^x = R^x$
\end{Folg}

\begin{DefBem}[Verallgemeinerung des Polynomrings]
\label{2.13}
Sei $R$ kommutativer Ring mit Eins, $(H,\cd)$ Halbgruppe.
\begin{enum}
\item $R[H] = \{(a_h)_{h \in H},\; a_h \neq 0$ nur für endlich viele
$h\in H\}$ ist mit den Verknüpfungen \[\begin{array}{ccc}
(a_h)+(b_h) &\defeqr& (a_h + b_h) \\
(a_h)\cd(b_h) &\defeqr& (c_h) = \sum_{h_1 h_2 = h} a_{h_1} b_{h_2}
\end{array}\]
ein Ring. $R[H]$ heißt \emp{Halbgruppenring} zu $H$ über $R$.
\newline Schreibweise: (auch) $\sum_{h \in H} a_h h$ für $(a_h)$
\item $R[(\mathbb{N},+)] \cong R[X],\; R[(\mathbb{N}^n,+)] \cong
R[X_1,\dots,X_n]$
\item $R[H] \left\{\begin{array}{l} \mbox{kommutativ} \\ \mbox{hat Eins}
\end{array}\right\} \lra H \left\{\begin{array}{l} \mbox{kommutativ} \\ \mbox{hat Eins}
\end{array}\right\}$
\item $(H,\cd) \mapsto (R[H],\cd), h \mapsto 1_R h$ ist injektiver
Halbgruppenhomomorphismus.
\item Ist $(H,\cd)$ Monoid, so ist $R \ra R[H], \; r\mapsto r\cd
1_H$ injektiver Ringhomomorphismus. \end{enum} \sbew{1.0}{-}
\end{DefBem}

\begin{Satz}[Universelle Abbildungseigenschaft des Monoidrings]
Sei $R$ Ring mit Eins, $(H,\cd)$ Monoid. Dann gibt es zu jedem
Homomorphismus $\varphi: R \ra R'$ von Ringen mit Eins und jedem
Monoidhomomorphismus $\sigma: H\ra (R',\cd)$ genau einen
Ringhomomorphismus $\Phi: R[H] \ra R'$ mit $\Phi_{|R} = \varphi$ und
$\Phi_{|H} = \sigma$. Dabei werden $R$ und $H$ wie in \ref{2.13} in
$R[H]$ eingebettet. \newline \sbew{1.0}{Es muß gelten: $\Phi(\sum_h
a_h h) = \sum_h \varphi(a_h) \sigma(h)$. Dies zeigt die
Eindeutigkeit, taugt aber auch als Definition von $\Phi$, was die
Existenz zeigt.}
\end{Satz}

\begin{DefBem}
\begin{enum}
\item $R\llbracket X\rrbracket = \{ (a_i)_{i \in \mathbb{N}}: a_i \in
R\}$ ist mit $+$ und $\cd$ wie im Polynomring ein kommutativer Ring
mit Eins. $R\llbracket X\rrbracket$ heißt \emp{Ring der (formalen)
Potenzreihen} über $R$.
\newline Schreibweise (auch): \[f = \sum_{i=0}^\infty a_i x^i\] für
$f=(a_i)_{i \in \mathbb{N}}$
\item Sei $0 \neq f = \sum_{i=0}^\infty a_i x^i \in
R\llbracket X\rrbracket$. Dann heißt $o(f) \mathrel{\mathop:}=
\min\{i \in \mathbb{N},\; a_i \neq 0 \}$ der \emp{Untergrad} von
$f$. Es gilt für alle $f,g \in R\llbracket X\rrbracket \setminus
\{0\}:$ \[o(f+g) \geq \min \{o(f), o(g)\} \mbox{ und } o(f\cd g)
\geq o(f) + o(g)\]

\item Ist $R$ Integritätsbereich, so ist $o(f\cd g) = o(f) + o(g)\;
\forall f,g \in R\llbracket X\rrbracket \setminus \{0\}$ und es
gilt: $R\llbracket X\rrbracket^x = \{ f = \sum_{i=0}^\infty a_i x^i
\in R\llbracket X\rrbracket : a_0 \in R^x \}$

\item Ist $R = K$ Körper, so ist $\mathcal(m) \defeqr K\llbracket
X\rrbracket \setminus K\llbracket X\rrbracket^x = \{ \sum a_i x^i :
a_0 = 0\}$ Ideal in $K\llbracket X\rrbracket$

\sbew{0.9}{(a), (b), (d) $\chk$\begin{enumerate}
\item[(c)] ''$\subseteq$'': Sei $f = \sum a_i x^i \in R\llbracket
X\rrbracket^x$. Dann gibt es $g = \sum b_i x^i \in R\llbracket
X\rrbracket$ mit $1=fg = a_0 b_0 + (a_1 b_0 + a_0 b_1)x + \dots \Ra
a_0 \in R^x$
\newline ''$\supseteq$'': Definiere $g = \sum b_i x^i$ rekursiv durch
$b_0 = a_0^{-1}, b_i \defeqr a_0^{-1} \cd \sum_{k=1}^i (-1)^k
a_k b_{i-k},\; i \geq 1$. Dann ist $fg = 1$
\newline\sbsp{0.9}{$i=1:b_i=a_0^{-1}(a_1 b_0$)}
\end{enumerate}
}\end{enum}
\end{DefBem}