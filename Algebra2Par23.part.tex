\section{Der Hilbert'sche Nullstellensatz}

\begin{Satz}[Hilbert'scher Nullstellensatz]
\label{Satz5}
  Sei $K$ ein Körper und $\mathcal{M}$ ein maximales Ideal in $K[X_1, \dots,
  X_n]$. Dann ist $L \defeqr K[X_1, \dots, X_n] / \mathcal{M}$ eine algebraische
  Körpererweiterung von $K$.
\end{Satz}

\begin{Bew}
  Für $n=1$ ist das aus Algebra I bekannt. Nimm dass als Induktionsanfang einer
  vollständigen Induktion nach $n$.
  $L$ wird als $K$-Algebra erzeugt von den Restklassen $x_1, \dots, x_n$ der
  $X_1, \dots, X_n$. Wenn $x_1, \dots, x_n$ algebraisch über $K$ sind, so auch
  $L$. Wir nehmen an, dass sei nicht der Fall, sei also $O\!\!E$ $x_1$ transzendent
  über $K$.\\
  Da $L$ Körper, liegt $K' \defeqr K(x_1)$ in $L$, so dass $L \subset K'[X_2,
  \dots, X_n]$ ein Faktorring von $K'[X_2, \dots, X_n]$ nach einem maximalen
  Ideal ist.\\
  $\overset{IV}{\Rightarrow} x_2, \dots, x_n$ sind algebraisch über $K'
  \Rightarrow \exists a_{i \nu} \in K'=K(x_1)$ mit $x^{n_i} + \sum_{\nu =
  0}^{n_i -1} a_{i \nu} x_i^{\nu} = 0$ für $i = 2, \dots, n$.
  Nennen wir den Hauptnenner der $a_{i \nu}$ von nun an $b \in K[X_1] \Rightarrow
  x_2, \dots, x_n$ sind ganz über $K[x_1, b^{-1}] \defeql R$.\\
  \textbf{Beh.:} $R$ ist Körper.\\
  \textbf{denn:} Sei $a \in R \setminus \{0\}$ und $a^{-1}$ das Inverse von $a$
  in $L$. Da $L$ ganz über $R$ ist, gibt es $\alpha_0, \dots, \alpha_{m-1} \in
  R$ mit $(a^{-1})^m + \sum_{i = 0}^{m-1} \alpha_i (a^{-1})^i = 0 \overset{
  \cdot a^m}{\Rightarrow} 1 = -\sum_{i=0}^{m-i} \alpha_i a^{m-1} = a
  (-\sum_{i=0}^{m-1} \alpha_i a^{m-2}) \Rightarrow R$ ist Körper $\Rightarrow$
  Widerspruch! $R$ kann niemals Körper sein.
\end{Bew}

\begin{Def}
  Sei $I \trianglelefteq K[X_1, \dots, X_n]$ ein Ideal. Dann heißt die Teilmenge
  $V(I) \subseteq K^n$, die durch $V(I) \defeqr \{(x_1, \dots, x_n) \in K^n:
  f(x_1, \dots, x_n) = 0 \; \forall f \in I\}$ bestimmt ist, die
  \emp{Nullstellenmenge}\index{Nullstellenmenge} von $I$ in $K^n$.
\end{Def}

\begin{nnBsp} 
  \begin{enumerate}
    \item[1.)] aus der LA bekannt: affine Unterräume des $K^n$ sind
               Nullstellenmenge von linearen Polynomen.
    \item[2.)] Anschaulicher Spezialfall von 1.):\\
               Punkte in $K^n: (x_1, \dots, x_n): V(X_1-x_1, X_2 - x_2, \dots,
               X_n - x_n)$.
  \end{enumerate}
\end{nnBsp}

\begin{BemDef}
  \begin{enumerate}
    \item \label{2.12a}Für 2 Ideale $I_1 \subseteq I_2$ gilt $V(I_1) \supseteq V(I_2)$.
    \item Definiert man für eine beliebige Teilmenge $V \subseteq K^n$ das                                         \emp{Verschwindungsideal}\index{Verschwindungsideal} von $V$ durch $I(V) \defeqr \{ f \in K[X_1,         \dots, X_n]: f(x_1, \dots, x_n) = 0 \; \forall (x_1, \dots, x_n) \in V\}$, so gilt $V \subseteq          V(I(V))$; ist $V$ bereits Nullstellenmenge $V(I)$ eines Ideals $I$ von $K[X_1, \dots, X_n]$, so          gilt sogar $V = V(I(V))$.
  \end{enumerate}
\end{BemDef}

\begin{Bew}
  \begin{enumerate}
    \item Sei $x \in V(I_2) \Rightarrow f(x) = 0 \; \forall f \in I_2 \supseteq I_1 \Rightarrow x \in V(I_1)$
    \item ''$\subseteq$'': Definition von $V$ und $I$\\
          ''$\supseteq$'': Sei $V = V(I)$ für $I \trianglelefteq K[X_1, \dots, X_n]$.
	  Nach Definition $I \subseteq I(V) \overset{\ref{2.12a}}{\Rightarrow} V(I(V)) \subseteq V(I) = V$
  \end{enumerate}
\end{Bew}

\begin{nnSatz}[Schwacher Nullstellensatz]
\label{SatzSchwach}
  Ist $K$ algebraisch abgeschlossenen, so ist für jedes echte Ideal $I \trianglelefteq K[X_1, \dots, X_n]$
   \[
   V(I) \not= \emptyset.
   \]
\end{nnSatz}

\begin{Bew}
  Sei $I \trianglelefteq K[X_1, \dots, X_n]$ echtes Ideal. Nach Algebra I gibt es dann maximales Ideal $\mathcal{M} \supseteq I$. Weiter gilt: $V(\mathcal{M}) \subseteq V(I)$, so können wir $O\!\!E$ annehmen, dass $I = \mathcal{M}$ maximal ist.
  Nach Satz \ref{Satz5} ist $K[X_1, \dots, X_n]/\mathcal{M}$ eine algebraische Körpererweiterung von $K$.
  Da $K$ algebraisch abgeschlossen $\Rightarrow K[X_1, \dots, X_n]/\mathcal{M} \cong K$.
  Seien nun $x_i$ die Restklasse von $X_i$ in $K[X_1, \dots, X_n]/\mathcal{M}$ und $x = (x_1, \dots, x_n)$.
  Für $f \in K[X_1, \dots, X_n]$ ist $f(x) = f(\bar{X_1}, \dots, \bar{X_n}) = \bar{f} \mbox{ mod } I \Rightarrow f(x) = 0\ \forall f \in I \Rightarrow x \in V(I)$.
\end{Bew}

\begin{nnSatz}[Starker Nullstellensatz]
  Ist $K$ algebraisch abgeschlossen, so gilt für jedes Ideal $I \trianglelefteq K[X_1, \dots, X_n]: I(V(I)) = \{ f \in K[X_1, \dots, X_n]: \exists d \ge 1: f^d \in I \} \defeql \sqrt[d]{I}$.
\end{nnSatz}

\begin{Bew}[Rabinovitsch-Trick]
  Sei $g \in I(V(I))$ und $f_1, \dots, f_m$ Idealerzeuger von $I \trianglelefteq K[X_1, \dots, X_n]$.\\
  Zu zeigen: $\exists d > 1 \mbox{ mit } g^d = \sum_{i = 1}^m a_i f_i$ für irgendwelche $a_i$.
  Sei 
  \[
  J \subseteq K[X_1, \dots, X_n, X_{n+1}]
  \]
  das von $f_1, \dots, f_m, gX_{n+1}-1$ erzeugtem Ideal.\\
  \textbf{Beh.:} $V(J) = \emptyset$\\
  \textbf{Bew.:} Sei $x = (x_1, \dots, x_n, x_{n+1}) \in V(J)$.
  Dann ist $f_i(x') = 0$ für $x' = (x_1, \dots, x_n)$ und $i = 1, \dots, m \Rightarrow x' \in V(I)$.
  Nach Wahl von $g (\in I(V(I)))$ ist also $g(x') = 0 \Rightarrow (gX_{n+1}-1)(x) = g(x') X_{n+1} - 1 = -1 \not= 0$.\\
  Nach schwachen Nullstellensatz ist $J = K[X_1, \dots, X_n,X_{n+1}] \Rightarrow \exists b_1,
  \dots, b_m$ und $b \in K[X_1, \dots, X_{n+1}]$ mit $\sum_{i=1}^m b_i f_i + b(gX_{n+1} - 1) = 1$.\\
  Sei $R \defeqr K[X_1, \dots, X_{n+1}]/ (gX_{n+1} - 1) \cong K[X_1, \dots, X_n][\frac{1}{g}]$. Unter dem Isomorphismus werden die $f_i$ auf sich selbst, die $b_i$ auf $\tilde{b_i} \in R$ abgebildet $\Rightarrow \sum_{i = 1}^m \tilde{b_i} f_i = 1 \mbox{ in } R$.
  Multipliziere mit dem Hauptnenner $g^d$ der $\tilde{b_i} \Rightarrow \sum_{i = 1}^m \underset{\in K[X_1, \dots, X_n]}{\underbrace{(g^d \tilde{b_i})}} f_i = g^d \Rightarrow I(V(I)) \subseteq \sqrt[d]{I}$.\\
  ''$\supseteq$'': klar.
\end{Bew}
