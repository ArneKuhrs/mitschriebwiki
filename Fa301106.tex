\documentclass[a4paper,11pt]{book}

\usepackage{amssymb}
\usepackage{amsmath}
\usepackage{amsfonts}
\usepackage{ngerman}
%\usepackage{graphicx}
\usepackage{fancyhdr}
\usepackage{euscript}
\usepackage{makeidx}
\usepackage{hyperref}
\usepackage[amsmath,thmmarks,hyperref]{ntheorem}
\usepackage{enumerate}
\usepackage{url}
\usepackage{mathtools}
\usepackage[arrow, matrix, curve]{xy}
%\usepackage{pst-all}
%\usepackage{pst-add}
%\usepackage{multicol}

\usepackage[latin1]{inputenc}

%%Zahlenmengen
%Neue Kommando-Makros
\newcommand{\R}{{\mathbb R}}
\newcommand{\C}{{\mathbb C}}
\newcommand{\N}{{\mathbb N}}
\newcommand{\Q}{{\mathbb Q}}
\newcommand{\Z}{{\mathbb Z}}
\newcommand{\K}{{\mathbb K}}
\newcommand{\ssL}{{\mathcal L}}
\newcommand{\sn}[1]{||#1||_{\infty}}
\newcommand{\eps}{{\varepsilon}}
\newcommand{\begriff}[1]{\textbf{#1}} %das sollte man noch ändern!
\newcommand{\eb}{\begin{flushright} \rule{1ex}{1ex} \end{flushright}}
\newcommand{\ind}{1\hspace{-0,9ex}\raisebox{-0,2ex}{1}}


% Seitenraender
\textheight22cm
\textwidth14cm
\topmargin-0.5cm
\evensidemargin0,5cm
\oddsidemargin0,5cm
\headheight14pt

%%Seitenformat
% Keine Einrückung am Absatzbeginn
\parindent0pt

\DeclareMathOperator{\unif}{Unif}
\DeclareMathOperator{\var}{Var}
\DeclareMathOperator{\cov}{Cov}


\def\AA{ \mathcal{A} }
\def\PM{ \EuScript{P} } 
\def\EE{ \mathcal{E} }
\def\BB{ \mathfrak{B} } 
\def\DD{ \mathcal{D} } 
\def\NN{ \mathcal{N} } 

% Komische Symbole
\def\folgt{\ensuremath{\implies}}
\newcommand{\folgtnach}[1]{\ensuremath{\DOTSB\;\xRightarrow{\text{#1}}\;}}
\def\equizu{\ensuremath{\iff}}
\def\d{\mbox{d}}
\def\fs{\stackrel{f.s.}{\rightarrow }}

%Nummerierungen
\newtheorem{Def}{Definition}[chapter]
\newtheorem{Sa}[Def]{Satz}
\newtheorem{Lem}[Def]{Lemma}
\newtheorem{Kor}[Def]{Korollar}
\theorembodyfont{\normalfont}
\newtheorem{Bsp}[Def]{Beispiel}
\newtheorem{Bem}[Def]{Bemerkung}
\theoremsymbol{\ensuremath{_\blacksquare}}
\theoremstyle{nonumberplain}
\newtheorem{Bew}[Def]{Beweis}
\setcounter{chapter}{1}
\setcounter{section}{4}
\setcounter{Def}{71}

% Kopf- und Fusszeilen
\pagestyle{fancy}
\fancyhead[LE,RO]{\thepage}
\fancyfoot[C]{}
\fancyhead[LO]{\rightmark}

\title{29.11.06}
\author{Das \texttt{latexki}-Team\\[8 cm]}

\date{Stand: \today}
\begin{document}

%1.5
\section{Standardkonstruktionen}
\textbf{A) Produkte}\\
Seien $X,Y$ nVRe. Dann:
\[
X \times Y = \{ (x,y): x \in X, y \in Y \} \text{ist ein nVR bzgl.}
\]
\[
||(x,y)||_p = \left\{
\begin{array}{cl}
(||x||_X^p + ||y||_Y)^{\frac1{p}} ,& 1 \leq p < \infty \\
\max\{ ||x||_X,||y||_Y\} ,& p = \infty
\end{array} \right.
\]
Diese Normen sind alle "aquivalent.\\
Sind $X,Y$ vollst"andig, dann ist $(X \times Y, ||\cdot||_p)$ ein BR.

%Definition ohne Nummer
\begin*{Def}
Sei $Z$ ein nVR und $P \in B(Z)$ mit $P = P^2$. Dann hei"st $P$ \begriff{Projektion}.
\end*{Def}
Hier ist die kanonische Projektion auf $X$ gegeben durch $P(x,y) = (x,0)$.\\
\\
\textbf{B) Diskrete Summe}

%Def 1.72
\begin{Def}
Seien $X_1,X_2$ abg. UVRe eines BRes $X$ mit $X_1 + X_2 = X$ und $X_1 \cap X_2 = \{ 0 \}$. Dann ist $X$ die \begriff{direkte Summe} von $X_1$ und $X_2$. Mann schreibt $X = X_1 \oplus X_2$.\\
$X_2$ hei"st dann \begriff{Komplement} von $X_1$ in $X$.
\end{Def}

%Lemma 1.73
\begin{Lem}
Sei $X$ ein BR und $P \in B(X)$ eine Projektion. Dann ist $Q = I-P \in B(X)$ auch eine Projektion und es gelten $R(P) = N(Q) =: X_1,\ N(P) = R(Q) =: X_2,\ X = X_1 \oplus X_2.$ Man hat $||P|| \geq 1$, wenn $P \not= 0.$
\end{Lem}

\begin{Bew}
$Q^2 = I - 2P + P^2 = I - P = Q$.\\
Falls $y = Px$ f"ur ein $x \in X \Longrightarrow Qy = Px - P^2x = 0.$\\
Falls $Qx = 0$ f"ur ein $x \in X \Longrightarrow x - Px = 0 \Longleftrightarrow x = Px \Longrightarrow x \in R(P) \Longrightarrow R(P) = N(Q)$.
Also ist $X_1 = N(Q) = R(P)$ abgeschlossen (1.16). Genauso: $X_2$.\\
Ist $x \in X \Longrightarrow x = Px + (I-P)x \in X_1 \oplus X_2$. Wenn $x \in X_1 \cap X_2 \Longrightarrow Px = 0$ und $x = Py$ f"ur ein $y \in X \Longrightarrow x = P^2 y = Px = 0 \Longrightarrow X_1 \cap X_2 = \{ 0 \}$. Schlie"slich: $||P|| = ||P^2|| \leq ||P||^2 \Longrightarrow ||P|| \geq 1$,\ falls $P \not= 0$.
\end{Bew}

Umkehrung:\\
Sei $X = X_1 \oplus X_2$. Dann existiert f"ur jedes $x \in X$ eindeutig bestimmte $x_1 \in X_1, x_2 \in X_2$ mit $x = x_1 + x_2$. Setze $Px = x_1$. Dann ist $P$ linear und $P=P^2$. Ferner ist $P$ stetig nach dem Homomorphiesatz (Kap. 3). Somit ist die Existenz direkter Zerlegungen und Projektionen "aquivalent.

%Beispiel 1.74
\begin{Bsp}
\begin{enumerate}

\item[a)] $X = L^1(\R).\ Pf := \ind_{\R^+} \cdot f,\ f \in X \Rightarrow P \in B(X), ||P|| = 1, P = P^2$. Ferner: $(I-P)f = \ind_{(-\infty,0)} \cdot f.$ Die Abbildung $J: R(P) \rightarrow L^1(\R^+),\ Jf = f_{|\R^+}$ ist stetig und linear mit stetiger Inverser
\[
J^{-1}g = \left\{
\begin{array}{cl}
g ,& \text{auf } \R^+ \\
0 ,& \text{auf } (-\infty,0)
\end{array} \right.
\]
$\Rightarrow R(P) \equiv L^1(\R^+)$. Entsprechend: $N(P) \equiv L^1(\R_-) \Rightarrow L^1(\R) \equiv L^1(\R^+) \oplus L^1(\R_-)$

\item[b)] $c_0$ hat kein Komplement in $\ell^{\infty}$ (Werner, IV 6.5)

\item[c)] $X = \R^2, P = \begin{pmatrix} 1 & t \\ 0 & 0 \end{pmatrix},\ t \in \R. P^2 = P$ und $||P|| = 1 + |t|$ bzgl. $||\cdot||_1.\ P$ ist Projektion auf $x-$Achse.
\end{enumerate}
\end{Bsp}

\textbf{Quotienten}\\
Seien $X$ nVR, $Y$ ein UVR.
\[
X_{/_Y} = \{ \hat{x} = x + Y,\ x \in X \} \quad \text{\begriff{Quotientenraum}}
\]
Die Quotientenabbildung $\Pi: X \rightarrow X_{/_Y}, \Pi X = \hat{X}$ ist wohldefiniert, linear und surjektiv. Man schreibt codim $Y = \dim X_{/_Y}.$ Es gilt $N(\Pi) = Y$. Definiere $||\hat{x}|| = \inf_{y \in Y} ||x-y|| := d(x,Y) \quad$ \begriff{Quotientennorm}. Gilt $\overline{x}+Y = x + Y$ f"ur gewisse $x,\overline{x} \in X$, dann gilt: $\overline{x}-x \in Y \Rightarrow d(x,Y) = d(\overline{x},Y) \Rightarrow$ Quotientennorm wohldefiniert.\\
Sei $\alpha \in \K \backslash \{0\}$. Dann:
\[
||\hat{\alpha x}|| = \inf_{y \in Y} ||\alpha x -\frac{\alpha}{\alpha} y|| = |\alpha| \inf_{z \in Y} ||x-z|| = |\alpha| \cdot ||\hat{x}||
\]
Seien $x_1,x_2 \in X$ und $\eps > 0$. Dann ex. $y_1,y_2 \in Y$ so, dass $||x_k - y_k|| \leq ||\hat{x_k}|| + \eps,\ k=1,2. \Rightarrow ||\hat{x_1}+\hat{x_2}|| = \inf_{y \in Y} ||x_1+x_2-y|| \leq ||x_1-y_1+x_2-y_2|| \leq ||\hat{x_1}|| + ||\hat{x_2}|| + 2 \eps \stackrel{\eps \rightarrow 0}{\Longrightarrow} ||\hat{x_1}+\hat{x_2}|| \leq ||\hat{x_1}|| + ||\hat{x_2}|| \Rightarrow ||\hat{x}||$ ist ein Halbnorm auf $X_{/_Y}$.\\
Sei nun $Y$ abgeschlossen. Ist $||\hat{x}|| = 0$, dann ex $y_n \in Y$ mit $||x - y_n|| \rightarrow 0 \ (n \rightarrow \infty).$ Da $Y$ abg $\Rightarrow x \in Y \Rightarrow \hat{x}=0$ und $X_{/_Y}$ ist nVR.\\
Weiter: $||\Pi(x)|| = ||\hat{x}|| \leq ||x||_X \Longrightarrow \Pi \in B(X,X_{/_Y})$ mit $||\Pi|| \leq 1$.

%Satz 1.75
\begin{Sa}
Sei $X$ ein BR und $Y$ ein abg UVR von $X$. Dann ist $(X_{/_Y},||\cdot||)$ ein BR ($||\cdot||$ Quotientennorm) und $\Pi \in B(X, X_{/_Y}),\ ||\Pi|| = 1.$
\end{Sa}

\begin{Bew}
Sei $\overline{x}$ wie in Lemma 1.51. Dann gilt: $||\Pi|| \geq ||\Pi \overline{x}|| = \inf_{y \in Y} ||\overline{x}-y|| \geq 1 - \delta$ f"ur ein bel $\delta \in (0,1) \stackrel{\delta \rightarrow 0}{\Rightarrow} ||\Pi|| = 1.$\\
Sei $(\hat{x_n})_{n \in \N}$ eine CF in $X_{/_Y}$. Dann ex eine Teilfolge $(\hat{x}_{n_k})_{k \in \N}$ mit $||\hat{x}_m-\hat{x}_{n_k}|| \leq 2^{-k}\ (\ast)$ f"ur alle $x \geq n_k$. Dann ex $y_{n_k} \in Y$ mit $||x_{n_{k+1}}-x_{n_k}-y_{n_k}|| \leq 2 \cdot 2^{-k}$. Setze $v_N = x_{n_1} + \sum_{k=1}^N z_k$, wobei $z_k = x_{n_{k+1}}-x_{n_k}-y_{n_k}.\ X$ BR $\Rightarrow \exists\, x:= \lim_{N \rightarrow \infty} v_n \in X$. Weiter gilt:
\[
v_N = x_{n_{N+1}} - \underbrace{\sum_{k=1}^N y_{n_k}}_{\in Y} \stackrel{\Pi \text{ stetig}}{=} \underbrace{\hat{v}_N}_{\hat{x}_{n_{N+1}}} \rightarrow \hat{x} \text{ in } X_{/_Y}
\]
Beachte: $\hat{v}_N = \hat{x}_{n_{N+1}}$. Wie in Th 1.42 folgt aus $(\ast)$, dass $\hat{x}_n \rightarrow \hat{x}$.
\end{Bew}

%Beispiel 1.76
\begin{Bsp}
\begin{enumerate}
\item[a)] Sei $X = Y \oplus Z,\ X$ BR. Setze $J: Z \rightarrow X_{/_Y},\ Jz = \hat{z} = z+Y \Rightarrow J$ ist linear und stetig. Sei $Jz = 0 \Rightarrow \hat{z} = 0 \Rightarrow z \in Y \stackrel{z \in X}{\Rightarrow} z \in X \cap Y \Rightarrow z = 0.$ F"ur alle $\hat{x} \in X_{/_Y}$ ex. $y \in Y, z \in Z$ mit $x = z+y$, also: $\hat{x} = Jz \Rightarrow J$ ist surjektiv. Mit dem Homomorphiesatz (Kap. 3) folgt: $J^{-1}$ ist steitg $\Rightarrow Z \cong X_{/_Y}$, z.B. $L^1(\R_+) \cong L^1(\R)_{/_{L^1(\R_+)}}$.\\
Beachte: $\ell_{/_{c_0}}^{\infty}$ kann nicht mit einem Teilraum von $\ell^{\infty}$ identifiziert werden, d.h. C) ist allgemeiner als B).

\item[b)] $c = c_0 \cdot \C$ mit Projektion $Px = x - \ell(x) \ind \ (\ell(x)=\lim_{n \rightarrow \infty} x_n)$, also ist $c_{/_{c_0}} \cong \C$, d.h. codim $c_0 = 1$.\\
Beachte: Mit anderem Isomorphismus gilt: $c \cong c_0$ (nach Bsp 1.71)
\end{enumerate}
\end{Bsp}

%Satz 1.77
\begin{Sa}
Sei $X$ nVR und $Y \subseteq X$ ein abg. UVR. Sei $T \in B(X)$ mit $TY = \{ Ty: y \in Y \} \subseteq Y$.\\
Dann def. $\hat{T} \hat{x} := Tx$ einen Operator $\hat{T} \in B(X_{/_Y},X)$ mit $||\hat{T}|| \leq ||T||$.
\end{Sa}

\begin{Bew}
Sei $\hat{x} = \hat{u} \Rightarrow x-u \in Y$. Dann: $T(x-u) \in Y$\\
...Vorlesungsende, Beweis in der n"achste Stunde fertig...
\end{Bew}

\end{document}