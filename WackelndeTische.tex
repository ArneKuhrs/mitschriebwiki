\documentclass{amsart}

\usepackage{german}
\usepackage[latin1]{inputenc}

\begin{document}

%
% � 2005 Joachim Breitner
% Bitte keine �nderungen vornehmen, alle Rechte vorbehalten. (soll hei�en: frag halt bevor ihr es irgendwie verwendet)
%


\title[Wackelfreie Tischpolsitionierung]{Wackelfreie Positionierung von vierbeinigen rechteckigen Tischen auf stetigen Fl�chen}
\author{Joachim Breitner}
%\email{mail@joachim-breitner.de}
%\urladdr{http://www.joachim-breitner.de/}
\dedicatory{Gewidmet dem Gartentisch in der Bothestr. 30, Heidelberg}

\begin{abstract}

Ein sich allj�hrlich wiederholendes Problem ist die Positionierung eines Tisches auf unebenem Untergrund, mit dem Ziel der Wackelfreiheit. Wackelfreiheit ist definiert als der Zustand, in dem jeder Fu� des Tisches den Boden ber�hrt. Es gibt verschiedene Methoden, diesen Zustand zu erreichen. Zum Teil ben�tigen diese Hilfsmittel, beispielweisweise Bierdeckel zum F�llen des leeren Raumes zwischen Boden und Tischbeinen oder Hosts�gen zum Verk�rzen von zu langen Tischbeinen.

Alternativ wird versucht, durch das Probieren verschiedener Positionen des Tisches eine zu finden, in der Wackelfreiheit gew�hrleistet ist. Beliebt ist dabei der Spezialfall des Drehens, und eben dieser wird hier mit den Mitteln der Mathematik untersucht.

\end{abstract}


\maketitle

\section{Vor�berlegungen}
\subsection{Boden}
Gegeben sei der Boden $B$ sowie der Tisch $T$. Unsere �berlegungen finden unter der Vorrausetzung eines stetigen Bodens statt, wir k�nnen also die H�he des Bodens als eine reellwertige Funktion zweier Koordinaten annehmen, diese sei $bh(x,y)$. Allerdings untersuchen wir ja das Finden einer wackelfreien Position allein durch Drehung, also interessiert uns nur das Verhalten des Bodens auf einer Kreislinie, genauer gesagt, auf der Kreislinie, den die Fu�spitzen beschreiben, sobald man den Tisch dreht. Wir untersuchen also die Funktion $h:\mathbb{R} \to \mathbb{R}$, die jedem Winkel (im Bogenma�) die H�he des Bodens an diesem Punkt auf der Kreisline zuordnet. Diese ist -- als Verkettung der stetigen Bodenh�hensfunktion $bh$ mit der Koordinatenumwandlungsfunktion -- auch stetig. Dar�ber hinaus ist diese Funktion zyklisch mit der Periode $2\pi$.

\subsection{Tisch}
Des weiteren muss die Eigenschaft "`Wackelfreiheit"' in mathematische Begriffe gefasst werden. Halten wir zuerst fest, dass der Boden unter den Beinen des rechteckigen, vierbeinigen Tisches, sofern man ein Bein auf den Punkt mit dem Winkel 0 setzt, die H�hen $h(0)$, $h(0+d)$, $h(\pi)$ sowie $h(\pi + d)$ hat mit $d \in (0,\pi)$. Der Tisch steht nun wackelfrei, wenn die Tischbeine alle den Boden ber�hren m�ssen, also die Tischbeinspitzenh�he mit der Bodenh�he am jeweiligen Punkt �bereinstimmen. Denkt man sich Achsen zwischen jeweils gegen�berliegenden Tischbeinspitzen, so schneiden sie sich in ihrem Mittelpunkt. Daraus ergibt sich f�r Wackelfreiheit am Punkt $x$ die folgende Formel:
$$ \frac{1}{2}(h(x) + h(x+\pi)) = \frac{1}{2}( h(x+d) + h(x+\pi+d)) \qquad (*) $$

\subsection{Einschr�nkungen}
Wir vernachl�ssigen hier die radiale Positions�nderung der Tischbeinspitzen bei einer starken Neigung des Tisches, was allerdings unter realistischen Bedingungen (geringe  Bodenh�henunterschiede bei gro�em Beinabstand) vertretbar ist. Ebenso wird die Tischbeinspitzenform vernachl�ssigt.


\section{Beweis f�r quadratische Tische}

Betrachten wir zuerst den Beweis f�r den Spezialfall eines quadratischen Tisches, da dieser mit einfacherer Mathematik auskommt. In dem Fall ist $d=\frac{1}2\pi$.

Wir formen die Bedingung $(*)$ um und erhalten:
$$ \underbrace{h(x) - h(x+\frac{1}{2}\pi) + h(x+\pi)) - h(x+\frac{3}{2}\pi)}_{:= f(x)} = 0$$
Die so definierte Funktion $f:\mathbb{R}->\mathbb{R}$ ist offensichtlich stetig. Betrachten wir nun $f$ am Punkt $x_0$. Ist die Funktion an diesem Punkt gleich Null, so ist die Bedingung $(*)$ erf�llt und der Tisch ist wackelfrei. Im anderen Fall ist sie ungleich Null:
$$ f(x_0) \ne 0 $$
Wir k�nnen o.B.d.A annehmen:
$$ f(x_0) > 0 $$
Betrachten wir nun $-f(x_0)$:
\begin{eqnarray*}
-f(x_0) & =& -(h(x_0) - h(x_0+\frac{1}{2}\pi) + h(x_0+\pi) - h(x_0+\frac{3}{2}\pi)) \\
        & =& -( h(x_0+2\pi) - h(x_0+\frac{1}{2}\pi) + h(x_0+\pi) - h(x_0+\frac{3}{2}\pi) )\\
        & =&-(- h(x_0+\frac{1}{2}\pi) + h(x_0+\pi) - h(x_0+\frac{3}{2}\pi) +h(x_0+2\pi))\\
        & =& h(x_0+\frac{1}{2}\pi) - h(x_0+\pi) + h(x_0+\frac{3}{2}\pi) - h(x_0+2\pi) \\
	&=& f(x_0+ \frac{1}{2}\pi) < 0 
\end{eqnarray*}

Aus $f(x_0) > 0 $ und $f(x_0 +\frac{1}2 \pi ) < 0$ folgt dann nach dem Zwischenwertsatz f�r stetige reellwertige Funktionen:
$$ \exists\xi \in[x_0,x_0+\frac{1}2\pi]: f(\xi) = 0$$
Womit gezeigt ist, dass eine wackelfreie Positoin f�r jeden Quadratischen, vierbeinigen Tisch auf einer stetigen Ebene existiert.

\section{Beweis f�r rechtwinkelige Tische}

Betrachten wir nun wieder den allgemeinen Fall, mit $d\in(0,\pi)$. Wieder formen wir die Bedingung $(*)$ um und erhalten:
$$ \underbrace{h(x) - h(x+d) + h(x+\pi)) - h(x+\pi+d}_{:= f(x)} = 0$$
Es sei $H=\int_0^{2\pi}h(x)dx$ und wegen der Periodiz�t von $h$ auch $H=\int_0^{2\pi}h(x+c)$, $c\in\mathbb{R}$. Nun Integrieren wir $f$ zwischen 0 und $2\pi$:

\begin{eqnarray*}
\int_0^{2\pi}f(x)dx & =& \int_0^{2\pi} h(x)dx - \int_0^{2\pi}h(x+d)dx + \int_0^{2\pi}h(x+\pi)dx - \int_0^{2\pi}h(x+\pi+d)dx \\
	& = & H - H + H - H \\
        & =& 0
\end{eqnarray*}

Da $f$ stetig ist, k�nnen wir den Mittelwertsatz der Integralrechnung anwendunden, der besagt, dass es ein $\xi \in [0,2\pi]$ gibt mit 
$$ \underbrace{\int_0^{2\pi}f(x)dx}_{=0} = f(\xi)\underbrace{(2\pi - 0)}_{\ne 0} \Rightarrow f(\xi) = 0 $$
Wir haben also wieder einen Punkt (besser gesagt: einen Winkel) $\xi$, so dass $f(\xi)=0$, der Tisch also an dieser Position nicht wackelt.

\section{Nachbemerkungen}

Diese Ergebnis ist von au�erordentlichem praktischem Wert, schlie�t er doch jegliche Diskussionen beim nachmittaglichen Kaffee und Kuchen �ber den Sinn oder Unsinn dieser Entwacklungsmethode mit mathematischer Gewissheit aus.

Dem interessierten Leser seien weitere �berlegungen nahe gelegt: Wie uneben muss der Boden sein, damit die Annahme, die Neigung des Tisches wirkt sich nicht relevant aus, nicht mehr gilt? Gilt der Beweis auch f�r 5- oder mehrbeinige Tische? Was passiert mit vierbeinigen Tischen anderer Form als der Rechteckform? Auch auf diese Fragen wird die Mathematik passende Antworten haben.
\end{document}
