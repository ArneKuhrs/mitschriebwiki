\documentclass{article}
\newcounter{chapter}
\setcounter{chapter}{11}
\usepackage{ana}

\title{Extremwerte unter Nebenbedingungen}
\author{Pascal Maillard}
% Wer nennenswerte Änderungen macht, schreibt euch bei \author dazu

\setlength{\parindent}{0pt}
\setlength{\parskip}{2ex}

\begin{document}
\maketitle

\begin{definition}
\indexlabel{Einschränkung einer Funktion}
Seien $M,N$ Mengen $\ne \emptyset,\ f:M\to N$ eine Funktion und $\emptyset \ne T \subseteq M$. Die Funktion $f_{|_T}: T \to N,\ f_{|_T}(x) := f(x)\ \forall x \in T$ heißt die \textbf{Einschränkung} von $f$ auf $T$.
\end{definition}

In diesem Paragraphen gelte stets: $\emptyset \ne D \subseteq \MdR^n,\ D$ offen, $f \in C^1(D,\MdR),\ p \in \MdN,\ p<n$ und $\varphi = (\varphi_1,\ldots,\varphi_p) \in C^1(D,\MdR^p)$. Es sei $T:=\{x\in D: \varphi(x) = 0\} \ne \emptyset.$

\begin{definition}
\indexlabel{lokales Extremum!unter einer Nebenbedingung}
$f$ hat in $x_0\in D$ ein \textbf{lokales Extremum unter der Nebenbedingung $\varphi = 0$} $:\equizu x_0 \in T$ und $f_{|_T}$ hat in $x_0$ ein lokales Extremum.
\end{definition}

Wir führen folgende Hilfsfunktion ein: Für $x=(x_1,\ldots,x_n) \in D$ und $\lambda = (\lambda_1,\ldots,\lambda_p) \in \MdR^p$ gilt: $$H(x,\lambda) := f(x) + \lambda\cdot\varphi(x) = f(x) + \lambda_1\varphi_1(x) + \cdots + \lambda_p\varphi_p(x)$$

Es ist $$H_{x_j} = f_{x_j} + \lambda_1\frac{\partial\varphi_1}{\partial x_j} + \cdots \lambda_p\frac{\partial\varphi_p}{\partial x_j}\ (j=1,\ldots,n),\ H_{\lambda_j} = \varphi_j$$

Für $x_0 \in D$ und $\lambda_0 \in \MdR^p$ gilt:

$H'(x_0,\lambda_0) = 0 \equizu f'(x_0) + \lambda_0\varphi'(x_0) = 0$ und $\varphi(x_0) = 0$\\
$\equizu f'(x_0) + \lambda_0\varphi'(x_0) = 0$ und $x_0 \in T$ (I)

\begin{satz}[Multiplikationenregel von Lagrange]
\indexlabel{Multiplikator}
$f$ habe in $x_0\in D$ eine lokales Extremum unter der Nebenbedingung $\varphi=0$ und es sei Rang $\varphi'(x_0) = p$. Dann existiert ein $\lambda_0 \in \MdR^p$ mit: $H'(x_0,\lambda_0) = 0$ ($\lambda_0$ heißt \textbf{Multiplikator}).
\end{satz}

\begin{folgerung}
$T$ sei beschränkt und abgeschlossen. Wegen 3.3 gilt: $\exists a,b \in T: f(a) = \max f(T),\ f(b) = \min f(T).$ Ist Rang $\varphi'(\underset{b}{a}) = p \folgt \exists \lambda_0 \in \MdR^p: H'(\underset{b}{a},\lambda_0) = 0$.
\end{folgerung}

\begin{beweis}
Es ist $x_0 \in T$ und

$$\varphi'(x_0) = 
\underbrace{
\left(
\begin{array}{ccc|}
\frac{\partial \varphi_1}{\partial x_1}(x_0) & \cdots & \frac{\partial \varphi_1}{\partial x_p}(x_0)\\
\vdots &  & \vdots\\
\frac{\partial \varphi_p}{\partial x_1}(x_0) & \cdots & \frac{\partial \varphi_p}{\partial x_p}(x_0)\\
\end{array}
\right.
}_{=:A}
\left.
\begin{array}{cc}
\cdots & \frac{\partial \varphi_1}{\partial x_n}(x_0)\\
 & \vdots\\
\cdots & \frac{\partial \varphi_p}{\partial x_n}(x_0)\\
\end{array}
\right)$$

Rang $\varphi'(x_0) = p \folgt$ o.B.d.A.: $\det A \ne 0.$

Für $x=(x_1,\ldots,x_n) \in D$ schreiben wir $x=(y,z)$, wobei $y=(x_1,\dots,x_p),\ z=(x_{p+1},\ldots,x_n).$ Insbesondere ist $x_0=(y_0,z_0)$. Damit gilt: $\varphi(y_0,z_0) = 0$ und $\det \frac{\partial \varphi}{\partial y}(y_0,z_0) \ne 0$.

Aus 10.1 folgt: $\exists$ offene Umgebung $U \subseteq \MdR^{n-p}$ von $z_0,\ \exists$ offene Umgebung $V \subseteq \MdR^p$ von $y_0$ und es existiert $g \in C^1(U,\MdR^p)$ mit:

\begin{itemize}
\item[(II)] $g(z_0)=y_0$
\item[(III)] $\varphi(g(z),z) = 0\ \forall z \in U$
\item[(IV)] $g'(z_0) = -(\frac{\partial \varphi}{\partial y}\underbrace{(g(z_0),z_0)}_{=x_0})^{-1}\frac{\partial \varphi}{\partial z}\underbrace{(g(z_0),z_0)}_{=x_0}$
\end{itemize}

(III) $\folgt (g(z),z) \in T\ \forall z \in U$. Wir definieren $h(z)$ durch
$$h(z) := f(g(z),z)\ (z \in U)$$

Dann hat $h$ in $z_0$ ein lokales Extremum (\emph{ohne} Nebenbedingung). Damit gilt nach 8.1:

$$0=h'(z_0) \gleichnach{5.4} f'(g(z_0),z_0)\cdot\left(
\begin{array}{c}
g'(z_0)\\
I
\end{array}\right) = \left(
\begin{array}{c|c}
\frac{\partial f}{\partial y}(x_0) & \frac{\partial f}{\partial z}(x_0)
\end{array}\right) \cdot \left(
\begin{array}{c}
g'(z_0)\\
I
\end{array}\right) = \frac{\partial f}{\partial y}(x_0) g'(z_0) + \frac{\partial f}{\partial z}(x_0)$$

$$\gleichnach{(IV)} \underbrace{\frac{\partial f}{\partial y}(x_0) \left(-\frac{\partial \varphi}{\partial y}(x_0)\right)^{-1}}_{=: \lambda_0 \in \MdR^p} \frac{\partial \varphi}{\partial z}(x_0) + \frac{\partial f}{\partial z}(x_0) \folgt \frac{\partial f}{\partial z}(x_0) + \lambda_0 \frac{\partial \varphi}{\partial z}(x_0) = 0\text{ (V)}$$

$\ds{\lambda_0 = \frac{\partial f}{\partial y}(x_0) \left(-\frac{\partial \varphi}{\partial y}(x_0)\right)^{-1} \folgt \frac{\partial f}{\partial y}(x_0) + \lambda_0 \frac{\partial \varphi}{\partial y}(x_0) = 0}$ (VI)

Aus (V),(VI) folgt: $f'(x_0) + \lambda_0\varphi'(x_0) = 0 \folgtnach{(I)} H'(x_0,\lambda_0) = 0.$

\end{beweis}

\begin{beispiel}
$(n=3,p=2)\ f(x,y,z) = x+y+z,\ T:=\{(x,y,z) \in \MdR^3: x^2+y^2=2,\ x+z=1\},\ \varphi(x,y,z) = (x^2+y^2-2,x+z-1).$

Bestimme $\max f(T),\ \min f(T)$. Übung: $T$ ist beschränkt und abgeschlossen $\folgtnach{3.3} \exists a,b \in T: f(a) = \max f(T),\ f(b) = \min f(T)$.

$$\varphi'(x,y,z) = \left(\begin{array}{ccc}
2x & 2y & 0\\
1  & 0  & 1
\end{array}\right)$$

Rang $\varphi'(x,y,z) = 1 < p=2 \equizu x=y=0.\ a,b\in T \folgt$ Rang $\varphi'(a) =$ Rang $\varphi'(b) = 2$

\def\shouldbe{\overset{!}{=}}

$H(x,y,z,\lambda_1,\lambda_2) = x+y+z+\lambda_1(x^2+y^2-2) + \lambda_2(x+z-1)$\\
\begin{tabbing}
$H_x=1+2\lambda_1x+\lambda_2$ \= $\shouldbe 0$ (1)\\
$H_y=1+2\lambda_1y          $ \> $\shouldbe 0$ (2)\\
$H_z=1+\lambda_2            $ \> $\shouldbe 0$ (3)\\
$H_{\lambda_1}=x^2+y^2-2    $ \> $\shouldbe 0$ (4)\\
$H_{\lambda_2}=x+z-1        $ \> $\shouldbe 0$ (5)
\end{tabbing}

(3) $\folgt \lambda_2=-1 \folgtnach{(1)} 2\lambda_1x=0$; (2) $\folgt \lambda_1\ne0 \folgt x=0 \folgtnach{(5)} z=1$; (4) $\folgt y = \pm \sqrt{2}$

11.2 $\folgt a,b \in \{(0,\sqrt{2},1),(0,-\sqrt{2},1)\}$

$f(0,\sqrt{2},1) = 1+\sqrt{2} = \max f(T);\ f(0,-\sqrt{2},1) = 1-\sqrt{2} = \min f(T)$

\end{beispiel}

\paragraph{Anwendung}
Sei $A$ eine reelle, \emph{symmetrische} $(n\times n)$-Matrix. Beh: $A$ besitzt einen reellen EW.

\begin{beweis}
$f(x) := x\cdot(Ax) = Q_A(x)\ (x \in \MdR^n),\ T:= \{x \in \MdR^n: ||x||=1\} = \partial U_1(0)$ ist beschränkt und abgeschlossen.

$\varphi(x) := ||x||^2-1 = x\cdot x-1;\ \varphi'(x) = 2x,\ f'(x) = 2Ax$.

3.3 $\folgt \exists x_0 \in T: f(x_0) = \max f(T);\ \varphi'(x) = 2(x_1,\ldots,x_n);\ x_0 \in T \folgt$ Rang $\varphi'(x_0) = 1\ (=p)$

11.2 $\folgt \exists \lambda_0 \in \MdR: H'(x_0,\lambda_0) = 0;\ h(x,\lambda) = f(x)+\lambda\varphi(x);\ H'(x,\lambda) = 2Ax+2\lambda x$

$\folgt 0 = 2(Ax_0+\lambda_0 x_0) \folgt Ax_0 = (-\lambda_0) x_0,\ x_0 \ne 0 \folgt -\lambda_0$ ist ein EW von $A$.
\end{beweis}
\end{document}
