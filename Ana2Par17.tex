\documentclass{article}
\newcounter{chapter}
\setcounter{chapter}{17}
\usepackage{ana}

\title{Quadrierbare Mengen}
\author{Joachim Breitner, Pascal Maillard und Wenzel Jakob}

\begin{document}
\maketitle

Sei $A \subseteq \MdR^n$. A hei�t \begriff{quadrierbar} (qb) $:\equizu 1_A \in L(\MdR^n)$ $(\equizu 1\in L(A))$

In diesem Fall hei�t $v_n(A) := \int_{\MdR^n} 1_A dx = \int_A1dx$ das $n$-dimensionale \begriff{Volumen} oder \begriff{Lebesguema�} von $A$.  
\textbf{Beachte:} $v_n(A) \in \MdR$

\begin{satz}
Sei $A\subseteq\MdR^n$ beschr�nkt. Ist $A$ offen oder abgeschlossen, dann ist $A$ quadrierbar.
\end{satz}
\begin{beweis}
16.11, 16.12
\end{beweis}

\begin{beispiele}
\item $\emptyset$ ist quadrierbar und $v_n(\emptyset) = 0$.
\item Sei $Q$ ein Quader im $\MdR^n \folgt 1_Q \in \T_n \subseteq L(\MdR^n) \folgt Q$ ist quadrierbar und $n$-dimensionale Volumen von oben gleich dem $n$-dimensionalen Volumen aus �15
\item Sei $\emptyset \ne D \subseteq \MdR^n$, $D$ beschr�nkt und abgeschlossen, $f\in C(D,\MdR)$ und $f\ge 0 $ auf $D$. $A:=\{(x,y) \in \MdR^{n+1}: x\in D, 0 \le y \le f(x)\}$. $A$ ist beschr�nkt und abgeschlossen $\folgtnach{17.1}$ $A$ ist quadrierbar (im $\MdR^{n+1}$). $v_{n+1}(A) = \int_A 1 d(x,y) \gleichnach{16.3} \int D \left(\int_0^{f(x)} 1 dy\right) dx = \int_D f(x)dx$
\item $A:=\{(x,y)\in\MdR^n: x^2+y^2\le r^2\} \ (r>0)$. $A=\overline{U_r(0)}$. $A$ ist beschr�nkt und abgeschlossen $\folgtnach{17.1}$ $A$ ist quadrierbar. $v_2(A) = \int_A 1 dx$. F�r $x\in[-r,r]$: $A_x=[-\sqrt{r^2-x^2}, \sqrt{r^2-x^2}] \folgt v_n(A) = \int _{-r}^r \left( \int _{-\sqrt{r^2-x^2}}^{\sqrt{r^2-x^2}} 1 dy \right) dx = \int_{-r}^r 2 \sqrt{r^2-x^2}dx \gleichnach{AI} \pi r^2$.
\end{beispiele}

\begin{satz}
$A,B,A_1,\ldots,A_m$ seien $\subseteq \MdR^n$ und quadrierbar.
\begin{liste}
\item $A \cap B, A \cup B, A\backslash B$ sind quadrierbar und\\ $v_n(A\cup B) = v_n(A)+v_n(B)-v_n(A\cap B)$.
\item Aus $A\subseteq B$ folgt $v_n(A) \le v_n(B)$.
\item $A_1 \cup A_2 \cup \ldots \cup A_m$ ist quadrierbar und\\ $v_n(A_1 \cup A_2 \cup \ldots \cup A_m)\le v_n(A_1)+\cdots+v_n(A_m)$
\end{liste}
\end{satz}

\begin{beweise}
\item $1_{A\cap B} = 1_A \cdot 1_B \folgtnach{16.5} 1_{A \cap B} \in L(\MdR^n) \folgt A\cup B$ ist quadrierbar. \\
      $1_{A\cup B} = 1_A + 1_B -1_{A\cap B} \folgtnach{16.5} 1_{A \cup B} \in L(\MdR^n) \folgt A\cup B$ ist quadrierbar. $v_n(A\cup B) = \int_{\MdR^n}1_{A\cup B} dx = \int_{\MdR^n}1_A dx + \int_{\MdR^n}1_B dx - \int_{\MdR^n}1_{A\cap B}dx = v_n(A) + v_n(B) - v_n(A\cap B)$.\\
      $1_{A\backslash B} = 1_A(1-1_B) \folgtnach{16.5} 1_{A\backslash B}\in L(\MdR^n) \folgt A\backslash B$ ist quadrierbar.
\item $A\subseteq B \folgt 1_A \le 1_B$ auf $\MdR^n \folgt v_n(A) = \int_{\MdR^n}1_A dx \le \int_{\MdR^n}1_B dx = v_n(B)$
\item folgt aus (1) mit Induktion
\end{beweise}

\begin{satz}[Prinzip von Cavalieri]
Sei $A\subseteq \MdR^n\times \MdR = \MdR^{n+1} = \{(x,z): x\in\MdR^n, z\in\MdR\}$ beschr�nkt und abgeschlossen (also quadrierbar im $\MdR^{n+1}$). Dann:
\begin{liste}
\item $\forall z\in\MdR$ ist $A_z$ beschr�nkt und abgeschlossen (also quadrierbar im $\MdR^n$).
\item $v_{n+1}(A) = \int_\MdR v_n(A_z)dz$
\end{liste}
\end{satz}

\begin{beweise}
\item �bung
\item $v_{n+1}(A) = \int_A 1 d(x,z) \gleichnach{16.3} \int_\MdR \underbrace{ \left( \int_{A_z} 1dx\right)} _{=v_n(A_z)} dz$. 
\end{beweise}

\begin{beispiele}
\item $A:=\{(x,y,z)\in\MdR^3: z\in[0,1], x^2+y^2\le z^2\}$. $A$ ist beschr�nkt und abgeschlossen $\folgt$ $A$ ist quadrierbar. F�r $z\notin [0,1]: A_z = \emptyset$. F�r $z \in [0,1]: A_z = \{(x,y)\in \MdR^n : x^2+y^2\le z^2\} \folgt v_2(A_z) = \pi z^2 \folgt v_3(A) = \int_0^1 \pi z^2 dz = \frac\pi 3$
\item Sei $[a,b]\subseteq \MdR$, $f\in C([a,b],\MdR)$ und $f\ge 0$ auf $[a,b]$. Graph von $f$ rotiert um die $x$-Achse $\longrightarrow$ Rotationsk�rper $A$. $A=\{(x,y,z)\in\MdR^3: x\in[a,b], y^2+z^2\le f(x)^2\}$. $A_x = \{ (y,z)\in\MdR^2 : y^2+z^2\le f(x)^2\}$ f�r $x\in[a,b]$. $v_2(A_X) = \pi f(x)^2$. $v_3(A)=\pi \int_a^b f(x)^2 dx$.

\textbf{Speziell:} $f(x) = \sqrt{r^2-x^2}$ ($r>0$), $x\in[-r,r]$.\\ Rotationsk�rper $A=\overline{U_r(0)} \subseteq \MdR^3$. $v_3(A) = \pi\int _{-r}^r (r^2-x^2)dx = \frac 4 3 \pi r^3$.
\end{beispiele}

\begin{definition}
Sei $N\subseteq \MdR^n$. $N$ hei�t eine \begriff{Nullmenge} genau dann, wenn $F$ quadrierbar und $v_n(N) = 0$ ist.
\end{definition}

\begin{satz}
Sei $N\subseteq \MdR^n$. $N$ ist eine Nullmenge $\equizu$ $\|1_N\|_1 = 0 $.
\end{satz}

\begin{beweis}
\glqq$\Rightarrow$\grqq{}: $N$ Nullmenge $\folgt 1_N \in L(\MdR^n)$. 16.5 $\folgt \|1_N\|_1 = \int_{\MdR^n}1_N dx  = v_n(N) = 0$.
\glqq$\Leftarrow$\grqq{}: Setze $(\varphi_k) := (0,0,0,\ldots)$; $(\varphi_k)$ ist eine Folge in $\T_n$: $\|1_N - \phi_k\|_1 = \|1_N\|_1 = \|1_N\|_1 \gleichnach{Vor.} 0 \ \forall k\in\MdN \folgt 1_N \in L(\MdR^n)$ und $\int 1_Ndx = \lim \int \varphi_k dx = 0 \folgt N$ ist quadrierbar und $v_n(N)=0$.
\end{beweis} 


\begin{satz}
$N,N_1,N_2,\ldots$ seien Nullmengen im $\MdR^n$.
\begin{liste}
\item Ist $M\subseteq N \folgt M$ ist eine Nullmenge.
\item $\displaystyle\bigcup_{k=1}^\infty N_k$ ist eine Nullmenge.
\end{liste}
\end{satz}

\begin{beweise}
\item $1_M \le 1_N \folgtnach{16.1} \|1_M\|_1 \le \|1_N\|_1 \gleichnach{17.4} 0 \folgt \|1_M\|_1 = 0 \folgt $ Beh.
\item $A:= \bigcup_{k=1}^\infty N_k$; $1_A \le \sum_{k=1}^\infty 1_{N_K} \folgtnach{16.1} \|1_A\|_1 \le \sum_{k=1}^\infty \|1_{N_k}\|_1 \gleichnach{17.4} 0 \folgt$ Beh.
\end{beweise}

\begin{beispiele}
\item Sei $x_0=(x_1,\ldots,x_n)\in\MdR^n$, $N:=\{x_0\} = \{x_1\}\times\{x_2\}\times\cdots\times\{x_0\}$. $N$ ist ein Quader, also quadrierbar, $v_n(N) = 0$, $N$ ist eine Nullmenge.
\item Beispiel (1) und 17.5(2) liefern: Jede abz�hlbare Teilmenge des $\MdR^n$ ist eine Nullmenge % Falsch, wenn nicht Ines da w�re.
\item Ist $Q = [a_1,b_1]\times[a_2,b_2]\times[a_3,b_3]\times \cdots \times [a_n,b_n] \subseteq\MdR^n$. $Q$ ist eine Nullmenge $\equizu$ $a_j=b_j$ f�r ein $j\in\{1,\ldots,n\}$.
\end{beispiele}

\begin{satz}
Sei $\emptyset \ne D \subseteq \MdR^n,\ D$ sei beschr�nkt und abgeschlossen, es sei $f\in C(D,\MdR)$ und $G_f:=\{(x,f(x)):x\in D\}\subseteq\MdR^{n+1}.$ Dann ist $G_f$ eine Nullmenge im $\MdR^{n+1}$.
\end{satz}

\begin{beweis}
$G_f$ ist beschr�nkt und abgeschlossen $\folgtnach{17.1} G_f$ ist qb. $v_{n+1}(G_f) = \int_{G_f}1d(x,y) \gleichnach{16.13} \int_D\left(\int_{f(x)}^{f(x)}1dy\right) dx = 0.$
\end{beweis}

\begin{definition}
\indexlabel{fast �berall}
Sei $A\subseteq\MdR^n$ und (E) eine Eigenschaft, welche die Elemente von $A$ betrifft. (E) gilt \textbf{fast �berall (f.�.)} auf $A :\equizu \exists$ Nullmenge $N\subseteq A$ mit: (E) gilt f�r alle $x\in A\backslash N$.
\end{definition}

\begin{beispiel}
$f:A\to\tilde\MdR$ sei eine Funktion. $f=0$ f.�. auf $A \equizu \exists$ Nullmenge $N\subseteq A:f(x)=0\ \forall x\in A\backslash N$.
\end{beispiel}

\begin{satz}
\begin{liste}
\item $f,g:\MdR^n\to\tilde\MdR$ seien Funktionen mit $f=g$ f.�. auf $\MdR^n$. Dann: $f\in L(\MdR^n) \equizu g\in L(\MdR^n).$ I. d. Fall: $\int fdx = \int gdx$.
\item Seien $A,B \subseteq\MdR^n,\ f\in L(A)\cap L(B)$ und $A\cap B$ sei eine Nullmenge. Dann: $f\in L(A\cup B)$ und $\int_{A\cup B}fdx = \int_A fdx + \int_B fdx$.
\end{liste}
\end{satz}

\begin{beweis}
\begin{liste}
\item $\exists$ Nullmenge $N\subseteq\MdR^n:f(x)=g(x)\ \forall x\in\MdR^n\backslash N$. Sei $f\in L(\MdR^n) \folgt \exists$ Folge $(\varphi_k)$ von Treppenfunktionen mit: $||f-\varphi_k||_1 \to 0$ und $\int fdx = \lim_{k\to\infty} \int\varphi_k dx.$

$f_k:=1_N\ (k\in N),\ h:=\sum_{k=1}^\infty f_k,\ ||h||_1 \overset{\text{16.1}}{\le} \sum_{k=1}^\infty ||f_k||_1 \gleichnach{17.4} 0 \folgt ||h||_1 = 0.$

Es ist $|g-\varphi_k|\le|f-\varphi_k|+h$ auf $\MdR^n \folgtnach{16.1} ||g-\varphi_k||_1\le||f-\varphi_k||_1+||h||_1 = ||f-\varphi_k||_1 \folgt ||g-\varphi_k||_1 \to 0 \folgt g\in L(\MdR^n)$ und $\int gdx = \lim\int\varphi_k dx = \int fdx.$

\item Wegen (1) o.B.d.A: $f=0$ auf $A\cap B$. Dann: $f_{A\cup B} = f_A+f_B \overset{\text{16.5}}\in L(\MdR^n) \folgt f\in L(A\cup B)$ und $\int_{A\cup B} fdx = \int_{\MdR^n} f_{A\cup B} dx = \int_{\MdR^n} f_A dx + \int_{\MdR^n} f_B dx = \int_A fdx + \int_B fdx.$
\end{liste}
\end{beweis}

\begin{satz}
$f:\MdR^n\to\tilde\MdR$ sei eine Funktion.
\begin{liste}
\item Ist $||f||_1<\infty$ und $N:=\{x\in\MdR^n:f(x)=\infty\} \folgt N$ ist eine Nullmenge. Dies ist z.B. der Fall, wenn $f\in L(\MdR^n)$ (16.5: $||f||_1 = \int|f|dx$)

\item $||f||_1=0\equizu f=0$ f.�. auf $\MdR^n$.
\end{liste}
\end{satz}

\begin{beweis}
\begin{liste}
\item Sei $\ep>0: 1_N\le\ep|f|$ auf $\MdR^n \folgtnach{16.1} ||1_N||_1 \le \ep||f||_1 \folgtwegen{\ep\to0} ||1_N||_1=0 \folgtnach{17.4}$ Beh.
\item "`$\Rightarrow$"': F�r $k\in N:N_k := \{x\in\MdR^n:|f(x)|\ge \frac{1}{k}\}$. Dann: $1_{N_k}\le k|f|$ auf $\MdR^n \folgtnach{16.1} ||1_{N_k}||_1\le k||f||_1=0 \folgtnach{17.4} N_k$ ist eine Nullmenge $\folgtnach{17.5} N:= \bigcup_{k=1}^\infty N_k$ ist eine Nullmenge. Es ist $N=\{x\in\MdR^n:f(x)\ne0\} \folgt f=0$ f.�. auf $\MdR^n$.

"`$\Leftarrow$"': $|f|=0$ f.�. auf $\MdR^n \folgtnach{17.7} |f|\in L(\MdR^n)$ und $\int|f|dx = \int0dx = 0.$ 16.5 $\folgt ||f||_1 = \int|f|dx = 0.$
\end{liste}
\end{beweis}

\begin{definition}
\indexlabel{Figur}
Seien $Q_1,Q_2,\ldots,Q_m$ \emph{abgeschlossene} Quader im $\MdR^n$ und $A:=Q_1\cup Q_2\cup\ldots\cup Q_m.$ Dann hei�t $A$ eine \textbf{Figur}.
\end{definition}

\begin{satz}
Sei $U\subseteq\MdR^n$ offen. Dann ex. Figuren $A_1,A_2,\ldots$ mit $A_1\subseteq A_2\subseteq \ldots$ und $U=\bigcup_{k=1}^\infty A_k$. Ist $U$ qb $\folgt v_n(U) = \ds{\lim_{k\to\infty}}v_n(A_k) = \sup\{v_n(A_k):k\in\MdN\}.$
\end{satz}

\def\Q{\ensuremath{\mathscr{Q}}}

\begin{beweis}
F�r $a\in\MdQ^n$ und $r\in\MdQ^+$ sei $W_r(a)$ wie im Beweis von 16.10.

$\Q:=\{W_r(a):a\in\MdQ^n,\ r\in\MdQ^+,\ W_r(a)\subseteq U\},\ U$ offen $\folgt \Q\ne\emptyset$.

Es ist $\Q=\{Q_1,Q_2,\ldots\},\ A_k:=Q_1\cup Q_2\cup\ldots\cup Q_k\ (k\in\MdN).\ (A_k)$ leistet das Verlangte.

Sei $U$ qb. $\varphi_k:=1_{A_k}\ (k\in\MdN) \folgt \varphi_k\in\T_n,\ \varphi_1\le\varphi_2\le\ldots$ auf $\MdR^n$ und $\varphi_k(x)\to1_U(x)\ \forall x\in\MdR^n$ und $\varphi_1\le\varphi_k\le1_U$ auf $\MdR^n \folgt \int\varphi_1dx\le\int\varphi_kdx\le\int1_Udx=v_n(U).$ 16.7 $\folgt \lim\underbrace{\int\varphi_kdx}_{v_n(A_k)} = \int1_udx = v_n(U).$
\end{beweis}

\begin{satz}
Sei $U\subseteq\MdR^n$ offen. Dann ex. Quader $Q_1,Q_2,\ldots \subseteq\MdR^n$ mit: $$U=\bigcup_{k=1}^\infty Q_k\text{ und }Q_k^\circ\cap Q_j^\circ = \emptyset\ (j\ne k)$$

Ist $U$ qb $\folgt v_n(U)=\sum_{k=1}^\infty v_n(Q_k).$
\end{satz}

\begin{beweis}
In der gr. �bung.
\end{beweis}

\begin{satz}
Sei $N\subseteq\MdR^n.\ N$ ist eine Nullmenge $\equizu \forall\ep>0\ \exists$ Quader $Q_1,Q_2,\ldots$ im $\MdR^n$ mit: (*) $N\subseteq\bigcup_{k=1}^\infty Q_k$ und $\sum_{k=1}^\infty v_n(Q_k)<\ep.$
\end{satz}

\begin{beweis}
"`$\Leftarrow$"': Sei $\ep>0$. Seien $Q_1,Q_2,\ldots$ wie in (*). Dann: $1_N\le\sum_{k=1}^\infty 1_{Q_k}$ auf $\MdR^n \folgtnach{16.1} ||1_N||_1\le\sum_{k=1}^\infty||1_{Q_k}||_1 = \sum_{k=1}^\infty\int1_{Q_k}dx = \sum_{k=1}^\infty v_n(Q_k) < \ep \folgt ||1_N||_1 = 0 \folgtnach{17.4} N$ ist eine Nullmenge.

"`$\Rightarrow$"': Sei $\ep>0$. Es gen�gt z.z.: $\exists$ offene Menge $U$ mit: $N\subseteq U,\ U$ ist qb und $v_n(U)<\ep$ (wegen 17.10).

$||2\cdot1_N||_1 = 2||1_N||_1 \gleichnach{17.4}0 \folgt \exists \Phi\in\H(2\cdot1_N):I(\Phi)<\ep.$ Sei $\Phi=\sum_kc_k1_{R_k}$, wobei $c_k\ge0,\ R_k$ offene Quader. O.B.d.A: $\Phi=\sum_{k=1}^\infty c_k1_{R_k}$.

$\varphi_m:=\sum_{k=1}^mc_k1_{R_k}\in\T_n;\ \varphi_1\le\varphi_2\le\ldots\le\Phi;\ \varphi_m(x)\to\Phi(x)\ \forall x\in\MdR^n.\ \int\varphi_1dx\le\int\varphi_mdx = \sum_{k=1}^mc_kv_n(R_k) \overset{m\to\infty}{\to} \sum_{k=1}^\infty c_kv_n(R_k) = I(\Phi) < \ep.$

16.7 $\folgt\Phi\in L(\MdR^n)$ und $\int\Phi dx = \lim\int\varphi_m dx = I(\Phi) < \ep.$

$U:=\{x\in\MdR^n:\Phi(x)>1\}.\ x\in N \folgt \Phi(x) \ge 2\cdot1_N(x) = 2\folgt x\in U.$ Also: $N\subseteq U.\ U$ offen, $U$ qb, $v_n(U)<\ep$.
\end{beweis}

\begin{folgerung}
Sei $N\subseteq\MdR^n$ eine Nullmenge und $\ep>0$. Dann existiert eine Menge $U\subseteq\MdR^n$: $U$ ist offen, $U$ ist quadrierbar, $N\subseteq U$ und $v_n(U)<\ep$.
\end{folgerung}

\begin{beweis}
Beweis von 17.11
\end{beweis}

\begin{satz}
Sei $A\subseteq\MdR^n$ beschr"ankt und abgeschlossen, $f:A\to\MdR$ sei beschr"ankt und fast "uberall stetig auf A. Dann: $f\in L(A)$.
\end{satz}

\begin{beweis}
$\exists \gamma\ge 0: |f|\le\gamma$ auf A. $\exists$ Nullmenge $N\subseteq A$: $f$ ist stetig auf $A\ \backslash\ N$. Sei $\ep>0$. 17.12 $\folgt\exists$ offene und quadrierbare Menge $U$ mit $N\subseteq U, v_n(U)<\ep$. $A\ \backslash\ U\subseteq A\ \backslash\ N$, $f$ stetig auf $A\ \backslash\ U$, $A\ \backslash\ U$ ist beschr"ankt und abgeschlossen. 16.12 $\folgt f\in L(A\ \backslash\ U)\folgt f_{A\backslash U}\in L(\MdR^n)\folgt \exists \varphi\in\T_n:\|f_{A\backslash U}-\varphi\|_1\le\ep$. Es ist $|f_A - f_{A\backslash U}|\le\gamma\cdot 1_U$ auf $\MdR^n$. $\overset{16.1}{\folgt}\|f_A-f_{A\backslash U}\|_1\le\gamma\|1_U\|_1\gleichnach{16.5}\gamma\int 1_U\text{d}x=\gamma v_n(U)<\gamma\ep$. Dann:
\begin{eqnarray*}
\|f_A-\varphi\|_1 & = & \|f_A-f_{A\backslash U}+f_{A\backslash U}-\varphi\|_1 \\
&\le&\|f_A-f_{A\backslash U}\|_1+\|f_{A\backslash U}-\varphi\|_1\\
&\le&\gamma\ep+\ep\\
&=&(\gamma+1)\ep
\end{eqnarray*}
D.h: $\forall k\in\MdN\ \exists\varphi_k\in\T_n:\|f_A-\varphi_k\|<\frac{\gamma+1}{k}\folgt f_a\in L(\MdR^n)\folgt f\in L(A)$.
\end{beweis}

\end{document}
