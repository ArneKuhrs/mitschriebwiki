\documentclass[parskip,a4paper,twoside,DIV15,BCOR12mm]{scrbook}
\usepackage{la}

\author{Die Mitarbeiter von \url{http://mitschriebwiki.nomeata.de/}}
\title{Lineare Algebra II}

\makeindex

\begin{document}
\maketitle

\renewcommand{\thechapter}{\Roman{chapter}}
\addcontentsline{toc}{chapter}{Inhaltsverzeichnis}

\tableofcontents

\chapter{Vorwort}

\section{Über dieses Skriptum}
Dies ist ein erweiterter Mitschrieb der Vorlesung \glqq Lineare Algebra II\grqq\ von Prof. Dr. Schmidt im
Sommersemester 2010 an der Universität Karlsruhe (KIT). Die Mitschriebe der Vorlesung werden mit
ausdrücklicher Genehmigung von Prof. Dr. Schmidt hier veröffentlicht, Prof. Dr. Schmidt ist für den
Inhalt nicht verantwortlich.

\section{Wer}
Gestartet wurde das Mitschriebwiki von Joachim Breitner. Beteiligt an diesem Mitschrieb sind
Rebecca Schwerdt und hoffentlich bald noch andere.

\section{Wo}
Alle Kapitel inklusive \LaTeX-Quellen können unter \url{http://mitschriebwiki.nomeata.de} abgerufen werden.
Dort ist ein \emph{Wiki} eingerichtet und von Joachim Breitner um die \LaTeX-Funktionen erweitert worden.
Das heißt, jeder kann Fehler nachbessern und sich an der Entwicklung
beteiligen. Auf Wunsch ist auch ein Zugang über \emph{Subversion} möglich.


\renewcommand{\thechapter}{\arabic{chapter}}
\renewcommand{\chaptername}{§}
\setcounter{chapter}{14}

\chapter{Multilineare Abbildungen und Tensorprodukte}
\section{Bilinearformen}
\begin{definition}
\index{Vektorraumpaarung}\index{Paarung}
\index{Bilinearform}
\index{ausgeartet!Paarung}
Seien $V,W$ $K$-VRme. Ein Abbildung $P:V\times W\to K$ heißt \textbf{\mbox{(Vektorraum-)}Paarung},
falls $P$ in jedem Argument linear ist, d.h. wenn für jedes feste $w_o\in W$
\[P(\cdot,w_0):V\to K, v\mapsto P(v,w_0)\]
und für jedes feste $v_0\in V$
\[P(v_0,\cdot):W\to K, w\mapsto P(v_0,w)\]
eine lineare Abbildung, also Linearform ist.\\
Im Fall $V=W$ heißt $P$ eine \textbf{Bilinearform} auf $V$.\\
Eine Paarung $P$ heißt \textbf{nicht ausgeartet}, wenn für jedes $w_0\in W$ und
für jedes $v_0\in V$ die Abbildung $P(\cdot,w_0)$ bzw. $P(v_0,\cdot)$ nicht die Nullabbildung
ist.
\end{definition}

\begin{comment}
Die Menge $\mathcal{P}(V,W)$ aller Paarungen von $V$ und $W$ ist ein untervektorraum
des $K$-VRms $\Abb(V\times W,K)$ aller Abbildungen von $V\times W$ nach $K$.
\end{comment}

\begin{example}
Auf dem Dualraum $W:=V^*(=\Hom(V,K))$ ist die nicht ausgeartete Paarung
\[P:V\times V^*\to K,(v,f)\mapsto f(v)\]
eine Bilinearform.\\
Für eine Paarung $P:V\times W\to K$ setzen wir $\rho_w(v):=P(v,w)$ und erhalten
so Linearformen $\rho_w\in V^*$ für alle $w\in W$.
\end{example}

\begin{theo}
\begin{enumerate}
\item Die Abbildung $\rho:W\to V^*,w\mapsto \rho_w$ ist ein Homomorphismus
von $K$-VRmen.
\item Die Zuordnung $\eta:P\to \rho$ ist ein Isomorphismus, es gilt:
\[\mathcal{P}(V,W)\stackrel{\sim}{\to} \Hom(W,V^*)\]
\end{enumerate}
\end{theo}

\begin{proof}
\begin{enumerate}
\item Es gilt für alle $\alpha\in K,w_1,w_2\in W$:
\begin{align*}
\rho_{\alpha w_1+w_2}&=(v\mapsto P(v,\alpha w_1+w_2))\\
&=(v\mapsto \alpha P(v,w_1)+P(v,w_2))\\
&=\alpha((v\mapsto P(v,w_1))+(v\mapsto P(v,w_2)))\\
&=\alpha\rho_{w_1}+\rho_{w_2} 
\end{align*}
\item Homomorphie selbst nachrechnen!\\
Die Umkehrabbildung ist:
\[\Hom(W,V^*)\to \mathcal{P}(V,W),\rho\mapsto P:=((v,w)\mapsto(\rho(w))(v))\]
\end{enumerate}
\end{proof}

\begin{remind}
Lineare Abbildungen sind bereits durch ihre Wirkung auf einer Basis eindeutig
bestimmt. Dieses Prinzip gilt auch für Paarungen.
\end{remind}

\begin{comment}
\index{bilinear!Fortsetzung}\index{Fortsetzung!bilineare}
Seien $V,W$ $K$-VRme mit jeweiliger Basis $B:=\{b_1,\ldots,b_m\}\subseteq V,
C:=\{c_1,\ldots,c_n\}\subseteq W$, so ist eine Paarung $P$ auf $V\times W$
Bereits durch ihre Einschränkung auf $B\times C$ festgelegt.\\
Für $v:=\sum_{i=1}^m \alpha_i b_i,w:=\sum_{j=1}^n \beta_j c_j$ gilt:
\[P(v,w)=\sum_{i=1}^m\sum_{j=1}^n \alpha_i\beta_j\cdot P(b_i,c_j)\]
Jede Abbildung $P':B\times C\to K$ definiert über diese Gleichung eine Paarung
$P':V\times W\to K$. Diese heißt \textbf{bilineare Fortsetzung}.
\end{comment}

\begin{definition}
\index{Fundamentalmatrix}
Die Matrix $D_{BC}(P):=(P(b_i,c_j))\in K^{m\times n}$ heißt \textbf{Fundamentalmatrix}
der Paarung $P$ bzgl. der Basen $B$ und $C$. Mit den Kkordinatenvektoren:
\[D_B(v)=\begin{pmatrix}\alpha_1\\\vdots\\\alpha_m\end{pmatrix}\text{ und }
D_C(w)=\begin{pmatrix}\beta_1\\\vdots\\\beta_n\end{pmatrix}\]
gilt nach obiger Gleichung:
\[P(v,w)=D_B(v)^T\cdot D_{BC}(P)\cdot D_C(w)\]
\end{definition}

\begin{theo}
Eine Paarung $P$ endlichdimensionaler $K$-VRme $V,W$ mit Basen $B,C$ ist genau dann
nicht ausgeartet, wenn die Dimensionen von $V$ und $W$ gleich und $D_{BC}(P)$ invertierbar
ist.
\end{theo}

\begin{proof}
Der Beweis erfolgt durch Implikation in beiden Richtungen:
\begin{enumerate}
\item["`$\impliedby$"'] Sei $\dim V=\dim W$ und $F:=D_{BC}(P)$ invertierbar.
Sei nun $w\ne 0\in W$, dann ist $D_C(w)\ne 0$.\\
Daraus folgt, dass auch $F\cdot D_C(w)$ nicht null ist. O.B.d.A sei die
$i$-te Koordinate ungleich null. Dann gilt:
\[P(b_i,w)=e_i^T\cdot F\cdot D_C(w)\ne 0\]
Insbesondere ist $P(\cdot,w)\ne 0$.
Analog folgt $P(v,\cdot)\ne 0$ für alle $0\ne v\in V$. Also ist $P$ nicht ausgeartet.
\item["`$\implies$"'] Sei $P$ nicht ausgeartet, dann ist insbesondere $\rho:W\to V^*,
w\mapsto \rho_w=(v\mapsto P(v,w))$ injektiv. Daraus folgt:
\[\dim V=\dim V^*\ge \dim W\]
Analog gilt:
\[\dim W=\dim W^*\ge \dim V\]
Also haben $V$ und $W$ gleiche Dimension.\\
\textbf{Annahme:} $F$ ist nicht invertierbar.\\
Dann existiert ein $D_C(w)\ne 0$, sodass $F\cdot D_C(w)$ gilt. Daraus folgt
für alle $v\in V$:
\[P(v,w)=D_B(v)^T\cdot F\cdot D_C(w)=0\ \lightning\]
Also ist $P(\cdot,w)=0$, was einen Widerspruch zur nicht Ausgeartetheit von $P$ darstellt.
\end{enumerate}
\end{proof}

\begin{theo}
Seien $B,\hat B$ Basen von $V$, $C,\hat C$ Basen von $W$ und $P$ eine Paarung von
$V$ und $W$. Dann gilt:
\[D_{BC}(P)=D_{\hat BB}(\id_V)^T\cdot D_{\hat B\hat C}(P)\cdot D_{\hat CC}(\id_W)\]
\end{theo}

\begin{proof}
Für $(v,w)\in V\times W$ gilt:
\begin{align*}
P(v,w)&=D_{\hat B}(v)^T\cdot D_{\hat B\hat C}(P)\cdot D_{\hat C}(w)\\
&=(D_{\hat BB}(\id_V)\cdot D_B(v))^T\cdot D_{\hat B\hat C}(P)\cdot(D_{\hat CC}(\id_W)\cdot D_C(w))\\
&=D_B(v)^T \cdot (D_{\hat BB}(\id_V)^T \cdot D_{\hat B\hat C}(P)\cdot D_{\hat CC}(\id_W))\cdot D_C(w)
\end{align*}
Aber es gilt auch:
\[P(v,w)=D_B(v)^T\cdot D_{BC}(P)\cdot D_C(w)\]
Durch einsetzen aller Basispaare $b_i,c_j$ folgt die Behauptung.
\end{proof}

\begin{comment}
Mit der Dualbasis $B^*=\{b_1^*,\ldots,b_m^*\}$ von $V^*$ zu $B$ (erinnere:
$b_k^*(b_i)=\delta_{ik}$) gilt für $\rho=\eta(P):W\to V^*$:
\[\rho(c_j)=\sum_{i=1}^n P(b_i,c_j)\cdot b_i^*\]
D.h. $D_{B^*C}(\rho)=D_{BC}(P)$.
\end{comment}

\begin{proof}
Es gilt:
\begin{align*}
\rho(c_j)&=P(\cdot,c_j)\\
&=(b_i\mapsto P(b_i,c_j))\\
&=\sum_{i=1}^n P(b_i,c_j)\cdot b_i^*
\end{align*}
\end{proof}

\begin{example}
Sei $W=V^*$ und für alle $f\in V^*$ sei $P(v,f)=f(v)$. Nehme nun die Dualbasis 
$C=B*$ zur Basis $B$ von $V$. Dann gilt:
\[P(b_i,b_k^*)=b_k^*(b_i)=\delta{ik}\]
Also ist $D_{BB}(P)=I_m$
\end{example}

Wir spezialisieren nun $W=V$.

\begin{definition}
\index{symmetrisch!Paarung}
\index{Orthogonalbasis (OGB)}
\index{Orthonormalbasis (ONB)}
Sei $P$ eine Paarung von $V$ und $V$.
\begin{enumerate}[(a)]
\item $P$ heißt \textbf{symmetrisch}, falls für alle $v,w\in V$ gilt:
\[P(v,w)=P(w,v)\]
\item Eine Basis $B=\{b_1,\ldots,b_m\}$ heißt \textbf{Orthogonalbasis} (OGB) von $V$
bezüglich $P$, wenn für alle $i\ne j$ gilt:
\[P(b_i,b_j)=0\]
\item Eine Basis $B=\{b_1,\ldots,b_m\}$ heißt \textbf{Orthonormalbasis} (ONB) von $V$
bezüglich $P$, wenn $B$ OGB ist und für alle $i\in\{1,\ldots,m\}$ gilt:
\[P(b_i,b_i)=1\]
\end{enumerate}
\end{definition}

\begin{comment}
Falls eine OGB $B$ existiert, so ist die Fundamentalmatrix $D_{BB}(P)$ diagonal, insbesondere
symmetrisch, also ist $P$ symmetrisch.
\end{comment}

\begin{theo}
Sei $K$ ein Körper mit $1+1\ne 0$ und $P$ eine symmetrische Bilinearform auf einem
$K$-VRm $V$ mit $\dim V=:n<\infty$. Dann existiert eine OGB von $V$ bzgl. $P$.
\end{theo}

\begin{proof}
Der beweis erfolgt durch vollständige Induktion nach der Dimension $n$.\\
Für $n=1$ ist die Behauptung offensichtlich wahr, nehmen wir also als Induktionsvoraussetzung an,
dass sie für $n-1$ erfüllt sei.\\
Da für $P=0$ jede Basis Orthogonalbasis ist, lässt sich im Folgenden o.B.d.A annehmen,
dass $P\ne 0$ ist. Also existieren $v,w\in V$ mit $P(v,w)\ne 0$, es gilt:
\begin{align*}
P(v+w,v+w)&=P(v,v)+P(w,w)+P(v,w)+P(w,v)\\
&= P(v,v)+P(w,w)+2P(v,w)
\end{align*}
Daraus folgt:
\[P(v,v)\ne 0 \vee P(w,w)\ne 0 \vee P(v+w,v+w)\ne 0\]
Also existiert ein $b_1\in V$ mit $P(b_1,b_1)\ne 0$. Nun betrachte:
\begin{align*}
W&:=\{v\in V\mid P(v,b_1)=0\}\\
&=\Kern(P(\cdot,b_1))
\end{align*}
Nach Dimensionsformel ist $\dim W=n-1$ und $V=K\cdot b_1 \oplus W$. Da die 
Einschränkung $P|_{W\times W}$ symmetrisch ist, existiert nach Induktionsvoraussetzung
eine OGB $\{b_2,\ldots,b_n\}$ von $W$ bzgl. $P|_{W\times W}$.\\
Da außerdem für alle $w\in W$ $P(w,b_1)=0$ ist, ist
$\{b_1,\ldots,b_n\}$ OGB von $V$ bzgl. $P$.
\end{proof}

\begin{comment}
\index{Fourierformel}
Die Basisdarstellung bzgl. eine ONB $B$ lautet:
\[v=\sum_{b\in B}P(v,b)\cdot b\]
\end{comment}

\begin{proof}
Leichte Übung!
\end{proof}

\section{Multilineare Abbildungen}
Veralgemeinere nun die Bilinearität und den Zielbereich.

\begin{definition}
\index{multilinear}
Seien $V_1,\ldots,V_n,W$ $K$-VRme und $M:V_1\times\ldots\times V_n\to W$ eine Abbildung.\\
$M$ heißt \textbf{(n-fach) multilinear}, falls für jedes $i\in\{1,\ldots,n\}$ bei fester
Wahl von $v_j\in V_j$ (für alle $j\ne i$) $M(v_1,\ldots,v_{i-1},\cdot,v_{i+1},\ldots,v_n):V_i\to W$
eine lineare Abbildubg ist.
\end{definition}

\begin{example}
Multilineare Abbildungen sind:
\begin{enumerate}
\item Die Determinantenabbildung:
\[\det:K^n\times\ldots\times K^n\to K\]
\item Die Skalarmultiplikation:
\[K\times V\to V,(\lambda,v)\mapsto \lambda\cdot v\]
\item Die Matrizenmultiplikation:
\[K^{p\times q}\times K^{q\times r}\times K^{r\times s}\to K^{p\times s},(A,B,C)\mapsto A\cdot B\cdot C\]
\end{enumerate}
\end{example}

\section{Tensorprodukte}
\begin{definition}
\index{universell!Abbildungseigenschaft (UAE)}\index{Abbildungseigenschaft!universelle (UAE)}
Seien $V,W$ $K$-VRme. Eine $K$-VRm $T$ mit einer bilinearen Abbildung $\tau:V\times W\to T$
heißt ein \textbf{Tensorprodukt} von $V$ und $W$, falls $\tau$ die folgende 
\textbf{universelle Abbildungseigenschaft} (UAE) erfüllt:\\
Zu jedem $K$-VRm $U$ und jeder bilinearen Abbildung $\beta:V\times W\to U$ existiert
genau eine lineare Abbildung $\Phi_\beta:T\to U$ derart, dass $\beta=\Phi_\beta\circ \tau$.\\
Schreibe: $T=:V\otimes_K W,\tau(v,w)=:v\otimes w$
\end{definition}

\begin{comment}
\begin{enumerate}
\item Falls $T$ existiert, so hat man eine Bijektion:
\[\Bil(V\times W,U)\stackrel{\sim}{\to}\Hom(T,U),\beta\mapsto\Phi_\beta\]
\item Sind $(T_1,\tau_1),(T_2,\tau_2)$ Tensorprodukte von $V$ und $W$, so existiert genau 
ein Isomorphismus $\Phi:T_1\to T_2$ mit $\tau_2=\Phi\circ\tau_1$.
\end{enumerate}
\end{comment}

\begin{task}
Beweise die Existenz von Tensorprodukten.
\end{task}

\begin{example}
\begin{enumerate}
\item Sei $V:=K^{n\times 1},W:=K^{m\times 1}, T:=K^{n\times m}$ und die bilineare
Abbildung:
\[\tau:K^n\times K^m\to T,(v,w)\mapsto v\cdot w^T\]
Für die Standartbasen $\{e_i\}\subseteq V,\{e'_j\}\subseteq W$ ist $\tau(e_i,e'_j)=E_{ij}$
die Elementarmatrix. $D:=\{E_{ij}\mid i\in\{1,\ldots,n\},j\in\{1,\ldots,m\}\}$ ist
Basis von $T$.\\
Im folgenden wollen wir die UAE nachweisen. Sei dazu $\beta:V\times W\to U$ bilinear.
Dann erhalten wir eine lineare Abbildung $\Phi:K^{n\times m}\to U$ für jede Vorgabe einer
Abbildung $D\to U$ (vgl. lineare Fortsetzung). Insbesondere also auch für die Vorgabe:
\[\forall i\in\{1,\ldots,n\},j\in\{1,\ldots,m\}:\Phi(E_{ij}):=\beta(e_i,e'_j)\]
Damit gilt dann:
\begin{align*}
\beta(v,w)&=\beta\left(\sum_{i=1}^n \alpha_i e_i,\sum_{j=1}^m \gamma_j e'_j\right)\\
&=\sum_{i,j}\alpha_i\gamma_j\cdot\beta(e_i,e'_j)\\
&=\sum_{i,j}\alpha_i\gamma_j\cdot\Phi(E_{ij})\\
&=\Phi\left(\sum_{ij}\alpha_i\gamma_j\cdot E_{ij}\right)\\
&=\Phi\left(\tau\left(\sum_{i=1}^n\alpha_i e_i,\sum_{j=1}^m \gamma_j e'_j\right)\right)\\
&=\Phi(\tau(v,w))=(\Phi\circ\tau)(v,w)
\end{align*}
Wir haben also gezeigt, dass $\beta=\Phi\circ\tau$ gilt.\\
Falls für ein $\Phi':T\to U$ auch $\beta=\Phi'\circ\tau$ gilt, folgt insbesondere
$\beta(e_i,e'_j)=\Phi'(\tau(e_i,e'_j))$ und damit:
\[\Phi'(E_{ij}=\beta(e_i,e'_j)=\Phi(\tau(e_i,e'_j))=\Phi(E_{ij)}\]
D.h. $\Phi|_D=\Phi'|_D$, also ist $\Phi=\Phi'$ eindeutig.
\item Seien $V,W$ beliebige VRme mit endlichen Dimensionen $\dim V=n,\dim W=m$.\\
Die Existenz des Tensorproduktes folgt etwa durch Koordinatenisomorphismen und
Beispiel 1. Für eine \textbf{koordinatenfreie Konstruktion} nehme $T:=\Hom(V^*,W)$ und
\[\tau:V\times W\to T,(v,w)\mapsto(V^*\to W,f\mapsto f(v)\cdot w)\]
Leichte Übung: $(T,\tau)$ ist Tensorprodukt von $V,W$ und für Basen $B,C$ von $V,W$ gilt:
\[D:=\{b\otimes c\in T\mid b\in B,c\in C\}\]
ist Basis von $T=V\otimes_K W$.
\end{enumerate}
\end{example}

\begin{theo}
Sind $V,W$ beliebige $K$-VRme, so existiert ein Tensorprodukt von $V$ und $W$.
\end{theo}

\begin{proof}
Kommt noch!
\end{proof}

\begin{application}
Das Tensorprodukt wird zur Erweiterung des Skalarbereiches eines VRms genutzt.
Sei $V$ $K$-VRm, $L$ ein Körper mit Teilkörper $K\le L$. Insbesondere ist also
$L$ ein $K$-VRm (vgl. früher). Nach Satz 5 existiert das Tensorprodukt 
$L\otimes_K V=:V_L$ ($K$-VRm).\\
Im Folgenden wollen wir zeigen, dass $V_L$ ein $L$-Vektorraum ist. Dazu fehlt
die Skalarmultiplikation $L\times V_L\to V_L$, die wir mittels der UAE definieren.
Für alle $l\in L$ ist:
\[\beta_l:L\times V\to V_L,(x,v)\mapsto lx\otimes v\]
bilinear, sodass $\beta_l(x,v)=\Phi_{\beta_l}(x\otimes v)$.\\
Nehme nun $\Phi_{\beta_l}$ als Skalarmultiplikation mit $l\in L$:
\[L\times V_L\to V_L,(l,u)\mapsto \Phi_{\beta_l}(u)\]
Leichte Übung: Dies erfüllt die Axiome für eine Skalarverknüpfung.
\end{application}

\begin{comment}
$V_L$ enthält $V$ als $K$-Untervektorraum über die Einbettung:
\[V\to V_L,v\mapsto 1\otimes v\]
Für eine Basis $B$ von $V$ ist das Bild $\{1\otimes b\mid b\in B\}\subseteq V_L$
eine Basis des $L$-VRms $V_L$. Insbesondere ist
\[L\otimes_K K^n\stackrel{\sim}{=}L^n\]
eine Isomorphie von $L$-VRmen.
\end{comment}

\chapter{Metriken, Normen und Skalarprodukte}
\begin{definition}
\index{Metrik}
\index{Abstand}
\index{positivdefinit}
\index{Dreiecksungleichung}
Metriken abstrahieren Abstände. Sei $X$ eine beliebige Menge. Eine Abbildung
$d:X\times X\to \mathbb{R}$ heißt eine \textbf{Metrik} oder ein \textbf{Abstand}
auf $X$, falls gilt:
\begin{enumerate}
\item $d$ ist \textbf{smmetrisch}, d.h. für alle $x,y\in X$ gilt:
\[d(x,y)=d(y,x)\]
\item $d$ ist \textbf{positivdefintit}, d.h. für alle $x,y\in X$ gilt:
\[d(x,y)\ge 0 \text{ und } (d(x,y)=0)\implies (x=y)\]
\item Für alle $x,y,z\in X$ gilt die \textbf{Dreicksungleichung}:
\[d(x,y)+d(y,z)\ge d(x,z)\]
\end{enumerate}
\end{definition}

\setcounter{chapter}{18}
\chapter{Isometrien}

\begin{task}
Studiere alle linearen Abbildungen, die \textbf{Abstände} von Punkten nicht ändern.
Z.B. Drehungen um einen Punkt im $\mathbb{R}^2$.
\end{task}

\section{Charakterisierung und orthogonale Gruppe}
\begin{definition}
\index{Morphismus}\index{Isometrie}\index{Automorphismus}\index{Automorphismengruppe}
Seien $V_1,V_2\ \mathbb{K}$-VRme mit Sesquilinearformen $s_1,s_2$.
\begin{enumerate}[(a)]
\item Ein \textbf{Morphismus von $\mathbb{K}$-VRmen mit Sesquilinearform} ins
$\Phi\in \Hom(V_1,V_2)$ mit:\\
\[\forall x,y\in V_1:s_2(\Phi(x),\Phi(y))=s_1(x,y)\]\\
Schreibe: $\Phi:(V_1,s_1)\to (V_2,s_2)$.
\item Ist $\Phi$ zusätzlich bijektiv, so heißt $\Phi$ eine \textbf{(lineare) Isometrie}.
\item Eine Isometrie $\Phi:(V,s)\to (V,s)$ heißt \textbf{Automorphismus} von s.\\
Die Gruppe $\Aut(s)\le\Aut(V)$ heißt die \textbf{Automorphismengruppe} von s.
\end{enumerate}
\end{definition}

\begin{example}
\index{Lorenzgruppe}
In der Relativitätstheorie wichtig ist die \textbf{Lorenzgruppe} $\Aut(s)$ zu\\
\[s:\mathbb{R}^4 \times \mathbb{R}^4 \to\mathbb{R}, (x,y)\mapsto x^T
\begin{pmatrix}
1&0&\cdots&0\\
0&\ddots&\ddots&\vdots\\
\vdots&\ddots&1&0\\
0&\cdots&0&-c
\end{pmatrix}
y\]\\ 
für $c:=$Lichtgeschwindigkeit.
\end{example}

\begin{definition}
\index{orthogonal!Gruppe}\index{Gruppe!orthogonale}
\index{orthogonal!Abbildung}\index{Abbildung!orthogonale}
\index{unitär!Gruppe}\index{Gruppe!unitäre}
\index{unitär!Abbildung}\index{Abbildung!unitäre}
Im Folgenden sei $s$ stets SKP.\\
\textbf{Fall $\mathbb{K}=\mathbb{R}$:}\\
$O(V,s):=\Aut(s)$ heißt \textbf{orthogonale Gruppe}. Die Elemente der Gruppe
heißen \textbf{orthogonale Abb. bzgl. s}.\\
\textbf{Fall $\mathbb{K}=\mathbb{C}$:}\\
$U(V.s):=\Aut(s)$ heißt \textbf{unitäre Gruppe}. Die Elemente der Gruppe heißen
\textbf{unitäre Abb. bzgl. s}.
\end{definition}

\begin{comment}
Eine wichtige Isometrie ist: abstrakter VRm $\cong$ Standartraum
\end{comment}

\begin{theo}
Sei $V$ VRm mit SKP $s$, $\dim(V)=n$ und ONB $B$. Dann ist die Koordinatendarstellung:\\
\[D_B:(V,s)\to(\mathbb{K}^n,\langle\cdot,\cdot\rangle)\] eine Isometrie.
\end{theo}

\begin{proof}
Sei $B=\{b_1,\ldots,b_n\}, x,y\in V$ mit $x=\sum_{i=1}^n{\alpha_i b_i},
y=\sum_{i=1}^n{\beta_i b_i}$. Dann gilt:
\begin{align*}
s(x,y)&=\sum_{i,j}{\alpha_i \overline{b_j}\cdot s(b_i,b_j)}\\
&= \sum_{i=1}^n{\alpha_i \overline{b_i}}\\
&= \langle
\begin{pmatrix}
\alpha_1\\
\vdots\\
\alpha_n
\end{pmatrix},
\begin{pmatrix}
\beta_1\\
\vdots\\
\beta_n
\end{pmatrix} 
\rangle\\
&= \langle D_B(x),D_B(y)\rangle
\end{align*}
\end{proof}

\begin{comment}
\begin{enumerate}
\index{Längentreue}\index{Winkeltreue}
\item Sei $\Phi: V_1\to V_2$ Morphismus von SKP-Räumen, dann ist $\Phi$ \textbf{längentreu}.
\[\iff\|x\|_1 = \|\Phi(x)\|_2\]
\textbf{Winkeltreue} für $K=\mathbb{R}$ bedeutet: 
\[\frac{\langle x_1,y_1\rangle_1}{\|x_1\|_1\|y_1\|_1} = \frac{\langle\Phi(x_1),
\Phi(y_1)\rangle_2}{\|\Phi(x_1)\|_2\|\Phi(y_1)\|_2}\]
\item $\Phi:(V,s)\to(V,s)$ Endomorphismus von SKP-Räumen und $\dim(V)<\infty
\implies \Phi$ ist Isomorphismus und Automorphismus, also orthogonal und unitär. 
\end{enumerate}
\end{comment}

\begin{theo}[Isometriekriterium]
Sei $V$ VRm mit SKP $s=\langle\cdot,\cdot\rangle$ und sei $\Phi\in\Aut(V)$.\\
Folgende Aussagen sind äquivalent:
\begin{enumerate}
\item $\Phi$ ist Isometrie (d.h. $\Phi\in\Aut(s)$).
\item $\Phi\in\End^a(V)$ und $\Phi^*=\Phi^{-1}$.
\item $\forall x\in V: \|x\|=\|\Phi(x)\|$
\item $\forall y\in V: (\|y\|=1)\implies(\|\Phi(y)\|=1)$ (Einheitssphärenabbildung).
\end{enumerate}
\end{theo}

\begin{proof}
Die Äquivalenz ergibt sich aus folgendem Ringschluss:
\begin{enumerate}
\item[1.$\implies$2.] Es gilt $\forall x,y\in V$, $z:=\Phi(y)$:
\begin{align*}
&\Phi \text{ Isometrie}\\
&\iff \forall x,y\in V:\langle\Phi(x),\Phi(y)\rangle =\langle x,y\rangle\\
&\stackrel{\Phi^{-1}\text{ ex.}}{\iff} \forall x,z\in V:\langle\Phi(x),z\rangle
=\langle x,\Phi^{-1}(z)\rangle
\end{align*}
Nach Definition der Adjungierten folgt daraus $\Phi^{-1} = \Phi^*$.
\item[2.$\implies$3.] Es gilt für alle $x\in V$:
\[\|\Phi(x)\|^2=\langle\Phi(x),\Phi(x)\rangle\stackrel{2.}{=}
\langle x,\Phi^*\Phi(x)\rangle=\langle x,x\rangle\]
\item[3.$\iff$4.] \checkmark
\item[3.$\implies$1.] Es gilt für alle $x,y,\in V,\alpha\in K:$
\begin{align*}
\langle\alpha x+y,\alpha x+y\rangle &= \langle\Phi(\alpha x+y),\Phi(\alpha x+y)\rangle\\
\iff\langle \alpha x,y\rangle +\langle y,\alpha x\rangle &= \langle\Phi(\alpha x),
\Phi(y)\rangle + \langle\Phi(y),\Phi(\alpha x)\rangle\\
\iff\alpha\langle x,y\rangle +\overline{\alpha}\overline{\langle x,y\rangle} &=
\alpha\langle\Phi(x),\Phi(y)\rangle + \overline{\alpha}\overline{\langle\Phi(x),\Phi(y)\rangle}
\end{align*}
\textbf{Fall $K=\mathbb{R}$:}\\ 
Mit $\alpha:=\frac12:\langle x,y\rangle =\langle\Phi(x),\Phi(y)\rangle$\\
\textbf{Fall $K=\mathbb{C}$:}\\ 
Mit $\alpha:=\frac12:\Real\langle x,y\rangle =\Real\langle\Phi(x),\Phi(y)\rangle$\\
Mit $\alpha:=\frac i2:\Imag\langle x,y\rangle =\Imag\langle\Phi(x),\Phi(y)\rangle$
\end{enumerate}
\end{proof}

\begin{corollary}
Sei $\dim(V)=n<\infty$, $B$ ONB von $V$ und $\Phi\in\End(V)$.\\
Folgende Aussagen sind äquivalent:
\begin{enumerate}
\item $\Phi$ ist Isometrie.
\item Es gilt für alle $x\in V:\|\Phi(x)\|=\|x\|$
\item $\Phi(B)$ ist ONB.
\item Es gilt $D_{BB}(\Phi)^{-1} = D_{BB}(\Phi^*)$, d.h. $D_{BB}(\Phi)$ ist unitär bzw. orthogonal.
\item Die Spalten (bzw. Zeilen) von $D_{BB}(\Phi)$ bilden eine ONB von $\mathbb{K}^n$ bzgl.
dem Standard-SKP.
\item Es existiert eine ONB $C$ von $V$ mit $D_{BB}(\Phi)=I_n$.
\end{enumerate}
\end{corollary}

\begin{proof}
Jede der Aussagen impliziert $\Phi\in\Aut(V)$.\\
Sei $B:=\{b_1,\ldots,b_n\}$.
\begin{enumerate}
\item[1.$\iff$2.$\iff$4.] Klar nach Isometriekriterium.
\item[4.$\iff$5.]Es gilt:
\begin{align*}
&D_{BB}(\Phi)^{-1} = D_{BB}(\Phi^*)\\
&	\implies D_{BB}(\Phi)\cdot D_{BB}(\Phi^*) = I_n\\
&\iff\{\text{Zeilen von }\Phi\}\text{ sind ONB bezgl. Standardform}\\
&\implies D_{BB}(\Phi^*)\cdot D_{BB}(\Phi) = I_n\\
&\iff\{\text{Spalten von }\Phi\}\text{ sind ONB bezgl. Standardform}
\end{align*}
\item[3.$\implies$2.]Da für alle $b_i,b_j\in B$ gilt:
\[\langle\Phi(b_i),\Phi(b_j)\rangle =\delta_{ij}=\langle b_i,b_j\rangle\]
Folgt für alle $x=\sum_{i=1}^n{\alpha_i b_i}\in V:$
\begin{align*}
\langle\Phi(x),\Phi(x)\rangle
&=\sum_{i,j}{\alpha_i\overline{\alpha_j}\langle\Phi(b_i),\Phi(b_j)\rangle}\\
&=\sum_{i,j}{\alpha_i\overline{\alpha_j}\langle b_i,b_j\rangle}\\
&=\langle x,x\rangle
\end{align*}
Also ist $\|\Phi(x)\|=\|x\|$ und $\Phi$ längenerhaltend.
\item[1.$\implies$3.] Da $\Phi$ Isometrie ist, gilt:
\[\implies \langle\Phi(b_i),\Phi(b_j)\rangle=\langle b_i,b_j\rangle =\delta_{ij}\]
D.h. $\Phi(B)$ ist ONB.
\item[3.$\implies$6.] Es existiert eine ONB $C=\{c_1,\ldots,c_n\}$, sodass gilt:
\[D_{BC}(\Phi) = I_n\]
Daraus folgt: $D_{BB}(\Phi)=D_{BC}(\Phi)\cdot M_{CB}=M_{CB}=:(\gamma_{ij})$\\
Also gilt für alle $b_j\in B$:
\[b_j=\sum_k{\gamma_{kj}\cdot c_k}\]
Daraus folgt:
\begin{align*}
\delta_{ij}&=\langle b_j,b_i\rangle\\
&=\langle\sum_k{\gamma_{kj}\cdot c_k},\sum_l{\gamma_{li}\cdot c_l}\rangle\\
&=\sum_{k,l}{\gamma_{kj}\cdot\overline{\gamma_{li}}\cdot\langle c_k,c_l\rangle}\\
&=\sum_k{\gamma_{kj}\cdot\overline{\gamma_{ki}}}\\
&=\sum_k{\overline{\gamma_{ki}}\cdot\gamma_{kj}}-(\overline{M}_{CB}^T\cdot M_{CB})_{ij}
\end{align*}
Es gilt also $M_{CB}^* = M_{CB}^{-1}$.
\end{enumerate}
\end{proof}

\section{Normalformen für Isometrien und normale Endomorphismen}
Sei $V$ VRm mit SKP $\langle\cdot,\cdot\rangle, \dim(V)=n<\infty$.

\subsection{Fall $\mathbb{K}=\mathbb{C}$}
\begin{lemma}
Ein Endomorphismus $\Phi$ ist genau dann unitär, wenn er normal ist und alle Eigenwerte 
Betrag 1 haben.
\end{lemma}

\begin{proof}
Da $\Phi$ unitär ist, also $\Phi^*=\Phi^{-1}$ gilt, ist $\Phi$ normal. 
Nach Spektralsatz existiert dann eine ONB $B=\{b_1,\ldots,b_n\}$ aus Eigenvektoren von $\Phi$.
Also gilt: 
\[\Phi(b_i)=\lambda_i b_i \text{ mit } \lambda_i\in\mathbb{C}\]
Mit dem Korollar folgt:\\
\begin{align*}
&\Phi \text{ unitär}\\
&\iff \Phi(B) \text{ ONB}\\
&\iff \delta_{ij}=\langle\Phi(b_i),\Phi(b_j)\rangle=\langle\lambda_i b_i,\lambda_j b_j\rangle
=|\lambda_i|^2\cdot\delta_{ij}\\
&\iff |\lambda_i|^2=1\\
&\iff |\lambda_i|=1
\end{align*}
\end{proof}

\begin{conclusion}
\index{Normalform}
$D_{BB}(\Phi)=\diag(\lambda_1,\ldots,\lambda_n)$ mit $|\lambda_i|=1$ heißt \textbf{Normalform}
der unitären Abb. $\Phi$ und ist bis auf die Reihenfolge der Eigenwerte eindeutig bestimmt.
\end{conclusion}

\begin{corollary}
\index{unitär!Basiswechsel}\index{Basiswechsel!unitärer}
Ist $A\in\mathbb{C}^{n\times n}$ normal, so existieren $M\in U_n$ und $\lambda_i\in\mathbb{C}$,
sodass gilt:
\begin{align*}
M^{-1}\cdot A\cdot M = \diag(\lambda_1,\ldots,\lambda_n)
\end{align*}
D.h. jedes normale $A$ erlangt durch \textbf{unitären Basiswechsel} Normalform.\\
Falls $A$ unitär ist, so existiert $\Phi_j\in\mathbb{R}$, sodass gilt:
\begin{align*}
\lambda_j=e^{i\Phi_j}=\cos\Phi_j+i\sin\Phi_j
\end{align*}
\end{corollary}



\chapter{Affine Räume}

Man möchte vom Anschauungsraum $\mathbb{R}^3$ abstrahieren:\\
\begin{itemize}
\item statt $\mathbb{R}$ \textbf{beliebige} Körper $K$
\item statt Dimension $3$ \textbf{beliebige} Dimensionen $< \infty$
\end{itemize}

\begin{task}
Finde die "`richtige"' Verallgemeinerung der vertrauten \textbf{geometrischen} 
Begriffe, sodass bekannte geometrische Sätze richtig bleiben.
\end{task}

Im Folgenden sei $K$ stets ein beliebiger Körper.

\section{Grunbegriffe}
\begin{definition}
\index{affin!Raum}\index{Raum!affiner}
\index{Richtungsvektorraum}
\index{Translation}
\index{Punkt}
\index{Translationsvektor}
Sei $V$ $K$-VRm mit $\dim(V)=n<\infty$.
\begin{enumerate}[(a)]
\item Eine Menge $A \ne \emptyset$ heißt \textbf{affiner Raum mit Richtungsvektorraum $V$},
falls $(V,+)$ auf $A$ operiert, d.h. es existiert eine Paarung "`$+$"' genannt
\textbf{Translation} $V \times A \to A$, $(x,P) \mapsto x+P$, mit der Eigentschaft:
\[\forall P,Q\in A\exists_1 x\in V: Q=x+P\]
\item Elemente von $A$ heißen \textbf{Punkte}.\\
Der zu gegebenen Punkten $P$, $Q$ eindeutig bestimmte Vektor $x$ mit $Q=x+P$ heißt
der \textbf{Translationsvektor von $P$ nach $Q$}.\\
Schreibe: $x:=\overrightarrow{PQ}$
\item $\dim(A) := \dim(V)$ heißt \textbf{Dimension von $A$}.
\end{enumerate}
\end{definition}

\begin{comment}
\begin{enumerate}
\item Vorsicht in 1. wird das Zeichen "`+"' für verschiedene Verknüpfungen benutzt.
\item Es gilt für $P,Q,R,\in A:$
\begin{align*}
\overrightarrow{PP}&=0\\
\overrightarrow{PQ}+\overrightarrow{QR} &= \overrightarrow{PR}\\
\overrightarrow{QP}&=-\overrightarrow{PQ}
\end{align*} 
\item $A$ besteht aus genau einer Bahn:
\[\forall P\in A: A=V+P:=\{x+P\mid x\in V\}\]
\end{enumerate}
\end{comment}

\begin{example}
\index{affin!Standardraum}\index{Standardraum!affiner}
Der \textbf{affine Standardraum} $\mathbb{A}_n(K)$ ist definiert als  Punktmenge
$\mathbb{A}:=K^n$ und $V:=K^n$, mit Translation $:=$ Addition in $K^n$, d.h.
für $P,Q\in K^n$ gilt:
\[\overrightarrow{PQ}=Q-P\]
\end{example}

\begin{definition}
\index{affin!Teilraum}\index{Teilraum!affiner}
\index{linear!Varietät}\index{Varietät!lineare}
\index{Gerade}
\index{Ebene}
\index{Hyperebene}
Eine Teilmenge $B \ne \emptyset$ eines affinen Raumes A heißt \textbf{(affiner) Teilraum}
oder \textbf{lineare Varietät} von $A$, falls ein VRm $U_B \le V$ existiert, sodass
$B$ affiner Raum ist, mit Richtungsvektorraum $U_B$ (unter der in $A$ gegebenen
Operation).\\
Auch $B = \emptyset$ werde affiner Teilraum genannt.\\
Spezielle affine Teilräume B sind:
\begin{enumerate}[(a)]
\item \textbf{Gerade} $\iff \dim(B) = 1$
\item \textbf{Ebene} $\iff \dim(B) = 2$
\item \textbf{Hyperebene} $\iff \dim(B) = \dim(A)-1$
\end{enumerate}
\end{definition}

\begin{lemma}
\index{Verbindungsgerade}
\begin{enumerate}
\item Ist $B\ne \emptyset$ affiner Teilraum, dann gilt:
\[U_B=\{\overrightarrow{PQ}\mid P,Q\in B\}\]
\item Sind $\emptyset\ne B \subseteq C$ affine Teilräume und $\dim(B)=\dim(C)$, dann ist $B=C$.
\item Durch zwei Punkte $P\ne Q$ in $A$ geht genau eine Gerade.
\[PQ := K \cdot \overrightarrow{PQ}+P = \{\lambda \cdot \overrightarrow{PQ}+P\mid\lambda\in K\}\le A\]
Diese wird die \textbf{Verbindungsgerade} von $P$ und $Q$ genannt.
\item Drei Punkte $P,Q,R \in A$ liegen genau dann auf \textbf{einer} Geraden, wenn gilt,
dass $\overrightarrow{PQ}$ und $\overrightarrow{QR}$ linear abhängig sind.
\end{enumerate}
\end{lemma}

\begin{proof}
\begin{enumerate}
\item 
\begin{enumerate}
\item["`$\supseteq$"']\checkmark
\item["`$\subseteq$"'] Da $B$ affiner Teilraum mit Richtung $U_B$ ist, gilt für alle $P,Q\in B$:
\[\exists_1 x\in B: x=\overrightarrow{PQ} \iff x+P = Q\]
\end{enumerate}
\item Aus 1. folgt mit $B\subseteq C$, dass $U_B\subseteq U_C$ gilt. Da diese die
gleiche Dimension haben muss dann schon $U_B=U_C$ gelten. Für $P\in B\cap C$ gilt dann:
\[B=U_B+P=U_C+P=C\]
\item Es ist klar, dass $P$ und $Q$ auf der Geraden $PQ$ liegen, daher muss lediglich die
Eindeutigkeit gezeigt werden.\\
Sei $B$ eine Gerade mit $P,Q\in B$ und $U:=U_B$. Da $P\ne Q$ ist, 
ist $\overrightarrow{PQ}\in U$ nicht der Nullvektor. Da außerdem $\dim U=1$ ist, gilt:
\[U=K\cdot\overrightarrow{PQ}\]
Daraus folgt:
\[B=U+P=PQ\]
\item Sei $x:=\overrightarrow{PQ}$ und $y:=\overrightarrow{QR}$. Es existiert genau dann
eine Gerade $B$ mit $P,Q,R\in B$, wenn gilt:
\[\exists\text{ VRm }U: \dim U=1, x,y\in U\]
Also genau dann, wenn $x$ und $y$ linear abhängig sind.
\end{enumerate}
\end{proof}

\begin{theo}[Teilraumkriterium]
Sei $A$ affiner Raum mit Richtung $V$ und sei $\emptyset \ne B\subseteq A$. Dann 
sind equivalent:
\begin{enumerate}
\item $B$ ist affiner Teilraum.
\item Es existieren $P\in A$ und $U\le V$, sodass gilt:
\[B=U+P\]
\item Falls $|K| >2$, so ist auch equivalent:
\[\forall P,Q\in B: P\ne Q \implies PQ \subseteq B\]
\item Falls $A=\mathbb{A}_n(K)$, so ist auch equivalent, dass $B$ Lösungsmenge eines LGS ist.
\end{enumerate}
\end{theo}

\begin{proof}
Die Äquivalenz ergibt sich aus folgendem Ringschluss:
\begin{itemize}
\item[1.$\implies$2.] Ist $B$ affiner Teilraum, so gilt:
\[\exists U\le V:\forall P\in U:B=U+P\]
\item[2.$\implies$1.] $B=U+P$ ist affiner Teilraum, denn $U$ operiert auf $B$
und für $Q,R\in B:$ gilt:
\[\exists x,y\in U: Q=x+P,R=y+P \text{ und}\] 
\[\exists_1 \text{ Translation } \overrightarrow{QR}=y-x\in U\]
Daraus folgt, dass $U$ affiner Teilraum ist.
\item[1.$\implies$3.] Sei $B$ affiner Teilraum mit $P,Q\in B, P\ne Q$. Dann gilt:
\begin{align*}
&\overrightarrow{PQ}\in U_B\\
&\implies\forall \lambda\in K:\lambda\cdot\overrightarrow{PQ}+P\in B\\
&\implies PQ\subseteq B
\end{align*}
\item[3.$\implies$2.] Setze $U:=\{\overrightarrow{PQ}\mid P,Q\in B\}\subseteq V$.\\
\textbf{Zeige zunächst:} Für alle $P\in B$ gilt:
\[U+P\subseteq B\] 
D.h. für alle $y\in U$ gilt:
\[y+P\in B\]
\begin{enumerate}
\item["`$\subseteq$"'] Sei also $0\ne y\in U$, dann existiert ein $Q\ne R\in B$, sodass gilt: 
\[ y=\overrightarrow{QR}\] 
Setze $z:=\overrightarrow{PQ}$.\\
\textbf{Fall $y,z$ linear abhängig:}\\
Aus dem Lemma folgt, dass $P,Q,R$ auf der Geraden $QR=\{\lambda\cdot y+P\mid \lambda\in K\}
\stackrel{3.}{\subseteq} B$ liegen. Insbesondere gilt: 
\[y+P\in B\]
\textbf{Fall $y,z$ linear unabhängig:}\\
Wähle $\lambda\in K\setminus\{0,1\}$. Betrachte $S:=\frac{\lambda}{\lambda-1}y+P$, 
$N:=\lambda z+P$.\\
Dann ist $N\in PQ\subseteq B$.\\
Annahme: $N=R$.
Dann gilt:
\begin{align*}
N&=\lambda z+p\\
&=R\\
&=y+z+P
\end{align*}
Daraus folgt, dass $y$ und $z$ linear abhängig sind. $\lightning$
Es gilt also $N\ne R$. Ferner gilt, dass $S,N,R$ auf einer Geraden liegen, denn:
\[\overrightarrow{NR} = y+z-\lambda z = y+(1-\lambda)z \text{ und}\]
\[\overrightarrow{SN} = \lambda z-\frac{\lambda}{1-\lambda}y=\frac{\lambda}{\lambda-1}
((\lambda-1)z-y)\]
sind linear abhängig.\\
Aus $N,R\in B$ folgt:
\[S\in NR \stackrel{3.}{\subseteq}B\]
Außerdem gilt: $S\ne P$, also $SP\in B$ und damit $y+P\in B$\\
Es gilt sogar: $B=U+P$, da für alle $Q\in B$ gilt:
\[Q=\overrightarrow{PQ}+P\in U+P\]
\item["`$\supseteq$"'] \checkmark
\end{enumerate}
\textbf{Es bleibt zu zeigen:} $U\le V$ (Untervektorraum)\\
Seien $x,y\in U,\alpha\in K$. O.B.d.A lässt sich $x\ne 0$ annehmen, etwa $x=\overrightarrow{PQ},
P,Q\in B$. Dann gilt:\\
\[\alpha x+P\in PQ \subseteq B \implies \alpha x\in U\]
Also genügt es zu zeigen, dass $x+y$ in $U$ liegt. Sei $P':=x+P$.\\
Dann gilt mit $x=\overrightarrow{PQ}$ und $y+P\in U+P\subseteq B$:\\
\begin{align*}
&(x+y)+P=x+(y+P)\in U+P'\subseteq B\\
&\implies x+y\in U
\end{align*}
\end{itemize}
\end{proof}

\section{Eigentschaften affiner Räume}
\begin{lemma}
Sei $I\ne\emptyset$ Indexmenge und $(B_i)_{i\in I}$ eine Familie affiner Teilräume von $A$.\\
Dann ist $B:= \bigcap_{i\in I} B_i$ affiner Teilraum von $A$ mit Richtung
$U_B=\bigcap_{i\in I} U_{B_i}$, falls $B\ne\emptyset$.
\end{lemma}

\begin{proof}
Sei $B\ne\emptyset$, dann existiert ein $P\in\bigcap_{i\in I}B_i$.
Setze $U:=\bigcap_{i\in I}U_{B_i} \le V$.\\
Dann gilt für ein $Q\in A$:\\
\begin{align*}
Q\in U+P &\iff \forall i\in I: Q\in U_{B_i}+P\\
&\iff Q\in\bigcap_{i\in I}B_i\\
&\iff Q\in B
\end{align*}
Daraus folgt: $B=U+P$
\end{proof}

\begin{definition}
\index{affin!Hülle}\index{Hülle!affine}
Sei $M$ Teilmenge von $A$, $C$ die Menge aller affinen Teilräume von $A$, die $M$ enthalten.\\
Dann heißt:
\[[M]:=\bigcap_{B\in C}B\]
die \textbf{affine Hülle} von $M$.\\
Für $M=\{P_1,\ldots,P_m\}$ schreibe: $[P_1,\ldots,P_m]:=[M]$.
\end{definition}

\begin{example}
Sei $P\ne Q$, dann ist $[P,Q]=PQ$ die Gerade durch $P$ und $Q$.
\end{example}

\begin{lemma}
\index{allgemein!Lage}\index{Lage!allgemeine}
Seien $P_0,\ldots,P_m\in A$ und sei $x_i:=\overrightarrow{P_0P_i}\in V$ für alle $i\in\{1,\ldots,m\}$.\\
Dann gilt:
\[[P_0,\ldots ,P_m]=\langle x_1,\ldots ,x_m \rangle +P_0\]
Insbesondere ist $\dim{[P_0,\ldots,P_m]}\le m$.\\
Falls gilt: $\dim{[P_0,\ldots,P_m]}=m$ sagt man, $P_0,\ldots,P_m$ sind in \textbf{allgemeiner Lage}.
\end{lemma}

\begin{proof}
\begin{enumerate}
\item["`$\subseteq$"'] Für alle $i\in\{1,\ldots,m\}$ gilt:
\[P_i=x_i+P_0\subseteq\langle x_1,\ldots,x_m\rangle+P_0\]
\item["`$\supseteq$"'] Sei $\sum_{i=1}^m{\alpha_i x_i+P_0}\in\langle x_1,\ldots,x_m\rangle+P_0$,
und sei $B\supseteq\{P_0,\ldots,P_m\}$ beliebiger affiner Teilraum. Dann gilt:
\begin{align*}
&\forall i\in\{1,\ldots,m\}: x_i=\overrightarrow{P_0P_i}\in U_B\\
&\implies \sum_{i=1}^m{\alpha_i x_i}\in U_B\\
&\implies \sum_{i=1}^m{\alpha_i x_i+P_0}\in U_B+P_0=B
\end{align*}
Da dies für einen beliebigen affinen Teilraum $B$ gilt, der $\{P_0,\ldots,P_m\}$ enthält, gilt dies für alle
solche Teilräume. Sei $C$ die Menge aller affinen Teilräume die $\{P_0,\ldots,P_m\}$ enthalten. Dann folgt:
\begin{align*}
&\forall B\in C: \sum_{i=1}^m{\alpha_i x_i+P_0}\in B\\ 
&\iff \sum_{i=1}^m{\alpha_i x_i+P_0}\in\bigcap_{B\in C}B\\
&\iff \sum_{i=1}^m{\alpha_i x_i+P_0}\in[P_0,\ldots,P_m]
\end{align*}
\end{enumerate}
\end{proof}

\begin{theo}
Seien $A_1:=U_1+P_1,A_2:=U_2+P_2$ affine Teilräume von A. Dann gilt:
\begin{enumerate}
\item $U_{[A_1\cup A_2]}=U_1+U_2+\langle\overrightarrow{P_1P_2}\rangle$
\item $A_1\cap A_2\ne\emptyset \implies \dim([A_1\cup A_2])=\dim{A_1}+\dim{A_2}-\dim{(A_1\cap A_2)}\\
A_1\cap A_2=\emptyset \implies \dim([A_1\cup A_2])=\dim{A_1}+\dim{A_2}-\dim{(U_1\cap U_2)}+1$
\end{enumerate}
\end{theo}

\begin{proof}
\begin{enumerate}
\item Sei $y:=\overrightarrow{P_1P_2}$ und $U:=U_1+U_2+\langle y\rangle$.\\
\textbf{Zu Zeigen:} $U+P_1=[A_1\cup A_2]$, d.h. $U_{[A_1\cup A_2]}=U$
\begin{enumerate}
\item["`$\subseteq$"'] Für einen beliebigen affinen Raum $B\supseteq A_1\cup A_2$ gilt: $U_B\ge U_1,U_2,
\langle y\rangle$.\\
Also gilt für $x=x_1+x_2+\alpha y\in U$ mit $x_1\in U_1, x_2\in U_2$:
\begin{align*}
&x=x_1+x_2+\alpha y\in U_B\\
&\implies x+P_1\in U_B+P_1=B\\
&\implies x+P_1\in \bigcap B=[A_1\cup A_2]
\end{align*}
\item["`$\supseteq$"'] \textbf{Zu zeigen:} $A_1\cup A_2\subseteq U+P_1$\\
\begin{align*}
A_1&=U_1+P_1\subseteq U+P_1\\
A_2&=U_2+P_2=U_2+y+P_1\subseteq U+P_1
\end{align*}
\end{enumerate}
\item 
\textbf{Fall $A_1\cap A_2\ne\emptyset$:} Nach Lemma gilt $U_{A_1\cap A_2}=U_1\cap U_2$,
und dass $P_1=P_2$ wählbar ist.\\ 
Daraus folgt $U=U_1+U_2$ (mit $y=0$). Also gilt: $[A_1\cup A_2]=U_1+U_2+P_1$ mit:
\begin{align*}
\dim{[A_1+A_2]}&=\dim{(U_1+U_2)}\\
&=\dim{U_1}+\dim{U_2}-\dim{(U_1\cap U_2)}\\
&=\dim{A_1}+\dim{A_2}-\dim{(A_1\cap A_2)}
\end{align*}
\textbf{Fall $A_1\cap A_2=\emptyset$}: Annahme: $y\in U_1+U_2$.\\
Dann ist $y=x_1+x_2$ für ein $x_1\in U_1,x_2\in U_2$. Daraus folgt:
\[x_1+P_1 = -x_2+y+P_1=-x_2+P_2\in A_1\cap A_2\ \lightning\]
Also ist $y\notin U_1+U_2$. Daraus folgt:
\[\dim{U}=\dim{(U_1+U_2)}+1\]
Der restliche Beweis erfolgt analog zum ersten Fall.
\end{enumerate}
\end{proof}

\begin{definition}
\index{Parallelität}
Affine Teilräume $B,C$ von $A$ heißen \textbf{parallel}, wenn gilt:
\[U_B\le U_C \text{ oder }U_C\le U_B\]
Schreibe: $B\parallel C$.
\end{definition}

\begin{example}
Man denke nicht nur an parallele Geraden oder Ebenen, sondern etwa auch an Gerade $\parallel$ Ebene.
\end{example}

\begin{comment}
\begin{enumerate}
\item Auf den Teilräumen einer festen Dimension ist Parallelität eine Äquivalenzrelation.
\item Aus $B\parallel C$ folgt: $(B\subseteq C) \vee (B\supseteq C) \vee (B\cap C=\emptyset)$
\item Für alle $P\in A$ und alle affinen Teilräume $B\ne\emptyset$ existiert genau ein affiner Teilraum
$C$ mit:
\begin{enumerate}[(a)]
\item $P\in C$
\item $B\parallel C$
\item $\dim{C}=\dim{B}$
\end{enumerate}
\end{enumerate}
\end{comment}

\begin{proof}
\begin{enumerate}
\item Leichte Übung!
\item Sei $P\in B\cap C$ und o.B.d.A $U_B\le U_C$. Dann gilt:
\[B=U_B+P\le U_C+P=C\]
\item Es muss $C=U_B+P$ gelten, da aus b) und c) folgt: $U_C=U_B$
\end{enumerate}
\end{proof}

\begin{theo}
Sei $A$ affiner Raum mit $\dim{A}=n>1$, $G\subseteq A$ Gerade und $H$ Hyperebene in $A$.\\
Dann gilt:
\begin{enumerate}
\item $G\cap H=\emptyset \implies G\parallel H$
\item $G\not\parallel H \implies \exists P: G\cap H=\{P\}$
\end{enumerate}
\end{theo}

\begin{comment}
$\dim{G\cap H}\le\dim{G}=1 \implies G\cap H=\begin{cases}
\emptyset\\
\text{Punkt}\\
\text{Gerade}
\end{cases}$
\end{comment}

\begin{proof}
\begin{enumerate}
\item Sei $G\cap H=\emptyset$, dann ist $G\cup H$ echte Obermenge von $H$. Es gilt also:
\[H \subsetneq G\cup H \subseteq [G\cup H]\]
Daraus folgt für die Dimensionen:
\begin{align*}
&n-1=\dim H < dim[G\cup H]\le n\\
&\implies \dim[G\cup H]=n\\
&\implies [G\cup H]=A
\end{align*}
Aus der Dimensionsformel für die affine Hülle folgt:
\begin{align*}
n&=\dim[G\cup H]\\
&= \dim G+\dim H -\dim(U_G\cap U_H)+1\\
&= n-\dim(U_G\cap U_H)+1
\end{align*}
Daraus folgt:
\begin{align*}
&\dim(U_G\cap U_H)=1=\dim U_G\\
&\implies U_G\cap U_H = U_G\\
&\implies U_G \subseteq U_H\\
&\implies G\parallel H
\end{align*}
\item Aus 1. folgt, dass $G\cap H$ nicht die leere Menge ist, wenn $G$ und $H$ nicht
parallel sind.\\
Sei nun $G':=G\cap H$ eine Gerade. Dann gilt:
\begin{align*}
&G'\subseteq G\\
&\implies G'=G\\
&\implies G\subseteq H\\
&\implies G\parallel H
\end{align*}
Also kann $G\cap H$ auch keine Gerade sein, wenn $G$ und $H$ nicht parallel sind. Mit der
Vorbemerkung folgt daraus, dass $G\cap H$ ein Punkt sein muss.
\end{enumerate}
\end{proof}

\chapter{Affine Koordinaten und affine Abbildungen}
\section{Grundbegriffe}
\begin{definition}
\index{affin!Koordinatensystem}\index{Koordinatensystem!affines}
\index{Ursprung}
\index{Koordinatenvektor}
\index{Koordinatendarstellung}
Sei $A$ ein affiner Raum mit Richtungs-VRm V der Dimension n.
\begin{enumerate}[(a)]
\item Sei $\mathcal{B}$ die Menge aller Basen von V. Ein Paar $\mathcal{K}:=(\mathcal{O},B)
\in A\times\mathcal{B}$ heißt \textbf{affines Koordinatensystem}, wobei $\mathcal{O}$ der
\textbf{Ursprung} heißt.
\item  Durch die Koordinatendarstellung $D_B:V\to K^n$ zur Basis $B$ definiert:
\[D_\mathcal{K}:A\to K^n,P\mapsto D_B(\overrightarrow{\mathcal{O}P})\]
den \textbf{Koordinatenvektor} $D_\mathcal{K}(P)$ von P bezüglich $\mathcal{O}$.
\item Die Abbildung $D_\mathcal{K}:A\to\mathbb{A}^n(K)$ heißt \textbf{Koordinatendarstellung}
zum Koordinatensystem $\mathcal{K}$.
\end{enumerate}
\end{definition}

\begin{task}
Was entspricht Homomorphismen von VRmen bei affinen Räumen?
\end{task}

\begin{definition}
\index{affin!Abbildung}\index{Abbildung!affine}
\index{Morphismus!affiner Räume}
Seien $A,B$ affine Räume mit Richtungen $V,W$.
Eine Abbildung $\varphi:A\to B$ heißt \textbf{affine Abbildung} oder \textbf{Morphismus affiner Räume},
falls ein $\Phi\in\Hom(V,W)$ existiert, sodass gilt:
\[\forall x\in V,\forall P\in A: \varphi(x+P)=\Phi(x)+\varphi(P)\]
Schreibe: $\Homaff(A,B):=\{\varphi:A\to B\mid\varphi$ affin $\}$.
\end{definition}

\begin{example}
\begin{enumerate}
\item  Die Identität $\id_A:A\to A$ ist affin mit zugehörigem $\Phi=\id_V$.
\item Für \textbf{festes} $Q\in B$ ist die konstante Abbildung $\varphi_Q:A\to B,
P\mapsto Q$ affin, mit der Nullabbildung als zugehörigem Homomorphismus.
\end{enumerate}
\end{example}

\begin{comment}
\begin{enumerate}
\item $\varphi:A\to B$ mit zugehörigem $\Phi\in\Hom(V,W)$ ist genau dann affin, wenn gilt:
\[\exists P_0\in A:\forall x\in V: \varphi(x+P_0)=\Phi(x)+\varphi(P_0)\]
\item Ist $\varphi\in\Homaff(A,B)$, so ist der zugehörige Homomorphismus \mbox{$\Phi=:\Lambda_\varphi$}
eindeutig bestimmt.
\item Die Hintereinanderausführung affiner Abbildungen ist affin, d.h.:
\[\Homaff(A,B)\times\Homaff(B,C)\to \Homaff(A,C), (\varphi,\psi)\mapsto\psi\circ\varphi\]
\item $\varphi\in\Homaff(A,B)$ ist genau dann injektiv (bzw. surjektiv), wenn $\Lambda_\varphi$
injektiv (bzw. surjektiv) ist.
\item Ist $\varphi\in\Homaff(A,B)$ bijektiv, so existiert $\varphi^{-1}\in\Homaff(B,A)$.
\end{enumerate}
\end{comment}

\begin{definition}
\index{Isomorphismus!affiner Räume}
\index{Affinität}
\index{affin!Automorphismus}\index{Automorphismus!affiner}
\index{affin!Gruppe}\index{Gruppe!affine}
Ein bijektives $\varphi\in\Homaff(A,B)$ heißt \textbf{Isomorphismus affiner Räume} oder \textbf{Affinität}.\\
Ist zusätzlich $A=B$, so heißt $\varphi\in\Homaff(A,A)$ \textbf{Automorphismus}. Diese Automorphismen
bilden die Gruppe $\Autaff(A)$, genannt die \textbf{affine Gruppe} von $A$.
\end{definition}

\begin{proof}
\begin{enumerate}
\item Sei $P\in A$ beliebig und $y:=\overrightarrow{P_0P}$. Dann gilt für alle $x\in V$:
\begin{align*}
\varphi(x+P) &= \varphi(x+y+P_0)\\
&= \Phi(x+y)+\varphi(P_0)\\
&= \Phi(x)+\Phi(y)+\varphi(P_0)\\
&= \Phi(x)+\varphi(y+P_0)\\
&= \Phi(x)+\varphi(P)
\end{align*}
\item Sei $\varphi\in\Homaff(A,B)$ gegeben, dann gilt für alle $P\in A,x\in V$:
\begin{align*}
&\varphi(x+P)=\Phi(x)+\varphi(P)\\
&\implies \Phi(x)=\overrightarrow{\varphi(P)\varphi(x+P)}
\end{align*}
Also ist $\Phi$ durch $\varphi$ eindeutig bestimmt.
\item Sei $\varphi\in\Homaff(A,B)$, $\psi\in\Homaff(B,C)$. Dann gilt für alle $P\in A,x\in V$:
\begin{align*}
\psi\circ\varphi(x+P) &= \psi(\varphi(x+P))\\
&= \psi(\Lambda_\varphi(x)+\varphi(P))\\
&= \Lambda_\psi(\Lambda_\varphi(x))+\psi(\varphi(P))\\
&= \Lambda_\psi\circ\Lambda_\varphi(x)+\psi\circ\varphi(P)
\end{align*}
Also ist $\psi\circ\varphi$ affin mit zugehörigem Homomorphismus $\Lambda_{\psi\circ\varphi}=
\Lambda_\psi\circ\Lambda_\varphi$.
\item Es gilt für $\varphi\in\Homaff(A,B)$:
\begin{align*}
\varphi \text{ injektiv }&\iff (\varphi(P)=\varphi(Q) \implies P=Q)\\
&\iff (\varphi(\overrightarrow{QP}+Q)=\varphi(Q) \implies P=Q)\\
&\iff (\Lambda_\varphi(\overrightarrow{QP})+\varphi(Q)=\varphi(Q) \implies P=Q)\\
&\iff (\Lambda_\varphi(\overrightarrow{QP})=0 \implies \overrightarrow{QP}=0)\\
&\iff \Lambda_\varphi \text{ ist injektiv}
\end{align*}
Der Beweis für Surjektivität erfolgt analog.
\item Leichte Übung!
\end{enumerate}
\end{proof}

\begin{theo}
Seien $A,B$ affine Teilräume mit Richtungen $V,W$.
Zu $(P_0,Q_0)\in A\times B$ und $\Phi\in\Hom(V,W)$ existiert genau eine affine Abbildung
$\varphi:A\to B$ mit $\Lambda_\varphi=\Phi$ und $\varphi(P_0)=Q_0$.
\end{theo}

\begin{proof}
Es ist $A=V+P_0$. Definiere $\varphi(x+P_0):=\Phi(x)+Q_0$, so ergibt sich nach Bemerkung 1.
eine affine Abbildung mit $\varphi(P_0)=Q_0$. Dies legt $\varphi$ bereits eindeutig fest.
\end{proof}

\begin{theo}
Die Koordinatenabbildung $D_\mathcal{K}:A\to\mathbb{A}^n(K)$ zu einem Koordinatensystem 
$\mathcal{K}=(\mathcal{O},B)$ ist ein affiner Isomorphismus (mit zugehöriger linearer Abbildung
$D_B:V\to K^n$).
\end{theo}

\begin{proof}
Es gilt:
\begin{align*}
D_\mathcal{K}(x+\mathcal{O})&\stackrel{Def.}{=} D_B(x)\\
&= D_B(x)+0\\
&= D_B(x)+D_\mathcal{K}(\mathcal{O})
\end{align*}
Nach dem letzten Satz existiert genau eine affine Abbildung, die dies tut.
Dass es sich bei $D_\mathcal{K}$ um eine Isometrie handelt, wurde bereits früher eingesehen,
da $D_B$ Isometrie ist.
\end{proof}

\begin{corollary}
Affine Räume über festen Körper sind genau dann isomorph, wenn ihre Dimension gleich ist.
\end{corollary}

\begin{theo}[Erhaltung von Teilräumen]
Sei $\varphi\in\Homaff(A,B)$ und $C\subseteq A$.
Falls $C$ affiner Teilraum  mit Richtung $U:=U_C$ ist, so ist $\varphi(C)\subseteq B$
affiner Teilraum mit Richtung $\Lambda_\varphi(U)$.\\
Ist $\varphi$ Isomorphismus, so gilt:
\begin{enumerate}
\item $C\subseteq A$ ist genau dann affiner Teilraum, wenn $\varphi(C)\subseteq B$ affiner Teilraum ist.
\item Es gilt $\dim C = \dim \varphi(C)$ für jeden affinen Teilraum $C$.
\item Sind $C,C'\subseteq A$ affine Teilräume, so gilt:
\[\varphi([C\cup C'])=[\varphi(C)\cup\varphi(C')]\]
und:
\[\varphi(C\cap C')=\varphi(C)\cap\varphi(C')\]
\item $C\parallel C' \implies \varphi(C)\parallel\varphi(C')$
\end{enumerate}
\end{theo}

\begin{proof}
Sei $\varphi\in\Homaff(A,B), P\in C$ (d.h. $C=U+P$). Nach Teilraumkriterium gilt dann:
\[\varphi(C)=\Phi(U)+\varphi(P)\]
Daraus folgt, dass $\varphi(C)$ affiner Teilraum ist.
\begin{enumerate}
\item Leichte Übung!
\item Leichte Übung!
\item Sogar für beliebige Teilmengen $C,C'\subseteq A$ gilt, wenn $\varphi$ bijektiv ist:
\[\varphi(C\cap C')=\varphi(C)\cap\varphi(C')\]
Für alle affinen Teilräume $D$, die $C\cup C'$ enhalten, gilt:
\[\varphi(D)\supseteq\varphi(C)\cup\varphi(C')\]
Also gilt insbesondere auch für $D:=[C\cup C']$:
\[\varphi([C\cup C'])\supseteq\varphi(C)\cup\varphi(C')\]
Daraus folgt (für jede affine Abbildung, also insbesondere auch für $\varphi^{-1}$):
\[\varphi([C\cup C'])\supseteq[\varphi(C)\cup\varphi(C')]\]
Insgesamt folgt:
\begin{align*}
[C\cup C']&\supseteq\varphi^{-1}([\varphi(C)\cup\varphi(C')])\\
&\supseteq[\varphi^{-1}(\varphi(C))\cup\varphi^{-1}(\varphi(C'))]\\
&= [C\cup C']
\end{align*}
Daraus folgt die Gleichheit.
\item Leichte Übung!
\end{enumerate}
\end{proof}

\subsection{Grundaufgaben im affinen Standartraum $\mathbb{A}_n(K)$}
Seien $P_0,\ldots,P_m,Q_0,\ldots,Q_s\in K^n$ und $B:=[P_0,\ldots,P_m], C:=Q_0,\ldots,Q_s$
gegeben. Ziel ist es $[B\cup C]$ und $B\cap C$ zu berechnen.\\

Mit $x_i:=\overrightarrow{P_0P_i}=P_i-P_0$ gilt:
\[B=\langle x_1,\ldots,x_m\rangle+P_0\]
Analog gilt mit $z_j:=\overrightarrow{Q_0Q_i}=Q_i-Q_0$:
\[C=\langle z_1,\ldots,z_s\rangle+Q_0\]
Daraus folgt dann mit $y:=\overrightarrow{P_0Q_0}$:
\[[B\cup C]=\langle x_1,\ldots,x_m,z_1,\ldots,z_s,y\rangle+P_0\]

\begin{enumerate}
\item Finde mit dem Gauß-Algorithmus eine Basis $\{b_1,\ldots,b_r\}$ von $U$, dann gilt:
\[[B\cup C]=[b_1+P_0,\ldots,b_r+P_0,P_0]\]
mit erzeugenden Punkten in allgemeiner Lage.
\item Interpretiere $B$ als Lösungsmenge $\mathcal{L}(A,b)$ eines LGS $Ax=b$.\\
Sei $x_0=P_0\in K^n$, dann liefert der Spezialfall $B=C$ in 1.:
\[B=U+x_0\]
wobei ${b_1,\ldots,b_r}$ Basis von $U$ ist.\\
Ziel ist es nun, eine Matrix $A\in K^{n-r\times n}$ zu finden, mit $U=\Kern(\Lambda_A)$.
Dazu betrachte die Matrix:
\[M:=
\begin{pmatrix}
b_1&\cdots&b_r
\end{pmatrix}\]
Offensichtlich gilt $\rank(M)=r$.\\
Betrachte nun die Rechtsmultiplikation:
% rho groß machen?
\[\rho_M:K^n\to K^r,y\mapsto yM\]
Dann hat der Kern von $\rho_M$ Dimension $n-r$ und eine Basis aus Zeilenvektoren
$\{c_1,\ldots,c_{n-r}\}$. Damit lässt sich nun die Matrix $A$ wie folgt definieren:
\[A:=
\begin{pmatrix}
c_1\\
\vdots\\
c_{n-r}
\end{pmatrix}\]
Da der Rang von A offensichtlich $n-r$ ist, ist die Dimension des Kerns genau $r$, und es gilt:
\begin{align*}
&\forall t\in\{1,\ldots,n-r\}: c_tM=0\\
&\iff\forall t\in\{1,\ldots,n-r\},j\in\{1,\ldots,r\}: c_t\cdot b_j=0\\
&\iff\forall j\in\{1,\ldots,r\}: Ab_j=0
\end{align*}
Also ist $U$ Unterraum von $\Kern(\Lambda_A)$ und aus der Gleichheit der Dimensionen
beider Räume folgt dann:
\[B=\mathcal{L}(A,b)\]
\item Durchschnittsberechnung:\\
Finde mit Hilfe von 2. Matrizen $A,A'$ und $b,b'\in K^n$, sodass $B=\mathcal{L}(A,b),
C=\mathcal{L}(A',b')$ ist. Dann gilt:
\[B\cap C = \mathcal{L}(D,d)\text{ mit }D:=
\begin{pmatrix}
A\\
A'
\end{pmatrix}, d=
\begin{pmatrix}
b\\
b'
\end{pmatrix}\]
Es genügt nun das LGS $Dx=d$ zu lösen, um $B\cap C$ zu erhalten. 
\end{enumerate}

\begin{example}
Betrachte den affinen Raum $\mathbb{A}_3(\mathbb{F}_2)=\{0,1\}^3$. Gegeben seien die
Ebenen:
\begin{align*}
&E:=\langle 
\begin{pmatrix}
1\\0\\0\\
\end{pmatrix},
\begin{pmatrix}
0\\1\\0
\end{pmatrix}\rangle +
\begin{pmatrix}
0\\0\\0
\end{pmatrix}=[e_1,e_2,0]\\
&F:=\langle 
\begin{pmatrix}
0\\0\\1\\
\end{pmatrix},
\begin{pmatrix}
1\\0\\0
\end{pmatrix}\rangle +
\begin{pmatrix}
1\\1\\1
\end{pmatrix}=[e_2,e_1+e_2,e_2+e_3]
\end{align*}
Zur Bestimmung von $E\cap F$ werden zunächst die zu $E$ und $F$ gehörigen Gleichungssysteme
aufgestellt:
\begin{align*}
&E=\{x\in\mathbb{F}_2^3\mid x_3=0\}=\mathcal{L}((0,0,1),0)\\
&F=\Kern(\Lambda_{(0,1,0)})+
\begin{pmatrix}
1\\1\\1
\end{pmatrix} = \mathcal{L}((0,1,0),1)
\end{align*}
Daraus folgt:
\[E\cap F=\mathcal{L}(
\begin{pmatrix}
0&0&1\\
0&1&0
\end{pmatrix},
\begin{pmatrix}
0\\1
\end{pmatrix})=\{e_2,e_1+e_2\}\]
\end{example}

\begin{theo}[Satz von Pappos]
In einem affinen Raum $A$ mit Dimension 2 seien $G,G'$ verschiedene Geraden
mit $G\cap G'=\{O\}\in A$. Ferner seien $P_1,P_2,P_3\in G\setminus\{O\}$
und $Q_1,Q_2,Q_3\in G'\setminus\{O\}$, sodass gilt:
\[ P_1Q_3\parallel P_3Q_1\text{ und }P_1Q_2\parallel P_2Q_1\]
Daraus folgt:
\[ P_2Q_3\parallel P_3Q_2\]
\end{theo}

\begin{proof}
Da $Q_3\notin G$ ist, sind $O,P_1,Q_3$ in allgemeiner Lage. Daraus erhalten
wir folgendes Koordinatensystem: 
\[\mathcal{K}:=(O,\{\overrightarrow{OP_1},\overrightarrow{OQ_3}\})\]
Da die Koordinatendarstellung $D_{\mathcal{K}}:A\stackrel{\sim}{\to}\mathbb{A}_2(K)$ 
Parallelitäten und Schnittpunkte erhält, können wir o.B.d.A annehmen:
\[A=\mathbb{A}_2(K) \text{ und }O=\begin{pmatrix}0\\0\end{pmatrix}\]
Dann gilt:
\begin{align*}
&P_1=\overrightarrow{OP_1}=\begin{pmatrix}\lambda_1\\0\end{pmatrix}=\begin{pmatrix}1\\0\end{pmatrix}
& &P_2=\begin{pmatrix}\lambda_2\\0\end{pmatrix}
& &P_3=\begin{pmatrix}\lambda_3\\0\end{pmatrix}\\
&Q_3=\overrightarrow{OQ_3}=\begin{pmatrix}0\\\mu_1\end{pmatrix}=\begin{pmatrix}0\\1\end{pmatrix}
& &Q_2=\begin{pmatrix}0\\\mu_2\end{pmatrix}
& &Q_3=\begin{pmatrix}0\\\mu_3\end{pmatrix}
\end{align*}
Wobei $\lambda_2,\lambda_3,\mu_2,\mu_3\ne 0$ sind. Daraus folgt für die Richtungen:
\begin{align*}
&\forall i,j\in \{1,2,3\}:U_{P_iQ_j}=\langle\overrightarrow{P_iQ_j}\rangle =\langle 
\begin{pmatrix}\lambda_i\\-\mu_j\end{pmatrix}\rangle\\
&\implies U_{P_1Q_3}=\langle \begin{pmatrix}1\\-1\end{pmatrix}\rangle
\end{align*}
Nach Vorraussetzung ist $\lambda_3=\mu_1$ und es existiert ein $\rho\in K^\times$, sodass gilt:
\[ \begin{pmatrix}\lambda_2\\-\mu_1 \end{pmatrix}=\rho\begin{pmatrix}1\\-\mu_2\end{pmatrix}\]
Daraus folgt mit $\lambda_3=\rho\mu_2=\lambda_2\mu_2$:
\begin{align*}
U_{P_2Q_3}&=\langle \begin{pmatrix}\lambda_2 \\-\mu_3 \end{pmatrix}\rangle\\
&=\langle \begin{pmatrix} \lambda_2\mu_2\\-\mu_2 \end{pmatrix}\rangle\\
&=\langle \begin{pmatrix} \lambda_3\\-\mu_2 \end{pmatrix}\rangle\\
&= U_{P_3Q_2}
\end{align*}
Also sind $P_2Q_3$ und $P_3,Q_2$ parallel.
\end{proof}

\section{Koordinatenwechsel und Darstellung affiner Abbildungen}
\begin{lemma}
Seien $\mathcal{K}=(O,B)$ und $\mathcal{L}=(Q,C)$ Koordinatensysteme des affinen Raums
$A$ mit Richtung $V$. Sei $M_{CB}:=D_{CB}(\id_V)$ die Basiswechselmatrix.\\
Dann rechnen sich Koordinaten eines Punktes $P$ bzgl. $\mathcal{K}$ in die
Koordinaten bzgl. $\mathcal{L}$ wie folgt um:
\[D_\mathcal{L}(P)=M_{CB}\cdot(D_\mathcal{K}(P)-D_\mathcal{K}(Q)\]
\end{lemma}

\begin{proof}
Es gilt:
\begin{align*}
D_\mathcal{L}(P)&=D_C(\overrightarrow{QP})\\
&=M_{CB}\cdot D_B(\overrightarrow{QP})\\
&=M_{CB}\cdot D_B(\overrightarrow{OP}-\overrightarrow{OQ})\\
&=M_{CB}\cdot (D_B(\overrightarrow{OP})-D_B(\overrightarrow{OP}))\\
&=M_{CB}\cdot(D_\mathcal{K}(P)-D_\mathcal{K}(Q)
\end{align*}
\end{proof}

\begin{application}
Ist ein beliebiges Koordinatensystem $\mathcal{L}=(Q,B)$ gegeben, so lässt sich ein 
Punkt $P$ einfach in das Koordinatensystem $\mathcal{K}=(0,S)$ von $\mathbb{A}_n(K)$ 
mit Standardbasis $S$ überführen. Schreibe dazu $B$ als:
\[B=\begin{pmatrix}b_1&\cdots &b_n\end{pmatrix}\in K^{(n\times n)}\]
Dann ist $M_{SB}=B$ und es gilt:
\[D_\mathcal{L}(P)=M_{BS}(P-Q)=B^{-1}(P-Q)\]
\end{application}

\renewcommand{\indexname}{Stichwortverzeichnis}
\printindex
\end{document}
