\documentclass[parskip,a4paper,twoside,DIV15,BCOR12mm]{scrbook}
\usepackage{la}

\author{Die Mitarbeiter von \url{http://mitschriebwiki.nomeata.de/}}
\title{Lineare Algebra II}

\makeindex

\begin{document}
\maketitle

\renewcommand{\thechapter}{\Roman{chapter}}
\addcontentsline{toc}{chapter}{Inhaltsverzeichnis}

\tableofcontents

\chapter{Vorwort}

\section{Über dieses Skriptum}
Dies ist ein erweiterter Mitschrieb der Vorlesung \glqq Lineare Algebra II\grqq\ von Prof. Dr. Schmidt im
Sommersemester 2010 an der Universität Karlsruhe (KIT). Die Mitschriebe der Vorlesung werden mit
ausdrücklicher Genehmigung von Prof. Dr. Schmidt hier veröffentlicht, Prof. Dr. Schmidt ist für den
Inhalt nicht verantwortlich.

\section{Wer}
Gestartet wurde das Mitschriebwiki von Joachim Breitner. Beteiligt an diesem Mitschrieb sind
Rebecca Schwerdt, Manuel Kaiser und Philipp Ost.

\section{Wo}
Alle Kapitel inklusive \LaTeX-Quellen können unter \url{http://mitschriebwiki.nomeata.de} abgerufen werden.
Dort ist ein \emph{Wiki} eingerichtet und von Joachim Breitner um die \LaTeX-Funktionen erweitert worden.
Das heißt, jeder kann Fehler nachbessern und sich an der Entwicklung
beteiligen. Auf Wunsch ist auch ein Zugang über \emph{Subversion} möglich.


\renewcommand{\thechapter}{\arabic{chapter}}
\renewcommand{\chaptername}{§}
\setcounter{chapter}{14}

\chapter{Multilineare Abbildungen und Tensorprodukte}
\section{Bilinearformen}
\begin{definition}
\index{Vektorraumpaarung}\index{Paarung}
\index{Bilinearform}
\index{ausgeartet!Paarung}
Seien $V,W$ $K$-VRme. Ein Abbildung $P:V\times W\to K$ heißt \textbf{\mbox{(Vektorraum-)}Paarung},
falls $P$ in jedem Argument linear ist, d.h. wenn für jedes feste $w_o\in W$
\[P(\cdot,w_0):V\to K, v\mapsto P(v,w_0)\]
und für jedes feste $v_0\in V$
\[P(v_0,\cdot):W\to K, w\mapsto P(v_0,w)\]
eine lineare Abbildung, also Linearform ist.\\
Im Fall $V=W$ heißt $P$ eine \textbf{Bilinearform} auf $V$.\\
Eine Paarung $P$ heißt \textbf{nicht ausgeartet}, wenn für jedes $w_0\in W$ und
für jedes $v_0\in V$ die Abbildung $P(\cdot,w_0)$ bzw. $P(v_0,\cdot)$ nicht die Nullabbildung
ist.
\end{definition}

\begin{comment}
Die Menge $\mathcal{P}(V,W)$ aller Paarungen von $V$ und $W$ ist ein untervektorraum
des $K$-VRms $\Abb(V\times W,K)$ aller Abbildungen von $V\times W$ nach $K$.
\end{comment}

\begin{example}
Auf dem Dualraum $W:=V^*(=\Hom(V,K))$ ist die nicht ausgeartete Paarung
\[P:V\times V^*\to K,(v,f)\mapsto f(v)\]
eine Bilinearform.\\
Für eine Paarung $P:V\times W\to K$ setzen wir $\rho_w(v):=P(v,w)$ und erhalten
so Linearformen $\rho_w\in V^*$ für alle $w\in W$.
\end{example}

\begin{theo}
\begin{enumerate}
\item Die Abbildung $\rho:W\to V^*,w\mapsto \rho_w$ ist ein Homomorphismus
von $K$-VRmen.
\item Die Zuordnung $\eta:P\to \rho$ ist ein Isomorphismus, es gilt:
\[\mathcal{P}(V,W)\stackrel{\sim}{\to} \Hom(W,V^*)\]
\end{enumerate}
\end{theo}

\begin{proof}
\begin{enumerate}
\item Es gilt für alle $\alpha\in K,w_1,w_2\in W$:
\begin{align*}
\rho_{\alpha w_1+w_2}&=(v\mapsto P(v,\alpha w_1+w_2))\\
&=(v\mapsto \alpha P(v,w_1)+P(v,w_2))\\
&=\alpha((v\mapsto P(v,w_1))+(v\mapsto P(v,w_2)))\\
&=\alpha\rho_{w_1}+\rho_{w_2} 
\end{align*}
\item Homomorphie selbst nachrechnen!\\
Die Umkehrabbildung ist:
\[\Hom(W,V^*)\to \mathcal{P}(V,W),\rho\mapsto P:=((v,w)\mapsto(\rho(w))(v))\]
\end{enumerate}
\end{proof}

\begin{remind}
Lineare Abbildungen sind bereits durch ihre Wirkung auf einer Basis eindeutig
bestimmt. Dieses Prinzip gilt auch für Paarungen.
\end{remind}

\begin{comment}
\index{bilinear!Fortsetzung}\index{Fortsetzung!bilineare}
Seien $V,W$ $K$-VRme mit jeweiliger Basis $B:=\{b_1,\ldots,b_m\}\subseteq V,
C:=\{c_1,\ldots,c_n\}\subseteq W$, so ist eine Paarung $P$ auf $V\times W$
Bereits durch ihre Einschränkung auf $B\times C$ festgelegt.\\
Für $v:=\sum_{i=1}^m \alpha_i b_i,w:=\sum_{j=1}^n \beta_j c_j$ gilt:
\[P(v,w)=\sum_{i=1}^m\sum_{j=1}^n \alpha_i\beta_j\cdot P(b_i,c_j)\]
Jede Abbildung $P':B\times C\to K$ definiert über diese Gleichung eine Paarung
$P':V\times W\to K$. Diese heißt \textbf{bilineare Fortsetzung}.
\end{comment}

\begin{definition}
\index{Fundamentalmatrix}
Die Matrix $D_{BC}(P):=(P(b_i,c_j))\in K^{m\times n}$ heißt \textbf{Fundamentalmatrix}
der Paarung $P$ bzgl. der Basen $B$ und $C$. Mit den Kkordinatenvektoren:
\[D_B(v)=\begin{pmatrix}\alpha_1\\\vdots\\\alpha_m\end{pmatrix}\text{ und }
D_C(w)=\begin{pmatrix}\beta_1\\\vdots\\\beta_n\end{pmatrix}\]
gilt nach obiger Gleichung:
\[P(v,w)=D_B(v)^T\cdot D_{BC}(P)\cdot D_C(w)\]
\end{definition}

\begin{theo}
Eine Paarung $P$ endlichdimensionaler $K$-VRme $V,W$ mit Basen $B,C$ ist genau dann
nicht ausgeartet, wenn die Dimensionen von $V$ und $W$ gleich und $D_{BC}(P)$ invertierbar
ist.
\end{theo}

\begin{proof}
Der Beweis erfolgt durch Implikation in beiden Richtungen:
\begin{enumerate}
\item["`$\impliedby$"'] Sei $\dim V=\dim W$ und $F:=D_{BC}(P)$ invertierbar.
Sei nun $w\ne 0\in W$, dann ist $D_C(w)\ne 0$.\\
Daraus folgt, dass auch $F\cdot D_C(w)$ nicht null ist. O.B.d.A sei die
$i$-te Koordinate ungleich null. Dann gilt:
\[P(b_i,w)=e_i^T\cdot F\cdot D_C(w)\ne 0\]
Insbesondere ist $P(\cdot,w)\ne 0$.
Analog folgt $P(v,\cdot)\ne 0$ für alle $0\ne v\in V$. Also ist $P$ nicht ausgeartet.
\item["`$\implies$"'] Sei $P$ nicht ausgeartet, dann ist insbesondere $\rho:W\to V^*,
w\mapsto \rho_w=(v\mapsto P(v,w))$ injektiv. Daraus folgt:
\[\dim V=\dim V^*\ge \dim W\]
Analog gilt:
\[\dim W=\dim W^*\ge \dim V\]
Also haben $V$ und $W$ gleiche Dimension.\\
\textbf{Annahme:} $F$ ist nicht invertierbar.\\
Dann existiert ein $D_C(w)\ne 0$, sodass $F\cdot D_C(w)$ gilt. Daraus folgt
für alle $v\in V$:
\[P(v,w)=D_B(v)^T\cdot F\cdot D_C(w)=0\ \lightning\]
Also ist $P(\cdot,w)=0$, was einen Widerspruch zur nicht Ausgeartetheit von $P$ darstellt.
\end{enumerate}
\end{proof}

\begin{theo}
Seien $B,\hat B$ Basen von $V$, $C,\hat C$ Basen von $W$ und $P$ eine Paarung von
$V$ und $W$. Dann gilt:
\[D_{BC}(P)=D_{\hat BB}(\id_V)^T\cdot D_{\hat B\hat C}(P)\cdot D_{\hat CC}(\id_W)\]
\end{theo}

\begin{proof}
Für $(v,w)\in V\times W$ gilt:
\begin{align*}
P(v,w)&=D_{\hat B}(v)^T\cdot D_{\hat B\hat C}(P)\cdot D_{\hat C}(w)\\
&=(D_{\hat BB}(\id_V)\cdot D_B(v))^T\cdot D_{\hat B\hat C}(P)\cdot(D_{\hat CC}(\id_W)\cdot D_C(w))\\
&=D_B(v)^T \cdot (D_{\hat BB}(\id_V)^T \cdot D_{\hat B\hat C}(P)\cdot D_{\hat CC}(\id_W))\cdot D_C(w)
\end{align*}
Aber es gilt auch:
\[P(v,w)=D_B(v)^T\cdot D_{BC}(P)\cdot D_C(w)\]
Durch einsetzen aller Basispaare $b_i,c_j$ folgt die Behauptung.
\end{proof}

\begin{comment}
Mit der Dualbasis $B^*=\{b_1^*,\ldots,b_m^*\}$ von $V^*$ zu $B$ (erinnere:
$b_k^*(b_i)=\delta_{ik}$) gilt für $\rho=\eta(P):W\to V^*$:
\[\rho(c_j)=\sum_{i=1}^n P(b_i,c_j)\cdot b_i^*\]
D.h. $D_{B^*C}(\rho)=D_{BC}(P)$.
\end{comment}

\begin{proof}
Es gilt:
\begin{align*}
\rho(c_j)&=P(\cdot,c_j)\\
&=(b_i\mapsto P(b_i,c_j))\\
&=\sum_{i=1}^n P(b_i,c_j)\cdot b_i^*
\end{align*}
\end{proof}

\begin{example}
Sei $W=V^*$ und für alle $f\in V^*$ sei $P(v,f)=f(v)$. Nehme nun die Dualbasis 
$C=B*$ zur Basis $B$ von $V$. Dann gilt:
\[P(b_i,b_k^*)=b_k^*(b_i)=\delta{ik}\]
Also ist $D_{BB}(P)=I_m$
\end{example}

Wir spezialisieren nun $W=V$.

\begin{definition}
\index{symmetrisch!Paarung}
\index{Orthogonalbasis (OGB)}
\index{Orthonormalbasis (ONB)}
Sei $P$ eine Paarung von $V$ und $V$.
\begin{enumerate}[(a)]
\item $P$ heißt \textbf{symmetrisch}, falls für alle $v,w\in V$ gilt:
\[P(v,w)=P(w,v)\]
\item Eine Basis $B=\{b_1,\ldots,b_m\}$ heißt \textbf{Orthogonalbasis} (OGB) von $V$
bezüglich $P$, wenn für alle $i\ne j$ gilt:
\[P(b_i,b_j)=0\]
\item Eine Basis $B=\{b_1,\ldots,b_m\}$ heißt \textbf{Orthonormalbasis} (ONB) von $V$
bezüglich $P$, wenn $B$ OGB ist und für alle $i\in\{1,\ldots,m\}$ gilt:
\[P(b_i,b_i)=1\]
\end{enumerate}
\end{definition}

\begin{comment}
Falls eine OGB $B$ existiert, so ist die Fundamentalmatrix $D_{BB}(P)$ diagonal, insbesondere
symmetrisch, also ist $P$ symmetrisch.
\end{comment}

\begin{theo}
Sei $K$ ein Körper mit $1+1\ne 0$ und $P$ eine symmetrische Bilinearform auf einem
$K$-VRm $V$ mit $\dim V=:n<\infty$. Dann existiert eine OGB von $V$ bzgl. $P$.
\end{theo}

\begin{proof}
Der beweis erfolgt durch vollständige Induktion nach der Dimension $n$.\\
Für $n=1$ ist die Behauptung offensichtlich wahr, nehmen wir also als Induktionsvoraussetzung an,
dass sie für $n-1$ erfüllt sei.\\
Da für $P=0$ jede Basis Orthogonalbasis ist, lässt sich im Folgenden o.B.d.A annehmen,
dass $P\ne 0$ ist. Also existieren $v,w\in V$ mit $P(v,w)\ne 0$, es gilt:
\begin{align*}
P(v+w,v+w)&=P(v,v)+P(w,w)+P(v,w)+P(w,v)\\
&= P(v,v)+P(w,w)+2P(v,w)
\end{align*}
Daraus folgt:
\[P(v,v)\ne 0 \vee P(w,w)\ne 0 \vee P(v+w,v+w)\ne 0\]
Also existiert ein $b_1\in V$ mit $P(b_1,b_1)\ne 0$. Nun betrachte:
\begin{align*}
W&:=\{v\in V\mid P(v,b_1)=0\}\\
&=\Kern(P(\cdot,b_1))
\end{align*}
Nach Dimensionsformel ist $\dim W=n-1$ und $V=K\cdot b_1 \oplus W$. Da die 
Einschränkung $P|_{W\times W}$ symmetrisch ist, existiert nach Induktionsvoraussetzung
eine OGB $\{b_2,\ldots,b_n\}$ von $W$ bzgl. $P|_{W\times W}$.\\
Da außerdem für alle $w\in W$ $P(w,b_1)=0$ ist, ist
$\{b_1,\ldots,b_n\}$ OGB von $V$ bzgl. $P$.
\end{proof}

\begin{comment}
\index{Fourierformel}
Die Basisdarstellung bzgl. eine ONB $B$ lautet:
\[v=\sum_{b\in B}P(v,b)\cdot b\]
\end{comment}

\begin{proof}
Leichte Übung!
\end{proof}

\section{Multilineare Abbildungen}
Veralgemeinere nun die Bilinearität und den Zielbereich.

\begin{definition}
\index{multilinear}
Seien $V_1,\ldots,V_n,W$ $K$-VRme und $M:V_1\times\ldots\times V_n\to W$ eine Abbildung.\\
$M$ heißt \textbf{(n-fach) multilinear}, falls für jedes $i\in\{1,\ldots,n\}$ bei fester
Wahl von $v_j\in V_j$ (für alle $j\ne i$) $M(v_1,\ldots,v_{i-1},\cdot,v_{i+1},\ldots,v_n):V_i\to W$
eine lineare Abbildubg ist.
\end{definition}

\begin{example}
Multilineare Abbildungen sind:
\begin{enumerate}
\item Die Determinantenabbildung:
\[\det:K^n\times\ldots\times K^n\to K\]
\item Die Skalarmultiplikation:
\[K\times V\to V,(\lambda,v)\mapsto \lambda\cdot v\]
\item Die Matrizenmultiplikation:
\[K^{p\times q}\times K^{q\times r}\times K^{r\times s}\to K^{p\times s},(A,B,C)\mapsto A\cdot B\cdot C\]
\end{enumerate}
\end{example}

\section{Tensorprodukte}
\begin{definition}
\index{universell!Abbildungseigenschaft (UAE)}\index{Abbildungseigenschaft!universelle (UAE)}
Seien $V,W$ $K$-VRme. Eine $K$-VRm $T$ mit einer bilinearen Abbildung $\tau:V\times W\to T$
heißt ein \textbf{Tensorprodukt} von $V$ und $W$, falls $\tau$ die folgende 
\textbf{universelle Abbildungseigenschaft} (UAE) erfüllt:\\
Zu jedem $K$-VRm $U$ und jeder bilinearen Abbildung $\beta:V\times W\to U$ existiert
genau eine lineare Abbildung $\Phi_\beta:T\to U$ derart, dass $\beta=\Phi_\beta\circ \tau$.\\
Schreibe: $T=:V\otimes_K W,\tau(v,w)=:v\otimes w$
\end{definition}

\begin{comment}
\begin{enumerate}
\item Falls $T$ existiert, so hat man eine Bijektion:
\[\Bil(V\times W,U)\stackrel{\sim}{\to}\Hom(T,U),\beta\mapsto\Phi_\beta\]
\item Sind $(T_1,\tau_1),(T_2,\tau_2)$ Tensorprodukte von $V$ und $W$, so existiert genau 
ein Isomorphismus $\Phi:T_1\to T_2$ mit $\tau_2=\Phi\circ\tau_1$.
\end{enumerate}
\end{comment}

\begin{task}
Beweise die Existenz von Tensorprodukten.
\end{task}

\begin{example}
\begin{enumerate}
\item Sei $V:=K^{n\times 1},W:=K^{m\times 1}, T:=K^{n\times m}$ und die bilineare
Abbildung:
\[\tau:K^n\times K^m\to T,(v,w)\mapsto v\cdot w^T\]
Für die Standardbasen $\{e_i\}\subseteq V,\{e'_j\}\subseteq W$ ist $\tau(e_i,e'_j)=E_{ij}$
die Elementarmatrix. $D:=\{E_{ij}\mid i\in\{1,\ldots,n\},j\in\{1,\ldots,m\}\}$ ist
Basis von $T$.\\
Im folgenden wollen wir die UAE nachweisen. Sei dazu $\beta:V\times W\to U$ bilinear.
Dann erhalten wir eine lineare Abbildung $\Phi:K^{n\times m}\to U$ für jede Vorgabe einer
Abbildung $D\to U$ (vgl. lineare Fortsetzung). Insbesondere also auch für die Vorgabe:
\[\forall i\in\{1,\ldots,n\},j\in\{1,\ldots,m\}:\Phi(E_{ij}):=\beta(e_i,e'_j)\]
Damit gilt dann:
\begin{align*}
\beta(v,w)&=\beta\left(\sum_{i=1}^n \alpha_i e_i,\sum_{j=1}^m \gamma_j e'_j\right)\\
&=\sum_{i,j}\alpha_i\gamma_j\cdot\beta(e_i,e'_j)\\
&=\sum_{i,j}\alpha_i\gamma_j\cdot\Phi(E_{ij})\\
&=\Phi\left(\sum_{ij}\alpha_i\gamma_j\cdot E_{ij}\right)\\
&=\Phi\left(\tau\left(\sum_{i=1}^n\alpha_i e_i,\sum_{j=1}^m \gamma_j e'_j\right)\right)\\
&=\Phi(\tau(v,w))=(\Phi\circ\tau)(v,w)
\end{align*}
Wir haben also gezeigt, dass $\beta=\Phi\circ\tau$ gilt.\\
Falls für ein $\Phi':T\to U$ auch $\beta=\Phi'\circ\tau$ gilt, folgt insbesondere
$\beta(e_i,e'_j)=\Phi'(\tau(e_i,e'_j))$ und damit:
\[\Phi'(E_{ij}=\beta(e_i,e'_j)=\Phi(\tau(e_i,e'_j))=\Phi(E_{ij)}\]
D.h. $\Phi|_D=\Phi'|_D$, also ist $\Phi=\Phi'$ eindeutig.
\item Seien $V,W$ beliebige VRme mit endlichen Dimensionen $\dim V=n,\dim W=m$.\\
Die Existenz des Tensorproduktes folgt etwa durch Koordinatenisomorphismen und
Beispiel (1) Für eine \textbf{koordinatenfreie Konstruktion} nehme $T:=\Hom(V^*,W)$ und
\[\tau:V\times W\to T,(v,w)\mapsto(V^*\to W,f\mapsto f(v)\cdot w)\]
Leichte Übung: $(T,\tau)$ ist Tensorprodukt von $V,W$ und für Basen $B,C$ von $V,W$ gilt:
\[D:=\{b\otimes c\in T\mid b\in B,c\in C\}\]
ist Basis von $T=V\otimes_K W$.
\end{enumerate}
\end{example}

\begin{theo}
Sind $V,W$ beliebige $K$-VRme, so existiert ein Tensorprodukt von $V$ und $W$.
\end{theo}

\begin{proof}
Finde einen \(K\)-Vektorraum \(T\) und eine lineare Abbildung 
\(\tau:\,V\times W\to T\) mit der universellen Abbildungseigenschaft. Dazu 
benutze den \(K\)-Vektorraum\break\(F:=\Abb(V\times W,K)_{0}\).\\
Sei \(f:\,V\times W\to K\) eine Abbildung mit endlichem Tr\"ager\break 
\(\Supp(f):=\{(v,w)\mid f(v,w)\neq0\}\).\\
\(B:=\{f_{(v,w)}\mid(v,w)\in V\times W\}\) ist eine Basis von \(F\) (da f\"ur 
beliebiges \(f\in F\) gilt: \(f(x,y)=\sum_{(v,w)\in\Supp(f)}{f(v,w)\cdot f_{(v,w)}}\)).\\
Setze \(\vp:\,V\times W\to F,\,(v,w)\mapsto f_{(v,w)}\). Vorsicht: \(\vp\) ist
nicht bilinear!

F\"ur die Bilinearit\"at ben\"otigen wir den Untervektorraum \(R\leq F\), 
erzeugt von den ``fehlenden Relationen''.
%split?
\[
f_{(\alpha v_{1}+v_{2},\beta w_{1}+w_{2})}-\alpha\beta f_{(v_{1},w_{1})}-
    \beta f_{(v_{2},w_{1})}-\alpha f_{(v_{1},w_{2})}-f_{(v_{2},w_{2})}\,
    \forall\alpha,\beta\in K,v_{i}\in V,w_{i}\in W
\]
\index{Abbildung!kanonische}
Bilde den Faktorraum \(T:=\frac{F}{R}\) versehen mit der kanonischen Abbildung
\[
\pi:\,F\to T,\,f\mapsto f+R=:[f]
\]
Betrachte
\[
\tau:\,V\times W\to T,\,(v,w)\mapsto\pi\left(\vp(v,w)\right)=[f_{(v,w)}]
\]
Nun gilt offenbar Bilinearit\"at:
\begin{align*}
\left[f_{(\alpha v_{1}+v_{2},\beta w_{1}+w_{2})}\right]&=
    \alpha\beta\left[f_{(v_{1},w_{1})}\right]+
    \beta\left[f_{v_{2},w_{1})}\right]+
    \alpha\left[f_{(v_{1},w_{2})}\right]+
    \left[f_{(v_{2},w_{2})}\right]\\
\tau(\alpha v_{1}+v_{2},\beta w_{1}+w_{2})&=\alpha\beta\tau(v_{1},w_{1})+
    \beta\tau(v_{2},w_{1})+\alpha\tau(v_{1},w_{2})+\tau(v_{2},w_{2})
\end{align*}
\paragraph{Nachweis der universellen Abbildungseigenschaft:}{Sei wieder 
\(\beta:\,V\times W\to U\) bilinear gegeben. Da 
\(F=\langle B\rangle=\langle\Bil(\vp)\rangle\) und \(\pi\) surjektiv sind,
folgt \(T=\langle\Bil(\tau)\rangle\).\\
Jede lineare Abbildung \(\phi:\,T\to U\) ist eindeutig bestimmt durch 
\(\left.\phi\right|_{\Bil(\tau)}\), also ist durch die Forderung 
\(\beta=\phi\circ\tau\) \(\phi\) eindeutig bestimmt (falls existent).}
\paragraph{Existenz von \(\phi\):}{Zun\"achst definiere die lineare Abbildung
\[
\phi_{F}:\,F\to U
\]
durch Vorgabe auf der Basis \(B\).
\[
\phi_{F}\left(f_{(v,w)}\right):=\beta(v,w)
\]
Da \(\beta\) bilinear ist, folgt
\[
\phi_{F}\left(f_{(\alpha v_{1}+v_{2},\beta w_{1}+w_{2})}-
    \alpha\beta f_{(v_{1},w_{1})}-\beta f_{(v_{2},w_{1})}-
    \alpha f_{(v_{1},w_{2})}-f_{(v_{2},w_{2})}\right)
=0
\]
also \(R\leq\Kern\phi_{F}\).\\
Mit dem Homomorphiesatz folgt: Es existiert eine lineare Abbildung 
\(\phi:\frac{F}{R}=T\to U\) mit 
\[
\phi([f])=\phi_{F}(f)
\]
und
\[
\phi\left(\tau(v,w)\right)=\phi\left([\vp(v,w)]\right)
    =\phi_{F}\left(f_{(v,w)}\right)=\beta(v,w)
\]
}
\end{proof}

\begin{application}
Das Tensorprodukt wird zur Erweiterung des Skalarbereiches eines VRms genutzt.
Sei $V$ $K$-VRm, $L$ ein Körper mit Teilkörper $K\le L$. Insbesondere ist also
$L$ ein $K$-VRm (vgl. früher). Nach Satz 5 existiert das Tensorprodukt 
$L\otimes_K V=:V_L$ ($K$-VRm).\\
Im Folgenden wollen wir zeigen, dass $V_L$ ein $L$-Vektorraum ist. Dazu fehlt
die Skalarmultiplikation $L\times V_L\to V_L$, die wir mittels der UAE definieren.
Für alle $l\in L$ ist:
\[\beta_l:L\times V\to V_L,(x,v)\mapsto lx\otimes v\]
bilinear, sodass $\beta_l(x,v)=\Phi_{\beta_l}(x\otimes v)$.\\
Nehme nun $\Phi_{\beta_l}$ als Skalarmultiplikation mit $l\in L$:
\[L\times V_L\to V_L,(l,u)\mapsto \Phi_{\beta_l}(u)\]
Leichte Übung: Dies erfüllt die Axiome für eine Skalarverknüpfung.
\end{application}

\begin{comment}
$V_L$ enthält $V$ als $K$-Untervektorraum über die Einbettung:
\[V\to V_L,v\mapsto 1\otimes v\]
Für eine Basis $B$ von $V$ ist das Bild $\{1\otimes b\mid b\in B\}\subseteq V_L$
eine Basis des $L$-VRms $V_L$. Insbesondere ist
\[L\otimes_K K^n\stackrel{\sim}{=}L^n\]
eine Isomorphie von $L$-VRmen.
\end{comment}

\chapter{Metriken, Normen und Skalarprodukte}
\begin{definition}
\index{Metrik}
\index{Abstand}
\index{positivdefinit}
\index{Dreiecksungleichung}
Metriken abstrahieren Abstände. Sei $X$ eine beliebige Menge. Eine Abbildung
$d:X\times X\to \mathbb{R}$ heißt eine \textbf{Metrik} oder ein \textbf{Abstand}
auf $X$, falls gilt:
\begin{enumerate}
\item $d$ ist \textbf{smmetrisch}, d.h. für alle $x,y\in X$ gilt:
\[d(x,y)=d(y,x)\]
\item $d$ ist \textbf{positivdefintit}, d.h. für alle $x,y\in X$ gilt:
\[d(x,y)\ge 0 \text{ und } (d(x,y)=0)\implies (x=y)\]
\item Für alle $x,y,z\in X$ gilt die \textbf{Dreicksungleichung}:
\[d(x,y)+d(y,z)\ge d(x,z)\]
\end{enumerate}
\end{definition}

\begin{example}
\index{diskret!Metrik}\index{Metrik!diskrete}
\begin{enumerate}
\item Die \textbf{diskrete Metrik} $d_0$ auf $X$:
\[d_0(x,y)=\begin{cases}
1&,(x\ne y)\\
0&,(x=y)
\end{cases}
\]
\item $X=\mathbb{C}$ oder $\mathbb{R}$ oder $\mathbb{Q}$ mit Betrag $|\cdot|$ hat die
Metrik:
\[d(x,y)=|x-y|\]
\end{enumerate}
Auf Vektorräumen über $K=\mathbb{R}$ entstehen natürliche Metriken aus sogenannten Normen.\\
Im Folgenden schreibe $\mathbb{K}$ für $\mathbb{R}$ oder $\mathbb{C}$.
\end{example}

\begin{definition}
\index{Norm}
\index{normiert!Raum}\index{Raum!normierter}
\index{Positivdefinitheit}
\index{Homogenität}
\index{Dreiecksungleichung}
Sei $V$ ein $\mathbb{K}$-VRm. Eine Abbildung:
\[\|\cdot\|:V\to\mathbb{R},x\mapsto \|x\|\]
heißt eine \textbf{Norm}, falls für $x,y\in V,\alpha\in\mathbb{K}$ gilt:
\begin{enumerate}
\item Die Abbildung ist \textbf{positivdefinit}, d.h.:
\[\|x\|\ge 0 \wedge (\|x\|=0\iff x=0)\]
\item Die Abbildung ist \textbf{homogen}, d.h.:
\[\|\alpha x\| = |\alpha|\cdot \|x\|\]
\item Es gilt die \textbf{Dreiecksungleichung}:
\[\|x+y\|\le\|x\|+\|y\|\]
\end{enumerate}
Ist dies erfüllt, so ist $(V,\|\cdot\|)$ ein \textbf{normierter Raum}.
\end{definition}

\begin{example}
\begin{enumerate}
\item Der Raum $V=\mathbb{K}$ mit $x=(x_1,\ldots,x_n)$ und den Normen:
\begin{align*}
&\text{(a) }\|x\|_2:=\sqrt{\sum_{i=1}^n |x_i|^2}\\
&\text{(b) }\|x\|_{\max}:= \max_{i=1,\ldots,n}|x_i|\\
&\text{(c) }\|x\|_1:=\sum_{i=1}^n |x_i|
\end{align*}
\item Der Raum der beschränkten Abbildungen $V=\Abb(M,\mathbb{K})$ mit einer 
beliebigen Menge $M$ und der Norm:
\[\|f\|_\infty := \sup_{m\in M}|f(m)|\]
\item Der Raum $V=C[a,b]$ der stetigen Abbildungen nach $\mathbb{K}$ mit der Norm:
\[\|f_1\|:=\int_a^b |f(t)|\text{ d}t\]
\end{enumerate}
\end{example}

\begin{comment}
\begin{enumerate}
\item Jeder normierte Raum $(V,\|\cdot\|)$ besitzt die Metrik:
\[d(x,y):=\|x-y\|\]
\item In der Linearen Algebra tauchen hauptsächlich Normen auf, die mit
\textbf{Skalarprodukten} definiert werden.
\end{enumerate}
\end{comment}

\begin{definition}
\renewcommand{\labelenumi}{(\alph{enumi})}
\index{Sesqilinearform}
\index{Schiefsymmetrie}
\index{hermitesch!Form}\index{Form!hermitesche}
\index{symmetrisch!Bilinearform}\index{Bilinearform!symmetrische}
\index{Skalarprodukt}
Sei $V$ ein $\mathbb{K}$-VRm. Für $\alpha\in\mathbb{K}$ sei $\overline\alpha$
die komplexe Konjugierte.
\begin{enumerate}
\item Eine Abbildung $s:V\times V\to\mathbb{K}$ heißt \textbf{Sesquilinearform}
("`sesqui"' bedeutet $1\frac12$), falls
\[s(\alpha x+y,z)=\alpha \cdot s(x,z)+s(y,z)\]
gilt, also $s(\cdot,z)$ linear ist und außerdem noch gilt:
\[s(x,\alpha y+z)=\overline\alpha \cdot s(x,y)+s(x,z)\]
\item Ist $s$ \textbf{schiefsymmetrisch}, d.h. es gilt:
\[s(y,x)=s(\overline{x,y})\]
so heißt es \textbf{hermitesche Form} ($\mathbb{K}=\mathbb{C}$) bzw.
\textbf{symmetrische Bilinearform} ($\mathbb{K}=\mathbb{R}$).
\item Eine schiefsymmetrische Sesquilinearform heißt \textbf{Skalarprodukt} auf $V$,
falls $s$ positivdefinit ist.
\end{enumerate}
\renewcommand{\labelenumi}{(\arabic{enumi})}
\end{definition}

\begin{comment}
\index{Darstellungsmatrix}
Ist $\dim V=n<\infty$ mit einer Basis $B=\{b_1,\ldots,b_n\}$, so ist eine
Sesquilinearform $s$ bestimmt durch die Werte $s(b_i,b_j)\in\mathbb{K}$, also
durch die \textbf{Darstellungsmatrix}
\[D_{BB}(s):=(s(b_i,bj))\in\mathbb{K}^{n\times n}\]
Jede beliebige Matrix $D=(d_{ij})\in\mathbb{K}^{n\times n}$ definiert ein
$s:=s_B^D$ mit:
\[s(\sum_{i=1}^n \alpha_i b_i,\sum_{j=1}^n \beta_j b_j)=\sum_{i=1}^n\sum_{j=1}^n \alpha_i\overline{\beta_j}\cdot d_{ij}\]
\end{comment}

\begin{definition}
\index{hermitesch}
\index{Positivdefinitheit}
$D$ heißt \textbf{hermitesch} (bzw. \textbf{positivdefinit}), wenn das zugehörige $s$
diese Eigenschaft hat.\\
Also gilt etwa:
\begin{align*}
&A=(a_{ij}) \text{ hermitesch}\\
\iff &\forall i,j=1,\ldots,n:a_{ij}=\overline{a_{ji}}\\
\iff &A=\overline{A}^T
\end{align*}
\end{definition}

\begin{example}
\index{Standardskalarproduk}
Das \textbf{Standardskalarprodukt} auf $V=\mathbb{K}^n$
\[D=I \implies s(\begin{pmatrix}\alpha_1\\\vdots\\\alpha_n \end{pmatrix},
\begin{pmatrix}\beta_1\\\vdots\\\beta_n \end{pmatrix})=\sum_{i=1}^n \alpha_i\overline{\beta_i}\]
ist hermitesch und positivdefinit.
\end{example}

\begin{theo}[Chauchy-Schwarzsche Ungleichung]
\index{Chauchy-Schwarzsche Ungleichung}
Ist $V$ VRm mit Skalarprodukt $\langle\cdot,\cdot\rangle$, so gilt für alle $x,y\in V$:
\[|\langle x,y\rangle|^2 \le \langle x,x\rangle \cdot \langle y,y\rangle\]
\end{theo}

\begin{proof}
Es gilt für alle $\alpha,\beta\in\mathbb{R}$:
\begin{align*}
0&\le \langle\alpha x+\beta y,\alpha x+\beta y\rangle\\
&=\alpha\langle x,\alpha x+\beta y\rangle+\beta\langle y,\alpha x+\beta y\rangle\\
&= \alpha^2\langle x,x\rangle +\alpha\beta(\langle x,y\rangle+\langle y,x\rangle)+
\beta^2\langle y,y\rangle
\end{align*}
Sei nun:
\begin{align*}
&F:=\langle x,x\rangle& &2G:=\langle x,y\rangle+\langle y,x\rangle& &H:=\langle y,y\rangle
\end{align*}
\textbf{Annahme:} $G^2>FH$\\
Dann hat $P(X):=FX^2+2GX+H\in\mathbb{R}[X]$ zwei verschiedene Nullstellen und es
existiert ein $\xi\in\mathbb{R}:P(\xi)<0$. Dies stellt einen Widerspruch zu obiger Überlegung da
(mit $\alpha=\xi$ und $\beta=1$). Also gilt:
\[G^2\le FH\]
\textbf{Fall 1:} $\langle x,y\rangle \in\mathbb{R}$\\
Dann gilt $G=\langle x,y\rangle=\langle y,x\rangle$ und mit $G^2\le FH$ folgt
die Cauchy-Schwarzsche Ungleichung.\\
\textbf{Fall 2:} $\langle x,y\rangle \notin\mathbb{R}$\\
Ersetze $y$ durch $\zeta	 y$ mit $\zeta\in\mathbb{C}^\times,|\zeta|=1$.
Wähle $\zeta:=\frac{\langle x,y\rangle}{|\langle x,y\rangle|}$. Dann gilt:
\begin{align*}
\langle x,\zeta y\rangle &= \overline{\zeta}\langle x,y\rangle\\
&=\frac{\langle\overline{x,y}\rangle}{|\langle x,y \rangle |}\cdot \langle  x,y\rangle\\
&=\frac{|\langle x,y \rangle |^2}{|\langle x,y \rangle |}\\
&=|\langle x,y \rangle |\in\mathbb{R}
\end{align*}
Nach Fall 1 folgt daraus:
\begin{align*}
|\langle x,y \rangle |^2&=|\langle x,\zeta y \rangle |^2\\
&\le \langle x,x\rangle\cdot\langle\zeta y,\zeta y\rangle\\
&=\langle x,x\rangle\cdot \zeta\overline\zeta\cdot\langle y,y\rangle\\
&=\langle x,x\rangle\cdot \langle y,y\rangle
\end{align*}
\end{proof}

\begin{comment}
Im Spezialfall $V=\mathbb{K}^n$ mit dem Standardskalarprodukt $\langle\cdot,\cdot\rangle$
gilt nach der Cauchy-Schwarzschen Ungleichung:
\[|\sum_{i=1}^n \xi_i\cdot\overline{\eta_i}|^2\le (\sum_{i=1}^n |\xi_i|^2)(\sum_{j=1}^n |\eta_j|^2)\]
\end{comment}

\begin{theo}
Jeder VRm $V$ mit einem Skalarprodukt ist normiert durch die Norm:
\[\|x\|:=\sqrt{\langle x,x\rangle}\]
\end{theo}

\begin{proof}
\begin{enumerate}
\item Es ist klar, das $\|x\|\in\mathbb{R}$ und $\|x\|\ge 0$ ist. Außerdem gilt:
\begin{align*}
\|x\|=0 &\implies \langle x,x\rangle =0\\
&\implies x=0
\end{align*}
\item  Es gilt:
\begin{align*}
&\langle x,y\rangle +\langle y,x\rangle\le 2\|x\|\cdot\|y\|\\
\iff &\langle x,x\rangle\langle x,y\rangle +\langle y,x\rangle\langle y,y\rangle
\le \langle x,x\rangle 2\|x\|\cdot\|y\|+\langle y,y\rangle\\
\iff &\langle x+y,x+y\rangle \le (\|x\|+\|y\|)^2\\
\iff &\|x+y\|\le \|x\|+\|y\|
\end{align*}
\item Es gilt:
\[\|\alpha x\|^2=\langle \alpha x,\alpha x\rangle = \alpha\overline\alpha
\langle x,x\rangle = |\alpha|^2\cdot\|x\|^2\]
\end{enumerate}
\end{proof}

\begin{comment}
\index{Parallelogrammgleichung}
\begin{enumerate}
\item Mit Hilfe der Norm lautet die Cauchy-Schwarzsche Ungleichung:
\[|\langle x,y\rangle|\le \|x\|\cdot\|y\|\]
\item Damit eine Norm von einem Skalarprodukt stammt, ist offenbar notwendig,
dass sie die \textbf{Parallelogrammgleichung} erfüllt:
\[\forall x,y\in V: \|x+y\|^2+\|x-y\|^2=2(\|x\|^2+\|y\|^2)\]
Denn falls die Norm $\|\cdot\|$ von einem Skalarprodukt $\langle\cdot,\cdot\rangle$
kommt, gilt:
\begin{align*}
\|x+y\|^2+\|x-y\|^2 &= \langle x+y,x+y\rangle +\langle x-y,x-y\rangle\\
&= \|x\|^2+\langle x,y\rangle +\langle y,x\rangle+\|y\|^2 + 
\|x\|^2-\langle x,y\rangle -\langle y,x\rangle+\|y\|^2\\
&= 2(\|x\|^2+\|y\|^2)
\end{align*}
Tatsächlich kommt eine Norm genau dann von einem Skalarprodukt, wenn sie die
Parallelogrammgleichung erfüllt. Dies wird jedoch ohne Beweis angegeben.
\end{enumerate}
\end{comment}

\chapter{Orthogonalsysteme}
\index{Orthogonalsysteme}
\section{Winkel und Orthogonalit\"at}
\index{Winkel}\index{orthognal}
\begin{prerem}
Sei \(V\) ein Vektorraum mit Skalaprodukt \(\SKP\) und zugeh\"origer Norm 
\(\lVert\cdot\rVert\), dann gilt nach Cauchy-Schwarz:
\[
\forall x,y\in V\setminus\{0\}:\quad\frac{\lvert\langle x,y\rangle\rvert}{\lVert x\rVert\cdot\lVert y\rVert}\leq 1
\]
\end{prerem}
\begin{definition}
\begin{enumerate}
\item Sei \(\mathbb{K}=\mathbb{R}\). F\"ur \(x,y\in V\setminus\{0\}\) sei 
\(\phi=\angle(x,y)\in[0,\pi]\) diejenige (eindeutig bestimmte) Zahl mit
\[
\cos\phi=\frac{\lvert\langle x,y\rangle\rvert}{\lVert x\rVert\cdot\lVert y\rVert}
\]
\(\phi\) hei\ss t der \textbf{Winkel} zwischen \(x\) und \(y\).
\item \(x,y\) hei\ss en \textbf{orthogonal} oder senkrecht zueinander, falls
\(\langle x,y\rangle=0\) gilt.\\
Schreibe: \(x\bot y\).
\item Teilmengen \(M,N\subseteq V\) hei\ss en orthogonal, falls gilt:
\[
\forall x\in M\forall y\in N:\quad x\bot y
\]
Schreibe: \(M\bot N\).
\item Eine Teilmenge \(B\subseteq V\) hei\ss t \textbf{Orthogonalsystem} (OGS),
falls f\"ur \(x,y\in B\) gilt: 
\[
x\neq y\implies x\bot y
\]
\item Ein Orthogonalsystem \(B\) hei\ss t \textbf{Orthonormalsystem} (ONS) wenn
gilt:
\[
\forall x\in B:\quad\lVert x\rVert=1
\]
\item Eine Basis \(B\) von \(V\) hei\ss t \textbf{Orthogonalbasis} (OGB), bzw.
\textbf{Orthonormalbasis} (ONB), falls \(B\) ein Orthogonalsystem, bzw.
Orthonormalsystem, ist.
\end{enumerate}
\end{definition}
\begin{example}
\begin{enumerate}
\item Sei \(V=\mathbb{K}^{n}\) mit Standardskalarprodukt \(\SKP\) und 
\(S:=\{e_{1},\ldots,e_{n}\}\) Standardbasis.\\
Dann ist \(S\) eine Orthonormalbasis und jede Teilmenge \(T\subseteq S\) ist
ein Orthonormalsystem.
\item Sei \(I:=[a,b]\) ein Intervall.\\
Sei 
\(V:=\{p\in\Abb(I,\mathbb{C})\mid\exists P\in\mathbb{C}[T]:\,p(t)=P(t)\}\).\\
\(w:I\to\mathbb{R}_{\geq0}\) sei stetig und mit der Eigenschaft \(w(t)=0\) nur
f\"ur endlich viele \(t\in I\).

Wir erhalten ein Skalarprodukt auf \(V\):
\[
\langle p,q\rangle_{w}:=\int\limits_{I}{w(t)p(t)\overline{q(t)}\mathrm{d}t}
\]
Eine Basis von \(V\) ist \(\{p_{n}(t)=:t^{n}\mid n\in\mathbb{N}_{0}\}\).

Gesucht ist eine Orthonormalbasis und ein Verfahren zu ihrer Bestimmung.
\end{enumerate}
\end{example}
\begin{comment}
Jedes Orthogonalsystem \(B\) mit \(0\not\in B\) ist linear unabh\"angig.
\end{comment}
\begin{proof}
Es ist
\[
\sum_{b\in B}{\alpha_{b}\cdot b}=0
\]
Dann gilt f\"ur alle \(c\in B\):
\begin{align*}
0&=\langle 0,c\rangle\\
&=\left\langle\sum_{b}{\alpha_{b}b},c\right\rangle\\
&=\sum_{b}{\alpha_{b}\langle b,c\rangle}\\
&\overset{b=0\forall b\neq c}{=}\alpha_{c}\underbrace{\langle c,c\rangle}_{\neq c}\\
&\implies\,\alpha_{c}=0
\end{align*}
\end{proof}

\section{Das E. Schmidtsche Orthogonalisierungsverfahren}
\index{E. Schmidt!Orthogonalisierungsverfahren}\index{Gram-Schmidt}
\begin{theo}
Sei \(V\) ein Vektorraum mit Skalarprodukt \(\SKP\) und sei 
\(M:=\{x_{0},x_{1},\ldots\}\) eine abz\"ahlbare Teilmenge von \(V\).
\begin{enumerate}
\item Es existiert ein Orthogonalsystem \(\{y_{0},y_{1},\ldots\}\) derart, 
da\ss \ gilt:
\begin{equation}
\label{eq: Gram-Schmidt; lineare Huelle}
\forall n:\quad\langle y_{0},y_{1},\ldots,y_{n}\rangle=\langle x_{0},\ldots,x_{n}\rangle\quad\text{(gleiche lineare H\"ulle)}
\end{equation}
\item Falls \(M\) linear unabh\"angig ist, so sind alle \(y_{i}\neq0\) und
\(B:=\{z_{0},z_{1},\ldots\}\) mit
\[
z_{i}:=\frac{1}{\lVert y_{i}\rVert}y_{i}
\]
ist ein Orthonormalsystem mit 
\[
\forall n:\quad\langle z_{0},z_{1},\ldots,z_{n}\rangle=\langle x_{0},x_{1},\ldots,x_{n}\rangle
\]
\end{enumerate}
\end{theo}
\begin{proof}
\begin{enumerate}
\item Wir beschreiben einen Algorithmus zum Auffinden der \(y_{n}\).\\
Start: \(y_{0}:=x_{0}\). Angenommen: alle \(y_{m}\) f\"ur \(m<n\) sind bereits
gefunden. Setze
\[
y_{n}:=x_{n}-\sum_{i=0}^{n-1}{\frac{\langle x_{n},y_{i}\rangle}{\langle y_{i},y_{i}\rangle}}y_{i}
\]
(nur \"uber \(i\) mit \(y_{i}\neq0\) summieren).\\
Damit folgt:
\begin{align*}
y_{n}&\in\langle\underbrace{y_{0},y_{1},\ldots,y_{n-1}}_{=\langle x_{0},\ldots,x_{n-1}\rangle},x_{n}\rangle=\langle x_{0},\ldots,x_{n}\rangle\\
x_{n}&\in\langle y_{0},y_{1},\ldots,y_{n}\rangle
\end{align*}
Daraus folgt \eqref{eq: Gram-Schmidt; lineare Huelle}.

Rest: F\"ur alle \(m<n:\,y_{n}\bot y_{m}\). Damit:
\begin{align*}
\langle y_{n},y_{m}\rangle&=\langle x_{n},y_{m}\rangle-\sum_{i=0}^{n-1}{\frac{\langle x_{n},y_{i}\rangle}{\langle y_{i},y_{i}\rangle}}\langle y_{i},y_{m}\rangle\\
&=\langle x_{n},y_{m}\rangle-\sum_{i=0}^{n-1}{\frac{\langle x_{n},y_{i}\rangle}{\langle y_{i},y_{i}\rangle}}\delta_{im}\langle y_{i},y_{i}\rangle\\
&=\langle x_{n},y_{m}\rangle-\langle x_{n},y_{m}\rangle\\
&=0
\end{align*}
\item \checkmark \ Leicht selbst zu verifizieren.
\end{enumerate}
\end{proof}
\begin{example}
Sei \(V=\mathbb{R}^{3}\) mit dem Standardskalarprodukt, 
\(M:=\left\{
\begin{pmatrix}1\\0\\1\end{pmatrix},
\begin{pmatrix}1\\1\\0\end{pmatrix},
\begin{pmatrix}0\\1\\1\end{pmatrix}
\right\}\).
Dann:
\begin{align*}
y_{0}&=x_{0},\quad\lVert y_{0}\rVert=\sqrt{2}\\
y_{1}&=x_{1}-\frac{\langle x_{1},y_{0}\rangle}{\langle y_{0},y_{0}\rangle}y_{0}
    =\begin{pmatrix}1\\1\\0\end{pmatrix}-\frac{1}{2}\begin{pmatrix}1\\0\\1\end{pmatrix}
    =\frac{1}{2}\begin{pmatrix}1\\2\\-1\end{pmatrix},
    \quad\lVert y_{1}\rVert=\sqrt{\frac{3}{2}}\\
y_{2}&=x_{2}-\frac{\langle x_{2},y_{0}\rangle}{\langle y_{0},y_{0}\rangle}y_{0}-
    \frac{\langle x_{2},y_{1}\rangle}{y_{1},y_{1}\rangle}y_{1}
    =\frac{2}{3}\begin{pmatrix}-1\\1\\1\end{pmatrix}
\end{align*}
\end{example}

Auswirkungen (des Orthogonalisierungsverfahrens) auf Matrizen:\\
Sei \(V\cong\mathbb{K}^{n}\) mit Standardskalarprodukt \(s\) und Basis
\(B:=\{b_{1},\ldots,b_{n}\}\).
Das Orthogonalisierungsverfahren liefert eine Orthogonalbasis 
\(C=\{c_{1},\ldots,c_{n}\}\).
\begin{align*}
c_{\nu}&=b_{\nu}-\sum_{i=1}^{n-1}{\frac{\SKP}{\SKP}\cdot c_{i}}=\ldots
    =\sum_{i=1}^{\nu}{\alpha_{i\nu}b_{i}}
    \longrightarrow A=M_{BC}=\left(\alpha_{i\nu}\right)=
    \begin{pmatrix}1&&*\\&\ddots&\\0&&1\end{pmatrix}\\
\frac{c_{\nu}}{\lVert c_{\nu}\rVert}
    &=\sum_{i=1}^{\nu}{\frac{\alpha_{i\nu}}{\lVert c_{i}\rVert}b_{i}}=z_{\nu}
    \longrightarrow M_{BZ}=
	\begin{pmatrix}*&&*\\&\ddots&\\0&&*\end{pmatrix}
\end{align*}
\begin{remind}[Darstellungsmatrix]
\[
D_{BB}(s)=\left(s\left(b_{\nu},b_{\mu}\right)\right)\in\mathbb{K}^{n}\\
\]
Da \(C\) eine Orthogonalbasis ist, folgt
\(s\left(c_{\nu},c_{\mu}\right)=\delta_{\nu\mu}\lVert c_{\nu}\rVert^{2}\), also
\[
D_{CC}(s)=\diag\left(\ldots,\lVert c_{\nu}\rVert^{2},\ldots\right)
\]
Falls \(Z\) eine Orthonormalbasis ist, so folgt \(D_{ZZ}(s)=I\).

Generell f\"ur beliebige Basen \(B,C\) und \(A=M_{BC}\):
\begin{align*}
D_{CC}(s)&=\left(s\left(c_{\nu},c_{\mu}\right)\right)\\
&=\left(s\left(\sum_{i=1}^{n}{\alpha_{i\nu}},\sum_{j}{\alpha_{j\mu}b_{j}}\right)\right)\\
&=\left(\sum_{i}{\sum_{j}{\alpha_{i\nu}\overline{\alpha_{i\mu}}\cdot s(b_{i},b_{j})}}\right)\\
&=A^{\top}\left(s(b_{i},b_{j})\right)\overline{A}
\end{align*}
\end{remind}
\begin{definition}
F\"ur beliebige \(D\in\mathbb{K}^{n\times n}\) setze \(D^{*}:=\overline{D}^{\top}\) (die sogenannte \textbf{Adjungierte}).
\[
D_{CC}(s)=A^{\top}D_{BB}(s)\overline{A}
\]
Speziell f\"ur jede Orthonormalbasis \(C\):
\[
D_{CC}(s)=I,
\]
das folgt wegen \(M_{BC}^{-1}=M_{CB}\).\\
Es ist 
\[
D_{BB}(s)=D^{*}D
\]
f\"ur \(D:=\overline{M}_{CB}\), wobei \(D\) obere Dreiecksmatrix ist.
\end{definition}
\begin{theo}
F\"ur \(P\in\mathbb{K}^{n\times n}\) ist \"aquivalent:
\begin{enumerate}
\item \(P\) ist hermitesch (symmetrisch) und positiv definit
\item Es gibt ein \(A\in\GL_{n}(\mathbb{K})\) mit \(P=A^{*}A\).
\end{enumerate}
\end{theo}
\begin{proof}
\begin{enumerate}
\item \(\implies\,(2)\):\\
Sei \(V=\mathbb{K}^{n},\,B=\{e_{1},\ldots,e_{n}\}\) die Standardbasis.\\
\(P\) definiert ein Skalarprodukt:
\begin{align*}
s(x,y)&=(\overline{\eta}_{1},\ldots,\overline{\eta}_{n})\cdot P\cdot
    \begin{pmatrix}\xi_{1}\\\vdots\\\xi_{n}\end{pmatrix}\\
P&=\left(s(e_{i},e_{j})\right)=D_{BB}(s)
\end{align*}
% Okay, das hier ist leicht seltsam -- mein Aufschrieb ist an der Stelle etwas
% wirr und unverstaendlich... Fuer einen Hinweis bin ich dankbar ;-) P.
Das Schmidtsche Orthogonalisierungsverfahren liefert eine Orthonormalbasis und
damit \(P=D^{*}D\).
\item\(\implies\,(1)\):\\
\(P=A^{*}A\) ist hermitesch; zu zeigen: \(P^{*}=P\)
\begin{align*}
P^{*}&=\left(A^{*}\cdot A\right)^{*}\\
&=\overline{\left(A^{*}\cdot A\right)}^{\top}\\
&=\overline{\left(\overline{A}^{\top}\cdot A\right)}^{\top}\\
&=\left(A^{\top}\cdot\overline{A}\right)^{\top}\\
&=\overline{A}^{\top}\cdot A\\
&=A^{*}A\\
&=P
\end{align*}
Daraus folgt: \(s(x,y)\) ist hermitesche Form. F\"ur alle \(x\in\mathbb{K}^{n}\)
gilt:
\begin{align*}
s(x,x)&=\overline{x}^{\top}\cdot P\cdot x\\
&=\overline{x}^{\top}\left(\overline{A}^{\top}\cdot A\right)x\\
&=\left(\overline{A}\overline{x}\right)^{\top}Ax\\
&=s_{0}(Ax,Ax)\\
&\geq0
\end{align*}
Weiterhin:
\begin{align*}
&s_{0}(Ax,Ax)=0\\
\Longleftrightarrow&Ax=0\\
\overset{A\text{ inv.}}{\Longleftrightarrow}&x=0\\
\implies&s\text{ positiv definit}
\end{align*}
\end{enumerate}
\end{proof}

Falls speziell \(B\) und \(C\) Orthonormalbasen sind, folgt:
\[
D_{BB}(s)=I=D_{CC}(s)
\]
und \(D:=\overline{M}_{CB}\).
\begin{conclusion}
Die Basiswechselmatrix \(A=M_{BC}\) einer Orthonormalbasis \(C\) in eine andere
Orthonormalbasis \(B\) geh\"ort zur \textbf{orthogonalen Gruppe}
\[
O(n):=\left\{A\in\GL_{n}(\mathbb{R})\mid A^{\top}\cdot A=I\right\}\quad
    \text{f\"ur } \mathbb{K}=\mathbb{R}
\]
beziehungsweise zur \textbf{unit\"aren Gruppe}
\[
U(n):=\left\{A\in\GL_{n}(\mathbb{C})\mid A^{\top}\cdot\overline{A}=I\right\}
    \quad\text{f\"ur } \mathbb{K}=\mathbb{C}
\]
\end{conclusion}
\begin{comment}
\(O(n)\), beziehungsweise \(U(n)\), ist eine Untergruppe von 
\(\GL_{n}(\mathbb{R})\), beziehungsweise \(\GL_{n}(\mathbb{C})\).
\end{comment}
\begin{conclusion}[Iwasawa-Zerlegung]
\index{Iwasawa-Zerlegung}
Jede Matrix \(g\in\GL_{n}(\mathbb{R})\) hat eine eindeutige Produktzerlegung
\[
g=k\cdot b
\]
mit \(k\in O(n)\) und \(b\in B(n):=\left\{
    \begin{pmatrix}\beta_{1}&&*\\&\ddots&\\0&&\beta_{n}\end{pmatrix}\mid
    \beta_{\nu}>0\right\}\).
Das hei\ss t:
\[
\GL_{n}(\mathbb{R})=O(n)\cdot B(n)
\]
Analog gilt: 
\[
\GL_{n}(\mathbb{C})=U(n)\cdot B(n)_{\mathbb{C}}
\]
mit
\[
B(n)_{\mathbb{C}}:=\left\{
    \begin{pmatrix}\beta_{1}&&*\\&\ddots&\\0&&\beta_{n}\end{pmatrix}\in
	\GL_{n}(\mathbb{C})\mid\beta_{\nu}\in\mathbb{R}_{>0}
    \right\}
\]
\end{conclusion}
\begin{proof}
F\"ur \(\mathbb{K}=\mathbb{R}\,g\in\GL_{n}(\mathbb{R})\) folgt: Die Spalten
\(b_{1},\ldots,b_{n}\) sind eine Basis von \(\mathbb{R}^{n}\).\\
Das Schmidtsche Orthogonalisierungsverfahren liefert eine Orthonormalbasis
\(\{c_{1},\ldots,c_{n}\}\) mit \"Ubergangsmatrix \(A=M_{BC}\in B(n)\). Denn:
\[
c_{\nu}:=\sum_{i=1}^{\nu}{\alpha_{i\nu}b_{i}}
\]
besagt \(g\cdot A=(c_{1},\ldots,c_{n})\) und \(k=(c_{1},\ldots,c_{n})\in O(n)\),
da \(k^{\top}\cdot k=\left(\langle c_{i},c_{j}\rangle\right)=I\).
\end{proof}

\section{Orthogonale Projektion und orthogonales Komplement}
\index{Projektion!orthogonale}\index{Komplement!orthogonales}
\begin{theo}[Satz von Pyhtogoras]
\index{Pythagoras!Satz von}
F\"ur \(x,y\in V\) mit \(x\bot y\) gilt:
\[
\lVert x+y\rVert^{2}=\lVert x\rVert^{2}+\lVert y\rVert^{2}
\]
\end{theo}
\begin{proof}
\begin{align*}
\lVert x+y\rVert^{2}&=\langle x+y,x+y\rangle\\
&=\langle x,x\rangle+\underbrace{\langle x,y\rangle}_{=0}+
    \underbrace{\langle y,x\rangle}_{=0}+\langle y,y\rangle\\
&=\lVert x\rVert^{2}+\lVert y\rVert^{2}
\end{align*}
\end{proof}
Ist \(\{x_{1},\ldots,x_{N}\}\) ein Orthogonalsystem, so folgt
\[
\left\lVert\sum_{\nu=1}^{N}{x_{\nu}}\right\rVert^{2}=\sum_{\nu=1}^{N}\lVert x_{\nu}\rVert^{2}
\]
Der Beweis folgt leicht mit vollst\"andiger Induktion.
\begin{theo}
Sei \(U\leq V\) mit \(\dim V<\infty\).
\begin{enumerate}
\item F\"ur alle \(x\in V\) existiert genau ein \(y\in U\) mit 
\(d:=\lVert x-y\rVert=\min\{\lVert x-u\rVert\mid u\in U\}\).
\item Dieses \(y\in U\) ist auch charakterisiert durch: \(x-y\bot U\).\\
Schreibe: \(y=:\Pi_{U}(x)\).
\item Die Abbildung \(\Pi_{U}\in\End(V)\) ist stetig; es gilt 
\(\Pi_{U}^{2}=\Pi_{U}\) und \(\lVert\Pi_{U}(x)\rVert\leq\lVert x\rVert\).\\
\index{Abstand}\index{Projektion!orthogonale}\index{Lot}\index{Lotfu\ss punkt}
\(d\) hei\ss t \textbf{Abstand} von \(x\) und \(U\), \(y=\Pi_{U}(X)\) die
\textbf{orthogonale Projektion} von \(x\) auf \(U\), \(z:=x-y\) hei\ss t 
\textbf{Lot} von \(x\) auf \(U\), \(y\) \textbf{Lotfu\ss punkt}.
\end{enumerate}
\end{theo}
% \(V=\mathbb{R}^{3},\,U=\langle u\rangle\)
% Hier kommt noch eine Zeichnung hin.
\begin{proof}
\begin{enumerate}
\item W\"ahle eine Orthonormalbasis \(S=\{e_{i}\mid i=1,\ldots,r\}\) in \(U\).
Setze \(y=\Pi_{U}(x):=\sum_{i=1}^{r}\langle x,e_{i}\rangle e_{i}\).
\begin{thesis}
\(\forall u'\in U:\quad x-y\bot y-u'\)
\end{thesis}
\begin{align*}
\langle x-y,y-u'\rangle&=\underbrace{\langle x,y\rangle-\langle y,y\rangle}_{\overset{!}{=}0}+\underbrace{\langle y,u'\rangle-\langle x,u'\rangle}_{\overset{!}{=}0}\\
\langle x,y\rangle&=\left\langle x,\sum_{i}{\langle x,e_{i}\rangle e_{i}}\right\rangle
    =\sum_{i}{\overline{\langle x,e_{i}\rangle}\langle x,e_{i}\rangle}
% <y,y> !?
\end{align*}
\(u'\) in Basisdarstellung: Mit \(u'=\sum_{j}{\alpha_{j}e_{j}}\) folgt
\begin{align*}
\langle y,u'\rangle&=\left\langle\sum_{i}{\langle x,e_{i}\rangle e_{i}},\sum_{j}{\alpha_{j}e_{j}}\right\rangle\\
&=\sum_{i}{\langle x,e_{i}\rangle\overline{\alpha}_{i}}
\end{align*}
Weiterhin gilt:
\begin{align*}
\langle x,u'\rangle&=\left\langle x,\sum_{j}{\alpha_{j}e_{j}}\right\rangle\\
&=\sum_{j}{\overline{\alpha}_{j}\langle x,e_{j}\rangle}
\end{align*}
Mit Pythagoras folgt:
\begin{align*}
\lVert x-u'\rVert^{2}&=\lVert x-y+y-u'\rVert^{2}\\
&=\lVert x-y\rVert^{2}+\underbrace{\lVert y-u'\rVert^{2}}_{\geq0}\\
&\geq\lVert x-y\rVert^{2}
\end{align*}
Es ist also \(\lVert x-u'\rVert\geq\lVert x-y\rVert\), wobei Gleichheit genau
f\"ur \(y-u'=0\) gilt. Damit folgt die Eindeutigkeit von \(y\).
\item Sei \(y\in U\) und \(x-y\bot U\). Dann gilt 
\(\langle x,e_{i}\rangle=\langle y,e_{i}\rangle\) f\"ur alle \(i\).\\
Es folgt:
\begin{align*}
y&=\sum_{i}{\langle y,e_{i}\rangle e_{i}}\\
&=\sum_{i}{\langle x,e_{i}\rangle e_{i}}\\
&=\Pi_{U}(x)
\end{align*}
\item Aus \(x-y\bot y\) folgt mit Pythagoras:
\begin{align*}
\lVert x\rVert^{2}&=\lVert x-y\rVert^{2}+\lVert y\rVert^{2}\\
&\geq\lVert y\rVert^{2}\\
&=\left\lVert\Pi_{U}(x)\right\rVert^{2}
\end{align*}
Es folgt: \(\Pi_{U}\) ist (Lipschitz-)stetig.\\
\(\Pi_{U}^{2}=\Pi_{U}\) ist leicht selbst zu verifizieren.
\end{enumerate}
\end{proof}
\begin{definition}
\index{Komplement!orthogonales}\index{Orthogonalraum}
Sei \(M\subseteq V\) Teilmenge. Der Vektorraum
\[
M^{\bot}:=\left\{y\in V\mid y\bot M\right\}
\]
hei\ss t \textbf{Orthogonalraum} oder \textbf{orthogonales Komplement} von 
\(M\).
\end{definition}
\begin{lemma}
\begin{enumerate}
\item \(M_{1}\subseteq M_{2}\implies M_{1}^{\bot}\geq M_{2}^{\bot}\)
\item \(\langle M\rangle^{\bot}=M^{\bot}\)
\item Aus \(M_{i}\subseteq V,\,(i=1,\ldots,n)\) folgt 
\[
\left(\bigcup_{i=1}^{n}M_{i}\right)^{\bot}=\bigcap_{i=1}^{n}M_{i}^{\bot}
\]
\item Aus \(U_{i}\leq V\) (Teilr\"aume) folgt
\[
\left(\bigcap_{i=1}^{n}U_{i}\right)^{\bot}\geq\sum_{i=1}^{n}{\left(U_{i}^{\bot}\right)}
\]
\item \(\langle M\rangle\leq\left(M^{\bot}\right)^{\bot}\) und
\(M^{\bot}=\left(\left(M^{\bot}\right)^{\bot}\right)^{\bot}\).
\item Im Spezialfall \(\dim V<\infty\) gilt:
\begin{enumerate}
\item Mit \(U\leq V\) folgt \(V=U\oplus U^{\bot}\) (insbesondere \(\dim U+\dim U^{\bot}=\dim V\)) und \(\left(U^{\bot}\right)^{\bot}=U\)
\item Mit \(U_{i}\leq V\) folgt 
\(\left(\bigcap_{i=1}^{n}U_{i}\right)^{\bot}=\sum_{i=1}^{n}{\left(U_{i}^{\bot}\right)}\)
\end{enumerate}
\end{enumerate}
\end{lemma}
\begin{proof}
\"Ubung!
\end{proof}

\chapter{Normale Endomorphismen}
\index{Endomorphismus}\index{normaler Endomorphismus}
\section{Die adjungierte lineare Abbildung}
\index{lineare Abbildung}\index{adjungierter Homomorphismus}\index{Adjungierte}
Seien \(V,\,W \mathbb{K}\)-Vektorr\"aume mit Skalarprodukt 
\(\langle\cdot,\cdot\rangle_{V},\,\langle\cdot,\cdot\rangle_{W}\)
\begin{lemma}
Sei \(\phi\in\Hom(V,W)\). Falls \(\Psi\in\Hom(W,V)\) mit der Eigenschaft
\[
\langle\phi(x),y\rangle_{W}=\langle x,\Psi(y)\rangle_{V}\,\forall x\in V,\,
y\in W,
\]
so ist \(\Psi\) hierdurch eindeutig bestimmt.
\end{lemma}
\begin{proof}
Sei \(\Psi':W\to V\) ein Homomorphismus mit derselben Eigenschaft\\
\(\implies\) F\"ur \(\Omega:=\Psi-\Psi'\in\Hom(W,V)\) gilt:
\begin{align*}
\forall x\in V,\,y\in W: \langle x,\Omega(y)\rangle_{V}&=
\langle x,\Psi(y)-\Psi'(y)\rangle_{V}\\
&=\langle x,\Psi(y)\rangle_{V}-\langle x,\Psi'(y)\rangle_{V}\\
&=\langle\phi(x),y\rangle_{W}-\langle\phi(x),y\rangle_{W}\\
&=0
\end{align*}
\(\implies \langle\Omega(y),\Omega(y)\rangle_{V}=0\quad\implies
\Omega(y)=0\,\forall y\)\\
Also: \(\Omega=0\), d.h. \(\Psi =\Psi'\).
\end{proof}

\begin{definition}
Falls \(\Psi\) existiert wie oben, so hei\ss t \(\Psi\) der zu \(\phi\) 
adjungierte Homomorphismus.\\
Schreibe: \(\Psi =:\phi^{*}\qquad\Hom^{a}(V,W):=\{\phi\in\Hom(V,W)\mid\phi^{*}\text{existiert}\}\)
\end{definition}

\begin{example}
\(V=\mathbb{K}^{n},\,W=\mathbb{K}^{m}\) mit Standardskalarprodukt.\\
\(A\in\mathbb{K}^{n\times m},\,\phi:=\Lambda_{A}:x\mapsto A\cdot x\)
\[
\langle\phi(x),y\rangle_{W}=\langle Ax,y\rangle_{W}=\overline{y}^{T}Ax=
(y^{*}A)x=(A^{*}y)^{*}=\langle x,A^{*}y\rangle_{V}=\langle x,\Lambda_{A^{*}}(y)\rangle
\]
Das hei\ss t: \((\Lambda_{A})^{*}=\Lambda_{A*}\). Insbesondere existiert die 
Adjungierte.
\end{example}
\begin{proposition}
\begin{enumerate}
\item \(\Hom^{a}(V,W)\leq\Hom(V,W)\)
\item F\"ur die Abbildung \(*:\,\Hom^{a}(V,W)\to\Hom(W,V),\,\phi\mapsto\phi^{*}\)
gilt:
\[
(\alpha\phi+\beta\Psi)^{*}=\overline{\alpha}\phi^{*}+\overline{\beta}\Psi^{*}
\]
Die Abbildung ist \textbf{semilinear}.
\item Aus \(\phi\in\Hom^{a}(V,W),\,\Theta\in\Hom^{a}(W,U)\) folgt
\(
\Theta\circ\phi\in\Hom^{a}(V,U)
\)
und
\(
(\Theta\circ\phi)^{*}=\phi^{*}\circ\Theta^{*}
\)
\item Aus \(\phi\in\Hom^{a}(V,W)\) folgt \(\phi^{*}\in\Hom^{a}(W,V)\) und
\((\phi^{*})^{*}=\phi\), sowie \(\Kern\phi=\Bil(\phi^{*})^{\bot}\).
\end{enumerate}
\end{proposition}
\begin{proof}
\begin{enumerate}
\item +(2) % ja, das ist h�sslich; kommt Zeit kommt bessere L�sung
Sei \(\phi,\Psi\in\Hom^{a}(V,W),\,\alpha,\beta\in\mathbb{C}\).\\
\(\overline{\alpha}\phi^{*}+\overline{\beta}\Psi^{*}\) ist die Adjungierte zu
\(\alpha\phi+\beta\Psi\), denn
\begin{align*}
\langle(\alpha\phi+\beta\Psi)(x),y\rangle
    &=\alpha\underbrace{\langle\phi(x),y\rangle}_{\langle x,\phi^{*}(y)\rangle}
    +\beta\underbrace{\langle\Psi(x),y\rangle}_{\langle x,\Psi^{*}(y)\rangle}\\
&=\langle x,\overline{\alpha}\phi^{*}(y)+\overline{\beta}\Psi^{*}(y)\rangle
\end{align*}
\item[(3)] F\"ur alle \(x\in V,\,y\in U\) gilt:
\begin{align*}
\langle\Theta\circ\phi(x),y\rangle&=\langle\Theta(\phi(x)),y\rangle\\
&=\langle\phi(x),\Theta^{*}(y)\rangle\\
&=\langle x,\phi^{*}(\Theta^{*}(y))\rangle
\end{align*}
\item[(4)] Es gilt
\begin{align*}
\langle\phi^{*}(y),x\rangle&=\overline{\langle x,\phi^{*}(y)\rangle}\\
&=\overline{\langle\phi(x),y\rangle}\\
&=\langle y,\phi(x)\rangle
\end{align*}
Das hei\ss t \(\phi^{*}\) hat die Adjungierte \(\left(\phi^{*}\right)^{*}=\phi\)

Weiterhin gilt:
\begin{align*}
x\in\Kern(\phi)&\Longleftrightarrow\phi(x)=0\\
&\Longleftrightarrow\forall y\in W:\,\underbrace{\langle\phi(x),y\rangle}_{\langle x,\phi^{*}(y)\rangle}=0\\
&\Longleftrightarrow x\bot\phi^{*}(w)
\end{align*}
\end{enumerate}
\end{proof}

Sei \(V\) ein Vektorraum mit Skalarprodukt \(\langle\cdot,\cdot\rangle,
\,\phi\in\End(V)\) mit \(\langle\phi(x),y\rangle=
\langle x,\phi^{*}(y)\rangle\).
\begin{lemma}
Sei \(\dim V<\infty,\,\phi\in\End(V)\). Dann gilt:
\[
\lambda\in\Spec(\phi)\implies\overline{\lambda}\in\Spec(\phi^{*})
\]
\end{lemma}
\begin{proof}
Sei \(u\neq0,\,\phi(u)=\lambda\cdot u\). Dann gilt f\"ur alle \(y\in V\):
\[
0=\langle(\phi-\lambda\id)(u),y\rangle=\langle u,e(\phi-\lambda\id)^{*}(y)\rangle
\]
Nach Proposition gilt \((\phi-\lambda\id)^{*}=\phi^{*}-\overline{\lambda}\id\).
\\
Dann ist \(0=\langle u,\underbrace{(\phi^{*}-\overline{\lambda}\id)(y)}_{\neq u}\rangle\)
(wegen der positiven Definitheit und \(u\neq0\)).

Daraus folgt: 
\begin{align*}
\phi^{*}-\overline{\lambda}\id\text{ ist nicht surjektiv}&\Longleftrightarrow
    \phi^{*}-\overline{\lambda}\id\text{ ist nicht injektiv}\\
&\Longleftrightarrow\exists v\neq0:\,\phi^{*}(v)=\overline{\lambda}v\\
&\implies\overline{\lambda}\in\Spec(\phi^{*})
\end{align*}
\end{proof}

\section{Der Spektralsatz}
\index{Spektralsatz}\index{Spektrum}

\begin{proposition}
Sei \(\phi\in\End^{a}(V)\)
\begin{enumerate}
\item F\"ur \(\lambda,\mu\in\Spec(\phi)\) mit \(\lambda\neq\mu\) gilt:
\[
E_{\lambda}(\phi)\bot E_{\mu}(\phi)
\]
\item Folgende Aussagen sind \"aquivalent:
    \begin{enumerate}
    \item\(\phi\circ\phi^{*}=\phi^{*}\circ\phi\)
    \item\(\forall x,y\in V:\,\langle\phi(x),\phi(y)\rangle=\langle\phi^{*}(x),\phi^{*}(y)\rangle\)\\
    \(\phi\) hei\ss t \textbf{normal}.
    \end{enumerate}
\item Ist \(\phi\) normal, dann folgt \(\Kern(\phi)=\Kern(\phi^{*})\), 
insbesondere \(E_{\lambda}(\phi)=E_{\overline{\lambda}}(\phi^{*})\).
\end{enumerate}
\end{proposition}
\begin{proof}
\begin{enumerate}
\item Seien \(u\in E_{\lambda}(\phi),\,v\in E_{\mu}(\phi)\). Dann gilt
\begin{align*}
\lambda\cdot\langle u,v\rangle&=\langle\lambda u,v\rangle\\
&=\langle\phi(u),v\rangle\\
&=\langle u,\phi^{*}(v)\rangle\\
&=\langle u,\overline{\mu}v\rangle\\
&=\mu\langle u,v\rangle\\
\end{align*}
Mit \(\lambda\neq\mu\) folgt \(\langle u,v\rangle=0\)
\end{enumerate}
\end{proof}

\begin{theo}[Spektralsatz]
Sei \(\dim V<\infty,\,\langle\cdot,\cdot\rangle\) Skalarprodukt mit 
\(\phi\in\End(V)\) normal.\\
Im Fall \(\mathbb{K}=\mathbb{R}\) habe das charakteristische Polynom 
\(f_{\phi}(T)\) nur reelle Nullstellen. Dann existiert eine Orthonormalbasis
aus Eigenvektoren von \(\phi\).
\end{theo}
\begin{proof}
Sei \(n:=\dim V,\,\lambda_{1}\in\Spec(\phi),\,b_{1}\neq0\in E_{\lambda_{1}}(\phi)\). Ohne Beschr\"ankung der Allgemeinheit sei \(\lVert b_{1}\rVert=1\).\\
Betrachte das orthogonale Komplement \(U:=b_{1}^{\top}\). Es gilt
\[
V=\langle b_{1}\rangle\oplus U,
\]
wobei \(\phi(U)\subseteq U,\,\phi^{*}(U)\subseteq U\) ist, denn f\"ur alle 
\(u\in U\) gilt
\begin{align*}
\langle\phi(u),b_{1}\rangle&=\langle u,\phi^{*}(b_{1})\rangle\\
&=\langle u,\overline{\lambda_{1}}b_{1}\rangle\\
&=\lambda_{1}\underbrace{\rangle u,b_{1}\rangle}_{=0}=0
\end{align*}
Daraus folgt \(\phi(u)\bot b_{1}\), das hei\ss t \(\phi(U)\bot b_{1}\), damit
folgt \(\phi(U)\subseteq U\).\\
F\"ur \(\phi^{*}\) ist die Vorgehensweise analog.\\
Insbesondere ist \(\left.\phi\right|_{U}\in\End(U)\).

Ferner gilt \(\left(\left.\phi\right|_{U}\right)^{*}=\left.\phi^{*}\right|_{U}\), also
\begin{align*}
\left.\phi\right|_{U}\left.\phi^{*}\right|_{U}&=\left.\left(\phi\phi^{*}\right)\right|_{U}\\
% das hier ist von der Ausrichtung noch nicht optimal...
&\overset{\phi\text{ normal}}{=}\left.\left(\phi^{*}\phi\right)\right|_{U}\\
&=\left.\phi^{*}\right|_{U}\left.\phi\right|_{U}
\end{align*}
Also ist \(\phi\) normal.

Vollst\"andige Induktion nach \(n\):\\
\(n-1 \leadsto n\): \(U\) hat eine Orthonormalbasis \(\{b_{2},\ldots,b_{n}\}\)
aus Eigenvektoren von \(\left.\phi\right|_{U}\).\\
Dann ist \(\{b_{1},b_{2},\ldots,b_{n}\}\) die gesuchte Orthonormalbasis.
\end{proof}
\begin{lemma}[Transfer zu Matrizen]
F\"ur beliebiges \(\phi\in\End(V)\) sei \(s_{\phi}\) die Sesquilinearform
\[
s_{\phi}(x,y):=\langle\phi(x),y\rangle
\]
\(B\) sei eine Orthonormalbasis von \(V\). Dann gilt:
\begin{enumerate}
\item\(D_{BB}(\phi^{*})=D_{BB}(\phi)^{*}\)
\label{Lemma: 18.2; 1; Beh. 1}
\item\(D_{BB}(s_{\phi})=D_{BB}(\phi)^{\top}\)
\label{Lemma: 18.2; 1; Beh. 2}
\item\(\phi\) ist normal, genau dann wenn f\"ur \(A:=D_{BB}(\phi)\) gilt:
\label{Lemma: 18.2; 1; Beh. 3}
\[
A\cdot A^{*}=A^{*}\cdot A
\]
\end{enumerate}
\end{lemma}
\begin{proof}
Sei \(B=\{b_{1},\ldots,b_{n}\}\).\\
Erinnere: \(D_{BB}(\phi)=(x_{ij})\) ist 
definiert durch \(\phi(b_{ij})=\sum_{i=1}^{n}{\alpha_{ij}b_{i}}\).\\
Es gilt
\begin{align*}
s_{\phi}(b_{j},b_{k})&=\langle\phi(b_{j}),b_{k}\rangle\\
&=\sum_{i=1}^{n}{\alpha_{ij}\underbrace{\langle b_{i},b_{k}\rangle}_{=\delta_{ik}}}\\
&=\alpha_{kj}
\end{align*}
Damit folgt die Behauptung \eqref{Lemma: 18.2; 1; Beh. 2}.

Sei \(D_{BB}(\phi^{*})=(\beta_{ij})\), das hei\ss t
\begin{align*}
\overline{\alpha_{ji}}&=\overline{\langle\phi(b_{i}),b_{j}\rangle}\\
&=\langle b_{j},\phi(b_{i})\rangle\\
&=\langle\phi^{*}(b_{j}),b_{i}\rangle\\
&=\beta_{ij}
\end{align*}
Damit folgt die Behauptung \eqref{Lemma: 18.2; 1; Beh. 1}.

Es bleibt noch Behauptung \eqref{Lemma: 18.2; 1; Beh. 3} zu zeigen:
\[
\phi\cdot\phi^{*}\Longleftrightarrow
\underbrace{D_{BB}(\phi\phi^{*})}_{=AA^{*}}=
\underbrace{D_{BB}(\phi^{*}\phi)}_{=A^{*}A}
\]
\end{proof}
\begin{corollary}[zum Spektralsatz]
F\"ur \(\lambda\in\Spec(\phi)\) sei \(U_{\lambda}:=E_{\lambda}(\phi)\) und
\(\Pi_{\lambda}:=\Pi_{U_{\lambda}}\) (orthogonale Projektion). Dann gilt f\"ur
\(p(T)\in\mathbb{K}[T]\):
\[
p(\phi)=\sum_{\lambda\in\Spec(\phi)}{p(\lambda)\cdot\Pi_{\lambda}}
\]
und
\[
\phi^{*}=\sum_{\lambda}{\overline{\lambda}\cdot\Pi_{\lambda}}
\]
\end{corollary}
\begin{proof}
Da \(U_{\lambda}\bot U_{\mu}\) f\"ur \(\lambda\neq\mu\) folgt \(\Pi_{\lambda}\Pi_{\mu}=\delta_{\lambda\mu}\Pi_{\lambda}\).\\
Spektralsatz: Aus \(V=\underset{\lambda}{\bigoplus}U_{\lambda}\) folgt 
\(\id_{V}=\sum_{\lambda}{\Pi_{\lambda}}\).\\
Aus \(\left.p(\phi)\right|_{U_{\lambda}}=p(\lambda)\cdot\id_{U_{\lambda}}\)
folgt \(p(\phi)=\sum_{\lambda}{p(\lambda)\Pi_{\lambda}}\).\\
\(\left.\phi^{*}\right|_{U_{\lambda}}=\overline{\lambda}\cdot\id_{U_{\lambda}}\)
liefert
\begin{align*}
\phi^{*}&=\phi^{*}\cdot\id_{U_{\lambda}}\\
&=\phi^{*}\cdot\sum_{\lambda}{\Pi_{\lambda}}\\
&=\sum_{\lambda}{\phi^{*}\Pi_{\lambda}}\\
&=\sum_{\lambda}{\overline{\lambda}\Pi_{\lambda}}
\end{align*}
\end{proof}
\begin{theo}
Seien \(\phi,\,\Psi\in\End(V)\) normal und \(\phi\cdot\Psi=\Psi\cdot\phi\).\\
Falls in \(V\) eine Orthonormalbasis aus Eigenvektoren existiert und eine
Orthonormalbasis aus Eigenvektoren zu \(\Psi\), dann existiert eine
Orthonormalbasis aus gemeinsamen Eigenvektoren zu \(\phi\) und \(\Psi\).
\end{theo}
\begin{proof}
Seien \(V=\underset{\lambda}{\bigoplus}U_{\lambda},\,U_{\lambda}:=E_{\lambda}(\phi)\).\\
Zeige: \(\Psi(U_{\lambda})\subseteq U_{\lambda}\) und \(\left.\Psi\right|_{U_{\lambda}}\) sind diagonalisierbar.

Dazu:
\begin{align*}
u\in U_{\lambda}&\implies\phi(u)=\lambda u\\
&\implies\Psi(\phi(u))=\Psi(\lambda u)=\lambda\Psi(u)\\
&\Longleftrightarrow\phi(\Psi(u))=\lambda\Psi(u)\quad\implies\Psi(u)\in U_{\lambda}
\end{align*}
Analog: \(\phi(E_{\mu}(\Psi))\subseteq E_{\mu}(\Psi)\).\\
Da \(V=\underset{\mu}{\bigoplus}E_{\mu}(\Psi)\), gilt insbesondere f\"ur alle
\(u\in U_{\lambda}:\,u=\sum_{\mu}{x_{\mu}}\in E_{\mu}(\Psi)\). Es gilt sogar:
jedes \(x_{\mu}\in U_{\lambda}\), denn:
\[
\phi(x_{\mu}=:x'_{\mu}\in E_{\mu}(\Psi)
\]
\begin{align*}
\lambda{\sum_{\mu}}^{\oplus}x_{\mu}&=\lambda u\\
&=\phi(u)\\
&=\sum_{\mu}{\phi(x_{\mu}})\\
&={\sum_{\mu}}^{\oplus}{x'_{\mu}}
\end{align*}
Da die Summe direkt ist, folgt f\"ur alle \(\mu\)
\[
\lambda\cdot x_{\mu}=x'_{\mu}=\phi(x_{\mu}),
\]
das hei\ss t \(x_{\mu}\in U_{\lambda}\).

Insgesamt gezeigt:
\[
U_{\lambda}=\underset{\mu}{\bigoplus}E_{\mu}(\Psi)\cap U_{\lambda}
\]
(d.h. \(\left.\Psi\right|_{U_{\lambda}}\) ist diagonalisierbar). Damit folgt
\[
V=\underset{\lambda}{\bigoplus}\underset{\mu}{\bigoplus}E_{\mu}(\Psi)\cap E_{\lambda}(\phi)
\]
\end{proof}

\section{Selbstadjungierte Endomorphismen}
\index{Endomorphismus!selbstadjungiert}\index{Matrix!hermitesch}
\index{Spektralradius}

\begin{definition}
\(\phi\in\End(V)\) hei\ss t \textbf{selbstadjungiert}, falls \(\phi^{*}=\phi\).
\end{definition}
\begin{comment}
\begin{enumerate}
\item \(\phi\) ist selbstadjungiert impliziert \(\phi\) ist normal.
\item Ist \(\dim V<\infty,\, B\) Orthonormalbasis und \(A:=D_{BB}(\phi)\), dann
ist \(\phi\) selbstadjungiert genau dann wenn \(A=A^{*}\), d.h. \(A\) ist 
hermitesch.
\end{enumerate}
\end{comment}
\paragraph{Hintergrund:}{Viele Problem in Physik und Technik f\"uhren auf 
hermitesche Matrizen.}
\begin{theo}
\begin{enumerate}
\item \(A\in\mathbb{C}^{m\times m}\) mit \(A=A^{\top}\) impliziert 
    \(\Spec(A)\subseteq\mathbb{R}\) (oder: das charakteristische Polynom hat nur
reelle Nullstellen).
\item F\"ur hermitesche \(A\) gilt:
\begin{center}
\(A\) ist positiv definit\(\Longleftrightarrow\forall\lambda\in\Spec(A):\,\lambda>0\)
\end{center}
\end{enumerate}
\end{theo}
\begin{proof}
\begin{enumerate}
\item Sei \(\lambda\in\Spec(A)\) und \(v\neq0\) mit \(Av=\lambda v\). Dann:
\begin{align*}
\lambda\langle v,v\rangle&=\langle\lambda v,v\rangle\\
&=\langle Av,v\rangle\\
&=\langle v,A^{*}v\rangle\\
&=\langle v,Av\rangle\\
&=\langle v,\lambda v\rangle\\
&=\overline{\lambda}\underbrace{\langle v,v\rangle}_{=\lVert v\rVert^{2}\neq0}
\end{align*}
Also gilt \(\lambda=\overline{\lambda}\), das hei\ss t \(\lambda\in\mathbb{R}\).
\item \(A\) ist nach Definition genau dann positiv definit wenn 
\(s_{A}(x,y)=x^{\top}A\overline{y}\) positiv definit ist.\\
% In der Vorlesung wurde an dieser Stelle an das komplexe Skalarprodukt
% erinnert, ist es sinnvoll/notwendig das hier auch zu tun?
F\"ur eine Orthonormalbasis \(\{b_{1},\ldots,b_{n}\}\) aus Eigenvektoren von
\(A=A^{*}\) gilt
\[
Ab_{i}=\lambda b_{i}
\]
und Basisdarstellung
\[
\overline{x}=\sum_{i=1}^{m}{\alpha_{i}b_{i}}\leadsto x=\sum_{i=1}^{m}{\overline{\alpha_{i}}\overline{b_{i}}}
\]
und somit
\begin{align*}
s_{A}(x,x)&=x^{\top}A\overline{y}\\
&=\sum_{i=1}^{m}{\overline{\alpha_{i}}\overline{b_{i}}}\sum_{j=1}^{m}{\alpha_{j}\underbrace{Ab_{j}}_{\lambda_{j}b_{j}}}\\
&=\sum_{i,j}{\overline{\alpha_{i}}\alpha_{j}\lambda_{j}\overline{b_{i}}^{\top}b_{j}}\\
&=\sum_{i,j}{\overline{\alpha_{i}}\alpha_{j}\lambda_{j}\underbrace{\overline{\langle b_{i},b_{j}\rangle}}_{=\delta_{ij}}}\\
&=\sum_{i=1}^{m}\lvert\alpha_{i}\rvert^{2}\lambda_{i}
\end{align*}
Also: \(s_{A}(x,x)=\sum_{i=1}^{m}{\lvert\alpha_{i}\rvert^{2}\lambda_{i}}\).\\
Dann folgt:\\
\[
s_{A}(x,x)\geq0\,\forall x\Longleftrightarrow\forall\lambda_{i}\geq0
\]
und
\[
s_{A}(x,x)=0\implies x=0
\]
genau dann, wenn alle \(\lambda_{i}\) gr\"o\ss er Null sind.
\end{enumerate}
\end{proof}
\begin{comment}
F\"ur selbstadjungierte, reelle \(A\) ist die Extravoraussetzung im Spektralsatz
immer erf\"ullt.
\end{comment}
\begin{corollary}
Ist \(V\) ein \(\mathbb{R}\)-Vektorraum mit Skalarprodukt, \(\dim V<\infty\) und
\(\phi\in\End(V)\) selbstadjungiert, so besitzt \(V\) eine Orthonormalbasis aus
Eigenvektoren zu \(\phi\).
\end{corollary}
\begin{definition}
Sei \(V\) ein \(\mathbb{C}\)-Vektorraum und \(\phi\in\End(V)\).\\
Dann hei\ss t \(\rho(\phi):=\sup\{\lvert\lambda\rvert\mid\lambda\in\Spec(\phi)\}\)
der \textbf{Spektralradius} von \(\phi\). F\"ur \(A\in\mathbb{C}^{m\times m}\)
setze \(\rho(A):=\rho(\Lambda_{A})\).
\end{definition}

\begin{comment}
Auf \(\mathbb{K}^{m\times n}\) ist durch
\[
\lVert A\rVert:=\sup\{\lVert A\rVert\mid x\in\mathbb{K}^{n},\lVert x\rVert\leq1\}
\]
eine Norm definiert.
\end{comment}
\begin{theo}
Es gilt \(\lVert A\rVert=\sqrt{\rho(A^{*}A)}\).\\
Falls \(m=n\) und \(A\) normal ist, gilt sogar \(\lVert A\rVert=\rho(A)\).
\end{theo}
\begin{proof}
\(A^{*}A\) ist selbstadjungiert, das hei\ss t es gilt 
\((A^{*}A)^{*}=A^{*}\cdot\left(A^{*}\right)^{*}=A^{*}A\).\\
Dann existiert eine Orthonormalbasis \(\{b_{1},\ldots,b_{n}\}\) aus 
Eigenvektoren, etwa \(A^{*}Ab_{i}=\mu_{i}b_{i}\) mit \(\mu_{i}\in\mathbb{R}\).\\
Dann gilt:
\begin{align*}
\lVert Ax\rVert^{2}&=\langle Ax,Ax\rangle\\
&=\langle x,A^{*}Ax\rangle\\
&\overset{x=\sum_{i=1}^{n}{\alpha_{i}b_{i}}}{=}=\left\langle x,\sum_{i=1}^{n}{\alpha_{i}\mu_{i}b_{i}}\right\rangle\\
&=\sum_{i=1}^{n}\lvert\alpha_{i}\rvert^{2}\underbrace{\overline{\mu_{i}}}_{=\mu_{i}}
\end{align*}
Au\ss erdem:
\begin{align*}
\lVert Ax\rVert^{2}&\leq\sum_{i}\lvert\alpha_{i}\rvert^{2}\underbrace{\max\{\lvert\mu_{i}\rvert\}}_{=\rho(A^{*}A)}\\
&=\rho(A^{*}A)\lVert x\rVert^{2}
\end{align*}
Sei \(x=\sum_{i}{\alpha_{i}b_{i}}\) die Basisdarstellung. Dann ist 
\(\lVert Ax\rVert^{2}=\sum_{i}{\lvert\alpha_{i}\rvert^{2}\mu_{i}}\), also alle
\(\mu_{i}\geq0\).\\
Weiterhin: \(A^{*}Ab_{i}=\mu_{i}b_{i}\) und \(\rho(A^{*}A)=\mu_{\max}=\mu_{i_{0}}\),
dazu \(b_{i_{0}}\). Mit \(x:=b_{i_{0}}\) folgt \(\lVert Ax\rVert^{2}=\mu_{\max}\).\\
\textbf{Speziell f\"ur normales \(A\)} (\(m=n\)):\\
Es gilt \(E_{\lambda}(A)=E_{\overline{\lambda}}(A^{*})\). Dann:
\[
\mu_{i}=\lambda_{i}\cdot\overline{\lambda_{i}}=\lvert\lambda_{i}\rvert^{2}
\]
und damit folgt 
\[
\lVert A\rVert=\lvert\mu_{\max}\rvert=\rho(A)
\]
\end{proof}
\begin{caution}
Im allgemeinen ist \(\lVert A\rVert\neq\rho(A)\).\\
Beispiel: 
\[
A=\begin{pmatrix}0&1\\0&0\end{pmatrix}
\] 
mit \(\rho(A)=0\) aber \(\lVert A\rVert=1\).
Es ist
\[
A^{*}=\begin{pmatrix}0&0\\1&0\end{pmatrix},\,A^{*}A=\begin{pmatrix}0&0\\0&1\end{pmatrix}
\]
\end{caution}

\chapter{Isometrien}

\begin{task}
Studiere alle linearen Abbildungen, die \textbf{Abstände} von Punkten nicht ändern.
Z.B. Drehungen um einen Punkt im $\mathbb{R}^2$.
\end{task}

\section{Charakterisierung und orthogonale Gruppe}
\begin{definition}
\index{Morphismus}\index{Isometrie}\index{Automorphismus}\index{Automorphismengruppe}
Seien $V_1,V_2\ \mathbb{K}$-VRme mit Sesquilinearformen $s_1,s_2$.
\begin{enumerate}[(a)]
\item Ein \textbf{Morphismus von $\mathbb{K}$-VRmen mit Sesquilinearform} ist
$\Phi\in \Hom(V_1,V_2)$ mit:\\
\[\forall x,y\in V_1:s_2(\Phi(x),\Phi(y))=s_1(x,y)\]\\
Schreibe: $\Phi:(V_1,s_1)\to (V_2,s_2)$.
\item Ist $\Phi$ zusätzlich bijektiv, so heißt $\Phi$ eine \textbf{(lineare) Isometrie}.
\item Eine Isometrie $\Phi:(V,s)\to (V,s)$ heißt \textbf{Automorphismus} von s.\\
Die Gruppe $\Aut(s)\le\Aut(V)$ heißt die \textbf{Automorphismengruppe} von s.
\end{enumerate}
\end{definition}

\begin{example}
\index{Lorenzgruppe}
In der Relativitätstheorie wichtig ist die \textbf{Lorenzgruppe} $\Aut(s)$ zu\\
\[s:\mathbb{R}^4 \times \mathbb{R}^4 \to\mathbb{R}, (x,y)\mapsto x^T
\begin{pmatrix}
1&0&\cdots&0\\
0&\ddots&\ddots&\vdots\\
\vdots&\ddots&1&0\\
0&\cdots&0&-c
\end{pmatrix}
y\]\\ 
für $c:=$Lichtgeschwindigkeit.
\end{example}

\begin{definition}
\index{orthogonal!Gruppe}\index{Gruppe!orthogonale}
\index{orthogonal!Abbildung}\index{Abbildung!orthogonale}
\index{unitär!Gruppe}\index{Gruppe!unitäre}
\index{unitär!Abbildung}\index{Abbildung!unitäre}
Im Folgenden sei $s$ stets SKP.\\
\textbf{Fall $\mathbb{K}=\mathbb{R}$:}\\
$O(V,s):=\Aut(s)$ heißt \textbf{orthogonale Gruppe}. Die Elemente der Gruppe
heißen \textbf{orthogonale Abb. bzgl. s}.\\
\textbf{Fall $\mathbb{K}=\mathbb{C}$:}\\
$U(V.s):=\Aut(s)$ heißt \textbf{unitäre Gruppe}. Die Elemente der Gruppe heißen
\textbf{unitäre Abb. bzgl. s}.
\end{definition}

\begin{comment}
Eine wichtige Isometrie ist: abstrakter VRm $\cong$ Standardraum
\end{comment}

\begin{theo}
Sei $V$ VRm mit SKP $s$, $\dim(V)=n$ und ONB $B$. Dann ist die Koordinatendarstellung:\\
\[D_B:(V,s)\to(\mathbb{K}^n,\langle\cdot,\cdot\rangle)\] eine Isometrie.
\end{theo}

\begin{proof}
Sei $B=\{b_1,\ldots,b_n\}, x,y\in V$ mit $x=\sum_{i=1}^n{\alpha_i b_i},
y=\sum_{i=1}^n{\beta_i b_i}$. Dann gilt:
\begin{align*}
s(x,y)&=\sum_{i,j}{\alpha_i \overline{b_j}\cdot s(b_i,b_j)}\\
&= \sum_{i=1}^n{\alpha_i \overline{b_i}}\\
&= \langle
\begin{pmatrix}
\alpha_1\\
\vdots\\
\alpha_n
\end{pmatrix},
\begin{pmatrix}
\beta_1\\
\vdots\\
\beta_n
\end{pmatrix} 
\rangle\\
&= \langle D_B(x),D_B(y)\rangle
\end{align*}
\end{proof}

\begin{comment}
\begin{enumerate}
\index{Längentreue}\index{Winkeltreue}
\item Sei $\Phi: V_1\to V_2$ Morphismus von SKP-Räumen, dann ist $\Phi$ \textbf{längentreu}.
\[\iff\|x\|_1 = \|\Phi(x)\|_2\]
\textbf{Winkeltreue} für $K=\mathbb{R}$ bedeutet: 
\[\frac{\langle x_1,y_1\rangle_1}{\|x_1\|_1\|y_1\|_1} = \frac{\langle\Phi(x_1),
\Phi(y_1)\rangle_2}{\|\Phi(x_1)\|_2\|\Phi(y_1)\|_2}\]
\item $\Phi:(V,s)\to(V,s)$ Endomorphismus von SKP-Räumen und $\dim(V)<\infty
\implies \Phi$ ist Isomorphismus und Automorphismus, also orthogonal und unitär. 
\end{enumerate}
\end{comment}

\begin{theo}[Isometriekriterium]
Sei $V$ VRm mit SKP $s=\langle\cdot,\cdot\rangle$ und sei $\Phi\in\Aut(V)$.\\
Folgende Aussagen sind äquivalent:
\begin{enumerate}
\item $\Phi$ ist Isometrie (d.h. $\Phi\in\Aut(s)$).
\item $\Phi\in\End^a(V)$ und $\Phi^*=\Phi^{-1}$.
\item $\forall x\in V: \|x\|=\|\Phi(x)\|$
\item $\forall y\in V: (\|y\|=1)\implies(\|\Phi(y)\|=1)$ (Einheitssphärenabbildung).
\end{enumerate}
\end{theo}

\begin{proof}
Die Äquivalenz ergibt sich aus folgendem Ringschluss:
\begin{enumerate}
\item[(1)$\implies$(2)] Es gilt $\forall x,y\in V$, $z:=\Phi(y)$:
\begin{align*}
&\Phi \text{ Isometrie}\\
&\iff \forall x,y\in V:\langle\Phi(x),\Phi(y)\rangle =\langle x,y\rangle\\
&\stackrel{\Phi^{-1}\text{ ex.}}{\iff} \forall x,z\in V:\langle\Phi(x),z\rangle
=\langle x,\Phi^{-1}(z)\rangle
\end{align*}
Nach Definition der Adjungierten folgt daraus $\Phi^{-1} = \Phi^*$.
\item[(2)$\implies$(3)] Es gilt für alle $x\in V$:
\[\|\Phi(x)\|^2=\langle\Phi(x),\Phi(x)\rangle\stackrel{(2)}{=}
\langle x,\Phi^*\Phi(x)\rangle=\langle x,x\rangle\]
\item[(3)$\iff$(4)] \checkmark
\item[(3)$\implies$(1)] Es gilt für alle $x,y,\in V,\alpha\in K:$
\begin{align*}
\langle\alpha x+y,\alpha x+y\rangle &= \langle\Phi(\alpha x+y),\Phi(\alpha x+y)\rangle\\
\iff\langle \alpha x,y\rangle +\langle y,\alpha x\rangle &= \langle\Phi(\alpha x),
\Phi(y)\rangle + \langle\Phi(y),\Phi(\alpha x)\rangle\\
\iff\alpha\langle x,y\rangle +\overline{\alpha}\overline{\langle x,y\rangle} &=
\alpha\langle\Phi(x),\Phi(y)\rangle + \overline{\alpha}\overline{\langle\Phi(x),\Phi(y)\rangle}
\end{align*}
\textbf{Fall $K=\mathbb{R}$:}\\ 
Mit $\alpha:=\frac12:\langle x,y\rangle =\langle\Phi(x),\Phi(y)\rangle$\\
\textbf{Fall $K=\mathbb{C}$:}\\ 
Mit $\alpha:=\frac12:\Real\langle x,y\rangle =\Real\langle\Phi(x),\Phi(y)\rangle$\\
Mit $\alpha:=\frac i2:\Imag\langle x,y\rangle =\Imag\langle\Phi(x),\Phi(y)\rangle$
\end{enumerate}
\end{proof}

\begin{corollary}
Sei $\dim(V)=n<\infty$, $B$ ONB von $V$ und $\Phi\in\End(V)$.\\
Folgende Aussagen sind äquivalent:
\begin{enumerate}
\item $\Phi$ ist Isometrie.
\item Es gilt für alle $x\in V:\|\Phi(x)\|=\|x\|$
\item $\Phi(B)$ ist ONB.
\item Es gilt $D_{BB}(\Phi)^{-1} = D_{BB}(\Phi^*)$, d.h. $D_{BB}(\Phi)$ ist unitär bzw. orthogonal.
\item Die Spalten (bzw. Zeilen) von $D_{BB}(\Phi)$ bilden eine ONB von $\mathbb{K}^n$ bzgl.
dem Standard-SKP.
\item Es existiert eine ONB $C$ von $V$ mit $D_{BC}(\Phi)=I_n$.
\end{enumerate}
\end{corollary}

\begin{proof}
Jede der Aussagen impliziert $\Phi\in\Aut(V)$.\\
Sei $B:=\{b_1,\ldots,b_n\}$.
\begin{enumerate}
\item[(1)$\iff$(2)$\iff$(4)] Klar nach Isometriekriterium.
\item[(4)$\iff$(5)]Es gilt:
\begin{align*}
&D_{BB}(\Phi)^{-1} = D_{BB}(\Phi^*)\\
&	\implies D_{BB}(\Phi)\cdot D_{BB}(\Phi^*) = I_n\\
&\iff\{\text{Zeilen von }\Phi\}\text{ sind ONB bezgl. Standardform}\\
&\implies D_{BB}(\Phi^*)\cdot D_{BB}(\Phi) = I_n\\
&\iff\{\text{Spalten von }\Phi\}\text{ sind ONB bezgl. Standardform}
\end{align*}
\item[(3)$\implies$(2)]Da für alle $b_i,b_j\in B$ gilt:
\[\langle\Phi(b_i),\Phi(b_j)\rangle =\delta_{ij}=\langle b_i,b_j\rangle\]
Folgt für alle $x=\sum_{i=1}^n{\alpha_i b_i}\in V:$
\begin{align*}
\langle\Phi(x),\Phi(x)\rangle
&=\sum_{i,j}{\alpha_i\overline{\alpha_j}\langle\Phi(b_i),\Phi(b_j)\rangle}\\
&=\sum_{i,j}{\alpha_i\overline{\alpha_j}\langle b_i,b_j\rangle}\\
&=\langle x,x\rangle
\end{align*}
Also ist $\|\Phi(x)\|=\|x\|$ und $\Phi$ längenerhaltend.
\item[(1)$\implies$(3)] Da $\Phi$ Isometrie ist, gilt:
\[\implies \langle\Phi(b_i),\Phi(b_j)\rangle=\langle b_i,b_j\rangle =\delta_{ij}\]
D.h. $\Phi(B)$ ist ONB.
\item[(3)$\implies$(6)] Es existiert eine ONB $C=\{c_1,\ldots,c_n\}$, sodass gilt:
\[D_{BC}(\Phi) = I_n\]
Daraus folgt: $D_{BB}(\Phi)=D_{BC}(\Phi)\cdot M_{CB}=M_{CB}=:(\gamma_{ij})$\\
Also gilt für alle $b_j\in B$:
\[b_j=\sum_k{\gamma_{kj}\cdot c_k}\]
Daraus folgt:
\begin{align*}
\delta_{ij}&=\langle b_j,b_i\rangle\\
&=\langle\sum_k{\gamma_{kj}\cdot c_k},\sum_l{\gamma_{li}\cdot c_l}\rangle\\
&=\sum_{k,l}{\gamma_{kj}\cdot\overline{\gamma_{li}}\cdot\langle c_k,c_l\rangle}\\
&=\sum_k{\gamma_{kj}\cdot\overline{\gamma_{ki}}}\\
&=\sum_k{\overline{\gamma_{ki}}\cdot\gamma_{kj}}-(\overline{M}_{CB}^T\cdot M_{CB})_{ij}
\end{align*}
Es gilt also $M_{CB}^* = M_{CB}^{-1}$.
\end{enumerate}
\end{proof}

\section{Normalformen für Isometrien und normale Endomorphismen}
Sei $V$ VRm mit SKP $\langle\cdot,\cdot\rangle, \dim(V)=n<\infty$.

\subsection{Fall $\mathbb{K}=\mathbb{C}$}
\begin{lemma}
Ein Endomorphismus $\Phi$ ist genau dann unitär, wenn er normal ist und alle Eigenwerte 
Betrag 1 haben.
\end{lemma}

\begin{proof}
Da $\Phi$ unitär ist, also $\Phi^*=\Phi^{-1}$ gilt, ist $\Phi$ normal. 
Nach Spektralsatz existiert dann eine ONB $B=\{b_1,\ldots,b_n\}$ aus Eigenvektoren von $\Phi$.
Also gilt: 
\[\Phi(b_i)=\lambda_i b_i \text{ mit } \lambda_i\in\mathbb{C}\]
Mit dem Korollar folgt:\\
\begin{align*}
&\Phi \text{ unitär}\\
&\iff \Phi(B) \text{ ONB}\\
&\iff \delta_{ij}=\langle\Phi(b_i),\Phi(b_j)\rangle=\langle\lambda_i b_i,\lambda_j b_j\rangle
=|\lambda_i|^2\cdot\delta_{ij}\\
&\iff |\lambda_i|^2=1\\
&\iff |\lambda_i|=1
\end{align*}
\end{proof}

\begin{conclusion}
\index{Normalform}
$D_{BB}(\Phi)=\diag(\lambda_1,\ldots,\lambda_n)$ mit $|\lambda_i|=1$ heißt \textbf{Normalform}
der unitären Abb. $\Phi$ und ist bis auf die Reihenfolge der Eigenwerte eindeutig bestimmt.
\end{conclusion}

\begin{corollary}
\index{unitär!Basiswechsel}\index{Basiswechsel!unitärer}
Ist $A\in\mathbb{C}^{n\times n}$ normal, so existieren $M\in U_n$ und $\lambda_i\in\mathbb{C}$,
sodass gilt:
\begin{align*}
M^{-1}\cdot A\cdot M = \diag(\lambda_1,\ldots,\lambda_n)
\end{align*}
D.h. jedes normale $A$ erlangt durch \textbf{unitären Basiswechsel} Normalform.\\
Falls $A$ unitär ist, so existiert $\Phi_j\in\mathbb{R}$, sodass gilt:
\begin{align*}
\lambda_j=e^{i\Phi_j}=\cos\Phi_j+i\sin\Phi_j
\end{align*}
\end{corollary}


\chapter{Affine Räume}

Man möchte vom Anschauungsraum $\mathbb{R}^3$ abstrahieren:\\
\begin{itemize}
\item statt $\mathbb{R}$ \textbf{beliebige} Körper $K$
\item statt Dimension $3$ \textbf{beliebige} Dimensionen $< \infty$
\end{itemize}

\begin{task}
Finde die "`richtige"' Verallgemeinerung der vertrauten \textbf{geometrischen} 
Begriffe, sodass bekannte geometrische Sätze richtig bleiben.
\end{task}

Im Folgenden sei $K$ stets ein beliebiger Körper.

\section{Grunbegriffe}
\begin{definition}
\index{affin!Raum}\index{Raum!affiner}
\index{Richtungsvektorraum}
\index{Translation}
\index{Punkt}
\index{Translationsvektor}
Sei $V$ $K$-VRm mit $\dim(V)=n<\infty$.
\begin{enumerate}[(a)]
\item Eine Menge $A \ne \emptyset$ heißt \textbf{affiner Raum mit Richtungsvektorraum $V$},
falls $(V,+)$ auf $A$ operiert, d.h. es existiert eine Paarung "`$+$"' genannt
\textbf{Translation} $V \times A \to A$, $(x,P) \mapsto x+P$, mit der Eigentschaft:
\[\forall P,Q\in A\exists_1 x\in V: Q=x+P\]
\item Elemente von $A$ heißen \textbf{Punkte}.\\
Der zu gegebenen Punkten $P$, $Q$ eindeutig bestimmte Vektor $x$ mit $Q=x+P$ heißt
der \textbf{Translationsvektor von $P$ nach $Q$}.\\
Schreibe: $x:=\overrightarrow{PQ}$
\item $\dim(A) := \dim(V)$ heißt \textbf{Dimension von $A$}.
\end{enumerate}
\end{definition}

\begin{comment}
\begin{enumerate}
\item Vorsicht in (1) wird das Zeichen "`+"' für verschiedene Verknüpfungen benutzt.
\item Es gilt für $P,Q,R,\in A:$
\begin{align*}
\overrightarrow{PP}&=0\\
\overrightarrow{PQ}+\overrightarrow{QR} &= \overrightarrow{PR}\\
\overrightarrow{QP}&=-\overrightarrow{PQ}
\end{align*} 
\item $A$ besteht aus genau einer Bahn:
\[\forall P\in A: A=V+P:=\{x+P\mid x\in V\}\]
\end{enumerate}
\end{comment}

\begin{example}
\index{affin!Standardraum}\index{Standardraum!affiner}
Der \textbf{affine Standardraum} $\mathbb{A}_n(K)$ ist definiert als  Punktmenge
$\mathbb{A}:=K^n$ und $V:=K^n$, mit Translation $:=$ Addition in $K^n$, d.h.
für $P,Q\in K^n$ gilt:
\[\overrightarrow{PQ}=Q-P\]
\end{example}

\begin{definition}
\index{affin!Teilraum}\index{Teilraum!affiner}
\index{linear!Varietät}\index{Varietät!lineare}
\index{Gerade}
\index{Ebene}
\index{Hyperebene}
Eine Teilmenge $B \ne \emptyset$ eines affinen Raumes A heißt \textbf{(affiner) Teilraum}
oder \textbf{lineare Varietät} von $A$, falls ein VRm $U_B \le V$ existiert, sodass
$B$ affiner Raum ist, mit Richtungsvektorraum $U_B$ (unter der in $A$ gegebenen
Operation).\\
Auch $B = \emptyset$ werde affiner Teilraum genannt.\\
Spezielle affine Teilräume B sind:
\begin{enumerate}[(a)]
\item \textbf{Gerade} $\iff \dim(B) = 1$
\item \textbf{Ebene} $\iff \dim(B) = 2$
\item \textbf{Hyperebene} $\iff \dim(B) = \dim(A)-1$
\end{enumerate}
\end{definition}

\begin{lemma}
\index{Verbindungsgerade}
\begin{enumerate}
\item Ist $B\ne \emptyset$ affiner Teilraum, dann gilt:
\[U_B=\{\overrightarrow{PQ}\mid P,Q\in B\}\]
\item Sind $\emptyset\ne B \subseteq C$ affine Teilräume und $\dim(B)=\dim(C)$, dann ist $B=C$.
\item Durch zwei Punkte $P\ne Q$ in $A$ geht genau eine Gerade.
\[PQ := K \cdot \overrightarrow{PQ}+P = \{\lambda \cdot \overrightarrow{PQ}+P\mid\lambda\in K\}\le A\]
Diese wird die \textbf{Verbindungsgerade} von $P$ und $Q$ genannt.
\item Drei Punkte $P,Q,R \in A$ liegen genau dann auf \textbf{einer} Geraden, wenn gilt,
dass $\overrightarrow{PQ}$ und $\overrightarrow{QR}$ linear abhängig sind.
\end{enumerate}
\end{lemma}

\begin{proof}
\begin{enumerate}
\item 
\begin{enumerate}
\item["`$\supseteq$"']\checkmark
\item["`$\subseteq$"'] Da $B$ affiner Teilraum mit Richtung $U_B$ ist, gilt für alle $P,Q\in B$:
\[\exists_1 x\in B: x=\overrightarrow{PQ} \iff x+P = Q\]
\end{enumerate}
\item Aus (1) folgt mit $B\subseteq C$, dass $U_B\subseteq U_C$ gilt. Da diese die
gleiche Dimension haben muss dann schon $U_B=U_C$ gelten. Für $P\in B\cap C$ gilt dann:
\[B=U_B+P=U_C+P=C\]
\item Es ist klar, dass $P$ und $Q$ auf der Geraden $PQ$ liegen, daher muss lediglich die
Eindeutigkeit gezeigt werden.\\
Sei $B$ eine Gerade mit $P,Q\in B$ und $U:=U_B$. Da $P\ne Q$ ist, 
ist $\overrightarrow{PQ}\in U$ nicht der Nullvektor. Da außerdem $\dim U=1$ ist, gilt:
\[U=K\cdot\overrightarrow{PQ}\]
Daraus folgt:
\[B=U+P=PQ\]
\item Sei $x:=\overrightarrow{PQ}$ und $y:=\overrightarrow{QR}$. Es existiert genau dann
eine Gerade $B$ mit $P,Q,R\in B$, wenn gilt:
\[\exists\text{ VRm }U: \dim U=1, x,y\in U\]
Also genau dann, wenn $x$ und $y$ linear abhängig sind.
\end{enumerate}
\end{proof}

\begin{theo}[Teilraumkriterium]
Sei $A$ affiner Raum mit Richtung $V$ und sei $\emptyset \ne B\subseteq A$. Dann 
sind equivalent:
\begin{enumerate}
\item $B$ ist affiner Teilraum.
\item Es existieren $P\in A$ und $U\le V$, sodass gilt:
\[B=U+P\]
\item Falls $|K| >2$, so ist auch equivalent:
\[\forall P,Q\in B: P\ne Q \implies PQ \subseteq B\]
\item Falls $A=\mathbb{A}_n(K)$, so ist auch equivalent, dass $B$ Lösungsmenge eines LGS ist.
\end{enumerate}
\end{theo}

\begin{proof}
Die Äquivalenz ergibt sich aus folgendem Ringschluss:
\begin{itemize}
\item[(1)$\implies$(2)] Ist $B$ affiner Teilraum, so gilt:
\[\exists U\le V:\forall P\in U:B=U+P\]
\item[(2)$\implies$(1)] $B=U+P$ ist affiner Teilraum, denn $U$ operiert auf $B$
und für $Q,R\in B:$ gilt:
\[\exists x,y\in U: Q=x+P,R=y+P \text{ und}\] 
\[\exists_1 \text{ Translation } \overrightarrow{QR}=y-x\in U\]
Daraus folgt, dass $U$ affiner Teilraum ist.
\item[(1)$\implies$(3)] Sei $B$ affiner Teilraum mit $P,Q\in B, P\ne Q$. Dann gilt:
\begin{align*}
&\overrightarrow{PQ}\in U_B\\
&\implies\forall \lambda\in K:\lambda\cdot\overrightarrow{PQ}+P\in B\\
&\implies PQ\subseteq B
\end{align*}
\item[(3)$\implies$(2)] Setze $U:=\{\overrightarrow{PQ}\mid P,Q\in B\}\subseteq V$.\\
\textbf{Zeige zunächst:} Für alle $P\in B$ gilt:
\[U+P\subseteq B\] 
D.h. für alle $y\in U$ gilt:
\[y+P\in B\]
\begin{enumerate}
\item["`$\subseteq$"'] Sei also $0\ne y\in U$, dann existiert ein $Q\ne R\in B$, sodass gilt: 
\[ y=\overrightarrow{QR}\] 
Setze $z:=\overrightarrow{PQ}$.\\
\textbf{Fall $y,z$ linear abhängig:}\\
Aus dem Lemma folgt, dass $P,Q,R$ auf der Geraden $QR=\{\lambda\cdot y+P\mid \lambda\in K\}
\stackrel{(3)}{\subseteq} B$ liegen. Insbesondere gilt: 
\[y+P\in B\]
\textbf{Fall $y,z$ linear unabhängig:}\\
Wähle $\lambda\in K\setminus\{0,1\}$. Betrachte $S:=\frac{\lambda}{\lambda-1}y+P$, 
$N:=\lambda z+P$.\\
Dann ist $N\in PQ\subseteq B$.\\
Annahme: $N=R$.
Dann gilt:
\begin{align*}
N&=\lambda z+p\\
&=R\\
&=y+z+P
\end{align*}
Daraus folgt, dass $y$ und $z$ linear abhängig sind. $\lightning$
Es gilt also $N\ne R$. Ferner gilt, dass $S,N,R$ auf einer Geraden liegen, denn:
\[\overrightarrow{NR} = y+z-\lambda z = y+(1-\lambda)z \text{ und}\]
\[\overrightarrow{SN} = \lambda z-\frac{\lambda}{1-\lambda}y=\frac{\lambda}{\lambda-1}
((\lambda-1)z-y)\]
sind linear abhängig.\\
Aus $N,R\in B$ folgt:
\[S\in NR \stackrel{(3)}{\subseteq}B\]
Außerdem gilt: $S\ne P$, also $SP\in B$ und damit $y+P\in B$\\
Es gilt sogar: $B=U+P$, da für alle $Q\in B$ gilt:
\[Q=\overrightarrow{PQ}+P\in U+P\]
\item["`$\supseteq$"'] \checkmark
\end{enumerate}
\textbf{Es bleibt zu zeigen:} $U\le V$ (Untervektorraum)\\
Seien $x,y\in U,\alpha\in K$. O.B.d.A lässt sich $x\ne 0$ annehmen, etwa $x=\overrightarrow{PQ},
P,Q\in B$. Dann gilt:\\
\[\alpha x+P\in PQ \subseteq B \implies \alpha x\in U\]
Also genügt es zu zeigen, dass $x+y$ in $U$ liegt. Sei $P':=x+P$.\\
Dann gilt mit $x=\overrightarrow{PQ}$ und $y+P\in U+P\subseteq B$:\\
\begin{align*}
&(x+y)+P=x+(y+P)\in U+P'\subseteq B\\
&\implies x+y\in U
\end{align*}
\end{itemize}
\end{proof}

\section{Eigentschaften affiner Räume}
\begin{lemma}
Sei $I\ne\emptyset$ Indexmenge und $(B_i)_{i\in I}$ eine Familie affiner Teilräume von $A$.\\
Dann ist $B:= \bigcap_{i\in I} B_i$ affiner Teilraum von $A$ mit Richtung
$U_B=\bigcap_{i\in I} U_{B_i}$, falls $B\ne\emptyset$.
\end{lemma}

\begin{proof}
Sei $B\ne\emptyset$, dann existiert ein $P\in\bigcap_{i\in I}B_i$.
Setze $U:=\bigcap_{i\in I}U_{B_i} \le V$.\\
Dann gilt für ein $Q\in A$:\\
\begin{align*}
Q\in U+P &\iff \forall i\in I: Q\in U_{B_i}+P\\
&\iff Q\in\bigcap_{i\in I}B_i\\
&\iff Q\in B
\end{align*}
Daraus folgt: $B=U+P$
\end{proof}

\begin{definition}
\index{affin!Hülle}\index{Hülle!affine}
Sei $M$ Teilmenge von $A$, $C$ die Menge aller affinen Teilräume von $A$, die $M$ enthalten.\\
Dann heißt:
\[[M]:=\bigcap_{B\in C}B\]
die \textbf{affine Hülle} von $M$.\\
Für $M=\{P_1,\ldots,P_m\}$ schreibe: $[P_1,\ldots,P_m]:=[M]$.
\end{definition}

\begin{example}
Sei $P\ne Q$, dann ist $[P,Q]=PQ$ die Gerade durch $P$ und $Q$.
\end{example}

\begin{lemma}
\index{allgemein!Lage}\index{Lage!allgemeine}
Seien $P_0,\ldots,P_m\in A$ und sei $x_i:=\overrightarrow{P_0P_i}\in V$ für alle $i\in\{1,\ldots,m\}$.\\
Dann gilt:
\[[P_0,\ldots ,P_m]=\langle x_1,\ldots ,x_m \rangle +P_0\]
Insbesondere ist $\dim{[P_0,\ldots,P_m]}\le m$.\\
Falls gilt: $\dim{[P_0,\ldots,P_m]}=m$ sagt man, $P_0,\ldots,P_m$ sind in \textbf{allgemeiner Lage}.
\end{lemma}

\begin{proof}
\begin{enumerate}
\item["`$\subseteq$"'] Für alle $i\in\{1,\ldots,m\}$ gilt:
\[P_i=x_i+P_0\subseteq\langle x_1,\ldots,x_m\rangle+P_0\]
\item["`$\supseteq$"'] Sei $\sum_{i=1}^m{\alpha_i x_i+P_0}\in\langle x_1,\ldots,x_m\rangle+P_0$,
und sei $B\supseteq\{P_0,\ldots,P_m\}$ beliebiger affiner Teilraum. Dann gilt:
\begin{align*}
&\forall i\in\{1,\ldots,m\}: x_i=\overrightarrow{P_0P_i}\in U_B\\
&\implies \sum_{i=1}^m{\alpha_i x_i}\in U_B\\
&\implies \sum_{i=1}^m{\alpha_i x_i+P_0}\in U_B+P_0=B
\end{align*}
Da dies für einen beliebigen affinen Teilraum $B$ gilt, der $\{P_0,\ldots,P_m\}$ enthält, gilt dies für alle
solche Teilräume. Sei $C$ die Menge aller affinen Teilräume die $\{P_0,\ldots,P_m\}$ enthalten. Dann folgt:
\begin{align*}
&\forall B\in C: \sum_{i=1}^m{\alpha_i x_i+P_0}\in B\\ 
&\iff \sum_{i=1}^m{\alpha_i x_i+P_0}\in\bigcap_{B\in C}B\\
&\iff \sum_{i=1}^m{\alpha_i x_i+P_0}\in[P_0,\ldots,P_m]
\end{align*}
\end{enumerate}
\end{proof}

\begin{theo}
Seien $A_1:=U_1+P_1,A_2:=U_2+P_2$ affine Teilräume von A. Dann gilt:
\begin{enumerate}
\item $U_{[A_1\cup A_2]}=U_1+U_2+\langle\overrightarrow{P_1P_2}\rangle$
\item $A_1\cap A_2\ne\emptyset \implies \dim([A_1\cup A_2])=\dim{A_1}+\dim{A_2}-\dim{(A_1\cap A_2)}\\
A_1\cap A_2=\emptyset \implies \dim([A_1\cup A_2])=\dim{A_1}+\dim{A_2}-\dim{(U_1\cap U_2)}+1$
\end{enumerate}
\end{theo}

\begin{proof}
\begin{enumerate}
\item Sei $y:=\overrightarrow{P_1P_2}$ und $U:=U_1+U_2+\langle y\rangle$.\\
\textbf{Zu Zeigen:} $U+P_1=[A_1\cup A_2]$, d.h. $U_{[A_1\cup A_2]}=U$
\begin{enumerate}
\item["`$\subseteq$"'] Für einen beliebigen affinen Raum $B\supseteq A_1\cup A_2$ gilt: $U_B\ge U_1,U_2,
\langle y\rangle$.\\
Also gilt für $x=x_1+x_2+\alpha y\in U$ mit $x_1\in U_1, x_2\in U_2$:
\begin{align*}
&x=x_1+x_2+\alpha y\in U_B\\
&\implies x+P_1\in U_B+P_1=B\\
&\implies x+P_1\in \bigcap B=[A_1\cup A_2]
\end{align*}
\item["`$\supseteq$"'] \textbf{Zu zeigen:} $A_1\cup A_2\subseteq U+P_1$\\
\begin{align*}
A_1&=U_1+P_1\subseteq U+P_1\\
A_2&=U_2+P_2=U_2+y+P_1\subseteq U+P_1
\end{align*}
\end{enumerate}
\item 
\textbf{Fall $A_1\cap A_2\ne\emptyset$:} Nach Lemma gilt $U_{A_1\cap A_2}=U_1\cap U_2$,
und dass $P_1=P_2$ wählbar ist.\\ 
Daraus folgt $U=U_1+U_2$ (mit $y=0$). Also gilt: $[A_1\cup A_2]=U_1+U_2+P_1$ mit:
\begin{align*}
\dim{[A_1+A_2]}&=\dim{(U_1+U_2)}\\
&=\dim{U_1}+\dim{U_2}-\dim{(U_1\cap U_2)}\\
&=\dim{A_1}+\dim{A_2}-\dim{(A_1\cap A_2)}
\end{align*}
\textbf{Fall $A_1\cap A_2=\emptyset$}: Annahme: $y\in U_1+U_2$.\\
Dann ist $y=x_1+x_2$ für ein $x_1\in U_1,x_2\in U_2$. Daraus folgt:
\[x_1+P_1 = -x_2+y+P_1=-x_2+P_2\in A_1\cap A_2\ \lightning\]
Also ist $y\notin U_1+U_2$. Daraus folgt:
\[\dim{U}=\dim{(U_1+U_2)}+1\]
Der restliche Beweis erfolgt analog zum ersten Fall.
\end{enumerate}
\end{proof}

\begin{definition}
\index{Parallelität}
Affine Teilräume $B,C$ von $A$ heißen \textbf{parallel}, wenn gilt:
\[U_B\le U_C \text{ oder }U_C\le U_B\]
Schreibe: $B\parallel C$.
\end{definition}

\begin{example}
Man denke nicht nur an parallele Geraden oder Ebenen, sondern etwa auch an Gerade $\parallel$ Ebene.
\end{example}

\begin{comment}
\begin{enumerate}
\item Auf den Teilräumen einer festen Dimension ist Parallelität eine Äquivalenzrelation.
\item Aus $B\parallel C$ folgt: $(B\subseteq C) \vee (B\supseteq C) \vee (B\cap C=\emptyset)$
\item Für alle $P\in A$ und alle affinen Teilräume $B\ne\emptyset$ existiert genau ein affiner Teilraum
$C$ mit:
\begin{enumerate}[(a)]
\item $P\in C$
\item $B\parallel C$
\item $\dim{C}=\dim{B}$
\end{enumerate}
\end{enumerate}
\end{comment}

\begin{proof}
\begin{enumerate}
\item Leichte Übung!
\item Sei $P\in B\cap C$ und o.B.d.A $U_B\le U_C$. Dann gilt:
\[B=U_B+P\le U_C+P=C\]
\item Es muss $C=U_B+P$ gelten, da aus b) und c) folgt: $U_C=U_B$
\end{enumerate}
\end{proof}

\begin{theo}
Sei $A$ affiner Raum mit $\dim{A}=n>1$, $G\subseteq A$ Gerade und $H$ Hyperebene in $A$.\\
Dann gilt:
\begin{enumerate}
\item $G\cap H=\emptyset \implies G\parallel H$
\item $G\not\parallel H \implies \exists P: G\cap H=\{P\}$
\end{enumerate}
\end{theo}

\begin{comment}
$\dim{G\cap H}\le\dim{G}=1 \implies G\cap H=\begin{cases}
\emptyset\\
\text{Punkt}\\
\text{Gerade}
\end{cases}$
\end{comment}

\begin{proof}
\begin{enumerate}
\item Sei $G\cap H=\emptyset$, dann ist $G\cup H$ echte Obermenge von $H$. Es gilt also:
\[H \subsetneq G\cup H \subseteq [G\cup H]\]
Daraus folgt für die Dimensionen:
\begin{align*}
&n-1=\dim H < dim[G\cup H]\le n\\
&\implies \dim[G\cup H]=n\\
&\implies [G\cup H]=A
\end{align*}
Aus der Dimensionsformel für die affine Hülle folgt:
\begin{align*}
n&=\dim[G\cup H]\\
&= \dim G+\dim H -\dim(U_G\cap U_H)+1\\
&= n-\dim(U_G\cap U_H)+1
\end{align*}
Daraus folgt:
\begin{align*}
&\dim(U_G\cap U_H)=1=\dim U_G\\
&\implies U_G\cap U_H = U_G\\
&\implies U_G \subseteq U_H\\
&\implies G\parallel H
\end{align*}
\item Aus (1) folgt, dass $G\cap H$ nicht die leere Menge ist, wenn $G$ und $H$ nicht
parallel sind.\\
Sei nun $G':=G\cap H$ eine Gerade. Dann gilt:
\begin{align*}
&G'\subseteq G\\
&\implies G'=G\\
&\implies G\subseteq H\\
&\implies G\parallel H
\end{align*}
Also kann $G\cap H$ auch keine Gerade sein, wenn $G$ und $H$ nicht parallel sind. Mit der
Vorbemerkung folgt daraus, dass $G\cap H$ ein Punkt sein muss.
\end{enumerate}
\end{proof}

\chapter{Affine Koordinaten und affine Abbildungen}
\section{Grundbegriffe}
\begin{definition}
\index{affin!Koordinatensystem}\index{Koordinatensystem!affines}
\index{Ursprung}
\index{Koordinatenvektor}
\index{Koordinatendarstellung}
Sei $A$ ein affiner Raum mit Richtungs-VRm V der Dimension n.
\begin{enumerate}[(a)]
\item Sei $\mathcal{B}$ die Menge aller Basen von V. Ein Paar $\mathcal{K}:=(\mathcal{O},B)
\in A\times\mathcal{B}$ heißt \textbf{affines Koordinatensystem}, wobei $\mathcal{O}$ der
\textbf{Ursprung} heißt.
\item  Durch die Koordinatendarstellung $D_B:V\to K^n$ zur Basis $B$ definiert:
\[D_\mathcal{K}:A\to K^n,P\mapsto D_B(\overrightarrow{\mathcal{O}P})\]
den \textbf{Koordinatenvektor} $D_\mathcal{K}(P)$ von P bezüglich $\mathcal{O}$.
\item Die Abbildung $D_\mathcal{K}:A\to\mathbb{A}^n(K)$ heißt \textbf{Koordinatendarstellung}
zum Koordinatensystem $\mathcal{K}$.
\end{enumerate}
\end{definition}

\begin{task}
Was entspricht Homomorphismen von VRmen bei affinen Räumen?
\end{task}

\begin{definition}
\index{affin!Abbildung}\index{Abbildung!affine}
\index{Morphismus!affiner Räume}
Seien $A,B$ affine Räume mit Richtungen $V,W$.
Eine Abbildung $\varphi:A\to B$ heißt \textbf{affine Abbildung} oder \textbf{Morphismus affiner Räume},
falls ein $\Phi\in\Hom(V,W)$ existiert, sodass gilt:
\[\forall x\in V,\forall P\in A: \varphi(x+P)=\Phi(x)+\varphi(P)\]
Schreibe: $\Homaff(A,B):=\{\varphi:A\to B\mid\varphi$ affin $\}$.
\end{definition}

\begin{example}
\begin{enumerate}
\item  Die Identität $\id_A:A\to A$ ist affin mit zugehörigem $\Phi=\id_V$.
\item Für \textbf{festes} $Q\in B$ ist die konstante Abbildung $\varphi_Q:A\to B,
P\mapsto Q$ affin, mit der Nullabbildung als zugehörigem Homomorphismus.
\end{enumerate}
\end{example}

\begin{comment}
\begin{enumerate}
\item $\varphi:A\to B$ mit zugehörigem $\Phi\in\Hom(V,W)$ ist genau dann affin, wenn gilt:
\[\exists P_0\in A:\forall x\in V: \varphi(x+P_0)=\Phi(x)+\varphi(P_0)\]
\item Ist $\varphi\in\Homaff(A,B)$, so ist der zugehörige Homomorphismus \mbox{$\Phi=:\Lambda_\varphi$}
eindeutig bestimmt.
\item Die Hintereinanderausführung affiner Abbildungen ist affin, d.h.:
\[\Homaff(A,B)\times\Homaff(B,C)\to \Homaff(A,C), (\varphi,\psi)\mapsto\psi\circ\varphi\]
\item $\varphi\in\Homaff(A,B)$ ist genau dann injektiv (bzw. surjektiv), wenn $\Lambda_\varphi$
injektiv (bzw. surjektiv) ist.
\item Ist $\varphi\in\Homaff(A,B)$ bijektiv, so existiert $\varphi^{-1}\in\Homaff(B,A)$.
\end{enumerate}
\end{comment}

\begin{definition}
\index{Isomorphismus!affiner Räume}
\index{Affinität}
\index{affin!Automorphismus}\index{Automorphismus!affiner}
\index{affin!Gruppe}\index{Gruppe!affine}
Ein bijektives $\varphi\in\Homaff(A,B)$ heißt \textbf{Isomorphismus affiner Räume} oder \textbf{Affinität}.\\
Ist zusätzlich $A=B$, so heißt $\varphi\in\Homaff(A,A)$ \textbf{Automorphismus}. Diese Automorphismen
bilden die Gruppe $\Autaff(A)$, genannt die \textbf{affine Gruppe} von $A$.
\end{definition}

\begin{proof}
\begin{enumerate}
\item Sei $P\in A$ beliebig und $y:=\overrightarrow{P_0P}$. Dann gilt für alle $x\in V$:
\begin{align*}
\varphi(x+P) &= \varphi(x+y+P_0)\\
&= \Phi(x+y)+\varphi(P_0)\\
&= \Phi(x)+\Phi(y)+\varphi(P_0)\\
&= \Phi(x)+\varphi(y+P_0)\\
&= \Phi(x)+\varphi(P)
\end{align*}
\item Sei $\varphi\in\Homaff(A,B)$ gegeben, dann gilt für alle $P\in A,x\in V$:
\begin{align*}
&\varphi(x+P)=\Phi(x)+\varphi(P)\\
&\implies \Phi(x)=\overrightarrow{\varphi(P)\varphi(x+P)}
\end{align*}
Also ist $\Phi$ durch $\varphi$ eindeutig bestimmt.
\item Sei $\varphi\in\Homaff(A,B)$, $\psi\in\Homaff(B,C)$. Dann gilt für alle $P\in A,x\in V$:
\begin{align*}
\psi\circ\varphi(x+P) &= \psi(\varphi(x+P))\\
&= \psi(\Lambda_\varphi(x)+\varphi(P))\\
&= \Lambda_\psi(\Lambda_\varphi(x))+\psi(\varphi(P))\\
&= \Lambda_\psi\circ\Lambda_\varphi(x)+\psi\circ\varphi(P)
\end{align*}
Also ist $\psi\circ\varphi$ affin mit zugehörigem Homomorphismus $\Lambda_{\psi\circ\varphi}=
\Lambda_\psi\circ\Lambda_\varphi$.
\item Es gilt für $\varphi\in\Homaff(A,B)$:
\begin{align*}
\varphi \text{ injektiv }&\iff (\varphi(P)=\varphi(Q) \implies P=Q)\\
&\iff (\varphi(\overrightarrow{QP}+Q)=\varphi(Q) \implies P=Q)\\
&\iff (\Lambda_\varphi(\overrightarrow{QP})+\varphi(Q)=\varphi(Q) \implies P=Q)\\
&\iff (\Lambda_\varphi(\overrightarrow{QP})=0 \implies \overrightarrow{QP}=0)\\
&\iff \Lambda_\varphi \text{ ist injektiv}
\end{align*}
Der Beweis für Surjektivität erfolgt analog.
\item Leichte Übung!
\end{enumerate}
\end{proof}

\begin{theo}
Seien $A,B$ affine Teilräume mit Richtungen $V,W$.
Zu $(P_0,Q_0)\in A\times B$ und $\Phi\in\Hom(V,W)$ existiert genau eine affine Abbildung
$\varphi:A\to B$ mit $\Lambda_\varphi=\Phi$ und $\varphi(P_0)=Q_0$.
\end{theo}

\begin{proof}
Es ist $A=V+P_0$. Definiere $\varphi(x+P_0):=\Phi(x)+Q_0$, so ergibt sich nach Bemerkung (1)
eine affine Abbildung mit $\varphi(P_0)=Q_0$. Dies legt $\varphi$ bereits eindeutig fest.
\end{proof}

\begin{theo}
Die Koordinatenabbildung $D_\mathcal{K}:A\to\mathbb{A}^n(K)$ zu einem Koordinatensystem 
$\mathcal{K}=(\mathcal{O},B)$ ist ein affiner Isomorphismus (mit zugehöriger linearer Abbildung
$D_B:V\to K^n$).
\end{theo}

\begin{proof}
Es gilt:
\begin{align*}
D_\mathcal{K}(x+\mathcal{O})&\stackrel{Def.}{=} D_B(x)\\
&= D_B(x)+0\\
&= D_B(x)+D_\mathcal{K}(\mathcal{O})
\end{align*}
Nach dem letzten Satz existiert genau eine affine Abbildung, die dies tut.
Dass es sich bei $D_\mathcal{K}$ um eine Isometrie handelt, wurde bereits früher eingesehen,
da $D_B$ Isometrie ist.
\end{proof}

\begin{corollary}
Affine Räume über festen Körper sind genau dann isomorph, wenn ihre Dimension gleich ist.
\end{corollary}

\begin{theo}[Erhaltung von Teilräumen]
Sei $\varphi\in\Homaff(A,B)$ und $C\subseteq A$.
Falls $C$ affiner Teilraum  mit Richtung $U:=U_C$ ist, so ist $\varphi(C)\subseteq B$
affiner Teilraum mit Richtung $\Lambda_\varphi(U)$.\\
Ist $\varphi$ Isomorphismus, so gilt:
\begin{enumerate}
\item $C\subseteq A$ ist genau dann affiner Teilraum, wenn $\varphi(C)\subseteq B$ affiner Teilraum ist.
\item Es gilt $\dim C = \dim \varphi(C)$ für jeden affinen Teilraum $C$.
\item Sind $C,C'\subseteq A$ affine Teilräume, so gilt:
\[\varphi([C\cup C'])=[\varphi(C)\cup\varphi(C')]\]
und:
\[\varphi(C\cap C')=\varphi(C)\cap\varphi(C')\]
\item $C\parallel C' \implies \varphi(C)\parallel\varphi(C')$
\end{enumerate}
\end{theo}

\begin{proof}
Sei $\varphi\in\Homaff(A,B), P\in C$ (d.h. $C=U+P$). Nach Teilraumkriterium gilt dann:
\[\varphi(C)=\Phi(U)+\varphi(P)\]
Daraus folgt, dass $\varphi(C)$ affiner Teilraum ist.
\begin{enumerate}
\item Leichte Übung!
\item Leichte Übung!
\item Sogar für beliebige Teilmengen $C,C'\subseteq A$ gilt, wenn $\varphi$ bijektiv ist:
\[\varphi(C\cap C')=\varphi(C)\cap\varphi(C')\]
Für alle affinen Teilräume $D$, die $C\cup C'$ enhalten, gilt:
\[\varphi(D)\supseteq\varphi(C)\cup\varphi(C')\]
Also gilt insbesondere auch für $D:=[C\cup C']$:
\[\varphi([C\cup C'])\supseteq\varphi(C)\cup\varphi(C')\]
Daraus folgt (für jede affine Abbildung, also insbesondere auch für $\varphi^{-1}$):
\[\varphi([C\cup C'])\supseteq[\varphi(C)\cup\varphi(C')]\]
Insgesamt folgt:
\begin{align*}
[C\cup C']&\supseteq\varphi^{-1}([\varphi(C)\cup\varphi(C')])\\
&\supseteq[\varphi^{-1}(\varphi(C))\cup\varphi^{-1}(\varphi(C'))]\\
&= [C\cup C']
\end{align*}
Daraus folgt die Gleichheit.
\item Leichte Übung!
\end{enumerate}
\end{proof}

\subsection{Grundaufgaben im affinen Standardraum $\mathbb{A}_n(K)$}
Seien $P_0,\ldots,P_m,Q_0,\ldots,Q_s\in K^n$ und $B:=[P_0,\ldots,P_m], C:=Q_0,\ldots,Q_s$
gegeben. Ziel ist es $[B\cup C]$ und $B\cap C$ zu berechnen.\\

Mit $x_i:=\overrightarrow{P_0P_i}=P_i-P_0$ gilt:
\[B=\langle x_1,\ldots,x_m\rangle+P_0\]
Analog gilt mit $z_j:=\overrightarrow{Q_0Q_i}=Q_i-Q_0$:
\[C=\langle z_1,\ldots,z_s\rangle+Q_0\]
Daraus folgt dann mit $y:=\overrightarrow{P_0Q_0}$:
\[[B\cup C]=\langle x_1,\ldots,x_m,z_1,\ldots,z_s,y\rangle+P_0\]

\begin{enumerate}
\item Finde mit dem Gauß-Algorithmus eine Basis $\{b_1,\ldots,b_r\}$ von $U$, dann gilt:
\[[B\cup C]=[b_1+P_0,\ldots,b_r+P_0,P_0]\]
mit erzeugenden Punkten in allgemeiner Lage.
\item Interpretiere $B$ als Lösungsmenge $\mathcal{L}(A,b)$ eines LGS $Ax=b$.\\
Sei $x_0=P_0\in K^n$, dann liefert der Spezialfall $B=C$ in (1):
\[B=U+x_0\]
wobei ${b_1,\ldots,b_r}$ Basis von $U$ ist.\\
Ziel ist es nun, eine Matrix $A\in K^{n-r\times n}$ zu finden, mit $U=\Kern(\Lambda_A)$.
Dazu betrachte die Matrix:
\[M:=
\begin{pmatrix}
b_1&\cdots&b_r
\end{pmatrix}\]
Offensichtlich gilt $\rank(M)=r$.\\
Betrachte nun die Rechtsmultiplikation:
\[ P_M:K^n\to K^r,y\mapsto yM\]
Dann hat der Kern von $\rho_M$ Dimension $n-r$ und eine Basis aus Zeilenvektoren
$\{c_1,\ldots,c_{n-r}\}$. Damit lässt sich nun die Matrix $A$ wie folgt definieren:
\[A:=
\begin{pmatrix}
c_1\\
\vdots\\
c_{n-r}
\end{pmatrix}\]
Da der Rang von A offensichtlich $n-r$ ist, ist die Dimension des Kerns genau $r$, und es gilt:
\begin{align*}
&\forall t\in\{1,\ldots,n-r\}: c_tM=0\\
&\iff\forall t\in\{1,\ldots,n-r\},j\in\{1,\ldots,r\}: c_t\cdot b_j=0\\
&\iff\forall j\in\{1,\ldots,r\}: Ab_j=0
\end{align*}
Also ist $U$ Unterraum von $\Kern(\Lambda_A)$ und aus der Gleichheit der Dimensionen
beider Räume folgt dann:
\[B=\mathcal{L}(A,b)\]
\item Durchschnittsberechnung:\\
Finde mit Hilfe von (2) Matrizen $A,A'$ und $b,b'\in K^n$, sodass $B=\mathcal{L}(A,b),
C=\mathcal{L}(A',b')$ ist. Dann gilt:
\[B\cap C = \mathcal{L}(D,d)\text{ mit }D:=
\begin{pmatrix}
A\\
A'
\end{pmatrix}, d=
\begin{pmatrix}
b\\
b'
\end{pmatrix}\]
Es genügt nun das LGS $Dx=d$ zu lösen, um $B\cap C$ zu erhalten. 
\end{enumerate}

\begin{example}
Betrachte den affinen Raum $\mathbb{A}_3(\mathbb{F}_2)=\{0,1\}^3$. Gegeben seien die
Ebenen:
\begin{align*}
&E:=\langle 
\begin{pmatrix}
1\\0\\0\\
\end{pmatrix},
\begin{pmatrix}
0\\1\\0
\end{pmatrix}\rangle +
\begin{pmatrix}
0\\0\\0
\end{pmatrix}=[e_1,e_2,0]\\
&F:=\langle 
\begin{pmatrix}
0\\0\\1\\
\end{pmatrix},
\begin{pmatrix}
1\\0\\0
\end{pmatrix}\rangle +
\begin{pmatrix}
1\\1\\1
\end{pmatrix}=[e_2,e_1+e_2,e_2+e_3]
\end{align*}
Zur Bestimmung von $E\cap F$ werden zunächst die zu $E$ und $F$ gehörigen Gleichungssysteme
aufgestellt:
\begin{align*}
&E=\{x\in\mathbb{F}_2^3\mid x_3=0\}=\mathcal{L}((0,0,1),0)\\
&F=\Kern(\Lambda_{(0,1,0)})+
\begin{pmatrix}
1\\1\\1
\end{pmatrix} = \mathcal{L}((0,1,0),1)
\end{align*}
Daraus folgt:
\[E\cap F=\mathcal{L}(
\begin{pmatrix}
0&0&1\\
0&1&0
\end{pmatrix},
\begin{pmatrix}
0\\1
\end{pmatrix})=\{e_2,e_1+e_2\}\]
\end{example}

\begin{theo}[Satz von Pappos]
In einem affinen Raum $A$ mit Dimension 2 seien $G,G'$ verschiedene Geraden
mit $G\cap G'=\{O\}\in A$. Ferner seien $P_1,P_2,P_3\in G\setminus\{O\}$
und $Q_1,Q_2,Q_3\in G'\setminus\{O\}$, sodass gilt:
\[ P_1Q_3\parallel P_3Q_1\text{ und }P_1Q_2\parallel P_2Q_1\]
Daraus folgt:
\[ P_2Q_3\parallel P_3Q_2\]
\end{theo}

\begin{proof}
Da $Q_3\notin G$ ist, sind $O,P_1,Q_3$ in allgemeiner Lage. Daraus erhalten
wir folgendes Koordinatensystem: 
\[\mathcal{K}:=(O,\{\overrightarrow{OP_1},\overrightarrow{OQ_3}\})\]
Da die Koordinatendarstellung $D_{\mathcal{K}}:A\stackrel{\sim}{\to}\mathbb{A}_2(K)$ 
Parallelitäten und Schnittpunkte erhält, können wir o.B.d.A annehmen:
\[A=\mathbb{A}_2(K) \text{ und }O=\begin{pmatrix}0\\0\end{pmatrix}\]
Dann gilt:
\begin{align*}
&P_1=\overrightarrow{OP_1}=\begin{pmatrix}\lambda_1\\0\end{pmatrix}=\begin{pmatrix}1\\0\end{pmatrix}
& &P_2=\begin{pmatrix}\lambda_2\\0\end{pmatrix}
& &P_3=\begin{pmatrix}\lambda_3\\0\end{pmatrix}\\
&Q_3=\overrightarrow{OQ_3}=\begin{pmatrix}0\\\mu_1\end{pmatrix}=\begin{pmatrix}0\\1\end{pmatrix}
& &Q_2=\begin{pmatrix}0\\\mu_2\end{pmatrix}
& &Q_3=\begin{pmatrix}0\\\mu_3\end{pmatrix}
\end{align*}
Wobei $\lambda_2,\lambda_3,\mu_2,\mu_3\ne 0$ sind. Daraus folgt für die Richtungen:
\begin{align*}
&\forall i,j\in \{1,2,3\}:U_{P_iQ_j}=\langle\overrightarrow{P_iQ_j}\rangle =\langle 
\begin{pmatrix}\lambda_i\\-\mu_j\end{pmatrix}\rangle\\
&\implies U_{P_1Q_3}=\langle \begin{pmatrix}1\\-1\end{pmatrix}\rangle
\end{align*}
Nach Vorraussetzung ist $\lambda_3=\mu_1$ und es existiert ein $\rho\in K^\times$, sodass gilt:
\[ \begin{pmatrix}\lambda_2\\-\mu_1 \end{pmatrix}=\rho\begin{pmatrix}1\\-\mu_2\end{pmatrix}\]
Daraus folgt mit $\lambda_3=\rho\mu_2=\lambda_2\mu_2$:
\begin{align*}
U_{P_2Q_3}&=\langle \begin{pmatrix}\lambda_2 \\-\mu_3 \end{pmatrix}\rangle\\
&=\langle \begin{pmatrix} \lambda_2\mu_2\\-\mu_2 \end{pmatrix}\rangle\\
&=\langle \begin{pmatrix} \lambda_3\\-\mu_2 \end{pmatrix}\rangle\\
&= U_{P_3Q_2}
\end{align*}
Also sind $P_2Q_3$ und $P_3,Q_2$ parallel.
\end{proof}

\section{Koordinatenwechsel und Darstellung affiner Abbildungen}
\begin{lemma}
Seien $\mathcal{K}=(O,B)$ und $\mathcal{L}=(Q,C)$ Koordinatensysteme des affinen Raums
$A$ mit Richtung $V$. Sei $M_{CB}:=D_{CB}(\id_V)$ die Basiswechselmatrix.\\
Dann rechnen sich Koordinaten eines Punktes $P$ bzgl. $\mathcal{K}$ in die
Koordinaten bzgl. $\mathcal{L}$ wie folgt um:
\[D_\mathcal{L}(P)=M_{CB}\cdot(D_\mathcal{K}(P)-D_\mathcal{K}(Q)\]
\end{lemma}

\begin{proof}
Es gilt:
\begin{align*}
D_\mathcal{L}(P)&=D_C(\overrightarrow{QP})\\
&=M_{CB}\cdot D_B(\overrightarrow{QP})\\
&=M_{CB}\cdot D_B(\overrightarrow{OP}-\overrightarrow{OQ})\\
&=M_{CB}\cdot (D_B(\overrightarrow{OP})-D_B(\overrightarrow{OP}))\\
&=M_{CB}\cdot(D_\mathcal{K}(P)-D_\mathcal{K}(Q)
\end{align*}
\end{proof}

\begin{application}
Ist ein beliebiges Koordinatensystem $\mathcal{L}=(Q,B)$ gegeben, so lässt sich ein 
Punkt $P$ einfach in das Koordinatensystem $\mathcal{K}=(0,S)$ von $\mathbb{A}_n(K)$ 
mit Standardbasis $S$ überführen. Schreibe dazu $B$ als:
\[B=\begin{pmatrix}b_1&\cdots &b_n\end{pmatrix}\in K^{(n\times n)}\]
Dann ist $M_{SB}=B$ und es gilt:
\[D_\mathcal{L}(P)=M_{BS}(P-Q)=B^{-1}(P-Q)\]
\end{application}

\begin{lemma}
\begin{enumerate}
\item Die Abbildung $\psi:K^n\to K^m$ ist genau dann affin, wenn gilt:
\[\exists A\in K^{m\times n},a\in K^m:\Psi(x)=Ax+a\]
Schreibe daher kurz: $\psi=:(A,a)$
\item Ist ferner $C\in K^{t\times m},c\in K^r$, so gilt:
\[(C,c)\circ(A,a)=(CA,Ca+c)\]
\item Ist $m=n$ und $A\in\GL_n(K)$, so ist $(A,a)$ bijektiv und es gilt:
\[(A,a)^{-1}=(A^{-1},-A^{-1}a)\]
\end{enumerate}
\end{lemma}

\begin{proof}
\begin{enumerate}
\item Die Abbildung $\psi$ ist genau dann affin, wenn ein $\Lambda_\vp=\Lambda_A\in\Homaff
(K^n,K^m)$ für ein $A\in K^{m\times n}$, sodass gilt:
\[\psi(x)=\psi(x+0)=\Lambda_\vp(x)+\psi(0)\]
Die Behauptung folgt mit $a:=\psi(0)$.
\item Leichte Übung!
\item Leichte Übung!
\end{enumerate}
\end{proof}

\begin{definition}
Seien $A,A'$ affine Räume mit Koordinatensystemen $\mathcal{K}=(O,B),\mathcal{K'}=(O',B')$
und zugehörigen Koordinatenisomorphismen $D_{\mathcal{K}},D_{\mathcal{K'}}$. Definiere:
\[D{\mathcal{K'K}}(\vp):=D_{\mathcal{K'}}\circ \vp\circ D_{\mathcal{K}}^{-1}\in\Homaff(K^n,K^m)\]
\end{definition}

\begin{lemma}
Es gilt:
\[D_{\mathcal{K'K}}(\vp)=(D_{B'B}(\Lambda_\vp),D_{B'}(\overrightarrow{O'\vp(O)}))\]
\end{lemma}

\begin{proof}
Sei $P\in A$ beliebig, so entspricht es $D_\mathcal{K}(P)\in K^n$. Dann gilt:
\begin{align*}
D_{\mathcal{K'K}}(\vp)(D_\mathcal{K}(P)) &\stackrel{Def.}{=} D_\mathcal{K'}(\vp(P))\\
&\stackrel{Def.}{=} D_{B'}(\overrightarrow{O'\vp(P)})\\
&= D_{B'}(\overrightarrow{O'\vp(O)}+\overrightarrow{\vp(O)\vp(P)})\\
&= D_{B'}(\overrightarrow{O'\vp(O)})+D_{B'}(\overrightarrow{\vp(O)\vp(P)})\\
&= D_{B'}(\overrightarrow{O'\vp(O)})+D_{B'}(\Lambda_\vp (\overrightarrow{OP}))\\
&= D_{B'B}(\Lambda_\vp)\cdot D_B(\overrightarrow{OP})+D_{B'}(\overrightarrow{O'\vp(O)})\\
&= (D_{B'B}(\Lambda_\vp),D_{B'}(\overrightarrow{O'\vp(O)}))(D_\mathcal{K}(P))
\end{align*}
Da also beide Abbildungen auf einen beliebigen Punkt $P$ die selbe Wirkung haben, 
müssen sie gleich sein.
\end{proof}

\begin{comment}
Das Zusammenfügen von kommutativen Diagrammen liefert für einen weiteren affinen Raum
$A''$ mit Koordinatensystem $\mathcal{K''}=(O'',B'')$ und einer affinen Abbildung $\Psi:
A'\to A''$:
\[D_{\mathcal{K''K}}(\psi\circ\vp)=D_{\mathcal{K''K'}}(\psi)\circ D_{\mathcal{K'K}}(\vp)\]
\end{comment}

\begin{corollary}
\begin{enumerate}
\item Speziell für eine affine Abbildung $\vp:A\to A$ und zwei Koordinatensysteme
$\mathcal{K,L}$ gilt:
\[D_{\mathcal{LL}}(\vp)=D_{\mathcal{LK}}(\id)\circ D_{\mathcal{KK}}(\vp)\circ D_{\mathcal{KL}}(\id)\]
\item Insbesondere gilt für $\vp=\id$ gilt:
\[D_{\mathcal{LL}}(\vp)=D_{\mathcal{KK}}(\vp)\]
\item Für $\mathbb{A}_n(K)$ mit Standardkoordinatensystem $\mathcal{K}=(0,S)$, sei
$D_{\mathcal{KL}}(\id)=:(M,b)$. Dann gilt für $\vp=(A,a)=D_{\mathcal{KK}}(\vp)$:
\[D_{\mathcal{LL}}(\vp)=(M^{-1}AM,M^{-1}((A-I)b+a))\]
\end{enumerate}
\end{corollary}

\begin{proof}
\begin{enumerate}
\item Folgt aus zweimaligem anwenden der obigen Bemerkung.
\item Folgt aus (1).
\item Es gilt:
\[D_{\mathcal{LK}}(\id)=(M,b)^{-1}=(M^{-1},-M^{-1}b)\]
Der restliche Beweis ergibt sich aus (1).
\end{enumerate}
\end{proof}

\section{Geometrische Eigenschaften von affinen Abbildungen}
Wir haben gesehen, dass Koordinaten für den Umgang mit affinen Abbildungen \textbf{nützlich}
sind. Nun stellen wir die Frage, inwiefern Koordinaten \textbf{nötig} sind, d.h.
welche Eigenschaften von der Koordinatenwahl abhängen.

\begin{definition}
\renewcommand{\labelenumi}{(\alph{enumi})}
\index{Fixpunkt}
\index{Fixraum}
\index{Fixrichtung}
Sei $A$ affiner Raum und $\vp\in\Homaff(A,A)$.
\begin{enumerate}
\item $P\in A$ heißt \textbf{Fixpunkt} von $\vp$, falls gilt:
\[\vp(P)=P\]
\item Ein affiner Teilraum $\emptyset\ne B\subseteq A$ heißt \textbf{Fixraum} 
von $\vp$, falls gilt:
\[\vp(B)\subseteq B\]
\item Ein Untervektorraum $U$ des Richtungsvektorraums $U_A$ heißt \textbf{Fixrichtung}
von $\vp$, falls gilt:
\[\Lambda_\vp(U)\subseteq U\]
\end{enumerate}
\renewcommand{\labelenumi}{(\arabic{enumi})}
\end{definition}

\begin{example}
\index{Translation}
\index{Streckung}
\begin{enumerate}
\item Sei $x\in V:=U_A$ fest und eine \textbf{Translation} 
\[\vp=\tau_x:A\to A,P\mapsto x+P\] 
gegeben, dann gilt:
\begin{enumerate}
\item Für alle $U\le V$ ist $U$ Fixrichtung, da $\Lambda_\vp=\id_V$ ist.
\item Für $x\ne 0$ existieren keine Fixpunkte.
\item Für eine Fixgerade $G$ muss gelten:
\begin{align*}
&\vp(G)=x+G\subseteq G\\
\iff &x\in U_G
\end{align*}
Also ist die Menge aller Fixgeraden für $X\ne 0$:
\[\{Kx+P\mid P\in A\}\]
Beachte dass eine Fixgerade hier \textbf{keinen} Fixpunkt enthält.
\end{enumerate}
\item Seien $\mu\in K\setminus\{0\}$ und $P\in A$ fest und eine \textbf{Streckung}
\[\vp:A\to A,x+P\mapsto \mu x+P\]
mit Zentrum $P$ und Streckungsfaktor $\mu$ gegeben.\\
Da im Fall $\mu=1$ $\vp=\id=\tau_0$ gilt, wollen wir im Folgenden $\mu\ne 1$ annehmen.
\begin{enumerate}
\item Die Menge der Fixpunkte ist gleich $\{P\}$.
\item Für alle $U\le V$ ist $U$ Fixrichtung.
\item Fixgeraden sind genau die Geraden, die $P$ enthalten.
\end{enumerate}
\end{enumerate}
\end{example}

\begin{lemma}
Für $A\in K^{n\times n},a\in K^n$ und $\vp=(A,a)$ gilt:
\begin{enumerate}
\item Die Fixpunkte bilden den affinen Teilraum $\mathcal{L}(A-I,-a)$.
\item Genau dann, wenn $1$ kein Eigenwert von $A$ ist, ist die Menge der Fixpunkte
einelementig.
\item $B$ ist genau dann Fixraum von $\vp$, wenn $U_B$ Fixrichtung ist und ein
Punkt $P\in B$ mit $\vp(P)\in B$ existiert.
\end{enumerate}
\end{lemma}

\begin{proof}
\begin{enumerate}
\item Es ist $\vp=Ax+a$, also gilt:
\begin{align*}
\vp(x)=x &\iff Ax+a=x\\
&\iff Ax-x=-a\\
&\iff (A-I)x=-a\\
&\iff x\in\mathcal{L}(A-I,-a)
\end{align*}
\item Ist $1$ kein Eigenwert von $\vp$, so existiert $(A-I)^{-1}$, daraus folgt für 
einen Fixpunkt $x$:
\[x=(A-I)^{-1}(-a)\]
Also ist $x$ eindeutig bestimmt.
\item $B=U+P$ ist genau dann Fixraum, wenn gilt:
\begin{align*}
\vp(B)\subseteq B &\iff \vp(U+P)\subseteq U+P=B\\
&\iff \Lambda_\vp(U)+\vp(P)\subseteq U+P\\
&\iff \vp(P)\in B\wedge \Lambda_\vp(U)\subseteq U
\end{align*}
Also ist $U$ Fixrichtung.
\end{enumerate}
\end{proof}

\section{Geometrische Charakterisierung von Affinitäten}
\begin{definition}
\index{Geradentreue}
Sei $A$ ein affiner Raum mit einer (nicht notwendigerweise affinen) Abbildung
$\vp:A\to A$. $\vp$ heißt \textbf{geradentreu}, wenn für $G\subseteq A$ gilt:
\[G \text{ Gerade } \iff \vp(G) \text{ Gerade}\]
\end{definition}

\begin{example}
\begin{enumerate}
\item Affinitäten sind geradentreu.
\item Die Abbildung:
\[\vp:\mathbb{A}_2(\mathbb{C})\to\mathbb{A}_2(\mathbb{C}), (\alpha,\beta)\mapsto
(\overline\alpha,\overline\beta)\]
ist geradentreu, aber \textbf{nicht} affin.
\end{enumerate}
\end{example}

\textbf{\color{Red}Hier fehlt noch der Mitschrieb vom 18.6.!}

\begin{proof}
\textbf{3. Schritt:} Zu Zeigen: $\forall T\in U_0:$ T Fixpunkt!\\
\[\exists \alpha,\beta\in K_0: T=\alpha P\beta Q\]
\[\{T\}=\underbrace{(KP+\underbrace{\beta Q}_{Fixpunkt})}_{\parallel [O,P]}\cap \underbrace{(KQ+\underbrace{\alpha P}_{Fixpunkt})}_{\parallel [O,Q]}\]
\end{proof}

\chapter{Euklidische Punkträume}

Hier sei stets $K=\mathbb{R}$. Neu in diesem Paragraphen sind \textbf{Abstände} zwischen 
Punkten im affinen Raum.

\section{Grundbegriffe}

\begin{definition}
\index{euklidisch!Raum}\index{Raum!euklidischer}
\index{Abstand}
\renewcommand{\labelenumi}{(\alph{enumi})}
\begin{enumerate}
\item $(E,\langle\cdot,\cdot\rangle)$ mit einem affinen Raum $E$ über $\mathbb{R}$
und einem Skalarprodukt $\langle\cdot,\cdot\rangle$ auf dem Richtungs-VRm $V=U_E$ 
von $E$ heißt \textbf{euklidischer Raum}.
\item Der \textbf{Abstand} von $P,Q\in E$ ist definiert als:
\[d(P,Q):=\|\overrightarrow{PQ}\| \left(=\sqrt{\langle\overrightarrow{PQ},\overrightarrow{PQ}\rangle}\right)\]
\end{enumerate}
\renewcommand{\labelenumi}{(\arabic{enumi})}
\end{definition}

\begin{example}
Der \textbf{euklidische Standardraum} $E=\mathbb{A}_n(\mathbb{R})=\mathbb{R}^n$ mit Standardskalarprodukt
\[d(\begin{pmatrix}x_1\\\vdots\\x_n\end{pmatrix},\begin{pmatrix}y_1\\\vdots\\y_n\end{pmatrix})
=\sqrt{\sum_{i=1}^n(x_i-y_i)^2}\]
\end{example}

\begin{comment}
Der Abstand $d$ eines euklidischen Raums $E$ definiert eine Metrik auf $E$ (Positivdefinitheit, Symmetrie
und Dreiecksungleichung).
\end{comment}

\begin{definition}
\index{Koordinatensystem!cartesisches}\index{cartesisch!Koordinatensystem}
\index{Längentreue}
\index{Bewegung}
\index{Isometrie}
\renewcommand{\labelenumi}{(\alph{enumi})}
\begin{enumerate}
\item Ein Koordinatensystem $\mathcal{K}=(O,B)$ auf dem euklidischen Raum $E$ heißt \textbf{cartesisch},
falls $B$ Orthonormalbasis (bzgl $\langle\cdot,\cdot\rangle$) ist.
\item Seien $E,F$ euklidische Räume und $\vp:E\to F$ eine beliebige Abbildung. $\vp$ heißt
\textbf{längentreu}, falls gilt:
\[\forall P,Q\in E: d(\vp(P),\vp(Q))=d(P,Q)\] 
\item $\vp$ heißt \textbf{isometrisch}, falls $\vp$ affin und längentreu ist.\\
$\vp$ heißt \textbf{Bewegung} von $E$, falls $\vp\in\Autaff(E)$ und isometrisch ist.
Ist ferner $\det(\Lambda_\vp)=1$, so heißt $\vp$ eine \textbf{eigentliche Bewegung}.
\end{enumerate}
\renewcommand{\labelenumi}{(\arabic{enumi})}
\end{definition}

\begin{comment}
Die Menge aller Bewegungen (schreibe $\Autdist(E)$) ist eine Gruppe mit Untergruppe
der Menge aller eigentlichen Bewegungen (schreibe $\Aut^+_\text{dist}(E)$).
\end{comment}

\begin{lemma}
Seien $E,F$ euklidische Räume und $\vp\in\Homaff(E,F)$. Falls $\Lambda_\vp$ ein 
Morphismus von Skalarprodukträumen ist, so ist $\vp$ isometrisch.
\end{lemma}

\begin{proof}
Sei $\Phi :=\Lambda_\vp$, dann gilt $\langle\Phi(x),\Phi(y)\rangle
=\langle x,y\rangle$ und es folgt:
\begin{align*}
d(\vp(P),\vp(Q))&=\|\overrightarrow{\vp(P)\vp(Q)}\|\\
&=\|\Phi(\overrightarrow{PQ})\|\\
&=\|\overrightarrow{PQ}\|\\
&= d(P,Q)
\end{align*}
\end{proof}

\begin{corollary}
Sei $\mathcal{K}=(O,B)$ cartesisches Koordinatensystem eines euklisischen Raums $E$. Dann
ist die Koordinatendarstellung $D_\mathcal{K}:E\to \mathbb{R}^n$ ein isometrischer affiner
Isomorphismus.\\
Daher genügt es meistens, den euklidischen Standardraum zu behandeln.
\end{corollary}

\begin{proof}
Es ist $D_\mathcal{K}(P)=D_B(\overrightarrow{OP})$ mit $B$ ONB. Daraus folgt:
\[D_B:V\stackrel{\sim}{\to}\mathbb{R}^n\] 
d.h. $D_B$ ist Isometrie von $V$ in den $\mathbb{R}^n$.
\end{proof}

\begin{comment}
Im Standardraum gilt:
\[\Autdist(\mathbb{R}^n)=\{(A,a)\in\Homaff(\mathbb{R}^n,\mathbb{R}^n)\mid A\in O_n\}\]
\end{comment}

\begin{definition}
\index{orthogonal!Teilräume}
$A,B$ affine Teilräume eines euklidischen Raumes $E$ heißen \textbf{orthogonal}, falls
$U_A \bot U_B$.
\end{definition}

\begin{task}
Bestimme den \textbf{Abstand} zwischen zwei Teilräumen $A,B$. Dieser ist wie folgt definiert:
\[d(A,B):=\min\{d(P,Q)\mid P\in A,Q\in B\}\]
\end{task}

\begin{method}
Lot fällen! (Dabei genügt es $E=\mathbb{R}^n$ zu betrachten.)
\end{method}

\begin{definition}
\index{gemeinsam!Lot}\index{Lot!gemeinsames}
\index{Lotfußpunkt}
Eine Gerade $G$ heißt \textbf{gemeinsames Lot} von $A,B$ mit \textbf{Lotfußpunkten} $P^+$ 
und $Q^+$, falls gilt:
\begin{align*}
G&\bot A &G&\bot B\\
G\cap A&=\{P^+\} &G\cap B&=\{Q^+\} 
\end{align*}
\end{definition}

\begin{theo}
Seien $A,B$ affine Teilräume von $\mathbb{R}^n$ mit $A\ne\emptyset\ne B$. Aus $\dim(U_A+U_B)<n$ folgt, dass
ein gemeinsames Lot $G$ mit Lotfußpunkten $P^+,Q^+$ und $d(A,B)=d(P^+,Q^+)$ existiert.
\end{theo}

\begin{proof}
Falls $G$ existiert, so gilt für alle $P\in A,Q\in B$:
\[d(P,Q)=\|\overrightarrow{PQ}\|=\|\underbrace{\overrightarrow{PP^+}}_{=:x\in U_A}+\underbrace{\overrightarrow{P^+Q^+}}_{=:y\in U_G}+\underbrace{\overrightarrow{Q^+Q}}_{=:z\in U_B}\|\]
wobei nach Vorraussetzung $y\bot y$ und $y \bot z$, also auch $y \bot (x+z)$ ist.
Nach Pythagoras gilt:
\[\|y+(x+z)\|^2=\|y\|^2+\|x+z\|^2\ge \|y\|^2\]
Mit Wurzelziehen folgt daraus:
\[d(P,Q)\ge \|y\|=d(P^+,Q^+)\]
Also ist $d(P^+,Q^+)=d(A,B)$, falls $G$ existiert.\\
Schreibe:
\begin{align*}
&A=\sum_{i=1}^r \mathbb{R}\cdot x_i+x_0 &&B=\sum_{j=1}^s \mathbb{R}\cdot y_j +y_0
\end{align*}
Es gelten folgende notwendige Bedingungen für $P^+,Q^+$:	
\begin{enumerate}
\item $P^+=\sum_{i}\lambda_i x_i+x_0$ mit $\lambda_i\in\mathbb{R}$.
\item $Q^+=\sum_{j}\mu_i y_j+y_0$ mit $\mu_j\in\mathbb{R}$.
\item $\forall i\in\{1,\ldots,r\}\langle x_i,P^+-Q^+\rangle = 0$
\item $\forall j\in\{1,\ldots,s\}\langle y_j,P^+-Q^+\rangle = 0$
\end{enumerate}
Daraus erhalten wir ein LGS für die unbestimmten $\lambda_i,\mu_j$, dessen
Lösung $P^+$ und $Q^+$ ergibt.
Das LGS ist genau dann lösbar, wenn gilt:
\[\exists P^+-Q^+:\sum_i \lambda_ix_i+x_0-\sum_j \mu_jy_j+y_0\in (U_A+U_B)^\bot\]
Wegen $\mathbb{R}^n=(U_A+U_B)\oplus(U_A+U_B)^\bot$ ist sicher
$x_0-y_0\in\langle x_i,y_j\rangle+(U_A+U_B)^\bot$, also ist das LGS lösbar.\\
Nach Vorraussetzung existiert ein $z\ne 0$ mit $z\in(U_A+U_B)^\bot$\\
Nehme:
\begin{align*}
G:=
\begin{cases}
[P^+,Q^+] &,P^+\ne Q^+\\
\mathbb{R}\cdot z+P^+ &,P^+=Q^+
\end{cases}
\end{align*}
\end{proof}

\begin{comment}
Sei $\{b_1,\ldots,b_t\}$ ONB von $(U_A+U_B)^\bot$. Dann gilt mit $\beta_\tau=\langle x_0-y_0, b_\tau\rangle$:
\[P^+-Q^+=\sum_{\tau=1}^t \beta_\tau\cdot b_\tau\]
Dann erhalten wir zwei Methoden zur Abstandsbestimmung:
\begin{enumerate}
\item Löse das LGS in $\lambda_i,\mu_j$!
\item Bestimme eine ONB $\{b_1,\ldots,b_t\}$ von $(U_A+U_B)\bot$, dann gilt:
\[d(A,B)\left(=\|P^+-Q^+\|\right) =\sqrt{\sum_{\tau=1}^t \langle x_0-y_0,b_\tau\rangle^2}\]
Diese Methode kommt \textbf{ohne} Berechnung von $P^+,Q^+$ aus.
\end{enumerate}
\end{comment}

\section{Bewegungen im $\mathbb{R}^2$}
\begin{task}
Klasseneinteilung von $\Autdist(\mathbb{R}^2)$.
\end{task}

\begin{method}
Die folgende Methode funktioniert analog zu der bei Affinitäten.\\ 
$\vp=(A,a)$ (bzgl. Standardkoordinatensystem $\mathcal{K}=(O,B)$) wird in ein 
anderes Koordinatensystem $\mathcal{L}=(P,B)$ umgerechnet:
\[D_{\mathcal{L}\mathcal{L}}(\vp)=\left((M^{-1}AM),M^{-1}((A-I)b+a)\right)=:(A',b')\]
wobei $(M,b):=D_{\mathcal{K}\mathcal{L}}(\id)$ mit $M=M_{SB}$ den Wechsel \textbf{cartesischer}
Koordinatensysteme beschreibt, so da\ss \ \((A',b')\) einfache Gestalt erh\"alt 
(``Normalform'').\\
\(A'\) hat folgende Form:\\
\(A'=D_{\alpha}=
\begin{pmatrix}
\cos\alpha&&-\sin\alpha\\
\sin\alpha&&\cos\alpha
\end{pmatrix}\) (Drehung) oder \(A'=C:=
\begin{pmatrix}
1&&0\\
0&&-1
\end{pmatrix}\) (Spiegelung)
\begin{itemize}
\item Fall \(D_{\alpha}\) mit \(0<\alpha <2\pi\):\\
Es gilt: \(1\not\in\Spec(A) \implies \vp\) hat genau einen Fixpunkt 
\(P\,\Longleftrightarrow\,(A-I)P+a=0\) wobei \((A-I)\) invertierbar ist.

W\"ahle Koordinatensystem \(\mathcal{L}:=(P,B)\,\rightarrow\,(A',b')=(D_{\alpha},0)\)
\item Fall \(A'=I\):\\
Sei \(\vp=(I,a)\) eine Translation, \(a\neq 0\).\\
W\"ahle \(\mathcal{L}:=(0,(b_{1},b_{2}))\) mit \(b_{1}:=\frac{a}{||a||}\). Dann
gilt:
\[
M_{SB} = (b_{1},b_{2}),\quad M_{SB}^{-1}(b_{1},b_{2})=(e_{1},e_{2})
\]
also \(b'=M_{SB}^{-1}a=\lambda_{e_{1}}\).\\
Dann ist \(D_{\mathcal{L}\mathcal{L}}(\vp)=(I,\lambda_{e_{1}})\) mit 
\(\lambda:=||a||>0\).
\item Fall \(A'=C\): analog
\end{itemize}
\end{method}

\begin{theo}
\index{Normalform}\index{Drehung}\index{Translation}\index{Spiegelung}
Zu \(\vp\in\Autdist(\mathbb{R}^{2})\) existiert ein cartesisches 
Koordinatensystem \(\mathcal{L}\) so, da\ss \ \(D_{\mathcal{L}\mathcal{L}}(\vp)\)
eine der folgenden Normalformen hat:
\begin{enumerate}
\item\((I,0) = \id\)
\item\((I,\lambda_{e_{1}})\quad\) Translation \((\lambda > 0)\), keine Fixpunkte
\item\((D_{\alpha})\quad\) Drehungen \((0<\alpha<2\pi)\), genau ein Fixpunkt 
\(O\).
\item\((C,0)\quad\)Spiegelung an einer Achse, die Achse ist die Menge der 
Fixpunkte
\item\((C,\lambda_{e_{1}})\quad\) Gleitspiegelung, kein Fixpunkt, genau eine
Fixgerade
\end{enumerate}
Eigentliche Bewegungen sind die Identit\"at, Translationen und Drehungen.
\end{theo}

\section{Geometrische Kennzeichnung von Bewegungen}
\index{Bewegung}

Betrachte zun\"achst generell eine l\"angentreue Abbildung 
\(\Psi:\mathbb{R}^{n}\to\mathbb{R}^{m}\) (nicht notwendig affin).
\begin{lemma}
Zu \(\lambda\in\mathbb{R},\,P\neq Q\in\mathbb{R}^{n}\) existiert genau ein
Punkt \(R\in\mathbb{R}^{n}\) mit
\begin{align*}
d(P,R)&=|\lambda|\cdot d(P,Q)\\
d(Q,R)&=|1-\lambda|\cdot d(P,Q)
\end{align*}
n\"amlich \(R:=\lambda y+P\) f\"ur \(y:=\overrightarrow{PQ}\).
\end{lemma}
\begin{proof}
\begin{align*}
d(P,R)&=||\lambda y|| = |\lambda|||y||=|\lambda|\cdot d(P,Q)\\
d(Q,R)&=||y+P-(\lambda y + P)||=|1-\lambda|||y||=|1-\lambda|\cdot d(P,Q)
\end{align*}

Sei \(S\) ein weiterer Punkt mit \(d(P,S)=d(P,R),\,d(Q,S)=d(Q,R)\). Etwa
\(S=x+R\).
Ohne Beschr\"ankung der Allgemeinheit sei \(P=0\) (nach Koordinatenwechsel),
also
\[
Q=y,\,R=\lambda y,\,S=x+\lambda y
\]
\(\implies ||R||=d(0,R)=d(0,S)=||S||\), also 
\[
\langle\lambda y,\lambda y\rangle = \langle x+\lambda y,x+\lambda y\rangle
\implies\langle x,x\rangle+2\lambda\langle x,y\rangle =0
\]
und
\[
||Q-R||=||Q-S||\implies ||y-\lambda y||=||y-\lambda y-x|| 
\overset{\text{analog}}{\implies}
\langle x,x\rangle+(2\lambda -2)\langle x,y\rangle =0
\]
Insgesamt: \(\langle x,y\rangle =0,\,\langle x,x\rangle=0\), also \(x=0\), d.h.
\(R=S\).
\end{proof}

\begin{corollary}
Ist \(\Psi:\mathbb{R}^{n}\to\mathbb{R}^{m}\) l\"angentreu, so gilt f\"ur alle
\(P,\,Q\in\mathbb{R}^{n},\,\lambda\in\mathbb{R}\):
\[
\Psi(\lambda\cdot\overrightarrow{PQ}+P)=\lambda\overrightarrow{\Psi(P)\Psi(Q)}+\Psi(P)
\]
Insbesondere ist \(\Psi\) geradentreu f\"ur \(n=m\).
\end{corollary}
\begin{proof}
Klar f\"ur \(P=Q\).\\
Sei nun \(y:=\overrightarrow{PQ}\neq 0,\,R:=\lambda y+P\). Da \(\Psi\) 
l\"angentreu, folgt nach Lemma
\begin{align*}
d(\Psi(P),\Psi(R))&=|\lambda|\cdot d(\Psi(P),\Psi(Q))\\
d(\Psi(Q),\Psi(R))&=|1-\lambda|d(\Psi(P),\Psi(Q))
\end{align*}
Lemma anwenden auf die Bildpunkte \(P':=\Psi(P),\,Q':=\Psi(Q),\,R':=\Psi(R)\)
liefert \(R'=\lambda\overrightarrow{P'Q'}+P'\).
\end{proof}

\begin{corollary}
\(\Psi(\mathbb{R}^{n})\) ist affiner Teilraum von \(\mathbb{R}^{n}\).
\end{corollary}
\begin{proof}
Nach dem vorhergehenden Korollar gilt f\"ur beliebige Punkte 
\(P',\,Q'\in\Psi(\mathbb{R}^{n})\), da\ss \ die Verbindungsgerade
\([P',Q']\subseteq\Psi(\mathbb{R}^{n})\).
Mit dem Teilraumkriterium folgt die Behauptung.
\end{proof}

\begin{corollary}
Sei \(n=m,\,\Psi:\mathbb{R}^{n}\to\mathbb{R}^{m}\) l\"angentreu, 
\(B={b_{1},\ldots,b_{n}}\) Orthonormalbasis und \(\Psi(0)=0\). Dann ist auch
\(\Psi(B)\) eine Orthonormalbasis.
\end{corollary}
\begin{proof}
\(n=1\): Klar.\\
Sei \(n>1\). Betrachte Abst\"ande \(d(\mathbb{R}\cdot b_{i},{b_{j}})\) f\"ur
\(i\neq j\).\\
\(\implies 0=\Psi(0)\in\Psi(\mathbb{R}\cdot b_{i})=[0,\Psi(b_{i})]\) hat 
minimalen Abstand von \(\Psi(b_{j})\).\\
Lotgerade \(G=[0,\Psi(b_{j})]\bot[0,\Psi(b_{i})]\), also \(\Psi(b_{i})\bot\Psi(b_{j})\)\\
Ferner ist: \(||\Psi(b_{i}||=d(0,\Psi(b_{i}))=d(0,b_{i})=||b_{i}||=1\).
Also ist \(\Psi(B)\) eine Orthonormalbasis.
\end{proof}

\begin{corollary}
\(\Psi:\mathbb{R}^{n}\to\mathbb{R}^{m}\) l\"angentreu \(\implies\,\Psi\) ist 
bijektiv.
\end{corollary}
\begin{proof}
Ohne Beschr\"ankung der Allgemeinheit sei \(\Psi(0)=0\), also 
\(\Psi(\mathbb{R}^{n})\) Untervektorraum von \(\mathbb{R}^{n}\) mit einer 
Orthonormalbasis von \(\mathbb{R}^{n}\,\implies\,\Psi(\mathbb{R}^{n})=\mathbb{R}^{n}\).\\
Injektiv: \(\Psi(P)=\Psi(Q)\)
\[
\implies 0=d(\Psi(P),\Psi(Q))=d(P,Q)\quad\implies P=Q
\]
\end{proof}

\begin{theo}
Jede l\"angentreue Abbildung \(\Psi:E\to E\) eines euklidischen Raumes \(E\) ist
eine Bewegung (also \(\Psi\in\Autaff(E)\)).
\end{theo}
\begin{proof}
Die Wahl eines cartesischen Koordinatensystems erlaubt ohne Beschr\"ankung der
Allgemeinheit \(E=\mathbb{R}^{n}\) zu nehmen.\\
Wechsel zu \(\Psi':=(x\mapsto\Psi(x)-\Psi(0))\) ergibt \(\Psi'(0)=0\).

Beachte: \(\Psi\) ist affin (bzw. l\"angentreu) genau dann, wenn \(\Psi'\) affin
(bzw. l\"angentreu) ist.

Also sei ohne Einschr\"ankung \(\Psi(0)=0\). Restbehauptung: \(\Psi\) ist eine
lineare Abbildung.\\
\(n=1\): Klar.\\
\(n>1\): \(\Psi\) ist geradentreu nach dem ersten Korollar, also 
% Querverweis von LaTeX erledigen lassen?
\(\mathbb{Q}\)-linear (nach 21.4), insbesondere additiv. 
\[
\lambda\in\mathbb{R}:\,\Psi(\lambda x)=\lambda\cdot\overrightarrow{\Psi(0)\Psi(x)}+\underbrace{\Psi(0)}_{=0}=\lambda\Psi(x)
\]
\end{proof}

\chapter{Analytische Geometrie}
\section{Quadriken}
\index{Quadrik}\index{Multiindex}

\(K\) sei ein K\"orper, \(f\in K[X_{1},\ldots,X_{n}]\)
\[
f(X_{1},\ldots,X_{n})=\sum_{i_{1},\ldots,i_{n}\in\mathbb{N}}{\alpha_{i_{1},\ldots,i_{n}}\underbrace{X_{1}^{i_{1}}\ldots X_{n}^{i_{n}}}_{=:\underline{X}^{\underline{i}}}}=
\sum_{\underline{i}\in\mathbb{N}^{n}}{\alpha_{\underline{i}}\cdot\underline{X}^{\underline{i}}}
\]

\(\underline{i}\) hei\ss t \textbf{Multiindex}.
\begin{definition}
F\"ur \(\underline{i}\in\mathbb{N}^{n}\) sei \(|\underline{i}|:=i_{1}+\ldots+i_{n}\) der \textbf{Grad} von \(\underline{X}^{\underline{i}}\).\\
Der \textbf{(Gesamt-)Grad} von \(f\) ist \(\deg(f):=\max\{|\underline{i}|:\alpha_{\underline{i}}\neq0\}\).
\end{definition}

Wunsch: Beschreibe die Nullstellenmenge
\(\mathcal{N}(f):=\{x=(x_{1},\dots,x_{n}\in K^{n}\mid f(x_{1},\dots,x_{n})=0\}\)
f\"ur ein Polynom \(f\) mit \(\deg(f)=2\).

\begin{comment}
Den Fall eines oder mehrerer Polynome vom Grad 1 erledigt die lineare Algebra.
Mehrere Polynome vom Grad \(\geq 2\) behandelt die \textbf{Kommutative Algebra 
und algebraische Geometrie}.
\end{comment}

\begin{prepwork}
Klassifiziere die Menge \(\mathcal{N}(f)\) durch
Klassifizierung der Polynome. (Erinnerung: Klasseneinteilung entspricht einer
\"Aquivalenzrelation)

Dazu sei \(G\leq\Autaff(K^{n})\) (Untergruppe).

Definiere hiermit eine \"Aquivalenzrelation auf Polynomen bzw. auf Teilmengen
\(T\subseteq K^{n}\).
\begin{align*}
f_{1}\approx_{G}f_{2}&:\Longleftrightarrow\exists\mu\in K^{\times}\exists\varphi
\in G:\,f_{2}=\mu\cdot (f_{1}\circ\phi)\\
M_{1}\sim_{G}M_{2}&:\Longleftrightarrow\exists\varphi\in G:\,M_{2}=\varphi(M_{1})
\end{align*}
Klar: \(f_{1}\approx_{G}f_{2}\implies\mathcal{N}(f_{1})\sim_{G}\mathcal{N}(f_{2})\)
\end{prepwork}

\begin{ziel}
Klassifiziere die Polynome f\"ur spezielle \(G\).
\begin{itemize}
\item Affine Klassifikation (f\"ur Char(K)\(\neq 2\)): 
\[
G=\Autaff(K^{n})=\{\varphi=(A,b)\mid A\in\GL_{n}(K),b\in K^{n}\}
\]
\item Euklidische Klassifikation: 
\[
G=\Autaff(\mathbb{R}^{n})=\{(A,b)\mid A\in O_{n},b\in\mathbb{R}^{n}\}
\]
\end{itemize}
\end{ziel}

Sei nun Char(K)\(\neq 2\).

\begin{prep}
Jedes Polynom \(f\) mit \(\deg(f)=2\) hat die Form
\[
f(X_{1},\dots,X_{n})=\sum_{i,j=1}^{n}{\alpha_{ij}X_{i}X_{j}}+2\sum_{i=1}^{n}{\beta_{i}X_{i}}+\gamma
\]
mit einer symmetrischen Matrix \(A=(\alpha_{ij})\neq 0,\,b=\begin{pmatrix}\beta_{1}\\\vdots\\\beta_{n}\end{pmatrix}\in K^{n},\,y\in K\).
\end{prep}

\begin{proof}(Symmetrie von A)\\
Falls \(A\) nicht symmetrisch ist, ersetze \(A\) durch 
\[
\frac{1}{2}\left(A+A^{\top}\right)=:A'
\]
\end{proof}

Beachte: F\"ur \(x=\begin{pmatrix} x_{1}\\\vdots\\ x_{n}\end{pmatrix}\in K^{n}\)
gilt
\[
f\left(x^{\top}\right)=x^{\top}Ax+2b^{\top}x+\gamma
\]
mit \(A^{\top}=A\).

\(Q:=\mathcal{N}(f)\) hei\ss t \textbf{affine Quadrik}.
\begin{lemma}
F\"ur \(f\) wie oben und \(\varphi=(C,d)\in\Autaff(K^{n})\) sei 
\(g(y):=(f\circ\varphi)(y)\). Dann ist
\[
g(y)=y^{\top}A'y+2b'^{\top}y+\gamma'
\]
wobei \(A':=C^{\top}AC,\,b':=C^{\top}(Ad+b),\,\gamma':=f(d)\).
\end{lemma}
\begin{proof}
\begin{align*}
f\left(\varphi(y)\right)&=f(Cy+d)\\
&=\underbrace{(Cy+d)^{\top}}_{y^{\top}C^{\top}+d^{\top}}A(Cy+d)+2b^{\top}(Cy+d)+\gamma\\
&=y^{\top}\underbrace{C^{\top}AC}_{=:A'}y+d^{\top}ACy+
    \underbrace{\overbrace{y^{\top}C^{\top}Ad}}^{\in K^{1\times 1}}_{=d^{\top}A^{\top}Cy = d^{\top}ACy}
    +2b^{\top}Cy+\underbrace{d^{\top}Ad+2b^{\top}d+\gamma}_{=:\gamma'}\\
&=y^{\top}A'y+2b'^{\top}y+\gamma'
\end{align*}
\end{proof}

\textbf{Prinzip der Klassifikation:} Zu gegebenem \(f\) finde \((C,d)\), so 
da\ss \ \(A',\,b",\,\gamma'\) eine einfache, \"ubersichtliche ``Normalform''
annehmen.

\begin{comment}
\(\vp\) bewirkt Wechsel des Koordinatensystems \(y=\vp^{-1}(x)=D_{\mathcal{L}}(x)\).
\(y\) beschreibt \(Q=\mathcal{N}(f)\) im Koordinatensystem \(\mathcal{L}\).
\end{comment}
\begin{theo}[Satz von der quadratischen Erg\"anzung]
Sei \(A\in K^{n\times n}\) symmetrisch vom Rang \(r\) und 
\(\text{Char}(K)\neq 2\). Dann existiert \(C\in\GL_{n}(K)\) so da\ss \ 
\[
C^{\top}AC=\diag(\alpha_{1},\ldots,\alpha_{r},0,\ldots,0)
\]
\end{theo}
\begin{proof}
Sei \(A=(\alpha_{ij})\) mit \(\alpha_{ij}=\alpha_{ji}\) f\"ur alle \(i,j\).\\
Nutze eine Variante des Gau\ss algorithmus; es gen\"ugt Diagonalgestalt zu 
erreichen, den Rest erledigen Vertauschungsmatrizen \(V_{ij}\).

\(\nu\)-ter Schritt: Angenommen die Zeilen (Spalten) mit einer Nummer kleiner
\(\nu\) haben die gew\"unschte Form. Dann ist:
\[
A_{\nu}:=
    \begin{pmatrix}
    *& & &0& & \\
     &\ddots& & &\ddots& \\
     & &*& & &0\\
    0& & &*&\cdots&* \\
     &\ddots& &\vdots&\ddots&\vdots\\
     & &0&*&\cdots&*\\
    \end{pmatrix}
\]
Unterscheide folgende F\"alle:
\begin{enumerate}
\item \(\alpha_{\nu\nu}\neq 0\):\\
Mache Zeilenumformung: Subtrahiere Vielfaches
der \(\nu\)-ten Zeile von der unteren Zeile, so da\ss \ dort die \(\nu\)-te 
Spalte Null wird.\\
Dies geht mit
\[
C':=I-\sum_{k=\nu+1}^{n}{\frac{\alpha_{k\nu}}{\alpha_{\nu\nu}}E_{k\nu}}:\quad
C'A_{\nu}=(\beta_{ij})
\]
mit \(\beta_{\nu+1,\nu}=\ldots =\beta_{n}=0\). Die \(\nu\)-te Zeile ist 
\(\beta_{\nu j}=\alpha_{\nu j}\left(=\alpha_{j\nu}\right)\).
\item \(\alpha_{\nu\nu}=0,\,\alpha_{kk}\neq 0\) f\"ur ein \(k>\nu\):\\
Bilde \(A'_{\nu}:=V_{\nu k}A_{\nu}V_{\nu k}\), dann weiter wie in Fall 1.
\item \(\alpha_{kk}=0\) f\"ur \(k=\nu,\ldots,n\):\\
Ist \(\alpha_{k\nu}=0\,\forall k\geq\nu\), so ist bereits \(A_{\nu}\) diagonal
bis Zeile \(\nu\).\\
Sonst sei \(\beta:=\alpha_{k\nu}\neq0\) f\"ur ein \(k>\nu\)
(addiere die \(k\)-te Zeile zur \(\nu\)-ten).
Nutze die Additionsmatrix \(T:=A_{\nu k}(1)=I+E_{\nu k}\) mit
\(T^{\top}=I+E_{k\nu}\).\\
Dann folgt \(A'_{\nu}:=TA_{\nu}T^{\top}\) hat \(\alpha'_{\nu\nu}=2\beta\neq0\)
(da Char\((K)\neq2\)). Fahre fort wie in Fall 1.
\end{enumerate}
\end{proof}
Vorsicht: Die \(\alpha_{1},\ldots,\alpha_{n}\) sind im allgemeinen nicht eindeutig.
\begin{theo}[Tr\"agheitssatz von Sylvester]
\index{Tr\"agheitssatz von Sylvester}
Sei \(A\in K^{n\times n}\) symmetrisch vom Rang \(r\).
\begin{enumerate}
\item F\"ur \(K=\mathbb{C}\) existiert \(C\in\GL_{n}(\mathbb{C})\) mit
\[
C^{\top}AC=\diag(\underbrace{1,\ldots,1}_{r},0,\ldots,0)
\]
\item F\"ur \(K=\mathbb{R}\) existiert \(C\in\GL_{n}(\mathbb{R})\) mit
\[
C^{\top}AC=\diag(\underbrace{1,\ldots,1}_{p},\underbrace{-1,\ldots,-1}_{q},0,\ldots,0)
\]
wobei \(p,q\) durch \(A\) eindeutig bestimmt sind.
\end{enumerate}
\end{theo}
\begin{proof}
\begin{enumerate}
\item Nach dem vorherigen Satz mit \(D:=\diag(\beta_{1},\ldots,\beta_{r},0,\ldots,0)\).\\
Weiteres umformen liefert:
\[
D^{\top}\cdot\diag(\alpha_{1},\ldots,\alpha_{r},0,\ldots,0)\cdot D=
\diag(\beta_{1}^{2}\alpha_{1},\ldots,\beta_{r}^{2}\alpha_{r},0,\ldots,0)
\]
Falls \(\beta_{i}\) Nullstelle von \(X^{2}-\frac{1}{\alpha_{i}}\) ist, existiert
in \(\mathbb{C}\) immer in eine Diagonalmatrix \(\diag(1,\ldots,1,0,\ldots,0)\).
\item \"Ahnlich f\"ur \(\mathbb{R}\). Vorzeichen ber\"ucksichtigen

Restbehauptung: \(p\) ist eindeutig bestimmt durch \(A\)\\
Behauptung: \(p=\max\left\{\dim U\mid U\leq\mathbb{R}^{n}\text{ mit } \left.
s_{A}\right|_{U}\text{ positiv definit}\right\}\) wobei 
\(s_{A}(x,y):=\langle Ax,y\rangle\).\\
Sei anderes \(C'\) mit zugeh\"origem \(p',\,q'\). Setze
\begin{align*}
b_{i}&:=Ce_{i}\\
b'_{i}&:=C'e_{i}\\
U&:=\langle b_{1},\ldots,b_{p}\rangle\\
U'&:=\langle b'_{1},\ldots,b'_{p'}\rangle\\
V&:=\langle b_{p+1},\ldots,b_{p+q}\rangle\\
V'&:=\langle b'_{p'+1},\ldots,b'_{p'+q'}\rangle\\
\end{align*}
Dann sind \(\left.s_{A}\right|_{U},\,\left.s_{A}\right|_{U'},\,\left.-s_{A}\right|_{V},\,\left.-s_{A}\right|_{V'}\) positiv definit.

F\"ur \(W:=\Kern(\Lambda_{A})=\langle b_{r+1},\ldots,b_{n}\rangle =\langle b'_{r+1},\ldots,b'_{n}\rangle\) gilt
\[
\mathbb{R}^{n}=U\oplus V\oplus W=U'\oplus V'\oplus W
\]
Damit folgt
\[
U\cap(V'\oplus W)=0,
\]
denn 
\begin{align*}
\forall x\in U\cap(V'\oplus W): s_{A}(x,x)\geq 0,s_{A}(x,x)\leq 0&\implies
s_{A}(x,x)=0\\
&\implies x=0
\end{align*}
Damit folgt \(p=\dim U\leq\dim U'=p'\), da 
\begin{align*}
\dim U'+\dim(V'+W)&=n\\
&\geq\dim(U+V'+W)\\
&=\dim(U\oplus V'+W)\\
&=\dim U + \dim (V'+W)
\end{align*}
Aus Symmetriegr\"unden folgt \(p=p'\).
\end{enumerate}
\end{proof}

\paragraph{Fortsetzung der Polynomklassifikation (\(\deg=2\)) \"uber beliebigem
K\"orper \(K\):}{Aus obigem Lemma und Satz folgt: Der quadratische Anteil der
Polynome \(f(x)=x^{\top}Ax+2b^{\top}x+\gamma\) l\"asst sich durch eine geeignete
affine Abbildung \(\vp=(C,d)\) auf folgende einfache Gestalt bringen:
\[
\alpha_{1}x_{1}^{2}+\alpha_{2}x_{2}^{2}+\ldots+\alpha_{n}x_{n}^{2}
\]
\begin{note}
Ab\"andern von \(C\), etwa 
\(C_{1}=\diag(\underbrace{1,\ldots,1}_{r},B)\) mit \(B\in\GL_{n-r}(K)\),
\"andert den quadratischen Anteil \textbf{nicht}.
\end{note}}
\paragraph{N\"achste Vereinfachung:}{linearer Term \(2b'^{\top}y\)\\
Kann eventuell \(2b'^{\top}y=0\) erreicht werden?
\[
b'\overset{\text{Def.}}{=}C^{\top}(Ad+b)=0\Longleftrightarrow Ad+b=0
\]
Das hei\ss t das LGS \(Az=-b\) hat die L\"osung \(z=d\).}
\begin{definition}
\index{Mittelpunkt}\index{Quadrik!Mittelpunkt der}
Falls eine L\"osung \(d\) existiert, so hei\ss t \(d\) \textbf{Mittelpunkt} der
Quadrik.
\end{definition}
\begin{note}
\[
y=d+t\in\mathcal{N}(f)\implies\,d-t\in\mathcal{N}(f)
\]
\end{note}
\begin{proof}
\(f(d+t)=0\), das hei\ss t:
\begin{align*}
&(d+t)^{\top}A(d+t)+2b^{\top}(d+t)+\gamma=0\\
&(d+t)^{\top}A(d+t)+2(-Ad)^{\top}(d+t)+\gamma=0\\
&(d+t)^{\top}A(d+t)-2d^{\top}A(d+t)+\gamma=0\\
\Longleftrightarrow&(d-t)^{\top}A(d+t)+\gamma=0
\end{align*}
\begin{align*}
f(d-t)&=(d-t)^{\top}A(d-t)+2(-Ad)^{\top}(d-t)+\gamma\\
&=-(d-t)^{\top}A(d+t)+\gamma\\
&=0
\end{align*}
\end{proof}
\subsubsection{Affine Klassifikation der Quadriken mit Mittelpunkt}
\begin{enumerate}
\item Fall \(K=\mathbb{C}\):
    \begin{enumerate}
    \item\(f=X_{1}^{2}+\ldots+X_{r}^{2}\quad(\gamma'=0)\)
    \item\(f=X_{1}^{2}+\ldots+X_{r}^{2}+1\quad(\gamma'\neq0)\)
    \end{enumerate}
\item Fall \(K=\mathbb{R}\):
    \begin{enumerate}
    \item\(f=X_{1}^{2}+\ldots+X_{p}^{2}-X_{p+1}^{2}-\ldots-X_{p+q}^{2}\quad
	(\text{ohne Einschr\"ankung }p\geq q)\)
    \item\(f=X_{1}^{2}+\ldots+X_{p}^{2}-X_{p+1}^{2}-\ldots-X_{p+q}^{2}+1\quad
	(\gamma'\neq0)\)
    \end{enumerate}
\end{enumerate}
\begin{example}
\(n=r=2\)
\begin{enumerate}
\item Ellipse: \(f=X_{1}^{2}+X_{2}^{2}-1\)
\item Hyperbel: \(f=X_{1}^{2}-X_{2}^{2}+1\)
\end{enumerate}
\end{example}
\subsubsection{Affine Klassifikation der Quadriken ohne Mittelpunkt}
Jetzt sei \(Az=-b\) unl\"osbar. Es ist aber auch f\"ur \(A\) Diagonalform 
erreichbar: \(A=\diag(\alpha_{1},\ldots,\alpha_{r},0,\ldots,0)\), wobei 
\(r=\rank(A)\) ist.\\
Daraus folgt: es existiert ein \(d\) mit
\[
Ad+b=:c\in\langle e_{r+1},\ldots,e_{n}\rangle\quad(c\neq0)
\]
Nun w\"ahle \(C_{1}=\diag(\underbrace{1,\ldots,1}_{r},B)\), so da\ss \ 
\(C_{1}^{\top}c=-e_{r+1}\).
\[
C_{1}^{\top}c=C_{1}^{\top}(Ad+b)=:b'
\]
\begin{note}
\(c\) bleibt unver\"andert wenn \(d\) durch \(d+y\) mit 
\(y\in\langle e_{r+1},\ldots,e_{n}=\Kern(\Lambda_{A})\) ersetzt wird.
\end{note}
Somit: 
\(f=\alpha_{1}X_{1}^{2}+\ldots+\alpha_{r}X_{r}^{2}
    \underbrace{-2X_{r+1}}_{2b'^{\top}X}+\gamma'\)

Schlie\ss lich: affine Transformation 
\(\vp=\left(I,\frac{1}{2}\gamma'e_{r+1}\right)\) f\"uhrt zu 
\begin{align*}
f&=\alpha_{1}X_{1}^{2}+\ldots+\alpha_{r}X_{r}^{2}-2\left(X_{r+1}+\frac{1}{2}\gamma'\right)+\gamma'\\
&=\alpha_{1}X_{1}^{2}+\ldots+\alpha_{r}X_{r}^{2}-2X_{r=1}
\end{align*}
\begin{enumerate}
\item Fall \(K=\mathbb{C}\): \(f=X_{1}^{2}+\ldots+X_{r}^{2}-2X_{r+1}\quad(\text{f\"ur }r<n)\)
\item Fall \(K=\mathbb{R}\): \(f=X_{1}^{2}+\ldots+X_{p}^{2}-X_{p+1}^{2}-\ldots-X_{p+q}^{2}-2X_{r+1}\quad(p\geq q)\)
\end{enumerate}
\subsubsection{Euklidische Klassifikation}
(\(\vp=(x\mapsto Cx+d)\) mit \(C\in O_{n}\) sei zugelassen)\\
Zur Diagonalisierung von \(A\) verwende den Spektralsatz. Das hei\ss t, es
existiert \(C\in O_{n}:\,C^{\top}AC=\diag(\lambda_{1},\ldots,\lambda_{r},0,\ldots,0)\), 
wobei ohne Beschr\"ankung der Allgemeinheit 
\(\lambda_{1}\geq\ldots\geq\lambda_{r}\) sei.

Der Rest ist wie oben. Damit erhalten wir folgende Normalformen:
\begin{enumerate}
\item\(\lambda_{1}X_{1}^{2}+\ldots+\lambda_{r}X_{r}^{2}\quad\)(bis auf einen
gemeinsamen Faktor \(\mu\neq0\) eindeutig)
\item\(\lambda_{1}X_{1}^{2}+\ldots+\lambda_{r}X_{r}^{2}+1\)
\item\(\lambda_{1}X_{1}^{2}+\ldots+\lambda_{r}X_{r}^{2}-2X_{r+1}\)
\end{enumerate}
\begin{definition}
Die Zahlen \(\lvert\lambda_{i}\rvert^{-\frac{1}{2}}\) hei\ss en 
\index{Halbachse}\index{Halbachsenl\"ange}
\textbf{Halbachsenl\"angen}, die Geraden \(\langle e_{i}\rangle\,(i=1,\ldots,r)\)
\index{Hauptachse}
hei\ss en die \textbf{Hauptachsen} der Quadrik in Normalform.
\index{Hauptachsentransformation}
Der \"Ubergang in Normalform hei\ss t auch \textbf{Hauptachsentransformation}.
\end{definition}

\section{Der Tangentialraum}
\index{Tangentialraum}
Sei \(K\) ein beliebiger K\"orper.
\begin{definition}
\index{Hyperfl\"ache}
Die Nullstellenmenge \(\mathcal{F}=\mathcal{N}(f)\) eines Polynoms
\(f\in K[X_{1},\ldots,X_{n}]\) hei\ss t \textbf{Hyperfl\"ache}.
\end{definition}
Im folgenden sei stets \(\mathcal{N}(f)\neq\varnothing\). Es sei 
\(P\in\mathcal{F}\subseteq K^{n}\) ein Punkt auf der Hyperfl\"ache.\\
Betrachte die Geraden \(G=P+\langle u\rangle\) mit  \(u\in K^{n}\).
\[
Q\in G:\,Q=P+\tau u\quad\text{mit }u\in K^{n}
\]
\begin{note}
\(\mathcal{P}\in\mathcal{F}\) impliziert \(\left.T\right|f(P+Tu)\in K[T]\) (da
Nullstelle bei \(T=0\)).
\end{note}
\begin{definition}
\index{Tangente}
Eine Gerade hei\ss t \textbf{Tangente} an \(\mathcal{F}\) in \(P\), falls
\(\left.T^{2}\right|f(P+Tu)\) gilt.
\end{definition}
\begin{ziel}
Bestimme alle Tangenten durch \(P\) (d.h. \(u\) variiert).
\end{ziel}
Dazu schreibe:
\[
f(P+Tu)=\alpha_{0}+\alpha_{1}T+\alpha_{2}T^{2}+\ldots
    \quad\text{mit }\alpha_{i}=\alpha_{i}(u)
\]
Es gilt: \(\alpha_{1}\in\Hom(K^{n},K)\). Daraus folgt: es existiert ein Vektor
\(J_{p}(f):=J_{p}\in K^{1\times n}\) mit \(\alpha_{1}(u)=J_{p}\cdot u\).
\index{Jacobi-Matrix}
\(J_{p}\) hei\ss t \textbf{Jacobi-Matrix}.

In Analogie zur Analysis schreibe 
\[
J_{p}=:\left.\frac{\partial f}{\partial x}\right|_{x=p}
\]
Es gilt
\[
J_{p}=\left.\left(\frac{\partial f}{\partial X_{1}},\ldots,\frac{\partial f}{\partial X_{n}}\right)\right|_{x=p}
\]
Das hei\ss t: 
\[
G=P+\langle u\rangle\text{ ist Tangente}\Longleftrightarrow J_{p}u=0
    \Longleftrightarrow u\in J_{p}^{\bot}
\]
\begin{definition}
\begin{enumerate}
\item \index{regul\"ar} \(P\in\mathcal{F}\) hei\ss t \textbf{regul\"ar}, falls
    \(J_{p}\neq0\).\\
\index{Tangentialraum}\index{Hyperebene}
Die Hyperebene \(T_{p}(\mathcal{F}):=P+J_{p}^{\bot}\) hei\ss t 
\textbf{Tangentialraum}.
\item \index{singul\"ar}\index{Singularit\"at}Sonst hei\ss t \(P\) \textbf{singul\"ar} (oder \textbf{Singularit\"at}).
\end{enumerate}
\end{definition}

\renewcommand{\indexname}{Stichwortverzeichnis}
\printindex
\end{document}
