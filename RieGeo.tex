\documentclass[a4paper,twoside,DIV15,BCOR12mm]{scrbook}
\usepackage{mathe}

\newcommand{\otm}{\stackrel{\circ}{\subset}} % Offene Teilmenge
\newcommand{\At}{\mathcal A}


\author{Die Mitarbeiter von \url{http://mitschriebwiki.nomeata.de/}}
\title{Riemann’sche Geometrie}
\makeindex

\begin{document}
\maketitle
 
\renewcommand{\thechapter}{\Roman{chapter}}
%\chapter{Inhaltsverzeichnis}
\stepcounter{chapter}
%\renewcommand{\tocname}{bla}
\addcontentsline{toc}{chapter}{\protect\numberline {\thechapter}Inhaltsverzeichnis}
\tableofcontents

 % Vorwort

\chapter{Vorwort}
%\addcontentsline{toc}{chapter}{Vorwort}

\section*{Über dieses Skriptum}
Dies ist ein Mitschrieb der Vorlesung \glqq Riemann’sche Geometrie\grqq\ von Herrn Dr. Leuzinger im
Sommersemester 07 an der Universität Karlsruhe (TH). Die Mitschriebe der Vorlesung werden mit
ausdrücklicher Genehmigung von Herrn Dr. Leuzinger hier veröffentlicht, Herr Dr. Leuzinger ist für  den
Inhalt nicht verantwortlich.

\section*{Wer}
Gestartet wurde das Projekt von Joachim Breitner. Bei der Erstellung wurde auf ein Skript für die englische Version der Vorlesung von Bernhard Konrad zurückgegriffen.


\section*{Wo}
Alle Kapitel inklusive \LaTeX-Quellen können unter \url{http://mitschriebwiki.nomeata.de} abgerufen werden.
Dort ist ein von Joachim Breitner programmiertes \emph{Wiki}, basierend auf \url{http://latexki.nomeata.de} installiert. 
Das heißt, jeder kann Fehler nachbessern und sich an der Entwicklung
beteiligen. Auf Wunsch ist auch ein Zugang über \emph{Subversion} möglich.


\setcounter{chapter}{0}
\renewcommand{\thechapter}{\arabic{chapter}}
\chapter{Mannigfaltigkeiten}

\section{Differenzierbare Mannigfaltigkeiten}

\subsection*{Erinnerung (LA/Analysis)}
%\begin{center}
\begin{tabular}{rl}
Euklidischer Raum & $\MdR^n, \langle\cdot,\cdot\rangle$ \\
Norm & $ \|a\| := \sqrt{\langle a, a\rangle} $ \\
Metrik & $d(a,b) := \| a-b\| $ \\
Winkel & $\cos \angle(a,b) := \frac{\langle a, b\rangle}{\|a\|\cdot\|b\|} $ 
\end{tabular}
%\end{center}

Die Funktion $f:U (\otm \MdR^n) \to \MdR$\footnote{$A\otm B := $ $A$ offen und $A\subset B$} ist \index{Glatte Funktion}glatt (oder \index{$C^\infty$}$C^\infty$) falls in jedem Punkt $p\in U$ alle gemischten partiellen Ableitungen existieren und stetig sind.

Die $C^\infty$-Funktion
\[ u^i:
\begin{aligned}
\MdR^n &\to \MdR \\
p = (p_1,\ldots,p_n) &\mapsto p_i =u^i(p)
\end{aligned}
\]
heißt $i$-te Koordinatenfunktion ($i=1,\ldots,n$). Eine Abbildung $\phi: U(\otm \MdR^n)\to \MdR^n$ heißt glatt falls jede der reellen Funktionen $u^i\circ\phi$ glatt ist ($i=1,\ldots,n$).

\subsubsection*{Karten und Atlanten}
Sei $M$ in topologischer Raum, der hausdorff’sch ist und eine abzählbare Basis hat.

\index{Koordinatensytem}
\index{Karte}
Ein Koordinatensystem (oder Karte) in $M$ ist ein Homöomorphismus
\[\varphi: U(\otm M) \to \varphi(U) (\otm\MdR^n) \]
Schreibt man $\varphi(p) = (x^i(p),\ldots,x^n(p))$, dann heißen die Funkionen $x^i := u^i\circ \varphi : U \to \MdR$ Koordinatenfunktionen von $\varphi$. $n$ heißt Dimension von $(\varphi,U)$.

\index{Atlas}
Ein $n$-dimensionaler, differenzierbarer Atlas für $M$ ist eine Kollektion $\At$ von $n$-dimensionalen Karten von $\M$. Es gilt:
\begin{itemize}
\item[$(\At1)$] Jeder Punkt von $M$ liegt im Definitionsbereich mindestens einer Karte, d.h. $M$ ist lokal euklidisch.
\item[$(\At2)$] Alle zu $\At$ gehörigen Kartenwechsel sind glatt, das heißt: Sind die Karten $\varphi: U\to \varphi(U)$ und $\psi:V\to\psi(V)$ in $\At$ und $V\cap U \ne \emptyset$, so sind $\varphi \circ \psi^{-1}: \psi(U\cap V) \to \varphi(U\cap V)$ sowie $\psi \circ \varphi^{-1}: \varphi(U\cap V) \to \psi(U \cap V)$, genannt Kartenwechsel, glatt.
\end{itemize}

Eine Karte $\psi$ von $M$ heißt mit $\At$ verträglich, wenn auch $\At \cup \{\psi\}$ ein differenzierbarer Atlas für $M$ ist.

\index{Atlas!vollständiger}
$\At$ ist vollständig (oder maximal) wenn jede mit $\At$ verträgliche Karte zu $\At$ gehört.

\index{Mannigfaltigkeik!$n$-dimensional und differenzierbar}
\begin{definition}
Eine $n$-dimensionale differenzierbare Mannigfaltigkeit ist ein topologischer Hausdorff-Raum mit abzählbarer Basis versehen mit einem vollständigen differenzierbaren $n$-dimensionalen Atlas.
\end{definition}

\begin{beispiele}
\index{Struktur! standard-differenzierbare}
\item $\MdR^n$: $\tilde\At = \{\text{id}\}$ ist ein Atlas. Durch erweiterung auf einen maximalen Atlas $\At$ erhalten wir die standard-differenzierbare Struktur auf $\MdR^n$.
\begin{bemerkung}
Auf $\MdR^n$, $n\ne 4$, existiert bis auf Diffeomorphismus %stimmt das?
genau eine differenzierbare Struktur. Auf $\MdR^4$ existieren „exotische“ differenzierbare Strukturen.
\end{bemerkung}
\end{beispiele}


\appendix
\chapter{Satz um Satz (hüpft der Has)}
\listtheorems{satz,wichtigedefinition}

\renewcommand{\indexname}{Stichwortverzeichnis}
\addtocounter{chapter}{1}
\addcontentsline{toc}{chapter}{\protect\numberline {\thechapter}Stichwortverzeichnis}
\printindex

\end{document}
