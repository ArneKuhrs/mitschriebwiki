\documentclass[a4paper,twoside,DIV15,BCOR12mm]{scrbook}
\usepackage{mathe}
\usepackage{skull}
\usepackage{faktor}

\newcommand{\otm}{\stackrel{\circ}{\subset}} % Offene Teilmenge
\newcommand{\At}{\mathcal A}
\renewcommand{\da}{\mathrel{\DOTSB\mathop:}=}
\newcommand{\ad}{=\mathrel{\DOTSB\mathop:}}

\newsavebox{\envbox}
%\newenvironment{definitionX}{
\renewenvironment{definition}{
	\begin{lrbox}{\envbox}
 	\begin{minipage}{1.0\textwidth}
	\begin{plaindefinition}
}{
	\end{plaindefinition}
 	\end{minipage}
	\end{lrbox}
	\par\fcolorbox{white}{blue!7!white}{\usebox\envbox}
}


\author{Die Mitarbeiter von \url{http://mitschriebwiki.nomeata.de/}}
\title{Riemann’sche Geometrie}
\makeindex

\begin{document}
\maketitle
 
\renewcommand{\thechapter}{\Roman{chapter}}
%\chapter{Inhaltsverzeichnis}
\stepcounter{chapter}
%\renewcommand{\tocname}{bla}
\addcontentsline{toc}{chapter}{\protect\numberline {\thechapter}Inhaltsverzeichnis}
\tableofcontents

 % Vorwort

\chapter{Vorwort}
%\addcontentsline{toc}{chapter}{Vorwort}

\section*{Über dieses Skriptum}
Dies ist ein Mitschrieb der Vorlesung \glqq Riemann’sche Geometrie\grqq\ von Herrn Dr. Leuzinger im
Sommersemester 07 an der Universität Karlsruhe (TH). Die Mitschriebe der Vorlesung werden mit
ausdrücklicher Genehmigung von Herrn Dr. Leuzinger hier veröffentlicht, Herr Dr. Leuzinger ist für  den
Inhalt nicht verantwortlich.

\section*{Wer}
Gestartet wurde das Projekt von Joachim Breitner. Bei der Erstellung wurde bisweilen auf ein Skript für die englische Version der Vorlesung, geTeXt von Bernhard Konrad, zurückgegriffen.


\section*{Wo}
Alle Kapitel inklusive \LaTeX-Quellen können unter \url{http://mitschriebwiki.nomeata.de} abgerufen werden.
Dort ist ein von Joachim Breitner programmiertes \emph{Wiki}, basierend auf \url{http://latexki.nomeata.de} installiert. 
Das heißt, jeder kann Fehler nachbessern und sich an der Entwicklung
beteiligen. Auf Wunsch ist auch ein Zugang über \emph{Subversion} möglich.


\setcounter{chapter}{0}
\renewcommand{\thechapter}{\arabic{chapter}}
\chapter{Mannigfaltigkeiten}

\section{Differenzierbare Mannigfaltigkeiten}

\subsection*{Erinnerung (LA/Analysis)}
%\begin{center}
\begin{tabular}{rl}
Euklidischer Raum & $\MdR^n, \langle\cdot,\cdot\rangle$ \\
Norm & $ \|a\| \da  \sqrt{\langle a, a\rangle} $ \\
Metrik & $d(a,b) \da  \| a-b\| $ \\
Winkel & $\cos \angle(a,b) \da  \frac{\langle a, b\rangle}{\|a\|\cdot\|b\|} $ 
\end{tabular}
%\end{center}

Die Funktion $f:U (\otm \MdR^n) \to \MdR$\footnote{$A\otm B \da  $ $A$ offen und $A\subset B$} ist \index{glatt!Funktion}glatt (oder \index{$C^\infty$}$C^\infty$) falls in jedem Punkt $p\in U$ alle gemischten partiellen Ableitungen existieren und stetig sind.

Die $C^\infty$-Funktion
\[ u^i:
\begin{aligned}
\MdR^n &\to \MdR \\
p = (p_1,\ldots,p_n) &\mapsto p_i =u^i(p)
\end{aligned}
\]
heißt $i$-te Koordinatenfunktion ($i=1,\ldots,n$). Eine Abbildung $\phi: U(\otm \MdR^n)\to \MdR^n$ heißt glatt falls jede der reellen Funktionen $u^i\circ\phi$ glatt ist ($i=1,\ldots,n$).

\subsubsection*{Karten und Atlanten}
Sei $M$ in topologischer Raum, der hausdorff’sch ist und eine abzählbare Basis hat.

\index{Koordinatensytem}
\index{Karte}
Ein Koordinatensystem (oder Karte) in $M$ ist ein Homöomorphismus
\[\varphi: U(\otm M) \to \varphi(U) (\otm\MdR^n) \]
Schreibt man $\varphi(p) = (x^i(p),\ldots,x^n(p))$, dann heißen die Funkionen $x^i \da  u^i\circ \varphi : U \to \MdR$ Koordinatenfunktionen von $\varphi$. $n$ heißt Dimension von $(\varphi,U)$.

\index{Atlas}
Ein $n$-dimensionaler, differenzierbarer Atlas für $M$ ist eine Kollektion $\At$ von $n$-dimensionalen Karten von $M$. Es gilt:
\begin{itemize}
\item[$(\At1)$] Jeder Punkt von $M$ liegt im Definitionsbereich mindestens einer Karte, d.h. $M$ ist lokal euklidisch.
\item[$(\At2)$] Alle zu $\At$ gehörigen Kartenwechsel sind glatt, das heißt: Sind die Karten $\varphi: U\to \varphi(U)$ und $\psi:V\to\psi(V)$ in $\At$ und $V\cap U \ne \emptyset$, so sind $\varphi \circ \psi^{-1}: \psi(U\cap V) \to \varphi(U\cap V)$ sowie $\psi \circ \varphi^{-1}: \varphi(U\cap V) \to \psi(U \cap V)$, genannt Kartenwechsel, glatt.
\end{itemize}

Eine Karte $\psi$ von $M$ heißt mit $\At$ verträglich, wenn auch $\At \cup \{\psi\}$ ein differenzierbarer Atlas für $M$ ist.

\index{Atlas!vollständiger}
$\At$ ist vollständig (oder maximal) wenn jede mit $\At$ verträgliche Karte zu $\At$ gehört.

\index{Mannigfaltigkeit!$n$-dimensional und differenzierbar}
\begin{definition}
Eine $n$-dimensionale differenzierbare Mannigfaltigkeit ist ein topologischer Hausdorff-Raum mit abzählbarer Basis versehen mit einem vollständigen differenzierbaren $n$-dimensionalen Atlas.
\end{definition}

\begin{beispiele}
\index{Struktur!standard-differenzierbare}
\item $\MdR^n$: $\tilde\At = \{ (\MdR^n,\text{id}) \}$ ist ein Atlas. Durch Erweiterung zu einem vollständigen Atlas $\At$ erhalten wir die standard-differenzierbare Struktur auf $\MdR^n$.
\begin{bemerkung}
Auf $\MdR^n$, $n\ne 4$, existiert bis auf Diffeomorphismus genau eine differenzierbare Struktur. Auf $\MdR^4$ existieren weitere, „exotische“ differenzierbare Strukturen.
\end{bemerkung}

\item Die Sphären $S^n \da  \{ p = (p_1,\ldots,p_{n+1})\in \MdR^{n+1} \mid \|p\| = 1\}$. Wir behaupten: $S^n$ ist eine $n$-dimensionale differenzierbare Mannigfaltigkeit.

Als Topologie wählen wir die Teilmengen-Topologie, d.h. $U\subset S^2$ offen $\gdw \exists U'\subset \MdR^{n+1}$ offen, so dass $U=S^n\cap U$. Daher folgt auch, dass die Sphären auch hausdorff’sch sind und eine abzählbare Basis haben.

%Um einen Atlas zu finden, definieren wir die offen Halb-Sphären
Seien $U_i^+$ bzw. $U_i^-$ die offenen Hemisphären, definiert durch
\begin{align*}
U_i^+ &\da \{ p \in S^n \mid p_i > 0 \} \\
U_i^- &\da \{ p \in S^n \mid p_i < 0 \}\,.
\end{align*}

Die Abbildungen $\varphi_i^{\pm}: U_i^\pm \to \MdR^n$ (Projektion in Richtung $i$-te Koordinaten-Achse) für $i=1,\dots,n+1$ mit
\[
\varphi_i^{\pm}(p) \da (u^1(p), \dots, u^{i-1}(p),u^{i+1}(p),\dots,u^{n+1}(p)) 
\]
sind Karten mit glatten ($C^\infty$) Kartenwechsel, was wir am Beispiel $n=2$ überprüfen:
\[
(u^1,u^2) \xmapsto{(\varphi_3^+)^{-1}} (u^1,u^2,\sqrt{1-(u^1)^2-(u^2)^2}) \xmapsto{\varphi_1^+} (u^2,\sqrt{1-(u^1)^2-(u^2)^2})\quad ((u^1)^2 +(u^2)^2 < 1)
\]

\item Kurven und Flächen in $\MdR^3$ sind $1$- bzw. 2-dimensionale Mannigfaltigkeiten

\index{Reel-Projektiver Raum}
\index{$P^n\MdR$}
\item[(4a)] Der $n$-dimensionale reell-projektiver Raum $P^n\MdR$
\begin{definition}
Auf $X\da \MdR^{n+1}\setminus\{0\}$ betrachte die Äquivalenz-Relation
\[ x \sim y \gdw \exists t \in \MdR,\; t\ne 0,\; y=tx, \text{ also } (y^1,\ldots,y^n) = (tx^1,\ldots,tx^{n+1})\]
Die Äquivalenzklassen sind also Geraden durch den Ursprung. Nun definieren wir:
% FIXME, Quotiontenraum schöner
\[ P^n\MdR \da \faktor{\MdR^{n+1}\setminus\{0\}}{\sim} \]
\end{definition}

Wir behaupten nun dass $P^n\MdR$ eine $n$-dimensionale differenzierbare Mannigfaltigkeit ist.

Die Topologie erhalten wir aus dem topologischen Raum $\MdR^{n+1}\setminus\{0\}$ über die Quotienten-Topologie, für die wir wir die surjektive Abbildung $\pi$ verwenden:
\[ \pi:
\begin{aligned}
\MdR^{n+1}\setminus\{0\} &\to P^n\MdR \\
x &\mapsto [x]_\sim
\end{aligned}
\]
Zur Erinnerung: Die Quotiententopologie ist allgemein: 
\[ X\subset \faktor{X}{\sim}\text{ offen } \gdw \pi^{-1}(U)\subset X \text{ offen } \]
Um zu Zeigen, dass $P^n\MdR$ eine abzählbare Basis hat, genügt es nach Lemma 1 des verteilen Blattes „Einige Grundbegriffe der Topologie“ zu zeigen, dass $\pi: \MdR^{n+1}\setminus\{0\} \to P^n\MdR$ offen ist. ($\pi$ ist offen wenn $\pi$-Bilder offener Mengen offen sind.) Dazu betrachten wir die Streckung $\alpha_t: X \to X$; $x\mapsto tx$ ($t\ne 0$). $\alpha_t$ ist ein Homöomorphismus mit $\alpha_t^{-1}=\alpha_{\frac 1 t}$.

Sei nun $U\subset X$ offen, so ist $\pi^{-1}(\pi(U)) = \bigcup_{t\ne 0}\alpha_t(U)$. Da jedes $\alpha_t(U)$ offen ist, ist $\pi^{-1}(\pi(U))$ offen. Nach der Definition der Quotiententopologie also ist $\pi(U)$ offen.

Weiter müssen wir zeigen, dass $P^n\MdR$ hausdorff’sch ist. Anschaulich heißt dass, um zwei „Geraden“ $[x]$ und $[y]$ je einen offenen „Kegel“ zu finden, welche disjunkt sind. Wir zeigen dies über das Lemma 2 des Blattes „Einige Grundbegriffe der Topologie“, wozu wir zeigen müssen:
$ R \da \{ (x,y) \in X \times X \mid x \sim y \} $ ist abgeschlossen.

Die Idee ist, auf $X\times X \subset \MdR^{n+1} \times \MdR^{n+1}$ die reelle Funktion $f$ zu betrachten:
\[
f(x,y) = f(x^1,\ldots,x^{n+1},y^1,\ldots,y^{n+1}) \da \sum_{i\ne j} |x^iy^j - x^jy^i|
\]
$f$ ist stetig und $f(x,y) = 0 \gdw y = t x$ für ein $t\ne 0$ $\gdw x\sim y$. Also ist $R = f^{-1}(\{0\})$. Da $f$ stetig ist, ist das Urbild einer abgeschlossenen Menge abgeschlossen, also ist $R$ abgeschlossen. Damit ist gezeigt, dass $\faktor{X}{\sim}$ hausdorff’sch ist.

Also ist $P^n\MdR$ ein topologischer Raum mit den gewünschten Eigenschaften. Es bleibt zu zeigen, dass für diese Menge ein vollständiger Atlas existiert.

Wir definieren also $n+1$ Karten $(U_i,\varphi_i)$ ($i=1,\ldots,n+1$). Es ist $\bar{U_i} \da \{x \in X\mid X^i\ne 0\}$ und $U_i \da \pi(\bar{U_i}) \subset P^n\MdR$. Damit ist $P^n\MdR$ abgedeckt ($\bigcup_{i=1,\ldots,n+1}U_i = P^n\MdR$). Weiter ist:
\[
\varphi_i :
\begin{aligned}
U_i &\to \MdR^n \\
[x] &\mapsto \left( \frac{x^1}{x^i},\ldots,\frac{x^{i-1}}{x^i},\frac{x^{i+1}}{x^i},\ldots,\frac{x^n}{x^i}\right)
\end{aligned}
\]
Diese Definition ist representanten-unabhängig und injektiv:
\begin{align*}
\varphi_i([x]) = \varphi_i([y]) 
& \folgt \frac{y^1}{y^i} = \frac{x^1}{x^i} \ad t \\
& \folgt y^1 = tx^1 \\
& \folgt y = tx \\
& \folgt [y] = [x]
\end{align*}
Auch ist $\varphi_i$ stetig, und surjektiv: $\varphi_i^{-1}(z^1,\ldots,z^n) = \pi(z^1,\ldots,z^{i-1},1,z^{i+1},\ldots,z^n)$.

Die Koordinatenwechsel $\varphi_j\circ \varphi_i^{-1}$ sind affin, also $C^\infty$ (Übungsaufgabe).

Diese Karten lassen sich zu einem vollständigen Atlas für $P^n\MdR$ erweitern, also liegt eine differenzierbare Mannigfaltigkeit vor.

\item[(4b)] $P^n\MdC$ ist eine $2n$-dimensionale differenzierbare Mannigfaltigkeit, was sich ähnlich zeigen lässt. Die doppelte Dimension kommt von der 2-dimenionalität von $K$.

\item[(5)] Wir wollen aus gegebenen Mannigfaltigkeiten neue Mannigfaltigkeiten erhalten.

\index{Untermannigfaltigkeit!offen}
Sei $M$ eine differenzierbare Mannigfaltigkeit mit vollständigem Atlas $\At$. Sei $\At'$ die Menge aller Koordinatensysteme mit Definitionsbereich in einer offenen Teilmenge $O\subset M$. $\At'$ ist ein Atlas für $O$. Die entsprechende differenzierbare Mannigfaltigkeit heißt offene Untermannigfaltigkeit.

\begin{beispiel}
Die allgemeine lineare Gruppe
\[ GL_n\MdR \da \{ A \in \MdR^{n\times n} \mid \det A \ne 0 \} \]
ist eine $n^2$-dimensionale differenzierbare Mannigfaltigkeit:
$\MdR^{n\times x} = \MdR^{n^2}$ ist eine $n^2$-differenzierbare Mannigfaltigkeit und $GL_n\MdR = \MdR^{n\times x} \setminus \{\det A = 0\}$ ist offen, da die Determinantenfunktion stetig ist, also  $\{\det A = 0\}$ abgeschlossen ist.
\end{beispiel}

\item[(6)] Die Produkt-Mannigfaltigkeit:\index{Produkt-Mannigfaltigkeit}
Sind $M^m$ und $N^n$ $m$- bzw. $n$-dimensionale Mannigfaltigkeiten, so ist das topologische Produkt $M\times N$ eine $(n+m)$-dimensionale Mannigfaltigkeit. Der Atlas besteht aus den Karten $\varphi\times \psi: U\times V \to \MdR^m\times\MdR^n = \MdR^{n+m}$ für Karten $(U,\varphi)$ von $M$ und $(V,\psi)$ von $N$.

\begin{beispiel}
($S^1$ ist der Einheitskreis im $\MdR^2$)
\begin{align*}
\MdR^n &= \underbrace{\MdR\times \cdots \times \MdR}_{\text{$n$ Faktoren}} \\
\mathbb{T}^n &= \overbrace{S^1\times \cdot \times S^1} \text{ $n$-dimensionaler Torus}
\end{align*}
\end{beispiel}

\item[(7)] Eine Lie-Gruppe\index{Lie-Gruppe} $G$ ist eine Gruppe die zugleich eine Mannigfaltigkeitsstruktur besitzt und zwar so, dass die Gruppenoperationen $i$ und $m$ differenzierbare Abbildungen (siehe nächster Abschnitt) sind. 
\begin{align*}
m&: G\times G \to G , &m(g_1,g_2) &= g_1g_2 \\
i&: G \to G , &i(g) &= g^{-1} 
\end{align*}
\begin{beispiele}
\item Die eindimensionalen Gruppen $GL_n\MdR$, $GL_1\MdR = (\MdR\setminus\{0\},\cdot)$ und  $(\MdR^+,\cdot)$ 
\item Die null-dimensionale Gruppe $(\MdZ,+)$.
\item Die spezielle Orthogonale Gruppe 
\[ SO(2) \da \left\{
\begin{pmatrix}
\cos \theta & \sin \theta \\ -\sin \theta & \cos \theta \end{pmatrix} \mid \theta \in [0,2\pi) \right\} \]
welche homöomorph zu $S^1$ ist.

\item Die spezielle unitäre Gruppe 
\[ SU(2) \da \left\{
\begin{pmatrix} \alpha & \beta \\ -\bar\beta & \bar\alpha \end{pmatrix} \mid \alpha, \beta \in \MdC,\ \alpha\bar\alpha + \beta\bar\beta =1 \right\} \]
welche homöomorph zu $S^3$ ist.
\end{beispiele}
\end{beispiele}

\section{Differenzierbare Abbildungen}
\begin{definition}[differenzierbare Abbildung]
\index{Differenzierbare Abbildung}
\index{Abbildung!differenzierbare}
\index{glatt!Abbildung}
Eine Abbildung $f:M^m\to N^n$ zwischen differenzierbaren Mannigfaltigkeiten heißt differenzierbar (oder glatt) im Punkt $p\in M$ falls für eine (und damit jede) Karte $\varphi: U \to U'=\varphi(U)\subset\MdR^m$ um $p$ und $\psi:V\to V'=\psi(V)\subset \MdR^n$ um $f(p)$ die Darstellung von $f$ in lokalen Koordinaten $\psi\circ f\circ\varphi^{-1}: U' \to V'$ glatt (oder $C^\infty$) ist.
\end{definition}

Die Unabhängigkeit der Aussage von der Wahl der Karte folgt aus der Definition des Atlases. Seien $\tilde\varphi$ und $\tilde\psi$ andere Karten um $p$ bzw. $f(p)$.
\begin{align*}
\tilde\psi \circ f \tilde\varphi^{-1} &= \tilde\psi \circ (\psi^{-1} \circ \psi) \circ f \circ (\varphi^{-1} \circ \varphi) \circ \tilde\varphi^{-1} \\
 &= \underbrace{(\tilde\psi \circ \psi^{-1})}_{\mathclap{\text{$C^\infty$, da Kartenwechsel}}} \circ \psi \circ f \circ \varphi^{-1} \circ \underbrace{(\varphi \circ \tilde\varphi^{-1})}_{\mathclap{\text{$C^\infty$, da Kartenwechsel}}}
\end{align*}
Also $\tilde\psi \circ f \circ \tilde \varphi^{-1}$ ist $C^\infty \gdw \psi \circ f \circ \varphi^{-1}$ ist $C^\infty$.

Spezialfälle sind:
\begin{itemize}
\item Falls $n=1$ heißt $f:M\to \MdR$ differenzierbare Funktion
\item Falls $m=1$ heißt $f:\MdR\to N$ heißt differenzierbare Kurve
\end{itemize}

\begin{definition}
$C^\infty(M)$ ist die Menge aller $C^\infty$-Funktionen auf einer differenzierbaren Mannigfaltigkeit $M$.
\end{definition}

\begin{bemerkung}
\index{$C^{\infty}(M)$}
$C^{\infty}(M)$ ist eine $\MdR$-Algebra bezüglich Addition, Multiplikation, skalare Multiplikation: ($p\in M$, $\lambda \in \MdR$)
\begin{align*}
(f+g)(p) &\da f(p) + g(p)   \\
(f\cdot g)(p) &\da f(p) \cdot g(p)  \\
(\lambda f)(p) &\da \lambda f(p) \\
\end{align*}
\end{bemerkung}

\begin{definition}[Diffeomorphismus]
\index{Diffeomorphismus}
Eine differenzierbare Abbildung $f:M\to N$ heißt Diffeomorphismus falls $f$ bijektiv und $f$ sowie $f^{-1}$ glatt sind.
\end{definition}

\begin{beispiele}
\item Identität auf $M$
\item Kartenwechsel
\end{beispiele}

Die Menge $\text{Diff}(M)$ aller (Selbst-)Diffeomorphismen von $M$ bilden eine Gruppe.


$\skull$ Ein differenzierbarer Homöomorphismus ist im allgemeinen \textbf{kein} Diffeomorphismus! So ist etwa $f:\MdR\to\MdR,\ x\mapsto x^3$ ein differenzierbarer Homöomorphismus, aber $f^{-1} :\MdR \to \MdR, \ x\mapsto\sqrt[3]{x}$ ist zwar stetig aber nicht glatt.

\section{Tangentialvektoren und -räume}

\subsection*{Erinnerung}
$v \in T_p\MdR^n = \{p\} \times \MdR^n$ und $f:U(p) (\otm \MdR^n) \to \MdR$ sei $C^\infty$. Dann ist die Richtungsableitung von $f$ in Richtung $v$:
\[
\partial_vf \da  \lim_{t\to0} \frac{f(p+tv) - f(p)}{t} = \left.\frac{d}{d^t}\right|_{t=0} f(p+tv)
\]
Für $v=e_i$ erhält man die $i$-te partielle Ableitung
\[ \frac{\partial f}{\partial x^i} = \partial_{e_i} f \]
Es gilt: ($a,b\in\MdR$, $f,g\in C^\infty(\MdR^n)$)
\begin{align*}
\partial_v(af + bg) &= a\partial_v f+ b\partial_v g \\
\partial_v(f \cdot g) &= f(p)\cdot\partial_v g+ g(p)\cdot\partial_v f
\end{align*}

\begin{definition}[Funktionskeim]
Zwei Funktionen $f$, $g$, die auf offenen Umgebungen von $p\in M$ differenzierbar sind, heißen äquivalent, falls sie auf einer Umgebung übereinstimmen. Die Äquivalenzklassen heißen Funktionskeime in $p\in M$: $C^\infty(p)$. Die Menge aller Funktionskeime in $p$ schrieben wir als $C^\infty(p)$.\index{Funktionskeim}
\end{definition}

\begin{definition}[Tangentialvektor]
Sei $M$ eine differenzierbare Mannigfaltigkeit und $p\in M$. Ein Tangentialvektor an $M$ in $p$ ist eine Funktion $v:C^\infty(p) \to \MdR$ so dass gilt: ($a,b\in\MdR$, $f,g\in C^\infty(p)$)
\begin{itemize}
\item[(T1)] $v$ ist $\MdR$-linear: $v(af + bg) = a v(f) + b v(g)$
\item[(T2)] Leibniz-Regel: $v(fg) = f(p)v(g) + g(p)v(f)$ 
\end{itemize}
\end{definition}

\begin{beispiel}
$v(f) \da 0$
\end{beispiel}
Sei $T_pM$ die Menge aller Tangentialvektoren von $M$ im Punkt $p$

\appendix
\chapter{Satz um Satz (hüpft der Has)}
\listtheorems{satz,wichtigedefinition}

\renewcommand{\indexname}{Stichwortverzeichnis}
\addtocounter{chapter}{1}
\addcontentsline{toc}{chapter}{\protect\numberline {\thechapter}Stichwortverzeichnis}
\printindex

\end{document}
