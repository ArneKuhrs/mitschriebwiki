\documentclass[a4paper,twoside,DIV15,BCOR12mm]{scrbook}

\usepackage{mathe}
\usepackage{saetze-leuzinger}
\usepackage{skull}
\usepackage{faktor}
\usepackage{enumerate}

\newcommand{\otm}{\stackrel{\circ}{\subset}} % Offene Teilmenge
\newcommand{\At}{\mathcal A}

\newcommand{\V}{\mathcal V}
\newcommand{\kan}{\text{kan}}
\DeclareMathOperator{\cut}{Cut}
\DeclareMathOperator{\inj}{Inj}
\DeclareMathOperator{\dist}{Dist}
\DeclareMathOperator{\vol}{vol}

\author{Die Mitarbeiter von \url{http://mitschriebwiki.nomeata.de/}}
\title{Riemann’sche Geometrie}
\makeindex

\begin{document}
\maketitle
 

\renewcommand{\thechapter}{\Roman{chapter}}
%\chapter{Inhaltsverzeichnis}
\stepcounter{chapter}
%\renewcommand{\tocname}{bla}
\addcontentsline{toc}{chapter}{\protect\numberline {\thechapter}Inhaltsverzeichnis}
\tableofcontents

 % Vorwort

\chapter{Vorwort}
%\addcontentsline{toc}{chapter}{Vorwort}

\section*{Über dieses Skriptum}
Dies ist ein Mitschrieb der Vorlesung \glqq Riemann’sche Geometrie\grqq\ von Herrn Dr. Leuzinger im
Sommersemester 07 an der Universität Karlsruhe (TH). Die Mitschriebe der Vorlesung werden mit
ausdrücklicher Genehmigung von Herrn Dr. Leuzinger hier veröffentlicht, Herr Dr. Leuzinger ist für  den
Inhalt nicht verantwortlich.
\section*{Wer}
Gestartet wurde das Projekt von Joachim Breitner. Bei der Erstellung wurde bisweilen auf ein Skript für die englische Version der Vorlesung, geTeXt von Bernhard Konrad, zurückgegriffen.


\section*{Wo}
Alle Kapitel inklusive \LaTeX-Quellen können unter \url{http://mitschriebwiki.nomeata.de} abgerufen werden.
Dort ist ein von Joachim Breitner programmiertes \emph{Wiki}, basierend auf \url{http://latexki.nomeata.de} installiert. 
Das heißt, jeder kann Fehler nachbessern und sich an der Entwicklung
beteiligen. Auf Wunsch ist auch ein Zugang über \emph{Subversion} möglich.


\setcounter{chapter}{0}
\renewcommand{\thechapter}{\arabic{chapter}}
\chapter{Mannigfaltigkeiten}

\section{Differenzierbare Mannigfaltigkeiten}

\subsection*{Erinnerung (LA/Analysis)}
%\begin{center}
\begin{tabular}{rl}
Euklidischer Raum & $\MdR^n, \langle\cdot,\cdot\rangle$ \\
Norm & $ \|a\| \da  \sqrt{\langle a, a\rangle} $ \\
Metrik & $d(a,b) \da  \| a-b\| $ \\
Winkel & $\cos \angle(a,b) \da  \frac{\langle a, b\rangle}{\|a\|\cdot\|b\|} $ 
\end{tabular}
%\end{center}

Die Funktion $f:U (\otm \MdR^n) \to \MdR$ ist \index{glatt!Funktion}glatt (oder \index{$C^\infty$}$C^\infty$) falls in jedem Punkt $p\in U$ alle gemischten partiellen Ableitungen existieren und stetig sind.\footnote{$A\otm B \da  $ $A$ offen und $A\subset B$}

Die $C^\infty$-Funktion
\[ u^i:
\begin{aligned}
\MdR^n &\to \MdR \\
p = (p_1,\ldots,p_n) &\mapsto p_i =u^i(p)
\end{aligned}
\]
heißt $i$-te Koordinatenfunktion ($i=1,\ldots,n$). Eine Abbildung $\phi: U(\otm \MdR^n)\to \MdR^n$ heißt glatt falls jede der reellen Funktionen $u^i\circ\phi$ glatt ist ($i=1,\ldots,n$).

\subsubsection*{Karten und Atlanten}
Sei $M$ ein topologischer Raum, der hausdorff’sch ist und eine abzählbare Basis hat.

\index{Koordinatensytem}
\index{Karte}
Ein Koordinatensystem (oder Karte) in $M$ ist ein Homöomorphismus
\[\varphi: U(\otm M) \to \varphi(U) (\otm\MdR^n) \]
Schreibt man $\varphi(p) = (x^i(p),\ldots,x^n(p))$, dann heißen die Funktionen $x^i \da  u^i\circ \varphi : U \to \MdR$ Koordinatenfunktionen von $\varphi$. $n$ heißt Dimension von $(\varphi,U)$.

\index{Atlas}
Ein $n$-dimensionaler, differenzierbarer Atlas für $M$ ist eine Kollektion $\At$ von $n$-dimensionalen Karten von $M$. Es gilt:
\begin{enumerate}[($\At$1)]
\item Jeder Punkt von $M$ liegt im Definitionsbereich mindestens einer Karte, d.h. $M$ ist lokal euklidisch.
\item Alle zu $\At$ gehörigen Kartenwechsel sind glatt, das heißt: Sind die Karten $\varphi: U\to \varphi(U)$ und $\psi:V\to\psi(V)$ in $\At$ und $V\cap U \ne \emptyset$, so sind $\varphi \circ \psi^{-1}: \psi(U\cap V) \to \varphi(U\cap V)$ sowie $\psi \circ \varphi^{-1}: \varphi(U\cap V) \to \psi(U \cap V)$, genannt Kartenwechsel, glatt.
\end{enumerate}

Eine Karte $\psi$ von $M$ heißt mit $\At$ verträglich, wenn auch $\At \cup \{\psi\}$ ein differenzierbarer Atlas für $M$ ist.

\index{Atlas!vollständiger}
$\At$ ist vollständig (oder maximal) wenn jede mit $\At$ verträgliche Karte zu $\At$ gehört.

\index{Mannigfaltigkeit!$n$-dimensional und differenzierbar}
\begin{definition}
Eine $n$-dimensionale differenzierbare Mannigfaltigkeit ist ein topologischer Hausdorff-Raum mit abzählbarer Basis versehen mit einem vollständigen differenzierbaren $n$-dimensionalen Atlas.
\end{definition}

\begin{beispiele}
\index{Struktur!standard-differenzierbare}
\item $\MdR^n$: $\tilde\At = \{ (\MdR^n,\text{id}) \}$ ist ein Atlas. Durch Erweiterung zu einem vollständigen Atlas $\At$ erhalten wir die standard-differenzierbare Struktur auf $\MdR^n$.
\begin{bemerkung}
Auf $\MdR^n$, $n\ne 4$, existiert bis auf Diffeomorphismus genau eine differenzierbare Struktur. Auf $\MdR^4$ existieren weitere, „exotische“ differenzierbare Strukturen.
\end{bemerkung}

\item Die Sphären $S^n \da  \{ p = (p_1,\ldots,p_{n+1})\in \MdR^{n+1} \mid \|p\| = 1\}$. Wir behaupten: $S^n$ ist eine $n$-dimensionale differenzierbare Mannigfaltigkeit.

Als Topologie wählen wir die Teilmengen-Topologie, d.h. $U\subset S^2$ offen $\gdw \exists U'\subset \MdR^{n+1}$ offen, so dass $U=S^n\cap U$. Daher folgt auch, dass die Sphären auch hausdorff’sch sind und eine abzählbare Basis haben.

%Um einen Atlas zu finden, definieren wir die offen Halb-Sphären
Seien $U_i^+$ bzw. $U_i^-$ die offenen Hemisphären, definiert durch
\begin{align*}
U_i^+ &\da \{ p \in S^n \mid p_i > 0 \} \\
U_i^- &\da \{ p \in S^n \mid p_i < 0 \}\,.
\end{align*}

Die Abbildungen $\varphi_i^{\pm}: U_i^\pm \to \MdR^n$ (Projektion in Richtung $i$-te Koordinaten-Achse) für $i=1,\dots,n+1$ mit
\[
\varphi_i^{\pm}(p) \da (u^1(p), \dots, u^{i-1}(p),u^{i+1}(p),\dots,u^{n+1}(p)) 
\]
sind Karten mit glatten ($C^\infty$) Kartenwechsel, was wir am Beispiel $n=2$ überprüfen:
\[
(u^1,u^2) \xmapsto{(\varphi_3^+)^{-1}} (u^1,u^2,\sqrt{1-(u^1)^2-(u^2)^2}) \xmapsto{\varphi_1^+} (u^2,\sqrt{1-(u^1)^2-(u^2)^2})\quad ((u^1)^2 +(u^2)^2 < 1)
\]

\item Kurven und Flächen in $\MdR^3$ sind $1$- bzw. 2-dimensionale Mannigfaltigkeiten

\index{Reell-Projektiver Raum}
\index{$P^n\MdR$}
\item[(4a)] Der $n$-dimensionale reell-projektiver Raum $P^n\MdR$
\begin{definition}
Auf $X\da \MdR^{n+1}\setminus\{0\}$ betrachte die Äquivalenz-Relation
\[ x \sim y \gdw \exists t \in \MdR,\; t\ne 0,\; y=tx, \text{ also } (y^1,\ldots,y^n) = (tx^1,\ldots,tx^{n+1})\]
Die Äquivalenzklassen sind also Geraden durch den Ursprung. Nun definieren wir:
% FIXME, Quotiontenraum schöner
\[ P^n\MdR \da \faktor{\MdR^{n+1}\setminus\{0\}}{\sim} \]
\end{definition}

Wir behaupten nun dass $P^n\MdR$ eine $n$-dimensionale differenzierbare Mannigfaltigkeit ist.

Die Topologie erhalten wir aus dem topologischen Raum $\MdR^{n+1}\setminus\{0\}$ über die Quotienten-Topologie, für die wir die surjektive Abbildung $\pi$ verwenden:
\[ \pi:
\begin{aligned}
\MdR^{n+1}\setminus\{0\} &\to P^n\MdR \\
x &\mapsto [x]_\sim
\end{aligned}
\]
Zur Erinnerung: Die Quotiententopologie ist allgemein: 
\[ U\subset \faktor{X}{\sim}\text{ offen } \gdw \pi^{-1}(U)\subset X \text{ offen } \]
Um zu Zeigen, dass $P^n\MdR$ eine abzählbare Basis hat, genügt es nach Lemma 1 des verteilen Blattes „Einige Grundbegriffe der Topologie“ zu zeigen, dass $\pi: \MdR^{n+1}\setminus\{0\} \to P^n\MdR$ offen ist. ($\pi$ ist offen wenn $\pi$-Bilder offener Mengen offen sind.) Dazu betrachten wir die Streckung $\alpha_t: X \to X$; $x\mapsto tx$ ($t\ne 0$). $\alpha_t$ ist ein Homöomorphismus mit $\alpha_t^{-1}=\alpha_{\frac 1 t}$.

Sei nun $U\subset X$ offen, so ist $\pi^{-1}(\pi(U)) = \bigcup_{t\ne 0}\alpha_t(U)$. Da jedes $\alpha_t(U)$ offen ist, ist $\pi^{-1}(\pi(U))$ offen. Nach der Definition der Quotiententopologie also ist $\pi(U)$ offen.

Weiter müssen wir zeigen, dass $P^n\MdR$ hausdorff’sch ist. Anschaulich heißt das, um zwei „Geraden“ $[x]$ und $[y]$ je einen offenen „Kegel“ zu finden, welche disjunkt sind. Wir zeigen dies über das Lemma 2 des Blattes „Einige Grundbegriffe der Topologie“, wozu wir zeigen müssen:
$ R \da \{ (x,y) \in X \times X \mid x \sim y \} $ ist abgeschlossen.

Die Idee ist, auf $X\times X \subset \MdR^{n+1} \times \MdR^{n+1}$ die reelle Funktion $f$ zu betrachten:
\[
f(x,y) = f(x^1,\ldots,x^{n+1},y^1,\ldots,y^{n+1}) \da \sum_{i\ne j} |x^iy^j - x^jy^i|
\]
$f$ ist stetig und $f(x,y) = 0 \gdw y = t x$ für ein $t\ne 0$ $\gdw x\sim y$. Also ist $R = f^{-1}(\{0\})$. Da $f$ stetig ist, ist das Urbild einer abgeschlossenen Menge abgeschlossen, also ist $R$ abgeschlossen. Damit ist gezeigt, dass $\faktor{X}{\sim}$ hausdorff’sch ist.

Also ist $P^n\MdR$ ein topologischer Raum mit den gewünschten Eigenschaften. Es bleibt zu zeigen, dass für diese Menge ein vollständiger Atlas existiert.

Wir definieren also $n+1$ Karten $(U_i,\varphi_i)$ ($i=1,\ldots,n+1$). Es ist $\bar{U_i} \da \{x \in X\mid X^i\ne 0\}$ und $U_i \da \pi(\bar{U_i}) \subset P^n\MdR$. Damit ist $P^n\MdR$ abgedeckt ($\bigcup_{i=1,\ldots,n+1}U_i = P^n\MdR$). Weiter ist:
\[
\varphi_i :
\begin{aligned}
U_i &\to \MdR^n \\
[x] &\mapsto \left( \frac{x^1}{x^i},\ldots,\frac{x^{i-1}}{x^i},\frac{x^{i+1}}{x^i},\ldots,\frac{x^n}{x^i}\right)
\end{aligned}
\]
Diese Definition ist representanten-unabhängig und injektiv:
\begin{align*}
\varphi_i([x]) = \varphi_i([y]) 
& \folgt \frac{y^1}{y^i} = \frac{x^1}{x^i} \ad t \\
& \folgt y^1 = tx^1 \\
& \folgt y = tx \\
& \folgt [y] = [x]
\end{align*}
Auch ist $\varphi_i$ stetig, und surjektiv: $\varphi_i^{-1}(z^1,\ldots,z^n) = \pi(z^1,\ldots,z^{i-1},1,z^{i+1},\ldots,z^n)$.

Die Koordinatenwechsel $\varphi_j\circ \varphi_i^{-1}$ sind affin, also $C^\infty$ (Übungsaufgabe).

Diese Karten lassen sich zu einem vollständigen Atlas für $P^n\MdR$ erweitern, also liegt eine differenzierbare Mannigfaltigkeit vor.

\item[(4b)] $P^n\MdC$ ist eine $2n$-dimensionale differenzierbare Mannigfaltigkeit, was sich ähnlich zeigen lässt. Die doppelte Dimension kommt von der 2-dimensionalität von $\MdC$.

\item[(5)] Wir wollen aus gegebenen Mannigfaltigkeiten neue Mannigfaltigkeiten erhalten.

\index{Untermannigfaltigkeit!offen}
Sei $M$ eine differenzierbare Mannigfaltigkeit mit vollständigem Atlas $\At$. Sei $\At'$ die Menge aller Koordinatensysteme mit Definitionsbereich in einer offenen Teilmenge $O\subset M$. $\At'$ ist ein Atlas für $O$. Die entsprechende differenzierbare Mannigfaltigkeit heißt offene Untermannigfaltigkeit.

\begin{beispiel}
Die allgemeine lineare Gruppe
\[ GL_n\MdR \da \{ A \in \MdR^{n\times n} \mid \det A \ne 0 \} \]
ist eine $n^2$-dimensionale differenzierbare Mannigfaltigkeit:
$\MdR^{n\times n} = \MdR^{n^2}$ ist eine $n^2$-differenzierbare Mannigfaltigkeit und $GL_n\MdR = \MdR^{n\times n} \setminus \{\det A = 0\}$ ist offen, da die Determinantenfunktion stetig ist, also  $\{\det A = 0\}$ abgeschlossen ist.
\end{beispiel}

\item[(6)] Die Produkt-Mannigfaltigkeit:\index{Produkt-Mannigfaltigkeit}
Sind $M^m$ und $N^n$ $m$- bzw. $n$-dimensionale Mannigfaltigkeiten, so ist das topologische Produkt $M\times N$ eine $(n+m)$-dimensionale Mannigfaltigkeit. Der Atlas besteht aus den Karten $\varphi\times \psi: U\times V \to \MdR^m\times\MdR^n = \MdR^{n+m}$ für Karten $(U,\varphi)$ von $M$ und $(V,\psi)$ von $N$.

\begin{beispiel}
($S^1$ ist der Einheitskreis im $\MdR^2$)
\begin{align*}
\MdR^n &= \underbrace{\MdR\times \cdots \times \MdR}_{\text{$n$ Faktoren}} \\
\mathbb{T}^n &= \overbrace{S^1\times \cdots \times S^1} \text{ $n$-dimensionaler Torus}
\end{align*}
\end{beispiel}

\item[(7)] Eine Lie-Gruppe\index{Lie-Gruppe} $G$ ist eine Gruppe die zugleich eine Mannigfaltigkeitsstruktur besitzt und zwar so, dass die Gruppenoperationen $i$ und $m$ differenzierbare Abbildungen (siehe nächster Abschnitt) sind. 
\begin{align*}
m&: G\times G \to G , &m(g_1,g_2) &= g_1g_2 \\
i&: G \to G , &i(g) &= g^{-1} 
\end{align*}
\begin{beispiele}
\item Die eindimensionalen Gruppen $GL_n\MdR$, $GL_1\MdR = (\MdR\setminus\{0\},\cdot)$ und  $(\MdR^+,\cdot)$ 
\item Die null-dimensionale Gruppe $(\MdZ,+)$.
\item Die spezielle Orthogonale Gruppe 
\[ SO(2) \da \left\{
\begin{pmatrix}
\cos \theta & \sin \theta \\ -\sin \theta & \cos \theta \end{pmatrix} \mid \theta \in [0,2\pi) \right\} \]
welche homöomorph zu $S^1$ ist.

\item Die spezielle unitäre Gruppe 
\[ SU(2) \da \left\{
\begin{pmatrix} \alpha & \beta \\ -\bar\beta & \bar\alpha \end{pmatrix} \mid \alpha, \beta \in \MdC,\ \alpha\bar\alpha + \beta\bar\beta =1 \right\} \]
welche homöomorph zu $S^3$ ist.
\end{beispiele}
\end{beispiele}

\section{Differenzierbare Abbildungen}
\begin{definition}[differenzierbare Abbildung]
\index{Differenzierbare Abbildung}
\index{Abbildung!differenzierbare}
\index{glatt!Abbildung}
Eine Abbildung $f:M^m\to N^n$ zwischen differenzierbaren Mannigfaltigkeiten heißt differenzierbar (oder glatt) im Punkt $p\in M$ falls für eine (und damit jede) Karte $\varphi: U \to U'=\varphi(U)\subset\MdR^m$ um $p$ und $\psi:V\to V'=\psi(V)\subset \MdR^n$ mit $f(U)\subset V$ die Darstellung von $f$ in lokalen Koordinaten $\psi\circ f\circ\varphi^{-1}: U' \to V'$ glatt (oder $C^\infty$) ist.
\end{definition}

Die Unabhängigkeit der Aussage von der Wahl der Karte folgt aus der Definition des Atlasses. Seien $\tilde\varphi$ und $\tilde\psi$ andere Karten um $p$ bzw. $f(p)$.
\begin{align*}
\tilde\psi \circ f \circ \tilde\varphi^{-1} &= \tilde\psi \circ (\psi^{-1} \circ \psi) \circ f \circ (\varphi^{-1} \circ \varphi) \circ \tilde\varphi^{-1} \\
 &= \underbrace{(\tilde\psi \circ \psi^{-1})}_{\mathclap{\text{$C^\infty$, da Kartenwechsel}}} \circ \psi \circ f \circ \varphi^{-1} \circ \underbrace{(\varphi \circ \tilde\varphi^{-1})}_{\mathclap{\text{$C^\infty$, da Kartenwechsel}}}
\end{align*}
Also $\tilde\psi \circ f \circ \tilde \varphi^{-1}$ ist $C^\infty \gdw \psi \circ f \circ \varphi^{-1}$ ist $C^\infty$.

Spezialfälle sind:
\begin{itemize}
\item Falls $n=1$ heißt $f:M\to \MdR$ differenzierbare Funktion
\item Falls $m=1$ heißt $f:\MdR\to N$ heißt differenzierbare Kurve
\end{itemize}

\begin{definition}
$C^\infty(M)$ ist die Menge aller $C^\infty$-Funktionen auf einer differenzierbaren Mannigfaltigkeit $M$.
\end{definition}

\begin{bemerkung}
\index{$C^{\infty}(M)$}
$C^{\infty}(M)$ ist eine $\MdR$-Algebra bezüglich Addition, Multiplikation, skalare Multiplikation: ($p\in M$, $\lambda \in \MdR$)
\begin{align*}
(f+g)(p) &\da f(p) + g(p)   \\
(f\cdot g)(p) &\da f(p) \cdot g(p)  \\
(\lambda f)(p) &\da \lambda f(p) \\
\end{align*}
\end{bemerkung}

\begin{definition}[Diffeomorphismus]
\index{Diffeomorphismus}
Eine differenzierbare Abbildung $f:M\to N$ heißt Diffeomorphismus falls $f$ bijektiv und $f$ sowie $f^{-1}$ glatt sind.
\end{definition}

\begin{beispiele}
\item Identität auf $M$
\item Kartenwechsel
\end{beispiele}

Die Menge $\text{Diff}(M)$ aller (Selbst-)Diffeomorphismen von $M$ bilden eine Gruppe.


$\skull$
Ein differenzierbarer Homöomorphismus ist im allgemeinen \textbf{kein} Diffeomorphismus! So ist etwa $f:\MdR\to\MdR,\ x\mapsto x^3$ ein differenzierbarer Homöomorphismus, aber $f^{-1} :\MdR \to \MdR, \ x\mapsto\sqrt[3]{x}$ ist zwar stetig aber nicht glatt.

\section{Tangentialvektoren und -räume}

\subsection*{Erinnerung}
$v \in T_p\MdR^n = \{p\} \times \MdR^n$ und $f:U(p) (\otm \MdR^n) \to \MdR$ sei $C^\infty$. Dann ist die Richtungsableitung von $f$ in Richtung $v$:
\[
\partial_vf \da  \lim_{t\to0} \frac{f(p+tv) - f(p)}{t} = \left.\frac{d}{d^t}\right|_{t=0} f(p+tv)
\]
Für $v=e_i$ erhält man die $i$-te partielle Ableitung
\[ \frac{\partial f}{\partial x^i} = \partial_{e_i} f \]
Es gilt: ($a,b\in\MdR$, $f,g\in C^\infty(\MdR^n)$)
\begin{align*}
\partial_v(af + bg) &= a\partial_v f+ b\partial_v g \\
\partial_v(f \cdot g) &= f(p)\cdot\partial_v g+ g(p)\cdot\partial_v f
\end{align*}

\begin{definition}[Funktionskeim]
Zwei Funktionen $f,g:M \to \MdR$, die auf offenen Umgebungen von $p\in M$ differenzierbar sind, heißen äquivalent, falls sie auf einer Umgebung übereinstimmen. Die Äquivalenzklassen heißen Funktionskeime in $p\in M$. Die Menge aller Funktionskeime in $p$ schreiben wir als $C^\infty(p)$.\index{Funktionskeim}
\end{definition}

\begin{definition}[Tangentialvektor]
Sei $M$ eine differenzierbare Mannigfaltigkeit und $p\in M$. Ein Tangentialvektor an $M$ in $p$ ist eine Funktion $v:C^\infty(p) \to \MdR$ so dass gilt: ($a,b\in\MdR$, $f,g\in C^\infty(p)$)
\begin{enumerate}[(T1)]
\item $v$ ist $\MdR$-linear: $v(af + bg) = a v(f) + b v(g)$
\item Leibniz-Regel: $v(fg) = v(f)g(p) + f(p)v(g)$ 
\end{enumerate}
Sei $T_pM$ die Menge aller Tangentialvektoren von $M$ im Punkt $p$
\end{definition}

\begin{beispiel}
$v(f) \da 0$
\end{beispiel}

Wie rechnet man mit Funktionskeimen? Praktisch genügt es mit Repräsentanten, also in $p$ differenzierbaren Funktionen zu rechnen.

\begin{lemma}
\begin{itemize}
\item[a)]
$v:C^\infty(p) \to \MdR$ erfülle (T1) und (T2) für Funktionen, die in $p$ differenzierbar sind. Falls $f$ und $g$ in einer Umgebung von $p$ übereinstimmen (d.h. $f\sim g \gdw [f] = [g]$) so ist $v(f) = v(g)$. Also insbesondere: $\tilde v([f]) \da v(f)$.
\item[b)]
Falls $h$ in einer Umgebung von $p$ konstant ist, so ist $v(h)=0$.
\end{itemize}
\end{lemma}

\begin{beweis}
\begin{itemize}
\item[a)] Da $v$ linear ist, genügt es zu zeigen: Falls $f=0$ in einer Umgebung $U$ von $p$, so ist $v(f)=0$. Dazu betrachte die „Abschneidefunktion“ $\tilde g$ mit 
\begin{enumerate}
\item Träger von $\tilde g \da \{ q\in M \mid \tilde g(q) \ne 0\} \subset U$
\item $0\le \tilde g \le 1$ auf $M$
\item $\tilde g = 1$ in einer Umgebung $V$ von $p$, $V \subset U$.
\end{enumerate}
Es ist dann $f\tilde g = 0$ auf $M$. Nun folgt aus den Axiomen (T1) und (T2) dass wegen $v(0)= v(0+0) = v(0) + v(0)$ gilt: $v(0)=0$. Somit ist
\[
0 = v(0) = v(f\tilde g) \gleichwegen{(T2)} v(f)\underbrace{\tilde g(p)}_{=1} + \underbrace{f(p)}_{=0}v(\tilde g) = v(f)\,.
\]
\item[b)] Nach a) können wir annehmen dass $h$ konstant $c$ auf $M$ ist. Es ist dann $v(h)= v(c\cdot1) = c\cdot v(1)$. Aus $v(1) = v(1\cdot 1) = v(1)\cdot 1 + 1\cdot v(1)$ folgt $v(1)=0$ und damit die Behauptung.
\end{itemize}
\end{beweis}

$T_pM$ ist ein $\MdR$-Vektorraum: ($v,w \in T_pM$, $f\in C^\infty(p)$, $a\in\MdR$)
\begin{align*}
\bigl(v + w\bigr)(f) &\da v(f) + w(f) \\
\bigl(a\cdot v\bigr)(f)   &\da a\cdot v(f) 
\end{align*}

% Der Satz ist irgendwie nicht sehr hilfreich:
%Was ist die Definition dieses Vektorraumes?

Weitere Beispiele von Tangentialvektoren via Karten:
\newcommand{\ptv}[1]{\left.\frac{\partial}{\partial x^{#1}}\right|_p}
\newcommand{\qtv}[1]{\left.\frac{\partial}{\partial x^{#1}}\right|_q}

Sei $\varphi=(x^1,\ldots,x^n)$ ein Koordinatensystem (eine Karte) von $M$ im Punkt $p$. (d.h. $x^i=u^i\circ\varphi$). Für $f\in C^{\infty}(p)$ setze:
\[
\frac{\partial f}{\partial x^i}(p) \da \frac{\partial (f\circ \varphi^{-1})}{\partial u^i}\bigl(\varphi(p)\bigr) % \qquad \text{$i$-te partielle Ableitung im $\MdR^n$}
\]

Eine direkte Rechnung zeigt:
\[
\ptv{i}:
\begin{aligned}
C^\infty{p} &\to \MdR \\
f &\mapsto \ptv{i}(f) \da \frac{\partial f}{\partial x^i}(p)
\end{aligned}
\]
ist ein Tangentialvektor in $p$.

\begin{satz}[Basis-Satz]
Sei $M$ eine $m$-dimensionale differenzierbare Mannigfaltigkeit und $\varphi=(x^i,\ldots,x^n)$ eine Karte um $p\in M$. Dann bilden die Tangentialvektoren $\ptv{i}$, $i=1,\ldots,n$, eine Basis von $T_pM$ und es gilt für alle $v\in T_pM$:
\[
v = \sum_{i=1}^m v(x^i) \ptv{i}
\]
Insbesondere ist $\dim T_pM = m = \dim M$.
\label{basissatz}
\end{satz}

Für diesen Satz benötigen wir noch das
\begin{lemma}[Analysis]
Sei $g$ eine $C^\infty$-Funktion in einer bezüglich $o$ sternförmigen offenen Umgebung von $o\in \MdR^n$. Dann gilt: $g = g(0) + \sum_{j=1}^n u^jg_j$ für $C^\infty$-Funktionen $g_j$, $j=1,\ldots,n$.
\label{lem2}
\end{lemma}

\begin{beweis}[Lemma \ref{lem2}]
Taylorintegralformel:
\[
g(u)-g(0) = \int_0^1 \frac d{dt} g(tu) dt = \sum_{j=1}^n u^j \int_0^1 \frac{\partial g}{\partial u^j}(tu) dt
\]
\end{beweis}

\begin{beweis}[Satz \ref{basissatz}]
\begin{enumerate}
\item[(a)] $\ptv{i}$ ist ein Tangentialvektor in $p$. (Rechnung hier ausgelassen) und für die $k$-te Koordinatensystem $x^k\da u^k\circ \varphi$ gilt:
\[
\ptv{i}(x^k) = \frac{\partial (x^k\circ\varphi^{-1})}{\partial u^i}\bigl(\varphi(p)\bigr) = \frac{\partial u^k}{\partial u^i}\bigl(\varphi(p)\bigr) = \delta_{ik}\,.
\]
\item[(b)] Die Vektoren $\ptv{i}$, $i=1,\ldots,n$, sind linear unabhängig:
Sei 
\[
\sum_{i=1}^m \lambda_i \ptv{i} = 0 
\qquad (\lambda_i\in\MdR)
\]
Dann ist für $k=1,\ldots,m$:
\[
0 = 0(x^k) = 
\sum_{i=1}^m \lambda_i \underbrace{\ptv{i}(x^k)}_{\delta_{ik}} = \lambda_k
\]
\item[(c)]  Die Vektoren $\ptv{i}$, $i=1,\ldots,n$, bilden ein Erzeugendensystem. Ohne Einschränkung gelte $\varphi(p)=0$ ($(*)$). Sei $v\in T_pM$ und $a_k \da v(x^k)$, $k=1,\ldots,m$. Setze 
\[w\da v-\sum_{k=1}^m a_k \ptv{k} \in T_pM\,.\]
Dann ist für alle $k=1,\ldots,m$: 
\[
w(x^k) = v(x^k) - \sum_{i=1}^m  a_i \ptv i (x^k) = a_k - \sum_{i=1}^m  a_i \delta_{ik} = 0 \quad (**)
\]
Nun wollen wir zeigen: $w=0$, d.h. $w(f) = 0$ für alle $f\in C^{\infty}(p)$.
Sei $f\in C^\infty(p)$. Dann ist $g \da f \circ \varphi^{-1} \in C^\infty\bigl(\varphi(p)\bigr)$.
\begin{align*}
w(f) &= w(f \circ \varphi^{-1} \circ \varphi) \\
&= w(g \circ \varphi) \\
&\gleichnach{Lemma \ref{lem2}} w\bigl( g(0) + \sum_{j=1}^m (u^j\circ\varphi )\cdot(g_j \circ \varphi ) \bigr) \\
& \gleichnach{(T1),(T2)} 0 + \sum_{j=1}^m \underbrace{w(x^j)}_{\gleichwegen{(**)}0} \cdot (g_j\circ\varphi)(p) + \underbrace{x^j(p)}_{\gleichwegen{(*)}0} \cdot w(g_j \circ\varphi ) ) \\
& = 0
\end{align*}
\end{enumerate}

\end{beweis}


\section{Tangentialabbildungen}

In diesem Abschnitt verwendete Notation: $\Phi: M\to N$ differenzierbar, $f\in C^\infty(M)$ oder $f\in C^\infty(p)$, $\varphi: U \to U'$ eine Karte.

Sei $\Phi:M^m\to N^n$ eine differenzierbare Abbildung zwischen differenzierbaren Mannigfaltigkeiten. Das Ziel ist $\Phi$ in jedem Punkt von $p\in M$ durch lineare Abbildungen $d\Phi_p: T_pM \to T_{\Phi(p)}N$ zu „approximieren“. 

\begin{definition}
\index{Differential}
\index{Tangentialabbildung}
Das Differential (oder die Tangentialabbildung) von $\Phi$ in $p$ ist:
$ d\Phi_p : T_pM \to T_{\Phi(p)}N $ mit $d\Phi_p(v) : C^\infty(\Phi(p)) \to \MdR$ gegeben durch
\[ d\Phi_p(v)(f) \da v(f \circ \Phi)\,. \]
\end{definition}

Nun ist zu zeigen dass $d\Phi_p(v) \in T_{\Phi(p)}N$:
\begin{itemize}
\item[(T1)]
\begin{align*}
d\Phi_p(v)(a\cdot f + b\cdot g) 
&= v( (a\cdot f+b\cdot g) \circ \Phi) \\
&=v(a\cdot f \circ \Phi + b\cdot g \circ \Phi) \\
&=a\cdot v(f\circ \Phi) + b\cdot v(g\circ \Phi) \\
&=a\cdot d\Phi_p(v)(f) + b\cdot d\Phi_p(v)(g)
\end{align*}
\item[(T2)]
\begin{align*}
d\Phi_p(v)(fg) 
&= v( (fg)\circ \Phi) \\
&= v( (f\circ \Phi)\cdot(g\circ\Phi) ) \\
&= v(f\circ \Phi)(g\circ \Phi)(p) + v(g\circ \Phi)(f\circ \Phi)(p) \\
&= d\Phi_p(v)(f) + \cdots 
\end{align*}
\end{itemize}

Beachte, dass aus der Definition direkt folgt: Ist $\Phi = \text{id}_M: M\to M,\> p\mapsto p$, so gilt $d\Phi_p(v) = d(\text{id})_p(v) = v$ für alle $v\in T_pM$.

\begin{lemma}
\label{lem3}
Sei $\Phi \in C^\infty(M,N)$, $\xi = (x^1,\ldots,x^m)$ eine Karte um $p\in M$ und $\eta = (y^1,\ldots,y^n)$ eine Karte um $\Phi(p)\in N$. Dann gilt:
\[
d\Phi_p\left(\ptv{j}\right) = \sum_{i=1}^n \frac{ \partial(y^i\circ \Phi)}{\partial x^j}(p) \left.\frac{\partial}{\partial y^i}\right|_{\Phi(p)} \qquad (*)
\]
\end{lemma}

\begin{beweis}
Sei $w\in T_{\Phi(p)}N$ die linke Seite von $(*)$. Dann gilt nach dem Basis-Satz (Satz \ref{basissatz}) ist 
\[ w = \sum w(y^i) \left.\frac{\partial}{\partial y^i}\right|_{\Phi(p)}\,.\]
Nach der Definition des Differentials ist
\[ w(y^i) = d\Phi_p(\ptv j)(y^i) = \frac{\partial(y^j\circ \Phi)}{\partial x^i}(p)\,.\]
\end{beweis}

\begin{definition}
Die Matrix 
\[
\left( \frac{\partial (y^j\circ \Phi)}{\partial x^j}(p)\right) 
= \left( \frac{\partial(y^j\circ \Phi \circ \xi^{-1})}{\partial u^j}\big(\xi(p)\big)\right)
\qquad (1\le i\le n, 1\le j\le m)
\]
heißt Jacobi-Matrix von $\Phi$ bezüglich $\xi$ und $\eta$.
\index{Jacobi-Matrix}
\end{definition}

\begin{lemma}[Kettenregel]
\label{lem4}
Falls $\Phi\in C^\infty(M,N)$ und $\Psi \in C^\infty(N,L)$, so gilt
\[
d(\Psi\circ\Phi)_p = d\Psi_{\Phi(p)} \circ d\Phi_p\,.
\]
\end{lemma}

\begin{beweis}
Mit einer Testfunktion $g$ überprüfen wir:
\[
d(\Psi \circ \Phi)(v)(g) = v(g \circ \Psi \circ \Phi) = d\Phi(v)(g \circ \Psi) = d\Psi(d\Phi(v))(g)
\]
\end{beweis}

\begin{bemerkung}
Falls $\Phi: M\to N$ ein Diffeomorphismus ist, so folgt wegen
\[ \text{id} = d(\text{id})_p = d(\Phi \circ \Phi^{-1})|_p \gleichnach{Lemma \ref{lem4}} d\Phi_p \circ d\Phi^{-1}_{\Phi(p)} \]
dass
\[ (d\Phi_p)^{-1} = d\Phi_{\Phi(p)}^{-1}\,. \]
Das heißt insbesondere, dass $d\Phi_p$ ein Vektorraum-Isomophismus ist, und $\dim M = \dim N$.
\end{bemerkung}

\begin{satz}[Inverser Funktionensatz für Mannigfaltigkeiten]
\label{invfunk}
\index{Lokaler Diffeomorphismus}
\index{Diffeomorphismus!lokaler}
Ist $\Phi \in C^\infty(M,N)$ und $d\Phi_p:T_pM \to T_{\Phi(p)}N$ ein Vektorraum-Isomorphismus für ein Punkt $p\in M$, dann existiert eine Umgebung $V$ von $p$ und eine Umgebung $W$ von $\Phi(p)$ so dass $\Phi|_V$ ein Diffeomorphismus von $V$ auf $\Phi(V)=W$ ist. $\Phi|_V$ nennen wir einen lokalen Diffeomorphismus.
\end{satz}

\begin{beweis}
Wähle eine Karte $\xi$ um $p\in M$ und eine Karte $\eta$ um $\Phi(p)\in N$. Nach dem Satz über inverse Funktionen (Analysis II) ist $\eta \circ \Phi \circ \xi^{-1}$ ein lokaler Diffeomorphismus (da $d(\eta \circ \Phi \circ \xi^{-1}) = d\eta \circ d\Phi \circ (d\xi)^{-1}$, was jeweils reguläre lineare Abbildungen sind).
\end{beweis}

\section{Tangentialvektoren an Kurven}

Die bisherige Herangehensweise an die Tangentialvektoren war sehr abstrakt, was Vor- und Nachteile hat. Ein weiterer Ansatz ist der Zugang über Kurven, den wir im Folgenden untersuchen.

\index{Kurve}
Eine Kurve ist eine $C^\infty$-Abbildung $c:I\to M$, wobei $I$ ein offenes Intervall in $\MdR$ (meist mit $0\in I$) und $M$ eine differenzierbare Mannigfaltigkeit ist.

Die erste (und einzige) Koordinatenfunktion der trivialen Karte von $I\subset\MdR$ schreiben wir als $u\da u^1$. Der Tangentialvektor ist dann $\frac{d}{d u}|_t \da\frac{\partial}{\partial u^1}|_t \in T_tI = T_t\MdR$.

\begin{definition}
Der Tangentialvektor an $c$ in $c(t)$ ist
\[
c'(t) \da dc_t \left(\left. \frac d{du}\right|_t\right) \in T_{c(t)}M\,.
\]
\end{definition}

Diese Tangentialvektoren haben interessante Eigenschaften:
\begin{enumerate}
\item Für $f\in C^\infty(M)$ ist $c'(t)(f) = \frac{d(f\circ c)}{du}(t)$ (Richtungsableitung)
\item Falls $v\in T_pM$ und $c$ eine Kurve mit $c(0)=p$ und $c'(0) = v$, dann gilt:
\[
v(f) = \frac{d}{dt}\big(f\circ c\big) (0)
\]
\item Ist $c:I\to M$ eine glatte Kurve und $\Phi:M\to N$ eine differenzierbare Abbildung, so ist $\Phi \circ c: I \to N$ eine glatte Kurve in $N$ und es gilt dass
\[
d\Phi_{c(t)}\big(c'(t)\big) = \big(\Phi \circ c\big)'(t)
\]
\begin{beweis}
$d\Phi\big(c'\big)(f) = c'(f\circ \Phi) = \frac{d}{du}\big(f\circ \Phi \circ c\big) (t) 
= \big(\Phi \circ c\big)'(t)(f)$
\end{beweis}
\item Ist $\varphi$ eine Karte um $p$ und $c_i(t):\varphi^{-1}(\varphi(p) + te_i)$, $i=1,\ldots,n$ die $i$-te Koordinatenline\index{Koordinatenlinie} um $p$ bezüglich $\varphi$, so gilt
\[
c_i'(0) = \ptv{i} \qquad (i=1,\ldots,n)
\]
\begin{beweis}
Sei $f\in C^\infty(p)$.
\begin{align*}c_i\big(p\big) (f) &= \frac d{dt}\big(f\circ c_i\big)(0)  \\
&= \frac d{dt} \big(f\circ \varphi^{-1}(\varphi(p) + te_i)\big)(0) \\
&= \frac{\partial}{\partial u^i}\big(f\circ \varphi^{-1}\big)(\varphi(p))\\
&= \ptv{i}(f)
\end{align*}
\end{beweis}
\end{enumerate}

\section{Untermannigfaltigkeiten und spezielle differenzierbare Abbildungen}

Eine $C^\infty$-Abbildung $\Phi:M^m\to N^n$ heißt 
\begin{itemize}
\item Immersion, falls $d\Phi_p: T_pM \to T_{\Phi(p)}N$ injektiv ist für alle $p\in M$.\index{Immersion}
\item Submersion, falls $d\Phi_p: T_pM \to T_{\Phi(p)}N$ surjektiv ist für alle $p\in M$.\index{Submersion}
\item Einbettung, $\Phi$ eine Immersion ist und $M$ homöomoph zu $\Phi(M)\subset N$ (versehen mit der Teilraum-Topologie) ist. \index{Einbettung}
\end{itemize}

Eine Teilmenge $M\subset N$ heißt (reguläre) Untermannigfaltigkeit,\index{Untermannigfaltigkeit!reguläre} falls die Inklusionsabbildung $i:M\hookrightarrow N$, $i(p) \da p$, eine differenzierbare Einbettung ist.

Manchmal definiert man eine (allgemeine) Untermannigfaltigkeit\index{Untermannigfaltigkeit!allgemein} als injektive Immersion $\Phi:M\to N$, so dass $M$ und $\Phi(M)$ diffeomorph sind. Dabei hat $\Phi(M)$ nicht notwendigerweise die Teilraum-Topologie.

\begin{beispiele}
\item Immersion:
\begin{align*}
\MdR^k &\to \MdR^{k+l} \\
(x^1,\ldots,x^k) &\mapsto (x^1,\ldots,x^k,0,\ldots,0)
\end{align*}
Man kann zeigen: Lokal sieht jede Immersion so aus.
\item Submersion:
\begin{align*}
\MdR^{k+l} &\to \MdR^k\\
(x^1,\ldots,x^{k+l}) &\mapsto (x^1,\ldots,x^k)
\end{align*}
Auch hier kann man zeigen, dass jede Submersion lokal so aussieht.

\item  Die Kurve $c : \MdR \to \MdR^2$, $t\mapsto (t^3, t^2)$ ist differenzierbar, aber keine Immersion, denn
\[ c'(o) = dc_0\Big(\underbrace{\frac d{dt}}_{\ne 0}\Big) = (0,0)\,.\]

\item  Die Kurve $c : \MdR \to \MdR^2$, $t\mapsto (t^3 - 4t, t^2 -4)$ ist eine Immersion, aber keine Einbettung.

\item $\MdR^2$ versehen mit der Äquivalenzrelation 
\[
(x, y)\sim(u,v) \gdw
\begin{aligned}
x &\equiv u \pmod {2\pi\MdZ}\\
y&\equiv v \pmod {2 \pi\MdZ}
\end{aligned}
\]
ergibt den zweidimensionalen Torus $T^2 \da \faktor{\MdR^2}{\sim}$. Wir betrachten nun die Kurve $c_\alpha: \MdR\to T^2$, $t\mapsto (e^{it}, e^{i\alpha t})$.
\begin{satz*}[Kronecker]
\begin{itemize}
\item $\alpha \in 2\pi\MdQ \folgt c_\alpha(\MdR)$ geschlossene Kurve.
\item $\alpha \notin 2\pi\MdQ \folgt c_\alpha(\MdR)$ dicht in $\MdR^2$
\end{itemize}
\end{satz*}
\begin{beweis}
\emph{Siehe V.I. Arnold: Gewöhliche Differenzialgleichungen}
\end{beweis}
Daraus folgt: Für $\alpha\notin 2\pi\MdQ$ ist $c_\alpha$ eine injektive Immersion, aber keine Einbettung, da $c_\alpha(\MdR)\subset\MdR^2$ mit der Teilraumtopologie nicht homöomoph zu $\MdR$ ist.
\end{beispiele}

\begin{bemerkungen}
\item Jede Immersion ist lokal eine Einbettung.
\item Einbettungs-Satz von Whitney (1936): Jede differenzierbare $n$-dimensionale Mannigfaltigkeit kann in $\MdR^{2n+1}$ eingebettet werden:
\[
\Phi: M^n \hookrightarrow \MdR^{2n+1}
\]
\emph{(Beweis: L.Führer: Topologie)}
\end{bemerkungen}

\section{Tangentialbündel und Vektorfelder}

\begin{satz}[Tangentialbündel]
Sei $M$ eine $n$-dimensionale differenzierbare Mannigfaltigkeit und 
\[TM \da \bigcup_{p\in M} T_pM = \{(p,v)\mid p\in M, v\in T_pM\}\,.\]
$TM$ ist eine $2n$-dimensionale differenzierbare Mannigfaltigkeit.
\label{satz3}
\end{satz}
$TM$ heißt Tangentialbündel\index{Tangentialbündel} und ist ein Spezialfall eines Vektorraumbündels. In der Physik entspricht dies dem Phasenraum (Ort, Geschwindigkeit).

\begin{beweis}
\emph{(Skizze)} Sei $(U_\alpha, \varphi_\alpha)_{\alpha \in A}$ ein Atlas für $M$. Ist $\varphi_\alpha = (x_\alpha^1,\ldots, x_\alpha^n)$, so gilt nach Basis-Satz (\mbox{Satz \ref{basissatz}}), dass $\{\ptv{i} \mid i = 1,\ldots,n\}$ eine Basis von $T_pM$ für alle $p \in U_\alpha$ ist. Für $v\in T_pM$ gilt also $v= \sum_{i=1}^n v(x^i)\ptv{i}$.

Somit erhalten wir für jedes $\alpha\in A$ eine bijektive Abbildung 
\[
h_\alpha:
\begin{aligned}
V_\alpha \da TU_\alpha = \bigcup_{p\in U_\alpha} T_pM &\to \MdR^{2n} \\
(p,v) &\mapsto \big(x^1(p), \ldots, x^n(p), v(x^1), \ldots, v(x^n)\big)
\end{aligned}
\]
\emph{Ohne Beweis:} $(V_\alpha, v_\alpha)_{\alpha\in A}$ ist ein differenzierbarer Atlas für $TM$.
\end{beweis}

\begin{definition}
Sei $M$ eine differenzierbarere Mannigfaltigkeit, $TM$ das Tangentialbündel von $M$ und $\pi: TM \to M$, $(p,v)\mapsto p$ die natürliche (oder kanonische) Projektion.

Ein Vektorfeld\index{Vektorfeld} (VF) auf $M$ ist eine Abbildung $V : M \to TM$, $p\mapsto v_p$ mit $\pi \circ V = \text{id}_M$, d.h. $v_p \in T_pM$.

Das Vektorfeld ist differenzierbar ($C^\infty$, glatt), falls $V: M \to TM$ eine differenzierbare Abbildung ist. Äquivalent dazu: Für alle $f\in C^\infty(M)$ ist $Vf\in C^\infty(M)$ mit $\big(Vf\big)(p) \da v_p(f)$.

Wir definieren für $p\in M$ und $f\in C^\infty(M)$:
\begin{itemize}
\item $\big(f\cdot V\big)(p) \da f(p) v_p$ sowie
\item $\big(V+W\big)(p) \da v_p + w_p$.
\end{itemize}
Damit ist $\V M$ (die Menge aller Vektorfelder auf $M$) ein $C^\infty(M)$-Modul.
\end{definition}

Die lokale Darstellung der Vektorfelder liefert uns Basisfelder:\index{Basisfeld}
Sei $\varphi = (x^1,\ldots,x^n)$ für $U\subset M$. Dann ist für $i=1,\ldots,n$
\[
\frac{\partial}{\partial x^i}:
\begin{aligned}
U&\to TU\\
p&\mapsto \ptv{i}
\end{aligned}
\]
ein Vektorfeld auf $U$, nämlich das $i$-te Koordinaten-Vektorfeld\index{Koordinaten-Vektorfeld} von $\varphi$ oder „begleitendes $n$-Bein\rlap{.}“

Nach dem Basissatz (Satz \ref{basissatz}) gilt: Jedes Vektorfeld $V\in \V M $ kann auf $U$ geschrieben werden als 
\[
V = \sum_{i=1}^n V(x^i) \frac{\partial}{\partial x^i}
\]

\begin{definition}
Eine Derivation\index{Derivation} von $C^\infty M$ ist eine Abbildung $\mathcal{D}: C^\infty M \to C^\infty M$ mit
\begin{enumerate}[(D1)]
\item $\mathcal{D}$ ist $\MdR$-linear: $\mathcal{D}(af+ bg) = a\mathcal D(f) + b\mathcal D(g)$
\item Leibnitz: $\mathcal D(f\cdot g) = \mathcal D(f)\cdot g + f\mathcal D(g)$
\end{enumerate}
\end{definition}

Aus den Axiomen (T1), (T2) für Tangentialvektoren folgt, dass $V\in\V M $ eine Derivation ist.

Umgekehrt gilt, dass Jede Derivation von einem Vektorfeld kommt:  Sei $\mathcal D$ eine Derivation. Definiere für jeden Punkt $p\in M$: $v_p(f) \da \mathcal D\big(f\big)(p)$. Aus (D1), (D2) folgt: $v_p\in T_pM$ und $V: M \to TM$, $p\mapsto v_p$ ist ein Vektorfeld.

Weiter gilt für alle $p\in M$: $\big(Vf\big)(p) = v_p\big(f\big) = \big(\mathcal D f\big) (p)$, also ist $Vf = \mathcal D f$, insbesondere ist $V$ glatt. Also entspricht $\V M $ den Derivationen auf $C^\infty M$.

Warum also führen wir Derivationen ein? Die entscheidende Eigenschaft ist dass das Produkt zwei Vektorfelder $V$ und $W$
\[
\big(V \cdot W\big) (f) \da V (W f)
\]
keine Derivation ist, da (D2) nicht erfüllt ist, also $(V\cdot W)$ kein Vektorfeld ist!

Dies korrigieren wir mit der Lie-Klammer\index{Lie-Klammer}
\[
[V,W] \da V\cdot W  - W\cdot V
\]
welche eine Derivation liefert! Insbesondere ist also $[V,W]$ wieder ein Vektorfeld.

Also 
\[[\cdot,\cdot]:
\begin{aligned}
\V M  \times \V M  &\to \V M  \\
(V,W) &\mapsto [V,W]
\end{aligned}
\]

\begin{lemma}
$\V M $ versehen mit der Lie-Klammer $[\cdot,\cdot]:\V M  \times \V M  \to \V M $ ist eine Lie-Algebra.
\label{lem5}
\end{lemma}

\begin{definition}
\index{Lie-Algebra}
Eine reelle Lie-Algebra ist ein $\MdR$-Vektorraum $L$ mit einer Verknüpfung $[\cdot,\cdot]: L\times L \to L$ mit 
\begin{enumerate}[(L1)]
\item $\MdR$-Linearität: $[ax+by, z] = a [x,z] + b[y,z]$ sowie
$[x,ay + bz] = a[x,y] + b[x,z]$ für $a,b\in\MdR$
\item Schiefsymetrie: $[x,y] = -[y,x]$
\item Jacobi-Identität: $\big[ [x,y], z\big] + \big[ [z,x], y\big] + \big[[y,z], x\big] = 0$
\end{enumerate}
\end{definition}

\subsection*{Vektorfelder und Differentialgleichungen}

Sei $V\in \V M $. Eine Integralkurve von V ist eine differenzierbare Kurve $\alpha: I\to M$ mit $\alpha'(t) = V\big(\alpha(t)\big)$ für alle $t\in I$.

In einem Koordinatensystem $\varphi=(x^1,\ldots,x^n)$ gilt:
\[
\alpha'(t) = \sum_{i=1}^n \frac d{dt}(x^i\circ \alpha) \left.\frac\partial{\partial x^i}\right|_{\alpha(t)}
\text{ sowie }
V\big(\alpha(t)\big) = \sum_{i=1}^n V(x^i\circ \alpha) \left.\frac\partial{\partial x^i}\right|_{\alpha(t)}
\]
Also gilt für $i=1,\ldots,n$
\[
\alpha'(t) = V(\alpha(t)) \gdw \frac d{dt}(x^i\circ \alpha) = V(x^i\circ\alpha)
\]
Dies ist ein System von $n$ gewöhnlichen Differenzialgleichungen erster Ordnung. Aus Existenz- und Eindeutigkeitssätzen für solche Systeme (zum Beispiel Königsberger II, 4.2) folgt

\begin{satz}[Existenz und Eindeutigkeit der Integralkurven]
Sei $V\in \V M $. Dann existiert für jeden Punkt $p\in M$ ein Intervall $I=I(p)$ um 0 und eine eindeutige Integralkurve $\alpha : I \to M$ von $V$ mit $\alpha(0) = p$.
\label{satz4}
\end{satz}

\begin{korrolar}
Ist $v\in T_pM$, dann existiert eine differenzierbare Kurve $\alpha: I\to M$ mit $\alpha (0)=p$ und $\alpha'(0)=v$.
\end{korrolar}
Beweisidee: Ergänze $v$ zu einem Vektorfeld in einer Umgebung von $p$ und wende Satz $\ref{satz4}$ an.

\chapter{Riemann’sche Metriken}

\section{Definition einer Riemann’schen Metrik und Struktur}
\newcommand{\asp}{\langle\cdot,\cdot\rangle} % Abstraktes Skalarprodukt
\newcommand{\ksp}[2]{\langle#1,#2\rangle} % Konkretes Skalarprodukt mit zwei Parameter
\newcommand{\lsp}[2]{\left\langle#1,#2\right\rangle} % großes Skalarprodukt mit zwei Parameter
\newcommand{\aasp}{\langle\langle\cdot,\cdot\rangle\rangle} % Abstraktes Skalarprodukt 2
\newcommand{\kksp}[2]{\langle\langle#1,#2\rangle\rangle} % großes Skalarprodukt 2

\index{Riemann’sche Metrik}
\index{Riemann’sche Struktur}
\index{Metrik!Riemann’sche}
\index{Struktur!Riemann’sche}
Eine Riemann’sche Metrik (oder Riemann’sche Struktur) auf einer differenzierbaren Mannigfaltigkeit $M$ ist dadurch gegeben, dass jedem Punkt $p\in M$ ein Skalarprodukt $\asp_p \equiv g_p(\cdot,\cdot)$ in $T_pM$ zugeordnet wird.

Diese Zuordnung soll differenzierbar sein, das heißt für alle lokalen Koordinaten $\phi: U \to \MdR^n$; $q\mapsto \big(x^1(q), \ldots, x^n(q)\big)$ sind die Funktionen 
\[
g_{ij}:
\begin{aligned}
U &\to \MdR \\
q &\mapsto g_{ij}(q) \da \left\langle \left.\frac\partial{\partial x^i}\right|_q, \left.\frac\partial{\partial x^j}\right|_q\right\rangle
\end{aligned}
\]
$C^\infty$ für $1\le i,j\le n$. Die $(n\times n)$-Matrix $\big( g_{ij}(q)\big)$ ist symmetrisch und positiv definit für alle $q\in  U$.

Insbesondere gilt für $v = \sum_{i=1}^n a_i \left.\frac\partial{\partial x^i}\right|_q$ und $w = \sum_{j=1}^n b_j \left.\frac\partial{\partial x^i}\right|_q \in T_pM$:
\[ \langle v,w\rangle_q = \sum_{i,j=1}^n a_i b_j g_{ij}(q) \]

\begin{definition}
\index{Riemann’sche Mannigfaltigkeit}
\index{Mannigfaltigkeit!Riemann’sche}
Eine Riemann’sche Mannigfaltigkeit ist ein Paar $(M,g)$ (oder $(M,\asp)$) bestehend aus einer differenzierbaren Mannigfaltigkeit $M$ und einer Riemann’schen Struktur auf $M$.
\end{definition}

\begin{bemerkung}
Ist $g$ nicht positiv definit (d.h. $g_p(v,v)\ge 0$ und $g_p(v,v) = 0 \gdw v=0$), sondern nur semi-definit, so heißt $g$ Pseudo-Riemann’sche Struktur. Zum Beispiel der $\MdR^4$ versehen mit der Form $x^2 + y^2 + z^2 - t^2$ modelliert die Minkowski-Raum-Zeit der speziellen Relativitätstheorie. Mehr dazu etwa in B. O’Neill: Semi-Riemannian Geometry.
\end{bemerkung}

Der Isomophie-Begriff auf Riemann’schen Mannigfaltigkeiten: Ein Diffeomorphismus $\Phi: (M,\asp) \to (N,\aasp)$ zwischen Riemann’schen Mannigfaltigkeiten heißt Isometrie\index{Isometrie} falls für alle $p\in M$ und alle $v,w\in T_pM$ gilt:
\[ \langle\langle d\Phi_p(v), d\Phi_p(w) \rangle\rangle_{\Phi_{p}} = \langle v,w\rangle_p \qquad (*) \]

Ein lokaler Diffeomorphismus $\Phi: U \to V$ ($U\subset M$, $V\subset N$) heißt lokale Isometrie\index{lokale Isometrie} falls $(*)$ gilt für alle $q\in U$ und alle $v,w \in T_qM$.


\section{Beispiele und Konstruktionen}
\subsection{$n$-dimensionaler Euklidischer Raum}
$M=\MdR^n$ mit Atlas $\{\text{id}\}$ ist eine Riemann’sche Struktur mit dem Standard-Skalarprodukt $\asp$. Dabei ist $g_{ij}(p) = \langle \ptv{i}, \ptv{j} \rangle = \langle e_i, e_j \rangle = \delta_{ij}$, also ist $\big(g_{ij}(p)\big)$ die Einheitsmatrix.

\subsection{$n$-dimensionale hyperbolische Räume}
$M = H^n \da \{ x = (x^,\ldots, x^n) \in \MdR^n \mid x_n > 0 \}$. Dies ist eine offene Teilmenge von $\MdR^n$, also eine offene Untermannigfaltigkeit.

Die Riemann’sche Metrik ist dann 
\[
g_{ij}(x) \da 
\begin{cases}
\frac 1 {(x^n)^2} & 1\le i = j \le n \\
0 & i \ne j
\end{cases}
\]
und die Matrix 
\[
\big(g_{ij}(x)\big) = 
\begin{pmatrix}
\frac 1 {(x^n)^2}& & 0 \\
& \ddots & \\
0 & & \frac 1 {(x^n)^2}
\end{pmatrix}
\]
positiv definit und symetrisch, also ist $(H^n, g)$ eine Riemann’sche Mannigfaltigkeit und ist ein Modell für $n$-dimensionale hyperbolische Geometrien.

\subsection{Konstruktion von neuen Riemann’schen Mannigfaltigkeiten aus gegebenen}

Sei $\Phi: M^m \to N^{n=m+k}$ sei eine Immersion. Weiter sei auf $N$ eine Riemann’sche Struktur $\aasp$ gegeben. Diese induziert eine Riemann’sche Metrik $\asp$ auf $M$:

Für $p\in M$, $u,v\in T_pM$ setze: $\langle u,v\rangle_p \da \langle\langle d\Phi_pu, d\Phi_p v \rangle\rangle_{\Phi(p)}$

Zu zeigen ist: $\asp_p$ ist symetrisch, bilinear und positiv definit. Die Symmetrie und Bilinearität ist klar. Zu überprüfen: Ist $\asp$ positiv definit? Es ist $\langle u,u\rangle_p \ge 0$. Ist $0 = \langle u,u\rangle_p = \langle\langle d\Phi_pu,d\Phi_pu\rangle\rangle_{\Phi(p)}$, so ist $d\Phi_pu = 0 \folgtnach{$\Phi$ injektiv} u = 0$

\index{Isometrische Immersion}
\index{Immersion!isometrische}
Die Abbildung $\Phi: (M,\asp) \to (N,\aasp)$ heißt isometrische Immersion von $M$ in $N$.

\begin{beispiel}
Flächen im $\MdR^3$ mit Standardskalarprodukt, wobei $\Phi=i: F \hookrightarrow \MdR^3$ die Inklusionsabbildung ist. Die so induzierte Riemann’sche Metrik auf $F$ heißt die 1. Fundamentalform von $F$. Für $u,v \in T_pF$ gilt dann:
\[
\langle u,v\rangle \da \langle di_pu, di_pv\rangle = \langle u, v\rangle
\]
wobei das letzte Skalarprodukt das Standardskalarprodukt ist.
\end{beispiel}

Analog kann man mit anderen Untermannigfaltigkeiten des $(\MdR^m,\asp)$ vorgehen. So kan man $S^n = \{x\in\MdR^{n+1}\mid \|x\|=1\}\subset \MdR^n+1$ mit der vom Standardskalarprodukt in $\MdR^n+1$ induzierten Riemann’schen Metrik versehen. Diese heißt sphärische Geometrie.\index{Sphärische Geometrie}

\begin{bemerkung}
Die klassischen Geometrien (euklidische, hyperbolische, sphärische) sind Spezialfälle der Riemman’schen Geometrien.
\end{bemerkung}

\subsection{Riemann’sche Produkte}
Seien $(M_1, \asp^{(1)})$, $(M_2, \asp^{(2)})$ zwei Riemann’sche Mannigfaltigkeiten. $M_1\times M_2$ ist eine differenzierbare Mannigfaltigkeit. Weiter haben wir die zwei kanonischen Projektionen auf die Faktoren:
\begin{align*}
\pi_1: M_1&\times M_2 \to M_1 &
\pi_2: M_1&\times M_2 \to M_2 \\
(m_1,m_2) &\mapsto m_1 &
(m_1,m_2) &\mapsto m_2 \\
\end{align*}

\begin{definition}[Riemann’sche Produktmetrik]
Riemann’sche Produktmetrik auf $M_1\times M_2$ ist für alle $u,v\in T_{(p,q)}(M_1\times M_2)$ und für alle $(p,q)\in M_1\times M_2$:
\begin{align*}
\ksp uv_{(p,q)} \da \ & \ksp {d{\pi_1}_{(p,q)} u}{ d{\pi_2}_{(p,q)}v}^{(1)} \\
+ &\ksp {d{\pi_2}_{(p,q)} u}{ d{\pi_2}_{(p,q)}v}^{(2)}
\end{align*}
(Kurz: $\|u\|^2 = \ksp uu = \ksp {u_1}{u_1}^{(1)} + \ksp {u_2}{u_2}^{(2)} = \|u_1\|^2 + \|u_2\|^2$)
\end{definition}

$\asp_{(p,q)}$ ist symetrisch und positiv bilinear. Es ist auch positiv definit:
\[
0 = \ksp uu \folgt 
\left.\begin{aligned}
d\pi_1 u = 0 \\
d\pi_2 u = 0
\end{aligned}\right\} \folgt u=0\,,
\]
da $u=d\pi_1u \oplus d\pi_2 u$.

\begin{beispiele}
\item $(\MdR^n,\asp) = \prod_{i=1}^n (\MdR^1,\asp)$. $(a_1,\ldots,a_n) = a \in T_x\MdR^n$; $\|a\|^2 = \sum_{i=1}^n a_i^2$
\item Flacher Torus:

$T^2 \da S^1 \times S^2$, wobei jeder Faktor $S^1$ mit der kanonischen Riemann’schen Metrik, induziert von $\MdR^2$, versehen ist. Wir betrachten lokale Koordinaten $(s,t)$. Dann:
\[
T_{(s,t)}(S^1\times S^1) = \MdR \left.\frac{\partial}{\partial s}\right|_s \oplus \MdR \left.\frac{\partial}{\partial t}\right|_t
\]
Sei nun $u,v\in T_{(s,t)}(S^1\times S^1)$ mit $u=a \frac\partial{\partial s} + b \frac\partial{\partial t}$ und $v=c \frac\partial{\partial s} + d \frac\partial{\partial t}$. Das heißt: $d\pi_1 u = a \frac\partial{\partial s}$ und $d\pi_2 u = b \frac\partial{\partial t}$.  
Ohne Einschränkung sei $\lsp {\frac\partial{\partial s}} {\frac\partial{\partial s}} = 1$ und $\lsp {\frac\partial{\partial t}} {\frac\partial{\partial t}} = 1$

Die Riemann’sche Produktmetrik auf $T^2$ bezüglich lokalen Koordinaten $(s,t)$:
\begin{align*}
g_{11}(s,t) &= \lsp {d\pi_1 ({\frac\partial{\partial s}} + 0)}{d\pi_1 ({\frac\partial{\partial s}} + 0)} + \lsp {d\pi_2 ({\frac\partial{\partial s}} + 0)}{d\pi_2 ({\frac\partial{\partial s}} + 0)} \\
&= \lsp {\frac\partial{\partial s}}{\frac\partial{\partial s}} + \lsp 0 0 = 1
\end{align*}
Analog: $g_{22}(s,t) = \cdots = \lsp {\frac\partial{\partial t}} {\frac\partial{\partial t}} = 1$
\begin{align*}
g_{12}(s,t) &= \lsp {d\pi_1 ({\frac\partial{\partial s}} + 0)}{d\pi_1 ({\frac\partial{\partial t}} + 0)} + \lsp {d\pi_2 ({\frac\partial{\partial s}} + 0)}{d\pi_2 ({\frac\partial{\partial t}} + 0)} \\
&= \lsp {\frac\partial{\partial s}}0 + \lsp 0 {\frac\partial{\partial s}} = 0
\end{align*}
also ist
\[
\Big( g_{ij}(s,t) \Big) = 
\begin{pmatrix}
1 & 0 \\ 0 & 1
\end{pmatrix}
\]
das heißt: $T^2$ mit Produktmetrik ist lokal isometrisch zur euklidischen Ebene. 

$\skull$ $T^2$ und $\MdR^2$ sind nicht global isometrisch (sonst wären sie homöomoph, aber $\MdR^2$ ist nicht kompakt, während $T^2$ kompakt ist).


\end{beispiele}


\section{Existenz von Riemann’schen Metriken}

\begin{satz}[Existenz der Riemann’schen Metrik]
Auf jeder $n$-dimensionalen differenzierbaren Mannigfaltigkeit existiert eine Riemann’sche Metrik.
\label{exriemet}
\end{satz}

\begin{beweis}
Wir gehen in zwei Schritten vor: %Füllsatz, LaTeX komisch.
\paragraph*{1. Schritt} (lokale Konstruktion für Kartengebiete)

Gegeben eine Karte $\varphi_\alpha: U_\alpha \to \MdR^n$, $p\mapsto \varphi_\alpha(p) = \big(x^1_\alpha(p),\ldots,x^n_\alpha(p)\big)$. Wir benötigen $\frac{n(n+1)}2$ $C^\infty$-Funktionen $g_{ij}: U_\alpha \to \MdR$, so dass die $n\times n$-Matrix $\big(g_{ij}(q)\big)$ positiv definit wird für alle $q\in U_\alpha$.

Eine Möglichkeit: Wähle Standardskalarprodukt $\asp$ auf $\varphi_\alpha(U_\alpha)\subset \MdR^n$, das heißt $\ksp {e_i}{e_j} = \delta_{ij}$ und setze für alle $u,v\in TqM$, $q\in U_\alpha$:
\[
g_\alpha(u,v) \da \lsp {d\varphi_{\alpha}|_q(u)} {d\varphi_{\alpha}|_q(v)} _{\varphi_\alpha(q)}\,,
\]
das heißt $\varphi_\alpha$ wird zu einer lokalen Isometrie gemacht.

Weil $d\varphi_\alpha|_q (\left.\frac\partial{\partial x^i}\right|_q) = e_i$ für $i=1,\ldots,n$ gilt, ist
\[
g_{ij}^{(\alpha)}(q) = g_\alpha\left( \left.\frac\partial{\partial x^i}\right|_q, \left.\frac\partial{\partial x^j}\right|_q \right) = \ksp {e_i} {e_j} = \delta_{ij}\,.
\]

\paragraph*{2. Schritt} (Globale Konstruktion)

Wir nehmen ein Hilfsmittel aus der Differential-Topologie:
\begin{satz}[„Zerlegung der Eins“]
\label{einszerl}
Sein $M$ eine differenzierbare Mannigfaltigkeit (insbesondere Hausdorff’sch und es existiert eine abzählbare Basis) und $(U_\alpha)_{\alpha\in A}$ eine (offene) Überdeckung von $M$ von Karten.

Dann existiert eine lokal endliche  Überdeckung $(V_k)_{k\in I}$ und $C^\infty$-Funktionen $f_k: M \to \MdR$ mit
\begin{enumerate}
\item Jedes $V_k$ liegt in einem $U_{\alpha=\alpha(k)}$.
\item $f_k\ge 0$ auf $\bar{V_k}$ und $f_k = 0$ auf dem Komplement von $\bar{V_k}$. (Das heißt: Der Träger von $f_k$ ist eine Teilmenge von $\bar{V_k}$)
\item $( \sum_{k\in I} f_k ) = 1$ für alle $p\in M$. Diese Summe ist immer endlich, da die Überdeckung lokal endlich ist. 
\end{enumerate}
\end{satz}
Lokal Endlich: {Für jeden Punkt $p\in M$ existiert eine Umgebung $U=U(p)$ mit $U\cap V_k \ne \emptyset$ für nur endlich viele $k\in I$}
\begin{beweis}[von Satz \ref{einszerl}]
siehe zum Beispiel: Gromoll-Klingenberg-Meyer, „Riemann’sche Geometrie im Großen“.
\end{beweis}

Für die Konstruktion einer Riemann’schen Metrik auf $M$ „verschmiert“ oder „glättet“ man jetzt alle im ersten Schritt konstruierten lokalen Riemann’schen Metriken $g_k : g_\alpha|_{V_k}$ wie folgt: 

Sei $p\in M$ beliebig und $u,v \in T_pM$. Setze 
\[
\ksp uv_p \da \sum_{k\in I} f_k(p) \cdot g_k\big(p\big) (u,v)
\]
Diese Summe ist endlich, da $f_k(p) \ne 0$ nur für endlich viele $k$.

Ist $\asp_p$ ein Skalarprodukt auf $T_pM$? 
\begin{itemize}
\item Symetrie und Bilinearität sind klar.
\item Positivität:
\[ \ksp uu_p = \sum_{k\in I} \underbrace{f_k(p)}_{\ge 0} \underbrace{g_k(u,u)}_{\ge 0} \ge 0\,.\]
\item Definitheit: Sei $\ksp uu_p = 0$, dann ist für jedes $k$ $f_k(p) g_k(u,u) = 0$. Wegen Punkt (3) von Satz \ref{einszerl} existiert mindestens ein $k_0\in I$, so dass $f_{k_0}(p) > 0$. Daher ist $g_{k_0}(u,u) = 0$, woraus $u=0$ folgt, da $g_{k_0} $positiv definit ist.
\end{itemize}

\end{beweis}


\section[Erste Anwendung von Riemann’schen Metriken: Länge von Kurven]{Anwendung: Länge von Kurven}

Sei $c: I \to M$ eine differenzierbare Kurve in einer Riemann’schen Mannigfaltigkeit $M$. Dann ist die Länge von $c$
\[
L(c) \da \int_I \sqrt{\ksp {c'(t)}{c'(t)}}dt  = \int _I \|c'(t)\|dt
\]
(Ein Spezialfall sind $C^\infty$-Kurven in $\MdR^n$ versehen mit Standardskalarprodukt)

Die Länge ist unabhängig von der Parametrisierung der Kurve und invariant unter Isometrien $\Phi : (M,\asp_1) \to (N,\asp_2)$, also $L(\Phi \circ c) = L(c)$, da $L(\Phi \circ c) = \int_I\|(\Phi \circ c)'\|_2dt = \int_I\|d\Phi_{c(t)}c'(t)\|_2dt \gleichnach{$\Phi$ iso.} \int_I\|c'\|_1dt = L(c)$.


\chapter{Affine Zusammenhänge und Parallelverschiebung}

\section{Motivation}

In $\MdR^n$ kann man Tangentialräume in verschiedenen Punkten vergleichen:
Die Tangentialräume von $x$ und $y$ sind $T_x\MdR^n = \{x\}\times \MdR^n \simeq \MdR^n$ und $T_y\MdR^n \simeq \MdR^n$. Es gibt dann eine Translation (Parallelverschiebung) $T_{y-x}:T_x\MdR^n \to T_y\MdR^n$; $(x,y)\mapsto (T_{y-x}(x), v)$, wobei $T_{y-x}(x) = x + (y-x) = y$.

Die Situation für Mannigfaltigkeiten ist lokal die gleiche: Ist $(U,\varphi)$ eine Karten, so gilt $TU \simeq U\times\MdR^n$ (vergleiche Basis-Satz, Satz \ref{basissatz}). Ist $p,q\in U$, so gilt: ($[\ldots]$ affine Hülle)
\[ T_pM = \left[ \ptv1,\ldots, \ptv n \right] \text{ und } T_qM = \left[ 
\qtv1 ,\ldots, \qtv n
 \right] \]
 Die Parallelverschiebung $T_pM \to T_qM$ bildet jetzt 
$v=\sum a_i \ptv i$ auf $\bar v = \sum a_i \qtv i$ ab.

Der globale Vergleich von Tangentialräumen erfordert jedoch eine Zusatzstruktur („Fernparallelismus“)

In der Flächentheorie realsisiert man die Parallelverschiebung via Kovariante Ableitung: Ist $c$ eine Flächenkurve der Fläche $F$, so ist $\frac D {dt} c'$ die orthogonale Projektion von $c''$  in die Tangentialebene $T_{c(t)}F$. Die Geodätischen in $F$ (die „verallgemeinerten Geraden“) sind definiert als Lösungen von $\frac D {dt} c' = 0$.

\section{Affine Zusammenhänge}

\begin{definition}[Affiner Zusammenhang]
\index{Affiner Zusammenhang}
\index{Zusammenhang!affiner}
Ein Affiner Zusammenhang $D$ auf einer differenzierbaren Mannigfaltigkeit $M$ ist eine Abbildung
\[
D: 
\begin{aligned}
\V M \times \V M &\to \V M \\
(X,Y) &\mapsto D_XY
\end{aligned}
\]
so dass für alle $X,Y,Z \in \V M$ und $f,g \in C^\infty M$ gilt:
\begin{enumerate}[(Z1)]
\item $D_{fX + gY} Z = fD_XZ + gD_YZ$
\item $D_X(Y + Z) = D_X Y + D_X Y$
\item $D_X(fY) = fD_XY + (Xf)Y$
\end{enumerate}
\end{definition}

\begin{beispiele}
\item Flächentheorie: $D_X Y \da Y_T'$
\item In $\MdR^n$: $X= \sum a_i \frac\partial{\partial x^i}$, $\sum b_i \frac\partial{\partial x^i}$, $D_XY\da \sum X(b_i) \frac\partial{\partial x^i}$
\end{beispiele}

$D$ ist ein lokaler Begriff: Wähle Karte $(U,\varphi)$ mit Basisfelder $X_i=\frac\partial{\partial x^i}$. $X,Y\in \V U$: $X=\sum_{i=1}^n v^iX_i$, $Y=\sum_{j=1}^n w^j X_j$. Dann: 
\begin{align*}
D_XY &= D_{\sum_i v^iX_i}(\sum_jw^jX_j) \\
     &\gleichnach{\clap{(Z1)}} \sum_i v^i D_{X_i}(\sum_jw^jX_j) \\
     &\gleichnach{\clap{(Z2)}} \sum_i v^i \sum_j D_{X_i} (w^j X_j) \\
     &\gleichnach{\clap{(Z3)}} \sum_{i,j} v^iw^j D_{X_i}X_j + \sum_{i,j} v^i X_i(w^j)X_j
\end{align*}
wobei $D_{X_i}X_j = \sum_{k=1}^n \Gamma^k_{ij} X_k$ (diese Darstellung existiert wegen dem Basissatz \ref{basissatz}) für lokal definierte $C^\infty$-Funktionen $\Gamma_{ij}^k: U \to \MdR$ (Christoffel-Symbole)\label{Christoffel-Symbole}.

Wir haben also:
\[
D_XY = \sum_{k=1}^n (\sum_{i,j=1}^n v^i w^j \Gamma_{ij}^k + X(w^k))X_k
\]
Die Formel zeigt, dass $D_XY(p)$ bestimmt ist durch $v^i(p)$, $w^j(p)$ und $X_p(w^k)$ (und $\Gamma_{ij}^k$). Insbesondere braucht man das Vektorfeld $Y$ (bzw. $w^k$) nur „in Richtung $X$“ zu kennen.

Wir folgern: Man kann Vektorfelder längs einer Kurve in Richtung dieser Kurve ableiten: Falls $Y$ ein Vektorfeld ist längs $c$ (also $Y\big(c(t)\big) = \sum_{i=1}^n w^k(t)X_k\big(c(t)\big)$), dann ist
\[
D_{c'}Y \da \sum_{k=1}^n\Big(\sum_{ij}^n {x^i}'(t) w^j(t) \Gamma_{ij}^k\big(c(t)\big) + {w^k}'(t)\Big)
\]
(wobei $\varphi \circ c(t) = \big(x^1(t),\ldots, x^n(t)\big)$ und damit $c' = \sum {x^i}'X_i$)

\begin{definition}
Ein Vektorfeld $Y$ längs einer Kurve $c$ heißt parallel\index{paralleles Vektorfeld} bezüglich einem affinen Zusammenhang $D$, falls $D_{c'}Y = 0$.
\end{definition}

\begin{beispiele}
\item Im $\MdR^n$ haben wir für ein paralleles Vektorfeld $Y$, dass $D_{c'}Y = \sum_{j=1}^n {w^j}'X_j = \sum_{j=1}^n 0 X_j= 0$, da bei Vektorfeldern in $\MdR^n$ parallel und konstant gleichwertig ist.
\item Ein Vektorfeld entlang eines Klein-Kreises der Sphäre ist nicht parallel. (Durch Skizze motiviert). Ein Vektorfeld entlang eines Groß-Kreises ist jedoch parallel, da $c''$ orthogonal zum Groß-Kreis zum Mittelpunkt zeigt, die Projektion auf die Sphäre also 0 ist.
\end{beispiele}

Später werden wir sehen, dass Geodätische (Kurven mit $D_{c'}c' = 0$) Geraden verallgemeinert.

\begin{satz}[Eindeutigkeit der Parallelverschiebung]
\label{eindpara}
Sei $M$ eine differenzierbare Mannigfaltigkeit mit affinem Zusammenhang $D$. Sei $c: I=[a,b] \to M$ eine differenzierbare Kurve und $v_o \in T_{c(a)}M$. Dann existiert genau ein paralleles Vektorfeld $V$ längs $c$ mit $V\big(c(a)\big)= v_0$.
\end{satz}

\begin{definition}
Der Vektor $V(t)$ (aus Satz \ref{eindpara}) heißt der längs $c$ parallel verschobene Vektor $V_0$. Die Abbildung
\[
c\|_a^t: 
\begin{aligned}
T_{c(a)}M &\to T_{c(t)}M \\
v_0&\to V(t)
\end{aligned}
\]
heißt Parallelverschiebung.
\end{definition}

\begin{beweis}
Im ersten Schritt betrachten wir die Situation lokal. Sei $t_1\in I$, so dass $c([a,t_1]) \subset U$ (Kartengebiet um $c(a)$). In der Karte $(U,\varphi)$ ist die Definitionsgleichung $D_{c'}V=0$ äquivalent zu:
\[
\sum_k \bigg(\frac{dv^k}{dt} + \sum_{i,j} \frac{dx^i}{dt}v^j \Gamma_{ij}^k \bigg) X_k = 0
\]
wobei $V = \sum_{i=1}^n v^i X_i$, $X_i = \frac\partial{\partial x^i}$, $\varphi \circ c(t) = \big(x^1(t),\ldots,x^n(t)\big)$, $c'(t) = \sum_i \frac{dx^i}{dt}(t)X_i\big(c(t)\big)$. Das heißt wir haben ein System von $n$ linearen Differentialgleichungen 1. Ordnung in $v_k(t)$:
\[
0 = \frac{dv^k}{dt} + \sum_{i,j} \Gamma_{ij}^k \frac{dx^i}{dt} v^j, \qquad k=1,\ldots,n
\]
Dieses System hat zu gegebenen Anfangsbedingungen $v(a)=v_0 = (v^1(a),\ldots,v^n(a))$ genau eine Lösung für alle $t\in [a,t_1]$. Dann existiert eindeutig ein Parallelfeld $V$ längs $c([a,t_1])$ mit $V(a)=v_0$.

Im zweiten Schritt sei $t_2\in I$ beliebig. Das Segment $c([a,t_2])$ ist kompakt in $M$ und kann daher mit endlich vielen Karten überdeckt werden. In jeder Karte existiert ein $V$ und ist eindeutig (nach Schritt 1). Daraus folgt, dass $V$ global eindeutig existiert auf $c([a,t_2])$ für beliebige $t_2$.
\end{beweis}

\section{Der Levi-Civita-Zusammenhang}

\paragraph{Motivation:} Ein Parallelfeld im Euklidischen Raum $(\MdR^n,\asp)$ ist eine Isometrie. 

\begin{definition}
Ein affiner Zusammenhang $D$ auf einer Riemann’schen Mannigfaltigkeit $(M,\asp)$ heißt verträglich \index{Zusammenhang!verträglicher}\index{verträglicher Zusammenhang}mit der Riemann’schen Struktur $\asp$ falls für jede differenzierbare Kurve $c:I\to M$ und jedes Paar von parallelen Vektorfeldern $V_1,V_2$ längs $c$ gilt:
\[
\lsp {V_1\big(c(t)\big)}{V_2\big(c(t)\big)}_{c(t)} \text{ ist für alle $t\in I$ konstant.}
\]
Das heißt dass die Parallelverschiebung $c\|_{t_1}^{t_2}: T_{c(t_1)}M \to T_{c(t_2)}M$ eine lineare Isometrie ist.
\end{definition}

\begin{satz}[Äquivalente Formulierung der Verträglichkeit]
Sei $(M,\asp)$ eine Riemann’sche Mannigfaltigkeit. Ein affiner Zusammenhang $D$ ist verträglich \index{Zusammenhang!verträglicher}\index{verträglicher Zusammenhang}mit $\asp$ genau dann, wenn für beliebige Vektorfelder $V$, $W$ längs einer beliebigen Kurve $c:I\to M$ für alle $t\in I$ gilt:
\[
\frac d{dt} \ksp {V(t)}{W(t)}_{c(t)} = \ksp {D_{c'}V}W_{c(t)} + \ksp V {D_{c'}W}_{c(t)} \quad (*)
\]
\end{satz}

\begin{beweis}
$(*) \folgt$ verträglich: $V$, $W$ parallel ist äquivalent zu $D_{c'}V = D_{c'}W = 0$, also $\frac d{dt} \ksp {V(t)}{W(t)}_{c(t)} = 0$, also verträglich.

Umgekehrt gilt: Sei $D$ verträglich, wir haben also eine Parallelverschiebung, die Isometrie ist. Wähle eine Orthonormalbasis $\{P_1(t_0),\ldots,P_n(t_0)\}$ von $T_{c(t_0)}M$. Mittels der der Parallelverschiebung erhalten wir wieder für alle $t\in I$ eine Orthonormalbasis $\{P_1(t),\ldots,P_n(t)\}$ von $T_{c(t)}M$. Wir können schreiben: 
\[ V(t) = \sum_{i=1}^n v_i(t) P_i(t)\quad \text{sowie}\quad
   W(t) = \sum_{i=1}^n w_i(t) P_i(t) \]
wobei $v_i, w_i \in C^\infty$. Also:
\[
D_{c'}V = \sum_{i=1}^n \underbrace{c'(v_i)}_{v_i'} P_i + \sum_{i=1}^n v_i \underbrace{D_iP_i}_{=0}
\]
das heißt: $D_{c'}V = \sum_{i=1}^n v_i' P_i$ und $D_{c'}W = \sum_{i=1}^n w_i' P_i$. Wir wollen zeigen, dass $(*)$ gilt. Die rechte Seite ist:
\begin{align*}
\ksp{D_{c'}V}W + \ksp V{D_{c'}W} &= \lsp {\sum v_i' P_i} {\sum w_jP_j} + \lsp {\sum v_i P_i} {\sum w_j'P_j} \\
&= \sum_{i,j} \left( v_i' w_j \ksp {P_i}{P_j} + v_iw_j' \ksp {P_1}{P_j}\right) \\
&= \sum_{i,j} \left( v_i' w_j \delta_{ij} + v_iw_j' \delta_{ij}\right) \\
&= \sum_{i=1}^n (v_i' w_j + v_i w_j')\\
&= \frac d{dt} \left( \sum_{i=1}^n v_i w_j \right)
\intertext{Die linke Seite ist:}
\frac d{dt} \ksp V W &= \frac d{dt} \ksp {\sum_i v_iP_i}{\sum_j w_j P_j} \\
&= \frac d{dt} \left( \sum_{ij} v_i w_j \ksp {P_i} {P_j} \right) \\
&= \frac d{dt} \left( \sum_{i=1}^n v_i w_j \right)
\end{align*}
\end{beweis}

Die Frage ist jetzt, ob zu einer gegebener Riemann’schen Struktur ein verträglicher Zusammenhang existiert.

\begin{definition}
Ein affiner Zusammenhang $D$ heißt symmetrisch\index{symmetrischer Zusammenhang} (oder torsionsfrei) falls für alle $X,Y\in \V M$:
\[
T(X,Y) \da D_XY - D_YX - [X,Y] = 0
\]
\end{definition}

\begin{bemerkung}
In lokalen Koordinaten $(U,\varphi)$ gilt für $D$ symmetrisch und Basisfelder $X_i = \frac\partial{\partial x^i}$:
\[
D_{X_i} X_j - D_{X_j}X_i  = [X_i,Y_j] = \left[\frac{\partial}{\partial x_i},\frac{\partial}{\partial x_i}\right] = 0
\]
da $\left[\frac{\partial}{\partial x_i},\frac{\partial}{\partial x_i}\right] f = \frac{\partial}{\partial x_i} \frac{\partial}{\partial x_j} f - \frac{\partial}{\partial x_j} \frac{\partial}{\partial x_i} f = 0$ wegen $f\in C^\infty$ und Vertauschbarkeit der partiellen Ableitungen. Weiter gilt:
\[
D_{X_i} X_j - D_{X_j}X_i  = \sum_k \Gamma_{ij}^kX_k - \sum_k \Gamma_{ji}^kX_k = \sum_k( \Gamma_{ij}^k - \Gamma_{ji}^k) X_k \folgt \Gamma_{ij}^k = \Gamma_{ji}^k
\]
\end{bemerkung}

\begin{satz}[Levi-Civita-Zusammenhang]
Auf jeder Riemann’schen Mannigfaltigkeit $(M,\asp)$ existiert genau ein affiner Zusammenhang $D$, so dass gilt:
\begin{enumerate}
\item $D$ ist symmetrisch
\item $D$ ist verträgliche mit $\asp$
\end{enumerate}
\end{satz}

Dieser eindeutige Zusammenhang $D$ heißt Levi-Civita-Zusammenhang\index{Levi-Civita-Zusammenhang} von $M$ bezüglich $\asp$.

\begin{beweis}
Wir nehmen an, dass ein solches $D$ existiert. Was sin die Eigenschaften?

$D$ verträglich:
\begin{align*}
X\ksp Y Z &= \ksp {D_X Y} Z  + \ksp Y {D_XZ} \\
\intertext{Der Trick ist jetzt, die Gleichung zyklisch zu vertauschen:}
Y\ksp Z X &= \ksp {D_Y Z} X  + \ksp Z {D_YX} \\
-Z\ksp X Y &= -\ksp {D_Z X} Y  - \ksp X {D_ZY} \\
\intertext{Summe der drei Gleichungen}
X \ksp Y Z + Y \ksp Z X - Z \ksp X Y &=
\ksp {[X,Z]}Y + \ksp {[X,Y]} Z + 2 \ksp Z {D_YX} + \ksp {[Y,Z]} X
\end{align*}
Wir erhalten die Kozul-Formel\index{Kozul-Formel}
\[
\ksp Z {D_YX} = \frac 12 \Big( X \ksp Y Z + Y \ksp Z X  - Z \ksp X Y -\ksp {[X,Z]}Y - \ksp {[X,Y]} Z  - \ksp {[Y,Z]} X \Big) \mathrlap{\quad (*)}
\]
Diese Formel zeigt, dass $D$ eindeutig durch die Riemann’sche Struktur $\asp$ bestimmt ist, denn seien $D$ und $\tilde D$ zwei affine Zusammenhänge, die (1) und (2) erfüllen, dann gilt $(*)$ für beide, also $\ksp Z {D_YX} = \ksp Z {\tilde D_Y X}$ für alle $X,Y,Z\in \V M$, was heißt dass $\ksp {D_YX - \tilde D_YX} Z = 0$, was heißt das $D_YX - \tilde D_YX=0$. Also ist $D=\tilde D$.

Die Existenz folgt daraus, dass man $D$ durch $(*)$ definieren kann.
\end{beweis}

\paragraph{Lokale Form von $D$} Gegeben eine Karte $(U,\varphi)$ mit Basisfelder $X_i \da \frac \partial {\partial x^i}$, $i=1,\ldots,n$, auf $U$.
Wir haben $g_{ij}=\ksp {X_i} {X_j}$, $D_{X_i}X_j = \sum_{k=1}^n \Gamma_{ij}^k X_k$, $[X_i, X_j] = 0$. Kozulformel:
\[
\ksp {X_k} {D_{X_i}X_j} = \frac 1 2 \Big( \frac {\partial g_{ik}}{\partial x^j} + \frac {\partial g_{jk}}{\partial x^i} - \frac {\partial g_{ij}}{\partial x^k} + 0 \Big) = \lsp {X_k} {\sum_{l=1}^n \Gamma_{ij}^l X_l} = \sum_{l=1}^n \Gamma_{ij}^l g_{kl}
\]
$[g_{kl}]$ hat inverse Matrix $[g^{mk}]$. Damit
\[
\Gamma_{ij}^m = \frac 1 2 \sum_{k=1}^n g^{mk} \Big( \frac {\partial g_{ik}}{\partial x^j} + \frac {\partial g_{jk}}{\partial x^i} - \frac {\partial g_{ij}}{\partial x^k} \Big)\,.
\]
Dieser Ausdruck zeigt nochmals: Levi-Civita-Zusammenhang ist eindeutig durch die Metrik bestimmt.

\begin{beispiel}
Im Euklidischer Raum ($\MdR^n$, Standardskalarprodukt) ist $g_{ij}=\delta_{ij}$, also $\Gamma_{ij}^k = 0$. Also: Der kanonische Zusammenhang ist der Levi-Civita-Zusammenhang.
\end{beispiel}

\chapter{Geodätische Linien}

Gegeben ist eine Riemann’sche Mannigfaltigkeit ($M,\asp)$ mit Levi-Civita-Zusammenhang $D$. Das Ziel ist es, ein Analogon für Geraden zu finden. Mögliche Charakterisierung von Geraden in der Euklidischen Geometrie sind:
\begin{itemize}
\item kürzeste Verbindung zweier Punkte (Variationseigenschaft).
\item Kurven $c(t)$ (mit Bogenlänge parametrisiert) mit $c''(t)=0$ (Differentialgleichung).
\end{itemize}

\section{Definition von Geodätischen}

\index{Geodätische}
Eine Geodätische (Linie) in $(M,\asp)$ ist eine differenzierbare Kurve $\gamma: I \to M$ so dass gilt: $D_{\gamma'(t)}\gamma'(t) = 0$ für alle $t\in I$. (Das heißt: Geodätische sind autoparallele Vektorfelder).

Das Tangentialvektorfeld ist parallel. $\gamma' = \frac {d\gamma}{dt} \da d\gamma(\frac\partial{\partial t})$.

Folgerungen aus der Definition:
\begin{enumerate}
\item $\|\gamma'(t)\|_{\gamma(t)}$ ist konstant.

\begin{beweis}
$\|\gamma'\|^2 = \ksp {\gamma'}{\gamma'}$. Also $\frac d{dt} \|\gamma'\|^2 = \frac d{dt} \ksp {\gamma'}{\gamma'} = \ksp{D_{\gamma'}\gamma'}{\gamma'} + \ksp {\gamma'}{D_{\gamma'}\gamma'} = 0$
\end{beweis}
Ein (entarteter) Spezialfall ist $\gamma(t)$ konstant $p\in M$.
\item Eine Geodätische ist proportional zur Bogenlänge parametrisiert:
\[
s(\gamma) \da \int_a^t \|\gamma'(\tau)\| d\tau = k |t-a|
\]
Ist $k=1$ so spricht man von einer normalen Geodätischen\index{normale Geodätische} sowie von isometrischen Kopien von Intervallen.
\item Ob eine Kurve eine „Geodätische“ ist hängt von der Parametrisierung ab, nicht nur vom Bild $\gamma(I)\subset M$.
\begin{beispiel}
$\gamma_1(t) = (t,0)$ ist eine Geodätische, aber $\gamma_2(t) = (t^3,0)$ nicht, da $\|\gamma_2'\|=3t^2$ nicht konstant ist.
\end{beispiel}
\end{enumerate}

\section{Lokale Darstellung und Differentialgleichung für Geodätische}
Sei $\gamma: I\to M$ eine Geodätische in $(M,\asp)$ und $(U,\varphi)$ eine Karte um $\gamma(t_0)$ mit $\varphi \circ \gamma(t) = (x^1(t),\ldots, x^n(t))$.

Dann ist $\gamma'(t) = \sum_{i=1}^n x_i' (t) \left. \frac \partial{\partial x^i}\right|_{\gamma(t)}$. Die allgemeine Formel für Parallelfelder imn lokalen Koordinaten (vgl 3.2) ergibt:
\[
0 = D_{\gamma '}\gamma' = \sum_{k=1}^n \Big(x''^k + \sum_{i,j=1}^n \Gamma_{ij}^k x'^i x'^j\Big) \frac{\partial}{\partial x^k}
\]
lokal gilt also: $D_{\gamma '}\gamma'= 0$ ist äquivalent zu dem System von $n$ Differentialgleichung 2. Ordnung
\[
x''^k(t) = - \sum_{i,j=1}^n \Gamma_{ij}^k(x(t)) x'^i(t) x'^j(t)\mathrlap{\qquad (1)}
\]

\section{Das Geodätische Vektorfeld auf $TM$}
Das System 2. Ordnung (1) ist äquivalent zu System 1. Ordnung:
\[
\begin{aligned}
x'^k &\ad y^k \\
y'^k &= -\sum_{i,j=1}^n \Gamma_{ij}^k(x(t)) y^i(t) y^j(t)
\end{aligned}\mathrlap{\qquad (2)}
\]
Was ist die Interpretation der von $y^i$ in der Mannigfaltigkeit? Die Geodätische $t\mapsto \gamma(t)$ in $M$ definiert differenzierbare Kurve $t \mapsto \big(\gamma(t), \gamma'(t)\big)$ in $TM = \{(p,v) \mid p\in M, v\in T_pM\}$. Lokale Koordinaten für $TM$: Sei $(U,\varphi)$ eine Karte in $M$, $TU \cong U\times \MdR^n$ (nach Basissatz). Dies ergibt eine Darstellung von  $(p,v)$ als $(x^1,\ldots,x^n,y^1,\ldots,x^n)$ mit $v=\sum y^i \frac\partial{\partial x^i}$. Speziell gilt $\big(\gamma(t),\gamma(t)\big) \to \big(x^1(t),\ldots,x^n(t),x'^1(t), \ldots x'^n(t)\big)$.

\begin{lemma}
Es existiert genau ein Vektorfeld $G\in \mathcal V(TM)$ auf $TM$ dessen Integralkurven (vergleiche 1.7) von der Form $\tilde \gamma(t) = \big(\gamma(t), \gamma'(t)\big)$ sind, wobei $\gamma(t)$ jeweils eine Geodätische in $M$ ist.
\end{lemma}
\begin{beweis}
\begin{enumerate}[(a)]
\item Eindeutigkeit (unter der Annahme der Existenz): Die Integralkurven von $G$ auf $TU$ sind nach Voraussetzung gegeben durch $\tilde \gamma(t) = \big( \gamma(t), \gamma'(t)\big)$. Diese Kurve ist aber Lösung von (2), also zu gegebener Anfangsbedingung eindeutig:
\[
\tilde G\big(\tilde \gamma(t)\big) = \tilde\gamma'(t) = G\big(\tilde\gamma(t)\big)
\]
\item Existenz: Wir definieren die Komponenten von $G$ bezüglich Basisfelder lokal durch (2). Wegen (a) ist $G$ auf ganz $TM$ eindeutig.
\end{enumerate}
\end{beweis}

\begin{definition}
$G$ heißt geodätisches Vektorfeld\index{geodätisches Vektorfeld}\index{Vektorfeld!geodätisches} auf $M$. ($G(p,v) \in T_{(p,v)}(TM) \subset T(TM)$)
\end{definition}

\begin{satz}[Lokale Integralkurve]
Für jede Karte $U$ und $p\in M$ existiert ein offenes $O\in TM$ mit $(p,o)\in O$ eine Zahl $\delta = \delta(p)$ und eine $C^\infty$-Abbildung $f: (-\delta,\delta)\times O \to TU\subset TM$, so dass $t \mapsto f\big(t,(q,v)\big)$ die eindeutige Integralkurve von $G$ ist mit $f\big(0,(q,v)\big)=(q,v)$ für alle $(q,v)\in O$.
\label{intkurv}
\end{satz}

\begin{beweis}
Nach 1.7 gilt lokal, dass Integralkurven von Vektorfeldern den Lösungen eines Systems von gewöhnlichen Differentialgleichung entspricht.
Die Existenz und Eindeutigkeit im Satz \ref{intkurv} folgt dann aus dem ensprechenden Satz über Existenz und Eindeutigkeit von Lösungen eines Differentialgleichungssystems zu gegebenen Anfangsbedingungen.

Dass $f$ differenzierbar ist folgt aus der Tatsache, dass Lösungen von Differentialgleichungen (gewöhnlich, 1. Ordnung) differenzierbar von den Anfangsbedingungen abhängen (vergleiche zum Beispiel Arnold, „Gewöhnliche Differenzialgleichungen“, Gromoll-Klingenberg-Meryer, „Differenzialgleichungen im Großen“, S.275)

Sei $\pi: TM\to M$; $(q,v)\mapsto q$ die kanonische Projektion. Die offene Menge $O\subset TM$ im Satz kann man wie folgt wählen: Es existiert ein $V\subset U$ (offene Umgebung von $p$) und $\ep_1>0$, so dass $O=\{(q,v)\in TU \cong U\times \MdR^n \mid q \in V, v\in T_qM, \|v\|<\ep_1\}$
\end{beweis}

Aus Satz \ref{intkurv} folgt dann:
\begin{satz}[Lokale Geodätische]
\label{lokgeo}
Zu $p\in M$ und einer Karte $U$ um $p$ existiert eine offene Menge $V$ von $p$, Zahlen $\delta=\delta(p)<0$, $\ep_1>0$  und eine $C^\infty$-Abbildung 
$\gamma \da \pi \circ f : (-\delta, \delta) \times O \to M$ (mit $O$ definiert wie oben), so dass die Kurve $t\mapsto \gamma(t,q,v)$ die  eindeutige Geodätische in $M$ ist mit $\gamma(0,q,v)=q$ und $\gamma'(0)=v$.
\end{satz}

\section{Die Expontential-Abbildung}

\begin{lemma}[Homogenität von Geodätischen]
\label{homgeo}
Sei $a\in \MdR$, $a>0$. Falls die Geodätische $\gamma(t,q,v)$ auf $(-\delta,\delta)$ definiert ist, so ist die Geodätische $\gamma(t,q,a\cdot v)$ auf $(-\frac \delta a, \frac \delta a)$ definiert und es gilt $\gamma(t,q,a\cdot v) = \gamma(a\cdot t, q ,v)$.
\end{lemma}

\begin{beweis}
Betrachte die Kurve $h: (-\frac \delta a, \frac \delta a) \to M$; $t\mapsto \gamma(at,q,v)$. Es gilt: $h(0)= \delta(0,q,v) = q$ sowie $h'(0) = \frac d{dt}\gamma(at,q,v)|_{t_0} = a \gamma'(0,q,v) = av$. Weiter ist $h'(t) = a\gamma'(at,q,v)$, also $D_{h'}h' = D_{a\gamma'}a\gamma' = a^2 D_{\gamma'}\gamma' = a^2\cdot 0 = 0.$

Das heißt: $h$ is eine Geodätische mit $h(0) = q$, $h'(0)=av$. Aus der Eindeutigkeit von Geodätischen (Satz \ref{lokgeo}) folgt, dass $\gamma(at,q,v) = h(t) = \gamma(t,q,av)$.
\end{beweis}

Nach Satz \ref{lokgeo} ist (für $q\in V=V(p)$) $\gamma(t,q,v)$ definiert für $|t|<\delta=\delta(p)$ und $\|v\| < \ep_1=\ep_1(p)$. Mit Lemma \ref{homgeo} folgt jetzt, dass $\gamma(t,q,\frac \delta 2 v)$ für $|t|<2$ definiert ist. Dann ist die Geodätische $\gamma(t,q,w)$ definiert für $q\in V, |t|<2$ und $w\in T_qM$, $\|w\|<\ep$. Damit ist gezeigt: 
\begin{satz}
\label{eindgeo}
Für jeden Punkt $p\in M$ existiert eine Umgebung $V$ von $p$, $\ep = \ep(p)>0$ und eine differenzierbare Abbildung:
\[
\gamma: (-2,2) \times \{(q,v)\in TM \mid q\in V, v\in T_qM, \|v\| < \ep \}
\]
so dass für ein festes $(q,v)$ die Abbildung $t\mapsto \gamma(t,q,v)$ die eindeutige Geodätische in $M$ ist mit Anfangsbedingung $\gamma(0,q,v) = q$, $\gamma'(0,q,v) = v$.
\end{satz}

Sei $p\in M$ und $O$ wie in Satz \ref{eindgeo}.
\begin{definition}
Die Expontential-Abbildung\index{Expontential-Abbildung} (auf $O$) ist:
\begin{align*}
\exp :\quad &O \subset TM \to M\\
\exp(q,v) &\da \gamma(1,q,v) \\
&= \gamma(1,q,\|v\| \frac v {\|v\|}) \\
&= \gamma(\|v\|,q,\frac v {\|v\|})
\end{align*}
\end{definition}

\begin{bemerkungen}
\item $\exp$ ist differenzierbar, da $\gamma$ differenzierbar ist (vergleiche Satz \ref{lokgeo})
\item Meistens benutzt man die Einschränkung von $\exp$ auf \emph{einen} Tangentialraum:
\[
\exp_p \da \exp(p,\cdot): B_\ep(0) (\subset TM) \to M
\]
Wobei $B_\ep(0)$ ein offener Ball mit Radius $\ep$ und $0$ ist.
\end{bemerkungen}

\begin{satz}
Für jeden Punkt $p$ einer $n$-dimensionalen differenzierbaren Riemann’schen Mannigfaltigkeit existiert ein $r=r(p)>0$, so dass die Abbildung $\exp_p: B_r(0)\subset T_pM \to \exp_p(B_r(0)) \subset M$ (mit $B_r(0) \da \{v\in T_pM \mid \|v\|<r\}$) ein Diffeomorphismus auf die offene Umgebung $V\da \exp_p(B_r(0))$ von $p$ ist.
\end{satz}

\begin{beweis}
Wir benutzen den Umkehrsatz für Mannigfaltigkeit (Satz \ref{invfunk}). Zu zeigen ist: $d\exp_p|_0: T_p(B_r(0)) \cong T_pM \to T_{\exp_p(0)}M = T_pM$ ist ein Vektorraum-Isomophismus.

Wähle dazu die Kurve $c(t) = tv$ (mit $c(0)=0$, $c'(0)=v$). Dann ist: $d\exp_p|_0(v) = \frac d{dt}|_0 (\exp _p \circ c)(t) = \frac d{dt}|_0 \exp_p(tv) = \frac d{dt}|_0 \gamma(1,p,tv) = \frac d{dt}|_0 \gamma(t,p,v) = v$. Also ist $d\exp_p|_0 = \text{id}$, und damit ein Vektorraum-Isomophismus.
\end{beweis}

\begin{definition}
Eine geodätische Normalumgebung\index{geodätische Normalumgebung}  von $p\in M$ ist eine Umgebung $U$ von $p$, so dass $\exp_p: V\to U$ ein Diffeomorphismus ist.

$B_r(p) \da \exp_p(B_r(0))$ heißt geodätischer\index{geodätischer Ball} Ball vom Radius $r$.

Die Koordinatenfunktionen der Karte $\exp_p^{-1} : U = \exp_p(V) \to V \otm T_pM \cong \MdR^n$ heißen geodätische Normalkoordinaten\index{geodätische Normalkoordinaten}.
\end{definition}

\begin{beispiele}
\item Im $\MdR^n$ mit Standardskalarprodukt sind die Geodätischen gerade die Geraden (mit Bogenlänge parametrisiert). Also, da $T_p\MdR^n \cong \MdR^n$, $\exp_0: \MdR^n \to \MdR^n$ ist die Identität.
\item In $M=S^n$ mit der von $\MdR^{n+1}$ induzierten Metrik sind die Geodätischen die Großkreise (mit Bogenlänge parametrisiert, siehe 4.5). Durch Skizzen für die Fälle $n=1,2$ motiviert: $\exp_0$ ist ein Diffeomorphismus auf $B_\pi(0)$.
\item Der Name „Exponentialabbildung“ kommt aus der Lie-Theorie. $G = U(1) \cong S^1$ sind die unitäre $(1\times 1)$-Matrizen, dann steht der Tangentialraum am Punkt 1 senkrecht, also $T_1U(1) \cong i\MdR$, und daher $\exp_0(t) = e^{t}$.
\item In der Lie-Gruppe $G=(\MdR_{>0},\cdot)$ ist $T_1\MdR \cong (\MdR,+)$. Hier ist $\exp_1(t) = e^t$.
\item $G=O(n)=\{A\in \MdR^{n\times n} \mid A A^t = E \}$. Hier ist $T_EO(n)$ die Menge der schiefsymetrischen Matrizen. Für $B\in T_EO(n)$ setze $A \da \exp(sB) \da E + sB + \frac{s^2}{2}B^2 + \frac{s^3}{3!}B^3 + \cdots$. Es gilt: $A\in O(n)$.
\end{beispiele}

\section{Minimaleigenschaft von Geodätischen}

Zur technischen Vorbereitung benötigen wir „Vektorfelder längs Flächen“. Sei $A$ eine zusammenhängende Menge in $\MdR^2$ mit stückweise differenzierbarem Rand und $M$ sei eine differenzierbare Mannigfaltigkeit. Eine parametrisierte Fläche\index{paramterisierte Fläche} in $M$ ist eine differenzierbare Abbildung $f: A (\subset \MdR^2) \to M$; $(u,v) \mapsto f(u,v)$.

Ein Vektorfeld längs $f$ ist eine differenzierbare Abbildung $V: A \to TM$ mit $V(u,v) \in T_{f(u,v)}M$. Die Parameterlinien $f(u,v_0)$ bzw. $f(u_0,v)$ mit $v_0$ bzw. $u_0$ fest definieren die „Tangential-Vektorfelder“ 
\[
\frac {\partial f}{\partial u}(u,v) \da df|_{(u,v)} \Big(\left.\frac\partial{\partial u}\right|_{(u,v)}\Big)
\quad\text{sowie}\quad
\frac {\partial f}{\partial v}(u,v) \da df|_{(u,v)} \Big(\left.\frac\partial{\partial v}\right|_{(u,v)}\Big)
\]

Weiter definieren wir die kovariante Ableitung für ein Vektorfeld $V$ längs $f$ wie folgt:
\[
\frac{DV}{\partial u}(u,v_0) \da D_{\frac{\partial f}{\partial u}(u,v_0)} V(u,v_0)
\quad\text{sowie}\quad
\frac{DV}{\partial v}(u_0,v) \da D_{\frac{\partial f}{\partial v}(u_0,v)} V(u_0,v)
\]


\begin{lemma}[Symmetrie]
\label{flaechsymm} Sei $M$ ein differenzierbare Mannigfaltigkeit und $D$ ein symmetrischer Zusammenhang auf $M$. Für eine parametrisierte Fläche $f: A \to M$ gilt:
\[
\frac D {\partial v} (\frac {\partial f}{\partial u}) =
\frac D {\partial u} (\frac {\partial f}{\partial v})
\]
\end{lemma}

\begin{beweis}
In lokalen Koordinaten $(U,\varphi)$ in der Umgebung eines Punktes von $f(A)\subset M$ sei $\varphi\circ f(u,v) = \big(x^1(u,v), \ldots, x^n(u,v)\big)$. Es gilt:
\begin{align*}
\frac D {\partial v} (\frac {\partial f}{\partial u}) &= \frac D{\partial v} (\sum_{i=1}^n \frac {\partial x^i}{\partial u} \frac{\partial}{\partial x^i}) \\
&= \sum_{i=1}^n \frac {\partial^2 x^i}{\partial v\partial u} \frac{\partial}{\partial x^i} + \sum_{i=1}^n \frac{\partial x^i}{\partial u} D_{\sum_j \frac{\partial x^j}{\partial v}\frac{\partial}{\partial x^j}} \frac \partial {\partial x^i} \\
&= \sum_{i=1}^n \frac {\partial^2 x^i}{\partial v\partial u} \frac{\partial}{\partial x^i} + \sum_{i=1,j}^n \frac{\partial x^i}{\partial u}\frac{\partial x^j}{\partial v}  D_{\frac{\partial}{\partial x^j}} \frac \partial {\partial x^i} \\
\intertext{Wegen der Symmetrie von D erhalten wir dann durch zurückrechnen}
&= \frac D {\partial u} (\frac {\partial f}{\partial v})
\end{align*}
\end{beweis}

\begin{satz}[Gauß-Lemma]
\label{gausslemma} Sei $(M,\asp)$ eine Riemann’sche Mannigfaltigkeit. Sei $p\in M$ und $v\in T_pM$ so dass $\exp_p(v)\ad q$ definiert ist. Für $w\in T_v(T_pM) \cong T_pM$ gilt:
\[
\lsp {d(\exp_p)|_v v} {d(\exp_p)|_v w}_q = \ksp v w_p
\]
\end{satz}
\begin{beweis}
Zerlege $w = w_T + w_N$, wobei $w_T$ die Komponente in Richtung $v$ und $w_N$ die dazu orthogonale Komponente ist (also $w_T = \ksp v w \frac v {\|v\|}$, $w_N = w- w_T$). Das Differential 
\[
d\exp_p|_v : T_v(T_pM) \cong T_pM \to T_{\exp_pv}M = T_qM
\]
ist linear. Also: 
\[
d\exp_p|_v (w_T + w_N) = d\exp_p|_v(w_T) + d\exp_p|_v(w_N)
\]

Es gilt zunächst
\[
d\exp_p|_v w_T = \left.\frac d{dt}\right|_0 \exp_p(v + tw_T) = \left.\frac d{dt}\right|_0 \exp_q(tw_T) = w_T
\]
sowie
\[
d\exp_p|_v v = \left.\frac d{dt}\right|_0 \exp_p(v + tv) = \left.\frac d{dt}\right|_0 \exp_q(tv) = v
\]
das heißt, das Gauß-Lemma gilt für $w=w_T$.

Ohne Einschränkung sei nun $w=w_N$. Nach Voraussetzung ist $q=\exp_p(v)$, also existiert ein $\ep>0$, so dass die Exponentialabbildung definiert ist für: $u\da tv(s)$ wobei $v(s)$ eine Kurve in $T_pM$ mit $v(0)=v$, $\|v(s)|$ konstant und $v'(0)=w\>(\bot v)$. Die Fläche $A\subset \MdR^2$ ist jetzt die Menge der $u$ für $0\le t\le 1$ und $-\ep < s < \ep$.

Betrachte jetzt die parametrisierte Fläche
\[
f: A \to M ; f(t,s) \da \exp_p(tv(s))
\]
Es gilt: $f(t,s_o)$ ist eine Geodätische für ein festes $s_0$, sowie $f(1,0) = \exp_p v = q$.

Wir haben für $t=1$
\[
\left.\frac{\partial f}{\partial s}\right|_{(1,0)} \gleichnach{Lemma \ref{lem3}} (d\exp_p)_v(v'(0)) = d\exp_p|_v (w)
\]
und für $s=0$
\[
\left.\frac{\partial f}{\partial t}\right|_{(1,0)} = v = d\exp_p|_v(v)
\]
Also ist zu zeigen:
\[
\lsp {\left.\frac{\partial f}{\partial s}\right|_{(1,0)} }{\left.\frac{\partial f}{\partial t}\right|_{(1,0)} }_q = 0 \mathrlap{\qquad (*)}
\]

Wir zeigen dazu zuerst: 
\[
\lsp {\left.\frac{\partial f}{\partial s}\right|_{(t,s)} }{\left.\frac{\partial f}{\partial t}\right|_{(t,s)} }_q
\]
ist unabhängig von $t$.
\begin{align*}
\frac\partial {\partial t} \lsp {\left.\frac{\partial f}{\partial s}\right|_{(t,s)} }{\left.\frac{\partial f}{\partial t}\right|_{(t,s)} }_q &\gleichnach{Verträglichkeit} 
\lsp {\frac D{\partial t}\frac{\partial f}{\partial s}}{\frac{\partial f}{\partial t}}_q + 
\lsp {\frac{\partial f}{\partial s}}{\underbrace{\frac D{\partial t}\frac{\partial f}{\partial t}}_{=0}}_q \\
&\gleichnach{Lemma \ref{lem3}} \lsp {\frac D{\partial s}\frac{\partial f}{\partial t}}{\frac{\partial f}{\partial t}} \\
&= \frac 1 2 \frac{\partial}{\partial s} \underbrace{\lsp {\frac{\partial f}{\partial t}}{\frac{\partial f}{\partial t}}}_{\mathclap{\text{konstant (Geodätische!)}}} = 0
\end{align*}
Es war $t$ beliebig, also wählen wir $t=0$.
\[
(*) \gdw 
\lsp {\left.\frac{\partial f}{\partial s}\right|_{(0,0)} }{\left.\frac{\partial f}{\partial t}\right|_{(0,0)} }_q = 0
\]
Aber für ein festes $t$ gilt:
\begin{align*}
\left.\frac{\partial f}{\partial s}\right|_{(t,0)} &= (d\exp_p)_{tv(s)} (tv'(s))\bigg|_{s=0} \\
&=d\exp_p|_{tv}(tw)
\end{align*}

\end{beweis}

\subsubsection*{Frage: Geodätische und Kürzeste}
Ein Segment $\gamma|_{[a,b]}$ einer Geodätischen $\gamma : I \to M
([a,b] \subset I)$ heißt minimierend, falls $L(\gamma|_{[a,b]})
\le L(c)$, wobei $c$ eine beliebige Kurve mit $c(a)=\gamma(a)$, $c(b)=
\gamma(b)$ und $L(.)$ die Länge ist.

\begin{satz}[Geodätische sind lokal minimierend]
\label{geolokmin}
Sei $p \in M$, U eine normale Umgebung von $p$ und $B \subset U$ ein normaler Ball mit Zentrum $p$. Sei $\gamma: [0,1] \to B$ ein geodätisches Segment mit $\gamma(0)=p$. Falls $c:[0,1] \to M$ eine beliebige, stückweise $C^\infty$ - Kurve mit $\gamma(0)=c(0)$ und $\gamma(1)=c(1)$ ist, dann gilt $L(\gamma) \le L(c)$, und falls $L(\gamma)=L(c)$, so ist $\gamma([0,1]) = c([0,1])$.
\end{satz}
\begin{beweis}
1. Fall: $c([0,1]) \subset B$. $\exp_p$ ist ein Diffeomorphismus, also können wir schreiben ("Polarkoordinaten"):
$c(t)=\exp_p(r(t)v(t))$, $t \in [0,1]$, wobei $v(t)$ ein Kurve in $T_pM$  ist mit $\|v(t)\|=1$ und $r:[0,1] \to \MdR^+$ stückweise differenzierbar. Sei $f(r,t) \da \exp_p(r\cdot v(t))$ eine parametrisierte Fläche in $B$, die $c$ enthält. Es gilt für fast alle $t$: 
\[ \frac{dc}{dt}=\frac{\partial f}{\partial r}r' + \frac{\partial f}{\partial t}\,.
\]
Nach dem Gauß-Lemma (\ref{gausslemma}) ist $\ksp {\frac{\partial f}{\partial t}} {\frac{\partial f}{\partial r}} = 0$, also 
\[
\|\frac{dc}{dt}\|^2 = \underbrace{\|\frac{df}{dr}\|^2}_{=1} |r'|^2 + \underbrace{\|\frac{df}{dt}\|^2}_{\ge 0} \ge |r'|^2\,.
\]

Damit gilt $\int_\ep^1 \|\frac{dc}{dt}\| dt \ge \int_\ep^1|r'|dt \ge \int_\ep^1r'dt = r(1) - r(\ep)$. (Beachte dass $r$ bei 0 nicht differenzierbar ist, aber $r(\ep) \to 0$ für $\ep \to 0$.) Also $L(c) = \int_0^1\|\frac{dc}{dt}\| dt \ge r(1) = c(1) = $ Endpunkt von $c = L(\gamma)$.

Die Gleichheit $L(c) = L(\gamma)$ gilt dann, wenn Gleichheit in allen Abschätzungen oben gilt, also $\|\frac{\delta f}{\delta t}\| = 0$, also $\frac{\delta f}{\delta t} = 0 \gleichnach{Kettenregel} d\exp_p(rv') \gdw rv'= 0\folgt v'=0 \folgt v$ ist kostant, a das heißt, dass die Richtung konstant ist. Weiter muss dann $\int_\ep^1 |r'|dt = \int_\ep^1 r' dt$ sein, also $|r'| = r' >0$, also ist $c$ eine monotone Parametrisierung von $\gamma$, insbesondere $c([0,1]) = \gamma([0,1])$.

2. Fall: $c([0,1]) \not\subset B$. Sei $\ep$ der Radius von $B$ und $t_1\in [0,1]$ der erste Parameterwert mit $c(t_1) \in S_\ep(p) = \partial B_\ep (p) = \partial B$. Dann ist
\[
L(c) > L(c|_{[0,t_1]}) \stackrel{\text{1. Fall}}\ge \ep \ge L(\gamma') \ge L(\gamma)\,.
\]
\end{beweis}

\paragraph{Fazit:} Aus „$\gamma$ ist eine Geodätische“ folgt „$\gamma$ ist lokal minimierend“. Doch gilt auch die Umkehrung?

\begin{satz}
\label{totnormumg}
Für jeden Punkt $p$ einer Riemann’schen Mannigfaltigkeit existiert eine Umgebung $W$ von $p$ und $\delta = \delta(p)>0$, so dass für alle $q\in W$ gilt:
\[
\exp_q : B_\delta(0)\ (\subset T_qM) \to M
\]
ist ein Diffeomorphismus mit $\exp_q (B_\delta(0)) \supseteq W$. Das heißt: $W$ ist eine normale Umgebung für jeden ihrer Punkte. Eine solche Umgebung von $p$ heißt total normal\index{total normale Umgebung}.
\end{satz}

\begin{beweis}
(Skizze) Betrachte $F:U\ (\subset TM)\to M\times M$, wobei $F(q,v) \da (q,\exp_q(v))$. Die Jacobi-Matrix von $F$ im Punkt $(p,0)$ ist $J_p \da \begin{pmatrix} E & 0 \\ E & E \end{pmatrix}$, wobei $E$ die $n\times n$-Einheitsmatrix ist. $J_F$ ist regulär in $(p,0)$. Weiter ist $F$ ein lokaler Diffeomorphismus von einer $(p,0)$-Umgebung $U'\subset U$ auf eine Umgebung $W'$ um $F(p,0) = (p,p)$. Wähle nun die Umgebung $W$ von $p$ so, dass $W\times W\subset W'$.
\end{beweis}

\begin{bemerkung}
Nach Satz \ref{geolokmin} und \ref{totnormumg} gilt: Für je zwei Punkte $q_1,q_2\in W$ (wie in Satz \ref{totnormumg}) existiert \emph{genau eine} minimierende Geodätische der Länge $<\delta$, welche $q_1$ und $q_2$ verbindet. Damit kann man die Radialsymetrie verlassen und hat mehr „Beweglichkeit“.
\end{bemerkung}

\begin{korrolar}[Geodätische erkennen]
Sei $\gamma$ eine stückweise differenzierbare Kurve $\gamma:[a,b]\to M$, mit der Bogenlänge parametrisiert. Falls $L(\gamma) \le L(c)$ für irgendeine
% muss das nicht für jede heißen?
stückweise differenzierbare Vergleichskurve $c$, die $\gamma(a)$ und $\gamma(b)$ verbindet, so ist $\gamma$ eine Geodätische.

Es gilt also: „$\gamma$ minimierend $\folgt$ $\gamma$ Geodätische“ (keine Lokalität zunächst!).
\end{korrolar}

\begin{beweis}
Sei $t\in[a,b]$ und $W$ eine total normale Umgebung von $\gamma(t)$. Dann existiert ein abgeschlossenes Intervall $[t_1,t_2] \subset [a,b]$, so dass $t\in I$ und $\gamma(I)\subset W$. $\gamma|_I$ ist stückweise differenzierbar und minimierend (sonst wäre $\gamma$ nicht minimierend). Nach Satz \ref{geolokmin} ist $L(\gamma|_I)$ die Länge eines radialen geodätischen Segments (da $W$ total normal) von $\gamma(t_1)$ nach $\gamma(t_2)$. Da $\gamma$ nach Bogenlänge parametrisiert ist, folgt nach Satz \ref{geolokmin}, dass $\gamma|_I$ eine Geodätische in der Umgebung von $t$ ist. $t$ war beliebig, woraus die Behauptung folgt.
\end{beweis}

\subsubsection*{Anwendungen:}
\begin{enumerate}
\item Eine Riemann’sche Isometrie bildet Geodätische auf Geodätische ab.
\begin{beweis}
$\varphi: M \to N$ sei eine Isometrie; $\gamma$ ist eine Geodätische, dann ist $L(\varphi \circ \gamma) = L(\gamma)$. Also: $\gamma$ ist minimierend, dann $\varphi\circ \gamma$ minimierend. Dann folgt die Behauptung aus dem Korrolar.

Alternativ: $D_{\gamma'}{\gamma'} = 0 \folgt D_{(\varphi\circ \gamma)'}(\varphi\circ\gamma)' = 0$.
\end{beweis}
\item Geodätische von $S^n=\{x \in \MdR^{n+1}\mid \|x\| = 1\}$ sind Großkreise.
\begin{beweis}
Sei $c$ ein Großkreis, das heißt $c = S^n \cap \sigma$, wobei $\sigma$ eine 2-dimensionale Ebene in $\MdR^{n+1}$ durch 0 ist. Wähle $p,q\in c$ genügend nahe, so dass es nach Satz \ref{totnormumg} genau eine Geodätische $\gamma$ zwischen $p,q$ existiert. Dann ist die euklidische Spiegelung $R$ an $\sigma$ eine euklidische Isometrie von $S^n$. $R$ fixiert $c$ punktweise und bildet $\gamma$  auf $\tilde\gamma$ ab (auch durch $p,q$). Also mus wegen Eindeutigkeit $\gamma = \tilde \gamma$ gelten und $\gamma$ bleibt punktweise fest, das heißt $\gamma\subset c = $ die Fixpunkte von $R$ $= S^n\cap \sigma$. Damit ist der Großkreis selber die Geodätische $\gamma$.
\end{beweis}
\end{enumerate}

\chapter{Krümmung}

% Historischer Ausgangspunkt Gauss fehlt...  

\section{Der Riemann’sche Krümmungstensor}

Gegeben sei eine Riemann’sche Mannigfaltigkeit $(M,\asp)$ mit Levi-Civita-Zusammenhang $D$. Der Riemann’sche Krümmungstensor\index{Riemann’sche Krümmungstensor}\index{Krümmungstensor} von $M$ bezüglich $D$ ist die Abbildung $R:\V M\times \V M \times \V M \to \V M$, $(X,Y,Z) \mapsto R(X,Y)Z$, wobei \[R(X,Y)Z \da D_YD_XZ - D_XD_YZ + D_{[X,Y]}Z\,.\]

\begin{beispiel}
Im $(\MdR^n,\asp)$, wobei $\asp$ das Standardskalarprodukt ist, betrachten wir das Vektorfeld $Z=(z^1,\ldots,z^n)\in \V \MdR^n$. Da $D_XZ = (Xz^1,\ldots,Xz^n)$, folgt: $D_YD_XZ = (YXz^1,\ldots,YXz^n)$. Wegen $[X,Y] = XY-YX$ folgt: $R(X,Y)Z = 0$.
\end{beispiel}

Das oben definierte $R$ ist somit ein „Maß“ für die Abweichung der Riemann’schen Mannigfaltigkeit $(M,\asp)$ von der euklidischen Geometrie.

\begin{bemerkung}
Bezüglich lokalen Basisfeldern $\frac{\partial}{\partial x^i}$ ($i=(1,\ldots,n)$) gilt: $[\frac\partial{\partial x^i},\frac\partial{\partial x^j}] = 0$ für $C^\infty$-Funktionen. Dann ist $R(\frac\partial{\partial x^i},\frac\partial{\partial x^j})\frac\partial{\partial x^k} = D_{\frac\partial{\partial x^j}}D_{\frac\partial{\partial x^i}} \frac\partial{\partial x^k} - D_{\frac\partial{\partial x^i}}D_{\frac\partial{\partial x^j}} \frac\partial{\partial x^k} $. „$R$ ist ein Maß für die Vertauschbarkeit der 2. kovarianten Ableitungen.“
\end{bemerkung}

\begin{definition}
Setze $\V_0M \da C^\infty M$, $\V_rM \da \V M\times \cdots \times \V M$. ($r$ Summanden). $\V_rM$ ist ein $C^\infty M$-Modul. Ein $(s,r)$-Tensorfeld\index{Tensorfeld} auf $M$ ist eine $r$-lineare Abbildung $T:\V_r \to \V_sM$ über dem Ring $C^\infty M$, das heißt 
\begin{align*}
T(X_1,\ldots,X_{i-1},fX + gY,X_{i+1},\ldots, X_r) &=
f T(X_1,\ldots,X_{i-1},X,X_{i+1},\ldots, X_r)\\ &+
g T(X_1,\ldots,X_{i-1},Y,X_{i+1},\ldots, X_r) 
\end{align*}
für alle Argumente von $T$, $X,Y \in \V M$
\end{definition}

\begin{satz}
\label{tensorfeld}
$R$ ist ein (1,3)-Tensorfeld
\end{satz}
\begin{beweis}
Exemplarisch für $R(X,Y)(fZ = fR(X,Y)Z\ \forall f\in C^\infty M$.

$D_YD_X(fZ) = D_Y(fD_XZ + (Xf)Z) = (Yf)D_XZ + fD_YD_X Z + (YXf) Z + (Xf)D_YZ$. Also:
$D_YD_X(fZ) - D_XD_Y(fZ) = f(D_YD_XZ - D_XD_YZ) + (YXf - XYf)Z$; $D_{[X,Y]}fZ = fD_{[X,Y]}Z + ([X,Y]f)Z\folgt R(X,Y)fZ = fR(X,Y)Z.$
\end{beweis}

\begin{satz}[Symmetrie-Eigenschaften]
\label{symeig}
$(M,\asp)$ sei eine Riemann’sche Mannigfaltigkeit. $D$ der Levi-Civita-Zusammenhang und $R$ ein Krümmungstensor. Dann gilt
\begin{enumerate}
\item $R(X,Y)Z + R(X,Z)X + R(Z,X)Y = 0$ (zyklisch Vertauschbar). „Bianchi-Identität“
\item $\ksp {R(X,Y)Z} T = - \ksp {R(Y,X)Z} T$
\item $\ksp {R(X,Y)Z} T = - \ksp {R(X,Y)T} Z$
\item $\ksp {R(X,Y)Z} T = \ksp {R(Z,T)X} Y$
\end{enumerate}
\end{satz}

\begin{beweis}
\begin{enumerate}
\item ist äquivalent zur Jacobi-Identität für Lie-Klammern.
\item folgt direkt aus der Definition.
\item ist äquivalent zu $\ksp {R(X,Y) W} W = 0$ (setzte $W=Z+T$ und verwende Satz \ref{tensorfeld}).

Es ist $\ksp {R(X,Y)W} W = \ksp {D_YD_XW - D_XD_YW + D_{[X,Y]}W} W$, $\ksp {D_YD_X W} W \gleichnach{Levi-Civita, vertraeglich} Y \ksp{D_XW} W  - \ksp {D_X W} {D_Y W}$,
analog $\ksp{D_XD_Y W} W$; $\ksp {D_{[X,Y]}W}W = \frac 12 [X,Y]\ksp W W$. Somit: $\ksp{R(X,Y)W}W = Y \ksp{D_XW}W - \ksp {D_XW}{D_YW} - X\ksp{D_YW}W + \ksp{D_YW}{D_XW} + \frac 12 [X,Y] \ksp W W = 0$.
\item Analog.
\end{enumerate}
\end{beweis}


\subsection*{Krümmungstensor in lokalen Koordinaten $(u,\varphi)$}

Die Basisfelder seien $X_i \da \frac{\partial}{\partial x^i}$, $i=1,\ldots,n$. Dann: $R(X_i,X_j)X_k \da \sum_{l=1}^n R_{ijk}^l X_l$ (per Basissatz), wobei $R_{ijk}^l$ die Komponenten des Krümmungstensors in lokalen Koordinaten sind, also $C^\infty$-Funktionen und symmetrisch bezüglich $i$, $j$.

Für beliebige Vektorfelder $X,Y,Z\in \V M$ mit 
\[ X =\sum_{i=1}^n u^i X_i,\quad Y=\sum_{j=1}^n v^jX_j,\quad Z=\sum_{k=1}^n w^k X_k \]
gilt wegen Satz \ref{tensorfeld}:
\[ R(X,Y)Z = \sum_{i,j,k,l=1}^n u^i v^j w^k R_{ijk}^l X_l \mathrlap(*) \]
(man muss alles an der Stelle $p$ kennen).


\begin{bemerkung} (Trägereigenschaft von $R$)
Die Formel $(*)$ zeigt, dass $(R(X,Y)Z)(p)$ nur von den Werten der Vektorfelder $X,Y,Z$ im Punkt $p$ abhängig ist.
\end{bemerkung}

\subsection*{Formel für $R_{ijk}^l$}
\begin{align*}
R(X_i,X_j)X_k &= D_{X_j}(D_{X_i}X_k) - D_{X_i}(D_{X-j}K_k) + D_{\underbrace{[X_i,X_j]}_{=0}}X_k \\
&=  D_{X_j}(\sum_{m=1}^n \Gamma_{ik}^m X_m) - D_{x_j}(\sum_{m=1}^n \Gamma_{jk}^n X_m) \\
&= \sum_{m=1}^n [ (X_j (\Gamma_{ik}^m X_m + \Gamma_{ik}^m \underbrace{D_{X_j}X_m}_{\sum_{l=1}^m \Gamma_{jm}^l X_l} ] - \sum_{m=1}^n [X_i (\Gamma_{ik}^m) X_m + \Gamma_{jk}^m \underbrace{D_{X_i}X_m}_{\sum_{l=1}^n \Gamma_{im}^l X_l } ] \\
\folgt R_{ijk}^l &= \frac{\partial}{\partial x^j} \Gamma_{ik}^l + \sum_{m=1}^n \Gamma_{ik}^n \Gamma_{jm}^l - \frac{\partial}{\partial x^i} \Gamma_{jk}^l - \sum_{m=1}^n \Gamma_{jk}^m \Gamma_{im}^l
\end{align*}
(so hatte es Riemann definiert)

Setze nun
\[
R_{ijks} \da \sum_{l=1}^n R_{ijk}^l \cdot g_{ls} = \ksp {R(X_i,X_j)X_k} {X_s}
\]
„Herunterziehen von Indizes“, „Ricci-Kalkül“. Nach Satz \ref{symeig} gilt:
\begin{itemize}
\item $R_{ijks} + R_{jkis} + R_{kijs} = 0$
\item $R_{ijks} = -R_{ijsk}$
\item $R_{ijks} = -R_{jiks}$
\item $R_{ijks} = R_{ksij}$
\end{itemize}

\begin{bemerkung}
Für $\dim M=2$ sind $i,j,k,s \in \{1,2\}$ und aufgrund obiger Symmetrien ist im wesentlichen nur $R_{1212} \ne 0$. Dies ist gerade die Gauß-Krümmung.
\end{bemerkung}


\subsection*{Riemann’scher Krümmungstensor}

Sei $(M,\asp)$ eine Riemann’sche Mannigfaltigkeit und $D$ der zugehöriger Levi-Civita-Zusammenhang. Dann ist
\[
R:
\begin{aligned}
\V M \times \V M \times \V M  &\to \V M \\
(X,Y,Z) &\mapsto R(X,Y)Z \da D_YD_XZ - D_XD_YZ - D_{[X,Y]}Z
\end{aligned}
\]
multilinear bezüglich $C^\infty M$.

\section{Schnittkrümmung}

Vorbemerkung aus der Linearen Algebra. Sei $V$ ein $\MdR$-Vektorraum mit Skalarprodukt $\asp$. Für $x,y\in V$ setze 
\[
| x \wedge y| \da \sqrt{\|x\|^2 \|y\|^2 - <x,y>^2} \ge 0
\]
(Flächeninhalt des des von $x$ und $y$ aufgespannten Parallelogramms). Für orthonormierte Vektoren ist $|x\wedge y| = 1$.

\begin{lemma}
Sei $(M,\asp)$ eine Riemann’sche Mannigfaltigkeit, $p\in M$, $\sigma$ ein 2-dimensionaler Untervektorraum von $T_pM$ mit Basis $x,y$. Dann ist 
\[
K(x,y) \da \frac{\ksp {R(x,y)x} y_p}{|x\wedge y|^2}
\] unabhängig von der Wahl der Basis.
\end{lemma}

In der Konsequenz macht folgende Definition Sinn:
\begin{definition}
Für $p\in M$, $\sigma \subset T_pM$ ein 2-dimensionaler Untervektorraum setze $K(p,\sigma) \da K(x,y)$ für eine beliebige Basis $\{x,y\}$ von $\sigma$. $K(p,\sigma)$ heißt Schnittkrümmung von $\sigma$ in $p\in M$.
\end{definition}

\begin{bemerkungen}
\item Für $n=2$ ist $K(p,\sigma) = K(p)$ die Gauß-Krümmung von $M$ im Punkt $p$. Die Menge der Krümmungstensoren $R$ im Punkt $p$ ist vollständig bestimmt.
\begin{beispiel}
Schnittkrümmung von $(\MdR^n,\kan)$ ist konstant null, da $R=0$.
\end{beispiel}
\item $(S^n,\kan)$. Behauptung: Schnittkrümmung ist konstant 1.
\begin{lemma}
Sei $f:(M,\asp) \to (N,\aasp)$ eine Riemann’sche Isometrie. Für $\sigma \subset T_pM$ ist $df|_p(\sigma) \subset T_{f(p)}N$ ein 2-dimensionaler Untervektorraum und $K^M(p,\sigma) = K^N(f(p),df|_p(\sigma))$. Das heißt: Schnittkrümmung ist invariant unter Isometrie.
\end{lemma}
\begin{beweis}
Es gilt (Übungsblatt 7 Aufgabe 1):
\begin{itemize}
\item $D_{dfx}^N df(y) = df(D_x^my)$
\item $[df(x), df(y)]^N = df([x,y]^M)$
\item $\kksp {df(x)}{df(y)} = \ksp x y$
\end{itemize}
$\folgt R^N(df(x), df(y)) df(z) = df(R^M(x,y)z)$. Es genügt zu zeigen: Zu $\sigma \subset T_xS^n$ und $\tau \subset T_yS^n$, jeweils 2-dimensionale Untervektorräume, existiert eine Isometrie $f: S^n \to S^n$ mit $df_x(\sigma) = \tau$.

Sei nun $\sigma = [u,v]$, $\tau = [\tilde u, \tilde v]$, wobei $u,v$ bzw. $\tilde u, \tilde v$ Orthonormalbasen sind. $l_1=x$, $l_2=u$, $l_3=v$.
\[
y =
\begin{pmatrix}
y_1 \\ \vdots \\ y_{n+1}
\end{pmatrix}
= f_1
\qquad
\tilde u =
\begin{pmatrix}
\tilde u_1 \\ \vdots \\ \tilde u_{n+1}
\end{pmatrix}
= f_2
\qquad
\tilde v =
\begin{pmatrix}
\tilde v_1 \\ \vdots \\ \tilde v_{n+1}
\end{pmatrix}
= f_3
\]
ergänze zu einer Orthonormalbasis $\{f_1,\ldots,f_{n+1}\}$ von $\MdR^{n+1}$. Dann ist $A \da [f_1,f_2,\ldots,f_{n+1}] \in O(n+1)$, also eine orthogonale $(n+1)\times(n+1)$-Matrix, mit $A_{li} = f_i$, $f: \MdR^{n+1} \to \MdR^{n+1}$; $w\mapsto Aw$ ist eine euklidische Isometrie (Rotation von $(\MdR^{n+1},\kan)$ die $S^n$ invariant lässt. Dies Induzier also eine Isometrie von $(S^n,\kan)$.

Da $f$ linear ist, $df=f$, also $df_x(\sigma) = df_x([u,v]) = [df_xu, df_yv] = [\tilde u, \tilde v] = \tau \folgt$ Behauptung. $S^n$ hat konstante Schnittkrümmung. Es glit $K=1$ (siehe später).
\end{beweis}

\item $n$-dimensionale hyperbolische Räume $H^n\MdR \da \{ x\in\MdR^n \mid x_n > 0\}$ mit der Identität als Karte und lokalen Koordinaten $x_1,\ldots,x_n$. Es ist
\[
(g_{ij}) \da 
\begin{pmatrix}
\frac{1}{(x_n)^2} & & 0 \\
& \ddots & \\
0 & & \frac 1 {(x_n)^2}
\end{pmatrix} = \frac 1 {(x_n)^2} 
\begin{pmatrix}
1 & & 0 \\
& \ddots & \\
0 & & 1
\end{pmatrix}
\] Berechnung der $R_{ijk}$ zeigt: Schnittkrümmung $R$ ist konstant $-1$.

\item Konfrome Änderung der Metrik $(M,g)$ einer Riemann’schen Mannigfaltigkeit um $\lambda \in C^\infty M$, $\lambda >0$: $\tilde g \da \lambda g$ ist wieder eine Riemann’sche Metrik.

Für konstantes $\lambda>0$ ist die Schnittkrümmung für $\tilde g$: $\tilde R=\frac 1\lambda R$. Insbesondere kann man jede Mannigfaltigkeit mit beliebiger, konstaner Krümmung durch Reskalierung der Riemann’schen Metrik zu $S^n$ oder $H^n$ formen. (besseres verb bitte?!?)

\end{bemerkungen}

\subsection*{Ergänzende Sätze (ohne Beweis, vergleiche: do Carmo, Kapitel 8)}

\begin{satz*}
$(M, \asp)$ hat konstante Schnittkrümmung, also
\[ K(p,\sigma) = K_0 \forall \sigma \subset T_pM
\forall p \in M \gdw \ksp {R(x,y)w} z = K_0 (\ksp x w \ksp y z - \ksp y w \ksp x z)
\]
insbesondere ist $\ksp {R(x,y)x} y = K_0(\|x\|^2\|y\|^2 - \ksp x y ^2)$.
\end{satz*}

\begin{satz*}[Hopf]
Eine vollständige, einfach zusammenhängende, ??? Riemann’sche Mannigfaltigkeit mit konstanter Krümmung 0,1 oder -1 ist isometrisch zu $\MdR^n$, $S^n$, $H^n\MdR$. Dabei heißt
\begin{itemize}
\item Vollständig: Jede Geometrie ist auf ganz $\MdR$? definiert
\item Einfach zusammenhängend: Jede geschlossene Kurve ist auf einen Punkt zusammenziehbar
\item „???“ = Riemann’sche Mannigfaltigkeit konstanter Krümmung.
\end{itemize}

\end{satz*}

\section{Ricci-Krümmung}

Sei $R$ der Krümmungstensor einer Riemann’schen Mannigfaltigkeit $(M,\asp)$ und $X,Y,Z \in \V M$. In jedem Punkt $p\in M$ ist $Y(p) \mapsto R(X(p),Y(p))Z(p)$ ein Endomorphismus von $T_pM$. Oder: Für $X,Z \in \V M$ fest ist $R(X,\cdot)Z$ ein (1,1)-Tensormodul.

Für ein beliebiges (1,1)-Tensorfeld $A$ ist $A(p):T_pM \to T_pM$ ein Endomorphismus und wir definieren die Spur von $A$ durch
\[
(\spur A)(p) \da \sum_{i=1}^n \ksp {A(p)e_i}{e_i}_p
\] wobei $[e_i]$ eine Orthonormalbasis von $T_pM$ ist. Linere Algebra: Es gibt einen Endomorphismus $\Phi$ mit Abbildungsmatrix $A$ und $\spur \Phi = \spur A = \sum_{i=1}^n A_{ii}$ (insbesondere für Orthonormalbasen, $a_{ii} = \ksp {A_{e_i}}{e_i}$).

Der Ricci-Tensor von $M$ ist der (0,2)-Tensor $Ric(x,y)\da \spur(y\mapsto R(x,y)z)$. (In manchen quellen noch mit $\frac1{n-1}$ normiert.) Die Ricci-Krümmung von $M$ in Richtung $v\in T_pM$ ist \[r(v) \da \frac{Ric(v,v)}{\|v^2\|}\,.\]

Für eine Orthonormalbasis $\{e_i\}$ von $T_pM$ ist $Ric(v,w) = \sum_{i=1}^n \ksp {R(v,e_i)w} {e_i}$. Also insbesondere ist der Ricci-Tensor symmetrisch und $r(e_1) = \sum_{i=2}^n K(p,[e_1,e_i])$.

Die Skalar-Krümmung ist eine differenzierbare Funktion auf $S:M\to \MdR$, $p\mapsto \sum_{j=1}^n r(e_j)$, wobei $\{e_j\}$ eine Orthonormalbasis von $T_pM$ ist.
\[
S(p) = \sum_{j=1}^n r(e_j) = \sum_{j=1}^n Ric(e_j, e_j) = \sum_{i,j=1}^n \ksp {R(e_j, e_i)e_j} e_i =\sum_{\mathclap{i,j=1,\,i\ne j}}^n K(p, [e_i, e_j])
\]

Eine Riemann’sche Mannigfaltigkeit $(M,g)$ heißt Einstein-Raum falls $Ric(x,y) = \lambda g(x,y) \forall x,y\in \V M$, wobei $\lambda:M \to \MdR$ eine differenzierbare Funktion ist.

\begin{beispiel}
Räume mit konstanter Krümmung sind Einstein-Räume: $K=c_0$ konstant. $Ric(X,X)=\sum_{i=1}^n k([x,e_i]) g(x,x) = (n-1) c_0 g(x,x)$
\end{beispiel}

\begin{bemerkung}
Der Einstein-Tensor ist\dots (bitte jemand eintragen!)
\end{bemerkung}

\chapter{Jacobi-Felder (Verbindung Geometrie--Krümmung)}
\section{Jacobi-Gleichung}

Sei $(M,\asp)$ eine Riemann’sche Mannigfaltigkeit. Für $v\in T_pM$ sei $\exp_p$ definiert. Wir betrachten die parametrisierte Fläche $f(t,s) \da \exp_p(t v(s))$ mit $0\le t \le 1$ und $-\ep \le s \le \ep$, wobei $v(s)$ eine Kurve in $T_pM$ mit $\|v(s)\| = \|v(0)\|$, $v(0) = v$, $v'(0) = w$ ist.

Es gilt (vergleiche Beweis Gauß-Lemma):
\[
d\exp_p|_v w  = \frac{\partial l}{\partial s}(1,0) = T_{\exp_p(v)}M\,.
\] $\|d\exp_p|_v w\|$ ist ein Maß dafür, wie schnell die Geodätischen $t\mapsto f(t,s)$ auseinanderlaufen.


Betrachte dazu das Vektorfeld $d\exp_p|_{tv} tw = \frac{\partial f}{\partial s}(t,v)$ längs $\gamma(t) \da \exp_p(tv)$, $0\le t\le 1$. Wir halten fest: Da $\gamma$ eine Geodätische ist, gilt für alle $t,s$: $\frac D{\partial t}\frac{\partial f}{\partial t}(t,s)=0$.

\begin{lemma}
$\ $\[
f: 
\begin{aligned}
A\subset \MdR^2 &\to M \\ (u,v) &\mapsto f(u,v)
\end{aligned}
\] sei eine parametrisierte Fläche und $V(u,v)$ sei ein Vektorfeld längs $f$. Dann gilt:
\[
\frac{D}{\partial V}\frac{D}{\partial U} V - \frac{D}{\partial U}\frac{D}{\partial V} V
 = R(\frac{\partial f}{\partial U}, \frac{\partial f}{\partial V})V
\]
wobei $\frac{D}{\partial U} = D_{\frac{\partial f}{\partial U}}$.
\end{lemma}
\begin{beweis}

Betrachte Karte $(U,\varphi)$. Dann sind die Basisfelder also $V=\sum_{i=1}^n v^iX_i$, $v^i= v^i(u,v)$, $\frac{D}{\partial U}V = \frac{D}{\partial U}(\sum_{i=1}^n u^ix_i) = \sum_{i=1}^n \frac{\partial u^i}{\partial U} x_i + \sum_{i=1}^n v^i \frac{D}{\partial u}x_i$. $\frac D {\partial u}(\frac D{\partial U} V) = \sum_{i=1}^n \frac{\partial ^2 u^i}{\partial v \partial u} x_i + \sum_{i=1}^n \frac{\partial v^i}{\partial u} \frac D{\partial v} x_i + \sum_{i=1}^n \frac{\partial u^i}{\partial v} \frac{\partial D}{\partial u} x_i$
$\folgt \frac{D}{\partial v}\frac D{\partial u} v - \frac D {\partial u}\frac D{\partial v} v = \sum_{i=1}^n v_i (\frac{D}{\partial v}\frac{D}{\partial u} x_i - \frac D{\partial u}\frac D{\partial v} x_i)\quad (+)$ (Bitte auf $v$- und $u$-Verwechsler prüfen!)

Berechne $\frac D{\partial v}\frac D{\partial u} x_i$: Für $f(u,v) = (x^1(u,v),\ldots, x^n(u,v,))$ ist $\frac {\partial f}{\partial u} = \sum_{j=1}^n \frac{\partial x^j}{\partial u} x_j$; $\frac {\partial f}{\partial v} = \sum_{k=1}^n \frac{\partial x^k}{\partial u} x_k$ und $\frac D {\partial u} x_i = D_{\frac {\partial f}{\partial u}} x_i = \sum_{j=1}^n \frac{\partial x^j}{\partial u} D_{x_j} x_j$. $\frac D{\partial v}\frac D {\partial u} v_i = \sum_{j=1}^n = \frac{\partial ^2 x^j}{\partial v\partial u} D_{x_j}  x_i + \sum \frac{\partial x^j}{\partial u} D_{\frac{\partial f}{\partial u}} (D_{x_j}x_i) = \sum_{j} \frac{\partial ^2 x^j}{\partial u\partial v} D_{x_j}x_i + \sum_d \frac{\partial x^i}{\partial u} (\sum_k \frac{\partial x^k}{\partial u} D_{x_k}D_{x_j}x_i) = \sum_{j}$ WTF\dots.

\end{beweis}


Weiter gilt:
\begin{align*}
0 = \frac D {\partial s} (\frac D {\partial t} \frac {\partial f}{\partial t}) &\gleichnach{Lemma 1} \frac D{\partial t} (\frac D {\partial s} \frac{\partial f}{\partial t}) - R(\frac {\partial f}{\partial s},\frac{\partial f}{\partial t}) \frac{\partial f}{\partial t} \\
&\gleichnach{Lemma 3 Kap 4 + schiefsym.}
\frac D{\partial t} (\frac D {\partial t} \frac{\partial f}{\partial s}) - R(\frac {\partial f}{\partial t},\frac{\partial f}{\partial s}) \frac{\partial f}{\partial t}
\end{align*}

Wir setzen $\gamma(t) = \exp_p(tv) = f(t,0)$ und $J(t)\equiv J(\gamma(t)) \da \frac{\partial f}{\partial s} (t,0)$ ein Vektorfeld längs $\gamma$. Dann gilt die Jacobi-Gleichung:
\[
\frac D{\partial t} \frac D{\partial t} J(t) + R(\gamma'(t), J(t))\gamma'(t) = 0
\]
mit der Kurzschreibweise
\[
\frac D{\partial t} \frac D{\partial t} J(t) \ad J''(t)
\]

\begin{definition}
Sei  $\gamma: [0,a] \to M$ eine Geodätische. Ein Vektorfeld $J$ längs $\gamma$ heißt Jacobi-Feld, falls $J$ für alle $t\in[0,a]$ die Jacobi-Gleichung\index{Jacobi-Gleichung} erfüllt.
\end{definition}

Es gilt: Ein Jacobi-Feld ist eindeutig bestimmt durch die Anfangsbedingungen $J(0)$ und $J'(0)\da D_{\gamma'} J(0)$.


Begründung: Betrachte orthonormale Parallelfelder $E_1(t),\ldots,E_n(t)$, wobei $E_i(t) = E_i(\gamma(t))$, längs $\gamma$. Dann kann man schreiben: $J(t) = \sum_{i=1}^n f_i(t) E_i(t)$ mit $f_i\in C^\infty$. Also $J'(t) = D_{\gamma'}J(t) = \sum_{i=1}^n D_{\gamma'} D_{\gamma'} (f_i E_i)= \sum_{i=1}^n (f_i' E_i+ f_i \underbrace{D_{\gamma'}E_i}_{=0}) 
= \sum_{i=1}^n f_i'(t) E_i(t)$ und $J''(t) = \sum_{i=1}^n f''_i(t) E_i(t)$.

Weiter sei $a_{ij}(t) \da \ksp {R(\gamma'(t), E_i(t))\gamma'(t)} {E_j(t)}_{\gamma(t)}$. Dann gilt $R(\gamma',J)\gamma' = \sum_j \ksp {R(\gamma',J)\gamma'} {E_j} E_j
= \sum_{j=1}^n \sum_{i=1}^n f_i \ksp {R(\gamma' E_i)\gamma'} {E_j} E_j
= \sum_{j=1}^n \sum_{i=1}^n f_i a_{ij}(t) E_j(t)$

Damit ist die Jacobi-Gleichung äquivalent zum System linearer Differentialgleichungen
2. Ordnung
\[
f_j''(t) + \sum_{i=1}^n a_{ij}(t) f_i(t) = 0,\quad j=1,\ldots,n
\]

Die Lösungen bilden einen Vektorraum der Dimension $2n$, wobei $n=\dim M$. Zu gegebener Anfangsbedingung $J(0)$, $J'(0)$ bzw. $f_1(0),\ldots,f_n(0),f_1'(0),\ldots,f_n'(0)$ existiert genau ein Jacobi-Feld längs ganz $\gamma$, also eine Lösung des obigen Differentialgleichungssystems für alle $t\in[0,a]$.

\begin{folgerung*}
Längs der Geodätischen $\gamma:[0,a] \to M$ existieren $2n$ linear unabhängige Jacobi-Felder, wobei $n=\dim M$.
\end{folgerung*}

\begin{bemerkung}
Gewisse Jacobi-Felder kann man direkt angeben:  $J(t) \da \gamma'(t)$ ist ein Jacobi-Feld, da $J'' + R(\gamma',J)\gamma' = \gamma''' + R(\gamma',\gamma')\gamma' = D_{\gamma'}\gamma'' + 0 = D_{\gamma'}D_{\gamma'} \gamma' = 0$.
\end{bemerkung}

Ansatz: $J(t)\da a(t)\gamma'(t)$ für $a:I\to \MdR$ ist Jacobi-Feld, genau dann, wenn $a(t)$ linear ist. Also: $J''= a''\gamma'$, $R(\gamma',J)\gamma' = R(\gamma',a \gamma')\gamma' = a R(\gamma',\gamma')\gamma'= 0$. Das heißt die Jacobi-Gleichung gilt $\gdw a''\gamma'=0 \gdw a''=0 \gdw a(t) = \alpha + t\beta$, $\alpha, \beta\in \MdR$.

\begin{folgerung*}
$J_1(t) \da \gamma'(t)$ und $J_2(t) \da t \gamma'(t)$ sind verschieden, da $J_1(0) = \gamma'(0) \ne J_2(0) = 0$, und spannen einen 2-dimensionalen Untervektorraum des Vektorraumes alles Jacobi-Felder längs $\gamma$ auf.

Es genügt dann den $2(n-1)$-dimensionalen Untervektorraum aller Jacobi-Felder orthogonal zu $\gamma'$ zu verstehen.
\end{folgerung*}

\begin{beispiel}[Jacobi-Felder für Riemann’sche Mannigfaltigkeiten konstanter Krümmung]
Sei $(M,\asp)$ eine Riemann’sche Mannigfaltigkeit mit konstanter Schnittkrümmung $k_0$, etwa $(\MdR^2,\kan): k_0=0$, $(S^2,\kan): k_0=1$, $(H^2\MdR,\kan): k_0 = -1$.

Weiter sei $\gamma: [0,a] \to M$ eine normale Geodätische und $J$ ein Jacobi-Feld längs $\gamma$, so dass $J(t) \bot \gamma'(t)$.

Für ein beliebigs Vektorfeld $X$ längs $\gamma$ gilt die Formel (vgl. 5.2):
\[
\ksp{R(\gamma',J) \gamma'}{X} = k_0 (\underbrace{\ksp{\gamma'}{\gamma'}}_{=1} \ksp J X - \ksp{\gamma'}X \underbrace{\ksp{J}{\gamma'}}_{=0}) = k_0 \ksp J X
\] also 
\[
R(\gamma', J) \gamma' = k_0 J
\]

Die Jacobi-Gleichung lautet hier:
\[
J'' + k_0 J = 0\mathrlap{\quad(*)}
\]
Es sei $E(t)$ ein Parallelfeld längs $\gamma$ mit $\|E(t)\|_{\gamma(t)}=1$ und $\ksp{E(t)}{\gamma'(t)}_{\gamma(t)} = 0$ für alle $t$. Dann ist 
\[
J(t) \da
\begin{cases}
\frac1{\sqrt {k_0}}\cdot {\sin(t\sqrt {k_0})}\cdot E(t), &k_0 > 0 \\
t \cdot E(t), & k_0 = 0 \\
\frac1{\sqrt{-k_0}}\cdot {\sinh(t \sqrt{-k_0})}\cdot E(t), & k_0 < 0
\end{cases}
\] eine Lösung von $(*)$ mit Anfangsbedingung $J(0) = 0$ und $J'(0) = E(0)$.

\begin{satz}
Sei $\gamma:[0,a]\to M$ eine normale Geodätische (also $\|\gamma'\| = 1$) und $J$ ein Jacobi-Feld längs $\gamma$ mit $J(0)=0$ und $J'(0)=\frac D{\partial t} J(0) = (D_{\gamma'}J)(0) \ad w$. Schließlich sei $v\da \gamma'(0)$.

Wir betrachten $w$ als Element von $T_{av}(T_{\gamma(0)} M)$ und wählen Kurve $v(s)$ in $T_{\gamma(0)}M$ mit $v(0) = av$, $v'(0)=aw$. Für die parametrisierte Fläche $f(t,s) \da \exp_{\gamma(0)}(\frac t a v(s))$, $|s|<\ep$, $0\le \frac t a \le 1$ ist $\bar J(t) \da \frac{\partial f}{\partial s}(t,0)$ ein Jacobi-Feld längs $\gamma$ mit $J(t) = \bar J(t)$ für alle $t\in[0,a]$.
\end{satz}

\begin{beweis}
Jacobi-Feld is durch Anfangsbedingungen vollständig bestimmt, das heißt es genüg zu zeigen: $J(0) = \bar J(0)$ und $J'(0) = \bar J'(0)$.

Es ist einfach zu sehen, dass $\bar J(0) = \frac {\partial f}{\partial s}(0,0) = 0$.

Weiter gilt
\begin{multline*}
\bar J'(t) = \frac D {\partial t} \frac {\partial f}{\partial s} (t,0) = \frac D {\partial t}(d\exp_p|_{\frac t a v(0)} \cdot \frac t a v'(0))  = \frac D{\partial t} (d\exp_p|_{tv} tw) \\ = \frac D{\partial t}( t d\exp_p|_{tv} w) = 1\cdot d\exp_p|_{tv} w + t \frac D{\partial t}(d\exp_p|_{tv} w)\,.
\end{multline*}
Daher ist $\bar J'(0) = d\exp_p|_0 w = w = J'(0)$.
\end{beweis}

\begin{bemerkungen}
\item Es gilt folgende Formel für ein Jacobi-Feld längs einer normalen Geodätischen $\gamma:[0,a] \to M$ mit $J(0) = 0$:
\[
J(t) = d\exp_p|_{t\gamma'(0)} (tJ'(0)),\quad t\in[0,a]
\]

\item Eine analoge Konstruktion (Jacobi-Felder erzeugen durch Variation einer Geodätischen) gilt auch für Jacobi-Felder mit Anfangsbedingung $J(0)\ne 0$.
\end{bemerkungen}

\section{Jacobi-Felder und Schnittkrümmung}

\begin{satz}
Sei $p\in M$, $\gamma:[0,a]\to M$ eine normale Geodätische mit $\gamma(0)=p$, $\gamma'(0)=v$ und $w \in T_v(T_pM) \cong T_pM$ mit $\|w\| = 1$.  Weiter sei $J(t) = d\exp_p|_{tv}(tw)$, $0\le t \le a$ ein Jacobi-Feld längs $\gamma$.

Dann gilt für die Taylorentwicklung von $\|J(t)\|^2_{\gamma(t)} = \ksp {J(t)}{J(t)}_{\gamma(t)}$ bei $t=0$:
\[
\|J(t)\|_{\gamma(t)}^2 = t^2 - \frac 13 \ksp {R(v,w)v}w_p t^4 + o(t^4)
\]
\end{satz}

\begin{beweis}
Es ist $J(0)=0$, $J'(0) = w$, $\|w\|=1$. Für die ersten drei Koeffizienten der Taylorreihe in $t$ folgt:
\begin{enumerate}
\item[(0)] $\|J(p)\|_p^2 = \ksp J J (0) = 0$
\item[(1)] $\ksp J J '(0) = 2\ksp {J'}J(0) = 0$
\item[(2)] $\ksp{J}J''(0) = 2\ksp {J''}J(0) + 2 \ksp{J'}{J'}(0) = 0 + 2\|w\|^2 = 2$
\item[(3)] $\ksp JJ'''(0) = 2\ksp {J'''} J (0) + 2 \ksp {J''}{J'} (0) + 4 \ksp {J''}{J'}(0) = 0 + 6 \ksp {-R(\gamma',J)\gamma'}{J'}(0) = 6 \ksp {-R(\gamma',0)\gamma'}{J'}(0) = 6 \ksp 0 {J'}(0) = 0$
\item[(4)] $\ksp JJ ''''(0) = 2 \ksp{J''''}J(0) + 2 \ksp{J'''}{J'}(0) + 6 \ksp{J'''}{J'}(0) + 6 \ksp{J''}{J''}(0) = 8 \ksp{J'''}{J'}(0) = -8 \ksp{R(\gamma',J')\gamma'}{J'}(0) = -8 \ksp{R(v,w)v}w_p$

Nebenrechnung für $J'''= -\frac D {\partial t} R(\gamma',J)\gamma'$. Dazu betrachten wir ein beliebiges Vektorfeld $Z$ mit $Z' = \frac D{\partial t}Z = D_{\gamma'}Z$. Es ist 
\begin{align*}
\ksp{\frac D{\partial t} R(\gamma',J) \gamma'}Z &= \frac d{dt} \ksp {R(\gamma',J)\gamma'}Z - \ksp{R(\gamma',J)\gamma'}{Z'}\\
&= \frac d{dt} \ksp {R(\gamma',Z)\gamma'}J - \ksp{R(\gamma',J)\gamma'}{Z'}\\
&= \ksp {\frac D{dt} R(\gamma',Z)\gamma'}J + \ksp {R(\gamma',Z)\gamma'} J' - \ksp{R(\gamma',J)\gamma'}{Z'}\,.\\
\intertext{Für $t=0$ ist $J(0)=0$, also:}
\ksp{\frac D{\partial t} R(\gamma',J) \gamma'}Z(0) &= 0 + \ksp {R(\gamma',Z)\gamma'} J'(0) - 0 \\
&= \ksp{R(\gamma',J')\gamma'}Z(0)
\end{align*}
Da $Z$ beliebig war, gilt $J'''(0) = - \frac D{\partial t} R(\gamma',J) \gamma'(0) = - R(\gamma',J')\gamma'(0)$

\end{enumerate}
\end{beweis}

\begin{korrolar}
Falls $\ksp v w_p =0$, ($v,w$ also orthonormiert) gilt: $\ksp {R(v,w)v}w_p = K(p,\sigma) =$ Schnittkrümmung der von $v$ und $w$ aufgespannten Ebene $\sigma$, also
\[
\|J(t)\|_{\gamma(t)}^2 =t^2 - \frac 1 3 K(p,\sigma)t^4 + o(t^4)
\]
sowie
\[
\|J(t)\|_{\gamma(t)} = t - \frac 1 6 K(p,\sigma)t^3 + o(t^3)
\]

\end{korrolar}

\begin{beweis}
Die Formel für $\|J(t)\|_{\gamma(t)}$ folgt aus einem Koeffizientenvergleich der Taylorreihen:
\begin{align*}
f(t) &= a + bt + ct^2 + dt^3+\cdots \\
(f(t))^2 &= a^2 + 2abt + \cdots
\end{align*}
\end{beweis}

\paragraph{Anwendung} Länge von geodätischen Kreisen. $p\in M$, $v,w\in T_pM$, $v\bot w$, $\|v\|=\|w\|=1$, $f(r,\theta) \da \exp_p(r (\cos \theta \cdot v + \sin \theta \cdot w))$. Für ein festes $r$ heißt $K_r(\theta) = f(r,\theta)$ für $0\le\theta\le2\pi$ ein geodätischer Kreis\index{geodätischer Kreis} von Radius $r$.

Die Länge von $K_r$ ist $L(K_r)\da \int_{0}^{2\pi} \|\frac d{d\theta} K_r(\theta) \| d\theta = \int_0^{2\pi} \|\frac{\partial f}{\partial \theta}\|d\theta $, wobei $\frac{\partial f}{\partial \theta}$ ein Jacobi-Feld längs $\gamma_\theta(r) = \exp_p(rv(\theta))$ ist. Daher \[ L(K_r) = \int_0^{2\pi} [r - \frac 1 6 K(p,\sigma) r^3 + o(r^3)] d\theta = 2\pi r(1 - \frac 1 6 K(p,\sigma)r^2 + o(r^2))\,.\] Das ist die klassiche Formel von Betrand-Puiseux (1848) für Flächen in $\MdR^3$.

Umgekehrt hat man $K(p,\sigma) = \frac3{\pi r^3} (2\pi r - L(K_r) + \sigma(r^3))$ oder \[ K(p,\sigma) = \lim_{r\to0} \frac 3{\pi r^3}(2\pi r - L(K_r))\,.\]

Im euklidischen ist $L(K_r) = 2\pi r$, also $K(p,\sigma) = 0$. Im sphärischen ist $L(K_r) = 2\pi \sin r = 2\pi(r - \frac{r^3}{3!} + \cdots )$, also $K(p,\sigma) = +1$. Im hyperbolischen ist $L(K_r) = 2\pi \sinh r = 2\pi (r + \frac{r^3}{3!} +\cdots)$, also $K(p,\sigma) = -1$.

\end{beispiel}



\chapter{Riemann’sche Mannigfaltigkeiten als metrische Räume}

Sei $(M,\asp)$ eine differenzierbare Mannigfaltigkeit, also insbesondere ein topologischer Raum (nach Definition), der Hausdorff’sch ist, eine abzählbare Basis hat und und lokal euklidisch ist. Solche topologischen Räume sind metrisierbar.

Bisher waren die Konzepte „lokal“, etwa die Geodätischen, die Exponentialabbildung, der Krümmungstensor, die Jacobi-Felder. Für globale Aussagen benötigen wir zusätzliche topologische Voraussetzungen.

Ein Prototyp für eine solche Voraussetzung ist der Satz von Gauß-Bounet in der Flächentheorie: Gegeben eine kompakte Fläche $S$ im $\MdR^3$ ohne Rand und die Gauß-Krümmung $K$, so ist $\int_S KdA = 2\pi \chi(S) = 2\pi (2 - 2g)$, wobei $\chi$ die Euler-Charakteristik ist. Diese Gleichung verbindet links eine Aussage über die „lokale Geometrie“ mit rechts einer topologische Invariante.

Die einfachste Globale Frage ist: Gegeben zwei Punkte $p,q\in M$, gibt es einen stetigen Weg zwischen $p$ und $q$? Notwendig dafür ist: $M$ zusammenhängend\footnote{Ein topologischer Raum $X$ heißt zusammenhängend wenn $X$ nicht in zwei disjunkte, offene, nichtleere Teilmengen zerlegt werden kann. Dazu äquivalent: $X$ und $\emptyset$ sind die einzigen Teilmengen von $X$ die sowohl offen als auch abgeschlossen sind.}. Zusammenhängend ist auch hinreichend:
\begin{lemma}
Ist $M$ zusammenhängend, so ist $M$ auch wegzusammenhängend. Das heißt, dass zu $p,q\in M$ ein stetiger Weg $c:[0,1]\to M$ mit $c(0)=p$ und $c(1)=q$ existiert.
\end{lemma}

\begin{bemerkung}
In allgemeinen topologischen Räumen gilt: Aus wegzusammenhängend folgt zusammenhängend, aber aus zusammenhängend folgt nicht zwingend wegzusammenhängend. Ein Beispiel dafür ist $X\da [(0,-1),(0,1)] \cup \{(x,\sin \frac1x)\in\MdR^2\mid x>0\}$ mit der von $\MdR^2$ induzierten Topologie. (Für einen Beweis siehe: Singer-Therpe, Elementary Topology \& Geometry, Seite 53)
\end{bemerkung}

\begin{beweis}
Sei $p\in M$ und $A\da \{q \in M\mid q$ ist mit $p$ durch einen stetigen Weg verbindbar$\}$.
\begin{itemize}
\item  $A\ne \emptyset$, da $p\in A$: $c:[0,1] \to M$; $c(t)\da p$.
\item  $A$ ist offen: Ist $q\in A$ und $r\in B_\ep(q)$ (= normale Umgebung von $q$), dann ist $r \in A$.
\item $A$ ist abgeschlossen, also $M\setminus A$ ist offen: Ist $q\in M \setminus A$, $r\in B_\ep(q)$, dann ist $r\in M\setminus A$.
\end{itemize}
\end{beweis}

Daher sei in diesem Kapitel stets vorausgesetzt, dass $M$ zusammenhängend ist. 

Daraus folgt: Zwei beliebige Punkte $p,q\in M$ sind durch stückweise differenzierbare (bzw. stückweise geodätische) Wege verbindbar.

\begin{beweis}
Wähle stetigen Weg zwischen $p$ und $q$: $c:[0,1]\to M$. $c([0,1])$ ist kompakt. Diese Menge kann durch endlich viele total normale Umgebungen überdeckt werden. In  diesem Umgebungen lässt sich der Weg wie gewünscht abändern.
\end{beweis}

\begin{bemerkung}
Im Allgemeinen existiert zwischen zwei Punkten $p$ und $q$ einer Riemann’schen Mannigfaltigkeit keine Geodätische! Etwa in $(\MdR^2\setminus\{0\},\kan)$, wo Geodätische Geradenstücken entsprechen, gibt es keine Geodätische zwischen $(0,-1)$ und $(0,1)$.
\end{bemerkung}

Setze $\Omega_{qp}\da \{$stückweise differenzierbare Kurven zwischen $p$ und $q\}$.

\begin{satz}[Längenmetrik]
$(M,\asp)$ sein eine Riemann’sche Mannigfaltigkeit. 
\[
d : 
\begin{aligned}
M \times M &\to \MdR_{\ge 0} \\
(p,q) &\mapsto \inf_{\mathclap{c\in \Omega_{pq}}} L(c)
\end{aligned}
\]
Dann ist $(M,d)$ ein metrischer Raum, also es gilt für $p,q,r\in M$: 
\begin{enumerate}
\item $d(p,q) = d(q,p) \ge 0$
\item $d(p,q) \le d(p,r) + d(r,q)$
\item $d(p,q) = 0 \gdw p = q$
\end{enumerate}
\end{satz}

\begin{beweis}
\begin{enumerate}
\item „rückwärts laufen“: $c[0,l]\to M$, $t\mapsto c(t)$, sei Kurve zwischen $p$ und $q$, also $c\in \Omega_{pq}$. Dann ist $\tilde c(t) \da c(t-l) \in \Omega_{qp}$ und $L(c) = L(\tilde c)$.
\item Da $\Omega_{pq}$ eine Obermenge der Wege von $p$ nach $q$ über $r$ ist, gilt $\inf_{c\in\Omega_{pq}} L(c) \le \inf_{c\in\Omega_{pr}} L(c) + \inf_{c\in\Omega_{rq}} L(c)$
\item Klar: Ist $p=q$, so ist hat der konstate Weg $c[0,1]\to M$; $t\mapsto p$ die Länge 0, also $d(p,q)=0$.

Annahme $p\ne q$. Wähle eine normale Umgebung $U_\ep(p)$ um $p$ mit $q\notin U_\ep(p)$ Dann gilt für eine beliebiges $c\in \Omega_{pq}$ nach Satz \ref{geolokmin}: $L(c) \ge \ep \folgt d(p,q) \ge \ep$.
\end{enumerate}
\end{beweis}

\begin{korrolar}
\begin{enumerate}
\item  Die Topologie des metrischen Raumes $(M,d)$ ist äquivalent zur ursprünglich auf $M$ gegebenen Topologie (also $U$ ist offen in $M$ $\gdw$ $U$ ist offen in $(M,d)$). Das heißt: Riemann’sche Mannigfaltigkeiten sind metrisierbar.
\item Für $p_0\in M$ ist $d_{p_0}: M \to \MdR$; $d_{p_0}(p) \da d(p_0,p)$ ist stetig (gilt für beliebige metrische Räume).
\item Ist $M$ kompakt, so ist der Durchmesser von $M$ beschränkt: \[ \operatorname{Diam}(M) = \sup_{p,q\in M} d(p,q) < \infty \]
\end{enumerate}
\end{korrolar}

\begin{beweis}
\begin{enumerate}
\item Nach Satz \ref{geolokmin} sind normale offene Bälle von genügend kleinem Radius $r$ identisch mit metrischen Bällen von Radius $r$ (bezüglich $d$):
\[
B_r^{(d)}(p) \da \{q\in M \mid d(p,q) < r\} = \exp_p(B_r(0))
\] wobei $B_r(0) = \{ v\in T_pM \mid \| v\|< r\}$.
\item $|d_{p_0}(p) - d_{p_0}(q)| = |d(p_0,p) - d(p_0,q)| \le d(p,q)$
\item Seien $p,q$ beliebig aus $M$ kompakt.
\[
d(p,q) \le d(p,p_0) + d(p_0,q) \le 2 \max_{r\in M} d(p_0,r) < \infty
\]
Das Maximum wird angenommen, da $M$ kompakt und $d_{p_0}$ stetig ist.
\end{enumerate}
\end{beweis}

\begin{definition}
$(M,d)$ ist vollständig\index{vollständig} genau dann, wenn jede Cauchy-Folge konvergiert.

$(M,\asp)$ ist geodätisch vollständig\index{geodätisch vollständig} genau dann, wenn für jedes $p\in M$ die Exponentialabbildung $\exp_p$ auf ganz $T_pM$ definiert ist, also jede Geodätische $\gamma(t)$ mit $\gamma(0) = p$ ist für alle $t\in \MdR$ definiert.
\end{definition}

\begin{satz}[Hopf-Rinow, 1931]
Sei $M$ eine zusammenhängende Riemann’sche Mannigfaltigkeit und $p\in M$. Folgende Aussagen sind äquivalent:
\begin{enumerate}
\item $\exp_p$ ist auf ganz $T_pM$ definiert.
\item Jede abgeschlossene Teilmenge $A\subset M$ mit beschränktem Durchmesser ist kompakt.
\item Der metrische Raum $(M,d)$ ist vollständig, das heißt, jede Cauchy-Folge konvergiert.
\item $M$ ist geodätisch vollständig.
\end{enumerate}
In diesem Falle gilt: Für jeden Punkt $q\in M$ existiert mindestens eine Geodätische $\gamma$, welche $p$ und $q$ verbindet und für die gilt: $L(\gamma) = d(p,q)$, das heißt, $\gamma$ realisiert die kürzeste Verbindung zwischen $p$ und $q$.
\end{satz}

\begin{beweis}
Die Vorgehensweise ist: $1 \folgt 2 \folgt 3 \folgt 4 \folgt 1$.
\begin{enumerate}[(1)$\folgt$(2)]
\item[(1)$\folgt$(2)] $A$ ist abgeschlossen. \[\operatorname{Diam}(A) = \sup_{a,b\in A}d(a,b) \le c < \infty\,.\]
Für ein festes $q_0\in A$ gilt für alle $q\in A$: \[d(p,q) \le d(p,q_0) + d(q_0,q) \le d(p,q_0) + c \ad R\,.\]
Das heißt: $A \subset \overline{B_{2R}(p)} \gleichwegen{(1)} \exp_p(\overline{B_{2R}(0)})$ ist kompakt, da $\exp_p$ stetig und $\overline{B_{2R}(0)}\subset T_pM$ kompakt ist. Also ist $A$ als abgeschlossene Teilmenge einer kompakten Menge selbst auch kompakt.
\item[(2)$\folgt$(3)] 
Eine Cauchy-Folge $\{p_n\}_{n\in \MdN}$ ist beschränkt: Wähle $\ep>0$, so gilt $d(p_n,p_m) < \ep$ für alle $m,n \ge n_0(\ep)$. Also ist nach (2) $\{p_n\}_{n\in \MdN}$ in einer kompakten Menge enthalten. Insbesondere hat $\{p_n\}_{n\in \MdN}$ eine konvergente Teilfolge. Da $\{p_n\}_{n\in \MdN}$ eine Cauchy-Folge ist, konvergiert $\{p_n\}_{n\in\MdN}$ selbst.
\item[(3)$\folgt$(4)] 
Sei $\gamma: I\to M$ eine normale Geodätische in $M$. Zu zeigen: $I$ ist offen und abgeschlossen in $\MdR$, also $I=\MdR$.

$I$ ist offen (und nicht leer): Aus der (lokalen) Eindeutigkeit und Existenz von Geodätischen (Satz \ref{eindgeo}) folgt: Ist $\gamma(t_0)$ definiert, so auch $\gamma(t_0+t)$ für genügend kleine $t$.

$I$ abgeschlossen: Sei $(t_n)_{n\in \MdN}$ eine monoton wachsende Folge in $I$, welche gegen $t_*$ konvergiert. Zu zeigen: $t_* \in I$. Zunächst ist für $m\ge n$ \[d(\gamma(t_n),\gamma(t_m)) \le L(\gamma|_{[t_n,t_m]}) = |t_n - t_m|\,.\] Daher ist $(\gamma(t_n))_{n\in\MdN}$ eine Cauchy-Folge in $M$ und nach Voraussetzung(3) konvergent.

Sei $p\da \lim_{n\to\infty} \gamma(t_n)$. Sei $W(p)$ eine total normale Umgebung um $p$. Satz \ref{totnormumg} besagt: Es existiert ein $\delta > 0$, so dass jede normale Geodätische, welche in $W(p)$ beginnt, auf $(-\delta,\delta)$ definiert ist. Wähle $n$ so groß, dass $|t_n - t_*|<\frac\delta2$ und $\gamma(t_n) \in W(p)$. Dann ist $\gamma(t)$ definiert für alle $t$ mit $|t_n - t| < \delta$, also insbesondere für $t_*$, das heißt $t_*\in I$ und $I$ ist abgeschlossen.
\item[(4)$\folgt$(1)] 
Klar.
\end{enumerate}
Es gelte nun (1), und wir zeigen die letzte Aussage des Satzes.

\begin{enumerate}[1. {Schritt:}]
\item Wir finden einen Kandidatenen für die Geodätische $\gamma$. Sei $r\da d(p,q)$ und für $0< \delta < r$ sei $B_\delta(p)$ ein normaler Ball um $p$ mit geodätischer Sphäre $S_\delta(p) = \partial B_\delta(p)$ als Rand. $S_\delta(p)$ ist kompakt.

Die Idee ist, $x_0$ als denjenigen Punktauf dem Rand zu wählen, wo die stetige Funktion $d_q|_{S_\delta} \da d(q,\cdot)|_{S_\delta}$ ein Minimum annimmt. Dann existiert nach Voraussetzung (1) ein $v\in T_pM$ mit $\|v\|=1$ und $x_0 = \exp_p(\delta v)$.

Definiere $\gamma(s) \da \exp_p(s v),$ $s\in \MdR$, was nach Voraussetzung (1) geht.

\item Wir zeigen, dass $\gamma$ die Punkte $p$ und $q$ verbindet. Zu zeigen ist also: $\gamma(r)=q$ (bzw. $d(q,\gamma(r)) = 0$). Betrachte dazu die Menge
\[A\da \{ s\in [0,r] \mid d(\gamma(s),q) = r - s \quad (*) \}\,.\]
Wir zeigen, dass $A = [0,r]$. $A$ ist abgeschlossen. $A \ne \emptyset$, da $0\in A$. Sei $s_0 \da \max A$.

Annahme: $s_0 < r$. Betrachte wieder normalen Ball $B_{\delta'}(\gamma(s_0))$ mit Rand $S'$ um $\gamma(s_0)$ mit $\delta'$ so klein, dass $q\notin B_{\delta'}(\gamma(s_0))$. $x_0'$ sei ein Punkt auf $S'$ in dem, $d_q|_{S'}$ ein Minimum annimmt.

\paragraph{Behauptung:} Es gilt $x_0' = \gamma(s_0 + \delta')\quad (**)$.
\begin{beweis}[der Behauptung]
Zunächst ist 
\[d(\gamma(s_0),q) = \delta' + \min_{x\in S'} d(x,q) = \delta' + d(x_0',q)\,.\]
Nach Voraussetzung ($(*)$ und Definition von $s_0$) ist $d(\gamma(s_0),q) = r - s_0$, also \[r-s_0 = \delta' + d(x_0',q)\,.\quad(***)\]
Weiter mit der Dreiecks-Ungleichung: \[d(p,x_0') \ge d(p,q) - d(q,x_0') \gleichwegen{(***)} r - (r - s_0 - \delta') = s_0 + \delta'\]
Ebenso gilt für die stückweise differenzierbare Kurve $c$: \[d(p,x_0') \le d(p,\gamma(s_0)) + d(\gamma(s_0),x_0') \le s_0 + \delta '\]
Die Kurve $c$ ist also minimierend und somit eine Geodätische, hat also keinen „Knick“ bei $s_0$. Daher gilt: $x_0' = \gamma(s_0+\delta')$.
\end{beweis}
Aus $(**)$ und $(***)$ folgt:
\[
r - (s_0 + \delta ') \gleichwegen{(***)} d(x_0',q) \gleichwegen{(**)} d(\gamma(s_0 + \delta'),q)
\]
Also gilt $(*)$ für $\delta_0 + \delta' > s_0$ im Widerspruch zur Definition von $s_0$. Daher ist die Annahme $s_0 < r$ falsch und $s_0 =r$.
\end{enumerate}
\end{beweis}

\begin{korrolar}
Eine zusammenhängende kompakte Riemann’sche Mannigfaltigkeit ist (geodätisch) vollständig.
\end{korrolar}

\begin{beweis}
$(M,d)$ ist vollständig, also nach Hopf-Rinow geodätisch vollständig.
\end{beweis}

\begin{korrolar}
$M$ sei eine zusammenhängende und vollständige, aber nicht kompakte, Riemann’sche Mannigfaltigkeit. Dann existiert ein geodätischer Strahl in $M$, also eine Geodätische $\gamma:[0,\infty)\to M$, welche für alle $t\in [0,\infty)$ minimierend ist: $d(\gamma(t),\gamma(0)) = t$.
\end{korrolar}

\begin{beweis}
Wähle Folge $(q_n)_{n\in\MdN}$ in $M$, so dass $d(q_0,q_n)\to \infty$ für $n\to\infty$. Schreibe $q_n = \exp(t_n v_n)$, $\|v_n\|=1$. Die Folge $(v_n)_{n\in\MdN} \subset \{ w \in T_{q_0}M\mid \|w\|=1\}$ kompakt, also hat $(v_n)_{n\in\MdN}$ eine konvergente Teilfolge, ohne Einschränkung sei diese $(v_n)_{n\in \MdN}$: $v_n \to v$.

Nun haben wir die Geodätischen $\gamma_n(t) \da \exp_{q_0}(tv_n)$ und  $\gamma(t) \da \exp_{q_0}(tv)$. Zu zeigen ist $d(\gamma(t_1), \gamma(t_2)) = |t_1 - t_2|$: Aber $\lim_{n\to\infty} \gamma_n(t) = \lim_{n\to\infty} \exp_{q_0}(tv_n) = \exp_{q_0}(\lim_{n\to\infty}tv_n) = \exp_{q_0}(tv) = \gamma(t) \folgt$ Behauptung.
\end{beweis}

\section{Schnittort einer vollständigen Riemann’schen Mannigfaltigkeit}
\label{schnittort}

\begin{lemma}
$M$ sei eine zusammenhängende vollständige Riemann’sche Mannigfaltigkeit und $\gamma:[a,b]\to M$ sei eine normale Geodätische.

\begin{enumerate}
\item Falls keine weitere Geodätische zwischen $\gamma(a)$ und $\gamma(b)$ existiert, die kürzer ist als $\gamma$, dann ist $\gamma$ minimierend auf $[a,b]$.
\item Falls eine Geodätische $c\ne \gamma$ zwischen $\gamma(a)$ und $\gamma(b)$ mit $L(c)=L(\gamma)$ existiert, so ist $\gamma$ nicht mehr minimierend auf $[a,b+\ep]$ für ein $\ep >0$.
\item Ist $\gamma$ minimierend auf einem Intervall $I$, so ist sie auch auf $J\subseteq I$ minimierend.
\end{enumerate}
\end{lemma}

\begin{beweis}
\begin{enumerate}
\item 
Nach Hopf-Rinow existiert eine minimierende Geodätische $\gamma^*$ zwischen $\gamma(a)$ und $\gamma(b)$. Es ist dann $L(\gamma^*)\le L(\gamma)$, also nach Voraussetzung $L(\gamma*)=L(\gamma)$, also muss $\gamma$ minimierend sein.
\item Sei $c$ eine Geodätische zwischen $\gamma(a)$ und $\gamma(b)$, $c\ne \gamma$, mit $L(c) = L(\gamma)$. Wähle $\delta>0$, so dass $W=W(\delta)$ eine total normale Umgebung von $\gamma(b)$ (siehe Satz \ref{totnormumg}). Betrachte die Kurve
\[
\alpha(t) \da 
\begin{cases}
c(t), & t\in [a,b] \\
\gamma(t), & t\in [b,b+\frac\delta2]
\end{cases}
\]
$\alpha$ verbindet $\gamma(0)$ und $\gamma(b+\frac\delta4)$. Da $W$ total normal ist existiert eine minimale Geodätische zwischen $\alpha(b-\frac\delta4)$ und $\alpha(b+\frac\delta4)$. $\alpha$ ist keine Geodätische (wegen dem „Knick“ bei $\gamma(b)$), also ist die Länge des minimalen geodätischen Segments zwischen $\alpha(b-\frac\delta4)$ und $\alpha(b+\frac\delta4)$ echt kleiner als das entsprechende Stück von $\alpha$. Daher existiert eine Kurve von $\gamma(a)$ nach $\gamma(\frac\delta4)$ die kürzer ist als $\alpha|_{[a,b+\frac\delta4]}$. Konstruktion ist $L(\gamma_{[a,b+\frac\delta4]}) = L(\alpha|_{[a,b+\frac\delta4]})$ und somit nicht mehr kürzeste nach $\gamma(b)$.
\item Annahme: $\gamma$ nicht minimierend auf $J\subseteq I$, dann wäre $\gamma$ nicht minimierend auf $J$.
\end{enumerate}
\end{beweis}

Für $p\in M$, $v\in T_pM$, $\|v\|=1$ sei $\gamma_v \da \exp_p(tv)$ die eindeutige normale Geodätische mit $\gamma_v(0)=p$, $\gamma_v'(0)=v$. Setze $I_v \da \{ t\in [0,\infty) \mid d(\gamma_v(t), \gamma_v(0))=t\}$, das heißt: $\gamma$ minimierend zwischen $\gamma_v(0)$ und $\gamma_v(t)$. $I_v \ne \emptyset$, da $[0,\ep]\subseteq I_v$ für genügend kleines $\ep$. $I_v$ ist abgeschlossen, da $d(\gamma_v(0),\cdot)$ stetig. Es gilt also entweder $I_v = [0,\infty)$, also $\gamma_v$ ist geodätischer Strahl, oder es existiert $s(v)>0$, so dass $I_v=[0,s(v)]$. In diesem Fall heißt $\gamma_v(s(v))$ Schnittpunkt\index{Schnittpunkt} von $p$ längs $\gamma_v$.

\begin{bemerkung}
\begin{enumerate}[(a)]
\item Es gilt (ohne Beweis): Die Abbildung 
\[
s:
\begin{aligned}
E_pM \da \{v \in T_pM \mid \|v\|=1\} &\to \MdR\cup\{\infty\} \\
v &\mapsto s(v)
\end{aligned}
\] ist stetig.
\item Für kompakte $M$ ist $s(v) < \infty$ für alle $v\in E_pM$, $p\in M$.
\item Ist $M$ nicht kompakt, so existiert nach dem Korrolar ein geodätischer Strahl und somit $p\in M$ und $v\in T_pM$, so dass $s(v)=\infty$.
\end{enumerate}
\end{bemerkung}

Für ein beliebiges, aber festes $p\in M$ ist
\[
U_p \da \{ w\in T_pM\setminus \{0\} \mid \|w\| < s(\frac w {\|w\|})\} \cup \{0\}
\] eine offene Umgebung von $0\in T_pM$. Der Rand von $U_p$, $\partial U_p$, ist die Menge $\{w\in T_pM | \|w\| = s(\frac w{\|w\|})\} = \{s(v) \cdot v \in T_pM \mid v \in E_pM\}$. Der Schnittort\index{Schnittort} von $p\in M$ ist \[ \cut(p) \da \exp_p(\partial U_p)\,.\]

\begin{beispiel}
In $(S^n,\kan)$ sind die Geodätischen mit Bogenlänge parametrisierte Großkreise der Länge $2\pi$. Also sind Geodätische $\gamma_v(t)$ minimierend für $t<\pi$. Also ist für alle $p\in S^n_1$ ist $U_p=\{w\in T_pM \mid \|w\| <\pi\}$, also \[\exp_p(U_p) = \{q\in S^n \mid d(p,q)<\pi \} = S^n\setminus \{-p\}\] und \[\cut(p) = \{q\in S^n \mid d(p,q)=\pi \} = \{-p\}\,.\] Man kann also die Sphäre disjunkt zerlegen in $\exp_p(U_p)$ und $\exp_p(\partial U_p) = \cut(p)$. Dies gilt allgemein!
\end{beispiel}

\begin{bemerkung}
Es gilt (ohne Beweis) der Satz: (Berker 1980)

Ist $(M,\asp)$ eine zusammenhängende kompakte Riemman’sche Mannigfaltigkeit mit $\operatorname{Diam}(M)=\pi$ und $\cut(p) = \{\text{Punkt}\}$ für alle $p\in M$, so ist $(M,\asp)$ isometrisch zu $(S^n,\kan)$. Solche Mannigfaltigkeiten heißen „Wiedersehen-Mannigfaltigkeit“.
\end{bemerkung}

\begin{satz}[Zerlegungssatz]
\label{zerlsatz}
Sei $M$ eine zusammenhängende, vollständige Riemman’sche Mannigfaltigkeit. Dann gilt für jeden Punkt $p\in M$ die disjunkte Zerlegung
\[
M = \exp_p(U_p) \uplus \cut(p)\,.
\]
\end{satz}

\begin{beweis}
Nach Hopf-Rinow gibt es für jeden Punkt $q\in M$ eine minimale Geodätische $\gamma_v$ zwischen $p$ und $q$ mit $q=\gamma_v(t_0 v)$, $t_0\le s(v)$, $\|v\|=1$. Insbesondere ist $t_0 v \in \overline U_p = U_p \cup \partial U_p$. Da $q$ beliebig ist $M \subset \exp_p(\overline U_p) = \exp_p(U_p) \cup \exp_p(\partial U_p) = \exp_p(U_p) \cup \cut(p) \subset M$.

Noch zu zeigen ist: $\exp_p(U_p) \cap \cut(p) = \emptyset$. Wir nehmen an, dass $q \in \exp_p(U_p) \cap \cut(p)$.

Da $q\in \exp_p(U_p)$ existiert eine minimierende Geodätische $\gamma:[a,b] \to M$ mit $\gamma(a) = p$ und $\gamma(b) = q$. $U_p$ ist offen also $\gamma$ auch minimierend auf $[a,b+\ep]$ für $\ep$ genügend klein.

$q\in \cut(p)$ heißt: $q$ ist Schnittpunkt einer von $p$ ausgehenden Geodätischen, das heißt es existiert eine minimierende Geodätische $c:[\alpha,\beta] \to M$ mit $c(\alpha)=p$, $c(\beta)=q$, die nach $c(\beta)$ nicht mehr minimierend ist (insbesondere $c\ne \gamma$), aber mit $L(c|_{[\alpha,\beta]}) = L(\gamma|_{[a,b]}) = d(p,q)$. Nach Lemma 2 (2) angewandt auf $\gamma$ ist die Geodätische $\gamma$ nicht mehr minimierend nach $\gamma(b)$, im Widerspruch zur Annahme!
\end{beweis}

\subsection*{Weitere Eigenschaften von $\cut(p)$}
\begin{enumerate}[(a)]
\item $\cut(p)$ hat keine inneren Punkte.
\begin{beweis}
Wir nehmen an es existiert ein $q$ im Inneren von $\cut(p)$ längs $\gamma$. Dann existiert ein $q' \in \gamma \cap \cut(p)$ „vor“ $q$. Nach Definition des Schnittortes existiert eine minimierende Geodätische $c$ zwischen Punkten $p$ und $q'$.
\begin{enumerate}[1. {Fall:}]
\item $c=y$: nach Definition des Schnittortes ist dann $\gamma$ nicht minimierend nach $q'$, im Widerspruch zur Annahme.
\item $c\ne y$: Nach Lemma 2 (2) ist $\gamma$ nicht mehr minimierend „nach“ $q'$, im Widerspruch zur Annahme.
\end{enumerate}
\end{beweis}
\item $\exp_p|_{U_p}$ ist injektiv. Es gilt sogar ohne Beschränkung: $\exp_p|_{U_p}$ ist eine differenzierbare Einbettung. Das heißt: für $q\in \exp_p(U_p)$ existiert genau eine minimierende Geodätische von $p$ nach $q$.
\begin{beweis}
Sei $q\in\exp_p(U_p)$ und $\exp_p(v_1) = q = \exp_p(v_2)$. Nehmen wir an, dass $v_1\ne v_2$, so hat man zwei minimierende Geodätische $\gamma_1\ne \gamma_2$ zwischen $p$ und $q$, das heißt nach Definition des Schnittortes bzw. Lemma 2(4), dass $q\in \cut(p)$, im Widerspruch zur Annahme.
\end{beweis}
\item Für ein kompaktes $M$ nimmt die stetige Funktion $s: E_pM \to \MdR$ ein Maximum bzw. Minimum an und ist somit beschränkt. Also ist $\overline {U_p} = \{tv \in T_pM \mid v\in E_pM, 0\le t\le s(v)\}$ homöomoph zum Einheitsball $B_p \da \{tv \in T_pM | v\in E_pM, 0\le t \le 1\}$.
\item 
Da $\exp_p: \overline{U_p} \to M$ surjektiv ist (Satz \ref{zerlsatz}) und auf $U_p$ injektiv ist, erhält man eine kompakte Riemann’sche Mannigfaltigkeit topologisch dadurch, dass man die Randpunkte eines Einheitsballs „geeignet“ identifiziert (beispielsweise werden für $S^n$ alle Punkte identifiziert). Die topologische Komplexität einer kompakten Mannigfaltigkeit steckt also im Schnittort.
\end{enumerate}

\begin{definition}
Der \index{Injektivitätsradius}Injektivitätsradius von $p$ ist definiert als
\[ \operatorname{Inj(p)} \da \dist (p,\cut(p))\,. \]
Der Injektivitätsradius von $M$ ist dann
\[ \operatorname{Inj(M)} \da \inf_{p\in M} \inj (p)\,.\]
\end{definition}

\begin{beispiel}
$\inj (S^n) = \pi$ und $\inj (p) = \pi$ für jeden Punkt $p \in S^n$.
\end{beispiel}

%$2\cdot \inj (p)$ ist die Länge der kürzesten von $p$ ausgehenden geschlossenen Geodätischen, falls es diese gibt.

\begin{satz}
(Ohne Beweis, vergleiche do Carmo, Kapitel 13, Proposition 2.2)

\begin{enumerate}
\item Sei $q\in \cut(p)$ mit $l\da d(p,q) = \dist (p,\cut(p)) = \inj (p)$. Dann gilt entweder, dass es eine Geodätische $\gamma$ zwischen $p$ und $q$ gibt, so dass $q$ zu $p$ längs ($\gamma$) konjugiert ist (das heißt, es gibt ein Jacobi-Feld $J$ längs $\gamma$, so dass $J(p)=0=J(q)$), oder es existieren genau zwei minimierende Geodätische $\gamma$ und $\tau$ zwischen $p$ und $q$ mit $\gamma'(l) = - \tau'(l)$.
\item Falls für $p\in M$ gilt: $\inj(p)=\inj(M)$, also $\dist(p,\cut(p))$ minimal in $M$, so gilt: Entweder ist $q$ zu $p$ konjugiert längs einer minimierenden Geodätischen oder $q$ ist Mittelpunkt einer geschlossenen Geodätischen (das heißt: differenzierbar in $p$ und $q$).
\end{enumerate}
\end{satz}

Beispiel für konjungierte Punkte: Breitner-Fläche (Vesperdose)

\begin{beispiel}
\begin{enumerate}
\setcounter{enumi}{-1}
\item $(\MdR^n,\kan)$ hat keine Schnittpunkte, da die Geodätischen Geraden sind. Also $\cut(p) = \emptyset$ für jeden Punkt $p\in \MdR^n$, und laut Zerlegungssatz gilt $\MdR^n=\exp_p(T_pM)$.
\item Hyperbolische Ebene $(H^2,\frac1{y^2}
\bigl(\begin{smallmatrix}
1 & 0 \\ 0 & 1
\end{smallmatrix}\bigr))$: Auch hier gilt $\cut(p) = \emptyset$ für jeden Punkt $p\in H^2$, also $H^2 = \exp_p(T_pH^2)$.
\begin{bemerkung}
Allgemeiner gilt der Satz von Hadamard-Cartan: Für einfach eine einfach zusammenhängende und zusammenhängende Riemann’sche Mannigfaltigkeit $M$ mit nichtpositiver Schnittkrümmung. So gilt $\cut(p) = \emptyset$ für alle $p\in M$.
\end{bemerkung}
\item $(\mathbb P ^n,\kan) = (\faktor{S^{\mathrlap{n}}}\sim,$ die von $S^n$ indizierte Metrik$)$, wobei $p\sim q$ genau dann, wenn $p$ und $q$ Antipoden sind. Hier sind Geodätische minimierend für $d(p,\cdot)<\frac\pi 2$. Sei $pr:S^n\to \mathbb{P}^n$; $pr(p)\da [p] = \{p,-p\}$ die Projektion von $S^n$ auf $\mathbb{P}^n$. Dann ist 
\[\cut(p)=pr(\text{„Äquator“}) = pr(q\in S^n\mid d(p,q)=\frac\pi2) = \faktor{S^{n-1}}\sim = \mathbb{P}^{n-1}\,.\]
Weiter ist $\exp_p(U_p)$ die offene „obere Hemisphäre“ und damit diffeomorph zu $\MdR^n$, etwa durch die Zentralprojektion
$Z: S_-^n \to \MdR^n$ oder die Orthogonalprojektion $O:S^n_- \to D^n = \{x\in\MdR^n \mid \|x\|<1\}$.

Nahc dem Zerlegungssatz gilt also topologisch betrachtet:
\[ \mathbb{P}^n = \exp_p(U_p) \uplus \cut(p) \cong \MdR^n \uplus \mathbb{P}^{n-1} \]
Mit Induktion nach $n= \dim \mathbb{P}^n$ folgt mit $\MdR^0\da\{0\}$:
\[ \mathbb{P}^n = \MdR^n \uplus \MdR^{n-1} \uplus \cdots \uplus \MdR^1 \uplus \MdR^0 \]
Beispielsweise ist $\mathbb{P}^2 = \MdR^2 \uplus \mathbb P^1 = \MdR^n \uplus \MdR^1 \uplus \MdR^0$. Dies nennt man auch eine Zellenzerlegung des projektiven Raumes.
\item Rotationszylinder.
\item Flacher Torus $T^2 = S^1 \times S^1 = \faktor\MdR\MdZ \times \faktor\MdR\MdZ$.
\end{enumerate}
\end{beispiel}

\begin{bemerkungen}
\item Der Schnittort ist im allgemeinen nicht differenzierbar, sondern in zweidimensionalen Mannigfaltigkeiten ein Graph, also ein simplizialer 1-Komplex.
\item Für obige Beispiele $M$ gilt immer, dass die Schnittorte eines jeden Punktes gleich aussehen. Dies liegt daran, dass $M$ jeweils homogen war, also für beliebige Punkte $p,q\in M$ exisistiert eine Isometrie $\varphi$ mit $\varphi(p)=q$.
\end{bemerkungen}

\begin{beispiel}
für einen Raum mit $\cut(p) \ne \cut(q)$: Rotationsellipsoid mit Rotationsachse $z$-Achse.
\end{beispiel}

\section{Volumenberechnung mit dem Zerlegungs-Satz}

Sei zuerst $G\subset M$ ein Gebiet, also offen, zusammenhängend und relativ kompakt, das ganz in einer Karte $(U,\varphi)$ liegt (mit $\varphi(p) = (x^1,\ldots,x^n)$ als Koordinaten).

In der linearen Algebra bezeichnen wir das Volumen des von $a_1,\ldots,a_n\in\MdR^n$ aufgespannten Parallelepipeds als $V=\sqrt{\det(\ksp {a_i} {a_j})}$ (Gramische Determinante).

\begin{definition}
Das Volumen\index{Volumen} des Gebietes bezeichnen wir als
\[
\vol(G) \da \int_{\varphi(G)} \sqrt{\det (g_{ij}(\varphi^{-1}(x)))} dx\mathrlap{^1}\ldots dx^n \ad \int_{\varphi(G)}d\vol
\] mit $g_{ij}(p) = \ksp{\frac{\partial}{\partial x^i}|_p}{\frac{\partial}{\partial x^j}|_p}_p$.
\end{definition}

Nach der Substitutionsregel für Integrale gilt: $\vol(G)$ ist unabhängig von der gewählten Karten und invariant unter Isometrien.

\begin{bemerkung}
Um das Volumen eines kompakten Gebietes $G$, das nicht ganz in einem Kartengebiet liegt, zu definieren benutzt man eine Überdeckung von $G$ durch (endlich viele) Karten $(U_i,\varphi_i)$ und eine zugehörige Zerlegung der Eins $(f_i)_{i\in I}$. Dann setzt man:
\[
\vol (G) \da \sum_{i\in I} \int_{\varphi_i(G\cap U_i)} f_i d \vol_i 
\]
wobei $d\vol_i \da \sqrt{\det(g_{ij}(\varphi^{-1}(p)))}dx_i^1\ldots x_i^n$.

Man kann zeigen, dass $\vol(G)$ nicht von der Wahl der Karten und der entsprechenden Zerlegung der Eins abhängt.
\end{bemerkung}

\begin{beispiele}
\item In der Flächentheorie ist $d\vol = dA = \sqrt{EG-F^2}du dv$.
\item $(\MdR^n,\text{id})$. Hier ist $d\vol = dx^{\mathrlap 1}\ldots dx^n$ bezüglich cartesischen Koordinaten.

Bezüglich Polarkoordinaten $(t,u)$, $u\in S^{n-1}$, ist $d\vol = t^{n-1}dtd\sigma$, wobei $d\sigma$ das Volumenelement auf der Einheitssphäre $S^{n-1}\subset \MdR^n$ ist.
\end{beispiele}

Wir wissen:
\begin{itemize}
\item $M=\exp_p(U_p) \uplus \cut(p)$.
\item $\cut(p)$ hat keine inneren Punkte, also ist $\vol(M) = \vol(\exp_p(U_p))$.
\item $\exp_p|_{U_p}$ ist ein Diffeomorphismus auf das Bild in $M$. 
\end{itemize}
Das heißt: Wir können $\exp_p^{-1}: \exp_p(U_p) \to T_pM\cong \MdR^n$ als Karte benutzen.

Sei $c(t)\da \exp_p(tu)$ eine normale Geodätische und $\{u,e_2,\ldots,e_n\}$ eine Orthonormalbasis von $T_pM$. Weiter seien $Y _i(t)$, $i=1,\ldots,n$, die eindeutigen Jacobifelder längs $c(t)$ mit $Y _i(0)=0$ und $Y _i(0)=e_i$. Es gilt (siehe \ref{schnittort}): $d\exp_p|_{tu}(u)= c'(t)$ und $d\exp_p|_{tu}(te_i)=Y _i(t)$, $i=2,\ldots,n$.

$(t=x^1,\ldots,x^n)$ seien die Koordinaten in $T_pM$ bezüglich der Orthonormalbasis $\{u,e_2,\ldots,e_n\}$. Dann gilt:
\begin{align*}
\frac{\partial}{\partial t}|_{c(t)} &= c'(t) \text{ und } c'(0)=u \\
\frac{\partial}{\partial x^i}|_{c(t)} &= \frac{\partial}{\partial s}|_0 \exp_p(tu+se_i) = d\exp_p|_{tu}(e_i) = \frac 1 t Y _i(t)
\end{align*}
Also:
\begin{align*}
g_{11}(c(t)) &= \ksp{\frac\partial{\partial t}}{\frac\partial{\partial t}}_{c(t)} = 1 \\
g_{1k}(c(t)) &= \ksp{\frac\partial{\partial t}}{\frac\partial{\partial x^k}}_{c(t)} = 0 \\
g_{ij}(c(t)) &= \ksp{\frac\partial{\partial x^i}}{\frac\partial{\partial x^j}}_{c(t)} = \frac 1 {t^2} \ksp{Y _i(t)}{Y _j(t)}_{c(t)} \\
\end{align*}
und
\[
\sqrt{\det (g_{ij}(c(t)))} = \frac 1 {t^{n-1}} \sqrt{\det \underbrace{\ksp{Y _i}{Y _j}_{c(t)}}_{\mathclap{\text{$(n-1)\times(n-1)$-Matrix}}}} \ad J(u,t)
\]
Also:
\begin{align*}
d\vol &= \sqrt{\det(g_ij)}dx^{\mathrlap 1}\ldots dx^n \\
 &= J(u,t)x^{\mathrlap 1}\ldots dx^n = J(u,t)t^{n-1}dt du
\end{align*}
wobei $du$ das euklidische Volumen-Element auf der Einheitssphäre $S^{n-1}\subset T_pM$ ist.

Daraus können wir folgende Volumenformel ableiten:

Ist $s(u) \in \MdR\cup\{\infty\}$ der Abstand von $p$ zum Schnittort von $p$ in Richtung $u\in T_pM$, $\|u\|=1$, so gilt:
\begin{align*}
\vol(M) &= \vol(\exp_p(U_p)) \\
&= \int_{S^{n-1}} \int_0^{s(u)} J(u,t) t^{n-1} dt du
\end{align*}
Allerdings kann man im allgemeinen $J(u,t)$ nicht explitzit berechnen, nur abschätzen.

Einfache Beispiele sind Mannigfaltigkeiten mit konstanter Krümmung.
\begin{enumerate}
\item $(\MdR^n,\kan)$ mit Schnittkrümmung konstant 0. Hier ist $Y _i(t) = t E_i(t)$, wobei $E_i(t) = \|_{c(t)}e_i$ ein Parallelfeld längs $c$ ist. Hier ist $J(t,u)=1$. 
\item $(S^n,\kan)$. Hier ist $Y_i(t) = \sin(t) E_i(t)$. Also 
\[ \vol(S^n)= \int_{S^{n-1}} \int_0^\pi \left(\frac{\sin(t)}{t}\right)^{n-1} t^{n-1} dt du = \vol(S^{n-1}) \int_0^\pi (\sin t)^{n-1}dt \]
Diese Rekursionsformel führt zu:
\begin{align*}
\vol(S^{2n}) &= \frac {2 (2\pi)^n} {(2n-1)(2n-3)\cdots 3\cdot 1} \\
\vol(S^{2n+1}) &= 2\frac {\pi^{n+1}} {n!}
\end{align*}
Das heißt auch: $\vol(S^n)\to 0$ für $n\to \infty$.
\item Hyperbolische Räume $H^n$. Hier ist $s(u)=\infty$ für alle $u\in S^{n-1}\subset T_pM$. $Y_i(t) = \sinh(t)E_i(t)$. $J(u,t) = \left( \frac {\sinh(t)} t \right)^{n-1}$.
Daraus ergibt sich $\vol(H^n)=\infty$. Betrachten wir also einen Ball von Radius $R$ (das heißt $B_R(p) \da \{ q\in H^n, d(p,q)\le R\}$):
\[
\vol(B_R(p)) = \int_{S^{n-1}} \int_0^R (\sinh t)^{n-1} dt du = \vol(S^{n-1}) \int_0^R (\sinh t)^{n-1} dt
\]
Für sehr große $R$ wächst $\vol(B_R(p))$ wie $e^{(n-1)R}$: Das Volumenwachstum von Bällen vom Radius $R$ in hyperbolischen Räumen ist exponentiell in $R$. Vergleiche das mit dem Volumenwachstum von Bällen von Radius $R$ in $\MdR^n$, welches polynomial ist.

Diese Beobachtungen waren Ausgangspunkt um den Krümmungsbegriff in allgemeinen metrischen Räumen einzuführen.
\end{enumerate}


\appendix

\chapter{Satz um Satz (hüpft der Has)}
\listtheorems{satz,wichtigedefinition}

\renewcommand{\indexname}{Stichwortverzeichnis}
\addtocounter{chapter}{1}
\addcontentsline{toc}{chapter}{\protect\numberline {\thechapter}Stichwortverzeichnis}
\printindex
\end{document}
