\section{Teilbarkeit}

Sei $R$ kommutativer Ring mit Eins.
\begin{DefBem}
Seien $a,b \in R,\; a\neq 0$.
\begin{enum}
\item $a$ \emp{teilt} $b$ (Schreibweise $a \mid b$) $:\lra b \in (a)$
($\lra \exists x \in R: b = ax$)

\item $d \in R$ hei�t \emp{gr��ter gemeinsamer Teiler} von $a$ und
$b$,(Schreibweise ggT($a$,$b$)) wenn gilt:
    \begin{enum}
        \item[(i)] $d \mid a$ und $d \mid b$ bzw. $a \in (d),b\in (d)$
        \item[(ii)] ist $d' \in R$ auch Teiler von $a$ und $b$, so
        gilt $d' \mid d$ bzw. $d \in (d')$
    \end{enum}
\item Ist $d \in R$ ein ggT von $a$ und $b$ und $e \in R^x$, so ist
auch $e \cd d$ ein ggT. Ist $R$ nullteilerfrei und
sind $d,d'$ beide ggT von $a$ und $b$, so gibt es $e \in R^x$ mit
$d' = ed$. \sbew{0.9}{Nach Definition gibt es $x,y \in R$ mit $d' = xd$ und
$d=yd' \Ra d' = xy d' \Ra d'(1-xy) = 0 \underset{\substack{d'\neq
0\\R \mbox{ \scriptsize nullteilerfrei}}}{\Longrightarrow} 1 = xy
\lra x,y \in R^x$}{\item[]}
\end{enum}
\end{DefBem}

\begin{DefBem}
\begin{enum}
\item Ein Integrit�tsbereich $R$ hei�t \emp{euklidisch}, wenn es
eine Abbildung: $\delta: R\setminus\{0\} \ra \mathbb{N}$ mit
folgender Eigenschaft gibt: zu $f,g \in R, g\neq 0$ gibt es $q,r \in
R$ mit $f = qg + r$ mit $r=0$ oder $\delta(r) < \delta(q)$.

\item Sei $R$ euklidisch, $a,b \in R\setminus\{0\}$. Dann gilt:
    \begin{enum}
        \item[(i)] in $R$ gibt es einen ggT von $a$ und $b$.
        \item[(ii)] $d \in (a,b)$ (dh $\exists x,y \in R$ mit $d=xa
        +yb$)
        \item[(iii)] $(d) = (a,b)$
    \end{enum}
\item Jeder euklidische Ring ist ein Hauptidealring.
\end{enum}
\sbsp{1.0}{$\mathbb{Z}$ mit $\delta(a) = |a|, K[X]$ mit $\delta(f)
=$ Grad($f$)}
\newline \bew{}{\item[(b)] \OE sei $\delta(a) \geq
\delta(b)$. Nach Voraussetzung gibt es $q_1,r_1 \in R$ mit $a = q_1
b + r_1$, $\delta(r_1) < \delta(b)$ oder $r_1 = 0$.
\newline Ist $r_1 = 0$, so ist $a \in (b) = (a,b)$ und
ggT($a,b$)$=b$. Sonst gibt es $q_2,r_2 \in R$ mit $b=q_2r_1 +r_2$
und $r_2 = 0$ oder $\delta(r_2) < \delta(r_1)$. usw... \[\Ra
\begin{array}{ccccc} r_i &= &q_{i+2} r_{i+1} &+& r_{i+2} \\
                    \vdots & & \vdots & & \\
                    r_{n-2} &=& q_n r_{n-1} && \end{array}\]
(da $\delta(r_{i+2}) < \delta(r_{i+1})$)
\newline \textbf{Beh.}: $d \defeqr r_{n-1}$ ist ggT von $a$ und $b$.
\newline\textbf{denn:} $d \mid r_{n-2}$ (vorletzte Zeile: $r_{n-3} = q_{n-1}
r_{n-2} + r_{n-1} \Ra d \mid r_{n-3}$)
\newline Induktion: $d \mid r_i$ f�r alle $i \Ra d \mid b \Ra d \mid a$
\newline \textbf{umgekehrt:} Sei $d'$ Teiler von $a$ und $b \Ra
d' \mid r_1 \underset{\mbox{\scriptsize Induktion}}{\Ra} d' \mid r_i\; \forall i
\Ra d' \mid d$. \newline noch zu zeigen:
\begin{enum}
\item[(ii)] $d \in (a,b)$ Nach Konstruktion ist $r_{i+2} \in
(r_i,r_{i+1}) \subset \dots \subset (a,b)\; \forall i$
\item[(iii)] $(d) = (a,b)$ ''$\subseteq$'' ist (ii) $\chk$
\newline ''$\supseteq$'' $a \in (d), b \in (d)$ nach Definition.
\end{enum}
\item[(c)] Sei $I \subseteq R$ Ideal, $I \neq \{0\}$. W�hle $a \in
I$ mit $\delta(a)$ minimal. Dann gilt f�r jedes $b \in I : b=qa+r$
mit $r \in I$ und $\delta(r) < \delta(a)\;\blitza$ also $r=0 \Ra I
=(a)$}
\end{DefBem}

\begin{DefBem}
Sei $R$ kommutativer Ring mit
Eins.
\begin{enum}
\item $x,y \in R$ hei�en \emp{assoziiert}, wenn es $e \in R^x$ mit
$y=xe$ gibt. ''assoziiert'' ist eine �quivalenzrelation.

\item $x \in R\setminus R^x$ hei�t \emp{irreduzibel}, wenn aus $x=y_1
\cd y_2$ mit $y_1,y_2 \in R$ folgt: $y_1 \in R^x$ oder $y_2 \in
R^x$.

\item $x \in R\setminus R^x$ hei�t \emp{prim} (oder
\emp{Primelement}), wenn $(x)$ ein Primideal ist, dh. aus $x \mid y_1
y_2$ folgt $x \mid y_1$ oder $x \mid y_2$.

\item Sind $x,y \in R \setminus R^x$ assoziiert, so ist $x$ genau
dann irreduzibel (bzw. prim), wenn $y$ irreduzibel(prim) ist.

\item Ist $R$ nullteilerfrei, so ist jedes Primelement irreduzibel.
\newline\sbew{0.9}{Sei $(x)$ Primideal und $x=y_1 y_2,\; y_1,y_2 \in R
\Ra$ \OE: $y_1 \in (x)$, dh. $y_1 = xa$ f�r ein $a \in R$ (R
nullteilerfrei, $x \neq 0$) $\Ra x = xay_2 \Ra x(1-ay_2) = 0 \underset{\mbox{\scriptsize $x\neq 0$}}{\Ra}
ay_2 = 1 \Ra
y_2 \in R^x$
}
\end{enum}

\bsp{$2 \cd 3 = 6 = (1+\sqrt{-5})(1-\sqrt{-5})$ \newline $R =
\mathbb{Z}[\sqrt{-5}] = \{a + b\sqrt{-5}:\;a,b \in \mathbb{Z} \}
\subset \mathbb{C}$
\newline $(a+b\sqrt{-5})(c+d\sqrt{-5}) = ac - 5bd +
(ad+bc)\sqrt{-5}$
\newline In $R$ ist $2$ kein Primelement, weder $1+\sqrt{-5}$ noch
$1-\sqrt{-5}$ sind durch $2$ teilbar, \textbf{aber} $2$ ist
irreduzibel!.
\newline \textbf{denn}: Sei $2 = (a+b\sqrt{-5})(c+d\sqrt{-5}) \Ra 4
= |2|^2 = (a+b\sqrt{-5})(a-b\sqrt{-5})(\dots) = (a^2 +
5b^2)(c^2+5d^2) = a^2c^2 + \underset{P \geq 0}{\underbrace{5P}} \Ra P =
0 \Ra b = d = 0 \Ra a^2 = 1,\; c^2 = 4$}
\end{DefBem}

\begin{PropDef}
\label{2.21}
Sei $R$ ein Integrit�tsbereich.
\begin{enum}
\item Folgende Eigenschaften sind �quivalent:
    \begin{enumerate}
        \item[(i)] Jedes $x \in R\setminus\{0\}$ l��t sich eindeutig
        als Produkt von Primelementen schreiben.
        \item[(ii)] Jedes $x \in R\setminus\{0\}$ l��t sich
        ''irgendwie'' als Produkt von Primelementen schreiben.
        \item[(iii)] Jedes $x \in R\setminus\{0\}$ l��t sich eindeutig als
        Produkt von irreduziblen Elementen schreiben.
    \end{enumerate}

\item Sind diese drei Eigenschaften f�r $R$ erf�llt, so hei�t $R$
\emp{faktorieller} Ring. (Oder \emp{ZPE-Ring} (engl.: UFD)). Dabei
ist in (a) ''eindeutig'' gemeint, bis auf Reihenfolge und
Multiplikation mit Einheiten. Pr�ziser: Sei $\mathcal{P}$ ein
Vertretersystem der Primelemnte ($\neq 0$) bez�glich
''assoziiert''.
\newline Dann hei�t (i) $\forall x \in R \setminus\{0\}\; \exists!\;e \in
R^x$ und f�r jedes $p \in \mathcal{P}$ ein $\ds\nu_p(x) \geq
0:x=e\prod_{p \in \mathcal{P}} p^{\nu_p}$. (beachte $\nu_p \neq 0$
nur f�r endlich viele $p$).
\end{enum}
\sbew{1.0}{\begin{description}\item[(i) $\Ra$ (ii)] $\chk$
\item[(ii) $\Ra$ (iii)] Sei $x \neq 0, x = ep_1\cd \dots \cd
p_r,\; p_i \in \mathcal{P},\; e \in R^x$. \newline Sei weiter $x = q_1 \cd \dots
\cd q_s$ mit irreduziblem Element $q_j$. \newline Es ist $x \in
(p_1) \Ra \exists j$ mit $q_j \in (p_1)$. \mbox{\OE}: $j=1$ dh. $q_1 =
\varepsilon_1 p_1$ mit $\varepsilon_1 \in R^x$ (da $q_1$ irreduzibel)
$\Ra \varepsilon_1 q_2 \cd \dots \cd q_s = e p_2 \cd \dots \cd p_r$.
Mit Induktion �ber $r$ folgt die Behauptung.
\item[(iii) $\Ra$ (i)] Noch zu zeigen: Jedes irreduzible Element in
$R$ ist prim.
\newline Sei $p \in R \setminus R^x$ irreduzibel, $x,y \in R$ mit
$xy \in (p)$, also $xy = pa$ f�r ein $a \in R$. Schreibe
$x=q_1,\dots,q_m,\;y=s_1,\dots,s_n,\;a=p_1,\dots,p_l$ mit
irreduziblen Elementen $q_i,s_j,p_k$.
\newline $\Ra xy = q_1\dots q_m s_1 \dots s_n = pa = p\cd p_1 \cd
\dots \cd p_l \overset{\mbox{\scriptsize
Eindeutigkeit}}{\Longrightarrow} p \in
\{q_1,\dots,q_m,s_1,\dots,s_n\}$ (bis auf
Einheiten)\end{description}}
\end{PropDef}

\begin{Bem}
Ist $R$ faktorieller Ring, so gibt es zu
allen $a,b \in R\setminus\{0\}$ einen ggT($a$,$b$).
\newline\sbew{1.0}{Sei $\mathcal{P}$ wie in \ref{2.21} Vertretersystem der
Primelemente. \[a = e_1 \prod_{p \in \mathcal{P}} p^{\nu_p(a)}, \; b
= e_2 \prod_{p \in \mathcal{P}} p^{\nu_p(b)} \Longrightarrow d
\defeqr \prod_{p \in \mathcal{P}} p^{\nu_p(d)}\] mit $\nu_p(d) =
\min(\nu_p(a),\nu_p(b))$ ist ggT von $a$ und $b$.}
\end{Bem}

\begin{Satz}
\label{Satz 9}
Jeder nullteilerfreie Hauptidealring ist faktoriell.
\newline\newline\bew{}{\item[(1)] Jedes $x\in R \setminus\{0\}$ l��t sich
als Produkt von irreduziblen Elementen schreiben.
\item[(2)] Jedes irreduzible $x \in R \setminus\{0\}$ erzeugt ein
maximales Ideal. Mit \ref{2.21} folgt dann die Behauptung.
\item[B(2)] Sei $p \in R\setminus\{0\}$ irreduzibel, $I$ Ideal in
$R$ mit $(p) \subseteq I \subset R$.
\newline Nach Voraussetzung gibt es $a \in R$ mit $I=(a)$, $a\not
\in R^x$, da $I \neq R$.
\newline Da $p \in (p) \subseteq I = (a)$, gibt es $\varepsilon \in
R$ mit $p = a \varepsilon \overset{p\mbox{ \scriptsize
irreduzibel}}{\Longrightarrow} \varepsilon \in R^x \Ra (p) = (a) =
I$
\item[B(1)] $x \in R\setminus\{0\}$ hei�e St�renfried, wenn $x$
nicht als Produkt von irreduziblen Elementen darstellbar ist.
\newline Sei $x$ St�renfried. Dann ist $x \not \in R^x$ und $x$
nicht irreduzibel, also $x = x_1 y_1$ mit $x_1,y_1 \not \in R^x$.
\newline \OE sei $x_1$ St�renfried (sonst ist $x$
doch Produkt von irredziblen Elementen). Also $x_1 = x_2y_2,\; x_2,
y_2 \not \in R^x$.
\newline \OE sei $x_2$ St�renfried. Induktiv erhalten
wir $x,x_1,x_2,\dots$ alles St�renfriede mit $(x) \subset (x_1)
\subset (x_2) \subset \dots$.
\newline Sei nun $I = \bigcup_{i\geq 1} (x_i)$. $I$ ist Ideal
$\chk \Ra$ \newline Es gibt $a \in R$ mit $I = (a) \Ra \exists i$
mit $a \in (x_i) \Ra x_j \in (x_i)$ f�r alle $j>i \blitzb$ }
\end{Satz}