\section{Der de Rham - Komplex}

$A$ (kommutative) $R$-Algebra.

$\Omega_A := \Omega_{A/R}$, $\Omega^i_A := \bigwedge\nolimits^i\Omega_A$ f"ur $i \geq 0$.

\begin{SatzDef}
\begin{enumerate}
\item[a)] F"ur jedes $i \geq 0$ gibt es eine eindeutig bestimmte $R$-lineare Abbildung $d_i : \Omega^i_A \rightarrow \Omega^{i+1}_A$ mit
\begin{enumerate}
\item[(i)] $d_i(f \cdot \omega) = df \wedge \omega + f d_i(\omega)$ f"ur alle $f \in A, \omega \in \Omega^i_A$
\item[(ii)] $d_{i+1} \circ d_i = 0$
\end{enumerate}

\item[b)]
Die Sequenz $\Omega^\bullet_A$:
$$A \overset{d_0}{\rightarrow} \Omega_A \overset{d_1}{\rightarrow} \Omega^2_A \overset{d_2}{\rightarrow} \cdots \overset{d_{n-1}}{\rightarrow} \Omega^n_A  \overset{d_n}{\rightarrow} \cdots$$
hei"st \emp{de Rham - Komplex}\index{de Rahm - Komplex} zu $A$.

\item[c)]
F\"ur jedes $i\geq 0$ hei\ss t $H_{dR}^i(A)\defeqr \textrm{Kern}(d_i)/Bild(d_{i-1})$ ($R$-Modul)
der $i$-te deRahm-Kohomologie-Modul von $A$. Dabei sei $d_{-1}=0$, d.h. $H_{dR}^0(A)=\textrm{Kern}(d)\supset R$.

\end{enumerate}

\begin{Bew}
\textbf{1. Fall} $A = R[X_1, \ldots X_n]$

Dann ist $\Omega^k_A$ freier $A$-Modul mit Basis
\[
\{d X_{i_1} \wedge \cdots \wedge d X_{i_n}:\ 1 \leq i_1 < \cdots < i_k \leq n\}
\]

f"ur $f \in A$ ist $df = d_A f = d_0 f = \sum_{i=1}^n \partial_i f d X_i$

Setze: $d_1(\sum_{i=1}^n f_i d X_i) = \sum_{i=1}^n d f_i \wedge d X_i \in \Omega^2_A$

konkretes Beispiel: $d_1(f_1 dX_1 + f_2 dX_2)$

$= (\frac{\partial f_1}{\partial X_1} dX_1 + \frac{\partial f_1}{\partial X_2} dX_2) \wedge dX_1 + (\frac{\partial f_2}{\partial X_1} dX_1 + \frac{\partial f_2}{\partial X_2} dX_2) \wedge dX_2 = (\frac{\partial f_2}{\partial X_1} - \frac{\partial f_1}{\partial X_2}) dX_1 \wedge dX_2$

Erkl"arung: $dX_1 \wedge dX_1 = 0$, $dX_2 \wedge dX_1 = - dX_1 \wedge dX_2$

allgemein:
\begin{equation*}
\begin{split}
 d_k(\displaystyle\sum_{1 \leq i_1 < \cdots < i_k \leq n} f_{i_1 \ldots i_k}
dX_{i_1} \wedge \cdots \wedge dX_{i_k}) =\\
\sum_{1 \leq i_1 < \cdots < i_k \leq n} d(f_{i_1 \ldots i_k}) \wedge dX_{i_1}
\wedge \cdots \wedge dX_{i_k} 
\end{split}
\qquad (\ast)
\end{equation*}

Diese $d_k$ erf"ullen (i):

Sei $\omega = \sum_{\underline{i}} f_{\underline{i}} \wedge dX_{i_1} \wedge \cdots \wedge dX_{i_k} \in \Omega^k$, $f \in A$.

$\Rightarrow$ $d_k(f \omega) = \sum_{\underline{i}} d(f f_{\underline{i}}) \wedge dX_{i_1} \wedge \cdots \wedge dX_{i_k}$

$= \underbrace{ \sum_{\underline{i}} f df_i \wedge dX_{i_1} \wedge \cdots \wedge dX_{i_k} }_{= f \cdot d_k(\omega)} + \underbrace{ \sum_{\underline{i}} f_i df \wedge dX_{i_1} \wedge \cdots \wedge dX_{i_k} }_{d f \wedge \omega}$

Aus (ii) folgt zwingend: $d_k(dX_{i_1}\wedge \cdots \wedge dX_{i_k}) = 0$ (mit 
Induktion \"uber $k$).

Eindeutigkeit der $d_k$: $d_0=d$ ist vorgegeben.\\
$d_1$: wegen (ii) ist $d_1(dX_i)=0$ f\"ur alle $i=1, \ldots, n$\\
wegen (i) ist $d_1(fdX_i)=df\wedge dX_i+f\underbrace{d_1(dX_i)}_{=0}$

$d_2$: zeige: $d_2(dX_{i_1}\wedge dX_{i_2})=0$\\
folgt aus (ii), da $dX_1\wedge dX_2=d_1(X_1 dX_2)=d_1(-X_2dX_1)$\\
wegen (i) ist $d_2(fdX_1\wedge dX_2)=df\wedge dX_1\wedge dX_2$\\
(die restlichen $d_k$ folgen mit Induktion).

Es bleibt noch 	u zeigen: $d_{k+1}\circ d_k(f\underbrace{dX_{i_1}\wedge \cdots
\wedge
dX_{i_k}}_{\defeql \omega})=0$ f\"ur alle $f\in A$, $1\leq i_1<\dots<i_k\leq n$\\
$d_{k+1}\circ d_k(f\omega)
\stackrel{(i)\ \textrm{f"ur}\ d_k}{=}d_{k+1}(df\wedge \omega +fd_k(\omega))
\stackrel{\textrm{Eind.}}{=}d_{k+1}(df\wedge \omega) 
=$\\
$d_{k+1}\left(\left(\sum_{i=1}^{n}\partial_i f_i dX_i\right)\wedge
\omega\right)
\stackrel{(\ast)}{=}\sum_{i=1}^{n}d(\partial_if_i)\wedge dX_i\wedge \omega=$\\
$\sum_{i=1}^n \sum_{j=1}^n \partial_j (\partial_i f_i)dX_j\wedge dX_{i}\wedge w=0$,
denn es ist $\partial_j(\partial_i f) = \partial_i(\partial_j f)$ f\"ur alle $i,j$
und $dX_i\wedge dX_j=-dX_j\wedge dX_i$ f\"ur alle $i,j$.

\textbf{2. Fall} $A$ beliebige $R$-Algebra.

Schreibe $A$ als Faktoralgebra eines Polynomrings $P$ (in eventuell unendlich vielen Variablen).\\
vornehm: Es gibt einen surjektiven $R$-Algebren-Homomorphismus $\varphi:P\to A$.\\
$\Omega$ ist Funktor, $\bigwedge\nolimits^i$ auch, $\varphi$ induziert also einen Homomorphismus 
$\varphi_i: \Omega_P^{i}\to \Omega_A^{i}$.

\[
\begin{xy}
\xymatrix{
\Omega_P^i\ar[rr]^{d_{i,P}}\ar[d]^{\varphi_i}&&\Omega_P^{i+1}\ar[d]^{\varphi_{i+1}}\\
\Omega_A^i\ar@{-->}[rr]^{d_{i,A}}&&\Omega_A^{i+1}\\
}
\end{xy}
\]

Es gilt: 
\begin{itemize}
\item $\textrm{Kern}(\varphi_i)\subseteq \textrm{Kern}(\varphi_{i+1}\circ d_{i,P})$
\item $\varphi_i$ surjektiv (\"U4A3a f\"ur $i=1$)
\end{itemize}

Dann induziert $d_{i,P}$ eine Abbildung $d_{i,A}: \Omega_A^{i}\to \Omega_A^{i+1}$.\\
Die Eigenschaften (i) und (ii) werden ``vererbt''.

\end{Bew}
\end{SatzDef}

\begin{nnBsp}
%% kein Label!!!
$A=K[X_1, \ldots, X_n]$, $\textrm{char}(K)=0$.\\
\textbf{Beh.}: $H_{dR}^i(A) = 0$ f\"ur alle $i>0$.\\
\textbf{Bew.: } $i=n$: \"U4A2\\
$i>n$: $\Omega_A^i=0$\\
$i=1$: Sei $\omega=\sum_{\nu=1}^nf dX_\nu\in \textrm{Kern}(d_1)$, also:
\[
0=\sum_{\nu=1}^n df_\nu \wedge dX_\nu=\sum_{\nu=1}^n \sum_{\mu=1}^n
\frac{\partial f_\nu}{\partial X_\mu}dX_\mu\wedge dX_\nu
\]
F\"ur alle $\nu\neq \mu$ ist also $\frac{\partial f_\nu}{\partial X_\mu}=\frac{\partial f_\mu}{\partial X_\nu}$
(da $dX_\mu\wedge dX_\nu = -dX_\nu\wedge dX_\mu$).\\
Zu zeigen: $\omega = df$ f\"ur ein $f\in A$, d.h. $f_\nu=\frac{\partial f}{\partial X_\nu}$, $\nu=1, \ldots, n$.\\
Schreibe $f_\nu=\sum_{\underline{i}} a_{\underline{i}}^{(\nu)}X_1^{i_1}\cdots X_n^{i_n}$.
Ansatz: $f=\sum_{\underline{i}} a_{\underline{i}} X_1^{i_1}\cdots X_n^{i_n}$.
$\Rightarrow \frac{\partial f}{\partial X_\nu}
=\sum_{\underline{i}=(i_1, \cdots, i_n), i_\nu\geq 1}a_{\underline{i}} X_1^{i_1}\cdots X_n^{i_n}$.

W\"ahle also $a_{\underline{i}}$ so, dass $i_\nu\cdot a_{\underline{i}} = a_{\underline{i}-\underline{e_\nu}}^{(\nu)}$,
$e_\nu=(0,\ldots,0,\underbrace{1}_{\nu},0,\ldots, 0)$

Es bleibt zu zeigen: $\frac{1}{i_\nu} a_{\underline{i}-\underline{e_\nu}}^{(\nu)}
=\frac{1}{i_\mu} a_{\underline{i}-e_\mu}^{(\mu)}$ f\"ur alle $\nu\neq \mu$.

\"Aquivalent: (*) $i_\mu\cdot a_{\underline{i}-e_{\nu}}^{(\nu)}=i_\nu \cdot a_{\underline{i}-\underline{e_\mu}}^{(\mu)}$

Beweis von (*): $
\sum_{\underline{i}} i_\mu a_{\underline{i}-\underline{e_\nu}}X^{i-e_\mu-e_\nu}
= \sum_{\underline{i}, i_\mu\geq 1} i_\mu a_{i}^{(\nu)} X^{i-e_\mu}=$\\
$\sum_{\underline{i}, i_\nu\geq 1}i_\nu a_{\underline{i}}^{(\mu)}X^{\underline{i}-\underline{e_\nu}}
= \sum_{\underline{i}} i_\nu a_{\underline{i}-e_\nu}^{(\mu)}X^{\underline{i}-\underline{e_\nu}-\underline{e_\mu}}$,
da $\frac{\partial f_\nu}{\partial X_\mu} = \frac{\partial f_\mu}{\partial X_\nu}$.
\end{nnBsp}

\begin{nnBsp}
%% kein Label !!!
$A=K[X, X^{-1}]=K[X,Y]/(XY-1)=\{f=\sum_{\nu=-n_0}^{n_1} a_\nu X^\nu: a_\nu\in K, n_0, n_1\in \mathbb{N}\}$

$\Omega_A=AdX$, $df=(\sum_{\nu\neq 0}\nu a_\nu X^{\nu-1})dX$
$\Rightarrow \Omega_A^2 = 0 \Rightarrow \textrm{Bild}(d)=\{fdX:f\in A, a_{-1}=0\}$,
d.h. $H_{dR}^{1}(A) = K\frac{dX}{X}$.
\end{nnBsp}
