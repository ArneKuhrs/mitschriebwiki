\chapter{Multilineare Algebra}

\section{Moduln}

Sei $R$ ein (kommutativer) Ring (mit Eins) (in der ganzen Vorlesung).

\begin{Def}
\label{1.1}
\begin{enumerate}
\item Eine abelsche Gruppe $(M,+)$ zusammen mit einer Abbildung
\[
\cdot : R \times M \to M
\]
heißt \emp{\RMod}\index{\RMod} (genauer:
$R$-Linksmodul), wenn für alle $a,b\in R$, $x,y\in M$ gilt:
\begin{enumerate}
\item[(i)] $a \cdot (x+y) = a \cdot x + a \cdot y$
\item[(ii)] $(a+b) \cdot x = a \cdot x + b \cdot x$
\item[(iii)] $(a \cdot b) \cdot x = a \cdot (b \cdot x)$
\item[(iv)] $1 \cdot x = x$
\end{enumerate}
\item Eine Abbildung 
\[
\varphi: M \to M'
\]
zwischen $R$-Modulen $M$, $M'$
heißt \emp{\RModHom}\index{\RMod!-Homomorphismus} (kurz
\emp{$R$-linear}\index{R-linear}), wenn für alle $a,b \in R$, $x,y \in M$
gilt:
\[
\varphi (a \cdot x + b \cdot y) = a \cdot \varphi (x) + b \cdot \varphi (y)
\]
\end{enumerate}
\end{Def}

\begin{nnBsp}
\begin{enumerate}
\item[(1)] $R = K$ Körper. Dann ist $R$-Modul = $K$-Vektorraum und $R$-linear =
linear.
\item[(2)] $R$ ist $R$-Modul. Jedes Ideal $I \subseteq R$ ist $R$-Modul.
\item[(3)] Jede abelsche Gruppe ist ein $\ZZ$-Modul, denn
\[
n \cdot x := \ub{x + x + \cdots + x}{n\text{-mal}}
\]
definiert die Abbildung $\cdot: \ZZ \times M \to M$ wie in \ref{1.1}
gefordert.
\end{enumerate}
\end{nnBsp}

\begin{BemDef}
\begin{enumerate}
\item Sind $M,M'$ $R$-Moduln, so ist
\[
\Hom[R]{M}{M'} := \{\varphi: M \to M' : \varphi \text{ ist } R\text{-linear}\}
\]
ein $R$-Modul durch 
\[
(\varphi_1 + \varphi_2)(x) := \varphi_1(x) + \varphi_2(x) 
\]
und
\[
(a \cdot \varphi_1)(x) := a \cdot \varphi_1(x).
\]
\item $M^* = \Hom[R]{M}{R}$ heißt dualer Modul.\index{R-Modul!dualer}
\end{enumerate}
\end{BemDef}

% fuer das grosse kartesische produkt im naechsten absatz
\newcommand{\BIGOP}[1]{\mathop{\mathchoice%
{\raise-0.22em\hbox{\huge $#1$}}%
{\raise-0.05em\hbox{\Large $#1$}}{\hbox{\large $#1$}}{#1}}}
\newcommand{\bigtimes}{\BIGOP{\times}} 

\begin{nnBsp}
Für $R = \ZZ$ ist 
\[
\Hom[R]{\ZZ/2\ZZ}{\ZZ} = \{ 0\}
\]
denn
\[
0 = \varphi(0) = \varphi(1 + 1) = \varphi(1)+\varphi(1) \Rightarrow \varphi(1) =
0.
\]
\end{nnBsp}

\begin{Bem}[Ähnlichkeiten von Moduln mit Vektorräumen]
Die $R$-Moduln bilden eine \emp{abelsche Kategorie}\index{Kategorie!abelsche} \emp{$R$-Mod}\index{Kategorie!R-Mod}.
\begin{enumerate}
\item Eine Untergruppe $N$ eines $R$-Moduls $M$ heißt $R$-Untermodul von
$M$, falls 
\[
R \cdot N \subseteq N.
\]
\item Kern und Bild $R$-linearer Abbildungen sind \RMods.
\item Zu jedem Untermodul $N \subseteq M$ gibt es einen Faktormodul $M/N$.
\item \textsc{Homomorphiesatz:} Für einen surjektiven Homomorphismus
\[
\varphi: M \to N 
\]
gilt:
\[
M/\K{\varphi} \cong N.
\]
\item \textsc{Direktes Produkt:} Sei ${\{M_{i}\}}_{i \in I}$ eine beliebige
Menge von \RMods.
Dann ist ihr direktes Produkt
\[
\prod_i M_i
\]
 gegeben durch die Menge aller Tupel ${(m_i)}_{i
\in I}$ mit $m_i \in M_i$ und der $R$-Aktion 
\[
{r(m_i)}_{i \in I} := {(rm_i)}_{i \in I}.
\]

\textsc{Direkte Summe:} Das gleiche wie beim direkten Produkt, jedoch dürfen in den 
Tupeln nur endlich viele $m_i \neq 0$ sein.
\end{enumerate}
\end{Bem}

\begin{Bew}
\begin{enumerate}
\stepcounter{enumi}
\item $\K{\varphi}$: Sei 
\[
\varphi: M \to N
\]
 lineare Abbildung. $m \in \K{\varphi}$, $r \in R$:
\[
\varphi(rm) = r\varphi(m) = 0
\] 
$\Rightarrow R \cdot \K{\varphi} \subseteq \K{\varphi}$; Untergruppe klar

$\B\varphi$: $n \in \B\varphi$, d.h. $\exists m\in M: n =
\varphi (m) \Rightarrow r \in R:$
\[
rn = r \varphi(m) = \varphi(rm) \in \B\varphi
\]
$\Rightarrow R
\cdot \B\varphi \subseteq \B\varphi$
\item $M$ abelsch $\Rightarrow$ jedes $N$ Normalteiler $\Rightarrow M/N$ ist
abelsche Gruppe.

Wir definieren $R$-Aktion auf $M/N$ durch
\[
r(m + N) = rm + N.
\]
Das ist wohldefiniert, denn
\[
r((m+n)+N)=r(m+n) + N= rm + \ub{rn}{\in N} + N = rm + N
\]
$r((m+N) + (m' + N ) ) = r((m+m')+N) = r(m+m') + N = rm + N + rm' + N =
r(m+N) + r(m'+N)$\\
Die restlichen drei Eigenschaften ergeben sich analog.
  
\item
$$\begin{xy}
\xymatrix{
M \ar[rr]^{\varphi} \ar[rd] &     &  N \\
&  M/\K{\varphi} \ar@{-->}[ur]_{\exists!\tilde{\varphi}}  & }
\end{xy}$$
Wohldefiniertheit von $\tilde{\varphi}$:
$k \in \K{\varphi}\Rightarrow \varphi(m+k) = \varphi(m)$

$\tilde{\varphi}$ surjektiv: $\forall n \in N: n = \varphi(m) =
\tilde{\varphi}(m+ \K{\varphi})$

$\tilde{\varphi}$ injektiv: $m, m' \in M$ mit $\varphi(m) = \varphi(m') = n
\in N \Leftrightarrow 
\varphi(m-m') = 0 \Rightarrow m + \K{\varphi} = 
m' + \K{\varphi}$

$\tilde{\varphi}$ ist $R$-linear: Klar, wegen $\varphi$ $R$-linear.
\end{enumerate}
\end{Bew}

\begin{Bem}
  \begin{enumerate}
    \item Zu jeder Teilmenge $X \subseteq M$ eines $R$-Moduls $M$ gibt es den von
          $X$ erzeugten Untermodul $$\langle X \rangle = \displaystyle 
          \bigcap_{\substack{M' \subseteq M\; \text{ Untermodul} \\ X \subseteq M'}} M' = \left\{
          \sum_{i=1}^n a_i x_i: n \in \NN, a_i \in R, x_i \in X \right\}$$
    \item $B \subset M$ heißt \emp{linear unabhängig}\index{linear unabhängig},
          wenn aus $\displaystyle \sum_{i=1}^n \alpha_i b_i = 0$ mit $n \in
          \NN, b_i \in B, \alpha_i \in R$ folgt $\alpha_i = 0$ für alle
          $i$.
    \item Ein linear unabhängiges Erzeugendensystem heißt
          \emp{Basis}\index{Basis}.
    \item Nicht jeder $R$-Modul besitzt eine Basis.

          Beispiel: $\ZZ/2\ZZ$ als $\ZZ$-Modul: $\{\bar{1}\}$
          ist nicht linear unabhängig, da $\ub{42}{\not= 0 \text{ in } \ZZ} \cdot 1 = 0$
    \item Ein $R$-Modul heißt \emp{frei}\index{R-Modul!freier}, wenn er eine
          Basis besitzt.
    \item Ein freier $R$-Modul $M$ hat die Universelle Abbildungseigenschaft eines Vektorraums. Ist $B$ eine
          Basis von $M$, $f: B \to M'$ eine Abbildung in einen $R$-Modul M', so
          gibt es genau eine $R$-lineare Abbildung $\varphi: M \to M'$ mit
          $\varphi|_B = f$.
    \item Sei $M$ freier Modul. Dann ist $M^*$ wieder frei und hat dieselbe
          Dimension wie $M$.
  \end{enumerate}
\end{Bem}

\begin{Bew}
\begin{enumerate}
\item[(f)] Sei $\{y_i\}_{i \in I}$ Familie von Elementen von $M'$.
Sei $x \in M$. Durch
\[
x=\sum_{i}a_ix_i
\]
ist $\{a_i\}_{i  \in I}$ eindeutig bestimmt. Wir setzen:
\[
\varphi(x):=\sum_i a_iy_i=\sum_ia_i\varphi(x_i).
\]

\textbf{Beh.1:} Falls $\{y_i\}_{i\in I}\;(y_i \neq y_j \text{ für } i\neq
j)$ Basis von $M'$ ist, dann ist $\varphi$ ein Isomorphismus.\\
\textbf{Bew.1:} Wir können den Beweis des Satzes rückwärts anwenden
$\Rightarrow \exists \psi:
M' \rightarrow M \text{mit } \psi(y_i)=x_i \ \forall i \in I \Rightarrow
\varphi \circ \psi = \id_N, \psi \circ \varphi = \id_M$.

\textbf{Beh.2:} Zwei freie Moduln mit gleicher Basis sind isomorph.\\
\textbf{Bew.2:} klar.
\end{enumerate}
\end{Bew}

\begin{PropDef}
  Sei $0 \to M' \overset{\alpha}{\to} M \overset{\beta}{\to} M'' \to 0$ kurze exakte Sequenz von $R$-Moduln (d.h.
  $M' \subseteq M$ Untermodul, $M'' = M/M'$). Dann gilt für jeden $R$-Modul $N$:
  \begin{enumerate}
    \item $0 \to \Hom[R]{N}{M'} \overset{\alpha_*}{\to} \Hom[R]{N}{M} \overset{\beta_*}{\to}
          \Hom[R]{N}{M''}$ ist exakt.
    \item $0 \to \Hom[R]{M'}{N} \overset{\beta^*}{\to} \Hom[R]{M}{N} \overset{\alpha^*}{\to}
          \Hom[R]{M''}{N}$ ist exakt.
    \item Im Allgemeinen sind $\beta_*$ bzw. $\alpha^*$ nicht surjektiv.
    \item Ein Modul $N$ heißt \emp{projektiv}\index{R-Modul!projektiver} (bzw.
          \emp{injektiv}\index{R-Modul!injektiver}), wenn $\beta_*$ (bzw.
          $\alpha^*$) surjektiv ist.
    \item Freie Moduln sind projektiv.
    \item Jeder $R$-Modul M ist Faktormodul eines projektiven $R$-Moduls.
    \item Jeder $R$-Modul M ist Untermodul eines injektiven $R$-Moduls.
  \end{enumerate}
\end{PropDef}

\begin{Bew}
  \begin{enumerate}
    \item $$
	  \begin{xy}
	  \xymatrix{
	         &                    &  N \ar[ld]_\varphi \ar[d]^\psi \ar[dr] & & \\
	0 \ar[r] & M' \ar[r]^{\alpha} & M \ar[r]^{\beta} & M'' \ar[r] & 0
	}
	\end{xy}
	$$

          $\alpha_*$ ist injektiv: Sei $\alpha \circ \varphi = 0 \overset{\alpha
          \text{ \scriptsize inj.}}{\Rightarrow} \varphi = 0$.

          $\B{\alpha_*} \subseteq \K{\beta_*}$:
          \[
          \beta_*(\alpha_*(\varphi)) = \underset{=0}{\underbrace{\beta \circ
          \alpha}} \circ \varphi = 0
          \]

          $\K{\beta_*} \subseteq \B{\alpha_*}$:
          Sei $\beta \circ \psi = 0$. Für jedes $x \in N$ ist $\psi(x) \in
          \K{\beta} = \B\alpha \Rightarrow$ zu $x \in N \;
          \exists y \in M' \text{ mit } \psi(x) = \alpha(y)$; $y$ ist
          eindeutig, da $\alpha$ injektiv.
          Definiere
          \[
          \begin{matrix}
          \varphi:& N &\to& M'\\
          &x &\mapsto& y
          \end{matrix}
          \]
          Zu zeigen: $\varphi$ ist $R$-linear\\
          Seien $x,x' \in N \Rightarrow \varphi(x+x')=z$ mit $\alpha(z) =
          \psi(x+x') = \psi(x) + \psi(x') = \alpha(y) + \alpha(y') =
          \alpha(y +y')$ mit $\varphi(x) = y, \; \varphi(x') = y'
          \overset{\alpha \text{ \scriptsize inj.}}{\Rightarrow} z = y + y'$

          Genauso: $\varphi(a \cdot x) = a \cdot \varphi(x)$
    \item \[
            \begin{xy}
              \xymatrix{
                0 \ar[r] & M' \ar[r] & M \ar[rr]^{\beta} \ar[rd]_{\beta^*(\varphi)} &  &  M'' \ar[dl]^{\varphi} \ar[r] & 0 \\
                & & & N & }
            \end{xy}
          \]
          $\beta^*$ injektiv, denn für $\varphi \in \Hom{M''}{N}$ ist
          $\beta^*(\varphi)=\varphi\circ \beta$\\
	  Sei $\beta^*(\varphi)= 0 \Rightarrow \varphi \circ \beta = 0 \overset{\beta
	  \text{ surj.}}{\Rightarrow}\varphi=0$.

	  $\B{\beta^*} \subseteq \K{\alpha^*}$:
	  \[
	  (\alpha^* \circ 
	  \beta^*)(\varphi)= \alpha^*(\varphi\circ \beta)=\varphi \circ
	  \ub{\beta \circ \alpha}{=0}=0
	  \]

	  $\K{\alpha^*}\subseteq\B{\beta^*}$: Sei $\psi \in 
	  \K{\alpha^*}$, d.h. $\psi \in \Hom{M}{N}$ mit $\psi
	  \circ\alpha=0$.
	  Weil $\psi$ auf $\B{\alpha}$ verschwindet, kommutiert
	  \[
            \begin{xy}
              \xymatrix{
                                 & M'' &\\
                M \ar[rd]_{\psi} \ar[ur]^{\beta} \ar[rr] &     &  M/\B{\alpha}
                \ar[dl]^\sigma \ar[ul]_{\cong}\\
                &  N  & }
            \end{xy}
          \]
		  $\Rightarrow \beta^*(\sigma)= \psi \Longrightarrow$ Beh.
	\item Im Allgemeinen sind $\beta_*$ und $\alpha^*$ nicht surjektiv\\
		z.B.: \begin{enumerate}
		\item[1.]
		\[
		0\rightarrow \ZZ \stackrel{\cdot2}{\stackrel{\alpha}\rightarrow} 
		\ZZ \stackrel\beta\rightarrow \ZZ / 2\ZZ\rightarrow 0
		\]
		mit $N:= \ZZ / 2\ZZ$.
		Es gilt: $\Hom{N}{\ZZ}=\{0\}$, aber
		$\Hom{N}{\ZZ/2\ZZ}=\{0, \id\}  \Longrightarrow  N$ nicht projektiv!
		\item[2.]
		\[
		0\rightarrow \ZZ \stackrel{\cdot4}{\stackrel{\alpha}\rightarrow} \ZZ 
		\stackrel\beta\rightarrow \ZZ / 4\ZZ\rightarrow 0
		\]
		mit $N:= 2\cdot \ZZ / 4\ZZ$.
		$\Hom{\ZZ}{N}= \{0, \psi\}$, wobei $\psi(1)=2$.\\
		Dann: $\alpha^*(\psi)=\psi\circ \alpha = 0$ $\Longrightarrow N$ nicht injektiv!
		\end{enumerate}
      \stepcounter{enumi}
      \item Sei $N$ frei mit Basis $\{e_i,i \in I\}$.
            Sei 
            \[
            \beta: M \to M''
            \]
            surjektive $R$-lineare Abbildung und
            \[
            \varphi: N \to M''
            \]
            $R$-linear. Für jedes $i \in I$ sei $x_i \in M$
            mit $\beta(x_i) = \varphi(e_i)$ (so ein $x_i$ gibt es, da $\beta$
            surjektiv). Dann gibt es genau eine $R$-lineare Abbildung
            \[
            \psi: N \to M
            \]
            mit $\psi(e_i) = x_i$. Damit gilt $\beta(\psi(e_i)) =
            \beta(x_i) = \varphi(e_i)$ für alle $i \in I \Rightarrow \beta \circ
            \psi = \varphi$
      \item \label{1.5fBew}
            Sei $M$ ein $R$-Modul. Sei $X$ ein Erzeugendensystem von $M$ als
            $R$-Modul (notfalls $X = M$). Sei $F$ der freie $R$-Modul mit Basis
            $X$,
            \[
            \varphi: F \to M
            \]
            die $R$-lineare Abbildung, die durch $x
            \mapsto x$ für alle $x \in X$ bestimmt ist. $\varphi$ ist surjektiv,
            da $X \subseteq \B\varphi$ und $\langle X \rangle = M$.
            Nach Homomorphiesatz ist $M \cong F/\K{\varphi}$.
  \end{enumerate}
\end{Bew}

\begin{Prop}
\label{1.6}
  Ein $R$-Modul $N$ ist genau dann projektiv, wenn es einen $R$-Modul $N'$ gibt,
  so dass 
  \[
  F \defeqr N \oplus N'
  \]
  freier Modul ist.
\end{Prop}

\begin{Bew}
  \glqq$\Rightarrow$\grqq:
  \[\xymatrix{
  &&N\ar[dl]\ar[d]^{\tilde{\varphi}}\ar[dr]^{\id} &&\\
  0\ar[r]&N'\ar[r]&F\ar[r]_\beta&N\ar[r]&0 
  }\]
  Sei $F$ freier $R$-Modul und $\beta: F \to N$ surjektiv (wie in Beweis von
  1.5\ref{1.5fBew}). Dann gibt es $\tilde{\varphi}: N \to F$ mit $\beta \circ
  \tilde{\varphi} = \id_N$ (weil $N$ projektiv ist).\\
  \textbf{Behauptung:}
  \begin{enumerate}
    \item[1.)] $F = \K\beta \oplus \B{\tilde{\varphi}} \cong
               N' \oplus N$
    \item[2.)] $\tilde{\varphi}$ injektiv
  \end{enumerate}
  \textbf{Beweis:}
  \begin{enumerate}
    \item[1.)] $\K\beta \cap \B{\tilde{\varphi}} = \{0\}$, denn:
               $\beta(\tilde{\varphi}(x)) = 0 \Rightarrow x = 0 \Rightarrow
               \tilde{\varphi}(x) = 0$.
  
               Sei $x \in F,\; y \defeqr
               \tilde{\varphi}(\beta(x)) \in \B\varphi$.
               Für $z = x - y$ ist
               \[
               \beta(z) = \beta(x) -
               \ub{\beta(\tilde{\varphi}}{\id}(\beta(x)))= 0
               \]
               und somit
               \[ 
               x = \ub{z}{\in
               \K\beta} + \ub{y}{\in \B{\tilde{\varphi}}}
               \]
    \item[2.)] $\tilde{\varphi}(x) = 0 \Rightarrow \ub{\beta(\tilde{\varphi}(x))}{= x} = 0$
  \end{enumerate}
  \glqq$\Leftarrow$\grqq:
  Sei $F = N \oplus N'$ frei, $\beta: M \to M''$ surjektiv und $\varphi: N \to
  M''$ R-linear. Gesucht ist 
  \[
  \psi: N \to M
  \]
  mit $\beta \circ \psi = \varphi$.
  Definiere 
  \[
  \begin{matrix}
  \tilde{\varphi}:& F& \to& M''\\
  &x + y &\mapsto&\varphi(x)
  \end{matrix}
  \]
  wobei jedes $z \in F$ eindeutig als $z = x + y$ mit $x \in N,\; y
  \in N'$ geschrieben werden kann.
  $F$ ist frei also projektiv $\Rightarrow \exists \tilde{\psi}: F \to M$ mit
  $\beta \circ \tilde{\psi} = \tilde{\varphi}$. Sei $\psi \defeqr
  \tilde{\psi}|_N$. Dann ist $\beta \circ \psi = \beta \circ \tilde{\psi}|_N =
  \tilde{\varphi}|_N = \varphi$.
\end{Bew}
