\documentclass[a4paper,DIV15,BCOR12mm]{article}
\newcounter{chapter}
\setcounter{chapter}{6}
\usepackage{ztheo}
%\usepackage{tikz}

\title{Primzahltests }

\begin{document}
\maketitle


Ein Primzahltest ist ein Algorithmus $Prim(m)$, der zu $m \in \MdN_+$ entscheidet, ob $m \in \MdP \vee m \not\in \MdP$.

Einteilung der Tests ($\neg$disjunkt):
\begin{itemize}
 \item[a)] 
  \begin{itemize}
   \item[+] Allgemeiner Test ($\forall m \in \MdN$)
   \item[-] Spezieller Test (nur gewisse $m \in \MdN$)
  \end{itemize}
 \item[b)] 
  \begin{itemize}
   \item[+] Voll bewiesener Test
   \item[-] Test abhängig von einer Vermutung (zB Riemann-Vermutung)
  \end{itemize}
 \item[c)] 
  \begin{itemize}
   \item[+] Sicherer Test
   \item[-] Propabilistischer Test (Monte-Carlo-Methode)
  \end{itemize}
 \item[d)] 
  \begin{itemize}
   \item[+] Praktikabler Test (geht für "`große"' $m$)
   \item[-] Unpraktischer Test
  \end{itemize}
\end{itemize}

\begin{beispiel}
 \begin{itemize}
  \item[a)] Pepins Test: nur für $F_n = 2^{2^n} + 1$
  \item[d)] Naiver Test: Probiere $a \mid m, \forall a \in \MdN, 1 < a \le \sqrt{m}$
  \item[d)] Wilsons Test: $m \in \MdP \Leftrightarrow (m-1)! \equiv -1 \mod m$, es sind mindestens $m$ "`Aktionen"' nötig
 \end{itemize}
\end{beispiel}

\begin{beweis}[Wilsons Test]
 \begin{itemize}
  \item[\underline{"`$\Rightarrow$"':}] $m = p \in \MdP$. In $\MdF_p:$\\
   $(m-1)! = \prod_{\alpha \in \MdF_p^\times}\alpha = \overline 1 \cdot (\overline{-1})$. Paare $\alpha\alpha^{-1}$ heben sich weg. Wenn $\alpha \not= \alpha^{-1}$ verbleibt $\alpha^2 = 1$, da $\alpha = \pm 1 \Rightarrow (m-1)! \equiv -1 \mod m$
  \item[\underline{"`$\Leftarrow$"':}] $m \not\in \MdP \Rightarrow \ggt ((m-1)!, m) = d > 1 \Rightarrow (m-1)! \not\equiv 1 \mod m$ (sonst $d \mid -1$)
 \end{itemize}
\end{beweis}

Prinzip moderner PZTests:\\
Meist ohne Einschränkung $m > 2, 2 \nmid m$. (Rechnung für große $m$ aufwändig, daher gewöhnlich erst $p \mid m$ probiert für die $p \in \MdP$, etwa $p \le 100000 \vee p \le 1000000$.). Man konstruiert Gruppe $G_m$ derart, dass die Struktur von $G_m$ für $m \in \MdP \wedge m \not\in \MdP$ verschieden ausfällt. Die Strukturverschiedenheit soll mit möglichst wenig und schnellen Rechnungen festgestellt werden.\\
EZT: Meist $G_m  = (\MdZ / m\MdZ)^\times$\\
Höhere ZT: Etwa $G_m = (\sigma_k / \sigma_k \cdot m)^\times$, webei $\sigma_k$ ein Ring "`ganzer algebraischer Zahlen "` im algebraischen Zahlenkörper $K$ ist.
\begin{beispiel}
 $K = \MdQ + \MdQ i, \sigma_k = \MdZ + \MdZ i$ (Ring der ganzen Gaußschen Zahlen)\\
 Algebraische Geometrie: $G_r$ konstruiert aus "`elliptischer Kruve"', die über $\MdZ$ definiert ist. Vorzug: Es gibt $\infty$ viele elliptische Kurven und Zahlenkörper. Man kann versuchen, möglichst "`geeignete"' zu finden. Hier $G_m = (\MdZ / m \MdZ)^\times$.
\end{beispiel}

\begin{itemize}
 \item[(A)] Ein $\neg$ganz geklückter Versuch\\
  Strukturaussage für $G_p (p \in \MdP)$:\\
  Satz von Euler-Fermat: $\overline a^{p-1} = 1$.
  \begin{definition}
   Sei ohne Einschränkung $m > 2, 2 \nmid m$. $a \in \MdZ$ heiße \underline{Carmichael-Zeuge} (für die Zerlegbarkeit von $m$), wenn gilt:
    \begin{itemize}
     \item[(i)] $\ggt (a,m) = 1$
     \item[(ii)] $a^{m-1} \not\equiv 1 \mod m$
    \end{itemize}
  \end{definition}
  Klar: Wenn Zeuge gefunden: $m \not\in \MdP$.\\
  Leider: $\exists m \in \MdN$ mit $m \not\in \MdP$, aber kein Zeuge vorhanden!
  \begin{definition}
   Solche $m \not\in \MdP$ (also die mit $\forall a \in \MdZ, 1 < a < m, \ggt (a,m) = 1$ ist $a^{m-1} \equiv 1 \mod m$) heißen \underline{Carmichael Zahlen}.
  \end{definition}
  \begin{satz}[Carcmichael, $\sim 1920$]
   Sei $m \in \MdN_+, m > 2, \MdP_m := \{p \in \MdP \big| p \mid m\}$. Dann: $m$ ist Carmichael Zahl $\Leftrightarrow$ Es gelten:
    \begin{itemize}
     \item[(i)] $2 \nmid m$
     \item[(ii)] $m$ ist qf (???) ($\forall p \in \MdP: v_p(m) \le 1$)
     \item[(iii)] $\forall p \in \MdP_m: p-1 \big| m-1$
     \item[(iv)] $m$ hat mindestens 3 verschiedene Primteiler ($\#\MdP_m \ge 3$)
    \end{itemize}
  \end{satz}
  \begin{beispiel}
   Kleinste Carmichael-Zahl: $m = 561 = 3 \cdot 11 \cdot 17$ - $2,10,16 \mid 560$
  \end{beispiel}
  \begin{beweis}
   \begin{itemize}
    \item[\underline{"`$\Leftarrow$"':}] 
     $\left.
      \begin{matrix}
       \text{Zeige } (i)-(iv)\\
       \ggt (a,m) = 1
       \end{matrix}\right
       \} \Rightarrow a^{m-1} \equiv 1 \mod m$.\\
     $\forall p \in \MdP_m:$ in $\MdF_p^\times: \ord \overline a \mid p-1 \stackrel{(iii)}{\mid} m-1 \Rightarrow \overline a^{m-1} = 1$ in $\MdF_p \Leftrightarrow a^{m-1} \equiv 1 \mod p \Leftrightarrow p \mid a^{m-1} -1 \stackrel{(ii)qf}{\Rightarrow} m = \prod_{p \in \MdP_m}p \mid a^{m-1}-1 \Rightarrow a^{m-1} \equiv 1 \mod m$
    \item[\underline{"`$\Rightarrow$"':}] ($-1$) kein Zeuge $\Rightarrow (-1)^{m-1} \equiv 1 \mod m$. Falls $2 \mid m \Rightarrow -1 \equiv 1 \mod m \Rightarrow m = 1,2$ (Widerspruch!). Also $2 \nmid m \leadsto (i)$.\\
    Zu (ii), (iii):\\
    Für $p \in \MdP_m$ ist $t:= v_p(m) \ge 1. \exists PW a \mod p$ mit $\ggt(a, m) = 1$ (Sei $w$ PW $\mod p$, lose das System $a \equiv w \mod p (ChRS), a \equiv 1 \mod q (q \in \MdP, q \not= p). \Rightarrow q \nmid a, p \nmid a \Rightarrow \ggt(a, m) = 1$)\\
    In $(\MdZ / p^t\MdZ)^\times$ ist $\overline a^{m-1} = 1$ (wegen $a^{m-1} \equiv 1 \mod m \Rightarrow a^{m-1} \equiv 1 \mod p^t) \Rightarrow \ord \overline a = \phi(p^t) = p^{t-1}(p-1) \mid m-1 \Rightarrow p-1 \mid m-1 \leadsto (iii)$\\
    Wäre $t > 1 \Rightarrow p \mid m-1$ (Wiederspruch zu $p \mid m$).\\
    Also $v_p(m) = 1 \leadsto$ (ii)\\
    Noch zu widerlegen: $\MdP_m = \{p, q\}, p \not= q$, etwa $2 < p < q (\star)$\\
    $m = pq$ laut (ii), $q-1 \stackrel{(iii)}{\mid} m-1 = pq - 1 = p(q-1) + p-1 \Rightarrow q-1 \mid q-1 \Rightarrow q \le p$ (Widerspruch $(\star)$)
   \end{itemize}
  \end{beweis}
 \item[(B)] Ein geglückter Versucht\\
  $m \in \MdN, m > 2, 2 \nmid m$. Schreibe $m-1 = 2^t \cdot u$ mit $t = v_2(m-1)$ also $2 \nmid u, t > 0$.
  \begin{definition}
   $a \in \MdN$ heiße \underline{Miller-Zeuge} (für die Zerlegbarkeit von $m$), wenn gilt:
   \begin{itemize}
    \item[(i)] $\ggt (a,m) = 1$
    \item[(ii)] $a^u \not\equiv 1 \mod m$
    \item[(iii)] $\forall s \in \{0,...,t-1\}: a^{u\dot 2^s} \not\equiv -1 \mod m$
   \end{itemize}
  \end{definition}
  \begin{satz}{Miller-Rabin-PZTest}
   Sei $m \in \MdN, m > 2, 2 \nmid m$. Dann: $m \not\in \MdP \Leftrightarrow \exists$ Miller-Zeuge $a$. $(0 < a < m)$
  \end{satz}
  Zusatz (Rabin): Es gibt dann höchstens $\frac{3}{4}\phi(m) \le \frac{3}{4}(m-1)$ $\neg$Zeugen\\
  $\leadsto$ Liefert voll bewiesenen Test:\\
  Test, ob $\frac{1}{4}(m-1)+1$ $a$s Zeugen sind.\\
  Sobald Zeugen gefunden $\Rightarrow m \not\in \MdP$.\\
  Kein Zeuge gefunden $\Rightarrow m \in \MdP$.\\
  Aber immer noch unpraktisch (ca $\frac{1}{4}m$ Aktionen). Es gibt einen sehr praktischen propabilistischen Test:\\
  Teste, ob $k$ zufällig ausgewählte Restklassen $\overline a$ $(1 < a < m)$ Zeuge sind (falls $\ggt (a,m) = d > 1$, so $m \not\in \MdP$, sonst $\ggt (a,m) = 1$). Falls Zeuge gefunden $\Rightarrow m \not\in \MdP$. Falls kein Zeuge gefunden: Die WK (???), dass man sich mit der Annahme "`$m$ ist prim"' irrt, ist $< \frac{1}{4^k}$.\\
  Für große $m$ scheint die WK sogar \underline{viel} kleiner als $\frac{1}{4^k}$. [experiment. Faktoren]
  
  $\begin{array}{cl}
   m < & \text{Zeuge, falls } m \not\in \MdP \\
   2047 & 2 \\
   1373653 & 2 \vee 3 \\
   3215031753 & 2,3 \vee 5
  \end{array}$
  \begin{beweis}
   \item[\underline{"`$\Leftarrow$"':}] $m = p \in \MdP, \overline a \in \MdF_p^\times$\\
    $ord \overline a \mid \phi(p) = p-1 = 2^t \dot u$\\
    $ord \overline a = 2^s \cdot v, 2 \nmid v, s \le t, v \mid u$
    \begin{itemize}
     \item[1. Fall:] $s = 0 \Rightarrow \overline a^v = 1 \Rightarrow \overline a^u = 1 \Rightarrow a^u \equiv 1 \mod p$, kein Zeuge
     \item[2. Fall:] $s > 0 \Rightarrow \overline a^{2^s \dot v} = 1, \overline a^{2^{s-1}\dot v} \equiv -1 \mod m$, $s \in \{0,...,t-1\} \Rightarrow$ kein Zeuge
    \end{itemize}
  \end{beweis}
\end{itemize}

Weiter bei der letzten Vorlesung:

$m-1 = 2^tu, 2 \nmid u$\\
\underline{Millerzeuge $a$}: $\ggt (a,m) = 1, a^u \not\equiv 1 \mod m$\\
$\forall s = 0, ..., t-1: a^{u2^s} \not\equiv 1 \mod m$

\underline{Rest:}\\
$m \not\in \MdP \Rightarrow \exists$ Millerzeuge
\begin{itemize}
    \item[Fall I:] $\#\MdP_m \ge 2, \MdP_m = \{p_1,...,p_l\}$\\
        $a \equiv -1 \mod p_1$\\
        $a \equiv 1 \mod p_j (j > 1)$\\
        (mit Chinesischem Restsatz lösen)\\
        $a^u \equiv (-1)^u \equiv -1 \mod p_1$, also ist $a^u \equiv 1 \mod m$ falsch (sonst $-1 \equiv 1 \mod p_2 \Rightarrow p_1 = 2$ [Widerspruch!]), also $a^u \not\equiv 1 \mod m$\\
        $a^{u2^s} \equiv 1^{u2^s} \equiv 1 \mod p_j (j > 1) \Rightarrow a^{u2^s} \equiv -1 \mod m$ ist falsch, also $a^{u2^s} \not\equiv 1 \mod m$\\
        Gesehen: $a$ ist Millerzeuge
    \item[Fall II:] $m = p^t, p \in \MdP, t > 1:$ ist $a$ Primitivwurzel $\mod m = p^t$, so ist $a$ Millerzeuge.\\
        $ord(\overline a) = \phi(p^t) = (p-1)p^{t-1}$
        \begin{itemize}
            \item{ $\Rightarrow$} $\overline a^u \not= 1$, weil sonst $ord(\overline a) \mid u \Rightarrow p \mid u \mid m-1$ (Widerspruch zu $p \mid m$)
            \item{ $\Rightarrow$} $\overline a^{u2^s} = -1 \Rightarrow \overline a^{us^{s+1}} = 1 \Rightarrow ord(\overline a) = (p-1)p^{t-1} \mid u2^{s+1} \Rightarrow p \mid u \mid m-1$ (Widerspruch!) $\Rightarrow a^{u2^s} \equiv -1 \mod m$
        \end{itemize}
\end{itemize}

\underline{Stand der Technik:}
\begin{itemize}
    \item[1.)] Primzahlen $< 10^{130}$ mit guter Sicherheit "`leicht"' auffindbar, z.B. mit Miller Rabin
    \item[2.)] Zahlen der Größe $> 10^{130}$, erstrecht $m = pq, p,q \ge 10^{130}$ können nicht faktorisiert werden.
\end{itemize}

Praktischer Test von Rumely, fast in Polynomial-Zeit, vorhanden
(Zeit $\approx \log(m)^{c\log \log \log m}$). Falls die
verallgemeinterte Riemann-Vermutung gilt, so ist dieser Test sogar
in Polynomial-Zeit.

Kayal, Saxena, Aal 2002: Voll bewiesener Primzahltest in
Polynomial-Zeit. Fraglich ob dies ein praktischer Test ist.

Faktorisierung großer Nichtprimzahlen schein ein viel härteres
Problem zu sein.

\underline{Idee von Fermat:}\\
$\MdN_+ \ni m = x^2 - y^2, x,y \in \MdN, m = (x-y)(x+y), x \ge y$
ist Faktorisierung, wenn $x-y \not= 1.m, x-y = 1$ und $x + y \not=
m.1, x+y = m \Rightarrow x = \frac{m+1}{2}, y = \frac{m-1}{2}$ also
echte Teiler, wenn $x,y \not= \frac{m \pm 1}{2}$

Viele moderne Tests arbeiten so: Suche $x,y \in \MdN$ mit $x^2
\equiv y^2 \mod m, x \not\equiv \pm y \mod m$

Gute Chance, dass $\ggt (m, x-y)$ oder $\ggt (m, x+y)$ echter Teiler
von $m$ ist. Sehr viel Test, um die Suche nach solchen $x,y$ zu
beschleunigen: Siehe z.B. Förster, Algorithmic number theory

\section{Anwendung der EZT in der Kryptographie}
Rivests öffentliches Chiffrier System. $m$ große Zahl.\\
Nachricht ist \underline{hier} $N \in \text{Versys}_m^\times = \{a
\in \MdN \big| 0 < a < m, \ge (a,m) = 1\}$ (Falls $m =
p_1^{n_1}\cdot ... \cdot p_l^{n_l}, p1 < ... < p_l \in \MdP, n_j \in
\MdN_+$, so sind alle $N \in \MdN$ mit $1 \le N < p_1$ im
$\text{Versys}_m$. $N$ kodiert Textabschnitt mit $k$ Zeichen, z.B.
Leerstelle = $000$, Jedes Zeichen erhält Ziffern $< 1000$.
\begin{beispiel}
    $\text{ }$\\
    $N =$ \begin{tabular}{ccccccccccr}
        K & O & M & M & & N & I & C & H & T & \\
        011 & 015 & 013 & 013 & 000 & 014 & 009 & 003 & 008 & 020 & $< 10^{3k}$
    \end{tabular}
\end{beispiel}

\begin{definition}
    \begin{itemize}
        \item[(i)] Eine Chiffre ist (für uns) eine bijektive Abbildung $P: \text{Versys}_m^\times \to \text{Versys}_m^\times, N' = P(N)$ ist die "`chriffrierte"' Nachricht.
        \item[(ii)] ein "`öffentliches Chiffresystem"' ist eine Liste ("`öffentliches Adressbuch"'):\\
            $(T, P_T), T \in \tau =$ Menge von Teilnehmern. $P_T$ Chiffre, derart, dass $T \not= T' \Rightarrow P_T \not= P_{T'}$
            \begin{itemize}
                \item[(a)] Jeder Teilnehmer $T \in \tau$ erhaält das Adressbuch $(T, P_T)_{T \in \tau}$
                \item[(b)] $T$ und nur $T$ erhält $P_T^{-1}$ (Umkehrabbildung von $P_T$)\\
                    Praktisch: $T$ muss $P_T^{-1}$ besonders gut sichern, gegen Diebstahl, Ausspähen, Hacker, usw.
            \end{itemize}   \end{itemize}
\end{definition}

\underline{Technische Anforderungen:}\\
\begin{itemize}
    \item[1.)] $P_T(N), P_T^{-1}(N)$ müssen in vernüftiger Realzeit berechenbar sein
    \item[2.)] Nicht einmal ein Supercomputer kann $P_T^{-1}$ aus $P_T$ ermitteln ($P_T$ \emph{Trapdoor}-Funktion)
    \item[3.)] Nur $T$ hat $P_T^{-1}$. Der Systemadministrator hat am Anfang die $P_T$'s und die $P_T^{-1}$'s. Nach Absenden von $P_T^{-1}$ an $T$ vernichtet er $P_T^{-1}$
\end{itemize}

\underline{Anwendungen:}\\
\begin{itemize}
    \item[I)] Geheime Nachricht über öffentlich  zugängliche Kanäle (etwa Internet) übermitteln $T$ von $A$ zu $B$, $A,B \in \tau$ ohne das Unbefugte $N$ gewinnen können.\\
        \underline{Methode:} $A$ berechnet $P(N) = N'$ und sendet $N'$ an $B$. \underline{Nur} $B$ kann aus $N'$ wieder $N = P_B^{-1}(N')$ ermitteln.\\
        \underline{Beispiel:}\\
            \begin{itemize}
                \item $A$ Spion des Geheimdienstes, $B =$ = Geheimdienstzentrale, $C,D$ die gegnerischen Geheimdienste
                \item $A$ ist Bank, $B$ ist Kunde, $N =$ Kontostand
            \end{itemize}
    \item[II)] Geheimnachricht mit elektronischer Unterschrift\\
        \underline{Methode:} $A$ sendet an $B$: "`$N = P_BP_A^{-1}(N), \text { Gruß }A$"'. Nur $A$ kann $N'$ herstellen, nur $B$ kann daraus $N = P_AP_B^{-1}(N')$ gewinnen.\\
        \underline{Beispiel:}\\
            $A =$ Kunde, $B =$ Bank, $N =$ "`Überweisen Sie 200'000.- von meinem Konto an $C$"'
    \item[III)] Sichere Speicherung von Nachrichten\\
        \underline{Methode:} Speichere $N' = P_{A_t}^{-1}(N)...P_{A_1}^{-1}(N)$. Benötigt werden $A_1,..., A_t \in \tau (t = 1)$. Nur mit Willen von allen Mitwirkenden $A_1,..., A_t$ kann $N$ aus $N'$ wieder rekonstruiert werden.
\end{itemize}

EZT kann z.B. zum Erfüllen der technischen Vorraussetzungen verwendet werden.\\
Rivests Vorschlag $\subseteq$ RSA-Code (Rinest, Shamn, Adleman 1978)\\
Adressbuch: Liste$(T, m_T, s_T), m_T, s_T \in \MdN, m_T = p_1^{n_1}...p_l^{n_l}, p_i$ zu Anfang dem Administrator bekannt, öffentlich nur $m_T$'s, $s_T$'s ziemlich groß.\\
Chiffre $P_T(N) := (N^{s_T} \mod m_i)$. Dann theoretisch
$P_T^{-1}(N') = N'^{t_T}$, wobei $t_Ts_T \equiv 1 \mod \phi(N)$
(Euler Funktion). Hiermit erhält $T$ auch noch $t_T$. $t_T$ ist nur
berechenbar, wenn $\phi(m) = m\prod_{p \mid m}(1-\frac{1}{p})$
bekannt, dass geht nur (nach heutigem Wissen), wenn Primzerlegung,
also die $p_i$ bekannt sind.
\end{document}
