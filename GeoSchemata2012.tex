%		UEBER DIESES DESIGN
%	
%	Dieses Vorlesungsskript soll dem Layout eines Buches entsprechen und alles beinhalten, 
%	was man fuer eine Mathevorlesung brauchen koennte. Ich habe mich fuer eine KOMA Script 
%	Klasse entschieden, einfach weil sie Optioen bieten, die man bei den Standardklassen 
%	nur ueber Umwege erreichen konnte. Der Code ist teilweise von mir selbst, teilweise aus 
%	allen moeglichen Quellen zusammengekratzt, ich habe mich bemueht ihn so sehr ich kann 
%	zu kommentieren.
%	Ich bin kein Experte fuer Typographie, alles wurde zwar nach meinem besten Gewissen getan, 
%	aber wenn jemand anderer Meinung ist, dann waere ich froh, wenn er meine Fehler korrigieren
%	koennte.
%

\documentclass[paper = A4, fontsize=12pt, numbers=noendperiod, chapterprefix=true]{scrbook}
% "numbers=noendperiod" macht dass beim Kapteln und aehnlichem kein Punkt am Ende steht
% normalerweise ist der Vorgang automatisiert, aber bei mir versagt es. ich mache die
% Punkte spaeter dann per hand wenn es sein muss.

% "chapterprefix=true" macht dass Kapitel, Abschnitte und aehnliches immer eine Ueber-
% schrift wie "Kapitel 1" hat. Gilt auch fuer den Anhang.

% eigener Rand (Normalerweise sollte der innere Rand hie Haelfte des aeusseren Randes betragen, aber
% ich musst ein wenig dazugeben, weil ein Teil beim Binden wegfaellt)
\usepackage[twoside]{geometry}
\geometry{top=2cm, inner=2cm,outer=2.5cm}

% Nummerierung der Paragraphen anpassen (sonst kommt etwas wie "Definition 2.9.1" heraus)
\renewcommand{\thesection}{\arabic{section}}

% Deutsche Sprache
\usepackage[ngerman]{babel}
\usepackage[ngerman]{translator}

% Verschiebt \sections auf die naechste seite falls sie zu tief sind. Muss vor
% hyperref kommen.
\usepackage[nobottomtitles]{titlesec}

% Schicke Schrift
\usepackage[utf8]{inputenc}
\usepackage[T1]{fontenc}
\usepackage{lmodern}

% Einheitliche Schriftart (KOMA Script verwendet fuer einige Ueberschriften eine serifenlose Schrift, mischt also
% Schriftarten. Ich habe mir die Argumente dafuer durchgelsen und war nicht ueberzeugt. Wenn jemand, der mehr als
% ich von der Materie versteht, anderer Meinung ist kann er diese Zeilen hier einfach auskommentieren)
\setkomafont{chapter}{\Huge\bfseries\rmfamily}
\setkomafont{chapterentry}{\bfseries\rmfamily}
\setkomafont{disposition}{\bfseries\rmfamily}
\setkomafont{descriptionlabel}{\bfseries\rmfamily}
%\setkomafont{section}{\Large\bfseries\rmfamily} % <- Scheint mit der Umdefinierung von disposition unnoetig zu sein

%Stellt das Paragraphenzeichen vor die Section-Nummer (das ist ein KOMA Script exklusiver Befehl)
% \ifstr{#1}{section} ueberprueft, ob der erste parameter (also die erste Ebene der Ueberschrift)
% section heisst, wenn ja, dann wird sie geaendert, ansonsten behalten wir den Standard
\renewcommand*{\othersectionlevelsformat}[3]{\ifstr{#1}{section}{\S\ #3\autodot}{#3\autodot}\enskip}

%Aendert die Kapitelbeschriftung in der Kopfzeile der linken Seiten
\renewcommand*{\chaptermarkformat}{\chapappifchapterprefix{\ }\thechapter\autodot:\enskip}

%Stellt ein Pragraphenzeichen vor den Abschnitt in der Kopfzeile der linken Seiten
\renewcommand*{\sectionmarkformat}{\S \thesection\autodot\enskip}

\setlength\parskip{\smallskipamount}
\setlength\parindent{0pt}
\tolerance=900

% Index erzeugen
\usepackage{index}
\newindex{default}{idx}{ind}{Stichwortverzeichnis}

%Symbolverzeichnis
\usepackage[toc]{glossaries}
\glossarystyle{super}
\makeglossaries

% links
\usepackage{xcolor}
\usepackage{hyperref}

\definecolor{rltred}{rgb}{0.75,0,0}
\definecolor{rltgreen}{rgb}{0,0.5,0}
\definecolor{rltblue}{rgb}{0,0,0.75}

%sichere Fraben, die sich auch bei einem SW-Druck unterscheiden lassen (Platzhalter momentan)
\definecolor{color1}{cmyk}{1,0,0,0} %cyan
\definecolor{color2}{rgb}{0,1,0} %green

\hypersetup{
  pdftitle={Algebraische Geometrie Prof. Herrlich},
  pdfsubject={Algebraische Geometrie},
  pdfkeywords={Algebraische Geometrie Herrlich},
  pdfproducer={pdfLaTeX},
  pdfpagemode={UseOutlines},
  colorlinks=true,
  bookmarksopen=true,
  bookmarksnumbered=true,
  urlcolor=rltblue,
  filecolor=rltgreen,
  linkcolor=rltblue,
  backref=true,
  pagebackref=true,
  pdfpagemode=None
}

% um Code schreiben zu koennen
\usepackage{listings}

% Mathe-Pakete
\usepackage{amssymb}
\usepackage{amsmath}
%\usepackage{amsthm} & <-- Konflikt mis ntheorem Paket
\usepackage{amsfonts}
\usepackage{stmaryrd}
\usepackage{mathtools}

% \mathcal unterstuetzt keine Kleinbuchstaben, das hier sollte eine brauchbare Alternative bieten
\DeclareMathAlphabet{\mathpzc}{OT1}{pzc}{m}{it}

% F\"ur TikZ-Diagramme
\usepackage{tikz}
\usetikzlibrary{matrix,arrows,calc,intersections}
\usetikzlibrary{decorations.text}

% Verschiedene items in enumerate Umgebungen (Das enumitem Paket bietet mehr Optionen)
\usepackage{enumerate}

% Erweiterte description umgebung
\usepackage{expdlist}

%fuer schoene Brueche
\usepackage{nicefrac}

\usepackage{etex}
\usepackage{pictex}
\usepackage{graphicx}

% Theorem-Umgebung
\usepackage[hyperref,amsmath,thmmarks,thref, amsthm]{ntheorem}

% Fuer verlaengerte Pfeile
\usepackage{extarrows}

% keine kursiv schrift in theorems
\theorembodyfont{}

% Theoreme definieren (konvention: Grossbuchstaben mit nummerierung, Kleinbuchstaben ohne)
\theoremstyle{break}
    \newtheorem{Satz}{Satz}
    \newtheorem{SatzDef}[Satz]{Satz + Definition}
    \newtheorem{UnterSatz}{Satz}[Satz]
    \newtheorem{Def}{Definition}[section]
    \newtheorem{DefBem}[Def]{Definition + Bemerkung}
    \newtheorem{DefErinn}[Def]{Definition + Erinnerung}
    \newtheorem{ErinnDefBem}[Def]{Erinnerung / Definition + Bemerkung}
    \newtheorem{ErinnDef}[Def]{Erinnerung / Definition}
    \newtheorem{Erinn}[Def]{Erinnerung}
    \newtheorem{DefSatz}[Def]{Definition + Satz}
    \newtheorem{Bem}[Def]{Bemerkung}
    \newtheorem{BemDef}[Def]{Bemerkung + Definition}
    \newtheorem{BemErinn}[Def]{Bemerkung + Erinnerung}
    \newtheorem{Prop}[Def]{Proposition}
    \newtheorem{PropDef}[Def]{Proposition + Definition}
    \newtheorem{Folg}[Def]{Folgerung}
    \newtheorem{FolgDef}[Def]{Folgerung + Definition}
    \newtheorem{Bsp}[Def]{Beispiel}
    \newtheorem{Bspe}[Def]{Beispiele}
    \newtheorem{DefProp}[Def]{Definition + Proposition}
    \newtheorem{anBew}[Def]{Beweis}
    \newtheorem{Kor}[Def]{Korollar}
    \newtheorem{Einsch}[Def]{Einschub}
    \newtheorem{Bez}[Def]{Bezeichnung}
    \newtheorem{Lemma}{Lemma}
    \newtheorem{Aufg}{Aufgabe}
	\newtheorem{Verallg}[Def]{Verallgemeinerung}
  \newtheorem{Loes}{L\"osung}
\theoremstyle{nonumberbreak}
	\newtheorem{bem}{Bemerkung}
	\newtheorem{bsp}{Beispiel}
	\newtheorem{bspe}{Beispiele}
	\newtheorem{satz}{Satz}
	\newtheorem{erinn}{Erinnerung}
	\newtheorem{folg}{Folgerung}
	\newtheorem{beo}{Beobachtung}
	\newtheorem{beh}{Behauptung}
	\newtheorem{frag}{Frage}
	\newtheorem{ziel}{Ziel}
	\newtheorem{strat}{Strategie}
	%fuer das Gaestebuch
	\newtheorem{gast}{G\"astebucheintrag}
	\newtheorem{komm}{Kommentar}
	\newtheorem{korr}{Korrktur}
	%fuer Beweise
	\theoremsymbol{\ensuremath{\Box}}
	\newtheorem{bew}{Beweis}
	\newtheorem{bewskiz}{Beweisskizze}
\theoremstyle{nonumberplain}

% Befehl fuer Anfuerungszeichen unten und oben
\newcommand{\quot}[1]{\textrm{\glqq}{#1}\textrm{\grqq}}

\newcommand{\emp}[1]{\textbf{\emph{#1}}}
\newcommand{\linetitle}[1]{\textbf{#1}}

% vertikales Gleichheitszeichen (rotatebox gehoert zum graphicx Paket)
\newcommand{\verteq}{\rotatebox{90}{$=$}}

% fuer Definitionen
\newcommand{\defterm}[1]{{\index{#1}}\emp{#1}}
\newcommand{\deftermspec}[2]{{\index{#2}}\emp{#1}}

\newcommand{\defeqr}[0]{\mathrel{\mathop:}=}
\newcommand{\defeql}[0]{=\mathrel{\mathop:}}

% Summen, Vereinigungen und Schnitte bei denen die Limits immer oben und unten stehen, nie nebendran
\newcommand{\Sum}{\sum\limits}
\newcommand{\Bigcup}{\bigcup\limits}
\newcommand{\Bigcap}{\bigcap\limits}

\newcommand{\folgtnach}[1]{\ensuremath{\DOTSB\;\xRightarrow{\text{#1}}\;}}
\newcommand{\folgtwegen}[1]{\ensuremath{\DOTSB\;\stackrel{#1}{\Rightarrow}\;}}
\newcommand{\equizunach}[1]{\ensuremath{\DOTSB\;\xLeftrightarrow{\text{#1}}\;}}
\newcommand{\equizuwegen}[1]{\ensuremath{\DOTSB\;\xLeftrightarrow{#1}\;}}
\newcommand{\tonach}[1]{\ensuremath{\DOTSB\;\xrightarrow{\text{#1}}\;}}
\newcommand{\towegen}[1]{\ensuremath{\DOTSB\;\xrightarrow{#1}\;}}
\newcommand{\tomit}[1]{\ensuremath{\DOTSB\;\xrightarrow{#1}\;}}
\newcommand{\gleichnach}[1]{\ensuremath{\DOTSB\;\stackrel{\text{#1}}{=}\;}}
\newcommand{\gleichwegen}[1]{\ensuremath{\DOTSB\;\stackrel{#1}{=}\;}}

\newcommand{\Abb}[5]{\ensuremath{#1:\begin{array}{ccc} #2 & \longrightarrow & #3 \\ #4 & \longmapsto & #5 \end{array}}}

% Doppelseitige Beweise
\newenvironment{twosidedproof}{\begin{enumerate}[\quot{$\Rightarrow$}:]}{\end{enumerate}}
\newenvironment{twosidedproofeq}{\begin{enumerate}[\quot{$\subseteq$}:]}{\end{enumerate}}
\newcommand{\proofforward}{\item[\quot{$\Rightarrow$}:]}
\newcommand{\proofreverse}{\item[\quot{$\Leftarrow$}:]}
\newcommand{\proofsubseteq}{\item[\quot{$\subseteq$}:]}
\newcommand{\proofsupseteq}{\item[\quot{$\supseteq$}:]}
\newcommand{\proofle}{\item[\quot{$\le$}]}
\newcommand{\proofge}{\item[\quot{$\ge$}]}

% Diagramm kommutiert.
\newcommand{\schraffiert}{\ensuremath{\nicefrac{\nicefrac{}{}}{\nicefrac{}{}}}}

\newcommand{\myref}[2]{%
\hyperref[#2]{#1~\ref*{#2}}%
}

% Mathe Operatoren
\DeclareMathOperator{\Aff}{Aff}
\DeclareMathOperator{\Alg}{Alg}
\DeclareMathOperator{\Aut}{Aut}
\DeclareMathOperator{\Bild}{Bild}
\DeclareMathOperator{\CaCl}{CaCl}
\DeclareMathOperator{\CaDiv}{CaDiv}
\DeclareMathOperator{\Cl}{Cl}
\DeclareMathOperator{\Ddef}{Def}%Definitionsbereich
\DeclareMathOperator{\Der}{Der}
\DeclareMathOperator{\Div}{Div}
\DeclareMathOperator{\GL}{GL}
\DeclareMathOperator{\Hom}{Hom}
\DeclareMathOperator{\Kern}{Kern}
\DeclareMathOperator{\Mod}{Mod}
\DeclareMathOperator{\Mor}{Mor}
\DeclareMathOperator{\Ob}{Ob}
\DeclareMathOperator{\Off}{Off}
\DeclareMathOperator{\Pic}{Pic}
\DeclareMathOperator{\Proj}{Proj}
\DeclareMathOperator{\Quot}{Quot}
\DeclareMathOperator{\Rang}{Rang}
\DeclareMathOperator{\Rat}{Rat}
\DeclareMathOperator{\Red}{Red}%reduziert
\DeclareMathOperator{\Rg}{Rg}
\DeclareMathOperator{\Sing}{Sing}
\DeclareMathOperator{\Spe}{Sp\'e}
\DeclareMathOperator{\Spec}{Spec}
\DeclareMathOperator{\Var}{Var}
\DeclareMathOperator{\VR}{VR}

\DeclareMathOperator{\chara}{char}
\DeclareMathOperator{\dd}{d} % das "d" zum Beispiel beim Integral am Ende "dx"
\DeclareMathOperator{\Ddiv}{div}
\DeclareMathOperator{\ddiv}{div}
\DeclareMathOperator{\Ht}{ht}
\DeclareMathOperator{\height}{ht}
\DeclareMathOperator{\id}{id}
\DeclareMathOperator{\kgv}{kgV}
\DeclareMathOperator{\kgV}{kgV}
\DeclareMathOperator{\modmodulo}{mod}%\modulo und \mod bereis belegt
\DeclareMathOperator{\op}{op}
\DeclareMathOperator{\ord}{ord}
\DeclareMathOperator{\pr}{pr}
\DeclareMathOperator{\red}{red}%reduziert
\DeclareMathOperator{\sign}{sign}
\DeclareMathOperator{\trdeg}{trdeg}

%Kategorien
\DeclareMathOperator{\Ab}{\underline{Ab}}
\DeclareMathOperator{\Fun}{\underline{Fun}}
\DeclareMathOperator{\Grp}{\underline{Grp}}
\DeclareMathOperator{\Ringe}{\underline{Ringe}}
\DeclareMathOperator{\Sch}{\underline{Sch}}
\DeclareMathOperator{\Set}{\underline{Set}}
\DeclareMathOperator{\Sets}{\underline{Sets}}
\DeclareMathOperator{\Sh}{\underline{Sh}}
\DeclareMathOperator{\Top}{\underline{Top}}

% zweigestrichene Buchstaben im Mathe Umgebungen
\usepackage{bbm} % <- \mathbb sieht ein wenig anders aus als \mathbbm
\newcommand{\A}{\mathbb{A}}
\newcommand{\C}{\mathbb{C}}
%\newcommand{\F}{\mathbb{F}}
\newcommand{\N}{\mathbb{N}}
\newcommand{\IP}{\mathbb{P}}%\P bereits belegt
\newcommand{\Q}{\mathbb{Q}}
\newcommand{\R}{\mathbb{R}}
\newcommand{\Z}{\mathbb{Z}}

% kursive Buchstaben im Mathe Umgebungen
\newcommand{\calC}{\mathcal{C}}
\newcommand{\calB}{\mathcal{B}}
\newcommand{\calE}{\mathcal{E}}
\newcommand{\calF}{\mathcal{F}}
\newcommand{\calG}{\mathcal{G}}
\newcommand{\calHom}{\mathcal{H}\mathpzc{om}}
\newcommand{\calI}{\mathcal{I}}
\newcommand{\calJ}{\mathcal{J}}
\newcommand{\calK}{\mathcal{K}}
\newcommand{\calL}{\mathcal{L}}
\newcommand{\calN}{\mathcal{N}}
\newcommand{\calO}{\mathcal{O}}
\newcommand{\calP}{\mathcal{P}}
\newcommand{\calW}{\mathcal{W}}
\newcommand{\calX}{\mathcal{X}}
\newcommand{\F}{\mathcal{F}} % <-- schnellere alternative
\newcommand{\G}{\mathcal{G}} % <-- schnellere alternative

%alte Buchstaben
\newcommand{\frakp}{\mathfrak{p}}
\newcommand{\frakP}{\mathfrak{P}}
\newcommand{\frakq}{\mathfrak{q}}
\newcommand{\p}{\mathfrak{p}} % <-- schnellere alternative
\newcommand{\q}{\mathfrak{q}} % <-- schnellere alternative

% Eintraege fuer das Symbolverzeichnis (Glossar)
\newglossaryentry{GL}{name={\ensuremath{\GL_n}},description={Allgemeine lineare Gruppe, Gruppe aller regul"aren $n \X n$-Matrizen mit Koeffizienten aus einem K"orper $K$},text={\ensuremath{\GL}},symbol={\ensuremath{\GL}},sort=GL}

%X als Malzeichen
\newcommand{\X}{\times}

% ein schoener aussehender Faktorraum anstaat einfach nur A/B
\newcommand{\FakRaum}[2]{
	\raisebox{0.7ex}{\ensuremath{#1}}
	\ensuremath{\mkern-3mu}\big/\ensuremath{\mkern-3mu}
	\raisebox{-0.6ex}{\ensuremath{#2}}}
\newcommand{\smallFakRaum}[2]{
	\scriptsize{\raisebox{0.7ex}{\ensuremath{#1}}
	\ensuremath{\mkern-3mu}\ / \ensuremath{\mkern-3mu}
	\raisebox{-0.6ex}{\ensuremath{#2}}}}

\renewcommand{\labelenumi}{\theenumi}    
\renewcommand{\theenumi}{(\alph{enumi})}

\newcommand{\ilim}{\mathop{\varprojlim}\limits}

% "ohne Einschraenkung" Zeichen
\renewcommand{\OE}{O\!\!E~}

\newcommand{\nsubset}{\subset\!\!\!\!\!/~}

% Weniger Abstand nach der Ueberschrift des Inhaltsverzeichnisses
\makeatletter
\let\@my@starttoc\@starttoc
\renewcommand*{\@starttoc}[1]{%
  \addvspace{-1.5em}%
  \@my@starttoc{#1}%
}
\makeatother

\setcounter{secnumdepth}{-1}

%\titlehead{inoffizielles Vorlesungsskript}
\subject{inoffizielles Skript}
\title{Geometrie der Schemata\\(Algebraische Geometrie II)}
\subtitle{Gehalten von Prof. Dr. F. Herrlich im Sommersemester 2012}
\author{getippt von Aleksandar Sandic\thanks{Aleksandar.Sandic@student.kit.edu}}


%--------------------------------------------------------------------
% ------------------- Hier beginnt das Skript -------------------
%--------------------------------------------------------------------

\begin{document}

\maketitle

% Inhaltsverzeichnis
\pdfbookmark[1]{Inhaltsverzeichnis}{contents}
\setlength\parskip{0.6pt}
\tableofcontents

% Liste der benannten Saetze
\section*{Benannte S\"atze}
\pdfbookmark[1]{Benannte S\"atze}{contents}

\theoremlisttype{optname}
\listtheorems{Satz,SatzDef,Def,DefBem,BemDef,Prop,PropDef,Bsp,DefProp}

\setlength\parskip{\smallskipamount}


%-------------------------- Vorwort --------------------------

\chapter{Vorwort}
\setcounter{secnumdepth}{2}
\section*{\"Uber dieses Skript}
Dies ist ein Mitschrieb der Vorlesung \glqq Geometrie der Schemata (Algebraische Geometrie II)\grqq\ von Prof. Dr. F. Herrlich im Sommersemster 2012 am Karlsruher Institut f\"ur Technologie (KIT). Prof. Dr. F. Herrlich ist f\"ur  den Inhalt nicht verantwortlich.

\textcolor{red}{Die Vorlesung l\"auft noch und dieses Skript wird w\"ahrend der Vorlesung erweitert. Der Inhalt sollte mit einer kleinen Verz\"ogerung der Vorlesung folgen. Stellen bei denen ich unsicher war sind rot markiert.}
\section*{Wer}
Getippt wurde das Skript von Aleksandar Sandic. Es basiert auf meinem Skript zur Vorlesung Algebraische Geometrie aus dem Wintersemster 2011/12.

\section*{Wo}
Link zum Skript: \url{http://mitschriebwiki.nomeata.de/GeoSchemata2012.html}\\
Link zum Mitschriebwiki: \url{http://mitschriebwiki.nomeata.de/}\\
Seite der Vorlesung: \url{http://www.math.kit.edu/iag3/lehre/geoschemata2012s/de}

\section*{To Do}\begin{itemize}
\item
  Viele Zeichnungen haben sich \"uberschneidende Pfeile. Es s"ahe besser aus, wenn die Pfeile sich nicht einfach \"uberkreuzen w"urden, sondern die unten liegenden Pfeile L"ucken h"atten. Das kann man machen, indem man direkt unter die oberhalb liegenden Pfeile eine etwas etwas dickere wei\ss e Linie zeichnet.
\item
	Eine Literaturliste anlegen und die B"ucherverweise korrekt aufschreiben.
\item
	Ich muss Farben finden, die sich auch bei einem SW-Ausdruck unterscheiden lassen und diese als Standard definieren.
%\item Design aufpolieren (weniger wichtig)
%\item L\"osungen der \"Ubungbl\"atter einbauen (weniger wichtig)
\end{itemize}

%-------------------------- Kapitel 1 --------------------------

\chapter{Schemata}

%-------------------------- Abschnitt 1 (Garben) --------------------------
%18. 4.
\section{Garben}

\begin{Def}
Sei $X$ ein topologischer raum, $\mathcal C$ eine Kategorie.\\
Eine \deftermspec{Pr\"agarbe}{Garbe!Pr\"a-} $\mathcal F$ auf $X$ mit Werten in $\mathcal C$ besteht aus einer Abbildung $\Off(X) \to \Ob \mathcal C$, $U \mapsto \mathcal F(U)$ und Morphismen $\rho_U^{U'}: \mathcal F(U') \to \mathcal F(U)$ f\"ur alle $U\subseteq U'$ offen, sodass gilt:
\begin{enumerate}[i)]
\item
  $\rho_U^U = \id_U$ f\"ur alle $U\in \Off(X)$
\item
  $\rho_U^{U''} = \rho_U^{U'} \circ \rho_{U'}^{U''}$ f\"ur alle $U\subseteq U' \subseteq U''$ in $\Off(X)$
\end{enumerate}\end{Def}

\begin{Bem}
Eine Pr\"agarbe $\mathcal F$ auf $X$ mit Werten in $\mathcal C$ ist dasselbe wie ein kontravarianter Funktor $\mathcal F: \Off(X) \to \mathcal C$.
\end{Bem}

\begin{Def}\label{1.3}
Eine Pr\"agarbe $\mathcal F$ auf $X$ mit Werten in $\mathcal C$ hei\ss t \defterm{Garbe}, wenn f\"ur jedes $U\in \Off(X)$, jede offene \"Uberdeckung $(U_i)_{i\in I}$ von $U$ und jede Familie $(s_i \in \mathcal F(U_i))_{i \in I}$ mit $\rho_{U_i\cap U_j}^{U_i}(s_i) = \rho_{U_i\cap U_j}^{U_j}(s_j)$ f\"ur alle $i,j \in I$ gilt:\\
Es gibt \emph{genau ein} $s \in \mathcal F(U)$ mit $\rho_{U_i}^U(s) = s_i$ f\"ur alle $i \in I$.
\end{Def}

\begin{bspe}\label{bsp1.3}\begin{enumerate}[1)]
\item
  $X$ quasi-projektive Variet\"at \"uber einem K\"orper $k$, $\mathcal O(U)$ Ring der regul\"aren Funktionen auf $U$.\\
  $\Rightarrow \mathcal O_X$ ist Garbe von Ringen auf $X$ ($k$-Algebren)
\item
  $X$ topologischer Raum, $\mathcal C_X(U) := \{f: U\to \R \text{ stetig}\}$\\
  $\mathcal C_X$ ist Garbe von Ringen
\item\label{bsp1.3.3}
  Sei $X$ topologischer Raum, $G$ Gruppe\\
  $\mathcal G(U) := G$ f\"ur alle $U \subseteq X$ offen\\
  $\rho_U^{U'} = \id_G$\\
  Seien $U, U'$ offen in $X$ mit $U\cap U' = \emptyset$\\
  $\widetilde U = U \cup U'$ ?!\\
  Finde kein $g\in \mathcal G(\widetilde U) = G$ mit $g = \left\{\begin{array}{rcl} \rho_{U}^{\widetilde U}(g) &=& g_1 \ne g_2 \\ \rho_{U'}^{\widetilde U}(g) &=& g_2 \ne g_1 \end{array}\right.$
\end{enumerate}\end{bspe}

\begin{Bem}
Sei $X$ topologischer Raum, $\mathcal F$ Garbe auf $X$.\\
Dann ist $\mathcal F(\emptyset)$ einelementig.
\end{Bem}

\begin{bew}
\"Uberdecke $\emptyset$ durch eine leere Menge von offenen Teilmengen! Jedes $s\in\mathcal F(\emptyset)$ erf\"ullt $\rho_{\emptyset}^{U_i}(s_i) = s$ f\"ur alle $i\in \emptyset$, $\emptyset = \bigcup\limits_{i\in \emptyset} U_i$. Also gibt es genau ein $s  \in\mathcal F(\emptyset)$.
\end{bew}

\begin{Def}
Sei $X$ ein topologischer Raum, $\mathcal F$, $\mathcal G$ Pr\"agarben auf $X$.\\
Ein \defterm{Morphismus} $\varphi: \mathcal F \to \mathcal G$ ist eine nat\"urliche Transformation der Funktoren $\mathcal F, \mathcal G: \Off(X) \to \mathcal C$, das hei\ss t $\varphi$ besteht aus Morphismen (in $\mathcal C$) $\varphi_U: \mathcal F(U) \to \mathcal G(U)$ f\"ur jedes $U\in \Off(X)$, die die folgenden Diagramme kommutativ machen:
\begin{center}\begin{tikzpicture}[baseline=0]
  %Punkte definieren
  \node (a) at (-1.5,1){$\mathcal F(U')$};
  \node (b) at (1.5,1){$\mathcal G(U')$};
  \node (c) at (-1.5,-1){$\mathcal F(U)$};
  \node (d) at (1.5,-1){$\mathcal G(U)$};
  \node (0,0){$\schraffiert$};
  \draw[->] (a) -- node[above] {$\varphi_{U'}$} (b);
  \draw[->] (c) -- node[below] {$\varphi_U$} (d);
  \draw[->] (a) -- node[left] {$\rho_U^{U'}$} (c);
  \draw[->] (b) -- node[right] {$\rho_U^{U'}$} (d);
\end{tikzpicture} $\qquad$ f\"ur alle $U\subseteq U'$ in $\Off(X)$\end{center}
\end{Def}

\begin{DefBem}
Sei $\mathcal F$ eine Pr\"agarbe auf X.
\begin{enumerate}[a)]
\item
  F\"ur ein $x\in X$ sei
    \[\mathcal F_X = \lim_{x\in U \in \Off(X)} \mathcal F(U) = \FakRaum{\{(U,s): x\in U\in \Off(X), s\in \mathcal F(U)\}}{{\sim}}\]
  wobei $(U,s)\sim(U',s') :\Leftrightarrow \exists x\in U'' \subseteq U\cap U'$ mit $\rho_{U''}^U(s) = \rho_{U''}^{U'}(s')$
\item
  F\"ur $x\in U \in \Off(X)$ sei $\mathcal F(U) \to \mathcal F_X, s\mapsto (U,s)_{\sim} =: s_x$ der \deftermspec{nat\"urliche Morphismus}{Morphismus!nat\"urlicher}.
\item
  (UAE)\\
  F\"ur jedes $C\in \Ob \mathcal C$ und jede konsistente Familie $\varphi_U: \mathcal F(U) \to C$ von Morphismen in $\mathcal C$ gibt es genau einen Morphismus $\varphi_x: \mathcal F_X \to C$ mit $\varphi_x \circ \sigma_X = \varphi_U$ f\"ur alle $U$
    \[(U,s)_{\sim} \mapsto \varphi_U(s)\]
\item
  Jeder Morphismus $\varphi: \mathcal F \to \mathcal G$ induziert f\"ur jedes $x\in X$ einen Morphismus
    \[\varphi_x: \mathcal F_X \to \mathcal G_X\]
  \begin{tikzpicture}
    \node (a) at (-1.5,1){$\mathcal F(U)$};
    \node (b) at (1.5,1){$\mathcal G(U)$};
    \node (c) at (-1.5,-0.5){$\mathcal F_X$};
    \node (d) at (1.5,-0.5){$\mathcal G_X$};

    \draw[->] (a) -- node[above] {$\varphi_{U}$} (b);
    \draw[->, dashed] (c) -- node[below] {$\varphi_x$} (d);
    \draw[->] (a) -- (c);
    \draw[->] (b) -- (d);
  \end{tikzpicture}
\end{enumerate}\end{DefBem}

%19. 4.

\begin{Bem}\label{1.7}
Sei $\calF$ Garbe von abelschen Gruppen auf $X$, $U\subseteq X$ offen, $s\in \calF(U)$.\\
Dann gilt:
  \[s=0 \Leftrightarrow s_x = 0 \text{ f\"ur alle } x\in U\]
\end{Bem}

\begin{bew}\begin{twosidedproof}
\proofforward
  $\checkmark$
\proofreverse
  F\"ur jedes $x\in U$ gibt es Umgebung $U_x$ mit $s|_{U_x} = 0$. ($s|_{U_x} = \rho_{U_x}^U$)\\
  Die $(U_x)_{x\in U}$ bilden offene \"Uberdeckungen, die $s|_{U_x}$ bilden konsistente Familie, $s$ und $0$ sind beides Amalgame $\xRightarrow[\text{eigenschaft}]{\text{Gruppen-}} s= 0$
\end{twosidedproof}\end{bew}

\begin{Prop}\label{1.8}
Seien $\calF, \calG$ Garben von abelschen Gruppen auf $X$, $\varphi: \calF \to \calG$ Morphismus.
\begin{enumerate}[a)]
\item\label{1.8a}
  $\varphi_U$ injektiv f\"ur jedes $U\in \Off(X) \Leftrightarrow \varphi_x$ injektiv f\"ur alle $x\in X$
\item
  $\varphi_U$ surjektiv f\"ur jedes $U\in \Off(X) \Rightarrow \varphi_x$ surjektiv f\"ur alle $x\in X$
\item
  $\varphi_U$ bijektiv f\"ur jedes $U\in \Off(X) \Leftrightarrow \varphi_x$ bijektiv f\"ur alle $x\in X$
\end{enumerate}\end{Prop}

\begin{bew}\begin{enumerate}[a)]
\item
  \begin{twosidedproof}
  \proofforward
    Sei $x\in X, s_x\in \calF$ mit $\varphi_x(s_x)=0$\\
    $\exists$ Umgebung von $x$ und $s\in \calF(U)$ mit $s_x =$ Keim von $s$ in $x$ mit $\varphi_x(s_x)=$ Keim von $\varphi_U(s)$ in $x = \varphi_U(s)_s$
    \begin{center}\begin{tikzpicture}
      \node (F) at (-2,1) {$\calF(U)$};
      \node (G) at (2,1) {$\calG(U)$};
      \node (Fx) at (-2,-1) {$\calF_x$};
      \node (Gx) at (2,-1) {$\calG_x$};
      
      \node (s) at (-2.5,1.5) {$s$};
      \node (phis) at (2.5,1.5) {$\varphi_U(s)$};
      \node (sx) at (-2.5,-1.5) {$s_x$};
      \node (0) at (2.5,-1.5) {$0$};
      
      \node (a) at (-0.5,3) {};
      
      \draw[->] (F) -- node[above] {$\varphi_{U}$} (G);
      \draw[->] (Fx) -- node[above] {$\varphi_{x}$} (Gx);
      \draw[->] (F) -- (Fx);
      \draw[->] (G) -- (Gx);
      
      \draw[|->] (s) to [out=20,in=160]  (phis);
      \draw[|->] (s) to [out=240,in=120] (sx);
      \draw[|->] (sx) to [out=340,in=200] (0);
      \draw[|->] (phis) to [out=300,in=60] (0);
    \end{tikzpicture}\end{center}
    $\Rightarrow$ \OE $\varphi_U(s) = 0 \xRightarrow[\text{injektiv}]{\varphi_U} s= 0$
  \proofreverse
    Sei $U\subset \Off(X), s\in \calF(U)$ mit $\varphi_U(s) = 0$\\
    $\Rightarrow$ f\"ur alle $x\in U$ ist $\varphi_x(s_x) = \varphi_U(s)_x = 0 \xRightarrow{\varphi \text{ injektiv}} s_x = 0 \xRightarrow{\ref{1.7}} s = 0$
  \end{twosidedproof}
\item
  \begin{twosidedproof}
  \proofforward
    Sei $g_x\in\calG_x$, $(U,g)$ Repr\"asentant\\
    $\Rightarrow \exists s\in \calF(U)$ mit $\varphi_U(s)=g \Rightarrow \varphi_x(s_x) = g$
    \begin{center}\begin{tikzpicture}
      \node (F) at (-2,1) {$\calF(U)$};
      \node (G) at (2,1) {$\calG(U)$};
      \node (Fx) at (-2,-1) {$\calF_x$};
      \node (Gx) at (2,-1) {$\calG_x$};
      
      \node (s) at (-2.5,1.5) {$s$};
      \node (phis) at (2.5,1.5) {$\varphi_U(s)$};
      \node (g) at (3.5,1.5) {$= g$};
      \node (sx) at (-2.5,-1.5) {$s_x$};
      \node (0) at (2.5,-1.5) {$0$};
      \node (gx) at (3.5,-1.5) {$g_x$};
      
      \node (a) at (-0.5,3) {};
      
      \draw[->] (F) -- node[above] {$\varphi_{U}$} (G);
      \draw[->] (Fx) -- node[above] {$\varphi_{x}$} (Gx);
      \draw[->] (F) -- (Fx);
      \draw[->] (G) -- (Gx);
      
      \draw[|->] (s) to [out=20,in=160]  (phis);
      \draw[|->] (s) to [out=240,in=120] (sx);
      \draw[|->] (sx) to [out=340,in=200] (0);
      \draw[|->] (phis) to [out=300,in=60] (0);
      \draw[|->] (g) to [out=300,in=60] (gx);
    \end{tikzpicture}\end{center}
  \end{twosidedproof}
  \begin{bsp}
  Sei $X = \C$, $\calO$ die Garbe der holomorphen Funktionen auf $\C$, $\calO^\X$ die Garbe der invertierbaren holomorphen Funktionen.\\
  $\varphi = \exp$, das hei\ss t f\"ur $f\in \calO(U)$ sei $\varphi(f) = e^{2\pi if}$.\\
  $\varphi$ ist Garbenhomomorphisms ($e^{f+g} = e^f \cdot e^g$). $\varphi_x$ ist surjektiv f\"ur jedes $x\in X$ (lokal gibt es zu jeder holomorphen Funktion ohne Nullstellen einen Logarithmus). $\varphi_{\C\setminus\{0\}}$ ist nicht surjektiv! (zum Beispiel gibt es keine holomorphe Funktion $\log z$ auf ganz $\C$)\\
  Schlimmer noch: $\varphi_U$ ist nicht injektiv f\"ur jedes $U \in \Off(\C)$, das nicht einfach zusammenh\"angend ist.
  \end{bsp}
\item
  \begin{twosidedproof}
  \proofforward
    $\checkmark$
  \proofreverse
    Sei $U\subseteq X$ offen, $g\in \calG(U)$.\\
    F\"ur jedes $x\in U$ gibt es $s_x\in \calF_x$ mit $\varphi_x(s_x) = g_x$.\\
    W\"ahle Repr\"asentanten $(U_x, s^{(x)})$ von $s_x$, sodass $\varphi_{U_x}(s^{(x)}) = g|_{U_x}$ (das geht!) (\emph{denn}: Sei $(U,\tilde s)$ Repr\"asentant von $s_x \Rightarrow \varphi_U(\tilde s)\sim_x g|_U \Rightarrow \exists x \in U_x \subset U: \varphi_U(\tilde s)|_{U_x} = g|_{U_x}$)\\
    Die $U_x$ bilden offene \"Uberdeckungen von $U$, die $s^{(x)}$ bilden konsistente Familie (*) $\Rightarrow \exists$ Amalgam $s\in \calF(U)$ mit $\varphi_U(s)|_{U_x} = \varphi_{U_x}(s^{(x)}) = g|_{U_x} \Rightarrow \varphi_U(s) = g$\\
    (*) zu zeigen: $s^{(x)}|_{U_x\cap U_y} = s^{(y)}|_{U_x\cap U_y}$\begin{description}[\setlabelstyle{\itshape}]
    \item[denn:] $\varphi_{U_x\cap U_y}(s^{(x)}|_{U_x\cap U_y} = \varphi_{U_x\cap U_y}(s^{(y)}|_{U_x\cap U_y}$, $\varphi_{U_x\cap U_y}$ injektiv nach Voraussetzung und a) $\Rightarrow $ Behauptung
    \end{description}
  \end{twosidedproof}
\end{enumerate}\end{bew}

\begin{PropDef}
Sei $X$ topoloischer Raum, $\calF$ Pr\"agarbe auf $X$ (mit Werten in $\calC$)
\begin{enumerate}[a)]
\item
  Es gibt genau eine Garbe $\calF^+$ auf $X$ und einen Morphismus $\vartheta: \calF \to \calF^+$, sodass $\vartheta_x:\calF_x \to \calF_x^+$ f\"ur jedes $x\in X$ ein Isomorphismus ist.
\item
  $\calF^+$ hei\ss t \deftermspec{zu $\calF$ assoziierte Garbe}{Garbe!assoziierte}.
\item(UAE)\\
  F\"ur jeden Morphismus $\varphi: \calF \to \calG$ in eine Garbe $\calG$ gibt es genau einen Morphismus $\varphi^+: \calF^+ \to \calG$ mit $\varphi = \varphi^+ \circ \vartheta$.
  \begin{center}\begin{tikzpicture}
    \node (F) at (-1,0) {$\calF$};
    \node (F+) at (1,0) {$\calF^+$};
    \node (G) at (0,-1.5) {$\calG$};
    
    \draw[->] (F) -- node[above]{$\vartheta$} (F+);
    \draw[->] (F) -- node[left]{$\varphi$} (G);
    \draw[->, dashed] (F+) -- node[right]{$\exists!$} (G);
  \end{tikzpicture}\end{center}
\end{enumerate}\end{PropDef}

\begin{bew}\begin{enumerate}[a)]\item
F\"ur $U\in \Off(X)$ sei
  \[\begin{array}{rcl}\calF^+(U) &:=& \{s: U\to \dot{\bigcup\limits_{x\in U}}\calF_x | s(x)\in \calF_x \forall\ x\in X \text{, zu jedem } x\in U \\ &&\text{ gibt es Umgebung } U_x \text{ und } f\in \calF(U) \text{ mit } s(y) = f_y \\&& \text{ f\"ur jedes } y \in U_y\}\end{array}\]
$\calF^+$ ist Garbe $\checkmark$\\
Sei $\vartheta: \calF \to \calF^+$ gegeben durch
  \[\boxed{\vartheta_U(f)(x) = f_x \quad (U\in \Off(X), f\in \calF(U))}\]
$\vartheta$ ist Morphismus: $\checkmark$\\
$\vartheta$ ist Isomorphismus: $\checkmark$
\end{enumerate}\end{bew}

\begin{DefBem}
Sei $\varphi:\calF \to \calG$ Morphismus von Garben abelscher Gruppen auf $X$.
\begin{enumerate}[a)]
\item
  Sei $\Kern(\varphi)$ die Pr\"agarbe $\Kern(\varphi)(U) := \Kern(\varphi_U)$.
\item
  $\Kern(\varphi)$ ist Garbe.
\item
  $\varphi$ hei\ss t \defterm{injektiv} (oder \deftermspec{Monomorphismus}{Morphismus!Mono-}) $:\Leftrightarrow \Kern(\varphi) = 0$
    \[\begin{array}{c}\mathcal E_1\overset{\psi_1}{\to}\\\mathcal E_2\overset{\psi_2}{\to}\end{array}\calF \overset{\varphi}{\to} \calG\]
  $\varphi$ Monomorphismus $\Leftrightarrow \varphi \circ \psi_1 = \varphi \circ \psi_2 \Rightarrow \psi_1  = \psi_2$
\item
  Sei $\calB_\varphi$ die Pr\"agarbe $\calB_\varphi(U) := \Bild(\varphi_U)$\\
  $\Bild(\varphi_U) := \calB_\varphi^+$
\item
  $\varphi$ hei\ss t \defterm{surjektiv} (oder \deftermspec{Epimorphismus}{Morphismus!Epi-}) $:\Leftrightarrow \Bild(\varphi) = \calG$
      \[\calF \overset{\varphi}{\to} \calG \begin{array}{c}\overset{\psi_1}{\to}\mathcal H_1\\\overset{\psi_2}{\to}\mathcal H_2\end{array}\]
  $\varphi$ Epimorphismus $\Leftrightarrow \psi_1 \circ \varphi = \psi_2 \circ \varphi \Rightarrow \psi_1 = \psi_2$
\end{enumerate}\end{DefBem}

\begin{bew}\begin{enumerate}[a)]
\item \textcolor{white}{a} %<-- ohne das wird das Design zerschossen
  \begin{center}\begin{tikzpicture}
    \node (FU') at (-2,1) {$\calF(U')$};
    \node (GU') at (2,1) {$\calG(U')$};
    \node (FU) at (-2,-1) {$\calF(U)$};
    \node (GU) at (2,-1) {$\calG(U)$};
    
    \draw[->] (FU') --node[above]{$\varphi_{U'}$} (GU');
    \draw[->] (FU') --node[left]{$\rho_{U}^{U'}$} (FU);
    \draw[->] (GU') --node[right]{$\rho_{U}^{U'}$} (GU);
    \draw[->] (FU) --node[above]{$\varphi_{U}$} (GU);
  \end{tikzpicture}\end{center}
\item
  Sei $(U_i)_{i\in I}$ offene \"Uberdeckung von $U\in \Off(X)$, $s_i \in \Kern(\varphi_{U_i}) \subseteq \calF(U_i)$ konsistente Familie.\\
  $\exists!$ Amalgam $s\in \calF(U)$.\\
  $\varphi_x(s_x) = 0$ f\"ur jedes $x\in U$ $\xRightarrow{\ref{1.8} \ref{1.8a})} \varphi_U(s) = 0$
%25.4.
\item[e)]
  $\Bild(\varphi) = \calG \Leftrightarrow \underbrace{\Bild(\varphi)_x}_{=\Bild(\varphi_x)} = \calG_x$ f\"ur alle $x$\\
  $\varphi$ Epimorphismus $\Leftrightarrow$ f\"ur jedes $x\in X$ ist $\varphi_x$ surjektiv, das hei\ss t $\Bild(\varphi_x) = \calG_x$.
\end{enumerate}\end{bew}

\begin{Def}
Seien $\calF, \calG$ Garben abelscher Gruppen auf $X$, $i:\calG \to \calF$ Monomorphismus.
\begin{enumerate}[a)]
\item
  $\calG$ hei\ss t \deftermspec{Untergarbe}{Garbe!Unter-} von $\calF$.
\item
  $U\mapsto \FakRaum{\calF(U)}{\calG(U)}$ ist Pr\"agarbe auf $X$, die assoziierte Garbe $\FakRaum{\calF}{\calG}$ hei\ss t \deftermspec{Quotientengarbe}{Garbe!Quotienten-}.
\end{enumerate}\end{Def}

\begin{bsp}
\begin{minipage}[t]{0.6\textwidth}
Sei $X=S^1$ (Einheitskreislinie)\\
$\calF = \calC$ (stetige Funktionen $S^1\to \R$)\\
$\calG = $ konstante Garbe $\Z$\\
$U = x$, $U_1, U_2$ wie im Bild
\end{minipage}
\begin{minipage}[t]{0.4\textwidth}
\textcolor{white}{[ ! BILD ! ]}\\
\begin{tikzpicture}
  \node at (-2.75,0) {\textcolor{color2}{$D_1$}};
  \node at (2.75,0) {\textcolor{color2}{$D_2$}};
  
  \node at (2.25,1.5) {$U_1$};
  \node at (2.25,-1.5) {$U_2$};
  
  \draw[yellow, line width=5] (2,0) arc (0:180:2);
  \draw[red, line width=5] (-2,0) arc (180:360:2);
  
  \draw[orange, line width=5] (-1.94,0.5) arc (165.5:194.5:2);
  \draw[orange, line width=5] (1.94,-0.5) arc (-14.5:14.5:2);
  
  \draw[very thick] (0,0) circle (2);

  \draw[color2, thick] (-2.25,0.5) -- (-2.25,-0.5) -- (-1.75,-0.5) -- (-1.75,0.5) -- cycle;
  \draw[color2, thick] (2.25,0.5) -- (2.25,-0.5) -- (1.75,-0.5) -- (1.75,0.5) -- cycle;
\end{tikzpicture}
\end{minipage}
  \[U_1 \cap U_2 = D_1 \cup D_2\]
Sei $f_1 \in \calF(U_1)$ mit $f_1|_{D_1} = 0, f_1|_{D_2} = 1, 0 = f_2 \in \calF(U_2)$\\
$\Rightarrow f_1|_{U_1\cap U_2} \in \calG(U_1\cap U_2)$ (!)\\
$\bar{f_1} = \bar{f_2}$ in $\FakRaum{\calF(U_1\cap U_2)}{\calG(U_1\cap U_2)}$\\
$\Rightarrow (\bar{f_i} \in \FakRaum{\calF(U_i)}{\calG(U_i)})_{i=1,2}$ ist konsistente Familie.\\
\emph{Aber:} Es gibt kein $f\in \calF(U)$ mit $f|_{U_1} = f_1, f|_{U_2} = f_2$
\end{bsp}

\begin{Prop}
Sei $X$ topologischer Raum, $U\subseteq X$ offen, $x\in X$.
\begin{enumerate}[a)]
\item
  Die Zuordnung $\Phi: \calF \mapsto \calF_x$ induziert exakten kovarianten Funktor von der Kategorie $\Sh(X)$ der Garben abelscher Gruppen auf $X$ in die Kategorie $\Ab$ der abelschen Gruppen.\\
  Dabei ist $\Phi_x(\varphi) = \varphi_x$ f\"ur $\varphi: \calF \to \calG$ Morphismus.
\item
  Die Zuordnung $\Phi_U: \F \to \F(U)$ induziert linksexakten kovarianten Funktor $\Sh(X) \to \Ab$ (mit $\Phi_U(\varphi) = \varphi_U$)
\end{enumerate}\end{Prop}

\begin{bew}
(*) Sei $0 \to \F' \overset{\varphi}{\to} \F \overset{\psi}{\to} \F'' \to 0$ exakte Sequenz in $\Sh(X)$. \textbf{Achtung:} Das bedeutet \emph{nicht}, dass f\"ur jedes $\widetilde U \in \Off(X)$ die Sequenz $0 \to \F'(\widetilde U) \to \F(\widetilde U) \to \F''(\widetilde U) \to 0$ exakt sein muss.\\
Aber: (*) ist \"aquivalent zu: $0 \to \F'_y \overset{\varphi_y}{\to} \F_y \overset{\psi_y}{\to} \F''_y \to 0$ ist exakt f\"ur jedes $y \in X$\\
$\Rightarrow $ a)
\begin{enumerate}[a)]\item[b)]
$\Phi_U$ linksexakt bedeutet:
  \[0 \to \F'(U) \overset{\varphi_U}{\to} \F(U) \overset{\psi_U}{\to} \F''(U) \to 0 \text{ ist exakt}\]
Das stimmt nach \ref{1.8} und \dots
\end{enumerate}\end{bew}

\begin{DefBem}
Sei $f: X\to Y$ stetig.
\begin{enumerate}[a)]
\item
  Sei $\F$ Garbe auf $X$.\\
  Dann ist die Pr\"agarbe $U\mapsto \F(f^{-1}(U))$ auf $Y$ eine Garbe, sie hei\ss t (direkte) \deftermspec{Bildgarbe}{Garbe!Bild-} von $\F$ (unter $f$). Bezeichnung: $f_*\F$
\item
  Sei $\G$ Garbe auf $Y$.\\
  Die zur Pr\"agarbe $U\mapsto \lim\limits_{\substack{f(U)\subseteq V\\ V\in \Off(Y)}} \G(V)$ assoziierte Garbe hei\ss t \deftermspec{Urbildgarbe}{Garbe!Urbild-} von $\G$.\\
  Bezeichnung: $f^{-1}(G)$
\item
  $f_*$ und $f^{-1}$ sind kovariante Funktoren.
\end{enumerate}\end{DefBem}

\begin{bew}\begin{enumerate}[a)]
\item
  Sei $U\subseteq Y$ offen, $(U_i)_{i\in I}$ offene \"Uberdeckung von $U$, also $(f^{-1}(U_i))_{i\in I}$ offene \"Uberdeckung von $f^{-1}(U)$. $s_i \in \underbrace{f_*\F(U_i)}_{=\F(f^{-1}(U_i))}$, $i \in I$, konsistente Familie.\\
  $\Rightarrow \exists$ Amalgam $s\in \underbrace{\F(f^{-1}(U))}_{=f_*\F(U)}$ mit $s|_{f^{-1}(U_i)} = s_i$ f\"ur alle $i \in I$.
\item[c)]
  Sei $\varphi:\F \to \G$ Morphismus von Garben auf $X$.
  \begin{enumerate}[i)]
  \item
    Definiere $\varphi_*: f_*\F \to f_* \G$ durch
      \[(\varphi_*)_U = \varphi_{f^{-1}(U)}: f_*\F(U) = \F(f^{-1}(U)) \to f_x\G(U) = \G(f^{-1}(U))\]
  \item
    Definiere $f^{-1}\varphi: f^{-1}\F \to f^{-1}\G$ durch $(f^{-1}_\varphi)_U = \lim\limits_{f^{-1}(U)\subseteq V\in \Off(Y)} \varphi_V$\\
    \begin{center}$V'\subseteq V \qquad$
    \begin{tikzpicture}[baseline = 0]
      \node (FV) at (-2,1) {$\F(V)$};
      \node (FV') at (2,1) {$\F(V')$};
      \node (GV) at (-2,-1) {$\G(V)$};
      \node (GV') at (2,-1) {$\G(V')$};
      
      \node(fF) at (4,2) {$f^{-1}\F(U)$};
      \node(fG) at (4,-2) {$f^{-1}\G(U)$};
      
      \draw[->] (FV) -- (FV');
      \draw[->] (FV) -- node[right]{$\varphi_V$} (GV);
      \draw[->] (GV) -- (GV');
      \draw[->] (FV') -- node[right]{$\varphi_{V'}$} (GV');
      
      \draw[->] (FV) to [out=30,in=170] (fF);
      \draw[->] (GV) to [out=330,in=190] (fG);
      \draw[->] (FV') -- (fF);
      \draw[->] (GV') -- (fG);
      \draw[->, dashed] (fF) to [out=300,in=60] (fG);
    \end{tikzpicture}\end{center}
  \end{enumerate}
\end{enumerate}\end{bew}

\begin{Prop}\label{1.14}
Sei $f:X\to Y$ stetig, $\F$ eine Garbe auf $X$, $\G$ Garbe auf $Y$.\\
Dann gibt es eine (nat\"urliche) Bijektion
  \[\Hom(f^{-1}\G, \F) \to \Hom(\G, f_*\F)\]
Das bedeutet: $f^{-1}$ ist linksadjungiert zu $f_*$.
\end{Prop}

\begin{bew}
Definiere $\varphi_\F: f^{-1}f_*\F \to \F$ und $\psi_\G: \G \to f_*f^{-1}\F$\\
Dann:
  \[ T_1:\left\{ \begin{array}{ccc}\Hom(f^{-1}\G,\F) &\to& \Hom(\G, f_*\F)\\
  \alpha &\mapsto& f_*(\alpha)\circ\psi_\G: \G \to f_*f^{-1}\G \overset{f_*\alpha}{\to} f_*\F\end{array}\right.\]
Analog: $T_2: \beta \mapsto \varphi_\F \circ f^{-1}(\beta)$\\
Rest: \"Ubung
\end{bew}

\newpage
%-------------------------- Abschnitt 2 (Affine Schemata) --------------------------
%26. 4.

\section{Affine Schemata}

\begin{description}
\item[Behauptung:]
  $\alpha:R\to R'$ Ringhomom., $\p\subset R'$ Primideal $\Rightarrow \alpha^{-1}(\p)$ Primideal in $R$
\item[Beweis:]
  Seien $f,g \in R$ mit $f\cdot g \in \alpha^{-1}(\p)$\\
  $\Rightarrow \underbrace{\alpha(f\cdot g)}_{=\alpha(f)\cdot\alpha(g)} \in \p \xRightarrow{\text{\OE}} \alpha(f) \in \p \Rightarrow f \in \alpha^{-1}(\p)$
\end{description}

\begin{DefBem}
Sei $R$ ein Ring (das hei\ss t kommutativer Ring mit Eins)
\begin{enumerate}[a)]
\item
  $\Spec R:= \{\p \subset R: \p \text{ Primideal}\}$ hei\ss t \defterm{Spektrum} von $R$.
\item
  F\"ur $I\subseteq R$ sei $V(I)=\{\p \in \Spec R: I \subseteq \p\}$. $V(I)$ hei\ss t \defterm{Nullstellenmenge} (vanishing set) von $I$, es ist $V(I) = V((I))$.
\item
  Die $V(I)$, $I\subseteq R$ Ideal, bilden die abgeschlossenen Mengen einer Topologie auf $\Spec R$, der \deftermspec{Zariski Topologie}{Topologie!Zariski-}.
\item
  F\"ur $V\subseteq \Spec R$ hei\ss t $I(V):= \bigcap\limits_{\p \in V}\p$ \deftermspec{Verschwindungsideal}{Ideal!Verschwindungs-} von $V$.
\end{enumerate}\end{DefBem}

\begin{bew}\begin{enumerate}[a)]\item[c)]
$\emptyset = V(R)$\\
$\Spec R = V(0)$\\
$\bigcap\limits_{i\in I} V(I_i) = \bigcap\limits_{i\in I}\{\p \in I_i \subseteq \p\} = \{\p: I_i \subseteq \p \forall\  i\} = V(\bigcup\limits_{i\in I} I_i) = V(\Sum_{i\in I} I_i)$\\
$V(I_1)\cup V(I_2) = \{\p \in \Spec R: I_1 \subseteq \p \text{ oder } I_2 \subseteq \p\} = V(I_1\cdot I_2) \overset{?}{=} V(I_1\cap I_2)$\\
\hspace*{10mm} $I_1\cdot I_2 \subseteq I_1 \cap I_2 \Rightarrow V(I_1\cdot I_2) \supseteq V(I_1\cap I_2)$\\
\hspace*{10mm} $\p \in V(I_1\cdot I_2) \Rightarrow I_1\cdot I_2 \subseteq \p \xRightarrow{\text{\OE}} I_1 \subseteq \p \Rightarrow I_1\cap I_2 \subseteq \p$\\
$\Rightarrow V(I_1\cdot I_2) = V(I_1\cap I_2)$
\end{enumerate}\end{bew}

\begin{Bem}\begin{enumerate}[a)]
\item
  $V(I(V)) = \bar{V}$ f\"ur jedes $V\subseteq \Spec R$
\item
  $I(V(I)) = \sqrt{I}$ f\"ur jedes ideal $I\subseteq R$
\end{enumerate}\end{Bem}

\begin{bew}\begin{enumerate}[a)]
\item
  \begin{twosidedproof}
  \proofsupseteq
    $V\subseteq V(I(V)) = \{\q \in \Spec R: \bigcap\limits_{\p \in V} \p \subseteq \q \}$
  \proofsubseteq
    Es ist $\bar{V} = \bigcap\limits_{V\subseteq V(I)}V(I)$\\
    Ist $I$ Ideal in $R$ mit $V\subseteq V(I)$, so ist $I\subseteq I(V) = \bigcap\limits_{\p \in V} \p$.\\
    $\Rightarrow V(I) \supseteq V(I(V))$\\
    $\Rightarrow V(I(V)) \subseteq \bigcap\limits_{I:V\subseteq V(I)} V(I) = \bar{V}$
  \end{twosidedproof}
\item
  $I(V(I)) = \bigcap\limits_{\p \in V(I)} \p = \bigcap\limits_{I\subseteq \p} \p \overset{!}{=} \sqrt{I}$ (\"Ubung)
\end{enumerate}\end{bew}

\begin{Prop}
Sei $V\subseteq \Spec R$ abgeschlossen, $V\ne 0$.\\
Dann gilt: $V$ irreduzibel $\Leftrightarrow I(V)$ Primideal
\end{Prop}

\begin{bew}
Wie in Algebraische Geometrie I, Proposition 4.4
\end{bew}

\begin{Bem}\label{2.4}
Jeder Ringhomomorphismus $\alpha: R \to R'$ induziert stetige Abbildung $f_\alpha:\Spec R' \to R$ durch $\p \mapsto \alpha^{-1}(\p)$, das hei\ss t $\Spec: \left\{\begin{array}{ccc} \Ringe &\to& \Top \\ R &\mapsto& \Spec R\end{array}\right.$ ist kontravarianter Funktor.
\end{Bem}

\begin{bew}
Noch zu zeigen: $f_\alpha$ stetig\\
Sei $V=V(I) \subseteq \Spec R$\\
$\Rightarrow f_\alpha^{-1}(V) = \{\p \in \Spec R': \alpha^{-1}(\p) \supseteq I\} = \{ \p: \p \supseteq \alpha(I)\} = V(\alpha(I))$
\end{bew}

\begin{Bem}
Sei $k$ algebraisch abgeschlossener K\"orper, $V\subseteq \A^n(k)$ affine Variet\"at.\\
Dann ist $m: \left\{ \begin{array}{ccc} V &\to& \Spec k[V]\\ x &\mapsto& m_x\end{array}\right.$ injektiv und stetig.
\end{Bem}

\begin{bew}
injektiv: $\checkmark$\\
$m$ stetig: Sei$V(I) \subseteq \Spec k[V]$ abgeschlossen.\\
$\Rightarrow m^{-1}(V(I)) = \{ x\in V: m_x \in V(I)\} = \{x: I \subseteq m_x\} = \{x: f(x) = 0$ f\"ur alle $f\in I\} = V(I)$
\end{bew}

\begin{bsp}
Seien $\p \subsetneq\q$. Dann ist $\q \in \overline{\{\p\}}$
\end{bsp}

\begin{DefBem}\begin{enumerate}[a)]
\item
  Ein Punkt $x\in X$ ($X$ topologischer Raum) hei\ss t \defterm{generisch}, wenn $\overline{\{x\}} = X$ ist.
\item
  Jede irreduzible Teilmenge von $\Spec R$ hat genau einen generischen Punkt.
\item
  Die irreduziblen Komponenten von $\Spec R$ entsprechen bijektiv den minimalen Primidealen in $R$.
\end{enumerate}\end{DefBem}

\begin{Bem}
F\"ur jedes $f\in R$ ist $D(f) = \Spec R\setminus V(f) = \{\p \in \Spec R : f \notin \p\}$ offen in $\Spec R$. Die $D(f), f\in R$ bilden eine Basis der Zariski-Topologie.
\end{Bem}

\begin{bew}
Sei $U\subseteq \Spec R$ offen, $V= \Spec R - U \Rightarrow \exists I \subseteq R$ Ideal mit $V=V(I)$\\
F\"ur $f\in I$ ist $f\in \p$ f\"ur jedes $\p \in V$, das hei\ss t $V(I) \subseteq V(f) \Rightarrow D(f) \subseteq U$
\end{bew}

\begin{Bem}
$\Spec R$ ist quasikompakt.
\end{Bem}

\begin{bew}
Sei $(U_i)_{i\in I}$ offene \"Uberdeckung von $\Spec R$.\\
\OE $U_i = D(f_i)$ f\"ur ein $f_i\in R$\\
Dann gilt : $\bigcup\limits_{i\in I}D(f_i) = \Spec R \Leftrightarrow\bigcap\limits_{i\in I}V(f_i) = \emptyset \Leftrightarrow$ Die $f_i$, $i \in I$, erzeugen $R$\\
$\Rightarrow \exists i_1,\ldots ,i_k$ mit $1=\Sum_{\nu=1}^k a_\nu f_{i_\nu}$ f\"ur gewisse $a_\nu \in R$\\
$\Rightarrow \bigcup\limits_{\nu = 1}^k D(f_{i_\nu}) = \Spec R$
\end{bew}

%2.5.
\begin{DefBem}
Sei $R$ ein Ring, $X = \Spec R$
\begin{enumerate}[a)]
\item
  F\"ur $f \in R$ sei $\calO_X(D(f)) := R_f$
\item
  Die Zuordnung $D(f) \mapsto R_f$ ist eine $\calB$-Garbe von Ringen auf $X$ f\"ur die Basis $\calB = \{D(f): f\in R\}$ der Zariski-Topologie auf $X$.
\item
  Es gibt eine eindeutig bestimmte Garbe $\calO_X$ von Ringen auf $X$ mit $\calO_X(D(f)) = R_f$ f\"ur jedes $f\in R$.\\
  $\calO_X$ hei\ss t \deftermspec{Strukturgarbe}{Garbe!Struktur-} auf $X$.
\item
  F\"ur beliebiges $U\subseteq X$ offen ist $\calO_X(U) = \{ s: U \to \bigsqcup\limits_{\p\in U}^{\cdot} R_\p \vert s(\p) \in R_\p$ f\"ur alle $\p \in U$; f\"ur jedes $\p \in U$ gibt es Umgebung $U_\p \subseteq U$ von $\p$ und $f,g \in R$ mit $g\notin \q$ f\"ur alle $\q \in U_\p$ sodass $s(\q) = \frac{f}{g}(\q)$ f\"ur alle $\q \in U_\p\}$\\
  \textcolor{gray}{$g\notin \q$ bedeutet $\q \in D(g)$; $\frac{f}{g}(\q) := \Bild$ von $\frac{f}{g}$ in $R_\q$}
\item
  $\calO_{X,\p} \cong R_\p$ f\"ur jedes $\p \in X$.
\end{enumerate}\end{DefBem}

\begin{bew}\begin{enumerate}[a)]\item[b)]
Seien $f,g \in R$ mit $D(f) \subseteq D(g)$.\\
$\Rightarrow V(g) \subseteq V(f) \Rightarrow f\in \bigcap\limits_{(g)\in \p} \p = \sqrt{(g)}$\\
$\Rightarrow \exists d \ge 1$ mit $f^d \in (g)$, das hei\ss t $\exists h \in R$ mit $f^d = g\cdot h$\\
$\Rightarrow $ erhalte Homomorphismus $\begin{array}{ccc} R_g &\to& R_f \\ \frac{a}{g^k} &\mapsto& \frac{a\cdot h^k}{f^{d\cdot k}}\end{array}$\\
\emph{Wohldefiniertheit:} $\frac{g}{1} \cdot \frac{h}{f^d} = 1$ in $R_f$, da $g\cdot h - f^d = 0$ in $R_f$.\\
\emph{Zeige:} $D(f) \mapsto R_f$ ist $\calB$-Garbe.\\
Sei also $f\in R$, $(D(f_i))_{i\in I}$ offene \"Uberdeckung von $D(f)$, $g_i \in R_{f_i}$ konsistente Familie (das hei\ss t $g_i = g_j$ in $\calO_X(D(f_i) \cap D(f_j)) = \calO(D(f_if_j)) = R_{f_if_j}$).\\
\emph{Zu zeigen:} $\exists! g \in R_f$ mit $g = g_i$ in $R_{f_i}$ f\"ur jedes $i$:\begin{description}[\setlabelstyle{\itshape}]
\item[$\bullet$]
  \OE $f = 1$, $I=\{1,\ldots ,n\}$ ($X$ ist quasikompakt)
\item[$\bullet$ Eindeutigkeit:]
  Ist $g=h$ in $R_{f_i}$, $i=1,\ldots ,n$, so ist $(g-h)\cdot f_i^d = 0$ f\"ur ein $d\ge 1$. Die $f_i^d$, $i=1,\ldots ,n$, erzeugen $R$ (!)
    \[(f_1,\ldots ,f_n)^{n\cdot d} \subseteq (f_1^d,\ldots ,f_n^d)\]
  $\Rightarrow g = h$
\item[$\bullet$ Existenz:]
  Schreibe $g_i = \frac{h_i}{f_i^N}$, $h_i \in R$, $N\ge 1$. Nach Voraussetzung ist $\overbrace{f_i^Nf_j^Ng_i}^{=f_j^Nh_i} = \overbrace{f_j^Nf_i^Ng_j}^{f_i^Nh_j}$ f\"ur ein (anderes) $N\ge 1$.\\
  $(f_1^N,\ldots ,f_n^N) =R \Rightarrow \exists b_i \in R$ mit $ 1 = \Sum_{i=1}^n b_if_i^N$\\
  Setze $g:= \Sum_{i=1}^n b_ih_i$\\
  Dann ist f\"ur $j=1,\ldots ,n$
    \[f_j^Ng = f_j \Sum_{i=1}^n b_ih_i = \Sum_{i=1}^n b_if_j^Nh_i = \underbrace{\Sum_{i=1}^n b_if_i^N}_{=1}h_j = h_j = f_j^Ng_j \text{ in } R_{f_j}\]
  $\Rightarrow g= g_j$ in $R_{f_j}$
\end{description}\end{enumerate}\end{bew}

\begin{Def}
Sei $R$ ein Ring, $X = \Spec R$, $\calO_X$ die Strukturgarbe.\\
Dann hei\ss t $(X,\calO_X)$ \deftermspec{affines Schema}{Schema!affines}.
\end{Def}

\begin{bspe}\begin{enumerate}[1)]
\item
  $R=k$ K\"orper $\Rightarrow X= \Spec k = \{(0)\}$, $\calO_X(X) = k$
\item
  $R= k[X], k$ K\"orper. Ist $\{(0)\}$ offen? Nein!\\
  $k = \Q$: $\p = (X^2+X+1)$ ist abgeschlossener Punkt
    \[\FakRaum{R_\p}{m_\p} \cong \FakRaum{\Q[X]}{(X^2+X+1)} = \Q(\zeta_3)\]
  $\Rightarrow \q = (X-a), a\in k$, $\FakRaum{R_\q}{m_\q} \cong \FakRaum{k[X]}{(X-a)} \cong k$
\end{enumerate}\end{bspe}

\begin{Bem}\label{2.11}
Sei $(X,\calO_X)$ affines Schema, $X= \Spec R$.\\
Dann ist f\"ur jedes $f\in R$ auch $(D(f), \calO_X(f))$ affines Schema. \emph{Genauer:} $(D(f), \calO_X(D(f))) = (\Spec R_f, \calO_{\Spec R_f})$
\end{Bem}

\begin{bew}
$D(f) = \{ \p \subset R \text{ Primideal}: f\notin R\}$\\
$\Spec R_f = \{\p \subset R \text{ Primideal}\}$
  \[\begin{array}{ccc} R &\to& R_f \\ a &\mapsto& \frac{a}{1}\end{array}\]
$\p \subsetneq \q \not\ni f$, $\p \cdot R_f = \q \cdot R_f$\\
Sei $x \in \q\setminus \p \overset{!}{\Rightarrow} \frac{x}{1} \notin \p \cdot R_f$\\
Sei $x= \frac{a}{f^d}, a \in \p, d\ge 1 \Rightarrow f^d \cdot x \in \p \lightning$\\
Sei $h = \frac{g}{f^d} \in R_f$. \emph{Zu zeigen:}
  \[ \underbrace{\calO_x|_{D(f)} (D(h))}_{\calO_X(D(f)\cap D(g) = R_{f\cdot g}} \cong \underbrace{\calO_{\Spec R_f}(D(h))}_{(R_f)_h = R_{f\cdot g}} \]
\end{bew}

\newpage
%-------------------------- Abschnitt 3 (Affine Schemata) --------------------------
%3. 5.

\section{(Allgemeine) Schemata}\label{par3}

\begin{Def}\begin{enumerate}[a)]
\item
  Ein \defterm{geringter Raum} ist ein topologischer Raum $X$ zusammen mit einer Garbe $\calO_X$ von Ringen.
\item
  Ein geringter Raum $(X, \calO_X)$ hei\ss t \deftermspec{lokal}{geringter Raum!lokal} geringter Raum, wenn f\"ur jedes $x\in X$ der Halm $\calO_{X,x}$ ein lokaler Ring ist.
\end{enumerate}\end{Def}

\begin{Bem}
Jedes affine Schema $(\Spec R, \calO_{\Spec R})$ ist ein lokal geringter Raum.
\end{Bem}

\begin{Def}\label{3.3}\begin{enumerate}[a)]
\item
  Ein \defterm{Morphismus} $f:(X,\calO_X) \to (Y, \calO_Y)$ ist eine stetige Abbildung $f:X \to Y$ zusammen mit einem Morphismus $f^\#: \calO_Y \to f_*\calO_X$ von Garben.
\item\label{3.3b}
  Ein Morphismus zwischen lokal geringten R\"aumen $(X, \calO_X)$ und $(Y, \calO_Y)$ ist ein Morphismus $f: (X, \calO_X) \to (Y, \calO_Y)$ wie in a) sodass f\"ur jedes $x\in X$ der auf den Halmen induzierte Homomophismus $f_x^\#: \calO_{Y,f(x)} \to \calO_{X,x}$ die Bedingung $f_x^\#(m_{f(x)}) \subseteq m_x$ erf\"ullt ($f_x^\#$ hei\ss t dann lokaloer Homomophismus).\\
  $(f_* \calO_X)_{f(x)} = \lim\limits_{f(x) \in U} \calO_X(\underbrace{f^{-1}(U)}_{x\in}) \to \calO_{X,x}$
\end{enumerate}\end{Def}

\begin{bsp}
Sei $R$ lokaler Ring, nullteilerfrei, $K= \Quot(R) \ne R$. Dann ist die Inklusion $R \hookrightarrow K$ \emph{nicht} lokal.
\end{bsp}

\begin{Prop}\label{3.4}
Die Kategorie der affinen Schemata mit Morphismen aus Definition \ref{3.3} \ref{3.3b}) ist (anti-)\"aquivalent zur Kategorie der Ringe.
\end{Prop}

\begin{bew}\begin{enumerate}[(i)]\item
Sie Zuordnung $R\to (\Spec R, \calO_{\Spec R})$ ist Funktor. Sei $\alpha: R\to S$ Ringhomomorphismus, $f_\alpha: \Spec S \to \Spec R, \p \mapsto \alpha^{-1}(\q)$. Nach Bemerkung \ref{2.4} ist $f_\alpha$ stetig und $f_\alpha^{-1}(D(g)) = D(\alpha(g))$.

Definiere $f_\alpha^\# : \calO_{\Spec R} \to (f_\alpha)_* \calO_{\Spec S}$ durch
  \[\begin{array}{rcccr} \underbrace{\calO_{\Spec R}(D(g))} &\to& (f_\alpha)_* \calO_{\Spec S} (D(g)) =& \calO_{\Spec S} (f_\alpha^{-1}(D(g)) =& \underbrace{\calO_{Spec S} (D(\alpha(g))}\\
    = R_g \ni \frac{a}{g^d} &\mapsto& \frac{\alpha(a)}{\alpha(g)^d} &\in& =S_{\alpha(g)}\end{array}\]
\emph{Noch zu zeigen:} $(f_\alpha^\#)_\q$ ist lokal f\"ur jedes $\q\in \Spec S$

Sei $\p = \alpha^{-1}(\q) \in \Spec R$, das hei\ss t $\p = f_\alpha(\q)$.\\
Das maximale Ideal $m_\p$ (beziehungsweise $m_\q$) in $\calO_{\Spec R, \p} (=R_\p)$ (beziehungsweise $\calO_{\Spec S, \q}$) ist $\p R_\p$ (beziehungsweise $\q R_\q$).\\
F\"ur $a = \frac{b}{f} \in m_\p$ ($b \in \p, f\notin \p$) ist $(f_\alpha^\#)_\q(a) = \frac{\alpha(b)}{\alpha(f)} \in m_\q$, da $b\in \q = \alpha^{-1}(\q)$, also $\alpha(b) \in \q$ und $f\notin \p = \alpha^{-1}(\q)$, also $\alpha(f) \notin \q$.
\end{enumerate}\end{bew}

\begin{bsp}[Fortsetzung des Beispiels]
$\alpha: R\hookrightarrow K$, $\dim R = 1$ (zum Beispiel diskreter Bewertungsring)
  \[\begin{array}{cccc} &\Spec K = \{(0)\} &\to& \Spec R = \{(0), m\}\\
    &(0) &\mapsto& (0)\\
    (f_\alpha^\#)_{(0)} :& R_{(0)} = K &\to& K
  \end{array}\]
\end{bsp}

\begin{bew}[Fortsetzung des Beweises von Proposition \ref{3.4}]\begin{enumerate}[i)]\item[(ii)]
Ist $(X,\calO_X)$ affines Schema, $X= \Spec R$, so ist $R = \calO_X(X)$.\\
Ein Morphismus $\Spec S \to \Spec R$ induziert Homomophismus
  \[ f^\#: \underbrace{\calO_{\Spec R}(\Spec R)}_{=R} \to \underbrace{f_* \calO_{\Spec S} (\Spec R)}_{\substack{= \calO_{\Spec S} (f^{-1}(\Spec R)) = S \\ f^{-1}(\Spec R) = \Spec S }} \]
\emph{Nachrechnen:} Die Funktoren in (i) und (ii) sind zueinander invers.
\end{enumerate}\end{bew}

\begin{Def}
Ein lokal geringter Raum $(X, \calO_X)$ hei\ss t \defterm{Schema}, wenn es eine offene \"Uberdeckung $(U_i)_{i\in I}$ von $X$ gibt und affine Schemata $(\Spec R_i, \calO_{\Spec R_i})$ f\"ur jedes $i\in I$, sodass
  \[(U_i, \calO_X|_{U_i}) \overset{\text{als lokal}}{\underset{\text{geringter Raum}}{\cong}} (\Spec R_i, \calO_{\Spec R_i})\]
\end{Def}

\begin{BemDef}
Sei $(X, \calO_X)$ Schema, $U\subseteq X$ offen.\\
Dann ist $(U, \calO_X|_{U})$ auch ein Schema. $(U, \calO_X|_{U})$ hei\ss t \deftermspec{offenes Unterschema}{Schema!Unter-!offenes} von $X$.
\end{BemDef}

\begin{bew}
Sei $X= \bigcup\limits_{i\in I} \Spec R_i$ eine offene \"Uberdeckung von $X$ durch affine Schemata.\\
$\Rightarrow U = \bigcup\limits_{i\in I} (\underbrace{U \cap \Spec R_i}_{\subset \Spec R_i \text{ offen}})$, wobei $U \cap \Spec R_i \subset \Spec R_i$ offen, also $= \bigcup\limits_{j \in J} D(f_{ij}), f_{ij} \in R_i$

$(D(f_{ij}), \calO_{\Spec R_i}|_{D(f_{ij})})$ ist affines Schema nach Bemerkung \ref{2.11}
\end{bew}

\begin{Prop}[Verkleben]
Seien $(X, \calO_X)$ und $(Y, \calO_Y)$ Schemata, $U\subseteq X$ und $V\subseteq Y$ offen und $\varphi: (U, \calO_X|_U) \to (V, \calO_Y|_V)$ Isomorphismus von Schemata (das hei\ss t von lokal geringten R\"aumen). Sei $Z= (X\cup Y)/_\sim$ der topologische Raum, der durch Verkleben von $X$ und $Y$ l\"angs $\varphi$ ntsteht.

Dann gibt es genau eine Garbe $\calO_Z$ auf $Z$ mit $\calO_Z|_X = \calO_X$ und $\calO_Z|_Y \cong \calO_Y$.
\end{Prop}

\begin{bew}
Die offenen Teilmengen von $X$ und von $Y$ bilden eine Basis der Topologie auf $Z$.
\end{bew}

\begin{bsp}\label{3bsp2}\textcolor{white}{\,}\\
$\begin{array}{l}X = \A^1, U = \A^1\setminus \{0\} \\ Y = \A^1, V = \A^1\setminus \{0\}\end{array} \varphi = \id \qquad$\begin{tikzpicture}[baseline = 0]
  \node (X) at (2,0.5) {$X$};
  \node (X) at (2,-0.5) {$Y$};
  \node (U) at (-1,0.75) {$\textcolor{color1}{U}$};
  \node (V) at (-1,-0.25) {$\textcolor{color1}{V}$};
  
  \draw[thick] (1.75,0.5) -- (-1,0.5);
  \draw[thick] (1.75,-0.5) -- (-1,-0.5);
  \draw[color1, fill] (0.375,0.5) circle(0.1);
  \draw[color1, fill] (0.375,-0.5) circle(0.1);
\end{tikzpicture}
\begin{tikzpicture}[baseline = 0]
  \node (Z) at (2,0) {$Z$};
    
  \draw[thick] (1.75,0) -- (0.45,0);
  \draw[thick] (0.3,0) -- (-1,0);
  \draw[fill] (0.375,0.25) circle(0.05);
  \draw[fill] (0.375,-0.25) circle(0.05);
  
\end{tikzpicture}
\end{bsp}

%9. 5.
\begin{bspe}\begin{enumerate}[1)]\item
Quasiprojektive Variet\"aten:

$V\subseteq \IP^n(k)$, $k$ K\"orper quasi-projektiver Variet\"aten. $V$ besitzt endliceh \"Uberdeckung durch affine Variet\"aten $V= \bigcup\limits_{i=1}^r X_i$.\\
$V$ ist \quot{Verklebung} dieser affinen Variet\"aten. Jedes $X_i$ bestimmt affines Schema $\Spec k[X_i]$. Verklebe die $\Spec k[X_i]$ zu Schema $(X,\calO_X)$. $X$ hat dieselben abgeschlossenen Punkte wie $V$ (falls $k$ algebraisch abgeschlossen).

\textbf{Beobachtung:} $(X, \calO_X)$ h\"angt (bis auf Isomorphie) nicht von der gew\"ahlten affinen \"Uberdeckung ab.
\end{enumerate}\end{bspe}

\begin{Prop}
Sei $k$ algebraisch abgeschlossener K\"orper. Dann gilt:
\begin{enumerate}[a)]
\item
	Die Zuordnung $V\mapsto \Spec k[V]$ ist ein volltreuer, auf Objekten injektiver Funktor $t$ von der Kategorie der affinen Variet\"aten$/k$ in die Kategorie der affinen Schemata.
\item
	$t$ setzt sich fort zu volltreuem, auf Objekten injektivem Funktor
		\[ \text{quasiprojektive Variet\"aten}/k \to \text{ Schemata} \]
	%In der Vorlesung wurde "volltreue, auf Objekten injektiv Funktor" durch "toll" abgekuerzt, ich habe es hier ausgeschrieben.
\end{enumerate}\end{Prop}

\begin{Bez}
$\A_k^n := \Spec k[X_1,\ldots ,X_n]$ (vergleiche $\A^n(k)$)
\end{Bez}

\begin{bspe}\begin{enumerate}[1)]\item[2)]
$X=Y=\A_k^1$, $U=V=D(T) = \A_k^1 \setminus \{(T)\}$, $\A_k^1 = \Spec k[T]$. Verklebe $X$ und $Y$ l\"angs $\id: U\to V$.\\
Erhalte Schema $Z$ mit offenen Einbettungen $i_X: X\to Z$, $i_Y: Y \to Z$ sodass $Z - \{ 0_X, 0_Y\}$ isomorph zu $U=V$ ist.
\begin{center}
$i_X\left((T)\right) =: 0_X$, $i_Y\left((T)\right) =: 0_Y$
\begin{tikzpicture}
  \node (Z) at (2.25,0) {$Z$};
  \node (W) at (-0.5,1) {$\textcolor{color1}{W}$};
  \node (KlammerRechts) at (1.5,0) {$\textcolor{color1}{)}$};
  \node (KlammerLinks) at (-1.5,0) {$\textcolor{color1}{(}$};
  \node (0X) at (0.5,0.6) {$0_X$};
  \node (0Y) at (0.5,-0.6) {$0_Y$};
    
  \draw[thick] (0.2,0) -- (2,0);
  \draw[thick] (-2,0) -- (-0.2,0);
  \draw[fill] (0,0.5) circle(0.05);
  \draw[fill] (0,-0.5) circle(0.05);
  
\end{tikzpicture}
\end{center}

\textbf{Es gilt:}\begin{enumerate}[(i)]
\item
	$Z$ ist irreduzibel.
\item
	Sei $W \subseteq Z$ offen, $0_X \in W$, $0_Y \in W$, $f\in \calO_Z(W)$. Dann ist $f(0_X) = f(0_Y)$.
\item
	Die Diagonale $\Delta = \{ (z_1,z_2) \in Z\times Z: z_1 = z_2\}$ ist nicht abgeschlossen.
\end{enumerate}
\textbf{Folgerung:} $Z$ ist nicht isomorph zu einem affinen Schema. Beweis in der \"Ubung.
\end{enumerate}\end{bspe}

\begin{DefBem}
Sei $S:= \bigoplus\limits_{d\ge 0} S_d$ graduierter Ring ($S_d \cdot S_e = S_{d+e}$)
\begin{enumerate}[a)]
\item
	$\Proj(S) := \{ \p \subset S: \p$ homogenes Primideal$, S_+ \nsubseteq \p\}$\\
	$(S_+:= \bigoplus\limits_{d> 0} S_d)$ hei\ss t \deftermspec{homogenes Spektrum}{Spektrum!homogenes} von $S$.
\item
	F\"ur ein homogenes Ideal $I\subseteq S$ sei $V(I): \{\p \in \Proj S, I \subseteq \p\}$. Die $V(I)$ bilden die abgeschlossenen Mengen einer Topologie auf $\Proj S$ (\defterm{Zariski Topologie}).
\item
	F\"ur homogenes $f \in S$ sei $D_+(f) := \Proj S - V(f)$. Die $D_+(f)$, $f\in S$ homogen, bilden Basis.
\item
	F\"ur $f \in S$ homogen sei
		\[ \calO_{\Proj S} (D_+(f)) := S_f^{\hom} = \left\{\frac{a}{f^d}: a \text{ homogen vom Grad } d\cdot \deg(f) \right\} \]
\item
	Es gibt genau eine Garbe $\calO_{\Proj S}$ von Ringen auf $\Proj S$ mit $\calO_{\Proj S}(D_+(f)) = S_f^{\hom}$.
\item
	F\"ur $\p \in \Proj S$ ist
		\[ \calO_{Proj S, \p} = S_\p^{\hom} = \left\{ \frac{a}{b}: a,b \text{ homogen}, \deg a = \deg b, b \notin \p \right\} \]
	\textcolor{gray}{(lokaler Ring mit maximalem Ideal $\p \cdot S_\p^{\hom} := \{ \frac{a}{b} \in S_\p^{\hom} : a \in \p \}$)}
\item
	$(\Proj S, \calO_{\Proj S})$ ist Schema.
\end{enumerate}\end{DefBem}

\begin{bew}\begin{enumerate}[a)]\item[g)]
$D_+(f), \underbrace{\calO_{\Proj S}(D_+(f))}_{\ni \p \mapsto \{ \frac{a}{f^d} \in S_f^{\hom}: a \in \p \}} \cong \Spec S_f^{\hom}$
\end{enumerate}\end{bew}

\begin{bsp}
$S= k[X_0,\ldots ,X_n]$

Dann: $\Proj S = t(\IP^n(k)) =: \IP_k^n$, denn $D_+(X_i) = \Spec k[\frac{X_0}{X_i},\ldots ,\frac{X_n}{X_i}]$
\end{bsp}
\newpage

%-------------------------- Abschnitt 4 (Abgeschlossene Unterschemata) --------------------------
%10. 5.

\section{Abgeschlossene Unterschemata}

\begin{BemDef}
Sei $R$ Ring, $I \subseteq R$ Ideal
\begin{enumerate}[a)]
\item
	Die Abbildung  $V(I) \to \Spec(\FakRaum{R}{I}), \p \mapsto \p \mod I$ ist ein Hom\"oomorphismus.
\item
	$(V(I), \calO_{\Spec (R/I)})$ hei\ss t \deftermspec{abgeschlossenes Unterschema}{Schema!Unter-!abgeschlossenes} von $\Spec R$.
\item
	Die abgeschlossenen Unterschemata von $\Spec R$ entsprechen bijektiv den Idealen in $R$.
\item
	F\"ur abgeschlossene Unterschemata $Z_i \in \Spec\FakRaum{R}{I_i}$ gilt: $Z_2$ ist abgeschlossenes Unterschema von $Z_1$ (\quot{$Z_2 \le Z_1$}) $\Leftrightarrow I_1 \subseteq I_2$.\\
	Es ist dann $Z_2 = V(I_2) \subseteq V(I_1) = Z_1$
\end{enumerate}\end{BemDef}

\begin{bsp}
$X = \A_k^1 - \Spec[X]$, $Z_1 = \Spec \FakRaum{k[X]}{(X^2)}$, $Z_2 = \FakRaum{k[X]}{(X^2-X)}$. Dann ist $Z_1\subseteq Z_2$ als topologische R\"aume aber nicht als abgeschlossene (Unter-) Schemata.
\end{bsp}

\begin{DefBem}\label{4.2}
Sei $I \subseteq R$ Ideal, $Z = \Spec \FakRaum{R}{I}$ das zugeh\"orige abgeschlossene Unterschema von $X = \Spec R$.
\begin{enumerate}[a)]
\item
	F\"ur $U\subseteq X$ offen sei $I(U):= I \cdot \calO_X(U)$, das Bild von $I$ unter Restriktion.\\
	$\mathcal I$ ist Garbe von Idealen auf $X$.
\item
	Sei $j: Z \to X$ die Inklusion. Dann ist $j_*\calO_Z \cong \FakRaum{\calO_X}{I}$
\end{enumerate}\end{DefBem}

\begin{bew}\begin{enumerate}[a)]\item[b)]
F\"ur $f \in R$ ist $j_*\calO_ZD(f) = \calO_Z j^{-1} D(f) = \calO_Z (D(f)n\Z) = \calO_Z (D(\overline f)) = (\FakRaum{R}{I}) \overline f = \FakRaum{R_f}{IR_f} = \FakRaum{\calO_X}{I(D(f))}$
\end{enumerate}\end{bew}

\begin{Folg}
In der Situation \ref{4.2} wird $j:Z \to X$ zum Schemamorphismus, wobei \begin{tikzpicture}[baseline=-20]\node (0) at (-1.5, 0) {$j^\#\calO_X$}; \node (1) at (1.5,0) {$j_*\calO_Z$}; \node (2) at (0, -1) {$\calO/\mathcal I$}; \draw[->] (0) -- (1);\draw[->] (0) -- (2);\draw[->] (2) -- node[right]{$\sim$ b)}(1);\end{tikzpicture} die Quotientenabbildung $\calO_X \to \FakRaum{\calO_X}{\mathcal I}$ ist.
\end{Folg}

\begin{Def}\label{4.4}
Sei $(X, \calO_X)$ ein Schema\begin{enumerate}[a)]
\item
	Eine Garbe $\calI$ (von abelschen Gruppen) auf $X$ hei\ss t \deftermspec{Idealgarbe}{Garbe!Ideal-}, wenn f\"ur jedes offene $U\subseteq X$ $\calI(X)$ ein Ideal in $\calO_X(U)$ ist und die Restriktionshomomorphismen $\calO_X(U)$-linear sind.
\item
	Ist $X = \Spec R$ affines Schema, so hei\ss t eine Idealgarbe $\calI$ auf $X$ \deftermspec{quasikoh"arent}{koh"arent!quasi-}, wenn es ein Ideal $I$ in R gibt mit $\calI(U) = \textcolor{red}{I\calO_X(U)}$ f\"ur jedes offene $U\subseteq X$.
\item
	Eine Idealgarbe $\calI$ auf $X$ hei\ss t \defterm{koh\"arent}, wenn f\"ur jedes offene affine Unterschema $U\textcolor{red}{\subset} X$ die Einschr\"ankung $\calI|_U$ quasikoh\"arent ist.
\end{enumerate}\end{Def}

\begin{Prop}
Eine Idealgarbe $\calI$ auf $X$ ist genau dann quasikoh\"arent, wenn es eine offene \"Uberdeckung $(U_i)_{i\in I}$ gibt durch affine Unterschemata $U_i$ gibt, sodass $\FakRaum{\calI}{U_i}$ quasikoh\"arent ist f\"ur jedes $I$. (Beweis \"Ubung)
\end{Prop}

\begin{DefBem}\begin{enumerate}[a)]
\item
	Sei $(X, \calO_X)$ ein Schema. Ein abgeschlossenes Unterschema von $X$ ist ein  Schema $(Y, \calO_Y)$, wobei $Y\subseteq X$ abgeschlossen und $\calO_Y = \FakRaum{\calO_X}{\calI}$ f\"ur eine quasikoh\"arente Untergarbe $\calI \subseteq \calO_X$.
\item
	Ist $(Y, \calO_Y)$ abgeschlossenes Unterschema, so gilt f\"ur jedes offene $U\subseteq X$: $U\cap Y$ ist das abgeschlossene Unterschema von $U$, das zu $\calI/U$ geh\"ort. Ist $U$ affin, so ist $\calI/U$ die von $\calI(U)$ induzierte Idealgarbe.
\end{enumerate}\end{DefBem}

\begin{DefBem}\begin{enumerate}[a)]
\item
	Sei $R$ ein Ring.\\
	$N_R := \sqrt{(0)} = \{x \in R | \exists n \ge 1: x^n = 0\}$ ist ein Ideal in $R$, das \deftermspec{Nilradikal}{Radikal!Nil-}.
\item
	Ein Ring $R$ hei\ss t \defterm{reduziert}, wenn $N_R = (0)$ ist.
\item
	Ist $X = \Spec R$, so hei\ss t $X_{\Red} := \Spec \FakRaum{R}{N_R}$ das zu $X$ assoziierte \deftermspec{reduzierte Schema}{Schema!reduziertes}.
\item
	$X_{\Red}$ ist abgeschlossenes Unterschema von $X$ und $X_{\Red} \hookrightarrow X$ ist Hom\"omorphismus.
\item
	Sei $X$ ein Schema, $\calN_X$ die durch $\calN_X(U) =$ Nilradikal in $\calO_X(U)$ definierte Idealgarbe.\\
	Dann gilt: $\calN_X$ ist quasikoh\"arent.
\item
	Das zu $\calN_X$ assoziierte abgeschlossene Unterschema von $X$ hei\ss t $\calX_{\red}$. $(X, \calO_X)$ hei\ss t \defterm{reduziert}, wenn $\calX_{\red} \cong X$ als Schema, das hei\ss t $\calN_X = 0$.
\end{enumerate}\end{DefBem}

\begin{bew}\begin{enumerate}[a)]\item[e)]
\emph{Zu zeigen:} F\"ur $f \in R$, $R$ Ring, gilt: $\calN_{(R_f)} = \calN_RR_f$.\begin{twosidedproof}
\proofsupseteq
	Sei $a \in \calN_R$, also $a^n = 0$ f\"ur ein $n \ge 1$. F\"ur $x\in R_f$ ist $ax \in \calN_{R_f}$
\proofsubseteq
	$x = \frac{a}{f^{\textcolor{red}{*}}} \in R_f, x^n = 0 \Rightarrow \frac{a^n}{f^{\textcolor{red}{*}n}} = 0 \Rightarrow a^n = 0$
\end{twosidedproof}\end{enumerate}\end{bew}

\newpage

%-------------------------- Abschnitt 5 (Faserprodukte) --------------------------
%16. 5.

\section{Faserprodukte}

\begin{DefBem}\label{5.1}
Seien $X, Y, S$ Mengen, $f: X\to Y$, $g: Y \to S$ Abbildungen.\begin{enumerate}[a)]
\item
	$X \X_S Y := \{(x,y) \in X \X Y: f(x) = g(y)\}$ hei\ss t \defterm{Faserprodukt}.
\item
	Es gilt: $X \X_S Y = \bigcup\limits_{s \in S} f^{-1}(s) \X g^{-1}(s)$
\item\label{5.1c}
	Das Faserprodukt erf\"ullt folgende UAE:\\
	F\"ur alle Mengen $Z$, Abbildungen $\varphi: Z \to X$, $\psi: Z \to Y$ mit $f \circ \varphi = g \circ \psi$ gibt es genau eine $h: Z \to X \X_S Y$ mit $\varphi = \pr_X \circ h$, $\psi = \pr_Y \circ h$.
	\begin{center}\begin{tikzpicture}
	\node (0) at (0,0) {\textcolor{color2}{$Z$}};
	\node (1) at (2,0.5) {\textcolor{color2}{$Y$}};
	\node (2) at (2,-0.5) {\textcolor{color2}{$X$}};
	\node (3) at (4,1.5) {$X \X_S Y$};
	\node (4) at (4,-1.5) {$S$};
	
	\draw[->, color2] (0) -- node[above] {$\psi$} (1);
	\draw[->, color2] (0) -- node[below] {$\varphi$} (2);
	
	\draw[->, dashed, color1] (0) to [out=90, in=135] node[above] {$\exists! h$} (3);
	\draw[->, color2] (0) to [out=270, in=225] (4);
	
	\draw[->] (3) -- node[above] {$\pr_Y$} (1);
	\draw[->] (3) to [out=315, in=0] node[right] {$\pr_X$} (2);
	\draw[->] (1) to [out=0, in=90] node[left] {$g$} (4);
	\end{tikzpicture}\end{center}
\item
	Das Faserprodukt ist der Limes des Diagramms
	\begin{center}\begin{tikzpicture}
	\node (0) at (-1.5,0) {$X$};
	\node (1) at (1.5,0) {$Y$};
	\node (2) at (0,-1) {$S$};
	
	\draw[->] (0) -- node[below] {$f$} (2);
	\draw[->] (1) -- node[below] {$g$} (2);
	\end{tikzpicture}\end{center}
\end{enumerate}\end{DefBem}

\begin{bew}\begin{enumerate}[a)]\item[c)]
Setze $h(z):= (\varphi(z), \psi(z))$
\end{enumerate}\end{bew}

\begin{bspe}\begin{enumerate}[1)]
\item
	$S= \{s\} \Rightarrow X \X_S Y = X \X Y$
\item
	$X\subseteq S, Y \subseteq S$, $f,g$ die Inklusionen $\Rightarrow X \X_S Y = X \cap Y$
\item
	$Y \subseteq S$, $g: Y \hookrightarrow S \Rightarrow X \X_S Y = f^{-1}(Y)$
\item
	$X=Y \Rightarrow X \X_S Y =$ Equalizer$(f,g)$
\end{enumerate}\end{bspe}

\begin{Def}
Seien $X, Y, S$ Schemata, $f: X \to S, g: Y \to S$ Morphismen.\\
Dann hei\ss t ein Schema $X \X_S Y$ zusammen mit Morphismen $\pr_X: X \X_S Y \to X$ und $\pr_Y: X \X_S Y \to Y$, sodass $f \circ \pr_X = g\circ \pr_Y$ ist, \defterm{Faserprodukt} von $X$ und $Y$ \"uber $S$, wenn die UAE aus \ref{5.1} \ref{5.1c}) erf\"ullt ist.
\end{Def}

\begin{DefBem}
Sei $S$ ein Schema.\begin{enumerate}[a)]
\item
	Ein \deftermspec{S-Schema}{Schema!$S$-} ist ein Schema $X$ zusammen mit einem Morphismus $f:X \to S$.
\item
	Die $S$-Schmata bilden eine Kategorie \underline{Sch/$S$}.
\item
	Das Faserprodukt $X \X_S Y$ ist das Produkt von $f: X \to S$ und $g: Y \to S$ in \underline{Sch/$S$}.
\end{enumerate}\end{DefBem}

\begin{bsp}
$S = \Spec$, $k$ K\"orper

Ein Morphismus $X \to \Spec k$ ist nach \"Ubung 3, Aufgabe 1 vollst\"andig bestimmt durch einen Ringhomomorphismus $k \to \calO_X(X)$. Dieser macht $\calO_X(X)$ zur $k$-Algebra und $\calO_X$ zu einer Garbe von $k$-Algebren. Insbesondere sind $k$-Variet\"aten \"uber den Funktor $t$ $k$-Schemata. Das Faserprodukt von $k$-Variet\"aten ist das Produkt der $k$-Variet\"aten (im Sinne von Algebraische Geometrie I)(siehe unten).
\end{bsp}

\begin{Satz}
Das Faserprodukt $X \X_S Y$ existiert f\"ur alle $S$-Schemata $f: X \to S$ und $g: Y \to S$. Es ist eindeutig bis auf eindeutigen Isomorphismus.
\end{Satz}

\begin{bew}\begin{enumerate}[(1)]
\item
	$X= \Spec A$, $Y =\Spec B$, $S =\Spec R$ affin. $f$ und $g$ machen $A$ und $B$ zu $R$-Algebren.
	
	\emph{Behauptung:} Das Tensorprodukt $A \otimes_R B$ erf\"ullt $\Spec (A \otimes_R B) = X \X_S Y$.
	
	\emph{Erinnerung:} Das Tensorprodukt $M \otimes_R N$ von $R$-Moduln $M,n$ \quot{linearisiert} die bilinieare Abbildung
	\begin{center}\begin{tikzpicture}
	\node (0) at (-1.5,0) {$M\X N$};
	\node (1) at (1.5,0) {$M\otimes_R N$};
	
	\node (2) at (-1.5,-0.5) {$(x,y)$};
	\node (3) at (1.5,-0.5) {$x\otimes y$};
	
	\node (4) at (0, -2) {$P$};
	
	\draw[->] (0) -- (1);
	\draw[|->] (2) -- (3);
	
	\draw[->] (2) -- node[left] {$\Phi$ bilinear}(4);
	\draw[->, dashed, color2] (3) -- node[right] {$\exists! \varphi$ linear} (4);
	\end{tikzpicture}\end{center}
	
	\begin{enumerate}[\textbullet]
	\item
		Sind $M=A$ und $N=B$ $R$-Algebren, so hat $A \otimes_R B$ eine Struktur als $R$-Algebra:
			\[(a_1 \otimes b_1) \cdot (a_2 \otimes b_2) := a_1a_2 \otimes b_1b_2\]
	\item
		$\begin{array}{l} \sigma_A: A \to A \otimes_R B, a \mapsto a \otimes 1 \\ \sigma_B: B \to A \otimes_R B, b \mapsto 1 \otimes b \end{array}$ sind $R$-Algebren-Homomophismen.
	\item
		$A \otimes_R B$ erf\"ullt die richtige UAE
		\begin{center}\begin{tikzpicture}
		\node (0) at (-4, 0) {$C$};
		\node (1) at (0, 0) {$A$};
		\node (2) at (4, 0) {$B$};
		
		\node (3) at (2, 1.5) {$A \otimes_R B$};
		\node (4) at (2, -1.5) {$R$};
		
		\draw[->, color1] (1) -- node[below] {$\alpha$} (0);
		\draw[->, color1] (2) to [in=15, out=165] node[above] {$\beta$} (0);
		
		\draw[->] (4) -- (1);
		\draw[->] (4) -- (2);
		\draw[->] (1) -- node[right] {$\sigma_A$} (3);
		\draw[->] (2) -- node[left] {$\sigma_B$}(3);
		
		\draw[->, color1] (4) to [in=315, out=180] (0);
		\draw[->, dashed, color2] (3) to [in=45, out=180] node[above] {$\exists! \gamma$} (0);
		\end{tikzpicture}\end{center}
		\quot{Beweis:} $\tilde \gamma: A \X B \to C$, $(a,b) \mapsto \alpha(a) \cdot \beta(b)$ ist bilinear, induziert also $\gamma: A \otimes B \to C$ linear. Nachrechnen: $\gamma$ Ringhomomorphismus, $\gamma$ eindeutig.
	\end{enumerate}
	\emph{Also:} $\Spec(A \otimes_R B)$ erf\"ullt die geforderte UAE f\"ur alle affinen Schemata $Z$.
	
	Ist $Z$ beliebiges Schema, so induzieren $\varphi: Z \to X$ und $\psi: Z \to Y$ $R$-Algebrahomomorphismen $\alpha: A \to \calO_Z(Z)$, $\beta: B \to \calO_Z(Z)$.\\
	$\alpha$ und $\beta$ induzieren $\gamma: A\otimes_R B \to \calO_Z(Z)$, also (\"Ubung 3, Aufgabe 1) Morphismus $h: Z \to \Spec (A \otimes_R B)$.
\item
	$X,Y,Z$ nicht notwendig affin.\\
	\"Uberdecke $S$ durch offene affine Schemata $S_i = \Spec R_i$ ($i\in I$). Sei $X_i:= f^{-1}(S_i)$, $Y_i:= g^{-1}(S_i)$ (offen in $X$ bewziehungsweise $Y$).\\
	\"Uberdecke $X_i$ durch offene affine Schemata $X_{ij} = \Spec A_{ij}$\\
	\"Uberdecke $Y_i$ durch offene affine Schemata $Y_{ij} = \Spec B_{ij}$\\
	Nach (1) existiert $X_{ij} \otimes_{S_i} Y_{ik}$ f\"ur alle $i,j,k$
	
	\emph{Behauptung 1:} Sei $T$ ein Schema, $V,W$ $T$-Schemata, $(V_l)_{l\in L}$ offene \"Uberdeckung von $V$. Existiert $V_l \X_T W$ f\"ur jedes $l$, so existiert $V \X_T W$.
	
	Wende Behauptung 1 an auf\begin{itemize}
	\item
		$T=S_i$, $V = X_i$, $W=Y_{ik}$, $V_l=X_{il}$ $\Rightarrow X_i \X_{S_i} Y_{ik}$ existiert $\forall\  i,k$
	\item
		$T=S_i$, $V = Y_i$, $W=X_{i}$, $V_l=Y_{il}$ $\Rightarrow X \X_{S_i} Y_{ik}$ existiert $\forall\  i$
	\end{itemize}

	\emph{Behauptung 2:} F\"ur jedes $i$ ist $X_i \X_{S_i} Y_i = X_i \X_{S} Y$
	
	Dann wende Behauptung an auf
		\[ T=S, V=X, W=Y, V_l = X_l \Rightarrow X \X_S Y \text{ existiert} \]
	
	\emph{Beweis 2:}
	\begin{center}\begin{tikzpicture}
		\node (0) at (-4, 0) {\textcolor{color2}{$Z$}};
		\node (1) at (0, 0) {$X_i$};
		\node (2) at (4, 0) {$Y_{\textcolor{color2}{i}}$};
		
		\node (3) at (2, 1.5) {$X_i \X_S Y_i$};
		\node (4) at (2, -1.5) {$S_{\textcolor{color2}{i}}$};
		
		\draw[->, color1] (0) -- node[below] {$\varphi$} (1);
		\draw[->, color1] (0) to [in=165, out=15] node[above] {$\psi$} (2);
		
		\draw[->] (1) -- node[left] {$f|_{X_i}$} (4);
		\draw[->] (2) -- node[right] {$g$}(4);
		\draw[->] (3) -- node[right] {$\pr_{X_i}$} (1);
		\draw[->] (3) -- node[left] {$\pr_{Y_i}$}(2);
		
		\draw[->, color1] (0) to [in=180, out=315] (4);
		\draw[->, color2] (0) to [in=180, out=45] node[above] {$h$} (3);
	\end{tikzpicture}\end{center}
		\[ \Psi(Z) \subseteq g^{-1} (\underbrace{f (\overbrace{\varphi(Z)}^{\subseteq X_i} )}_{\subseteq S_i} ) \subseteq Y_i \]
	\emph{Beweis 1:} Verklebe die $V_l \X_T W$ l\"angs $U_{lm} = \pr_l^{-1} (V_l \cap V_m) \subseteq V_l \X_T W$. Es gilt: $U_{lm} = (V_l \cap V_m) \X_T W$. Dann ist $U_{lm} \quot{=} U_{ml}$, lassen sich also verkleben zu Schema $V$. Zeige: $\tilde V = V \X_T W$
\end{enumerate}\end{bew}

% 23. 5.

\begin{Bem}\begin{enumerate}[a)]
\item
	$X \X_S S \cong X$ f\"ur jedes $S$-Schema
\item
	$(X \X_S T) \X_T Y \cong X \X_S Y$ f\"ur alle\ldots
\end{enumerate}\end{Bem}

\begin{bew}\begin{enumerate}[a)]
\item
	\emph{Zeige:} $X$ erf\"ullt die UAE von $X \X_S S$\\
	\begin{minipage}[c]{0.4\textwidth}\begin{tikzpicture}[baseline=0]
		\node (0) at (0, 1.5) {$X$};
		\node (1) at (0, -1.5) {$S$};
		\node (2) at (1.5, 0) {$S$};
		\node (3) at (-1.5, 0) {$X$};
		\node (4) at (-4, 0) {\textcolor{color1}{$Z$}};
		
		\draw[->] (0) --node[above]{$\id$} (3);
		\draw[->] (0) --node[above]{$f$} (2);
		\draw[->] (3) --node[below]{$f$} (1);
		\draw[->] (2) --node[below]{$\id$} (1);
		\draw[->, color1] (4) --node[below]{$\varphi$} (3);
		
		\draw[->, dashed, color1] (4) to [in=180, out=45] node[left]{$\varphi$} (0);
		\draw[->, color1] (4) to [in=150, out=30] node[below, near start]{$\psi$} (2);
	\end{tikzpicture}\end{minipage}
	\begin{minipage}[c]{0.55\textwidth}Es gilt $\id_S \circ \psi = f \circ \varphi$ im unteren Dreieck, also auch im Oberen.\end{minipage}
\item
	\emph{Zeige:} $X \X_S Y$ erf\"ullt die UAE von $(X \X_S T) \X_T Y$\\
	\emph{Erhalte:} $h:X \X_S Y \to X \X_S T$ aus \begin{tikzpicture}[baseline = 0]
		\node (0) at (0,1.5) {$X \X_S T$};
		\node (1) at (-1.5,0) {$X$};
		\node (2) at (1.5,0) {$T$};
		\node (3) at (0,-1.5) {$S$};
		\node (4) at (-4,0) {$X \X_S Y$};
		
		\draw[->] (0) -- node[above]{\mbox{\small{$\pr_X$}}} (1);
		\draw[->] (0) -- node[above]{\mbox{\small{$\pr_T$}}} (2);
		\draw[->] (1) -- node[left]{\mbox{\small $f$}} (3);
		\draw[->] (2) -- node[right]{\mbox{\small$f \circ g$}} (3);
		
		\draw[->, color1] (4) -- node[below]{\mbox{\small $\pr_X$}} (1);
		\draw[->, dashed, color1] (4) to [in=180, out=45] node[above]{\mbox{\small $h$}} (0);
		\draw[->, color1] (4) to [in=150, out=30] node[below,sloped, near start]{\mbox{\small $g \circ \pr_Y$}} (2);
	\end{tikzpicture}$\begin{array}{rccc} g: &Y &\to& T \\ f: &X &\to &S \\ h: &T &\to &S\end{array}$\\
	\emph{Betrachte:} 
	\begin{tikzpicture}[baseline=0]
		\node (0) at (0, 1.5) {$X \X_S Y$};
		\node (1) at (0, -1.5) {$T$};
		\node (2) at (2, 0) {$Y$};
		\node (3) at (-2, 0) {$X \X_S T$};
		\node (4) at (-5, 0) {\textcolor{color1}{$Z$}};
		
		\draw[->] (0) --node[above]{$h$} (3);
		\draw[->] (0) --node[above]{$\pr_Y$} (2);
		\draw[->] (3) --node[below]{$\pr_T$} (1);
		\draw[->] (2) --node[below]{$g$} (1);
		\draw[->, color1] (4) --node[below]{$\varphi$} (3);
		
		\draw[->, dashed, color1] (4) to [in=180, out=45] node[left]{$\exists! \Phi$} (0);
		\draw[->, color1] (4) to [in=160, out=30] node[below, near start]{$\psi$} (2);
	\end{tikzpicture}
	
	\emph{Es gilt:} $f \circ \pr_X = t \circ g \circ \pr_Y$
	\begin{center}\begin{tikzpicture}
		\node (0) at (0, 1.5) {$X \X_S Y$};
		\node (1) at (0, -1.5) {$S$};
		\node (2) at (2, 0) {$Y$};
		\node (3) at (-2, 0) {$X$};
		\node (4) at (-5, 0) {\textcolor{color1}{$Z$}};
		
		\draw[->] (0) --node[above]{\textcolor{color1}{$\pr_X$}} (3);
		\draw[->] (0) --node[above]{$\pr_Y$} (2);
		\draw[->] (3) --node[left]{$f$} (1);
		\draw[->] (2) --node[right]{$t \circ g$} (1);
		\draw[->, color1] (4) --node[below]{$\textcolor{black}{\pr_X} \circ \varphi$} (3);
		
		\draw[->, color1] (4) to [in=160, out=30] node[below, near start]{$\psi$} (2);
	\end{tikzpicture}\end{center}
	\emph{Zu zeigen:} \fcolorbox{color2}{white}{$f \circ \pr_X$} $\circ \varphi = t \circ g \circ \psi =$ \fcolorbox{color2}{white}{$t \circ \pr_T$} $\circ \varphi$\\
	\begin{center}$\Rightarrow$ \begin{tikzpicture}[baseline=0]
		\node (0) at (0, 1.5) {$X \X_S Y$};
		\node (1) at (0, -1.5) {$S$};
		\node (2) at (2, 0) {$Y$};
		\node (3) at (-2, 0) {$X$};
		\node (4) at (-5, 0) {\textcolor{color1}{$Z$}};
		
		\draw[->] (0) --node[above]{\textcolor{color1}{$\pr_X$}} (3);
		\draw[->] (0) --node[above]{$\pr_Y$} (2);
		\draw[->] (3) --node[left]{$f$} (1);
		\draw[->] (2) --node[right]{$t \circ g$} (1);
		\draw[->, color1] (4) --node[below]{$\textcolor{black}{\pr_X} \circ \varphi$} (3);
		
		\draw[->, dashed, color1] (4) to [in=180, out=45] node[left]{$\exists! \Phi$} (0);
		\draw[->, color1] (4) to [in=160, out=30] node[below, near start]{$\psi$} (2);
	\end{tikzpicture}\end{center}
	\emph{Wir wissen:} $\pr_T \circ \varphi = g \circ \psi$\\
	\emph{Zu zeigen:}(i) $h \circ \Phi = \varphi$\\
	\textcolor{white}{\emph{Zu zeigen:}}(ii) $\pr_Y \circ \Phi = \psi \checkmark$
	
	F\"ur (i) ist zu zeigen: (i$_1$) $\underbrace{\pr_X \circ h}_{\color{color1}\pr_X} \circ \Phi = \pr_X \circ \varphi \checkmark$\\
	\textcolor{white}{F\"ur (i) ist zu zeigen:} (i$_2$) $\underbrace{\underbrace{\pr_T \circ h}_{\color{color2}g \circ \pr_Y} \circ \Phi}_{\color{color2}g \circ \psi} = \pr_T \circ \varphi$
	
	Damit ist die Existenz von $\Phi$ gezeigt. Eindeutigkeit in der \"Ubung.
\end{enumerate}\end{bew}

\newpage

%-------------------------- Abschnitt 6 (Punkte) --------------------------

\section{Punkte}

\begin{DefBem}\label{6.1}
Sei $X$ ein Schema, $x \in X$
\begin{enumerate}[a)]
\item
	$\kappa(x) := \FakRaum{\calO_{X,x}}{m_x}$ hei\ss t \deftermspec{Restklassenk\"orper}{K\"orper!Restklassen-} von $X$ im Punkt $x$.
	
	\textbf{Beispiele}\begin{enumerate}[1)]
	\item
		$X = \Spec \Z$\\
		$x = p \Rightarrow \kappa(x) = \mathbb F_p$\\
		$x = (0) \Rightarrow \kappa(x) = \Q$
	\item
		$X = \A_k^1$\\
		$x = (X-a)$ ($a \in k$) $\Rightarrow \kappa(x) = k$\\
		$x = (0) \Rightarrow \kappa(x) = k(X) = \Quot(k[X])$
	\item
		$X = \A_k^2 = \Spec k[X,Y]$\\
		$x = (f)$, $f$ irreduzibel $\Rightarrow \kappa(x) = \Quot(k[V]) = k(V)$ ($V = V(f)$)
	\end{enumerate}
\item\label{6.1b}
	Sei $f: X \to S$ ein Morphismus, $s:= f(x)$. $f$ induziert Homomorphismus $\kappa(s) \to \kappa(x)$.
\item\label{6.1c}
	F\"ur einen K\"orper $k$ gibt es genau dann einen Morphismus $\iota: \Spec k \to X$ mit $\iota(0) = x$, wenn $\kappa(x)$ isomorph zu einem Teilk\"orper von $k$ ist.
\item
	In der Situation c) hei\ss t $x$ \deftermspec{k-wertiger Punkt}{Punkt!$k$-wertiger} von $x$.
\end{enumerate}\end{DefBem}

\begin{bew}\begin{enumerate}[a)]
\item[b)]
	$f$ induziert lokalen Homomorphismus $f_x^\#: \calO_{S, f(x)} \to \calO_{X,x}$, das hei\ss t $f_x^\#(m_s) \subseteq m_x \Rightarrow f_x^\#$ induziert $\kappa(s) \to \kappa(x)$.
\item[c)]
	Sei $U = \Spec R$ affine Umgebung von $x$.\\
	$\iota$ exisiert $\Leftrightarrow\exists \alpha: R \to k$ mit $\Kern(\alpha) = \p$, wobei $\p$ das zu $x$ geh\"orige Primideal in $R$ ist.
	
	Es ist $\calO_{X, x} \cong R_{\p}$, also $\kappa(x) = \FakRaum{R_{\p}}{\p \cdot R_{\p}}$\\
	\emph{Also:} $\iota$ existiert $\Leftrightarrow \exists  \alpha: \begin{array}{ccc}R &\to& k \\ \p &\mapsto& (0)\end{array}$, also $\overline \alpha: \kappa(x) \to k$
	\begin{twosidedproof}\proofreverse $\alpha: R \to R_\p \to \FakRaum{R_{\p}}{\p \cdot R_{\p}} \xrightarrow{\overline \alpha} k$
\end{twosidedproof}\end{enumerate}\end{bew}

\begin{Bem}\label{6.2}
Seien $X, Y$ $S$-Schemata.\\
Dann ist die Abbildung $\begin{array}{ccc} X \X_S Y &\to& \{(x,y) \in X \X Y: f(x)=g(y)\} \\ z &\mapsto& (\pr_X(z), \pr_Y(z))\end{array}$ surjektiv.
\begin{center}
\begin{tikzpicture}
	\node (0) at (0,1) {$X \X_S Y$};
	\node (1) at (-2,0) {$X$};
	\node (2) at (2,0) {$Y$};
	\node (3) at (0,-1) {$S$};
	
	\draw[->] (0) -- node[above]{$\pr_X$}(1);
	\draw[->] (0) -- node[above]{$\pr_Y$}(2);
	\draw[->] (1) -- node[below]{$f$}(3);
	\draw[->] (2) -- node[below]{$g$}(3);
\end{tikzpicture}
\end{center}
\end{Bem}

\begin{bew}
Abbildung wohldefiniert: $\checkmark$

Seien $x\in X, y \in Y$ mit $f(x) = g(y) =: s \in S$. Seien $\kappa: \kappa(s), \kappa(x), \kappa(y)$ die Restklassenk\"orper. \OE $\kappa \subseteq \kappa(x)$, $\kappa \subseteq \kappa(y)$. Sei $k$ ein K\"orper mit $\kappa(x) \subseteq k$, $\kappa(y) \subseteq k$ (zum Beispiel Komposition). Sei $Z:= \Spec k$.

Nach \ref{6.1} \ref{6.1c}) gibt es Morphismen $\varphi: Z \to X, \varphi(0) = x$, $\psi: Z \to Y, \psi(0) = y$. Es ist $f \circ \varphi = g \circ \psi \Rightarrow  \exists h: Z \to X \X Y$ mit $\pr_X \circ h = \varphi, \pr_Y \circ h = \psi$. Setze $z := h(0)$.
\end{bew}

%24. 5.

\begin{DefBem}
Sei $f:X \to Y$ Morphismus von Schemata, $y \in Y$\begin{enumerate}[a)]
\item
	$X_y = f^{-1}(y) = X \X_Y \Spec \kappa(y)$ hei\ss t \defterm{Faser} von $f$ \"uber $y$. Dabei ist $\iota: \Spec \kappa(y) \to Y$ der zu $y$ geh\"orige Morphismus aus \ref{6.1}.
\item
	$\pr_X: X_y \to X$ ist injektiv.
\item
	$\pr_X(X_y) \to \{ x\in X: f(x) = y \}$ ist bijektiv.
\item
	Ist $y$ abgeschlossen, so ist $X_y$ abgeschlossenes Unterschema.
\end{enumerate}\end{DefBem}

\begin{bew}\begin{enumerate}[a)]
\item[c)]
	Folgt aus b) und \ref{6.2}.
\item[d)]
	Folgt aus c).
\item[b)]
	Seien $x_1, x_2 \in X_y$ mit $\pr_X(x_1) = \pr(x_2) =: x \in X \Rightarrow f(x) = y$. Sei $Z = \Spec \kappa(x)$ und $\iota: Z \to X$ mit $\iota(0) = x$. Sei $\psi: Z \to \Spec \kappa(y)$ der von $f_*^\#$ induzierte Morphismus (\ref{6.1} \ref{6.1b})).\\
	Nach \ref{6.1} \ref{6.1b}) ist $\kappa(x) \subseteq \kappa(x_i)$, $i=1,2$.
	\begin{center}\begin{tikzpicture}
		\node (0) at (0, 1.5) {$X_Y$};
		\node (1) at (0, -1.5) {$Y$};
		\node (2) at (2, 0) {$\Spec \kappa(y)$};
		\node (3) at (-2, 0) {$X$};
		\node (4) at (-5, 0) {$Z$};
		
		\draw[->] (0) --node[above]{$\pr_X$} (3);
		\draw[->] (0) -- (2);
		\draw[->] (3) -- (1);
		\draw[->] (2) -- (1);
		\draw[->] (4) --node[below]{$\iota$} (3);
		
		\draw[->] (4) to [in=180, out=45] node[above]{$h_1$} node[below]{$h_2$} (0);
		\draw[->] (4) to [in=165, out=15] node[above]{$\psi$} (2);
	\end{tikzpicture}\end{center}
	$\xRightarrow{\ref{6.1} \ref{6.1c}} \exists$ Morphismen $h_i: Z \to X_y$ mit $h_i(0) = x_i$, $i = 1,2$\\
	Es gilt: $\pr_Z \circ h_i = \psi$, $i = 1,2$\\
	\textcolor{white}{Es gilt:} $\pr_X \circ h_i = \iota$ nach Definition von $h_i$
	
	$\xRightarrow{\text{Eindeutigkeit}} h_1 = h_2 \Rightarrow x_1 = x_2$
\end{enumerate}\end{bew}

\begin{bspe}\begin{enumerate}[1)]
\item
	$f: \underset{=\Spec k[X]}{\quot{\A_k^1}} \to \A_k^1, \quot{x \mapsto x^2}$; $f$ werde induziert von $\alpha: k[X] \to k[X], X \mapsto X^2$\\
	Sei $y=(X-a) \Rightarrow X_y = \A_k^1 \X_{\A_k^1} \Spec k = \Spec ( \underbrace{k[X] \otimes_{k[X]} k}_{\cong k[X]/\alpha(X-a)} )$\\
	$\FakRaum{k[X]}{\alpha(X-a)} = \FakRaum{k[X]}{(X^2-a)} = \begin{cases} k \oplus k & \text{falls } a \in (k^\X)^2 \\ \FakRaum{k[X]}{(X^2)} & \text{falls } a = 0\end{cases}$
\item
	$X = (x,y)$-Ebene $\cup \ (z,w)$-Ebene in $\A_k^4$\\
	$\textcolor{white}{X} = V(z,w) \cup V(x,y) = V(xz, yz, xw, yw)$
	
	$f: X \to \A_k^2, (x,y,z,w) \mapsto (x+z, y+w)$ wird induziert von $\alpha: k[s,t] \to k[X,Y,Z,W] \to k[V], s \mapsto X + Z, t \mapsto Y + W$.
	
	Sei $ y = \quot{(0,0)} = (s,t) \Rightarrow V_y = V \X_{\A_k^2} \Spec k = \Spec (k \otimes_{k[s,t]} k) \cong \FakRaum{k[V]}{\alpha(s,t)} = \FakRaum{k[V]}{(X+Z, Y+W)} = \FakRaum{k[X,Y,Z,W]}{(X+Z, Y+W, XZ, YZ, XW, YW)}$\\
	$ = \FakRaum{k[X,Y]}{(-X^2, -XY, -Y^2)} =: R$
	
	\emph{Beachte:} $\dim_k R = 3$
\end{enumerate}\end{bspe}

\begin{DefBem}
Sei $X$ ein Schema, $T$ ein weiteres Schema.
\begin{enumerate}[a)]
\item
	Ein $T$-\deftermspec{wertiger Punkt}{Punkt!$T$-wertiger} von $X$ ist ein Morphismus $T \to X$.
\item
	Der Funktor $h_X: \Sch \to \Sets$, $T \mapsto \Hom(T,X)$ hei\ss t \deftermspec{Punktfunktor}{Funktor!Punkt-} zu $X$. $h_X$ ist kontravarianter Funktor.
\item
	Die $H_X$ definieren Funktor $h: \Sch \to \Fun(\text{Sch}^{\op}, \text{Sets})$. Dieser Funktor ist Kovariant. \textcolor{gray}{($\Fun(\text{Sch}^{\op}, \text{Sets})$ ist die Kategorie der kontravarianten Funktoren von Schemata nach Mengen; $\op$ steht f\"ur \quot{opposite})}
\end{enumerate}\end{DefBem}

\begin{bspe}\begin{enumerate}[1)]
\item
	Sei $T = \Spec(\FakRaum{k[\epsilon]}{(\epsilon^2)})$ ($k$ ein K\"orper), $X = \A_k^2 = \Spec k[X,Y]$. Ein $T$-wertiger Punkt von $X$ ist ein Ringhomomorphismus $\alpha: k[X,Y] \to \FakRaum{k[\epsilon]}{(\epsilon^2)}$. Sei $\alpha$ surjektiv, $\alpha^{-1}((\epsilon)) = (X,Y)$.
	
	Also $\alpha(X) = a\epsilon$, $\alpha(Y) = b\epsilon$ ($a,b \in k$) $\Rightarrow \alpha(bX - aY) = 0$. $\alpha$ bestimmt also nicht nur einen Punkt $x$ von $X$, sondern auch eine \quot{Richtung} in $x$.
\item
	$T = \Spec R$, $R$ diskreter Bewertungsring.
	
	$T = \{t_0, t_1\}$, $t_0 \in \overline{\{t_1\}}$, $K:= \Quot R$, $X$ ein Schema, $\kappa(t_0)=k$, $\kappa(t_1) = K$
	\[\begin{array}{lcr}\Hom(T,X) &=& \{ (x_0,x_1,x_2): x_0, x_1 \in X, x_0 \ne x_1, x_0 \in \overline{\{x_1\}}, \iota: \kappa(x_1) \to K\\
		&&\text{ Homomorphismus mit } \iota(\calO_{\overline{\{x_1\}}, x_0}) \subseteq R \text{ und } \iota(m_{x_0}) \subseteq m \}\end{array}\]
\end{enumerate}\end{bspe}

\newpage

%-------------------------- Abschnitt 7 (EndEndlichkeitseigenschaften) --------------------------
%30. 5.

\section{Endlichkeitseigenschaften}

\begin{Def}
Sei $X$ ein Schema.
\begin{enumerate}[a)]
\item
	$X$ hei\ss t \deftermspec{lokal noethersch}{noethersch!lokal}, wenn es eine offene \"Uberdeckung $(U_i)_{i\in I}$ von $X$ durch affine Schemata $U_i = \Spec R_i$ gibt, sodass die $R_i$ noethersch sind.
\item
	$X$ hei\ss t \defterm{noethersch}, wenn es eine endliche \"Uberdeckung wie in a) gibt.
\end{enumerate}\end{Def}

\begin{bsp}
Quasiprojektive Variet\"aten sind noethersch.
\end{bsp}

\begin{Prop}\label{7.2}\begin{enumerate}[a)]
\item
	Ein affines Schema $X = \Spec R$ ist genau dann noethersch, wenn $R$ noethersch ist.
\item
	Ein Schema $X$ ist genau dann lokal noethersch, wenn f\"ur jedes offene affine Unterschema $U = \Spec R$ gilt: $R$ ist noethersch
\end{enumerate}\end{Prop}

\begin{bew}\begin{enumerate}[a)]
\item
	folgt aus b)
\item
	Sei $X = \bigcup\limits_{i \in I} U_i$, $U_i = \Spec R_i$ offen in $X$, $R_i$ noethersch. Sei $U = \Spec R$ offen in $X$.
	
	\textbf{Zu zeigen:} $R$ ist noethersch
	
	Es gilt: $U \cap U_i$ ist offen in $U_i$ f\"ur jedes $i$. $\Rightarrow U \cap U_i = \bigcup\limits_{j \in J_i} D(f_{ij})$ f\"ur geeignete $f_{ij} \in R_i$.\\
	$D(f_{ij}) = \Spec (R_i)_{f_{ij}}$, $R_{ij} := (R_i)_{f_{ij}}$ ist noethersch\\
	$D(f_{ij})$ ist auch offen in $U$.\\
	$\Rightarrow \exists g_{ijk} \in R$ mit $D(f_{ij}) = \bigcup\limits_k D(g_{ijk})$
	
	Sei $\varphi_{ij}: R \to R_{ij}$ der von $D(f_{ij}) \hookrightarrow U$ induzierte Ringhomomorphismus\\
	$\Rightarrow R_{g_{ijk}} \overset{(!)}{\cong} (R_{ij})_{\varphi_{ij}(g_{ijk})} \Rightarrow R_{g_{ijk}}$ ist noethersch
	
	Die $D(g_{ijk})$ \"uberdecken $U$.
	
	$U$ ist quasikompakt $\Rightarrow $ endlich viele der $g_{ijk}$ gen"ugen zum \"Uberdecken. Nenne sie $g_1, \ldots, g_r$. Sei nun $I_1 \subseteq I_2 \subseteq \ldots $ Kette von idealen in $R$.
	
	F\"ur $i=1,\ldots ,r$ sei $\varphi_i: R \to R_{g_i}$ der nat\"urliche Homomorphismus $\Rightarrow \varphi_i(I_1) \cdot R_{g_i} \subseteq \varphi_i(I_2) \cdot R_{g_i} \subseteq \ldots$ wird station"ar
	
	\textbf{Behauptung:} F\"ur jedes Ideal $I \subseteq R$ gilt:
		\[ I = \bigcap_{i=1}^r \varphi_i^{-1} (\varphi_i(I) \cdot R_{g_i}) \]
	
	\textbf{Beweis der Behauptung:}\begin{twosidedproof}
	\proofsubseteq $\checkmark$
	\proofsupseteq
		Sei $b \in \bigcup\limits_{i=1}^r \varphi_i^{-1} \left(\varphi_i(I)\cdot R_{g_i}\right)$
		
		F\"ur jedes $i = 1,\ldots ,r$ gibt es $a_i \in I$, $n\in \N$ mit $\varphi_i(b) = \frac{b}{1} = \frac{a_i}{g_i^n}$ in $R_{g_i}$.\\
		$\Rightarrow \exists m_i$ mit $g_i^{m_i} (g_i^{n_i}b - a_i) = 0$ in $R$\\
		$\Rightarrow g_i^{m_i + n_i} b = g_i^{m_i} a_i \in I$\\
		$\Rightarrow \exists M$ mit $g_i^M b \in I$ f\"ur $i = 1,\ldots ,r$
		
		Nach Voraussetzung ist $\left.\begin{array}{r} (g_1,\ldots ,g_r) = R \\ \Rightarrow (g_1^M,\ldots ,g_r^M) = R\end{array} \right\} \Rightarrow b \in I$
\end{twosidedproof}\end{enumerate}\end{bew}

\begin{DefProp}
Sei $f: X \to Y$ Morphismus von Schemata.
\begin{enumerate}[a)]
\item
	$f$ hei\ss t \deftermspec{lokal von endlichem Typ}{Typ!von endlichem!lokal}, wenn es eine offene affine \"Uberdeckung $(U_i = \Spec A_i)_{i\in I}$ von $Y$ gibt und f\"ur jedes $i \in I$ eine offene affine \"Uberdeckung $(U_{ij} = \Spec B_{ij})_{j\in J_i}$ von $f^{-1}(U_i) \subseteq X$, so dass $B_{ij}$ (durch den von $f$ induzierten Homomorphismus) endlich erzeugte $A_i$-Algebra ist $\forall\ i \in I, j \in J_i$.
\item
	$f$ hei\ss t \deftermspec{von endlichem Typ}{Typ!von endlichem}, wenn in a) jedes $f^{-1}(U_i)$ eine endliche \"Uberdeckung der gew"unschten Art hat.
\item
	Ist $f$ (lokal) von endlichem Typ, so gibt es f\"ur jedes offene affine $U = \Spec A \subseteq Y$ eine endliche offene affine \"Uberdeckung $U_i = \Spec B_i$ von $f^{-1}(U)$, so dass $B_i$ endlich erzeugte $A$-Algebra ist.
\end{enumerate}\end{DefProp}

\begin{bew}\begin{enumerate}[a)]\item[c)]
"Ahnlich \ref{7.2}
\end{enumerate}\end{bew}

\begin{Bspe}\begin{enumerate}[1)]
\item
	Jeder Morphismus von quasiprojektiven Variet\"aten/$k$ ist von endlichem Typ.
\item
	Insbesondere ist f\"ur jede quasiprojektive Variet\"at $V/k$ der \quot{Strukturmorphismus} $V \to \Spec k$ von endlichem Typ.
\item
	$\Spec \C \to \Spec \Q$ ist nicht lokal von endlichem Typ.
\end{enumerate}\end{Bspe}

\begin{Def}
Ein Morphismus $f: X \to Y$ von Schemata hei\ss t \defterm{endlich}, wenn es eine offene affine \"Uberdeckung $(U_i = \Spec A_i)_{i\in I}$ von $Y$ gibt, so dass f"ur jedes $i \in I$ $f^{-1}(U_i)$ affin ist (also $f^{-1}(U_i) = \Spec B_i$) und dabei $B_i$ als $A_i$-Modul endlich erzeugt ist.
\end{Def}

\begin{Bem}\label{7.6}
Ist $f: X \to Y$ endlich, so ist $f^{-1}(y)$ endlich f\"ur jedes $y \in Y$.
\end{Bem}

\begin{bew}
Sei $U = \Spec A$ affine Umgebung von $y$ $\Rightarrow f^{-1}(y) \subset f^{-1}(U) = \Spec B$

$B$ ist nach Voraussetzung endl. erzeugter $A$-Modul. Weiter ist $f^{-1}(y) = \Spec (B \otimes_A \kappa(y))$.

$B \otimes_A \kappa(y)$ ist endlich-dimensionaler $\kappa(y)$-Vektorraum $\Rightarrow B \otimes_A \kappa(y)$ hat nur endlich viele Primideale
\end{bew}

\newpage

%-------------------------- Abschnitt 7 (eigentliche Morphismen) --------------------------
%31. 5.

\section{Eigentliche Morphismen}

\begin{Def}
Sei $f: X \to X$ ein Morphismus von Schemata.
\begin{enumerate}[a)]
\item
	Der von $\id_X$ induzierte Morphismus $\Delta = \Delta_f: X \to X \X_S X$ hei\ss t \deftermspec{Diagonalmorphismus}{Morphismus!Diagonal-} (oder Diagonale\index{Diagonale}) zu $f$.
	\begin{center}\begin{tikzpicture}
	\node (0) at (0,1.5) {$X \X_S X$};
	\node (1) at (-2.5,0) {$X$};
	\node (2) at (2.5,0) {$X$};
	\node (3) at (0,-1.5) {$S$};
	\node (4) at (-3.5,1) {$X$};
	
	\draw[->, font=\scriptsize] (0) -- node[above]{$\pr_1$} (1);
	\draw[->, font=\scriptsize] (0) -- node[above]{$\pr_2$} (2);
	\draw[->, font=\scriptsize] (1) -- node[below]{$f$} (3);
	\draw[->, font=\scriptsize] (2) -- node[below]{$f$} (3);
	
	\draw[->, font=\scriptsize] (4) to[in=135, out=45] node[above, sloped]{$\Delta = \Delta_f$} (0);
	\draw[->, font=\scriptsize] (4) -- node[below]{$\id$} (1);
	\draw[->, font=\scriptsize] (4) to[in=165, out=15] node[below, near end]{$\id$} (2);
	\end{tikzpicture}
	
	Es ist $\pr_1(\Delta(X)) = \pr_2(\Delta(X))$\end{center}
\item
	$f$ hei\ss t \defterm{separiert} (oder auch $X$ hei\ss t separiert \"uber $S$), wenn $\Delta$ einen abgeschlossene Einbettung ist.
\end{enumerate}\end{Def}

\begin{Erinn}
Ein topologischer Raum $X$ ist genau dann hausdorffsch, wenn $\Delta = \{ (x,x) \in X \X X \}$ abgeschlossene Teilmenge von $X \X X$ ist
\end{Erinn}

\begin{bsp}
Sei $X = $ \begin{tikzpicture}[baseline=-4]
  \draw[thick] (-1.5,0) -- (-0.2,0);
  \draw[thick] (0.2,0) -- (1.5,0);
  \draw[fill, font=\scriptsize] (0,0.25) circle(0.05) node[right] {$0_1$};
  \draw[fill, font=\scriptsize] (0,-0.25) circle(0.05) node[right] {$0_1$};
\end{tikzpicture} $\A_k^1$ mit doppeltem Nullpunkt (\S \ref{par3} Beispiel 2), $F:X \to \Spec k$ der Strukturmorphismus.

$f$ ist nicht separiert, da $\Delta(X)$ nicht abgeschlossen is in $ X \X_k X$, denn $(0_1,0_2) \notin \Delta$ aber $\in \Delta$.
\end{bsp}

\begin{Bem}\label{8.3}
Jeder Morphismus affiner Schemata ist separiert.
\end{Bem}

\begin{bew}
Sei $f: X = \Spec B \to \Spec A = Y$ Morphismus, induziert von Ringhomomorphismus $\alpha: A \to B$. Dann ist $X \X_Y X = \Spec (B \otimes_A B)$.

$\Delta: X \to X \X_Y X$ wird induziert von $\mu: \begin{array}{ccc} B \otimes_A B &\to& B \\ b_1 \otimes b_2 &\mapsto& b_1 \cdot b_2\end{array}$\\
$\mu$ ist surjektiv, also ist $\Delta$ abgeschlossene Einbettung.
\end{bew}

\begin{Bem}\label{8.4}
Offene und abgeschlossene Einbettungen sind separiert.
\end{Bem}

\begin{bew}
Sei $i: U \hookrightarrow X$ offene abgeschlossene Einbettung.

$\Rightarrow U \X_X U \cong U$ und f\"ur $\Delta: U \to U \X_X U \cong U$ gilt $\Delta = \id_U$.
\end{bew}

\begin{Def}
Sei $f:X \to Y$ ein Morphismus von Schemata.
\begin{enumerate}[a)]
\item
	$f$ hei\ss t \deftermspec{universell abgeschlossen}{abgeschlossen!universell}, wenn f\"ur jeden Morphismus $g: Y' \to Y$ gilt: $f': X \X_Y Y' \to Y'$ ist abgeschlossen.
\item
	$f$ hei\ss t \defterm{eigentlich}, wenn es von endlichem Typ, separiert und universell abgeschlossen ist.
\end{enumerate}\end{Def}

\begin{bsp}
$\A_k^1 \to \Spec$ ist abgeschlossen, aber nicht universell abgeschlossen.

\emph{Denn:}
\begin{center}\begin{tikzpicture}
	\node (0) at (-2,0.75) {\textcolor{white}{$\A_k^1 \X_k \A_k^1$}};
	\node at (-2.5,0.75) {$\A_k^2 = \A_k^1 \X_k \A_k^1$};
	\node (1) at (2,0.75) {$\A_k^1$};
	\node (2) at (-2,-0.75) {$\A_k^1$};
	\node (3) at (2,-0.75) {$\Spec k$};

	\draw[->] (0) -- (1);
	\draw[->, font=\scriptsize] (0) -- node[right]{$\pr_1$} (2);
	\draw[->] (2) --(3);
	\draw[->] (1) --  (3);
\end{tikzpicture}\end{center}

$\pr_1$ ist nicht abgeschlossen.

$V = V(X^2 + Y^2 - 1) \subseteq \A_k^2$, $\pr_1(V) = ?$\\
$V = V(XY - 1) \Rightarrow \pr_1(V) = \A_k^1 - \{0\}$

$\Rightarrow \pr_1(V) = \A_k^1 - \{0\}$ ist nicht abgeschlossen (\OE $k$ algebraisch abgeschlossen)
\end{bsp}

\begin{Def}\begin{enumerate}[a)]
\item
	Ein nullteilerfreier Ring $R$ hei\ss t \defterm{Bewertungsring}, wenn f\"ur jedes $x \in K = \Quot R$ gilt: $x \in R$ oder $x^{-1} \in R$. $R$ ist lokaler Ring mit maximalem Ideal $m = \{ x \in R: x^{-1} \notin R\}$, $(x+y)^{-1} =$ "Ubung??
\item
	Sei $f: X \to Y$ ein Morphismus von Schemata, $R$ ein Bewertungsring, $K = \Quot R$, $U = \Spec K$, $ T = \Spec R$.
	
	Ein kommutatives Diagramm \begin{tikzpicture}[baseline=0]
		\node (0) at (-1,0.5) {$U$};
		\node (1) at (1,0.5) {$X$};
		\node (2) at (-1,-0.5) {$T$};
		\node (3) at (1,-0.5) {$Y$};
		
		\draw[->, font=\scriptsize] (0) -- node[above]{$h_0$} (1);
		\draw[->, font=\scriptsize] (0) -- (2);
		\draw[->, font=\scriptsize] (1) -- node[right]{$f$} (3);
		\draw[->, font=\scriptsize] (2) -- node[below]{$h_1$} (3);
	\end{tikzpicture} hei\ss t \defterm{Bewertungsdiagramm} f\"ur $f$.
\end{enumerate}\end{Def}

\begin{Satz}
Sei $f: X \to Y$ ein Morphismus von Schemata, $X$ noethersch, $f$ von endlichem Typ f\"ur \quot{eigentlich}. Dann gilt:
\begin{enumerate}[a)]
\item
	$f$ ist genau dann $\left\{\begin{array}{c}\text{separiert}\\ \text{eigentlich}\end{array}\right\}$, wenn es zu jedem Bewertungsdiagramm
	\begin{center}\begin{tikzpicture}[baseline=0]
		\node (0) at (-3,1) {$\Spec K = U$};
		\node (1) at (3,1) {$X$};
		\node (2) at (-3,-1) {$\Spec R = T$};
		\node (3) at (3,-1) {$Y$};
		
		\draw[->, font=\scriptsize] (0) -- node[above]{$h_2$} (1);
		\draw[->, font=\scriptsize] (0) -- node[left]{$i$} (2);
		\draw[->, font=\scriptsize] (1) -- node[right]{$f$} (3);
		\draw[->, font=\scriptsize] (2) -- node[below]{$h_1$} (3);
		\draw[->, dashed, font=\scriptsize] (2) -- node[above]{$h$}(1);
	\end{tikzpicture} (*)\end{center}
	$\left\{\begin{array}{c}\text{h"ochstens}\\ \text{genau}\end{array}\right\}$ einen Morphismus $h: T \to X$ gibt sodass (*) kommutiert
	
	Dabei sei $R$ ein Bewertungsring und $K = \Quot(R)$.
\item
	Sind $X$ und $Y$ noethersch und $f$ von unendlichem Typ, so gen"ugt es, Bewertugnsdiagramme zu diskreten Bewertungsringen zu betrachten.
\end{enumerate}\end{Satz}

\begin{erinn}
$R$ Bewertungsring $:\Leftrightarrow R$ nullteilerfrei, f\"ur $x \in \Quot(R)^\X$ ist $x \in R$ oder $x^{-1} \in R$.

\emph{Bewertung:} $G$ abelsche Gruppe, $\le$ Totalordnung auf $G$, sodass aus $x \le y$ folgt: $x + a \le y + a \ \forall\ a \in G$, $v: k^\X \to G$ Homomophismus mit $v(a + b) \ge \min (v(a), v(b))$.
\end{erinn}

\begin{bsp}
Sei $X = \A_k^1$, $Y = \Spec k$, $f: X \to Y$\ldots\\
$K = k(T)$, $R = \{ \frac{g}{h}: g,h \in k[T], \deg h \ge \deg g\}$, $R$ diskreter Bewertungsring, $K = \Quot(R)$
\begin{center}\begin{tikzpicture}
	\node (0) at (-2,1) {$\Spec k(T)$};
	\node (1) at (2,1) {$\A_k^1$};
	\node (2) at (-2,-1) {$\Spec R$};
	\node (3) at (2,-1) {$\Spec k$};
	
	\draw[->, font=\scriptsize] (0) -- (1);
	\draw[->, font=\scriptsize] (0) -- (2);
	\draw[->, font=\scriptsize] (1) -- node[right]{$f$}(3);
	\draw[->, font=\scriptsize] (2) -- (3);
\end{tikzpicture}
\begin{tikzpicture}
	\node (0) at (-2,1) {$k(T)$};
	\node (1) at (2,1) {$k[T]$};
	\node (2) at (-2,-1) {$R$};
	\node (3) at (2,-1) {$k$};
	
	\draw[->, font=\scriptsize] (1) -- (0);
	\draw[->, font=\scriptsize] (2) -- (0);
	\draw[->, font=\scriptsize] (3) -- (1);
	\draw[->, font=\scriptsize] (3) -- (2);
	\draw[->, dashed, font=\scriptsize] (1) -- node[above]{$h?$}(2);
\end{tikzpicture}\end{center}
Es gibt kein $h$, da $k[T] \hookrightarrow k(T)$ \emph{nicht} \"uber $R$ faktorisiert: $T \notin R$
\end{bsp}

% 6. 6.

\begin{bem}

\end{bem}

\begin{bewskiz}\begin{enumerate}[I)]
\item
	\quot{separiert}
	\begin{twosidedproof}
	\proofforward
		Seien $h$, $h'$ Fortsetzungen von $h_0$.
		\begin{center}\begin{tikzpicture}[baseline=0]
			\node (0) at (-2,0.75) {$U$};
			\node (1) at (2,0.75) {$X$};
			\node (2) at (-2,-0.75) {$T$};
			\node (3) at (2,-0.75) {$Y$};
			
			\draw[->, font=\scriptsize] (0) -- (1);
			\draw[->, font=\scriptsize] (0) -- node[left]{$i$} (2);
			\draw[->, font=\scriptsize] (1) -- node[right]{$f$} (3);
			\draw[->, font=\scriptsize] (2) -- (3);
			\draw[transform canvas={yshift=0.3ex}, ->, font=\scriptsize] (2) -- node[above]{$h$}(1);
			\draw[transform canvas={yshift=-0.3ex}, ->, font=\scriptsize] (2) -- node[below]{$h'$}(1);
		\end{tikzpicture}\end{center}
		Sei $\tilde h: T \to X \X_Y X$ der von $h$ und $h'$ induzierte Morphismus.
		\begin{center}\begin{tikzpicture}[baseline=0]
			\node (0) at (-2,0.75) {$X \X_Y X$};
			\node (1) at (2,0.75) {$X$};
			\node (2) at (-2,-0.75) {$X$};
			\node (3) at (2,-0.75) {$Y$};
			\node (4) at (-4.5,2) {$T$};
			
			\draw[->, font=\scriptsize] (0) -- node[above]{$\pr_2$} (1);
			\draw[->, font=\scriptsize] (0) -- node[left]{$\pr_1$} (2);
			\draw[->, font=\scriptsize] (1) -- node[right]{$f$} (3);
			\draw[->, font=\scriptsize] (2) -- node[above]{$f$} (3);
			
			\draw[->, font=\scriptsize] (4) to[in=150,out=0] node[above]{$h'$} (1);
			\draw[->, font=\scriptsize] (4) to[in=180,out=270] node[below]{$h$} (2);
			\draw[->, font=\scriptsize, dashed] (4) -- node[below]{$\overline h$} (0);
		\end{tikzpicture}\end{center}
		Nach Voraussetzung ist $h(t_1) = h'(t_1) = h_0(t_1) =: x_1 \Rightarrow \tilde h(t_1) \in \Delta(X)$
		
		$\Delta(X)$ ist nach Voraussetzung abgeschlossen $\Rightarrow \tilde h(t_0) \in \overline{\{\tilde h(t_1)\}} \subseteq \Delta(X) \Rightarrow h(t_0) = h'(t_0) \Rightarrow h = h'$, weil $h^\#$ und $h'^\#$ durch $h_0$ festgelegt sind.
	\proofreverse
		Gen"ugt zu zeigen: $\Delta(X)$ ist abgeschlossen in $X \X_Y X$. Weil $X$ noethersch ist, k"onnen wir verwenden:
		\begin{Prop}\label{8.7}
		Sei $f: X \to Y$ ein quasikompakter Morphismus von Schemata.
		
		Dann gilt: $f(X)$ ist abgeschlossen in $Y$ $\Leftrightarrow$ f\"ur jedes $y_1 \in f(X)$ und jedes $y_0 \in \overline{\{y_1\}}$ ist $y_0 \in f(X)$ (\quot{abgeschlossen unter Spezialisierung})
		\end{Prop}
		\begin{bew}
		$\text{[H]}$ Hartshorne: Algebraic Geometry, Springer GT 52\\
		Chapter II, Lemma 4.5
		\textcolor{red}{Buchverweis}
		\end{bew}
		Sei also $x_1 \in \Delta(X)$, $x_0 \in \overline{\{x_1\}} \subseteq X \X_Y X$. Sei $Z := \overline{\{x_1\}}$ mit der reduzierten Struktur $\calO := \calO_{Z, x_0}$, $K = \calO_{Z, x_1} = \kappa(x_1) = \Quot \calO$.
		\begin{PropDef}\label{8.8}
		Sei $K$ ein K\"orper, $R \subset K$ ein lokaler Ring.\begin{enumerate}[a)]
		\item
			$(R_1,m_1)$ \deftermspec{dominiert}{dominieren} $(R_2,m_2)$, wenn $R_2 \subseteq R_1$ und $m_2 = m_1 \cap R_2$.
		\item
			$R$ ist Bewertungsring $\Leftrightarrow R$ ist maximal bez\"uglich Dominanz
		\item
			$R$ wird dominiert von einem Bewertungsring.
		\end{enumerate}\end{PropDef}
		\begin{bew}
		$\text{[AM]}$ Atiyah-McDonald: Intro to Cummutative Algebra\\
		Chapter 5, Theorem 5.11
		\textcolor{red}{Buchverweis}
		\end{bew}
		Sei also $R \subset K$ Bewertungsring, der $\calO$ dominiert.
		
		Nach Vor"uberlegung gibt es Morphismus $h: T = \Spec R \to X \X_Y X$ mit $h(t_1) = x_1$, $h(t_0) = x_0$.
		
		Sei $h_i := \pr_i \circ h$, $i=1,2$
		\begin{center}\begin{tikzpicture}
			\node (0) at (-3,1) {$X \X_Y X$};
			\node (1) at (3,1) {$X$};
			\node (2) at (-3,-1) {$X$};
			\node (3) at (3,-1) {$Y$};
			\node (4) at (-6,1) {$T$};
			\node (5) at (-4.5,2.5) {$\textcolor{color1}{U}$};
			
			\draw[->, font=\scriptsize] (0) -- node[below]{$\pr_2$} (1);
			\draw[->, font=\scriptsize] (0) -- node[left]{$\pr_1$} (2);
			\draw[->, font=\scriptsize] (1) -- node[right]{$f$} (3);
			\draw[->, font=\scriptsize] (2) -- node[below]{$f$}(3);
			\draw[->, font=\scriptsize] (4) -- node[below]{$h$}(0);
			\draw[->, font=\scriptsize, color1] (5) -- node[left]{$i$}(4);
			
			\draw[->, font=\scriptsize, color2] (4) to[in=180,out=270] node[below]{$h_1$} (2);
			\draw[->, font=\scriptsize] (4) to[in=165,out=25] node[below]{$h_2$} (1);
			\draw[->, font=\scriptsize, color1] (5) to[in=135,out=15] node[above]{$h_0$} (1);
		\end{tikzpicture}\end{center}
		$\Rightarrow f \circ h_1 = f\circ h_2$
		
		Da $x_1 \in \Delta(X)$ ist $h_1|_U = h_2|_U$, $U = \Spec K$.
		
		$\xRightarrow{\text{Vor.}} h_1 = h_2 \Rightarrow h$ faktorisiert \"uber $\Delta \Rightarrow x_0 \in \Delta(X)$.
		\begin{center}\begin{tikzpicture}
			\node (0) at (-3,1) {$X \X_Y X$};
			\node (1) at (3,1) {$X$};
			\node (2) at (-3,-1) {$X$};
			\node (3) at (3,-1) {$Y$};
			\node (4) at (-5,1) {$X$};
			\node (5) at (-6,2.5) {$T$};
			
			\draw[->, font=\scriptsize] (0) -- (1);
			\draw[->, font=\scriptsize] (0) -- (2);
			\draw[->, font=\scriptsize] (1) -- node[right]{$f$} (3);
			\draw[->, font=\scriptsize] (2) -- node[below]{$f$}(3);
			\draw[->, font=\scriptsize] (4) -- node[below]{$\Delta$}(0);
			\draw[->, font=\scriptsize] (5) -- node[left]{$h_1$}(4);
			\draw[->, font=\scriptsize] (5) -- node[left]{$h$}(0);
			
			\draw[->, font=\scriptsize] (4) -- node[below]{$\id$} (2);
			\draw[->, font=\scriptsize] (4) to[in=165,out=25] node[below]{$\id$} (1);
			\draw[->, font=\scriptsize] (5) to[in=140,out=10] node[above]{$h_1$} (1);
		\end{tikzpicture}\end{center}\end{twosidedproof}
		
\item \quot{eigentlich}
	\begin{twosidedproof}
	\proofforward
		Eindeutigkeit von $h$ folgt aus I
		
		\emph{Existenz von $h$}: Im Basiswechseldiagramm
			\begin{center}\begin{tikzpicture}
				\node (0) at (-1.5,0.75) {$X \X_Y T$};
				\node (1) at (1.5,0.75) {$X$};
				\node (2) at (-1.5,-0.75) {$T$};
				\node (3) at (1.5,-0.75) {$Y$};
				
				\draw[->] (0) -- (1);
				\draw[->] (2) -- node[below]{$h_1$}(3);
				\draw[->] (0) -- node[left]{$f'$}(2);
				\draw[->] (1) -- node[right]{$f$}(3);
			\end{tikzpicture}\end{center}
		ist $f'$ nach Voraussetzung abgeschlossen.
		
		Sei $\varphi: U \to X \X_Y T$ der von $h_0$ und $i$ induzierte Morphismus
			\begin{center}\begin{tikzpicture}
				\node (0) at (0,0.75) {$X \X_Y T$};
				\node (1) at (2.5,0.75) {$X$};
				\node (2) at (-1.25,-0.75) {$T$};
				\node (3) at (1.25,-0.75) {$Y$};
				\node (4) at (-2.5,0.75) {$U$};
				
				\draw[->] (0) -- (1);
				\draw[->] (2) -- node[below]{$h_1$}(3);
				\draw[->] (0) -- node[left]{$f'$}(2);
				\draw[->] (1) -- node[right]{$f$}(3);
				
				\draw[->,dashed] (4) -- node[above]{$\varphi$}(0);
				\draw[->] (4) -- node[left]{$i$}(2);
				\draw[->] (4) to[in=135,out=45] node[above]{$h_0$}(1);
			\end{tikzpicture}\end{center}
		Da $i = f' \circ \varphi$ ist und $i$ dominant, ist auch $f'$ dominant $\xRightarrow{f' \text{ abg.}} f'$ surjektiv
		
		Sei $z_1 = \varphi(t_1) \in X \X_Y T$, also $f'(z_1) = t_1$ (generischer Punkt), $Z:= \overline{\{z_1\}}$ mit reduzierter Struktur.
		
		Auch $f'|_Z$ ist surjektiv, also gibt es $z_0 \in Z$ mit $f'(z_0) = t_0$. $f'$ induziert lokalen Ringhomomorphismus $R = \calO_{T,t_0} \to \calO_{Z, z_0}$ und Einbettung $K = \kappa(t_1) \hookrightarrow \kappa(z_1)$. $\varphi$ induziert $\kappa(z_1) \hookrightarrow \kappa(t_1) = K$, also $\kappa(z_1) \cong K$.
		
		$\xRightarrow{\text{Prop. \ref{8.8}}} R \cong \calO_{Z, z_0} \xRightarrow{\text{\S 3 Bsp.2}} \exists\ h: t \to X$ mit $h(t_i) = \pr_X(z_i)$, $i= 0,1$
% 13. 6.
	\proofreverse
		Zu zeigen: Wenn es zu jedem Bewertungsdiagramm genau eine Fortsetzung $h$ von $h_1$ gibt, so ist $f$ eigentlich.
		
		Es gen"ugt zu zeigen: $f'$ ist universell abgeschlossen. Sei also Bewertungsdiagramm
			\begin{center}\begin{tikzpicture}
				\node (0) at (-2,0.75) {$X' = X \X_Y Y'$};
				\node (1) at (2,0.75) {$X$};
				\node (2) at (-2,-0.75) {$Y'$};
				\node (3) at (2,-0.75) {$Y$};
				
				\draw[->] (0) -- (1);
				\draw[->] (0) -- node[left]{$f'$} (2);
				\draw[->] (1) -- node[right]{$f$} (3);
				\draw[->] (2) -- (3);
			\end{tikzpicture}\end{center}
		\emph{Zu zeigen:} $f'$ ist abgeschlossen. Sei daf"ur $Z' \subseteq X'$ abgeschlossen, $y_1 = f'(z_1) \in f'(Z')$ und $y_0 \in \overline{\{y_1\}}$.
		
		\emph{Zu zeigen:} $y_0 \in f'(Z')$ (das gen"ugt nach Proposition \ref{8.7})
		
		Sei $\calO = \calO_{Z, y_0}$, wobei $Z = \overline{\{y_1\}}$ (mit reduzierter Struktur)
		
		$\Quot(\calO) = \kappa(y_1) \underset{(f')^\#}{\hookrightarrow} \kappa(z_1) =: K$
		
		$K|\kappa(y_1)$ ist endliche K\"orpererweiterung (da $f$ von endlichem Typ) $\xRightarrow{\text{Prop.\ref{8.8}}}$ Es gibt Bewertungsring $R$ von $K$, der $\calO$ dominiert $\Rightarrow$ Es gibt Morphismus $h_1: T = \Spec R \to Y'$ mit $h_1(t_i) = y_i$, $i=0,1$. Dann ist
			\begin{center}\begin{tikzpicture}
				\node (0) at (-2,1) {$X'$};
				\node (1) at (2,1) {$X$};
				\node (2) at (-2,-1) {$Y'$};
				\node (3) at (2,-1) {$Y$};
				\node (4) at (-6,1) {$\Spec K = U$};
				\node (5) at (-6,-1) {$T$};
				
				\draw[->] (0) -- (1);
				\draw[->] (0) -- node[left]{$f'$} (2);
				\draw[->] (1) -- node[right]{$f$} (3);
				\draw[->] (2) -- (3);
				\draw[->] (4) -- (0);
				\draw[->] (4) -- (5);
				\draw[->] (5) -- (2);
				
				\draw[->,dashed,color1] (5) -- node[above]{$h'$} (0);
				\draw[->,dashed,color2] (5) -- node[below]{$h$} (1);
			\end{tikzpicture}\end{center}
		ein Bewertungsdiagramm f"ur $f$.
		
		Nach Voraussetzung gibt es $h: T \to X$ mit \ldots
		
		Die UAE des Faserprodukts liefert $h': T \to X'$ mit $f'(h'(t_0)) = h_1(t_0) = y_0 \Rightarrow y_0 \in f'(Z')$.
		
		\textcolor{gray}{$h'(t_0) := z_0 \in \overline{\{h'(t_1)\}} = \overline{\{z_1\}} \in Z'$}
\end{twosidedproof}\end{enumerate}\end{bewskiz}

\begin{Folg}\label{8.9}
F\"ur Morphismen noetherscher Schemata gilt:
\begin{enumerate}[a)]
\item\label{8.9a}
	Die Komposition $\left\{\begin{array}{c}\text{separierter}\\ \text{eigentlicher}\end{array}\right\}$ Morphismen ist $\left\{\begin{array}{c}\text{separiert}\\ \text{eigentlich}\end{array}\right\}$.
\item\label{8.9c}
	$\left\{\begin{array}{c}\text{separiert}\\ \text{eigentlich}\end{array}\right\}$ ist stabil unter Basiswechsel.
\item\label{8.9c}
	Ist $g \circ f$ $\left\{\begin{array}{c}\text{separiert}\\ \text{eigentlich und } g \text{ separiert}\end{array}\right\}$, so ist $f$ $\left\{\begin{array}{c}\text{separiert}\\ \text{eigentlich}\end{array}\right\}$.
\end{enumerate}\end{Folg}

\begin{bew}
Bewertungskriterium anwenden
\end{bew}

\begin{Prop}\label{8.10}
Der Strukturmorphismus $\IP_\Z^n = \Proj \Z[X_0,\ldots ,X_n] \to \Spec \Z$ ist eigentlich.
\end{Prop}

\begin{Folg}
Sei $k$ ein K\"orper
\begin{enumerate}[a)]
\item
	$\IP_k^n$ ist eigentlich \"uber $\Spec k$.
\item
	Sind $V, V'$ projektive Variet\"aten$/k$, $f: V \to V'$ Morphismus, so ist der induzierte Morphismus $t(V) \to t(V')$ eigentlich.
\end{enumerate}\end{Folg}

\begin{bew}\begin{enumerate}[a)]\item[b)]
Sei $\underset{\subseteq \IP_k^n}{X:= t(V)}$, $\underset{\subseteq \IP_k^m}{Y:= t(V')}$ (abgeschlossene Unterschemata)
	\begin{center}\begin{tikzpicture}
		\node (0) at (-2.5,2) {$X$};
		\node (1) at (2.5,2) {$Y$};
		\node (2) at (-2.5,0) {$\IP_k^n$};
		\node (3) at (2.5,0) {$\IP_k^m$};
		\node (4) at (0,-1.5) {$\Spec k$};
		
		\draw[->] (0) -- node[above]{$f$} (1);
		\draw[->,color1] (0) -- (2);
		\draw[->,color1] (1) -- (3);
		\draw[->,color1] (2) -- node[below]{a)} (4);
		\draw[->,color1] (3) -- node[below]{a)} (4);
		\draw[->,color2] (0) -- node[right]{$g \circ f$} (4);
		\draw[->,color2] (1) -- node[left]{$g$} (4);
	\end{tikzpicture}\end{center}
Aus \ref{8.9} \ref{8.9a}) und \ref{8.9} \ref{8.9c}) folgt: $f$ ist eigentlich.
\end{enumerate}\end{bew}

\begin{bew}[von Proposition \ref{8.10}]
$\IP_k^n$ ist von endlichem Typ \"uber $\Spec \Z$ $\checkmark$

Sei \begin{tikzpicture}[baseline=0]
	\node (0) at (-1.5,0.75) {$U$};
	\node (1) at (1.5,0.75) {$\IP_\Z^n$};
	\node (2) at (-1.5,-0.75) {$T$};
	\node (3) at (1.5,-0.75) {$\Spec \Z$};
	
	\draw[->] (0) -- (1);
	\draw[->] (0) -- node[left]{$i$} (2);
	\draw[->] (1) -- (3);
	\draw[->] (2) -- (3);
	\draw[->,dashed] (2) -- node[above,sloped]{$\exists! h$} (1);
\end{tikzpicture} ein Bewertungsdiagramm.

\emph{Zu zeigen}: $\exists! h: T \to \IP_\Z^n$

Sei $\xi_1: h_0(t_1)$; \OE $\xi_1 \in \bigcap\limits_{i=0}^n U_i$ ($U_i = D(X_i)$) (sonst ist $\xi_1 \in \IP_\Z^{n-1}$ Induktion \"uber $n$) $\Rightarrow \frac{x_i}{x_j} \in \calO_{\IP_\Z^n, \xi_1}^\X$ f\"ur alle $i,j \Rightarrow$ Das Bild $\tilde f_{ij}$ von $\frac{x_i}{x_j}$ in $\underbrace{\kappa(\xi_1)}_{= \calO_{\IP_\Z^n,\xi_1}/m_{\xi_1}}$ ist $\ne 0$ $\Rightarrow f_{ij} := h_0^\#(\tilde f_{ij}) \in K^\X$

Sei $v: K^\X \to G$ die zu $R$ geh"orige Bewertung. W"ahle $j \in \{1,\ldots ,n\}$, sodass $v(f_{j0}) = \min\limits_{k=1}^n v(f_{k0})$ $\Rightarrow v(f_{ij}) = v(f_{i0}) - v(f_{j0}) \ge 0$ f\"ur $i= 0,\ldots ,n \Rightarrow f_{ij} \in R$ f\"ur $i= 0,\ldots ,n \Rightarrow \frac{X_i}{X_j} \mapsto f_{ij}$ definiert Ringhomomorphismus $\Z[\frac{X_0}{X_j},\ldots ,\frac{X_n}{X_j}] \to R$, also Morphismus $\textcolor{color2}{h}: T \to U_j \hookrightarrow \IP_\Z^n$

\emph{Eindeutigkeit von $h$}: Sei $\underset{\ne h}{h'}: T \to U_k$ eine weitere Fortsetzung von $h_0$.

Dann ist $k \ne j$, weil $U_j \to \Spec \Z$ separiert ist (\ref{8.3}). Sei $\beta: \Z[\frac{X_0}{X_k},\ldots ,\frac{X_n}{X_k}] \to R$ der zugeh"orige Ringhomomorphismus $\Rightarrow \beta(\frac{X_i}{X_k}) = h_0^\#(\frac{X_i}{X_k}) = f_{ik} \in R^{\times}$

Es ist $\underset{\in R^\X}{f_{ik}} = \underset{\in R^\X}{f_{ij}} \cdot \underset{\in R^\X}{f_{jk}} \Rightarrow \beta$ induziert denselben Morphismus $T \to \IP_\Z^n$ wie $\alpha$.
\end{bew}

\newpage

%-------------------------- Kapitel 2 --------------------------

\chapter{Garben und Divisoren}
\setcounter{section}{8}

%-------------------------- Abschnitt 9 ($\calO_X$-Modulgarben) --------------------------

\section{$\calO_X$-Modulgarben}

\begin{Def}
Sei $(X, \calO_X)$ ein lokal geringter Raum. Eine Garbe $\calF$ von abelschen Gruppen auf $X$ hei\ss t $\calO_X$-Modulgarbe, wenn gilt:
\begin{enumerate}[(i)]
\item
	$\calF(U)$ \quot{ist} $\calO_X(U)$-Modul f\"ur jedes offene $U \subseteq X$
\item
	$\rho_{U'}^U: \calF(U) \to \calF(U')$ ist $\calO_X(U)$-Modul-Homomophismus f\"ur $U' \subseteq U \subseteq X$ offen (via $\calO_X(U) \to \calO_X(U')$)
\end{enumerate}\end{Def}

\begin{bem}
Die $\calO_X$-Modulgarben auf $X$ bilden eine Kategorie. Gegenbeispiel: $\calO_X^\X$ ist \emph{keine} $\calO_X$-Modulgarbe.
\end{bem}

\begin{bspe}[f\"ur $\calO_X$-Modulgarben]\begin{enumerate}[1)]
\item
	Idealgarben
\item
	Sei $X$ eine nichtsingul"are Kurve (\"uber einem algebraisch abgeschlossenen K\"orper $k$), $D := \Sum_{\substack{P \in X \\ \text{(abg.)}}} n_P P$ ein Divisor.
	
	$\calO_X$ soll die Garbe der regul\"aren Differentiale auf $X$ sein.
	
	F\"ur $U \subseteq X$ offen sein $\calL(D)(U) := \{ f \in k(X): \ddiv f|_U + D|_U \ge 0\}$, das hei\ss t f\"ur alle $P \in U$ gilt $\ord_P f \ge -m_P$.
	
	$\calL(D)(U)$ ist $\calO_X(U)$-Modul: f\"ur $g \in \calO_X(U)$, $f \in \calL(D)(U)$ und $P \in U$ ist $\ord_P(fg) = \ord_P(f) + \underbrace{\ord_P(g)}_{\ge 0}$
	
	$\Rightarrow \calL(D)$ ist Modulgarbe
\item
	Sei $X$ weiterhin nichtsingul"are Kurve.
	
	\textbf{Erinnerung:} Sei $R$ ein Ring, $A$ eine $R$-Algebra, $M$ ein $A$-Modul, $\Der_R (A,M) = \{ \delta: A \to M, R$-liniear, $\delta(f \cdot g) = f \cdot \delta(g) + g \cdot \delta(f)\}$.
	
	Es gibt $\left\{\begin{array}{ll} \Omega_{A/R} & A\text{-Modul} \\ d: \begin{array}{rcl} A &\to& \Omega_{A/R} \\ a &\mapsto& da\end{array} & \text{Derivation} \end{array}\right\}$ sodass \begin{tikzpicture}[baseline=0]
		\node (A) at (-1.25,0.75) {$A$};
		\node (O) at (1.25,0.75) {$\Omega_{A/R}$};
		\node (M) at (0,-0.75) {$M$};
		
		\draw[->,font=\scriptsize] (A) -- (O);
		\draw[->,font=\scriptsize] (A) -- node[left]{$\delta$} (M);
		\draw[->,font=\scriptsize,dashed] (O) -- node[right,align=left]{$\exists! \varphi$ \\ $A$-linear} (M);
	\end{tikzpicture}
	
	$A = k[X]$\\
	$\Rightarrow \boxed{\Omega_{A/k} = A \cdot dX}$\\
	$\Omega_{k(X)/k} = k(X) \cdot dX$
	
	\emph{Es gilt}: Ist $X$ irreduzible Kurve, so ist $\Omega_{k(X)/k}$ 1-dimensionaler Vektorraum \"uber $k(X)$ (Beispiel $y^2 = x^3 + ax + b \Rightarrow 2ydy = 3x^2dx + adx$)
	
	F\"ur $\omega \in \Omega_{k(X)/k}$ und $P \in X$ sei $\ord_P \omega$ wie folgt definiert: sei $t_P$ ein Erzeuger von $m_P$ ($=$ maximales Ideal in $\calO_{X,P}$)
	
	$\Rightarrow \exists\ f_P \in k(X)$ mit $\omega = f_P dt_P$. Setze $\ord_P \omega := \ord_P f_P$
	
	F\"ur $\omega \in \Omega_{k(X)/k}$ sei $\ddiv \omega := \Sum_{P \in X} \ord_P \omega$ (wobei $d(t_P - c) = dt_P$).
	
	F\"ur $U \subseteq X$ offen: $\Omega_X(U) := \{ \omega \in \Omega_{k(X)/k}: \ddiv \omega|_U \ge 0\}$
	
	$\calO_X$ ist $\calO_X$-Modulgarbe, da $\ddiv(f \cdot \omega) = \ddiv f + \ddiv \omega$.
\end{enumerate}\end{bspe}

\begin{DefBem}
Sei $(X, \calO_X)$ lokal geringter Raum, $\calF, \calG$ $\calO_X$-Modulgarben.
\begin{enumerate}[a)]
\item
	$\calF \otimes_{\calO_X} \calG$ sei die zur Pr\"agarbe $U \mapsto \calF(U) \otimes_{\calO_X(U)} \calG(U)$ assoziierte Garbe, $\calF \otimes_{\calO_X} \calG$ ist eine $\calO_X$-Modulgarbe.
\item
	F\"ur $U \subseteq X$ offen sei
		\[ \calHom(\calF, \calG)(U) := \Hom_{\calO_X|_U} (\calF|_U, \calG|_U) \]
	Das ist eine $\calO_X$-Modulgarbe.
\end{enumerate}\end{DefBem}

\begin{bew}\begin{enumerate}[a)]\item[b)]
\ldots
\end{enumerate}\end{bew}

\begin{DefBem}\label{9.3}
Sei $f: X \to Y$ ein Morphismus lokal geringter R"aume
\begin{enumerate}[a)]
\item\label{9.3a}
	F\"ur jede $\calO_X$-Modulgarbe $\calF$ auf $X$ ist $f_* \calF$ eine $\calO_Y$-Modulgarbe.
\item
	F\"ur jede $\calO_Y$-Modulgarbe $\calG$ auf $Y$ ist $f^{-1} \calG$ eine $f^{-1} \calO_Y$-Modulgarbe und $f^* \calG := f^{-1} \calG \otimes_{f^{-1}\calO_Y} \calO_X$ eine $\calO_X$-Modulgarbe.
\end{enumerate}\end{DefBem}

\begin{bew}\begin{enumerate}[a)]
\item
	F\"ur $U \subseteq Y$ offen ist $f_*\calF(U) = \calF (f^{-1}(U))$ ein $\calO_X(f^{-1}(U))$-Modul, $f^\#: \calO_Y \to f_*\calO_X$ induziert $f_U^\#: \calO_Y(U) \to f_*\calO_X(U) = \calO_X(f^{-1}(U))$. Dadurch wird $\calF(f^{-1}(U))$ zu einem $\calO_Y(U)$-Modul.
\item
	Zu Definition von $f^*\calG$ wird Garbenhomomorphismus $f^{-1}\calO_Y \to \calO_X$ ben"otigt. $f^\#: \calO_Y \to f_*\calO_X$ induziert $f^{-1}\calO_Y \to f^{-1}f_*\calO_X \xrightarrow{\ref{1.14}} \calO_X$
\end{enumerate}\end{bew}

\begin{erinn}
$X = \Spec R$ affines Schema, $I \subseteq R$ Ideal, $f \in R$, $\tilde I (D(f)) = I \cdot R_f = I \cdot \calO_X(D(f))$, $\tilde I$ hei\ss t quasikoh"arente\index{koh"arent!quasi-} Idealgarbe\index{Garbe!Ideal-}.
\end{erinn}

\begin{DefBem}\begin{enumerate}[a)]
\item
	Sei $X = \Spec R$ affines Schema, $M$ ein $R$-Modul. Dann sei $\tilde M$ die (!) Garbe auf $X$ mit $\tilde M(D(f)) := M_f$ (der von $M$ erzeugte Modul \"uber $R_f$) f\"ur jedes $f \in R$.
	$\tilde M$ ist $\calO_X$-Modulgarbe, $(\tilde M)_\p = M_\p$ f\"ur jedes Primideal $\p \in \Spec R$.
%20.6
\item
	Sei $X$ ein Schema. Eine $\calO_X$-Modulgarbe $\calF$ hei\ss t \deftermspec{quasikoh"arent}{koh"arent!quasi-}, wenn f\"ur jede affine Teilmenge $U = \Spec R \subseteq X$ ein $R$-Modul $M_U$ existiert, sodass $\calF|_U \cong \tilde M_U$.
\item
	$\calF$ (wie in b)) ist genau dann quasikoh"arent, wennn es eine offene affine \"Uberdeckung $U_i = \Spec R_i$ von X gibt mit $\calF|_{U_i} \cong \tilde M_i$ f\"ur geeignete $R_i$-Moduln $M_i$.
\item
	Ist $X$ noethersch, so hei\ss t $\calF$ \defterm{koh"arent}, wenn in b) (beziehungsweise c)) alle $R$-Moduln $M_U$ endllich erzeugbar sind.
\end{enumerate}\end{DefBem}

\begin{bew}\begin{enumerate}[a)]\item[c)]
Wie "Ubung 4, Aufgabe 3
\end{enumerate}\end{bew}

\begin{Bem}
Sei $X = \Spec R$ ein affines Schema.

Die Zuordnung $M \mapsto \tilde M$ ist ein kovarianter auf Objekten injektiver Funktor von der Kategorie $R$-Mod in $\calO_X$-Mod; Umkehrfunktor: $\calF \to \calF(X)$.
\end{Bem}

\begin{bew}
Sei $0 \to M' \to M \to M'' \to 0$ exakt.

\emph{Zu zeigen}: $0 \to \tilde M' \to \tilde M \to \tilde M'' \to 0$ ist exakt.

\emph{Gen"ugt zu zeigen}: $\begin{array}{cccccccccl} 0 &\to& \tilde M_\p' &\to& \tilde M_\p &\to& \tilde M_\p'' &\to& 0 & \text{exakt f"ur jedes Prim-}\\
	&&\verteq&&\verteq&&\verteq&&&\text{ideal } \p \in \Spec R \\
	0 &\to& M_\p' &\to& M_\p &\to& M_\p'' &\to& 0 & \text{Das stimmt!}\end{array}$
\end{bew}

\begin{Bem}
Sei $X = \Spec R$ affines Schema.
\begin{enumerate}[a)]
\item
	F\"ur $R$-Moduln $M$ und $N$ gilt:
		\[\tilde M \otimes_{\calO_X} \tilde N \cong \widetilde{M \otimes_R N}\]
\item
	F\"ur $R$-Moduln $M_i$, $i \in I$ gilt:
		\[ \bigotimes_{i\in I} \tilde M_i \cong \widetilde{\bigotimes_{i \in I} M_i} \]
\end{enumerate}\end{Bem}

\begin{bew}\begin{enumerate}[a)]
\item
	$\widetilde{(M \otimes_R N)}_\p = (M \otimes_R N)_\p \cong M_\p \otimes_{R_\p} N_\p = \tilde M_\p \otimes_{R_\p} \tilde N_\p \cong (\tilde M \otimes_R \tilde N)_\p$
\item
	genauso
\end{enumerate}\end{bew}

\begin{Bem}\label{9.7}
Seien $X= \Spec R$, $Y = \Spec R'$ affine Schemata, $f: X \to Y$ Morphismen, $\alpha = f_X^\#: R' \to R$.
\begin{enumerate}[a)]
\item
	F\"ur jeden $R$-Modul gilt:
		\[ f_*\tilde M = _\alpha\tilde M (= \text{der via } \alpha \text{ als } R' \text{-Modul aufgefasste } R \text{-Modul } M) \]
\item\label{9.7b}
	F\"ur jeden $R'$-Modul $N$ gilt:
		\[ f^*\tilde N \cong \widetilde{(N \otimes_{R'} R)} \]
\end{enumerate}\end{Bem}

\begin{bew}\begin{enumerate}[a)]
\item
	F\"ur $U \subseteq Y$ offen ist $f_* \tilde M (U) = \tilde M (f^{-1}(U))$; das wird durch $f_U^\#: \calO_Y(U) \to \calO_X(f^{-1}(U))$ zum $\calO_Y(U)$-Modul (vergleiche \ref{9.3} \ref{9.3a})).
\item
	$f^* \tilde N (X) = f^{-1} \tilde N (X) \otimes_{f^{-1} \calO_Y(X)} \calO_X (X) = N \otimes_R R$; Entsprechendes gilt f\"ur jedes $U \subseteq X$ offen.
\end{enumerate}\end{bew}

\begin{Prop}
Sei $f: X \to Y$ Morphismus von Schemata.
\begin{enumerate}[a)]
\item
	Ist $\calG$ quasikoh"arent auf $Y$, so ist $f^*\calG$ quasikoh"arent auf $X$.
\item
	Sind $X$ und $Y$ noethersch und ist $\calG$ koh"arent, so ist $f^*\calG$ koh"arent.
\item
	Ist $X$ noethersch und $\calF$ quasikoh"arent auf $X$, so ist $f_*\calF$ quasikoh"arent auf $Y$.
\end{enumerate}\end{Prop}

\begin{bew}\begin{enumerate}[a)]
\item
	\OE $Y = \Spec R'$ affin, $\calG = \tilde N$ f\"ur einen $R'$-Modul $N$. \OE $X = \Spec R$ affin.
	
	Damit folgt die Behauptung aus \ref{9.7} \ref{9.7b}).
	
	\textcolor{gray}{$x_1,\ldots ,x_n$ seien Erzeuger von $N$ also $R'$-Modul $\Rightarrow x \in N$ hat Darstellung $\Sum_{i=1}^n a_i x_i$\\
	\emph{Behauptung}: $x_1\otimes 1,\ldots ,x_n \otimes 1$ erzeugen $N \otimes_{R'} R$ als $R$-Modul\\
	Sei $x \otimes a \in N \otimes_{R'} R, x = \sum a_i x_i$ ($a_i \in R'$) $\Rightarrow x \otimes a = a \cdot \sum a_i (x_i \otimes 1)$}
\item
	Ist $N$ endlich erzeugt als $R'$-Modul, so ist $N \otimes_{R'} R$ endlich erzeugt als $R'$-Modul.
\item
	\OE $Y$ affin. Sei $U_1,\ldots ,U_r$ offene affine \"Uberdeckung von $X$. \OE $U_i \cap U_j$ affin(!)? ("Ubung)
	
	\[\begin{array}{ccccccl} 0 &\to& \textcolor{color1}{f_*}\calF &\to& \bigoplus\limits_{i=1}^r \textcolor{color1}{f_*}\calF|_{U_i} &\to& \bigoplus\limits_{i < j} \textcolor{color1}{f_*}\calF|_{U_i \cap U_j} \\
		&& m &\mapsto& (m|_{u_i})_{i=1,\ldots ,r} && \\
		&&&& (m)_{i=1,\ldots ,r} &\mapsto& (m_i|_{u_i \cap u_j} - m_j|_{u_i \cap u_j})_{i<j}\end{array}\]
	ist exakt (weil $\calF$ Garbe ist, \textcolor{color1}{weil $f_*$ linksexakt ist})
	
	$\Rightarrow f_*\calF$ ist als Kern einer Morphismus von quasikoh"arenten Garben selbst quasikoh"arent.
\end{enumerate}\end{bew}

\newpage
%-------------------------- Abschnitt 9 ($\calO_X$-Modulgarben) --------------------------

\section{Lokal Freie Garben}

\begin{DefBem}
Sei $(X, \calO_X)$ lokal geringter Raum, $\calF$ eine $\calO_X$-Modulgarbe.
\begin{enumerate}[a)]
\item
	$\calF$ hei\ss t \defterm{frei} vom Rang $n \ge 1$, wenn $\calF \cong \calO_X^n = \bigoplus\limits_{i=1}^n \calO_X$.
\item
	$\calF$ hei\ss t \deftermspec{lokal frei}{frei!lokal} vom Rang $n$, wenn es eine offene \"Uberdeckung $(U_i)_{i\in I}$ von $X$ gibt, sodass $\calF|_{U_i} \cong (\calO_X|_{U_i})^n$ f\"ur jedes $i \in I$.
\item
	Ist $X$ Schema, so sind lokal freie Garben quasikoh"arent (und sogar koh"arent wenn $X$ noethersch ist)
\end{enumerate}\end{DefBem}

\begin{bew}\begin{enumerate}[a)]\item[c)]
Ist $U = \Spec R$, so ist $\calF|_U$ frei vom Rang $ n \Leftrightarrow\calF \cong \tilde R^n$.
\end{enumerate}\end{bew}

\begin{Bsp}
Sei $X$ nichtsingul"are Kurve (\"uber $k$\ldots) und $D$ ein Divisor auf $X$.

Dann ist $\calL(D)$ lokal frei vom Rang 1: "Ubung 9, Aufgabe 3b)

\textcolor{gray}{$\calL(D)(U) = \{ f \in k(X): \ddiv(f|_U) + D|_U \ge 0 \}$}
\end{Bsp}

%21. 6.
\begin{bsp}
Sei $X$ nichtsingul"are Kurve$/k$, $k$ algebraisch abgeschlossen, $\calL$ lokal freie Garbe vom Rang 1 auf $X$, $\calL \subseteq k(X)$ (konstante Garbe) $\Rightarrow \exists$ Divisor auf $X$ mit $\calL \cong \calL(D)$
\begin{description}[\setlabelstyle{\itshape}]\item[Denn:]
Sei $(U_i)_{i = 1,\ldots ,r}$ offene \"Uberdeckung von $X$ mit $\calL|_{U_i} \cong \calO_X|_{U_i}$, $i=1,\ldots ,r$

Nach Voraussetzung ist dann $\calL(U_i) = f_i \cdot \calO_X(U_i)$ f\"ur ein $f_i \in k(X)$. Setze $D|_{U_i} := \ddiv(\frac{1}{f_i})$

\emph{Behauptung:} Auf $U_i \cap U_j$ ist $\ddiv(\frac{1}{f_i}) = \ddiv(\frac{1}{f_j})$

"Aquivalent: $\frac{f_i}{f_j} \in \calO_X(U_i \cap U_j)^*$

\emph{denn:} $\calL(U_i \cap U_j) = f_i\calO_X(U_i \cap U_j) = f_i \calO_X(U_i \cap U_j)$
\end{description}\end{bsp}

\begin{Bsp}\label{10.3}
Sei $X$ eine Mannigfaltigkeit (reelle, differenzierbare, komplexe) und $\calO_X$ die Garbe der stetigen (differenzierbaren, holomorphen) Funktionen auf $X$. Sei $E$ eine weitere Mannigfaltigkeit, $p: E \to X$ eine stetige (differenzierbare, holomorphe) Abbildung.

$(E,p)$ hei\ss t \defterm{Vektorb\"undel} vom Rang $n$ \"uber $X$, wenn es eine offene "Uberdeckung $(U_i)$ von $X$ gibt und Isomorphismen $\varphi_i: p^{-1}(U_i) \to U_i \X \R^n$ sodass
	\begin{center}\begin{tikzpicture}
		\node (0) at (-2,1) {$p^{-1}(U_i)$};
		\node (1) at (2,1) {$U_i \X \R^n$};
		\node (2) at (0,-0.5) {$U_i$};
		\node at (0,0.35) {$\schraffiert$};
		
		\draw[->,font=\scriptsize] (0) -- node[above]{$\varphi_i$} (1);
		\draw[->,font=\scriptsize] (0) -- node[left]{$p$} (2);
		\draw[->,font=\scriptsize] (1) -- node[right]{$\pr_{U_i}$} (2);
	\end{tikzpicture}\end{center}
mit $p = \pr_{U_i} \circ \varphi_i$, sodass f"ur alle $i,j$ gilt: $\varphi_i \circ \varphi_j^{-1}: (U_i \cap U_j) \X \R^n \to (U_i \cap U_j) \X \R^n$ induziert f"ur jedes $x \in U_i \cap U_j$ eine lineare Abbildung $\R^n \to \R^n$, die stetig (differenzierbar, holomorph) von $x$ abh"angt, das hei\ss t $\varphi_i \circ \varphi_j^{-1} \in \gls{GL}_n (\calO_X(U_i \cap U_j))$.

Sei $\calE$ die Garbe der Schnitte in $E$, das hei\ss t $\calE(U) = \{ s: U \to E$ stetig (differenzierbar, holomorph) $: p \circ s = \id_U\}$. $\calE$ ist lokal frei vom Rang $n$.
\begin{description}[\setlabelstyle{\itshape}]
\item[Denn:]
	F"ur jedes $i \in I$ ist $\calE(U_i) = \{s: U_i \to \R^n$ stetig (differenzierbar, holomorph)$\} = (\calO_X(U_i))^n \Rightarrow \calE|_{U_i} \cong (\calO_X|_{U_i})^n$
\item[Umgekehrt:]
	Sei $\calE$ lokal frei vom Rang $n$ auf $X$. Sei $U_i \subset X$ offen, $\varphi_i: \calE|_{U_i} \to (\calO_X|_{U_i})^n$ Isomorphismus ($i \in I$, $(U_i)$ "Uberdeckung).
	
	F"ur $i,j \in I$ und $x \in U_i \cap U_j$ ist die induzierte Abbildung $(\varphi_i \circ \varphi_j^{-1})_x: \kappa_x^n \to \kappa_x^n$ ein Isomorphismus von $\kappa(x)$-Vektorraum ($\kappa(x) = \R$ beziehungsweise $\C$).
	
	$\Rightarrow (\varphi_i \circ \varphi_j^{-1})_x \in \GL_n(\overset{\C}{\R})$ und $\varphi_i \circ \varphi_j^{-1} \in \GL_n (\calO_X(U_i \cap U_j))$
	
	Sei $E_i := U_i \X \R^n$. Verklebe $E_i$ und $E_j$ "uber $(U_i \cap U_j) \X \R^n$ via $\varphi_i \circ \varphi_j^{-1}$. Erhalte $E$!
\end{description}\end{Bsp}

\begin{Def}
Sei $(X, \calO_X)$ ein Schema, $p: E \to X$ ein Morphismus von Schemata.

$(E, p)$ hei\ss t (geometrisches) \defterm{Vektorb\"undel} "uber $X$, wenn es eine offene \"Uberdeckung $(U_i)_{i \in I}$ von $X$ gibt und Isomorphismen $\varphi_i: p^{-1}(U_i) \to \A_\Z^n \X_{\Spec \Z} U_i$, sodass f\"ur alle $i,j$ und f\"ur alle offenen affinen Teilmengen $\underset{= \Spec R}{U} \subseteq U_i \cap U_j$ gilt:

$\varphi_i \circ \varphi_j^{-1}|_{\A_U^n}$ wird vone einem linearen Automorphismus $\alpha_{ij}$ von $R[X_1,\ldots ,X_n]$, das hei\ss t $\alpha_{ij}(a) = a$ f\"ur alle $a \in R$ und $\alpha_{ij}(x) = \sum_{j=1}^n a_{ij} X_j$ f\"ur gewisse $a_{ij} \in R$ ($\Leftrightarrow \alpha_{ij}$ ist $R$-Algebra-Homomorphismus), induziert.

\textcolor{gray}{\emph{Anmerkung}: $\A_U^n = \Spec R[X_1,\ldots ,X_n]$}
\end{Def}

\begin{Prop}
Sei $X$ ein Schema. Die Isomorphieklassen von lokal freien Garben vom Rang $n$ auf $X$ ensprechen bijektiv den Isomorphieklassen von Vektorb"undeln vom Rang $n$ \"uber $X$.
\end{Prop}

\begin{bew}
Analog Beispiel \ref{10.3}; "Ubung?
\end{bew}

\begin{DefProp}\label{10.6}
Sei $(X, \calO_X)$ lokal geringter Raum, $\calE$ lokal freie Garbe vom Rang $n$ auf $X$.
\begin{enumerate}[a)]
\item
	$\calE^* = \calHom_{\calO_X} (\calE, \calO_X)$ ist lokal freie Garbe vom Rang $n$, sie hei\ss t zu $\calE$ \deftermspec{duale}{dual} Garbe.
\item
	$(\calE^*)^* \cong \calE$
\item\label{10.6c}
	F\"ur jede $\calO_X$-Modulgarbe $\calF$ ist $\calHom_{\calO_X}(\calE, \calF) \cong \calE^* \otimes_{\calO_X} \calF$.
\item
	Ist $\calE'$ eine weitere lokal freie Garbe vom Rang $m$, so ist $\calE \otimes_{\calO_X} \calE'$ lokal frei vom Rang $n \cdot m$.
\item
	Ist $f: Y \to X$ Morphismus lokal geringter R"aume, so ist $f^* \calE$ lokal frei vom Rang $n$ auf $Y$.
\end{enumerate}\end{DefProp}

\begin{bew}\begin{enumerate}[a)]
\item
	Lokal ist $\calE \cong \calO_X^n$ und $\calHom (\calO_X^n, \calO_X) \cong \calO_X^n$.
\item
	Wie in Lineare Algebra I
\item
	F\"ur $U \subseteq X$ offen ist
	
	$\begin{array}{rcl} (\calE^* \otimes_{\calO_X} \calF)(U) = \Hom(\calE|_U, \calO_X|_U) \otimes_{\calO_X(U)} \calF(U) &\cong& \Hom(\calE|_U, \calF|_U) = \calHom_{\calO_X}(\calE, \calF) (U)\\
	l \otimes v &\mapsto& v \mapsto (x \mapsto l(x)v)\end{array}$
\item
	$\calO_X^n \otimes _{\calO_X} \calO_X^m \cong \calO_X^{n \cdot m}$
\item
	Ist $U \subseteq X$ offen mit $\calE|_U = (\calO_X|_U)^n$, so ist $f^*\calE (f^{-1}(U)) = (f^{-1} \calO_X|_U \otimes_{f^{-1}\calO_X|_U} \calO_Y) (f^{-1}(U)) = (\calO_Y(f^{-1}(U)))^n$
\end{enumerate}\end{bew}

% 27. 6.

\begin{BemDef}
Sei $(X, \calO_X)$ lokal geringter Raum.
\begin{enumerate}[a)]
\item
	F"ur jede lokal freie Garbe $\calL$ auf $X$ von Rang $1$ gilt:
		\[ \calL \otimes_{\calO_X} \calL^* \cong \calO_X \]
\item
	Die Isomorphieklassen von lokal freien Garben vom Rang $1$ auf $X$ bilden eine Gruppe $\Pic(X)$ (\deftermspec{Picard-Gruppe}{Gruppe!Picard-} von $X$).
\item
	Eine $\calO_X$-Modulgarbe $\calL$ auf $X$ hei\ss t \defterm{invertierbar}, wenn sie lokal frei vom Rang $1$ ist.
\end{enumerate}\end{BemDef}

\begin{bew}\begin{enumerate}[a)]\item
Nach \ref{10.6} \ref{10.6c}) ist $\calL \otimes \calL^* \cong \Hom_{\calO_X}(\calL, \calL)$. Sei $\varphi: \calO_X \to \Hom_X(\calL, \calL)$ gegeben durch $1 \mapsto \id$. $\varphi$ ist injektiver Morphismus von Garben.

$\varphi$ ist surjektiv, da $\varphi_X$ surjektiv ist f"ur jedes $x \in X$: sei dazu $\alpha \in \Hom_{\calO_X}(\calL_X, \calL_X)$ und $(U, \tilde \alpha)$ Vertreter von $\alpha$, sodass $\calL|_U = \calO_X|_U$. Dann ist $\tilde \alpha$ durch $\tilde \alpha(1) \in \calL(U) = \calO_X(U)$ bestimmt $\Rightarrow \tilde \alpha = \tilde \alpha(1) \cdot \id \in \Bild \varphi_U$.
\end{enumerate}\end{bew}

\newpage
%-------------------------- Abschnitt 10 (Divisore) --------------------------

\section{Divisoren}

Sei $X$ ein noethersches, $\underbrace{\text{irreduzibles, reduziertes}}_{=:\text{ integer (engl. integral)}}$ Schema.

\begin{DefBem}\begin{enumerate}[a)]
\item
	Ein \deftermspec{Primdivisor}{Divisor!Prim-} auf $X$ ist ein integres abgeschlossenes Unterschema der Kodimension $1$.
\item
	Die freie abelsche Gruppe $\Div(X)$, die von den Primdivisoren erzeugt wird, hei\ss t die gruppe der \deftermspec{Weil-Divisoren}{Divisor!Weil-}\index{Gruppe!Weil-Divisoren-}.
\item
	Ist $X$ eine Kurve "uber einem algebraisch abgeschlossenen K"orper, so sind die Primdivisoren die abgeschlossenen Punkte auf $X$.
\item
	Sei $W$ ein Primdivisor auf $X$, $\eta_W$ der generische Punkt von $W$, $\calO_{X,W} = \calO_{X,\eta_W}$ ($\dim \calO_{X,W} = 1$). F"ur $f \in \calO_{X,W} \setminus \{0\}$ sei $\ord_W(f) := \dim_{\kappa(W)} \FakRaum{\calO_{X,W}}{(f)}$. F"ur $f = \frac{g}{h} \in K = k(X) = \Quot (\calO_{X,W})$ sei $\ord_W(f) = \ord_W(g) - \ord_W(h)$ (wohldefiniert).
	
	Es gilt: $\ord_W(f \cdot g) = \ord_W(f) + \ord_W(g)$.
	
	\begin{center}\begin{tikzpicture}[gray, scale=0.5]
		\draw (0,2) -- (0,-2);
		\draw (-3,0) -- (3.5,0);
		
		\draw (3,1.5) to[in=0,out=225] (0,0);
		\draw (3,-1.5) to[in=0,out=135] (0,0);
		
		\node[font=\small] at (-2,1) {$Y^2-X^3$};
		\node[font=\scriptsize] at(-2,-2) {$\ord_{(0,0)}y = 3$};
	\end{tikzpicture}
	
	\textcolor{gray}{\small{$\FakRaum{\FakRaum{k[X,Y]}{(y^2-x^3)_{(x,y)}}}{(y)} = k\cdot \overline 1 + k \cdot \overline x + k \cdot \overline x^2 \leadsto \ord_{(0,0)}x = 2$}}
	\end{center}
\item
	F"ur $f \in k^\X$ sei $\ddiv(f) := \Sum_{W\text{Primdiv.}} \ord_W(f) \cdot W$; $\ddiv(f)$ ist Divisor.
\item
	$\Cl(X) := \FakRaum{\Div(X)}{\underset{\textcolor{gray}{\text{Hauptdivisoren}}}{\Div_H(X)}}$ hei\ss t \deftermspec{Weil-Divisorengruppe}{Gruppe!Weil-Divisoren-} von $X$.
\end{enumerate}\end{DefBem}

\begin{Bsp}
$X = \IP_\Z^n = \Proj \Z[X_0,\ldots ,X_n]$

Jeder Primdivisor $W$ auf $\IP^n$ ist von der Form $W = V(F)$ f"ur ein irreduzibles homogenes Polynom $F \in \Z[X_0,\ldots ,X_n]$. Sei $d := \deg F$.

\textbf{Behauptung:} $W - d \cdot H$ ist hauptdivisor f"ur $H := V(X_0)$

\emph{denn:} $\frac{F}{X_0^d} \in K = k(X)$, $\ddiv \frac{F}{X_0^d} = W - d \cdot H \Rightarrow \Cl(\IP^n) \cong \Z$ \textcolor{gray}{(als Gruppe)}
\end{Bsp}

\begin{Prop}\label{11.3}
Sei $R$ faktorieller Ring, $X = \Spec R$. Dann ist $\Cl(X) = 0$.
\end{Prop}

\begin{bew}
Sei $W$ Primdivisor, also $W = V(\p)$ f"ur ein Primideal $\p$ der H"ohe $1$.

\begin{description}[\setlabelstyle{\bfseries}]
	\item[Behauptung:]$\p$ ist Hauptideal
\end{description}

Dann ist $\p(f)$ f"ur ein Primideal $f \in R \Rightarrow \ddiv(f) = W$

\begin{description}[\setlabelstyle{\itshape}]
	\item[Beweis der Behauptung:]Sei $0 \ne f \in \p$, $f = f_1 \cdot \ldots \cdot f_r$ die Primfaktorzerlegung $\Rightarrow$ \OE $f_1 \in \p \Rightarrow (0) \subsetneq (f_1) \subseteq \p$ ist Primidealkette $\xRightarrow[\Ht(\p) = 1]{} \p = (f_1)$
\end{description}\end{bew}

\begin{bsp}
Sei $R = \FakRaum{k[X,Y]}{(Y^2-X^3+X)} = k[E]$, $E = V(Y^2-X^3+X) \subseteq \A_k^2$. Dann ist f"ur $P \in E$ abgeschlossen $1 \cdot P$ kein Hauptdivisor!
\end{bsp}

% 28. 6.

\begin{DefBem}
Sei $X$ ein integres Schema, $K = k(X)$, der Funktionenk"orper von $X$.
\begin{enumerate}[a)]
\item
	Ein \deftermspec{Cartier-Divisor}{Divisor!Cartier-} auf $X$ ist eine "Aquivalenzklasse von Familien $(U_i, f_i)_{i \in I}$, wobei $(U_i)_{i \in I}$ offene "Uberdeckung von $X$, $f_i \in K^\X$, sodass $\frac{f_i}{f_j} \in \calO_X(U_i \cap U_j)^\X$. Dabei sei $(U_i, f_i)_{i\in I} \sim (V_j, g_j)_{j \in J} :\Leftrightarrow \frac{f_i}{g_j} \in \calO_X(U_i \cap V_j)$.
\item
	Die Cartier-Divisoren auf $X$ bilden eine Gruppe $\CaDiv(X)$ mit folgender Verkn"upfung: Seien $D_1 = (U_i, f_i), D_2 = (V_j, g_j) \in \CaDiv(X)$, $(W_k)_k$ gemeinsame Verfeinerung, also $I = J$, $U_i = V_i$, $D_1 + D_2 := (U_i, f_i \cdot g_i)_{i \in I}$.
\item
	$\CaDiv(X) \cong \FakRaum{\calK_X^\X}{\calO_X^\X} (X)$, wobei $\calK_X^\X$ die konstante Garbe $K^\X$ sei.
\item
	$D \in \CaDiv(X)$ hei\ss t \deftermspec{Haptdivisor}{Divisor!Haupt-}, wenn es einen Vertreter von $D$ der Form $(X, f)$ gibt.
		\[ \CaCl(X) := \FakRaum{\CaDiv}{\CaDiv_H}(X) \]
\end{enumerate}\end{DefBem}

\begin{bew}\begin{enumerate}[a)]\item[c)]
Sei $D = (U_i, f_i)_i \in \CaDiv(X)$, $U_i$ \emph{klein}. F"ur $i \in I$ sei $\varphi_i \in \FakRaum{K^\X}{\calO_X}(U_i) = \FakRaum{\calK_X^\X}{\calO_X^\X}(U_i)$ die Restklasse von $f_i$. $\varphi_i$ ist durch $D$ eindeutig bestimmt.

Auf $U_i \cap U_j$ ist $\varphi_i = \varphi_j$ $\Rightarrow \exists \varphi \in \FakRaum{K_X^\X}{\calO_X^\X}(X)$ mit $\varphi|_{U_i} = \varphi_i$

Sei umgekehrt $\varphi \in \FakRaum{\calK_X^\X}{\calO_X^\X}(X)$. F"ur $x \in X$ sei $(U_x, \varphi^{(x)})$ Vertreter von $\varphi_x \in (\calK^\X, \calO^\X)_x$ \textcolor{gray}{($U_x$ Umgebung von $x \in K^\X$)}

$\Rightarrow (U_x, \varphi^{(x)})_{x \in X}$ ist Cartier-Divisor.
\end{enumerate}\end{bew}

\begin{bsp}
Sei $X = \IP_k^n$, $k$ K"orper, $U_i = D(X_i)$, $i = 0,\ldots ,n$, $f_i := \frac{X_0}{X_i}$, $f_0 := 1$.
\begin{description}[\setlabelstyle{\normalfont}]
\item[\textbf{Behauptung:}]
	$(U_i, f_i)$ ist Cartier-Divisor
\item[\itshape{Denn:}]
	F"ur $i \ne 0 \ne j$ ist $\frac{f_i}{f_j} = \frac{X_j}{X_i} \in \calO_X(U_i \cap U_j)^\X$.
	
	$\frac{f_i}{f_0} = \frac{X_0}{X_i} \in \calO_X(U_0 \cap U_i)$
	
	$\ddiv(f_i)|_{U_i} = V(X_0)$
	
	$(U_i, f_i)_{i=0,\ldots ,n}$ \quot{induziert} also den Weil-Divisor $V(X_0)$.
\end{description}\end{bsp}

\begin{Satz}
Sei $X$ noethersches intetegres separiertes Schema. Dann gilt:
\begin{enumerate}[a)]
\item
	$\CaCl(X) \cong \Pic(X)$
\item
	Es gibt nat"urlichen Homomophismus $\alpha: \CaCl(X) \to \Cl(X)$.
\item
	Ist $\calO_{X,x}$ faktoriell f"ur jedes $x \in X$, so ist $\alpha$ Isomorphismus.
\end{enumerate}\end{Satz}

\begin{bew}\begin{enumerate}[a)]
\item[b)]
	Sei $D = (U_i, f_i) \in \CaDiv(X)$, $W$ Primdivisor auf $X$. W"ahle $i$ mit $U_i \cap W \ne \emptyset$. Setze $n_W := \ord_W(f_i)$.
	\begin{description}[\setlabelstyle{\itshape}]
	\item[$n_W$ ist wohldefininiert:]
		Sei $j \in I$ mit $U_j \cap W \ne \emptyset \xRightarrow{W\text{ irred.}} U_i \cap U_j \cap W \ne \emptyset$, $\frac{f_i}{f_j} \in \calO_X(U_i \cap U_j)^\X \Rightarrow (f_i) = (f_j)$ in $\calO_{X,W} \Rightarrow \ord_W(f_i) = \ord_W(f_j)$
	\end{description}
	\OE $I$ endlich, da $X$ noethersch $\Rightarrow D = \Sum_{W \text{ Primid.}} n_W W$ ist Weil-Divisor
\item[c)]
	Umkehrabbildung zu $\alpha$
	\begin{description}[\setlabelstyle{\normalfont}]
		\item[\textbf{Behauptung:}]
			Zu jedem Weil-Divisor $D = \sum n_W W$ auf $X$ gibt es "Uberdeckung $(U_i)_i$ von $X$ und $f_i \in K^\X$ mit $D|_{U_i} = \ddiv (f_i)|_{U_i}$.
		\item[\itshape{Dann gilt:}]
			$\beta(D) := (U_i, f_i)$ ist Cartier-Divisor.
		\item[\itshape{Denn:}]
			$\ddiv(f_i)|_{U_i \cap U_j} = \ddiv(f_j)|_{U_i \cap U_j} = D|_{U_i \cap U_j} \Rightarrow \frac{f_i}{f_j} \in \calO_X(U_i \cap U_j)^\X$
		\item[\itshape{Beweis der Behauptung:}]
			\OE $X = \Spec R$. Sei $\xi \in X$ abgeschlossener Punkt.
			
			$\Rightarrow \calO_{X,\xi} = R_{m_\xi}$ (nach Voraussetzung faktoriell!)
			
			F"ur jeden Primdivisor $W = V(\p)$ von $X$ gilt: $\xi \in W \Leftrightarrow \p \subseteq m_\xi$
			
			Da $\Ht(\p) = 1$ ist nach \ref{11.3} $\p \cdot \calO_{X,\xi} = \underset{\in K^\X}{(f_w)}$ ein Hauptideal. Sei $f_\xi := \prod\limits_{\xi \in W} f_W^{n_W}$ und $U_\xi := X - \bigcup\limits_{\substack{W \text{ Primid.} \\ \xi \notin W \\ n_W \ne \ord_W f_\xi}} W$
	\end{description}
\item[a)]
	Sei $D = (U_i, f_i) \in \CaDiv(X)$, $\calL(D)(U_i) := \frac{1}{f_i} \cdot \calO_X(U_i) \subset K$ (als $\calO_X(U_i)$-Untermodul von $K$). Da $\frac{f_i}{f_j} \in \calO_X(U_i \cap U_j)^\X$ gilt $\frac{1}{f_i} \calO_X(U_i \cap U_j) = \frac{1}{f_j} \calO_X(U_i \cap U_j)$.
	\begin{description}[\setlabelstyle{\bfseries}]
	\item[Behauptung 1:]
		$\calL(D)$ ist lokal freie Garbe vom Rang $1$. $\calL$ induziert Homomophismus $\CaCl(X) \to \Pic X$
	\item[Behauptung 2:]
		Jede lokal freie $\calO_X$-Modulgarbe von $\calK^\X$ ist von der Form $\calL(D)$ f"ur ein $D \in \CaDiv(X)$.
	\item[Behauptung 3:]
		Jede lokal freie Garbe vom Rang $1$ auf $X$ ist isomorph zu einer Untergarbe von $\calK$.
	\item[Beweis 3:]
		$\calL \hookrightarrow \calL \otimes_{\calO_X} \calK \cong \calK$
	\end{description}
\end{enumerate}\end{bew}

\newpage

%-------------------------- Kapitel 3 --------------------------

% 4. 7.
\chapter{Homologie von Garben}
\setcounter{section}{11}

$0 \to \calF' \to \calF \to \calF'' \to 0$ kurze exakte Sequenz von Garben auf $X$, $U \subseteq X$ offen

$\Rightarrow 0 \to \calF'(U) \to \underset{\substack{\verteq \\ \Gamma(U,\calF)}}{\calF(U)} \to \calF''(U) \to \ldots $? exakt

%-------------------------- Abschnitt 12 (Abgeleitete Funktoren) --------------------------

\section{abgeleitete Funktoren}

\begin{DefBem}\begin{enumerate}[a)]
\item
	Eine Kategorie $\calC$ hei\ss t \defterm{abelsch}, wenn gilt:
	\begin{enumerate}[(i)]
	\item
		$\Hom_\calC(A,B)$ ist abelsche Gruppe f"ur alle $A, B \in \Ob(\calC)$
	\item
		F"ur Morphismen gelten die Distributivgesetze (bez"uglich $+$ und $\cdot$)
	\item
		Direkte Summen, Kerne, Kokerne existieren
	\item
		Der Homomorphiesatz gilt
	\end{enumerate}
\item
	$\Ab$, $\underline{k-\VR}$, $\underline{R-\Mod}$, \underline{Ab$(X)$}, $\underline{\calO_X-\Mod}$ sind abelsche Kategoriem
\item
	$\Grp$, $\Set$ sind nicht abelsch
\end{enumerate}\end{DefBem}

\begin{Def}
Sei $\calC$ abelsche Kategorie
\begin{enumerate}[a)]
\item
	Ein \defterm{Komplex} in $\calC$ ist eine Sequenz $C^0 := \ldots \to C^{i-1} \xrightarrow{d^{i-1}} C^i \xrightarrow{d^i} C^{i+i} \to \ldots$ mit Objekten $C^i$ in $\calC$, Morphismen $d^i \in \Hom_\calC(C^i, C^{i+1})$ sodass f"ur jedes $i \in \Z$ gilt: $\Bild(d^{i-1}) \subseteq \Kern(d^i)$
\item
	F"ur einen Komplex $C^0$ hei\ss t $H^i(C^0) := \FakRaum{\Kern(d^i)}{\Bild(d^{i-1})}$ \deftermspec{$i$-tes Kohomologieobjekt}{Kohomologieobjekt!$i$-tes}.
\item
	$d^i \circ d^{i-1} = 0 \ \forall \ i$
\end{enumerate}\end{Def}

\begin{Prop}
Sei $\calC$ eine abelsche Kategorie
\begin{enumerate}[a)]
\item
	Die Komplexe in $\calC$ bilden eine Kategorie $\calC^0$ mit Morphismen\ldots 
\item
	$H^i$ ist Funktor $\calC^0 \to \calC \checkmark$
\item
	Zu jeder kurzen exakten Sequenz $0 \to C'^0 \to C^0 \to C''^0 \to 0$ in $\calC^0$	gibt es eine lange exakte Kohomologiesequenz
		\[ 0 \to H^0(C'^0) \to H^0(C^0) \to H^0(C''^0) \xrightarrow{d} H^1(C'^0) \to H^1(C^0) \to \ldots \]
	\begin{center}\begin{tikzpicture}
		\node(11) at (0,0) {$0$};
		\node(12) at (2,0) {$0$};
		\node(13) at (4,0) {$0$};
		\node(14) at (6,0) {$0$};
		
		\node(21) at (0,-1) {$C'^{i-1}$};
		\node(22) at (2,-1) {$C'^{i}$};
		\node[shape=rectangle,rounded corners=1mm,gray,draw](23) at (4,-1) {$C'^{i+1}$};
		\node(24) at (6,-1) {$C'^{i+2}$};
		
		\node(31) at (0,-2) {$C^{i-1}$};
		\node(32) at (2,-2) {$C^{i}$};
		\node(33) at (4,-2) {$C^{i+1}$};
		\node(34) at (6,-2) {$C^{i+2}$};
		
		\node(41) at (0,-3) {$C''^{i-1}$};
		\node[shape=rectangle,rounded corners=1mm,gray,draw](42) at (2,-3) {$C''^{i}$};
		\node(43) at (4,-3) {$C''^{i+1}$};
		\node(44) at (6,-3) {$C''^{i+2}$};
		
		\node(51) at (0,-4) {$0$};
		\node(52) at (2,-4) {$0$};
		\node(53) at (4,-4) {$0$};
		\node(54) at (6,-4) {$0$};
		
		\node (dl) at (-2,-2) {\ldots}; \node (dr) at (8,-2) {\ldots};
		
		\draw[->] (11) -- (21); \draw[->] (21) -- (31); \draw[->] (31) -- (41); \draw[->] (41) -- (51);
		\draw[->] (12) -- (22); \draw[->] (22) -- (32); \draw[->] (32) -- (42); \draw[->] (42) -- (52);
		\draw[->] (13) -- (23); \draw[->] (23) -- (33); \draw[->] (33) -- (43); \draw[->] (43) -- (53);
		\draw[->] (14) -- (24); \draw[->] (24) -- (34); \draw[->] (34) -- (44); \draw[->] (44) -- (54);
		
		\draw[->] (21) -- (22); \draw[->] (22) -- (23); \draw[->] (23) -- (24);
		\draw[->] (dl) -- (31); \draw[->] (31) -- (32); \draw[->] (32) -- (33); \draw[->] (33) -- (34); \draw[->] (34) -- (dr);
		\draw[->] (41) -- (42); \draw[->] (42) -- (43); \draw[->] (43) -- (44);
		
		\draw[->, gray] ([xshift=-2mm]42.north) -- node[left, font=\scriptsize]{Urbild} ([xshift=-2mm]32.south);
		\draw[->, gray] ([yshift=2mm]32.east) -- ([yshift=2mm]33.west);
		\draw[->, gray] ([xshift=-2mm]33.north) -- node[left, font=\scriptsize]{Urbild} ([xshift=-2mm]23.south);
		\draw[->, gray, dashed] ([yshift=2mm]42.east) to[out=0,in=270] ([xshift=2mm]23.south);
	\end{tikzpicture}\end{center}
\end{enumerate}\end{Prop}

\textbf{Ziel:} Sei $X$ ein Schema, $\calF$ Garbe auf $X$. Suche Gruppen (beziehungsweise $\calO_X$-Moduln) $H^i(X, \calF)$, $i \ge 0$, sodass f"ur jede kurze exakte Sequenz $0 \to \calF' \to \calF \to \calF'' \to 0$ in \underline{Ab$(X)$} (beziehungsweise $\underline{\calO_X-\Mod}$) gilt:

$0 \to H^0(X, \calF') \to H^0(X, \calF) \to H^0(X, \calF'') \to H^1(X, \calF') \to H^1(X, \calF) \to \ldots$ ist exakt und $H^0(X, \calF) = \Gamma(X, \calF) = \calF(X)$.

\begin{Bem}
Sei $H^i(X,\cdot)$ so ein Funktor, (*) $0 \to \calF \to \calG^0 \to \calG^1 \to$ exakte Sequenz in \underline{Ab$(X)$} (\quot{Aufl"osung von $\calF$}).

Ist $H^i(X, \calG^j) = 0$ f"ur alle $j \ge 0$ und alle $i \ge 1$ ($\calG^j$ ist \defterm{azyklisch}), so gilt $H^i(X, \calF) = H^i(\Gamma(X, \calG^0))$
\end{Bem}

\begin{bew}
Induktion "uber $i$:\begin{enumerate}[$a = 0$:]
\item[$i = 0$:]
	$0 \to \Gamma(X, \calF) \to \Gamma(X, \calG^0) \to \Gamma(X, \calG^1)$ ist exakt.\\
	$H^0(\Gamma(X, \calG^0)) = \Kern (\Gamma(X, \calG^0) \to \Gamma(X, \calG^1)) \cong \Gamma(X, \calF)$
	
	Aus (*) folgt $\begin{array}{cccccccccl} 0 &\to& \calF &\to& \calG^0 &\to& \FakRaum{\calG^0}{\calF} &\to& 0 & \text{ist exakt und} \\
		0 &\to& \FakRaum{\calG^0}{\calF} &\to& \calG^1 &\to& \calG^2 &\to& \ldots & \text{ist exakt}\end{array}$
\item[$i = 1$:]
	Nach Voraussetzung gibt es lange exakte Sequenz
		\[ 0  \to H^0(X, \calF) \to H^0(X, \calG^0) \to H^0(X, \FakRaum{\calG^0}{\calF}) \to H^1(X, \calF) \to 0 \to H^1(X, \FakRaum{\calG^0}{\calF}) \to \ldots  \]
		$\Rightarrow H^1(X, \calF) = \FakRaum{H^0(X, \calG^0/\calF)}{\Bild(H^0(X, \calG^0))}$\\
		\textcolor{gray}{($H^0(X, \FakRaum{\calG^0}{\calF}) \cong \Kern(H^0(X, \calG^1) \to H^0(X, \calG^2))$)}
		
		$\Rightarrow H^1(X, \calF) \cong \FakRaum{\Kern( H^0(X, \calG^1) \to H^0(X, \calG^2))}{\Bild( H^0(X, \calG^0) \to H^0(X, \calG^1))}$\\
		$\textcolor{white}{\Rightarrow H^1(X, \calF)} = H^1(\Gamma(X, \calG^0))$
\end{enumerate}\end{bew}

\begin{DefBem}\begin{enumerate}[a)]
\item
	Ein Objekt $I$ in einer abelschen Kategorie $\calC$ hei\ss t \defterm{injektiv}, wenn $\Hom_\calC(\cdot, I)$ exakt ist.
\item
	$I$ ist genau dann injektiv, wenn f"ur jedes solches Diagramm ein $\varphi \in \Hom_\calC(C, I)$ existiert mit $\tilde \varphi \circ \alpha = \varphi$.
	\begin{center}\begin{tikzpicture}
		\node (0) at (-1,0) {$0$};
		\node (1) at (0,0) {$C'$};
		\node (2) at (1,0) {$C$};
		\node (3) at (0,-1) {$I$};
		
		\draw[->] (0) -- (1);
		\draw[->] (1) -- node[above,font=\scriptsize]{$\alpha$} (2);
		\draw[->] (1) -- node[left,font=\scriptsize]{$\varphi$} (3);
		\draw[->,dashed] (2) -- node[right,font=\scriptsize]{$\tilde \varphi$} (3);
	\end{tikzpicture}\end{center}
	
	$0 \to C' \xrightarrow{\alpha} C \to C'' \to 0$
	
	$0 \to \Hom(C'', I) \to \Hom(C, I) \to \Hom(C', I)$
\end{enumerate}\end{DefBem}

\begin{bsp}
$\FakRaum{\Q}{\Z}$ ist injektiv in $\Ab$, \emph{denn}:

Seien $G' \subseteq G$ abelsche Gruppe, $\varphi: G' \to \FakRaum{\Q}{\Z}$ Homomorphismus. F"ur $a \in G$ sei
	\[ \tilde \varphi(a) := \begin{cases} \frac{1}{n} \varphi(n \cdot a) & \text{, falls } n \text{ minimal mit } n \cdot a \in G' \\
		0 & , n \cdot a \notin G' \text{ f"ur alle } n > 0 \end{cases}\]
\end{bsp}

\begin{Bem}
Sei $I$ injektives Objekt in \underline{Ab$(X)$} und \begin{tikzpicture}[baseline=-3.75]
	\node (0) at (0,0) {$0$}; \node (I) at (1,0) {$I$}; \node (F) at (2,0) {$\calF$}; \node (1) at (3,0) {$\calF''$}; \node (2) at (4.25,0) {$\calF''$}; \node (3) at (5.25,0) {$0$};
	\draw[->] (0) -- (I); \draw[->] (I) -- (F); \draw[->] (F) -- (1); \draw[->] (1) -- (2); \draw[->] (2) -- (3); \draw[->,dashed] (F) to[out=225,in=315] node[below,font=\scriptsize]{$p$} (I);\end{tikzpicture} exakt.

\begin{enumerate}[a)]
\item
	Dann gibt es $p: \calF \to I$ mit $p \circ \alpha = \id_I$.
\item
	$\calF \cong \calF'' \oplus I$, \emph{denn:}
	
	$I \cap \Kern(p) = 0$, $\beta|_{\Kern(p)}: \Kern(p) \to \calF''$ ist Isomorphismus, $I + \Kern(p) = \calF$
\item
	$0 \to H^0(X, I) \to H^0(X, \calF) \to H^0(X, \calF'') \to 0 = H^1(X, I)$ ist exakt.
\end{enumerate}\end{Bem}

% 5. 7.
\begin{Prop}

\end{Prop}

\begin{bew}

\end{bew}

\begin{DefBem}
\begin{enumerate}[a)]
\item
	
\item
	
\item
	
\end{enumerate}\end{DefBem}

\begin{Prop}
\begin{enumerate}[a)]
\item
	
\item
	
\item
	
\item
	
\end{enumerate}\end{Prop}

\begin{Verallg}

\end{Verallg}

\newpage

%-------------------------- Abschnitt 13 (Cech Kohomologie) --------------------------

\section{\v{C}ech-Kohomologie}

\begin{DefBem}\begin{enumerate}[a)]
\item
	
\item
	
\item
	
\item
	
\end{enumerate}\end{DefBem}

\begin{bew}\begin{enumerate}[a)]\item[d)]

\end{enumerate}\end{bew}

\begin{Bsp}
\begin{enumerate}[a)]
\item
	
\item
	
\end{enumerate}\end{Bsp}

\begin{Def}\begin{enumerate}[a)]
\item
	
\item
	
\item
	
\item
	
\end{enumerate}\end{Def}

\begin{bew}\begin{enumerate}[a)]\item[d)]

\end{enumerate}\end{bew}

\begin{Prop}

\end{Prop}

\begin{bew}

\end{bew}




%-_-_-_-_-_-_-_-_-_-_-_-_-_-_ Anhang -_-_-_-_-_-_-_-_-_-_-_-_-_-_-_-_

\appendix

%-_-_-_-_-_-_-_-_-_-_-_-_-_-_ Uebungen -_-_-_-_-_-_-_-_-_-_-_-_-_-_-_-_

\chapter{\"Ubungen}

% Die Benennung der "section" so aendern, dass "\"Ubung 123 vom " am Anfang steht
% Der Code ist fast genau der vom Anfang der Praeambel, dort steht die Erklaerung
\renewcommand*{\othersectionlevelsformat}[3]{\ifstr{#1}{section}{\"Ubung\ #3\ vom\ }{#3\autodot\enskip}}

% Das Format der "section" in Kopfzeile der rechten Seiten
\renewcommand*{\sectionmarkformat}{\"Ubung \thesection\autodot\ vom\enskip}

\setcounter{section}{-1}

%-------------------------- Uebung 0 --------------------------
\section{24. April 2012}
\setcounter{Aufg}{0} %Damit die Aufgaben jedes Mal bei Aufgabe 1 anfangen
\setcounter{Loes}{0}

\begin{Aufg}
 Sei $X$ ein topologischer Raum, $A$ eine abelsche Gruppe und $x \in X$. 

Die "`Wolkenkratzergarbe"' auf $X$ ist definiert durch 
$\mathcal{W}(U) \coloneqq \left\{ \begin{array}{ll}
                                   A & ,\, x \in U\\
                                   \{0\} & ,\,x \notin U
                                  \end{array}
\right.$ f\"ur $U \subseteq X$ offen, mit Restriktionsabbildungen 
$$\rho_V^U = \left\{ \begin{array}{rcll}
                     \{0\} \ni &0 \mapsto 0& \in \{0\}& ,\, x \notin V,\, x \notin U\\
                     A \ni &a \mapsto 0& \in \{0\}& ,\, x \notin V,\, x \in U\\
                     & \id_A && ,\,x \in V,\, x \in U
                    \end{array} \right.$$ f\"ur $V \subseteq U \subseteq X$ offen.
\begin{enumerate}%[topsep=0cm, parsep=0cm, label=\alph*)]
 \item Zeige, dass $\mathcal{W}$ eine Garbe von abelschen Gruppen auf $X$ ist.
 \item Berechne die Halme der Garbe.
\end{enumerate}
\end{Aufg}



\begin{Aufg}
 Sei $X$ ein topologischer Raum und $\mathcal{F}$ eine Pr\"agarbe von abelschen Gruppen auf $X$. Ist $x \in X$ ein Punkt, so bezeichne $\mathcal{F}_x$ den Halm von $\mathcal{F}$ in $x$. Nun definiert man die Menge
\begin{center}
Sp\'e$(\mathcal{F}):= \stackrel{.}{\bigcup\limits_{x\in X}} \mathcal{F}_x$.
\end{center}
Definiere weiter eine Projektionsabbildung
$\pi \colon \textup{Sp\'e}(\mathcal{F}) \rightarrow X$ durch $\mathcal{F}_x \ni s \mapsto x$. \\
F\"ur jede offene Menge $U \subseteq X$ und f\"ur jedes $s \in \mathcal{F}(U)$ erh\"alt man eine Abbildung $\overline{s} \colon U \rightarrow \textup{Sp\'e}(\mathcal{F})$ durch
$x \mapsto s_x$, wobei $s_x$ der Keim von $s$ in $x$ sei. Alle diese Abbildungen sind Schnitte von $\pi$ \"uber $U$, d.\,h. f\"ur jedes dieser $\overline{s}$ gilt: $\pi \circ \overline{s} = \textup{id}_U$.
Nun macht man Sp\'e$(\mathcal{F})$ zu einem topologischen Raum, indem man ihm die feinste Topologie gibt, so dass alle diese $\overline{s}$ f\"ur alle $s$ und alle $U$ stetig werden. Versehen mit dieser Topologie hei\ss t $\textup{Sp\'e}(\mathcal{F})$ der \textit{Espace Etal\'e} zur Pr\"agarbe $\mathcal{F}$.

Sei nun $\mathcal{F}^+$ die zu $\mathcal{F}$ assoziierte Garbe. Zeige:\\
F\"ur jede offene Menge $U \subseteq X$ ist $\mathcal{F}^+(U)$ die Menge aller \textbf{stetigen} Schnitte von $\pi$ \"uber $U$.
 Insbesondere ist $\mathcal{F}$ genau dann eine Garbe, wenn f\"ur jedes offene $U \subseteq X$ die Menge aller stetigen Schnitte von $\pi$ \"uber $U$ gerade $\mathcal{F}(U)$ ist.
\end{Aufg}




\begin{Loes}\begin{enumerate}[a)]
\item
  Sei $U$ offen, $U = \bigcup\limits_{i\in I} U_i$, $s_i$ konsistente Familie.
  \begin{description}[\setlabelstyle{\itshape}]
  \item[Fall 1:]
    $x \in U \Rightarrow \exists i \in I: x\in U_i \Rightarrow \calW(U_i) = A \Rightarrow  \calW(U) = A$\\
    \emph{Behauptung:} $s=s_i$ ist Amalgam\\
    $\rho_{U_i}^U= \id_A \leadsto s = s_i$ ist einziger Kandidat\\
    F\"ur $j\in I$\begin{description}
    \item{$\bullet$ mit $x\in U_j$:}\\
      $s_j = \id_A(s_j) = \rho_{U_i\cap U_j}^{U_j}(s_j) = \rho_{U_i\cap U_i}^{U_j}(s_j)$\\
      $\textcolor{white}{s_j = \id_A(s_j)}= \id_A(s_i) = s_i = s = \rho_{U_j}^U(s)$
    \item{$\bullet$ mit $x\notin U_j$:}\\
      $s_j = 0, \rho_{U_0}^U(s) = 0$
    \end{description}
  \item[Fall 2:]
    $x \notin U \Rightarrow x\notin U_j \forall\  j \in I$\\
    $\Rightarrow \calW(U) = \calW(U_j) = \{0\}$
  \end{description}
\item
  Sei $y \in X$. Bestimme den Halm von $\calW$ in $y$.\\
  $\calW_y = \{[(U,s)]_y \vert s\in \calW(U)\}$ mit $[(U,s)]_y = [(U',s')]_y \Leftrightarrow \exists U'' \subseteq U \cap U', y \in U'': s|_{U''} = s'|_{U''}$ \begin{description}[\setlabelstyle{\itshape}]
  \item[Fall 1:]$\exists U\subseteq X$ offen, $y\in U, x\notin U$\\
    $\Rightarrow [U',s] = [U,0]$ f\"ur alle $U'\subseteq X$ offen, $s\in \calW(U')$\\
    $\Rightarrow \calW_y = \{0\}$
  \item[Fall 2:] $\forall\ $ offenen Umgebungen $U$ von $y: x\in U$\\
    $\Rightarrow \calW_y \cong A$, \emph{denn:} Sei $U$ offene Umgebung von $y$
      \begin{equation*}\begin{split} \varphi:&\left\{\begin{array}{rcl} A &\to& \calW_y \\
      a &\mapsto& [U,a]\end{array}\right.\\
      \psi:& \left\{\begin{array}{rcl} \calW_y &\to& A \\
      \,[U,a] &\mapsto& a\end{array}\right.\end{split}\end{equation*}
    $[U,a]_y = [U,b]_y \Leftrightarrow a=a|_{U''} = b|_{U''} = b$
  \end{description}
  
\end{enumerate}\end{Loes}

\begin{Loes}
\begin{minipage}[t]{0.6\textwidth}
$s\in \calF(U), x\in U \leadsto [U,s]_x \in \calF_x$\\
$\bar s :\left\{\begin{array}{rcl}U &\to& \Spe(\calF)\\ x &\mapsto& [U,s]_x\in\calF_x\end{array}\right., \pi \circ \bar s = \id_U$\end{minipage}
\begin{minipage}[c]{0.4\textwidth}
\begin{tikzpicture}
  %Punkte definieren
  \node (x) at (0,-0.5){$x \in X$};
  \node (Fx) at (0,2.5){$\calF_x$};
  \node (s) at (2.25,1.25){\textcolor{color1}{$s$}};
  \node (X) at (3,0){$X$};
  \node (KlAuf) at(-2,0) {\textcolor{color1}{(}};
  \node (KlZu) at(2,0) {\textcolor{color1}{)}};
  \node (U) at(1.75,-0.5) {\textcolor{color1}{$U$}};

  \draw (-2.5,0) -- (2.5,0);
  %senkrechte Linien zeichnen
  \draw (-1.5,0) -- (-1.5,2);
  \draw (-1,0) -- (-1,2);
  \draw (-0.5,0) -- (-0.5,2);
  \draw (0,-0.25) -- (0,2);
  \draw (1.5,0) -- (1.5,2);
  \draw (1,0) -- (1,2);
  \draw (0.5,0) -- (0.5,2);
  %Schlangenlinie
  \draw[color1] (-2,1) to [out=30,in=180] (-1,1.25) to [out=0,in=150] (0,1) to [out=330,in=180] (1,0.75) to [out=0,in=210] (2,1);
  %gestrichelte Linien
  \draw[dashed,color1] (-2,1) -- (-2,0);
  \draw[dashed,color1] (2,1) -- (2,0);
\end{tikzpicture}
\end{minipage}
$\calF^+$ assoziiert Garbe zu $\calF$, also f\"ur $U\subseteq X$ offen:
  \[\begin{array}{rccr}\calF^+(U) &:= \{s: U\to \dot{\bigcup\limits_{x\in U}}\calF_x &\vert& \forall\  x\in X: s(x)\in \calF_x \text{ und } \exists \text{ offene Umgebung } U_x \\ &&& \text{ von } x, \exists f\in \calF(U_x), s(y) = f_y := [(U_x,f)]_y\}\end{array}\]
\emph{Zu zeigen:} $\calF^+(U) = \{\text{stetige Schnitte von } \pi \text{ \"uber } U\}$
\begin{twosidedproof}
\proofsubseteq
  Sei $s\in \calF^+(U)$\\
  $\pi \circ s = \id \Rightarrow s$ Schnitt\\
  Noch zu zeigen: $s$ ist stetig\\
  Nach Definition von $\calF^+$ gibt es \"Uberdeckung $U = \bigcup\limits_{i \in I} U_i$ und Schnitte $s_i \in\calF(U_i)$ mit $s|_{U_i} = \bar{s_i}$. Die $\bar{s_i}: U_i \to \Spe(\calF)$ sind stetig.\\
  Sei $\widetilde U \subseteq \Spe(\calF)$ offen.\\
  $\bar s_i^{-1}(\widetilde U) = s^{-1}(\widetilde U) \cap U_i$ offen\\
  $s^{-1}(\widetilde U) = \bigcup\limits_{i\in I} (s^{-1}(\widetilde U)\cap U_i)$ ist offen
\proofsupseteq
  Sei $s: U\to \Spe(\calF)$ stetiger Schnitt von $\pi$ \"uber $U$, das hei\ss t $\forall\  x \in X: s(x) \in \calF_x$.\\
  \emph{Noch zu zeigen:} $\forall\  x\in X \exists U_x\in \Off(X)$ mit $x\in U_x$ und ein $f\in \calF(U_x)$ sodass $\forall\  y \in U$ gilt $s(y) = f_y$\\
  Sei $x\in X$, $s(x)=[U',s']_x\in\calF_x$, $s'\in\calF(U')$\\
  $\widetilde U:= \{[U',s']_y \in \calF_y | y\in U'\}$ ist offen in $\Spe(\calF)$ und enth\"alt $s(x)$.
    \[[U',s']_y = [U'',s'']_y\]
  $\Rightarrow \exists U''' \subseteq U'\cap U'': s'|_{U'''} = s''|_{U'''}$\\
  $s''^{-1}(\widetilde U \cap s'(U''')) = s'^{-1}(\widetilde U \cap s'(U'''))$ offen\\
  Setze $U_x:= s^{-1}(\widetilde U)$\\
  $s$ stetig $\Rightarrow U_x$ offen, $x\in U_x$\\
  Au\ss erdem: $\forall\  y \in U_x: \bar{s'}(y) = \underbrace{[U', s']_y}_{=[U_x, s'|_{U_x}]_y} \xlongequal{\pi \circ s = \id} s(y)$\\
  $\Rightarrow s'|_{U_x} \in \calF(U_x)$ erf\"ullt Garbeneigenschaft.
\end{twosidedproof}

\textbf{zur L\"osung:}

In der \"Ubung wollte ich zeigen, dass f\"ur offenes $U' \subseteq X$ mit $x \in U'$ und $s' \in \mathcal{F}(U')$ die Menge $\tilde{U} \coloneqq \{ [U',s']_y \in \mathcal{F}_y \mid y \in U' \}$ offen ist bez\"uglich der oben definierten Topologie auf Sp\'e$(\mathcal{F})$.

Auf einmal scheint das Argument auf meinem Zettel wieder zu stimmen (war nur etwas knapp formuliert), deshalb schreibe ich es hier nochmal ausf\"uhrlicher auf.

Zu zeigen ist ja, dass f\"ur jedes offene $U'' \subseteq X$ und jedes $s'' \in \mathcal{F}(U'')$ die Menge $\left(\overline{s''}\right)^{-1}\left(\tilde{U}\right)$ offen in $X$ ist. Sei also $U'' \in \Off(X)$ und $s'' \in \mathcal{F}(U'')$.

Ist $\left(\overline{s''}\right)^{-1}\left(\tilde{U}\right) = \emptyset$, so ist nichts weiter zu zeigen, da $\emptyset$ ja offen ist. Ist andernfalls $y \in \left(\overline{s''}\right)^{-1}\left(\tilde{U}\right)$, so ist $ \overline{s''}(y) \in \tilde{U}$ und wegen $\overline{s''}(y) \in \mathcal{F}_y$ folgt $[U'',s'']_y = [U',s']_y$ ($\tilde{U}$ enth\"alt ja pro Halm h\"ochstens ein Element). Folglich existiert ein offenes $U''' \subseteq U' \cap U''$ mit $y \in U'''$ und $s'|_{U'''} = s''|_{U'''}$. Somit ist $U''' \subseteq \left(\overline{s''}\right)^{-1}\left(\tilde{U}\right)$ eine offene Umgebung von $y$ und $\left(\overline{s''}\right)^{-1}\left(\tilde{U}\right)$ offen.

\end{Loes}

%-------------------------- Uebung 1 --------------------------
\newpage
\section{1. Mai 2012}
\setcounter{Aufg}{0}
\setcounter{Loes}{0}

\begin{Aufg}[4 Punkte]
\begin{enumerate}[a)]
 \item Sei $X:=\{0,1\}$ mit der diskreten Topologie versehen (d.\,h. jede Teilmenge ist offen).
Bestimme alle Garben $\mathcal{F}$ von Mengen auf $X$.
 \item Bestimme die Halme $\mathcal{F}_x$ der Garben $\mathcal{F}$ aus Teil a) in den Punkten $x=0$ und $x=1$.
\end{enumerate}
\end{Aufg}

\begin{Aufg}[4 Punkte]
Sei $X$ ein topologischer Raum und $\mathcal{F}$ die Pr\"agarbe der konstanten Funktionen auf $X$. Genauer sei f\"ur $U \subseteq X$ offen 
$\mathcal{F}(U) \coloneqq \{ \, f \colon U \to \Z \mid f \textrm{ ist konstant } \}$ und f\"ur $V \subseteq U \subseteq X$ offen $\rho_V^U\colon \mathcal{F}(U) \to \mathcal{F}(V), f \mapsto f|_{V}$.
\begin{enumerate}[a)]%[topsep=0cm, parsep=0cm, label=\alph*)]
\item Zeige, dass $\mathcal{F}$ im allgemeinen keine Garbe ist.
\item Berechne die zu $\mathcal{F}$ assoziierte Garbe $\mathcal{F}^+$.
\end{enumerate}
\end{Aufg}

\begin{Aufg}[4 Punkte]
Sei $X$ ein topologischer Raum und $\mathcal{B}$ eine Basis\footnote{d.\,h. alle offenen $U \subseteq X$ lassen sich als Vereinigung von Elementen aus $\mathcal{B}$ schreiben} der Topologie von $X$. Weiter sei $\mathcal{C}$ eine Kategorie.
Wir definieren eine $\mathcal{B}$-Garbe auf $X$ als kontravarianten Funktor $\mathcal{F}'$ von $(\mathcal{B}, \subseteq)$ nach $\mathcal{C}$, der die $\mathcal{B}$-Garbeneigenschaft erf\"ullt:

Sind $U = \bigcup_{i\in I} U_i$ mit $U, U_i \in \mathcal{B}$ und $(s_i)_{i \in I}$ mit $s_i \in \mathcal{F}'(U_i)$ gegeben, so dass $\rho^{U_i}_V(s_i) = \rho^{U_j}_V(s_j)$ f\"ur alle $V \subseteq U_i \cap U_j, V \in \mathcal{B}$ gilt, dann existiert genau ein $s \in \mathcal{F}'(U)$ mit $\rho^{U}_{U_i}(s) = s_i$ f\"ur alle $i \in I$.

Sei zun\"achst $\mathcal{C}$ die Kategorie der Mengen. 
\begin{enumerate}[a)]%[topsep=0cm, parsep=0cm, label=\alph*)]
 \item Zeige, dass sich $\mathcal{F}'$ eindeutig zu einer Garbe $\mathcal{F}$ auf $X$ fortsetzen l\"asst.\\
  \textit{Hinweis:} Inverser Limes
 \item Es seien $\mathcal{F}$ und $\mathcal{G}$ Garben auf $X$ und f\"ur $U \in \mathcal{B}$ Abbildungen $\tilde{\varphi}(U)\colon \mathcal{F}(U) \to \mathcal{G}(U)$ gegeben, die mit den Restriktionsabbildungen $\rho^U_{U'}$ kommutieren. 
 
 Zeige, dass es einen eindeutigen Garbenmorphismus $\varphi\colon \mathcal{F} \to \mathcal{G}$ gibt mit $\varphi(U) = \tilde{\varphi}(U)$ f\"ur alle $U \in \mathcal{B}$.
 \item Mache dir klar, dass a) und b) auch in der Kategorie der (abelschen) Gruppen, der Ringe und der $R$-Moduln gelten.
\end{enumerate}
\end{Aufg}

\begin{Loes}
\begin{enumerate}[a)]%[topsep=0cm, parsep=2mm, label=\alph*)]
 \item Die offenen Mengen von $X$ sind $\Off(X)= \{ \emptyset, \{0\}, \{1\}, \{0,1\}\}$. Eine Garbe von Mengen auf $X$ ordnet jedem Element aus $\Off(X)$ eine Menge zu, also 
$$
\emptyset  \rightarrow  M_\emptyset,  \quad
\{0\}  \rightarrow  M_0,\quad
\{1\} \rightarrow  M_1,\quad
X  \rightarrow  M_X.
$$
Zus\"atzlich brauchen wir noch Restriktionsabbildungen
$$
M_0 \rightarrow  M_\emptyset,\quad
M_1  \rightarrow  M_\emptyset,\quad
M_X  \rightarrow  M_\emptyset,\quad
M_X  \rightarrow  M_0,\quad
M_X  \rightarrow  M_1,\quad
$$
so dass das folgende Diagramm kommutiert:
\begin{center}
\begin{tikzpicture}
   \matrix (m) [matrix of math nodes, row sep=2em, column sep=2em]
  { M_X & M_0 \\ M_1 & M_\emptyset \\};
  \path[->]
  (m-1-1) edge (m-2-1)
      edge (m-1-2)
      edge (m-2-2)
  (m-2-1) edge (m-2-2)
  (m-1-2) edge (m-2-2);
\end{tikzpicture}
\end{center}
In der Vorlesung wurde bewiesen, dass $M_\emptyset$ einelementig sein muss. Dann kommutiert das obige Diagramm automatisch.

Da die offenen Mengen $\{0\}$ und $\{1\}$ die offene Menge $\{0,1\}$ \"uberdecken, besagt die Garbeneigenschaft, dass es zu jedem $a \in M_0$ und jedem $b\in M_1$ ein eindeutig bestimmtes $c \in M_X$ geben muss mit $\rho^X_{\{0\}}(c) = a$ und $\rho^X_{\{1\}}(c)=b$. Das bedeutet, dass $M_X \cong M_0 \times M_1$ ist. Damit haben wir alle Garben von Mengen auf $X$ charakterisiert.

 \item Der Halm der Garbe $\mathcal{F}$ in $x$ ist definiert als 
\begin{eqnarray*}
\mathcal{F}_x \coloneqq \{\, (U,f) \mid U \in \Off(X), x \in U, s \in \mathcal{F}(U) \,\}/\sim,
\end{eqnarray*}
wobei zwei Paare $(U,s)$ und $(U',s')$ \"aquivalent sein sollen, wenn es ein $U'' \subseteq U \cap U'$ gibt mit $\rho^U_{U''}(s) = \rho^{U'}_{U''}(s')$.

Zwei Punkte $(X,(a,b))$ und $(X,(a',b'))$ beschreiben genau dann den gleichen Punkt in $\mathcal{F}_0$, wenn $a = \rho^X_{\{0\}}((a,b)) = \rho^{X}_{\{0\}}((a',b'))=a'$.
Also ist $\mathcal{F}_0 \cong M_0$. Ebenso ist $\mathcal{F}_1 \cong M_1$.
\end{enumerate}
\end{Loes}

\begin{Loes}
\begin{enumerate}[a)]%[topsep=0cm, parsep=2mm, label=\alph*)]
 \item Falls es in $X$ offene Mengen $U$ und $U'$ gibt mit $U \cap U' = \emptyset$, dann ist $\mathcal{F}$ keine Garbe:
 
 Seien $f_1 \in \mathcal{F}(U)$ mit $f_1 \equiv 1$ und $f_2 \in \mathcal{F}(U')$ mit $f_2 \equiv 2$. Da $U \cap U' = \emptyset$, ist $f_1|_{U \cap U'} = {f_2}|_{ U \cap U'}$ und $(f_i)_{i\in \{1,2\}}$ ist eine konsistente Familie f\"ur die offene \"Uberdeckung $U \cup U'$. Da aber f\"ur $f \in \mathcal{F}(U \cup U')$ nicht gleichzeitig $f|_U \equiv 1$ und $f|_{U'} \equiv 2$ gelten kann, gibt es zu der konsistenten Familie kein Amalgam.
 
 \item Zun\"achst berechnen wir die Halme von $\mathcal{F}$: 
 $$[(U,f)]_x = [(U',f')]_x \Leftrightarrow \exists U'' \subseteq U \cap U' \textrm{ mit } x \in U'' : f|_{U''} = f'|_{U''} \Leftrightarrow f(x) = f'(x)$$
 Die Halme sind also \"uber $[(U,f)]_x \mapsto f(x)$ alle isomorph zu $\Z$.
 
 Es gilt 
 $$\begin{array}{cl}
    \mathcal{F}^+(U) \coloneqq \{ \, s \colon U \to \stackrel{.}{\bigcup\limits_{x \in U}} \mathcal{F}_x \mid 
    & \forall\  x \in U : s(x) \in \mathcal{F}_x \textrm{ und } \exists \textrm{ offene Umgebung } U_x \textrm{ von } x, \\
    & \exists f \in \mathcal{F}(U_x) \, \forall\  y \in U_x : s(y) = [(U_x,f)]_y \eqqcolon f_y \, \} \, .
   \end{array}$$
 Benutzen wir die Isomorphismen $\mathcal{F}_x \ni [(U,f)]_x \mapsto f(x) \in \Z$ der Halme mit $\Z$, sowie die Tatsache, dass $f \in \mathcal{F}(U_x)$ konstant ist, so besagt die letzte Bedingung an $s$:
 $$\forall\  x \in U \; \exists \textrm{ offene Umgebung } U_x \textrm{ von } x: \forall\  y \in U_x: s(y) = s(x) \, .$$
 Damit gilt
  $$\begin{array}{rcccl}
    \mathcal{F}^+(U) &= \{ & s \colon U \to \Z &\mid 
    & \forall\  x \in U \, \exists U_x \in \Off(X), x \in U_x : s|_{U_x} \textrm{ ist konstant}\, \}\\
    &= \{ & s \colon U \to \Z & \mid & s \textrm{ ist lokal konstant} \, \} \, .
   \end{array}$$
\end{enumerate}
\end{Loes}

\begin{Loes}
\begin{enumerate}[a)]%[topsep=0cm, parsep=0.1cm, label=\alph*)]
 \item Wir nutzen den Hinweis und definieren $\F$ mit Hilfe des inversen Limes. F\"ur $U \in \Off(X)$ setzen wir 
 $$\begin{array}{rcl}
    \F(U) &\coloneqq& \varprojlim\limits_{V \subseteq U, V \in \mathcal{B}} \F'(V) \\
    &=& \{ \, (s_V)_{V \subseteq U, \, V \in \mathcal{B}} \in \prod_{V \subseteq U, V \in \mathcal{B}} \F'(V) \mid \forall\  W \subseteq V \subseteq U, V,W \in \mathcal{B}: \rho^V_W(s_V) = s_W \, \}
   \end{array}
$$
und f\"ur $U' \subseteq U \subseteq X$ offen sei
$$
\rho^U_{U'} \colon \F(U) \to \F(U'), (s_V)_{V \subseteq U, V \in \mathcal{B}} \mapsto (s_V)_{V \subseteq U', V \in \mathcal{B}}\,.
$$
Die Abbildung $\rho^U_{U'}$ bildet die Familie $(s_V)_{V \subseteq U, V \in \mathcal{B}}$ auf die Teilfamilie \"uber den $V \subseteq U'$ ab, also gilt $\rho^U_U = \id_U$ und $\rho^{U'}_{U''} \circ \rho^U _ {U'} = \rho^U_{U''}$
und wir haben eine Pr\"agarbe von Mengen definiert.

Die Pr\"agarbe $\F$ stimmt auf den Mengen $V \in \mathcal{B}$ mit $\F'$ \"uberein: 

Die Abbildung
$$
\psi(V)\colon \F'(V) \to \F(V), s \mapsto (s|_{V'})_{V' \subseteq V, V' \in \mathcal{B}}
$$
ist wohldefiniert, da eine $\mathcal{B}$-Garbe insbesondere ein kontravarianter Funktor von $(\mathcal{B}, \subseteq)$ nach $\mathcal{C}$ ist. Au\ss erdem liefert uns die $\mathcal{B}$-Garbeneigenschaft eine wohldefinierte Umkehrabbildung
$$
\psi^{-1}(V)\colon \F(V) \to \F'(V),  (s_{V'})_{V' \subseteq V, V' \in \mathcal{B}} \mapsto s_V\,.
$$
Sowohl $\psi$ als auch $\psi^{-1}$ kommutieren mit den Restriktionsabbildungen.

Weiter ist $\F$ sogar eine Garbe von Mengen auf $X$:

Sei $U = \bigcup_{i \in I} U_i$ eine offene \"Uberdeckung, $s_i = (s_{i,V} )_{V \subseteq U_i, V \in \mathcal{B}} \in \F(U_i)$ und $(s_i)_{i\in I}$ eine konsistente Familie. 
Ist $s \coloneqq (s_V)_{V \subseteq U, V \in \mathcal{B}}$  ein Amalgam zu der konsistenten Familie $(s_i)_{i \in I}$, so muss f\"ur alle $V \subseteq U_i$, $V \in \mathcal{B}$ wegen $\rho^{U_i}_V(s) = s_i$ die Gleichheit $s_V = s_{i,V}$ gelten. Da die Familie konsistent ist, gilt $(s_{i,V})_{V \subseteq U_i\cap U_j, V \in \mathcal{B}} = \rho^{U_i}_{U_i \cap U_j}(s_i) = \rho^{U_j}_{U_i \cap U_j}(s_j) = (s_{j,V} )_{V \subseteq U_i\cap U_j, V \in \mathcal{B}}$ und $s_V$ ist auch f\"ur $V \in U_i \cap U_j$ wohldefiniert. 

Sei nun $V \subseteq U, V \in \mathcal{B}$ beliebig. Dann sind die $(s_{V'})_{V' \subseteq V \cap U_i, V' \in \mathcal{B}}$ bereits alle definiert und stimmen auf den Schnitten $U_i \cap U_j$ \"uberein. Da $\mathcal{B}$ eine Basis der Topologie ist, ist $V = \bigcup\limits_{i \in I} \bigcup\limits_{V' \subseteq V \cap U_i, V' \in \mathcal{B}} V'$ und nach der $\mathcal{B}$-Garbeneigenschaft gibt es dann genau ein $s_V \in \F'(V) \cong \F(V)$ mit $\rho^V_{V'}(s_V) = s_{V'}$ f\"ur alle $i \in I$, $V' \subseteq V \cap U_i$, $V' \in \mathcal{B}$. Folglich existiert das Amalgam $s$ und da wir bei der Konstruktion keine Wahl zu treffen hatten, ist es auch eindeutig. 

Es bleibt zu zeigen, dass $\F$ bis auf Isomorphie eindeutig ist. Daf\"ur verwenden wir den Aufgabenteil b):

Sei also $\G$ eine weitere Garbe von Mengen auf $X$, mit $\G(V) = \F'(V) = \F(V)$ f\"ur alle $V \in \mathcal{B}$.
Auf den Basismengen $V$ der Topologie sind also $\id(V)\colon \F(V) \to \G(V)$ und $\id(V)\colon \G(V) \to \F(V)$ wohldefinierte Abbildungen, die mit den Restriktionen kommutieren. Nach b) existieren dann genau ein Garbenmorphismus $\varphi\colon \F \to \G$ und ein Garbenmorphismus $\varphi'\colon \G \to \F$, der $\id$ fortsetzt.

Weiterhin ist $\varphi' \circ \varphi(V) = \id_{\F(V)}$ und, wiederum nach b), gibt es genau einen Garbenmorphismus von $\F$ nach $\F$, der diese fortsetzt. Da sowohl der Identit\"atsfunktor $\id_{\F}$, als auch $\varphi' \circ \varphi$ solche eine Fortsetzung sind, gilt $\varphi' \circ \varphi \cong \id_{\F}$. Genauso zeigt man $\varphi \circ \varphi' \cong \id_G$ und hat damit bewiesen, dass $\G \cong \F$ ist.

\item Seien also $\F$ und $\G$ zwei Garben und f\"ur alle $V \in \mathcal{B}$ eine Abbildung $\varphi(V)\colon \F(V) \to \G(V)$ gegeben, so dass die $\varphi(V)$ mit den Restriktionsabbildungen kommutieren. Sei nun $U \in \Off(X)$ beliebig. Da $\mathcal{B}$ eine Basis der Topologie auf $X$ ist, gilt $U = \bigcup\limits_{V \subseteq U, V \in \mathcal{B}} V$. Damit ist $\varphi(U)$ bereits eindeutig bestimmt:

\begin{center}
\begin{tikzpicture}
   \matrix (m) [matrix of math nodes, row sep=2em, column sep=2em]
  { \F(U) & \G(U) \\
    \F(V) & \G(V) \\
   };
  \path[->,font=\scriptsize]
  (m-1-1) edge node[auto] {$\varphi(U)$} (m-1-2)
      edge node[auto,swap] {$\rho^U_V$} (m-2-1)
  (m-2-1) edge node[auto,swap] {$\varphi(V)$} (m-2-2)
  (m-1-2) edge node[auto] {$\rho^U_V$} (m-2-2);
\end{tikzpicture}
\end{center}

Sei $f \in \F(U)$. F\"ur alle $V \subseteq U, V \in \mathcal{B}$ sind die Schnitte $\rho^U_V(\varphi(U)(f)) = \varphi(V)(\rho^U_V(f)) \in \G(V)$ festgelegt und bilden eine konsistente Familie in $\G$. Da $\G$ eine Garbe ist, ist $\varphi(U)(f) \in \G(U)$ damit eindeutig bestimmt.

Ist $U' \subseteq U \subseteq X$ offen, dann ist jedes $V \subseteq U', V \in \mathcal{B}$ auch in $U$ enthalten. Es gilt $\rho^{U'}_V(\rho^U_{U'}(\varphi(U)(f))) = \rho^U_V(\varphi(U)(f)) = \varphi(V)(\rho^U_V(f))$, also ist $\rho^U_{U'}(\varphi(U)(f))$ ein Amalgam f\"ur die $\varphi(V)(\rho^U_V(f))$ und es gilt $\rho^U_{U'}(\varphi(U)(f)) = \varphi(U')(\rho^U_ {U'}(f))$. 

\item Hier muss man sich klar machen, dass $\F(U)$ eine (abelsche) Gruppe/ein Ring/ein $R$-Modul ist, falls die $\F'(V)$ es sind und dass die Restriktionsabbildungen in der entsprechenden Kategorie liegen. Auch in b) muss gezeigt werden, dass die $\varphi(U)$ Morphismen aus der richtigen Kategorie sind.
\end{enumerate}

\textit{Anmerkung:} Nat\"urlich h\"atte man bei dieser Aufgabe alles noch ein wenig kategorieller formulieren k\"onnen und vor allem die UAE des inversen Limes mit ins Spiel bringen.
\end{Loes}

%-------------------------- Uebung 2 --------------------------
\newpage
\section{7. Mai 2012}
\setcounter{Aufg}{0}
\setcounter{Loes}{0}

Auf diesem Blatt bezeichne $R$ immer einen kommutativen Ring mit Eins und $k$ einen %algebraisch abgeschlossenen 
K\"orper.

\begin{Aufg}[4 Punkte]
 Beweise die Proposition 1.14 aus der Vorlesung:
Sei $f\colon X\to Y$ stetig, $\mathcal{F}$ eine Garbe auf $X$ und $\G$ eine Garbe auf $Y$.
Dann ist $f^{-1}$ linksadjungiert zu $f_\ast$ in der Kategorie der Garben, d.\,h. es gibt eine nat\"urliche Bijektion 
$$\Hom(f^{-1}\G, \F) \to \Hom(\G, f_\ast \F) \,.$$
\end{Aufg}

 
\begin{Aufg}[4 Punkte]
Zeige das folgende Lemma von Krull:
\begin{enumerate}%[topsep=0cm, parsep=0cm, label=\alph*)]
 \item Sei $S \subseteq R$ ein multiplikatives System und $I \subseteq R$ ein Ideal, das disjunkt zu $S$ ist.
Dann gibt es ein Primideal $\wp \subseteq R$, das $I$ enth\"alt und ebenfalls zu $S$ disjunkt ist.
 \item Es gilt: $$\bigcap_{\stackrel{\wp \textrm{ Primideal in } R}{I \subseteq \wp}} \wp = \sqrt{I}\,.$$
\end{enumerate}
\end{Aufg}

\begin{Aufg}[2 Punkte]
 Zeige:
\begin{enumerate}%[topsep=0cm, parsep=0cm, label=\alph*)]
 \item Jede \textbf{abgeschlossene, nicht leere,} irreduzible Teilmenge von $\Spec R$ hat genau einen generischen Punkt.
 \item Die irreduziblen Komponenten von $\Spec R$ entsprechen bijektiv den minimalen Primidealen in $R$.
\end{enumerate}
\end{Aufg}

\begin{Aufg}[6 Punkte]
 Bestimme $\Spec R$ f\"ur folgende Ringe $R$:
\begin{enumerate}%[topsep=0cm, parsep=0.1cm, label=\alph*)]
 \item $R = \Z/n\!\Z$ f\"ur $n\in \N$.
 \item $R = k[X]/(X^2)$.
 \item $R = k[X]_{(X)} = k[X]_S$ mit $S = k[X]\setminus(X)$.
 \item $R = \C[A]$, die von $A\in \C^{n\times n}$ im Matrizenring $\C^{n\times n}$ erzeugte $\C$-Unteralgebra.
 
 \textit{Hinweis: Hier kann die Lineare Algebra helfen.}
\end{enumerate}
\end{Aufg}

\textbf{Direkter Limes:}

$(I, \le)$ gerichtet $:\Leftrightarrow$ teilgeordnet und jede endliche Teilmenge hat obere Schranke

Gerichtetes System in Kategorie $\mathcal K$ zu $(I, \le)$ ist kovarianter Funktor von $(I, \le)$ nach $\mathcal K$. Der direkte Limes (falls er existiert) zu diesem gerichteten System ist ein $\varinjlim A_i \in \Ob(\mathcal K)$ zusammen mit $\psi_j: A_{j} \to \varinjlim A_i \in \Mor_{\mathcal K}(A_{j}, \varinjlim A_i)$,  sodass:\\
\begin{center}
$j\le k$
\begin{tikzpicture}[baseline = 0]
	\node (0) at (-1.25, 1.25) {$A_j$};
	\node (1) at (1.25, 1.25) {$A_k$};
	\node (2) at (0, -0.5) {$\varinjlim A_i$};
	\node (3) at (0, -1.75) {$B$};
	\node (4) at (0, 0.5) {$\schraffiert$};
	\node (5) at (-1, -0.5) {$\schraffiert$};
	\node (6) at (1, -0.5) {$\schraffiert$};
	
	\draw [->] (0) -- node[left]{$\psi_j$} (2);
	\draw [->] (1) -- node[right]{$\psi_k$} (2);
	\draw [->] (0) -- node[above]{$\rho_k^j$} (1);
	\draw [->] (2) -- node[right]{$\exists!$} (3);
	
	\draw [->] (0) to [out=225,in=165] node[left]{$h_j$} (3);
	\draw [->] (1) to [out=315,in=15] node[right]{$h_k$} (3);
\end{tikzpicture}
\begin{tikzpicture}[baseline = 0]
	\node (0) at (-1, 0.5) {$A_j$};
	\node (1) at (1, 0.5) {$A_k$};
	\node (2) at (0, -1) {$B$};
	\node (3) at (0, 0) {$\schraffiert$};

	\draw [->] (0) -- node[left]{$h_j$} (2);
	\draw [->] (1) -- node[right]{$h_k$} (2);
	\draw [->] (0) -- node[above]{$\rho_k^j$} (1);
\end{tikzpicture}
\end{center}
$\forall\  B \in \Ob(\mathcal K)$ zusammen mit $h_{j}: A_{j} \to B $ mit Diagramme kommutieren (siehe oben)\\
$\Rightarrow \exists! f: \varinjlim A_i \to B : f \circ \psi_k = h_k \forall\  k \in I$

\begin{Loes}
$X, Y$ R\"aume , $f: X \to Y$ stetig, $\calF$ Garbe auf $X$, $\calG$ Garbe auf $Y$ (von abelschen Gruppen)

\textbf{Erinnerung:}\begin{itemize}
\item direkte Bildgarbe: $\forall\   V \subseteq Y: f_{*}\calF(V) = \calF(f^{-1} V)$
\item Urbildgarbe: $\forall\  U\subseteq X: f^{-1} \calG(U) = \varinjlim\limits_{f(U) \subseteq V \in \Off(Y)} \calG(V)$
\end{itemize}
\emph{Zu zeigen:} $f^{-1}$ adjungiert zu $f_{*}$, das hei\ss t es gibt eine in $\calF$ und $\calG$ nat\"urliche Bijektion $\Hom(f^{-1}\calG, \calF) \to \Hom(\calG, f_*\calF)$

Sei $\alpha \in \Hom(f^{-1}\calG, \calF)$\\
$\Rightarrow f_*\alpha : f_*f^{-1}\calG \to f_*\calF$ durch $(f_*\alpha)(V) = \alpha (f^{-1}(V))$ $\leadsto$ es fehlt noch: $\calG \to f_{*}f^{-1}\calG$\\
Sei $V \in \Off(Y)$:
	\[ f_{*}f^{-1}\calG(V) = f^{-1}\calG(f^{-1}(V)) = \varinjlim\limits_{f(f^{-1}(V)) \subseteq W \in \Off(Y)} \calG(W)\]
da $f(f^{-1}(V)) \subseteq V$\\
Zum direkten Limes geh\"ort $\psi(V): \calG(V) \to \varinjlim\limits_{f(f^{-1}(V)) \subseteq W \in \Off(Y)} \calG(W)$.\\
Sei $\beta \in \Hom(\calG, f_*\calF) \Rightarrow f^{-1}\beta: f^{-1}\calG \to f^{-1} f_*\calF$. Sei $U\in \Off(X)$:
	\[f^{1}f_*\calF(U) = \varinjlim\limits_{f(U) \subseteq V \in \Off(Y)} f_*\calF(V) = \varinjlim\limits_{f(U) \subseteq V \in \Off(Y)}\calF(f^{-1}(U)) \]
$f^{-1}\calG(U) = \varinjlim\limits_{f(U) \subseteq V \in \Off(Y)} \calG(V)$\\
Seien $V', V'' \in \Off(Y)$, $f(U) \subseteq V' \subseteq V''$, $\exists! f^{-1}\beta$
\begin{center}
\begin{tikzpicture}[baseline = 0]
	\node (0) at (-2,2) {$\calG(V'')$};
	\node (1) at (2,2) {$\calG(V')$};
	\node (2) at (-2,0) {$f_*\calF(V'')$};
	\node (3) at (2,0) {$f_*\calF(V')$};
	\node (4) at (0,-2.5) {$\varinjlim\limits_{f(U) \subseteq V} f_*\calF(V)$};
	
	\node at (0, 1.25) {$\schraffiert$};
	\node at (0, -1) {$\schraffiert$};
	
	\draw[->] (0) -- node[above]{$\rho_{V'}^{V''}$} (1);
	\draw[->] (2) -- node[above]{$\rho_{V'}^{V''}$} (3);
	
	\draw[->] (0) -- node[left]{$\beta(V'')$} (2);
	\draw[->] (1) -- node[right]{$\beta(V')$} (3);
	
	\draw[->] (2) -- node[left]{$\widetilde \psi(V'')$}(4);
	\draw[->] (3) -- node[right]{$\widetilde \psi(V'')$}(4);
\end{tikzpicture}
$\Longrightarrow$
\begin{tikzpicture}[baseline = 0]
	\node (0) at (-2, 1) {$\calG(V')$};
	\node (1) at (2, 1) {$f^{-1}\calG(U)$};
	\node (2) at (0, -1) {$f^{-1}f_*\calF(U)$};

	\draw [->] (0) -- node[left]{$\widetilde \psi (V') \circ \beta(V')$} (2);
	\draw [->] (1) -- node[right]{$f^{-1}\beta$} (2);
	\draw [->] (0) -- node[above]{$\psi_{f^{-1}\calG}$} (1);
\end{tikzpicture}

\end{center}
Es fehlt noch: $\varphi_\calF: f^{-1}f_*\calF \to \calF$\\
Seien $V',V''$ wie eben. Dann: $U\subseteq f^{-1}(V') \subseteq f^{-1}(V'')$
\begin{center}
\begin{tikzpicture}[baseline = 0]
	\node (0) at (-2, 0.5) {$\calF(f^{-1}(V''))$};
	\node (1) at (2, 0.5) {$\calF(f^{-1}(V'))$};
	\node (2) at (0, -1) {$\calF(U)$};
	\node (3) at (0, 0) {$\schraffiert$};

	\draw [->] (0) -- node[left]{$\rho$} (2);
	\draw [->] (1) -- node[right]{$\rho$} (2);
	\draw [->] (0) -- node[above]{$\rho$} (1);
\end{tikzpicture}
\end{center}
$\Rightarrow \exists! \varphi_\calF(U): f^{-1}f_*\calF(U): \varphi_\calF \circ \tilde \psi = \rho$\\
Definiere:
\begin{equation*}\begin{array}{rlcccl}
	T_1:& \Hom(f^{-1}\calG, \calF) &\ni &\alpha &\mapsto &f_* \alpha \circ \psi_\calG \in \Hom(\calG, f_*\calF)\\
	T_2:& \Hom(\calG, f_*\calF) &\ni &\beta &\mapsto & \textcolor{red}{\text{[FEHLT]}}
\end{array}\end{equation*}
Sei $\alpha \in \Hom(f^{-1}\calG, \calF)$, $U\in \Off(X)$. Sei dazu $V\in \Off(Y)$ mit $f(U)\subseteq V$, also $U\subseteq f^{-1}(f(U)) \subseteq f^{-1}(V)$.
\begin{center}\begin{tikzpicture}
	\node (0) at (1,0) {$f^{-1}f_*\calF(U)$};
	\node (1) at (-4,0) {$\calF(f^{-1}(V)) = f_* \calF(V)$};
	
	\node (2) at (-4,3) {$f^{-1}\calG(f^{-1}(V)) = f_*f^{-1}\calG(V)$};
	\node (3) at (2,3) {$\calG(V)$};
	\node (4) at (7,3) {$f^{-1}\calG(U)$};
	
	\node (5) at (-1,-3) {$\calF(U)$};
	
	\node at (1.5,1.5) {$\schraffiert$};
	\node at (-2.5,1.5) {$\schraffiert$};
	\node at (1.5,4.5) {$\schraffiert$};
	\node at (-2,-1.5) {$\schraffiert$};
	\node at (4.5,0) {$\schraffiert$};
	
	\draw[->] (1) -- node[below]{$\tilde \psi(V)$} (0);
	\draw[->] (2) -- node[left]{$\alpha(f^{-1}(V))=f_*\alpha(V)$} (1);
	\draw[->] (3) -- node[above]{$\psi_\calG(V)$}(2);
	\draw[->] (3) -- node[above]{$\psi_{f^{-1}\calG}(V)$} (4);
	\draw[->, dashed] (3) -- (1);
	\draw[->, dashed] (4) -- node[below, sloped]{$f^{-1}(f_*\alpha \circ \psi_\calG)(U)$} (0);
	\draw[->] (0) -- node[right]{$\varphi_\calF(U)$} (5);
	\draw[->] (1) to [out=270,in=180] node[left]{$\rho_U^{f^{-1}(V)}$} (5);
	\draw[->] (2) to [out=45,in=135] node[above]{$\rho_U^{f^{-1}}(V)$} (4);
	\draw[->] (4) to [out=315,in=0] node[right]{$\hat \alpha(U)$}(5);
\end{tikzpicture}\end{center}
Zu zeigen: $\hat \alpha = \alpha$
\end{Loes}

\begin{Loes}
\begin{enumerate}%[topsep=0cm, parsep=2mm, label=\alph*)]
  \item Betrachte den kanonischen Ringhomomorphismus $\varphi\colon R \to R_S$. Sei $I'=\left(\varphi(I)\right)$ das von $\varphi(I)$ erzeugte Ideal in $R_S$, also $I'=\{\frac{f}{a}\in R_S\mid f\in I,\ a\in S\}$.

    Zun\"achst \"uberlegen wir uns, dass $I'\neq R_S$, also $1\notin I'$. Denn w\"are $1\in I'$, dann g\"abe es ein $f\in I$ und ein $a\in S$
    mit $1=\frac{f}{a}$ in $R_S$, d.h. es g\"abe ein
    \[s\in S\text{ mit }s(f-a)=0.\] 
    Dann w\"are $sf=sa$ sowohl in $I$ als auch in $S$ -- ein Widerspruch zur Disjunktheit!

    Somit ist $I'$ ein echtes Ideal und damit in einem maximalen Ideal $\mathfrak{m}'$ enthalten. Dann ist $\mathfrak{p}:=\varphi^{-1}(\mathfrak{m}')$ ein
    Primideal, enth\"alt $I$ und hat leeren Schnitt mit $S$, denn sonst w\"are f\"ur $s\in S\cap \mathfrak{p}$ das Bild $\varphi(s) \in \mathfrak{m}'$ eine Einheit in $R_S$.
  \item \begin{itemize}
         \item["`$\supseteq$"'] Ist $f \in \sqrt{I}$, so existiert ein $n \in \N$ mit $f^n \in I$. Folglich gilt f\"ur alle $\mathfrak{p} \in \Spec R$ mit $I \subseteq \mathfrak{p}$, dass $f^n \in \mathfrak{p}$, also auch $f \in \mathfrak{p}$. 
         \item["`$\subseteq$"'] Sei $a\in R\setminus\!\sqrt{I}$ und $S:=\{a^n\mid n\in\N_0\}$. Dann ist $S$ ein multiplikatives System und $I\cap S=\emptyset$. Nach a) gibt es dann ein Primideal $\mathfrak{p}$ mit $\mathfrak{p} \supseteq I$ und $\mathfrak{p} \cap S=\emptyset$. Damit liegt $a$ nicht im Schnitt \"uber alle Primideale, die $I$ enthalten.
        \end{itemize}
\end{enumerate}
\end{Loes}

\begin{Loes}
$R$ kommutativer Ring mit Eins.
\begin{enumerate}[a)]
\item
	$V \ne \emptyset$, abgeschlossen, irreduzibel, $V \subseteq \Spec R$\\
	$\Rightarrow \underbrace{V = V(I)}_{\{\p \in \Spec R | I \subseteq \p\}}$, $I$ Primideal\\
	$\Rightarrow \overline{\{I\}} = V(I(\{I\})) = V(I) = V$\\
	Seien $\p,\q \in \Spec R$ mit $\overline{\{\p\}} = \overline{\{\q\}}$\\
	$\q \in \overline{\{\p\}} = V(\p) = \{ \tilde \p \in \Spec R | \p \subseteq \tilde \p \} \Rightarrow \q \subseteq \p$\\
	Analog $\q \subseteq \p \Rightarrow \p = \q$
\item
	Seien $\p, \q \in \Spec R$\\
	$\p \subseteq \q\Leftrightarrow \q \in \overline{\{\p\}} \Leftrightarrow \overline{\{\q\}} \subseteq \overline{\{\p\}}$
\end{enumerate}\end{Loes}

\begin{Loes}
$\Spec: \underline{\text{Ringe}} \to \underline{\text{Top.}}$ kontravarianter Funktor, $\alpha: R \to R'$, $\Spec(\alpha):\left\{\begin{array}{ccc}\Spec R' \to \Spec R \\ \p \mapsto \alpha^{-1}(\p)\end{array}\right.$\\
Sei $R$ Ring, $I\subseteq R$ ein Ideal und $\pi: R \twoheadrightarrow R/I$ die kanonische Projektion, dann ist $\Spec \pi \colon \Spec R/I \hookrightarrow \Spec R$ injektiv.
	\[ \Bild(\Spec \pi) = \{\p \in \Spec R | I \subseteq \p\} \]
\begin{enumerate}[a)]
\item
	$R= \FakRaum{\Z}{n\Z}$, $I:=n\Z$, $\pi: \Z \to \FakRaum{\Z}{n\Z}$, $\Bild(\Spec \pi) = \{(p) \in \Spec \Z | p \in \IP, p|n\}$, $x,y\in \FakRaum{\Z}{n\Z}: D(x) \subseteq D(y) \Leftrightarrow \exists z \in \FakRaum{\Z}{n\Z}, m \in \N: x^m = z\cdot y$\\
	\textcolor{gray}{$\left( x \in \FakRaum{\Z}{n\Z} , D(x) = \p \in \Spec R | x\in \p \right)$}\\
	$\rho_{D(x)}^{D(y)}: R_y \to R_x, \frac{a}{y^k}\mapsto \frac{a\cdot z^k}{x^{m\cdot k}}$

 \item Es gilt $\Spec(k[X]) = \{(0)\} \cup \{(f) \mid f \in k[X] \textrm { irreduzibel}\}$.
 Sei $R = k[X]/(X^2)$, $I \coloneqq (X^2)$ und $\pi\colon k[X] \twoheadrightarrow k[X] / I$. Dann ist 
 $$\Bild(\Spec \pi) = \{(f) \in \Spec k[X]/(X^2) \mid f \in k[X] \textrm{ irreduzibel}, (X^2) \subseteq (f) \} = \{(X)\}$$ und somit
 $Y \coloneqq \Spec R = \{(X)\}$ einelementig. Die Strukturgarbe ist dann offensichtlich:
 $$\calO_Y(Y)=R \stackrel{\rho^Y_\emptyset \textrm{ eindeutig}}{\longrightarrow} \{p\} = \calO_Y(\emptyset)$$ 
 
 \item Nun sei $R = k[X]_{(X)} = k[X]_S$ mit $S= k[X] \setminus (X)$. Der Ring ist als Teilring von $k(X)$ nullteilerfrei und somit ist $(0)$ ein Primideal. Au\ss erdem ist der Ring lokal, mit maximalem Ideal $\mathfrak{m} = R \setminus R^{\times} = (\frac{X}{1})$. Sei nun $\emptyset \neq \mathfrak{p} \in \Spec R$ und $f \in \mathfrak{p}$. Dann schreiben wir $f = {\frac{X}{1}}^e \cdot \frac{a}{b}$ mit $a \in k[X], b \in k[X] \setminus (X), X \nmid a$. Dann ist $a \in S$ und $\frac{a}{b}$ somit offensichtlich invertierbar, also gilt $\frac{X}{1}^e \in \mathfrak{p}$. Da $\mathfrak{p}$ prim ist, ist $\frac{X}{1} \in \mathfrak{p}$ und wegen $\mathfrak{p} \subseteq (\frac{X}{1})$ folgt $\mathfrak{p} = (\frac{X}{1})$. Folglich besteht das Spektrum von $R$ aus genau zwei Punkten: $Y \coloneqq \Spec k[X]_{(X)} = \{(0), (\frac{X}{1})\}$. Die offenen Mengen sind $\emptyset, D(\frac{X}{1}) = \{(0)\}$ und $Y$. Auch hier ist die Strukturgarbe leicht anzugeben:
 
 \begin{center}
\begin{tikzpicture}
   \matrix (m) [matrix of math nodes, row sep=1em, column sep=0em]
	{ \calO_Y(Y) & = & R & = & k[X]_{(X)}\\
	  \calO_Y(D(\frac{X}{1})) & = & R_{\frac{X}{1}} & = & \Quot(k[X]) & = & k(X)\\
	  \calO_Y(\emptyset) &=& \{p\}\\
	 };
  \path[right hook->,font=\scriptsize]
	(m-1-5) edge (m-2-7);
  \path[->, font=\scriptsize]
	(m-1-1) edge [bend right=60] node[auto,swap] {!} (m-3-1)
	(m-2-7) edge node[auto] {!} (m-3-3);
\end{tikzpicture}
\end{center}
 
 
 \item Wir betrachten den Einsetzungshomomorphismus $\Psi_A: \C[X] \twoheadrightarrow \C[A]$. Der Kern von $\Psi_A$ ist $(\mathfrak{m}_A)$, wobei $\mathfrak{m}_A = \prod_{i=1}^e (X-\lambda_i)^{r_i}$ das Minimalpolynom von $A$ ist, mit den Eigenwerten $\lambda_i$ von $A$. Nach dem Homomorphiesatz ist $\C[A] \cong \C[X]/(\mathfrak{m}_A)$. Unsere  Vor\"uberlegungen liefern dann, genau wie in a) und b), 
 $$\Bild(\Spec \Psi_a) = \{ \mathfrak{q} \in \Spec \C[X] \mid (\mathfrak{m}_A) \subseteq \mathfrak{q} \} = \{(X- \lambda_i) \mid \lambda_i \textrm{ ist EW von } A \}$$
 und somit $\Spec \C[A] = \{(A- \lambda_i) \mid \lambda_i \textrm{ ist EW von } A\}$.
\end{enumerate}

\end{Loes}

%-------------------------- Uebung 3 --------------------------
\newpage
\section{15. Mai 2012}
\setcounter{Aufg}{0}
\setcounter{Loes}{0}

Auf diesem Blatt bezeichne $R$ immer einen kommutativen Ring mit Eins.

\begin{Aufg}[4 Punkte]\label{a3.1}
 Sei $\left(X, \mathcal{O}_X \right)$ ein Schema und $(\Spec R,\calO_{\Spec R})$ ein affines Schema.
Zeige, dass die Zuordnung 
$$\begin{array}{rcl}
   \Mor(X, \Spec R) &\to& \Hom(R, \mathcal{O}_X(X))\,,\\
   (\varphi, \varphi^\sharp) &\mapsto& \varphi^\sharp(\Spec R)
  \end{array}
$$ 
bijektiv ist.
\end{Aufg}


\begin{Aufg}[6 Punkte]
 \begin{enumerate}%[topsep=0cm, parsep=0cm, label=\alph*)]
 \item Zeige, dass $(\Spec \Z, \calO_{\Spec \Z})$ ein terminales Objekt in der Kategorie der affinen Schemata ist, dass es also zu jedem affinen Schema $(\Spec R,\calO_{\Spec R})$ genau einen Morphismus $(\varphi, \varphi^\sharp)$ von lokal geringten R\"aumen nach $(\Spec \Z, \calO_{\Spec \Z})$ gibt. 
  \item Finde jeweils einen Ring $R$, so dass der Morphismus $(\varphi, \varphi^\sharp)$ aus a)
 \begin{itemize}
  \item genau einen Punkt als Bild hat,
  \item genau zwei Punkte als Bild hat,
  \item unendlich viele Punkte als Bild hat, aber nicht surjektiv ist,
  \item surjektiv ist.
 \end{itemize}
\end{enumerate}
\end{Aufg}


\begin{Aufg}[4 Punkte]
Zeige:
\begin{enumerate}%[topsep=0cm, parsep=0cm, label=\alph*)]
 \item Ein Element $e\in R$ ist genau dann idempotent (d.h. es gilt $e^2 = e$), wenn $1-e$ idempotent ist.
 \item Gibt es in $R$ zwei Elemente $e_1$, $e_2 \not\in R^\times$, so dass $e_1+e_2 =1$ und $e_1\cdot e_2$ nilpotent ist, so enth\"alt $R$ mindestens drei idempotente Elemente.
 
 \textit{Hinweis: Betrachte das Ideal $I_n = (e_1^n, e_2^n)$ f\"ur $n\in \N$.}
 \item $\Spec R$ ist genau dann zusammenh\"angend, wenn $R$ h\"ochstens zwei idempotente Elemente enth\"alt.
 \item Gib einen Ring $R$ an, sodass $\Spec R$ nicht zusammenh\"angend ist.
\end{enumerate}
\end{Aufg}

\begin{Loes}
$(X, \calO_X)$ Schema, $(\Spec R, \calO_{\Spec R})$ affines Schema\\
\emph{Zu zeigen:} $\begin{array}{rcl} \Mor(X, \Spec R) &\to& \Hom(R, \calO_X(X))\\ (\varphi, \varphi^\#) &\mapsto& \varphi^\# (\Spec R) \end{array}$ ist bijektiv

Konstruiere die Umkehrabbildung. Sei $\Phi: R\to \calO_X(X)$ Ringhomomorphismus. Konstruiere $\varphi: X\to \Spec R$ stetig (bez\"uglich Zariski-Topologie).\\
Sei $p \in X$, $R \xrightarrow{\Phi} \calO_X(X) \xrightarrow{\Psi_X} \calO_{X,p} = \varinjlim\limits_{p\in U \in \Off(X)} \calO_X(U)$\\
$\calO_{X,p}$ ist lokaler Ring mit maximalem Ideal $m_p$. Setze $\varphi(p):=(\Psi_X \circ \Phi)^{-1}(m_p)$.

\emph{Behauptung:} $\varphi$ ist stetig\\
\emph{denn:} Es reicht das f\"ur Basismengen zu \"uberpr\"ufen $\leadsto$ auf affinen Teilen reicht.

Sei $\Spec S \subseteq X$, $p\in S\Spec S$
\begin{center}\begin{tikzpicture}
	\node (0) at (-5, 1) {$R$};
	\node (1) at (-3, 1) {$\calO_X(X)$};
	\node (2) at (3, 1) {$\calO_{X, p}$};
	\node (3') at (-3, -1) {\textcolor{white}{$\calO_X(\Spec S)$}}; %auf diesen Knoten wird der Pfeil gezeichnet
	\node (3) at (-5, -1) {$S = \calO_{\Spec S}(\Spec S) = \calO_X(\Spec S)$}; %dieser Knoten wird zu sehen sein
	\node (4) at (3, -1) {\textcolor{white}{$\calO_{\Spec S|_p}$}};
	\node (4') at (3.5, -1) {$\calO_{\Spec S|_p} = S_p$};
	
	\node at (0, 0) {$\schraffiert$};
	
	\draw[->] (0) -- node[above]{$\Phi$} (1);
	\draw[->] (1) -- node[above]{$\Psi_X$} (2);
	\draw[->] (3) -- node[below]{$\Psi_{\Spec S}$} (4');
	\draw[->] (1) -- node[left]{$\rho_{\Spec S}^X$} (3');
	\draw[->] (2) -- node[right]{$\cong$} (4);
\end{tikzpicture}\end{center}
$(\Psi_X \circ \Phi)^{-1}(m_p) = (\Psi_{\Spec S} \circ \rho_{\Spec S}^X \circ \Phi)^{-1} (p) = \Phi^{-1} (\rho_{\Spec S}^X (p))$ $\Rightarrow $ auf allen affinen Teilen gilt: $\varphi|_{\Spec S}(p) = \Phi^{-1}(p)$ $\leadsto$ nach Bemerkung \ref{2.4} stetig.

Konstruiere Garbenmorphismus $\varphi^\#: \calO_{\Spec R} \to \varphi_*\calO_X$. Sei $f\in R$, $D(f) = \{p \in \Spec R | f\notin p\} \subseteq \Spec R$.
\begin{center}(*)\begin{tikzpicture}[baseline=0]
	\node (0) at (-3, 1) {$R$};
	\node (1) at (3, 1) {$\calO_X(X)$};
	\node (2') at (-3, -1) {\textcolor{white}{$R_f$}};
	\node (2) at (-4.25, -1) {$\calO_{\Spec R}(D(f))=R_f$};
	\node (3') at (3, -1) {\textcolor{white}{$\varphi_*\calO_X(D(f))$}};
	\node (3) at (4.75, -1) {$\varphi_*\calO_X(D(f)) = \calO_X(\varphi^{-1}(D(f))$};

	\node (4) at (-4, 0.75) {$v$};
	\node (5) at (-4, -0.5) {$\frac{v}{1}$};
	
	\node at (0, 0) {$\schraffiert$};
	
	\draw[->] (0) -- node[above]{$\Phi$} (1);
	\draw[->] (0) -- node[right]{$i$} (2');
	\draw[->] (1) -- node[right]{$\rho_{\varphi^{-1}(D(f))}^X$} (3');
	\draw[->, dashed] (2) -- node[below]{$h = \varphi^\#(D(f))$} (3);
	\draw[|->] (4) -- (5);

\end{tikzpicture}
\\ \ \\$S=\{f^n|n\in \N\}$\end{center}
\emph{UAE Lokalisierung:} Falls $(\rho \circ \Phi)(S) \subseteq \varphi_*\calO_X(D(f))^\times \Rightarrow \exists! h : R_f \to \varphi_*\calO_*(D(f))$\\
$\rho \circ \Phi = h \circ i$

$\varphi^{-1}(D(f)) = \{q \in X | \varphi(q) \in D(f)\} = \{q \in X | f \notin \varphi(q)\}$\\
F\"ur $q \in \Spec S \subseteq X$ gilt:
	\[f \notin \varphi(q) = \Phi^{-1}((\rho_{\Spec S}^X)^{-1}(q)) \Leftrightarrow \rho_{\Spec S}^X (\Phi(f)) \in q \Leftrightarrow q \in D(\rho_{\Spec S}^X(\Phi(f))) = S_{\rho_{\Spec S}^X (\Phi (f))} \]
$\Rightarrow \rho_{\varphi^{-1}(D(f)) \cap \Spec S}^X (\Phi(f))$ ist invertierbar.\\
$\calO_X$ Garbe $\Rightarrow $ Inverse von $\varphi(\Phi(f))$ setzen sich zu \quot{globalem} Inversen zusammen.\\
$\xRightarrow{\text{UAE}} \exists! \varphi^\#(D(f)) : \calO_{\Spec R}(D(f)) \to \varphi_*\calO_X(D(f))$ mit (*) $\leadsto$ passt auch auf $D(f) \cap D(g)$ zusammen.\\
$\xRightarrow[\text{Aufgabe 3}]{\text{Blatt 1}}$ eindeutige Forsetzung Garbenmorphismus $\varphi^\#: \calO_{\Spec R} \to \varphi_*\calO_X$

\emph{Zeige:} $\varphi^\#$ induziert lokale Ringhomomorphismen auf Halmen.

F\"ur $q \in \Spec S \subseteq X$ gilt $\calO_{X,q} = \calO_{\Spec S, q} = S_q$. Setze $p:= \varphi(q) \subseteq \Spec R$.\\
$\Rightarrow \varphi_p^\#(p): \left\{ \begin{array}{rcl} \calO_{\Spec R, p} = R_p &\to& S_q \\ \frac{a}{b} &\mapsto& \frac{\rho_{Spec S}^X (\Phi(a))}{\rho_{Spec S}^X (\Phi(b))} \end{array}\right.$

\emph{Noch zu zeigen:} $f^\#(\Spec R) = \Phi$

$\Spec R = D(1) \checkmark$\\
$(\varphi, \varphi^\#) \to \varphi^\#(\Spec R) \to (\varphi', (\varphi^{\#})' )$ ist Identit\"at.
\end{Loes}

\begin{Loes}\begin{enumerate}[a)]
\item
	$(\Spec \Z, \calO_{\Spec \Z})$ ist terminales Objekt in der Kategorie der Schemata (\quot{affin} ist \"uberfl\"ussig). Ist $(X, \calO_X)$ Schema, $\exists! \Phi:\left\{\begin{array}{rcl} \Z &\to& \calO_X(X)\\ 1 &\mapsto& 1\end{array}\right.$ Ringhomomorphismus $\xRightarrow{\text{Aufgabe 1}} \exists! (\varphi, \varphi^\#): X \to \Spec \Z$ Schemamorphismus
\item
	\begin{description}[\setlabelstyle{\itshape}]
	\item[Genau ein Punkt im Bild:]
		$R=K$ K\"orper $\Rightarrow \Spec K= \{(0)\}$\\
		$\Phi:\left\{ \begin{array}{rcl} \Z &\to& K \\ 1 &\mapsto& 1 \end{array}\right.$, $\Phi^{-1}((0)) = (p) \in \Spec \Z$, $p = \chara K$, $\varphi((0)) = (p)$\\
		$\varphi_p^\#: \left\{\begin{array}{rcl} \calO_{\Spec \Z, p} \cong \Z_p &\to& \calO_{\Spec K, (0)} = K_{(0)} = k \\ \frac{f}{g} &\mapsto& \FakRaum{\Phi(f)}{\Phi(g)}\end{array}\right.$
	\item[Genau zwei Punkte:]
		$R=\Z_{(p)}$, $\Spec R = \{ (0), (p)\}$\\
		$\Phi: \left\{\begin{array}{rcl} \Z &\to& \Z(p) \\ 1 &\mapsto& \frac{1}{1}\end{array}\right.$\\
		$\begin{array}{l} \varphi((0)) = \Phi^{-1}((0)) = (0) \\ \varphi((p)) = (p)\end{array} \Rightarrow $ Bild zweielementig
	\item[unendliches Bild, nicht surjektiv:]
		Bemerkung: Ist $R$ kommutativer Ring mit $1$, $S\subseteq R$ multiplikatives System, $\Phi: R \to R_S$, $\Spec \Phi: \Spec R_S \to \Spec R$ $\Rightarrow \Bild(\Spec \Phi) = \{p \in \Spec R | p \cap S = \emptyset\}$. $\Spec \Phi$ ist Bijektion auf das Bild (sogar Hom\"oomorphismus).
		
		$R= \Z_n, n\in \N$\\
		$\Rightarrow \Bild(\Spec \Phi) = \{(p) \in \Spec \Z |p \in \IP, p \nmid n\}$\\
		$\Rightarrow \Bild(\Spec \Phi) \ne \Spec \Z$ f\"ur $n \ge 2$
	\item[surjektiv:]$\id: \Z \to \Z$\\
		$\id: \Spec\Z \to \Spec\Z$
	\end{description}
\end{enumerate}\end{Loes}

\begin{Loes}\begin{enumerate}[a)]
\item
	Sei $e^2 = e$ dann ist $(1-e)^2 = 1 -2e + e^2 = 1-e$. Mit $1-(1-e) = e$ folgt die andere Richtung.
\item
	Seien $e_1,e_2 \in R\setminus R^{\times}$ mit $e_1 + e_2 =1$, $e_1e_2$ nilpotent $\Rightarrow R$ enth\"alt mindestens drei idempotente Elemente.
	
	$0$, $1$ sind idempotent\\
	$I_n = (e_1^n, e_2^n)$, $n\in \N$\begin{enumerate}[\textbullet]
	\item
		$\sqrt{I_n} = R$, denn: $e_1, e_2 \in \sqrt{I_n} \Rightarrow 1= e_1 + e_2 \in \sqrt{I_n}$
	\item
		$I_n = R$, denn sonst: $I_n \subseteq m$, $m$ maximales Ideal $\Rightarrow \sqrt{I_n} \subseteq m$ $\lightning$ zu $\sqrt{I_n} = R$
	\end{enumerate}
	$\Rightarrow \forall\  n \in \N \exists  \alpha_n, \beta_n \in R: \alpha_n e_1^n + \beta_n e_2^n = 1$\\
	$e_1e_2$ nilpotent $\Rightarrow \exists k \in \N: (e_1e_2)^k = 0$\\
	$x = \alpha_ke_1^k = 1 - \beta_ke_2^k$ ist idempotent\\
	$x^2 = \alpha_ke_1^k (1- \beta_ke_2^k) = \alpha_k e_1^k - \alpha_k \beta_k (\underbrace{e_1e_2}_{=0})^k = x$\\
	Es bleibt zu zeigen, dass $x$ weder $1$ noch $0$ ist. W"are $x=0$, so w"are $\beta e_2^k = 1$, also $e_2 \in R^\times$, was im Widerspruch zur gegenteiligen Voraussetzung steht. Analog folgt aus $x = \alpha e_1^k = 1$, dass $e_1 \in R^\times$.
  
  \item Sei zun"achst $\Spec R$ zusammenh"angend. Wir nehmen an, es g"abe drei verschiedenen idempotente Elemente, $0,1$ und $x$. Nach a) ist dann auch $1-x$ idempotent. Setze $V_1 = V(x)$ und $V_2 = V(1-x)$. Die Mengen sind abgeschlossen (klar) und nichtleer: 
  W"are $V_1 = \emptyset$, so w"are $x$ in keinem echten Ideal enthalten, also invertierbar und wegen $x^2 = x$ gleich $1$. Genauso folgt aus $V_2 = \emptyset$, dass $1-x\in R^{\times}$ und mit $(1-x)^2 = (1-x)$, dass $x = 0$.

  Angenommen es g"abe ein $\p \in V(x)\cap V(1-x)$. F"ur dieses g"alte $x$ und $1-x \in \p$, also auch $1\in \p$, ein Widerspruch. Andererseits ist $V(x)\cup V(1-x) = V(x(1-x))= V(0) = \Spec(R)$. Damit haben wir eine disjunkte Zerlegung von $\Spec(R)$ in abgeschlossene, nichtleere Teilmengen gefunden, im Widerspruch zur Voraussetzung. Also gibt es h"ochstens zwei idempotente Elemente in $R$.

  Nun sei $\Spec(R)$ nicht zusammenh"angend. Wir finden ein drittes idempotentes Element. Sei $\Spec(R) = V_1 \cup V_2$ eine Zerlegung in abgeschlossene, disjunkte, nichtleere Teilmengen. Es ist $V_1 = V(I_1)$ und $V_2 = V(I_2)$, f"ur Ideale $I_1, I_2 \subseteq R$. Da ihr Durchschnitt $V(I_1) \cap V(I_2) = V(I_1+I_2) = \emptyset = V(1)$ ist, folgt $I_1+I_2 = R$, und es gibt $e_1\in I_1$, $e_2\in I_2$ mit $e_1+e_2 = 1$. Andererseits ist weder $e_1$ noch $e_2$ eine Einheit, denn beide Verschwindungsmengen sind nichtleer. Au\ss erdem ist $V(I_1)\cup V(I_2) = V(I_1\cdot I_2) = \Spec(R) = V(\sqrt{0})$ gleich der Verschwindungsmenge des Nilradikals. Also folgt $I_1\cdot I_2 \subseteq \sqrt{0}$ und damit ist $e_1e_2$ nilpotent.

  \item Es sei $R$ ein beliebiger Ring mit $1$ und $S= R\times R$ mit der komponentenweisen Verkn"upfung. Dann ist $(1,0)$ ein idempotentes Element in $S$, dass weder $(0,0)$ noch $(1,1)$ ist. Also ist $\Spec S$ nach c) nicht zusammenh"angend.
\end{enumerate}\end{Loes}

%-------------------------- Uebung 4 --------------------------
\newpage
\section{22. Mai 2012}
\setcounter{Aufg}{0}
\setcounter{Loes}{0}

\begin{Aufg}[4 Punkte]
 Seien $R$, $S$ Ringe, $X\coloneqq\textup{Spec}(R)$, $Y\coloneqq\textup{Spec}(S)$ sowie $\varphi \colon R \to S$ ein Ringhomomorphismus und $g \colon Y \to X$ der von $\varphi$ induzierte Schemamorphismus. Zeige:
\begin{enumerate}%[topsep=0cm, parsep=0cm, label=\alph*)]
\item Ein Element $f \in R$ ist genau dann nilpotent, wenn $D(f)$ leer ist.
\item Der Ringhomomorphismus $\varphi$ ist genau dann injektiv, wenn der induzierte Morphismus $g^\sharp \colon \mathcal{O}_X \to g_* \mathcal{O}_Y$ injektiv ist. Ist das der Fall, so ist $g$ dominant, d.h. $g(Y)$ ist dicht in $X$.
\item Ist $\varphi$ surjektiv, so ist $g$ ein Hom"oomorphismus von $Y$ auf eine abgeschlossene Teilmenge von $X$ und $g^\sharp \colon \mathcal{O}_X \rightarrow g_* \mathcal{O}_Y$ ist surjektiv.
\end{enumerate}
\end{Aufg}

\begin{Aufg}[4 Punkte]
 Finde einen Ring $R$ und eine Idealgarbe $\mathcal{I}$ auf $\Spec(R)$, die nicht quasikoh"arent ist (und beweise, dass du solche $R$ und $\mathcal{I}$ gefunden hast).

\textit{Hinweis: Diskrete Bewertungsringe k"onnen helfen.}
\end{Aufg}

\begin{Aufg}[4 Punkte]
 Beweise die Proposition 4.5 aus der Vorlesung:

Eine Idealgarbe $\mathcal{I}$ auf einem Schema $X$ ist genau dann quasikoh"arent, wenn es eine offene "Uberdeckung $(U_i)_{i \in M}$ von $X$ durch affine Schemata $U_i$ gibt, so dass $\mathcal{I}|_{U_i}$ f"ur jedes $i \in M$ quasikoh"arent ist.
\end{Aufg}

\begin{Loes}
\begin{enumerate}%[topsep=0cm, parsep=2mm, label=\alph*)]
  \item[c)] Ist $\varphi\colon R \to S$ surjektiv, so ist $S \cong R/I$ mit $I = \Kern(\varphi)$. Bei der L"osung von Aufgabe 4 auf \"Ubungsblatt 2 haben wir bereits eingesehen, dass $g = \Spec(\varphi)$ injektiv ist und dass $\Bild(g) = \{ \p \in \Spec R \mid I \subseteq \p \} = V(I)$. Die Umkehrabbildung $g^{-1} \colon V(I) \to \Spec(S), \p \mapsto \varphi(\p) = \p + I$ ist folglich wohldefiniert.
  
  Es bleibt zu zeigen, dass $g^{-1}$ stetig ist. Sei dazu $V(J) \subseteq \Spec S$ abgeschlossen. Dann ist 
  $$ \begin{array}{rcl}
        (g^{-1})^{-1}(V(J))
  = g(V(J))
  &=& g(\{\q \in \Spec S \mid J \subseteq \q \})\\
  &=& \{ \p \in V(I) \mid J \subseteq g^{-1}(\p) = \varphi(\p)\}\\
  &=& \{p \in \Spec R \mid \varphi^{-1}(J) \subseteq \p \} = V(\varphi^{-1}(J))
     \end{array}
  $$
  abgeschlossen.
  Somit ist $g$ ein Hom"oomorphismus auf $V(I)$.
  
  Nun sollte noch gezeigt werden, dass $g^\sharp$ surjektiv ist, falls $\varphi$ surjektiv ist. Nach Bemerkung 1.10 aus der Vorlesung sollten wir dazu zeigen, dass f"ur alle $\p \in \Spec R$ die induzierte Abbildung $g^\sharp_{\p}$ surjektiv ist.
  
  Sei zun"achst $\p \notin V(I)$. Dann gibt es ein offenes $U \subseteq X$ mit $\p \in U$ und $U \cap V(I)= \emptyset$. Dann ist $g_\ast \mathcal{O}_Y(U) = \mathcal{O}_Y(g^{-1}(U)) = \mathcal{O}_Y(\emptyset) = \{0\}$ und somit auch $(g_\ast \mathcal{O}_Y)_{\p} = \{0\}$. Folglich ist $g_{\p}^\sharp\colon \mathcal{O}_{X,\p} \to (g_\ast\mathcal{O}_Y)_{\p}$ surjektiv. 
  
  Ist $\p \in V(I)$, dann ist $(g_\ast \mathcal{O}_Y)_{\p} = \varinjlim\limits_{\p \in U} \mathcal{O}_Y(g^{-1}(U)) \cong \varinjlim\limits_{\q \in V} \mathcal{O}_Y(V) = \mathcal{O}_{Y,\q}$ mit $\q = g^{-1}(\p)$. Der Isomorphismus zwischen den Halmen kommt daher, dass g ein Hom"oomorphismus auf $V(I)$ ist und $\p \in V(I)$.
  Weiter gilt $\mathcal{O}_{Y,\q} = S_{\q} \cong (R/I)_{\p+I} \cong R_{\p}/I  R_{\p}$. Mit der Identifizierung $S \cong R/I$ wird $g^\sharp_{\p}$ zur kanonischen Projektion $R_{\p} \to R_{\p}/I R_{\p}$ und ist damit surjektiv.
\end{enumerate}
\end{Loes}

\begin{Loes}

\end{Loes}

\textbf{Definition \ref{4.4}} (Pr\"azisierung)
\begin{enumerate}[a)]\item
Eine Garbe $\calI$ von abelschen Gruppen auf $X$ (Schema $(X, \calO_X)$, Restriktionabbildung $\rho$) hei\ss t \deftermspec{Idealgarbe}{Garbe!Ideal-}, wenn $\forall\  U \in \Off(X) \calI(U) $ Ideal in $\calO_X(U)$ ist und die Restriktionsabbildungen $\tilde \rho$ $\calO_X$-linear sind:
	\[ \forall\  U'\subseteq U \in \Off(X), r\in \calO_X(U), i \in \calI(U): \tilde{\rho}_{U'}^U (r, i) = \underbrace{\rho_{U'}^U(r)}_{\in \calO_X(U')} \cdot \tilde{\rho}_{U'}^U (i) \]
\end{enumerate}

\begin{Loes}
$\calF$ Idealgarbe auf Schema $(X, \calO_X)$ mit Restriktionsabbildung $\tilde \rho$ beziehungsweise $\rho$.

\emph{Behauptung:} $\calF$ quasikoh\"arent $\Leftrightarrow \exists$ offene, affine \"Uberdeckung $X = \bigcup\limits_{i\in I} U_i: \calF|_{U_i}$ quasikoh\"arent\begin{twosidedproof}
\proofforward $\checkmark$ nach Definition

\proofreverse
Schreibweise: $R$ Ring, $\calJ \subseteq R$ Ideal definiert quasikoh\"arente Idealgarbe auf $Y = \Spec R$ durch $\tilde{\calJ}(U) := \rho_U^Y(\calJ) \cdot \calO_Y(U)$\\
(*) $U_i \cong \Spec R_i$, $\calF|_{U_i}$ quasikoh\"arent, das hei\ss t $\exists \calJ_i \subseteq R_i$ Ideal mit $\calF|_{U_i} \cong \tilde{\calJ}_i$.

Verfeinern der \"Uberdeckung erh\"alt (*) $\Rightarrow$ \OE bilden $U_i$ eine Basis der Topologie.\\
\emph{Zu zeigen:} $\forall\  \hat U \subseteq X$ offen, affin: $\calF_{\hat U}$ quasikoh\"arent\\
$\hat U = \bigcup\limits_{i\in I'} U_i \Rightarrow $ \OE $X = \Spec R$, also affin und \OE $U_i = D(g_i) \subseteq R$ mit $g_i \in R$ $\leadsto$ Insbesodnere $\calO_X(U_i) = R_g$, $U_i \cong \Spec R_{g_i}$, $\calJ := \calF(X)$

\textbf{Voraussetzung:} $\calF|_{U_i} \cong \tilde{\calJ}_i$, \emph{zu zeigen:} $\calF \cong \tilde \calJ$

Definiere Garbenmorphismus $\alpha: \tilde \calJ \to \calF$ durch
	\[ \text{f\"ur } f\in R: \alpha(D(f)): \begin{array}{rcl} \tilde{\calJ}(D(f)) = \calJ \cdot R_f &\to& \calF(D(f)) \subseteq R_f \text{ Ideal} \\ \frac{a}{f^n} &\mapsto& \frac{1}{f^n} \tilde{\rho}_{D(f)}^X(\alpha) \end{array} \]

\textbf{Behauptung 1:} $\forall\ i\in I: \alpha(\underbrace{D(g_i)}_{U_i})$ ist Isomorphismus\\
$\xRightarrow[\text{Aufg. }3\text{b})]{\text{Blatt }1} \alpha$ Garbenisomorphismus

\textbf{Behauptung 2:} $\forall\ s \in \calF(X)$ mit $\tilde \rho_{D(f)}^X(s) = 0$ f\"ur ein $f\in R$\\
$\Rightarrow \exists n \ge 0: f^n\cdot s = 0$ in $R$

\textbf{Behauptung 3:} Sei $f \in R$\\
$\forall\ t \in \calF(D(f)) \exists n \ge 0: \exists s \in \calF(X): \tilde \rho_{D(f)}^X(s) = f^n(f)$

\textbf{Beweis 1:}\begin{description}[\setlabelstyle{\itshape}]
\item[\textbullet $\alpha(U_i)$ injektiv:] Sei $\frac{a}{g_i^n} \in \tilde \calJ(U_i)$, also $a \in \calJ$, $n\in \N$ mit $\underbrace{\alpha_i(U_i)(\frac{a}{g_i^n})}_{\frac{1}{g_i^n}\cdot \tilde \rho_{U_i}^X(a)} = 0$ in $R_{g_i}$.\\
	$\Rightarrow \exists m \in \N: \underbrace{\tilde \rho_{U_i}^X(a) \cdot g_i^m}_{\tilde \rho_{U_i}^X(a \cdot g_i^m)} = 0$ in $R$\\
	$\xRightarrow{\text{Beh. }2} \exists m' \in \N: g_i^{m'} \cdot a = 0$ in $R$\\
	$\Rightarrow \frac{a}{g_i^n} = 0$ in $R$
\item[\textbullet $\alpha(U_i)$ surjektiv:] Sei $t \in \calF(U_i) \xRightarrow{\text{Beh. }3} \exists n \ge 0: \exists s \in \calF(X): \tilde \rho_{U_i}^X(s) = g_i^n \cdot t$\\
	$\Rightarrow \alpha(U_i)(\frac{s}{g_i^n}) = \frac{1}{g_i^n} \cdot \tilde \rho_{U_i}^X(s) = t$
\end{description}

\textbf{Beweis 2:}

$s \in \calF(X)$, $\tilde \rho_{D(f)}^X(s) = 0$ f\"ur ein $f \in R$\\
$s_i:= \tilde \rho_{U_i}^X(s) \in \calF(U_i) \subseteq R_{g_i}$\\
$D(f)\cap D(g_i) = D(g_if) \subseteq D(g_i)$, $\calF(D(fg_i)) = \tilde \calJ_i(D(fg_i)) = \calJ_i R_{fg_i}$\\
$\tilde \rho_{D(f)\cap U_i}^{U_i}(s_i) = \tilde \rho_{D(f)\cap U_i}^X(s) = 0$ in $R_{fg_i}$\\
$\xRightarrow[\leadsto \text{\"Uberd. endl.}]{X \text{ quasikoh.}} \exists N > 0: \forall\ i$ ist $f^Ns_i = 0$ in $R_{g_i}$ $\xRightarrow{\text{Garbe}} f^Ns = 0$ in $R$

\textbf{Beweis 3:}

$t \in \calF(D(f))$, $\rho_{D(fg_i)}^{D(f)}(f) \in \calJ_i R_{g_if} \Rightarrow \exists n \ge 0, t_i \in \calJ_i: \tilde{\rho}_{D(fg_i)}^{D(f)}(t) = \frac{t_i}{f^n}$\\
$\xRightarrow[\text{endlich}]{\text{\"Uberd.}} \exists N > 0: \forall\  i \in I \exists t_i \in \calJ_i: \tilde \rho_{D(g_if)}^{D(f)}(t) = \frac{t_i}{f^N}$\\
$\Rightarrow \rho_{D(fg_ig_j)}^{D(g_i)}(t_i) = \rho_{D(g_ig_jf)}^{D(g_if)} \underbrace{\rho_{D(g_if)}^{D(g_i)} (t_i)}_{f^N\rho_{D(g_if)}^{D(f)}(t)} = \rho_{D(fg_ig_j)}^{D(f)} (f^N(t))$\\
$\xRightarrow{\text{Beh. }2} \exists m > 0: f^m(t_i - t_j) = 0$ in $D(g_ig_j)$\\
$\xRightarrow[\text{\"Uberd.}]{\text{endliche}} \exists M > 0: \forall\ i,j \in I$ gilt $f^M(t_i - t_j) = 0$ in $D(g_ig_j)$\\
$\Rightarrow f^Mt_i$ bilden konsistente Familie $\Rightarrow \exists s \in \calF(X): \tilde \rho_{D(f)}^X(s) = f^{M+N}t$
\end{twosidedproof}\end{Loes}

%-------------------------- Uebung 5 --------------------------
\newpage
\section{29. Mai 2012}
\setcounter{Aufg}{0}
\setcounter{Loes}{0}

\begin{Aufg}[4 Punkte]
Es sei $R$ ein kommutativer Ring mit $1$. Bestimme die folgenden Faserprodukte:
\begin{enumerate}%[topsep=0cm, parsep=0cm, label=\alph*)]
 \item $\Spec R[X] \times_{\Spec R} \Spec R[Y]$ f"ur die Inklusionen $R\to R[X]$ und $R\to R[Y]$.
 \item $\Spec R[X] \times_{\Spec R[Y]} \Spec R$ f"ur die Ringhomomorphismen
 \[R[Y] \to R\ ,\quad Y \mapsto 0 \qquad \text{und}\qquad R[Y] \to R[X]\ , \quad Y \mapsto X^2.\]
 \item $\Spec L \times_{\Spec k} \Spec \overline{k}$\ , wobei $L/k$ eine endliche separable K\"orpererweiterung ist und $\overline{k}$ der algebraische Abschluss von $k$.
\end{enumerate}
\end{Aufg}


\begin{Aufg}[4 Punkte]
Seien $S$ ein Schema und $T$ ein $S$-Schema. Erg"anze die Zuordnung $X \mapsto X \times_S T$ zu einem kovarianten Funktor von der Kategorie der $S$-Schemata in die Kategorie der $T$-Schemata.
\end{Aufg}


\begin{Aufg}[4 Punkte]
Es seien $k$ ein algebraisch abgeschlossener K\"orper und $\lambda\in k\setminus\{0,1\}$. Wir betrachten die elliptische Kurve $E = V(Y^2 - X(X-1)(X-\lambda)) \subset \mathbb{A}^2(k)$. Au\ss erdem sei $f\colon E \to \mathbb{A}^1(k)$ der von 
\[k[X] \to k[X,Y]/(Y^2 - X(X-1)(X-\lambda)) = k[E]\ , \quad X\mapsto X\]
induzierte und $g\colon E \to \mathbb{A}^1(k)$ der von 
$k[Y] \to k[E], Y\mapsto Y$
induzierte Schemamorphismus.
\begin{enumerate}%[topsep=0cm, parsep=0cm, label=\alph*)]
 \item Wir betrachten das Faserprodukt $V = E \times_{\mathbb{A}^1(k)} E$ mit
\begin{minipage}{3cm}
  \begin{tikzpicture}
   \matrix (m) [matrix of math nodes, row sep=1em, column sep=2em]
	{V & E \\
	 E & \mathbb{A}^1(k)\\
	 };
  \path[->, font=\scriptsize]
	(m-1-1) edge (m-1-2)
			edge (m-2-1)
	(m-2-1) edge node[auto,swap] {$f$} (m-2-2)
	(m-1-2) edge node[auto] {$f$} (m-2-2);
\end{tikzpicture}
\end{minipage}
. \\
Zeige, dass $V$ reduzibel und singul\"ar ist.
\item Sei nun $Y = E \times_{\mathbb{A}^1(k)} E$ mit
\begin{minipage}{3cm}
\begin{tikzpicture}
   \matrix (m) [matrix of math nodes, row sep=1em, column sep=2em]
	{Y & E \\
	 E & \mathbb{A}^1(k)\\
	 };
  \path[->, font=\scriptsize]
	(m-1-1) edge (m-1-2)
			edge (m-2-1)
	(m-2-1) edge node[auto,swap] {$f$} (m-2-2)
	(m-1-2) edge node[auto] {$g$} (m-2-2);
\end{tikzpicture}
\end{minipage}
.\\ Ist das Faserprodukt noch immer singul\"ar?
\end{enumerate}
\textit{Hinweis: Durch den Funktor $t$ aus Proposition 3.8 k\"onnen affine Variet\"aten als Schemata aufgefasst werden. Damit ist ihr Faserprodukt definiert.}
\end{Aufg}


\begin{Loes}\begin{enumerate}[a)]
\item
	$\Spec R[X] \X_{\Spec R} \Spec R[Y] = \Spec(R[X] \otimes_R R[Y])$ mit $R \hookrightarrow R[X], R \hookrightarrow R[Y]$
	
	\begin{minipage}[c]{0.4\textwidth}\begin{tikzpicture}
		\node (0) at (-1.5,1) {$R[X,Y]$};
		\node (1) at (1.5,1) {$R[Y]$};
		\node (2) at (-1.5,-1) {$R[X]$};
		\node (3) at (1.5,-1) {$R$};
		\node (4) at (-3,2) {$C$};
		
		\node at (-1,2) {$\schraffiert$};
		\node at (-2.5,0.25) {$\schraffiert$};
		
		\draw[right hook->] (3) -- (1);
		\draw[right hook->, font=\scriptsize] (2) -- node[right] {$j$} (0);
		\draw[left hook->, font=\scriptsize] (1) -- node[above] {$i$} (0);
		\draw[left hook->] (3) -- (2);
		
		\draw[->, dashed] (0) -- (4);
		\draw[->, font=\scriptsize] (1) to[out=135, in=45] node[above] {$\psi$} (4);
		\draw[->, font=\scriptsize] (2) to[out=180, in=225] node[left] {$\varphi$} (4);
	\end{tikzpicture}\end{minipage}
	\begin{minipage}[c]{0.55\textwidth}\textbf{UAE Polynomring:} $\exists! f: R[X,Y] \to C$ mit $f(X) = \varphi(X), f(Y) = \psi(Y)$ und $\psi: R[Y] \to C$ mit $\psi(Y) = f(Y)$ ist eindeutig $\Rightarrow f \circ i = \psi$\end{minipage}
	
	Genauso: $\varphi = f \circ i$
\item
	$\Spec R[X] \X_{\Spec R} \Spec R[Y]$, $\begin{array}{ccc} R[Y] &\twoheadrightarrow& R \\ Y &\mapsto& 0\end{array}$, $\begin{array}{ccc} R[Y] &\to& R[X] \\ Y &\mapsto& X^2\end{array}$, $R = \FakRaum{R[Y]}{(Y)}$ als $k[Y]$-Algebra.
	
	\emph{Behauptung:}$A \otimes_R \FakRaum{R}{I} \cong \FakRaum{A}{\varphi(I)A}$, wobei $A$ durch $\varphi: R \to A$ zur $R$-Algebra wird.
	
	\emph{Beweis:}
	\begin{center}\begin{tikzpicture}
		\node (0) at (0,1.5) {$\FakRaum{A}{\varphi(I)A}$};
		\node (1) at (-2.5,0) {$A$};
		\node (2) at (2.5,0) {$\FakRaum{R}{I}$};
		\node (3) at (0,-1.5) {$R$};
		\node (4) at (0,3) {$C$};
		
		\draw[->, font=\scriptsize] (3) -- node[below]{$\varphi$} (1);
		\draw[->, font=\scriptsize] (3) -- node[below]{$p$} (2);
		\draw[->, font=\scriptsize] (1) -- node[above]{$\pi$} (0);
		\draw[->] (2) -- (0);
		
		\draw[->, dashed, font=\scriptsize] (0) -- node[right]{$\exists! f$} (4);
		\draw[->, font=\scriptsize] (1) to[out=90, in=180] node[above]{$\alpha$} (4);
		\draw[->, font=\scriptsize] (2) to[out=90, in=0] node[above]{$\beta$} (4);
	\end{tikzpicture}\end{center}
	Es gilt $\alpha(\varphi(I)) = \beta(p(I)) = \beta(0) = 0 \xRightarrow[\text{Satz von } A/\varphi(I)A]{\text{Homomorphie-}} \exists! f: \FakRaum{A}{\varphi(I)A} \to C: f \circ \pi = \alpha$\\
	$f \circ h \circ p = \alpha \circ \varphi = \beta \circ p \xRightarrow{p \text{ surj.}} f \circ h = \beta$ $\qquad \Box$
	
	$\Rightarrow  \Spec(R[X]) \otimes_{R[X]} \FakRaum{R[Y]}{(Y)} = \Spec\left(\FakRaum{R[X]}{(X^2)}\right)$
\item
	$L/K$ endliche seperable K\"orpererweiterung, $\overline k$ algebraischer Abschluss von $k$, $\Spec L \X_{\Spec k} \Spec \overline k \cong \Spec(L \otimes_k \overline k)$
	
	Satz von primitiven Element $\Rightarrow L = k(\alpha), \alpha \in L$ algebraisch, $f_a$ sei Minimalpolynom von $\alpha$.\\
	$L \cong \FakRaum{k[X]}{(f_\alpha)}$, $\FakRaum{k[X]}{(f_{\alpha})} \otimes_k \overline{k} \cong \FakRaum{\overline{k}[X]}{(f_{\alpha})}$
	\begin{center}\begin{tikzpicture}
		\node (0) at (0,1.5) {$\FakRaum{\overline k[X]}{(f_\alpha)}$};
		\node (1) at (-2.5,0) {$\FakRaum{k[X]}{(f_\alpha)}$};
		\node (2) at (2.5,0) {$\overline k$};
		\node (3) at (0,-1.5) {$k$};
		\node (4) at (0,3) {$C$};
		\node (5) at (-4.5,-1) {$k[X]$};
		
		\draw[left hook->] (3) -- (1);
		\draw[right hook->] (3) -- (2);
		\draw[right hook->] (1) -- (0);
		\draw[left hook->] (2) -- (0);
		
		\draw[->, dashed] (0) -- (4);
		\draw[->, font=\scriptsize] (1) to[out=90, in=225] node[above]{$\alpha$} (4);
		\draw[->, font=\scriptsize] (2) to[out=90, in=315] node[above]{$\beta$} (4);
		
		\draw[->] (5) to[out=90, in=180] (4);
		\draw[->] (5) -- (1);
	\end{tikzpicture}\end{center}
	$f_\alpha = \prod\limits_{i=1}^d (X-\beta_i)$, $d = \deg f_\alpha$, $\beta_i \in \overline k$
	
	$\xRightarrow[\text{Restsatz}]{\text{Chinesischer}} \FakRaum{\overline k[X]}{(f_\alpha)} = \prod\limits_{i=1}^d \FakRaum{\overline k[X]}{(X-\beta_i)} \cong \prod\limits_{i=1}^d \overline k$
	
	$\Rightarrow \Spec \ldots = \Spec\left( \prod \FakRaum{\overline k[X]}{(X-\beta_i)} \right) = \coprod \Spec \left( \FakRaum{\overline k[X]}{(X-\beta_i)} \right) = \coprod \Spec \overline k$
	
	Koprodukt ($\coprod$) in der Kategorie der Schemata ist die disjunkte Vereinigung.
\end{enumerate}\end{Loes}

\begin{Loes}
$S$ Schema, $T$ $S$-Schema, das hei\ss t $T \underset{b}{\to} S$, $X \mapsto X \X_S T$
\begin{center}\begin{tikzpicture}[baseline=0]
	\node (0) at (-1.5,0.5) {$X$};
	\node (1) at (1.5,0.5) {$Y$};
	\node (2) at (0,-0.5) {$S$};
	\node at (0,0) {$\schraffiert$};
	
	\draw[->, font=\scriptsize] (0) -- node[above]{$f$} (1);
	\draw[->] (0) -- (2);
	\draw[->] (1) -- (2);
\end{tikzpicture} $\overset{\text{konstruiere}}{\leadsto}$ \begin{tikzpicture}[baseline=0]
	\node (0) at (-1.75,0.75) {$X \X_S T$};
	\node (1) at (1.75,0.75) {$Y \X_S T$};
	\node (2) at (0,-0.75) {$T$};
	\node at (0,0) {$\schraffiert$};
	
	\draw[->, font=\scriptsize] (0) -- node[above]{$f \X_S T$} (1);
	\draw[->] (0) -- (2);
	\draw[->] (1) -- (2);
\end{tikzpicture}\end{center}
\begin{center}\begin{tikzpicture}[baseline=0]
	\node (0) at (-2,1.5) {$X \X_S T$};
	\node (1) at (2,1.5) {$X$};
	\node (2) at (-2,0) {$T$};
	\node (3) at (2,0) {$S$};
	\node (4) at (-2,-1.5) {$Y \X_S T$};
	\node (5) at (2,-1.5) {$Y$};

	\node at (-2.5,0) {$\schraffiert$};
%	\node at (2.5,0) {$\schraffiert$};

	\node at (0,0.75) {$\Box$};
	\node at (0,-0.75) {$\Box$};
	
	\draw[->, font=\scriptsize, color1] (0) -- (1);
	\draw[->, font=\scriptsize] (2) -- node[below]{$b$} (3);
	\draw[->, font=\scriptsize] (4) -- (5);
	
	\draw[->, font=\scriptsize, color1] (0) -- (2);
	\draw[->, font=\scriptsize] (4) -- (2);
	\draw[->, font=\scriptsize] (1) -- node[right]{$a$} (3);
	\draw[->, font=\scriptsize] (5) -- node[right]{$c$} (3);
	
	\draw[->, font=\scriptsize, color1] (1) to[in=45, out=315] (5);
	\draw[->, font=\scriptsize] (0) to[in=135, out=225] node[left, align=right]{UAE $Y \X_S T$:\\ $\exists! f \X_S T$} (4); %Die tikz Version im Mitschriebwiki ist veraletet und vertaegt den Befehlt align=right nicht
%	\draw[->, font=\scriptsize] (0) to[in=135, out=225] node[left]{UAE $Y \X_S T$: $\exists! f \X_S T$} (4);
\end{tikzpicture}\end{center}
\emph{Noch zu zeigen:} $\id_X \X_S T = \id_{X \X_S T}$, $(f \circ g) \X_S T = (f \X_S T) \circ (g \X_S T)$
\begin{center}\begin{tikzpicture}
	\node (0) at (-4,1.5) {$X \X_S T$};
	\node (1) at (4,1.5) {$X$};
	\node (2) at (-4,0) {$Y \X_S T$};
	\node (3) at (4,0) {$Y$};
	\node (4) at (-4,-1.5) {$Z \X_S T$};
	\node (5) at (4,-1.5) {$Z$};
	\node (6) at (-1.5,0) {$T$};
	\node (7) at (1.5,0) {$S$};
	
	\node at (0,1.25) {$\Box$};
	\node at (0,-1.25) {$\Box$};
	\node at (0,-0.25) {\textcolor{color1}{$\Box$}};
	
	\node at (4.5,0) {$\schraffiert$};
	
	\draw[->, font=\scriptsize] (0) -- (2);
	\draw[->, font=\scriptsize] (2) -- (4);
	\draw[->, font=\scriptsize] (1) -- node[right]{$f$} (3);
	\draw[->, font=\scriptsize] (3) -- node[right]{$g$} (5);
	
	\draw[->, font=\scriptsize] (0) -- (1);
	\draw[->, font=\scriptsize, color1] (2) -- (6);
	\draw[->, font=\scriptsize, color1] (6) -- (7);
	\draw[->, font=\scriptsize, color1] (3) -- (7);
	\draw[->, font=\scriptsize] (4) -- (5);
	
	\draw[->, font=\scriptsize] (0) -- (6);
	\draw[->, font=\scriptsize] (4) -- (6);
	\draw[->, font=\scriptsize] (1) -- (7);
	\draw[->, font=\scriptsize] (5) -- (7);
	
	\draw[->, font=\scriptsize] (0) to[in=135, out=225] (4);
	\draw[->, font=\scriptsize] (1) to[in=45, out=315] (5);
	
	\draw[->, font=\scriptsize, color1] (2) to[in=195, out=345] (3);
\end{tikzpicture}\end{center}
\end{Loes}

\begin{Loes}
$k$ algebraisch abgeschlossen, $\lambda \in k \setminus \{0,1\}$, $E = V(Y^2 - X (X-1)(X-\lambda)) \subseteq \A^2(k)$
\begin{enumerate}[a)]
\item
	$V = E \X_{\A^1(k)} E$, $I:= (Y^2 - X (X-1)(X-\lambda))$, $k[E] = \FakRaum{k[X,Y]}{I}$, $t(E) = \Spec (k[E])$, $t(\A^1(k)) = \Spec (k[X])$, $\Spec A \X_{\Spec B} \Spec C \cong \Spec (A \otimes_B C)$
	\begin{center}\begin{tikzpicture}
		\node (0) at (-1.5, 1) {$V$};
		\node (1) at (1.5, 1) {$E$};
		\node (2) at (-1.5, -1) {$E$};
		\node (3) at (1.5, -1) {$\A^1(k)$};
		\node (4) at (-3, 2) {$C$};
		
		\draw[->, font=\scriptsize] (0) -- (1);
		\draw[->, font=\scriptsize] (0) -- (2);
		\draw[->, font=\scriptsize] (1) -- node[right]{$f$} (3);
		\draw[->, font=\scriptsize] (2) -- node[below]{$f$} (3);
		
		\draw[->, dashed, font=\scriptsize] (4) -- (0);
		\draw[->, font=\scriptsize] (4) to[in=135, out=0] (1);
		\draw[->, font=\scriptsize] (4) to[in=135, out=270] (2);
	\end{tikzpicture}\end{center}
	\begin{center}\begin{tikzpicture}
		\node (0) at (-2.5,1) {$R$};
		\node (1) at (2.5,1) {$k[E]$};
		\node (2) at (-2.5,-1) {$k[E]$};
		\node (3) at (2.5,-1) {$k[X]$};
		\node (4) at (-4.5,3.25) {$C$};
		
		\node(5) at (-2,1.25) {$x$};
		\node(6) at (2,1.25) {$x$};
		\node(7) at (-2,1.75) {$z$};
		\node(8) at (2,1.75) {$y$};
		
		\node(9) at (-3,0.75) {$x$};
		\node(10) at (-3,-0.75) {$x$};
		
		\node(11) at (-2,-1.25) {$x$};
		\node(12) at (2,-1.25) {$x$};
		
		\node(13) at (3.25,0.75) {$x$};
		\node(14) at (3.25,-0.75) {$x$};
		
		\draw[left hook->, font=\scriptsize] (1) -- (0);
		\draw[->, font=\scriptsize] (3) -- (2);
		\draw[right hook->, font=\scriptsize] (2) -- (0);
		\draw[->, font=\scriptsize] (3) -- (1);
		
		\draw[->, dashed, font=\scriptsize] (0) -- (4);
		\draw[->, font=\scriptsize] (1) to[in=0, out=90] node[above]{$h_2$} (4);
		\draw[->, font=\scriptsize] (2) to[in=270, out=180] node[left]{$h_1$} (4);
		
		\draw[|->, font=\scriptsize] (6) --(5);
		\draw[|->, font=\scriptsize] (8) --(7);
		
		\draw[|->, font=\scriptsize] (10) --(9);
		
		\draw[|->, font=\scriptsize] (12) --(11);

		\draw[|->, font=\scriptsize] (14) --(13);
	\end{tikzpicture}\end{center}
	$R = \FakRaum{k[X,Y,Z]}{\underbrace{(Y^2 - X (X-1)(X-\lambda), Z^2-Y^2)}_{=:I}}$\\
	\textcolor{gray}{\emph{Anmerkung:} $I= ((Y^2 - X (X-1)(X-\lambda), Z^2 - X (X-1)(X-\lambda))$}
	
	UAE Polynomring: $\exists! h': k[X,Y,Z] \to C$ mit $h'(X) = h_1(X) = h_2(X)$, $h'(Y) = h_1(Y)$, $h'(Z) = h_2(Y)$
	
	$h'(I') = 0 \Rightarrow h'$ faktorisiert \"uber $R$\\
	$\Rightarrow t(V) = t(E \X_{\A^1(k)} E) = \Spec(\FakRaum{k[X,Y,Z]}{I'})$\\
	$\Rightarrow V = V(I') = V(Y^2 - Z^2, Y^2 - X^2(X-1)(X-\lambda))$\\
	$\textcolor{white}{\Rightarrow V = V(I')} = V((Y-Z)(Y+Z), Y^2 - X^2(X-1)(X-\lambda))$
	
	$(0,0,0) \in V(Y-Z, Y^2 - X(X-1)(X-\lambda)) \cap V(Y+Z, Y^2 - X(X-1)(X-\lambda))$
	
	$\Rightarrow (0,0,0)$ singul\"arer Punkt
\item
	$R = \FakRaum{k[X,Y,Z]}{I''}$
	\begin{center}\begin{tikzpicture}
		\node (0) at (-2.5,1) {$R$};
		\node (1) at (2.5,1) {$k[E]$};
		\node (2) at (-2.5,-1) {$k[E]$};
		\node (3) at (2.5,-1) {$k[X]$};
		
		\node(4) at (-2,1.25) {$x$};
		\node(5) at (2,1.25) {$x$};
		\node(6) at (-2,1.75) {$y$};
		\node(7) at (2,1.75) {$y$};
		
		\node(8) at (-3,0.75) {$x$};
		\node(9) at (-3,-0.75) {$y$};
		\node(10) at (-3.5,0.75) {$z$};
		\node(11) at (-3.5,-0.75) {$x$};
		
		\node(12) at (3.25,0.75) {$x$};
		\node(13) at (3.25,-0.75) {$x$};
		
		\node(14) at (-2,-1.25) {$y$};
		\node(15) at (2,-1.25) {$x$};
		
		\draw[->] (1) -- (0);
		\draw[->] (2) -- (0);
		\draw[->] (3) -- (1);
		\draw[->] (3) -- (2);
		
		\draw[|->] (5) -- (4);
		\draw[|->] (7) -- (6);
		
		\draw[|->] (9) -- (8);
		\draw[|->] (11) -- (10);
		
		\draw[|->] (13) -- (12);

		\draw[|->] (15) -- (14);
	\end{tikzpicture}\end{center}
	$I'' = (Y^2 - X(X-1)(X-\lambda), X^2 - Z(Z-1)(Z-\lambda))$
	
	$k[E] \otimes_{k[X]} k[E] = R$
	
	$\Rightarrow E \X_{\A^1(k)} E = V(Y^2 - X(X-1)(X-\lambda), X^2 - Z(Z-1)(Z-\lambda))$
		\[ A := \tiny\begin{pmatrix} -(X-1)(X-\lambda) - X(X-\lambda) - X(X-1) & 2Y & 0\\
		2X & 0 & -(Z-1)(Z-\lambda) - Z(Z-\lambda) - Z(Z-1)\normalsize\end{pmatrix} \]
		
	\emph{Jacobi-Kriterium:} $(x,y,z) \in V$ ist singul\"ar $\Leftrightarrow \Rang(A(x,y,z)) < \underbrace{n}_{=3} - \dim V$
	
	$\dim V \ge 1$
	
	Sei $(x,y,z) \in V$ mit $\Rang(A(x,y,z)) < 2$
	
	1. Spalte $\ne \begin{pmatrix}0\\0\end{pmatrix}$\\
	$2y = 0 \wedge 1 - (Z-1)(Z-\lambda) - Z(Z-1) - Z(Z-\lambda) = 0 \lightning (x,y,z) \in V$
	
	$\Rightarrow V$ ist regul\"ar.
\end{enumerate}\end{Loes}

\begin{bem}
$t$ ist injektiv auf Objekten und volltreu $\Rightarrow$ man kann $E \X_{\A^1(k)} E$ auch in Kategorie der affinen Variet\"aten berechnen

$\leadsto$ dort ist Faserprodukt $\cong \{ (x,y) \in X \X Y \ |\ g(x) = f(y)\}$, $g: X \to S, f: Y \to S$
\end{bem}


%-------------------------- Uebung 6 --------------------------
\newpage
\section{5. Juni 2012}
\setcounter{Aufg}{0}
\setcounter{Loes}{0}

\begin{Aufg}[4 Punkte]
Betrachte die Abbildung $X \times_S Y \to \{(x,y) \in X \times Y \mid f(x) = g(y) \}$ aus Bemerkung 6.2, die das Faserprodukt von zwei $S$-Schemata $f \colon X \to S$ und $g \colon Y \to S$ in das topologische Faserprodukt von $X$ und $Y$ "uber $S$ abbildet.

Zeige, dass die Abbildung im Allgemeinen nicht injektiv ist.

\textit{Hinweis: Es gibt ein Beispiel, bei dem das topologische Faserprodukt $\{(x,y) \in X \times Y \mid f(x) = g(y)\}$ einelementig ist.}
\end{Aufg}

\begin{Aufg}[4 Punkte]
Seien $S$ ein Schema, $\xi \colon X \to S$ ein $S$-Schema und $Y \hookrightarrow S$ ein abgeschlossenes Unterschema von $S$. Zeige:
\begin{enumerate}%[topsep=0cm, parsep=0cm, label=\alph*)]
 \item Der topologische Raum $X \times_S Y$ ist hom"oomorph zu $\xi^{-1}(Y)$.
 \item Ist $X \to S$ ein abgeschlossenes Unterschema von $S$, so ist $X \times_S Y$ hom"oomorph zu $X \cap Y$.
\end{enumerate}
\end{Aufg}


\begin{Aufg}[4 Punkte]
Sei $k$ ein algebraisch abgeschlossener K"orper und $Y = \textup{Spec}(k[t])$.
\begin{enumerate}%[topsep=0cm, parsep=0.2cm, label=\alph*)]
\item Seien weiter 
$$X = \textup{Spec}(k[x,y,t]/(ty-x^2))$$
und $f\colon X \rightarrow Y$ der Schemamorphismus, der von dem nat"urlichen $k$-Algebren-Homomorphismus $k[t] \rightarrow k[x,y,t]/(ty-x^2), t \mapsto t$ induziert wird.

Zeige: F"ur $a \in k \setminus\{0\}$ ist die Faser $X_{(t-a)}$ irreduzibel und reduziert, wohingegen die Faser $X_{(t)}$ zwar irreduzibel, aber nicht reduziert ist.
\item Nun sei
$$Z =\textup{Spec}(k[x,y,t]/(xy-t))$$

 und $f \colon Z \to Y$ der Schemamorphismus, der von dem nat"urlichen Homomorphismus ${k[t] \rightarrow k[x,y,t]/(xy-t)}, t \mapsto t$ induziert wird. 
 
 Zeige:
Auch hier ist die allgemeine Faser $Z_{(t-a)}$ irreduzibel und reduziert, wohingegen $Z_{(t)}$ reduziert, aber nicht irreduzibel ist.
\end{enumerate}
\end{Aufg}

\begin{Loes}
\emph{Blatt 5, Aufgabe 1c):} $\Spec L \X_{\Spec X} \Spec \overline k = \overbrace{\coprod\limits_{i=1}^n \Spec \overline k}^{n\text{-elementig}}$, $L|k$ seperabel, endlich, $\Spec L, \Spec k, \Spec \overline k$ alle einelementig

$\Rightarrow $ topologisches Faserprodukt ist auch einelementig.

\emph{Beispiel:} $\C = \R(i)$\\
$\Spec \C \X_{\Spec \R} \Spec \C$ zweielementig
\end{Loes}

\begin{Loes}
$S$ Schema, $\xi: X \to S$, $\eta: Y \hookrightarrow S$, $Y$ abgeschlossenes Unterschema von $S$.
\begin{enumerate}[a)]
\item
	\emph{Zu zeigen:} $X \X_S Y \cong \xi^{-1}(Y)$ hom"oomorph
	
	Sei zun"achst $S = \Spec R$ affin $\Rightarrow Y = \Spec \FakRaum{R}{I}$, $I \subseteq R$ Ideal
	
	Schreibe $X = \bigcup\limits_{i\in I} U_i$, $U_i \subseteq X$ offen und affin, $U_i = \Spec A_i$
	
	$\xi_i := \xi|_{U_i}: \underset{=\Spec A_i}{U_i} \to \underset{=\Spec R}{S}$ $\leadsto \varphi_i: R \to A_i$
	
	$X \X_S Y = (\bigcup\limits_{i\in I} U_i) \X_S Y = \bigcup\limits_{i\in I}(U_i \X_S Y) = \bigcup\limits_{i\in I} \Spec (\underbrace{A_i \otimes_R \FakRaum{R}{I}}_{\scriptsize{\raisebox{0.7ex}{\ensuremath{A_i}}\ensuremath{\mkern-3mu}\ / \ensuremath{\mkern-3mu}\raisebox{-0.6ex}{\ensuremath{\varphi_i(I)\cdot A}}}}) = \bigcup\limits_{i \in I} \underbrace{\Spec (\FakRaum{A_i}{\varphi_i(I) \cdot A_i})}_{=: Z_i}$
	
	$Z_i \cong \{ p \in \Spec A_i | \varphi_i(I) \subseteq p \} = V(\varphi_i(I) A_i) = \xi_i^{-1}(Y)$
	
	Sei $p \in Z_i \Rightarrow \underbrace{\varphi^{-1}(\varphi_i(I))}_{\subseteq I} \subseteq \varphi_i^{-1}(p) = \xi(p)$ $\Rightarrow \xi(p) \in \Spec (\FakRaum{R}{I}) = Y$ $\Rightarrow Z_i \subseteq \xi^{-1}(Y)$
	
	$\xRightarrow{X \X_S Y \twoheadrightarrow \xi(Y)} X \X_S Y = \xi^{-1}(Y)$
	
	\begin{center}in tolopogischen R"aumen: \begin{tikzpicture}[baseline=0]
		\node (0) at (-3,1) {$\{ (x,y) \in X \X Y | f(x) = g(x)\}=f^{-1}(Y)$};
		\node (1) at (3,1) {$X$};
		\node (2) at (-3,-1) {$Y$};
		\node (3) at (3,-1) {$S$};
		\node at (0,0) {$\Box$};
		
		\draw[->] (0) -- (1);
		\draw[->] (0) -- (2);
		\draw[->] (1) -- (3);
		\draw[right hook->] (2) -- (3);
	\end{tikzpicture}\end{center}
	
	Sei nun $S$ beliebig: $S = \bigcup\limits_{i \in I} S_i$, $S_i$ offen und affin, $X_i := \xi^{-1}(S_i)$, $Y_i := \eta^{-1}(S_i)$
	\begin{center}\begin{tikzpicture}
		\node (0) at (-2,1) {?};
		\node (1) at (2,1) {$\bigcup\limits_{i \in I} X_i$};
		\node (2) at (-2,-1) {$\bigcup\limits_{i \in I} Y_i$};
		\node (3) at (2,-1) {$\bigcup\limits_{i \in I} S_i$};
		\node at (0,0) {$\Box$};
		
		\draw[->,dashed] (0) -- (1);
		\draw[->,dashed] (0) -- (2);
		\draw[->] (1) -- (3);
		\draw[->] (2) -- (3);
	\end{tikzpicture}\end{center}
	
	$X \X_S Y = \bigcup\limits_{i \in I} X \X_S Y \overset{\text{Vorl.}}{=} \bigcup\limits_{i \in I} X_i \X_{S_i} Y_i \overset{\text{s. o.}}{=} \bigcup\limits_{i\in I} \xi^{-1} (Y_i) = \xi^{-1} (\bigcup\limits_{i\in I} Y_i) = \xi^{-1}(Y)$
\item
	$\xi: X \hookrightarrow S$ abgeschlossenes Unterschema $\Rightarrow X \X_S Y \overset{\text{a)}}{=} \xi^{-1}(Y) \overset{\xi \text{ Inklusion}}{=} X \cap Y$
\end{enumerate}\end{Loes}

\begin{Loes}
$k$ algebraisch abgeschlossen, $Y = \Spec(k[t])$
\begin{enumerate}[a)]
\item
	$X = \Spec (\FakRaum{k[x,y]}{(ty-x^2)})$\\
	\textcolor{gray}{Anmerkung: Gro\ss buchstaben stehen in dieser Aufgabe f\"ur Schemata, Kleinbuchstaben f\"ur Variablen.}
	
	$f: X \to Y$ induziert durch $\begin{array}{ccc} k[t] &\to& R \\ t &\mapsto& t\end{array}$
	
	Berechne Faser $X_{(t-a)}$:\\
	$p:= (t-a) \in Y$\\
	$\calO_{Y,p} = k[t]_p$\\
	Restklassenk"orper: $\kappa(p) = \FakRaum{\calO_{Y,p}}{m_{Y,p}}$
	
	$\kappa(p) = \FakRaum{k[t]_p}{p \cdot k[t]_p} = \FakRaum{k[t]_p}{(t-a)}$ \textcolor{gray}{Anmerkung: $(\FakRaum{R}{I})_{\pi(S)} = \FakRaum{R_S}{I\cdot R_S}$, $\pi: R \twoheadrightarrow \FakRaum{R}{I}$}
	
	$X_{(t-a)} = X \X_Y \Spec \kappa(p)$\\
	$\textcolor{white}{X_{(t-a)}} = \Spec( \underbrace{\FakRaum{k[x,y,t]}{(ty-x^2)}}_{A} \underbrace{\otimes_{k[t]}}_{R} \underbrace{\FakRaum{k[t]}{(t-a)}}_{\scriptsize{\raisebox{0.7ex}{\ensuremath{R}}\ensuremath{\mkern-3mu}\ / \ensuremath{\mkern-3mu}\raisebox{-0.6ex}{\ensuremath{I}}}})$\\
	$\textcolor{white}{X_{(t-a)}} = \Spec\left(\FakRaum{k[x,y,t]}{(ty - x^2, t-a)}\right)$\\
	$\textcolor{white}{X_{(t-a)}} \cong \Spec\left(  \FakRaum{k[x,y]}{(ay-x^2)} \right) \cong V(ay-x^2) \subseteq \Spec k[x,y]$
	
	$I := (ay - x^2)$
	
	$X_{(t-a)}$ ist irreduzibel $\Leftrightarrow \sqrt I$ prim
	
	\begin{enumerate}[\underline{$a=0$}:]
	\item[\underline{$a \ne 0$}:]
		$ay$ ist kein Quadrat in $k[y]$\\
		$\Rightarrow ay - x^2$ irreduzibel in $k[y][x]$\\
		$\Rightarrow I$ prim\\
		$\Rightarrow \sqrt I= I$ prim\\
		$\Rightarrow X_{(t-a)}$ irreduzibel\\
		Au\ss erdem $\FakRaum{k[x,y]}{I}$ ist nullteilerfrei $\Rightarrow X_{(t-a)}$ reduziert
	\item[\underline{$a=0$}:]
		$X_{(t)} = \Spec ( \FakRaum{k[x,y]}{x^2} )$\\
		$I = (x^2) \Rightarrow \sqrt I = (x)$ prim\\
		$\textcolor{white}{I = (x^2)} \Rightarrow X_{(t)}$ irreduzibel\\
		$X$ nilpotent in $\FakRaum{k[x,y]}{x^2} \Rightarrow X_{(t)}$ ist nicht reduziert
	\end{enumerate}
	\begin{center}\begin{tikzpicture}
		%Achsenkreuz zeichnen
		\path[draw,->,very thick,name path=t Achse] (-5,0) -- (5,0);
		\draw[->,very thick,color1] (0,-5) -- (0,5);
		\draw[dashed,very thick] (-4,4) -- (0,0);
		\draw[->,very thick] (0,0) -- (4,-4);
				
		%Beschriftungen
		\node at (5.25,0) {$t$};
		\node at (-0.25,5) {$y$};
		\node at (4.25,-4.25) {$x$};
		
		\node at (2.5,-0.5) {\textcolor{color1}{$X_{(t-1)}$}};
		\node at (-1,-4) {\textcolor{color1}{$X_{(t+1)}$}};
		
		%horizontale Parabeln
		\path[rotate around={-90:(0,2.5)},name path=obere parabel] (-0.75,7) parabola bend (0,2.5) (0.75,7);
		\path[rotate around={90:(0,-2.5)},name path=untere parabel] (-0.75,2) parabola bend (0,-2.5) (0.75,2);
		
		%obere vertikale Parabeln (unsichtbar)
		\path[name path=obere rechte parabel] (3.25,5) parabola bend (3.75,0) (4.25,5);
		\path[name path=obere mittlere parabel] (2,5) parabola bend (2.5,0) (3,5);
		\path[name path=obere linke parabel] (0.75,5) parabola bend (1.25,0) (1.75,5);
		
		%Schnittpunkte
		\path[name intersections={of=obere parabel and obere rechte parabel, name=oben rechts}];
		\path[name intersections={of=obere parabel and obere mittlere parabel, name=oben mitte}];
		\path[name intersections={of=obere parabel and obere linke parabel, name=oben links}];
		
		%der sichtbare Teil der oberen horizontalen Parabel
		\begin{scope}
			\clip (oben rechts-4) rectangle (0,2.5);
			\path[draw,rotate around={-90:(0,2.5)}] (-0.75,7) parabola bend (0,2.5) (0.75,7);
		\end{scope}\begin{scope}
			\clip (oben rechts-1) rectangle (0,2.5);
			\path[draw,rotate around={-90:(0,2.5)}] (-0.75,7) parabola bend (0,2.5) (0.75,7);
		\end{scope}
		
		%die sichtbaren oberen Parabeln
		\begin{scope}
			\clip (oben rechts-1) rectangle (3.75,0);
			\path[draw] (3.25,5) parabola bend (3.75,0) (4.25,5);
		\end{scope}\begin{scope}		
			\clip (oben rechts-4) rectangle (3.75,0);
			\path[draw] (3.25,5) parabola bend (3.75,0) (4.25,5);
		\end{scope}
		
		\begin{scope}
			\clip (oben mitte-1) rectangle (oben mitte-3);
			\path[draw] (3.25,5) (2,5) parabola bend (2.5,0) (3,5);
		\end{scope}\begin{scope}
			\clip (oben mitte-3) rectangle (2.5,0);
			\path[draw,dashed] (3.25,5) (2,5) parabola bend (2.5,0) (3,5);
		\end{scope}\begin{scope}		
			\clip (oben mitte-4) rectangle (2.5,0);
			\path[draw] (3.25,5) (2,5) parabola bend (2.5,0) (3,5);
		\end{scope}
		
		\begin{scope}
			\clip (oben links-1) rectangle (oben links-3);
			\path[draw,color1] (0.75,5) parabola bend (1.25,0) (1.75,5);
		\end{scope}\begin{scope}
			\clip (oben links-3) rectangle (1.25,0);
			\path[draw,color1,dashed] (0.75,5) parabola bend (1.25,0) (1.75,5);
		\end{scope}\begin{scope}
			\clip (oben links-4) rectangle (1.25,0);
			\path[draw,color1] (3.25,5) (0.75,5) parabola bend (1.25,0) (1.75,5);
		\end{scope}
		
		
		%untere vertikale Parabeln
		\path[name path=untere linke parabel] (-3.25,-5) parabola bend (-3.75,0) (-4.25,-5);
		\path[name path=untere mittlere parabel] (-2,-5) parabola bend (-2.5,0) (-3,-5);
		\path[name path=untere rechte parabel] (-0.75,-5) parabola bend (-1.25,0) (-1.75,-5);
		
		%Schnittpunkte
		\path[name intersections={of=untere parabel and untere rechte parabel, name=unten rechts}];
		\path[name intersections={of=untere parabel and untere mittlere parabel, name=unten mitte}];
		\path[name intersections={of=untere parabel and untere linke parabel, name=unten links}];
		
		%der sichtbare Teil der unteren horizontalen Parabel
		\begin{scope}
			\clip (unten links-4) rectangle (unten links-3);
			\path[draw,rotate around={90:(0,-2.5)},name path=untere parabel] (-0.75,2) parabola bend (0,-2.5) (0.75,2);
		\end{scope}\begin{scope}
			\clip (unten links-3) rectangle (0,-2.5);
			\path[draw,dashed,rotate around={90:(0,-2.5)},name path=untere parabel] (-0.75,2) parabola bend (0,-2.5) (0.75,2);
		\end{scope}\begin{scope}
			\clip (unten links-1) rectangle (0,-2.5);
			\path[draw,rotate around={90:(0,-2.5)},name path=untere parabel] (-0.75,2) parabola bend (0,-2.5) (0.75,2);
		\end{scope}
		
		%die sichtbaren unteren Parabeln
		\begin{scope}
			\clip (unten links-4) rectangle (-3.75,0);
			\path[draw,dashed,name path=untere linke parabel] (-3.25,-5) parabola bend (-3.75,0) (-4.25,-5);
		\end{scope}\begin{scope}
			\clip (unten links-1) rectangle (-3.75,0);
			\path[draw,name path=untere linke parabel] (-3.25,-5) parabola bend (-3.75,0) (-4.25,-5);
		\end{scope}
		
		\begin{scope}
			\clip (unten mitte-4) rectangle (-2.5,0);
			\path[draw,dashed,name path=untere mittlere parabel] (-2,-5) parabola bend (-2.5,0) (-3,-5);
		\end{scope}\begin{scope}
			\clip (unten mitte-1) rectangle (-2.5,0);
			\path[draw,name path=untere mittlere parabel] (-2,-5) parabola bend (-2.5,0) (-3,-5);
		\end{scope}
		
		\begin{scope}
			\clip (unten rechts-4) rectangle (-1.25,0);
			\path[draw,dashed,color1,name path=untere rechte parabel] (-0.75,-5) parabola bend (-1.25,0) (-1.75,-5);
		\end{scope}\begin{scope}
			\clip (unten rechts-1) rectangle (-1.25,0);
			\path[draw,color1,name path=untere rechte parabel] (-0.75,-5) parabola bend (-1.25,0) (-1.75,-5);
		\end{scope}
		
		%Markierungen auf der t-Achse
		\path[draw,thick,name intersections={of=obere linke parabel and t Achse, name=t}] (t-1) ++ (0,0.25) -- ++(0,-0.5) node[below,color1] {$t$};
		\path[draw,thick,name intersections={of=untere rechte parabel and t Achse, name=minus eins}] (minus eins-1) ++ (0,-0.25) -- ++(0,0.5) node[above,color1] {$-1$};
	\end{tikzpicture}\end{center}
	
\item
	$Z:= \Spec \left(\FakRaum{k[x,y,t]}{(xy-t)} \right)$
	
	\begin{center}\begin{tikzpicture}
		%Achsenkreuz zeichnen
		\draw[->,very thick,color1] (-5,0) -- (5,0);
		\draw[->,very thick,color1] (0,-5) -- (0,5);
		
		%Beschriftungen
		\node at (5.25,0) {$x$};
		\node at (-0.25,5) {$y$};
		
		\node at (0.75,4.8) {\textcolor{color1}{$t=0$}};
		\node at (5,1.5) {$t=2$};
		\node at (5,0.5) {$t=1$};
		\node at (5.5,-0.5) {$t=-1$};
		\node at (-0.75,-4.8) {$t=1$};
		\node at (-2,-4.8) {$t=2$};
		\node at (-5.5,0.5) {\textcolor{color1}{$-1=t$}};
		
		%die Hyperbeln
		\draw[thick, bend right=45, looseness=1.75] (0.5,4.5) to (4.5,0.5);
		\draw[thick, bend left=45, looseness=1.75] (-4.5,-0.5) to (-0.5,-4.5);
		\draw[thick, bend right=45, looseness=1.75] (-4.5,0.5) to (-0.5,4.5);
		\draw[thick, bend left=45, looseness=1.75] (0.5,-4.5) to (4.5,-0.5);
		
		\draw[thick, bend right=45, looseness=1.75] (1.5,4.5) to (4.5,1.5);
		\draw[thick, bend left=45, looseness=1.75] (-4.5,-1.5) to (-1.5,-4.5);
		
		\draw[thick, bend right=45, looseness=1.75, dashed, color1] (-4.5,0.5) to (-0.5,4.5);
		\draw[thick, bend left=45, looseness=1.75, dashed, color1] (0.5,-4.5) to (4.5,-0.5);
	\end{tikzpicture}\end{center}
	\textcolor{gray}{Anmerkung: Falls man es nicht in der Zeichnung sieht, die Hyperbeln im zweiten und vierten Quadranten enthalten auch noch eine gestrichelte Hyperbel in der anderen Farbe.}
	
	$\kappa((t-a)) \cong \FakRaum{k[t]}{(t-a)}$
	
	$Z_{(t-a)} = \Spec \left( \FakRaum{k[x,y,t]}{(xy-t)} \otimes_{k[t]} \FakRaum{k[t]}{(t-a)} \right) = \Spec \left( \FakRaum{k[x,y]}{(xy-a)} \right)$
	
	\begin{enumerate}[\underline{$a=0$}:]
	\item[\underline{$a \ne 0$}:]
		$(xy-a)$ ist prim $\Rightarrow Z_{(t-a)}$ reduziert und irreduzibel
	\item[\underline{$a = 0$}:]
		$Z_{(t)} = \Spec \left( \FakRaum{k[x,y]}{(xy)} \right) \cong V(xy) \subseteq \Spec k[x,y]$\\
		$V(xy) = V(x) \cup V(y)$\\
		$\Rightarrow Z_{(t)}$ reduzibel\\
		$\sqrt{(xy)} = (xy) \Rightarrow$ keine nilpotenten Elemente in $\FakRaum{k[x,y]}{(xy)} \Rightarrow Z_{(t)}$ reduziert
	\end{enumerate}
\end{enumerate}\end{Loes}

%-------------------------- Uebung 7 --------------------------
\newpage
\section{12. Juni 2012}
\setcounter{Aufg}{0}
\setcounter{Loes}{0}

\begin{Aufg}[4 Punkte]
Ein Schemamorphismus $f \colon X \to Y$ hei"st {\em quasikompakt}, wenn es eine offene, affine "Uberdeckung $Y = \bigcup\limits_{i \in I} V_i$ gibt, sodass jedes $f^{-1}(V_i)$ quasikompakt ist. Zeige:
\begin{enumerate}%[topsep=0cm, parsep=0cm, label=\alph*)]
\item Ein Schemamorphismus $f \colon X \to Y$ ist genau dann quasikompakt, wenn f"ur jede offene, affine Teilmenge $V \subseteq Y$ das Urbild $f^{-1}(V)$ quasikompakt ist.
\item Ein Schemamorphismus $f \colon X \to Y$ ist genau dann von endlichem Typ, wenn er lokal von endlichem Typ und quasikompakt ist.
\end{enumerate}
\end{Aufg}

\begin{Aufg}[3 Punkte]
Es sei $k$ ein K"orper und $X$ die affine Gerade "uber $k$ mit doppeltem Nullpunkt. $X$ ist also das Schema, das entsteht, wenn wir $U = \Spec k[X]$ mit $V = \Spec k[Y]$ entlang $D(X)$ und $D(Y)$ verm"oge des Isomorphismus $k[Y,Y^{-1}] \to k[X,X^{-1}]$, $Y\mapsto X$ verkleben.

Zeige, dass $X$ nicht separiert "uber $k$ ist.
\end{Aufg}


\begin{Aufg}[3 Punkte]
Zeige, dass ein Schemamorphismus $f\colon X \rightarrow Y$ genau dann separiert ist, wenn das Bild von $X$ unter der von $f$ induzierten Diagonalabbildung $\Delta_f\colon X \rightarrow X \times_Y X$ abgeschlossen ist.
\end{Aufg}

\begin{Aufg}[4 Punkte]
Es seien $S$ ein Schema und $X$, $Y$ zwei $S$-Schemata. Weiter seien zwei Morphismen $f$ und $g$ von $X$ nach $Y$ gegeben, die auf einer offenen, dichten Teilmenge von $X$ "ubereinstimmen. Zeige:
\begin{enumerate}%[topsep=0cm, parsep=0cm, label=\alph*)]
 \item Ist $X$ reduziert und $Y$ separiert "uber $S$, so gilt $f = g$.
 \item Die Aussage ist falsch, wenn $X$ nicht reduziert ist oder wenn $Y$ nicht separiert ist.
 \end{enumerate}
 \end{Aufg}

\begin{Loes}
$f: X \to Y$ quasikomp. $:\Leftrightarrow \exists$ offene, affine "Uberd. $Y = \Bigcup_{i\in I} V_i$ mit $\forall i \in I: f^{-1}(V_i)$ quasikomp.
\begin{enumerate}[a)]
\item
	\emph{Behauptung:} $f$ quasikompakt $\Leftrightarrow$ f"ur alle $V \subseteq Y$ offen, affin gilt $f^{-1}(V)$ quasikompakt
	\begin{twosidedproof}
	\proofforward
		$\checkmark$
	\proofreverse
		Sei $Y = \Bigcup_{i\in I} V_i$, $V_i = \Spec R_i$, $f^{-1}(V_i)$ quasikomp., $V = \Spec R \subseteq Y$, $V = \Bigcup_{i \in I} V \cap V_i$.
		
		"Uberdecke $\underbrace{f^{-1}(V_i)}_{\text{quaskomp.}}$ affin: $f^{-1}(V_i) = \Bigcup_{j \in J_i} \Spec A_{ij} \Rightarrow$ \OE $J_i$ endlich
		
		$f|_{\Spec A_{ij}}: \Spec A_{ij} \to \Spec R_i \leadsto \varphi_{ij}: R_i \to A_{ij}$
		
		Sei $x \in V \cap V_i \Rightarrow x \in \Spec R_i \Rightarrow \exists \ g_{x,i} \in R_i: x \in D(g_{x,i})$, $D(g_{x,i}) = \Spec R_i [\frac{1}{g_{x,i}}]$
		
		$f^{-1}(D(g_{x,i})) \cap \Spec A_{ij} = \Spec A_{ij}[\frac{1}{\varphi_{ij}(g_{x,i})}]$ quasikompakt
		
		$V = \Bigcup_{i \in I} \Bigcup_{x \in V \cap V_i} (D(g_{x,i})) \overset{V}{\underset{\text{quasikomp.}}{=}} \Bigcup_{\text{endl.}} D(g_{x,i})$
		
		$\Rightarrow f^{-1}(V) = \Bigcup_{\text{endl.}} f^{-1} (D(g_{x,i})) = \Bigcup_{\substack{\text{endl.} \\ g_{x,i}}} \Bigcup_{j \in J_i} f^{-1} (D(g_{x,i})) \cap \Spec A_{ij} = \Bigcup_{\text{endl.}} \bigcup_{j \in J_i} \Spec A_{ij} [\frac{1}{\varphi_{ij}(g_{x,i})}]$ ist quasikompakt, da endliche Vereinigung von quasikompakten Mengen.
	\end{twosidedproof}
\item
	$f$ lokal von endlichem Typ $\Leftrightarrow \exists \ Y = \bigcup \Spec R_i: f^{-1}(\Spec R_i) = \bigcup_{j \in J} \Spec A_{ij}$, $A_{ij}$ endlich erzeugte $R_i$-Algebra
	
	$f$ von endlichem Typ $\Leftrightarrow$ zus"atzlich $J$ endlich
	
	\emph{Zu zeigen:} von endlichem Typ $\Leftrightarrow$ lokal von endlichem Typ und quasikompakt
	\begin{twosidedproof}
	\proofreverse
		$\checkmark$
	\proofforward
		\emph{Zu zeigen:} $f$ quasikompakt
		
		$Y = \bigcup \Spec R_i$, $f^{-1}(\Spec R_i) = \Bigcup_{j \in J} \Spec A_{ij}$, $J$ endlich
		
		$\Rightarrow$ endliche Vereinigung von quasikompakten Mengen $\checkmark$ (endliche Vereinigung von quasikompakten Mengen ist quasikompakt)
	\end{twosidedproof}
\end{enumerate}\end{Loes}

\begin{Loes}
Wir wollen eine offene, affine Teilmenge $W \subseteq X \times_{\Spec k} X$ finden, so dass $W \cap \Delta(X)$ kein abgeschlossenes Unterschema von $W$ ist. Damit das klappen kann, sollte $W$ etwas mit $U \cap V$ zu tun haben.
Es gilt: $U \cap V \cong D(x) \cong D(y) \cong \Spec k[t, t^{-1}]$, durch $k[x,x^{-1}] \to k[t,t^{-1}], x \mapsto t$ und $k[y,y^{-1}] \to k[t,t^{-1}], y \mapsto t$.

\begin{minipage}{5cm}
\begin{tikzpicture}
     \matrix (m) [matrix of math nodes, row sep=1.5em, column sep=2em]{
		  & X \\
		  & X\times_k X\\
		  X & & X \\
		  &\Spec k \\
	  };
    \path[->]
	(m-1-2) edge [bend right=30] node[above] {$\id\quad$}  (m-3-1)
            edge [bend left=30] node[above] {$\quad\id$} (m-3-3)
            edge [densely dotted] node[right] {$\Delta$} (m-2-2)
    (m-2-2) edge node[above] {$p_1$} (m-3-1)
			edge node[above] {$p_2$} (m-3-3)
	(m-3-1) edge (m-4-2)
	(m-3-3) edge (m-4-2);   
\end{tikzpicture}
\end{minipage}
\textbf{Beh.:} Es gilt $p_1^{-1}(U) \cap p_2^{-1}(V) \cong U \times_{\Spec k} V \subseteq X \times_{\Spec k} X$.

Wir zeigen, dass $p_1^{-1}(U) \cap p_2^{-1}(V)$ die UAE von $U \times_{\Spec k} V$ erf"ullt. Dazu seien ein Schema $T$ mit $\Spec k$-Morphismen $\varphi_1 \colon T \to U$ und $\varphi_2 \colon T \to V$ vorgegeben mit $s \circ i_1 \circ \varphi_1 = s \circ i_2 \circ \varphi_2$.

\begin{minipage}{10.5cm}
\begin{tikzpicture}[]
    \matrix (m) [matrix of math nodes,
         row sep=2em, column sep=2em,
         text height=1.5ex,
         text depth=0.25ex]{
         T&                                  &              &     \\
          &p_1^{-1}(U)\cap p_2^{-1}(V)   &              & V \\
          & 			             & X\times_{\Spec k} X  & X \\
          & 				U  & X             & \Spec k \\
	  };
    \path[->]
	(m-1-1) edge [bend right=15] node[left] {$\varphi_1$}  (m-4-2)
                edge [bend left=15] node[below] {$\varphi_2$} (m-2-4)
                edge [densely dotted] node[above] {$\theta$} (m-2-2)
        (m-2-2) edge node[below,sloped] {$\subset$}   (m-3-3)
        (m-2-2) edge node[above] {$p_2$}   (m-2-4)
        (m-2-2) edge node[left] {$p_1$}  (m-4-2)
        (m-3-3) edge node[left] {$p_1$}   (m-4-3)
        (m-3-3) edge node[above] {$p_2$}  (m-3-4)
        (m-4-2) edge node[below] {$i_1$} (m-4-3)
        (m-2-4) edge node[right] {$i_2$} (m-3-4)
        (m-3-4) edge node[right] {$s$} (m-4-4)
        (m-4-3) edge node[below] {$s$} (m-4-4);        
\end{tikzpicture}
\end{minipage}
\begin{minipage}{5.4cm}
  Aus der UAE von $X\times_k X$, angewendet auf $i_1 \circ \varphi_1$ und $i_2 \circ \varphi_2$, erhalten wir $\theta\colon T\to X\times_{\Spec k} X$ das diese Morphismen faktorisiert. Daraus folgt, dass $p_1(\theta(T)) \subseteq U$ und $p_2(\theta(T))\subseteq V$ gilt, wir also einen eindeutigen Morphismus $\theta \colon T \to p_1^{-1}(U_i)\cap p_2^{-1}(U_j)$ erhalten haben.
\end{minipage}

Wir erhalten $\Delta^{-1}(U \times_{\Spec k} V) = \Delta^{-1}(p_1^{-1}(U) \cap p_2^{-1}(V)) = U \cap V$, also einen Morphismus  $\Delta|_{U \cap V} \colon U \cap V \to U \times_{\Spec k} V \cong \Spec k[x,y]$ von affinen Schemata mit zugeh"origem Ringhomomorphismus 
$$\varphi: \left\{\begin{array}{rcl}
                  k[x,y] & \to& k[t,t^{-1}] \\ x &\mapsto& t \\ y &\mapsto& t                                                                                                                                                                                                                                                                                                                                                                                                                                                                                                            \end{array} \right. .$$
Da $\varphi$ nicht surjektiv ist, wird der Ringhomomorphismus nicht von einem Ideal in $k[x,y]$ induziert und $\Delta_\ast \calO_X|_{U \times_{\Spec k} V}$ ist nicht quasikoh"arent. Also ist $U \times_{\Spec k} V$ das $W$, das wir finden wollten.
\end{Loes}

\begin{Loes}
$f: X \to Y$ separiert $\Leftrightarrow \Delta_f(X) \subseteq X \X_Y X$ abgeschlossen
\begin{center}\begin{tikzpicture}[baseline=0]
	\node (0) at (0, 2) {$X$};
	\node (1) at (0, 1) {$X \X_Y X$};
	\node (2) at (-2, 0) {$X$};
	\node (3) at (2, 0) {$X$};
	\node (4) at (0, -1) {$Y$};
	\node at (0,0) {$\Box$};
	
	\draw[->] (0) -- (1);
	\draw[->] (1) -- node[above, font=\scriptsize]{$p_1$} (2);
	\draw[->] (1) -- node[above, font=\scriptsize]{$p_2$} (3);
	\draw[->] (2) -- node[below, font=\scriptsize]{$f$} (4);
	\draw[->] (3) -- node[below, font=\scriptsize]{$f$} (4);
	\draw[->] (0) to[out=180, in=135] node[above, font=\scriptsize]{$\id$} (2);
	\draw[->] (0) to[out=0, in=45] node[above, font=\scriptsize]{$\id$} (3);
\end{tikzpicture}
\begin{minipage}{7cm}\begin{flushright}$f$ separiert $\Leftrightarrow \Delta_f(X) \subseteq X \X_Y X$ abgeschlossenes Unterschema\end{flushright}\end{minipage}\end{center}

\begin{twosidedproof}
\proofforward
	Nach Definition
\proofreverse
	\begin{enumerate}[$\bullet$]
		\item $\Delta_f$ hat Linksinverse $p_1 \Rightarrow \Delta_f$ injektiv
		\item Noch zu zeigen: $\underbrace{\calO_{\Delta_f(X)}}_{(\Delta_f)_*\calO_X} = \calO_{X \X_Y \smallFakRaum{X}{\calI}}$, $\calI$ quasikoh"arente Idealgarbe auf $X \X X$.
	\end{enumerate}
	$\leadsto$ finde offene, affine "Uberdeckung $X \X_Y X = \Bigcup_{i\in I} U_i$, sodass $\forall i \in I: \calI_{U_i}$ quasikoh"arent.
	
	$Y = \bigcup \Spec R_i$\\
	$X = \bigcup \underbrace{f^{-1}(\Spec R_i)}_{X_i}$, $X_i = \bigcup \Spec A_{ij}$
	
	$\Rightarrow \Spec A_{ij} \X_{\Spec R_i} \Spec A_{ij}$ "uberdecken $\Delta_f(X)$\\
	$f|_{\Spec A_{ij}}: \underbrace{\Spec A_{ij}}_{=: Z_{ij}} \to \Spec R_i$ separiert
	
	$\Rightarrow \Delta_f|_{Z_{ij}}: Z_{ij} \to Z_{ij} \X_{\Spec R_i} Z_{ij}$ ist abgeschlossene Einbettung
	
	$\Rightarrow \calO_{\Delta_f(X)|_{Z_{ij}}} = \calO_{Z_{ij} \X_{\Spec R} \smallFakRaum{Z_{ij}}{I}}$, $I$ quasikoh"arent
	
	$\calI|_{Z_{ij}} = I$
\end{twosidedproof}\end{Loes}


\begin{Loes}
$X, Y$ $S$-Schemata, $f: X \to Y$, $g: X \to Y$, $f = g$ auf $U \subseteq X$ offen, dicht.
\begin{enumerate}[a)]
\item
	\emph{Behauptung:} $X$ reduziert, $Y \to S$ separiert $\Rightarrow f = g$
	
	\begin{minipage}{6cm}\begin{tikzpicture}[baseline=0]
		\node (0) at (0, 2.5) {$X$};
		\node (1) at (0, 1) {$Y \X_S Y$};
		\node (2) at (-2, 0) {$Y$};
		\node (3) at (2, 0) {$Y$};
		\node (4) at (0, -1) {$S$};
		\node (5) at (2, 2.5) {$U$};
		\node at (0,0) {$\Box$};
		\node at (1.5,1.3) {$\schraffiert$};
		\node at (-1.5,1.3) {$\schraffiert$};
		
		\draw[->, dashed] (0) -- node[left, font=\scriptsize]{$(f,g)_S$} (1);
		\draw[->] (1) -- node[above, font=\scriptsize]{$p_1$} (2);
		\draw[->] (1) -- node[above, font=\scriptsize]{$p_2$} (3);
		\draw[->] (2) -- (4);
		\draw[->] (3) -- (4);
		\draw[->] (0) to[out=180, in=135] node[above, font=\scriptsize]{$f$} (2);
		\draw[->] (0) to[out=0, in=45] node[above, font=\scriptsize]{$g$} (3);
		\draw[left hook->] (5) -- node[above, font=\scriptsize]{$j$} (0);
		\draw[->,dashed] (5) -- (1);
	\end{tikzpicture}\end{minipage}\begin{minipage}{10cm}
	$f \circ j = g \circ j : U \to Y$ faktorisiert "uber $p_i \circ (f,g)_S$
	
	$\Rightarrow (f,g)_S \circ j = (f \circ j, g \circ j)_S = (f \circ j, f \circ j)_S$\end{minipage}
	
	\begin{minipage}{6cm}\begin{tikzpicture}[baseline=0]
		\node (0) at (0, 2.5) {$T$};
		\node (1) at (0, 1) {$Z \X_S Z$};
		\node (2) at (-2, 0) {$Z$};
		\node (3) at (2, 0) {$Z$};
				
		\node at (1.5,1.3) {$\schraffiert$};
		\node at (-1.5,1.3) {$\schraffiert$};
		
		\draw[->, dashed] (0) -- node[left, font=\scriptsize]{$(m,m)_S$} (1);
		\draw[->] (1) -- node[above, font=\scriptsize]{$\tilde p_1$} (2);
		\draw[->] (1) -- node[above, font=\scriptsize]{$\tilde p_2$} (3);
		\draw[->] (0) to[out=180, in=135] node[above, font=\scriptsize]{$m$} (2);
		\draw[->] (0) to[out=0, in=45] node[above, font=\scriptsize]{$m$} (3);
		\draw[->] (3) to[out=180, in=270] node[left, font=\scriptsize]{$\Delta_Z$} (1);
	\end{tikzpicture}\end{minipage}\begin{minipage}{10cm}
	$\Rightarrow \tilde p_i \circ \Delta_Z = \id_Z \Rightarrow \tilde p_i \circ \Delta_Z \circ m = m$
	
	$\Rightarrow (m,m)_S = \Delta_Z \circ m$\end{minipage}
	
	$\Rightarrow \underbrace{(f,g)_S}_{=:h} \circ j = (f \circ j, f \circ j) = \Delta_f \circ (f \circ j)$
	
	$h(j(U)) \subseteq \Delta_Y(Y) \Rightarrow j(U) \subseteq h^{-1}(\Delta_Y(Y))$. $\Delta_Y(Y)$ ist abgeschlossen $\xRightarrow{h \text{ stetig}} h^{-1}(\Delta_Y(Y))$ ist abgeschlossen $\xRightarrow{U \text{ dicht}} h^{-1}(\Delta_Y(Y)) = X \Rightarrow h(X) \subseteq \Delta_Y(Y) \Rightarrow \forall \ y \in h(X): p_1(y) = p_2(y)$. F"ur alle $x \in X$ gilt $f(x) = p_1(h(x)) = p_2(h(x)) = g(x)$ $\Rightarrow$ topologisch: $f = g$
	
	\emph{Noch zu zeigen:} $f^\# = g^\#: \calO_Y \to f_* \calO_X$, beziehungsweise $f^\# - g^\# = 0$
	
	\begin{description}[\setlabelstyle{\itshape}]
	\item[Behauptung:]
		F"ur alle $V \subseteq Y$ offen, affin, $W \subseteq f^{-1}(Y)$ offen, affin gilt $f = g: W \to V \Rightarrow f = g$
	\item[Denn:]
		$Y = \bigcup_{i \in I} V_i$ affine, offene "Uberdeckung, $V' \subseteq Y$ offen, $S \in \calO_Y(V')$
	\end{description}
	
	$t := (f_{V'}^\# - g_{V'}^\#)(S) \in \calO_X(f^{-1}(V')) = 0 \Leftrightarrow t_i := t|_{f^{-1}(V_i \cap V')} = 0 \ \forall \ i \in I$, $f^{-1}(V_i \cap V') = \bigcup W_j$ offen, affin. Nach Voraussetzung: $f = g: W_j \to V_i \cap V' \Rightarrow t_i|_{W_j} = (f^\#|_{W_j} - g^\#|_{W_j})(S) = 0 \xRightarrow{\text{Garbe}} t_i = 0$
	
	Sei also  \OE $X = \Spec B$, $Y = \Spec A$. Seien $\varphi, \psi: A \to B$ Ringhomomorphismen zu $f, g$, $f|_U = g|_U$ als Garbenmorphismus $\Rightarrow$ f"ur alle $\p \in U$ ist $\varphi_\p = \psi_\p: A_{\varphi^{-1}(\p)} \to B_\p$
	
	Sei $a \in A$. \emph{Annahme:} $\varphi(a) \ne \psi(a)$
	
	\begin{description}[\setlabelstyle{\itshape}]
	\item[Behauptung:]
		$D(\varphi(a) - \psi(a)) = \emptyset$, sonst $U \cap D(\varphi(a) - \psi(a)) \ne \emptyset$, enth"alt $q$
	\end{description}
	
	$\Rightarrow 0 = \varphi_q(\frac{a}{1}) - \psi_q(\frac{a}{1}) = \frac{\varphi(a) - \psi(a)}{1} \in B_q \Leftrightarrow t \notin q: \underset{\notin q}{t}(\underbrace{\varphi(a) - \psi(a)}_{\notin q}) = \underset{\in q}{0} \in B \lightning$
	
	$\xRightarrow[\text{Aufg. 1}]{\text{Blatt 4}} \varphi(a) - \psi(a) \in \sqrt 0 \overset{B \text{ red.}}{=} \{0\}$
\item
	$X$ aus Aufgabe 2 ist nicht separiert $\leadsto$ w"ahle das $X$ als $Y$
	
	$\Rightarrow Y = U \cup V$, $f: \Spec k[x] \overset{\sim}{\to} U$, $g: \Spec k[x] \overset{\sim}{\to} V$
	
	$D(X) \xrightarrow{f|_{D(X)}} U \cap V \xrightarrow{g^{-1}|_{U \cap V}} D(X)$ ist Identit"at $\Rightarrow f \equiv g$ auf $D(X) \subseteq \Spec k[x]$ offen dicht, aber $f((x)) \ne g((x))$.
	
	In der "Ubung hat noch ein Beispiel daf"ur gefehlt, dass die Aussage aus a) nicht gilt, falls $X$ nicht reduziert ist.
	
	Ist $X = \Spec k[x,y]/(xy,y^2)$, $Y = \Spec k[x,y]$ und seinen $\overline{x}$ und $\overline{y}$ die Restklassen von $x$ und $y$ in $k[x,y]/(xy,y^2)$. Dann sind
		\[\varphi : k[x,y] \to k[x,y]/(xy,y^2)\ ,\quad x\mapsto \overline{x}\ ,\quad y\mapsto \overline{y}\]
	und
		\[\psi : k[x,y] \to k[x,y]/(xy,y^2)\ ,\quad x\mapsto \overline{x}\ ,\quad y\mapsto 0\]
	zwei unterschiedliche Ringhomomorphismen, liefern also unterschiedliche Schemamorphismen
		\[f: \Spec k[x,y]/(xy,y^2) \to \Spec k[x,y]\]
	und
		\[g: \Spec k[x,y]/(xy,y^2) \to \Spec k[x,y].\]
	Diese sind beide abgeschlossene Einbettungen mit Bild jeweils $V(xy,y^2) = V(y) \subset \Spec k[x,y]$, also die ``$x$-Achse'' in der affinen Ebene. Weiter ist $D(x) \cap V(y)$ offen und dicht in $V(y)$, also auch $D(\overline{x}) = f^{-1}(D(x)) = g^{-1}(D(x)) \subset \Spec k[x,y]/(xy,y^2)$ offen und dicht. Wir behaupten dass $f$ und $g$ auf $D(\overline{x})$ "ubereinstimmen. Dazu reicht es, zu zeigen, dass die induzierten Morphismen
		\[\varphi_{{x}} = \psi_{{x}} : k[x,y,x^{-1}] \to (k[x,y]/(xy,y^2))_{\overline{x}}\]
	"ubereinstimmen. Aber in $(k[x,y]/(xy,y^2))_{\overline{x}}$ gilt $\overline{x}\overline{y} = 0$ und $0 = \overline{x}^{-1}\overline{x}\overline{y} = \overline{y}$, also stimmen $\varphi_x$ und $\psi_x$ auf $x$ und $y$ -- und damit "uberall -- "uberein.
\end{enumerate}

\end{Loes}
 
%-------------------------- Uebung 8 --------------------------
\newpage
\section{19. Juni 2012}
\setcounter{Aufg}{0}
\setcounter{Loes}{0}

\begin{Aufg}
Zeige, dass f"ur Morphismen von noetherschen Schemata gilt:
\begin{enumerate}%[topsep=0cm, parsep=0cm, label=\alph*)]
 \item Die Verkettung von separierten Morphismen ist separiert.
 \item Seien $f \colon X \to Y$ und $g \colon Y \to Z$ Schemamorphismen. Ist $g\circ f$ separiert, so auch $f$.
 \item Die Eigenschaft, separiert zu sein, ist stabil unter Basiswechsel.
\end{enumerate}
\end{Aufg}

\begin{Aufg}
Es sei $k$ ein algebraisch abgeschlossener K"orper. 
\begin{enumerate}%[topsep=0cm, parsep=0cm, label=\alph*)]
 \item Zeige: Ist $Y \subset \IP^2(k)$ eine irreduzible Kurve und $y \in \IP^2_k$ ihr generischer Punkt, so ist $\calO_{\IP^2_k,y}$ ein diskreter Bewertungsring.
 \item Finde eine irreduzible quasiprojektive Fl"ache $X$ "uber $k$ und eine irreduzible Kurve $Y \subset X$ mit generischem Punkt $y \in t(X)$, so dass $\calO_{t(X),y}$ kein diskreter Bewertungsring ist.
\end{enumerate}
\end{Aufg}

\begin{Aufg}
Es sei $X$ ein Schema von endlichem Typ "uber einem K"orper $k$, d.h. $X \to \Spec k$ ist von endlichem Typ. Zeige:
\begin{enumerate}%[topsep=0cm, parsep=0cm, label=\alph*)]
 \item Ein Punkt $x\in X$ ist genau dann abgeschlossen, wenn die K"orpererweiterung $\kappa(x) / k$ endlich ist. Hierbei sei $\kappa(x)$ der Restklassenk"orper im Punkt $x$.
 \item Die abgeschlossenen Punkte von $X$ sind dicht in $X$.
 \item Wenn $X$ nicht von endlichem Typ "uber einem K"orper ist, dann stimmen a) und b) im Allgemeinen nicht.
\end{enumerate}
\textit{Hinweis: Erinnere dich an die algebraische Version von Hilberts Nullstellensatz.}
\end{Aufg}


%-------------------------- Uebung 9 --------------------------
\newpage
\section{26. Juni 2012}
\setcounter{Aufg}{0}
\setcounter{Loes}{0}

\begin{Aufg}
Ein Schemamorphismus $f\colon X\to Y$ hei\ss t \textbf{quasi-endlich}, wenn $f^{-1}(y)$ f"ur alle $y\in Y$ eine endliche Menge ist.

Finde ein Beispiel f"ur einen Schemamorphismus $f\colon X\to Y$, so dass
\begin{enumerate}%[topsep=0cm, parsep=0cm, label=\alph*)]
 \item $f$ lokal von endlichem Typ, aber nicht von endlichem Typ ist.
 \item $f$ von endlichem Typ, aber nicht endlich ist.
 \item $f$ quasi-endlich, aber nicht endlich ist.
 \item $f$ eigentlich, aber nicht endlich ist.
 \item $f$ quasi-endlich, aber nicht lokal von endlichem Typ ist.
 \item $f$ von endlichem Typ, aber nicht quasi-endlich ist.
\end{enumerate}
\end{Aufg}


\begin{Aufg}
Ein Schemamorphismus $f\colon X \rightarrow Y$ hei\ss t \textbf{projektiv}, wenn es f"ur ein $n \in \N_0$ ein kommutatives Diagramm 
\begin{minipage}{4cm}
\begin{tikzpicture}
    \matrix (m) [matrix of math nodes, row sep=2em, column sep=2em, text height=1ex, text depth=0.25ex]{
          X & & \mathbb{P}^n_Y\\
          & Y \\
	  };
    \path[->, font=\small]
	  (m-1-1) edge node[auto,swap] {$f$}  (m-2-2)
      (m-1-3) edge node[auto] {$\textrm{pr}_Y$}   (m-2-2);
    \path[right hook->, font=\small]
      (m-1-1) edge node[above] {$i$} (m-1-3);
\end{tikzpicture}
\end{minipage}
gibt, in dem $i$ eine abgeschlossene Einbettung ist. Dabei sei $\mathbb{P}^n_Y \coloneqq \mathbb{P}^n_{\Z} \times_{\Spec(\Z)} Y$.

Zeige:
\begin{enumerate}%[topsep=0cm, parsep=0cm, label=\alph*)]
\item Ist $Y=\Spec(R)$ affin, so ist $\mathbb{P}^n_Y \cong \mathbb{P}^n_R$.
\item Die Komposition von projektiven Morphismen ist projektiv.
\item Abgeschlossene Einbettungen sind projektiv.
\item Projektiv zu sein ist stabil unter Basiswechsel.
\item Projektive Morphismen noetherscher Schemata sind eigentlich.
\end{enumerate}

\textit{Hinweis:} Erinnere dich an die Segre-Einbettung von Blatt 7, Aufgabe 4, aus der Algebraischen Geometrie I.
\end{Aufg}


\begin{Aufg}
Es sei $X$ eine nichtsingul"are, projektive Kurve "uber einem algebraisch abgeschlossenen K"orper $k$. Zeige:
\begin{enumerate}%[topsep=0cm, parsep=0cm, label=\alph*)]
 \item F"ur jedes $f \in k(X)^\times$ ist $\mathcal{L}(\textrm{div}(f))$ isomorph zu $\mathcal{O}_X$.
 \item F"ur jeden Divisor $D$ auf $X$ gibt es eine offene "Uberdeckung $X = \bigcup_{i \in I} U_i$ %von $X$ 
 und rationale Funktionen $f_i \in k(X)$, sodass $\mathcal{L}(D)(U_i) = \frac{1}{f_i} \mathcal{O}_X(U_i)$ f"ur jedes $i \in I$ gilt.
 \item F"ur je zwei Divisoren $D_1$ und $D_2$ auf $X$ ist $\mathcal{L}(D_1 + D_2)$ isomorph zu $\mathcal{L}(D_1) \otimes_{\mathcal{O}_X} \mathcal{L}(D_2)$.
\end{enumerate}
\end{Aufg}


\begin{Loes}
Schemamorphismus $f: X \to Y$ quasi-endlich $\Leftrightarrow \forall y \in Y: |f^{-1}(y)| < \infty$

\begin{center}\begin{tikzpicture}
	\node (0) at (0,1.5) {eigentlich};
	\node (1) at (0,0) {von endlichem Typ};
	\node (2) at (-3.5,0) {endlich};
	\node (3) at (-6.5,0) {quasi-endlich};
	\node (4) at (5,0) {lokal von endlichem Typ};
	
	\draw[double,-implies] (0) -- (1);
	\draw[double,-implies] (2) -- (1);
	\draw[double,-implies] (2) -- (3);
	\draw[double,-implies] (1) -- (4);
	
	\draw[double,-implies] (4) to[out=205,in=335] node[sloped]{$/$}node[below]{a)}(1);
	\draw[double,-implies] (1) to[out=205,in=335] node[sloped]{$/$}node[below]{b)}(2);
	\draw[double,-implies] (3) to[out=30,in=150] node[sloped]{$/$}node[above]{c)}(2);
	\draw[double,-implies] (0) -- node[sloped]{$/$}node[above]{d)}(2);
	\draw[double,-implies] (node cs: name=3, angle=210) to[out=330,in=210] node[sloped]{$/$}node[below]{e)} (node cs: name=4, angle=330);
	\draw[double,-implies] (1) to[out=225,in=315] node[sloped]{$/$}node[above]{f)}(3);
\end{tikzpicture}\end{center}
\begin{enumerate}[a)]
\item
	$Y = \Spec k$, $X = \bigcup\limits_{i \in N}^\cdot \Spec k[t]$, $k \hookrightarrow k[t]$ induziert $f: X \to Y$. $f$ ist lokal von endlichem Typ. $X = \Spec (\underbrace{\prod\limits_{i \in \N} k[t]}_{\substack{\text{nicht endl.}\\\text{erz. "uber }k}})$
\item
	$f$ von endlichem Typ, aber nicht endlich.
	
	$X = \Spec k[t]$, $Y = \Spec k$, $k \hookrightarrow k[t]$ induziert $f: X \to Y$, $f^{-1}(Y) = \Spec k[t]$, $k[t]$ nicht endlich erzeugt als $k$-Modul.
\item
	Sei $L|K$ unendliche K"orpererweiterung.	$\Rightarrow L$ ist nicht endlich erzeugt als $k$-Modul $\Rightarrow f: \Spec L \to \Spec k$ nicht endlich. $f^{-1}(\Spec k) = \Spec L$ ist endlich.
\item
	$f: \IP_k^n \to \Spec k$ ist eigentlich, $f^{-1}(\Spec k) = \IP_k^n$ ist nicht affin.
\item
	Beispiel aus c): $L$ ist nicht endlich erzeugt als $k$-Algebra.
\item
	Beispiel aus b): $f^{-1}(\Spec k) = \Spec k[t]$ ist unendlich.
\end{enumerate}
\textbf{Anmerkungen zur L"osung:}
Die Aufgabe haben wir in der "Ubung bereits gel"ost. Da aber die Diskussion zu der L"osung zu einem falschen Ende gekommen ist, hier noch eine Anmerkung:

In der Definition von quasi-endlich ist mit $f^{-1}(y)$ die Faser (als Schema) "uber $y$ gemeint.
In meinen Gegenbeispielen f"ur c), e), f) hatte ich immer $Y = \Spec k$, also $Y$ einelementig, gew"ahlt.
Auf "Ubungsblatt 6 Aufgabe 2 haben wir aber schon eingesehen, dass $f^{-1}(Y)$, aufgefasst als topologischer Raum, hom"oomorph zum Urbild von $Y$ unter $f$ ist, wenn wir $f$ als stetige Abbildung zwischen den topologischen R"aumen $X$ und $Y$ auffassen. In meinen Gegenbeispielen gab es daher keinen Unterschied zwischen \quot{topologisches urbild nehmen} und \quot{Ausrechnen der Faser mit anschlie\ss endem Vergessen der Strukturgarbe}.
\end{Loes}

\begin{Loes}
$f: X \to Y$ projektiv $:\Leftrightarrow \exists\ n \in \N$ mit \begin{tikzpicture}[baseline=-4]
	\node (0) at (-1,0) {$X$};
	\node (1) at (1,0) {$\IP_Y^n$};
	\node (2) at (1,-1) {$Y$};
	\node at (0.5,-0.35) {\tiny{\schraffiert}};
	
	\draw[right hook->] (0) --node[above,font=\scriptsize]{$i$} (1);
	\draw[->] (1) --node[right,font=\scriptsize]{$\pr_Y$} (2);
	\draw[->] (0) --node[below,font=\scriptsize]{$f$} (2);
\end{tikzpicture}, $i$ abgeschlossene Einbettung.
\begin{center}\begin{tikzpicture}
	\node (0) at (0,1.5) {$\IP_Y^n$};
	\node[anchor=east] (1) at (-2.5,0) {$\Proj(\Z[X_0,\ldots ,X_n]) = \IP_\Z^n$};
	\node[anchor=west] (2) at (2.5,0) {$Y$};
	\node (3) at (0,-1.5) {$\Spec \Z$};
	
	\draw[->] (0) -- (node cs: name=1, anchor=north east);
	\draw[->] (0) -- node[above, font=\scriptsize]{$\pr_Y$}(2);
	\draw[->] (node cs: name=1, anchor=south east) -- (3);
	\draw[->] (2) -- (3);
\end{tikzpicture}\end{center}
\begin{enumerate}[a)]
\item
	\emph{Zu zeigen:} $Y = \Spec R \Rightarrow \IP_Y^n \cong \IP_R^n$
	
	$D_+(X_i) \cong \Spec \Z[\frac{X_0}{X_i},\ldots ,\frac{X_n}{X_i}] \cong \A_\Z^n$ "uberdecken $\IP_\Z^n$
	
	$\A_n^\Z \X_{\Spec \Z} \Spec R = \Spec( \Z[\frac{X_0}{X_i},\ldots ,\frac{X_n}{X_i}] \otimes_\Z R) = \Spec(R[\frac{X_0}{X_i}, \ldots ,\frac{X_n}{X_i}]) = \A_R^n$
	
	$\Rightarrow \IP_Y^n = \IP_\Z^n \X \Spec R \cong \IP_R^n$
\item
		\begin{description}[\setlabelstyle{\bfseries}]
	\item[Behauptung 1:]
		Abgeschlossene Einbettung zu sein ist stabil unter Basiswechsel.
	\item[Beweis:]
		Sei $i: X \hookrightarrow Y$ abgeschlossene Einbettung, $g: Z \to Y$
		
		\begin{minipage}[c]{0.3\textwidth}\begin{tikzpicture}
		\node (0) at (-1.5,1) {$X \X_Y Z$};
		\node (1) at (1.5,1) {$X$};
		\node (2) at (-1.5,-1) {$Z$};
		\node (3) at (1.5,-1) {$Y$};
				
		\node at (0,0) {$\schraffiert$};
				
		\draw[->] (0) -- (1);
		\draw[->] (0) -- node[left]{$i'$} (2);
		\draw[->] (2) -- node[below]{$g$} (3);
		\draw[->] (1) -- node[right]{$i$} (3);
		\end{tikzpicture}\end{minipage}
		\begin{minipage}[c]{0.55\textwidth}\emph{Zu zeigen:} $i'$ ist abgeschlossene Einbettung
	
		$i'$ ist injektiv $\checkmark$ (Monotonie, UAE)\end{minipage}
		
		Sei \OE $X$, $Y$, $Z$ affin, $X = \Spec R$, $Y = \Spec S$, $Z = \Spec 1$.
		
		\begin{minipage}[c]{0.3\textwidth}\begin{tikzpicture}
		\node (0) at (-1.5,1) {$R \otimes_S T$};
		\node (1) at (1.5,1) {$R$};
		\node (2) at (-1.5,-1) {$T$};
		\node (3) at (1.5,-1) {$S$};
				
		\node at (0,0) {$\schraffiert$};
				
		\draw[->] (1) -- (0);
		\draw[->] (2) -- (0);
		\draw[->] (3) -- (2);
		\draw[->>] (3) -- (1);
		\end{tikzpicture}\end{minipage}
		\begin{minipage}[c]{0.55\textwidth}$R$ wird als $S$-Modul von $1$ erzeugt $\Rightarrow \{1 \otimes t \ \vert \ t \in T\}$ erzeugt $R \otimes_S T$ als $S$-Modul.
		
		$\Rightarrow \begin{array}{ccc} T &\to& R \otimes_S T \\ t &\mapsto& 1 \otimes t \end{array}$ surjektiv\end{minipage}
	\end{description}

	 Seien also $f \colon X \to Y$ und $g \colon Y \to Z$ projektiv und $m, n$ die nat"urlichen Zahlen, f"ur die es eine abgeschlossene Einbettungen $i \colon X \hookrightarrow \IP^n_Y$ bzw. $i' \colon Y \hookrightarrow \IP^m_Z$ gibt, "uber die $f$ bzw. $g$ faktorisiert. 
 
 Betrachte das Faserprodukt von $\IP^n_{\Z} \times_{\Spec \Z} \IP^m_Z$ und $Y$ "uber $\IP^m_Z$.
 Da f"ur Schemata allgemein $A \times_B B = B$ und $(A \times_R B) \times_S C =  A \times_R (B \times_S C)$ gilt, gilt auch $\IP^n_{\Z} \times_{\Spec \Z} \IP^m_Z \times_{\IP^m_Z} Y \cong \IP^n_Y$. Somit ist das obere rechte Viereck im kommutativen Diagramm 
 
 \begin{minipage}{7.9cm}
 \centering
\begin{tikzpicture}
     \matrix (m) [matrix of math nodes, row sep=2em, column sep=2em]{
		  X & \IP^n_Y & \IP^n_{\Z} \times_{\Spec \Z} \IP^m_Z\\
		  & Y & \IP^m_Z\\
		  & & Z \\
	  };
    \path[->, font=\footnotesize]
    (m-1-1) edge node[auto,swap] {$f$}  (m-2-2)
    (m-1-2) edge node[auto] {$\textrm{pr}_Y$}  (m-2-2)
    (m-2-2) edge node[auto,swap] {$g$}  (m-3-3)
    (m-1-3) edge (m-2-3)
    (m-2-3) edge node[auto] {$\textrm{pr}_Z$}  (m-3-3);
    \path[right hook->, font=\footnotesize]
    (m-1-1) edge node[auto] {$i$}  (m-1-2)
    (m-1-2) edge node[auto] {$\tilde{i}$}  (m-1-3)
    (m-2-2) edge node[auto] {$i'$}  (m-2-3);
\end{tikzpicture}
\end{minipage}
\begin{minipage}{7cm}
  ein Faserprodukt und nach Behauptung 1 ist damit $\tilde{i}$ eine abgeschlossene Einbettung.
\end{minipage}

 Weiter erinnern wir uns an die Segre-Einbettung $\Psi \colon \IP^n(k) \times \IP^m(k) \to \IP^N(k)$ mit $N = (n+1)(m+1)-1$ von "Ubungsblatt 7 Aufgabe 3 aus der Algebraischen Geometrie I. Dort haben wir gelernt, dass das Bild abgeschlossen in $\IP^N(k)$ ist, n"amlich $V(J)$ f"ur ein $J$, das wir konkret angegeben hatten. 
 Nun kann man sich klar machen, dass die durch $J$ gegebenen Gleichungen alle schon "uber $\Z$ definiert sind und die Konstruktion somit auch "uber $\Z$ klappt. Alternativ (da wir das Tensorprodukt dieses Semester nicht mehr scheuen) definieren wir in Analogie zu $\Psi$ einen Schemamorphismus $\Psi' \colon \IP^n_{\Z} \times \IP^m_{\Z} \to \IP^N_{\Z}$ lokal auf $D_+(x_r) \times D_+(y_s)$ durch den surjektiven Ringhomomorphismus
 $$\begin{array}{ccc}
 \Z[\frac{z_{ij}} {z_{rs}} \mid (i,j) \in \{0,\dots,n\}\times \{0,\dots,m\}] & \longrightarrow & \Z[\frac{x_0}{x_r},\dots,\frac{x_n}{x_r}] \otimes \Z[\frac{y_0}{y_s},\dots,\frac{y_m}{y_s}] \\
 \frac{z_{ij}} {z_{rs}} & \mapsto & \frac{x_i}{x_r} \otimes \frac{y_j}{y_s}
 \end{array}.
 $$
Aus der lokalen Darstellung von $\Psi'$ folgt sofort, dass $\Psi'$ eine abgeschlossene Einbettung ist.

Basiswechsel mit dem Morphismus $\textrm{pr}_{\IP^N_{\Z}}$ liefert $j \colon \IP^n_{\Z} \times \IP^m_{\Z} \times Z \to \IP^N_Z$. Nach Behauptung 1 ist $j$ wieder eine abgeschlossene Einbettung. Die Komposition von abgeschlossenen Einbettungen ist eine abgeschlossene Einbettung, also ist $j \circ \tilde{i} \circ i$ eine abgeschlossene Einbettung und das folgende kommutative Diagramm beweist, dass $g \circ f$ ein projektiver Morphismus ist.
 \begin{center}
\begin{tikzpicture}
     \matrix (m) [matrix of math nodes, row sep=2em, column sep=2em]{
		  X & \IP^n_Y & \IP^n_{\Z} \times_{\Spec \Z} \IP^m_Z & \IP^N_Z\\
		  & Y & \IP^m_Z\\
		  & & Z \\
	  };
    \path[->, font=\footnotesize]
    (m-1-1) edge node[auto,swap] {$f$}  (m-2-2)
    (m-1-2) edge node[auto] {$\textrm{pr}_Y$}  (m-2-2)
    (m-2-2) edge node[auto,swap] {$g$}  (m-3-3)
    (m-1-3) edge (m-2-3)
    (m-2-3) edge node[auto,swap] {$\textrm{pr}_Z$}  (m-3-3)
    (m-1-4) edge node[auto] {$\textrm{pr}_Z$} (m-3-3);
    \path[right hook->, font=\footnotesize]
    (m-1-1) edge node[auto] {$i$}  (m-1-2)
    (m-1-2) edge node[auto] {$\tilde{i}$}  (m-1-3)
    (m-2-2) edge node[auto] {$i'$} (m-2-3)
    (m-1-3) edge node[auto] {$j$} (m-1-4);
\end{tikzpicture}
\end{center}
\item
	\emph{Zu zeigen:} abgeschlossene Einbettungen sind projektiv
	
	$\IP_\Z^0 \cong \Spec \Z \Rightarrow Y \cong Y \X_{\Spec \Z} \Spec \Z \cong Y \X \IP_\Z^0 = \IP_Y^0$
	\begin{center}\begin{tikzpicture}
		\node (0) at (0,0) {$Y$};
		\node (1) at (-2.5,0) {$X$};
		\node (2) at (2.5,0) {$\IP_Y^0$};
		
		\draw[->] (1) --node[above,font=\scriptsize]{$f$} (0);
		\draw[->] (0) --node[above,font=\scriptsize]{$\sim$} (2);
		\draw[->] (1) to[out=30, in=150] node[above,font=\scriptsize]{$i$} (2);
		\draw[->] (2) to[out=225, in=315] node[above,font=\scriptsize]{$\sim$}node[below,font=\scriptsize]{$\pr_Y$} (0);
	\end{tikzpicture}\end{center}
	$\Rightarrow f$ projektiv

\item
	Sei $f \colon X \to Y$ projektiv und $g \colon Y' \to Y$ ein beliebiger Schemamorphismus (der Basiswechsel).
 
  \begin{minipage}{5.5cm}
\begin{tikzpicture}
     \matrix (m) [matrix of math nodes, row sep=2em, column sep=2em]{
		  X \times_Y Y' & X & \IP^n_{Y}\\
		  Y' & Y\\
	  };
    \path[->, font=\footnotesize]
    (m-1-1) edge node[auto] {$\textrm{pr}_X$}  (m-1-2)
		    edge node[auto,swap] {$f'$}  (m-2-1)
    (m-1-2) edge node[auto] {$f$}  (m-2-2)
    (m-2-1) edge node[auto,swap] {$g$}  (m-2-2)
    (m-1-3) edge node[auto] {$\textrm{pr}_Y$} (m-2-2);
    \path[right hook->, font=\footnotesize]
    (m-1-2) edge node[auto] {$i$}  (m-1-3);
\end{tikzpicture}
  \end{minipage}
 Zu zeigen ist nun, dass auch $f'$ projektiv ist.
 
 Dazu machen wir uns klar, dass $\IP^n_{Y'} = \IP^n_Y \times_Y Y'$ und, dass $\IP^n_{Y'} \times_{\IP^n_Y} X = Y' \times_Y X$. Dann erhalten wir das folgende kommutative Diagramm: 
   \begin{center}
\begin{tikzpicture}
     \matrix (m) [matrix of math nodes, row sep=2em, column sep=2em]{
		  X \times_Y Y' & X\\
		  \IP^n_{Y'} & \IP^n_{Y}\\
		  Y' & Y\\
	  };
    \path[->, font=\footnotesize]
    (m-1-1) edge node[auto] {$\textrm{pr}_X$}  (m-1-2)
		    edge node[auto,swap] {$i'$}  (m-2-1)
    (m-1-2) edge [bend left=50] node[auto] {$f$}  (m-3-2)
    (m-3-1) edge node[auto,swap] {$g$}  (m-3-2)
    (m-2-1) edge node[auto,swap] {$\textrm{pr}_{Y'}$} (m-3-1)
    (m-2-2) edge node[auto,swap] {$\textrm{pr}_{Y}$} (m-3-2)
    (m-2-1) edge (m-2-2);
    \path[right hook->, font=\footnotesize]
    (m-1-2) edge node[auto] {$i$}  (m-2-2);
\end{tikzpicture}
  \end{center}
 Der Morphismus $i$ ist nach Voraussetzung eine abgeschlossene Einbettung, nach Behauptung 1 also auch $i'$. 
 Die beiden inneren Vierecke sind Faserprodukte und somit auch das "au\ss ere Viereck. Damit gilt $\textrm{pr}_{Y'} \circ i' = f'$ und $f'$ ist ein projektiver Schemamorphismus.
\item
	 Seinen $X,Y$ noethersche Schemata und $f\colon X \to Y$ ein Schemamorphismus. Wir zeigen, dass $f = \textrm{pr}_Y \circ i$ eigentlich ist. 
 
 Nach Proposition \ref{8.10} ist $\IP^n_{\Z} \to \Spec \Z$ eigentlich. Alle beteiligten Schemata sind noethersch, also ist nach Folgerung \ref{8.9}. "`eigentlich"' stabil unter Basiswechsel und $\textrm{pr}_Y$ eigentlich.
 
 Abgeschlossene Einbettungen sind nach Bemerkung \ref{8.4} separiert und nach Bemerkung \ref{7.6} endlich (also insbesondere von endlichem Typ). Nach Behauptung~1 ist eine abgeschlossene Einbettung auch universell abgeschlossen. Damit ist $i$ eigentlich. 
 
 Die Komposition von eigentlichen Morphismen ist eigentlich und somit ist $f$ eigentlich. 
\end{enumerate}

\end{Loes}

\begin{Loes}
$X$ nichtsingul"are projektive Kurve "uber algebraisch abeschlossenem K"orper $k$ $\Rightarrow \forall p \in X: \calO_{X,p}$ ist diskreter Bewertungsring
\begin{enumerate}[a)]
\item
	\emph{Zu zeigen:} $\forall f \in k(X)^\X: \calL(\ddiv(f)) \cong \calO_X$
	
	$\ddiv(f) = \Sum_{p \in X} \ord_p(f) \cdot p$
	
	\[\calL(\ddiv(f)) = \{g \in k(X)^\X \vert \underbrace{\ddiv g + \ddiv f \ge 0}_{\substack{\Leftrightarrow \ddiv(g \cdot f) \ge 0 \\ \Rightarrow g \in \calL(\ddiv(f)) \Leftrightarrow g \cdot f \in \calO_X(U)}}\} \cup \{0\}\]
\item
	$D = \Sum_{p \in X} e_p \cdot p$, $P = \{p \in X \vert e_p \ne 0\} = \{p_1,\ldots ,p_n\}$. Sei $U \subseteq X$ offen, $|U \cap P| \le 1 \leadsto$ solche $U$ "uberdecken $X$.
	
	\emph{Zu zeigen:} $\calL(D)(U) \cong \calO_X(U)$
	
	\begin{description}{\setlabelstyle{\bfseries}}
	\item[1. Fall:] $U \cap P = \emptyset$
	
		$\Rightarrow \calL(D)(U) = \{g \in k(X)^\X \vert \forall \ p \in U: \ord_p(g) \ge 0\} \cup \{0\} = \calO_X(U)$
	\item[2. Fall:] $U \cap P = \{p_i\}$
		
		$\Rightarrow \calL(D)(U) = \{g \in k(X) \vert \forall \ q \in U, q \ne p_i: \ord(g) \ge 0 \text{ und } \ord_{p_i}(g) + e_{p_i} \ge 0\}$
		
		W"ahle Uniformisierende $t_i$ von $\calO_{X,p_i} \Rightarrow \ord_{p_i}(t_i) = 1$. Mache $U$ eventuell kleiner, so dass $\ord_q(t_i) = 0 \ \forall \ q \in U \setminus \{p_i\}$.
		
		$\Rightarrow \ord_{p_i} (t_i^{e_{p_i}} g) = e_{p_i} \cdot \underbrace{\ord_{p_i}(t_i)}_{=1} + \ord_{p_i}(g)) \ \forall \ q \in U \setminus \{p_i\}: \ord_q(t_i^{e_{p_i}} g) = \ord_q(g)$
		
		$\Rightarrow g \in \calL(D)(U) \Leftrightarrow t_i^{e_{p_i}} \cdot g \in \calO_X(U) \Leftrightarrow g \in \frac{1}{t_i^{e_{p_i}}} \cdot \calO_X(U)$
	\end{description}
\item
	\emph{Zeige:} $\calL(D_1 + D_2) \cong \calL(D_1) \otimes_{\calO_X} \calL(D_2)$
	
	$\calL(D_1) \otimes_{\calO_X} \calL(D_2)(U)$ ist garbenassoziiert zu $U \mapsto \calL(D_1)(U) \otimes_{\calO_X} \calL(D_2)(U)$. W"ahle "Uberdeckung $X = \bigcup\limits_{i \in I} U_i$ wie in b). Auf $U_i$ sei $t_i$ Uniformisierende.
	
	$\Rightarrow \calL(D_1)(U_i) = \frac{1}{t_i^{e_{p_i}}} \calO_X(U_i)$, $\calL(D_2)(U_i) = \frac{1}{t_i^{\tilde e_{p_i}}} \calO_X(U_i)$ \textcolor{gray}{($D_1 = \sum e_p \cdot p$, $D_2 = \sum \tilde e_p \cdot p$)}
		\[ \calL(D_1 + D_2)(U) = \frac{1}{t_i^{e_{p_i} + \tilde e_{p_i}}} \]
	$\Rightarrow$ lokal haben wir Isomorphismus
		\[\begin{array}{ccc} \calL(D_1)(U_i) \otimes_{\calO_X(U_i)} \calL(D_2)(U_i) &\to& \calL(D_1 + D_2)(U_i)\\
			\frac{f}{t_i^{e_p}} \otimes \frac{g}{t_i^{\tilde e_p}} &\mapsto& \frac{fg}{t_i^{e_p + \tilde e_p}}\\
			h(\frac{1}{t_i^{e_p}} \otimes \frac{1}{t_i^{\tilde e_p}}) &\mapsfrom& \frac{h}{t_i^{e_p + \tilde e_p}}\end{array}\]
\end{enumerate}
\end{Loes}

%-------------------------- Uebung 10 --------------------------
\newpage
\section{3. Juli 2012}
\setcounter{Aufg}{0}
\setcounter{Loes}{0}

\begin{Aufg}
 Sei $X$ ein topologischer Raum, 
$U \subseteq X$ eine offene Teilmenge von $X$ und $\mathcal{F}$ eine Garbe auf $U$. Ist $j\colon U \hookrightarrow X$ die Inklusion, so ist die \deftermspec{durch Null fortgesetzte Garbe}{Garbe!durch Nullen fortgesetzte}, $j_!(\mathcal{F})$, definiert als die zur Pr"agarbe
$$
V \mapsto \left\{ \begin{array}{cc} \mathcal{F}(V) & \textup{f"ur } V \subseteq U \\ 0 & \textup{sonst}
\end{array} \right. 
$$
assoziierte Garbe auf $X$.

Sei nun $X$ ein Schema und $U \neq X$ ein offenes, dichtes Unterschema von $X$.\\
Zeige, dass $j_!(\mathcal{O}_U)$ nicht quasikoh"arent ist.
\end{Aufg}


\begin{Aufg}
Seien $X$ und $Y$ noethersche Schemata und $f\colon X \to Y$ ein endlicher Morphismus. 
\begin{enumerate}%[topsep=0cm, parsep=0cm, label=\alph*)]
 \item Zeige: Ist $\mathcal{F}$ eine koh"arente $\mathcal{O}_X$-Modulgarbe, so ist $f_*(\mathcal{F})$ eine koh"arente $\mathcal{O}_Y$-Modulgarbe.
 \item Finde ein Beispiel daf"ur, dass a) selbst f"ur Variet"aten "uber einem algebraisch abgeschlossenen K"orper $k$ falsch ist, wenn $f$ nicht endlich ist.
\end{enumerate}
\end{Aufg}


\begin{Aufg}
Sei $X$ ein noethersches Schema und $\F$ eine koh"arente $\mathcal{O}_X$-Modulgarbe auf $X$. Zeige:
\begin{enumerate}%[topsep=0cm, parsep=0.1cm, label=\alph*)]
 \item Die Garbe $\F$ ist genau dann lokal frei, wenn f"ur jedes $x \in X$ der Halm $\F_x$ %, $x \in X$, 
 ein freier $\calO_{X,x}$-Modul ist.
 \item Ist $A$ ein lokaler Ring und sind $M$ und $N$ endlich erzeugte $A$-Moduln mit $M \otimes_A N \cong A$, dann gilt $M \cong A \cong N$. 
 
 \textit{Hinweis: Lemma von Nakayama}
 \item Gibt es eine koh"arente $\mathcal{O}_X$-Modulgarbe $\G$ auf $X$ mit $\F \otimes_{\calO_X} \G \cong \calO_X$, so ist $\F$ lokal frei vom Rang $1$, d.h. invertierbar.
\end{enumerate}
\end{Aufg}

\begin{Loes}
$X$ top. Raum, $U \subseteq X$, $\calF$ Garbe auf $U$, $j: U \hookrightarrow X$, Pr"agarbe $\calP: V \mapsto \begin{cases} \calF(V) &, V \subseteq U \\ 0 &, \text{sonst}\end{cases}$ $\leadsto$ assoziierte Garbe $j_!(\calF)$

$X$ Schema, $U \subseteq X$ offen, dicht, $U \ne X$

\emph{Zu zeigen:} $j_!(\calO_U)$ nicht quasikoh"arent

Sei $\Spec R \cong V \subseteq X$ offen, affin, irreduzibel, $V \not\subseteq U$ (existiert, da $U \ne X$). $U$ dicht $\Rightarrow V \cap U \ne 0$. Sei $x \in U \cap V \leadsto x \in \Spec R \Rightarrow \calO_{U,x} = R_X$
	\[ j_!(\calO_U)_X = \calO_{U \cap V, x} = \calO_{U,x} = R_X \]
	
\emph{Annahme:} $j_!(\calO_U)$ ist quasikoh"arent

$\Rightarrow j_!(\calO_U)|_V = \widetilde{j_!(\calO_U)(V)}$ \textcolor{gray}{(die Tilde geht "uber den ganzen Term)}

Sei $s \in j_!(\calO_U)(V)$, $x \in V \setminus U$. F"ur alle $U_x$ Umgebung von $x$ gilt $\calP(U_x) = 0$

$\Rightarrow j_!(\calO_U)_X = 0$\\
$\Rightarrow s_X = 0$\\
$\Rightarrow \exists \ U_x \subseteq V$ offene Umgebung von $x$ mit $s|_{D(f)} = 0$, insbesondere gibt es $f \in R \setminus \{0\}$ mit $s|_{D(f)} = 0$ $\Rightarrow \exists \ n \in \N: f^n\cdot s = 0 \xRightarrow[R\text{ nullteilerfrei}]{f\ne0} s = 0$

$\Longrightarrow j_!(\calO_U)|_V \equiv 0 \lightning$
\end{Loes}

\begin{Loes}
$X, Y$ noethersche Schemata, $f:X \to Y$ endlich
\begin{enumerate}[a)]
\item
	\emph{Zu zeigen:} $\calF$ koh"arente $\calO_X$-Modulgarbe $\Rightarrow f_*(\calF)$ koh"arente $\calO_Y$-Modulgarbe
	
	$\calF$ koh"arent $:\Leftrightarrow X$ noethersch und f"ur $\overset{=\Spec R}{\text{alle }U \subseteq X}$ offen, affin gibt es endlich erzeugten $R$-Modul $M_U$ mit $\calF|_U = \tilde M_U$
	
	Sei $\Spec S \cong V \subseteq Y$ offen, affin. $f^{-1}(V) = \Spec R$ affin, offen in $X$, $R$ endlich erzeugter $S$-Modul ($S$-Modul durch $f$).
	
	$\calF$ koh"arent $\Rightarrow \exists$ endlich erzeugter $R$-Modul $N$ mit $\calF|_{f^{-1}(V)} = \tilde N$, $f|_{f^{-1}(V)}: \Spec R \to \Spec S$, $\leadsto \alpha: S \to R$
	
	Bemerkung \ref{9.7} $\Rightarrow f_*\tilde N = _\alpha\tilde N$, wobei $_\alpha N = N$, aufgefasst als $S$-Modul durch $\alpha$.
	
	$N$ endlich erzeugter $R$-Modul, $R$ endlich erzeugter $S$-Modul $\Rightarrow _\alpha N$ endlich erzeugter $S$-Modul
	
	$\Rightarrow f_*\calF|_V = f_*\tilde N = _\alpha \tilde N \Rightarrow f_* \calF$ koh"arent
\item
	$k \hookrightarrow k[x]$, $k[x]$ nicht endl. erzeugbar als $k$-Modul $\Rightarrow f: \underbrace{\Spec k[x]}_{X} \to \underbrace{\Spec k}_{Y}$ nicht endlich
	
	$\calF = \calO_X \Rightarrow \calF$ koh"arent $\checkmark$
	
	$f_*\calO_X(Y) = \calF(X) = k[x]$ $\rightarrow$ nicht koh"arent
\end{enumerate}\end{Loes}

\begin{Loes}
$X$ noethersches Schema, $\calF$ koh"arenter $\calO_X$-Modulring
\begin{enumerate}[a)]
\item
	\emph{Zu zeigen:} $\calF$ lokal frei $\Leftrightarrow \forall \ x \in X: \calF_X$ ist freier $\calO_{X,x}$-Modul
	
	\textcolor{gray}{Hier habe ich in der "Ubung einen Fehler gemacht (das mit dem von den Relationen erzeugten Untermodul macht leider doch keinen Sinn), daher hier noch einmal mein Weg (Jonathans gilt noch), in richtig:}
	\begin{twosidedproof}
	\proofforward $\checkmark$
	\proofreverse
		Sei $U = \Spec R$ eine offene affine Umgebung von $x \in X$. Die Garbe $\F$ ist koh"arent, also existiert ein endlich erzeugter $R$-Modul $M$, so dass $\F|_U \cong \widetilde{M}$. Weiter ist $\F_x = M_x$ ein freier $\calO_{X,x} = R_x$-Modul. Wie in der "Ubung, w"ahle ich freie Erzeuger $[(U_1, s_1)], \dots, [(U_n,s_n)]$ von $\F_x$ als $R_x$-Modul. Da wir ohne Einschr"ankung $U$ durch $\bigcap_{i=1}^n U_i$ ersetzen k"onnen, gilt ohne Einschr"ankung $U = U_1 = \dots = U_n$. Dann k"onnen wir die $s_i$ zu einem Erzeugendensystem $\{s_1, \dots, s_n, s_{n+1}, \dots, s_{n+m}\}$ von $M$ erg"anzen. In $\F_x$ gilt 
 $s_{n+j} = \sum_{i=1}^n \alpha_{i,j} s_i$ mit $\alpha_{i,j} \in R_x$. Es gibt demnach eine offene affine Umgebung $V_j \subseteq U$ von $x$, auf der gilt $s_{n+j} = \sum_{i=1}^n \alpha_{i,j} s_i$. Wieder schneiden wir diese (endlich vielen) $V_j$ und finden im Schnitt eine offene, affine Umgebung $V$ von $x$ mit $\F|_V = \widetilde N$, wobei $N = \langle s_1|_V, \dots, s_n|_V \rangle _{R-Mod}$. Dieser Modul ist frei, den jede Relation in $N$ w"urde eine Relation in $\F_x$ ergeben (w"ahle $V$ irreduzibel - wie in Aufgabe 1), im Widerspruch dazu, das $M_x = N_x$ frei in den $s_i$ ist.
	\end{twosidedproof}
\item
	\emph{Zeige:} $A$ lokaler Ring, $M, N$ endlich erzeugte $A$-Moduln, $M \otimes_A N \cong A \Rightarrow M \cong N \cong A$
	
	Sei $m$ maximales Ideal von $A$, $k := \FakRaum{A}{m}$
		\[ \FakRaum{A}{m} \otimes_A M \otimes_A N \cong \FakRaum{A}{m} \otimes_A A = \FakRaum{A}{m} = k \]
	\begin{description}[\setlabelstyle{\itshape}]
	\item[Behauptung:] $k \otimes_A (M \otimes_A N) \cong (k \otimes_A M) \otimes_k (k \otimes_A N)$
	\item[Denn:] $a \otimes m \otimes n \mapsto (a \otimes m) \otimes (1 \otimes n)$
	\end{description}
	$A$-lineare Fortsetzung $\leadsto \varphi: k \otimes_A \ldots  \to \ldots $
	
	$(ab \otimes m) \otimes (1 \otimes n) = (a \otimes m) \otimes (b \otimes n) \Rightarrow \varphi$ surjektiv $\Rightarrow \dim_k (k \otimes_A M) \otimes_k (k \otimes_A N) \le 1$
	
	\emph{Annahme:} \quot{$=0$}
	
	$\Rightarrow k \otimes_A M = 0$ oder $k \otimes_A N = 0$
	
	\OE $k \otimes_A M = 0$ \textcolor{gray}{($k \otimes_A M = \FakRaum{A}{m} \otimes_A M = \FakRaum{M}{mM}$)}
	
	$\Rightarrow M = mM + 0 \xRightarrow{\text{Nakayama}} M = 0$\\
	$\Rightarrow A = M \otimes_A N = 0 \lightning$
	
	$\Rightarrow \dim_k (k \otimes_A M) = 0$
	
	$\FakRaum{M}{mM} \cong \FakRaum{A}{m} \xRightarrow{\text{Nakayama}} M \cong A$, $N \cong A$ analog
	
	$M = mM + (X)$ $A$-Modul
\item
	\emph{Zeige:} $\calG$ koh"arente $\calO_X$-Modulgarbe, $\calF \otimes_{\calO_X} \calG \cong \calO_X$ $\Rightarrow \calF$ lokal frei von Rang $1$
	
	$\underset{=\calF_X \otimes_{\calO_{X,x}} \calG_X}{(\calF \otimes_{\calO_X} \calG)_x} \cong \calO_{X,x}$
	
	$\xRightarrow{b)} \calF_X \cong \calO_{X,x} \xRightarrow{\text{Bew. v. a)}} \calF$ ist lokal frei von Rang $1$.
\end{enumerate}\end{Loes}


%-------------------------- Uebung 11 --------------------------
\newpage
\section{10. Juli 2012}
\setcounter{Aufg}{0}
\setcounter{Loes}{0}

%-------------------------- Uebung 12 --------------------------
\newpage
\section{17. Juli 2012}
\setcounter{Aufg}{0}
\setcounter{Loes}{0}

%-------------------------- Uebung 13 --------------------------
% \newpage
% \section{24. Juli 2012}
% \setcounter{Aufg}{0}
% \setcounter{Loes}{0}



%-_-_-_-_-_-_-_-_-_-_-_-_-_-_ Stichwortverzeichnis -_-_-_-_-_-_-_-_-_-_-_-_-_-_-_-_

\def\indexspace{\par\medskip}
%\printindex[default]
\printindex[default][\phantomsection\addcontentsline{toc}{chapter}{Stichwortverzeichnis}\vspace{-1.2em}]

%-_-_-_-_-_-_-_-_-_-_-_-_-_-_ Symbolverzeichnis -_-_-_-_-_-_-_-_-_-_-_-_-_-_-_-_

\printglossaries




%-_-_-_-_-_-_-_-_-_-_-_-_-_-_ Gaestebuch -_-_-_-_-_-_-_-_-_-_-_-_-_-_-_-_


\chapter{G\"astebuch}
Hier kann jeder, der g\"o\ss ere \"Anderungen oder Korrekturen am Skript vorgenommen hat, seinen Namen und einen Kommentar hinterlassen. Kleinigkeiten wie Rechtschreibfehler m\"ussen nicht unbedingt sein (und bei meinen Tippfehlern w\"are das G\"astebuch gr\"o\ss er als das Skript), aber ein korrigierter Satz, ein vervollst\"andigter Beweis, solche Sachen, einfach damit jemand der eine \"altere Version des Skriptes hat schnell den Unterschied finden kann. Oder hinterlasst einfach ein Paar nette Worte :-)

Zum Schreiben im G\"astebuch stehen euch folgende Umgebungen zur Verf\"ugung (neben den \"ublichen aus dem Skript):
\begin{center}\begin{minipage}{0.3\textwidth}\begin{verbatim}
  \begin{gast}
    ...
  \end{gast}
\end{verbatim}\end{minipage}
\begin{minipage}{0.3\textwidth}\begin{verbatim}  
  \begin{komm}
    ...  
  \end{komm}
\end{verbatim}\end{minipage}
\begin{minipage}{0.3\textwidth}\begin{verbatim}  
  \begin{korr}
    ...  
  \end{korr}
\end{verbatim}\end{minipage}\end{center}
G\"astebucheintr\"age sind f\"ur alle Arten von Eintr\"agen gedacht. Kommentare sollten nur f\"ur wichtige Sachen verwendet werden und auch blo\ss,  wenn man sie nicht direkt in das Skript einbauen kann. Korrekturen sind da um gr\"o\ss ere und wichtige Korrekturen festzuhalten, damit man schnell wei\ss was sich seit der letzten Version wichtiges ge\"andert hat.

Dieser Teil ist absichtlich am Ende, damit man ihn beim Drucken einfach weglassen kann. Wenn jemand einen weiteren Anhang einf\"ugen m\"ochte, dann tut das bitte \emph{vor} dem G\"astebuch.\\

\begin{gast}[von Aleks am 23. April 2012]
Ich habe das G\"astebuch angelegt. Ihr k\"onnt hier Aufz\"ahlungen verwenden:
\begin{enumerate}[1)]
\item
  Ist das nicht toll?
\item
  Ich kann sogar einen Link zu einem Eintrag setzen, wie zum Beispiel Definition \ref{1.3}
\end{enumerate}
\end{gast}

\begin{komm}[von Aleks am 13. Mai 2012]
Eine Sache, die mich immer gewurmt hat, ist dass das Skript einfach runtergeleiert aussah, es sah nicht wie ein Buch aus. Das war auch kein Wunder, schlie\ss lich hatte ich die Dokumentenklasse \verb!report! verwendet gehabt. Ich habe das endlich ge\"andert, und wo ich schon dabei war dachte ich mir, ich k\"onnte auch gleich diese \verb!KOMA Script! Klassen ausprobieren, also in diesem Fall \verb!scrbook!.

Das war eine gute Idee, die Klasse ist sehr anpassbar, ohne gro\ss en Aufwand, man muss sich nur in der riesigen Dokumentation zurechtfinden. Wo ich schon dabei war habe ich den Code ein wenig besser kommentiert und aufger\"aumt, falls jemand in Zukunft dieses Skript als Vorlage verwenden m\"ochte.
\end{komm}

\end{document}
