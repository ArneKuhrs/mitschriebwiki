\documentclass[11pt]{book}
\usepackage{amssymb}
\usepackage[ansinew]{inputenc}
\usepackage{german}
\usepackage[german]{babel}
\usepackage[T1]{fontenc}
\usepackage{geometry}
\usepackage{paralist}
\usepackage[explicit]{titlesec}
\usepackage{amsmath}
\usepackage{setspace}
\usepackage{polynom}
\usepackage{stmaryrd}
\usepackage[arrow, matrix, curve]{xy}
\usepackage{float}
\usepackage{titletoc}
\usepackage{pgfplots}
\usepackage{graphicx} 
\usepackage{wrapfig}
\usepackage{mathabx}
\usepackage{caption}
\usepackage{subcaption}
\usepackage{booktabs}
\allowdisplaybreaks
\usepackage{cmap}

% Meta-Daten fuer Latexki
\usepackage{latexki}
\title{Hyperbolische Geometrie}
\lecturer{Prof. Dr. Enrico Leuzinger}
\semester{Sommersemester 2016}
\scriptstate{complete}


%%%%THEOREM 

\usepackage[amsmath,thmmarks]{ntheorem}
\newcounter{dummy}
\numberwithin{dummy}{section}

\newtheorem{theorem}{Satz}[section]
\newtheorem{lemma}[theorem]{Lemma}
\newtheorem{proposition}[theorem]{Proposition}
\newtheorem{corollary}[theorem]{Korollar}
\newtheorem{motivation}[theorem]{Motivation}
\newtheorem{definition}[theorem]{Definition}
\newtheorem{remark}[theorem]{Bemerkung}
\newtheorem{example}[theorem]{Beispiel}
\newtheorem*{exampleo}[theorem]{Beispiel}
\newtheorem{bemdefini}[theorem]{Bemerkung + Definition}
\newtheorem{definibem}[theorem]{Definition + Bemerkung}
\newtheorem{erinndefini}[theorem]{Erinnerung + Definiton}
\newtheorem{erinner}[theorem]{Erinnerung}
\newtheorem{propdefini}[theorem]{Proposition + Definiton}
\newtheorem{definiprop}[theorem]{Definiton + Proposition}
\newtheorem{folg}[theorem]{Folgerung}
\newtheorem{problem}[dummy]{Aufgabe}
\newtheorem{erinnbemer}[theorem]{Erinnerung + Bemerkung}
\newenvironment{proof}[1][\it Beweis.]{\begin{trivlist} \item[\hskip \labelsep {#1}]}{\end{trivlist}}
\newenvironment{solution}[1][\it Lösung.]{\begin{trivlist} \item[\hskip \labelsep {#1}]}{\end{trivlist}}
\theoremstyle{nonumberbreak} 
\theoremseparator{:} 
\theoremindent0.5cm 
\theoremheaderfont{\scshape} 
\theorembodyfont{\normalfont} 
\theoremsymbol{\ensuremath{_\bigboxvoid}} 
\RequirePackage{amssymb} 
\qedsymbol{\ensuremath{_\bigboxvoid}}
\newenvironment{defin}[1][]{\ifthenelse{\equal{#1}{}}{\definition}{\definition[#1]}\rm}{\enddefinition}
\newenvironment{prob}[1][]{\ifthenelse{\equal{#1}{}}{\problem}{\problem[#1]}\rm}{\endproblem}
\newenvironment{motiv}[1][]{\ifthenelse{\equal{#1}{}}{\motivation}{\motivation[#1]}\rm}{\endmotivation}
\newenvironment{pr}[1][]{\ifthenelse{\equal{#1}{}}{\proof}{\proof[#1]}\rm}{\endproof}
\newenvironment{sol}[1][]{\ifthenelse{\equal{#1}{}}{\solution}{\solution[#1]}\rm}{\endsolution}
\newenvironment{ex}[1][]{\ifthenelse{\equal{#1}{}}{\example}{\example[#1]}\rm}{\endexample}
\newenvironment{exo}[1][]{\ifthenelse{\equal{#1}{}}{\exampleo}{\exampleo[#1]}\rm}{\endexampleo}
\newenvironment{bemdefin}[1][]{\ifthenelse{\equal{#1}{}}{\bemdefini}{\bemdefini[#1]}\rm}{\endbemdefini}
\newenvironment{erinndefin}[1][]{\ifthenelse{\equal{#1}{}}{\erinndefini}{\erinndefini[#1]}\rm}{\enderinndefini}
\newenvironment{erinnbem}[1][]{\ifthenelse{\equal{#1}{}}{\erinnbemer}{\erinnbemer[#1]}\rm}{\enderinnbemer}
\newenvironment{propdefin}[1][]{\ifthenelse{\equal{#1}{}}{\propdefini}{\propdefini[#1]}\rm}{\endpropdefini}
\newenvironment{definprop}[1][]{\ifthenelse{\equal{#1}{}}{\definiprop}{\definiprop[#1]}\rm}{\enddefiniprop}
\newenvironment{er}[1][]{\ifthenelse{\equal{#1}{}}{\erinner}{\erinner[#1]}\rm}{\enderinner}
\newenvironment{definbem}[1][]{\ifthenelse{\equal{#1}{}}{\definibem}{\definibem[#1]}\rm}{\enddefinibem}
\newcommand{\QED}{\hskip\hsize\hskip-\marginparwidth\hskip-\marginparsep\qedsymbol} 




%%%%ABKÜRZUNGEN
\newcommand{\spec}{\mathrm{Spec} \hspace{1pt} }
\newcommand{\p}{\mathfrak{p}}
\newcommand{\q}{\mathfrak{q}}
\newcommand{\K}{k}
\newcommand{\LL}{\mathcal{L}}
\newcommand{\ideal}{\subseteq}


%%Kategorien

\newcommand{\ringe}{\underline{\mathrm{Ringe}}}
\newcommand{\topraum}{\underline{\mathrm{Top}}}
\newcommand{\affsch}{\underline{\mathrm{Aff. Sch.}}}
\newcommand{\Hom}{\mathrm{Hom}\hspace{1pt}}
\newcommand{\sets}{\underline{\mathrm{Sets}}}
\newcommand{\Oxmod}{\underline{\mathcal{O}_X\textrm{-}\mathrm{Mod}}}
\newcommand{\Rmod}{\underline{R\textrm{-}\mathrm{Mod}}}
\newcommand{\AbX}{\underline{\mathcal{A}b}(X)}
\newcommand{\Ab}{\underline{\mathrm{Ab}}}
\newcommand{\kvec}{\underline{k\textrm{-}\mathrm{vec}}}
\newcommand{\grp}{\underline{\mathrm{Groups}}}
\newcommand{\topsp}{\underline{\mathrm{Top}}}
\newcommand{\topstar}{\underline{\mathrm{Top}}_*}
\newcommand{\hotop}{\underline{\mathrm{HoTop}}}
\newcommand{\hotopstar}{\underline{\mathrm{HoTop}}_{*}}
\newcommand{\toptwo}{\underline{\mathrm{Top}}^{(2)}}
\newcommand{\cov}{\underline{\mathrm{Cov}}}




%%ABKÜRZUNGEN
\newcommand{\bild}{\mathrm{Bild}\hspace{1-5pt} }
\newcommand{\Imm}{\mathrm{Im}\hspace{1.5pt}}
\newcommand{\arctanh}{\mathrm{arctanh} \hspace{1.5pt}}
\newcommand{\Ree}{\mathrm{Re}\hspace{1.5pt}}
\newcommand{\kernel}{\mathrm{ker}\ }
\newcommand{\Proj}{\mathrm{Proj} \hspace{1.5pt}}
\newcommand{\C}{\mathbb{C}}
\newcommand{\Z}{\mathbb{Z}}
\newcommand{\N}{\mathbb{N}}
\newcommand{\E}{\mathbb{E}}
\newcommand{\Loid}{\mathbb{L}}
\newcommand{\R}{\mathbb{R}}
\newcommand{\Sph}{\mathbb{S}}
\newcommand{\He}{\mathbb{H}}
\newcommand{\KK}{\mathbb{K}}
\newcommand{\RR}{\overline{\mathbb{R}}}
\newcommand{\CC}{\overline{\mathbb{C}}}
\newcommand{\D}{\mathbb{D}}
\newcommand{\F}{\mathcal{F}}
\newcommand{\G}{\mathcal{G}}
\newcommand{\obc}{\mathrm{ob}(\mathcal{C})}
\newcommand{\homcxy}{\mathrm{Hom}\hspace{1pt}_{\mathcal{C}}(X,Y)}
\newcommand{\la}{\longrightarrow}
\newcommand{\Img}{\mathrm{Im} \hspace{1.5pt}}
\newcommand{\id}{\mathrm{id}}
\newcommand{\homoecc}{\mathcal{H}\textit{omöo}(\CC)}
\newcommand{\homoekcc}{\mathcal{H}\textit{omöo}_K(\CC)}
\newcommand{\mob}{\textup{Möb}^+}
\newcommand{\amob}{\textup{Möb}}
\newcommand{\amobh}{\textup{Möb}(\mathbb{H})}
\newcommand{\mobh}{\textup{Möb}^+(\mathbb{H})}
\newcommand{\amobr}{\textup{Möb}(\overline{\mathbb{R}})}
\newcommand{\mobr}{\textup{Möb}^+(\overline{\mathbb{R}})}
\newcommand{\ssp}{\langle \cdot, \cdot \rangle}

\newcommand{\matx}[4]{\begin{pmatrix}[rr]#1 & #2 \\[-6pt] #3 & #4 \end{pmatrix}}

\newcommand{\matxx}[4]{\begin{pmatrix}[rr]#1 & #2 \\ #3 & #4 \end{pmatrix}}
\newcommand{\Matx}[9]{\begin{pmatrix}[rrr]#1 & #2 & #3 \\ #4 & #5 & #6 \\ #7 & #8 & #9 \end{pmatrix}}
\newcommand{\tx}[1]{\begin{pmatrix}[rr]1 & #1 \\[-6pt] 0 & 1 \end{pmatrix}}
\newcommand{\vecdr}[3]{\begin{pmatrix}[rr] #1 \\[-6pt] #2 \\[-6pt] #3 \end{pmatrix}}
\newcommand{\Vecdr}[3]{\begin{pmatrix}[rr] #1 \\ #2 \\ #3 \end{pmatrix}}

%%%%TITLES
\titlecontents{chapter}[2em]{\addvspace{2pc}\bfseries}{\contentslabel{1.7em}}{}{\titlerule*[0.5pc]{.}\contentspage}
\titlecontents{section}[5.5em]{}{\contentslabel{3.5em}}{}{\titlerule*[0.5pc]{.}\contentspage}
\renewcommand*\thechapter{\arabic{chapter}}
\titlespacing{\chapter}{0pt}{*5}{*1.5}
\titlespacing{\section}{0pt}{*5}{20pt}
\titlespacing{\subsection}{0pt}{16pt}{0pt}



%\titlespacing{\chapter}{0pt}{*5}{*1.5}
%\titlespacing{\section}{0pt}{*5}{20pt}
%\titlespacing{\subsection}{0pt}{16pt}{0pt}






\geometry{a4paper, top=30mm, left=25mm, right=25mm, bottom=30mm, headsep=10mm, footskip=15mm}
\renewcommand{\labelenumi}{(\roman{enumi})}
\renewcommand{\labelenumii}{(\arabic{enumii})}
\setlength{\parindent}{0pt}
\newcommand{\slant}[2]{{\raisebox{.1em}{$#1$}\left/\raisebox{-.1em}{$#2$}\right.}}
\newcommand{\bigslant}[2]{{\raisebox{.2em}{$#1$}\left/\raisebox{-.2em}{$#2$}\right.}}
\usepackage{graphicx}
\newcommand\tabrotate[1]{\rotatebox{90}{#1\hspace{\tabcolsep}}}
\newcommand\verschiebung[1][-.75\normalbaselineskip]{\hspace{#1}}
\makeatletter
\renewcommand*{\env@matrix}[1][*\c@MaxMatrixCols c]{
  \hskip -\arraycolsep
  \let\@ifnextchar\new@ifnextchar
  \array{#1}}
\makeatother
\renewcommand{\chaptermark}[1]{ 
  \markboth{ 
     \MakeUppercase{\thechapter \quad #1} 
  }{} 
} 
\renewcommand{\sectionmark}[1]{ 
  \markright{ 
     \MakeUppercase{\thesection \quad #1} 
  } 
}

\usepackage[hypertexnames=false]{hyperref}
\newenvironment{rcases}
  {\left.\begin{aligned}}
  {\end{aligned}\right\rbrace}


\begin{document}
\begin{titlepage}

\textrm{ }\\[64pt]

\begin{center}
{\fontsize{32}{32} \selectfont \textbf{Hyperbolische Geometrie}}
\end{center}
\textrm{ } \\[36pt]
\begin{center} \large{\textrm{gelesen von Prof. Dr. Enrico Leuzinger im Sommersemester 2016 am KIT}} \end{center}
\textrm{ } \\[320pt]
\begin{center} \large{\textit{Geschrieben in } \LaTeX \textit{ von Arthur Martirosian, arthur.martirosian.93@gmail.com}}\end{center}
\textrm{ }\\[24pt]
\begin{center} \large{\today} \end{center}

\end{titlepage}
\thispagestyle{empty}



\begin{spacing}{1.6}
\setcounter{tocdepth}{2}
\tableofcontents
\thispagestyle{empty}
\end{spacing}
\newpage



\begin{spacing}{1.4}
\thispagestyle{empty}


\chapter{Möbiustransformationen} %KAPITEL I
\setlength\abovedisplayshortskip{0pt}
\setlength\belowdisplayshortskip{10pt}
\setlength\abovedisplayskip{10pt}
\setlength\belowdisplayskip{10pt}

Ziel des Kapitels ist es, möglichst viele Transformationen der hyperbolischen Räume zu erhalten. 




\section{Die Riemannsche Zahlenkugel} %PARAGRAPH I.1

\thispagestyle{empty}




\begin{defin}
\begin{compactenum}
\item Sei $\C$ der Körper der komplexen Zahlen und setze $\CC:= \C \cup \{\infty\}$. $\CC$ heißt \textit{Riemannsche Zahlenkugel}. Auf ihr wollen wir im Folgenden eine Topologie definieren. Sei dazu für eine komplexe Zahl $z \in \C$ und $\epsilon >0$ 
$$U_{\epsilon}(z) := \left\{ w \in \C \ \vert \ \vert w-z \vert < \epsilon \right\},$$
$$U_{\epsilon}(\infty) := \left\{ w \in \C \ \vert \ \vert w \vert > \epsilon \right\}.$$
\item Eine Menge $U \subseteq \CC$ heißt \textit{offen}, falls für jeden Punkt $x \in U$ ein $\epsilon >0$ existiert mit $U_{\epsilon}(x) \subseteq U$.
\item $z \in \CC$ heißt \textit{Grenzwert} der Folge $(z_n)_{n \in \mathbb{N}} \subseteq \CC$, falls für jedes $\epsilon >0$ ein $N_0 := N_0(\epsilon)$ existiert, sodass $z_n \in U_{\epsilon}(z)$ für alle $n \geqslant N_0$ ist. In diesem Fall sagt mann, dass $(z_n)_{n \in \mathbb{N}}$ \textit{gegen} $z$ \textit{konvergiert} und schreibt $\lim_{n\to \infty} z_n = z$ oder $z_n \xrightarrow{n \to \infty}z$.
\item Eine Abbildung $f: \CC \la \CC$ heißt \textit{stetig in} $z \in \CC$, falls für alle konvergenten Folgen $(z_n)_{n \in \mathbb{N}} \subseteq \CC$ mit $z=\lim_{n \to \infty} z_n$ gilt:
$$f\left( \lim_{n \to \infty} z_n\right) = f(z) \overset{!}{=} \lim_{n \to \infty} f(z_n).$$
$f$ heißt \textit{stetig}, falls $f$ stetig ist in allen Punkten $z \in \CC$.

\end{compactenum}


\end{defin}



\begin{ex}
\begin{compactenum}
\item Die Folge $\left( \frac{1}{n}\right)_{n \in \mathbb{N}} \subseteq \CC$ konvergiert gegen $0$.
\item Die Folge $\left(n\right)_{n \in \mathbb{N}} \subseteq \CC$ konvergiert gegen $\infty \in \CC$.
\end{compactenum}
\end{ex}



\begin{remark}[Stereographische Projektion]
Betrachte die $2$-Sphäre $$\mathbb{S}^2 := \left\{(x,y,z) \in \mathbb{R}^3 \ \vert \ x^2+y^2+z^2=1 \right\}.$$ $\mathbb{S}^2$ ist abgeschlossen und beschränkt im $\mathbb{R}^3$, also kompakt (nach Heine-Borel). Sei $N=(0,0,1) \in \mathbb{S}^2$ der Nordpol und definiere
$$\sigma: \mathbb{S}^2 \setminus \{N\} \la \C, \qquad P=(x,y,z) \mapsto \overline{NP} \cap \C = \left( \frac{x}{1-z}, \frac{y}{1-z}, 0 \right).$$
Dann ist $\sigma$ die gewöhnliche stereographische Projektion, welche uns die Identität $\mathbb{S}^2 \setminus \{N\} \cong \mathbb{C} \cong \mathbb{R}^2$ liefert. Wir können $\sigma$ nun durch
$$\overline{\sigma}: \mathbb{S}^2 \la \CC, \qquad P \mapsto \begin{cases} \ \sigma(P), & \ \textrm{falls }P \neq N, \\ \ \infty, &\ \textrm{falls }P=N \end{cases}$$
auf die ganze Sphäre erweitern. 

\end{remark}

\hypertarget{propeinseinsvier}{}
\begin{proposition}
Die erweiterte stereographische Projektion $\overline{\sigma}: \mathbb{S}^2 \la \CC$ ist ein Homoömorphismus, das heißt $\sigma$ ist bijektiv und $\sigma, \sigma^{-1}$ sind stetig. Insbesondere ist damit die Riemannsche Zahlenkugel als stetiges Bild eines Kompaktums ebenfalls kompakt. 
\begin{pr}
Übung (\hyperlink{Aeinsdrei}{Aufgabe 1.3}). $\hfill \Box$
\end{pr}
\end{proposition}



\begin{remark}[Kreise in $\CC$]
Ein Kreis in $\CC$ ist entweder ein eukidischer Kreis in $\C \cong \mathbb{R}^2$ oder die Vereinigung einer euklidischen Geraden mit $\{\infty\}$. Dies wollen wir durch Gleichungen explizit beschreiben.
Betrachte den euklidischen Kreis mit Zentrum $z_0=x_0 + i y_0$ und Radius $r >0$. Dann liegt $z \in \C$ auf dem Kreis genau dann wenn gilt 
$$r^2= \vert z-z_0\vert^2 = (z-z_0)(\overline{z-z_0}) = z\overline{z} - z_0\overline{z}- \overline{z}_0z + z_0 \overline{z}_0.$$
Beschreibt nun $ax+by+c=0$ eine euklidische Gerade, so gilt für $z \in \C$ mit $x= \Ree(z)= \frac{z+\overline{z}}{2}$ und $y=i\Imm(z)=\frac{z-\overline{z}}{2}$ gerade

$$\frac{a}{2}(z + \overline{z}) + \frac{b}{2} i (z-\overline{z}) + c=\frac{1}{2}(a+bi)z + \frac{1}{2}(a-bi)\overline{z} + c =: \beta z + \overline{\beta} \overline{z} + c =0.$$

Wir sehen damit, dass jeder Kreis in $\CC$ gegeben ist als Lösungsmenge einer Gleichung der Form 
$$\alpha z \overline{z} + \beta z + \overline{\beta} \overline{z} + \gamma = 0$$
für gewisse $\beta \in \C$, $\alpha, \gamma \in \mathbb{R}$ (mit $\alpha=0$ genau dann, wenn der Kreis eine euklidische Gerade ist und $\alpha\neq0$ genau dann, wenn der Kreis ein euklidischer Kreis ist). Dabei ist $\infty$ eine Lösung, falls es eine Folge von Lösungen $(z_n)_{n \in \N}$ gibt mit $z_n \xrightarrow{n \to \infty} \infty.$

\end{remark}





\section{Möbiustransformationen} %%  1.2     Möbiustransformationen




Eine Abbildung $\phi: \CC \la \CC$ heißt \textit{Homöomorphismus}, falls $\phi$ bijektiv und $\phi, \phi^{-1}$ stetig sind. Definiere
$$\homoecc := \left\{ \phi: \CC \la \CC \ \vert \ \phi \textrm{ ist Homöomorphismus} \right\}.$$
Dann wird $\homoecc$ mit der Komposition von Abbildungen zur Gruppe.

\hypertarget{beispieleinszweieins}{}
\begin{ex} Sei $p=a_nz^n+a_{n-1}z^{n-1} + \ldots + a_1z + a_0$ ein komplexe Polynom. Dann ist die Abbildung
$$\phi_p: \CC \la \CC, \qquad z \mapsto \begin{cases} \ p(z) & \ \textrm{ falls } z \neq \infty, \\ \ \infty & \ \textrm{ falls } z=\infty. \end{cases}$$
ein Homöomorphismus genau dann, wenn $n=1$. In der Tat ist für $q(z)=\frac{1}{a}z-\frac{b}{a}$ die Abbildung $\phi_q$ eine stetige Umkehrabbildung. Ist umgekehrt $n=0$, so ist $\phi$ nicht injektiv und damit kein Homöomorphismus. Für $n \geqslant 2$ ist $\phi$ offenbar nicht bijektiv.
\end{ex}

\begin{definbem}
Es ist 
$$\homoekcc := \left\{ \phi \in \homoecc \ \vert \ \phi \textrm{ bildet Kreise auf Kreise ab } \right\}$$
die Menge der kreistreuen Homöomorphismen. Dabei gilt $\homoekcc \subsetneq \homoecc$, denn betrachte folgende Abbildung:

$$\phi: \CC \la \CC, \qquad z \mapsto \begin{cases} \ z& \ \textrm{ falls } \mathrm{Re}(z) \leqslant 0, \\ \ z + i Re(z) & \ \textrm{ falls } \mathrm{Re}(z) >0, \\ \ \infty & \ \textrm{ falls } z = \infty. \end{cases}$$
Dann kann man nachrechnen, dass $\phi$ ein Homöomorphismus ist, für den Kreis $\mathbb{R} \cup \{\infty\}$ aber $\phi(\mathbb{R} \cup \{\infty\})$ kein Kreis in $\CC$ ist.
\end{definbem}


\hypertarget{propeinszweidrei}{}
\begin{proposition}
Für $a,b \in \C$, $a \neq 0$ sei 
$$\rho: \CC \la \CC, \qquad z \mapsto \begin{cases} \ az + b & \ \textrm{ falls } z \neq \infty, \\ \ \infty & \ \textrm{ falls } z = \infty. \end{cases}$$
Dann ist $\rho \in \homoekcc$.
\begin{pr}
Nach \hyperlink{beispieleinszweieins}{Beispiel 1.2.1} ist $\rho$ ein Homöomorphismus. Es ist also lediglich die Kreistreue zu zeigen. Sei also $A \subseteq \CC$ ein Kreis. Dann ist $A$ die Lösungsmenge eine Gleichung
$$\alpha z \overline{z} + \beta z + \overline{\beta} \overline{z} + \gamma = 0, \qquad \alpha, \gamma \in \mathbb{R}, \ \beta \in \C.$$
Betrachte nun zwei Fälle:
\begin{compactenum}
\item[\textbf{Fall (a)}] $A$ ist eine euklidische Gerade, also 
$$A= \left\{ z \in \C \ \vert \ \beta z + \overline{\beta} \overline{z} + \gamma = 0 \right\} \cup \{\infty\}.$$
Wir müssen zeigen, dass die Punkte in $\rho(A)$ ebenfalls eine Kreisgleichung erfüllen. Sei also $w= az+b = \rho(z) \in \rho(A) \subseteq \CC$. Dann ist (da $a \neq 0$) $z= \frac{1}{a} (w-b)$ und damit
\setlength{\abovedisplayskip}{5.5pt}
\setlength{\belowdisplayskip}{5.5pt}
\begin{alignat*}{5}
0 \ \ &=&& \ \ \beta z + \overline{\beta} \overline{z} + \gamma \\
&=&& \ \ \beta \frac{1}{a}(w-b) + \overline{\beta} \frac{1}{a} (\overline{w-b}) + \gamma \\
&=&& \ \ \frac{\beta}{a} w + \frac{\overline{\beta}}{\overline{a}} \overline{w} - \frac{\beta}{a} b - \overline{\left(\frac{\beta}{a} b \right)} + \gamma,
\end{alignat*}
$w$ erfüllt also eine Kreigleichung wie gewünscht.
\item[\textbf{Fall (b)}] Ist nun $A$ ein euklidischer Kreis, so lässt sich analog verfahren. $\hfill \Box$
\end{compactenum}
\end{pr}



\hypertarget{propeinszweivier}{}

\end{proposition}

\begin{proposition}
Die Abbildung 
$$\mathcal{J}: \CC \la \CC, \qquad z \mapsto \begin{cases} \ 0 & \ \textrm{ falls } z = \infty, \\ \ \frac{1}{z} & \ \textrm{ falls } z \in \C \setminus \{0\}, \\ \ \infty & \ \textrm{ falls }z=0. \end{cases}$$
ist ein kreistreuer Homöomorphismus.
\begin{pr}
Sei $A \subseteq \CC$ ein Kreis beschrieben durch die Gleichung $\alpha z \overline{z} + \beta z + \overline{\beta} \overline{z} + \gamma = 0$. Setze $w= \mathcal{J}(z) = \frac{1}{z}$ für $z \in \C \setminus \{0\}$. Dann liefert die Gleichung 
$$\alpha \frac{1}{w} \frac{1}{\overline{w}} + \beta \frac{1}{w} + \overline{\beta} \frac{1}{\overline{w}} + \gamma = 0$$
durch Multiplikation mit $w\overline{w}$ 
$$\alpha + \beta \overline{w} + \overline{\beta} w + \gamma w \overline{w} = 0,$$
was wegen $w \overline{w} \in \mathbb{R}$ einer Kreisgleichung entspricht. $\hfill \Box$

\end{pr}

\end{proposition}


\begin{remark}
Geometrisch ist $\mathcal{J}$ die Spiegelung am Einheitskreis, denn: Für $z \in \C$ mit $\vert z \vert =1$ gilt auch $\vert \mathcal{J}(z) \vert = \big \vert \frac{1}{z} \big\vert = 1$. Ist $z= r e^{i \theta}$, so ist
$$ \mathcal{J}(z) =  \frac{1}{z} = \frac{1}{r} e^{-i\theta}$$ und $\mathcal{J}(\mathcal{J}(z)) = z$, also $\mathcal{J} = \mathcal{J}^{-1}$.

\end{remark}



\begin{definbem}
\begin{compactenum}
\item Seien $a,b,c,d \in \C$ mit $ad-bc \neq 0$. Dann heißt 
$$m: \CC \la \CC, \qquad z \mapsto \frac{az+b}{cz+d}$$
\textit{Möbiustransformation}.
\item Mit der Konvention $\frac{a}{0} = \infty$ für $a \neq 0$ gilt (mit erzwungener Stetigkeit von $m$)
$$m(\infty) = m\left( \lim_{z \to \infty} z\right) =\lim_{z \to \infty} m(z)= \lim_{z \to \infty} \frac{az+b}{cz+d} = \frac{a}{c}.$$
\item Es bezeichne $\mob$ die Menge der Möbiustransformationen.
\item Es gilt $m(\infty) = \infty$ genau dann, wenn $c =0$ sowie $m(0)=0$ genau dann, wenn $b=0$ (beachte, dass wegen der Bedingung an die Koeffizienten niemals beide Fälle auftreten).
\item Gilt $ad-bc = 0$, so ist $m$ nicht injektiv, also kein Homöomorphismus.
\item Polynome von Grad $1$ und die Inversion am Einheitskreis sind Möbiustransformationen:
$$\mathcal{J}(z) = \frac{1}{z} = \frac{0 \cdot z +1}{1 \cdot z +0}, \qquad \rho(z)=az+b = \frac{a\cdot z+b}{0\cdot z+1}.$$

\end{compactenum}
\end{definbem}


\hypertarget{satzeinszweisieben}{}
\begin{theorem}       %%%Satz 1.3 
Jede Möbiustransformation ist Komposition von Möbiustransformationen der Form $\mathcal{J}(z)=\frac{1}{z}$ und $\rho(z) = az+b$.
\begin{pr}
Sei also $m(z)= \frac{az+b}{cz+d}$ eine Möbiustransformation.
\begin{compactenum}
\item[\textbf{Fall (a)}] Sei $c=0$. Dann ist $m(z)= \frac{az+b}{d} = \frac{a}{d} z + \frac{b}{d} =: \rho(z)$.
\item[\textbf{Fall (b)}] Sei $c \neq 0$. Dann ist 

$$m(z)= \frac{az+b}{cz+d} = \frac{(az+b)c}{(cz+d)c} = \frac{acz+bc}{c^2z+dc} = \frac{acz+ad-(ad-bc)}{c^2z+dc}=\frac{a}{c} - \frac{ad-bc}{c^2z+dc}.$$
Mit $$\rho_1(z)= c^2z+dc, \qquad \rho_2(z) = (bc-ad) z + \frac{a}{c}$$
folgt
$$m(z) = \frac{a}{c} - \frac{ac-bc}{c^2z+dc} = (\rho_2 \circ \mathcal{J} \circ \rho_1 )(z),$$
was zu zeigen war. $\hfill \Box$

\end{compactenum}
\end{pr}

\end{theorem}

\hypertarget{satzeinszweiacht}{}
\begin{theorem}   %%Satz 1.4
Möbiustransformationen sind kreistreu.
\begin{pr}
Folgt direkt aus \hyperlink{satzeinszweisieben}{Satz 1.2.7} und den Propositionen \hyperlink{propeinszweidrei}{1.2.3} und \hyperlink{propeinszweivier}{1.2.4}. $\hfill \Box$
\end{pr}
\end{theorem}






\section{Transitivitätseigenschaften von Möbiustransformationen} %% 1.3 Transitivitätseigenschaften



Erinnerung aus der linearen Algebra: Eine lineare Abbildung zwischen endlichdimensionalen Vektorräumen ist durch endlich viele Vektoren (Basisvektoren) und ihre Bilder bestimmt. Analog wollen wir Möbiustransformationen durch endlich viele Punkte festlegen - und zwar durch genau drei Stück.

\begin{defin}     %%Fixpunkt
Ein \textit{Fixpunkt} einer Möbiustransformation $m \in \mob$ ist ein Punkt $z \in \C$ mit $m(z)=z$.
\end{defin}


\hypertarget{satzeinsdreizwei}{}
\begin{theorem}     %%satz 1.5
Sei $m \in \mob$ eine Möbiustransformation. Fixiert $m$ drei verschiedene Punkte in $\C$, so ist $m$ bereits die Identität.
\begin{pr}
Schreibe $m(z)= \frac{az+b}{cz+d}$. Wir zeigen nun, dass $m$ höchstens zwei Fixpunkte besitzt, falls $m \neq \id$.
\begin{compactenum}
\item[\textbf{Fall (a)}]
Sei zunächst $c=0$. Dann gilt nach 2.6 $m(\infty) = \infty$. Weiter ist $m(z) = \frac{a}{d}z + \frac{b}{d}$, also gilt für weitere Fixpunkte $z \in \C$ die Gleichung
$z= m(z) = \frac{a}{d} z + \frac{b}{d}$. Ist $\frac{a}{d} = 1$, so folgt $\frac{b}{d} = 0$, also $m= \id$, ein Widerspruch. Für $\frac{a}{d} \neq 1$ ergibt sich ein zweiter und letzter Fixpunkt $z= \frac{b}{d-a}$.
\item[\textbf{Fall (b)}] Ist nun $c \neq 0$, so ist $m(\infty) \neq \infty$. Alle Fixpunkte sind also Lösungen der Gleichung 
$$z=m(z) = \frac{az+b}{cz+d} \quad \Longleftrightarrow \quad cz^2+(d-a)z - b = 0.$$
Diese ist quadratisch in $\CC$ und besitzt ebenfalls eine oder zwei paarweise verschiedene Lösungen.


\end{compactenum}
Insgesamt muss damit eine Möbiustransformation mit drei Fixpunkten die Identität sein. $\hfill \Box$
\end{pr}
\end{theorem}


\hypertarget{satzeinsdreidrei}{}
\begin{theorem}    %%Satz 1.6
Die Menge $\mob$ ist \textit{einfach $3$-Punkt transitiv}, das heißt für zwei geordnete Tripel $(z_1, z_2, z_3), (w_1, w_2, w_3)$ von paarweise verschiedenen Punkten in $\CC$ existiert genau eine Möbiustransformation $m\in \mob$ mit $m(z_i) = w_i$ für $i \in \{1,2,3\}$. 


\begin{pr}
wir zeigen zunächst die Eindeutigkeit einer solchen Möbiustransformation. Seien also $m_1, m_2 \in \mob$ mit $m_1(z_i) = w_i = m_2(z_i)$ für $i \in \{1,2,3\}$. Setze $m:= m_1^{-1} \circ m_2$. Dann ist $m$ ebenfalls eine Möbiustransformation und es gilt $m(z_i) = z_i$ für $i \in \{1,2,3\}$, nach dem vorangegangenen Satz ist $m$ also die Identität, woraus $m_1 = m_2$ folgt.\\
Zur Existenz. Es genügt zu zeigen, dass es für jedes Tripel $(z_1, z_2, z_3)$ von paarweise verschiedenen Punkten in $\CC$ eine Möbiustransformation $m_z \in \mob$ gibt, welche dieses auf $(0,1,\infty)$ abbildet. Durch $m:= m_w^{-1} \circ m_z$ erhalten wir die gwünschte Möbiustransformation. Seien also $z_1, z_2, z_3 \in \CC$ paarweise verschieden. Wir unterscheiden zwei Fälle.
\begin{compactenum}
\item[\textbf{Fall (a)}] Seien $z_1, z_2, z_3 \in \C$. Mit dem Ansatz $m(z) = \frac{az+b}{cz+d}$ und 
$$m(z_1) = \frac{az_1 +b}{cz_1 + d} = 0, \qquad \frac{az_2+b}{cz_2+d}=1, \qquad \frac{az_3+d}{cz_3+d}= \infty$$ findet man
$$m(z) = \frac{z-z_1}{z-z_3} \frac{z_2-z_3}{z_2-z_1} = \frac{(z_2-z_3)z - (z_2-z_3)z_1}{(z_2-z_1)z - (z_2-z_1)z_3} =: \frac{\alpha z + \beta}{\gamma z + \delta}$$
mit $m(z_1)=0$, $m(z_2)=1$, $m(z_3)=\infty$ sowie
\begin{alignat*}
\alpha \delta - \beta \gamma \ \ &=&& \ \  (z_2-z_3)(-z_3(z_2-z_3)) - (-z_1(z_2-z_3))(z_2-z_1)\\
& =&& \ \  (z_2-z_3)(z_1-z_3)(z_1-z_2)\\
& \neq&& \ \  0,
\end{alignat*}
$m$ ist also eine Möbiustransformation mit den gewünschten Eigenschaften.
\item[\textbf{Fall (b)}] Ist $z_1 = \infty$, so geht man analog vor. $\hfill \Box$


\end{compactenum}

\end{pr}
\end{theorem}

\begin{remark}    %%Bemerkung
Tripel sind geordnete Mengen. Für ungeordnete Mengen $\{z_1, z_2, z_3\}$ existiert eine Möbiustransformation wie in \hyperlink{satzeinsdreidrei}{Satz 1.3.3} zwar, ist aber nicht eindeutig.
\end{remark}


\begin{corollary}   %%Korollar
$\mob$ operiert einfach 3-Punkt transitiv auf $\CC$.

\end{corollary}

\hypertarget{satzeinsdreisechs}{}
\begin{theorem}    %%satz 1.7
Die Gruppe $\mob$ opertiert transitiv auf der Menge $\mathcal{K}$ der Kreise in $\CC$. 

\begin{pr}
Wir zeigen zunächst, dass ein Tripel von drei verschiedenen Punkten in $\CC$ einen eindeutigen Krei in $\CC$ definiert. Sei also $(z_1, z_2, z_3)$ ein solches Tripel.
\begin{compactenum}
\item[\textbf{Fall (a)}] Sind $z_1, z_2, z_3 \in \C$ nicht kolinear, so existiert genau ein euklidischer Kreis durch $z_1, z_2, z_3$ (Mittelsenkrechtenkonstruktion) sowie keine euklidische Gerade.
\item[\textbf{Fall (b)}] Sind $z_1, z_2, z_3 \in \C$ kolinear, so gibt es genau eine Gerade durch die drei Punkte.
\item[\textbf{Fall (c)}] Ist ohne Einschränkung $z_1= \infty$ und $z_2, z_3 \in \C$, so ändert sich nichts.
\end{compactenum}
Seien nun $A,B$ Kreise in $\CC$. Wir müssen eine Möbiustransformation $m \in \mob$ finden mit $m(A)=B$. Wähle jeweils drei verschiedene Punkte $z_1, z_2, z_3 \in A$ und $w_1, w_2, w_3 \in B$. Dann existiert genau ein $m \in \mob$ mit $m(z_i) = w_i$ für $i \in \{1,2,3\}$. Weiter ist $m$ kreistreu, $m(A)$ und $B$ sind also beide Kreise, welche $w_1, w_2, w_3$ enthalten, also $m(A) = B$. $\hfill \Box$


\end{pr}
\end{theorem}


\begin{remark}    %%Bemerkung
Die Operation aus \hyperlink{satzeinsdreisechs}{Satz 1.3.6} ist nicht einfach transitiv, denn durch eine andere wahl der $z_i$ oder $w_i$ erhält man verschiedene Möbiustransformationen mit $m(A)=B$.

\end{remark}

\begin{defin}     %%Scheibe
Eine \textit{Scheibe} $D$ in $\CC$ ist eine Zusammenhangskomponente von $\CC \setminus K$, wobei $K$ ein Kreis in $\CC$ ist. Also ist $D$ eine Kreisscheibe, das Komplement einer Kreisscheibe oder eine Halbebene. Da wir $\CC$ mit $\mathbb{S}^2$ identifizieren können, fallen diese drei Begriffe zusammen und jede Scheibe $D$ ist topologisch homöomorph zu dem Inneren des Einheitskreises.

\end{defin}

\begin{theorem}   %%Satz 1.8
$\mob$ operiert transitiv auf der Menge $\mathcal{S}$ der Scheiben in $\CC$.
\begin{pr}
Seien $D=D_+,E=E_+ \in \mathcal{S}$ Scheiben in $\CC$ und $\partial D = A$, $\partial E = B$ die Randkreise. weitere seien $D_- := \CC \setminus \{A, D\}$ sowie $E_- := \CC \setminus \{B, E\}$. Nach Satz 3.6 existiert $m \in \mob$ mit $B=m(A)$. wegen der Stetigkeit von $m$ und der Kreistreue gilt nun bereits $m(D) \in \{E_+,E_-\}$. Ist $m(D) = E$, so sind wir fertig. Andernfalls benötigen wir noch eine Möbiustransformation, die $E_+$ und $E_-$ vertauscht und $B$ auf sich selbst abbildet. Betrachte dazu den speziellen Kreis $K= \mathbb{R} \cup \infty= \overline{\mathbb{R}}$. Dann gilt für
$$\mathcal{J}: \CC \la \CC, \qquad z \mapsto \begin{cases} \ 0, & \ \textrm{ falls } z = \infty \\ \ \frac{1}{z}, & \ \textrm{ falls } z \in \C \setminus \{0\}, \\ \ \infty, & \ \textrm{ falls } z=0 \end{cases}$$
gerade $\mathcal{J}(K)=K$. Weiter ist $\mathcal{J}(i)=\frac{1}{i} = -i$ und da $\mathcal{J}$ ein Homöomorphismus ist, folgt $\mathcal{J}(\mathbb{H}^{+}) = \mathbb{H}^-$ und $\mathcal{J}(\mathbb{H}^{-})= \mathbb{H}^{+}$. Sei nun $K_1$ ein beliebiger Kreis in $\CC$ und $n \in \mob$ mit $n(A)=K$ (eine solche Möbiustransformation existiert nach Satz 3.6). Dann gilt für $M:= n^{-1}\circ \mathcal{J} \circ n$:
$$M(A) = (n^{-1}\circ \mathcal{J} \circ n)(A) = n^{-1}(\mathcal{J}(K)) = n^{-1}(K) = A$$
und $M$ vertauscht die durch $A$ definierten Scheiben, was zu zeigen war. Damit liefert die Möbiustransformation $N:= M \circ m$ die gewünschte Eigenschaft $N(A)=B$ und $N(D)=E$. $\hfill \Box$
\end{pr}
\end{theorem}









\section{Das Doppelverhältnis: Eine numerische Invariante} %% 1.4 Das Doppelverhältnis



Eine Funktion $f: \CC^k \la \CC, (z_1, \ldots, z_k) \mapsto f(z_1, \ldots, z_k)$ heißt \textit{invariant unter} $\mob$, falls für jede Möboiustransformation $m \in \mob$ und für alle $(z_1, \ldots, z_k) \in \CC^k$ gilt
$$f\left( m(z_1), \ldots, m(z_k)\right) = f(z_1, \ldots, z_k).$$
Beispielsweise ist die Abbildung
$$f: \CC \la \CC, \qquad z \mapsto \begin{cases} \ z^2, & \ \textrm{ falls } z \in \C \\ \ \infty, & \ \textrm{ falls } z = \infty \end{cases}$$
nicht invariant unter $\mob$ (Aufgabe 2.3). Was ist die größte Untergruppe vom $\mob$, die $f$ invariant lässt? 

\begin{remark}
Da $\mob$ einfach $3$-Punkt transitiv auf $\He^2$ operiert, sind für $1 \leqslant k \leqslant 3$ die einzigen invarianten Funktionen die Konstanten, denn: Sei zum Beispiel $f: \CC^3 \la \CC$ invariant unter $\mob$ und $c:=f(0,0,0)$. Für $z_1, z_2, z_3 \in \CC$ gibt es $m \in \mob$ mit $m(z_i)=0$ für $i\in \{1,2,3\}$. Dann gilt aber bereits
$$f(z_1, z_2, z_3)= f(m(z_1), m(z_2), m(z_3)) = f(0,0,0)=c.$$

\end{remark}


\begin{defin}
Die Abbildung 
$$[\ ]: \C^{4} \la \C, \qquad (z_1, z_2, z_3, z_4) \mapsto [z_1, z_2; z_3, z_4] = \frac{(z_1-z_4)(z_3-z_2)}{(z_1-z_2)(z_3-z_4)}$$
heißt \textit{Doppelverhältnis}. Ist $z_k = \infty$ für ein $k$, so erweitern wir die Abbildung sinnvoll zu
$$[\infty, z_2; z_3, z_4] = \lim_{z \to \infty} [z, z_2; z_3, z_4] = \frac{z_3-z_2}{z_3-z_4}$$
und analog für $z_2=\infty$, $z_3= \infty$, $z_4= \infty$. 

\end{defin}


\begin{proposition}
Das Doppelverhältnis $[ \ ]$ ist invariant unter $\mob$. 
\begin{pr}
Nach \hyperlink{satzeinszweisieben}{Satz 1.2.7} genügt es zu zeigen, dass das Doppelverhältnis invariant unter $\mathcal{J}$ und unter linearen Polynomen von Grad $1$ ist. In der Tat gilt für paarweise verschiedene $z_1, z_2, z_3, z_4 \in \CC$
\setlength{\abovedisplayskip}{5.5pt}
\setlength{\belowdisplayskip}{5.5pt}
\begin{alignat*}{5}
[\mathcal{J}(z_1),\mathcal{J}(z_2);\mathcal{J}(z_3),\mathcal{J}(z_4),] \quad &=&& \quad \left[ \frac{1}{z_1}, \frac{1}{z_2}; \frac{1}{z_3}, \frac{1}{z_4} \right] \\
&=&& \quad \frac{\left( \frac{1}{z_1} - \frac{1}{z_4} \right)\left( \frac{1}{z_3} - \frac{1}{z_2}\right)}{\left( \frac{1}{z_1} - \frac{1}{z_2}\right)\left( \frac{1}{z_3}- \frac{1}{z_4}\right)} \\
&=&& \quad \frac{ \left( \frac{z_4-z_1}{z_4 z_1}\right) \left( \frac{z_2-z_3}{z_3z_2}\right)}{\left( \frac{z_2-z_1}{z_1z_2}\right)\left( \frac{z_4-z_3}{z_4z_3}\right)} \\
&=&& \quad \frac{(z_4-z_1)(z_2-z_3)}{(z_2-z_1)(z_4-z_3)} \\
&=&& \quad [z_1, z_2; z_3, z_4]
\end{alignat*}
sowie für $\rho(z) = az+b$ mit $a \neq 0$
\setlength{\abovedisplayskip}{5.5pt}
\setlength{\belowdisplayskip}{5.5pt}
\begin{alignat*}{5}
[\rho(z_1), \rho(z_2); \rho(z_3), \rho(z_4)] \quad &=&& \quad [az_1+b, az_2+b; az_3+b, az_4+b] \\
&=&& \quad \frac{(az_1-az_4)(az_3-az_2)}{(az_1-az_2)(az_3-az_4)} \\
&=&& \quad [z_1, z_2; z_3, z_4],
\end{alignat*}
woraus die Behauptung folgt. $\hfill \Box$


\end{pr}

\end{proposition}

\hypertarget{propeinsviervier}{}
\begin{proposition}   %%PROPOSITION 1.10
Seien $z_1,z_2,z_3,z_4$ vier paarweise verschiedene Punkte in $\CC$. Dann liegen {$z_1,z_2,z_3,z_4$ }auf einem Kreis in $\CC$ genau dann, wenn das Doppelverhältnis $[z_1,z_2;z_3,z_4]$ reell ist.
\begin{pr}
Übung (\hyperlink{Azweivier}{Aufgabe 2.4}). $\hfill \Box$
\end{pr}
\end{proposition}




\section{Möbiustransformationen und Matrizen} %% 1.5 Möbiustransformationen und Matrizen



Erinnerung: Es ist
\begin{spacing}{1.2}
$$\mathrm{GL}_2(\C) = \left\{ A=\begin{pmatrix} a & b \\ c & d \end{pmatrix} \in \C^{2 \times 2} \ \bigg \vert \ \det A = ad-bc \neq 0 \right\}$$
\end{spacing}
die allgemeine lineare Gruppe sowie
\begin{spacing}{1.2}
$$\mathrm{SL}_2(\C) = \left\{ A \in \mathrm{GL}_2(\C) \ \big \vert \ \det A =1 \right\}$$
\end{spacing}die spezielle lineare Gruppe.


\hypertarget{propeinsfunfeins}{}
\begin{proposition}   %%Proposition 1.11
Die Abbildung
$$\Phi: \mathrm{GL}_2(\C) \la \mob, \qquad \begin{pmatrix} a & b \\[-6pt] c & d \end{pmatrix} \mapsto \left( m(z) =\frac{az+b}{cz+d} \right)$$
ist ein surjektiver Gruppenhomomorphismus mit Kern
$$\kernel \Phi := \mathcal{Z} = \left\{ \lambda E_2 = \begin{pmatrix} \lambda & 0 \\[-6pt] 0 & \lambda \end{pmatrix} \ \bigg\vert \ \lambda \in \C\setminus \{0\} \right\}.$$
Insbesondere liefert der Isomorphiesatz
$$\mob \cong \slant{ \mathrm{GL}_2(\C)}{\mathcal{Z}} =: \mathrm{PGL}_2(\C).$$
\begin{pr}
Seien Matrizen $A,B \in \mathrm{GL}_2(\C)$ mit 

$$A= \begin{pmatrix} a & b \\[-6pt] c & d \end{pmatrix}, \qquad B= \begin{pmatrix} \tilde{a} & \tilde{b} \\[-6pt] \tilde{c} & \tilde{d} \end{pmatrix} $$

gegeben mit $a,b,c,d, \tilde{a}, \tilde{b}, \tilde{c}, \tilde{d} \in \C$ und $ad-bc \neq 0 \neq \tilde{a}\tilde{d} - \tilde{b} \tilde{c}$. Dann gilt 

\setlength{\abovedisplayskip}{5.5pt}
\setlength{\belowdisplayskip}{5.5pt}
\begin{alignat*}{5}
(\Phi(A) \circ \Phi(B) ) (z) \quad &=&& \quad \left(\Phi \left( \begin{pmatrix} a & b \\[-6pt] c & d \end{pmatrix} \right) \circ \Phi\left(\begin{pmatrix} \tilde{a} & \tilde{b} \\[-6pt] \tilde{c} & \tilde{d} \end{pmatrix}\right) \right)(z) \\[5pt]
&=&& \quad \frac{ a \frac{\tilde{a} z + \tilde{b}}{\tilde{c}z+\tilde{d}} + b}{c \frac{\tilde{a}z+\tilde{b}}{\tilde{c}z+\tilde{d}} + d } \quad = \quad  \frac{(a\tilde{a} + b \tilde{c})z + a\tilde{b} + b \tilde{d}}{(c\tilde{a} + d \tilde{c})z + c\tilde{b} + d \tilde{d}} \\[5pt]
&=&& \quad \Phi \left( \begin{pmatrix} a \tilde{a} + b \tilde{c} & a \tilde{b} + b \tilde{d} \\ c \tilde{a} + d \tilde{c} & c \tilde{b} + d \tilde{d} \end{pmatrix} \right) (z) \\[5pt]
&=&& \quad \Phi(AB)(z).
\end{alignat*}
Surjektivität ist klar: Für $m(z) = \frac{az+b}{cz+d}$ wähle die Matrix
$$A= \begin{pmatrix} a&b\\[-6pt] c&d\end{pmatrix}$$

Dann gilt offensichtlich $\Phi(A)(z) = m(z)$. Zu zeigen ist noch die Aussage über den Kern. Sei also $A= \begin{pmatrix} a&b \\[-6pt]c&d \end{pmatrix}\in \kernel \Phi$. Dann gilt 
$$z= \id(z) = \Phi(A)(z) = \frac{az+b}{cz+d}.$$
Setzen wir nun $z=0,1, \infty$ ein, liefert das $0 = \id(0) = \Phi(A)(0) = \frac{b}{d}$, also $b=0$. Analog folgt $\infty= \id(\infty) = \Phi(A)(\infty) =  \frac{a}{c}$, also $c=0$. Schließlich liefert $z=1$ dann $a=d:=\lambda$ und $\lambda \neq 0$, da wir außerdem $ad-bc \neq 0$ benötigen. Insgesamt erhalten wir also die Behauptung. $\hfill \Box$




\end{pr}

\end{proposition}




\section{Spiegelungen} %% 1.6 Spiegelungen





Wir wollen nun eine umfassendere Gruppe von kreistreuen Homöomorphismen in $\CC$ haben.


\hypertarget{propeinssechseins}{}
\begin{proposition}    %%Proposition 1.12
Die komplexe Konjugation
$$C: \CC \la \CC, \qquad z=x+iy\ \mapsto \ \begin{cases} \ \overline{z} = x-iy, & \ \textrm{ falls } z \neq \infty \\ \ \infty, & \ \textrm{ falls } z= \infty \end{cases}$$
ist ein kreistreuer homöomorphismus, aber keine Möbiustransformation, das heißt es gilt $C \in \homoekcc \setminus \mob$. 
\begin{pr}
Wir zeigen zu erst, dass $C \in \homoekcc$. Sicherlich ist $C$ ein Homöomorphismus, denn wegen $C^{-1}=C$ ist $C$ bijektiv und $C(U_{\epsilon}(z)) = U_{\epsilon}(C(z))$ liefert die Stetigkeit von $C$ und $C^{-1}$. Wir müssen noch die Kreistreue zeigen. Dazu genügt es, dass Kreisgleichungen unter $C$ wieder auf Kreisgleichungen abgebildet werden. Sei also durch $X:\alpha z \overline{z} + \beta z + \overline{\beta} \overline{z} + \gamma =0$ ein Kreis gegeben. Dann gilt 
$$C(X): \alpha \overline{z} z + \overline{\beta} z + \beta \overline{z} + \gamma=0,$$
was also einer Kreisgleichung entspricht. Wir zeigen nun, dass $C$ keine Möbiustransformation ist. Das sieht man schnell ein, denn $C$ fixiert offenbar jeden Punkt in $\overline{\mathbb{R}}$ und hat damit insbesondere mehr als $3$ Fixpunkte. Wäre $C$ eine Möbiustransformation, so wäre $C$ nach \hyperlink{satzeinsdreizwei}{Satz 1.3.2} bereits die Identität, was wegen $C(i)=i \neq i$ nicht wahr ist, weshalb $C$ keine Möbiustransformation ist. $\hfill \Box$

\end{pr}
\end{proposition}

\begin{remark}[Geometrische Interpretation]
$C$ ist die Spiegelung am Kreis $\mathbb{R}$, denn es gilt $C(z)=z$ für $z \in \mathbb{R}$ und $C=C^{-1}$. Ist nun $A \subseteq \CC$ ein beliebiger Kreis, so definiert man die Spiegelung an $A$ durch $C_A := m \circ C \circ m^{-1}$, wobei $m \in \mob$ eine Möbiustransformation ist, welche $\overline{\mathbb{R}}$ auf $A$ abbildet. Beachte: Diese Definition ist unabhängig von der Wahl von $m$ (vgl. Aufgabe 3.1).

\end{remark}


\begin{defin}
Die \textit{allgemeine Möbius-Gruppe} ist die von $\mob$ und $C$ erzeugte Gruppe $\amob = \langle \mob, C \rangle$. Das bedeutet, dass jede allgemeine Möbiustransformation $m \in \amob$ geschrieben werden kann als
$m= m_1 \circ \ldots\circ m_n$
mit $n \in \mathbb{N}$ und $m_i \in \mob$ oder $m_i =C$. 

\end{defin}

\hypertarget{satzeinssechsvier}{}
\begin{theorem}   %%Satz 1.13
Es gilt $\amob \subseteq \homoekcc$.
\begin{pr}
Da $C \in \homoekcc$ nach \hyperlink{propeinssechseins}{Proposition 1.6.1} und $\mob \subseteq \homoekcc$ nach \hyperlink{satzeinszweiacht}{Satz 1.2.8} ist jede Komposition aus $C$ und $m_i \in \mob$ ebenfalls in $\homoekcc$, also $\amob \subseteq \homoekcc$. $\hfill \Box$
\end{pr}

\end{theorem}

\begin{remark}
In \hyperlink{satzeinssechsvier}{Satz 1.6.4} gilt sogar Gleichheit, also $\amob = \homoekcc$. Man hat also eine Charakterisierung von allgemeinen Möbiustransformationen als genau die kreistreuen Homöomorphismen der Riemannschen Zahlenkugel $\CC$.
\end{remark}


\hypertarget{propeinssechssechs}{}
\begin{proposition} %%Proposition 1.14
 Jede Möbiustransformation in $\amob = \langle \mob, C\rangle$ hat entweder die Form $m(z)= \frac{az+b}{cz+d}$ oder $m(z)= \frac{a \overline{z}+b}{c\overline{z}+d}$ mit $a,b,c,d \in \C$ und $ad-bc \neq 0$.
\begin{pr}
Sei $m(z) = \frac{az+b}{cz+d} \in \mob$ und $n(z) = \frac{\alpha\overline{z} + \beta}{\gamma \overline{z} + \delta} = (\tilde{n} \circ C)(z) \in \amob$ mit $\tilde{n} = \frac{\alpha z + \beta}{\gamma z + \delta} \in \mob$. Dann sind auch die möglichen Kompositionen 
$$(m \circ C)(z) = \frac{a \overline{z} b}{c \overline{z} +d}$$
$$(m \circ n)(z) = \frac{(a\alpha + b \gamma)z + a\beta + b \delta}{(c \alpha + d\gamma)z + c \beta + d \delta}$$
und ebenso $n \circ m$ von derselben Gestalt. $\hfill \Box$

\end{pr}

\end{proposition}


\begin{ex}
Betrachte den Kreis $\mathbb{S}^1 = \{z \in \C \ \vert \ \vert z \vert = 1 \}$. Um eine Möbiustransformation $m \in \mob$ mit $m(\overline{\mathbb{R}})=\Sph^1$ zu finden, genügt es, das Tripel $(0,1,\infty)$ auf $(-i,1,i)$ abzubilden. Mit dem Ansatz 
$$m(z)= \frac{az+b}{cz+d}$$
führt das mit
$$-i = m(0)= \frac{b}{d}, \qquad 1 = m(1) = \frac{a+b}{c+d}, \qquad i = m(\infty) = \frac{a}{c}$$
zu $d=ib$, $c=-ia$ und $$1= \frac{a+b}{-ia+ib} = i\ \frac{a+b}{a-b}.$$
Damit erhalten wir $a-b= i (a+b)$, also $a(1-i)=b(i+1)$, also $b= a \frac{1-i}{i+1}$. Wähle zum Beispiel $a= i$, dann folgt $b=1$, $c=1$ und $d=i$ und damit 
$$m(z)= \frac{iz+1}{z+1}.$$ Um die inverse Möbiustransformation und letztlich die Spiegelung an $\mathbb{S}^1$ zu bestimmen, gehen wir zur Matrixschreibweise über, um die Rechnung zu vereinfachen. Demnach ist die zu $m$ zugehörige Matrix ist
$$M_m = \begin{pmatrix}[rr]i & 1 \\[-6pt] 1 & i \end{pmatrix}, $$
die zu $m^{-1}$ also
$$M_{m^{-1}} = (M_m)^{-1}= \begin{pmatrix}[rr] i & 1 \\[-6pt] 1 & i \end{pmatrix}^{-1} = -\frac{1}{2} \begin{pmatrix}[rr]i & -1 \\[-6pt] -1 & i \end{pmatrix},$$
also $$m^{-1}(z)= \frac{ - \frac{i}{2} z + \frac{1}{2}}{\frac{1}{2}z - \frac{i}{2}} = \frac{-i z +1}{z-i}.$$
Die Spieglung an $\mathbb{S}^1$ ist also
$$C_{\mathbb{S}^1}(z) = (m \circ C \circ m^{-1})(z) = (m \circ C) \left( \frac{-i z + 1}{ z - i}\right) = m \left( \frac{i\overline{z} +1}{\overline{z}+i}\right) = \frac{i \frac{i\overline{z}+1}{\overline{z}+i} +1}{\frac{i \overline{z} +1}{\overline{z}+i} +1} = \frac{1}{\overline{z}} = (\mathcal{J} \circ C)(z).$$

\end{ex}


\hypertarget{propeinssechsacht}{}
\begin{proposition}   %%Proposition 1.15
Jede Möbiustransformation $m \in \mob$ kann als Verknüpfung von endlich vielen Spiegelungen an Kreisen in $\CC$ geschrieben werden.
\begin{pr}
Übung (\hyperlink{Adreizwei}{Aufgabe 3.2}). $\hfill \Box$
\end{pr}
\end{proposition}


\newpage


\section{$\mathbb{H}$-invariante Möbiustransformationen} %% 1.7 H-oinvariante Möbiustransformationen




Es sei $\mathbb{H}=\mathbb{H}^+= \{ z \in \C \ \vert \  \mathrm{Im}(z) >0 \}$ die obere Halbebene. Weiter setzen wir

$$\amobh := \{ m \in \amob \ \vert \ m(\He) = \He\},$$
$$\mobh := \{ m \in \mob \ \vert \ m(\He) = \He\},$$
$$\amobr := \{m \in \amob \ \vert \ m(\RR)= \RR \}.$$

Nach \hyperlink{propeinssechssechs}{Proposition 1.6.6} und durch Normieren können wir annehmen, dass jede Möbiustransformation $m \in \amob$ von den Formen
$$(1) \quad m(z) = \frac{az+b}{cz+d}, \qquad \quad (2) \quad m(z) = \frac{a\overline{z} + b}{c \overline{z} +d}$$
mit $a,b,c,d \in \CC$ und $ad-bc =1$ ist. Dabei ist im zweiten Fall $m(z) = (\tilde{m} \circ C)(z)$ mit $\tilde{m}$ wie in (1).
Als erstes interessieren wir uns für $\amobr$.

\hypertarget{propeinssiebeneins}{}
\begin{proposition}
Sei $m \in \mob$. Dann sind die Koeffizienten von $m$ allesamt reell oder aber alle rein imaginär genau dann, wenn $m \in \mobr$.

\begin{pr}
Sei zunächst $m \in \mob(\RR)$ und schreibe $$m(z)= \frac{az+b}{cz+d}, \qquad m^{-1} = \frac{1}{ad-bc} \frac{dz-b}{-cz+a} = \frac{dz-b}{-cz+a}.$$ Dann gilt nach Voraussetzung $m(\RR)=\RR$, also 
$$m(\infty) = \frac{a}{c} \in \RR, \qquad m^{-1}(\infty) = -\frac{d}{c} \in \RR, \qquad m^{-1}(0) = -\frac{b}{a} \in \RR.$$
Wir unterscheiden nun drei Fälle.
\begin{compactenum}
\item[\textbf{Fall (a)}] Sei $a \neq 0 \neq c$. Dann sind alle obigen Werte in $\R$ und jeder Koeffizient ist ein Vielfaches von $c$:
$$a= m(\infty) c, \qquad b = -m^{-1}(0) a = -m^{-1}(0) m(\infty) c, \qquad d = -m^{-1}(\infty) c.$$
Dann gilt 
$$1= ad-bc = c^2 (-m(\infty) m^{-1}(\infty) + m(\infty) n^{-1}(0)) \in c^2 \cdot \R$$
und also auch $c^2 \in \RR$. Damit ist $c \in \R$ oder $c$ rein imaginär. Aus den obigen Beziehungen folgt dann, dass bereits $a,b,c,d \in \R$ oder $a,b,c,d \in \C \setminus \R^{\times}$.

\item[\textbf{Fall (b)}] Sei $c=0$. Dann ist $m(z)= \frac{a}{d} z + \frac{b}{d}$ und damit $m(0) = \frac{b}{d} \in \R$, $m(1) = \frac{a}{d} + \frac{b}{d} \in \R$ und somit $\mu:= \frac{a}{d} = m(1)-m(0) \in \R$. Es folgt
$$1=ad-bc = ad = \mu d^{2}$$
und mit derselben Argumentation wie im ersten Fall ist $d$ reell oder rein imaginär. Aus $a = \mu d$ und $b = m(0) d$ folgt $a,b,d \in \R$ oder $a,b,d \in \C \setminus \R^{\times}$.
\item[\textbf{Fall (c)}] Sei schließlich $a=0$. Dann ist $m(z) = \frac{b}{cz+d}$ und damit $m(0) = \frac{b}{d} \in \R$, $m(1) = \frac{b}{c+d} \in \R$. Es ist dann auch $\frac{d}{b} \in \R$ und $\frac{c+d}{b} \in \R$ und $\mu:=\frac{c}{b} \in \R$. Es gilt dann 
$$1=ad-bc = -bc = -b^2 \mu$$
und dieselbe Argumentation liefert die Behauptung. 

\end{compactenum}
Sind nun umgekehrt alle Koeffizienten reell oder alle rein imaginär, so sieht man leicht ein, dass $m(\RR) = \RR$. $\hfill \Box$

\end{pr}

\end{proposition}


\hypertarget{coreinssiebenzwei}{}
\begin{corollary}
Jedes Element $m \in \amobr$ hat eine der 4 Formen 
\setlength{\abovedisplayskip}{5.5pt}
\setlength{\belowdisplayskip}{5.5pt}
\begin{alignat*}{5}
(1) \quad m(z) &=&& \frac{az+b}{cz+d}, \quad a,b,c,d \in \R, \quad &&ad-bc = 1,\\
(2) \quad m(z) &=&& \frac{a\overline{z}+b}{c\overline{z}+d}, \quad a,b,c,d \in \R, \quad &&ad-bc = 1,\\
(3) \quad m(z) &=&& \frac{az+b}{cz+d}, \quad a,b,c,d \in \C \setminus \R^{\times}, \quad &&ad-bc = 1,\\
(4) \quad m(z) &=&& \frac{a \overline{z} +b}{c \overline{z} +d}, \quad a,b,c,d \in \C \setminus \R^{\times}, \quad &&ad-bc = 1.
\end{alignat*}
\begin{pr}
Folgt direkt aus den Propositionen \hyperlink{propeinssechssechs}{1.6.6} und \hyperlink{propeinssiebeneins}{1.7.1}. $\hfill \Box$

\end{pr}
\end{corollary}



Als nächstes sollen die Möbiustransformationen klassifiziert werden, welche die obere Halbebene invariant lassen.


\hypertarget{satzeinssiebendrei}{}
\begin{theorem} %% Satz 1.17
Jedes Element $m \in \amobh$ hat eine der 2 Formen
\setlength{\abovedisplayskip}{5.5pt}
\setlength{\belowdisplayskip}{5.5pt}
\begin{alignat*}{5}
(1) \quad m(z) &=&& \frac{az+b}{cz+d}, \quad a,b,c,d \in \R, \quad &&ad-bc = 1,\\
(2) \quad m(z) &=&& \frac{a \overline{z} +b}{c \overline{z} +d}, \quad a,b,c,d \in \C \setminus \R^{\times}, \quad &&ad-bc = 1.
\end{alignat*}

\begin{pr}
Der Kreis $\RR$ bestimmt zwei Scheiben $\He = \He^{+}$ und $\He^- = C(\He^+)$ in $\CC$. Da Möbiustransformationen stetig sind, lässt jedes Element in $\amobr$ die beiden Halbebenen invariant oder vertauscht sie. Daraus sieht man leicht: Ein Element in $\amobr$ liegt in $\amobh$ genau dann, wenn $\mathrm{Im}(m(i)) >0$ gilt. Wir rechnen dies für die vier Fälle aus \hyperlink{coreinssiebenzwei}{Korollar 1.7.2} direkt nach:

\setlength{\abovedisplayskip}{5.5pt}
\setlength{\belowdisplayskip}{5.5pt}
\begin{alignat*}{15}
&(1)&& \quad \mathrm{Im}(m(i))\  &&=&&\ \mathrm{Im} \left( \frac{ai+b}{ci+d}\right) \  &&=&& \ \mathrm{Im} \left( \frac{(ai+b)(-ci+d)}{\vert ci+d\vert^2} \right) \ &&=&& \ \frac{ad-bc}{\vert ci +d \vert^2} \ &&=&&\  \frac{1}{\vert ci+d\vert^2} \ >\ 0 \\
&(2)&& \quad \mathrm{Im}(m(i))\  &&=&& \ \mathrm{Im}\left( \frac{-ai+b}{-ci+d} \right) \  &&=&& \  \mathrm{Im}\left( \frac{(-ai+b)(ci+d)}{\vert -ci+d\vert^2} \right) \  &&=&& \ \frac{-ad+bc}{\vert ci +d \vert^2} \  &&=&& \  \frac{-1}{\vert ci+d\vert^2} \  <\  0 \\
&(3)&& \quad \mathrm{Im}(m(i))\  &&=&& \ \mathrm{Im}\left( \frac{ai+b}{ci+d} \right) \  &&=&& \  \mathrm{Im}\left( \frac{(ai+b)(ci-d)}{\vert ci+d\vert^2} \right) \  &&=&& \ \frac{-ad+bc}{\vert ci +d \vert^2} \  &&=&& \  \frac{-1}{\vert ci+d\vert^2} \  <\  0 \\
&(4)&& \quad \mathrm{Im}(m(i))\  &&=&& \ \mathrm{Im}\left( \frac{-ai+b}{-ci+d} \right) \  &&=&& \  \mathrm{Im}\left( \frac{(-ai+b)(-ci-d)}{\vert -ci+d\vert^2} \right) \  &&=&& \ \frac{ad-bc}{\vert ci +d \vert^2} \  &&=&& \  \frac{-1}{\vert ci+d\vert^2} \  >\  0 \\
\end{alignat*}
(Beachte: Es gilt stets $ad-bc =1$ sowie $a,b,c,d \in \R$ in (1) und (2) sowie $a,b,c,d \in \C \setminus \R^{\times}$ in (3) und (4)). Daraus folgt insgesamt dann die Behauptung. $\hfill \Box$

\end{pr}

\end{theorem}



\hypertarget{korollareinssiebenvier}{}
\begin{corollary}   %%Korollar 1.18
Es gilt
$$\amobh \cong \mathrm{PSL}_2(\R) \cup \left(\mathrm{PSL}_2(\mathbb{R}) \circ \sigma\right).$$

\begin{pr}
Wir zeigen, dass die beiden Typen (1) und (2) aus \hyperlink{satzeinssiebendrei}{Satz 1.7.3} jeweils einer Komponente der rechten Seite der Behauptung entspricht.\\
Wir haben bereits in \hyperlink{propeinsfunfeins}{Proposition 1.5.1} gesehen, dass $\mob \cong \mathrm{PGL}_2(\C)$ gilt. Die Normalisierung $ad-bc=1$ gibt uns Determinante $1$ der Matrizen und reelle Koeffizienten implizieren auch reelle Matrixeinträge. Insgesamt folgt daraus
$$\amobh \vert_{\mathrm{Typ\ (1)}} \cong \mathrm{PSL}_2(\R).$$

Betrachte nun die From (2) in \hyperlink{satzeinssiebendrei}{Satz 1.7.3} und schreibe $a= \alpha i$, $b= \beta i$, $c= \gamma i$, $d = \delta i$ mit $\alpha, \beta, \gamma, \delta \in \R$. Dann gilt $\alpha \delta - \beta \gamma = \frac{1}{i^2} (ad-bc) = -1$. Ist nun $\sigma (z) := -C(z)$ die Spiegelung an der imaginären Achse, so ist $\sigma \in \amobh$. (2) wird dann zu 
$$m(z) = \frac{a\overline{z} + b}{c\overline{z}+d} = \frac{\alpha i \overline{z} +\beta i}{\gamma i \overline{z} + \delta i} = \frac{(-\alpha)(-\overline{z}) + \beta}{(-\gamma)(-\overline{z}) + \delta} = (\tilde{m} \circ \sigma)(z)$$
mit 
$$\tilde{m}(z) = \frac{-\alpha z + \beta}{-\gamma z + \delta}, \qquad \alpha, \beta, \gamma, \delta \in \R, \qquad (-\alpha) \delta - \beta (-\gamma) = 1,$$
also $\tilde{m}$ wie in (1). Das liefert 
$$\amobh \vert_{\mathrm{Typ\ (2)}} \cong \left(\mathrm{PSL}_2(\R) \circ \sigma \right)$$

und insgesamt die Behauptung. $\hfill \Box$


\end{pr}




\end{corollary}



\newpage

\section{Konformität von Möbiustransformationen} %% 1.8 Konformität




Sein $C_1, C_2$ differenzierbare Kurven in $\CC$, die sich in einem Punkt $z_0 \in \C$ schneiden. Der \textit{Winkel} $\sphericalangle_{z_0} (C_1, C_2)$ zwischen $C_1$ und $C_2$ in $z_0$ ist definiert als der orientierte (also im Gegenuhrzeigersinn laufende) Winkel zwischen den Tangenten $T_1$ und $T_2$ von $C_1$ und $C_2$ in $z_0$ modulo $\pi$. Ein Diffeomorphismus $f: \CC \la \CC$ heißt \textit{konform} oder \textit{winkeltreu}, falls er den absoluten Wert von Winkeln zwischen Kurven erhält. Beachte: Ist $m \in \amob$ eine Möbiustransformation und $m(z_0) = \infty$, so definieren wir den Winkel zwischen $m(C_1)$ und $m(C_2)$ ind $m(z_0)$ als den Winkel zwischen $C_1$ und $C_2$ in $z_0$.


\hypertarget{satzeinsachteins}{}
\begin{theorem}    %% Satz 1.19
Möbiustransformationen sind konform.
\begin{pr}
Da Winkel über Tangentialvektoren definiert sind, genügt es zu zeigen, dass für beliebige sich in $z_0 \in \C$ schneidende euklidische Geraden $X_1, X_2$ in $\C$ und Möbiustransformationen $m \in \amob$ gilt:
$$\sphericalangle_{m(z_0)} (m(X_1), m(X_2)) = \sphericalangle_{z_0} (X_1, X_2).$$
Seien also $X_1, X_2$ sich im Punkt $z_0$ schneidende euklidische Geraden in $\C$ und wähle Punkte $z_k \in X_k$ mit $z_k \neq z_0$ für $k \in \{1,2\}$. Für die Steigungen $s_k$ der Geraden $X_k$ erhalten wir 
$$s_k = \frac{ \mathrm{Im}(z_k-z_0)}{\mathrm{Re}(z_k - z_0)} = \tan \theta_k,$$
wobei $\theta_k$ den Winkel zwischen $z_k-z_0$ und der reellen Achse bezeichne. Insbesondere erhalten wir also 
$$\sphericalangle_{z_0}(X_1, X_2) = \theta_2 - \theta_1 = \arctan s_2 - \arctan s_1.$$
Nun wird $\amob(\CC)$ erzeugt von Transformationen der Form 
$$(i) \quad f(z) = az+b, \qquad (ii) \quad \mathcal{J}(z) = \frac{1}{z}, \qquad (iii) \quad C(z) = \overline{z}.$$
Es genügt also, Konformität für diese drei Abbildungen zu zeigen.
\begin{compactenum}
\item Sei $m(z) = f(z) = az+b$ mit $a,b \in \C$, $a \neq 0$. Schreibe $a= \rho e^{i \beta}$. Da $f(\infty) = \infty$, sind $f(X_1)$ und $f(X_2)$ wieder euklidische Geraden in $\C$. Die Bildgeraden $f(X_k)$ verlaufen durch die Punkte $f(z_0)$ und $f(z_k)$ und besitzen die Steigung 
$$t_k = \frac{\mathrm{Im}(f(z_k) - f(z_0))}{\mathrm{Re}(f(z_k)-f(z_0))} = \frac{\mathrm{Im}(a (z_k-z_0)}{\mathrm{Re}(a (z_k-z_0)}  = \frac{\mathrm{Im}(e^{i\beta}(z_k-z_0))}{\mathrm{Re}(e^{i \beta} (z_k-z_0))} = \tan ( \theta_k + \beta).$$
für $k \in \{1,2\}$. Dann folgt für den Winkel
$$\sphericalangle_{f(z_0)} (f(X_1), f(X_2)) = \arctan t_2 - \arctan t_1 = \theta_2 - \theta_1 = \sphericalangle_{z_0} (X_1, X_2).$$

\item Sei $m(z) = \mathcal{J}(z) = \frac{1}{z}$. Dann gibt es drei Fälle zu betrachten: $\mathcal{J}(X_1)$ und $\mathcal{J}(X_2)$ sind zwei euklidische Kreise, zwei euklidische Geraden und ein euklidischer Kreis und eine euklidische Gerade. Wir betrachten lediglich den ersten, die beiden anderen Fälle verbleiben als Übung (\hyperlink{Avierdrei}{Aufgabe 4.3}). Seien also die Geraden $X_k$ gegeben durch die Gleichungen $\beta_k z + \overline{\beta}_k \overline{z} +1 =0$ mit $\beta_k \in \C$. Mit $z=x+iy$ erhalten wir also 
$$\beta_k (x+iy) + \overline{\beta_k}(x-iy) + 1 = x(\beta_k + \overline{\beta}_k) + yi (\beta_k - \overline{\beta}_k) + 1 = x \mathrm{Re}(\beta_k) + iy \mathrm{Im}(\beta_k) + 1 = 0$$
und damit ist die Steigung von $X_k$ gerade 
$$s_k = \frac{\mathrm{Re}(\beta_k)}{\mathrm{Im}(\beta_k)}.$$
Die Bildkreise sind gegeben durch 
$$\mathcal{J}(X_k): \beta_k \frac{1}{z} + \overline{\beta}_k \frac{1}{\overline{z}} + 1 = 0 \quad \Longleftrightarrow \quad \beta_k \overline{z} + \overline{\beta}_k z + z \overline{z} = 0.$$
Umformen liefert die Gleichung
$$\mathcal{J}(X_k): \vert z + \beta_k \vert ^2 = (z+ \beta_k)(\overline{z} + \overline{\beta}_k) = \vert \beta_k \vert^2,$$
das heißt $\mathcal{J}(X_k)$ sind Kreise mit Zentrum $-\beta_k$ und Radius $\vert \beta_k \vert$. Die Steigung der Tangente in $\mathcal{J}(X_k)$ in $0$ ist
$$t_k = - \frac{\mathrm{Re}(\beta_k)}{\mathrm{Im}(\beta_k)} = - \tan \theta_k = \tan (-\theta_k),$$
es folgt also 
$$\sphericalangle_0 (\mathcal{J}(X_1), \mathcal{J}(X_2)) = - \theta_2 - (-\theta_1) = - \sphericalangle_{\infty} (X_1, X_2).$$
Aus Symmetriegründen gilt bereits 
$$\sphericalangle_{\mathcal{J}(z_0)} (\mathcal{J}(X_1), \mathcal{J}(X_2)) = \sphericalangle_{\mathcal{J}(\infty)} (\mathcal{J}(X_1), \mathcal{J}(X_2)),$$
woraus die Konformität von $\mathcal{J}$ in diesem Fall folgt.
\item Übung (\hyperlink{Avierdrei}{Aufgabe 4.3}). $\hfill \Box$


\end{compactenum}




\end{pr}


\end{theorem}










































\chapter{Differentialgeometrie und Geometrie von Dreiecken} %KAPITEL II
\setlength\abovedisplayshortskip{0pt}
\setlength\belowdisplayshortskip{10pt}
\setlength\abovedisplayskip{10pt}
\setlength\belowdisplayskip{10pt}
 



\section{Ebene euklidische Geometrie} %PARAGRAPH II.1

\thispagestyle{empty}


Es bezeichne $\ssp$ das Standardskalarprodukt in $\R^2$, für $a=(a_1,a_2)$ und $b=(b_1,b_2)$ gilt also $\langle a,b \rangle = a_1 b_1 + a_2 b_2$. Es sei $\mathbb{E}^2 = (\mathbb{R}^2, \ssp)$ das Modell für zweidimensionale euklidische Geometrie. Der \textit{Tangentialraum} in $p \in \mathbb{E}^2$ ist 
$$T_p\mathbb{E}^2 = \{p \} \times \mathbb{R}^2 = \{ (p,v) \ \vert \ v \in \R^2 \} \cong \R^2.$$
Die \textit{Länge} eines Tangentialvektors $(p,u) =: u$ ist 
$$\Vert u \Vert := \sqrt{ \langle u, u \rangle}.$$
Der \textit{Winkel} zwischen 2 Tangentialvektoren $u,v \in T_p \mathbb{E}^2$ ist 
$$\cos \sphericalangle(u,v) = \frac{\langle u,v \rangle}{\Vert u \Vert \Vert v \Vert}.$$
Ist $c: [a,b] \la \R^2$, $c(t) = (x(t), y(t))$ eine differenzierbare Kurve, so ist die \textit{euklidische Länge von} $c$ definiert als 
$$L_e(c) := \int_a^b \  \bigg\Vert \frac{\mathrm{d}c}{\mathrm{d}t}(t) \bigg\Vert \mathrm{d}t = \int_a^b \sqrt{ (x'(t))^2 + (y'(t))^2} \mathrm{d} t$$
(motiviert durch Polygon-Approximation). Beispielsweise hat das parametrisierte Geradensegment $g: [0,1] \la \R^2$, $g(t) = (1-t)p + tq$ für beliebige $p,q \in \R^2$ 
$$L_e(g) = \int_0^1 \Vert p-q \Vert \mathrm{d} t = \Vert p-q \Vert.$$



\hypertarget{lemmazweieinseins}{}
\begin{lemma}   %%Lemma 2.1
\begin{compactenum}
\item Die euklidische Länge einer differenzierbaren Kurve ist unabhängig von der Parametrisierung.
\item Die euklidische Länge einer differenzierbaren Kurve ist invariant unter euklidischen Isometrien, also Translationen, Rotationen und Spiegelungen.
\end{compactenum}
\begin{pr}[Beweisidee.]
Subsitutionsregel und $\Vert(I \cdot c)'\Vert = \Vert I \cdot c' \Vert = \Vert c' \Vert$ für Isometrien $I$.
\end{pr}

\end{lemma}


\begin{defin}
Für zwei Punkte $p,q \in \mathbb{E}^2$ definieren wir
$$\Omega_{pq} := \left\{ \gamma: [0,1] \la \R^2 \ \vert \ \gamma(0)=p, \ \gamma(1) = q, \ \gamma \textrm{ ist stückweise differenzierbar } \right\}.$$


\end{defin}



\hypertarget{lemmazweieinsdrei}{}
\begin{lemma} %%%Lemma 2.2
Die kürzesten Verbindungskurven zwischen zwei Punkten in $\mathbb{E}^2$ sind genau die Geradensegmente.
\begin{pr}
Seien $p,q \in \mathbb{E}^2$ und bilde nun durch geeignete euklidische Isometrien das Punktepaar $(p,q)$ auf $((0,0), (0,l)$ ab. Dann ist nach \hyperlink{lemmazweieinseins}{Lemma 2.1.1} Die Länge aller Kurven in $\Omega_{pq}$ unverändert und es gilt
$$L_e(c) = \int_0^1 \sqrt{ (x'(t))^2 + (y'(t))^2} \mathrm{d}t \geqslant \int_0^1 \sqrt{(y'(t))^2} \mathrm{d}t \geqslant \int_0^1 \vert y'(t) \vert \mathrm{d} t \geqslant \int_0^1 y'(t) \mathrm{d}t = l$$
für alle Kurven $c \in \Omega_{pq}$. Das Minimum $L_e(c)=l$ wird mit der Geraden angenommen. Ist umgekehrt $c$ eine längenminimierende Kurve zwischen $p$ und $q$, so gilt in der obigen Ungleichungskette Gleichheit an jeder Stelle, also $x'(t)=0$ und $\vert y'(t)\vert = y'(t)$, woraus wiederum folgt, dass $c$ eine Gerade ist. $\hfill \Box$


\end{pr}
\end{lemma}


\begin{theorem}    %%satz 2.3

Auf $\mathbb{E}^2$ definere die Längenfunktion
$$d_e: \mathbb{R}^2 \times \R^2 \la \R, \qquad (p,q) \mapsto d_e(p,q) = \mathrm{inf}_{c \in \Omega_{pq}} \ L_e(c).$$
Dann wird $(\mathbb{E}^2, d_e)$ zu einem metrischen Raum.
\begin{pr}
Nach \hyperlink{lemmazweieinsdrei}{Lemma 2.1.3} gilt $d_e(p,q) = \Vert p-q \Vert$, womit die Behauptung unmittelbar folgt. Alternativ lässt sich die Behauptung auch konzeptionell zeigen. Dazu zeigen wir die erforderlichen Eigenschaften einer Metrik.
\begin{compactenum}
\item Für die Symmetrie beachte, dass für jede Kurve $c:[0,1] \la \R^2$, $t \mapsto c(t)$ in $\Omega_{pq}$ die Kurve $\tilde{c}: [0,1] \la \R^2$, $t \mapsto c(1-t)$ in $\Omega_{qp}$ ist und dieselbe Länge besitzt, also $d_e(q,p) \leqslant d_(p,q)$. Analog folgt aber auch die umgekehrte Ungleichung, insgesamt also die Symmetrie.
\item  Seien $p,q,r \in \R^2$. Dann gilt
$$\Omega_{rq} \circ \Omega_{pr} := \left\{ c_1 \circ c_2 \ \vert \ c_1 \in \Omega_{rq}, \ c_2 \in \Omega_{pr} \right\} \subseteq \Omega_{pq},$$
also auch $d_e(p,q) \leqslant d_e(p,r) + d_e(r,q)$.
\item Für die Definitheit betrachte $p=q \in \R^2$. Dann verbindet der konstante Weg $c_0: [0,1] \la \R^2$, $t \mapsto p$ die beiden und besitzt wegen $\Vert c'(t) \Vert = 0$ für alle $t$ Länge $0$, also $d_e(p,p)=0$. Ist hingegen $p \neq q$, so gilt nach \hyperlink{lemmazweieinsdrei}{Lemma 2.1.3}, dass das Geradensegment $g: [0,1] \la \R^2$, $t \mapsto (1-t)p + tq$ den Abstand realisiert und Länge $\Vert q-p \Vert > 0$ hat. $\hfill \Box$
\end{compactenum}
\end{pr}

\end{theorem}

\begin{defin}
\begin{compactenum}
\item Eine \textit{Isometrie} eines metrischen Raums $(M,d)$ ist eine bijektive Abbildung $\phi: M \la M$ mit $d(\phi(p), \phi(q)) = d(p,q)$ für alle $p,q \in M$. Die Isometrien von $M$ bilden eine Gruppe $\mathrm{Isom}(M)$. 
\item $(M,d)$ heißt homogen, falls es für je zwei Punkte $p,q \in M$ eine Isometrie $\phi \in \mathrm{Isom}(M)$ gibt mit $\phi(p) = q$. 
\item $(M,d)$ heißt $2$\textit{-Punkt-homogen}, falls es für je zwei Punktepaare $(p_1,p_2), (q_1,q_2)$ mit $d(p_1, p_2) = d(q_1, q_2)$ eine Isometrie $\phi \in \mathrm{Isom}(M)$ gibt mit $\phi(p_1) =q_1$ und $\phi(p_2) = \phi(q_2)$. 
\end{compactenum}
\end{defin}


\begin{corollary}   %%korollar 2.4
\begin{compactenum}
\item Translationen, Rotationen und Spieglungen sind Isometrien von $\E^2 = (\R^2, d_e)$.
\item Die euklidische Ebene $\E^2$ ist $2$-Punkt-homogen. Insbesondere gilt $$\E^2 \cong \slant{\mathrm{Isom}(\E^2)}{\mathrm{Stab}_{\mathrm{Iso}(\E^2)}(0)}.$$
\end{compactenum}
\begin{pr}
\begin{compactenum}
\item Folgt aus \hyperlink{lemmazweieinseins}{Lemma 2.1.1}.
\item Seien $(p_1, p_2), (q_1, q_2)$ zwei Paare von Punkten in $\E^2$, ohne Einschränkung gelte $q_1= (0,0)$ und $q_2=(0,\alpha)$. Führe dann zunächst die Translation $T: \R^2 \la \R^2$, $v \mapsto v-p_1$ durch und erhalte $T(p_1) = q_1=0$ und $T(p_2) = p_2-p_1$. Führe anschließend eine Drehung $D$ um $(0,0)$ durch mit $D(p_2-p_1)= (0,\alpha)$ und $D(0)=0$.  $\hfill \Box$
\end{compactenum}
\end{pr}

\end{corollary}






\section{Zweidimensionale sphärische Geometrie} %PARAGRAPH II.2





Die Sphäre mit Radius $R >0$ um $0 \in \R^3$ ist definert als
$$\Sph^2(R) := \left\{ (x,y,z) \in \R^3 \ \vert \ x^2+y^2+z^2 = R^2 \right\} \subseteq \R^3.$$
Es sei weiter für $p,q \in \Sph^2(R)$
$$\Omega_{pq}(\Sph^2(R)) := \left\{ c \in \Omega_{pq} \ \vert \ c(t) \in \Sph^2(R) \textrm{ für alle } t \in [0,1] \right\}$$
Für $c \in \Omega_{pq}(\Sph^2(R))$, $c(t) = (x(t), x(t), z(t))$ sei die \textit{sphärische Länge} definiert durch 
$$L_s(c) := \int_0^1 \Vert c'(t) \Vert \mathrm{d}t = \int_0^1 \sqrt{ (x'(t))^2 + (y'(t))^2 + (z'(t))^2 } \mathrm{d}t.$$
Die zu ihr assoziierte Längenfunktion ist analog wie im Euklidischen definiert als
$$d_s: \Sph^2(R) \times \Sph^2(R) \la \R, \qquad (p,q) \mapsto d_s(p,q) = \mathrm{inf}_{c \in \Omega_{pq}(\Sph^2(R))} L_s(c).$$


\hypertarget{lemmazweizweieins}{}
\begin{lemma} %%Lemma 2.5
Die sphärische Länge einer differenzierbaren Kurve aus $\Sph^2(R)$ ist invariant unter Rotationen von $\R^3$.
\begin{pr}
Sei $D: \R^3 \la \R^3$ eine Rotation von $\R^3$ um $0$. Dann gilt 
$$\langle Dx, Dx \rangle = \Vert Dx \Vert_e^2 = \Vert x \Vert_e^2 = \langle x, x \rangle,$$
da $D$ orthogonal ist. Wegen der Linearität von $D$ folgt außerdem 
$(D \circ c(t))' = D \circ c'(t)$
und damit 
$$\Vert (D \circ c(t))' \Vert_e = \Vert D \circ c'(t) \Vert_e = \Vert c'(t)\Vert_e,$$
woraus die Behauptung folgt. $\hfill \Box$
\end{pr}

\end{lemma}


\hypertarget{lemmazweizweizwei}{}
\begin{lemma} %%Lemma 2.6
Die kürzesten Verbindungen zwischen Punkten auf $\Sph^2(R)$ sind gerade die Großkreise, also Schnitte von $\Sph^2(R)$ mit $2$-dimensionalen Untervektorräumen von $\R^3$.
\begin{pr}
Seien $p,q \in \Sph^2(R)$ beliebige Punkte. Dann existiert eine Rotation von $\R^3$, die $p$ auf $p^*=(0,0,R)$ und $q$ auf $q^*=(0,y,z)$ abbildet. Nach \hyperlink{lemmazweizweieins}{Lemma 2.2.1} genügt es also, die kürzeste Vebindung zwischen $p^*$ und $q^*$ zu finden. Wähle hierzu geographische Koordinaten $\phi, \theta$ auf $\Sph^2(R)$. Damit können wir jede Kurve $c \in \Omega_{p^*q*}$ schreiben
$$c(t) = R \cdot \begin{pmatrix}[r] \sin \theta(t) \cos \phi(t) \\ \sin \theta(t) \sin \phi(t) \\ \cos \theta(t) \end{pmatrix}$$
und damit 
$$c'(t) = R \cdot \begin{pmatrix}[r] \theta'(t) \cos \theta(t) \cos \phi(t) - \phi'(t) \sin \theta(t) \sin \phi(t) \\ \theta'(t) \cos \theta(t) \sin \phi(t) + \phi'(t) \sin \theta(t) \cos \phi(t) \\ - \theta'(t) \sin \theta(t) \end{pmatrix}$$
Also ist 
$$\Vert c'(t) \Vert_e^2 = R^2 ( (\theta'(t))^2 + (\phi'(t))^2 \sin ^2 \theta(t))$$
und wir erhalten für die Länge von $c$
$$L_s(c) = \int_0^1 \Vert c'(t) \Vert_e^2 \mathrm{d}t =R \int_0^1 \sqrt{ (\theta'(t))^2 + (\phi'(t))^2 \sin^2 \theta(t) } \ \mathrm{d} t \geqslant R \int_0^1 \theta'(t) \ \mathrm{d} t = R\ (\theta(1) - \theta(0)).$$
Das Großkreissegment zwischen $p^*$ und $q^*$ realisiert jedoch genau diese Länge. Für jede weitere Kurve $\tilde{c}$ dieser Länge gilt Gleichheit an jeder Stelle, also insbesondere $\tilde{\phi}'(t)=$ für alle $t \in [0,1]$, womit $\tilde{c}$ ebenfalls ein Großkreis ist. $\hfill \Box$ 
\end{pr}
\end{lemma}

\hypertarget{satzzweizweidrei}{}
\begin{theorem}    %Satz 2.7
$(\Sph^2(R), d_s)$ ist ein metrischer Raum (also ein Modell für die 2-dimensionale sphärische Geometrie).
\begin{pr}
Übung
 (\hyperlink{Asechseins}{Aufgabe 6.1}).
\end{pr}
\end{theorem}


\begin{corollary}
\begin{compactenum}
\item Rotationen von $\R^3$ eingeschränkt auf $\Sph^2(R)$ sind Isometrien, d.h. $\mathcal{O}(3) \subseteq \mathrm{Isom}(\Sph^2(R))$.
\item $(\Sph^2(R), d_s)$ ist $2$-Punkt-homogen.
\end{compactenum}
\begin{pr}
\begin{compactenum}
\item Folgt aus \hyperlink{lemmazweizweieins}{Lemma 2.2.1}.
\item Übung (wurde schon im Beweis von \hyperlink{lemmazweizweizwei}{Lemma 2.2.2} benutzt): Sind $(p_1, p_2)$ und $(q_1, q_2)$ zwei Paare von Punkten auf $\Sph^2(R)$, so wende zunächst eine Rotation $A \in \mathcal{O}(3)$ mit $Ap_1 = q_1$ und anschließend eine Rotation $B \in \mathrm{Stab}_{\mathcal{O}(3)}(p_1=q_1)$ mit $B (Ap_2) = q_2$ an. $\hfill \Box$
\end{compactenum}
\end{pr}
\end{corollary}






\section{Zweidimensionale hyperbolische Geometrie} %PARAGRAPH II.3


Ziel soll es sein, auf der obere Halbebene $\He^2 = \{ z \in \C \ \vert \ \mathrm{Im}(z) >0 \}$ eine Längenmessung für stückweise differenzierbare Kurven (also eine Längenmetrik $d_h$) zu definieren, sodass $\amob(\He^2) \subseteq \mathrm{Isom}(\He^2, d_h)$. Dies führen wir konstruktiv durch. Sei dazu $A= \begin{pmatrix}a&b\\[-7pt]c&d\end{pmatrix} \in \mathrm{SL}_2(\R)$ und $T_A$ die zugehörige Möbiustransformation 
$$T_A: \He^2 \la \He^2, \qquad z \mapsto T_A(z) = \frac{az+b}{cz+d}$$
in $\amob(\He)$. Sei weiter
$$\gamma: [a,b] \la \He^2, \qquad t \mapsto \gamma(t) = x(t) + i y(t)$$
eine differenzierbare Kurve in $\He^2$ sowie $$w(t) = T_A(\gamma(t)) = \frac{a \gamma(t) +b}{c \gamma(t) +d}$$
die Bildkurve. Dann gilt für Tangentialvektoren der Bildkurve
$$w'(t) = \frac{\mathrm{d}w}{\mathrm{d}t}(t)= \frac{\mathrm{d}w}{\mathrm{d}\gamma} \ \frac{\mathrm{d}\gamma}{\mathrm{d}t} (t) = \frac{a (c \gamma(t) + d) - (a \gamma(t) +b)c}{(c \gamma(t) +d)^2} \ \gamma'(t) = \frac{\gamma'(t)}{(c \gamma(t) +d)^2}$$
sowie für den Imaginärteil der Bildkurve
$$\mathrm{Im}(w(t)) = \frac{1}{2i} (w(t)- \overline{w(t)} = \frac{1}{2i} \frac{ \gamma(t) - \overline{\gamma(t)}}{ \vert c \gamma(t) + d\vert^2} = \frac{\mathrm{Im}(\gamma(t))}{\vert c \gamma(t) +d \vert^2}$$
Dies motiviert die folgende Definition.

\begin{defin}[hyperbolische Längenmessung nach Poincaré]
Sei $\gamma:[a,b] \la \He^2$, $t \mapsto \gamma(t) = x(t) + i y(t)$ eine differenzierbare Kurve. Dann heißt 
$$L_h(\gamma) := \int_a^b \frac{\Vert \gamma'(t) \Vert_e}{ \mathrm{Im}(\gamma(t))} \ \mathrm{d} t = \int_a^b \frac{\sqrt{(x'(t))^2+ (y'(t))^2}}{y(t)} \ \mathrm{d}t$$
\textit{hyperbolische Länge von} $\gamma$.

\end{defin}


\hypertarget{lemmazweidreizwei}{}
\begin{lemma}    %%Lemma 2.9
Die hyperbolische Länge eine differenzierbaren Kurve $z$ in $\He^2$ is invariant unter Möbiustransformationen von $\amob(\He^2)$. Insbesondere gilt für alle $A\in \mathrm{SL}_2(\R)$ also $L_h(T_A \circ z) = L_h(z)$.
\begin{pr}
Nach obigen Formeln gilt für $A= \begin{pmatrix}a&b\\[-6pt] c&d\end{pmatrix} \in \mathrm{SL}_2(\R)$
$$\frac{\Vert (T_A \circ \gamma)'(t)\Vert_e}{\mathrm{Im}( (T_A\circ \gamma)(t))} = \frac{\Vert w'(t)\Vert_e}{\mathrm{Im}(w(t))} =  \frac{ \Vert \gamma'(t) \Vert_e}{\Vert (c \gamma(t) +d)^2 \Vert_e} \ \frac{\vert c \gamma(t) + d \vert^2}{\mathrm{Im}(\gamma(t))} = \frac{\Vert \gamma'(t)\Vert_e}{\mathrm{Im}(\gamma(t))},$$
für alle $t \in [a,b]$, es folgt also die Behauptung.
\end{pr}

\end{lemma}


\begin{remark}
Die Definition der Länge von Kurven bewirkt, dass Vektoren derselben euklidischen Länge hyperbolisch länger sind, falls sie näher an der reellen Achse liegen. So wird beispielsweise die kürzeste Verbindung zwischen zwei Punkten, welche beide der reellen Achse nahe sind, nicht etwa durch eine horizontal verlaufende Kurve realisiert, sondern vielmehr durch einen nach oben ragenden Bogen. Wir werden später einsehen, dass die kürzesten Verbindungen, die sogenannten Geodätischen, gerade die Halbkreise sind.

\end{remark}

Analog zur ebenen und sphärischen Geometrie definieren für $p,q \in \He^2$
$$\Omega_{pq}(\He^2) := \{ c \in \Omega_{pq} \ \vert \ c(t) \in \He^2 \textrm{ für alle t } \in [a,b] \}$$
sowie 
$$d_h(p,q):= \inf_{c \in \Omega_{pq}(\He^2)} L_h(c).$$

\begin{theorem}    %%Satz 2.10
$(\He^2, d_h)$ ist ein metrischer Raum, also ein Modell für die zweidimensionale hyperbolische Geometrie.
\begin{pr}
Wir zeigen die erforderlichen Eigenschaften einer Metrik.
\begin{compactenum}
\item Aus früheren Argumentationen folgt mit $\Omega_{pq} \cong \Omega_{qp}$ die Symmetrie $d(p,q) = d(q,p)$ für alle $p,q \in \He^2$.
\item Analog ist wegen $\Omega_{rq}(\He^2) \circ \Omega_{pr}(\He^2) \subseteq \Omega_{pq}(\He^2)$ die Dreiecksungleichung $d_h(p,q) \leqslant d_h(p.r) + d_h(r,q)$ für alle $p,q,r \in \He^2$ erfüllt.
\item Mit dem konstanten Weg folgt $d_h(p,p)=0$ für alle $p \in \He^2$. Es bleibt zu zeigen, dass für $p \neq q \in \He^2$ folgt, dass $d_h(p,q) >0$. Seien also $p,q \in \He^2$, $p\neq q$ und $c \in \Omega_{pq}(\He^2)$. Wähle einen euklidischen Ball $B_{\epsilon}(p)$ um $p$ mit Radius $\epsilon$, sodass $B_{\epsilon}(p) \subseteq \He^2$ und $q \notin B_{\epsilon}(p)$. Da $c$ stetig ist existiert nach dem Zwischenwertsatz ein (minimales) $\delta \in [0,1]$, sodass $c(\delta) \in \partial B_{\epsilon}(p)$ (und wegen der Minimalität $c([0,\delta]) \subseteq \overline{B_{\epsilon}(p)}$). Weiter gilt $\mathrm{Im}(c(t)) \leqslant \mathrm{Im}(p) + \epsilon$ für alle $t \in [0,\delta]$ und es folgt
$$L_h(c) \  = \  \int_0^1 \frac{\Vert c'(t)\Vert_e}{\mathrm{Im}(c(t))} \ \mathrm{d}t \  \geqslant  \  \int_0^{\delta} \frac{\Vert c'(t)\Vert_e}{\mathrm{Im}(p) + \epsilon} \ \mathrm{d} t \ \geqslant \ \frac{\epsilon}{\mathrm{Im}(p) + \epsilon},$$
also $$d_h(p,q) \ =\ \inf_{c \in \Omega_{pq}(\He^2)} L_h(c) \ \geqslant\ \frac{\epsilon}{\mathrm{Im}(p) + \epsilon} \ >\ 0,$$
was zu zeigen war. $\hfill \Box$


\end{compactenum}


\end{pr}
\end{theorem}




\begin{er}[Gruppenoperationen] Seien $G$ eine Gruppe, $X$ eine Menge sowie eine Gruppenoperation $\cdot: G \times X \la X$, $(g,x) \mapsto g \cdot x$ gegeben.
\begin{compactenum}
\item Für $x_0 \in X$ heißt $G\cdot x_0 := \{ g \cdot x_0 \ \vert \ g \in G\}$ \textit{Bahn von} $x_0$.
\item $X$ heißt homogen, falls es nur eine Bahn gibt.
\item Für $x_0 \in X$ heißt $G_{x_0} := \{ g \in G \ \vert \ g \cdot x_0 = x_0 \}$ \textit{Stabilisator von} $x_0$.
\item (Bahnensatz) Für $x_0 \in X$ gilt $G \cdot x_0 \cong \slant{G}{G_{x_0}}$.
\end{compactenum}
\begin{pr}[\textrm{\textit{Beweis von (iv)}}] Wir geben die Bijektion explizit an. Definiere
$$\phi: \slant{G}{G_{x_0}} \la G \cdot x_0, \qquad g \ G_{x_0} \mapsto g \cdot x_0.$$
Dann ist $\phi$ offensichtlich surjektiv und für $g,h \in G$ mit $g \cdot x_0 = h \cdot x_0$ gilt $(h^{-1} g) \cdot x_0 = x_0$, also $h^{-1}g =: k \in G_{x_0}$. Dann folgt aber
$$g \ G_{x_0} = hk \ G_{x_0} = h \ G_{x_0},$$
$\phi$ ist also injektiv, was den Satz beweist. $\hfill \Box$
\end{pr}

\end{er}








\hypertarget{korollarzweidreisechs}{}
\begin{corollary} %% Korollar 2.3.6
Es gilt:
\begin{compactenum}
\item $\amob(\He^2) \subseteq \mathrm{Isom}(\He^2, d_h)$.
\item $(\He^2, d_h)$ is $2$-Punkt-homogen.
\end{compactenum}
\begin{pr}
\begin{compactenum}
\item Für $T_A$ mit $A \in \mathrm{SL}_2(\R)$ folgt die Aussage aus der Definition von $d_h$ und \hyperlink{lemmazweidreizwei}{Lemma 2.3.2}. Es bleibt noch zu zeigen, dass $\sigma(z) := -\overline{z}$ eine hyperbolische Isometrie ist. Seien also $p,q \in \He^2$ und $c \in \Omega_{pq}(\He^2)$ mit $c(t) = x(t) + i y(t)$. Dann gilt 
$$\frac{ \Vert (\sigma \circ c)'(t) \Vert_e}{\mathrm{Im}((\sigma \circ c)(t))} = \frac{\Vert-x'(t) + i y'(t)\Vert_e}{\mathrm{Im}(-x(t) + i y(t))} = \frac{\Vert x'(t) + i y'(t) \Vert_e}{y(t)} = \frac{\Vert c'(t)\Vert_e}{\mathrm{Im}(c(t))}, $$
also
$$d_h(\sigma(p), \sigma(q)) = \inf_{c \in \Omega_{pq}(\He^2)} \left\{ \int_a^b \frac{ \Vert (\sigma \circ c)'(t) \Vert_e}{\mathrm{Im}((\sigma \circ c)(t))} \ \mathrm{d} t \right\} = \inf_{c \in \Omega_{pq}(\He^2)} \left\{ \int_a^b \frac{\Vert c'(t)\Vert_e}{\mathrm{Im}(c(t))} \ \mathrm{d}t \right\} = d_h(p,q), $$
womit $\sigma$ eine Isometrie ist.
\item Wir zeigen lediglich, dass $(\He^2, d_h)$ homogen bzgl. $\mathrm{SL}_2(\R)$ ist - die $2$-Punkt-Homogenität verbleibt als Übung (Aufgabe 6.2). Wir haben eine Gruppenoperation 
$$\cdot: SL_2(\R) \times \He^2 \la \He^2, \qquad \left( A= \begin{pmatrix}a&b\\[-6pt] c&d\end{pmatrix}, z \right) \mapsto A \cdot z := T_A(z) = \frac{az+b}{cz+d}.$$
Dann heißt Homogenität von $\He^2$ bzgl. $\mathrm{SL}_2(\R)$ gerade Transitivität der Gruppenoperation, dass es also nur eine Bahn gibt. Hierfür zeigen wir, dass für die Bahn von $i$ gilt $\mathrm{SL}_2(\R) \cdot i = \He^2$. Aus der Disjunktheit der Bahnen folgt dann die Behauptung. Für $z= x+iy \in \He^2$ wähle 
$$A= \begin{pmatrix} 1 & x \\[-2pt] 0 & 1 \end{pmatrix} \ \begin{pmatrix} \sqrt{y} & 0 \\[-2pt] 0  & \frac{1}{\sqrt{y}} \end{pmatrix}$$
Dann gilt 
$$A \cdot i = \left( \begin{pmatrix} 1 & x \\[-2pt] 0 & 1 \end{pmatrix} \ \begin{pmatrix} \sqrt{y} & 0 \\[-2pt] 0  & \frac{1}{\sqrt{y}} \end{pmatrix} \right) \cdot i \ =\ \begin{pmatrix} 1 & x \\[-2pt] 0 & 1 \end{pmatrix} \cdot (iy) \ = \ x + iy,$$
woraus nun die Homogenität von $\He^2$ folgt. $\hfill \Box$


\end{compactenum}
\end{pr}
\end{corollary}





\begin{ex}
Wir wollen mithilfe des Bahnensatzes eine algebraische Beschreibung der Halbebene erhalten. Nach \hyperlink{korollarzweidreisechs}{Korollar 2.3.6} gilt $\mathrm{SL}_2(\R) \cdot i = \He^2$. Um den Stabilisator von $i$ zu bestimmen, betrachte für $A = \begin{pmatrix}a&b \\[-6pt]c&d \end{pmatrix} \in \mathrm{SL}_2(\R)$

$$i= A\cdot i = T_A(i) = \frac{ai+b}{ci+d} \quad \Longleftrightarrow \quad (ad)i = - (b+c) \quad \Longleftrightarrow \quad a=d, \ b=-c.$$
Mit $1=ad-bc = a^2+b^2$ erhalten wir 
$$({\mathrm{SL}_2(\R)})_{i} \ \cong \ \left\{ \begin{pmatrix}[rr] \cos \theta & \sin \theta \\[-6pt] - \sin \theta & \cos \theta \end{pmatrix} \ \bigg \vert \ \theta \in [0,2\pi] \right\} \ \cong \ \mathrm{SO}(2)$$
Damit erhalten wir die Darstellung 
$$\He^2 = \mathrm{SL}_2(\R) \cdot i \ \cong \ \slant{\mathrm{SL}_2(\R)}{(\mathrm{SL}_2(\R))_i} = \slant{\mathrm{SL}_2(\R)}{\mathrm{SO}(2)}.$$
Eine analoge Darstellung erhalten wir auch in der ebenen und sphärischen Geometrie:
$$\E^2 \cong \slant{\mathrm{O}(2) \ltimes \R^2}{\mathrm{O}(2)},$$
$$\mathbb{S}^2 \cong \slant{\mathrm{SO}(3)}{\mathrm{SO}(2)}$$
wobei die Stabilisatoren des Ursprungs bzw. des Nordpols herangezogen wurden.


\end{ex}


















\section{Kürzeste Verbindungen in $(\He^2, d_h)$} %PARAGRAPH II.4





\begin{theorem}
Kürzeste Verbindungen bzw. Geodätische zwischen Punkten in $\He^2$ sind geeignet parametrisierte euklidische Halbkreise mit Zentrum auf der reellen Achse oder euklidische Halbgeraden parallel zur imaginären Achse.
\begin{pr}
Seien $p,q \in \He^2$, $p\neq q$. Sei $K$ der eindeutige euklidische Kreis mit Zentrum auf der reellen Achse bzw. die eindeutige euklidische Halbgerade orthogonal zur reellen Achse durch $p,q$. Nach Satz \hyperlink{satzeinsdreisechs}{1.3.6} und \hyperlink{satzeinssiebendrei}{1.7.3} existiert eine Möbiustransformation $T_A \in \amob(\He^2)$ mit $A \in \mathrm{SL}_2(\R)$, welche $K$ auf die imaginäre Achse abbildet. $T_A$ ist eine Isometrie von $(\He^2, d_h)$, lässt also insbesondere Längen und Abstände invariant. Wir können also ohne Einschränkung $p=ia$ und $q=ib$ für $b>a \in \R$ annehmen. Sei nun $$c:[\alpha,\beta] \la \He^2, \qquad t \mapsto c(t) = x(t) + i y(t)$$ eine differenzierbare Kurve mit $c(\alpha)=ia$ und $c(\beta)=ib$. Dann gilt für die hyperbolische Länge von $c$
$$L_h(c) = \int_{\alpha}^{\beta} \frac{\sqrt{ (x'(t))^2 + (y'(t))^2}}{y(t)} \ \mathrm{d}t \ \geqslant\ \int_{\alpha}^{\beta} \frac{\vert y'(t)\vert}{y(t)} \ \mathrm{d} t \ \geqslant \ \int_{\alpha}^{\beta} \frac{y'(t)}{y(t)} \ \mathrm{d}t \ =\ \left[ \ln y(t)\right]^{\beta}_{\alpha} \ =\ \ln \frac{b}{a}.$$
Diese Länge wird aber gerade durch das Geradensegment
$$g: [\ln a,\ln b] \la \He^2, \qquad t \mapsto e^{t}i$$
realisiert. Für jede weitere minimale Kurve gilt Gleichheit an jeder Stelle, also insbesondere $x'(t) = 0$ für $t \in [\alpha, \beta]$, womit diese auch ein geeignet parametrisierter Teil der imaginären Achse als Bild besitzt. $\hfill \Box$

\end{pr}
\end{theorem}


\begin{remark}
\begin{compactenum}
\item Isometrien bilden Geodätische auf Geodätische ab.
\item Ursprünglich sind Geraden bzw. Geodätische in einem metrischen Raum $(M,d)$ definiert als abstandserhaltende Abbildungen $g: \R \la M$, $t \mapsto g(t)$, das heißt mit 
$$d\left( g(t_1), g(t_2)\right) = d_{\R}(t_1, t_2) = \vert t_2 - t_1 \vert$$
für alle $t_1, t_2 \in \R$ (Parametrisierung nach Bogenlänge). Eine Parametrisierung der imaginären Achse (diese ist eine Geodätische) nach Bogenlänge ist gegeben durch
$$g: \R \la \He^2, \qquad t \mapsto e^{t}i = \frac{e^{\frac{t}{2}} i + 0}{ 0 i + e^{-\frac{t}{2}}} = T_{A_t}(i), \qquad A_{t} = \begin{pmatrix}e^{\frac{t}{2}} & 0 \\[-2pt] 0 & e^{-\frac{t}{2}}\end{pmatrix},$$
denn es gilt 
$$d_h( g(t_1), g(t_2)) = d_h(e^{t_1}i, e^{t_2}i) = \ln e^{t_2} - \ln e^{t_1} = t_2 - t_1$$
für alle $t_1, t_2 \in \R$, $t_1 < t_2$.
\item $(\He^2, d_h)$ ist eine zweidimensionale, nichteuklidische Geometrie, das heißt das Parallelenaxiom gilt nicht.

\end{compactenum}
\end{remark}



\begin{theorem}[Explizite Abstandsformel]     %%satz 2.13
Seien $z,w \in \He^2$, $z\neq w$. Die kürzeste Verbindung zwischen $z$ und $w$ habe Endpunkte $z^{*}, w^{*}$ auf der reellen Achse. Dann gilt
$$d_h(z,w) = \ln [z,z^*;w,w^*].$$
\begin{pr}
Sei $A \in \mathrm{SL}_2(\R)$ sodass die Möbiustransformation $T_A \in \amob(\He^2)$ die Geodätische zwischen $z$ und $w$ auf die imaginäre Achse abbildet. Nach eventuellem Anwenden weiterer Möbiustransformationen der Form
$$s: \He^2 \la \He^2, \qquad \zeta \mapsto s(\zeta) = \lambda \zeta = \frac{\sqrt{\lambda}\zeta + 0}{0\zeta + \frac{1}{\sqrt{\lambda}}} = T_{S_{\lambda}}(\zeta), \qquad S_{\lambda} = \begin{pmatrix} \sqrt{\lambda} & 0 \\[-2pt] 0 & \frac{1}{\sqrt{\lambda}}\end{pmatrix}\in \mathrm{SL}_2(\R)$$
für $\lambda>0$ (Streckung) und 
$$v: \He^2 \la \He^2, \qquad \zeta \mapsto v(\zeta) = -\frac{1}{\zeta} = \frac{0\zeta -1}{\zeta+0} = T_V(\zeta), \qquad V=\begin{pmatrix}0 & -1\\[-2pt] 1 & 0 \end{pmatrix} \in \mathrm{SL}_2(\R)$$
(Vertauschung von $0$ und $\infty$) kann man annehmen, dass 
$$T_A(z) = i, \qquad T_A(w) = ri, \qquad T_A(z^*)= 0, \qquad T_A(w^*) = \infty$$
für ein $r>1$. Nach dem Beweis von Satz 2.4.1 erhalten wir 
$$d_h(z,w) = d_h(T_A(z), T_A(w)) = d_h(i, ri) = \ln \frac{ri}{i} = \ln r.$$
Andererseits gilt für das Doppelverhältnis
$$[z,z^*;w,w^*] = [T_A(z), T_A(z^*); T_A(w), T_A(w^*)] = [i,0;ri,\infty] = \frac{(i-\infty)(0-ri)}{(i-0)(ri-\infty)} = r,$$
es folgt also die Behauptung. $\hfill \Box$



\end{pr}
\end{theorem}


\hypertarget{satzzweiviervier}{}
\begin{theorem}[Weitere explizite Abstandsformeln]    %%satz 2.14
Für $z,w \in \He^2$ gilt
\begin{compactenum}
\item $$d_h(z,w) = \ln \frac{\vert z - \overline{w} \vert + \vert z-w \vert}{\vert z - \overline{w}\vert - \vert z - w \vert}$$
\item $$ \cosh \left( d_h(z,w)\right) = 1 + \frac{ \vert z-w \vert^2}{ 2 \mathrm{Im}(z) \mathrm{Im}(w)}.$$
\item $$\sinh\left( \frac{1}{2} d_h(z,w)\right) = \frac{\vert z - w \vert}{2 \sqrt{\mathrm{Im}(z) \mathrm{Im}(w)}}.$$
\item $$\cosh\left( \frac{1}{2} d_h(z,w)\right) = \frac{ \vert z - \overline{w}\vert}{2 \sqrt{\mathrm{Im}(z) \mathrm{Im}(w)}}.$$
\item $$\tanh\left( \frac{1}{2} d_h(z,w)\right) = \frac{\vert z - w \vert }{\vert z - \overline{w}\vert}.$$
\end{compactenum}
\begin{pr}
Wir beweisen exemplarisch (iii). Der Rest folgt durch Umformen (Übung, \hyperlink{Asiebeneins}{Aufgabe 7.1}). Für die linke Seite gilt für eine Möbiustransformation $T_A$ mit $A \in \mathrm{SL}_2(\R)$
$$\sinh \left( \frac{1}{2} d_h(z,w)\right) = \sinh \left( \frac{1}{2} d_h(T_A(z), T_A(w))\right).$$
Aber auch die rechte Sseite ist invariant unter Möbiustransformationen, denn 
ist $A$ gegeben durch $A=\matx{a}{b}{c}{d}$, so gilt
\begin{alignat*}{5}
\vert T_A(z)-T_A(w)\vert \ = \ \Bigg \vert \frac{az+b}{cz+d} - \frac{aw+b}{cw+d}\Bigg \vert \ \ &=&& \ \ \Bigg\vert \frac{(az+b)(cw+d)-(aw+b)(cz+d)}{(cz+d)(cw+d)} \Bigg \vert \\
&=&& \ \ \Bigg\vert \frac{azd+bcw-awd-bcw}{(cz+d)(cw+d)} \Bigg \vert \\
&=&& \ \ \frac{\vert z-w \vert}{\vert cz+d\vert \vert cw+d\vert}
\end{alignat*}
und mit $$\mathrm{Im}(T_A(z)) =\mathrm{Im}\left( \frac{az+b}{cz+d}\right) = \mathrm{Im}\left( \frac{(az+b)(c\overline{z}+d)}{\vert cz+d\vert^2}\right) =   \frac{\mathrm{Im}(z)}{\vert cz+d\vert^2}$$
schließlich
$$\frac{\vert T_A(z)-T_A(w)\vert}{2 \sqrt{\mathrm{Im}(T_A(z)) \mathrm{Im}(T_A(w))}} = \frac{\vert z - w\vert}{2 \vert cz+d \vert \vert cw+d\vert \sqrt{\frac{\mathrm{Im}(z)}{\vert cz+d\vert^2} \frac{\mathrm{Im}(w)}{\vert cw+d\vert^2}}} = \frac{\vert z - w \vert}{2 \sqrt{\mathrm{Im}(z)\mathrm{Im}(w)}},$$
also die gewünschte Invarianz.
Sei also $L$ die eindeutige Geodätische durch $z$ und $w$ und $T_0\in \amob(\He^2)$, welche $L$ auf die imaginäre Achse abbildet. Dann genügt es nach obigen, die Aussage für $z=ia$, $w=ib$ für $b>a \in \R$ zu zeigen. es gilt aber $d_h(ia, ib) = \ln \frac{b}{a}$ und 
$$\sinh\left( \frac{1}{2} d_h(ia, ib)\right) = \frac{1}{2} \left( e^{\ln \sqrt{\frac{b}{a}}} - e^{- \ln \sqrt{\frac{b}{a}}} \right) = \frac{1}{2} \left( \sqrt{\frac{b}{a}} - \sqrt{\frac{a}{b}} \right) = \frac{1}{2} \frac{b-a}{\sqrt{ab}}= \frac{\vert z-w\vert}{2 \sqrt{\mathrm{Im}(z) \mathrm{Im}(w)}},$$
was zu zeigen war. $\hfill \Box$


\end{pr}

\end{theorem}






\section{Alle Isometrien von $(\He^2, d_h)$} %PARAGRAPH II.5




Wir haben in \hyperlink{satzeinssiebendrei}{Satz 1.7.3} bzw. \hyperlink{Adreivier}{Aufgabe 3.4} bereits gesehen:
$$\amob(\He^2) \cong \mathrm{PSL}_2(\R) \cup \left( \mathrm{PSL}_2(\R) \circ \sigma\right).$$
Korollar 2.11 besagte $\amob(\He^2) \subseteq \mathrm{Isom}(\He^2, d_h)$. Wir zeigen nun, dass die Möbiustransformationen bereits alle Isometrien der Halbebene sind.


\hypertarget{satzzweifuenfeins}{}
\begin{theorem}    %%satz 2.15
Es gilt $\mathrm{Isom}(\He^2, d_h) = \amob(\He^2)$.
\begin{pr}
Es ist noch $\mathrm{Isom}(\He^2, d_h) \subseteq \amob(\He^2)$ zu zeigen. Sei also $\phi$ eine beliebige Isometrie von $(\He^2, d_h)$, das heißt für alle $z,w \in \He^2$ gilt $d_h(\phi(z), \phi(w)) = d_h(z,w)$. Wir wissen bereits, dass $\phi$ Geodätische auf solche abbildet. Die imaginäre Achse $I$ ist eine hyperbolische Geodätische, $\phi(I)$ also ebenfalls. Dann existiert jedoch eine Möbiustransformation $m \in \amob(\He^2)$ mit $m(\phi(I)) = I$. Nach weiterem Anwenden von Möbiustransformationen der Gestalt $z\mapsto \lambda z$ für ein $\lambda>0$ sowie $z \mapsto \frac{1}{z}$ gilt ohne Einschränkung $(m\circ\phi)(i) = i$ sowie $(m\circ \phi)([i,\infty))=[i,\infty)$ und $(m \circ \phi)((0,i]) = (0,i]$. Da $m, \phi$ Isometrien sind, wird $I$ also von $m \circ \phi$ punktweise festgehalten. Sei nun $z=x+iy \in \mathbb{H}^2$ ein beliebiger Punkt der Halbebene sowie $w:=(m \circ \phi)(z) = u + i v$. Nach Satz 2.13(iii) folgt dann 
$$\frac{\vert z - it\vert}{2 \sqrt{ \mathrm{Im}(z) \mathrm{Im}(it)}} = d_h(z,it) = d_h((m\circ \phi)(z), (m\circ \phi)(it)) = d_h(w,it) = \frac{\vert w - it\vert}{2 \sqrt{\mathrm{Im}(w) \mathrm{Im}(it)}}.$$
also 
$$\frac{\vert x + i(y-t) \vert}{2 \sqrt{yt}} = \frac{ \vert u + i(v-t)\vert}{2 \sqrt{vt}}$$
und schließlich
$$\frac{(x^2+(y-t)^2)v}{t^2} = \frac{(u^2+(v-t)^2)y}{t^2}$$
Für $t \to \infty$ folgt $y=v$ und damit auch $x^2=u^2$. Letzteres impliziert $x=\pm u$.
\begin{compactenum}
\item[\textbf{Fall (a)}] Gilt $x=u$, so ist $(m \circ \phi)(z) = w = u+iv = x+iy = z$, wegen der Stetigkeit von $(m \circ \phi)$ also bereits $m \circ \phi = \id$. Damit ist $\phi = m^{-1}$ eine Möbiustransformation.
\item[\textbf{Fall (b)}] Gilt $x=-u$, so ist $(m \circ \phi)(z) = w = u+iv = -x+iy = -\overline{z} = \sigma(z)$ und wegen der Stetigkeit wiederum $m \circ \phi = \sigma$. Damit ist auch in diesem Fall $\phi = m^{-1} \circ \sigma$ eine Möbiustransformation und es folgt die Behauptung. $\hfill \Box$
\end{compactenum}



\end{pr}
\end{theorem}







\section{Das Einheitskreismodell der hyperbolischen Geometrie} %PARAGRAPH II.6




Betrachte die Einheitskreisscheibe $\mathbb{D}^2 = \left\{ z \in \C \ \vert \ \vert z \vert <1\right\} \subseteq \C$. Erinnerung: Die Möbiustransformation
$$m: \C \la \C, \qquad z \mapsto \frac{iz+1}{z+i}$$
bildet die obere Halbebene $\He^2$ auf $\D^2$ ab. Insbesondere bildet $m$ auch $\RR$ auf $\mathbb{S}^1$ ab.


\begin{defin}
Für eine Kurve $c: [a,b] \la \D^2$ mit $c(t) = x(t) + i y(t)$ definieren wir die hyperbolische Längenmessung in $\D^2$ durch 
$$L_h^{*}(c) := \int_a^b \frac{2 \sqrt{(x'(t))^2+ (y'(t))^2}}{1- ((x(t))^2 + (y(t))^2)} \ \mathrm{d}t.$$
Mit der bereits bekannten Notation 
$$\Omega_{pq}(\D^2) := \left\{ c \in \Omega_{pq} \ \vert \ c(t) \in \D^2 \textrm{ für alle } t \in [a,b] \right\}$$
erhalten wir durch 
$$d_h*: \D^2 \times \D^2 \la \R, \qquad (p,q) \mapsto d_h^{*}(p,q) := \mathrm{inf}_{c \in \Omega_{pq}(\D^2)} L_h^{*}(c)$$
eine Längenmetrik auf $\D^2$.


\end{defin}


\begin{remark}
Die Punkte in $\mathbb{S}^1$, dem "Rand" von $\D^2$ sind gerade dadurch charakterisiert, dass sie unendlichen Abenstand von allen anderen Punkten in $\D^2$ besitzen. Wir bezeichnen daher $\mathbb{S}^1$ auch als den \textit{Rand im Unendlichen} von $\D^2$. Analog ist $\RR$ der Rand im Unendlichen von $\He^2$.
\end{remark}









\section{Riemannsche Metriken und Winkel} %PARAGRAPH II.7





\begin{defin}
Sei $U \subseteq \R^2$ offen und $\mathrm{Sym}(2)$ die Gruppe der positiv definiten, symmetrischen, reellen $2\times2$-Matrizen. Dann ist eine \textit{Riemannsche Metrik} $g$ \textit{auf} $U$ eine Abbildung
$$g: U \la \mathrm{Sym}(2), \qquad p=(x,y) \mapsto g(p) = \begin{pmatrix}[rr] g_{11}(x,y) & g_{12}(x,y)\\[-6pt] g_{21}(x,y) & g_{22}(x,y) \end{pmatrix},$$
wobei die Funktionen $g_{ij}$ differenzierbar sind. Dann definiert $g(p)$ ein Skalarprodukt auf $T_pU=\{p\} \times \R^2 \cong \R^2$. Damit haben wir eine Längen- und Winkelmessung auf $T_pU$ gegeben: für $u,v \in T_pU$ gilt
$$\Vert u \Vert_p := \sqrt{ g(p)(u,u)} = \sqrt{u^{T} g(p) u}, \qquad \cos \sphericalangle (u,v) := \frac{\sqrt{ g(p)(u,v)}}{\Vert u \Vert_p \Vert v\Vert_p}.$$
Für eine differenzierbare Kurve $c:[a,b] \la U$ gilt $c'(t) \in T_{c(t)}U$ und mit
$$L_g(c) = \int_a^b \sqrt{ g(c(t))(c'(t), c'(t))} \ \mathrm{d} t = \int_a^b \Vert c'(t) \Vert_{c(t)} \ \mathrm{d}t
$$
einen Längenbegriff für differenzierbare Kurven.

\begin{ex}
\begin{compactenum}
\item Die euklidische Metrik auf $\R^2$ ist gegeben durch die konstante Riemannsche Metrik 
$$g(p) = g(x,y) = \begin{pmatrix}1&0\\[-6pt] 0&1 \end{pmatrix}$$
\item Die hyperbolische Metrik auf $\He^2$ ist für $(x,y) \in \He^2$ gegeben durch 
$$g(x,y) = \begin{pmatrix}[rr] \frac{1}{y^2} & 0 \\[-6pt] 0 &\frac{1}{y^2} \end{pmatrix} = \frac{1}{y^2} \begin{pmatrix}1&0\\[-6pt] 0&1 \end{pmatrix}$$
\item Die hyperbolische Metrik auf $\D^2$ ist für $z\in \D^2$ gegeben durch 
$$g(z) = \begin{pmatrix}[rr] \frac{4}{(1-\vert z \vert^2)^2} & 0 \\[-6pt] 0 & \frac{4}{(1-\vert z \vert^2)^2} \end{pmatrix} = \frac{4}{(1-\vert z \vert^2)^2} \begin{pmatrix}1&0\\[-6pt] 0&1 \end{pmatrix}.$$
\end{compactenum}
Offenbar unterscheiden sich die beiden hyperbolischen Metriken von der euklidischen nur um einen konformen Faktor.
\end{ex}


\begin{defin}
Der Riemannsche Winkel zwischen zwei sich schneidenden Kurven ist der Winkel zwischen den Tangentialvektoren um Schnittpunkt.

\end{defin}

\begin{proposition}
Der Riemannsche Winkel zwischen zwei sich schneidenden Geodätischen in $\D^2$ oder $\He^2$ stimmt mit dem euklidischen Winkel überein. \textit{(Ohne Beweis.)}
\end{proposition}


\end{defin}











\section{Der hyperbolische Flächeninhalt} %PARAGRAPH II.8




Der hyperbolische Flächeninhalt einer Teilmenge $A\subseteq \He^2$ ist definiert als
$$\mu(A) := \int_A \ \frac{1}{y^2} \ \mathrm{d}x \mathrm{d}y,$$
falls das Integral existiert.

\begin{theorem}   %%Satz 2.17
Der hyperbolische Flächeninhalt ist invariant unter Isometrien. 
\begin{pr}
Nach \hyperlink{satzzweifuenfeins}{Satz 2.5.1} und \hyperlink{korollareinssiebenvier}{Korollar 1.7.4} genügt es zu zeigen, dass für $T \in \mathrm{PSL}_2(\R)$ gilt
$$\mu(T(A)) = \mu(A)$$
für alle Teilmengen $A\subseteq \He^2$ (denn der Flächeninhalt ist invariant unter $\sigma$). Wir erinnern uns an den Transformationssatz aus der Analysis: Für einen Homöomorphismus $\phi$ gilt 
$$\int_{\phi(U)} \ f(y) \mathrm{d}y = \int_U \ f(\phi(x)) \ \vert \det D \phi(X) \vert \ \mathrm{d}x.$$
Schreibe nun
\setlength{\abovedisplayskip}{5.5pt}
\setlength{\belowdisplayskip}{5.5pt}
\begin{alignat*}{5}
T(z=x+iy) = \frac{az+b}{cz+d} \ \ &=&& \ \  \frac{ax+b+ iay}{cx+d + icy}\\
&=&&  \ \ \frac{(ax+b+iay)(cx+d-icy)}{(cx+d)^2+(cy)^2}\\
& =&& \  \  \frac{(ax+b)(cx+d) + acy^2 + i (ay (cx+d) - cy(ax+b))}{(cx+d)^2+(cy)^2} \\
&=&& \ \ \frac{(ax+b)(cx+d) + acy^2}{(cx+d)^2+(cy)^2} + i \frac{ay (cx+d) - cy(ax+b)}{(cx+d)^2+(cy)^2}\\
&=&& \ \ \frac{acx^2+adx+bcx+ acy^2+bd}{(cx+d)^2+c^2y^2} + i \frac{y}{(cx+d)^2+c^2y^2}
\end{alignat*}
Fasse $T$ also als Abbildung 
$$T: \R^2 \la \R^2, \qquad (x,y) \mapsto (T_1(x,y), T_2(x,y))= \left( \frac{(ax+b)(cx+d) + acxy}{(cx+d)^2+(cy)^2}, \frac{y}{(cx+d)^2+(cy)^2} \right)$$
auf. Für die Jacobimatrix gilt 
$$D T(x,y) = \begin{pmatrix}[rr] \frac{\partial T_1}{\partial x}(x,y) & \frac{\partial T_1}{\partial y}(x,y) \\[6pt] \frac{\partial T_2}{\partial x}(x,y) & \frac{\partial T_2}{\partial y}(x,y) \end{pmatrix} = \begin{pmatrix}[rr] \frac{(cx+d)^2-c^2y^2}{((cx+d)^2+c^2y^2)^2} & \frac{2cy(cx+d)}{((cx+d)^2+c^2y^2)^2} \\[6pt] \frac{-2cy(cx+d)}{((cx+d)^2+c^2y^2)^2} & \frac{(cx+d)^2-c^2y^2}{((cx+d)^2+c^2y^2)^2} \end{pmatrix},$$
also 
$$\det D T(x,y) = \frac{1}{((cx+d)^2+c^2y^2)^2}.$$
Mit $h(x,y) = \frac{1}{y^2}$ folgt wegen
$$(h \circ T)(x,y) = \frac{((cx+d)^2+c^2y^2)^2}{y^2}$$
schließlich
\setlength{\abovedisplayskip}{5.5pt}
\setlength{\belowdisplayskip}{5.5pt}
\begin{alignat*}{5}
\mu(T(A)) \ \ &=&& \ \ \int_{T(A)} \frac{1}{y^2} \ \mathrm{d}x \mathrm{d}y \\
&=&& \ \  \int_A\ (h \circ T)(x,y) \ \vert \det D T(x,y)\vert \ \mathrm{d}x \mathrm{d}y \\
&=&& \ \  \int_A \ \frac{((cx+d)^2+c^2y^2)^2}{y^2} \ \frac{1}{((cx+d)^2+c^2y^2)^2} \ \mathrm{d}x \mathrm{d}y \\
&=&& \ \ \int_A \ \frac{1}{y^2}\ \mathrm{d}x \mathrm{d}y\\
&=&& \ \ \mu(A),
\end{alignat*}
was zu zeigen war. $\hfill \Box$
\end{pr}


\end{theorem}



\begin{defin}
Ein \textit{hyperbolisches Polygon} $P$ \textit{mit $n$ Seiten} ist ein von $n$ Geodätischen beschränkte Teilmenge von $\overline{\He}^2 = \He^2 \cup \RR$ bzw. von $\overline{\D}^2 = \D \cup \mathbb{S}^1$. Schneiden sich zwei der Geodätischen in einem Punkt $S$, so heißt dieser Schnittpunkt \textit{Ecke} von $P$. Beachte: Ecken im Rand im Unendlichen sind erlaubt (während es für Seiten nicht erlaubt ist).

\end{defin}


\begin{ex}
Im folgenden wollen wir hyperbolischen Dreiecke ($3$-Gone) diskutieren. Dabei unterscheiden wir, wie viele der $3$ Ecken auf dem Rand im Unendlichen liegen.
\begin{compactenum}
\item Sind keine Ecken im Unendlichen, so haben Dreiecke die Gestalt

$$
\begin{tikzpicture}[scale=0.75]
\draw[->] (-5,0) -- (4,0);
\draw[dashed] (0.5,0) arc (0:180:2);
\draw[thick] (0,1.3229) arc (41.4096: 117:2);
\draw[thick] (2.4167,1.7776) arc (36.3367:143.6633:3);
\draw[thick] (2.4167,1.7776) arc (62.7209:138.5896:2);
\draw[dashed] (3.5,0) arc (0:180:2);
\draw[dashed] (3,0) arc (0:180:3);
\end{tikzpicture}
$$

\item Ist eine Ecke auf dem Rand im Unendlichen, so erhalten wir Dreiecke der Gestalt
$$
\begin{tikzpicture}[scale=0.75]
\draw[->] (-5,0) -- (9,0);
\draw[dashed] (0.5,0) arc (0:180:2);
\draw[thick] (3.5,0) arc (0: 138.5906:3);
\draw[thick] (3.5,0) arc (0: 138.3757:2);
\draw[thick] (0,1.3229) arc(41.6243: 97.1808:2);
\draw[dashed] (3.5,0) arc (0:180:2);
\draw[dashed] (3.5,0) arc (0:180:3);
\draw[dashed] (8.5,0) arc (0:180:1.5);
\draw[thick] (8,1.1180) arc (48.1888: 90:1.5);
\draw[dashed] (7,0) -- (7,3.5);
\draw[thick] (7,1.5) -- (7,3.5);
\draw[dashed] (8,0) -- (8,3.5);
\draw[thick] (8,1.1180) -- (8,3.5);
\end{tikzpicture}
$$

\item Sind zwei Randpunkte im Unendlichen, so erhalten wir
$$
\begin{tikzpicture}[scale=0.75]
\draw[->] (-5,0) -- (9,0);
\draw[dashed] (-1,0) arc(0:180:1.5);
\draw[dashed] (4,0) arc(0:180:2.5);
\draw[dashed] (3,0) arc(0:180:3.5);
\draw[thick] (2,2.4495) arc (44.4153:180:3.5);
\draw[thick] (2,2.4495) arc ( 78.4631:180:2.5);
\draw[thick] (-1,0) arc (0:180:1.5);

\draw[dashed] (8.5,0) arc (0:180:2);
\draw[dashed] (8.5,0) -- (8.5,4.0);
\draw[dashed] (6.5,0) -- (6.5,4.0);
\draw[thick] (8.5,0) arc (0:90:2);
\draw[thick] (8.5,0) -- (8.5,3.5);
\draw[thick] (6.5,2) -- (6.5,3.5);


\end{tikzpicture}
$$


\item Sind letztlich alle drei Ecken im Rand im Unendlichen, so ergeben sich Bilder

$$
\begin{tikzpicture}[scale=0.75]
\draw[->] (-6,0) -- (8,0);
\draw[thick] (2,0) arc (0:180: 1.5);
\draw[thick] (-1,0) arc (0:180:2);
\draw[thick] (2,0) arc (0:180:3.5);

\draw[thick] (7.5,0) arc (0:180:1.5);
\draw[dashed] (7.5,0) -- (7.5,4);
\draw[thick] (7.5,0) -- (7.5,3.5);
\draw[dashed] (4.5,0) -- (4.5,4);
\draw[thick] (4.5,0) -- (4.5,3.5);


\end{tikzpicture}
$$



\end{compactenum}

\end{ex}





\begin{theorem}[Satz von Gauß-Bonnet für hyperbolische Dreiecke]
Der Flächeninhalt eines hyperbolischen Dreiecks ist allein durch die Winkel bestimmt. Genauer gilt für ein Dreieck $\Delta$ mit Winkeln $\alpha, \beta, \gamma$ die Gleichung
$$\mu(A) = \pi- \alpha - \beta - \gamma,$$
also insbesondere $\alpha+\beta+\gamma <\pi$.

\begin{pr}
Wir unterscheiden zwei Fälle:
\begin{compactenum}
\item[\textbf{Fall (a)}] Es gibt eine Ecke $E$ von $\Delta$ im Rand $\R \cup \{\infty \}$. Nach Anwendung einer Möbiustransformation gilt ohne Einschränkung $E=\infty$.
Nach Anwenden von $z \mapsto z+k$ für ein $k\in \R$ und $z \mapsto \lambda z$ für ein $\lambda \in \R$ liegt die Seite $c$ auf $\mathbb{S}^1$. Dann erhalten wir folgendes Bild:

$$
\begin{tikzpicture}
\draw[->] (-2,0) -- (5,0);
\draw[->] (2,-1) -- (2,4); 
\draw[dashed] (4,0) arc (0:180: 2);
\draw[thick] (3.5, 1.3229) arc (41.4096: 120:2);
\draw[dashed] (3.5,0) -- (3.5, 3.5);
\draw[thick] (3.5, 1.3229) -- (3.5, 3);
\draw[dashed] (1,0) -- (1,3.5);
\draw[thick](1, 1.7321) -- (1, 3);
\draw (2,0) -- (4.5, 2.2215);
\draw (2.8,0) arc (0:41.4101:0.8);
\draw (3.5, 2.1229) arc ( 90:131.4101:0.8);
\draw (2.5, 2.4568) -- (4.5,0.1890);
\draw (3.5,0) node[below] {$b$};
\draw (2.3,0.225) node[right] {$\beta$};
\draw (3.06,1.8) node[right] {$\beta$};
\draw (1,0) node[below] {$a$};
\draw (2,0) -- (0.5,2.5982);
\draw (1.8, 2.1940) -- (0, 1.1548);
\draw (1.6,0.6928) arc (120:180:0.8);
\draw (1.3,0.225) node[right] {$\alpha$};
\draw (1.59, 2.0521) arc (30:90:0.7);
\draw (0.92,2.1) node[right] {$\alpha$};



\end{tikzpicture}
$$


Für den Flächeninhalt gilt dann mit $b=\cos \beta$, $a= \cos (\pi-\alpha)$
\setlength{\abovedisplayskip}{5.5pt}
\setlength{\belowdisplayskip}{5.5pt}
\begin{alignat*}{5}
\mu(A) \ \ &=&& \ \ \int_{\Delta} \ \frac{1}{y^2} \ \mathrm{d}x \mathrm{d}y \\
&=&& \ \ \int_{a}^b \int_{\sqrt{1-x^2}}^{\infty} \ \frac{1}{y^2} \ \mathrm{d}x \mathrm{d}y\\
&=&& \ \ \int_a^b \ \frac{1}{\sqrt{1-x^2}} \ \mathrm{d}x\\
&=&& \ \ \int_{\pi-\alpha}^{\beta} \ \frac{1}{\sqrt{1-\cos^2 \theta}} (-\sin \theta) \mathrm{d} \theta \\
&=&& \ \ - (\beta - (\pi-\alpha))\\
&=&& \ \ \pi - \alpha - \beta - \gamma.
\end{alignat*}

\item[\textbf{Fall (b)}] Es sei nun keine Ecke auf dem Rand im Unendlichen. Durch das Verlängern einer Seite bis auf den Rand erhalten wir zwei neue Dreiecke $\Delta_1$ und $\Delta_2$ wie abgebildet:


$$
\begin{tikzpicture}
\draw[->] (-5,0) -- (4,0);
\draw[dashed] (0.5,0) arc (0:180:2);
\draw[thick] (0,1.3229) arc (41.4096: 117:2);
\draw[thick] (2.4167,1.7776) arc (36.3367:143.6633:3);
\draw[thick] (2.4167,1.7776) arc (62.7209:138.5896:2);
\draw[thick, color=red] (3,0) arc (0: 36.3367:3);
\draw[dashed] (3.5,0) arc (0:180:2);
\draw[dashed] (3,0) arc (0:180:3);
\draw[dashed] (3,0) arc(0:180:1.7917);
\draw[thick, color=red] (3,0) arc (0:132.4077:1.7917);

\draw (-2,2.2) node[right] {$\alpha$};
\draw (1.35,2.23) node[right] {$\gamma$};
\draw (0,1.35) node[above] {$\beta$};
\draw (1,1) node[right] {$\delta$};
\draw (1.2,1.2)-- (1,1.9);
\draw (2.5,0.2) node[left] {$\epsilon$};
\draw (2.5,0.2) -- (3,0.2);
\end{tikzpicture}
$$
Dabei ist $\Delta_1$ das von den Winkeln $\alpha, \beta+\delta, \epsilon$ und $\Delta_2$ das von den Winkeln $\delta, \pi-\gamma, \epsilon$ Dreieck und beide haben eine Ecke im Rand im Unendlichen. Mit Fall (a) folgt dann 
$$\mu(\Delta) = \mu(\Delta_1) - \mu(\Delta_2) = (\pi- \alpha - (\beta + \delta)) - (\pi-\delta - (\pi- \gamma)) = \pi-\alpha-\beta-\gamma,$$
was zu zeigen war. $\hfill \Box$

\end{compactenum}



\end{pr}

\end{theorem}












\section{Trigonometrie der hyperbolischen Ebene} %PARAGRAPH II.9




Erinnerung: Im Euklidischen haben wir folgende Identitäten für ein Dreieck $\Delta$ mit Winkeln $\alpha, \beta, \gamma$ und Seiten $a,b,c$ gegenüber von $\alpha, \beta, \gamma$:
\begin{compactenum}
\item \textit{Sinussatz}: Es gilt
$$\frac{a}{\sin \alpha} = \frac{b}{\sin \beta} = \frac{c}{\sin \gamma}.$$
\item \textit{Kosinussatz}: Es gilt
$$c^2=a^2+b^2- 2 ab \cos \gamma$$
\end{compactenum}

(für die Beweise Siehe \hyperlink{Aeinseins}{Aufgabe 1.1}).Im Hyperbolischen gelten ähnliche Beziehungen.


\begin{theorem}
Sei $\Delta$ ein hyperbolisches Dreieck mit Seiten $a,b,c$ und Winkel $\alpha, \beta, \gamma$.
\begin{compactenum}
\item \textit{Sinussatz}: Es gilt
$$\frac{\sinh a}{\sin \alpha} = \frac{\sinh b}{\sin \beta} = \frac{ \sinh c}{\sin \gamma}.$$
\item \textit{1. Kosinussatz}: Es gilt
$$\cos \gamma= \frac{\cosh a  \cosh b - \cosh c}{\sinh a \sinh b}$$
\item \textit{2. Kosinussatz}: Es gilt
$$\cosh c = \frac{ \cos \alpha \cos \beta + \cos \gamma}{\sin \alpha \sin \beta}.$$
\end{compactenum}

\begin{pr}
\begin{compactenum}
\item[(ii)] Wir benutzen das Einheitskreismodell. Seien $E_a, E_b$ und $E_c$ die entsprechenden Ecken gegenüber von $a,b$ und $c$. Ohne Einschränkung sieht $\Delta$ aus wie folgt:




$$
\begin{tikzpicture}
\draw (2,0) arc (0:360:2);
\draw (0,0) arc (0:360:0.05);
\draw (0,0) node[below] {$E_c$};
\draw (1.5,0) arc(0:360:0.05);
\draw (1.5,0) node[below]{$E_a$};
\draw (0,0) -- (1.4,0);
\draw (0.7,1.2) arc(0:360:0.05);
\draw (-0.03,0.04) -- (0.62,1.16);
\draw (0.7,1.5) node[left] {$E_b$};






\end{tikzpicture}
$$



also $E_c=0$, $\Imm(E_a)=0$ und $\Ree( E_a) >0$. Wir benutzen die aus der Übung bekannte Formel
$$d_h^{*}(0,r)= \ln \frac{1+r}{1-r}$$
für $0 <r<1$. Da
$$\tanh x = 1 - \frac{2}{e^{2x}+1}$$
können wir schreiben
$$E_a= \tanh \frac{b}{2}, \qquad E_b = e^{i\gamma}\tanh \frac{a}{2},$$
wobei wir die Kurzschreibweisen
$$b= d_h^{*}(0,E_a), \qquad a= d_h^{*}(0,E_b)$$
benutzen.
Weiter ist 
\setlength{\abovedisplayskip}{5.5pt}
\setlength{\belowdisplayskip}{5.5pt}
\begin{alignat*}{5}
\cosh c \ \ &=&& \ \ \cosh \left( d_h^{*}(E_a, E_b)\right) \\
&=&& \ \ 2 \sinh \left( \frac{1}{2} d_h^{*}(E_a, E_b)\right)^2 + 1\\
&=&& \ \ 2 \frac{\vert E_a- E_b\vert^2}{(1-\vert E_a\vert^2)(1-\vert E_b\vert)^2} + 1.
\end{alignat*}
Die Gültigkeit der letzten Gleichung macht man sich dadurch klar, dass sie für $E_a=0$, $E_b=r>0$ gilt und beide Seiten invariant unter Möbiustransformationen sind. Nun folgt
\setlength{\abovedisplayskip}{5.5pt}
\setlength{\belowdisplayskip}{5.5pt}
\begin{alignat*}{5}
\cosh c \ \ &=&& \ \ 2 \frac{ \vert E_a - E_b\vert^2}{ (1-\vert E_a\vert^2)(1-\vert E_b\vert)^2} +1 \\
&=&& \ \ \cosh a \cosh b -  \sinh a \sinh b \cos \gamma,
\end{alignat*}
wobei man für die letzte Gleichheit $E_a = \tanh \frac{b}{2}$ und $E_b = e^{i\gamma} \tanh \frac{a}{2}$ einsetzt und Additionstheoreme verwendet. $\hfill \Box$


\item[(i)] Wegen (ii) gilt 
\begin{alignat*}{5}
\left( \frac{ \sinh c}{\sin \gamma}\right)^2 \ &=&& \ \frac{ \sinh^2c}{1 - \left( \frac{\cosh a \cosh b - \cosh c}{\sinh a \sinh b} \right)^2} = \frac{ (\sinh a \sinh b \sinh c)^2}{(\sinh a \sinh b)^2 - (\cosh a \cosh b - \cosh c)^2}\\
& =:&& \ \frac{Z(a,b,c)}{N(a,b,c)}.
\end{alignat*}
Um (i) zu zeigen, genügt es also zu zeigen, dass (*) symmetrisch in $a,b,c$ ist. Für den Zähler ist das offensichtlich, für den Nenner folgt mit $\cosh^2x - \sinh^2x =1$ und den Kurzschreibweisen $C=\cosh$, $S=\sinh$ und $\Delta=N(a,b,c)-N(b,c,a)$ beispielsweise
\begin{alignat*}{5}
\Delta \ \ &=&& \ \ S(a)^2S(b)^2 - (C(a)C(b)-C(c))^2 - S(b)^2S(c)^2+(C(b)C(c)-C(a))^2 \\
&=&& \ \ S(a)^2S(b)^2-C(a)^2C(b)^2-C(c)^2-S(b)^2S(c)^2+C(b)^2C(c)^2+C(a)^2 \\
&=&& \ \ S(b)^2\left( S(a)^2-S(c)^2\right) - C(b)^2\left( C(a)^2-C(c)^2\right) + C(a)^2-C(c)^2 \\
&=&& \ \ S(b)^2\left( S(a)^2-S(c)^2\right) + \left( C(a)^2-C(c)^2\right)\left( 1-C(b)^2\right) \\
&=&& \ \ S(b)^2\left( S(a)^2-S(c)^2\right) - \left( C(a)^2-C(c)^2\right) S(b)^2\\
&=&& \ \ S(b)^2\left( S(a)^2-S(c)^2 - C(a)^2 + C(c)^2\right) \\
&=&& \ \ S(b)^2 (-1+1)\\
&=&& \ \ 0.
\end{alignat*}
Die anderen beiden Symmetrien zeigt man analog.

\item[(iii)] Setze $A:= \cosh a$, $B:=\cosh b$, $C:= \cosh c$. Dann folgt aus (i)
$$\cos \gamma = \frac{AB-C}{\sqrt{A^2-1}\sqrt{B^2-1}},$$
wobei erneut die Identität $\cosh^2x-\sinh^2x=1$ benutzt wurde. Damit erhalten wir
$$\sin^2 \gamma = 1 - \cos^2 \gamma =\frac{(A^2-1)(B^2-1) - (AB-C)^2}{(A^2-1)(B^2-1)} =  \frac{D}{(A^2-1)(B^2-1)}$$
mit
$$\qquad D:=(A^2-1)(B^2-1) - (AB-C)^2= 1 + 2ABC - (A^2+B^2+C^2).$$
Aus dieser Gleichung folgt $D \geqslant 0$ sowie Symmetrie von $D$ in $A,B,C$. Entsprechend erhalten wir Formeln für $\cos \alpha$, $\cos \beta$, $\sin \alpha$ und $\sin \beta$:
$$\cos \alpha = \frac{BC-A}{\sqrt{B^2-1}\sqrt{C^2-1}}, \qquad \cos \beta = \frac{CA-B}{\sqrt{C^2-1}\sqrt{A^2-1}}$$
$$\sin \alpha = \frac{\sqrt{D}}{\sqrt{B^2-1}\sqrt{C^2-1}}, \qquad \sin \beta = \frac{\sqrt{D}}{\sqrt{A^2-1}\sqrt{C^2-1}}$$
und es folgt
\setlength{\abovedisplayskip}{5.5pt}
\setlength{\belowdisplayskip}{5.5pt}
\begin{alignat*}{5}
\frac{\cos \alpha \cos \beta + \cos \gamma}{\sin \alpha \sin \beta} \ \ &=&& \ \ \frac{(BC-A)(CA-B) + (C^2-1)(AB-C)}{\sqrt{A^2-1}\sqrt{B^2-1}(C^2-1)} \cdot \frac{\sqrt{A^2-1}\sqrt{B^2-1}(C^2-1)}{\sqrt{D} \sqrt{D}} \\
&=&& \ \ \frac{(BC-A)(CA-B)+ (C^2-1)(AB-C)}{D}\\
&=&& \ \ \frac{ABC^2-B^2C-A^2C+AB+ABC^2-AB - C^3 -C}{D} \\
&=&& \ \ \frac{ C (2ABC - B^2 - A^2- C^2- 1)}{D} \\
&=&& \ \ C,
\end{alignat*}
die Behauptung. $\hfill \Box$
\end{compactenum}
\end{pr}



\end{theorem}



\begin{theorem}[Parallel-Winkel]
Sei $\Delta$ ein hyperbolisches Dreieck mit Winkeln $\alpha=0, \beta=\frac{\pi}{2}$ und einer endlichen Seite $a$. Dann gilt für den 3. Winkel $p(a):= \beta_a$
$$\sin p(a) = \frac{1}{\cosh a}.$$
%bzw. 
%$$\tan p(a) = \frac{1}{\cosh a}.$$

\begin{pr}
Durch Isometrien können wir das Dreieck auf folgende Weise annehmen:
$$
\begin{tikzpicture}
\draw[->] (-4,0) -- (4,0);
\draw[dashed] (2,0) arc(0:180:2);
\draw[dashed] (0,0) -- (0,3.5);
\draw[dashed] (1.5,0) -- (1.5,3.5);
\draw[thick] (0,2) -- (0,3);
\draw[thick] (1.5, 1.3229) -- (1.5,3);
\draw[thick] (1.5, 1.3229) arc(41.4096:90:2);
\draw (1.5,2.3229) arc(90:146:1);
\draw (0.49,1.95) arc (-15:90:0.5);
\draw (0,2.2) node[right] {$\beta$};
\draw (1.55, 1.95) node[left] {$\beta_a$};
\draw (0.75,1.5) node {$a$};
\draw (1.5,1.55) node[right] {$z_0$};
\draw[thin] (0,0) -- (1.5,1.3229);
\draw (1,0) arc(0:41.4096:1);
\draw (0.7,-0.05) node[above] {$\beta_a$};

\end{tikzpicture}
$$
das heißt, der Winkel $0$ wird in der Ecke im Unendlichen angenommen und $a$ ist auf dem Einheitskreis. Nach der hyperbolischen Abstandsformel aus \hyperlink{satzzweiviervier}{Satz 2.4.4 (ii)} gilt dann
\setlength{\abovedisplayskip}{5.5pt}
\setlength{\belowdisplayskip}{5.5pt}
\begin{alignat*}{5}
\cosh a \ \ &=&& \ \ 1 + \frac{ \vert z_0 - i\vert^2}{ 2 \mathrm{Im} \hspace{1pt} z_0} \\
&=&& \ \ 1 + \frac{\left(2 \sin \left( \frac{1}{2} (\frac{\pi}{2} - \beta_a)\right)\right)^2}{2 \sin \beta_a} \\
&=&& \ \ \frac{\sin \beta_a + 2 \sin^2\left( \frac{\pi}{4} - \frac{\beta_a}{2}\right)}{\sin \beta_a} \\
&=&& \ \ \frac{ \sin \beta_a + 2 \left( \frac{1 - \cos \left( \frac{\pi}{2} - \beta_a\right)}{2}\right)}{\sin \beta_a} \\
&=&& \ \ \frac{1}{\sin \beta_a},
\end{alignat*}
was zu zeigen war. $\hfill \Box$

\end{pr}
\end{theorem}












\section{Hyperbolische Dreiecke sind dünn} %PARAGRAPH II.10


\hypertarget{satzzweizehneins}{}
\begin{theorem}    %%Satz 2.22
Sei $\Delta = \overline{pqr}$ ein Dreieck mit Eckpunkten $p,q,r$ in $\He^2$ und $x \in \overline{pq}$. Dann existiert ein Punkt $y \in \overline{pr} \cup \overline{qr}$, sodass für den hyperbolischen Abstand $d_h(x,y)$ gilt 
$$d_h(x,y) \leqslant \ln( 1 + \sqrt{2}).$$
Dabei ist $\delta:=\ln 1 + \sqrt{2}$ eine uniforme, vom Dreieck unabhängige Konstante. Für 
$$U_{\delta}(\overline{pr}) := \left\{ z \in \He^2 \ \vert \ d_h(z, \overline{pr}) \leqslant \delta \right\}$$
besagt der Satz gerade 
$$\overline{pq} \subseteq U_{\delta}(\overline{pr}) \cup U_{\delta}(\overline{qr}).$$
Achtung: Diese Aussage gilt im Euklidischen nicht!


\end{theorem}

Für den Beweis benötigen wir die folgenden drei Lemmata:

\hypertarget{lemmazweizehnzwei}{}
\begin{lemma}    %%Lemma 2.23
Seien $g_1, g_2$ vertikale Geodeätische in $\He^2$. Falls $p \in g_1$ gegen die reelle Achse geht, wächst der Abstand $d_h(p,g_2)$ monoton.
\begin{pr}
Ohne Einschränkung sei $g_2 = i \mathbb{R}$ die imaginäre Achse. 
$$
\begin{tikzpicture}
\draw[->] (-2,0) -- (5,0);
\draw (0,0) -- (0,4.5);
\draw (2,0) -- (2,4.5);
\draw[] (4,0) arc (0:90:4);
\draw (2,4) node[right] {$g_1$};
\draw[] (2,0) arc (0:90:2);
\draw[fill=black] (2.05,3.4641) arc (0:360:0.05);
\draw (2,3.5641) node[right] {$p$};
\draw (0,4) node[left] {$\lambda i$};
\draw (0,2) node[left] {$i$};
\draw (0,0) -- (2,3.4641);
\draw (0,3) node[left] {$g_2$};
\draw (1,1.7320) node[right] {$z_0$};
\draw[fill=black] (1.05,1.7320) arc(0:360:0.05);
\draw[fill=black] (3.55,1.9365) arc(0:360:0.05);
\draw (3.5,1.9365) node[right] {$p'$};
\draw[fill=black] (1.8, 0.9682) arc(0:360:0.05);
\draw (1.7, 0.9682) node[below] {$z_0'$};
\draw[dashed] (0,0) -- (3.5, 1.9365);


\end{tikzpicture}
$$
Ist $p'$ nun näher an der reellen Achse als $p$, so gilt, da $z \mapsto \lambda z$ eine Isometrie ist
$$d_h(g_2,p') = d_h(z_0', i) \geqslant d_h(z_0, i) = d_h(p, \lambda i) = d_h(g_2, p),$$
woraus die Monotonie folgt. $\hfill \Box$

\end{pr}
\end{lemma}

\hypertarget{lemmazweizehndrei}{}
\begin{lemma}    %%Lemma 2.24
Seien $p,q$ Punkte auf einem geodätischen Halbkreis in $\He^2$ mit 
$$p= (\lambda \cos \phi, \lambda \sin \phi), \qquad q = ( \lambda \cos \theta, \lambda \sin \theta)$$
mit $\theta > \phi$. Dann gilt 
$$d_h(p,q) = \ln \frac{\sin \theta (1 + \cos \phi)}{\sin \phi (1+ \cos \theta)}$$
\begin{pr}
Da $z \mapsto \lambda z$ Isometrie ist, gilt $d_h(p,q) = d_h(p',q')$ mit 
$$p'=(\cos \phi, \sin \phi), \qquad q' = (\cos \theta, \sin \theta).$$
$p',q'$ liegen auf dem Einheitskreis, welchen wir parametrisieren durch 
$$c: [0,2\pi] \la \C, \qquad t \mapsto c(t) = (\cos t, \sin t)$$
mit $\Vert c'(t)\Vert = 1$ für alle $t \in [0,2\pi]$. Das abstandsrealisierende geodätische Segment ist dann gerade $c \vert_{[\phi, \theta]}$ und es folgt
$$d_h(p,q) = d_h(p',q') = \int_{\phi}^{\theta} \frac{\Vert c'(t)\Vert}{ \sin t} \ \mathrm{d}t = \int_{\phi}^{\theta} \frac{1}{\sin t}\ \mathrm{d}t = \left[ \ln \frac{\sin t}{1+ \cos t} \right]^{\theta}_{\phi}$$
und damit die Behauptung. $\hfill \Box$

\end{pr}
\end{lemma}


\hypertarget{lemmazweizehnvier}{}
\begin{lemma}   %%Lemma 2.25
Sei $g$ eine Gerade in $\He^2$ und $p \in \He^2$, $p \notin g$. Dann gilt für den Lotfußpunkt $p' \in g$
$$d_h(p,g) = \min_{q \in g} d_h(p,q) = d_h(p,p').$$
\begin{pr}
Sei $q \in g$, sodass die Geodätische $c$ zwischen $q$ und $p$ mit $g$ den Winkel $\alpha \neq \frac{\pi}{2}$ einschließt ($q$ also kein Lotfußpunkt). Dann gilt nach dem Sinussatz
$$\sinh c = \frac{\sinh c}{\frac{\pi}{2}} = \frac{\sinh d_h(p,p')}{\sin \alpha} > \sinh d_h(p,p'),$$
also gerade $c >d_h(p,p')$, was zu zeigen war. $\hfill \Box$

\end{pr}
\end{lemma}



\begin{pr}[\textit{Beweis von Satz 2.10.1.}]
Sei $\Delta= \overline{pqr} \subseteq \He^2$ ein hyperbolisches Dreieck, ohne Einschränkung sei dabei $\overline{pq} \subseteq \mathbb{S}^1$ und $\mathrm{Re}(p) = \mathrm{Re}(r)$ (Jedes Dreieck ist kongruent zu einem solchen speziellen Dreieck).

$$
\begin{tikzpicture}
\draw[very thin, ->] (-5,0) -- (5,0);
\draw[dashed] (2,0) arc(0:180:2);
\draw[thick] (1,1.7321) arc(60:120:2);
\draw[dashed] (-1,0) -- (-1,4.1);
\draw[thick] (-1,1.7321) -- (-1,3.5);
\draw[dashed] (1.4255,0) arc(0:120:3.7380);
\draw[thick] (1,1.7321) arc (27.6050:69.4440:3.7380);
\draw (-1,3.7) node[left] {$r$};
\draw (-1,1.8) node[left] {$p$};
\draw (1,1.8) node[right] {$q$};
\draw[fill=black] (0.05,2) arc(0:360:0.05);
\draw (0,2) node[below] {$x$};
\draw[thin] (0,2) arc(63.4350:90:2.2361);
\draw (-0.8, 2.235) arc( -7:90:0.2);
\draw (-0.92, 2.3251) node {$\cdot$};
\draw (0.4, 2.5720) arc (120:170:0.8);
\draw (0.25,2.7214) arc(130:220:0.2);
\draw (0.27,2.5720) node {$\cdot$};

\end{tikzpicture}
$$

Die folgenden Operationen vergrößern das Dreieck und damit den Abstand von $x$ zu $\overline{pr} \cup \overline{qr}$:
\begin{compactenum}
\item Bewege $r$ in Richtung $\infty$. Dann bleibt der Abstand von $x$ zu $\overline{pr}$ unverändert, während der Abstand von $x$ zu $\overline{qr}$ größer wird.
\item Verschieben von $p$ in Richtung $p'=-1$ bzw. $q$ in Richtung $q'=1$ (entlang des Einheitskreises) vergrößert $\Delta$ und die beiden Abstände ebenfalls.

\end{compactenum}
Wir erhalten also ein ideales Dreieck (mit größtmöglichen Abständen):
\newpage

$$
\begin{tikzpicture}
\draw[very thin,->] (-5,0) -- (5,0);
\draw[very thin, ->] (0,-0.5) -- (0,5);
\draw[thick] (2,0) arc(0:180:2);
\draw[thick] (2,0) -- (2,4);
\draw[dashed] (2,0) -- (2,4.5);
\draw[dashed] (-2,0) -- (-2,4.5);
\draw[thick] (-2,0) -- (-2,4);
\draw (2,0) node[below] {$q'=1$};
\draw (-2,0) node[below] {$p'=-1$};
\draw (2,4) node[right] {$r'=\infty$};
\draw[fill=black] (-0.95,1.7321) arc(0:370:0.05);
\draw (-1,1.7321) node[below] {$x$};
\draw (-1,1.7321) arc (60:90:2);
\draw (2,3.4641) arc (90:150:3.4641);
\draw (2,3.6641) arc (90:182:0.2);
\draw (1.92, 3.5341) node {$\cdot$};
\draw (-1.8,1.99) arc (-7:90:0.2);
\draw (-1.92, 2.08) node {$\cdot$};

\end{tikzpicture}
$$
Im nächsten Schritt überlegen wir uns, für welche Wahl von $x$ der Abstand von $x$ zu $\overline{p'r'} \cup \overline{q'r'}$ maximal wird. 
\begin{compactenum}
\item[(iii)] Nach \hyperlink{lemmazweizehnzwei}{Lemma 2.10.2} vergrößert sich der Abstand von $x$ zu $\overline{p'r'}$, wenn man $x$ in Richtung $q'$ verschiebt monoton. Analog vergrößert sich der Abstand von $x$ zu $\overline{q'r'}$, wenn man $x$ in Richtung $p'$ verschiebt monoton. Der maximal Abstand von $x$ zu beiden Seiten wird also gerade durch $x=i$ realisiert.
\end{compactenum}
Damit ist nach \hyperlink{lemmazweizehnvier}{Lemma 2.10.4} der Abstand von $x$ zu der Vereinigung der beiden Seiten gerade die Länge des geodätischen Segments zwischen $x=i$ und $f$, dem Lotfußpunkt auf (beispielsweise) $\overline{p'r'}$.


$$
\begin{tikzpicture}
\draw[very thin, ->] (0,-0.5) -- (0,5);
\draw[very thin,->] (-6,0) -- (5,0);
\draw[thick] (2,0) arc(0:180:2);
\draw[thick] (2,0) -- (2,4);
\draw[dashed] (2,0) -- (2,4.5);
\draw[dashed] (-2,0) -- (-2,4.5);
\draw[thick] (-2,0) -- (-2,4);
\draw (2,0) node[below] {$q'=1$};
\draw (-2,0) node[below] {$p'=-1$};
\draw (2,4) node[right] {$r'=\infty$};
\draw[fill=black] (0.05,2) arc(0:370:0.05);
\draw (0,2) node[below] {$i$};
\draw[dashed] (0.8284,0) arc(0:180:2.8284);
\draw (0,2) arc (45:90:2.8284);
\draw[fill=black] (-1.95,2.8284) arc(0:360:0.05);
\draw (-2,2.8284) node[left] {$f$};
\draw (-2,0) -- (0,2);
\draw (-0.8,0) arc(0:45:1.2);
\draw (-1.3,0) node[above] {$\frac{\pi}{4}$};
\end{tikzpicture}
$$
Nach \hyperlink{lemmazweizehndrei}{Lemma 2.10.3} gilt also
\setlength{\abovedisplayskip}{5.5pt}
\setlength{\belowdisplayskip}{5.5pt}
\begin{alignat*}{5}
d_h(x, \overline{q'r'} \cup \overline{p'r'}) \ \ &=&& \ \ \int_{\frac{\pi}{4}}^{\frac{\pi}{2}} \ \frac{1}{\sin \psi} \ \mathrm{d} \psi\\ 
&=&& \ \ \ln \frac{\sin \frac{\pi}{2}}{1 + \cos \frac{\pi}{2}} - \ln \frac{\sin \frac{\pi}{4}}{1+ \cos \frac{\pi}{4}} \\
&=&& \ \ 0 - \ln \frac{\frac{1}{\sqrt{2}}}{1+ \frac{1}{\sqrt{2}}} \\
&=&& \ \ \ln( 1+\sqrt{2}).
\end{alignat*}
Da durch die Wahl von $\Delta= \overline{p'q'r'}$ ein Dreieck mit größtmöglichen Abständen gegeben ist, folgt die Behauptung für alle hyperbolischen Dreiecke. $\hfill \Box$
\end{pr}



















\section{Lineare isoperimetrische Ungleichungen} %PARAGRAPH II.11






In diesem Abschnitt wollen wir einen weiteren Unterschied zwischen der euklidischen und hyperbolischen Geometrie betrachten. Betrachte zunächst den euklidischen Fall:\\
Es bezeichne 
$\mathcal{G}_{rk}^{\infty}(\R^2)$ die Menge aller relative kompakten Gebiete des $\mathbb{R}^2$ mit stückweise glattem Rand. Für $\Omega \in \mathcal{G}_{rk}^{\infty}(\R^2)$ bezeichne $A_r(\Omega)$ den euklidischen Flächeninhalt von $A$ sowie $L_e(\partial \Omega)$ die euklidische Länge der Randkurve $\partial \Omega$. Dann gilt die euklidische, quadratische isoperimetrische Ungleichung
$$A_e(\Omega) \leqslant \frac{1}{4\pi} L_e^2(\partial \Omega)$$
mit Gleichheit genau dann, wenn $\Omega$ kongruent zu einem Ball mit Radius $R$ um $0$ ist. Zur Herleitung der Ungleichung betrachte die euklidische isoperimetrische Konstante
$$I_e(\R^2) := \inf_{\Omega \in \mathcal{G}_{rk}^{\infty}(\R^2)} \frac{L_e(\partial \Omega)}{A_e(\Omega)}.$$
Schätze nun wie folgt ab: Sei $l:= L_e(\partial \Omega)$ und wähle $x_0 \in \partial \Omega$. Unterteile dann $\partial \Omega$ in Segmente der Länge $1$ und erhalte durch Verbinden der durch die Einteilung entstehenden Punkte "Dreiecke" mit Seiten $ \sim d,d,1$, wobei wegen der Dreiecksungleichung $d \leqslant l$ gilt. Dann können wir die Fläche von $\Omega$ stückweise aufaddieren und erhalten 
$$A_e(\Omega)= \sum \{\textrm{Dreiecke} \} = l \ \frac{d}{2} \leqslant \frac{l^2}{2} = \frac{1}{2} L_e^2(\partial \Omega).$$
Für die euklidische isomperimetrische Konstante gilt folglich $I_e(\R^2) = 0$. \\

Für den hyperbolischen Fall definieren wir analog 
$$I_h(\He^2) := \inf_{ \Omega \in \mathcal{G}_{rk}^{\infty}(\He^2)} \frac{L_h(\partial \Omega)}{A_h(\Omega)}$$
mit den entsprechenden hyperbolischen Größen. Sei nun $\Omega \in \mathcal{G}_{rk}^{\infty}(\He^2)$. Zur Vereinfachung nehmen wir an, dass $\partial \Omega_1$ der Graph 
$$\partial \Omega_1 = \{ (x,f(x)) \in \He^2 \ \vert \ x\in [a,b] \}$$
einer Funktion $f$ sei.

$$
\begin{tikzpicture}
\draw[very thin, ->] (-4,0) -- (4,0);
\draw (-2,0) -- (-2,4);
\draw (2,0) -- (2,4);
\draw (2,0) node[below] {$b$};
\draw (-2,0) node[below] {$a$};
\draw (2,2) arc(60:130:3.5);
\draw (0,2.8) node {$\partial \Omega_2$};
\draw (-2,1.65) arc (210:340:2.21);
\draw (0,1) node[above] {$\Omega$};
\draw (0,0) node[above] {$\partial \Omega_1$};
\end{tikzpicture}
$$

Dann ist
$$L_h(\partial \Omega) = L_h(\partial \Omega_1) + L_h(\partial \Omega_2) \geqslant L_h(\partial \Omega_1) = \int_a^b \frac{ \sqrt{1 + (f'(x))^2}}{f(x)} \ \mathrm{d}x \geqslant \int_a^b \ \frac{1}{f(x)} \ \mathrm{d}x$$
sowie 
$$A_h(\Omega) \leqslant \int_a^b \left(\int_{f(x)}^{\infty} \frac{\mathrm{d}y}{y^2} \right) \mathrm{d}x = \int_a^b \left[ -\frac{1}{y} \right] ^{\infty}_{f(x)} \mathrm{d}x = \int_a^b \frac{\mathrm{d}x}{f(x)}$$
also
$$A_h(\Omega) \leqslant L_h(\partial \Omega).$$
Dies hat eine lineare isoperimetrische Gleichung zur Folge und es gilt 
$$I_h(\He^2) = \inf_{\Omega \in \mathcal{G}_{rk}^{\infty}(\He^2)} \frac{L_h(\partial \Omega)}{A_h(\Omega)} \geqslant 1.$$
Indem wir in das Einheitskreismodell der hyperbolischen Geometrie wechseln, zeigen wir im Folgenden, dass die Abschätzung für Bälle optimal ist, Gleichheit also gerade durch Bälle (wie im Euklidischen) auftritt. Für $R >0$ sind hyperbolische Bälle bzw. Kreise mit Radius $R$ um $0$ gegeben durch 
$$B_R(0) := \{ z \in \R^2 \ \vert \ d_h^*(0,z) \leqslant R \} \subseteq \D^2$$
$$\partial B_R(0) := \{ z \in \R^2 \ \vert \ d_h^*(0,z) = R \} \subseteq \D^2$$
Beachte: Mengentheoretisch entspricht $\partial B_R(0)$ gerade einem euklidischen Kreis mit Radius $\rho$, wobei der Zusammenhang zwischen $R$ und $\rho$ gegeben ist durch 
$$R= \int_0^{\rho}  \ \frac{2 \mathrm{d}r}{1-r^2} = \ln \frac{1+\rho}{1-\rho},$$
also 
$$\rho= \frac{e^R-1}{e^R+1} = \frac{\cosh R-1}{\sinh R}.$$
Dann gilt für die Fläche von $B_R(0)$
$$A_h(B_R(0)) = \int_{0}^{2\pi} \int_0^{\rho} \ \frac{4r}{(1-r^2)^2} \ \mathrm{d}r \mathrm{d} \theta = \int_0^{2 \pi} \frac{2\rho^2}{1-\rho^2} \ \mathrm{d}\theta = \frac{4\pi \rho^2}{1-\rho^2}$$
und mit 
$$1-\rho^2 = 1-\left( \frac{\cosh R -1}{\sinh R}\right)^2 = \frac{2 (\cosh R -1 )}{\sinh^2 R}$$
also 
$$A_h(B_R(0)) = \frac{4 \pi \rho^2}{1 - \rho^2} = 2\pi \frac{\rho^2 \sinh^2R}{\cosh R-1} = 2\pi \frac{(\cosh R-1)^2}{\cosh R-1} = 2\pi ( \cosh R-1) \sim 2\pi e^R$$
für $R \gg1$, wobei $\rho \sinh R = \cosh R - 1$ genutzt wurde. Für die Länge der Kreislinie $\partial B_R(0)$ rechnen wir
$$L_h(\partial B_R(0)) = \int_0^{2 \pi} \ \frac{2 \rho}{1-\rho^2} \ \mathrm{d} \theta = \frac{4 \pi \rho}{1-\rho^2} = 2 \pi \frac{(\rho \sinh R )\sinh R}{\cosh R -1} = 2\pi \sinh R \sim 2 \pi e^R.$$
Insgesamt gilt damit 
$$\frac{L_h(\partial B_R(0))}{A_h(B_R(0))} = \frac{2 \pi \sinh R}{2 \pi (\cosh R-1)} \overset{R \to \infty}{\la}1$$
und damit gilt für die hyperbolische isoperometrische Konstante
$$I_h(\He^2) = I_h(\D^2) = 1.$$












































\newpage
\thispagestyle{empty}












\chapter{Hyperbolische Parkettierungen und Fuchssche Gruppen} %KAPITEL III
\setlength\abovedisplayshortskip{0pt}
\setlength\belowdisplayshortskip{10pt}
\setlength\abovedisplayskip{10pt}
\setlength\belowdisplayskip{10pt}
 



\section{Parkettierungen und Gruppen} %PARAGRAPH III.1

\thispagestyle{empty}


Die allgemeine Situation ist die folgende: Gegeben sei ein metrischer Raum $(X,d)$ sowie eine abgeschlossene Teilmenge $P\subseteq X$. Eine \textit{Parkettierung} bzw. \textit{Pflasterung von $X$ mit Fundamentalbereich} $P$ ist eine Überdeckung $X= \bigcup_{i \in I} P_i$ mit einer abzählbaren Indexmenge $I$, sodass gilt:
\begin{compactenum}
\item Für jedes $i \in I$ ist $P_i= T_i(P)$ für eine Isometrie $T_i \in \mathrm{Isom}(X)$. Insbesondere sind alle $P_i$ kongruent zu $P$.
\item Für je zwei $i \neq j$ in $I$ gilt $\overset{\circ}{P_i} \cap \overset{\circ}{P_j} = \emptyset$. 

\end{compactenum}


\begin{ex}
\begin{compactenum}
\item $\R^2$ lässt sich beispielsweise durch gleichseitige, kongruente Dreiecke oder Quadrate parkettieren. Auch Variationen sind möglich, haben aber eine geringere Symmetrie zur Folge. Die Symmetriegruppe der Parkettierung
$$
\begin{tikzpicture}
\draw[thin, ->] (-3.5,0) -- (3.5,0);
\draw[thin, ->] (0,-2.5) -- (0,2.5);
\draw[dashed] (-3.5,-2) -- (3.5,-2);
\draw[dashed] (-3.5,-1) -- (3.5,-1);
\draw[dashed] (-3.5,1) -- (3.5,1);
\draw[dashed] (-3.5,2) -- (3.5,2);
\draw[dashed] (-3,-2.5) -- (-3,2.5);
\draw[dashed] (-2,-2.5) -- (-2,2.5);
\draw[dashed] (-1,-2.5) -- (-1,2.5);
\draw[dashed] (1,-2.5) -- (1,2.5);
\draw[dashed] (2,-2.5) -- (2,2.5);
\draw[dashed] (3,-2.5) -- (3,2.5);
\draw[thick] (-2,-2) -- (-1,-2) -- (-1,-1) -- (-2,-1) -- (-2,-2);
\draw (-1.5,-1.5) node {$P$};

\end{tikzpicture}
$$
von $\R^2$ besitzt beispielweise $\mathbb{Z}^2$ als Untergruppe, denn sie ist invariant unter Translationen in $x$- und $y$-Richtung. Allgemeiner gilt für eine Parkettierung $\mathcal{P}$
$$G:= \langle \{T_i \}_{i \in I} \rangle \subseteq \mathrm{Isom}(X) \cap \mathrm{Sym}(\mathcal{P}).$$


\item Die Sphäre $X= \mathbb{S}^2$ lässt sich durch die vier Viertelsphären parkettieren.
\item Betrachte Die Parkettierung
$$
\begin{tikzpicture}
\draw[dashed] (-1,2) -- (1,-2);
\draw[dashed] (-2,0) -- (2,0);
\draw[dashed] (-1,-2) -- (1,2);
\draw[thick] (2,0) -- (0,0) -- (1,2);
\draw (1,0.8) node {$P$};


\end{tikzpicture}
$$
Dann ist die Symmetriegruppe gegeben durch die zyklische Gruppe $G$ der Ordnung $6$, erzeugt von den Rotationen $T_k$ um den Winkel $\frac{2 \pi k}{6}$.



\end{compactenum}
\end{ex}

$\quad$ Operiert nun umgekehrt eine Gruppe $G$ auf $X$, so erhalten wir eine dazugehörige Parkettierung auf $X$ (Modell für den Bahnenraum). Dies wollen wir im Folgenden explizit beschreiben. Sei $(X,d)$ ein zusammenhängender, metrischer Raum und $G \subseteq \mathrm{Isom}(X)$ eine Gruppe von Isometrien von $X$. Dann operiert $G$ diskret auf $X$, falls jede Bahn $G \cdot x:= \{ g \cdot x \ \vert \ g \in G\} \subseteq X$ diskret ist, das heißt für jedes $x \in X$ ein $\epsilon_x >0$ existiert, sodass
$$B_{\epsilon_x}(x) \cap B_{\epsilon_x}(g\cdot x) = B_{\epsilon_x}(x) \cap g \cdot B_{\epsilon_x}(x) = \emptyset$$
für alle $g \in G \setminus \{e_G\}$.


\begin{ex}
\begin{compactenum}
\item $\Z$ operiert diskret auf $\R$ durch Translationen:
$$\cdot: \Z \times \R \la \R, \qquad (k,t) \mapsto k\cdot t = k+t.$$
\item Analog opertiert $\Z^2$ diskret auf $\R^2$ durch 
$$\cdot: \Z^2 \times \R^2 \la \R^2, \qquad ((k,l),(x,y)) \mapsto (k,l) \cdot (x,y) = (x+k,y+l)$$
\item Betrachte nun die hyperbolische Isometrie
$$T: \He^2 \la \He^2, \qquad z \mapsto 2z.$$
Dann ist $\langle T \rangle = \{ \ldots, T^{-2}, T^{-1}, \id, T, T^2, \ldots \} \cong \mathbb{Z}$ und erhalten dadurch eine diskrete Operation auf $\He^2$ (Achtung: Funktioniert im Euklidischen nicht, da $z \mapsto 2z$ keine euklidische Isometrie ist).
\item Die Gruppe $\mathrm{PSL}_2(\Z) \subseteq \mathrm{PSL}_2(\R)$ operiert diskret auf $\He^2$ (siehe später).


\end{compactenum}
\end{ex}



\begin{theorem}[Dirichlet-Parkettierung]
Sei $G$ eine abzählbare Gruppe, welche diskret auf einem metrischen Raum $(X,d)$ operiere und $p \in X$ ein Punkt in $X$, der durch kein Element aus $G \setminus \{e_G\}$ fixiert werde. Dann ist der \textit{Dirichlet-Bereich von $G$ zentriert in $p$}
\begin{alignat*}{5}
D_p(G) \ \ &:=&& \ \ \{ x \in X \ \vert \ d(x,p) \leqslant d(x,g\cdot p) \textrm{ für alle } g \in G\} \\
&=&& \ \ \{x \in X \ \vert \ d(x,p) \leqslant d(g\cdot x,p) \textrm{ für alle } g \in G\}
\end{alignat*}
ein Fundamentalbereich $P$ für eine Parkettierung $\mathcal{P}$ von $X$. Dabei gilt die Mengengleichheit wegen der Invarianz von $d$ unter Isometrien.
\end{theorem}
\begin{pr}[\textit{Bemerkung:}] Jede Bahn von $G$ trifft jeden Punkt von $\overset{\circ}{D_p(G)}$ genau einmal.
\end{pr}
\begin{pr}
Sei $D:=D_p(G)$ und $y \in X$ beliebig. Wir müssen zeigen, dass $y$ in einer Bahn $g\cdot D$ für ein $g \in G$ enthalten ist. Betrachte nun die Bahn $G\cdot y$ von $y$. Nach Voraussetzung ist $G \cdot y$ diskret, es existiert also $y_0 \in G \cdot y$ mit $d(y_0, p)  = \min_{g \in G} d(g \cdot y, p)$. Dann ist aber $d(y_0,p) \leqslant d(g \cdot y_0,p)$ für alle $h \in G$, also gerade $y_0 \in D$. Wegen $G \cdot y = G \cdot y_0$ folgt der erste Teil der Behauptung.
Es bleibt noch zu zeigen, dass die Inneren der Translate disjunkt sind. Hierfür genüg es zu zeigen, dass falls $x_1, x_2 \in \overset{\circ}{D}$ für $x_1 \neq x_2$, dass dann $x_1$ und $x_2$ nicht zu derselben Bahn gehören. Wir überlegen uns hierfür zunächst, dass falls $d(x,p) = d(g\cdot x,p)$ für $g \in G\setminus \{e_G\}$ gilt, dass dann $d(x,p) = d(x, g^{-1} \cdot p)$ und $x$ damit in der Mittelebene von $p$ und $g^{-1}\cdot p$
$$M(p,g^{-1}) := \{x \in X \ \vert \ d(x,p) = d(x,g^{-1} \cdot p\}$$
liegt. Dann gilt entweder $x \notin D$ oder $x \in \partial D$. Insbesondere gilt dann für $x \in \overset{\circ}{D}$ also $d(x,p)<d(g \cdot x, p)$ für alle $g \in G \setminus \{e_G\}$. Sind nun also $x_1 \neq x_2$ in derselben Bahn, also $x_2 = g_0 \cdot x_1$, so folgt 
$$d(x_1, p) < d(g_0 \cdot x_1,p) = d(x_2,p) < d(g_0^{-1} \cdot x_2, p) = d(x_1, p),$$
ein Widerspruch. $\hfill \Box$
\end{pr}


\begin{ex}
\begin{compactenum}
\item Betrachte die Gruppenoperation von $\Z$ auf $\R$ von oben. Dann wird $p=0$ in $\R$ von keinem $k \in \Z\setminus \{0\}$ fix gelassen und Anwenden von Satz 1.3 liefert den Fundamentalbereich $D_0(\Z) = \left[-\frac{1}{2}, \frac{1}{2}\right]$. Die relevanten Mitteleben sind dabei $M(0,1): x = \frac{1}{2}$ sowie $M(0,-1): x = -\frac{1}{2}$.

\item Analog lässt sich dies für die Gruppenaktion von $\Z^2$ auf $\R^2$ anwenden. 

$$
\begin{tikzpicture}
\draw[thin, ->] (-4.5,0) -- (4.5,0);
\draw[thin, ->] (0,-2.5) -- (0,2.5);
\draw[dashed] (-4.5,-2) -- (4.5,-2);
\draw[dashed] (-4.5,2) -- (4.5,2);
\draw[dashed] (-4,-2.5) -- (-4,2.5);
\draw[dashed] (-2,-2.5) -- (-2,2.5);
\draw[dashed] (2,-2.5) -- (2,2.5);
\draw[dashed] (4,-2.5) -- (4,2.5);
\draw[fill=black] (0.05,0) arc(0:360:0.05);
\draw[fill=blue] (2.05,0) arc(0:360:0.05);
\draw[fill=blue] (-1.95,0) arc(0:360:0.05);
\draw[fill=blue] (0.05,2) arc(0:360:0.05);
\draw[fill=blue] (0.05,-2) arc(0:360:0.05);
\draw[fill=green] (2.05,2) arc(0:360:0.05);
\draw[fill=green] (2.05,-2) arc(0:360:0.05);
\draw[fill=green] (-1.95,-2) arc(0:360:0.05);
\draw[fill=green] (-1.95,2) arc(0:360:0.05);
\draw[color=blue] (-3.5,1) -- (4.5,1);
\draw[color=blue] (4.5,1) node[right] {$M(0,(0,1))$};
\draw[color=blue] (-3.5,-1) -- (4.5,-1);
\draw[color=blue] (4.5,-1) node[right] {$M(0,(0,-1))$};
\draw[color=blue] (-1,2.5) -- (-1,-2.5);
\draw[color=blue] (1,2.5) -- (1,-2.5);
\draw[very thick] (1,1) -- (1,-1) -- (-1,-1) -- (-1,1) -- (1,1);
\draw[color=blue] (-1,-2.5) node[left] {$M(0,(-1,0))$};
\draw[color=blue](1,-2.5) node[right] {$M(0,(1,0))$};
\draw[color=green] (-2.5,-0.5) -- (0.5,2.5);
\draw[color=green] (-0.5,2.5) -- (2.5,-0.5);
\draw[color=green] (-2.5,0.5) -- (0.5,-2.5);
\draw[color=green] (-0.5,-2.5) -- (2.5,0.5);
\end{tikzpicture}
$$
Als Fundamentalbereich er gibt sich nun $D_{(0,0)}(\Z^2) = \left\{ (x,y) \in \R^2 \ \vert \ \vert x \vert < \frac{1}{2}, \ \vert y \vert < \frac{1}{2} \right\}.$

\item Das Beispiel aus (ii) lässt sich verallgemeinern. Für zwei linear unabhängige Vektoren $v_1, v_2 \in \R^2$ ist $G:= \{mv_1+ nv_2 \ \vert \ m,n \in \Z \} \cong \Z^2$und wir erhalten durch die Mittelebenenkonstruktion ein Hexagon als Fundamentalbereich.

$$
\begin{tikzpicture}
\draw[thin, ->] (-4.5,0) -- (4.5,0);
\draw[thin, ->] (0,-2.5) -- (0,2.5);
\draw[thick,color=red,->] (0,0) -- (2,0);
\draw[color=red] (1,0) node[below] {$v_1$};
\draw[thick,color=red,->] (0,0) -- (1,2);
\draw[fill=black] (0.05,0) arc(0:360:0.05);
\draw[fill=blue] (2.05,0) arc(0:360:0.05);
\draw[fill=blue] (-1.95,0) arc(0:360:0.05);
\draw[fill=blue] (1.05,2) arc(0:360:0.05);
\draw[fill=green] (-2.95,-2) arc(0:360:0.05);
\draw[fill=green] (3.05,2) arc(0:360:0.05);
\draw[fill=green] (1.05,-2) arc(0:360:0.05);
\draw[fill=blue] (-0.95,-2) arc(0:360:0.05);
\draw[fill=green] (-0.95,2) arc(0:360:0.05);
\draw[color=blue] (1,2.5) -- (1,-2.5);
\draw[color=blue] (-1,2.5) -- (-1,-2.5);
\draw[color=blue] (-1.5,-0.5) -- (1.5,-2);
\draw[color=blue] (-1.5,2) -- (2.5,0);
\draw[color=green] (-2.5,0) -- (1.5,2);
\draw[color=green] (-1.5,-2) -- (2.5,0);
\draw[color=green] (1,2) -- (2.25,-0.5);
\draw[color=green] (-2.25,0.5) -- (-1,-2);

\draw[very thick] (-1,0.75) -- (0,1.25) -- (1,0.75) -- (1,-0.75) -- (0,-1.25) -- (-1,-0.75) -- (-1,0.75);









\end{tikzpicture}
$$

\item Betrachte $T: \He^2 \la \He^2, \ z \mapsto 2z$. Dann ist $i$ nicht fixiert und $G=\langle T \rangle \cong \Z$ operiert auf $\He^2$. Die Mittelebenen sind diesmal hyperboilsche Geodätische, also Halbkreise und wir erhalten folgenden Fundamentalbereich:


$$
\begin{tikzpicture}
\draw[thin, ->] (-4,0) -- (4,0);
\draw[thin, ->] (0,-0.5) -- (0,4.5);
\draw[fill=black] (0.05,2) arc(0:360:0.05);
\draw[fill=black] (0.05,1) arc(0:360:0.05);
\draw[fill=black] (0.05,4) arc(0:360:0.05);
\draw (0,1) node[left] {$\frac{i}{2}$};
\draw (0,2) node[left] {$i$};
\draw (0,4) node[left] {$2i$};
\draw (2.6,2.6) node {$M(i,2)$};
\draw (1.5,0.3) node[left] {$M(i,\frac{1}{2})$};
\draw[very thick] (3,0) arc (0:180:3);
\draw[very thick] (1.5,0) arc (0:180:1.5);
\draw[very thick] (3,0) -- (1.5,0);
\draw[very thick] (-3,0) -- (-1.5,0);

\end{tikzpicture}
$$
\end{compactenum}
\end{ex}



\hypertarget{satzdreieinsfuenf}{}
\begin{theorem}
Die Gruppe
$$\mathrm{SL}_2 \Z := \left\{ T= \begin{pmatrix}a&b\\[-6pt] c&d \end{pmatrix} \in \mathrm{SL}_2\R \ \bigg\vert \ a,b,c,d \in \Z \right\}$$
operiert durch Möbiustransformationen diskret auf $\He^2$, d.h. alle Bahnen sind diskret.
\end{theorem}

\begin{remark}
Die Gruppe $\mathrm{SL}_2\R$ wird mit der von $\R^4$ induzierten Teilraumtopologie versehen.

\end{remark}

\begin{lemma}
Sei $z_0=x_0+iy_0 \in \He^2$ und $K \subseteq \He^2$ kompakt. Dann ist die Menge 
$$E_{z_0}:=E= \left\{T= \begin{pmatrix}a&b\\[-6pt] c&d \end{pmatrix} \in \mathrm{SL}_2\R \ \bigg\vert \ T\cdot z_0 = \frac{az_0+b}{cz_0+d} \in K \right\}$$
kompakt in $\mathrm{SL}_2\R$.
\begin{pr}
Wir müssen zeigen, dass $E$ abgeschlossen und beschränkt bezüglich der von $\R^4$ induzierten Teilraumtopologie ist. Die Abbildung
$$\Psi: \mathrm{SL}_2\R \la \He^2, \qquad \begin{pmatrix}a&b \\[-6pt]c&d\end{pmatrix} \mapsto \frac{az_0+b}{cz_0+d}$$
ist stetig, also ist $E= \Psi^{-1}(K)$ als stetiges Urbild einer kompakten Menge abgeschlossen. Es bleibt also noch die Beschränktheit zu zeigen. Dabei ist $T= \begin{pmatrix}a&b\\[-6pt] c&d \end{pmatrix} \in \mathrm{SL}_2\R$ bezüglich der Teilraumtopologie beschränkt genau dann, wenn die Koeffizienten $a,b,c,d$ beschränkt in $\R$ sind. Nun ist $K$ beschränkt in $\He^2$, das heißt es existiert eine Konstante $M_1 >0$, sodass
$$\vert \Psi(T) \vert = \bigg \vert \frac{az_0+b}{cz_0+d} \bigg\vert \leqslant M_1$$
für alle $T= \begin{pmatrix}a&b\\[-6pt] c&d \end{pmatrix} \in \mathrm{SL}_2\R$, denn eine bezüglich $d_h$ beschränkte Menge ist auch euklidisch beschränkt. Weiter ist $K$ kompakt in $\He^2$, es gibt also $M_2>0$, sodass
$$\mathrm{Im} \hspace{1.5pt} \Psi(z_0) = \mathrm{Im} \hspace{1.5pt} \frac{az_0+b}{cz_0+d} = \Imm \frac{(az_0+b)(c\overline{z}_0+d)}{\vert cz_0+d\vert^2} =\Imm \frac{adz_0+bc\overline{z}_0}{\vert cz_0+d\vert^2}= \frac{\mathrm{Im}\hspace{1.5pt} z_0}{\vert cz_0+d\vert^2} \geqslant M_2.$$
Damit folgt
$$\vert cz_0+d\vert \leqslant \sqrt{\frac{\mathrm{Im}\hspace{1.5pt} z_0}{M_2}}$$
sowie
$$\vert az_0+b\vert \leqslant M_1 \vert cz_0+d\vert \leqslant M_1 \sqrt{\frac{\mathrm{Im} \hspace{1.5pt} z_0}{M_2}}.$$
Wegen 
$$\vert az_0+b\vert = \vert a(x_0+iy_0)+b\vert = \sqrt{ (ay_0)^2 + (ax_0+b)^2} \geqslant \vert ay_0\vert$$
folgt damit 
$$\vert a \vert \leqslant \frac{M_1}{\vert y_0\vert} \sqrt{\frac{\mathrm{Im} \hspace{1.5pt} z_0}{M_2}}$$
und
$$\vert b \vert = \vert az_0+b-az_0 \vert \leqslant \vert az_0+b\vert + \vert az_0 \vert \leqslant M_1 \sqrt{\frac{\mathrm{Im}\hspace{1.5pt} z_0}{M_2}} \left( 1 + \frac{\vert z_0\vert}{\vert y_0\vert} \right).$$
Analog zeigt man die Beschränktheit von $c$ und $d$ und erhält die Behauptung. $\hfill \Box$

\end{pr}
\end{lemma}



\begin{pr}[\textit{Beweis von Satz 3.1.5.}]
Sicherlich ist $\mathrm{SL}_2\Z$ diskret in $\mathrm{SL}_2\R$.Sei nun $z_0 \in \He^2$ und $\epsilon >0$. Die Menge 
\begin{alignat*}{5}
S \ \ &=&& \ \ \left\{T= \begin{pmatrix}a&b\\[-6pt] c&d \end{pmatrix} \in \mathrm{SL}_2\Z \ \bigg\vert \  T \cdot z_0 = \frac{az_0+b}{cz_0+d} \in B_{\epsilon}(z_0) \right\}\\
&=&& \ \ \left\{T= \begin{pmatrix}a&b\\[-6pt] c&d \end{pmatrix} \in \mathrm{SL}_2\R \ \bigg\vert \  T \cdot z_0 = \frac{az_0+b}{cz_0+d} \in B_{\epsilon}(z_0) \right\} \cap \mathrm{SL}_2 \Z
\end{alignat*}
ist als Durchschnitt einer kompakten mit einer diskreten Menge endlich. Durch verkleinern von $\epsilon$ erreicht man 
$$B_{\epsilon}(z_0) \cap B_{\epsilon}(T \cdot z_0) = B_{\epsilon}(z_0) \cap T \cdot B_{\epsilon}(z_0) = \emptyset$$
für alle $T \in \mathrm{SL}_2\Z \setminus \{ \id\}$, was den Satz beweist. $\hfill \Box$


\end{pr}


$\quad$ Im Folgenden wollen wir nun den Dirichlet-Bereich für $\mathrm{SL}_2\Z$ konstruieren. 

\begin{remark}
Ist $(X,d)$ ein metrischer Raum und $p,q \in X$, so ist die Mittelgerade gegeben durch 
$$M_{pq} = \{ x \in X \ \vert \ d(p,x) = d(q,x) \}.$$
Diese entspricht im Euklidischen der Mittelsenkrechten. Sind nun $p,q \in \He^2$ beliebige Punkte, so lassen sich wegen der $2$-Punkt-Homogenität der hyperbolischen Ebene $p,q$ in die spezielle Lage $p=z$ und $q= -\overline{z}$ überführen:

$$
\begin{tikzpicture}
\draw[thin, ->] (-7,0) -- (-1,0);
\draw[thin, ->] (-4,-0.5) -- (-4,2.5);
\draw[thin, ->] (1,0) -- (7,0);
\draw[thin, ->] (4,-0.5) -- (4,2.5);
\draw[dashed] (6,0) arc(0:180:2);
\draw[dashed] (-2,0) arc(0:180:2);
\draw[fill=black] (5.05,1.7321) arc(0:360:0.05);
\draw[fill=black] (3.05,1.7321) arc(0:360:0.05);
\draw[fill=black] (-2.45,1.3229) arc(0:360:0.05);
\draw[fill=black] (-3.45,1.9365) arc(0:360:0.05);
\draw (-3.5,1.9365) node[above] {$p$};
\draw (-2.5,1.3229) node[above] {$q$};
\draw (3,1.7321) node[above] {$-\overline{z}$};
\draw (5,1.7321) node[above] {$z$};
\draw[->] (-1,2) -- (1,2);
\draw (0,2) node[above] {$\mathrm{MT}$};




\end{tikzpicture}
$$



Durch die expliziten Abstandsformeln rechnet man nach, dass in der speziellen Lage die Mittelgerade gerade durch die imaginäre Achse gegeben ist.
\end{remark}

$\quad$ Für die Konstruktion des Dirichlet-Bereichs benötigen wir einen Punkt, der nicht fixiert wird.

\begin{lemma}
Für $\lambda>1$ gilt $T(\lambda i) \neq \lambda i$ für alle Möbiustransformationen $T \in \mathrm{SL}_2\Z \setminus \{\pm \id\}$.
\begin{pr}
Sei $T= \begin{pmatrix}a&b\\[-6pt] c&d\end{pmatrix} \in \mathrm{SL}_2 \Z$, das heißt es gilt $a,b,c,d \in \Z$ sowie $ad-bc=1$. Fixiert $T$ nun $\lambda i$, so erhalten wir
$$\lambda i = T(\lambda i) = \frac{\lambda ai + b}{\lambda c i +d} = \frac{(a\lambda i + b)(d-c\lambda i )}{c^2\lambda^2+d^2}  = \frac{bd+ac\lambda^2}{c^2\lambda^2+d^2} + \frac{\lambda i}{c^2\lambda^2+d^2}.$$
Koeffizientenvergleich liefert zum einen $c^2\lambda^2 + d^2 = 1$, woraus wegen $\lambda >1$ bereits $c=0$ und damit $d=\pm 1$ folgt. Weiter ist $bd+ac\lambda^2 = \pm b =0$, also auch $b=0$. Die Normiertheit impliziert schließlich $a=d$, also $T=\pm I_2$, womit die zu $T$ assoziierte Möbiustransformation also bereits die Identität ist.
 $\hfill \Box$


\end{pr}
\end{lemma}


Wir können nun also $p=\lambda i$ als Zentrum für den Dirichlet-Bereich wählen.

\begin{theorem}
Das geodätische Dreieck
$$\Delta= \left\{ z \in \He^2 \ \big\vert \ \vert z \vert \geqslant 1, \ \vert \mathrm{Re} \hspace{1.5pt} z \vert \leqslant \frac{1}{2} \right\}$$
ist der Dirichletbereich $D_{\lambda i}(\mathrm{SL}_2 \Z)$ von $\mathrm{SL}_2\Z$ zentriert in $\lambda i$.
$$
\begin{tikzpicture}
\draw[thin, ->] (-4,0) -- (4,0);
\draw[thin, ->] (0,-0.5) -- (0,5);
\draw[dashed] (2,0) arc(0:180:2);
\draw (0,1.7) node[left] {$i$};
\draw (0,4) node[left] {$\lambda i$};
\draw[fill=black] (0.05,2) arc(0:360:0.05);
\draw[fill=black] (0.05,4) arc(0:360:0.05);
\draw[thick] (1,1.7321) arc(60:120:2);
\draw[thick] (1,1.7321) -- (1,4);
\draw[thick] (-1,1.7321) -- (-1,4);
\draw[dashed] (-1,0) -- (-1,4.5);
\draw[dashed] (1,0) -- (1,4.5);
\draw (2,0) node[below] {$1$};
\draw (1,0) node[below] {$\frac{1}{2}$};
\draw (-1,0) node[below] {$-\frac{1}{2}$};
\draw (-2,0) node[below] {$-1$};



\end{tikzpicture}
$$


\begin{pr}
Die Isometrien
$$T(z) = z + 1 = \begin{pmatrix}[rr] 1 & 1 \\[-6pt] 0 & 1 \end{pmatrix} \cdot z$$
$$S(z) = -\frac{1}{z} = \begin{pmatrix}[rr] 0 & -1 \\[-6pt] 1&0 \end{pmatrix} \cdot z$$
sind in $\mathrm{SL}_2\Z$ (und erzeugen die Gruppe sogar) und es gilt 
$$S(\infty) = 0, \qquad S(0) = \infty, \qquad S(i) = i, \qquad S(\lambda i ) = \frac{1}{\lambda} i.$$
Die drei Mittelgeraden von $T(\lambda i), T^{-1}(\lambda i), \lambda i$ und $S(\lambda i)$ sind gerade die Seiten von $\Delta$.

$$
\begin{tikzpicture}
\draw[thin, ->] (-4,0) -- (4,0);
\draw[thin, ->] (0,-0.5) -- (0,5);
\draw[dashed] (2,0) arc(0:180:2);
\draw (0,1.7) node[left] {$i$};
\draw (0,4) node[left] {$\lambda i$};
\draw[fill=black] (0.05,2) arc(0:360:0.05);
\draw[fill=black] (0.05,4) arc(0:360:0.05);
\draw[thick] (1,1.7321) arc(60:120:2);
\draw[thick] (1,1.7321) -- (1,4);
\draw[thick] (-1,1.7321) -- (-1,4);
\draw[dashed] (-1,0) -- (-1,4.5);
\draw[dashed] (1,0) -- (1,4.5);
\draw (2,0) node[below] {$1$};
\draw (1,0) node[below] {$\frac{1}{2}$};
\draw (-1,0) node[below] {$-\frac{1}{2}$};
\draw (-2,0) node[below] {$-1$};
\draw (2,4) node[right] {$T(\lambda i)$};
\draw (-2,4) node[left] {$T^{-1}(\lambda i)$};
\draw[fill=black] (2.05,4) arc(0:360:0.05);
\draw[fill=black] (-1.95,4) arc(0:360:0.05);
\draw[fill=black] (0.05,1) arc(0:360:0.05);
\draw (0,1) node[left] {$S(\lambda i)$};


\end{tikzpicture}
$$
Dies zeigt bereits $D_{\lambda i}(\mathrm{SL}_2\Z) \subseteq \Delta$. Angenommen es gelte keine Gleichheit. Dann ist $\Delta$ kein Fundamentalbereich, also überlappen sich die Translate von $\Delta$, das heißt es gibt $z \in \overset{\circ}{\Delta}$ und $R\in \mathrm{SL}_2\Z$ sodass $R(z) \in \overset{\circ}{\Delta}$. Sei $R= \begin{pmatrix}a&b\\[-6pt] c&d \end{pmatrix}$. Dann gilt
\begin{alignat*}{5}
\vert cz+d\vert^2 \ \ &=&& \ \ c^2 \vert z \vert^2 + 2\vert cd \vert \mathrm{Re} \hspace{1.5pt} z + d^2 \\
&>&& \ \ c^2 + d^2 - \vert cd\vert \\
&=&& \ \ (\vert c \vert - \vert d \vert)^2 + \vert cd \vert \\
&\geqslant&& \ \ 1,
\end{alignat*}
wobei die letzte Gleichheit gilt, da der Term ganzzahlig, nichtnegativ und verschieden von Null ist, denn andernfalls wäre $\vert c \vert = \vert d \vert$ und $0=\vert cd\vert = \vert c\vert^2$ und damit $c=d=0$, ein Widerspruch zu $ad-bc=1$. Damit erhalten wir die strikte Ungleichung
$$\mathrm{Im} \hspace{1.5pt} R(z) = \frac{\mathrm{Im} \hspace{1.5pt} z}{\vert cz+d\vert^2} < \mathrm{Im} \hspace{1.5pt} z.$$

Dasselbe Aurgument mit $z$ und $h$ ersetzt durch $R(z)$ und $R^{-1}$ liefert aber auch
$$\mathrm{Im} \hspace{1.5pt} z = \mathrm{Im} \hspace{1.5pt} R^{-1}(R(z)) < \mathrm{Im} \hspace{1.5pt} R(z),$$
ein Widerspruch. Somit folgt $\Delta = D_{\lambda i}(\mathrm{SL}_2\Z)$. $\hfill \Box$
\end{pr}
\end{theorem}

\begin{remark}
Analog wie im Beweis von \hyperlink{satzdreieinsfuenf}{Satz 3.1.5} zeigt man: Eine diskrete Untergruppe von $\mathrm{SL}_2 \R$ oder $\mathrm{PSL}_2 \R$ operiert diskret auf $\He^2$ und liefert einen Dirichletbereich für eine Parkettierung von $\He^2$. Solche Gruppen heißen \textit{Fuchssche Gruppen}.

\end{remark}








 



\section{Dreiecksgruppen und zugehörige Parkettierungen} %PARAGRAPH III.2



Sir betrachten zunächst die Spiegelung an Geodätischen in den uns drei bekannten Modellen der Geometrie. In der euklidischen Ebene $\E^2$ ist die Spiegelung an der euklidischen Geodätischen $g_0: x=0$ (der $y$-Achse) gegeben durch 
$$\sigma_{g_0}: \E^2 \la \E^2, \qquad (x,y) \mapsto (-x,y).$$
Ist $g$ nun beliebig, so wähle eine euklidische Isometrie $\phi \in \mathrm{Isom}(\E^2)$ mit $\phi(g_0) = g$ und setze $\sigma_g := \phi \circ \sigma_{g_0} \circ \phi^{-1}$. Dan ist $\sigma_g$ die Spiegelung an der Geraden $g$. Man sieht leicht ein, dass diese Definition unabhängig von der Wahl von $\phi$ ist.\\
$\quad$ Auf der Sphäre $\Sph^2$ verfahren wir analog. Die sphärischen Geodätischen sind gerade die Großkreise, also speziellen Großkreis wählen wir $g_0$ in der $x$-$z$-Ebene. Dann ist durch 
$$\sigma_{g_0}: \Sph^2 \la \Sph^2, \qquad (x,y,z) \mapsto (x,-y,z)$$
die Spiegelung and $g_0$ gegeben. Für einen beliebigen Großkreis $g$ wähle eine sphärische Isometrie $\phi$ mit $\phi(g_0) = g$ und setze $\sigma_g := \phi \circ \sigma_{g_0} \circ \phi^{-1}$.
In der hyperbolischen Halbebene funktioniert das genau so: Als spezielle hyperbolische Geodätische wählen wir die imaginäre Achse $i\R$. Die Formel 
$$\sinh^2 \left( \frac{1}{2} d_h(z,w)\right) = \frac{\vert z-w \vert^2}{4\  \Img z \Img w}$$
zeigt, dass die Abbildung
$$\sigma_{g_0}: \He^2 \la \He^2, \qquad z \mapsto - \overline{z}$$
eine hyperbolische Isometrie ist, die $g_0$ punktweise festhält und $\sigma_{g_0}^2= \id_{He^2}$ erfüllt, also eine Spiegelung an $g_0$ ist. Ist $g$ nun eine beliebige hyperbolische Geodätische, so wähle eine hyperbolische Isometrie $\phi$ mit $\phi(g_0) = g$ und setze $\sigma_g := \phi \circ \sigma_{g_0} \circ \phi^{-1}$.

\begin{proposition}
Seien $g_1, g_2$ zwei sich in einem Punkt $p$ schneidene Geodätische in $\E^2$ (bzw. in $\Sph^2$ oder $\He^2$) sowie $\sigma_1, \sigma_2$ die Spieglungen an $g_1, g_2$. Dan ist $\sigma_1 \circ \sigma_2$ eine Drehung um $p$ um den Winkel $2 \sphericalangle (g_1, g_2)$.
\begin{pr}
Wir zeigen die Behauptung für den euklidischen Fall; der sphärische und hyperbolische verbleiben als Übung. Es sei ohne Einschränkung $p=0$ und $g_1$, $g_2$ seien euklidische Geodätische gegeben durch 
$$g_1=\{(x, \tan \theta) \ \vert \ x \in \R\}, \qquad g_2=\{(x,\tan \psi) \ \vert \ x \in \R\},$$
mit $\mu, \lambda \in \left[0, \frac{\pi}{2}\right)$ und ohne Einschränkung gelte $\psi> \theta$. Es ist dann $\sphericalangle_p(g_1,g_2) = \psi - \theta$. Wir wollen die Spiegelungen an $g_1$ und $g_2$ nun wie oben beschreiben. Wir suchen also Isometrien $A$ und $B$, welche $g_0$ auf $g_1$ und $g_2$ abbilden. Diese sind aber gerade gegeben durch Drehungen um die Winkel $\theta - \frac{\pi}{2}$ bzw. $\psi - \frac{\pi}{2}$, also durch die Matrizen
$$A=\begin{pmatrix}[rr] \cos\left( \theta - \frac{\pi}{2} \right)& - \sin\left( \theta - \frac{\pi}{2}\right) \\ \sin \left(\theta - \frac{\pi}{2}\right) & \cos\left( \theta - \frac{\pi}{2} \right)\end{pmatrix}, \qquad B=\begin{pmatrix}[rr] \cos\left( \psi - \frac{\pi}{2} \right)& - \sin\left( \psi - \frac{\pi}{2}\right) \\ \sin \left(\psi - \frac{\pi}{2}\right) & \cos\left( \psi - \frac{\pi}{2} \right)\end{pmatrix}.$$
Mit
$$\sigma_{g_0}=\matx{-1}{0}{0}{1}$$
folgt
\begin{alignat*}{5}
\sigma_{g_1} =  A \circ \sigma_{g_0} \circ A^{-1} \ \ &=&& \ \ \begin{pmatrix}[rr] \cos\left( \theta - \frac{\pi}{2} \right)& - \sin\left( \theta - \frac{\pi}{2}\right) \\ \sin \left(\theta - \frac{\pi}{2}\right) & \cos\left( \theta - \frac{\pi}{2} \right)\end{pmatrix} \matx{-1}{0}{0}{1} \begin{pmatrix}[rr] \cos\left( \theta - \frac{\pi}{2} \right)& - \sin\left( \theta - \frac{\pi}{2}\right) \\ \sin \left(\theta - \frac{\pi}{2}\right) & \cos\left( \theta - \frac{\pi}{2} \right)\end{pmatrix} \\
&=&& \ \ \begin{pmatrix}[rr] - \cos^2\left( \theta - \frac{\pi}{2}\right) + \sin^2 \left( \theta - \frac{\pi}{2}\right) & - 2 \sin \left( \theta - \frac{\pi}{2}\right) \cos \left( \theta - \frac{\pi}{2}\right) \\ - 2 \sin \left( \theta - \frac{\pi}{2}\right) \cos \left( \theta - \frac{\pi}{2}\right) & - \sin^2 \left( \theta - \frac{\pi}{2}\right) + \cos^2\left( \theta - \frac{\pi}{2}\right) \end{pmatrix} \\
&=&& \ \ \begin{pmatrix} - \cos (2\theta - \pi) & -\sin ( 2 \theta - \pi) \\ - \sin (2 \theta - \pi) & \cos (2\theta - \pi) \end{pmatrix},
\end{alignat*}
und genau so 
$$\sigma_{g_2}=\begin{pmatrix} - \cos (2\psi - \pi) & -\sin ( 2 \psi - \pi) \\ - \sin (2 \psi- \pi) & \cos (2\psi - \pi) \end{pmatrix}.$$
Damit ist die Hintereinanderausführung der beiden Spiegelungen gerade
\begin{alignat*}{5}
\sigma_1 \circ \sigma_2 \ \ &=&& \ \ \begin{pmatrix} - \cos (2\theta - \pi) & -\sin ( 2 \theta - \pi) \\ - \sin (2 \theta - \pi) & \cos (2\theta - \pi) \end{pmatrix}\begin{pmatrix} - \cos (2\psi - \pi) & -\sin ( 2 \psi - \pi) \\ - \sin (2 \psi- \pi) & \cos (2\psi - \pi) \end{pmatrix}\\
&=&& \ \ \begin{pmatrix} \cos(2 (\psi - \theta)) & \sin( 2(\psi - \theta)) \\ -\sin(2(\psi-\theta)) & \cos(2 (\psi-\theta)) \end{pmatrix},
\end{alignat*}
was also der Drehung um den doppelten Schnittwinkel entspricht. $\hfill \Box$

\end{pr}

\end{proposition}



Sei nun $\Delta = ABC$ ein euklidisches (bzw. sphärisches oder hyperbolisches) Dreieck mit Winkeln $\alpha, \beta, \gamma$ und Schenkeln $a,b,c$. Wir spiegeln nun das Dreieck an den Seiten von $\Delta$. Als eine notwendige Bedingung dafür, dass sich dadurch eine Parkettierung des zugrundeliegenden Raumes ergibt, müssen sich die Innenwinkel $\alpha, \beta$ und $\gamma$ jeweils zu $2 \pi$ vervielfachen, d.h. $l \alpha = 2\pi$ für ein $l \in \mathbb{N}$, $n \geqslant 3$ und ebenso für $\beta$ und $\gamma$. Weiter fordert man, dass sich die von $2$ Spiegelungen erzeugten Drehungen ebenfalls zu $2\pi$ vervielfachen, also $2k \alpha = 2 \pi$ für $k \geqslant 2$ und analog für $\beta$ und $\gamma$. Dies führt zu folgender Definition:

\begin{defin}
Sei $\Delta=ABC$ ein euklidisches (bzw. sphärisches oder hyperbolisches Dreieck) mit Schenkeln $a,b,c$ und Winkeln $\alpha = \frac{\pi}{l}, \beta = \frac{\pi}{m}, \gamma= \frac{\pi}{n}$ für $l,m,n \geqslant 2$. Dann heißt die von den Spiegelungen $\sigma_a, \sigma_b, \sigma_c$ erzeugte Gruppe \textit{Dreiecksgruppe} und wird notiert als $\Delta(l,m,n) := \langle \sigma_a, \sigma_b, \sigma_c \rangle \subseteq \mathrm{Isom}(X)$, wobei $X= \E^2$ (bzw. $X=\Sph^2$ oder $X=\He^2$). Es gelten folgende Relationen in $\Delta(l,m,n)$:
$$\sigma_a^2 = \sigma_b^2 = \sigma_c^2= \id_X, \qquad (\sigma_c \circ \sigma_b)^{l} = \id_X, \qquad  (\sigma_c \circ \sigma_a)^m = \id_X, \qquad (\sigma_a \circ \sigma_b)^n = \id_X$$

\end{defin}


Im Folgende wollen wir die Möglichkeiten für $l,m,n$ untersuchen. Dies müssen wir für jede Geometrie separat tun.\\

\begin{theorem}[Satz von Gauß-Bonnet für euklidische Dreiecke]
Ist $\Delta\subseteq \E^2$ ein euklidisches Dreieck mit Innenwinkeln $\alpha, \beta, \gamma$, so ist $\alpha + \beta + \gamma = \pi$.
\end{theorem}

Eine euklidische Dreiecksgruppe erfüllt also 
$$\frac{1}{l} + \frac{1}{m} + \frac{1}{n} = 1$$
mit $l,m,n \geqslant 2$. Es ergeben sich durch sukzessives Steigern der Konstanten die $3$ Lösungen $\Delta_1=\Delta(2,3,6), \Delta_2=\Delta(2,4,4)$ sowie $\Delta_3=\Delta(3,3,3)$. Diese liefern folgende Parkettierungen:

$$
\begin{tikzpicture}
\draw[thick, fill=green, opacity=0.2] (-1,0) -- (0,0) -- (-1,2) -- (-1,0);
\draw[thick] (-1,0) -- (0,0) -- (-1,2) -- (-1,0);
\draw[thin] (-1,0) -- (-1,-2) --(0,0);
\draw[thin] (0,0) -- (1,-2) --(1,0) -- (0,0);
\draw[thin] (0,0) -- (1,2) --(1,0);
\draw[thin] (-1,-2) -- (1,-2);
\draw[thin] (-1,2) -- (1,2);
\draw[thin] (0,-2) -- (0,2);
\draw[thin] (-1,2) -- (-2,0) -- (-1,-2);
\draw[thin] (-2,0) -- (-1,0);
\draw[thin] (1,0) -- (2,0);
\draw[thin] (1,2) -- (2,0) -- (1,-2);
\draw[thick, fill=blue, opacity=0.2] (4,0) -- (5,0) -- (4,1) -- (4,0);
\draw[thick] (4,0) -- (5,0) -- (4,1) -- (4,0);
\draw[thin] (4,0) -- (4,-1) --(5,0);
\draw[thin] (5,0) -- (6,-1) --(6,0) -- (5,0);
\draw[thin] (5,0) -- (6,1) --(6,0);
\draw[thin] (4,-1) -- (6,-1);
\draw[thin] (4,1) -- (6,1);
\draw[thin] (5,-1) -- (5,1);
\draw[thin] (4,1) -- (3,0) -- (4,-1);
\draw[thin] (3,0) -- (4,0);
\draw[thin] (6,0) -- (7,0);
\draw[thin] (6,1) -- (7,0) -- (6,-1);



\draw[thick,fill=red,opacity=0.2] (9,0) -- (10.5,0) -- (9.75,1.3) -- (9,0);
\draw[thick] (9,0) -- (10.5,0) -- (9.75,1.3) -- (9,0);
\draw (9,0) -- (10.5,0) -- (9.75,-1.3) -- (9,0);
\draw (10.5,0) -- (12,0) -- (11.25, 1.3) -- (10.5,0);
\draw (10.5,0) -- (12,0) -- (11.25, -1.3) -- (10.5,0);
\draw (9.75,1.3) -- (11.25,1.3);
\draw (9.75,-1.3) -- (11.25,-1.3);



\end{tikzpicture}
$$

Für den sphärischen Fall betrachte zunächst folgende Proposition:

\begin{proposition}[Satz von Gauß-Bonnet für sphärische Dreiecke] %%Proposition 3.4
Ist $\Delta \subseteq \Sph^2$ ein sphärisches Dreieck mit Innenwinkeln $\alpha, \beta, \gamma$, so ist $\alpha + \beta + \gamma > \pi$.
\begin{pr}
Für zwei Großkreise auf der Sphäre, die sich im Winkel $\alpha \leqslant \pi$ schneiden bezeichne $M(\alpha)$ die eingeschlossene Fläche. Für diese gilt $M(\alpha) = M(\Sph^2_R) \cdot \frac{\alpha}{2\pi}$. Weiter ist
$$\Sph^2_R = 2 (M(\alpha) + M(\beta) + M(\gamma)) - 2 \Delta,$$
also
$$4 \pi R^2 = A(\Sph^2_R) = 4 R^2 ( \alpha + \beta + \gamma) - 2 A(\Delta),$$
und folglich
$$0 < A(\Delta) = 2R^2 ( \alpha + \beta + \gamma - \pi),$$
woraus $\alpha + \beta + \gamma >\pi$ folgt. $\hfill \Box$
\end{pr}
\end{proposition}

Für die Sphäre erhalten wir also als Bedingung für eine Dreiecksgruppe 
$$\frac{1}{l} + \frac{1}{m} + \frac{1}{n} >1$$
mit $l,m,n \geqslant 2$. Durch sukzessives Steigern der Konstanten erhalten wir die Lösungen $\Delta_{1_n} = \Delta(2,2,n)$, $\Delta_2 = \Delta(2,3,3)$, $\Delta_3 = \Delta(2,3,4)$ und $\Delta_4 = \Delta(2,3,5)$ für $n \geqslant 2$ beliebig.\\

Für $\He^2$ wissen wir bereits, dass gilt $\alpha + \beta + \gamma <\pi$, die Bedingung für Dreiecksgruppen ergibt sich hier also zu 
$$\frac{1}{l} + \frac{1}{m} + \frac{1}{n} <1$$
für $l,m,n \geqslant 2$. Hierfür gibt es unendlich viele Lösungen. Beachte: $l=m=n=\infty$ ist erlaubt.



\begin{remark}
Wir haben nun für die euklidische, sphärische und hyperbolische Geometrie jeweils eine Version des Satzes von Gauß-Bonnet gesehen. Doch was haben diese mit dem Satz aus der Topologie zu tun? Für eine zweidimensionale, kompakte, orientierbare Mannigfaltigkeit $M$ mit Rand $\partial M$, Gaußkrümmung $\kappa$ und geodätische Randkrümmung $k_g$ auf $\partial M$ gilt:
$$2 \pi \chi(M) = \int_M \kappa\ \mathrm{d}A + \int_{\partial M} k_g \ \mathrm{d}s.$$
Betrachte $M=\Delta=\Delta(\alpha, \beta, \gamma) \subseteq X$ mit $X\in \{\E^2, \Sph^2, \He^2\}$. Aus der Differentialgeometrie wissen wir, dass diese drei Räume von konstanter Krümmung $0,1$ bzw. $-1$ sind. Weiter ist die Eulercharakteristik eines Dreiecks sicherlich $1$ und die geodätische Krümmung des Randes verschwindet überall außer in den Ecken. Damit ergibt sich das Integral zu 
$$2\pi = \int_{\Delta} \kappa \ \mathrm{d}A + \int_{\partial \Delta} k_g \ \mathrm{d}s = \mu(\Delta) \kappa + k_g(A) + k_g(B) + k_g(C) = \mu(\Delta) \kappa + (\pi-\alpha) + (\pi-\beta) + (\pi-\gamma).$$
Wir erhalten damit
\begin{compactenum}
\item Im Euklidischen 
$2\pi = \pi - \alpha +\pi - \beta + \pi - \gamma$, also $\pi = \alpha + \beta + \gamma$,
\item Im Sphärischen $2 \pi = \mu(A) + \pi - \alpha +\pi - \beta + \pi - \gamma$, also $\mu(A)=\alpha+\beta+\gamma-\pi$ und 
\item Im Hyperbolischen $2\pi = -\mu(A)+\pi - \alpha +\pi - \beta + \pi - \gamma$, also $\mu(A) = \pi-\alpha-\beta-\gamma$,
\end{compactenum}
also gerade die drei vorgestellten Versionen.


\end{remark}




\begin{remark}
Man kann zeigen: Hyperbolische Dreiecksgruppen operieren diskret auf $\He^2$ und induzieren also hyperbolische Parkettierungen (Siehe z.B. [Caratheodory, Funktionentheorie II, S: 159 ff]).

\end{remark}





\newpage
\thispagestyle{empty}


































































\chapter{Höhere Dimensionen und weitere Modelle} %KAPITEL IV
\setlength\abovedisplayshortskip{0pt}
\setlength\belowdisplayshortskip{10pt}
\setlength\abovedisplayskip{10pt}
\setlength\belowdisplayskip{10pt}
 



\section{Höhere Dimensionen} %PARAGRAPH IV.1

\thispagestyle{empty}


Für die euklidische Ebene $\E^2= (\R^2, d_e)$ ist der $n$-dimensionale euklidische Raum $\E^n = (\R^2, d_e)$ mit 
$$d_e: \R^n \times \R^n \la \R, \qquad (x,y) \mapsto d_e(x,y) = \sqrt{\sum_{k=1}^n (x_k-y_k)^2}$$
die naheliegende Verallgemeinerung. Durch Nullsetzen der Koordinaten $3,...,n$ sehen wir in ihr die ebene euklidische Geometrie. Analog können wir die zweidimensionale sphärische Geometrie auf $\Sph^2$ auf $$\Sph^n:= \left\{ x=(x_1, \ldots, x_{n+1} \in \R^{n+1} \ \bigg\vert \ \sum_{k=1}^{n+1} x_k^2 = 1 \right\}$$
übertragen. Auch hier findet man $\Sph^2$ wieder. Für die zweidimensionale hyperbolische Geometrie haben wir bereits das Halbebenenmodell
$$\He^2 = \{ (x_1, x_2) \in \R^2 \ \vert \ x_2 >0 \}, \qquad \mathrm{d}s_{h}^2 = \frac{\mathrm{d}x_1^2 + \mathrm{d}x_2^2}{x_2^2}$$
kennengelernt. Für eine differentierbare Kurve $c:[a,b] \la \He^2$, $c(t) = (x_1(t), x_2(t))$ liefert dies eine Längenmessung
$$L_h(c) = \int_a^b \Vert c'(t)\Vert^{h}_{c(t)} \ \mathrm{d}t = \int_a^b \frac{\sqrt{(x_1'(t))^2 + (x_2'(t))^2}}{x_2(t)} \ \mathrm{d}t$$
ein Modell für die $n$-dimensionale hyperbolische Geometrie ist dann kanonisch gegeben durch 
$$\He^n = \{(x_1, \ldots, x_n) \in \R^n \ \vert \ x_n >0 \},$$
dem \textit{oberen Halbraum}. Für das Wegelement gilt 
$$\mathrm{d}s_{\mathbb{H}^n}^2 = \frac{\sum_{k=1}^n \mathrm{d}x_k^2}{x_n^2},$$
wir erhalten für eine differenzierbare Kurve $c(t) = (c_1(t), \ldots, c_n(t))$ also durch 
$$L_{\mathbb{H}^n}(c) =\int_a^b \Vert c'(t) \Vert^{\mathbb{H}^n}_{c(t)} \ \mathrm{d}t = \int_a^b \frac{\sqrt{\sum_{k=1}^n (c_k'(t))^2 }}{c_n(t)} \ \mathrm{d}t$$
eine Längenmessung. Diese liefert eine Längenmetrik
$$d_{\He^n}: \He^n \times \He^n \la \R, \qquad (p,q) \mapsto d_{\He^n}(p,q) := \inf_{c \in\Omega_{pq}(\He^n)} L_{\mathrm{hyp}}(c).$$
Damit wird $(\He^n, d_{\He^n})$ zu einem metrischen Raum. Dieser ist ein Modell für die $n$-dimensionale hyperbolische Geometrie.\\
Weiter hatten wir bereits das Einheitskreismodell für die ebene hyperbolische Geometrie kennengelernt. Dieses lässt sich ebenfalls verallgemeinern zu 
$$\D^n := \{ (x_1, \ldots, x_n) \in \R^n \ \vert \sum_{k=1}^n x_k^2 <1 \},$$
wobei für das Wegelement
$$\mathrm{d}s_{\D^n}^2 = \frac{4 \sum_{k=1}^n \mathrm{d}x_k^2}{\left( 1 - \sum_{k=1}^n x_k^2\right)^2}$$
gilt. Damit wird $(\D^n, d_{\D^n})$ zu einem metrischen Raum, dem \textit{Einheitsballmodell} der $n$-dimensionalen hyperbolischen Geometrie.

\begin{remark}
\begin{compactenum}
\item Wir werden sehen: Die beiden vorgestellten Modelle $(\He^n, d_{\He^n})$ und $(\D^n, d_{\D^n})$ sind isometrisch.
\item $\He^n$ und $\D^n$ enthalten (viele) Kopien von $\He^2, \D^2$ (beispielsweise durch Nullsetzen von Koordinaten).
\end{compactenum}

\end{remark}





 

\newpage





\section{Das Hyperboloidmodell} %PARAGRAPH IV.2



Definiere auf $\R^{n+1}$ die quadratische Form 
$$\star: \R^{n+1} \times \R^{n+1} \la \R, \qquad (x,y) \mapsto x\star y = \sum_{k=1}^n x_ky_k - x_{n+1}y_{n+1}.$$
Definiere für $c \in \R$ dann $\Loid^n_c = \{ x \in \R^{n+1} \ \vert \ x \star x = c \}$.
$\Loid_0$ heißt auch \textit{Nullkegel}. $\Loid^3_{-c^2}$ spielt in der speziellen Relativitätstheorie eine wichtige Rolle (Lorentz-Minkowski-Raumzeit). Für unser hyperbolisches Modell setze
$$\Loid^n := \Loid^n_{-1} := \{ x=(x_1, \ldots, x_{n+1}) \in \R^{n+1} \ \vert \ x_1^2 + \ldots + x_n^2 - x_{n+1}^2 = -1 \}.$$
Im Folgenden wollen wir eine Längemessung auf $\Loid^n$ definieren. Für eine differenzierbare Kurve  $c: I \la \Loid^n, \ t \mapsto c(t)=(c_1(t), \ldots, c_{n+1}(t))$ gilt $c(t) \star c(t) = -1$ für alle $t \in I$, also 
$$-1 = c(t) \star c(t) = c_1(t)^2 + \ldots + c_n(t)^2 - c_{n+1}(t)^2.$$
Ableiten auf beiden Seiten liefert
$$0 = c_1'(t) c_1(t) + \ldots + c_n'(t) c_n(t) - c_{n+1}'(t) c_{n+1}(t) = c'(t) \star c(t).$$
der Tangentialraum an $c(t)$ in $p \in \Loid^n$ ist also 
\begin{alignat*}{5}
T_p \Loid^n \ \ &=&& \ \ \{ c'(0) \in \R^{n+1} \ \vert \ c:(-\epsilon, \epsilon) \la \Loid^n \textrm{ ist differenzierbar mit }c(0) = p\} \\
&=&& \ \ \{ x \in \R^{n+1} \ \vert \ x \star p = 0 \}\\
&=&& \ \ p^{\perp_{\star}},
\end{alignat*}
der Orthogonalraum von $p$ bezüglich der quadratischen Form $\star$.


\hypertarget{lemmavierzweieins}{}
\begin{lemma} %%Lemma 4.1
Die quadratische Form $\star$ auf $\R^{n+1}$ eingeschränkt auf den Tangentialraum $T_p \Loid^n$ von $\Loid^n$ im Punkt $p$ ist positiv definit für alle $p \in \Loid^n$.
\begin{pr}
Sei $p=(p_1, \ldots, p_n, p_{n+1}) = (\hat{p}, p_{n+1}) \in \Loid^n$ mit $\hat{p} \in \R^n$ und $x=(\hat{x},x_{n+1}) \in T_p \Loid^n$ mit $\hat{x} \in \R^n$. Sei weiter mit $\langle \cdot, \cdot \rangle$ das Standardskalarprodukt auf $\R^n$ gegeben. Wir betrachten zwei Fälle.
\begin{compactenum}
\item[\textbf{Fall (a)}] Es gilt $x_{n+1}=0$. Dann ist $x\star x = \langle \hat{x}, \hat{x}\rangle$ und $\star$ ist damit wegen der positiven Definitheit von $\langle \cdot, \cdot \rangle$ ebenfalls positiv definit.
\item[\textbf{Fall (b)}] Sei nun $x_{n+1}\neq 0$. Dann gilt auch $x=( \hat{x}, x_{n+1}) \neq 0$. Wir müssen also zeigen, dass $x \star x >0$. Nach Definition des Tangentialraums ist 
$$0=x \star p = x_1p_1 + \ldots + x_np_n - x_{n+1} p_{n+1} = \langle \hat{x}, \hat{p} \rangle - x_{n+1} p_{n+1}$$
und da $p\in \Loid^n$  weiter
$$-1= p \star p = \langle \hat{p}, \hat{p} \rangle - p_{n+1}^2.$$
Die Cauchy-Schwarz-Ungleichung für $\langle \cdot, \cdot \rangle$ ergibt 
$$\langle \hat{x} \hat{x} \rangle \langle \hat{p} \hat{p} \rangle \geqslant \langle \hat{x}, \hat{p} \rangle^2 = (x_{n+1} p_{n+1})^2 = x_{n+1}^2 (1 + \langle \hat{p}, \hat{p} \rangle )$$
und damit
$$(x \star x) \langle \hat{p}, \hat{p} \rangle = \left( \langle \hat{x}, \hat{x} \rangle - x_{n+1}^2\right) \langle \hat{p}, \hat{p} \rangle \geqslant x_{n+1}^2 >0.$$
Da $\langle \hat{p}, \hat{p} \rangle >0$, folgt $x \star x >0$, was zu zeigen war. $\hfill \Box$.


\end{compactenum}

\end{pr}
\end{lemma}


\begin{remark}
\begin{compactenum}
\item Aus \hyperlink{lemmavierzweieins}{Lemma 4.2.1} folgt, das $\star$ für alle $p \in \Loid^n$ ein Skalarprodukt auf $T_p\Loid^n$ definiert, also eine Riemannsche Metrik ist. Das Wegelement
$$\mathrm{d}s_{\Loid^n}^2 = \sum_{k=1}^n \mathrm{d}x_k^2 - \mathrm{d}x_{n+1}^2$$
induziert für differenzierbare Kurven $c: I \la \Loid^n$, $c(t) = (c_1(t), \ldots, c_{n+1}(t))$ eine Längenmessung
$$L_h(c) = \int_I \sqrt{ \sum_{k=1}^n c_k'(t)^2 - c_{n+1}'(t)^2} \ \mathrm{d}t.$$
Diese liefert eine Abstandsfunktion $d_{\Loid^n}$, womit $(\Loid^n, d_{\Loid^n})$ zu einem metrischen Raum wird, dem Hyperboloidmodell für die $n$-dimensionale hyperbolische Geometrie.
\item Die Kurve
$$c: \R \la \Loid^n, \qquad t \mapsto c(t) = \left( \sinh t, 0, \ldots, 0 , \cosh t\right)$$
ist eine isometrische Einbettung (d.h. $c(\R)$ ist eine hyperbolische Geodätische), denn für jedes $t\in \R$ ist $c(t) \star c(t) = \sinh^2 t - \cosh^2 t = -1$ sowie
$$\Vert c'(t)\Vert_{\Loid^n}^2 = \cosh^2 t - \sinh^2 t = 1,$$
also 
$$d_{\Loid^n}(c(t_1), c(t_2)) = \int_{t_1}^{t_2} \Vert c'(t) \Vert _{\Loid^n}^2  \ \mathrm{d}t = \int_{t_1}^{t_2} \ \mathrm{d}t = t - s = d_e(s,t),$$
das heißt $c$ ist eine abstandstreue Kurve, also eine Isometrie.

\end{compactenum}

\end{remark}













 



\section{Die Modelle sind isometrisch} %PARAGRAPH IV.3




\hypertarget{satzvierdreieins}{}    %%SATZ 4.2
\begin{theorem}
Die metrischen Räume $(\He^n, d_{\He^n}), (\D^n, d_{\D^n})$ und $(\Loid^n, d_{\Loid^n})$ sind isometrische Modelle der $n$-dimensionalen hyperbolischen Geometrie.

\begin{pr}
Für den Beweis wollen wir ein weiteres Modell angeben und anschließend explizite Isometrien in die drei obigen Modelle angeben. Betrachte die obere Hemisphäre
$$\KK^n := \left\{ (x_1, \ldots, x_{n+1}) \in \R^{n+1} \ \bigg\vert \ \sum_{i=1}^{n+1} x_i^{2} = 1, \ x_{n+1} >0 \right\}$$
mit dem Wegelement
$$\mathrm{d}s_{\KK^n}^2 = \frac{\mathrm{d}x_1^2 + \ldots + \mathrm{d}x_{n+1}^2}{x_{n+1}^2}.$$
Diese wird zu einem metrischen Raum $(\KK^n, d_{\KK^n})$.  Betrachte nun die Abbildung
\begin{alignat*}{5}
\alpha: \KK^n \ \ &\la&& \ \ \He^n := \{ (1,x_2, \ldots, x_{n+1}) \in \R^{n+1} \ \vert \ x_{n+1}>0 \}\\
x=(x_1, \ldots, x_{n+1}) \ \ & \mapsto&& \ \ \alpha(x) =: y = \left( 1, \frac{2x_2}{x_1+1}, \ldots, \frac{2x_{n+1}}{x_1+1} \right)=:(y_1,\ldots, y_{n+1})
\end{alignat*}
Dann ist $\alpha$ gerade die Zentralprojektion von $(-1,0, \ldots, 0)$ aus.

$$
\begin{tikzpicture}

\draw[very thin,->] (-3,0) -- (4,0);
\draw (4,0) node[below] {$x_1$};
\draw[very thin,->] (0,-2.5) -- (0,3.5);
\draw (0,3.5) node[left] {$x_{n+1}$};
\draw[thick] (2,0) arc(0:180:2);
\draw[dashed] (-2,0) arc (180:360:2);
\draw (-1.3,1.7) node[above] {$\KK^n$};
\draw[fill=black] (-1.95,0) arc (0:360:0.05);
\draw (-2,-0.5) node[left] {$(-1,0, \ldots, 0)$};
\draw[thick] (2,0) -- (2,3);
\draw[dashed] (2,-2.5) -- (2,3.5);
\draw (2,1) node[right] {$\He^n$};
\draw[fill=black] (1.05,1.7321) arc(0:360:0.05);
\draw (1,1.7321) node[above] {$x$};
\draw (-2,0) -- (2,2.3094);
\draw[fill=black] (2.05,2.3094) arc(0:360:0.05);
\draw (2,2.3094) node[right] {$y$};
\draw (1,-0.1) -- (1,0.1);
\draw[dashed] (1,0) -- (1,1.7321);
\draw (1,0) node[below] {$x_1$};
\draw (-0.1,1.7321) -- (0.1, 1.7321);
\draw[dashed] (0,1.7321) -- (1,1.7321);
\draw (0,1.5) node[left] {$x_{n+1}$};

\end{tikzpicture}
$$

Die Formel für $\alpha$ ergibt sich direkt durch den Strahlensatz. Definiere nun
\begin{alignat*}{5}
\beta: \KK^n \ \ &\la&& \ \ \D^n = \left\{ (x_1, \ldots, x_{n+1}) \in \R^{n+1} \ \bigg\vert \ x_{n+1} = 0, \ \sum_{i=1}^n x_i^2 <1 \right\} \\
x= (x_1, \ldots, x_{n+1}) \ \ &\mapsto&& \ \  \beta(x) =: y = \left( \frac{x_1}{x_{n+1}+1}, \ldots, \frac{x_n}{x_{n+1}+1}, 0 \right) = (y_1, \ldots, _{n+1})
\end{alignat*}

Dann ist $\beta$ die Zentralprojektion von $(0,\ldots, 0,-1)$ aus:

$$
\begin{tikzpicture}

\draw[very thin,->] (-3,0) -- (4,0);
\draw (4,0) node[below] {$x_1$};
\draw[very thin,->] (0,-2.5) -- (0,3.5);
\draw (0,3.5) node[left] {$x_{n+1}$};
\draw[thick] (2,0) arc(0:180:2);
\draw[dashed] (-2,0) arc (180:360:2);
\draw (-1.3,1.7) node[above] {$\KK^n$};

\draw[thick] (-2,0) -- (2,0);
\draw (-1,0) node[below] {$\D^n$};
\draw (0,-2.2) node[right] {$(0,\ldots, 0,-1)$};
\draw[fill=black] (0.05,-2) arc (0:360:0.05);
\draw[fill=black] (1.05,0) arc(0:360:0.05);
\draw (0,-2) -- (1.6,1.2);
\draw (1.6,1.2) node[right] {$x$};
\draw[fill=black] (1.65,1.2) arc(0:360:0.05);
\draw (0.9,0) node[above] {$y$};



\end{tikzpicture}
$$

Auch hier ergibt sich die Formel unmittelbar. Definiere schließlich noch
$$\gamma: \Loid^n \la \KK^n, \qquad x=(x_1, \ldots, x_{n+1}) \mapsto  \gamma(l)=:y= \left( \frac{x_1}{x_{n+1}}, \ldots, \frac{x_n}{x_{n+1}}, \frac{1}{x_{n+1}} \right)=(y_1, \ldots, y_{n+1})$$
Auch $\gamma$ ist die Zentralprojektion von $(0,\ldots, 0 ,-1)$ aus. Wir zeigen nun (exemplarisch): Für eine differenzierbare Kurve $c: I \la \KK^n, \ t \mapsto (x_1(t), \ldots, x_{n+1}(t))$ gilt
$$L_{\He^n}(\alpha \circ c) = L_{\KK^n}(c),$$
das heißt $\alpha$ is längentreu. Daraus folgt mit den Definitionen der Metriken bereits $d_{\He^n}(\alpha(p), \alpha(q)) = d_{\KK^n}(p,q)$ für alle $p,q \in \KK^n$, also dass $\alpha$ eine Isometrie ist. Parametrisiere nun die Bildkurve $\alpha \circ c$ durch 
$$y_2(t) := \frac{2 x_2(t)}{x_1(t) +1},\ldots, y_{n+1}(t) := \frac{2 x_{n+1}(t)}{x_1(t)+1}.$$
Es gilt dann 
$$y_k'(t) = \frac{ \mathrm{d} y_k(t)}{\mathrm{d}t} = \frac{2 x_k'(t) (x_1(t) + 1) - 2 x_k(t) x_1'(t)}{(x_1(t)+1)^2} = \frac{2}{x_1(t)+1} \left( x_k'(t) - \frac{x_k(t)}{x_1(t)+1} x_1'(t) \right).$$
Wegen
$$\sum_{i=1}^{n+1} x_i(t)^2 = 1$$
folgt
$$\sum_{i=2}^{n+1} x_i(t)^2 = 1 - x_1(t)^2$$
und
$$x_1(t) x_1'(t) = - \sum_{i=2}^{n+1} x_i(t) x_i'(t).$$
Damit erhalten wir
$$L_{\He^n}(\alpha \circ c) = \int_I \frac{1}{y_{n+1}(t)} \sqrt{ \sum_{i=2}^{n+1} y_i'(t)^2} \ \mathrm{d}t, \qquad L_{\KK^n}(c) = \int_I \frac{1}{x_{n+1}(t)} \sqrt{ \sum_{i=1}^{n+1} x_i'(t)^2} \ \mathrm{d}t.$$
Für die Integranden rechnen wir mit $x_k:= x_k(t)$:
\begin{alignat*}{5}
\frac{1}{y_{n+1}^2} \left( \sum_{i=2}^{n+1} (y_i')^2 \right) \ \ &=&& \ \ \frac{(x_1+1)^2}{4 x_{n+1}^2}  \frac{4}{(x_1+1)^2} \left( \sum_{i=2}^{n+1} \left( x_i' - \frac{x_i}{x_1+1} x_1'\right)^2 \right) \\
&=&& \ \ \frac{1}{x_{n+1}^2} \left(  \sum_{i=2}^{n+1} (x_i')^2 - \frac{2x_1'}{x_1+1} x \sum_{i=2}^{n+1} x_i' x_i + \frac{(x_1')^2}{(x_1+1)^2} \sum_{i=2}^{n+1} x_i^2 \right) \\
&=&& \ \ \frac{1}{x_{n+1}^2} \left( \sum_{i=2}^{n+1} (x_i')^2 + \frac{2x_1'}{x_1+1} x_1 x_1' + \frac{(x_1')^2}{(x_1+1)^2} (1-x_1^2) \right) \\
&=&& \ \ \frac{1}{x_{n+1}^2} \left( \sum_{i=2}^{n+1} (x_i')^2 + (x_1')^2 \frac{2x_1}{x_1+1} + (x_1')^2 \frac{(1-x_1)(1+x_1)}{(x_1+1)^2} \right) \\
&=&& \ \ \frac{1}{x_{n+1}} \left( \sum_{i=2}^{n+1} (x_i')^2 + \frac{(x_1')^2}{x_1+1} \left( 2x_1+1-x_1\right) \right) \\
&=&& \ \ \frac{1}{x_{n+1}^2} \left( \sum_{i=1}^{n+1} (x_i')^2 \right),
\end{alignat*}

woraus die Behauptung folgt. Für $\beta$ und $\gamma$ verfahre analog. $\hfill \Box$



\end{pr}



\end{theorem}











\section{Isometrien und Geodätische des Halbraum-Modells} %PARAGRAPH IV.4


\hypertarget{satzvierviereins}{}
\begin{theorem}
Die folgenden Selbstabbildungen sind Isometrien von $(\He^n, d_{\He^n})$:
\begin{compactenum}
\item Euklidische Drehungen um die $x_n$-Achse.
\item Translationen parallel zur Hyperebene $x_n=0$.
\item Streckungen $x \mapsto \lambda x$ für $\lambda >0$. 
\end{compactenum}
\begin{pr}
\begin{compactenum}
\item Eine Rotation um die $x_n$-Achse ist gegeben durch eine Matrix
$$R= \begin{pmatrix} A & 0 \\[-6pt] 0 & 1 \end{pmatrix} \ \in \mathcal{O}(n), \qquad A \in \mathcal{O}(n-1).$$
Ist $c: I \la \He^n$ eine differenzierbare Kurve mit $c(t) = (x_1(t), \ldots, x_n(t))$, so ist 
\begin{alignat*}{5}
L_{\He^n}(R \circ c) \ \ &=&& \ \ \int_I \frac{1}{(R \circ c)_n(t)} \sqrt{ \sum_{i=1}^{n} (R \circ c)_i'(t)^2} \ \mathrm{d}t \\
&=&& \ \ \int_I \frac{1}{x_n(t)} \Vert (R \circ c)' \Vert_e \ \mathrm{d}t \\
&=&& \ \ \int_I \frac{1}{x_n(t)} \Vert R c'(t) \Vert_e \ \mathrm{d}t \\
&=&& \ \ \int_I \frac{1}{x_n(t)} \Vert c'(t)\Vert_e \ \mathrm{d}t\\
&=&& \ \ L_{\He^n}(c),
\end{alignat*}
wobei die Linearität von $R$ sowie $R \in \mathrm{Isom}(\E^n)$ benutzt wurde. 
\item Eine Translation in $(x_1, \ldots, x_{n-1})$-Richtung ist gegeben durch die Abbidung 
$$T_a: \He^n \la \He^m, \qquad (x_1, \ldots, x_n) \mapsto (x_1+ a_1, \ldots, x_{n-1} + a_{n-1}, x_n)$$
mit $(a_1, \ldots, a_{n-1}, 0) \in \R^n$. Dann gelten jedoch erneut 
$(T_a \circ c)'(t) = c'(t)$ sowie $\frac{1}{(T_a \circ c)(t)} = \frac{1}{x_n(t)}$, woraus $L_{\He^n}(c) = L_{\He^n}(T_a \circ c)$ folgt.
\item Ein Streckung ist durch eine Abbildung 
$$s_{\lambda}: \He^n \la \He^n, \qquad (x_1, \ldots, x_n) \mapsto (\lambda x_1, \ldots, \lambda x_n)$$
gegeben Dann gilt
$$\frac{1}{(s_{\lambda} \circ c)(t)} \sqrt{ \sum_{i=1}^n (s_{\lambda} \circ c)'_i(t)^2} \ = \frac{1}{\lambda x_n(t)} \sqrt{ \lambda^2 \sum_{i=1}^n x_i'(t)^2} = \frac{1}{x_n(t)} \sqrt{ \sum_{i=1}^n x_i'(t)^2}$$
und es folgt die Behauptung. $\hfill \Box$
\end{compactenum}
\end{pr}
\end{theorem}


\begin{remark}

Mit \hyperlink{satzvierviereins}{Satz 4.4.1} hat man viele Isometrien: $(\He^n, d_{\He^n})$ ist homogen bezüglich der Isometriegruppe: Für $p_0=e_n$ und beliebigem $p \in \He^n$ strecke $p_0$ um den Faktor $p_n$ und verschiebe ihn anschließend um $(p_1, \ldots, p_{n-1}, 0)$. Dann ist 
$$p=\left( T_{(p_1, \ldots, p_{n-1},0)} \circ s_{p_n}\right)(p_0).$$
\item Man kann zeigen: Die Isometrien aus \hyperlink{satzvierviereins}{Satz 4.4.1} sind nicht alle Isometrien von $(\He^n, d_{\He^n})$. 


\end{remark}



\hypertarget{lemmaviervierzwei}{}
\begin{lemma}   %% Lemma 4.4
Der Teilraum $U \subseteq \He^n$ gegeben durch die Gleichungen $x_i=0$ für $i \in \{1, \ldots, n-2\}$ mit der von $\He^n$ induzierten Riemannschen Metrik ist isometrisch zu $(\He^2, d_{\He^2})$.
\begin{pr}
Die Riemannsche Metrik von $\He^n$ ist 
$$\mathrm{d}s_{\He^n}^2 = \frac{1}{x_n^2} \left( \mathrm{d}x_1^2 + \ldots + \mathrm{d}x_n^2\right).$$
Die auf $U$ induzierte Metrik ist dann 
$$\mathrm{d}s_{\He^n}^2 \vert_U = \frac{1}{x_n^2} \left( \mathrm{d}x_{n-1}^2 + \mathrm{d}x_n^2 \right),$$
welche bis auf die Indizies mit der Metrik der hyperbolischen Ebene übereinstimmt. $\hfill \Box$
\end{pr}

\end{lemma}

\begin{lemma}
Die Projektion $$\mathrm{pr}: \He^n \la U \cong \He^2, \qquad (x_1, \ldots, x_n) \mapsto (0,\ldots, 0, x_{n-1}, x_n)$$ verkürzt die Länge von Kurven. Insbesondere sind Geodätische in $U$ bereits Geodätische in $\He^n$, das heißt $U$ ist \textit{total geodätisch}.
\begin{pr}
Sei $c: [0,1] \la \He^n$, $c(t) = (x_1(t), \ldots, x_n(t))$ eine differenzierbare Kurve. Dann ist 
$$(\mathrm{pr} \circ c)(t) = (0,\ldots, 0, x_{n-1}(t), x_n(t))$$
und damit
$$L_{\He^n}(c) \ = \ \int_a^b \frac{1}{x_n(t)} \sqrt{ \sum_{i=1}^n x_i'(t)^2} \ \mathrm{d}t \ \geqslant \ \int_a^b \frac{1}{x_n(t)} \sqrt{ x_{n-1}'(t)^2 + x_n'(t)^2} \ \mathrm{d}t \ = \ L_U(\mathrm{pr} \circ c),$$
woraus die Behauptung folgt. $\hfill \Box$

\end{pr}

\end{lemma}


\begin{theorem}
Die Geodätischen in $\He^n$ sind (parametrisierte) Halbgeraden senkrecht zur Hyperebene $x_n=0$ oder parametrisierte Halbkreise orthogonal zur Hyperebene $x_n=0$. Insbesondere gibt es durch zwei verschiedene Punkte genau eine Geodätische.
\begin{pr}
Seien $p,q \in \He^n$. Es seien $p', q'$ die Projektionen von $p,q$ auf die Hyperebene $H_n: x_n=0$. Betrachte nun die zu $H_n$ orthogonale Ebene $\sigma= \overline{pp'qq'}$. Gilt $p'=q'$, so wähle eine beliebige Gerade $g$ in $H_n$ durch $p'$, ansonsten setze $g= \overline{p'q'}$. Nach \hyperlink{satzvierdreieins}{Satz 4.3.1} existieren Translationen und Rotationen in $\mathrm{Isom}(\He^n)$, welche $\sigma$ auf $U$ abbilden. Da $U$ isometrisch zu $\He^2$ ist, ist auch $\sigma$ isometrisch zu $\He^2$. Weiter ist nach \hyperlink{lemmaviervierzwei}{Lemma 4.4.2} $\sigma$ total geodätisch. Damit ist die Geodätische durch $p$ und $q$ gegeben durch die Geodätische durch $p$ und $q$ in $U$, welche ein Halbkreis oder eine Halbgerade orthogonal zu $g$ ist. $\hfill \Box$

\end{pr}

\end{theorem}


\begin{remark}
\begin{compactenum}
\item Wie $\He^2$ kann man auch Kopien von $\He^k$ für $k\leqslant n$ in $\He^n$ konstruieren. Setze dazu $x_1= \ldots = x_{n-k} = 0$. 
\item Geodätische in anderen Modelle erhält man durch folgende Proposition: Die Zentralprojektion $\alpha, \beta,\gamma$ aus \hyperlink{satzvierdreieins}{Satz 4.3.1} sind kreis- und winkeltreu (vgl. Cannon). Damit zeigt man: Geodätische in $\D^n$ sind euklidische Kreissegmente in $\D^n$, welche orthogonal zu $\partial \D^n = \Sph^n$ sind. Geodätische im Hyperboloidmodell sind gerade die Schnitte von $\Loid^n$ mit zweidimensionalen Untervektorräumen des $\R^{n+1}$. 
\end{compactenum}

\end{remark}


















\section{Alle Isometrien des Hyperboloid-Modells} %PARAGRAPH IV.5




Seien $x,y \in \R^{n+1}$. Das \textit{Lorentz-Minkowski-Produkt} von $x$ und $y$ ist definiert als 
$$x \star y := \sum_{k=1}^n x_ky_k - x_{n+1}y_{n+1}.$$
Dies definiert eine symmetrische Bilinearform
$$B: \R^{n+1} \times \R^{n+1} \la \R, \qquad (x,y) \mapsto B(x,y)= x \star y$$
und die zu $B$ assoziierte quadratische Form 
$$Q: \R^{n+1} \la \R, \qquad x \mapsto Q(x)=B(x,x)= x \star x$$
auf $\R^{n+1}$.

\begin{remark}
Die Bilinearform $B$ ist nicht ausgeartet.
\begin{pr}
Es gelte $B(x,y)=0$ für alle $y \in \R^{n+1}$. Sein $\mathcal{E}=\{e_1, \ldots, e_{n+1}\}$ die Standardbasis des $\R^{n+1}$. Dann gilt für $x \in \R^{n+1}$ insbesondere
$$0= B(x,e_i)= x \star e_i = x_i, \qquad 0 = B(x,e_{n+1}) = x\star e_{n+1} = -x_{n+1},$$
also $x_i=0$ für alle $i \in \{1, \ldots, n+1\}$ und damit $x=0$. $\hfill \Box$
\end{pr}
\end{remark}

Das Hyperboloidmodell der $n$-dimensionalen hyperbolischen Geometrie ist nun gegeben durch 
$$\Loid^n = \{ x \in \R^{n+1} \ \vert \ Q(x) = x \star x = -1, \ x_{n+1}>0\}.$$

\begin{defin}
\begin{compactenum}
\item Eine \textit{lineare Isometrie} $f: \Loid^n \la \Loid^n$ ist die Einschränkung einer linearen Abbildung $F: \R^{n+1} \la \R^{n+1}$, die das Lorentz-Minkowski-Produkt erhält. Für $x,y \in \R^{n+1}$ gilt also $B(F(x), F(y)) = F(x) \star F(y) = x \star y = B(x,y)$ und insbesondere $Q(F(x)) = Q(x)$.
\item Eine \textit{Riemannsche Isometrie} $f: \Loid^n \la \Loid^n$ ist ein Diffeomorphismus, sodass für differenzierbare Kurven $c_1, c_2$ mit $c_1(0) = c_2(0) = p \in \Loid^n$ gilt 
$$\mathrm{d}s_{\Loid^n}^2(f(p)) \left( (f \circ c_1)'(0), (f \circ c_2)'(0) \right) = \mathrm{d}s_{\Loid^n}^2(p)(c_1'(0), c_2'(0)),$$
das heißt $f$ lässt die riemannsche Metrik invariant.
\item Eine \textit{topologische Isometrie} $f: \Loid^n \la \Loid^n$ ist ein Homöomorphismus, der die Abstandsfunktion $d_{\Loid^n}$ invariant lässt, also 
$$d_{\Loid^n}\left( f(x), f(y) \right) = d_{\Loid^n}(x,y)$$
für alle $x,y \in \Loid^n$. 

\end{compactenum}

\end{defin}


\hypertarget{satzvierfuenfdrei}{}
\begin{theorem}   %% Satz 4.7
Eine reelle $(n+1) \times (n+1)$-Matrix $M$ mit Spalten $m_1, \ldots, m_{n+1}$ definiert eine lineare Isometrie von $\Loid^n$ genau dann, wenn die folgenden beiden Bedingungen gelten:
\begin{compactenum}
\item[(1)] Für alle Indizes $i,j \in \{1, \ldots, n+1\}$ gilt $m_i \star m_j = e_i \star e_j$, wobei $\mathcal{E}:= \{e_1, \ldots, e_{n+1}\}$ die Standardbasis der $\R^{n+1}$ bezeichne.
\item[(2)] Der letzte Eintrag in der Spalte $m_{n+1}$ ist positiv.
\end{compactenum}
Dabei ist die erste Bedingung $(1)$ äquivalent zu 
\begin{compactenum}
\item[(1')] $M$ ist invertierbar mit inverser Matrix $M^{-1} = J M^{T} J$, wobei 
$$J = \begin{pmatrix}[rr] I_{n-1} & 0 \\ 0 & -1\end{pmatrix}$$
\end{compactenum}

\begin{pr}
Offensichtlich ist $J$ die Abbildungsmatrix des Lorentz-Minkowski-Pridukts bezüglich der Standardbasis, denn es gilt 
$$e_i \star e_j = J_{ij} = \begin{cases} \ 1, & \ \textrm{ für }i=j, \ i,j \in \{1, \ldots, n \} \\ \ -1, & \ \textrm{ für }i=n+1=j \\ \ 0, & \ \textrm{ für }i \neq j \end{cases} $$
Also gilt für $x,y \in \R^{n+1}$: $x \star y = x^{T} J y$. Somit ist $Mx \star My= x^{T} M^{T} J M y$ und $M$ ist invariant genau dann, wenn $x^{T}M^{T} JMy=x \star y$, also genau dann, wenn $M^{T} J M = J$. Der $(i,j)$-te EIntrag von $M^{T}JM$ ist $m_i \star m_j$ und jener von $J$ ist $e_i \star e_j$. Da $J$ invertierbar ist, folgt aus $(1)$, dass $M$ ebenfalls invertierbar ist, denn es gilt
$$\det M = \det m_{ij} = \det(m_i \star m_j) = \det (e_i \star e_j) = \det J = -1.$$
\end{pr}


\end{theorem}


\begin{remark}
Die linearen Abbildungen aus \hyperlink{vierfuenfdrei}{Satz 4.5.3} definieren eine Gruppe $\mathcal{O}^+(n,1) \subseteq \mathcal{O}(n,1)$, wobei
$$\mathcal{O}(n,1) = \{ M \in \mathbb{R}^{(n+1) \times (n+1)} \ \vert \ M^TJM = J\}$$
eine \textit{semi-orthogonale Gruppe} ist. Diese ist ein Spezialfall von 
$$\mathcal{O}(p,q) = \{ M \in \mathbb{R}^{(p+q+1) \times (p+q+1)} \ \vert \ M^T \tilde{J} M = \tilde{J} \}$$
mit 
$$\tilde{J} = \begin{pmatrix} I_p & 0 \\ 0 &-I_q \end{pmatrix}$$

\end{remark}


\begin{theorem}    %%Satz 4.8
Sei $f: \Loid^n \la \Loid^n$ eine Abbildung. Dann sind folgende Aussagen äquivalent:
\begin{compactenum}
\item $f$ ist eine lineare Isometrie.
\item $f$ ist eine Riemannsche Isometrie.
\item $f$ ist eine topologische Isometrie.
\end{compactenum}
Insbesondere ist also $\mathrm{Isom}(\He^n)=\mathcal{O}^+(n,1)$.
Weiter gilt für $p,q \in \Loid^n$
$$p \star q = - \cosh \left( d_{\Loid^n}(p,q)\right)$$


\begin{pr}
\begin{compactenum}
\item["$(i) \rightarrow (ii)$"] Sei zunächst $f$ eine lineare Isometrie, das heißt es gibt eine lineare Abbildung $F: \R^{n+1} \la \R^{n+1}$, welche $\star$ erhält, $\Loid$ in sich selbst abbildet und $F\vert_{\Loid^n} = f$ erfüllt. Die Riemannsche Metrik $\mathrm{d}s^{2}_{\Loid^n}$ ist in jedem Punkt $x \in \Loid^n$ ein Skalarprodukt im Tangentialraum $T_x\Loid^n$, gegeben durch 
$$\mathrm{d}s^2_{\Loid^n}(x): T_x\Loid^n \times T_x\Loid^n \la \R, \qquad \mathrm{d}s_{\Loid^n}^2(x)(c_1'(0), c_2'(0)) =  c_1'(0) \star c_2'(0)$$
Für müssen zeigen, das $f$ dieses Skalarprodukt invariant lässt. Es gilt für $x \in \Loid^n$ und $c_1'(0), c_2'(0) \in T_x\Loid^n$:
\begin{alignat*}{5}
\mathrm{d}s^2_{\Loid^n}\left( f(x)\right) \left( (f \circ c_1)'(0), (f \circ c_2)'(0)\right) \ \ &=&& \ \ \mathrm{d}s^2_{\Loid^n} \left( f(x)\right) \left( (\mathrm{d}f \circ c_1')(0), (\mathrm{d}f\circ c_2')(0) \right) \\
&=&& \ \ \mathrm{d}s^2_{\Loid^n}\left( F(x)\right)\left( (\mathrm{d}F \circ c_1')(0), (\mathrm{d}F\circ c_2')(0) \right) \\
&=&& \ \ \mathrm{d}s^2_{\Loid^n}\left( F(x)\right)\left( (F ( c_1'(0)), F(c_2'(0)) \right) \\
&=&& \ \ F(c_1'(0)) \star F(c_2'(0)) \\
&=&& \ \ c_1'(0) \star c_2'(0) \\
&=&& \ \ \mathrm{d}s_{\Loid^n}^2(x)(c_1'(0), c_2'(0)),
\end{alignat*}
also ist $f$ eine Riemannsche Isometrie.
\item["$(ii)\rightarrow (iii)$"] Sei nun $f$ eine Riemannsche Isometrie. Dann gilt für jede differenzierbare Kurve $c \in \Omega_{pq}(\Loid^n)$
\begin{alignat*}{5}
L_{\Loid^n}(f \circ c) \ \ &=&& \ \ \int_I \sqrt{\mathrm{d}s^2_{\Loid^n}\left( (f \circ c)(t)\right)\left( (f \circ c)'(t), (f \circ c)'(t) \right)} \ \mathrm{d}t \\
&=&& \ \ \int_I \sqrt{ \mathrm{d}s_{\Loid^n}^2 \left( c(t)\right) \left( c'(t), c'(t)\right) } \ \mathrm{d} t\\
&=&& \ \ L_{\Loid^n}(c),
\end{alignat*}
und für $p,q \in \Loid^n$ also auch 
$d_{\Loid^n}(f(p), f(q)) = d_{\Loid^n}(p,q)$.
\end{compactenum}

Für die fehlende Implikation zu zeigen, beweisen wir zunächst die Abstandsformel. Seien $p,q \in \Loid^n$ beliebig, $t= d_{\Loid^n}(p,q)$. Den Abstand erhält man durch Integration der Riemannschen Metrik entlang der eindeutigen Geodätischen zwischen $p$ und $q$. Diese Metrik ist nach obigem aber invariant unter linearen Isometrien, wir können also $p$ und $q$ zunächst durch eine lineare Isometrie in eine spezielle Position überführen und anschließend erst die Distanz berechnen. Sei dazu $m_1$ der Einheitstangentialvektor (bzgl. $\mathrm{d}s_{\Loid^n}^2(p)$) im Punkt $p$ an die Schnittkurve $[p,q,0] \cap \Loid^n$ (welche gerade die Geodätische zwischen $p$ und $q$ ist). Weiter sei $m_{n+1}=p$. Beachte: $m_1$ und $m_{n+1}$ sind orthogonal bzgl. $\star$. Nach dem Gram-Schmidt-Orthogonalisierungsverfahren können wir $m_1, m_{n+1}$ zu einer $\star$-ONB $\{m_1, m_2, \ldots, m_n, m_{n+1}\}$ von $\R^{n+1}$ erweitern, sodass also gilt $m_i \star m_j = e_i \star e_j$ für alle $i,j$. Nach Satz 4.7 definiert die Matrix $M=(m_1 \vert m_2 \vert \ldots \vert m_{n+1}) \in \mathcal{O}^+(n,1)$ eine lineare Isometrie. Die inverse Abbildung $M^{-1}$ ist ebenfalls eine lineare Isometrie von $\Loid^n$ und bildet $p$ auf $e_{n+1}$ und $[p,q]=[m_1, m_{n+1}] \subseteq \R^{n+1}$ auf den Unterraum $\sigma = [e_1, e_{n+1}]$ ab. Der Schnitt $\sigma \cap \Loid^n$ ist ein Hyperbel-Ast durch die Punkte $M^{-1}(p)=e_{n+1}$ und $M^{-1}(q)$. Da $t= d_{\Loid^n}(p,q) = d_{\Loid^n}(M^{-1}(p), M^{-1}(q))$ und die Geodätische von $e_1$ nach $e_{n+1}$ nach Bogenlänge parametrisiert ist, können wir annehmen, dass $M^{-1}(q) =( \sinh t, 0, \ldots, 0, \cosh t)$. Dann gilt 
$$p \star q = M^{-1}(p) \star M^{-1}(q) =  (0,\ldots, 0,1) \star (\sinh t, 0 ,\ldots, 0 , \cosh t) = - \cosh t = - \cosh \left( d_{\Loid^m}(p,q)\right),$$
was gerade zu zeigen war. Wir können nun die fehlende Implikation beweisen:

\begin{compactenum}
\item["$(iii) \rightarrow (i)$"] Sei $f: \Loid^n \la \Loid^n$ eine topologische Isometrie, es gelte also
$$d_{\Loid^n}(f(p), f(q)) = d_{\Loid^n}(p,q)$$
für alle $p,q \in \Loid^n$. Wähle eine Basis $\{v_1, \ldots, v_{n+1}\} \subseteq \Loid^n $ von $\R^{n+1}$. Sei $F: \R^{n+1} \la \R^{n+1}$ die lineare Abbildung gegeben durch $F(v_i) = f(v_i)$ für alle $i$. Dann erhält $F$ das Lorentz-Minkowski-Produkt $\star$, d.h. $F\vert_{\Loid^n}$ ist eine lineare Isometrie, denn: Schreibe die Standarsbasisvektoren $e_i$ als
$e_i = \sum_{j=1}^{n+1} a_{ij} v_j.$
Dann gilt 
\begin{alignat*}{5}
F(e_i) \star F(e_j) \ \ &=&& \ \ \sum_{k=1}^{n+1} \sum_{l=1}^{n+1} a_{ik}a_{jl} f(v_k) \star f(v_l) \\
&=&& \ \ \sum_{k=1}^{n+1} \sum_{l=1}^{n+1} a_{ik}a_{jl} \left(-\cosh \left( d_{\Loid^n}(f(v_k), f(v_l))\right) \right)\\
&=&& \ \ \sum_{k=1}^{n+1} \sum_{l=1}^{n+1} a_{ik}a_{jl} \left(-\cosh \left( d_{\Loid^n}(v_k, v_l)\right) \right)\\
&=&& \ \ \sum_{k=1}^{n+1} \sum_{l=1}^{n+1} a_{ik}a_{jl} v_k \star v_l \\
&=&& \ \ e_i \star e_j
\end{alignat*}
\end{compactenum}
Zeige noch, dass $F\vert_{\Loid^n} = f$. Betrachte dazu $\tilde{f}= F^{-1} \circ f$. Dann sind offensichtlich äquivalent:
\begin{compactenum}
\item[(1)] $F \vert_{\Loid^n} = f$.
\item[(2)] $\tilde{f}= \id_{\Loid^n}$.
\item[(3)] Für alle $x \in \Loid^n$ gilt $\tilde{f}(x) - x =0 $.
\item[(4)] Für alle $x \in \Loid^n$ und $i \in \{1, \ldots, n+1\}$ gilt $(\tilde{f}(x) - x)\star e_i=0$
\item[(5)] Für alle $x \in \Loid^n$ und $i \in \{1, \ldots, n+1\}$ gilt $\tilde{f}(x) \star e_i = x \star e_i$.
\end{compactenum}
Dies folgt daraus, dass $\star$ nicht ausgeartet ist. Seien also $x,i$ beliebig aber fest. Dann gilt 
\begin{alignat*}{5}
\tilde{f}(x) \star e_i \ \ &=&& \ \ \tilde{f}(x) \star \sum_{j=1}^{n+1} a_{ij} v_j \\
&=&& \ \ \sum_{j=1}^{n+1} a_{ij} \left( \tilde{f}(x) \star v_j \right)\\
&=&& \ \ \sum_{j=1}^{n+1} a_{ij} \left( \tilde{f}(x) \star \tilde{f}(v_j)\right) \\
&=&& \ \ \sum_{j=1}^{n+1} a_{ij} \left( - \cosh \left( d_{\Loid^n}(\tilde{f}(x), \tilde{f}(v_j))\right) \right) \\
&=&& \ \ \sum_{j=1}^{n+1} a_{ij} \left( - \cosh \left( d_{\Loid^n}(x,v_j)\right) \right) \\
&=&& \ \ \sum_{j=1}^{n+1} a_{ij} \left( x \star v_j\right) \\
&=&& \ \ x \star e_i,
\end{alignat*}
was zu zeigen war. $\hfill \Box$
\end{pr}

\end{theorem}



\begin{corollary}
\begin{compactenum}
\item $(\Loid^n, d_{\Loid^n})$ (und damit auch jedes andere Modell der $n$-dimensionalen hyperbolischen Geometrie) ist $2$-Punkt homogen. Insbesondere ist $\Loid^n$ homogen, d.h. die Isometriegruppe $\mathcal{O}^+(n,1)$ operiert transitiv auf $\Loid^n$. Algebraisch erhalten wir
$$\Loid^n \cong \slant{\mathcal{O}^+(n,1)}{\mathrm{Stab}_{\mathcal{O}^+(n,1)}(e_{n+1})} \cong \slant{\mathcal{O}^+(n,1)}{\mathcal{O}(n)}.$$
\item Geodätische in $\Loid^n$ sind (parametrisierte) Schnitte von $\Loid^n$ mit zweidimensionalen Untervektorräumen des $\R^{n+1}$. 
\end{compactenum}
\begin{pr}
\begin{compactenum}
\item Der Beweis von Satz 4.5.5 zeigt: Für $p,q \in \Loid^n$ mit $d_{\Loid^n}(p,q) = t$ existiert eine Isometrie $f_{pq}: \Loid^n \la \Loid^n$ gegeben durch eine Matrix $M \in \mathcal{O}^+(n,1)$, die das Paar $(p,q)$ auf das Standardpaar in der Ebene $[e_1, e_{n+1}]$ abbildet:
$$f_{pq}(p) = e_{n+1}, \qquad f_{pq}(q) = e_1 \sinh t + e_{n+1} \cosh t.$$
Insbesondere ist $p=f_{pq}^{-1}(e_{n+1})$, $\Loid^n$ ist also eine $\mathcal{O}^+(n,1)$-Bahn. Für den Stabilisator von $e_{n+1}$ gilt 
\begin{alignat*}{5}
\mathrm{Stab}_{\mathcal{O}^+(n,1)}(e_{n+1}) \ \ &=&& \ \ \{M \in \mathcal{O}^+(n,1) \ \vert \ M(e_{n+1}) = e_{n+1} \} \\
&=&& \ \ \left\{ \begin{pmatrix}[rrr|r] &&& 0 \\[-6pt] &M'&&\vdots \\[-6pt] &&& 0 \\[-3pt] \hline 0 & \ldots & 0 & 1 \end{pmatrix} \ \in \mathcal{O}^+(n,1) \ \bigg\vert \ M' \in \mathcal{O}(n) \right\} \\
&\cong&& \ \ \mathcal{O}(n)
\end{alignat*}
\item Für eine parametrisierte Hyperbel $\gamma_0: \R \la \Loid^n$, $\gamma_0(t) = (\sinh t,0, \ldots, 0, \cosh t)$ gilt

$$d_{\Loid^n}(\gamma_0(t_1), \gamma_0(t_2)) = \int_{t_1}^{t_2} \ \Vert \gamma_0'(t)\Vert_{\gamma_0(t)}^{\Loid^n} \ \mathrm{d}t = \int_{t_1}^{t_2} \sqrt{ \cosh^2t - \sinh^2t} \ \mathrm{d}t = t_2 - t_1,$$
das heißt $\gamma_0$ ist eine Geodätische und es gilt $\gamma_0(\R) = \Loid^n \cap [e_1, e_{n+1}]$. Sei nun $f_{pq}$ die Isometrie, welche $(p,q)$ nach $[e_1, e_{n+1}] \cap \Loid^n$ abbildet. Dann gilt für die Geodätische $\gamma$ durch $p$ und $q$
$$\gamma(\R) = f^{-1}(\gamma_0(\R)) = f^{-1}([e_1, e_{n+1}] \cap \Loid^n) = f^{-1}([e_1, e_{n+1}]) \cap \Loid^n,$$
wobei die Linearität von $f^{-1}$ und $f^{-1}(\Loid^n) = \Loid^n$ ausgenutzt wurde. Damit ist $\gamma(\R)$ ein zweidimensionaler Untervektorraum von $\R^{n+1}$, was den Beweis abschließt. $\hfill \Box$ 


\end{compactenum}
\end{pr}

\end{corollary}













\newpage
\thispagestyle{empty}




































































\chapter{Hyperbolische Geometrie im 20. Jahrhundert} %KAPITEL V
\setlength\abovedisplayshortskip{0pt}
\setlength\belowdisplayshortskip{10pt}
\setlength\abovedisplayskip{10pt}
\setlength\belowdisplayskip{10pt}
 



\section{$\delta$-hyperbolische Räume nach Gromov} %PARAGRAPH V.1

\thispagestyle{empty}



Sei $(X,d)$ ein metrischer Raum. Eine \textit{Geodätische} oder  ein \textit{geodätisches Segment} von $x\in X$ nach $y \in X$ ist eine Abbildung $c: [0,l] \la X$, $c(0)=x$, $c(l)=y$, sodass für alle $t_1\leqslant t_2 \in [0,l]$ gilt: $d(c(t_1), c(t_2)) = t_2-t_1$. Für das Segment schreiben wir $[x,y]$.  Ein metrischer Raum heißt \textit{geodätisch}, falls es zwischen je $2$ Punkten stets ein geodätisches Segment gibt.

\begin{ex}
Die Standardräume $\E^n, \Sph^n, \He^n$ sind alle geodätisch. Dabei sind $\E^n$ und $\He^n$ sogar eindeutig geodätisch, die Geodätischen zwischen je zwei Punkten sind also eindeutig. Zwischen antipodalen Punkten auf der Sphäre hingegen gibt es unendlich viele Geodätische. 
\end{ex}

Sei nun $\delta \geqslant 0$. Ein geodätischer, metrischer Raum $(X,d)$ heißt $\delta$-\textit{hyperbolisch}, falls alle geodätischen Dreiecke in $X$ $\delta$-dünn sind, d.h. für alle $x_1,x_2,x_3 \in X$ und ein Dreieck $\Delta$ beschreibende geodätischen Segmente $[x_1,x_2]$, $[x_2,x_3]$, $[x_1,x_3]$ gilt
$$[x_i,x_j] \subseteq U_{\delta}(\Delta \setminus [x_i,x_j])$$




Für eine Teilmenge $Y \subseteq X$ ist dabei 
$$U_{\delta}(Y) = \{ x \in X \ \vert \ d(x,y) \leqslant \delta \textrm{ für ein }y \in Y\}$$
die $\delta$-\textit{Umgebung} von $Y$. $(X,d)$ heißt \textit{Gromov-hyperbolisch}, falls $(X,d)$ $\delta$-hyperbolisch ist für ein $\delta \geqslant 0$.



\begin{ex}
\begin{compactenum}
\item Sei $(X,d)$ ein beschränkter metrischer Raum, d.h. es gilt $$\mathrm{diam}(X) = \max_{x,y \in X} d(x,y) < \infty.$$ Dann ist $X$ ist Gromov-hyperbolisch mit $\delta =  \mathrm{diam}(X)$, dem Durchmesser von $X$. Damit ist $\delta$-hyberbolisch eine Eigenschaft der "Grob-Geometrie" ("large scale").
\item Kombinatorische Räume (als eindimensionale Simplizialkomplexe) mit der induzierten Längenmetrik (d.h. jeder $1$-Simplex besitzt die Länge $1$): Erinnerung: Ein \textit{kombinatorischer Graph} $G$ besteht aus Ecken $E(G)$ ($0$-Simplices) und Kanten $K(G)$ ($1$-Simplices). Ein \textit{Baum} ist ein Graph ohne Kreise, wobei ein Kreis ein geschlossener Weg ist.
$$
\begin{tikzpicture}
\draw[fill=black] (2.05,1) arc(0:360:0.05);
\draw[fill=black] (1.05,-1) arc(0:360:0.05);
\draw[fill=black] (1.05,2) arc(0:360:0.05);
\draw[fill=black] (-0.95,1) arc(0:360:0.05);
\draw[fill=black] (-3.95,-1) arc(0:360:0.05);
\draw[fill=black] (-3.95,2) arc(0:360:0.05);
\draw (-4,2) -- (-1,1) -- (-4,-1);
\draw (2,1) -- (1,2) -- (-1,1) -- (1,-1);

\end{tikzpicture}
$$

Die Metrik auf einem Graphen ist für $x,y \in E(G)$ definiert wie folgt:
$$d_G(x,y) = \inf_{c\ \mathrm{Kantenzug}} L(c)$$
wobei in Kantenzug zwei Ecken verbindet und jede Kante isometrisch zu $[0,1]$ ist. Man kann als Verfeinerung auch den Abstand für Punkte auf den Kanten definieren.  Man sieht leicht, dass Bäume $0$-hyperbolisch sind.

\item Die euklidische Ebene $\E^2$ ist nicht Gromov-hyperbolisch, da Dreiecke beliebig groß werden können.

\item Ein Graph, der kein Baum ist, kann trotzdem $\delta$-hyperbolisch sein (für $\delta >0$):

$$
\begin{tikzpicture}
\draw[fill=black] (0.05,0.5) arc(0:360:0.05);
\draw[fill=black] (0.55,0) arc(0:360:0.05);
\draw[fill=black] (-0.45,0) arc(0:360:0.05);
\draw[fill=black] (0.05,-0.5) arc(0:360:0.05);
\draw (0,0.5) -- (-0.5,0) -- (0,-0.5) -- (0.5,0) -- (0,0.5);
\draw[fill=black] (-1.95,0) arc(0:360:0.05);
\draw[fill=black] (2.05,0) arc(0:360:0.05);
\draw[fill=black] (-2.45,-0.25) arc(0:360:0.05);
\draw[fill=black] (-2.45,0.25) arc(0:360:0.05);
\draw[fill=black] (2.55,0.25) arc(0:360:0.05);
\draw[fill=black] (2.55,-0.25) arc(0:360:0.05);
\draw[fill=black] (0.05,-1.5) arc(0:360:0.05);
\draw[fill=black] (-0.45,-2) arc(0:360:0.05);
\draw[fill=black] (0.55,-2) arc(0:360:0.05);
\draw (0.5,-2) -- (0,-1.5) -- (-0.5,-2);
\draw (0,-1.5) -- (0,-0.5);
\draw (2.5,0.25) -- (2,0) -- (2.5,-0.25);
\draw (-2.5,0.25) -- (-2,0) -- (-2.5,-0.25);
\draw (2,0) -- (0.5,0);
\draw (-2,0) -- (-0.5,0);


\end{tikzpicture}
$$

\item Es gibt auch Graphen, die nicht $\delta$-hyperbolisch sind:

$$
\begin{tikzpicture}
\draw (-4.5,3) -- (4.5,3);
\draw (-4.5,2) -- (4.5,2);
\draw (-4.5,1) -- (4.5,1);
\draw (-4.5,0) -- (4.5,0);
\draw (-4.5,-1) -- (4.5,-1);
\draw (-4.5,-2) -- (4.5,-2);
\draw (-4.5,-3) -- (4.5,-3);
\draw (-4,3.5) -- (-4,-3.5);
\draw (-3,3.5) -- (-3,-3.5);
\draw (-2,3.5) -- (-2,-3.5);
\draw (-1,3.5) -- (-1,-3.5);
\draw (-0,3.5) -- (0,-3.5);
\draw (1,3.5) -- (1,-3.5);
\draw (2,3.5) -- (2,-3.5);
\draw (3,3.5) -- (3,-3.5);
\draw (4,3.5) -- (4,-3.5);
\draw[thick, color=red] (-3,-2) -- (1,-2) -- (1,-1) -- (0,-1) -- (0,1) -- (-1,1) -- (-1,2) -- (-2,2) -- (-2,3) -- (-3,3) -- (-3,-2);
\draw[fill=black] (-2.95,-2) arc(0:360:0.05);
\draw[fill=black] (1.05,-2) arc(0:360:0.05);
\draw[fill=black] (-2.95,3) arc(0:360:0.05);


\end{tikzpicture}
$$

\item $\E^n$ ist nicht Gromov-hyperbolisch für $n \geqslant 2$. $\E^1$ hingegen ist $0$-hyperbolisch.
\item Die klassischen hyperbolischen Räume $(\He^n, d_{\He^n})$, $(\Loid^n, d_{\Loid^n})$ und $(\D^n, d_{\D^n})$ sind $\delta$-hyperbolisch für $\delta = \ln ( 1+ \sqrt{2})$ (bereits gezeigt für $n=2$, $n\geqslant 3$ als Übung (Aufgabe 13.1)).




\end{compactenum}

\end{ex}

Wir wissen nun, dass kombinatorische Bäume $0$-hyperbolisch sind. Gilt auch die Umkehrung? Nicht ganz! Die Ebene $\R^2$ ausgestattet mit der SNCF-Metrik
$$d_{\mathrm{SNCF}}(x,y) = \begin{cases} \ \Vert x-y\Vert_e, & \ \textrm{ falls }x=\lambda y \textrm{ für ein } \lambda \in \R \\ \ \Vert x \Vert_e + \Vert y \Vert_e, &\  \textrm{ andernfalls} \end{cases}$$
ist $0$-hyperbolisch, aber kein Baum. 




\begin{defin}
\begin{compactenum}
\item Sei $(X,d)$ ein geodätischer, metrischer Raum.
Ein \textit{topologisches Segment} zwischen $x,y \in X$ ist ein Homöomorphismus $\gamma:[0,1]=I \la \gamma(I) \subseteq X$ mit $\gamma(0) = x$ und $\gamma(1) = y$. 
\item Ein \textit{reeller Baum} (oder $\mathbb{R}$-\textit{Baum}) ist ein geodätischer metrischer Raum $(X,d)$ mit der Eigenschaft, dass es für je zwei Punkte $x,y \in X$ genau ein topologisches Segment zwischen $x$ und $y$ gibt.
\end{compactenum}
\end{defin}


\begin{theorem}   %%Satz 5.1
Ein geodätischer metrischer Raum ist $0$-hyperbolisch genau dann, wenn er ein $\R$-Baum ist.
\begin{pr}
\begin{compactenum}
\item["$\Leftarrow$"] Sei $(X,d)$ ein $\R$-Baum, $x,y,z \in X$ und $[x,y] \cup [y,z] \cup [x,z]$ das durch die drei Punkte eindeutig festgelegte geodätische Dreieck. Wir zeigen exemplarisch $[x,y] \subseteq [y,z] \cup [x,z]$. Angenommen es gelte $[x,y] \nsubseteq [y,z] \cup [x,z]$. Wähle dann einen Teilweg $[x',y'] \subseteq [x,y]$ sodass $[x',y'] \cap \left( [x,z] \cup [y,z] \right) = \{x',y'\}$. Dann ist der der Weg $\overline{x'zy'}$ jedoch ein von $[x',y']$ verschiedenes topologisches Segment zwischen $x'$ und $y'$, ein Widerspruch zur Eindeutigkeit.
\item["$\Rightarrow$"] Sei nun $(X,d)$ $0$-hyperbolisch. Da in einem $0$-hyperbolischen Raum geodätische Segmente eindeutig sind, ist lediglich zu zeigen, dasss jedes topologische Segment auch ein geodätisches Segment ist. Seien $x,y \in X$ und $\gamma: I \la \gamma(I)$ ein topologisches Segment, das $x$ und $y$ verbindet. Weiter sei $[x,y]$ das eindeutige geodätische Segment zwischen $x$ und $y$.
\begin{compactenum}
\item[\textbf{Beh. (a)}] Für jedes $\epsilon >0$ gilt $[x,y] \subseteq U_{\epsilon}( \mathrm{Bild}\ \gamma)$.
\end{compactenum}
Mit der Behauptung folgt: $\mathrm{Bild} \ \gamma = [x,y]$. Denn als homöomorphes Bild eines Intervalls ist $\mathrm{Bild}\ (\gamma)$ vollständig. Für alle $z \in [x,y]$ existiert also $t_n \in I$ mit $d(z, \gamma(t_n)) \leqslant \frac{1}{n}$. Die Dreiecksungleichung liefert dann für $n \leqslant m$
$$d(\gamma(t_n), \gamma(t_m)) \leqslant \frac{2}{n},$$
$\left( \gamma(t_n)\right)_{n \in \mathbb{N}}$ ist also eine Cauchyfolge und konvergiert, da $\mathrm{Bild} \ \gamma$ vollständig ist. Damit gilt $\gamma(t_n) \overset{ n \to \infty}{\la} \gamma(t^*)$ für ein $t_* \in I$. Es folgt
$$d(z, \gamma(t^*)) = \lim_{n \to \infty} d (z, \gamma(t_n)) \leqslant \lim_{n \to \infty} \frac{1}{n} =0,$$
also $z = \gamma(t^*)$. Insgesamt gilt also $[x,y] \subseteq \mathrm{Bild}\ \gamma$ und schließlich $\mathrm{Bild} \ \gamma = [x,y]$, was zu zeigen war. Es bleibt noch die Behauptung zu beweisen.
\begin{compactenum}
\item[\textbf{Bew. (a)}] Sei $\epsilon >0$. Da $\mathrm{Bild} \ \gamma$ als stetiges Bild eines Kompaktums ebenfalls kompakt ist, existieren $N \in \mathbb{N}$ und Punkte $x=x_0, x_1, \ldots, x_{N-1}, x_N=y$ auf $\mathrm{Bild} \ \gamma$, sodass $d(x_{i-1}, x_i) \leqslant 2 \epsilon$ für alle $i \in \{1, \ldots, N\}$. Das eindeutige geodätische Segment $[x_{i-1}, x_i]$ liegt dann in einer $\epsilon$-Umgebung von $\{x_{i-1}, x_i\}$, also auch in einer $\epsilon$-Umgebung von $\mathrm{Bild} \ \gamma$. Da $X$ nach Voraussetzung $0$-hyperbolisch ist, folgt (mit Induktion)
$$[x,y] = [x_0,x_N] \subseteq \bigcup_{i=1}^N [x_{i-1},x_i] \subseteq U_{\epsilon} \left( \mathrm{Bild} \ \gamma \right),$$
woraus die Behauptung folgt. $\hfill \Box$
\end{compactenum}


\end{compactenum}
\end{pr}

\end{theorem}







 



\section{Quasi-Isometrien und Quasi-Geodätische} %PARAGRAPH V.2


\begin{defin}

Sei $f:(X,d_X) \la (Y,d_Y)$ eine Abbildung zwischen metrischen Räumen.
\begin{compactenum}
\item $f$ heißt \textit{topologische Isometrie}, falls 
$$d_X(x,y) = d_Y(f(x), f(y))$$
für alle $x,y \in X$.
\item $f$ heißt $\lambda$-\textit{Bilipschitz-Abbildung}, falls eine Konstante $\lambda \geqslant1$ existiert, sodass
$$\frac{1}{\lambda} d_X(x,y) \leqslant d_Y(f(x), f(y)) \leqslant \lambda d_X(x,y)$$
für alle $x,y \in X$.  Insbesondere sind $1$-Bilipschitz-Abbildungen gerade topologische Isometrien.
\item $f$ heißt $(\lambda, \epsilon)$\textit{-quasi-isometrische Einbettung}, falls Konstanten $\lambda \geqslant1, \epsilon \geqslant 0$ existieren, sodass
$$\frac{1}{\lambda} d_X(x,y) - \epsilon \leqslant d_Y(f(x), f(y)) \leqslant \lambda d_X(x,y) + \epsilon$$
für alle $x,y \in X$. Insbesondere sind $(\lambda,0)$-quasi-isometrische Einbettungen gerade $\lambda$-Bilipschitz-Abbildungen.
\item $f$ heißt $(\lambda, \epsilon)$-\textit{Quasi-Isometrie}, falls $f$ eine $(\lambda,\epsilon)$-quasi-isometrische Einbettung ist und zusätzlich eine Konstante $D \geqslant 0$ existiert, sodass jeder Punkt von $Y$ in einer $D$-Umgebung von $f(X) \subseteq Y$ liegt, das Bild also "$D$-dicht" ist. Insbesondere ist $f$ für $D=0$ surjektiv.

\item Zwei metrische Räume $(X,d_X)$ und $(Y,d_Y)$ heißen \textit{quasi-isometrisch}, falls eine Quasi-Isometrie $f: (X,d_X) \la (Y,d_Y)$ existiert.

\end{compactenum}

\end{defin}

Im Folgenden werden wir quasi-isometrische Invarianten bestimmen, um das Klassifikationsproblem lösen zu können.


\begin{ex}
\begin{compactenum}
\item Ein metrischer Raum $(X,d)$ ist quasi-isometrisch zu einem Punkt genau dann, wenn $\mathrm{diam} X < \infty$.
\item $\Z$ mit der von $\R$ induzierten Metrik ist quasi-isometrisch zu $\R$ (mit $f: \Z \hookrightarrow \R$, $\lambda =1, \epsilon=0$ und $D=1$). Genau so sind $\Z^n$ und $\R^n$ quasi-isometrisch.
\item die stetigen Funktionen $\mathcal{C}(\R,d_{\R})$ sind quasi-isometrisch zu $\R$.
\item Aus der geometrischen Gruppentheorie: Jede endlich erzeugte Gruppe $G= \langle S\rangle$ definiert auf natürliche Weise eine Metrik auf $G$, wodurch $G$ zum metrischen Raum $(G,d_S)$ wird (beachte, dass die Metrik vom Erzeugendensystem abhängt). Ist $S'$ ein weiteres Erzeugendensystem, so sind die resultierenden metrischen Räume $(G,d_S)$ und $(G,d_{S'})$ quasi-isometrisch.

\end{compactenum}
\end{ex}



Was passiert nun mit Geodätischen unter Quasi-Isometrien? Erinnerung: Ist $(X,d)$ ein metrischer Raum, so ist eine Geodätische eine isometische Einbettung $\gamma: I \la X$ für ein Intervall $I \subseteq \R$. Analog ist eine $(\lambda, \epsilon)$-\textit{Quasi-Geodätische} in einem metrischen Raum $(X,d)$ eine $(\lambda, \epsilon)$-quasi-isometrische Einbettung $c: I \la X$, wobei $I\subseteq \R$ ein Intervall in $\R$ oder in $\R \cap \Z$ ist, das heißt es gilt 
$$\frac{1}{\lambda} \vert t- t'\vert - \epsilon \leqslant d(c(t), c(t')) \leqslant \lambda \vert t-t'\vert + \epsilon$$
für alle $t,t' \in I$. Falls $I= [0,\infty) = \R_+^0$, so heißt $c$ \textit{geodätischer Strahl}. Beachte: Für $\epsilon\neq0$ ist eine $(\lambda, \epsilon)$-Quasi-Geodätische nicht zwingend stetig! Wir werden sehen: In hyperbolischen Raumen sind Quasi-Geodätische jedoch "nahe" bei Geodätischen (man nennt dies \textit{Trapping} oder \textit{Stabilität} von Geodätischen). Dies wird zeigen, dass Hyperbolizität eine quasi-isometrische Invariante ist.


\begin{remark}
Beachte: Trapping gilt in der eukldischen Ebene $\E^2$ nicht: Die Spirale
$$c:[0,\infty) \la \E^2, \qquad t \mapsto \begin{pmatrix} t \cos \left( \log 1+t \right) \\ t \sin \left( \log 1+t\right) \end{pmatrix}$$
ist ein quasi-geodätischer Strahl, kann aber nicht durch einen geodätischen Strahl "eingefangen" werden.

\end{remark}



Sei nun $(X,d)$ ein metrischer Raum und $c:[a,b] \la X$ eine stetige Kurve. Wir definieren die \textit{Länge von} $c$ durch 
$$l(c) := \sup_{a=t_0 \leqslant \ldots \leqslant t_n=b} \ \sum_{i=0}^{n-1} d(c(t_i), c(t_{i+1}).$$
$c$ heißt rektifizierbar, falls $l(c) < \infty$.

\begin{remark}[Parametrisierung nach Bogenlänge]
Sei $c:[a,b] \la X$ rektifizierbar mit $l:=l(c)$. Weiter sei 
$$\lambda:[a,b] \la [0,l], \qquad t \mapsto \lambda(t) = l\left( c\vert_{[a,t]}\right).$$
Dann ist $\tilde{c}:[0,l] \la X$ mit $\tilde{c} \circ \lambda = c$ rektifizierbar und es gilt $l\left( \tilde{c}\vert_{[a,t]}\right) = t$.
\begin{pr}
Siehe Bridson-Haifliger, S.13. $\hfill \Box$
\end{pr}
\end{remark}

\begin{theorem}
Sei $(X,d)$ ein $\delta$-hyperbolischer, geodätischer metrischer Raum und $c:[a,b]\la X$ ein stetiger, rektifizierbarer Weg in $X$ mit Anfangspunkt $p=c(a)$ und Endpunkt $q=c(b)$. Sei Weiter $[p,q]$ ein geodätischer Segment, das $p$ und $q$ verbindet. Dann gilt für alle $x \in [p,q]$:
$$d \left( x, \mathrm{Bild} \ c\right) \leqslant \delta \vert \log_2 l(c)\vert +1.$$

\begin{pr}
Ist $l(c) \leqslant 1$, so gilt 
$$d(x, \mathrm{Bild} \ c) \leqslant d(x,p) \leqslant d(p,q) \leqslant l(c) \leqslant 1 \leqslant \delta \vert \log_2 l(c) \vert +1,$$
die Behauptung ist also klar. Sei also $l(c) >1$. Nach Bemerkung 2.4 können wir ohne Einschränkung annehmen, dass $c$ eine Abbildung $c:[0,1] \la X$ ist, die $\mathrm{Bild} \ c$ proportional zur Bogenlänge parametrisiert, das heißt es gilt 
$$d(c(t), c(t')) = l(c) \vert t-t'\vert$$
für $t,t' \in [0,1]$. Wähle $N \in \mathbb{N}$ so groß, dass 
$$\frac{l(c)}{2^{N+1}} < 1 \leqslant \frac{l(c)}{2^N}.$$
Betrachte nun das Dreieck
$$\Delta_1:= \Delta \left( \left[c(0), c\left(\frac{1}{2}\right)\right], \left[ c\left(\frac{1}{2}\right), c(1)\right], \left[ c(0), c(1) \right] \right)$$

$$
\begin{tikzpicture} [scale=1.5]
\draw (-3,0) node[below] {$p=c(0)$};
\draw (3,0) node[below] {$q=c(1)$};
\draw (3,0) arc(0:180:3);
\draw (1,3) node[right] {$\mathrm{Bild} \ c$};
\draw (0,3.05) node[above] {$c\left(\frac{1}{2}\right)$};
\draw[fill=black] (-2.95,0) arc(0:360:0.05);
\draw[fill=black] (3.05,0) arc(0:360:0.05);
\draw[fill=black] (0.05,3) arc(0:360:0.05);
\draw[fill=green, opacity=0.2] (-3,0) to[bend right=20] (0,3) to[bend right=20] (3,0) to[bend right=20] (-3,0);
\draw (-3,0) to[bend right=20] (0,3) to[bend right=20] (3,0) to[bend right=20] (-3,0);
\draw (0,1.5) node {$\Delta_1$};
\draw[fill=black] (-1,0.53) arc(0:360:0.05);
\draw (-0.95, 0.5) node[below] {$x$};
\draw[fill=black] (-1.3,1.06) arc(0:360:0.05);
\draw (-1.35,1.1) node[above] {$y_1$};
\draw (-1.05,0.53) arc(5:50:0.8);
\draw[fill=black] (-2.0713, 2.1213) arc(0:360:0.05);
\draw (-2.4213,2.1213) node[above] {$c\left(\frac{1}{4}\right)$};
\draw (-3,0) to[bend right=20] (0,3) to[bend left=20] (-2.1213,2.1213) to[bend left=20] (-3,0);
\draw[fill=red, opacity=0.2] (-3,0) to[bend right=20] (0,3) to[bend left=20] (-2.1213,2.1213) to[bend left=20] (-3,0);
\draw (-1.2,1.8) node {$\Delta_2$};
\draw[fill=black] (2.1713,2.1213) arc(0:360:0.05);
\draw (2.1213,2.1213) node[right] {$c\left(\frac{3}{4}\right)$};
\draw[fill=black] (-2.14,1.53) arc(0:360:0.05);
\draw (-1.35,1.06) arc(40:82:1.4);
\draw (-2.19,1.7) node[right] {$y_2$};

\draw[fill=black] (-2.7216,1.1481) arc(0:360:0.05);
\draw (-2.7716,1.1481) node[left] {$c\left(\frac{1}{8}\right)$};
\draw (-3,0) to[bend right =20] (-2.1213,2.1213) to[bend left=20] (-2.7716,1.1481) to[bend left=20] (-3,0);
\draw[fill=blue, opacity=0.1] (-3,0) to[bend right =20] (-2.1213,2.1213) to[bend left=20] (-2.7716,1.1481) to[bend left=20] (-3,0);


\end{tikzpicture}
$$
Dann ist $\Delta_1$ ein geodätisches Dreieck, dass die Seite $[p,q]$ enthält. Sei nun $x \in [p,q]=[c(0), c(1)]$. Da $X$ $\delta$-hyperbolisch ist, können wir $y_1 \in \left[c(0), c\left(\frac{1}{2}\right)\right] \cup \left[ c\left(\frac{1}{2}\right)\right]$ derart whälen, dass $d(x,y_1) \leqslant \delta$. Falls $y_1 \in \left[c(0), c\left(\frac{1}{2}\right)\right]$, betrachte ein weiteres Dreieck 
$$\Delta_2 := \Delta \left( \left[c(0), c\left(\frac{1}{2}\right)\right], \left[ c(0), c\left(\frac{1}{4}\right)\right], \left[ c\left(\frac{1}{4}\right), c\left(\frac{1}{2}\right)\right] \right).$$
Dann ist $\Delta_2$ ebenfalls ein geodätisches Dreieck und besitzt $\left[ c(0), c\left(\frac{1}{2}\right) \right]$ als gemeinsame Seite mit $\Delta_1$. Ist hingegen $y_1$ auf der anderen Seite, also $y_1 \in \left[ c\left(\frac{1}{2}\right)\right]$, so definiere $\Delta_2$ wie folgt:
$$\Delta_2 := \Delta \left( \left[c\left(\frac{1}{2}\right), c\left(1\right)\right], \left[c\left(\frac{1}{2}\right), c\left(\frac{3}{4}\right)\right], \left[ c\left(\frac{3}{4}\right), c\left(1\right)\right] \right).$$
In jedem Fall existiert wieder wegen der Hyperbolizität von $X$ ein $y_2 \in \Delta_2 \setminus \Delta_1$ mit $d(y_1, y_2) \leqslant \delta$.\\
Diese Konstruktion setzen wir iterativ fort: Im $n+1$-ten Schritt berachte das geodätische Dreieck $\Delta_{n+1}$, das die Seite $[c(t_n), c(t_n')]$ mit $\Delta_n$ gemeinsam hat und $y_n$ enthält. Weiter sei $c(t_{n+1})$ die dritte Ecke des Dreiecks mit $t_{n+1} =\frac{t_n+t_n'}{2}$. Wann wählen wir $y_{n+1} \in \Delta_{n+1} \setminus [c(t_n), c(t_n')]$ mit $d(y_n, y_{n+1}) \leqslant \delta$. \\
Im $N$-ten Schritt erhalten wir somit einen Punkt $y_N$ mit
$$d(x,y_N) \leqslant d(x,y_1) + d(y_1, y_2) + \ldots + d(y_{N-2}, y_{N-1}) + d(y_{N-1}, y_N) \leqslant N \delta,$$
der auf einem geodätischen Segment der Länge $\frac{l(c)}{2^N}$ (beachte, dass $c$ nach Bogenlänge parametrisiert ist, also sukzessives Halbieren der Intervallenlänge auch eine Halbierung der Bildkurve herbeiführt) mit Endpunkten in $\mathrm{Bild} \ c$ liegt. Sei $y$ derjenige Endpunkt, der am nächsten bei $y_N$ liegt. Dann gilt 
$$d(x,y) \leqslant d(x,y_N) + d(y_N,y) \leqslant N \delta +  \frac{1}{2}\frac{l(c)}{2^{N}} = N \delta + \frac{l(c)}{2^{N+1}}.$$
Nach der Wahl von $N$ ist $\frac{l(c)}{2^{N+1}} <1$ und $2^N \leqslant l(c)$, also $N \leqslant \log_2 l(c)$ und damit 
$$d(x, \mathrm{Bild} \ c) \leqslant d(x,y) \leqslant N \delta + \frac{l(c)}{2^{N+1}} \leqslant \delta \vert \log_2 l(c)\vert +1,$$
was gerade zu zeigen war. $\hfill \Box$








\end{pr}

\end{theorem}



\begin{remark}
Eine vergleichbare Aussage in $\E^2$ gibt es nicht.
\end{remark}


\begin{defin}
Sei $(X,d)$ ein metrischer Raum und $A,B \subseteq X$ Teilmengen von $X$. Dann ist 
$$\mathrm{Hd}(A,B) := \inf \{ \epsilon \geqslant 0  \ \vert \ A \subseteq U_{\epsilon}(B), \ B \subseteq U_{\epsilon}(A) \}$$
der \textit{Hausdorff-Abstand} von $A$ und $B$.
\end{defin}


\begin{theorem}[Stabilität von Quasi-Geodätischen]    %%%Satz 5.3
Sei $(X,d)$ ein $\delta$-hyperbolischer, geodätischer, metrischer Raum, $c:[a,b] \la X$ eine $(\lambda, \epsilon)$-Quasi-Geodätische mit Anfangs- und Endpunkten $c(a)=p$, $c(b)=q$ und $[p,q]$ ein geodätisches Segment, dass $p$ und $q$ verbindet. Dann existiert eine Konstante $R=R(\delta, \lambda, \epsilon)$, sodass
$$\mathrm{Hd}([p,q], \mathrm{Bild} \ c) \leqslant R.$$
\end{theorem}




Bevor wir die Aussage beweisen, leiten wir die wichtigste Folgerung her. Ein $(\lambda, \epsilon)$-quasi-geodätisches Dreieck besteht aus drei $(\lambda, \epsilon)$-Quasi-Geodätischen $q_i: [0,T_i] \la X$ ($i \in \{1,2,3\}$) mit 
$$q_1(T_1) = q_2(0), \qquad q_2(T_2) = q_3(0), \qquad q_3(T_3) = q_1(0).$$
Das Dreieck heißt $K$-\textit{dünn} für eine Konstante $K \geqslant 0$, falls für jedes $i \in \{1,2,3\}$ gilt: Jeder Punkt $x \in \mathrm{Bild} \ q_i$ ist in einer $K$-Umgebung von $ \Delta \setminus\mathrm{Bild} \ q_{i}$ enthalten.



\begin{corollary}   %% Korollar 5.4
Ein geodätischer, metrischer Raum $(X,d)$ ist Gromov-hyperbolisch genau dann, wenn für jedes $\lambda \geqslant 1$ und jedes $\epsilon \geqslant 0$ eine Konstante $M \geqslant 0$ existiert, sodass jedes $(\lambda, \epsilon)$-quasi-geodätische Dreieck $M$-dünn ist.


\end{corollary}



\begin{theorem}     %%Satz 5.5
Seien $(X,d_X)$, $(Y,d_Y)$ geodätische, metrische Räume und $f: Y\la X$ eine $(\lambda, \epsilon)$-quasi-isometrische Einbettung. Dann gilt: Ist $X$ $\delta$-hyperbolisch, so ist $Y$  $\delta'$-hyperbolisch für ein $\delta \geqslant 0$.
\begin{pr}
Sei $\Delta \subseteq Y$ ein geodätisches Drieieck in $Y$. Wir müssen zeigen, dass $\Delta$ mit einer uniformen Konstante dünn ist. Es bezeichnen $c_1([0,1]), c_2([0,1]), c_3([0,1])$ die Seiten von $\Delta$. Dann ist $f(\Delta)$ ein $(\lambda, \epsilon)$-quasi-isometrisches Dreieck mit Seiten $(f\circ c_1)([0,1])$, $(f\circ c_2)([0,1])$ und $(f \circ c_3)([0,1])$. Nach Korollar 2.9 ist $f(\Delta)$ $M$-dünn für eine Konstante $M:= M(\lambda, \epsilon) \geqslant 0$, das heißt es gibt (beispielsweise) für $x \in \mathrm{Bild} \ c_1$ ein $y \in \mathrm{Bild} \ c_2 \cup \mathrm{Bild} \ c_3$, sodass $d_X(f(x), f(y)) \leqslant M$. Da $f$ eine $(\lambda, \epsilon)$-quasi-isometrische Einbettung ist, gilt weiter
$$d_Y(x,y) \leqslant \lambda d_X(f(x), f(y)) + \epsilon \leqslant \lambda M + \epsilon  =: \delta'$$
und analog für die anderen Seiten. Damit $\Delta$ $\delta'$-dünn, was zu zeigen war. $\hfill \Box$



\end{pr}

\end{theorem}


\begin{lemma}   %% Lemma 5.6
Sei $(X,d)$ ein geodätischer, metrischer Raum und $c:[a,b] \la X$ eine $(\lambda, \epsilon)$-Quasi-Geodätische. Dann existiert eine stetige $(\lambda, \epsilon')$-Quasi-Geodätische $c':[a,b] \la X$ mit
\begin{compactenum}
\item $c$ und $c'$ haben dieselben Anfangs- und Endpunkte, das heißt es gilt $c(a)=c'(a)$ sowie $c(b)=c'(b)$.
\item $\epsilon'=2(\lambda+\epsilon)$.
\item Für alle $t,t' \in [a,b]$ gilt 
$$l\left( c'\vert_{[t,t']}\right) \leqslant k_1  d\left( c'(t), c'(t')\right) + k_2,$$
wobei $k_1=\lambda(\lambda + \epsilon)$ und $k_2= (\lambda \epsilon' + 3)(\lambda + \epsilon)$.
\item Für den Hausdorff-Abstand der beiden Quasi-Geodätischen gilt 
$$\mathrm{Hd}\left( \mathrm{Bild} \ c, \mathrm{Bild} \ c'\right) \leqslant \lambda + \epsilon.$$
\end{compactenum}


\begin{pr}
Es sei $\Sigma := \Z \cap (a,b) \cup \{a,b\} =\{a=s_0, s_1, \ldots, s_{n-1}, s_n=b\}$. Für $s \in \Sigma$ setze $c'(s):= c(s)$. zwischen aufeinanderfolgenden Punkten $c(s_i)$ und $c(s_{i+1})$ wähle ein linear, proportional zur Bogenlänge parametrisiertes geodätisches Segment. Definiere anschließend $c'$ als die Verkettung dieser Segmente. Zu den einzelnen Eigenschaften:
\begin{compactenum}
\item Nach Definition von $c'$ gilt $c'(a)= c(a)$ sowie $c'(b) = c(b)$. 
\item[(iv)] Für die Länge jedes Teilsegments gilt
$$l\left( c'\vert_{[s_i, s_{i+1}]}\right) = d \left( c'(s_i), c'(s_{i+1})\right) = d\left( c(s_i), c(s_{i+1})\right) \leqslant \lambda d(s_i, s_{i+1}) + \epsilon \leqslant \lambda + \epsilon.$$
Da jeder Punkt von $\mathrm{Bild} \ c \cup \mathrm{Bild} \ c'$ in einer $\frac{\lambda + \epsilon}{2}$-Umgebung von $c(\Sigma) = c'(\Sigma)$ liegt, folgt bereits die Behauptung.
\item[(ii)] Für beliebiges $t \in [a,b]$ sei $[t]$ derjenige Punkt von $\Sigma$, welcher am nächsten zu $t$ ist. Da $c$ eine $(\lambda, \epsilon)$-Quasi-Geodätische ist und $c([t])= c'([t])$ gilt, folgt
\begin{alignat*}{5}
d\left( c'(t), c'(t')\right) \ \ & \leqslant && \ \ d\left( c'(t), c'([t]) \right) + d\left( c'([t]), c'([t']) \right) + d\left( c'([t'], c'(t')\right) \\
&\leqslant&& \ \ \frac{\lambda + \epsilon}{2} + \ \ d\left( c'([t]), c'([t'])\right) + \frac{\lambda + \epsilon}{2}\\
&=&& \ \ d \left( c([t]), c([t']) \right) + \lambda + \epsilon \\
&\leqslant&& \ \ \lambda d_{\R}([t], [t']) + \epsilon + \lambda + \epsilon \\
&=&& \ \ \lambda \vert [t]-[t']\vert + \lambda + 2 \epsilon \\
&\leqslant&& \ \ \lambda \left( \vert [t]-t \vert + \vert t-t'\vert + \vert t'-[t'] \vert \right) + \lambda + 2\epsilon \\
&\leqslant&& \ \ \lambda\left(  \vert t-t'\vert + 1 \right) + \lambda + 2\epsilon \\
&=&& \ \ \lambda \vert t-t'\vert + 2 (\lambda + \epsilon).
\end{alignat*}

Wir erhalten außerdem 
\begin{alignat*}{5}
\frac{1}{\lambda} \vert t-t'\vert  - 2(\lambda + \epsilon) \ \ &\leqslant&& \ \ \frac{1}{\lambda} \left( \vert t-t'\vert -1\right) - (\lambda + 2\epsilon) \\
&\leqslant&& \ \ \frac{1}{\lambda} \left( \vert t-[t]\vert + \vert [t]-[t']\vert + \vert [t']-t'\vert -1 \right) - (\lambda + 2 \epsilon) \\
&\leqslant&& \ \ \frac{1}{\lambda} \left( \frac{1}{2} + \vert [t]-[t'] \vert + \frac{1}{2} - 1 \right) - (\lambda + 2 \epsilon) \\
&=&& \ \ \frac{1}{\lambda} \vert [t]-[t'] \vert - \epsilon - (\lambda + \epsilon) \\
&\leqslant&& \ \ d \left( c([t]), c([t']) \right) - (\lambda + \epsilon) \\
&=&& \ \ d \left( c'([t]), c'([t']\right) - (\lambda + \epsilon) \\
&\leqslant&& \ \ d(c'([t]), c'(t)) + d(c'(t), c'(t')) + d(c'(t'). c'([t'])) - (\lambda + \epsilon) \\
&\leqslant&& \ \  \frac{\lambda + \epsilon}{2} + d(c'(t), c'(t')) + \frac{\lambda + \epsilon}{2} - (\lambda + \epsilon) \\
&=&& \ \ d \left( c'(t), c'(t')\right),
\end{alignat*}
wobei die erste Ungleichung wegen 
$$\frac{1}{\lambda} \left( \vert t-t'\vert -1\right) - (\lambda + 2 \epsilon) = \frac{1}{\lambda} \vert t-t'\vert - \frac{1}{\lambda} - (\lambda + 2\epsilon) \geqslant \frac{1}{\lambda} \vert t-t'\vert - \lambda - (\lambda + 2 \epsilon) = \frac{1}{\lambda} \vert t-t'\vert - 2(\lambda + \epsilon)$$
gilt. Für $\epsilon':= 2(\lambda+ \epsilon)$ gilt also 
$$\frac{1}{\lambda} \vert t-t'\vert - \epsilon' \leqslant d(c(t), c(t')) \leqslant \lambda \vert t-t'\vert + \epsilon'$$
für alle $t,t' \in [a,b]$, $c':[a,b]\la X$ ist also eine $(\lambda, \epsilon')$-quasi-isometrische Einbettung und damit $(\lambda, \epsilon')$-quasi-geodätisch.
\item[(iii)] Für ganze Zahlen $m\leqslant n \in [a,b]$ gilt
$$l\left( c'\vert_{[m,n]}\right) = \sum_{i=m}^{n-1} d\left( c'(i), c'(i+1)\right) = \sum_{i=m}^{n-1} d\left( c(i), c(i+1)\right) \leqslant (\lambda + \epsilon) \vert n-m\vert$$
sowie 
$$l\left( c'\vert_{[a,n]}\right) \leqslant (\lambda + \epsilon)(m-a+1), \qquad l\left( c'\vert_{[n,b]}\right) \leqslant (\lambda + \epsilon) (b-n+1).$$
Für $t,t' \in [a,b]$ folgt also 
\begin{alignat*}{5}
l\left( c'\vert_{[t,t']}\right) \ \ &\leqslant&& \ \ l \left( c'\vert_{[t,[t]]}\right) + l \left( c'\vert_{[[t],[t']]}\right) + l\left( c'\vert_{[[t'],t']}\right) \\
&\leqslant && \ \ (\lambda + \epsilon) \left( [t]-[t'] +2 \right)
\end{alignat*}
Außerdem gilt
$$d\left( c'(t), c'(t')\right) \geqslant \frac{1}{\lambda} \vert t-t'\vert -  \epsilon' \geqslant \frac{1}{\lambda} \left( [t]-[t'] -1  \right) - \epsilon'$$
und damit
\begin{alignat*}{5}
l\left( c'\vert_{[t,t']}\right) \ \ &\leqslant&& \ \ (\lambda + \epsilon)\left( [t]-[t'] +2\right)\\
&=&& \ \ \lambda ( \lambda + \epsilon) \left( \frac{1}{\lambda}\left( [t]-[t'] -1 \right) + \frac{3}{\lambda} \right)\\
&=&& \ \ \lambda (\lambda + \epsilon)  \left( \frac{1}{\lambda}\left([t]-[t'] -1 \right) - \epsilon' \right) + (\lambda + \epsilon)(3+\lambda \epsilon')\\
&\leqslant&& \ \ \lambda(\lambda + \epsilon) d\left( c'(t), c'(t')\right) + (\lambda + \epsilon) ( 3 + \lambda \epsilon') \\
&=&& \ \ k_1 d\left( c'(t), c'(t')\right) + k_2,
\end{alignat*}
was zu zeigen war. $\hfill \Box$




\end{compactenum}
\end{pr}


\end{lemma}


\begin{pr}[\textrm{\textit{Beweis von Satz 2.8}}]
Zunächst ersetzen wir die Quasi-Geodätische $c$ durch eine stetige Quasi-Geodätische $c'$ wie in Lemma 2.11. Sei $[p,q]$ das geodätische Segment zwischen den Endpunkten von $c$ bzw. $c'$. Weiter sei 
$$D:= \max_{x \in [p,q]} \left\{ d(x, \mathrm{Bild}\ c'\right\}$$
und $x_0 \in [p,q]$ ein Punkt, in dem das Maximum angenommen wird. 

$$
\begin{tikzpicture}

\draw (-4,0) to[bend left=20] (4,0);
\draw (4,0) arc(30:150:4.62);
\draw[fill=black] (4.05,0) arc(0:360:0.05);
\draw[fill=black] (-3.95,0) arc(0:360:0.05);
\draw[fill=black] (0.05,0.8) arc(0:360:0.05);
\draw (4,0) node[below] {$q$};
\draw (-4,0) node[below] {$p$};
\draw (0,0.8) node[below] {$x_0$};
\draw (2.4,2) node[right] {$\mathrm{Bild} \ c'$};
\draw[dashed] (1.25,0.71) arc(-10:190:1.3);
\draw (2.5,0.48) node[below] {$z$};
\draw (-2.5,0.48) node[below] {$y$};
\draw[fill=black] (2.5,0.48) arc(0:360:0.05);
\draw[fill=black] (-2.4,0.48) arc(0:360:0.05);
\draw[fill=black] (-2.9,1.25) arc(0:360:0.05);
\draw[fill=black] (3,1.25) arc(0:360:0.05);
\draw (-3.05, 1.25) node[left] {$y'$};
\draw (3.05, 1.25) node[right] {$z'$};
\draw (-2.45,0.48) arc(20:45:2.1);
\draw (2.95,1.25) arc(135:160:2.1);
\end{tikzpicture}
$$

Der offene Ball mit Radius $D$ und Zentrum $x_0$ trifft dann $\mathrm{Bild} \ c'$ nicht. Idee: Benutze Satz 5.2.5, um eine Schranke für $D$ zu finden, die nicht von der Kurve abhängt (sondern nur von $\lambda, \epsilon$ und $\delta$). Wähle hierzu $y \in [p,x_0] \subseteq [p,q]$, sodass $d(y,x_0) =2D$. Ist $d(p,x_0)< 2D$, so wähle stattdessen $y=p$. Analog finden wir einen Punkt $z \in [x_0, q]$. Wähle weiter $y',z' \in \mathrm{Bild} \ c'$ mit $d(y,y') \leqslant D$ und $d(z,z') \leqslant D$ sowie geodätische Segmente $[y,y']$ und $[z',z]$. Sei nun $\gamma$ der parametrisierte Weg, welcher zunächst entlang $[y,y']$, dann $c'$ und anschließend $[z',z]$ läuft. Dann liegt $\mathrm{Bild} \ \gamma$ nach Konstruktion außerhalb des $D$-Balls um $x_0$. Es gilt 
$$d(y',z') \leqslant d(y',y) + d(y,z) + d(z,z') \leqslant D + 4D + D = 6D.$$
Aus Lemma 2.11 (iii) folgt nun 
$$l(\gamma) \leqslant d(y,y') + k_1 d(y',z') + k_2 + d(z,z') \leqslant k_1 6D + k_2 + 2D$$
und Satz 2.5 liefert wegen $D= d(x_0, \mathrm{Bild} \ c')$
$$D-1 \leqslant \delta \vert \log_2 l(\gamma)\vert \leqslant \delta \vert \log_2 (k_1 6D + k_2 + 2D)\vert$$
Diese Ungleichung zwischen eine linearem Term in $D$ und einem logarithmischen Term in $D$ kann nur erfüllt sein, falls $D$ beschränkt ist, das heißt Satz 5.2.5 liefert implizit eine obere Schranke für $D$, welche nur von $\lambda, \epsilon$ und $\delta$ abhängt. Sei $D_0$ eine solche Schranke.
\begin{compactenum}
\item[\textbf{Beh. (a)}] Es gilt $\mathrm{Bild} \ c' \subseteq U_{R'}([p,q])$ mit $R'=D_0(1+k_1) + \frac{k_2}{2}$.
\item[\textbf{Bew. (a)}] 
Betrachte ein maximales Teilintervall $[a',b'] \subseteq [a,b]$, sodass $c'\vert_{[a',b']}$ außerhalb der $D_0$-Umgebung $U_{D_0}([p,q])$ liegt (existiert kein solches, so folgt die Behauptung mit $R'=D_0$). Jeder Punkt von $[p,q]$ liegt per Definition von $D_0$ in $U_{D_0}(\mathrm{Bild} \ c')$, das heißt es gilt 
$$[p,q] \subseteq U_{D_0}\left(\mathrm{Bild} \ c'\vert_{[a,a']}  \cup \mathrm{Bild} \ c'\vert_{[b',b]} \right).$$
Da Intervalle zusammenhängend sind, existieren $w \in [p,q]$, $t \in [a,a']$ und $t' \in [b',b]$, sodass
$$d(w,c'(t)) \leqslant D_0, \qquad d(w,(c'(t')) \leqslant D_0.$$
Mit der Dreiecksungleichung folgt insbesondere
$$d(c'(t), c'(t')) \leqslant d(c'(t), w) + d(w, c'(t')) \leqslant 2D_0$$
und nach Lemma 2.11 (iii) gilt damit
$$l\left( c'\vert_{[t,t']}\right) \leqslant k_1 d(c'(t), c'(t')) + k_2 \leqslant k_1 2D_0 + k_2.$$

Für beliebiges $z=c'(t) \in \mathrm{Bild}\ c'$ gibt es also $t' \in [a,a'] \cup [b',b]$ mit
$$d(c'(t), c'(t')) \leqslant \frac{1}{2} \left( k_1 2D_0 + k_2\right) = k_1D_0 + \frac{k_2}{2}$$
und es folgt
$$d(z, [p,q]) \leqslant d(z,x_0) \leqslant d(z,c'(t')) + d(c'(t')) \leqslant k_1D_0 + \frac{k_2}{2} + D_0 \leqslant D_0(1+k_1) + \frac{k_2}{2} =R',$$
also $z \in U_{R'}([p,q])$ und damit $\mathrm{Bild}\ c' \subseteq U_{R'}([p,q])$.

\end{compactenum}


\end{pr}

Wir können den Beweis nun abschließen. Nach Lemma 2.11 (iv) gilt 
$$\mathrm{Hd} \left( \mathrm{Bild} \ c , \mathrm{Bild} \ c'\right) \leqslant \lambda + \epsilon$$
und da $\mathrm{Hd}$ eine Metrik auf der Menge der abgeschlossenen Teilmengen von $X$ bildet folgt
$$\mathrm{Hd}\left( [p,q], \mathrm{Bild} \ c\right) \leqslant \mathrm{Hd}\left([p,q], \mathrm{Bild} \ c'\right) + \mathrm{Hd} \left( \mathrm{Bild}\ c', \mathrm{Bild} \ c\right) \leqslant R' + \lambda + \epsilon =: R = R(\lambda, \epsilon, \delta),$$
die Behauptung des Satzes. $\hfill \Box$












\section{Hyperbolische Gruppen} %PARAGRAPH V.3






In diesem Abschnitt soll auf Verbindungen zwischen Geometrie und Gruppentheorie eingegangen werden. Sei $G$ eine endlich erzeugte Gruppe mit $G= \langle S \rangle$ für ein Erzeugendensystem $S=\{s_i \vert i \in I\}$ mit endlicher Indexmenge $I$ und $S^{-1}=S$, das heißt für alle $g \in G$ gibt es $s_1, \ldots, s_k \in S$ mit $g=s_1 \cdots s_k$ (Beachte, dass $S$ keineswegs eindeutig ist).

\begin{ex} Für $G=(\Z, +)$ wähle $S=\{-1,1\}$ oder $\tilde{S}=\{-3,-2,2,3\}$. Wegen $\mathrm{ggT}(2,3)=1$ gilt $\langle \tilde{S} \rangle = \Z$.
\end{ex}


Dies wollen wir nun geometrisieren (nach A. Cayley, 1878): Zu $G=\langle S\rangle$ assoziiere einen Graphen, einen \textit{Cayley-Graphen} $\mathrm{Cay}(G,S)$ zu $(G,S)$ mit $G$ als Menge der Eckpunkte sowie Kanten zwischen $g,h \in G$ genau dann, wenn es $s \in S$ gibt, sodass $h=gs$. 

\begin{ex}
Betrachte zu nächst $(\Z,+)= \langle \{-1,1\}=: \langle S\rangle$. Dann existiert zwischen zwei ganzen Zahlen $m,n \in \Z$ eine (nicht orientierte) Kante genau dann, wenn sie sich um ein Element in $S$ unterscheiden, also genau dann, wenn $\vert m-n\vert =1$. Der zugehörig Cayley-Graph $\mathrm{Cay}(\Z, S)$ sieht dann wie folgt aus:
$$
\begin{tikzpicture}[scale=1.7]
\draw[fill=black] (4.05,0) arc(0:360:0.05);
\draw[fill=black] (3.05,0) arc(0:360:0.05);
\draw[fill=black] (2.05,0) arc(0:360:0.05);
\draw[fill=black] (1.05,0) arc(0:360:0.05);
\draw[fill=black] (0.05,0) arc(0:360:0.05);
\draw[fill=black] (-0.95,0) arc(0:360:0.05);
\draw[fill=black] (-1.95,0) arc(0:360:0.05);
\draw[fill=black] (-2.95,0) arc(0:360:0.05);
\draw[fill=black] (-3.95,0) arc(0:360:0.05);
\draw (4,0) -- (-4,0);
\draw[dashed] (4.5,0) -- (4,0);
\draw[dashed] (-4.5,0) -- (-4,0);
\draw (4,-0.1) node[below] {$m+4$};
\draw (3,-0.1) node[below] {$m+3$};
\draw (2,-0.1) node[below] {$m+2$};
\draw (1,-0.1) node[below] {$m+1$};
\draw (0,-0.1) node[below] {$m$};
\draw (-1,-0.1) node[below] {$m-1$};
\draw (-2,-0.1) node[below] {$m-2$};
\draw (-3,-0.1) node[below] {$m-3$};
\draw (-4,-0.1) node[below] {$m-4$};
\end{tikzpicture}
$$
\end{ex}

\begin{ex}
Betrachte nun $(\Z, +) = \langle \{-3,-2,2,3\}\rangle=: \langle \tilde{S}\rangle$ Wir erhalten dann ein den Cayley-Graphen $\mathrm{Cay}(\Z, \tilde{S}\rangle$
$$
\begin{tikzpicture}[scale=1.5]
\draw[fill=black] (4.05,0) arc(0:360:0.05);
\draw[fill=black] (3.05,0) arc(0:360:0.05);
\draw[fill=black] (2.05,0) arc(0:360:0.05);
\draw[fill=black] (1.05,0) arc(0:360:0.05);
\draw[fill=black] (0.05,0) arc(0:360:0.05);
\draw[fill=black] (-0.95,0) arc(0:360:0.05);
\draw[fill=black] (-1.95,0) arc(0:360:0.05);
\draw[fill=black] (-2.95,0) arc(0:360:0.05);
\draw[fill=black] (-3.95,0) arc(0:360:0.05);


\draw (-4,0) to[bend left=40] (-2,0);
\draw (-3,0) to[bend left=40] (-1,0);
\draw (-2,0) to[bend left=40] (0,0);
\draw (-1,0) to[bend left=40] (1,0);
\draw (-0,0) to[bend left=40] (2,0);
\draw (1,0) to[bend left=40] (3,0);
\draw (2,0) to[bend left=40] (4,0);
\draw (3,0) to[bend left=20] (4,0.4);
\draw[dashed] (4,0.4) to[bend left=20] (5,0);
\draw[dashed] (-5,0) to[bend left=20] (-4,0.4);
\draw (-4,0.4) to[bend left=20] (-3,0);
\draw[dashed] (4,0) to[bend left=20] (5,0.4);
\draw[dashed] (-5,0.4) to[bend left=20] (-4,0);
\draw (-4,0) to[bend right=60] (-1,0);
\draw (-3,0) to[bend right=60] (0,0);
\draw (-2,0) to[bend right=60] (1,0);
\draw (-1,0) to[bend right=60] (2,0);
\draw (0,0) to[bend right=60] (3,0);
\draw (1,0) to[bend right=60] (4,0);


\end{tikzpicture}
$$

\end{ex}

\begin{ex}
Sei $G=(\Z^2,+)$ mit $S=\{\pm e_1, \pm e_2\}$. Der zugehörige Cayley-Graph hat dann die Gestalt
$$
\begin{tikzpicture}
\draw[fill=black] (0.05,0) arc(0:360:0.05);
\draw[fill=black] (1.05,0) arc(0:360:0.05);
\draw[fill=black] (2.05,0) arc(0:360:0.05);
\draw[fill=black] (3.05,0) arc(0:360:0.05);
\draw[fill=black] (-0.95,0) arc(0:360:0.05);
\draw[fill=black] (-1.95,0) arc(0:360:0.05);
\draw[fill=black] (-2.95,0) arc(0:360:0.05);

\draw[fill=black] (0.05,-1) arc(0:360:0.05);
\draw[fill=black] (1.05,-1) arc(0:360:0.05);
\draw[fill=black] (2.05,-1) arc(0:360:0.05);
\draw[fill=black] (3.05,-1) arc(0:360:0.05);
\draw[fill=black] (-0.95,-1) arc(0:360:0.05);
\draw[fill=black] (-1.95,-1) arc(0:360:0.05);
\draw[fill=black] (-2.95,-1) arc(0:360:0.05);

\draw[fill=black] (0.05,-2) arc(0:360:0.05);
\draw[fill=black] (1.05,-2) arc(0:360:0.05);
\draw[fill=black] (2.05,-2) arc(0:360:0.05);
\draw[fill=black] (3.05,-2) arc(0:360:0.05);
\draw[fill=black] (-0.95,-2) arc(0:360:0.05);
\draw[fill=black] (-1.95,-2) arc(0:360:0.05);
\draw[fill=black] (-2.95,-2) arc(0:360:0.05);

\draw[fill=black] (0.05,1) arc(0:360:0.05);
\draw[fill=black] (1.05,1) arc(0:360:0.05);
\draw[fill=black] (2.05,1) arc(0:360:0.05);
\draw[fill=black] (3.05,1) arc(0:360:0.05);
\draw[fill=black] (-0.95,1) arc(0:360:0.05);
\draw[fill=black] (-1.95,1) arc(0:360:0.05);
\draw[fill=black] (-2.95,1) arc(0:360:0.05);

\draw[fill=black] (0.05,2) arc(0:360:0.05);
\draw[fill=black] (1.05,2) arc(0:360:0.05);
\draw[fill=black] (2.05,2) arc(0:360:0.05);
\draw[fill=black] (3.05,2) arc(0:360:0.05);
\draw[fill=black] (-0.95,2) arc(0:360:0.05);
\draw[fill=black] (-1.95,2) arc(0:360:0.05);
\draw[fill=black] (-2.95,2) arc(0:360:0.05);

\draw (-3,2) -- (3,2);
\draw (-3,1) -- (3,1);
\draw (-3,0) -- (3,0);
\draw (-3,-1) -- (3,-1);
\draw (-3,-2) -- (3,-2);
\draw (-3,-2) -- (-3,2);
\draw (-2,-2) -- (-2,2);
\draw (-1,-2) -- (-1,2);
\draw (-0,-2) -- (0,2);
\draw (1,-2) -- (1,2);
\draw (2,-2) -- (2,2);
\draw (3,-2) -- (3,2);

\draw[dashed] (-3.5,2) -- (-3,2);
\draw[dashed] (-3.5,1) -- (-3,1);
\draw[dashed] (-3.5,0) -- (-3,0);
\draw[dashed] (-3.5,-1) -- (-3,-1);
\draw[dashed] (-3.5,-2) -- (-3,-2);

\draw[dashed] (3.5,2) -- (3,2);
\draw[dashed] (3.5,1) -- (3,1);
\draw[dashed] (3.5,0) -- (3,0);
\draw[dashed] (3.5,-1) -- (3,-1);
\draw[dashed] (3.5,-2) -- (3,-2);

\draw[dashed] (-3,-2.5) -- (-3,-2);
\draw[dashed] (-2,-2.5) -- (-2,-2);
\draw[dashed] (-1,-2.5) -- (-1,-2);
\draw[dashed] (-0,-2.5) -- (0,-2);
\draw[dashed] (1,-2.5) -- (1,-2);
\draw[dashed] (2,-2.5) -- (2,-2);
\draw[dashed] (3,-2.5) -- (3,-2);


\draw[dashed] (-3,2.5) -- (-3,2);
\draw[dashed] (-2,2.5) -- (-2,2);
\draw[dashed] (-1,2.5) -- (-1,2);
\draw[dashed] (0,2.5) -- (0,2);
\draw[dashed] (1,2.5) -- (1,2);
\draw[dashed] (2,2.5) -- (2,2);
\draw[dashed] (3,2.5) -- (3,2);



\end{tikzpicture}
$$

\end{ex}





\begin{ex}
Für die freie Gruppe $F_2=F(a,b)$ in zwei Erzeugern mit $S=\{a,b\}$ erhalten wir einen Cayley-Graphen
$$
\begin{tikzpicture}
\draw (-1.5,0) -- (1.5,0);
\draw (0,-1.5) -- (0,1.5);
\draw (-0.5,1) -- (0.5,1);
\draw (-0.5,-1) -- (0.5,-1);
\draw (-1,-0.5) -- (-1,0.5);
\draw (1,-0.5) -- (1,0.5);
\draw (0.375,0.875) -- (0.375,1.125);
\draw (-0.375,0.875) -- (-0.375,1.125);
\draw (0.375,-0.875) -- (0.375,-1.125);
\draw (-0.375,-0.875) -- (-0.375,-1.125);
\draw (-0.125,1.375) -- (0.125,1.375);
\draw (-0.125,-1.375) -- (0.125,-1.375);
\draw (0.875,-0.375) -- (1.125,-0.375);
\draw (0.875,0.375) -- (1.125,0.375);
\draw (-0.875,-0.375) -- (-1.125,-0.375);
\draw (-0.875,0.375) -- (-1.125,0.375);
\draw (1.375,-0.125) -- (1.375,0.125);
\draw (-1.375,-0.125) -- (-1.375,0.125);
\end{tikzpicture}
$$
\end{ex}

Nun wollen wir auf einem Cayley-Graphen eine topologische Struktur definieren. Definiere hierfür die \textit{Wortmetrik} auf $G$ bzw. auf $\mathrm{Cay}(G,S)$
$$d_S(g,h) := \min \left\{ m \in \mathbb{N}_0 \ \vert \ \textrm{ es existieren } s_1, \ldots, s_m \in S \textrm{ mit } g^{-1}h = s_1 \cdots s_m \right\} = d_S(e,g^{-1}h).$$
Dann ist $d_S$ eine linksinvariante Metrik auf $\mathrm{Cay}(G,S)$ und entspricht der kombinatorischen Graphenmetrik. Eine Gruppe $G=\langle S\rangle$ wird so zu einem geometrischen Objekt. Allerdings hängt die geometrische Struktur von der Wahl es Erzeugendensystems $S$ von $G$ ab. Welcher Zusammenhang besteht zwischen den entstehenden metrischen Räumen? Wir haben einen entscheidenden Satz:

\begin{theorem}
Sei $G$ eine Gruppe sowie $S, \tilde{S}$ Erzeugendensysteme für $G$. Dann induziert die Identitätsabbildung $\mathrm{id}_G$ eine $(\lambda,0)$-Quasi-Isometrie zwischen den metrischen Räumen $(\mathrm{Cay}(G,S), d_S)$ und $(\mathrm{Cay}(G,\tilde{S}), d_{\tilde{S}})$.
\begin{pr}
Vorlesung über Geometrische Gruppentheorie oder Übung. $\hfill \Box$
\end{pr}
\end{theorem}

Eine geometrische Eigenschaft einer endlich erzeugten Gruppe entspricht also einer Eigenschaft eines Cayley-Graphen, welche invariant unter Quasi-Isometrien ist (Gromov, 1987).
Wir führen nun die Definition einer hyperbolischen Gruppe ein:

\begin{defin}
Eine endlich erzeugte Gruppe $G= \langle S\rangle$ heißt \textit{hyperbolisch}, falls der Cayley-Graph $\mathrm{Cay}(G,S)$ als metrischer Raum $\delta$-hyperbolisch ist für ein $\delta = \delta(S) \geqslant 0$.
\end{defin}

Nach Satz 5.4 ist Hyperbolizität eines geodätischen, metrischen Raums eine Quasi-Isometrie-Invariante, Hyperbolizität einer endlich erzeugten Gruppe ist also eine geometrische Eigenschaft.

\begin{ex} 
\begin{compactenum}
\item $F_2=F(a,b)$ ist $\delta$-hyperbolisch, da sein Graph ein Baum und damit $0$-hyperbolisch ist.
\item $\Z^2$ ist nicht hyperbolisch, wie wir in Beispiel 5.1.2 bereits gesehen haben.
\end{compactenum}
\end{ex}

Eine wichtige Beobachtung: Jeder metrische Raum $(X,d)$, der quasi-isometrisch zu $(\mathrm{Cay}(G,S), d_S)$ ist, ist auch quasi-isometrisch zu $(\mathrm{Cay}(G,\tilde{S}), d_{\tilde{S}})$ ist. Die Frage ist nun: Wie findet man zu einer gegebenen Gruppe $G=\langle S\rangle$ einen zu $(\mathrm{Cay}(G,S), d_S)$ quasi-isometrischen metrischen Raum $(X,d)$? Eine Antwort bietet der folgende Satz:

\begin{theorem}[Satz von Schwarz-Milnor]
Sei $G$ eine endlich erzeugte Gruppe, welche auf einem geodätischen, metrischen Raum $(X,d)$ durch Isometrien operiere. Weiter existiere eine Teilmenge $B \subseteq X$ mit $\mathrm{diam} B < \infty$ und 
\begin{compactenum}
\item $X=\bigcup_{g \in G} g \cdot B$
\item Die Menge $S=\{g \in G \ \vert g \cdot B \cap B \neq \emptyset \}$ ist endlich.
\end{compactenum}
Dann gilt: Für alle $x \in X$ ist die Bahn-Abbildung 
$$\phi_x: G \la G\cdot x \subseteq X, \qquad g \mapsto g \cdot x$$
eine Quasi-Isometrie.
\begin{pr}
Siehe Vorlesung über Geometrische Gruppentheorie oder Übung. $\hfill \Box$
\end{pr}
\end{theorem}



\begin{ex}
Die Gruppe $(\Z^2, +)$ operiert auf $\R^2$ durch Isometrien via 
$$\cdot: \Z^2 \times \R^2 \la \R^2, \qquad (m,n) \cdot (x,y) \mapsto (x+m,y+n).$$
Wähle $B= \left[ -\frac{1}{2}, \frac{1}{2} \right] \times \left[ -\frac{1}{2}, \frac{1}{2} \right]$:

$$
\begin{tikzpicture}
\draw[thin, ->] (-5,0) -- (5,0);
\draw[thin, ->] (0,-4.5) -- (0,4.5);
\draw[fill=black] (2.05,0) arc (0:360:0.05);
\draw[fill=black] (-1.95,0) arc (0:360:0.05);
\draw[fill=black] (0.05,2) arc (0:360:0.05);
\draw[fill=black] (0.05,-2) arc (0:360:0.05);
\draw[fill=black] (0.05,0) arc(0:360:0.05);
\draw[fill=black] (4.05,0) arc(0:360:0.05);
\draw[fill=black] (2.05,2) arc(0:360:0.05);
\draw[fill=black] (2.05,-2) arc(0:360:0.05);
\draw[fill=black] (-1.95,-2) arc (0:360:0.05);
\draw[fill=black] (-1.95,2) arc (0:360:0.05);
\draw[fill=black] (-3.95,0) arc (0:360:0.05);
\draw[fill=red, opacity=0.2] (1,1) -- (1,-1) -- (-1,-1) -- (-1,1) -- (1,1);
\draw (1,1) -- (1,-1) -- (-1,-1) -- (-1,1) -- (1,1);

\draw (3,1) -- (3,-1) -- (1,-1) -- (1,1) -- (3,1);
\draw (-1,1) -- (-1,-1) -- (-3,-1) -- (-3,1) -- (-1,1);
\draw (1,3) -- (1,1) -- (-1,1) -- (-1,3) -- (1,3);
\draw (1,-1) -- (1,-3) -- (-1,-3) -- (-1,-1) -- (1,-1);

\draw[fill=green, opacity=0.2] (3,1) -- (3,-1) -- (1,-1) -- (1,1) -- (3,1);
\draw[fill=green, opacity=0.2] (-1,1) -- (-1,-1) -- (-3,-1) -- (-3,1) -- (-1,1);
\draw[fill=green, opacity=0.2] (1,3) -- (1,1) -- (-1,1) -- (-1,3) -- (1,3);
\draw[fill=green, opacity=0.2] (1,-1) -- (1,-3) -- (-1,-3) -- (-1,-1) -- (1,-1);

\draw (3,3) -- (3,1) -- (1,1) -- (1,3) -- (3,3);
\draw (-1,-1) -- (-1,-3) -- (-3,-3) -- (-3,-1) -- (-1,-1);
\draw (-1,3) -- (-1,1) -- (-3,1) -- (-3,3) -- (-1,3);
\draw (3,-1) -- (3,-3) -- (1,-3) -- (1,-1) -- (3,-1);

\draw[fill=green, opacity=0.2] (3,3) -- (3,1) -- (1,1) -- (1,3) -- (3,3);
\draw[fill=green, opacity=0.2] (-1,-1) -- (-1,-3) -- (-3,-3) -- (-3,-1) -- (-1,-1);
\draw[fill=green, opacity=0.2] (-1,3) -- (-1,1) -- (-3,1) -- (-3,3) -- (-1,3);
\draw[fill=green, opacity=0.2] (3,-1) -- (3,-3) -- (1,-3) -- (1,-1) -- (3,-1);

\draw (0,0.5) node[right] {$B$};

\draw[dashed] (3,1) -- (4,1);
\draw[dashed] (3,3) -- (4,3);
\draw[dashed] (3,3) -- (3,4);
\draw[dashed] (1,3) -- (1,4);
\draw[dashed] (-1,3) -- (-1,4);
\draw[dashed] (-3,3) -- (-3,4);
\draw[dashed] (-3,3) -- (-4,3);
\draw[dashed] (-3,1) -- (-4,1);
\draw[dashed] (-3,-1) -- (-4,-1);
\draw[dashed] (-3,-3) -- (-4,-3);
\draw[dashed] (-3,-3) -- (-3,-4);
\draw[dashed] (-1,-3) -- (-1,-4);
\draw[dashed] (1,-3) -- (1,-4);
\draw[dashed] (3,-3) -- (3,-4);
\draw[dashed] (3,-3) -- (4,-3);
\draw[dashed] (3,-1) -- (4,-1);

\end{tikzpicture}
$$
Offenbar gilt 
$$\bigcup_{(m,n)  \in \Z^2} g \cdot B = \R^2$$
sowie 
\begin{alignat*}{4}
S\ \ &=&& \ \ \{ (m,n) \in \Z^2 \ \vert \ (m,n) \cdot B \neq \emptyset\}\\
 &=&& \ \  \{(1,0), (-1,0), (0,1), (0,-1), (1,1), (-1,-1), (1,-1), (-1,1) \}
\end{alignat*}
Satz 3.9 liefert also eine Quasi-Isometrie.
\end{ex}

\begin{ex} Mit Dreiecksgruppen können ähnliche Konstruktionen durchgeführt werden.
\end{ex}




Welche Quasi-Isometrie-Invarianten gibt es noch?
\begin{compactitem}
\item z.B. Wachstumsfunktionen (quadratisch, exponentiell,...)
\item isoperimetrische Ungleichungen
\end{compactitem}



Aus der Topologie betrachten wir nun die Klassifikation von kompakten, orientierbaren, $2$-dimensionalen Mannigfaltigkeiten ("Flächen"): Hier für bezeichne das Geschlecht $g$ eine Fläche die Anzahl der "Henkel" der Mannigfaltigkeit. Dann gibt es zu jedem $g$ bis aus Homöomorphie genau eine solche Fläche $F_g$: Für $g=0$ erhalten wir die Sphäre, $g=1$ entspricht dem Torus (welcher durch das Anbringen eines Henkels an die Sphäre hervorgeht), für $g=2$ die Sphäre mit zwei Henkeln usw.
\begin{compactitem}
\item Für $g=0$ ist $F_0$ die $2$-Sphäre. Sie hat konstante Krümmung $1$.
\item $F_1$ ist der flache Torus $\mathbb{T}^2 = \slant{\R^2}{\Z^2}$. Er ist lokal euklidisch (als Quotient der eukldischen Ebene), hat also konstante Krümmung $0$. 
\item Für $g\geqslant 2$ gilt wegen des Satzes von Gauß-Bonnet, welcher
$$\int_{F_g} \kappa \mathrm{d}A = 2 \pi \chi(F_g) = 2\pi (2-2g)$$
besagt, dass $F_g$ negativ gekrümmt ist. Mann kann zeigen, dass $F_g$ als Quotient von $\He^2$ nach einer direkten Gruppe realisiert wird.
\end{compactitem}









\newpage
\thispagestyle{empty}























\clearpage
\titlespacing{\chapter}
             {0pc}{*7}{*5}[0pc]
\renewcommand*\thechapter{\Alph{chapter}}

\setcounter{chapter}{0}


\titlecontents{chapter}[2.3em]{\addvspace{2pc}\bfseries}{\contentslabel{2em}}{}{\titlerule*[0.5pc]{. \ }\contentspage}
\titlecontents{section}[2.3em]{}{\contentslabel{4em}}{}{\titlerule*[0.5pc]{. \ }\contentspage}

\titleformat{\chapter}{\bf\Huge}{\Alph{chapter} \quad #1 }{5em}{}


\renewcommand{\chaptermark}[1]{ 
  \markboth{ 
     \MakeUppercase{#1} 
  }{} 
} 
\renewcommand{\sectionmark}[1]{ 
  \markright{ 
     \MakeUppercase{\hspace{-4pt}#1} 
  } 
}




\chapter{Literatur}


\thispagestyle{empty}


\textrm{ }\\[55pt]


\begin{itemize}
\item J.W. Anderson, \textit{Hyperbolic Geometry}, Springer Undergraduate Mathematic Series, Springer Verlag, London 1999.
\item S. Stahl, \textit{The Poincaré Half-Plane, A Gateway to Modern Geometry}. Jones and Bartlett, Boston 1993.
\item S. Katok, \textit{Fuchsian Groups}, Chicago Lecture in Mathematics, Chicago UP, 1992.
\item J. Cannon, W. Floyd, R. Kenyon, W. Parry, \textit{Hyperbolic Geometry}, in Levy, S. (ed), \textit{Flavours of Geometry}, MSRI Publ. 31. Ca,bridge UP, 1997.
\item R. Benedetti, C. Petronio, \textit{Lectures on Hyperbolic Geometry}, Universitext, Springer Berlin, 1992.
\item J. Ratcliffe, \textit{Foundations of Hyperbolic Manifolds}, Graduate texts in Math. 149, Springer, New York, 1994.
\item M. Gromov, \textit{Hyperbolic Groups}, in: \textit{Essays in Group Theory}, ed. S. Gersten, MSRI Publ. 8, Springer, 1987.
\item M. Bridson, A. Haefliger, \textit{Metric spaces of Non.positive Curvature}, Springer, 1999.
\item J. Milnor, \textit{Hyperbolic Geometry: the first 150 Years}, Bulletin of the AMS, Bd. 6, 1982.



\end{itemize}



\newpage

\thispagestyle{empty}









\chapter{Übungen} %KAPITEL VII

\setlength\abovedisplayshortskip{0pt}
\setlength\belowdisplayshortskip{10pt}
\setlength\abovedisplayskip{10pt}
\setlength\belowdisplayskip{10pt}

\thispagestyle{empty}
\setcounter{section}{0}





\titlespacing*{\section}{-16.5pt}{0pt}{20pt}
\renewcommand*\thesection{}
\section{Übungsblatt 1} %PARAGRAPH A.1
\renewcommand*\thesection{\arabic{section}}


\hypertarget{Aeinseins}{}
\begin{prob}     %%AUFGABE 1.1
Für zwei Strahlen $g(t):=tv+w$ und $h(s)=su+w$ mit $u,v,w \in \R^2$, $s,t \in \R_{\geqslant 0}$ ist der Winkel im Schnittpunkt $w$ definiert als die eindeutig bestimmte Zahl $\eta\in [0,\pi]$ mit 
$$\cos \eta = \frac{\langle v,u \rangle}{\Vert v \Vert \Vert u \Vert}.$$
Es sei nun $\Delta=\Delta ABC$ ein Dreieck in der euklidischen Ebene $\E^2$ mit Kanten der Länge $a,b,c$ (jeweils gegenüber von $A,B,C$) und sei $\alpha=\sphericalangle CAB$ der Winkel bei $A$, sowie $\beta$ und $\gamma$ die Winkel bei $B$ bzw. bei $C$. Zeigen Sie
\begin{compactenum}
\item den \textit{Kosinussatz}:
$$c^2=a^2+b^2-2ab \cos \gamma.$$
\item den \textit{Sinussatz}:
$$\frac{a}{\sin\alpha} = \frac{b}{\sin \beta} = \frac{c}{\sin \gamma}.$$
\end{compactenum}

\begin{sol}
\begin{compactenum}
\item Es seien $g(t)=tv+C$ und $h(s)=su+C$ mit $v,u \in \Sph^1$, $t \in [0,b]$, $s \in [0,a]$ die parametrisierten Kanten von $C$ nach $B$ bzw. nach $A$. Dann lässt sich die Kante zwischen $B$ und $A$ schreiben als 
$$k(r)=r \left( \frac{h(a)-g(b)}{c}\right) + B$$
mit $r \in [0,c]$ und es gilt
\begin{alignat*}{5}
c^2 \ \ &=&& \ \ \Vert k(c)\Vert^2 \\
&=&& \ \ \Vert h(a) - g(b)\Vert^2 \\
&=&& \ \ \langle h(a) - g(b), h(a) - g(b)\rangle \\ 
&=&& \ \ \langle h(a), h(a)\rangle - 2 \langle h(a),g(b)\rangle + \langle g(b), g(b)\rangle\\
&=&& \ \ a^2+b^2 - 2ab \cos \gamma,
\end{alignat*}
was den Kosinussatz beweist.
\item Nach dem Kosinussatz gilt
$$a^2=b^2+c^2- 2 b c \cos \alpha,$$
also 
$$(a^2-b^2-c^2)^2 = (2-bc \cos \alpha)^2 = 4 b^2c^2 \cos^2 \alpha.$$
Genau so gelten
$$(b^2-a^2-c^2)^2 = 4a^2c^2 \cos^2 \beta, \qquad (c^2-a^2-b^2)^2 = 4a^2b^2 \cos^2 \gamma.$$
Es folgt schließlich
\begin{alignat*}{5}
\frac{\sin^2 \alpha}{\sin^2 \beta} \ \ &=&& \ \ \frac{1- \cos^2 \alpha}{1- \cos^2 \beta} \\[6pt]
&=&& \ \ \frac{4a^2b^2c^2 (1-\cos^2 \alpha)}{4a^2b^2c^2(1-\cos^2 \beta)} \\[6pt]
&=&& \ \ \frac{4a^2b^2c^2 - a^2 (a^2-b^2-c^2)^2}{4a^2b^2c^2 - b^2(b^2-a^2-c^2)^2} \\[6pt]
&=&& \ \ \frac{4a^2b^2c^2 - a^2(a^4+b^4+c^4-2a^2b^2-2a^2c^2+2b^2c^2)}{4a^2b^2c^2 - b^2(a^4+b^4+c^4-2a^2b^2 + 2a^2c^2 - 2b^2c^2)} \\[6pt]
&=&& \ \ \frac{a^2(-a^4-b^4-c^4+2a^2b^2+2a^2c^2+2b^2c^2)}{b^2(-a^4-b^4-c^4+2a^2b^2 + 2a^2c^2+2b^2c^2)}\\[6pt]
&=&& \ \ \frac{a^2}{b^2}.
\end{alignat*}
Da $\sin\vert_{[0,\pi]} \geqslant 0$ folgt damit 
$$\frac{a}{\sin\alpha} = \frac{b}{\sin\beta}.$$
Die zweite Gleichheit folgt analog. $\hfill \Box$


\end{compactenum}

\end{sol}
\end{prob}









\begin{prob}    %%AUFGABE 1.2
Sei
$$\mathcal{J}: \CC \la \CC, \qquad z \mapsto \mathcal{J}(z):= \begin{cases} \ \frac{1}{z} & \textrm{ für } z \in \C \setminus \{0\}, \\ \ \infty & \textrm{ für } z=0, \\ \ 0 & \textrm{ für }z=\infty. \end{cases}$$
Zeigen Sie, dass $\mathcal{J}$ stetig ist.

\begin{sol}
Wir zeigen Folgenstetigkeit auf allen drei Teilgebieten, auf denen $\mathcal{J}$ definiert ist. Sei $(z_n)_{n \in \N}$ eine konvergente Folge mit $z_n \xrightarrow{n\to \infty} z$ in $\C$.
\begin{compactenum}
\item[$z=0.$] Ist $(z_n)$ eine Nullfolge, so gilt $\frac{1}{z_n} \xrightarrow{n\to \infty} \infty$ und damit
$$ \lim_{n \to \infty} \mathcal{J}(z_n) = \lim_{n\to \infty} \frac{1}{z_n} = \infty = \mathcal{J}(0) = \mathcal{J}\left( \lim_{n\to\infty} z_n\right),$$
$\mathcal{J}$ ist also stetig in $z=0$.
\item[$z=\infty$.] Sei nun also $z_n \xrightarrow{n \to \infty}\infty$. Dann gilt offenbar $\mathcal{J}(z_n) =\frac{1}{z_n} \xrightarrow{n \to \infty} 0 = \mathcal{J}(\infty)$ und damit die Stetigkeit in $z=\infty$.
\item[$z \in \C\setminus \{0\}$.] Klar. $\hfill \Box$
\end{compactenum}

\end{sol}

\end{prob}




\hypertarget{Aeinsdrei}{}
\begin{prob}    %%AUFGABE 1.3
Beweisen sie \hyperlink{propeinseinsvier}{Proposition 1.1.4}, indem Sie zeigen, dass die stereographische Projektion $\overline{\sigma}: \Sph^2 \la \CC$ ein Homöomorphismus ist.
\begin{sol}
Es gilt 
$$\overline{\sigma}: \Sph^2 \la \CC, \qquad P=(x,y,z) \mapsto \overline{\sigma}(P) = \ \begin{cases} \ \sigma(P)= \frac{x}{1-z}+ i \frac{y}{1-z} & \textrm{ für } P \neq (0,0,1), \\ \ \infty & \textrm{ für } P=(0,0,1). \end{cases}$$
mit der gewöhnlichen stereographischen Projektion $\sigma$. $\overline{\sigma}$ ist damit stetig auf $\Sph^2 \setminus \{N\}$. Sei nun $(z_n=(x_n,y_n,z_n))_{n \in \N}$ mit $z_n \xrightarrow{n \to \infty} N= 0,0,1)$. Dann gilt 
$$\overline{\sigma}(z_n) = \left(\frac{x_n}{1-z_n}, \frac{y_n}{1-z_n}, 0 \right)  \xrightarrow{n \to \infty}  = \infty =  \overline{\sigma}(N) = \overline{\sigma}\left( \lim_{n \to \infty} z_n\right),$$
$\overline{\sigma}$ is also stetig. Weiter ist durch 
$$\tau: \CC \la \Sph^2, \qquad z=x+iy \mapsto \tau(z)=\ \begin{cases} \ \left( \frac{2x}{x^2+y^2+1}, \frac{2y}{x^2+y^2+1}, \frac{x^2+y^2-1}{x^2+y^2+1} \right) & \textrm{ für } z \neq \infty, \\ \ (0,0,1) & \textrm{ für }z=\infty. \end{cases}$$
eine stetige Umkehrabbildung für $\overline{\sigma}$ gegeben, denn für $z=x+iy\in \C$ gilt
\begin{alignat*}{5}
(\overline{\sigma} \circ \tau)(z)\ \ &=&& \ \  \overline{\sigma} \left( \frac{2x}{x^2+y^2+1}, \frac{2y}{x^2+y^2+1}, \frac{x^2+y^2-1}{x^2+y^2+1} \right) \\[6pt]
&=&& \ \ \frac{2x}{x^2+y^2+1} \left( 1- \frac{x^2+y^2-1}{x^2+y^2+1}\right)^{-1} + i \frac{2y}{1+x^2+y^2} \left( 1- \frac{x^2+y^2-1}{x^2+y^2+1}\right)^{-1} \\[6pt]
&=&& \ \ \frac{2x}{x^2+y^2+1} \left(\frac{2}{x^2+y^2+1}\right)^{-1} + i \frac{2y}{1+x^2+y^2} \left(\frac{2}{x^2+y^2+1}\right)^{-1} \\[6pt]
&=&& \ \ x+iy
\end{alignat*}
und für $(x,y,z) \in \Sph^1$ gilt
\begin{alignat*}{5}
(\tau \circ \overline{\sigma})(x,y,z) \ \ &=&& \ \ \tau \left( \frac{x}{1-z} + i \frac{y}{1-z} \right) \\[6pt]
&=&& \ \ \left( \frac{\frac{2x}{1-z}}{\frac{x^2}{(1-z)^2}+\frac{y^2}{(1-z)^2}+1}, \frac{\frac{2y}{1-z}}{(\frac{x^2}{(1-z)^2}+\frac{y^2}{(1-z)^2}+1}, \frac{ \frac{x^2}{(1-z)^2} + \frac{y^2}{(1-z)^2} -1}{\frac{x^2}{(1-z)^2} + \frac{y^2}{(1-z)^2} + 1} \right) \\[6pt]
&=&& \ \ \left( \frac{2x (1-z)}{x^2+y^2 + (1-z)^2}, \frac{2y(1-z)}{x^2+y^2+(1-z)^2}, \frac{x^2+y^2-(1-z)^2}{x^2+y^2+(1-z)^2} \right) \\[6pt]
&=&& \ \ \left( \frac{2x(1-z)}{2-2z}, \frac{2y(1-z)}{2-2z}, \frac{2-2z + 2 (1-z)^2}{2-2z} \right) \\[6pt]
&=&& \ \ (x,y,z).
\end{alignat*}
Insgesamt folgt also, dass $\overline{\sigma}$ ein Homöomorphismus ist. $\hfill \Box$
\end{sol}

\end{prob}








\newpage




\titlespacing*{\section}{-16.5pt}{0pt}{20pt}
\renewcommand*\thesection{}
\section{Übungsblatt 2} %PARAGRAPH A.2
\renewcommand*\thesection{\arabic{section}}


\begin{prob}     %%AUFGABE 2.1
Ein topologischer Raum $X$ heißt \textit{kompakt}, wenn zu jeder Überdeckung $X=\bigcup_{i \in I} U_i$ durch offene Mengen $U_i \subseteq X$ eine endliche Teilüberdeckung $X=\bigcup_{k=1}^n U_{i_k}$ existiert. Zeigen Sie ohne die stereographische Projektion zu verwenden, dass $\CC$ mit der in der Vorlesung beschriebenen Topologie kompakt ist.

\begin{sol}
Wir bemerken zunächst, dass eine Teilmenge $U \subseteq \CC$ mit $\infty \notin U$ offen in $\CC$ ist genau dann, wenn $U \subseteq \C$ offen ist. Ist $\infty \in U$, so ist $U$ offen genau dann, wenn $U^C$ kompakt ist, denn: Ist $\infty \in U$, so gibt es ein $\epsilon >0$ mit $U_{\epsilon} = \{z \in \CC \ \vert \ \vert z \vert > \epsilon\} \subseteq U$. Damit ist $U^C \subseteq \overline{B}_{\epsilon}$ als abgeschlossene Teilmenge einer kompakten Teilmenge wieder kompakt. Wir zeigen nun, dass $\CC$ kompakt ist. Sei dazu $\{U_i\}_{i \in I}$ eine offene Überdeckung von $\CC$. Dann gibt es einen Index $i_1\in I$ mit $\infty \in U_{i_1}$, $\CC \setminus U_{i_1}$ ist also kompakt. Da $\{U_i\}_{i \in I\setminus \{i_1\}}$ dann $\CC \setminus U_{i_1}$ überdeckt, gibt es $i_2, \ldots, i_n \in I\setminus \{i_1\}$ sodass $\bigcup_{k=2}^n U_{i_k} = \CC \setminus U_{i_1}$. Damit ist $$\CC= \CC \setminus U_{i_1} \cup U_{i_1} = \bigcup_{k=1}^n U_{i_k},$$
$\CC$ ist also kompakt. $\hfill \Box$

\end{sol}

\end{prob}





\begin{prob}   %%AUFGABE 2.2
\begin{compactenum}
\item Es sei die \textit{Bewegungsgruppe der euklidischen Ebene} definiert als 
$$\mathcal{O}(2) \ltimes \R^2 := \{(A,b) \in \mathcal{O}(2) \times \R^2\}$$
mit der Verknüpfung
$$(A,b) \circ (C,d) := (AC,b+Ad).$$
Zeigen Sie, dass diese Gruppe auf der euklidischen Ebene $\E^2$ via 
$$\cdot: \R^2 \times (\mathcal{O}(2) \ltimes \R^2) \la \R^2, \qquad (A,b) \cdot x := Ax+b$$
transitiv operiert.

\item Bestimmen Sie eine Möbiustransformation, die die obere Halbebene $\He^2$ auf die Einheitskreisscheibe $\D^2$ abbildet.
\end{compactenum}

\begin{sol}
\begin{compactenum}
\item Wir zeigen zunächst, dass $\cdot$ tatsächlich eine Gruppenoperation definiert. Offenbar ist das neutrale Element der Bewegungsgruppe gegeben durch $e_{\mathcal{O}(2) \ltimes \R^2}=(I_2, 0)$. Dann gilt 
$e_{\mathcal{O}(2) \ltimes \R^2} \cdot x =x.$ Sind nun $(A,b),(C,d) \in \mathcal{O}(2) \ltimes \R^2$, so erhalten wir weiter
\begin{alignat*}{5}
\left( (A,b) \circ (C,d)\right) \cdot x \ \ &=&& \ \ (AC,b+Ad) \cdot x\\
&=&& \ \ ACx+b + Ad \\
&=&& \ \ A(Cx+d) + b \\
&=&& \ \ (A,b) \cdot \left( (C,d) \cdot x\right),
\end{alignat*}
das heißt, $\cdot$ operiert auf $\E^2$. Für die Transitivität seien $y_1, y_2 \in \E^2$ beliebig. Dann gilt 
$$(I_2, -y_1+y_2) \cdot y_1 = y_1 + (-y_1+y_2) =y_2.$$
\item Die Idee ist es, den reellen Kreis $\RR$ auf $\Sph^1$ abzubilden. Aus Stetigkeitsgründen muss dann bereits $m(\He^2) \in \{ \mathbb{D}^2, \CC \setminus \mathbb{D}^2\}$ gelten. Fordert man dann noch $\vert m(i)\vert <1$, so ist $m$ bestimmt. Wir haben den allgemeinen Ansatz 
$$m(z) = \frac{az+b}{cz+d}$$
mit $a,b,c,d \in \C$, $ad-bc \neq0$. Bilde das Tripel $(0,1, \infty)$ auf das Tripel $(1,i,-1)$ ab. In beiden Tripeln liegt die für uns relevante Scheibe links des orientierten Weges $0\la1\la\infty$ bzw. $1 \la 1\la-1$, d.h. es gilt damit tatsächlich $m(\He^2)=\mathbb{D}^2$.Unsere Bedingungen liefern
$$1=m(0)=\frac{b}{d}, \qquad i=m(1)=\frac{a+b}{c+d}=\frac{b+a}{b-a}, \qquad -1=m(\infty)=\frac{a}{c},$$
also wegen ersterem $b=d$ sowie letzterem $a=-c$. Die Zzweite Bedingung besagt $a+b=i(b-a)$, was beispielsweise für $b=1$ und $a=i$ erfüllt ist (beachte, dass diese Wahl nicht eindeutig ist). Damit ist $m$ gegeben durch 
$$m(z)=\frac{az+b}{cz+d}=\frac{iz+1}{-iz+1}$$
und es gilt tatsächlich $\vert m(i)\vert = 0 <1$, also $m(\He^2)=\D^2$. $\hfill \Box$

\end{compactenum}
\end{sol}

\end{prob}





\begin{prob}   %%AUFGABE 2.3
Zeigen Sie, dass die Abbildung 
$$f: \CC \la \CC, \qquad z \mapsto \ \begin{cases} \ \infty & \textrm{ falls }z=\infty, \\ \ z^2 & \textrm{ sonst } \end{cases} $$
\textit{nicht} invariant unter $\mob$ ist. Bestimmen Sie außerdem die größte Untergruppe von $\mob$, welche $f$ invariant lässt.

\begin{sol}
Betrachte die Möbiustransformation $\mathcal{J}(z)=\frac{1}{z}$. Dann gilt offenbar 
$$f(m(\infty))=f(0) = 0 \neq \infty = f(\infty).$$
Um zu bestimmen, welche Möbiustransformationen $f$ invariant lassen, machen wir den allgemeinen Ansatz 
$$m(z)=\frac{az+b}{cz+d}$$
für gewisse $a,b,c,d\in \CC$ und $ad-bc\neq 0$. Einsetzen von gewissen Bedingungen in $f\circ m =f$ liefert
$$f(\infty)=\infty \overset{!}{=}f(m(\infty)) = f\left(\frac{a}{c}\right) = \frac{a^2}{c^2},$$
also $c=0$, 
$$f(0)=0\overset{!}{=} f(m(0))=f\left(\frac{b}{d}\right) = \frac{b^2}{d^2},$$
also $b=0$. Wegen $ad-bc=ad\neq0$ gilt außerdem $a\neq0\neq d$, also $m(z)=\frac{a}{d}z$.Weiter ist *$$z^2=f(z)\overset{!}{=} f(m(z))= f\left(\frac{a}{d}z\right))=\frac{a^2}{d^2}z^2,$$
also $a^2=d^2$ und somit $a=d$ oder $a=-d$. Mehr Bedingungen gibt es nicht, also ist $\{id, -\id\} \cong \slant{\Z}{2\Z}$ die größte Untergruppe von $\mob$, welche $f$ invariant lässt. $\hfill \Box$



\end{sol}



\end{prob}


\hypertarget{Azweivier}{}
\begin{prob}    %%AUFGABE 2.4
Beweisen Sie \hyperlink{propeinsviervier}{Proposition 1.4.4} aus der Vorlesung: Seien $z_1,z_2,z_3,z_4$ vier paarweise verschiedene Punkte in $\CC$. Dann liegen {$z_1,z_2,z_3,z_4$ }auf einem Kreis in $\CC$ genau dann, wenn das Doppelverhältnis $[z_1,z_2;z_3,z_4]$ reell ist.
\begin{sol}
Als Erinnerung sei erwähnt, dass Kreise in $\CC$ von der Form 
$$K=\{z \in \CC \ \vert \ \alpha z \overline{z} + \beta z + \overline{\beta} \overline{z} + \gamma=0\}$$
sind, wobei $\alpha, \gamma \in \R$, $\beta \in \C$ Konstanten sind. 
\begin{compactenum}
\item["$\Leftarrow$"] Sei zunächst $[z_1,z_2;z_3,z_4] \in \R$ sowie $K$ der von $z_1,z_2,z_3$ eindeutig festgelegte Kreis in $\CC$. Sei weiter $m\in \mob$ eine Möbiustransformation mit 
$$m(z_1)=0, \qquad m(z_2)=1, \qquad m(z_3)=\infty,$$
sodass also $m(K)=\RR$. Da das Doppelverhältnis invariant unter $\mob$ ist, gilt
\begin{alignat*}{5}
\R \ni [z_1,z_2;z_3,z_4] \ \ &=&& \ [m(z_1),m(z_2);m(z_3),m(z_4)] \\
&=&& \ \ [0,1;\infty, m(z_4)]\\
&=&& \ \ \frac{(0-m(z_4))(\infty-1)}{(0-1)(\infty-m(z_4))}\\
&=&& \ \ m(z_4),
\end{alignat*}
wobei die üblichen Konventionen $\frac{\infty}{\infty}=1$ und $\infty-1=\infty$ genutzt wurden. Damit ist $m(z_4) \in \R=m(K)$ und damit $z_4 \in K$, was zu zeigen war.
\item["$\Rightarrow$"] Seien nun $z_1,z_2,z_3,z_4$ auf einem Kreis $K$ in $\CC$ sowie $m \in \mob$ eine Möbiustransformation mit $m(K)=\RR$. Nach obiger Rechnung gilt dann $[z_1,z_2,z_3,z_4]= m(z_4) \in \R$, woraus die Behauptung schließlich folgt. $\hfill \Box$

\end{compactenum}


\end{sol}

\end{prob}









\newpage




\titlespacing*{\section}{-16.5pt}{0pt}{20pt}
\renewcommand*\thesection{}
\section{Übungsblatt 3} %PARAGRAPH A.3
\renewcommand*\thesection{\arabic{section}}







\begin{prob}    %%AUFGABE 3.1
Es sei 
$$C: \CC \la \CC, \qquad z \mapsto \ \begin{cases} \ \overline{z} & \textrm{ falls }z \in \C, \\ \ \infty & \textrm{ falls }z=\infty \end{cases}$$
die Spiegelung am Kreis $\RR$. Weiter sei $A \subseteq \CC$ ein beliebiger Kreis und $m\in \amob$ eine Möbiustransformation, welche $\RR$ auf $A$ abbildet. Die \textit{Spiegelung an} $A$ definiert man nun als
$$C_A:=m\circ C \circ m^{-1}.$$
Zeigen Sie, dass $C_A$ wohldefiniert ist, also nicht von der Wahl von $m$ abhängt.
\begin{sol}
Seien $m,n \in \amob$ Möbiustransformationen mit $m(\RR)=A=n(\RR)$. Dann ist $M:= n^{-1} \circ m \in \amob(\RR)$. Wir zeigen zunächst, dass $M$ und $C$ kommutieren. Dazu betrachten wir die 4 Formen, welche $M$ nach Korollar 1.7.2 annehmen kann.
\begin{compactenum}
\item Sei zunächst $M(z)=\frac{az+b}{cz+d}$ mit Koeffizienten $a,b,c,d \in \R$ und $ad-bc=1$. Dann ist 
$$(C\circ M)(z)=\overline{\left( \frac{az+b}{cz+d}\right)}= \frac{a\overline{z}+b}{c\overline{z}+d} = (M \circ C)(z),$$
also $C\circ M=C \circ M$.
\item Im Fall $M(z)=\frac{a\overline{z}+b}{c\overline{z}+d}$ mit $a,b,c,d \in \R$ und $ad-bc=1$ folgt 
$$(C\circ M)(z)=\overline{\left(\frac{a\overline{z}+b}{c\overline{z}+d}\right)}= \frac{az+b}{cz+d} = (M\circ C)(z).$$
\item Ist $M(z)=\frac{az+b}{cz+d}$ mit reinimaginären Koeffizienten $a,b,c,d \in \C \setminus \R^{\times}$ und $ad-bc=1$, so folgt
$$(C\circ M)(z)=\overline{\left(\frac{az+b}{cz+d}\right)}= \frac{-a\overline{z}-b}{-c\overline{z}-d} = (M \circ C)(z).$$
\item Ist schließlich $M(z)=\frac{a\overline{z}+b}{c\overline{z}+d}$ mit $a,b,c,d \in \C \setminus \R^{\times}$ und $ad-bc=1$, so erhalten wir
$$(C\circ M)(z)=\overline{\left(\frac{a\overline{z}+b}{c\overline{z}+d}\right)}= \frac{-az-b}{-cz-d} = (M\circ C)(z).$$

\end{compactenum}
Damit erhalten wir mit $m=n \circ M$
$$m \circ C \circ m^{-1}= n \circ M \circ C \circ (n \circ M)^{-1} = n \circ C \circ M \circ M^{-1} \circ n^{-1} = n \circ C \circ n^{-1},$$
die Definition der Spiegelung an $A$ ist also unabhängig von der Wahl von $m$. $\hfill \Box$

\end{sol}

\end{prob}






\hypertarget{Adreizwei}{}
\begin{prob}    %%AUFGABE 3.2
Beweisen Sie \hyperlink{propeinssechsacht}{Proposition 1.6.8}:
Es sei $\amob:= \langle \mob,C\rangle$ die allgemeine Möbiusgruppe, wobei $C$ die Spiegelung am Kreis $\RR$ bezeichne. Zeigen Sie, dass sich jedes Element von $\amob$ als Komposition endlich vieler Spiegelungen an Kreisen in $\CC$ schreiben lässt.
\textit{(Hinweis: Es genüg dies für die Transformationen $C$, $\mathcal{J}$ und $f(z)=az+b$ mit $a,b \in \C, a \neq 0$ zu zeigen. Warum?)}

\begin{sol}
Es ist $C(z)=\overline{z}$ bereits eine Spiegelung sowie
$$\mathcal{J}(z)=(C\circ \overline{\mathcal{J}})(z)$$
mit der Spiegelung am Einheitskreis $\overline{\mathcal{J}}(z)=\frac{1}{\overline{z}}$ ebenfalls Komposition von Spiegelungen an Kreisen. Es bleibt also noch die Aussage für $f$ zu zeigen. Hierfür gehen wir in Polarkoordinaten $a=r_1 e^{i \phi_1}$, $b=r_2 e^{i\phi_2}$ und nutzen, dass sich $f$ als Komposition einer Rotation, Streckung und Translation schreiben lässt. Die Rotation ist gegeben durch 
$f_1(z)= e^{i\phi_1} z.$ Setze
$$m_2(z):= e^{i \frac{\phi_1}{2}} z, \qquad m_2^{-1}(z)=e^{-i\frac{\phi_1}{2}} z.$$
Dann ist
$$\left(m_2 \circ C \circ m_2^{-1}\right)(z)=e^{i\frac{\phi_1}{2}} \left( e^{i\frac{\phi_1}{2}} z \right) = e^{i\phi_1} z = f_1(z),$$
$f_1$ ist also eine Spiegelung. Die Streckung ist gegeben durch $f_2(z)=r_1z$. Betrachte die Spiegelung am Einheitskreis
$$m_1(z)=\frac{iz+1}{z+i}, \qquad m_1^{-1}(z)=\frac{iz-1}{-z+i}$$
sowie
$$m_3(z)=\frac{i \sqrt{r_1}z + \sqrt{r_1}}{z+i}, \qquad m_3^{-1}(z)=\frac{iz-\sqrt{r_1}}{-z+i\sqrt{r_1}}.$$
Dann gilt
\begin{alignat*}{5}
\left(m_3 \circ C \circ m_3^{-1} \circ m_1 \circ C \circ m_1^{-1}\right)(z) \ \ &=&& \ \ \left(m_3 \circ C \circ m_3^{-1}\right)\left( \frac{i \frac{-i\overline{z}-1}{-\overline{z}-i} +1}{\frac{-i\overline{z}-1}{-\overline{z}-i} +i}\right)  \\
&=&& \ \ (m_3 \circ C \circ m_3^{-1})\left( \frac{\overline{z}-i-\overline{z}-i}{-i\overline{z}-1-i\overline{z}+1}\right) \\
&=&& \ \ (m_3 \circ C \circ m_3^{-1})\left( \frac{1}{\overline{z}}\right) \\
&=&& \ \ (m_3 \circ C) \left( \frac{\frac{i}{\overline{z}} - \sqrt{r_1}}{-\frac{1}{\overline{z}} + i \sqrt{r_1}}\right) \\
&=&& \ \ (m_3 \circ C) \left(\frac{i-\sqrt{r_1} \overline{z}}{-1+i \sqrt{r_1}\overline{z}} \right) \\
&=&& \ \ m_3 \left( \frac{-i - \sqrt{r_1}z}{-1-i \sqrt{r_1}z} \right) \\
&=&& \ \ \frac{i \sqrt{r_1} \frac{-i-\sqrt{r_1}z}{-1-i\sqrt{r_1}z} + \sqrt{r_1}}{\frac{-i-\sqrt{r_1}z}{-1-i\sqrt{r_1}z} + i} \\
&=&& \ \ \frac{\sqrt{r_1}-ir_1z  - \sqrt{r_1} - i r_1 z}{-i - \sqrt{r_1}z - i + \sqrt{r_1}z} \\
&=&& \ \ r_1z\\
&=&& \ \ f_2(z),
\end{alignat*}
also ist auch die Streckung Komposition zweier Spiegelungen. Betrachte nun noch die Translation $f_3(z)=z+b=z+r_2e^{i\phi_2}$. Setze dazu
$$m_4(z)= e^{i \left(\phi_2+\frac{\pi}{2}\right)} z, \qquad m_4^{-1}(z)=e^{-i\left( \phi_2 + \frac{\pi}{2}\right)} z$$
und 
$$m_5(z)=e^{i\left(\phi_2 + \frac{\pi}{2}\right)} z + \frac{r_2}{2} e^{i\phi_2}, \qquad m_5^{-1}(z)=\frac{z-\frac{r_2}{2} e^{i\phi_2}}{e^{i\left(\phi_2 + \frac{\pi}{2}\right)}}.$$
Dann folgt
\begin{alignat*}{5}
\left(m_5 \circ C \circ m_5^{-1} \circ m_4 \circ C \circ m_4^{-1}\right)(z) \ \ &=&& \ \ (m_5 \circ C \circ m_5^{-1})\left( e^{i\left(\phi_2 + \frac{\pi}{2}\right)} \left( e^{i\left(\phi_2+ \frac{\pi}{2}\right)} \overline{z} \right) \right)\\
&=&& \ \ (m_5 \circ C \circ m_5^{-1})\left(-e^{2i\phi_2} \overline{z}\right) \\
&=&& \ \ (m_5 \circ C)\left(\frac{-e^{2i\phi_2}\overline{z} - \frac{r_2}{2} e^{i\phi_2}}{e{i\left(\phi_2 + \frac{\pi}{2}\right)}}\right) \\
&=&& \ \ m_5 \left( \frac{-e^{-2i\phi_2} z - \frac{r_2}{2} e^{-i\phi_2}}{e^{-i\left(\phi_2+\frac{\pi}{2}\right)}}\right) \\
&=&& \ \ e^{i\left(\phi_2+\frac{\pi}{2}\right)} \left(\frac{-e^{-2i\phi_2} z - \frac{r_2}{2} e^{-i\phi_2}}{e^{-i\left(\phi_2+\frac{\pi}{2}\right)}}\right) + \frac{r_2}{2} e^{i\phi_2} \\
&=&& \ \ -e^{2i\phi_2}\left( - e^{-2i\phi_2}z - \frac{r_2}{2} e^{-i\phi_2}\right) + \frac{r_2}{2} e^{i\phi_2}\\
&=&& \ \ z + r_2 e^{i\phi_2} \\
&=&& \ \ f_3(z).
\end{alignat*}
und auch $f_3$ ist eine Verkettung von Spiegelungen an Kreisen. Insgesamt kann also $m=f_3 \circ f_2 \circ f_1$ als Komposition endlich vieler Spiegelungen an Kreisen von $\CC$ geschrieben werden, was zu zeigen war. $\hfill \Box$




\end{sol}


\end{prob}


\newpage

\begin{prob}   %%AUFGBE 3.3
Sei $\mathrm{SL}_2(\R):=\{A \in \R^{2\times2} \ \vert \ \det A=1\}$ die spezielle lineare Gruppe. Zeigen Sie, dass $\mathrm{SL}_2(\R)$ von den Matrizen der Form 
$$T_x:=\begin{pmatrix}[rr]1 & x \\[-6pt] 0 & 1 \end{pmatrix}, \qquad S:=\begin{pmatrix}[rr]0 & -1 \\[-6pt] 1 & 0 \end{pmatrix}$$
für $x \in \R$ erzeugt wird.

\begin{sol}
Zunächst stellen wir fest, dass
$$T_x T_y = \tx{x}\tx{y} = \tx{x+y}$$
und 
$$S^2=\matx{0}{-1}{1}{0}^2 = \matx{-1}{0}{0}{-1}, \qquad S^3 = S^{-1}.$$
Weiter ist für $x,y,z \in \R$
$$T_xST_YST_z =  \tx{x} \matx{0}{-1}{1}{0} \tx{y} \matx{0}{-1}{1}{0} \tx{z} = \matx{1-xy}{xyz-x-z}{y}{1-yz}.$$
Es sei also 
$$A=\begin{pmatrix}[rr]a & b \\[-6pt] c & d \end{pmatrix} \in \mathrm{SL}_2(\R)$$
beliebig mit $ad-bc=1$. 
\begin{compactenum}
\item[\textbf{Fall (a)}] Sei zunächst $c \neq 0$. Dann wähle $y=c$ und erhalte
$$a=1-xy = 1-xc, \qquad d=1-yz=1-cz,$$
also $x=\frac{1-a}{c}$ und $\frac{1-d}{c}$. Für $b$ gilt dann automatisch wegen $ad-bc=1$
$$b=\frac{ad-1}{c} = \frac{(1-xy)(1-yz)-1}{y} = \frac{1-xy-yz+xy^2z-1}{y}=xyz-x-z,$$
es gilt also 
\begin{alignat*}{5}
A\ \ &=&& \ \ \matx{a}{b}{c}{d}=T_{\frac{1-a}{c}} S T_c S T_{\frac{1-d}{c}} \\
&=&& \ \ \tx{\frac{1-a}{c}} \matx{0}{-1}{1}{0} \tx{c} \matx{0}{-1}{1}{0} \tx{\frac{1-d}{c}}.
\end{alignat*}

\item[\textbf{Fall (b)}] Sei nun $c=0$. Dann gilt $1=ad-bc=ad$, also insbesondere $a\neq 0$. Wegen 
$$SA= \matx{0}{-1}{1}{0} \matx{a}{b}{c}{d} = \matx{0}{-d}{a}{b}$$
ist 
$$A=S^{-1} \matx{0}{-d}{a}{b} = S^3\matx{0}{-d}{a}{b} \overset{(a)}{=} S^3 T_{\frac{1}{a}} S T_a S T_{\frac{1b}{a}},$$
wobei Fall $(a)$ genutzt wurde. $\hfill \Box$

\end{compactenum}


\end{sol}

\end{prob}






\hypertarget{Adreivier}{}
\begin{prob} %%AUFGABE 3.4
Es sei $\sigma: \CC \la \CC$, $z \mapsto - \overline{z}$ die Spiegelung an der imaginären Achse. Zeigen Sie, dass gilt:
$$\amob(\He) \cong \mathrm{PSL}_2(\R) \cup \left( \mathrm{PSL}_2(\R) \circ \sigma \right).$$
\begin{sol}
Siehe Korollar 1.7.4 $\hfill \Box$
\end{sol}
\end{prob}


















\newpage




\titlespacing*{\section}{-16.5pt}{0pt}{20pt}
\renewcommand*\thesection{}
\section{Übungsblatt 4} %PARAGRAPH A.4
\renewcommand*\thesection{\arabic{section}}


\begin{prob}   %%AUFGABE 4.1
\begin{compactenum}
\item Bestimmen Sie all Möbiustransformationen $m \in \amob(\He)$ mit $m(i)=i$.
\item Bestimmen Sie all Möbiustransformationen $m \in \amob(\He)$, welche die Parallelen zur imaginären Achse auf solche abbilden.
\end{compactenum}

\begin{sol}
\begin{compactenum}
\item Nach Satz 1.7.3 kann $m$ zwei Gestalten annehmen. Wir untersuchen beide getrennt:
\begin{compactenum}
\item Sei zunächst $m(z)=\frac{az+b}{cz+d}$ mit $a,b,c,d \in \R$ und $ad-bc=1$. Dann erhalten wir
\begin{alignat*}{5}
i=m(i)=\frac{ai+b}{ci+d} &=&& \ \  \frac{(ai+b)(-ci+d)}{(ci+d)(-ci+d)}\\  &=&& \ \  \frac{1}{c^2+d^2} \left( ac+bd + i(ad-bc)\right) \\ &=&& \ \  \frac{ac+bd}{c^2+d^2} + i \ \frac{1}{c^2+d^2}.
\end{alignat*}
Koeffizientenvergleich liefert $c^2+d^2=1$ und $ac+bd=0$, also 
$$0=acd+bd^2 = (1+bc)c+bd^2 = c + b(c^2+d^2)=b+c$$
und damit $b=-c$. Dies impliziert wegen $1=ad-bc=ad+c^2$ auch $a=d$ und somit 
$$\amob(\He)_i:= \mathrm{Stab}_{\amob(\He)}(i)\supseteq \left\{ \matx{d}{c}{-c}{d} \ \bigg\vert \ c,d \in \R, \ c^2+d^2=1 \right\} \cong \mathcal{O}(2).$$

\item Ist nun $m(z)= \frac{a\overline{z}+b}{c\overline{z}+d}$ mit $a,b,c,d \in \C \setminus \R^{\times}$ und $ad-b=1$, so liefert die Forderung
\begin{alignat*}{5}
i=m(i)=\frac{-ai+b}{-ci+d} &=&& \ \  \frac{(-ai+b)(-ci-d)}{(ci+d)(-ci-d)}\\  &=&& \ \  -\frac{1}{c^2+d^2} \left( -ac-bd + i(ad-bc)\right) \\ &=&& \ \  \frac{ac+bd}{c^2+d^2} - i \ \frac{1}{c^2+d^2}.
\end{alignat*}
Koeffizientenvergleich liefert $c^2+d^2=-1$ und analog wie oben $b=-c$. Damit ist 
$$\amob(\He)_i\supseteq \left\{ \matx{d}{c}{-c}{d} \ \bigg\vert \ c,d \in \C \setminus \R^{\times}, \ c^2+d^2=-1 \right\} \cong i\ \mathcal{O}(2)$$
\end{compactenum}
Insgesamt folgt damit
$$\amob(\He)_i= \mathcal{O}(2) \cup i\ \mathcal{O}(2)$$

\item Wir verfahren analog wie in (i) und betrachten die beiden Fälle getrennt.
\begin{compactenum} 
\item Sei $m(z)=\frac{az+b}{cz+d}$ mit $a,b,c,d \in \R$ und $ad-bc=1$. Betrachte die Gerade $g: \mathrm{Re}(z)=0$. Dann sind $0,i \in g$ und 
$$m(0)=\frac{b}{d}, \qquad m(i)= \frac{ai+b}{ci+d} = \frac{bd+ac+i}{c^2+d^2}.$$
Damit $m(g)$ auch parallel zur imaginären Achse ist, genügt es, dass $\mathrm{Re}(m(i))= \mathrm{Re}(m(0))$ gilt. Dies liefert
$$\mathrm{Re}(m(0))=\frac{b}{d}\overset{!}{=} \frac{bd+ac}{c^2+d^2} = \mathrm{Re}(m(i)),$$
also 
$$b=\frac{bd^2+acd}{c^2+d^2} = \frac{bd^2+c(bc+1)}{c^2+d^2} = \frac{b(c^2+d^2)+c}{c^2+d^2} = b + \frac{c}{c^2+d^2}$$
und damit $c=0$. Damit ist wegen $1=ad-bc=ad$ somit $d=\frac{1}{a}$ und $m(z)=a^2z+ab$. Wir zeigen nun noch, dass dieser Spezialfall bereits ausreicht. Da Geraden auf Geraden abbilden, wird auch jeder weitere Punkt in $g$ auf die Gerade durch $m(0)$ und $m(i)$ abgebildet. Sind nun $\lambda, \mu \in\R$ beliebig, $h: \mathrm{Re}(z)=\lambda$ eine beliebige Gerade, so gilt für die Punkte $\lambda, \lambda + i \mu$ auf $h$
$$\mathrm{Re}(m(\lambda))=\mathrm{Re}(a^2\lambda + ab)=a^2\lambda + ab = \mathrm{Re}(a^2(\lambda + i \mu) + ab) =\mathrm{Re}(m(\lambda + i \mu)).$$
Insgesamt haben wir also gezeigt, dass
$$\amob(\He)_P \supseteq \left\{ \matx{a}{b}{0}{\frac{1}{a}} \ \bigg \vert \ a,b \in \R, \ a \neq 0 \right\},$$
wobei wir mit $\amob(\He)_P$ jene Möbiustransformationen bezeichnen, welche Parallelen zur imaginären Achse auf solche abbilden.

\item Ist nun $m(z)= \frac{a\overline{z}+b}{c\overline{z}+d}$ mit $a,b,c,d \in \C \setminus \R^{\times}$ und $ad-bc=1$, so gilt 
$$m(0)= \frac{b}{d}, \qquad m(i)= \frac{-ai+b}{-ci+d} = \frac{(-ai+b)(-ci-d)}{-ci+d)(-ci-d)} = -\frac{-ac-bd+i}{c^2+d^2} = \frac{ad+bc-i}{c^2+d^2},$$
es folgt also dieselbe Rechnung wie in (1).
\end{compactenum}
Damit ist insgesamt 
$$\amob(\He)_P = \left\{ \matx{a}{b}{0}{\frac{1}{a}} \ \bigg \vert \ a,b \in \R, \ a \neq 0 \right\},$$
wir sind also fertig.$\hfill \Box$



\end{compactenum}
\end{sol}


\end{prob}

\newpage

\begin{prob}    %%AUFGABE 4.2
\begin{compactenum}
\item Bestimmen Sie die Gruppe aller Möbiustransformationen, die $\Sph^1$ invariant lassen.
\item Bestimmen Sie die Gruppe aller Möbiustransformationen, die $\D^2$ invariant lassen.

\end{compactenum}


\begin{sol}
\begin{compactenum}
\item Betrachte die Möbiustransformation $n(z)=\frac{iz+1}{z+i}$ mit Inverser $n^{-1}(z)=\frac{-iz+1}{z-i}$. Es gilt $n(\RR)=\Sph^1$. Für eine Möbiustransformation $p \in \amob$ gilt also $p(\Sph^1) = \Sph^1$ genau dann, wenn $(n^{-1}\circ p \circ n)(\RR)=\RR$ gilt, also wenn $p=n\circ m \circ n^{-1}$ für ein $m \in \amob(\RR)$ gilt. In der Tat liefert eine weitere Mlbiustransformation $\tilde{n} \in \amob(\He)$ mit $\tilde{n}(\RR)=\Sph^1$ die Mengengleichheit
$$A:=\{n \circ m \circ n^{-1} \ \vert \ m \in \amob(\RR)\} = \{ \tilde{n} \circ m \tilde{n}^{-1} \ \vert \ m \in \amob(\RR) \}=:\tilde{A},$$
denn wegen $(n^{-1} \circ \tilde{n})(\RR)=\RR$ können wir $n^{-1}\circ \tilde{n} =:s \in \amob(\RR)$ schreiben. Damit ist für $m \in \amob(\RR)$, also $n \circ m \circ n^{-1} \in A'$ auch
$$n \circ m \circ n^{-1}=\tilde{n} \circ s^{-1} \circ m \circ s \circ \tilde{n} = \tilde{n} \circ (s^{-1} \circ m \circ s) \circ \tilde{n} = \tilde{n} \circ \tilde{m} \circ \tilde{n}^{-1} \in A'$$
und genauso umgekehrt. Damit ist 
$$\amob(\Sph^1) = \left\{ \left( z \mapsto \frac{iz+1}{z+i}\right) \circ m \circ \left( z \mapsto \frac{-iz+1}{z-i}\right) \ \bigg \vert \ m \in \amob(\RR) \right\}$$
und wir können die exakte Gestalt einer solchen Möbiustransformation berechnen, indem wir die 4 möglichen Gestalten von $m \in \amob(\RR)$ betrachten.
\begin{compactenum}
\item Sei zunächst $m(z)=\frac{az+b}{cz+d}$ mit $a,b,c,d \in \R$ und $ad-bc=1$. Dann ist die zu $p=n \circ m \circ n^{-1}$ zugehörige Matrix
$$\matx{i}{1}{1}{i} \matx{a}{b}{c}{d} \matx{-i}{1}{1}{-i} = \matx{a+d+i(b-c)}{b+c+i(a-d)}{b+c+i(d-a)}{a+d+i(c-)},$$
also 
$$p(z)= \frac{(a+d+i(b-c))z + b+ c+ i(a-d)}{(b+c+i(d-a))z + a+ d+ i (c-b)} =: \frac{\alpha z + \beta}{\overline{\beta} z + \overline{\alpha}}.$$
\item Sei $m(z)=\frac{az+b}{cz+d}$ mit $a,b,c,d \in \C \setminus \R^{\times}$ und $ad-bc=1$. Dann ist genau so 
$$p(z)= \frac{(a+d+i(b-c))z + b+ c+ i(a-d)}{(b+c+i(d-a))z + a+ d+ i (c-b)} =: \frac{\alpha z + \beta}{-\overline{\beta} z - \overline{\alpha}}.$$
\item Ist $m(z)=\frac{a\overline{z}+b}{c\overline{z}+d}$ mit $a,b,c,d \in \R$ und $ad-bc=1$, so ist
$$(m \circ n^{-1})(z) = m\left(\frac{-iz+1}{z-i}\right) = \frac{a \overline{\left( \frac{-iz+1}{z-i}\right)} + b}{c \overline{\left( \frac{-iz+1}{z-i}\right)} + d} = \frac{(ai+b) \overline{z} + a+bi}{(ci+d)\overline{z} + c + di}$$
und die zu $p=n\circ m \circ n^{-1}$ zugehörige Matrix ist
$$\matx{i}{1}{1}{i} \matx{ai+b}{a+bi}{ci+d}{c+di} = \matx{d-a+i(b+d)}{c-b+i(a+d)}{b-c+i(a+d)}{a-d+i(b+d)}$$
und damit mit 
$$\gamma:= d-a + i(b+d), \qquad \delta := c-b + i (a+d)$$
schließlich 
$$p(z)=\frac{(d-a+i(b+d))\overline{z} + c-b+i(a+d)}{(b-d+i(a+d))\overline{z}+a-d+i(b+d)} = \frac{\gamma \overline{z} + \delta}{-\overline{\gamma} \overline{z} - \overline{\delta}}.$$
\item Ist $m(z)=\frac{a\overline{z}+b}{c\overline{z}+d}$ mit $a,b,c,d \in \C \setminus \R^{\times}$ und $ad-bc=1$, so folgt analog 
$$p(z)=\frac{(d-a+i(b+d))\overline{z} + c-b+i(a+d)}{(b-d+i(a+d))\overline{z}+a-d+i(b+d)} = \frac{\gamma \overline{z} + \delta}{\overline{\gamma} \overline{z} + \overline{\delta}}.$$
\end{compactenum}
\item Jede Möbiustransformation, welche $\D^2$ invariant lässt, bildet $\Sph^1$ auf $\Sph^1$ ab, wir müssen also llediglich $n$ in (i) so wählen, dass $n(\He)=\D^2$. Die erfüllt beispielsweise 
$$n(z)=\frac{z-i}{-iz+1}, \qquad n^{-1}(z)=\frac{-z-i}{-iz-1}.$$
Dann gilt 
$$\amob(\D^2)=\left\{n \circ m \circ n^{-1} \ \vert \ m \in \amob(\He)\right\}.$$
\end{compactenum}

\end{sol}


\end{prob}



\hypertarget{Avierdrei}{}
\begin{prob}    %%AUFGABE 4.3
Führen Sie den Beweis von \hyperlink{satzeinsachteins}{Satz 1.8.1} zu Ende: Es seien $X_1,X_2 \subseteq \C$ sich zwei schneidene euklidische Geraden.
\begin{compactenum}
\item Zeigen Sie, dass $\mathcal{J}(z)=\frac{1}{z}$ konform ist, indem Sie die in der Vorlesung offen gelassenen Fälle
\begin{compactenum}
\item $\mathcal{J}(X_1), \mathcal{J}(X_2)$ sind zwei euklidische Geraden.
\item $\mathcal{J}(X_1), \mathcal{J}(X_2)$ sind ein euklidischer Kreis und eine euklidische Gerade.
\end{compactenum}
beweisen.
\item Zeigen Sie, dass $C(z)=\overline{z}$ konform ist.
\end{compactenum}


\begin{sol}


\begin{compactenum}

\item Wir betrachten zunächst einen allgemeinen Kreis $X: \alpha z \overline{z} + \beta z + \overline{\beta}\overline{z} + \gamma=0$ in $\C$. Dann ist $X$ eine euklidische Gerade genau dann, wenn $\alpha=0$ und ein euklidischer Kreis genau dann, wenn $\alpha \neq0$. Ist $X$ eine Gerade, also $\alpha=0$, so ist $\mathcal{J}(X)$ gegeben durch $\beta \overline{z} + \overline{\beta}z + \gamma z \overline{z}=0$, also ist $\mathcal{J}(X)$ ein euklidischer Kreis genau dann, wenn $\gamma=0$ (also $0 \in X$) und eine Gerade, falls $\gamma=0$ (also $0\notin X$). Ist hingegen $X$ ein euklidischer Kreis, so ist $\mathcal{J}(X)$ gegeben durch $\alpha + \beta \overline{z} + \overline{\beta} z + \gamma z\overline{z} =0$, also ist $\mathcal{J}(X)$ ein euklidischer Kreis, falls $\gamma \neq 0$ und eine Gerade, falls $\gamma=0$. Nun zur Aufgabe: Seien $X_1,X_2$ sich im Punkt $z_0$ schneidende euklidische Geraden der Steigung $s_1,s_2$. In der Vorlesung haben wir gesehen, dass der Schnittwinkel in $z_0$ gerade 
$$\sphericalangle_{z_0}(X_1,X_2) = \arctan s_2 - \arctan s_1$$
ist, wobei aus den Gleichungen $X_k: \beta_k z + \overline{\beta}_k \overline{z} + \gamma_k=0$ die Steigungen $s_k$ als $s_k= \frac{\mathrm{Re}(\beta_k)}{\mathrm{Im}(\beta_k)}$ berechnet werden.
\begin{compactenum}
\item Sind $\mathcal{J}(X_1)$ und $\mathcal{J}(X_2)$ euklidische Geraden, so gilt $\gamma_k=0$ und die Bildkreise sind gegeben durch $\mathcal{J}(X_k): \overline{\beta}_k z + \beta_k \overline{z} =0$. Die Bildsteigungen sind dann
$$\tilde{s}_k=\frac{\mathrm{Re}(\overline{\beta})}{\mathrm{Im}(\overline{\beta})} = - \frac{\mathrm{Re}(\beta_k)}{\mathrm{Im}(\beta_k)} = - s_k = - \arctan \theta_k,$$
woraus also 
$$\sphericalangle_{\mathcal{J}(z_0)}(\mathcal{J}(X_1), \mathcal{J}(X_2)) = \arctan \tilde{s}_2 - \arctan \tilde{s}_1 = \theta_1 - \theta_2 = - \sphericalangle_{z_0}(X_1,X_2)$$
folgt.
\item Ist $X_1$ ein euklidischer Kreis und $X_2$ euklidische Gerade, so sind die Geradengleichungen $X_1: \beta_1 z + \overline{\beta}_1 \overline{z} + \gamma_1=0$ und $X_2: \beta_2 z + \overline{\beta}_2 \overline{z} =0$. Weiter hat $\mathcal{J}(X_1)$ die Steigung (Siehe Proposition 1.8.1) gerade
$$\tilde{s}_1=-s_1 = -\arctan \theta_1$$
sowie die Steigung von $\mathcal{J}(X_2)$ nach (1) gerade
$$\tilde{s}_2=-s_2= - \arctan \theta_2.$$
Damit folgt wie in (1) die Behauptung.
\end{compactenum}

\item Für eine Gerade $g: \beta z + \overline{\beta}\overline{z} + \gamma=0$ ist $C(g):\beta \overline{z} +  \overline{\beta} z + \gamma=0$, das Bild also wieder eine Gerade. Genauso bildet $C$ euklidische Kreise auf solche ab. Daraus folgt bereits die Konformität von $C$. $\hfill \Box$

\end{compactenum}

\end{sol}

\end{prob}







\newpage




\titlespacing*{\section}{-16.5pt}{0pt}{20pt}
\renewcommand*\thesection{}
\section{Übungsblatt 5} %PARAGRAPH A.5
\renewcommand*\thesection{\arabic{section}}



\begin{prob}   %%AUFGABE 5.1
Es sei $M:=\R^2$ und $d_{KVV}:M \times M \la \R$ die durch 
$$d_{KVV}((x_1,x_2),(y_1,y_2))  \ := \ \begin{cases} \ \vert x_2\vert + \vert x_1-y_1\vert + \vert y_2\vert & \textrm{ falls } x_1 \neq y_1, \\ \ \vert x_2 - y_2\vert & \textrm{ falls }x_1 = y_1 \end{cases}$$
definierte Abbildung.
Zeigen Sie, dass $(M,d_{KVV})$ ein metrischer Raum ist. Überlegen Sie sich, woher der Name kommt.

\begin{sol}Es sind die Axiome einer Metrik zu zeigen.
\begin{compactenum}
\item Da $\vert \cdot \vert$ symmetrisch ist, gilt $\vert x_1-y_1\vert = \vert y_1 - x_1\vert$ und $\vert x_2 - y_2 \vert = \vert y_2 - x_2\vert$ für alle $(x_1,x_2), (y_1,y_2) \in \R^2$ und damit
$$d_{KVV}((x_1,x_2),(y_1,y_2)) = d_{KVV}((y_1,y_2),(x_1,x_2)).$$
\item Da $\vert x\vert \geqslant 0$ für alle $x\in \R^2$ ist $d_{KVV}((x_1,x_2),(y_1,y_2)) \geqslant 0$ für all $(x_1,x_2), (y_1,y_2) \in \R^2$. Weiter ist $d_{KVV}((x_1,x_2),(x_1,x_2)) = 0$ für alle $(x_1,x_2) \in \R^2$. Ist nun umgekehrt $d_{KVV}((x_1,x_2),(y_1,y_2))=0$, so führt der Fall $x_1 \neq y_1$ zum Widerspruch $$0=d_{KVV}((x_1,x_2),(y_1,y_2))=\vert x_2\vert + \vert x_1-y_1\vert+\vert y_2\vert \geqslant \vert x_1-y_1\vert >0$$ und im Fall $x_1=y_1$ folgt $x_2-y_2=0$, also $(x_1,x_2)=(y_1,y_2)$. Insgesamt ist $d_{KVV}$ damit positiv definit.

\item Es bleibt noch die Dreiecksungleichung zu zeigen. Seien dazu $(x_1,x_2), (y_1,y_2), (z_1,z_2) \in \R^2$ beliebig. Wir müssen einige Fallunterscheidungen treffen.
\begin{compactenum}
\item[\textbf{Fall (1)}] Sei zunächst $x_1\neq z_1$. 
\begin{compactenum}

\item[\textbf{Fall (1.1)}] Sei $y_1 \neq x_1$ und $y_1 \neq z_1$. Dann ist
\begin{alignat*}{5}
d_{KVV}((x_1,x_2), (y_1,y_2)) \ \ &=&& \ \ \vert x_2\vert + \vert x_1-z_1\vert + \vert z_2 \vert \\
&=&& \ \ \vert x_2 \vert + \vert x_1-y_1+y_1-z_1\vert + \vert z_2\vert \\
&\leqslant&& \ \ \vert x_2\vert + \vert x_1 - y_1\vert + \vert y_1-z_1\vert + \vert z_2\vert \\
&\leqslant&& \ \ \vert x_2 \vert + \vert x_1-y_1\vert + 2 \vert y_2\vert + \vert y_1-z_1\vert + \vert z_2\vert \\
&=&& \ \ d_{KVV}((x_1,x_2),(y_1,y_2)) + d_{KVV}((y_1,y_2),(z_1,z_2)).
\end{alignat*}

\item[\textbf{Fall (1.2)}] Sei $y_1=x_1$ und $y_1\neq z_1$. Dann ist
\begin{alignat*}{5}
d_{KVV}((x_1,x_2),(z_1,z_2)) \ \ &=&& \ \ \vert x_2 \vert + \vert x_1-z_1\vert + \vert z_2\vert \\
&=&& \ \ \vert x_2-y_2+y_2\vert + \vert x_1-z_1\vert + \vert z_2\vert \\
&\leqslant&& \ \ \vert x_2-y_2\vert + \vert y_2\vert + \vert x_1-z_1\vert + \vert z_2\vert \\
&=&& \ \ \vert x_2 - y_2 \vert + \vert y_2\vert + \vert y_1-z1\vert + \vert z_2\vert \\
&=&& \ \ d_{KVV}((x_1,x_2),(y_1,y_2)) + d_{KVV}((y_1,y_2),(z_1,z_2)).
\end{alignat*}
\item[\textbf{Fall (1.3)}] Ist $y_1\neq x_1$ und $y_1=z_1$, so folgt analog
\begin{alignat*}{5}
d_{KVV}((x_1,x_2),(z_1,z_2)) \ \ &=&& \ \ \vert x_2\vert + \vert x_1-z_1\vert + \vert z_2\vert \\
&=&& \ \ \vert x_2\vert + \vert x_1-y_1\vert + \vert z_2-y_1+y_2\vert \\
&\leqslant&& \ \ \vert x_2\vert + \vert x_1-y_1\vert + \vert y_2\vert + \vert z_2-y_2\vert\\
&=&& \ \ d_{KVV}((x_1,x_2),(y_1,y_2)) + d_{KVV}(y_1,y_2),(z_1,z_2)).
\end{alignat*}
\end{compactenum}
\item[\textbf{Fall (2)}]Sei nun $x_1=z_1$. Dann gibt es zwei Unterfälle:
\begin{compactenum}
\item[\textbf{Fall (2.1)}] Ist $y_1=x_1$ (und damit auch $y_1=z_1$), so gilt
\begin{alignat*}{5}
d_{KVV}((x_1,x_2),(z_1,z_2)) \ \ &=&& \ \ \vert x_2-z_2\vert \\
&\leqslant&& \ \ \vert x_2 - y_2\vert + \vert y_2-z_2\vert \\
&=&& \ \ d_{KVV}((x_1,x_2),(y_1,y_2)) + d_{KVV}((y_1,y_2),(z_1,z_2)).
\end{alignat*}
\item[\textbf{Fall (2.2)}] Ist $y_1\neq x_1$ (und damit auch $y_1\neq z_1$), so erhalten wir
\begin{alignat*}{5}
d_{KVV}((x_1,x_2),(z_1,z_2)) \ \ &=&& \ \ \vert x_2 - z_2\vert \\
&\leqslant&&\ \  \vert x_2\vert + \vert z_2\vert \\
&\leqslant&& \ \  \vert x_2\vert + \vert x_1-y_1\vert + \vert y_2\vert + \vert y_2\vert + \vert y_1-z_1\vert + \vert z_2\vert \\
&=&& \ \ d_{KVV}((x_1,x_2),(y_1,y_2)) + d_{KVV}((y_1,y_2),(z_1,z_2)).
\end{alignat*}
\end{compactenum}
\end{compactenum}
Damit gilt in allen Fällen die Dreiecksungleichung, womit $d_{KVV}$ zu einer Metrik und $(M,d_{KVV})$ zum metrischen Raum wird.
\end{compactenum}
Zu zwei gegebenen Punkten in der Ebene misst die Metrik die Länge des Weges, wenn man senkrecht von einem Punkt zur $x$-Achse, dann entlang der $x$-Achse zum Lotfußpunkt des anderen Punktes und anschließend senkrecht zum zweiten Punkt entlang läuft. Dieser Weg erinnert an das Straßenbahnnetz in Karlsruhe, wo (fast) all Bahnlinien entlang der Kaiserstraße verlaufen und sich erst im Osten bzw. Westen in die verschiedenen Stadtteile abspalten. $\hfill \Box$



\end{sol}



\end{prob}




\begin{prob}   %%AUFGABE 5.2
Zeigen Sie, dass die Translationen $T(\R^2) =\{T_a: \R^2 \la \R^2, \ x \mapsto x+a \ \vert \ a \in \R^2\}$ ein Normalteiler in der euklidischen Isometriegruppe $\mathrm{Isom}(\E^2)=\mathcal{O}(2) \ltimes \R^2$ ist.

\begin{sol}
Wir wenden zunächst das Untergruppenkriterium an, um zu zeigen, dass $T(\R^2)$ eine Untergruppe ist. Sicherlich ist $T(\R^2)\neq \emptyset$, da $\id_{\E^2} \in T(\R^2)$. Weiter gilt für $a,b\in \R^2$
$$(T_a\circ T_{-b})(x)= T_a(x-b)=x+a-b = T_{a-b}(x) \in T(\R^2),$$
also ist $T(\R^2)$ eine Untergruppe (beachte, dass $T_b^{-1}=T_{-b}$). Es bleibt noch die Normalteilereigenschaft zu überprüfen. Sei dazu $\Phi_{A,a} \in \mathrm{Isom}(\E^2)$ beliebig, wobei $\Phi_{A,a}(x)=Ax+a$ ein beliebiges Element der Isometriegruppe von $\E^2$ ist. Dann gilt $\Phi_{A,a}^{-1}(x)=A^{-1}x-A^{-1}a$, denn es gilt
$$(\Phi_{A,a} \circ \Phi_{A,a}^{-1})(x)=\Phi_{A,a}\left( A^{-1}x-A^{-1}a\right) = A(A^{-1}x-A^{-1}a) +a = x-a+a=x$$
und
$$(\Phi_{A,a}^{-1}\circ \Phi_{A,a})(x)=\Phi_{A,a}\left(Ax+a\right) = A^{-1}(Ax+a) - A^{-1}a= x+a-a=x.$$
Dann gilt für jedes $b \in \R^2$
\begin{alignat*}{5}
\left(\Phi_{A,a}^{-1} \circ T_b \circ \Phi_{A,a}\right)(x) \ \ &=&& \ \ \left(\Phi_{A,a}^{-1}\circ T_b\right) \left( Ax+a\right) \\
&=&& \ \ \Phi_{A,a}^{-1}\left(Ax+a+b\right) \\
&=&& \ \ A^{-1}(Ax+a+b) - A^{-1}a \\
&=&& \ \ x+A^{-1}b \\
&=&& \ \ T_{A^{-1}b}(x),
\end{alignat*}
also $\Phi_{A,a}^{-1}\circ T_b \circ \Phi_{A,a} \in T(\R^2)$, was zu zeigen war. $\hfill \Box$

\end{sol}

\end{prob}







\begin{prob}   %%AUFGABE 5.3
Ein Längenraum ist ein topologischer Raum $X$, der hausdorffsch ist, zusammen mit einer Klasse $W$ von zulässigen Wegen in $X$ und einer Abbildung $L: W \la \R^+\cup \{\infty\}$ mit den Bedingungen
\begin{compactenum}
\item[$(w1)$] $W$ ist abgeschlossen unter Restriktionen, d.h. wenn $\gamma:[a,b] \la X$ in $W$ liegt, so auch $\gamma\vert_{[c,d]}$ für alle $a\leqslant c\leqslant d\leqslant b$.
\item[$(w2)$] $W$ ist abgeschlossen unter Verkettung, das heißt für Wege $\gamma_1:[a,b] \la X, \gamma_2:]c,d]\la X$ in $W$ mit $\gamma_1(b)=\gamma_2(c)$ ist $\gamma:=\gamma_2 \circ \gamma_1$ in $W$.
\item[$(w3)$] $W$ ist abgeschlossen unter Umparametrisierungen.
\item[$(w4)$] $W$ enthält für jeden Punkt $x \in X$ einen Weg $\gamma:[a,b]\la X$ mit $x=\gamma(t)$ für ein $t\in[a,b]$.
\end{compactenum}
und
\begin{compactenum}
\item[$(l1)$] $L$ ist additiv, das heißt für jeden Weg $\gamma:[a,c]\la X$ gilt $L(\gamma)= L(\gamma\vert_{[a,b]} + L(\gamma\vert_{[b,c]})$.
\item[$(l2)$] Für jeden Weg $\gamma:[a,b] \la X$ ist $$L(\gamma,a,\cdot): [a,b] \la \R^+ \cup \{\infty\}, \ t \mapsto L(\gamma,a,t)=L(\gamma\vert_{[a,t]})$$ eine stetige Funktion.
\item[$(l3)$] $L$ ist invariant unter Umparametrisierungen von Wegen.
\item[$(l4)$] $L$ verträgt sich mit der Topologie von $X$, das heißt für jeden Punkt $x \in X$ und jede offene Umgebung $U_x$ von $x$ gilt
$$\inf \{L(\gamma\vert_{[a,b]}) \ \vert \ \gamma \in W \textrm{  und } \gamma(a)=x, \ \gamma(b) \in X \setminus U_x \} >0.$$
\end{compactenum}
Auf einem solchen Längenraum $(X,W,L)$ kann man eine Abbildung 
$$d_L: X \times X \la \R^+ \cup \{\infty\}, \qquad (x,y)\mapsto \inf \{L(\gamma\vert_{[a,b]}) \ \vert \ \gamma \in W \textrm{ und } \gamma(a)=x, \ \gamma(b)=y\}$$
definieren. Zeigen Sie, dass $(X,d_L)$ ein metrischer Raum (mit nicht-endlicher Metrik) ist.

\begin{sol}
Für den Beweis zeigen wir die Axiome einer Metrik.
\begin{compactenum}
\item
Zur Symmetrie: Seien $x,y \in X$. Ist $d_L(x,y)=\infty$, so gibt es keinen Weg in $W$ von $x$ nach $y$, also wegen $(w3)$ auch keinen von $y$ nach $x$. Daraus ergibt sich $d_L(y,x)=\infty=d_L(x,y)$. Sei nun $d_L(x,y)<\infty$. Für jeden Weg $\gamma:[a,b] \la X$ mit $\gamma(a)=x$ und $\gamma(b)=y$ ist $\overline{\gamma}:[a,b] \la X, \ \overline{\gamma}(t)=\gamma(a+b-t)$ ein Weg von $y$ nach $x$ (exstiert nach $(w3)$). Aus $(l3)$ folgt $L(\gamma)=L(\overline{\gamma})$ und damit $d_L(y,x)\leqslant _L(x,y)$. Dasselbe Argument andersherum liefert die andere Ungleichung und damit $d_L(x,y)=d_L(y,x)$.

\item Seien $x,y,z\in X$. Wir müssen zeigen, dass $d_L$ die Dreiecksungleichung erfüllt. Es gilt 
\begin{alignat*}{5}
d_L(x,z)\ \  &=&& \ \ \inf \left\{ L(\gamma\vert_{[a,b]}) \ \big\vert \ \gamma \in W \textrm{ und }\gamma(a)=x, \ \gamma(b)=z \right\} \\
&=&& \ \ \inf \left\{ L(\gamma\vert_{[0,2]}) \ \big\vert \ \gamma \in W \textrm{ und }\gamma(0)=x, \ \gamma(2)=z \right\} \\
&\leqslant&& \ \ \inf \left\{ L(\gamma\vert_{[0,2]}) \ \big\vert \ \gamma \in W \textrm{ und }\gamma(0)=x, \ \gamma(2)=z, \ \gamma\left(1\right)=y \right\} \\
&=&& \ \ \inf \left\{ L(\gamma\vert_{[0,1]}) + L(\gamma\vert_{[1,2]}) \ \big\vert \ \gamma \in W \textrm{ und }\gamma(0)=x, \ \gamma(2)=z, \ \gamma\left(1\right)=y \right\} \\
&=&& \ \ \inf \left\{ L(\gamma\vert_{[0,1]}) \ \big\vert \ \gamma \in W \textrm{ und }\gamma(0)=x, \gamma(1)=y\right\} + \\
&&& \ \ \inf \left\{ L(\gamma\vert_{[1,2]}) \ \big\vert \ \gamma \in W \textrm{ und }\gamma(1)=y, \gamma(2)=z\right\}\\
&=&& \ \ d_L(x,y) + d_L(y,z).
\end{alignat*}
Dabei wurde in der zweiten Gleichheit die Invarianz unter Umparametrisierungen, in der dritten Gleichheit die Additivität und in der vierten Gleichheit die Abgeschlossenheit unter Kompositionen benutzt.

\item Da $L(\gamma) \geqslant 0$ ist für alle $\gamma \in W$ gilt $d_L(x,y)\geqslant 0$ für alle $x,y \in X$. Zeige nun $d_L(x,x)=0$. Wähle einen Weg $\gamma_x:[a,b] \la X$ mit $\gamma(t)=x$. Dann gilt
\begin{alignat*}{5}
L(\gamma\vert_{[a,b]}) \ \ &=&& \ \ L(\gamma\vert_{[a,t]}) + L(\gamma\vert_{[t,t]}) + L(\gamma\vert_{[t,b]})\\
&=&& \ \ L(\gamma\vert_{[a,b]}) + L(\gamma\vert_{[t,t]}),
\end{alignat*}
also 
$$d_L(x,x) \leqslant L(\gamma\vert_{[t,t]}) = L(\gamma\vert_{[a,b]}) - L(\gamma\vert_{[a,b]}) =  0$$
und damit $d_L(x,x)=0$. Seien nun umgekehrt $x \neq y$. Wir müssen zeigen, dass dann $d_L(x,y)>0$. Da $X$ hausdorffsch ist, können wir eine Umgebung $U_x$ von $x$ wählen mit $y \notin U_x$. Dann gilt 
\begin{alignat*}{5}
d_L(x,y) \ \ &=&& \ \  \inf \left\{ L(\gamma\vert_{[a,b]}) \ \big\vert \ \gamma \in W \textrm{ und } \gamma(a)=x, \ \gamma(b) = y \right\} \\
&\geqslant && \ \ \inf \left\{ L(\gamma\vert_{[a,b]}) \ \big\vert \ \gamma \in W \textrm{ und } \gamma(a)=x, \ \gamma(b) \in X \setminus U_x \right\}\\
&=&& >0,
\end{alignat*}
$d_L$ ist also positiv definit und damit eine Metrik. $\hfill \Box$
\end{compactenum}

\end{sol}




\end{prob}









\newpage




\titlespacing*{\section}{-16.5pt}{0pt}{20pt}
\renewcommand*\thesection{}
\section{Übungsblatt 6} %PARAGRAPH A.6
\renewcommand*\thesection{\arabic{section}}









\hypertarget{Asechseins}{}
\begin{prob}  %%AUFGABE 6.1
\begin{compactenum}
\item Beweisen Sie \hyperlink{satzzweizweidrei}{Satz 2.2.3}, indem Sie zeigen, dass $(\Sph^2(R), d_s)$ ein metrischer Raum ist.
\item Zeigen Sie, das $\iota: (\Sph^2, d_s) \la (\Sph^2(R), \frac{1}{R}d_s)$, $x \mapsto Rx$ eine Isometrie ist.
\item Bestimmen Sie den Durchmesser $\mathrm{diam}\hspace{1.5pt} \Sph^2(R) := \sup \{d_s(x,y) \ \vert \ x,y \in \Sph^2(R)\}$.
\end{compactenum}

\begin{sol}
\begin{compactenum}
\item Wir zeigen die Axiome der Metrik 
$$d_s: \Sph^2(R) \times \Sph^2(R) \la \R^+, \qquad (p,q) \mapsto d_s(p,q)=\inf_{c \in \Omega_{pq}(\Sph^2(R))} L_s(c).$$
\begin{compactenum}
\item Da $\Vert c'(t)\Vert_e \geqslant 0$ für jede Kurve $c \in \Omega_{pq}(\Sph^2(R))$, gilt $d_s(p,q)\geqslant 0$ für alle $p, q\in \Sph^2(R)$. Weiter gilt für den konstanten Weg $c:[a,b] \la \Sph^2(R), \ t \mapsto c(t)=p$ für beliebiges $p \in \Sph^2(R)$ bereits $c'(t)=0$, also $d_s(p,p)=0$. Sei nun umgekehrt $p,q\in \Sph^2(R)$, $p\neq q$. Wir müssen zu zeigen, dass ihr Abstand nicht verschwindet. Wir gehen hierfür vor wie im Euklidischen. Wähle $\epsilon>0$, sodass der euklidische Ball $B_{\epsilon}(p) \subseteq \R^3$ den Punkt $q$ nicht enthält und wähle einen Weg $c:[a,b] \la \Sph^2(R)$ mit $c(a)=p$ und $c(b)=q$. Nach dem Zwischenwertsatz gibt es $\delta \in [a,b]$ mit $c(\delta) \in \partial B_{\epsilon}(p)$ und $c([a,\delta]) \subseteq B_{\epsilon}(p)$, das heißt $\delta$ ist der Zeitpunkt, zu dem die Kurve $c$ den Ball erstmalig verlässt. Dann gilt für die Länge der Kurve
$$L_s(c) = \inf_a^b \Vert c'(t)\Vert_e \ \mathrm{d}t \geqslant \int_a^{\delta} \Vert c'(t)\Vert_e \ \mathrm{d}t \geqslant d_s(p,c(\delta)) \geqslant d_e(p,c(\delta)) = \epsilon$$
und damit 
$$d_s(p,q) = \inf_{c \in \Omega_{pq}(\Sph^2(R))} L_s(c) \geqslant \epsilon >0,$$
$d_s$ ist also positiv definit.
\item Für $p,q,r \in \Sph^2(R)$ gilt
$$\Omega_{qr}(\Sph^2(R)) \circ \Omega_{pq}(\Sph^2(R)) := \{ \gamma \circ \eta \ \vert \ \gamma \in \Omega_{rq}(\Sph^2(R)), \ \eta \in \Omega_{pq}(\Sph^2(R)) \} \subseteq \Omega_{pr}(\Sph^2(R))$$
und damit 
$d_s(p,r) \geqslant d_s(p,q) + d_s(q,r)$, die Dreiecksungleichung.
\item Da für jeden Weg auch der umgekehrte Weg existiert und dieselbe Länge hat, folgt Symmetrie.

\end{compactenum}

\item Wir müssen zeigen, dass für $p,q \in \Sph^2$ gilt $d_s(p,q)=\frac{1}{R} d_s(\iota(p), \iota(q)) = \frac{1}{R} d_s(Rp,Rq)$. Es gilt 
\begin{alignat*}{5}
\frac{1}{R} d_s(\iota(p),\iota(q)) \ \ &=&& \ \ \frac{1}{R} \inf_{\gamma \in \Omega_{Rp Rq}(\Sph^2(R))} L_s(\gamma) \\
&=&& \ \ \frac{1}{R} \inf_{\gamma \in \Omega_{RpRq}(\Sph^2(R))} \int_a^b \sqrt{x'(t)^2 + y'(t)^2 + z'(t)^2} \ \mathrm{d}t \\
&=&& \ \ \inf_{\gamma \in \Omega_{RpRq}(\Sph^2(R))} \int_a^b \sqrt{ \left(\frac{x'(t)}{R}\right)^2 + \left( \frac{y'(t)}{R}\right)^2 + \left( \frac{z'(t)}{R}\right)^2} \ \mathrm{d}t \\
&=&& \ \ \inf_{\gamma \in \Omega_{RpRq}(\Sph^2(R))} \int_a^b \sqrt{ \left(\left(\frac{x(t)}{R}\right)'\right)^2 + \left(\left( \frac{y(t)}{R}\right)'\right)^2 + \left(\left( \frac{z(t)}{R}\right)'\right)^2} \ \mathrm{d}t \\
&\geqslant&& \ \ \inf_{\gamma \in \Omega_{pq}(\Sph^2)} \int_a^b \sqrt{x'(t)^2 + y'(t)^2 + z'(t)^2} \ \mathrm{d}t \\
&=&& \ \ d_s(p,q)
\end{alignat*}
wobei genutzt wurde, dass für $\gamma \in \Omega_{RpRq}(\Sph^2(R))$ mit $\gamma(t)=(x(t),y(t),z(t))$ die Kurve $\tilde{\gamma}(t)=\left( \frac{x(t)}{R}, \frac{y(t)}{R}, \frac{z(t)}{R}\right)$ eine Kurve in $\Omega_{pq}(\Sph^2)$ ist. Genau so folgt
\begin{alignat*}{5}
R d_s(p,q) \ \ &=&& \ \ R \inf_{\gamma \in \Omega_{pq}(\Sph^2)} L_s(\gamma) \\
&=&& \ \ R \inf_{\gamma \in \Omega_{pq}(\Sph^2)} \int_a^b \sqrt{x'(t)^2+y'(t)^2 + z'(t)^2} \ \mathrm{d}t \\
&=&& \ \ \inf_{\gamma \in \Omega_{pq}(\Sph^2)} \int_a^b \sqrt{ (Rx'(t))^2 + (Ry'(t))^2 + (Rz'(t))^2} \ \mathrm{d}t \\
&=&& \ \  \inf_{\gamma \in \Omega_{pq}(\Sph^2)} \int_a^b \sqrt{ ((Rx(t))')^2 + ((Ry(t))')^2 + ((Rz'(t))')^2} \ \mathrm{d}t \\
&\geqslant && \ \ \inf_{\gamma \in \Omega_{RpRq}(\Sph^2(R))} \int_a^b \sqrt{x'(t)^2+y'(t)^2+z'(t)^2} \ \mathrm{d}t\\
&=&& \ \ d_s(\iota(p), \iota(q)),
\end{alignat*}
die umgekehrte Ungleichung.

\item Es ist $d_s(x,y)=L_s(\gamma)$ für den kürzesten Weg $\gamma$, der entlang eines Großkreises auf $\Sph^2(R)$ verläuft. Zu zwei Punkten $p,q\in \Sph^2(R)$ gibt es einen eindeutig bestimmten Großkreis und zwei Segmente, die $p$ und $q$ verbinden. Das kürzere davon ist kürzer als der der halbe Umfang, also $d_s(p,q) \leqslant \frac{U}{2}=\pi R$. Andererseits haben Antipodenpunkte genau den Abstand $\pi R$, also gilt $\mathrm{diam} \hspace{1.5pt} (\Sph^2(R)) = \pi R$. $\hfill \Box$

\end{compactenum}
\end{sol}


\end{prob}



\begin{prob}   %%AUFGABE 6.2
Zeigen Sie:
\begin{compactenum}
\item Die Sphäre $(\Sph^2(R), d_S)$ ist $2$-Punkt-homogen.
\item Der hyperbolische Raum $(\He^2, d_h)$ ist $2$-Punkt-homogen.
\end{compactenum}

\begin{sol}

\begin{compactenum}
\item Seien $(a,b), (x,y)$ zwei Punktepaare in $\Sph^2(R)$ mit $d_s(a,b)=d_s(x,y)$. Wir müssen zeigen, dass es eine Isometrie $\Phi\in \mathrm{Isom}(\Sph^2(R))$ gibt mit $\Phi(x)=a$ und $\Phi(y)=b$. Da $\Sph^2(R)$ homogen ist, können wir ohne Einschränkung $a=x$ annehmen. Außerdem operiert $\mathcal{O}(3)$ transitiv auf der Menge der Vektoren des $\R^3$ mit Norm $R$, wir können also weiter $a=(0,0,R)$ annehmen. Wir brauchen also eine Isometrie $\Phi$ mit $\Phi(0,0,R)=(0,0,R)$ und $\Phi(y)=b$, es muss also $\Phi \in \mathrm{Isom}(\Sph^2(R))_a \cong \mathcal{O}(2)$ enthalten sein. Aus $d_s(a,b)=d_s(x,y)=d_s(a,y)$ folgt $b_3=y_3$ und damit aus 
$$ \Bigg\Vert \begin{pmatrix} b_1\\[-6pt] b_2\\[-6pt] b_3\end{pmatrix}\Bigg\Vert^2 = b_3^2 + \Bigg \Vert \begin{pmatrix} b_1 \\[-6pt] b_2\end{pmatrix} \Bigg \Vert^2 = \Bigg \Vert \begin{pmatrix} y_1 \\[-6pt] y_2\end{pmatrix} \Bigg \Vert^2 + y_3^2  = \Bigg\Vert \begin{pmatrix} y_1\\[-6pt] y_2\\[-6pt] y_3\end{pmatrix}\Bigg\Vert^2  $$
auch $$r:=\Bigg \Vert \begin{pmatrix} b_1 \\[-6pt] b_2\end{pmatrix} \Bigg \Vert = \Bigg \Vert \begin{pmatrix} y_1 \\[-6pt] y_2\end{pmatrix} \Bigg \Vert.$$
Da $\mathcal{O}(2)$ transitiv auf Vektoren des $\R^3$ mit Norm $r$ operiert, gibt es eine solche Isometrie $\Phi \in \mathcal{O}(2)$ mit $\Phi(y_1,y_2)=(b_1,b_2)$ und $\Phi(b_3)=b_3=y_3$, insgesamt also $\Phi(y)=b$, was zu zeigen war.

\item Für $\alpha + i \beta \in \He^2$ gilt mit 
$$A=\begin{pmatrix}[rr] \frac{1}{\beta} & -\frac{\alpha}{\sqrt{\beta}} \\ 0 & \sqrt{\beta} \end{pmatrix} \in \mathrm{PSL}_2(\R)$$ 
bereits $T_A(i)=\alpha + i\beta$, es bleibt noch zu zeigen, dass für $z,w\in \He^2$ mit $d_h(z,i)=d_h(w,i)$ eine Isometrie $\Phi \in \mathrm{Isom}(\He^2)$ gibt mit $\Phi(w)=z$ und $\phi(i)=i$. Wir wissen bereits
$$\mathrm{PSL}_2(\R)_i\cong \mathrm{PO}(2).$$
Es sind also $z,w$ auf dem hyperbolischen Kreis mit Radius $r:=d_h(z,i)=d_h(w,i)$ um $i$, man kann also $z$ mit einer Drehung um $i$ (welche $i$ selbst fix lässt) auf $w$ abbilden. Dies wollen wir im folgenden präzisieren. Betrachte dazu für $P:=ie^{r}$ die $\mathrm{PO}(2)$-Bahn
\begin{alignat*}{5}
\mathrm{PO}(2) \cdot ie^r \ \ &=&& \ \ \left\{ \frac{aie^r + b}{-bie^r + a} \ \bigg\vert \ a^2+b^2=1 \right\}\\
&=&& \ \ \left\{ \frac{(aie^r + b)(a+bie^r)}{b^2e^{2r}+a^2} \ \bigg\vert \ a^2+b^2 =1 \right\} \\
&=&& \ \ \left\{ \frac{-	ab e^{2r} + a^2ie^r + b^2ie^r  + ab}{b^2e^{2r}+a^2} \ \bigg\vert \ a^2+b^2=1\right\}\\
&=&& \ \ \left\{\frac{ab(1-e^{2r}) + ie^r}{b^2e^{2r}+a^2} \ \bigg\vert \ a^2+b^2=1\right\} \\
&=&& \ \ \left\{\frac{ab(1-e^{2r}) + ie^{r}}{a^2(1-e^{2r}) + e^{2r}} \ \bigg\vert \ a^2+b^2 = 1 \right\} \\
&=&& \ \ \left\{ \frac{e^r-e^{-r}}{2}(u+iv) + \frac{e^r+e^{-r}}{2} i \ \bigg \vert \ u^2 +v^2 =1 \right\}\\
&=&& \ \ \left\{ \sinh r (u+iv) + \cosh r \ \bigg\vert \ u^2+v^2=1 \right\}\\
&=&& \ \ B_r(i),
\end{alignat*}
wobei folgendermaßen substituiert wurde:
$$u=\frac{ab(1-e^{2r})}{a^2(1-e^{2r})+e^{2r}}, \qquad v=\frac{2e^{2r}-(1+e^{2r})(a^2(1-e^{2r})+e^{2r})}{(e^{2r}-1)(a^2(1-e^{2r})+e^{2r})}$$
(eine aufwändige Rechnung zeigt, dass $a^2+b^2=1$ genau dann, wenn $u^2+v^2=1$).
Daraus folgt, dass $z$ und $w$ in derselben Bahn sind und damit $2$-Punkt-Homogenität von $\He^2$. $\hfill \Box$


\end{compactenum}

\end{sol}


\end{prob}
\newpage


\begin{prob} %%AUFGABE 6.3
Es sei $R>0$ und 
$$\gamma_R: [0,1] \la \R^2, \qquad t \mapsto \  \begin{pmatrix}[c] R \cos 2\pi t \\ R \sin 2\pi t + 2R\end{pmatrix}.$$
\begin{compactenum}
\item Skizzieren Sie das Bild von $\gamma_R$ für $R=1,R=2$ und $R=3$.
\item Berechnen Sie die euklidische Länge von $\gamma_R$.
\item Zeigen Sie, dass die hyperbolische Länge von $\gamma_R$ unabhängig von $R$ ist.
\end{compactenum}
\begin{sol}
\begin{compactenum}
\item Wir erhalten folgendes Bild:

$$
\begin{tikzpicture}[scale=0.8]
\draw[->] (-5,0) -- (5,0);
\draw[->] (0,-1) -- (0,10);
\draw (1,2) arc(0:360:1);
\draw (2,4) arc(0:360:2);
\draw (3,6) arc(0:360:3);
\draw (0.9,1.5) node[right] {$\gamma_1$};
\draw (1.8,3.1) node[right] {$\gamma_2$};
\draw (3,6) node[right] {$\gamma_3$};

\end{tikzpicture}
$$

\item Es gilt 
$$\gamma'(t)= \begin{pmatrix}[r] -2 \pi R \sin 2\pi t \\ 2 \pi R \cos 2\pi r \end{pmatrix},$$
also $\Vert \gamma'(t)\Vert_e= 2\pi R$. Für die euklidische Länge folgt also 
$$L_e(\gamma_R)= \int_0^1 \Vert \gamma_R'(t)\Vert_e \ \mathrm{d}t = \int_0^1 2 \Pi R \ \mathrm{d}t = 2 \pi R.$$

\item Es gilt $\mathrm{im}(\gamma_R(t)) = R( 2 + \sin 2 \pi t)$. Für die hyperbolische Länge gilt also 
$$L_h(\gamma_R)= \int_0^1 \ \frac{\Vert \gamma_R'(t)\Vert_e}{\mathrm{Im}(\gamma_R(t))} \ \mathrm{d}t = \int_0^1 \frac{2\pi}{2 + \sin 2 \pi t} \ \mathrm{d}t,$$
was nicht von $R$ abhängt. $\hfill \Box$

\end{compactenum}
\end{sol}
\end{prob}





\begin{prob}  %%AUFGABE 6.4
\begin{compactenum}
\item Berechnen Sie den Flächeninhalt eines hyperbolische $n$-Gons.
\item Zeigen Sie, dass es ein hyperbolische Achteck gibt, dessen Innenwinkel all $90^{\circ}$ betragen.
\end{compactenum}

\begin{sol}
\begin{compactenum}
\item sei $\Delta$ ein hyperbolisches $n$-Eck mit Innenwinkeln $\alpha_1, \ldots, \alpha_n$ und betrachte die $n$ Dreiecke $\Delta_1, \ldots, \Delta_n$, die entstehen, wenn man die Eckpunkte mit einem beliebigen inneren Punkt (durch Geodätische) verbindet.
$$
\begin{tikzpicture}
\draw (3,0) arc(0:360:3);
\draw[dashed] (-3,0) -- (3,0);
\draw[dashed] (0,-3) -- (0,3);
\draw[dashed] (2.1213,2.1213) -- (-2.1213,-2.1213);
\draw[dashed] (-2.1213,2.1213) -- (2.1213,-2.1213);
\draw[thick] (-2.25,0) -- (2.25,0);
\draw[thick] (0,-2.25) -- (0,2.25);
\draw[thick] (1.5910,1.5910) -- (-1.5910,-1.5910);
\draw[thick] (1.5910,-1.5910) -- (-1.5910,1.5910);
\draw[thick] (2.25,0) to[bend left=30] (1.5910,1.5910) to[bend left=30] (0,2.25) to[bend left=30] (-1.5910,1.5910) to[bend left=30] (-2.25,0) to[bend left=30] (-1.5910,-1.5910) to[bend left=30] (0,-2.25) to[bend left=30] (1.5910,-1.5910) to[bend left=30] (2.25,0);
\draw (0.8,0) arc (0:45:0.8);
\draw (0.5,-0.08) node[above] {$\delta_i$};
\draw (1.0607,1.0607) arc (210:269:0.75);
\draw (1.0607,1.0607) node[right] {$\beta_i$};
\draw (1.8,0.5) arc(120:178:0.6);
\draw (1.8,-0.08) node[above] {$\gamma_i$};
\end{tikzpicture}
$$
Bezeichne die Innenwinkel mit $\beta_i, \gamma_i, \delta_i$, wobei $\beta_i+\gamma_{i-1}=\alpha_i$ für $i \in \{2, \ldots, n\}$ und $\alpha_1=\beta_1+\gamma_n$. Dann gilt 
\begin{alignat*}{5}
\mu(\Delta) \ \ &=&& \ \ \sum_{i=1}^n \mu(\Delta_i) \\
&=&& \ \ \sum_{i=1}^n \pi - \beta_i - \gamma_i - \delta_i \\
&=&& \ \ n \pi - \left( \sum_{i=1}^n \delta_i\right) - \left( \sum_{i=1}^n \beta_i+\gamma_i\right) \\
&=&& \ \ (n-2)\pi - \left( \sum_{i=1}^n \alpha_i\right).
\end{alignat*}
\item Betrachte das regelmäßige Achteck mit Ecken auf dem Rand im Unendlichen (im Einheitskreismodell). Dann gilt $\alpha_i=0$ für alle $i \in \{1,\ldots, 8\}$ und nach Teil (i) $\mu(\Delta)=6 \pi$. Verschiebe nun die Ecken gleichmäßig zum Ursprung. Dann gilt $\mu(\Delta_8) \la 0$ und damit 
$$\alpha:= \alpha_i= \frac{1}{8} \sum_{i=1}^8 \alpha_i = \frac{6 \pi - \mu(\Delta)}{8} \la \frac{3\pi}{4}.$$
Dieser Vorgang ist stetig un der Zwischenwertsatz liefert uns für jedes $0 \leqslant \alpha \leqslant \frac{3\pi}{4}$ ein Dreieck mit Innenwinkeln $\alpha$, also insbesondere auch für $\alpha=\frac{\pi}{2}$. $\hfill \Box$

\end{compactenum}

\end{sol}

\end{prob}














\newpage




\titlespacing*{\section}{-16.5pt}{0pt}{20pt}
\renewcommand*\thesection{}
\section{Übungsblatt 7} %PARAGRAPH A.7
\renewcommand*\thesection{\arabic{section}}



\hypertarget{Asiebeneins}{}
\begin{prob}   %%AUFGABE 7.1
Beweisen Sie \hyperlink{satzzweiviervier}{Satz 2.4.4}: Für $z,w \in \He^2$ gilt
\begin{compactenum}
\item $$d_h(z,w) = \ln \frac{\vert z - \overline{w} \vert + \vert z-w \vert}{\vert z - \overline{w}\vert - \vert z - w \vert}$$
\item $$ \cosh \left( d_h(z,w)\right) = 1 + \frac{ \vert z-w \vert^2}{ 2 \mathrm{Im}(z) \mathrm{Im}(w)}.$$
\item $$\sinh\left( \frac{1}{2} d_h(z,w)\right) = \frac{\vert z - w \vert}{2 \sqrt{\mathrm{Im}(z) \mathrm{Im}(w)}}.$$
\item $$\cosh\left( \frac{1}{2} d_h(z,w)\right) = \frac{ \vert z - \overline{w}\vert}{2 \sqrt{\mathrm{Im}(z) \mathrm{Im}(w)}}.$$
\item $$\tanh\left( \frac{1}{2} d_h(z,w)\right) = \frac{\vert z - w \vert }{\vert z - \overline{w}\vert}.$$
\end{compactenum}

\begin{sol}
\begin{compactenum}
\item[(iii)] Wurde in der Vorlesung gezeigt.
\item[(iv)] Es gilt 
\begin{alignat*}{5}
\cosh \left( \frac{1}{2} d_h(z,w)\right) \ \ &=&& \ \  \sqrt{ 1 + \sinh^2 \left( \frac{1}{2} d_h(z,w)\right)} \\
&=&& \ \ \sqrt{ 1 + \frac{\vert z-w \vert^2}{ 4 \mathrm{Im}(z) \mathrm{Im}(w)} } \\
&=&& \ \ \frac{\sqrt{4 \mathrm{Im}(z) \mathrm{Im}(w) + \vert z-w \vert^2}}{2\sqrt{\Imm(z) \Imm(w)}} \\
&=&& \ \ \frac{\sqrt{4 \Imm(z) \Imm(w) + ( \Ree(z) - \Ree(w))^2 + (\Imm(z)-\Imm(w))^2}}{2\sqrt{\Imm(z) \Imm(w)}} \\
&=&& \ \ \frac{\sqrt{(\Ree(z)_\Ree(w))^2 + (\Imm(z) + \Imm(w))^2}}{2 \sqrt{\Imm(z) \Imm(w)}} \\
&=&& \ \ \frac{\vert z- \overline{w}\vert}{2 \sqrt{\Imm(z)\Imm(w)}}.
\end{alignat*}
\item[(v)] Es gilt nach (iii) und (iv)
$$\tanh \left( \frac{1}{2} d_h(z,w)\right) = \frac{\sinh \left( \frac{1}{2} d_h(z,w)\right)}{\cosh \left( \frac{1}{2} d_h(z,w)\right)} = \frac{\vert z-w\vert}{\vert z - \overline{w}\vert}.$$
\item[(i)] Mit (v) und der Identität $\arctanh(x) = \frac{1}{2} \ln \left(\frac{1+x}{1-x}\right)$ gilt
\begin{alignat*}{5} 
\ln \left( \frac{\vert z - \overline{w}\vert + \vert z-w\vert}{\vert z -\overline{w}\vert - \vert z- w\vert}\right) \ \ &=&& \ \ 2 \arctanh \left( \frac{\vert z-w\vert}{\vert z - \overline{w}\vert}\right) \\
&=&& \ \ 2 \arctanh \left( \tanh\left( \frac{1}{2} d_h(z,w)\right) \right) \\
&=&& \ \ d_h(z,w).
\end{alignat*}
\item[(ii)] Schließlich gilt mit $\cosh(2x) = 2 \sinh^2(x) + 1$ 
$$\cosh^2\left( d_h(z,w)\right) = \frac{\vert z-w\vert^2}{2 \Imm(z) \Imm(w)} + 1,$$
was die letzte Formel beweist. $\hfill \Box$


\end{compactenum}
\end{sol}

\end{prob}




\begin{prob}   %%AUFGABE 7.2
\begin{compactenum}
\item Zeigen Sie, dass die Möbiustransformation 
$$m: (\He^2, d_h) \la (\D^2, d_h^*), \qquad z \mapsto m(z)=\frac{iz+1}{z+i}$$
eine Isometrie ist.
\item Zeigen Sie des weiteren die folgenden Formeln für die Metrik $d_h^*$:
\begin{compactenum}
\item Für $0<r<1$ gilt $d_h^*((0,0),(0,r)) = \ln \frac{1+r}{1-r}$.
\item Für $z,w \in \D^2$ gilt 
$$\cosh^2 \left( \frac{1}{2} d_h^*(z,w)\right) = \frac{\vert 1 - z \overline{w}\vert^2}{(1-\vert z \vert^2)(1-\vert w \vert ^2)}.$$
\end{compactenum}

\end{compactenum}

\begin{sol}
\begin{compactenum}
\item Wir zeigen, dass für jeden Weg $\gamma:[a,b] \la \He^2$ gilt $L_h(\gamma)=L_h^*(m\circ\gamma)$, also 
$$L_h(\gamma)= \int_a^b \frac{\Vert \gamma'(t)\Vert}{\Imm (\gamma(t))} \ \mathrm{d}t \overset{!}{=} \int_a^b \frac{2 \Vert (m\circ \gamma)'(t)\Vert}{1-\Vert (m\circ \gamma)(t)\Vert^2}\ \mathrm{d}t = L_h^*(m \circ \gamma),$$
indem wir Gleichheit der Integranden zeigen. Mit der Kettenregel gilt $(m \circ \gamma)'(t)= m'(\gamma(t))\gamma'(t)$. $m$ abgeleitet ergibt
$$m'(z) = \frac{\mathrm{d}}{\mathrm{d}z} \left( \frac{iz+1}{z+i}\right) = \frac{i(z+i)-(iz+1)}{(z+i)^2} = -\frac{2}{(z+i)^2},$$
es gilt also
\begin{alignat*}{5}
\frac{2 \Vert (m\circ \gamma)'(t)\Vert}{1-\Vert (m\circ \gamma)(t)\Vert^2} \ \ &=&& \ \  \frac{2 \big\Vert \frac{-2}{(\gamma(t)+i)^2} \gamma'(t) \big\Vert}{1-\big\Vert \frac{i\gamma(t) + 1}{\gamma(t)+i}\big\Vert^2}\\[6pt]
&=&& \ \ \frac{4 \Vert \gamma'(t)\Vert}{\Vert \gamma(t)+i\Vert^2 -\Vert i \gamma(t) + 1 \Vert} \\[6pt]
&=&& \ \ \frac{4 \Vert \gamma'(t)\Vert}{x(t)^2 + (y(t)+1)^2 - (x(t)^2 + (y(t)-1)^2)} \\[6pt]
&=&& \ \ \frac{4 \Vert \gamma'(t)\Vert}{4 \gamma(t)} \\[6pt]
&=&& \ \ \frac{\Vert \gamma'(t)\Vert}{\Imm(\gamma(t))},
\end{alignat*}
woraus nun die Behauptung folgt.

\item \begin{compactenum}
\item Es gilt $m(i)=(0,0)$ sowie
$$m\left( \frac{1+r}{1-r}i\right) = \frac{- \frac{1+r}{1-r} +1}{\frac{1+r}{1-r}i + i} = \frac{-1-r+1-r}{(1+r)i+i-ir} = -\frac{2r}{2i} = ri = (0,r),$$
also 
$$d_h^*((0,0),(0,r)) = d_h^*\left( m(i), m\left(\frac{1+r}{1-r}i\right)\right) = d_h\left(i, \frac{1+r}{1-r}i\right) = \ln \frac{1+r}{1-r},$$
was zu zeigen war.
\item Es gilt
\begin{alignat*}{5}
\cosh^2 \left( d_h^*(z,w)\right) \ \ &=&& \ \ \cosh^2 \left( d_h^*\left( m\left( \frac{-iz+1}{z-i}\right), m\left( \frac{-iw+1}{w-i}\right)\right)\right) \\[6pt]
&=&& \ \ \cosh^2 \left( d_h\left(\frac{-iz+1}{z-i}, \frac{-iw+1}{w-i}\right) \right) \\[6pt]
&=&& \ \ \left( \frac{ \bigg \vert \frac{-iz+1}{z-i} - \overline{\left( \frac{-iw+1}{w-i}\right)} \bigg\vert }{2 \sqrt{ \Imm \left( \frac{-iz+1}{z-i}\right) \Imm\left( \frac{-iw+1}{w-i}\right)}}\right)^2
\end{alignat*}
Zwischenrechnungen ergeben
\begin{alignat*}{5}
\Bigg\vert \frac{-iz+1}{z-i} - \frac{i\overline{w} +1}{\overline{w}+i} \Bigg\vert \ \ &=&& \ \ \Bigg\vert \frac{(-iz+1)(\overline{w}+i) - (i \overline{w}+1)(z-i)}{(z-i)(\overline{w}+i)} \Bigg \vert \\[6pt]
&=&& \ \ \frac{2 \vert 1-z \overline{w}\vert^2}{\vert z-i\vert \vert \overline{w}+i\vert} \\[6pt]
&=&& \ \ \frac{2 \vert 1-z \overline{w}\vert^2}{\vert z-i\vert \vert w-i\vert} \\
\end{alignat*}
sowie mit $z=x+iy$
\newpage
\begin{alignat*}{5}
\Imm\left( \frac{-iz+1}{z-i}\right) \ \ &=&& \ \ \Imm \left( \frac{-i(x+iy)+1}{x+iy-1}\right) \\[6pt]
&=&& \ \ \Imm\left( \frac{y+1-ix}{x+i(y-1)}\right) \\[6pt]
&=&& \ \ \Imm\left( \frac{(y+1-ix)(x-i(y-1))}{x^2+(y-1)^2}\right) \|[6pt]
&=&& \ \ \frac{1-x^2y^2}{x^2+(y-1)^2}\\[6pt]
&=&& \ \ \frac{1-\vert z \vert^2}{\vert z-i\vert^2},
\end{alignat*}
ebenso
$$\Imm\left( \frac{-iz+1}{z-i}\right) = \frac{1-\vert w \vert^2}{\vert w - i \vert^2}.$$
Diese liefern dann 

$$ \cosh^2 \left( d_h^*(z,w)\right) =  \left( \frac{ \bigg \vert \frac{-iz+1}{z-i} - \overline{\left( \frac{-iw+1}{w-i}\right)} \bigg\vert }{2 \sqrt{ \Imm \left( \frac{-iz+1}{z-i}\right) \Imm\left( \frac{-iw+1}{w-i}\right)}}\right)^2 =  \frac{\vert 1 - z \overline{w}\vert^2}{(1-\vert z \vert^2)(1-\vert w \vert ^2)},$$
was zu zeigen war. $\hfill \Box$
\end{compactenum}

\end{compactenum}

\end{sol}

\end{prob}


\begin{prob}   %%AUFGABE 7.3
\begin{compactenum}
\item Beweisen Sie, dass hyperbolische Kreise euklidische Kreise (mit anderen Radien und anderen Mittelpunkten) sind.
\item Berechnen Sie außerdem den hyperbolischen Radius für einen euklidischen Kreis in $\D^2$ mit Zentrum in $(0,0)$ und euklidischem Radius $\rho$.
\end{compactenum}

\begin{sol}
\begin{compactenum}
\item Sei $K$ ein hyperbolischer Kreis in $\He^2$ um $i$ mit Radius $\ln R$ für ein $R >1$. Sei $x=x+iy\in \He^2$ ein beliebiger Punkt. Dann gilt
\begin{alignat*}{5}
z \in K \ \ &\Longleftrightarrow && \quad d_h(i,z) = \ln R = d_h(i,iR) \\
&\Longleftrightarrow && \quad \frac{\vert i-z\vert}{\vert i - \overline{z}\vert} = \frac{\vert i - iR\vert}{\vert i + i R
\vert}\\
&\Longleftrightarrow && \quad  \frac{\vert i-z\vert^2}{\vert i - \overline{z}\vert^2} = \frac{\vert i - iR\vert^2}{\vert i + i R \vert^2} \\
&\Longleftrightarrow && \quad \vert i \vert \vert 1 + R\vert^2 \vert i -z \vert^2 = \vert i \vert \vert 1-R \vert^2 \vert i - \overline{z}\vert^2\\
&\Longleftrightarrow && \quad (1+R)^2 (1+iz-i\overline{z} + z\overline{z}) = (1-R)^2(1+z\overline{z} - iz + i \overline{z})\\
&\Longleftrightarrow && \quad (1+R^2)(iz - i \overline{z}) + 2R ( 1+ z \overline{z}) = (1-R^2)(i\overline{z} -iz) -2R(1 + z \overline{z}) \\
&\Longleftrightarrow && \quad 4R z \overline{z} + 2iz -2i \overline{z} + 4R = 0,
\end{alignat*}
was einer euklidischen Kreisgleichung entspricht.

\item Im Einheitskreismodell haben die Punkte $\rho$, $i\rho$ und $-\rho$ Abstand $\ln \frac{1+\rho}{1-\rho}$ vom Urpsrung. Betrachtet man nun einen hyperbolischen Kreis um $i$ mit Radius $r$ in $\He^2$, so liegen die beiden Punkte $ie^r$ und $ie^{-r}$ auf dem Kreis. Dieser entspricht also einem eulidischen Kreis mit Mittelpunkt $\frac{e^r+e^{-r}}{2}=\cosh r$ und Radius $\frac{e^r-e^{-r}}{2}=\sinh r$. $\hfill \Box$


\end{compactenum}
\end{sol}


\end{prob}



















\newpage




\titlespacing*{\section}{-16.5pt}{0pt}{20pt}
\renewcommand*\thesection{}
\section{Übungsblatt 8} %PARAGRAPH A.8
\renewcommand*\thesection{\arabic{section}}




\begin{prob}   %%AUFGABE 8.1
Zeigen Sie: Die Geodätischen in $(\D^2, d_h^*)$ sind Segmente von euklidischen Kreisen Orthogonal zum Einheitskreis $\Sph^1$.

\begin{sol}
Sei $g$ eine Geodätische in $(\D^2, d_h^*)$. Sei $m: \He^2 \la \D^2$ die übliche Isometrie und betrachte $m^{-1}(g)$. Dann ist $m^{-1}(g)$ eine Geodätische in $(\He^2, d_h)$, also der Schnitt von $\He^2$ mit einem verallgemeinerten Kreis in $\CC$, der senkrecht auf $\RR$ ist. Da $g=m(m^{-1}(g))$ und $m\in \amob$ kreistreu und konform ist, ist auch $g$ Schnitt eines verallgemeinerten Kreises in $\CC$ mit $\D^2$, der senkrecht auf $m(\RR) = \Sph^1$ ist. $\hfill \Box$

\end{sol}


\end{prob}

\begin{prob}   %%%AUFGABE 8.2
Zeigen Sie, dass \textit{Äquidistanten} zu einer hyperbolischen Geodätischen $h$, d.h. Kurven bestehend aus Punkten mit konstantem Abstand zu $h$ keine hyperbolischen Geodätischen sind.
\begin{sol}
Sei $g$ zunächst eine Parallele zur imaginären Achse mit Fußpunkt $x \in \R$ und $a$ eine Äquidistante zu $g$ mit Abstand $\delta >0$. Sei $y \in \He^2$ mit $d_h(y,g)= \delta$, also ohne Einschränkung $y \in a$. Sei $m_{\lambda}$ für $\lambda>0$ diejenige Möbiustransformation, welche zentisch um $x$ mit dem Faktor $\lambda$ streckt (also $m(z)=\lambda (z-x)$). Betrachte den Strahl 
$$A:=\{m_{\lambda}(y)\ \vert \ \lambda >0\}$$
der Strahl von $x$ durch $y$. Wir behaupten jetzt, dass $A$ äquidistant zu $g$ ist. 
$$
\begin{tikzpicture}[scale=0.9]
\draw[->] (-4,0) -- (4,0);
\draw[->] (-2.5,-0.5) -- (-2.5,4);
\draw[dashed] (3,0) arc(0:180:3);
\draw[dashed] (2,0) arc(0:180:2);
\draw[color=blue] (0,-0.5) -- (0,3.5);
\draw[fill=black] (0.05,0) arc(0:360:0.05);
\draw (0.5,0) node[below] {$x$};
\draw[very thick, color=red] (1.4142,1.4142) arc(45:90:2);
\draw[very thick, color=green] (2.1213,2.1213) arc(45:90:3);
\draw[fill=black] (0.05,2) arc(0:360:0.05);
\draw[fill=black] (0.05,3) arc(0:360:0.05);
\draw (1.4142,1.4142) node[right] {$z$};
\draw (2.1213,2.1213) node[right] {$y$};
\draw (0,2.3) node[left] {$z'$};
\draw (0,3.3) node[left] {$y'$};
\draw[fill=black] (2.1713,2.1213) arc(0:360:0.05);
\draw[fill=black] (1.4642,1.4142) arc(0:360:0.05);

\draw (0,0) -- (3,3);
\draw (3,3) node[below] {$A$};
\draw[color=blue] (0,1) node[left] {$g$};


\end{tikzpicture}
$$

Der Abstand $\delta$ von $y$ zu $g$ wird realisiert vom grünen Kreissegment. Ist $z \in A$ nun beliebig, so wird der Abstand von $z$ zu $g$ realisiert vom roten Kreissegment. Die Länger dieser Segmente ist nach Lemma 2.10.3 ist jedoch identisch, daher ist $A=a$ eine Äquidistante zu $g$ mit Abstand $\delta$, aber nicht mehr senkrecht zu $\RR$, also keine Geodätische. Ist $h$ nun eine beliebige Gerade, so wähle eine Möbiustransformation $m \in \mob(\He^2)$ mit $m(h)=g$. Dann ist für jede Äquidistante $a$ von $g$ auch $m^{-1}(a)$ eine Äquidistante zu $h$ und keine Geodätische, da sonst auch $a= m(m^{-1}(a))$ eine Geodätische wäre, was nach obiger Erkenntnis falsch ist. $\hfill \Box$

\end{sol}
\end{prob}



\begin{prob} %%%AUFGABE 8.3
\begin{compactenum}
\item Beweisen Sie Lemma 2.10.4: Sei $g$ eine Geodätische in $\He^2$ und $p\in \He^2$, $p \notin g$. Dann gilt für den Lotfußpunkt $p'\in g$
$$d_h(p,g)= \min_{q \in g} d_h(p,q) = d_h(p,p').$$
\item Um den hyperbolischen Mittelpunkt eines Kreises $K$ in $\He^2$ zu ermitteln, kann man folgende Konstruktion (mit Zirkel und Lineal) durchführen:
\begin{compactenum}
\item Man zeichne die vertikale Geodätische $g$ durch den euklidischen Mittelpunkt $Z_{\mathrm{eukl}}$ von $K$ und erhält den Schnittpunkt $x$ von $g$ mit der reellen Achse.
\item Man zeichne die beiden Tangenten an $K$, die durch $x$ laufen und erhält die beiden Berührpunkte $p,q$ auf $K$.
\item Man zeichne die hyperbolische Geodätische $h$ durch $p$ und $q$ und erhält den Schnittpunkt $z$ von $h$ und $g$.
\end{compactenum}
Zeigen Sie, dass $z$ der hyperbolische Mittelpunkt $Z_{\mathrm{hyp}}$ von $K$ ist.
\end{compactenum}
$$
\begin{tikzpicture}
 \draw[->] (-4,0) -- (4,0);
 \draw (1,3) arc(0:360:1);
 \draw (0.8,3.75) node[above] {$K$};
 \draw (0.1,3.3) node[left] {$Z_{\mathrm{eukl}}$};
 \draw[fill=black] (0.05,3) arc(0:360:0.05);
 \draw (0,0) -- (0,5);
 \draw (0,4.5) node[left] {$g$};
 \draw (0,0) node[below] {$x$};
 \draw (0,0) -- (1.5,4.2);
\draw (0,0) -- (-1.5,4.2);
\draw[fill=black] (0.99,2.65) arc(0:360:0.05);
\draw[fill=black] (-0.89,2.65) arc(0:360:0.05);
\draw (0.94,2.65) node[right] {$q$};
\draw (-0.94,2.65) node[left] {$p$};
\draw[fill=black] (0.05,2.8118) arc(0:360:0.05);
\draw (-0.1,2.5) node[right] {$Z_{\mathrm{hyp}}$};
\draw[dashed] (2.8118,0) arc(0:180:2.8118);
\draw (0.94,2.65) arc(70.47:109.53:2.8118);

\end{tikzpicture}
$$

\begin{sol}
\begin{compactenum}
\item Sei $p' \in g$ und betrachte das Dreieck $\Delta= \Delta(p'pq)$ mit Innenwinkel $\alpha$ bei $q$. Nach dem Sinussatz gilt 
$$\frac{\sinh d_h(p,q)}{\sin 90^{\circ}} = \frac{\sinh d_h(p,p')}{\sin \alpha},$$
also 
$$d_h(p,p') = \sin \alpha \sinh d(p,p') \leqslant d_h(p,q),$$
woraus die Behauptung folgt. 
\item Es gilt: $p$ und $q$ sind die beiden einzigen Punkte auf $K$ mit maximalem Abstand zu $g$ (denn $p$ und $q$ haben beide als einziges maximalen Polarwinkel) und $Z_{\mathrm{hyp}}$ ist der Lotfußpunkt von $p$ bzw. $q$ auf $g$. Sei nun $w\neq Z_{\mathrm{hyp}}$ auf $g$ und innerhalb von $K$. Angenommen $w$ ist der hyperbolische Mittelpunkt von $K$. Wähle $r \in K$, sodass $w$ der Lotfußpunkt von $r$ ist. Damit ist 
$$d_h(g,p) > d_h(g,r) \overset{(i)}{=} d_h(w,r) = d_h(w,p) > d_h(g,p),$$
ein Widerspruch. Ist $z$ außerhalb von $K$, so ist $z$ sicherlich nicht der hyperbolische Mittelpunkt von $K$. Damit ist $Z_{\mathrm{hyp}}$ der hyperbolische Mittelpunkt, was zu zeigen war. $\hfill \Box$


\end{compactenum}
\end{sol}



\end{prob}




















\newpage




\titlespacing*{\section}{-16.5pt}{0pt}{20pt}
\renewcommand*\thesection{}
\section{Übungsblatt 9} %PARAGRAPH A.9
\renewcommand*\thesection{\arabic{section}}




\begin{prob}    %%AUFGABE 9.1

\begin{compactenum}
\item Zeigen Sie, dass das Flächenelement für das Kreisscheibenmodell $(\D^2, d_h^*)$ in Polarkoordinaten $r, \theta$ gegeben ist durch 
$$\mathrm{d}A=\frac{4r}{(1-r^2)^2}\ \mathrm{d}r \mathrm{d}\theta.$$
\item Berechnen Sie den hyperbolischen Umfang $L_h(\partial B_R)$ einer hyperbolischen Kreisscheibe
mit Radius $R>0$. 
\item Berechnen Sie den hyperbolischen Flächeninhalt $\mu(B_R)$ einer hyperbolischen Kreisscheibe $B_R$ mit Radius $R>0$.
\end{compactenum}


\begin{sol}
\begin{compactenum}
\item In $(\He^2, d_h)$ gilt für $A \subseteq \He^2$
$$\mu(A) = \int_A \mathrm{d}A = \int_A \frac{1}{y^2} \ \mathrm{d}x \mathrm{d}y,$$
also $\mathrm{d}A(x,y)=\frac{1}{y^2}.$ Fass die Isometrie
$$m: \D^2 \la \He^2, \qquad z \mapsto \frac{-iz+1}{z-i}$$
als Funktion 
$$m: \R^2 \la \R^2, \qquad (x,y) \mapsto \frac{1}{x^2+(1-y)^2} \left( 2x, 1-x^2-y^2\right)$$
auf. Dann gilt 
$$(\mathrm{d}A \circ m)(x,y) = \frac{(x^2+(1-y)^2)^2}{(1-x^2-y^2)^2}$$
und 
$$D m(x,y) = \begin{pmatrix}[rr] \frac{\partial m_1}{\partial x}(x,y) & \frac{\partial m_1 }{\partial y}(x,y) \\ \frac{\partial m_2}{\partial x}(x,y) & \frac{\partial m_2}{\partial y}(x,y) \end{pmatrix} = \frac{1}{(x^2+(1-y)^2)^2} \begin{pmatrix} 2(y-1)^2-2x^2 & 4x(y-1) \\ 4x (y-1) & 2 (y-1)^2 - 2x^2 \end{pmatrix} $$
Es ist 
\begin{alignat*}{5}
\det D m(x,y)  \ \ &=&& \ \ \frac{1}{(x^2+(1-y)^2)^4} \left( (2 (y-1)^2-2x^2)^2 - (4x(y-1))^2 \right) \\
&=&& \ \ \frac{1}{(x^2+(1-y)^2)^4} \left( 4 (x^2+(1-y)^2)^2 \right) \\
&=&& \ \ \frac{4}{(x^2+(1-y)^2)^2}.
\end{alignat*}
Die Transformationsformel liefert nun für $A \subseteq \D^2$
\begin{alignat*}{5}
\mu(A) \ \ &=&& \ \ \mu(m^{-1}(A)) \\
&=&& \ \ \int_{m^{-1}(A)}  \mathrm{d}A(x,y) \\
&=&& \ \ \int_{A} \ \mathrm{m}(\mathrm{d}A(x,y)) \vert \det D m(x,y)\vert \ \mathrm{d}x \mathrm{d}y\\
&=&& \ \ \int_{m(A)} \frac{4}{(1-x^2-y^2)^2} \ \mathrm{d}x \mathrm{d}y,
\end{alignat*}
also $\mathrm{d}A= \frac{4}{(1-x^2-y^2)^2}$. Übergang zu Polarkoordinaten liefert die Behauptung. 
\item Wähle ohne Einschränkung $B_R= B_R(0) \subseteq \D^2$. In der Vorlesung haben wir gesehen, dass der hyperbolische Radius $R$ dem euklidischen Radius 
$$\rho = \frac{e^R-1}{e^R+1}$$
entspricht. Definiere nun
$$c:[0,2\pi] \la \D^2, \qquad t \mapsto \begin{pmatrix}[r] R \cos t \\ R \sin t\end{pmatrix}.$$
Dann ist 
$$c'(t)= \begin{pmatrix}[r] - \rho \sin t \\ \rho \cos t \end{pmatrix}$$
und 
\begin{alignat*}{5}
L_h(\partial B_R) = L_h(c) \ &=&& \ \ \int_0^{2\pi} \frac{2 \sqrt{(-\rho \sin t)^2 + (\rho \cos t)^2}}{1- \left( (\rho \cos t)^2 + (\rho \sin t)^2\right)} \ \mathrm{d}t \\
&=&& \ \ \int_0^{2\pi} \frac{2\rho}{1-\rho^2} \ \mathrm{d}t \\
&=&& \ \ \frac{4\pi \rho}{1-\rho^2}.
\end{alignat*}
\item Wir wählen denselben Ball und rechnen 
\begin{alignat*}{5}
A_h(B_R)=\mu(B_R(0)) \ \ &=&& \ \ \int_0^{2\pi} \int_0^{\rho} \frac{4r}{(1-r^2)^2} \ \mathrm{d}r \mathrm{d}\theta \\
&=&& \ \ \int_0^{2\pi} \left[ \frac{?}{1-r^2} \right]^{\rho}_{0} \ \mathrm{d}\theta \\
&=&& \ \ \frac{4\pi}{1-\rho^2} - 4 \pi \\
&=&& \ \ \frac{4\pi \rho^2}{1-\rho^2}.
\end{alignat*}
Für $R \la \infty$ gilt $\rho\la 1$ und damit $\rho^2 \sim \rho$. Für die isoperimetrische Konstante gilt dann
$$\frac{\mu(B_R(0))}{L(\partial B_R)} = \rho \la 1,$$
was wir in der Vorlesung bereits gesehen haben. $\hfill \Box$
\end{compactenum}

\end{sol}

\end{prob}





\begin{prob}    %%AUFGABE 9.2
Sei $\Gamma$ eine Fuchssche Gruppe. Zeigen Sie, dass der Flächeninhalt eines Fundamentalbereichs eine Invariante der Gruppe $\Gamma$ ist: Es Seien $P_1, P_2 \subseteq \He^2$ zwei Fundamentalbereiche für $\Gamma$, sodass die Ränder hyperbolischen Flächeninhalt Null haben und es gelte $\mu(P_1)<\infty$. Dann gilt $\mu(P_1) = \mu(P_2)$.

\begin{sol}
Da $\mu(\partial P_i)=0$, gilt $\mu(P_i)= \mu(\overset{\circ}{P}_i)$ für $i=1,2$. Es gilt weiter 
$$P_1 \supseteq \bigcup_{\gamma \in \Gamma} (P_1 \cap \gamma \overset{\circ}{P}_2).$$
Da $P_2$ ein Fundamentalbereich für $\Gamma$ ist, gilt 
$$\gamma \overset{\circ}{P}_2 \cap \tilde{\gamma} \overset{\circ}{P}_2 = \emptyset$$
für alle $\gamma \neq \tilde{\gamma}\in \Gamma$, es folgt also
$$\mu(P_1)\geqslant \mu\left( \bigcup_{\gamma \in \Gamma} P_1 \cap \gamma \overset{\circ}{P}_2\right) = \sum_{\gamma \in \Gamma} \mu\left(P_1 \cap \gamma \overset{\circ}{P}_2\right) = \sum_{\gamma \in \Gamma} \mu(\gamma^{-1}(P_1) \cap \overset{\circ}{P}_2).$$
Da auch $P_1$ ein Fundamentalbereich für $\Gamma$ ist, gilt $\bigcup_{\gamma \in \Gamma} \gamma^{-1}(P_1) = \He^2$ und damit 
$$\mu(P_1) \geqslant \sum_{\gamma \in \Gamma}\mu(\gamma^{-1}(P_1) \cap \overset{\circ}{P}_2) = \mu\left( \left( \bigcup_{\gamma \in \Gamma} \gamma^{-1}(P_1)\right) \cap \overset{\circ}{P}_2\right) = \mu(\overset{\circ}{P}_2) = \mu(P_2).$$
Analog zeigt man die umgekehrte Ungleichung, woraus die Behauptung dann folgt. $\hfill \Box$
\end{sol}

\end{prob}


\begin{prob}   %%AUFGABE 9.3
Bestimmen Sie den Flächeninhalt des kleinsten hyperbolischen Dreieck, dessen Innenwinkel von der Form $\frac{\pi}{k}, \frac{\pi}{l}, \frac{\pi}{m}$ mit $k,l,m \in \mathbb{N}\setminus \{1\}$ sind.
\begin{sol}
Mit der Bedingung
$$\frac{\pi}{k} + \frac{\pi}{l} + \frac{\pi}{m} <\pi$$
und der Flächenformel
$$\mu(\Delta)= \pi-\frac{\pi}{k}-\frac{\pi}{l}-\frac{\pi}{m} = \pi\left( 1- \frac{1}{k} - \frac{1}{l}- \frac{1}{m}\right)$$
für hyperbolische Dreiecke findet man durch sukzessives Probieren $k=2$, $l=3$ und $m=7$. $\hfill \Box$
\end{sol}
\end{prob}
























\newpage




\titlespacing*{\section}{-16.5pt}{0pt}{20pt}
\renewcommand*\thesection{}
\section{Übungsblatt 10} %PARAGRAPH A.10
\renewcommand*\thesection{\arabic{section}}



\begin{prob}    %%AUFGABE 10.1

Sei $g \subseteq \He^2$ eine hyperbolische Geodätische. Die \textit{Spiegelung an} $g$ ist die Abbildung $\sigma_g; \He^2 \la \He^2$, welche durch folgende Eigenschaft charakterisiert ist: Für alle $z \in \He^2$ trifft das (eindeutig bestimmte) geodätische Segment $\overline{z \sigma_g(z)}$ von $z$ nach $\sigma_g(z)$ die Geodätische $g$ orthogonal und wird von dieser halbiert. Zeigen Sie:
\begin{compactenum}
\item Ist $g_0$ die positive imaginäre Achse, so gilt $\sigma_{g_0}(z)=-\overline{z}$ für alle $z \in \He^2$.
\item Ist $g$ eine beliebige Geodätische und $\phi$ eine hyperbolische Isometrie mit $g=\phi(g_0)$, dann gilt $\sigma_g= \phi \circ \sigma_{g_0}\circ \phi^{-1}$ und dies ist unabhängig von der Wahl von $\phi$.
\end{compactenum}


\begin{sol}
\begin{compactenum}
\item Es sei $z=(\lambda \cos \theta, \lambda \sin \theta)$ in Polarkoordinaten gegeben. Dann ist der Spiegelpunkt $-\overline{z} = (-\lambda \cos \theta, \lambda \sin \theta)= (\lambda \cos \pi- \theta, \lambda \sin \pi-\theta)$ und das geodätische Segment zwischen $z$ und $-\overline{z}$ ist gerade der parametrisierte Halbkreis mit Zentrum $0$ und Radius $\lambda$, schneidet $g_0$ also im Punkt $m:=(0,\lambda) = \left(\lambda \cos \frac{\pi}{2}, \lambda \sin \frac{\pi}{2}\right)$. Dann gilt Nach Lemma 2.10.3
$$d_h(m,z)=\ln \frac{\sin \frac{\pi}{2} \left( 1 + \cos \theta\right)}{\sin \theta \left( 1 + \cos \frac{\pi}{2}\right)} = \ln \frac{1 + \cos \theta}{\sin \theta}$$
und wegen 
$$(1+\cos \theta)(1-\cos\theta) = 1 - \cos^2\theta) = \sin^2\theta \quad \Longleftrightarrow \quad \frac{1+\cos\theta}{\sin\theta} = \frac{\sin \theta}{1-\cos \theta}$$
folgt
$$d_h(m,z)= \ln \frac{1+\cos\theta}{\sin \theta} = \ln \frac{\sin \theta}{1-\cos \theta}  = \ln \frac{ \sin \pi-\theta \left(1+\cos \frac{\pi}{2}\right)}{\sin \frac{\pi}{2} \left( 1+ \cos \pi-\theta\right)}=d_h(-\overline{z},m),$$
$g$ halbiert also die Geodätische. Weiter ist der Tangentialvektor an der Geodätischen senkrecht zu $g_0$, denn ist $\overline{-\overline{z} z}$ parametrisiert durch 
$c:[\theta,\pi-\theta] \la \He^2,\ t \mapsto \lambda ( \sin t + i \cos t),$
so gilt offenbar $m=c(0)$ und $c'(0)=-1 \perp i \parallel g_0$.

\item Sei nun $g$ beliebig und $\phi$ eine hyperbolische Isometrie mit $\phi(g_0)=g$. Für $z \in \He^2$ gilt dann 
\begin{alignat*}{5}
\overline{z \sigma_g(z)} \perp g \ \ &\Longleftrightarrow&& \ \ \overline{\phi^{-1}(z) \phi^{-1}(\sigma_g(z))} \perp \phi^{-1}(g) \\
&\Longleftrightarrow&& \ \ \overline{\phi^{-1}(z) \phi^{-1}(\sigma_g(z))} \perp g_0 \\
&\Longleftrightarrow&& \ \ \phi^{-1}(\sigma_g(z)) = \sigma_{g_0}(\phi^{-1}(z)) \\
&\Longleftrightarrow&& \ \ \phi^{-1} \circ \sigma_g = \sigma_{g_0}\circ \phi^{-1} \\
&\Longleftrightarrow&& \ \ \sigma_g = \phi \circ \sigma_{g_0} \circ \phi^{-1},
\end{alignat*}
wobei (i) und die Konformität von $\phi$ benutzt wurde. Wegen der Längentreue von $\phi$ gilt außerdem $d(z,g) = d(\sigma_g(z),g)$ genau dann, wenn $d\left( \phi^{-1}(z), g_0\right) = d\left( \phi^{-1}(\sigma_g(z)), g_0\right) = d\left( \sigma_{g_0}(\phi^{-1}(z)), g_0\right)$, was wir in (i) bereits gezeigt haben. Auch die Winkeltreue (also der senkrechte Schnitt) folgt aus der Konformität. Es bleibt noch die Unabhängigkeit von $\phi$ zu zeigen. Sei dazu $\psi \in \mathrm{Ism}(\He^2)$ eine weitere hyperbolische Isometrie mit $\psi(g_0)=g$. Wir müssen zeigen, dass dann 
$$\phi \circ \sigma_{g_0} \circ \phi^{-1} = \psi \circ \sigma_{g_0} \circ \psi^{-1}.$$
Betrachte dazu $m:= \psi^{-1} \circ \phi$. Dann gilt $m(g_0)=g_0 = i \R$. Welche Möbiustransformationen erhalten $i\R$? Sicher zum einen $C(z)=-\overline{z}$. Für die weiteren machen wir den Ansatz $$m= \matx{a}{b}{c}{d}$$ mit $m(i\R)=i\R$. Dann erhalten wir zwei Fälle:
\begin{compactenum}
\item[\textbf{Fall (a)}] $m(0)= \frac{b}{d}=0$, also $b=0$ und $d\neq 0$. Damit ist $m_1(z)= \frac{az}{cz+d}$. Schreibe $m_1(i)=ui$ für ein $u >0$. Dann ist 
$$ui=m_1(i)=\frac{ai}{ci+d},$$
woraus $c=0$ folgt, also $m_1(z)= \frac{a}{d}z$.
\item[\textbf{Fall (b)}] $m(\infty)=\frac{a}{c} = \infty$, also $d=0$ und $b \neq 0$. Damit ist $m_2(z)=\frac{az+b}{cz}$. Schreibe $m_2(i)=vi$ für ein $v >0$. Dann ist
$$vi=m_2(i) = \frac{ai+b}{ci},$$
woraus $a=0$ folgt, also $m_2(z)= \frac{b}{cz}$.
\end{compactenum}
Wir sehen nun 
$$(C \circ \sigma_{g_0} \circ C^{-1})(z) = ( C \circ \sigma_{g_0})(-\overline{z}) = C( z) = - \overline{z} = \sigma_{g_0}(z),$$
$$(m_1 \circ \sigma_{g_0} \circ m_1^{-1})(z) = (m_1 \circ \sigma_{g_0})\left( \frac{d}{a}z\right) = (m_1\left( - \frac{d}{a} \overline{z}\right) = - \overline{z} = \sigma_{g_0}(z),$$
$$(m_2 \circ \sigma_{g_0} \circ m_2^{-1})(z) = (m_2 \circ \sigma_{g_0})\left( \frac{b}{cz}\right) = (m_2\left( - \frac{b}{c \overline{z}}\right) = - \overline{z} = \sigma_{g_0}(z),$$
also ist auch $m\in \langle C,m_1,m_2\rangle$ und damit $m= m \circ \sigma_{g_0} \circ m^{-1}$. Damit folgt 
$$\phi \circ \sigma_{g_0} \circ \phi^{-1} = \psi \circ (\psi^{-1} \circ \phi) \circ \sigma_{g_0} \circ ( \phi^{-1} \circ \psi) \circ \psi^{-1} = \psi \circ (\psi^{-1} \circ \sigma_{g_0} \circ \psi ) \circ \psi^{-1} = \psi \circ \sigma_{g_0} \circ \psi^{-1}, $$
was zu zeigen war. $\hfill \Box$

\end{compactenum}

\end{sol}

\end{prob}




\begin{prob}   %%AUFGABE 10.2
Bestimmen Sie die Dreiecksgruppe $\Gamma_{\Delta}$ für das Fundamentaldreieck $\Delta=(2,3,7)$ \textit{(Schwarz Dreieck)}.


\begin{sol}
$\Delta$ is das Dreieck mit Innenwinkeln $\frac{\pi}{2}, \frac{\pi}{3}, \frac{\pi}{7}$. Nach Definition ist 
$$\Gamma_{\Delta} = \langle \sigma_a, \sigma_b, \sigma_c \ \vert \ \sigma_a^2, \sigma_b^2, \sigma_c^2, (\sigma_a\sigma_b)^7, (\sigma_b\sigma_c)^2, (\sigma_a \sigma_c)^3 \rangle.$$

\end{sol}

\end{prob}




\begin{prob} %%%AUFGABE 10.3
\begin{compactenum}
\item Bestimmen Sie einen Dirichlet-Fundamentalbereich $D(\Gamma_1)$ für die Gruppenoperation von $\Gamma_1= 3 \Z \times \Z$ auf $\E^2$ durch Translationen.
\item Bestimmen Sie einen Dirichlet-Fundamentalbereich $D(\Gamma_2)$ für die Gruppenoperation von $\Gamma_2=\langle z \mapsto z+a \rangle$ für $a >0$ auf $\He^2$.
\item Bestimmen Sie einen Dirichlet-Fundamentalbereich $D(\Gamma_3)$ für die Gruppenoperation von $\Gamma_3=\Big\langle z \mapsto \frac{\frac{\sqrt{3}}{2}z - \frac{1}{2}}{\frac{1}{2}z + \frac{\sqrt{3}}{2}z} \Big\rangle$ auf $\He^2$.


\end{compactenum}

\begin{sol}
In allen Teilaufgaben wollen wir Satz 3.1.3 benutzen.
\begin{compactenum}
\item Es wird der Punkt $p=(0,0)$ offenbar von keinem Gruppenelement $(z,w) \in 3\Z \times \Z$ fixiert, wir können also durch 
$$D(\Gamma_1)=D_p(\Gamma_1) = \{ (x,y) \in \E^2 \ \times d_e((x,y),(0,0)) \leqslant d_e((x+z,y+w),(0,0)) für alle (z,w) \in \Gamma_1 \} $$
einen Fundamentalbereich bestimmen. Wie in Beispiel 3.1.4 erhalten wir durch Mittelebenenkonstruktion
$$
\begin{tikzpicture}[scale=1.2]
\draw (0.05,0) arc(0:360:0.05);
\draw[fill=black] (0.05,0) arc(0:360:0.05);

\draw (03.05,0) arc(0:360:0.05);
\draw[fill=green] (3.05,0) arc(0:360:0.05);

\draw (-2.95,0) arc(0:360:0.05);
\draw[fill=green] (-2.95,0) arc(0:360:0.05);

\draw (0.05,1) arc(0:360:0.05);
\draw[fill=green] (0.05,1) arc(0:360:0.05);

\draw (0.05,-1) arc(0:360:0.05);
\draw[fill=green] (0.05,-1) arc(0:360:0.05);

\draw (3.05,1) arc(0:360:0.05);
\draw[fill=blue] (3.05,1) arc(0:360:0.05);

\draw (3.05,-1) arc(0:360:0.05);
\draw[fill=blue] (3.05,-1) arc(0:360:0.05);

\draw (-2.95,1) arc(0:360:0.05);
\draw[fill=blue] (-2.95,1) arc(0:360:0.05);

\draw (-2.95,-1) arc(0:360:0.05);
\draw[fill=blue] (-2.95,-1) arc(0:360:0.05);

\draw[thin, ->] (-4,0) -- (4,0);
\draw [thin, ->] (0,-1.5) -- (0,1.5);

\draw[color=green] (-3.5,0.5) -- (3.5,0.5);
\draw[color=green] (-3.5,-0.5) -- (3.5,-0.5);
\draw[color=green] (1.5,1) -- (1.5,-1);
\draw[color=green] (-1.5,-1) -- (-1.5,1);

\draw[color=blue] (-1.75,-0.25) -- (-1.25,1.25);
\draw[color=blue] (-1.75,0.25) -- (-1.25,-1.25);
\draw[color=blue] (1.25,1.25) -- (1.75,-0.25);
\draw[color=blue] (1.25,-1.25) -- (1.75,0.25);
\draw[thick, color=black] (-1.5,0.5) -- (1.5,0.5) -- (1.5,-0.5) -- (-1.5,-0.5) -- (-1.5,0.5);
\draw[fill=black, opacity=0.2] (-1.5,0.5) -- (1.5,0.5) -- (1.5,-0.5) -- (-1.5,-0.5) -- (-1.5,0.5);



\end{tikzpicture}
$$
gerade
$$D(\Gamma_1) = \left[ -\frac{3}{2}, \frac{3}{2}\right] \times \left[-\frac{1}{2}, \frac{1}{2}\right].$$
\newpage

\item Als nicht fixierten Punkt wählen wir $p=i$. Durch Mittelebenenkonstruktion erhalten wir 
$$
\begin{tikzpicture}
\draw[->,thin] (-4,0) -- (4,0);
\draw[->,thin] (0,-0.5) -- (0,3.5);
\draw[fill=black] (0.05,2) arc(0:360:0.05);

\draw[fill=green] (2.05,2) arc(0:360:0.05);
\draw (2.05,2) arc(0:360:0.05);

\draw[fill=green] (4.05,2) arc(0:360:0.05);
\draw (4.05,2) arc(0:360:0.05);

\draw[fill=green] (-1.95,2) arc(0:360:0.05);
\draw (-1.95,2) arc(0:360:0.05);

\draw[fill=green] (-3.95,2) arc(0:360:0.05);
\draw (-3.95,2) arc(0:360:0.05);

\draw[color=green] (-1,-0.25) -- (-1,2.5);
\draw[color=green] (1,-0.25) -- (1,2.5);

\draw[thick] (-1,2.5) -- (-1,0) -- (1,0) -- (1,2.5);
\draw[thick,dashed] (-1,2.5) -- (-1,3);
\draw[thick,dashed] (1,2.5) -- (1,3);

\draw[fill=black, opacity=0.2] (-1,3) -- (-1,0) -- (1,0) -- (1,3) -- (-1,3);
\end{tikzpicture}
$$
Das gibt uns 
$$D(\Gamma_2)=\left[-\frac{a}{2}, \frac{a}{2}\right] \times (0,\infty) = \left\{ z \in \He^2 \ \times \ \vert \Ree(Z)\vert \leqslant \frac{a}{2}\right\}.$$

\item Wegen 
$$A:= \begin{pmatrix}[rr] \frac{\sqrt{3}}{2} & -\frac{1}{2} \\ \frac{1}{2} & \frac{\sqrt{3}}{2} \end{pmatrix} = \begin{pmatrix}[rr] \cos \frac{\pi}{6} & -\sin \frac{\pi}{6} \\ \sin \frac{\pi}{6} & \cos \frac{\pi}{6} \end{pmatrix} \in \mathrm{SO}(2) = \mathrm{Stab}_{\mathrm{Isom}(\He^2)}(i)$$
ist, wird $i$ von allen $\gamma \in \Gamma_3$ fixiert, $i$ liegt also auf dem Rand aller Translate jedes Fundamentalbereichs. Weiter ist $A^{12}=I_2$ und $A^k\neq I_2$ für $0\leqslant k <12$, also $\vert \Gamma_3 \vert =12$, ein Fundamentalbereich besitzt also 12 Translate. Wähle $p=2i$. Die Punkte $\gamma \cdot p$ liegen auf dem Kreis um $i$ und sind Schnittpunkte von Geodätischen durch $i$ mit Winkeln $\frac{\pi}{6}, \frac{2\pi}{6}, \ldots$ zu $i \R_{\geqslant 0}$. Durch Mittelebenenkonstruktion erhält man wieder einen Fundamentalbereich. Geht man stattdessen via 
$$m: \He^2 \la \D^2, \qquad z \mapsto \frac{iz+1}{z+i}$$
zum Einheitskreismodell über, so ist $q=m(p)=\frac{1}{3}i$ und $0=m(i)$, die Geodätischen von oben verlaufen also durch den Ursprung, sind also Geraden. Der Ein Fundamentalbereich ist also durch das Zwölftel des Einheitskreises gegeben. $\hfill \Box$
\end{compactenum}
\end{sol}

\end{prob}






















\newpage




\titlespacing*{\section}{-16.5pt}{0pt}{20pt}
\renewcommand*\thesection{}
\section{Übungsblatt 11} %PARAGRAPH A.11
\renewcommand*\thesection{\arabic{section}}



\begin{prob}   %%AUFGABE 11.1
Es sei $r \in \mathbb{N}$, $r \geqslant 2$ und 
$$\Gamma(r)= \left\{ \matx{a}{b}{c}{d}\in \mathrm{PSL}_2(\Z) \ \Bigg \vert \matx{a}{b}{c}{d} \cong \matx{1}{0}{0}{1} \equiv r \right\}$$
Zeigen Sie, dass $\Gamma(r)$ für $r \geqslant 3$ torsionsfrei ist, also keine Untergruppen endlicher Ordnung besitzt. Gehen Sie dabei wie folgt vor:
\begin{compactenum}
\item Zeigen Sie: Ist $A=I+ rB \in \Gamma(r)$ und $A^m=I_2$, so gilt $rd \vert m$, wobei $d$ den größten gemeinsamen Teiler aller Einträge von $B$ ist.
\item Zeigen Sie weiter, dass für jedes nichttriviale Element $A\in \Gamma(r)$ gilt $r=\mathrm{ord}(A)$.
\end{compactenum}


\begin{sol}
\begin{compactenum}
\item Sei $\tilde{B}:= \frac{1}{\mathrm{ggt}(B)} B = \frac{1}{d}B$. Dann gilt 
\begin{alignat*}{5}
I_2 = A^m \ \ &=&& \ \ (I_2+rB)^m\\
&=&& \ \ (I_2+rd\tilde{B})^m \\
&=&& \ \ \sum_{k=0}^m \binom{m}{k} I_2^{m-k} \tilde{B}^k (rd)^k \\
&=&& \ \ I_2 + rdm\tilde{B} + \sum_{k=2}^m \binom{m}{k} r^kd^k \tilde{B}^k \\
&=&& \ \ I_2 + rdm \tilde{B} + (r^2d^2) \sum_{k=2}^m \binom{m}{k} r^{k-2}d^{k-2} \tilde{B}^k \\
&\equiv&& \ \ I_2 + rdm \tilde{B} \mod r^2d^2,
\end{alignat*}
also $rdm \tilde{B} \equiv 0 \mod r^2d^2$. Da $\mathrm{ggt}(\tilde{B})=1$, folgt $rdm \equiv 0 \mod r^2d^2$, also $r^2d^2 \vert rdm$ und damit $rd \vert m$, was zu zeigen war.

\item Es gelte nun $A^m=I_2$ für $A \in \Gamma(r)$, $A\neq I_2$. Sei zunächst $m=p$ eine Primzahl und $r = 2$. Dann folgt nach (i) $d=1$ und damit $r=m=2= \mathrm{ord}(A)$. Sei nun $r \geqslant 3$, also $m=p=r\geqslant 3$ und wieder $d=1$. Dann ist $mrd=r^2$. Außerdem gilt $r \big\vert \binom{m}{k}$ für $m\neq k \neq 0$. Damit folgt 
\begin{alignat*}{5}
I_2=A^m \ \ &=&& \ \ I_2 + rdm \tilde{B}  + r^2d^2 \sum_{k=2}^m \binom{m}{k} r^{k-2}d^{k-2} \tilde{B}^k \\
&=&& \ \ I_2 + r^2 \tilde{B} + r^2d^2 \binom{m}{2} \tilde{B}^2 + (r^2d^2) \sum_{k=3}^m \binom{m}{k} r^{k-2} \tilde{B}^k \\
\end{alignat*}
Dies gibt modulo $r^3d^3$ gerade $r^2 \tilde{B} \equiv 0 \mod r^3$, ein Widerspruch zu $\mathrm{ggt}(\tilde{B}) = 1$. Ist $m$ nun keine Primzahl, so gilt für jeden Primfaktor $p$ von $m$
$$I_2=A^m = \left( A ^{\frac{m}{p}}\right)^p,$$
ein Widerspruch zur obigen Betrachtung. $\hfill \Box$
\end{compactenum}

\end{sol}

\end{prob}




\begin{prob}    %%AUFGABE 11.2
Vervollständigen Sie den Beweis von Satz 4.3.1:
\begin{compactenum}
\item Zeigen Sie, dass die Abbildung
$$\beta: \mathbb{K}^n \la \D^n, \qquad (x_1, \ldots, x_{n+1}) \mapsto \left( \frac{x_1}{1+x_{n+1}}, \ldots, \frac{x_n}{1+x_{n+1}}, 0 \right)$$
eine Isometrie ist.
\item Zeigen Sie, dass die Abbildung
$$\gamma: \Loid^n \la \mathbb{K}^n, \qquad (x_1,\ldots, x_n) \mapsto \left( \frac{x_1}{x_{n+1}}, \dots, \frac{x_n}{x_{n+1}}, \frac{1}{x_{n+1}}\right)$$
eine Isometrie ist.
\end{compactenum}

\begin{sol}
Wir verfahren in völliger Analogie zum Beweis, dass $\alpha$ eine Isometrie ist und zeigen, dass $\beta$ und $\gamma$ Weglängen erhalten.
\begin{compactenum}
\item Sei also
$$c:[0,1] \la \mathbb{K}^n, \qquad t \mapsto c(t)=\left(x_1(t), \ldots, c_{x+1}(t)\right)=: (x_1, \ldots, x_{n+1})$$
eine differenzierbare Kurve in $\mathbb{K}^n$ sowie $$\beta \circ c: [0,1] \la \D^n, \qquad t \mapsto \left(\frac{x_1(t)}{1+x_{n+1}(t)}, \ldots, \frac{x_n(t)}{1+x_{n+1}(t)}, 0\right) := (y_1,\ldots  y_{n+1})$$ die Bildkurve. Es gilt
$$L_{\mathbb{K}^n}(c)= \int_0^1 \frac{1}{x_{n+1}(t)} \sqrt{\sum_{k=1}^{n+1} x_k'(t)^2} \ \mathrm{d}t =: \int_0^1 A(t) \ \mathrm{d}t$$
sowie 
$$L_{\Loid^n}(\beta \circ c) = \int_0^1 \frac{2}{1-\sum_{k=1}^n y_k(t)^2} \sqrt{\sum_{k=1}^n y_k'(t)^2} \ \mathrm{d}t=: \int_0^1 B(t) \ \mathrm{d}t,$$
wir zeigen also Gleichheit der Integranden. Es gilt 
$$y_k' = \frac{\mathrm{d}y_k}{\mathrm{d}t} = \frac{x_k'(1+x_{n+1}) - x_{n+1}' x_k}{(1+x_{n+1})^2} = \frac{1}{1+x_{n+1}} \left( x_k' - \frac{x_kx_{n+1}'}{1+x_{n+1}}\right).$$
Weiter erhalten wir aus $\sum_{k=1}^{n+1} x_k^2 =1$ durch Ableiten
$$x_{n+1} x_{n+1}' = - \sum_{k=1}^n x_k x_k'.$$
Das Quadrat des zweiten Integranden ergibt sich also zu 
\begin{alignat*}{5}
B(t)^2 \ \ &=&& \ \ \left( \frac{2}{1-\sum_{k=1}^n y_k^2} \sqrt{\sum_{k=1}^n (y_k')^2}\right)^2 \\
&=&& \ \ \frac{4}{\left( 1- \sum_{k=1}^n y_k^2\right)^2} \left( \sum_{k=1}^n (y_k')^2\right) \\
&=&& \ \ \frac{4}{\left( 1- \frac{1}{1+x_{n+1}^2} \sum_{k=1}^n x_k^2\right)^2} \sum_{k=1}^n \left( \frac{1}{1+x_{n+1}} \left( x_k' - \frac{x_kx_{n+1}'}{1+x_{n+1}}\right) \right)^2 \\
&=&& \ \ \frac{4}{\left( 1- \frac{1-x_{n+1}^2}{1+x_{n+1}^2}\right)^2} \frac{1}{(1+x_{n+1})^2} \left( \sum_{k=1}^n (x_k')^2 + \sum_{k=1}^n \frac{(x_{n+1}')^2x_k^2}{(1+x_{n+1})^2} - \sum_{k=1}^n \frac{2x_kx_k'x_{n+1}'}{1+x_{n+1}} \right)\\
&=&& \ \ \frac{4}{\left( \frac{2 x_{n+1}^2}{1+x_{n+1}}\right)^2} \frac{1}{(1+x_{n+1})^2} \left( \sum_{k=1}^n (x_k')^2 + (x_{n+1}')^2 \left( \frac{1-x_{n+1}^2}{(1+x_{n+1})^2} + \frac{2x_{n+1}}{1+x_{n+1}}\right) \right) \\
&=&& \ \ \frac{1}{x_{n+1}^2} \left( \sum_{k=1}^n (x_k')^2 + (x_{n+1}')^2 \left( \frac{1-x_{n+1}^2 + 2 x_{n+1} + 2 x_{n+1}^2}{(1+x_{n+1})^2} \right) \right) \\
&=&& \ \ \frac{1}{x_{n+1}^2} \left( \sum_{k=1}^n (x_k')^2 + (x_{n+1}')^2  \frac{(1+x_{n+1})^2}{(1+x_{n+1})^2} \right) \\
&=&& \ \ \frac{1}{x_{n+1}^2} \left( \sum_{k=1}^{n+1} (x_k')^2\right)\\
&=&& \ \ A(t)^2
\end{alignat*}
Insgesamt folgt damit für $p,q \in \mathbb{K}^n$
$$d_{\mathbb{K}^n}(p,q) = \inf_{c \in \Omega_{pq}(\mathbb{K}^n)} L_{\mathbb{K}^n}(c) = \inf_{c \in \Omega_{pq}(\mathbb{K}^n)} L_{\mathbb{D}^n}(\beta \circ c) \geqslant \inf_{c \in \Omega_{\beta(p)\beta(q)}(\mathbb{D}^n)}(c) = d_{\D^n}(\beta(p), \beta(q))$$
und auch umgekehrt, also die Behauptung.

\item Wir verfahren analog. Sei
$$c:[0,1] \la \Loid^n, \qquad t \mapsto c(t)=(x_1(t), \ldots, x_{n+1}(t)) =: (x_1, \ldots, x_{n+1})$$
eine differenzierbare Kurve und
$$\gamma \circ c: [0,1] \la \mathbb{K}^n, \qquad t \mapsto \left(\frac{x_1(t)}{x_{n+1}(t)}, \ldots, \frac{x_n(t)}{x_{n+1}(t)}, \frac{1}{x_{n+1}(t)}\right) =: (y_1, \ldots, y_n)$$
die Bildkurve. Es gilt
$$L_{\Loid^n}(c) = \int_0^1 \sqrt{\sum_{k=1}^n x_k'(t)^2 - x_{n+1}'(t)^2} \ \mathrm{d}t =: \int_0^1 A(t) \ \mathrm{d}t$$
sowie
$$L_{\mathbb{K}^n}(\gamma \circ c) = \int_0^1 \frac{1}{y_{n+1}(t)} \sqrt{\sum_{k=1}^{n+1} y_k'(t)^2} \ \mathrm{d}t =: \int_0^1 B(t) \ \mathrm{d}t,$$
wir zeigen also Gleichheit der Integranden. Es gilt
$$y_k'= \frac{\mathrm{d}y_k}{\mathrm{d}t} = \frac{x_k x_{n+1} - x_{n+1}' x_k}{x_{n+1}^2} = \frac{1}{x_{n+1}} \left( x_k' - \frac{x_{n+1}' x_k}{x_{n+1}}\right)$$
für $k \in \{1, \ldots, n\}$ sowie 
$$y_{n+1}' = \frac{\mathrm{d}}{\mathrm{d}t} \frac{1}{x_{n+1}} = - \frac{x_{n+1}'}{x_{n+1}^2}.$$
Weiter folgt aus
$$\sum_{k=1}^n x_k^2 - x_{n+1}^2 = -1 \quad \Longleftrightarrow \quad \sum_{k=1}^n x_k^2 = x_{n+1}^2 -1$$
durch Ableiten 
$$\sum_{k=1}^n x_k x_k' = x_{n+1} x_{n+1}'.$$
Dann erhalten wir 
\begin{alignat*}{5}
B(t)^2 \ \ &=&& \ \ \frac{1}{y_{n+1}^2} \left( \sum_{k=1}^{n+1} (y_k')^2\right) \\
&=&& \ \ x_{n+1}^2 \left(\sum_{k=1}^n \left( \frac{1}{x_{n+1}} \left( x_k' - \frac{x_{n+1}'x_k}{x_{n+1}}\right) \right)^2 + \left( \frac{x_{n+1}'}{x_{n+1}^2}\right)^2 \right) \\
&=&& \ \ \sum_{k=1}^n (x_k')^2 + \sum_{k=1}^n \frac{x_{n+1}'^2 x_k^2}{x_{n+1}^2} - 2 \sum_{k=1}^n \frac{x_k x_k' x_{n+1}'}{x_{n+1}} +  \frac{(x_{n+1}')^2}{x_{n+1}^2} \\
&=&& \ \ \sum_{k=1}^n (x_k')^2 + \frac{(x_{n+1}')^2}{x_{n+1}^2} \left( x_{n+1}^2-1\right) - 2 \frac{x_{n+1}'}{x_{n+1}} x_{n+1} x_{n+1}' + \frac{(x_{n+1}')^2}{x_{n+1}^2} \\
&=&& \ \ \sum_{k=1}^n (x_k')^2 +(x_{n+1}')^2 - \frac{(x_{n+1}')^2}{x_{n+1}^2} - 2 (x_{n+1}')^2 + \frac{(x_{n+1}')^2}{x_{n+1}^2} \\
&=&& \ \ \sum_{k=1}^n (x_k')^2 - (x_{n+1}')^2 \\
&=&& \ \ A(t)^2,
\end{alignat*}
woraus wie oben die Behauptung folgt. $\hfill \Box$


\end{compactenum}

\end{sol}

\end{prob}



\begin{prob}   %%AUFGABE 11.3
Es sei $\Loid^n$ das Hyperboloid-Modell des $n$-dimensionalen hyperbolischen Raumes und $q_L(x,y):= x \star y := x_1y_1 + \ldots + x_n y_n - x_{n+1}y_{n+1}$ das Lorentz-Produkt von $x,y \in \R^{n+1}$. Weiter sei $\Vert x\Vert_L:= \sqrt{q_L(x,x)}$ und $l(x,y):=\Vert x-y \Vert_L$. Zeigen Sie:
\begin{compactenum}
\item $q_L$ ist symmetrisch und bilinear.
\item $l$ ist symmetrisch und positiv definit auf $\Loid^n$ \textit{(Hinweis: Sie können ohne Einschränkung $x=(0,\ldots, 0,x_{n+1})$ annehmen)}.
\item $l$ ist keine Metrik auf $\Loid^n$.

\end{compactenum}


\begin{sol}

\begin{compactenum}

\item Es gilt für $x,y \in \R^{n+1}$
$$q_L(x,y) = \sum_{k=1}^n x_ky_k - x_{n+1}y_{n+1} =\sum_{k=1}^n y_k x_k - y_{n+1}x_{n+1} = q_L(y,x)$$
und für $x,y,z\in \R^{n+1}$, $\alpha \in \R$
\begin{alignat*}{5}
q_L(\alpha x + y, z ) \ \ &=&& \ \ \sum_{k=1}^n (\alpha x_k+y_k) z_k - (\alpha x_{n+1} + y_{n+1})z_{n+1} \\ 
&=&& \ \ \alpha \sum_{k=1}^n x_k z_k + \sum_{k=1}^n y_kz_k - \alpha x_{n+1}z_{n+1} - y_{n+1}z_{n+1}\\
&=&& \ \ \alpha \left( \sum_{k=1}^n x_kz_k - x_{n+1}z_{n+1}\right) + \sum_{k=1}^n y_kz_k - y_{n+1}z_{n+1} \\
&=&& \ \ \alpha q_L(x,z) + q_L(y,z).
\end{alignat*}

\item Für $x,y \in \Loid^n$ gilt 
$$l(x,y)=\Vert x-y \Vert_L = \sqrt{q_L(x-y,x-y)} = \sqrt{(-1)^2 q_L(y-x,y-x)} = \Vert y-x \Vert_L = l(y,x).$$
Weiter gilt freilich $l(x,x)=0$. Seien nun $(\hat{x},x_{n+1})=x\neq y=(\hat{y},y_{n+1}) \in \Loid^n$. Wir müssen zeigen, dass $l(x,y) >0$. Gelte zunächst $x_{n+1}=y_{n+1}$. Dann ist $\hat{x}\neq \hat{y}$ und damit
$$l(x,y)^2 = (x_1-y_1)^2 + \ldots + (x_n-y_n)^2 - (x_{n+1}-y_{n+1})^2 = \Vert \hat{x} - \hat{y} \Vert_e >0.$$
Sei nun also $x_{n+1}\neq y_{n+1}$. Dann gilt 
\begin{alignat*}{5}
0<(x_{n+1}-y_{n+1})^2 \ \ &\Longleftrightarrow&& \ \ x_{n+1}^2 - 2x_{n+1}y_{n+1} + y_{n+1}^2 >0 \\
&\Longleftrightarrow&& \ \ x_{n+1}^2 y_{n+1}^2 - 2x_{n+1} y_{n+1} +1 > x_{n+1}^2 y_{n+1}^2 - x_{n+1}^2 - y_{n+1}^2  1 \\
&\Longleftrightarrow&& \ \ 0 \leqslant (x_{n+1}y_{n+1}-1)^2 > (x_{n+1}^2-1)(y_{n+1}^2-1) \\
&\Longleftrightarrow&& \ \ x_{n+1}y_{n+1}-1 > \sqrt{(x_{n+1}^2-1)(y_{n+1}^2-1)} \\
&\Longleftrightarrow&& \ \ -1 \geqslant \sqrt{(x_{n+1}^2-1)(y_{n+1}^2-1)} - x_{n+1}y_{n+1} \quad (*)
\end{alignat*}
Weiter gilt 
\begin{alignat*}{5}
l(x,y)^2 \ \ &=&& \ \ \sum_{k=1}^n (x_k-y_k)^2 - (x_{n+1}-y_{n+1})^2 \\
&=&& \ \ \sum_{k=1}^n x_k^2 - 2 \sum_{k=1}^n x_ky_k + \sum_{k=1}^n y_k^2 - x_{n+1}^2 + 2x_{n+1}y_{n+1} - y_{n+1}^2 \\
&=&& \ \left( \sum_{k=1}^n x_k^2 - x_{n+1}^2\right) + \left( \sum_{k=1}^n y_k^2 - y_{n+1}^2\right) - 2 \sum_{k=1}^n + 2 x_{n+1}y_{n+1}\\
&=&& \ \ -2 -2\left( \sum_{k=1}^n x_ky_k -x_{n+1}y_{n+1}\right).
\end{alignat*}
Die Cauchy-Schwarz-Ungleichung liefert
$$\sum_{k=1}^n x_k y_k \leqslant \sqrt{\left( \sum_{k=1}^n x_k^2\right) \left( \sum_{k=1}^n y_k^2\right)},$$
wir erhalten also 
\begin{alignat*}{5}
l(x,y)^2 \ \ &=&& \ \ \sum_{k=1}^n (x_k-y_k)^2 - (x_{n+1}-y_{n+1})^2 \\
&=&& \ \ -2 -2\left( \sum_{k=1}^n x_ky_k -x_{n+1}y_{n+1}\right) \\
&\geqslant&& \ \ -2 -2 \left(\sqrt{\left( \sum_{k=1}^n x_k^2\right) \left( \sum_{k=1}^n y_k^2\right)} - x_{n+1}y_{n+1}\right) \\
&=&& \ \ -2 -2\left(\sqrt{(x_{n+1}^2-1)(y_{n+1}^2-1)} - x_{n+1}y_{n+1}\right) \\
&\overset{(*)}{\hspace{-3.0 pt}> }&& \ \ -2 -2 (-1)\\
&=&& \ \ 0,
\end{alignat*}
was zu zeigen war.

\item Offenbar muss die Dreiecksungleichung verletzt sein. In der Tat gilt für 
$$x=(1, 0, \ldots, 0, \sqrt{2}), \qquad y=(0,\ldots, 0, 1), \qquad z=(-1,0,\ldots, 0, \sqrt{2})$$
die Ungleichung
$$l(x,y)+l(y,z) = \sqrt{1-(\sqrt{2}-1)^2} + \sqrt{1-(\sqrt{2}-1)^2} < 1+1=2 = l(x,z),$$
also definiert $l$ keine Metrik auf $\Loid^n$.

\end{compactenum}
\end{sol}

\end{prob}
























\newpage




\titlespacing*{\section}{-16.5pt}{0pt}{20pt}
\renewcommand*\thesection{}
\section{Übungsblatt 12} %PARAGRAPH A.12
\renewcommand*\thesection{\arabic{section}}






\begin{prob}   %%AUFGABE 12.1

Ein Punkt $z \in \R^{n+1}$ heißt \textit{zeitartig}, falls $q_L(z)<0$. Seien $x,y \in \R^{n+1}$ mit $x_{n+1}y_{n+1}>0$. Dann definiert man den zeitartigen Winkel $\sphericalangle(x,y)$ zwischen $x$ und $y$ durch 
$$\cosh \sphericalangle(x,y) = \frac{x\star y }{\Vert x\Vert_L \Vert y\Vert_L}.$$
Zeigen Sie:
\begin{compactenum}
\item Es gilt $\sphericalangle(x,y)=0$ genau dann, wenn $x=ty$ für ein $t\in \R$.
\item Für $x,y \in \Loid^{n}$ gilt $\sphericalangle(x,y)= d_{\Loid^n}(x,y)$.
\end{compactenum}

\begin{sol}
\begin{compactenum}
\item Seien zunächst $x$ und $y$ linear abhängig, also $x=ty$ für ein $r \in R$. Dann gilt 
$$\frac{x\star y}{\Vert x \Vert_L \Vert y \Vert_L} = \frac{t (y \star y)}{ \Vert ty \Vert_L \Vert y \Vert_L} = \frac{t \Vert y \Vert_L^2}{t \Vert y \Vert_L^2} = 1,$$
also
$\sphericalangle(x,y) = \mathrm{areacosh} \hspace{1.5pt} 1 = 0.$
Seien nun $x$ und $y$ nicht linear abhängig. Dann schreibe $y= \alpha x + x^{\perp}$ mit einem zu $y$ orthogonalen Anteil $x^{\perp}\neq0$. Dann gilt 
$$x \star y= x \star ( \alpha x + x^{\perp}) = \alpha (x \star x) + x \star x^{\perp} = \alpha (x \star x)$$
und 
$$y \star y = (\alpha x + x^{\perp})\star (\alpha x + x^{\perp}) = 2 \alpha (x \star x^{\perp}) + x^{\perp} \star x^{\perp} + \alpha^2 (x \star x) = \Vert x^{\perp}\Vert_L^2 + \alpha^2 \Vert x \Vert_L^2.$$
Weiter gilt 
$$x \star x^{\perp} = \sum_{k=1}^n x_kx_k^{\perp} - x_{n+1}x_{n+1}^{\perp} \leqslant \sqrt{\sum_{k=1}^n x_k^2} \sqrt{\sum_{k=1}^n \left(x_k^{\perp}\right)^2} - x_{n+1}x_{n+1}^{\perp}.$$
Nach Voraussetzung ist $x$ zeitartig, es gilt also 
$$ \sum_{k=1}^n x_k^2 < x_{n+1}^2$$
 und wir können ohne Einschränkung $x_{n+1}, y_{n+1}>0$ annehmen. Dann folgt
 $$0 \leqslant x_{n+1} \sqrt{\sum_{k=1}^n \left(x_k^{\perp}\right)^2} - x_{n+1}x_{n+1}^{\perp} = x_{n+1}\left( \sqrt{\sum_{k=1}^n \left( x_k^{\perp}\right)^2} - x_{n+1}^{\perp}\right),$$
 also 
 $$\vert x_{n+1}^{\perp} \vert < \sqrt{\sum_{k=1}^n \left( x_k^{\perp}\right)^2}$$
 und damit 
 $$x^{\perp} \star x^{\perp} = \sum_{k=1}^n \left( x_k^{\perp}\right)^2 - \left(x_{n+1}^{\perp}\right)^2 >0,$$
 wegen der Zeitartigkeit von $x$ also $-(x \star x)\left( x^{\perp} \star x^{\perp}\right) >0$.
 Wir erhalten
 $$(x \star y)^2 = \alpha^2 ( x \star x)^2 = (x \star x)\left((y \star y) - (x^{\perp} \star x^{\perp})\right)> (x\star x)(y \star y) = \Vert x \Vert_L^2 \Vert y \Vert_L^2$$
und damit $$\cosh \sphericalangle (x,y) = \frac{x \star y}{\Vert x \Vert_L \Vert y \Vert_L} >1,$$
was $\sphericalangle (x,y) \neq 0$ impliziert.

\item Aus der Vorlesung (Satz 4.5.5) wissen wir 
$$ \cosh ( d_{\Loid^n}(x,y)) = - (x \star y) =\frac{x \star y}{\Vert x \Vert_L \Vert y \Vert_L} = \cosh \sphericalangle(x,y).$$
Injektivität von $\cosh$ auf $[0,\infty)$ liefert $d_{\Loid^n}(x,y)=\sphericalangle(x,y)$, was zu zeigen war. $\hfill \Box$
 


\end{compactenum}
\end{sol}


\end{prob}







\begin{prob}
\begin{compactenum}
\item Zeigen Sie, dass für $P \in \mathrm{SL}_2(\R)$ die Abbildung 
$$\Phi_P: \Loid^2 \la \Loid^2, \qquad (x,y,z) \mapsto (x',y',z')$$
definiert durch 
$$\matx{z'-y'}{x'}{x'}{z'+y'} = P \matxx{z-y}{x}{x}{z+y}P^T$$
eine lineare Isometrie von $\Loid^2$ ist.
\item Zeigen Sie, dass $\sigma: \mathrm{SL}_2(\R) \la \mathcal{O}^+(2,1)$, $P\mapsto \Phi_P$ ein Gruppenhomomorphismus mit $\ker \sigma = \{\pm I_2\}$ ist.
\end{compactenum}

\begin{sol}
\begin{compactenum}
\item Bestimme zunächst die Abbildungsmatrix $M_P$. Schreibe $P=\matx{a}{b}{c}{d}$ mit $ad-bc=1$. Dann gilt
\begin{alignat*}{5}
A\ \ &:=&& \ \ \matx{z'-y'}{x'}{x'}{z'+y'}\\
&=&& \ \ \matxx{a}{b}{c}{d} \matx{z-y}{x}{x}{z+y} \matx{a}{c}{b}{d} \\
&=&& \ \ \matxx{az-ay+bx}{ax+bz+by}{cz-cy+dx}{cx+dz+dy} \matx{a}{c}{b}{d} \\
&=&& \ \ \matxx{a^2z-a^2y+abx+abx+b^2z+b^2y}{acz-acy+bcx+bdx+bdz+bdy}{acz-acy+bcx+bdx+bdz+bdy}{c^2z-c^2y+cdx+cdx+d^2z+d^2y} \\
&=&& \ \ \matxx{2abx+y(b^2-a^2)+z(b^2+a^2)}{x(ad+bc)+y(bd-ac)+z(ad+bc)}{x(ad+bc)+y(bd-ac)+z(bd+ac)}{2cdx+y(d^2-c^2)+z(d^2+c^2)},
\end{alignat*}
also 
$$x'=x(ad+bc)+y(bd-ac) + z(bd+ac),$$
$$y'=\frac{\left( (z'+y')-(z'-y')\right)}{2} = x(cd-ab) + \frac{y}{2}(a^2-b^2-c^2+d^2) + \frac{z}{2}(-a^2-b^2+c^2+d^2),$$
$$z'=\frac{\left((z'+y')+(z'-y')\right)}{2} = x(ab+cd) + \frac{y}{2}(-a^2+b^2-c^2+d^2) + \frac{z}{2}(a^2+b^2+c^2+d^2).$$
Die Abbildungsmatrix ist dann gegeben durch 
$$M_P:= \Matx{ad+bc}{bd-ac}{bd+ac}{cd-ab}{\frac{1}{2}(a^2-b^2-c^2+d^2}{\frac{1}{2}(-a^2-b^2+c^2+d^2)}{ab+cd}{\frac{1}{2}(-a^2+b^2-c^2+d^2)}{\frac{1}{2}(a^2+b^2+c^2+d^2)}$$
Wir zeigen nun, dass $M \in \mathcal{O}^+(2,1)$. Es gilt $m_{33}=\frac{1}{2}(a^2+b^2+c^2+d^2) >0$ und mit 
$$J=\Matx{1}{0}{0}{0}{1}{0}{0}{0}{-1}$$
rechnet man 
$$M_PJM_P^T J= \Matx{b2c^2+a^2d^2-2abcd}{0}{0}{0}{b^2c^2+a^2d^2-2abcd}{0}{0}{0}{b^2c^2+a^2d^2-2abcd} = I_3$$
und Satz 4.5.3 liefert die Behauptung.


\item Für $P,Q \in \mathrm{SL}_2(\R)$ mit Abbildunsgmatrizen $M_P, M_Q \in \mathcal{O}^+(2,1)$ und $(x,y,z) \in \Loid^2$ gilt mit Regeln aus der linearen Algebra
\begin{alignat*}{5}
\left( \sigma(P) \circ \sigma(Q)\right)(x,y,z)
& = && \ \sigma(P)\left( \sigma(Q)(x,y,z)\right) \\
 &=&& \ \sigma(P)(M_Q(x,y,z))\\
& =&& \ (M_P M_Q)(x,y,z)\\
 &=&& \ M_{PQ}(x,y,z)\\
 &=&& \ \sigma(PQ)(x,y,z).
 \end{alignat*}
Zeige nun noch die Aussage über den Kern: Ist $P \in \ker \sigma$, so gilt $\Phi_P=\id$, also $M_P=I_3$. Dann gilt (zum Beispiel)
$$0=bd-ac=bd+ac,$$
also $b=0$ oder $d=0$ sowie $a=0$ oder $c=0$. Das sind vier Fälle:
\begin{compactenum}
\item[\textbf{Fall (a)}] $a=0=b$ widerspricht $m_{11}=1$.
\item[\textbf{Fall (b)}] $c=0=d$ widerspricht $m_{11}=1$.
\item[\textbf{Fall (c)}] $a=0=d$ liefert $1=\frac{1}{2}(-b^2-c^2) <0$, einen Widerspruch.
\item[\textbf{Fall (d)}] Dann folgt $b=0=c$. Wegen $ad=1$ und $1=\frac{1}{2}(a^2+d^2)$, $0=\frac{1}{2}(a^2-d^2)$ folgt $a^2=d^2=1$, also $a=d=\pm 1$ und damit $P=\pm I_2$.
\end{compactenum}

\end{compactenum}
\end{sol}

\end{prob}


\begin{prob}   %%AUFGABE 12.3
Fortsetzung von Aufgabe 12.2.
\begin{compactenum}
\item Zeigen Sie, dass sich jede Matrix $M \in \mathcal{O}^+(2,1)$ als Produkt $M=R_{\alpha} L_x R_{\beta}$ von Matrizen der Form 
$$R_{\alpha}=\Matx{\cos \alpha}{\sin \alpha}{0}{-\sin \alpha}{\cos\alpha}{0}{0}{0}{1}, \qquad L_x = \Matx{\cosh x}{0}{\sinh x}{0}{1}{0}{\sinh x}{0}{\cosh x}$$
schreiben lässt.
\item Folgern Sie, dass $\sigma: \mathrm{SL}_2(\R) \la \mathcal{O}^+(2,1)$ surjektiv ist.
\item Folgern Sie $\mathrm{PSL}_2(\R) \cong \mathcal{O}^+(2,1)$.
\end{compactenum}

\begin{sol}
\begin{compactenum}
\item Für $\alpha, \beta,x \in \R$ gilt 
\begin{alignat*}{5}
A\ \ &:=&& \ \ \Matx{\cos \alpha}{\sin \alpha}{0}{-\sin \alpha}{\cos \alpha}{0}{0}{0}{1} \Matx{\cosh x}{0}{\sinh x}{0}{1}{0}{\sinh x}{0}{\cosh x} \Matx{\cos \beta}{\sin \beta}{0}{-\sin \beta}{\cos \beta}{0}{0}{0}{1} \\
&=&& \ \ \Matx{\cos \alpha \cos \beta \cosh x - \sin \alpha \sin \beta}{\sin \alpha \cos \beta + \cos \alpha  \sin \beta \cosh x}{\cos \alpha \sinh x}{-\sin \alpha \cos \beta \cosh x - \cos \alpha \sin \beta}{\cos \alpha \cos \beta - \sin \alpha \sin \beta \cosh x}{-\sin \alpha \sinh x}{\cos \beta \sinh x}{\sin \beta \sinh x}{\cosh x}
\end{alignat*}
Sei nun $$M=(m_1\vert m_2 \vert m_3)= \Matx{m_{11}}{m_{12}}{m_{13}}{m_{21}}{m_{22}}{m_{23}}{m_{31}}{m_{32}}{m_{33}} \in \mathcal{O}^+(2,1).$$
Dann gilt
$$-1=m_3 \star m_3 = m_{13}^2+m_{23}^2 -m_{33}^2 \geqslant - m_{33}^2,$$
also $m_{33} \geqslant 1$. Nach dem Zwischenwertsatz gibt es also $x \in \R$ mit $m_{33}=\cosh x$. Mit obiger Gleichung folgt nun 
$$m_{13}^2+m_{23}^2 = -1 + m_{33}^2 = -1+ \cosh^2x = \sinh^2x,$$
also $(m_{13}, m_{23}) \in \Sph^1(\sinh x)$. Wir können damit $\tilde{\alpha} \in \R$ so wählen, dass 
$$m_{13}= \cos \tilde{\alpha} \sinh x, \qquad m_{23}= \sin \tilde{\alpha} \sinh x.$$
Setze nun $\alpha:=-\tilde{\alpha}$. Dann ist
$$m_3 = \Vecdr{m_{13}}{m_{23}}{m_{33}} = \Vecdr{\cos \alpha \sinh x}{- \sin \alpha \sinh x}{\cosh x}.$$
Nach Definition ist $M \in \mathcal{O}^+(2,1)$ genau dann, wenn $M^T \in \mathcal{O}^+(2,1)$, es muss also auch die Gleichung
$$-1= m^T_3 \star m^T_* = m_{31}^2 + m_{32}^2 - m_{33}^2 = m_{31}^2 + m_{32}^2 - \cosh^2x$$
gelten, also 
$$m_{31}^2 + m_{32}^2 = -1 + \cosh^2x = \sinh^2x.$$
So wie eben können wir also $\beta \in \R$ so wählen, dass 
$$m_{31} = \cos \beta \sinh x, \qquad m_{32} = \sin \beta \sinh x.$$
Es gilt nun also 
$$M=\Matx{m_{11}}{m_{12}}{\cos \alpha \sinh x}{m_{21}}{m_{22}}{-\sin \alpha \sinh x}{\cos \beta \sinh x}{\sin \beta \sinh x}{\cosh x}.$$
Eine weitaus aufwändigere Rechnung zeigt, dass die übrigen Bedingungen die verbleibenden Einträge ebenfalls eindeutig festlegen.


\item Sei die lineare Isometrie $F:\Loid^n \la \Loid^n$ gegeben durch $A \in \mathcal{O}^+(2,1)$ und schreibe $A=R_{\alpha}L_x R_{\beta}$. Betrachte die Matrizen 
$$P_{\alpha}=\matxx{\cos \frac{\alpha}{2}}{\sin \frac{\alpha}{2}}{-\sin \frac{\alpha}{2}}{\cos \frac{\alpha}{2}}, \qquad Q_x=\matxx{\cosh \frac{x}{2}}{\sinh \frac{x}{2}}{\sinh \frac{x}{2}}{\cosh \frac{x}{2}}, \qquad P_{\beta}=\matxx{\cos \frac{\beta}{2}}{\sin \frac{\beta}{2}}{-\sin \frac{\beta}{2}}{\cos \frac{\beta}{2}}.$$
in $\mathrm{SL}_2(\R)$. Mit der Formel aus Aufgabe 12.2(i) und den Additionstheoreme ergeben sich die Abbildungsmatrizen gerade zu 
$$M_{\sigma(P_{\alpha})}=\Matx{\cos^2 \frac{\alpha}{2}- \sin^2 \frac{\alpha}{2}}{2 \cos \frac{\alpha}{2} \sin \frac{\alpha}{2}}{0}{-2 \cos \frac{\alpha}{2} \sin \frac{\alpha}{2}}{\cos^2 \frac{\alpha}{2}- \sin^2 \frac{\alpha}{2}}{0}{0}{0}{1} = \Matx{\cos \alpha}{\sin \alpha}{0}{-\sin \alpha}{\cos \alpha}{0}{0}{0}{1} = R_{\alpha},$$
$$M_{\sigma(P_{\beta})}=R_{\beta},$$
$$M_{\sigma(Q_x)} = \Matx{\cosh^2 \frac{x}{2} + \sinh^2 \frac{x}{2}}{0}{2 \cosh \frac{x}{2} \sinh \frac{x}{2}}{0}{1}{0}{2 \cosh \frac{x}{2} \sinh \frac{x}{2}}{0}{\cosh^2 \frac{x}{2} + \sinh^2 \frac{x}{2}}= \Matx{\cosh x}{0}{\sinh x}{0}{1}{0}{\sinh x}{0}{\cosh x}.$$
Setze nun $S:=P_{\alpha}Q_xP_{\beta} \in \mathrm{SL}_2(\R)$. Dann gilt 
$$M_{\sigma(S)} = M_{\sigma(P_{\alpha}Q_xP_{\beta})} = M_{\sigma(P_{\alpha}) \circ \sigma(Q_x) \circ P_{\beta}} = M_{\sigma(P_{\alpha})} M_{\sigma(Q_x)} M_{\sigma(P_{\beta})} = R_{\alpha} L_x R_{\beta} = A,$$
also ist $F=\phi_S=\sigma(S)$ und es folgt Sujektivität von $\sigma$.

\item Der Isomorphiesatz sowie die vorangegangen Teilaufgaben liefern
$$\mathcal{O}^+(2,1) = \mathrm{Bild}\hspace{1.5pt}(\sigma) \cong \slant{\mathrm{SL}_2(\R)}{\mathrm{Kern}\hspace{1.5pt} (\sigma)} \cong \slant{\mathrm{SL}_2(\R)}{\{\pm I_2\}} \cong \mathrm{PSL}_2(\R),$$
was zu zeigen war. $\hfill \Box$

 
\end{compactenum}

\end{sol}

\end{prob}





















\newpage




\titlespacing*{\section}{-16.5pt}{0pt}{20pt}
\renewcommand*\thesection{}
\section{Übungsblatt 13} %PARAGRAPH A.13
\renewcommand*\thesection{\arabic{section}}




\begin{prob} %%%AUFGABE 13.1
Zeigen Sie, dass für alle $n \in \mathbb{N}$ der $n$-dimensionale hyperbolische Raum $\He^n$ ein $\delta$-hyperbolischer Raum mit $\delta= \ln(1+ \sqrt{2})$ ist.

\begin{sol}
Die Seiten eines beliebigen Dreiecks $\Delta\subseteq \He^3$ sind auf einer Halbkugelschale senkrecht zur Ebene $x_3=0$, bilden also ein geodätisches Dreieck in $\mathbb{K}^2 \cong \He^2$. Dann folgt die Behauptung per Induktion. $\hfill \Box$


\end{sol}

\end{prob}








\begin{prob} %%AUFGABE 13.2
Es sei die Abbildung $c$ gegeben durch 
$$c:[0,\infty) \la \E^2, \qquad t \mapsto \begin{pmatrix}[r] t \cos\log(1+t) \\ t \sin \log(1+t) \end{pmatrix}$$
\begin{compactenum}
\item Zeigen Sie, dass $c$ ein quasi-geodätischer Strahl ist.
\item Skizzieren Sie das Bild von $c$.
\item Zeigen Sie, dass es keinen geodätischen Strahl in $\E^2$ gibt, der endlichen Abstand von $c$ hat.

\end{compactenum}


\begin{sol}
\begin{compactenum}

\item Es gilt 
$$c'(t)=\begin{pmatrix}[r] \cos \log(1+t) - \frac{1}{1+t} \sin \log(1+t) \\ \sin \log(1+t) + \frac{t}{1+t} \cos \log(1+t) \end{pmatrix}$$
und damit
\begin{alignat*}{5}
\Vert c'(t)\Vert^2 \ \ &=&& \ \ \left( \cos \log(1+t) - \frac{t\sin \log(1+t)}{1+t} \right)^2 + \left( \sin \log(1+t) + \frac{t\cos \log(1+t)}{1+t}\right)^2 \\
&=&& \ \ \cos^2 \log(1+t) - \frac{2t}{1+t}\cos \log(1+t) \sin \log(1+t) + \frac{t^2}{(1+t)^2} \sin^2 \log(1+t) \\
&\hspace{2.8pt}+&& \ \ \sin^2 \log(1+t) + \frac{2t}{1+t} \cos \log(1+t)\sin\log(1+t) + \frac{t^2}{(1+t)^2} \cos^2\log(1+t) \\
&=&& \ \ 1+ \frac{t^2}{(1+t)^2}\\
&<&& \ \ 2,
\end{alignat*}
also $\Vert c'(t)\Vert^2 <\sqrt{2}$. Folglich ist $c$ Lipschitz-stetig mit $L=\sqrt{2}$ es gilt also 
$$d(c(t), c(t')) \leqslant \sqrt{2} \vert t-t'\vert$$
für all $t,t' \in [0,\infty)$. Weiter gilt
\begin{alignat*}{5}
\Vert c(t)-c(t')\Vert^2 \ \ &=&& \ \ \Bigg\Vert \begin{pmatrix}[r] t \cos \log(1+t) - t' \cos \log(1+t') \\ t \sin \log(1+t) - t' \sin \log(1+t') \end{pmatrix} \Bigg \Vert^2 \\
&=&& \ \ t^2 + t'^2 - 2tt'\cos \log(1+t) \cos \log(1+t') -2tt' \sin \log(1+t) \sin \log(1+t') \\
&=&& \ \ t^2+t'^2 -2tt' \left( \cos \log(1+t) \cos \log(1+t') + \sin \log(1+t) \sin \log(1+t') \right) \\
&=&& \ \ t^2+t'^2 - 2tt' \cos \log \frac{1+t}{1+t'} \\
&\geqslant&& \ \ t^2+t'^2 - 2tt' \\
&=&& \ \ \vert t-t'\vert \\
&\geqslant&& \ \ \frac{1}{\sqrt{2}} \vert t-t'\vert,
\end{alignat*}
wir erhalten insgesamt damit
$$\frac{1}{\sqrt{2}} \vert -t'\vert \leqslant d_e(c(t),c(t')) \leqslant \sqrt{2} \vert t-t'\vert$$
für all $t,t' \in [0,\infty)$, $c$ ist also ein geodätischer Strahl.

\item Wir erhalten folgendes Bild:

\usetikzlibrary{calc}

\pgfdeclarelayer{background}
\pgfsetlayers{background,main}

\newcommand\GoldenRatio{1.6180339887}
\newcommand\Side{10}
\newcommand\Sqrtwo{1.4142135624}
$$
\begin{tikzpicture}[line cap=round, scale=0.4]
\coordinate (origin) at (0,0);
\foreach \angle in {0,90,...,1000} {
  \coordinate (corner) at ($(origin)+(45+\angle:\Side*\Sqrtwo)$);
  \begin{pgfonlayer}{background}
  \end{pgfonlayer}
  \draw[thick,red] (origin) arc(\angle-90:\angle:\Side);
  \coordinate (origin) at (corner);
  \pgfmathsetmacro{\Side}{\Side/\GoldenRatio}
  \xdef\Side{\Side}
}
\draw [->] (-6,11.7) -- (12,11.7);
\draw[->] (2.76,-1) -- (2.75,18);
\end{tikzpicture}
$$


\item Sei $\gamma$ ein beliebiger geodätischer Strahl mit $\gamma(0)=0=c(t)$. Dann hat $\gamma$ die Gestalt
$$\gamma:[0,\infty) \la \E^2, \qquad t \mapsto \begin{pmatrix}[r] t \\ \alpha t\end{pmatrix}$$
oder 
$$\gamma:[0,\infty) \la \E^2, \qquad t \mapsto \begin{pmatrix}[r] 0 \\  t\end{pmatrix}$$
oder
$$\gamma:[0,\infty) \la \E^2, \qquad t \mapsto \begin{pmatrix}[r] 0 \\ - t\end{pmatrix}$$
ohne Einschränkung betrachten wir den dritten Fall. Angenommen es gäbe ein $R >0$ mit $\mathrm{Hd}(\mathrm{Bild}\ (\gamma), \mathrm{Bild}\ (c))\leqslant R$. Wähle $k \in \mathbb{N}$ sodass $\tilde{t} := e^{\left(2k + \frac{1}{2}\right) \pi} -1 >R$. Dann gilt
$$c(\tilde{t})=\begin{pmatrix} \tilde{t} \cos \log(1+\tilde{t}) \\ \tilde{t} \sin \log(1+\tilde{t})\end{pmatrix} = \begin{pmatrix} \tilde{t} \cos \frac{\pi}{2} \\ \tilde{t} \sin \frac{\pi}{2} \end{pmatrix} = \begin{pmatrix} 0 \\ \tilde{t} \end{pmatrix}$$
Dann folgt aber 
$$R\leqslant \tilde{t} = d(0,c(\tilde{t})) = \min_{t \in [0,\infty)} d(0,c(t)),$$
also $c(\tilde{t}) \notin U_{R}(\mathrm{Bild}(\gamma))$, ein Widerspruch zur Annahme. $\hfill \Box$

\end{compactenum}

\end{sol}

\end{prob}







\begin{prob}   %%AUFGABE 13.3
Es seien $(X,d_X)$, $(Y,d_Y)$ und $(Z,d_Z)$ metrische Räume. Zeigen Sie:
\begin{compactenum}
\item Wenn es eine $(\lambda, \epsilon)$-Quasi-Isometrie $f: X \la Y$ gibt, so gibt es auch eine $(\lambda', \epsilon')$-Quasi-Isometrie $g:Y \la X$ und ein $k \geqslant 0$, sodass
$$d_X((g\circ f)(x),x) \leqslant k, \qquad d_Y((f\circ g)(y),y) \leqslant k$$
für alle $x \in X, y \in Y$.
\item Es seien $g:X \la Y$ und $h: Y \la Z$ $(\lambda_g,\epsilon_g)$- bzw. $(\lambda_h,\epsilon_h)$-quasi-isometrische Einbettungen. Dann ist $h\circ g$ eine $(\lambda_g \lambda_h, \epsilon_h + \lambda_h \epsilon_g)$-quasi-isometrische Einbettung.
\item Die Verkettung zweier Quasi-Isometrien ist wieder eine Quasi-Isometrie.
\end{compactenum}

\begin{sol}
\begin{compactenum}
\item Nach Voraussetzung gilt für $x_1, x_2 \in X$
$$\frac{1}{\lambda} d_X(x_1,x_2) - \epsilon \leqslant d_Y(f(x_1),f(x_2)) \leqslant \lambda d_X(x_1,x_2) + \epsilon.$$
Außerdem gibt es $D \geqslant 0$, sodass $Y\subseteq U_D(f(X))$, für alle $y \in Y$ gibt es also $x\in X$ mit $d_Y(y,f(x))\leqslant D$; insbesondere ist $f^{-1}(U_D(y))$ also nichtleer. Definiere nun $g$ durch 
$$g: Y \la X, \qquad y \mapsto x \in f^{-1}(U_D(y)).$$
Dann gilt $f(g(y)) \in U_D(y)$ und damit
$$d_Y((f\circ g)(y),y) = d_Y(f(g(y)),y) \leqslant D$$
sowie
$$d_X((g\circ f)(x),x) = d_X(g(f(x)),x) \leqslant \lambda d_Y\left(f(g(f(x))),f(x)\right)+\lambda \epsilon.$$
Da $g(f(x)) \in f^{-1}\left(U_D(f(x))\right)$, ist $f(g(f(x))) \in U_D(f(x))$ und somit 
$$d_X((g\circ f)(x),x) \leqslant \lambda D + \lambda \epsilon = \lambda (D+\epsilon).$$ 
Da $D\leqslant \lambda(D+\epsilon)$, setze $k:= \lambda (D+\epsilon)$ und erhalte
$$d_X((g\circ f)(x),x) \leqslant k, \qquad d_Y((f\circ g)(y),y) \leqslant k$$
für alle $x \in X$ und $y \in Y$. Es bleibt nun noch zu zeigen, dass $g$ eine $(\lambda', \epsilon')$-Quasi-Isometrie ist. Sei $x \in X$ beliebig. Wähle $y:=f(x) \in Y$. Dann gilt nach obigenm
$$d_X(x,g(y))=d_X(x,g(f(x))) \leqslant \lambda(D+\epsilon)=: D',$$
wir müssen also nur noch zu zeigen, dass $g$ eine quasi-isometrische Einbettung ist. Für $y_1,y_2 in X$ gilt
\begin{alignat*}{5}
d_X(g(y_1),g(y_2)) \ \ &\leqslant&& \ \ \lambda \epsilon + \lambda d_Y\left(f(g(y_1)), f(g(y_2))\right) \\
&\leqslant && \ \ \lambda \epsilon + \lambda \left( d_Y(y_1,y_2) + 2D\right)\\
&=&&\ \ \lambda d_2(y_1,y_2) +\lambda (2D+\epsilon),
\end{alignat*}
wobei die zweite Ungleichung aus $f(g(y_1)) \in U_D(y_1)$, $f(g(y_2)) \in U_D(y_2)$ und der Dreiecksungleichung folgt. Weiter ist 
\begin{alignat*}{5}
d_X(g(y_1), g(y_2)) \ \ &\geqslant&& \ \ \frac{1}{\lambda} d_Y(f(g(y_1)), f(g(y_1)) - \frac{\epsilon}{\lambda} \\
&\geqslant&& \ \ \frac{1}{\lambda} \left( d_Y(y_1,y_1) - 2D\right) - \frac{\epsilon}{\lambda} \\
&=&& \ \ \frac{1}{\lambda} d_Y(y_1,y_2) - \frac{1}{\lambda} (2D+\epsilon) \\
&\geqslant&& \ \ \frac{1}{\lambda}d_Y(y_1,y_2) - \lambda(2D+\epsilon).
\end{alignat*}
Mit $\lambda':=\lambda$ und $\epsilon':=\lambda(2D+\epsilon)$ folgt also aus den beiden Ungleichungen, dass $g$ eine $(\lambda',\epsilon')$-quasi-isometrische Einbettung ist, insgesamt also eine $(\lambda',\epsilon')$-Quasi-Isometrie.

\item Wir haben für $x_1,x_2 \in X$ sowie $y_1,y_2\in Y$ die beiden Ungleichungsketten
$$\frac{1}{\lambda_g} d_X(x_1,x_2) - \epsilon_g \leqslant d_Y(g(x_1), g(x_2)) \leqslant \lambda_g d_X(x_1,x_2) + \epsilon_g$$
sowie 
$$\frac{1}{\lambda_h} d_Y(y_1,y_2) - \epsilon_h \leqslant d_Z(h(y_1), h(y_2)) \leqslant \lambda_h d_Y(y_1,y_2) + \epsilon_h.$$
Damit erhalten wir 
\begin{alignat*}{5}
d_Z\left((h\circ g)(x_1), (h\circ g)(x_2)\right) \ \ &\leqslant&& \ \ \lambda_h d_Y(g(x_1),g(x_2)) + \epsilon_h\\
&\leqslant&& \ \ \lambda_h\left( \lambda_g d_X(x_1,x_2) + \epsilon_g\right) + \epsilon_h\\
&=&& \ \ \lambda_g\lambda_h d_X(x_1,x_2) + \epsilon_h+\lambda_h\epsilon_g
\end{alignat*}
und
\begin{alignat*}{5}
d_Z\left((h\circ g)(x_1), (h\circ g)(x_2)\right) \ \ &\geqslant&& \ \ \frac{1}{\lambda_h} d_Y\left(g(x_1),g(x_2)\right) - \epsilon_h \\
&\geqslant&& \ \ \frac{1}{\lambda_h} \left( \frac{1}{\lambda_g} d_X(x_1,x_2) - \epsilon_g\right) - \epsilon_h \\
&=&&\ \ \frac{1}{\lambda_g\lambda_h} d_X(x_1,x_2) - \left(\frac{1}{\lambda_h} \epsilon_g + \epsilon_h\right) \\
&\geqslant&& \ \ \frac{1}{\lambda_g \lambda_h} d_X(x_1,x_2) - \left(\lambda_h \epsilon_g + \epsilon_g\right),
\end{alignat*}
was zu zeigen war.

\item Wir schließen den Beweis an jenen von (ii) an und zeigen noch die Quasi-Dichtheit von $h\circ g$. Seien also $D_g, D_h\geqslant 0$, sodass $U_{D_g}(g(X))\subseteq Y$ und $U_{D_h}(h(Y))\subseteq Z$. Wir müssen $D\geqslant 0$ finden, sodass $U_D((h\circ g)(X))\subseteq Z$, für jedes $z \in Z$ ein $x \in X$ mit $d_Z((h \circ g)(x),z)\leqslant D$. Sei also $z \in Z$ beliebig und wähle $y \in Y$ und anschließend $x \in X$ mit 
$$d_Z(h(y),z) \leqslant D_h, \qquad \qquad d_Y(g(x),y)\leqslant D_g.$$
Dann gilt 
\begin{alignat*}{5}
d_Z((h \circ g)(x),z)\ \ &\leqslant&& \ \ d_Z((h \circ g)(x), h(y)) + d_Z(h(y),z) \\
&\leqslant&& \ \ \lambda_h d_Y(g(x),y) +\epsilon_h + D_h \\
&\leqslant&& \ \ \lambda_h D_g + \epsilon_h + D_h\\
&=:&& \ \ D,
\end{alignat*}
was den Beweis abschließt. $\hfill \Box$


\end{compactenum}

\end{sol}

\end{prob}





















\end{spacing}

\end{document}


