\documentclass[12pt]{scrreprt}

\usepackage{schnaubelt}
\usepackage[utf8]{inputenc}
\usepackage{ngerman, url}
\usepackage{hyperref}
\usepackage{tikz}

\author{Die Mitarbeiter von \url{http://mitschriebwiki.nomeata.de/}}
\title{Analysis I}
\makeindex

\begin{document}
\maketitle

\tableofcontents

\setcounter{chapter}{-1}
\chapter{Vorbemerkungen}
\label{cha:vor}

\section{Bezeichnungen}
\label{sec:vor.bezeichnungen}
\subsection*{Allgemeine Bezeichnungen}
\begin{itemize}
\item griechische Buchstaben: s. Übungsblatt.
\item Thm = Theorem = Hauptsatz.
\item Def. = Definition, "`{\da}"' heißt "`steht für"'.
\item Lem. = Lemma = Hilfssatz.
\item Bew. = Beweis.
\item Beh. = Behauptung.
\item Ann. = Annahme.
\item n.V. = nach Voraussetzung.
\item Vor. = Voraussetzung.
\item Bsp. = Beispiel.
\item Bem. = Bemerkung.
\item $\Box$ = Beweisende.
\end{itemize}

\subsection*{Logische Symbole}
\begin{itemize}
\item $\neg$ = nicht.
\item $\wedge$ = und.
\item $\vee$ = oder.
\item $\ra$ = impliziert.
\item $\Longleftrightarrow$ = equivalent.
\item $\forall$ = für alle.
\item $\exists$ = es existiert.
\item $\exists!$ = es existiert genau eines.
\end{itemize}

\subsection*{Etwas zu Mengen}
Mengen werden durch die Angabe ihrer Elemente definiert, z.B. $M = \{1; 2; 3\} = \{2; 1; 3\} =$ die Menge, die aus 1, 2 und 3 besteht.
\begin{itemize}
\item $M = \set{N} =$ die Menge der natürlichen Zahlen.
\item $M = \{\nat{x}: x$ ist gerade\} = gerade Zahlen.
\item $\varnothing$ = leere Menge = $\{\}$.
\end{itemize}

\subsection*{Operationen mit Mengen $M, N$:}
\begin{itemize}
\item $x \in M$ -- "`$x$ ist ein Element von M."' (Bsp.: $1 \in \{1; 2; 3\}$)
\item $x \notin M$ -- "`$x$ ist kein Element von M."'
\item $M \subseteq N$ -- "`M ist Teilmenge (TM) von M,"' d.h. wenn $x \in M$, dann auch $x \in N$, \textit{oder}: $x \in M \folgt x \in N$.
\item $M = N$ -- "`$M \subseteq N$ und $N \subseteq N$"' \textit{oder}: $M$ und $N$ haben die gleichen Elemente.
\item $M \cap N = \{x: x \in M$ und $x \in N\}$ = Schnittmenge = Menge der $x$, die in beiden Mengen liegen.
\item $M \cup N = \{x: x \in M$ oder $x \in N\} =$ Vereinigungsmenge = Menge der $x$, die in einer der beiden Mengn liegen (oder auch in beiden).
\item $M \times N = \{(x,y): x \in M$, $y \in N\} =$ Menge der geordneten Paare aus $M$ und $N$. Ferner: $M^2 = M \times M$, $M^n = M \times \ldots \times M$ ($n$-fach) ($n \in N$).
\item $M \setminus N = \{x \in M: x \notin N\} =$ Differenzmenge = Menge der $x$ aus $M$, die nicht in $N$ liegen.
\item $\mathcal{P}(M) = \{N: N \subseteq M\} =$ Potenzmenge = die Menge aller Teilmengen von $M$.
\item $\mathcal{P}(\{1; 2; 3\}) = \{\varnothing, \{1\}, \{2\}, \{1; 2\}, \{2; 3\}, \{1; 3\}, \{1; 2; 3\}\}$. Zugehörige Rechenregeln $\ra$ LA.
\end{itemize}

\subsection*{Abbildungen (Abb.) oder Funktionen (Fkt.):}
Seien $M$ und $N$ Mengen. Eine Funktion $f: M \ra N, x \mapsto f(x)$ besteht aus dem Definitions\-bereich $M$, dem Bildbereich $N$ und der Abbildungsvorschrift $f$, die jedem "`Urbild"' $x \in M$ genau ein "`Bild"' $f(x) \in N$ zuordnet.

Streng genommen ist die Funktion das Tripel $(f, M, N)$, man schreibt meistens nur $f$. Bsp.: $f: \set{N} \ra \set{N}, x \mapsto 2x$. Hier schreibt man auch $f: \set{N} \ra \set{N}, f(x) = 2x$.

\section{Vollständige Induktion}
\label{sec:vor.vollstInd}
Wir setzen die natürlichen Zahlen $\set{N} = \{1; 2; 3; \ldots\}$, ($\set{N}_0 = \{0; 1; 2; \ldots\}$), die ganzen Zahlen $\set{Z}$ und die Brüche $\set{Q}$ samt ihren Rechenregeln vorraus.

Dann gilt das \textit{Prinzip der vollständigen Induktion} (vollst. Ind.).

$M \subseteq \set{N}$ erfülle die beiden folgenden Bedingungen:

(IA) $1 \subseteq M$

(IS) Wenn ein $\nat{n}$ zu $M$ gehört, dann gehört auch der Nachfolger $n + 1$ zu $M$.

\noindent\textit{Beh.} Dann gilt $M = \set{N}$.
\begin{proof}
(indirekt) \textit{Ann.} Die Beh. sei falsch. Dann existiert ein $\nat{m} \setminus M$. Nach (IA) ist $1 \in M$. Dann liefert (IS), dass $2 = 1 + 1 \in M$. Diesen Schritt wiederholt man $(m - 1)$ mal. Somit erhält man mit $\nat{m}$ einen Widerspruch ($\blitz$) zu $\nat{m} \setminus M$. Also muss die Annahme falsch sein, d.h. die Behauptung ist wahr.
\end{proof}

Eine \textit{Aussage} ist ein "`Satz"', der entweder wahr oder falsch ist, z.B. $7 + 5 = 12, 3 + n = n$ sind Aussagen.

$n + 1$ ist keine Aussage.

\subsection*{Beweisprinzip der vollständigen Induktion}

Es seien für jedes $\nat{n}$ Aussagen $A(n)$ gegeben. Wir wollen zeigen, dass alle Aussagen wahr sind, d.h. $M \da \{\nat{n}: A(n)$ ist wahr$\}$ muss gleich $\set{N}$ sein. Nach dem Prinzip der vollst. Ind. muss man also die folgenden Beh. zeigen:

(IA) Induktionsanfang: Man zeigt, dass $A(1)$ wahr ist.

(IS) Induktionsschluss: Es gelte die Induktionsvoraussetzung (IV): Für ein (festes, aber beliebiges) $\nat{n}$ ist $A(n)$ wahr.

Dann zeigt man, dass auch $A(n+1)$ wahr ist.

$\leadsto$ Dann folgt, dass alle $A(n)$ wahr sind.

\begin{bsp}
Zeige: $1 + 2 + 3 + \ldots + n = \frac{1}{2}n(n+1)$, $\forall \nat{n}$.
\begin{proof} (per vollst. Ind.)

Es sei $A(n): 1 + \ldots + n = \frac{1}{2}n(n+1)$, $\nat{n}$.

IA: $n = 1: 1 = \frac{1}{2}\cdot 1 \cdot 2 \folgt A(1)$ ist wahr.

IS: Es gelten $A(n)$ für ein $\nat{n}$ (IV).

\textit{Dann}: $(1+ \ldots + n) + n+1 \stackrel{(IV)}{=} \frac{1}{2}n(n+1) + (n+1)\cdot\frac{1}{2}\cdot 2 = \frac{1}{2}(n+1)(n+2)$

$\folgt A(n+1)$ ist wahr.

$\folgt$ IS ist gezeigt.

$\folgt$ nach vollst. Ind. folgt Beh.
\end{proof}

Unbefriedigend ist die Schreibweise "`$+\ldots +$"'.

Dafür: "`rekursive Def."' des Summenzeichens:

Gegeben seien $\rat{a_j}$ für jedes $\zahl{j}$ mit $j \geq m$ für ein festes $\zahl{m}$. Dann setzen wir:
\[\sum_{j=m}^m a_j \da a_m.\]

Wir nehmen an, dass $\displaystyle \sum_{j=m}^{m+n} a_j$ für ein festes, aber beliegiges $\nat{n}$ def. sei.

Dann definieren wir:
\[\sum_{j=m}^{m+n+1} a_j \da \left( \sum_{j=m}^{m+n} a_j\right) + a_{m+n+1}.\]

Nach dem Induktionsprinzip ist die Menge: $\displaystyle M = \{\nat{n}: \sum_{j=m}^{m+n} a_j$ ist def.$\}$ gleich $\set{N}$ (Korrektur: Hier braucht man das Induktionsprinzip für $\set{N}_0 = \{0; 1; 2; \ldots\} \folgt$ siehe Übung).

Wir haben also den Ausdruck $\displaystyle \sum_{j=m}^k a_j$ für alle $\zahl{k}$, $k \geq m$ def.

Man schreibt oft $\displaystyle \sum_{j=m}^k a_j = a_m + \ldots + a_k$.

Genauso def. man: $\displaystyle \prod_{j=m}^k a_j = a_m \cdot a_{m+1} \cdot \ldots \cdot a_k$.

Es gelten die üblichen Rechenregeln, wie man per Induktion zeigt. Dazu ein Beispiel, wobei $m = 1$.

Gegeben seien $\rat{a_j, b_j}$, $\nat{j}$. Dann gilt:
\[A(n): \sum_{j=1}^n a_j + \sum_{j=1}^n b_j = \sum_{j=1}^n (a_j + b_j), \forall \nat{n}.\]

\begin{proof} (per Ind.)

IA: $\displaystyle n = 1: \sum_{j=1}^1 a_j + \sum_{j=1}^1 b_j \stackrel{Def.}{=} a_1 + b_1 \stackrel{Def.}{=} \sum_{j=1}^1 (a_j + b_j) \folgt A(n)$ ist wahr.

IS: Es gelte $A(n)$ für ein $\nat{n}$ (IV). Dann:
\[\sum_{j=1}^{n+1} a_j + \sum_{j=1}^{n+1} b_j \stackrel{Def.}{=} \left(\sum_{j=1}^n a_j + a_{n+1}\right) + \left(\sum_{j=1}^n b_j + b_{n+1}\right)\]
\[\stackrel{Def.}{=} \sum_{j=1}^n (a_j + b_j) + (a_{n+1} + b_{n+1}) \stackrel{Def.}{=} \sum_{j=1}^{n+1} (a_j + b_j).\]

$\folgt A(n+1)$ ist wahr.

$\folgt$ IS gilt.

$\folgt A(m)$ gilt $\forall \nat{m}$.
\end{proof}
\end{bsp}

\begin{bsp} (geometrische Summenformel)

Gegeben sei $\rat{q}\setminus \{1\}$. \textit{Beh.} Dann gilt:
\[A(n): \sum_{j=0}^n q^j = \frac{q^{n+1}-1}{q-1}, \forall\nat{n}.\]
\begin{proof} (per Ind.)

IA: $\displaystyle (n = 1): \left.\begin{array}{l}\displaystyle\sum_{j=0}^1 q^j = q^0 + q^1 = 1+q\\\displaystyle \frac{q^2 - 1}{q-1} = \frac{(q+1)(q-1)}{q-1} = 1+q\end{array}\right\} = \folgt A(1)$ ist wahr.
\medskip

IS: Es gelte $A(n)$ für ein $\nat{n}$ (IV).

\textit{Dann}:
\[\displaystyle \sum_{j=0}^{n+1} q^j = \sum_{j=0}^n q^j + q^{n+1} \stackrel{(IV)}{=} \frac{q^{n+1}-1}{q-1} + q^{n+1} = \frac{q^{n+1}-1+q^{n+2}-q^{n+1}}{q-1} = \frac{q^{n+2}-1}{q-1}.\]

$\folgt A(n+1)$ gilt $\folgt$ (IS) ist gezeigt.

$\folgt$ vollst. Ind. zeigt, dass $A(n)$ für alle $\nat{n}$ gilt.
\end{proof}

Eine weitere rekursive Definition:

\textit{Fakultät}: $0! = 1$, $1! = 1$.

Wenn $n!$ für ein $\nat{n}$ def. ist, dann setzt man $(n+1)! = (n+1)\cdot n!$.

Man schreibt: $n! = 1 \cdot 2 \cdot \ldots \cdot n$.

\begin{dfn*} Binomialkoeffizienten.

Seien $n, j\in\set{N}_0$ und $n \geq j$.

Dann setzt man $\displaystyle \left(\begin{array}{c}n\\j\end{array}\right) \da \frac{n!}{j!(n-j)!} = \frac{1\cdot 2\cdot\ldots\cdot n}{(1\cdot 2\cdot\ldots\cdot j)(1\cdot 2\cdot\ldots\cdot (n-j))}$.

\textit{Eigenschaften}: ($n, j\in\set{N}_0$, $n \geq j$)
%sollten eigentlich Zahlen sein ....
\begin{enumerate}
\item $\displaystyle\bin{n}{n-j} = \frac{n!}{(n-j)!(n-n+j)!} = \bin{n}{j}.$
\begin{equation}
\displaystyle\bin{n}{0} = \frac{n!}{0!n!} = 1 = \bin{n}{n}.\label{binkoff1}
\end{equation}

\item Sei $j \geq 1$. Dann: $\displaystyle\bin{n}{j-1}+\bin{n}{j} = \frac{n!\cdot j}{(j-1)!(n-j+1)!\cdot j} + \frac{n!(n-j+1)}{j!\cdot(n-j)!(n-j+1)}$
\begin{equation}\stackrel{Def.\ Fak.}{=} \frac{j\cdot n! + (n-j+1)n!}{j!(n-j+1)!} \stackrel{Def.\ Fak.}{=} \frac{(n+1)!}{j!(n+1-j)!} \stackrel{Def.}{=} \bin{n+1}{j}\label{binkoff2}\end{equation}
\end{enumerate}
\end{dfn*}
\end{bsp}

\begin{bsp} (Binomischer Satz)

Seien $\rat{a, b}$, $\nat{n}$. Dann $\displaystyle A(n): (a+b)^n = \sum_{j=0}^n\bin{n}{j}a^{n-j}b^j,$ $\nat{n}$.
\begin{proof}(per vollst. Ind.)

IA: $(n=1)$
\[\sum_{j=0}^1\bin{1}{j}a^{1-j}b^j \stackrel{\ref{binkoff1}}{=} 1\cdot a^1\cdot b^0 + 1\cdot a^0\cdot b^1 = (a+b)^1.\]
$\folgt A(1)$ ist wahr.

IS: $A(n)$ gelte für ein $\nat{n}$ (IV).
\[(a+b)^{n+1} = (a+b)(a+b)^n \stackrel{(IV)}{=} (a+b)\sum_{j=0}^n\bin{n}{j}a^{n-j}b^j\]
\[= \sum_{j=0}^n\bin{n}{j}a^{n-j+1}\cdot b^j + \sum_{j=0}^n\bin{n}{j}a^{n-j}\cdot b^{j+1}\]

\textit{setze} $l = j+1\ (\gdw j = l-1)$

\[=\sum_{j=0}^n\bin{n}{j}a^{n+1-j}\cdot b^j + \sum_{l=1}^{n+1}\bin{n}{l-1}a^{n+1-l}\cdot b^l\]
%\[\stackrel{\ref{binkoff1}}{=} \begin{array}{c r c c c c}a^{n+1}&\sum_{j=1}^n&\left(\bin{n}{j}+\bin{n}{j-1}\right)&a^{n+1-j}\cdot b^j&+&1\cdot a^0\cdot b^{n+1}\\(j=0)&&\stackrel{\ref{binkoff2}}{=}\bin{n+1}{j}&(j=l \mathrm{gesetzt})&&(j=n+1)\end{array}\]
\[\stackrel{\ref{binkoff1}}{=} \underbrace{a^{n+1}}_{(j=0)}\sum_{j=1}^n\underbrace{\left(\bin{n}{j}+\bin{n}{j-1}\right)}_{\stackrel{\ref{binkoff2}}{=}\scriptsize\bin{n+1}{j}}\underbrace{a^{n+1-j}\cdot b^j}_{(j=l\ \mathrm{gesetzt})}+\underbrace{1\cdot a^0\cdot b^{n+1}}_{(j=n+1)}\]
\[= \sum_{j=0}^{n+1}\bin{n+1}{j}a^{n+1-j}\cdot b^j.\]

$\folgt A(n+1)$ ist gezeigt $\folgt$ (IS) gilt.

$\folgt$ Beh. folgt mit vollst. Ind.
\end{proof}
\end{bsp}

\chapter{Reelle und komplexe Zahlen}
\label{cha:zahlen}
Wir definieren die reellen Zahlen "`axiomatisch"', d.h.: Man legt in einer Definition die 
Eigenschaften der reellen Zahlen fest, die im folgenden verwendet werden dürfen.
Ausblick: \set{R} ist ein "`ordnungsvollständiger, geordneter Körper"'.

\section{Geordnete Körper}
\label{sec:zahlen.geordnete-koerper}
\begin{dfn*}
Sei $M$ eine nichtleere Menge. Eine Abbildung $\ast: M\times M \ra M(x, y) 
\mapsto x \ast y$ heißt Verknüpfung auf $M$.
\end{dfn*}

\begin{dfn}
\label{dfn:zahlen.koerper}
Seien $K$ eine Menge, $0 \in K$, $1 \in K$ mit $0 \ne 1$, und "`$+$"' und "`$\cdot$"' Verknüpfungen auf 
$K$. Dann heißt $(K, 0, 1, +, \cdot)$ ein \emph{Körper}, wenn die folgenden Eigenschaften für alle 
$x$, $y$, $z \in K$ gelten:

\begin{enumerate}
\item Assoziativgesetze:
\[(x + y) + z = x + (y + z)\]
\[(x \cdot y) \cdot z = x \cdot (y \cdot z)\]

\item neutrale Elemente:
\[x + 0 = x, x \cdot 1 = x\]

\item inverse Elemente:
\begin{itemize}
  \item Zu jedem $x \in K$ existiert ein $a \in K$ mit $x + a = 0$.
  \item Zu jedem $x \in K \backslash \{0\}$ existiert ein $b \in K$ mit $x \cdot b = 1$.
\end{itemize}

\item Kommutativgesetze:
\[x + y = y + x,\ x \cdot y = y \cdot x\]

\item Distributivgesetz:
\[(x + y) \cdot z = (x \cdot z) + (y \cdot z)\]
\end{enumerate}

Man schreibt oft $K$ anstelle $(K, 0, 1, +, \cdot)$.
\end{dfn}

\begin{bsp*}
\begin{enumerate}
\item \set{Q} mit den üblichen 0, 1, $+$, $\cdot$ ist ein Körper.
\item \set{Z} ist kein Körper, da es kein $b \in$ \set{Z} gibt mit $2b = 1$.
\item Weitere Beispiele in linearer Algebra und Analysis I.
\end{enumerate}
\end{bsp*}

...
% ...
\subsection*{Eigenschaften von \set{R} und sup, inf}
\begin{satz}\label{satz:zahlen.sup-schranke}
Sei $M \subset \set{R}$ nichtleer und nach oben (unten) beschränkt
und $S \in \set{R}$. Dann sind äquivalent:
\begin{enumerate}
\item \label{satz:zahlen.sup-schranke.a} 
$S = \sup M$ ($S = \inf M$)
\item \label{satz:zahlen.sup-schranke.b}
$S$ ist eine obere (untere) Schranke und $\forall \varepsilon > 0\, \exists x_\varepsilon \in M: S - \varepsilon < x_\varepsilon \le S $ $( S \le x_\varepsilon < S+\varepsilon)$
\end{enumerate}
\end{satz}
\begin{proof}
(Nur für $\sup$):
Sei $B$ die Menge der oberen Schranken von $M$.
\ref{satz:zahlen.sup-schranke.a} $\gdw S= \min B \gdw S$ ist obere Schranke von $M$ und $\forall \varepsilon > 0 : s - \varepsilon \notin B$
(da $S$ kleinste obere Schranke) \gdw $S$ ist obere Schranke von $M$ und 
$\forall \varepsilon > 0\,\exists x_\varepsilon \in M: S-\varepsilon < x_\varepsilon \gdw \ref{satz:zahlen.sup-schranke.b}$
\end{proof}

\begin{satz}\label{satz:zahlen.min-in-nat}
Sei $M \subseteq \set{N}$ nichtleer. Dann exisitiert $\min M$.
\end{satz}
\begin{proof}
Da $\set{N} \subset \set{R}$ und 1 eine untere Schranke von $\set{N}$ ist,
exisitert $x = \inf M$. Nach Satz~\ref{satz:zahlen.sup-schranke} mit $\varepsilon = \frac{1}{3}$ existiert
ein $m_0 \in M$ mit $ x \le m_0 < x + \frac{1}{3} \le m + \frac{1}{3}$ für alle $ m \in M$.
Für $m \in \set{N}$ mit $m \ne m_0$ gilt $\abs{m - m_0} \ge 1$. Also gilt $m_0 \le m$ für alle $ m\in M \folgt m_0 = \min M$.
\end{proof}

\begin{satz}\label{satz:zahlen.arch-ord}
\begin{enumerate}
\item \label{satz:zahlen.arch-ord.a}
$\set{R}$ ist "`archimedisch geordnet"', d.h. $\forall x \in \set{R}\,\, \exists n_x \in \set{N} : n_x > x$
\item \label{satz:zahlen.arch-ord.b}
$\forall \varepsilon \in \set{R}$ mit $\epsilon > 0\,\exists n_\varepsilon \in \set{N}$, sodass $\frac{1}{n_\varepsilon} < \varepsilon$.
\item \label{satz:zahlen.arch-ord.c}
Sei $x \in \set{R}$. Wenn $0 \le x \le \frac{1}{n} $ für alle $ n \in \set{N}$, dann $x = 0$.
\end{enumerate}
\end{satz}
\begin{proof}
\begin{enumerate}
\item Annahme: Die Behauptung sei falsch, d.h. $\exists x_0 \in \set{R}\,\forall n \in \set{N} : n \le x_0$.
Somit exisitert $s = \sup \set{N} \in \set{R}$. Nach Satz~\ref{satz:zahlen.sup-schranke} mit $\varepsilon = \frac{1}{2}$
exisitert dann $m\in \set{N}$ mit $s-\frac{1}{2} < m \folgtwegen{+1} s < s+\frac{1}{2} < m +1$. Da $m+1 \in \set{N}$, kann $s$ kein Supremum
sein. $\blitz \folgt$ \ref{satz:zahlen.arch-ord.a} gilt.
\item Sei $\varepsilon > 0$ gegeben. Setze $x = \frac{1}{\varepsilon} \in \set{R}$. Nach \ref{satz:zahlen.arch-ord.a} existiert $n_x \in \set{N}$
mit $n_x > x = \frac{1}{\varepsilon} \folgt \varepsilon > \frac{1}{n_x} \folgt$ Beh.~\ref{satz:zahlen.arch-ord.b} mit $n_\varepsilon = n_x$
\item folgt direkt aus \ref{satz:zahlen.arch-ord.b}
\end{enumerate}
\end{proof}

\begin{dfn*}
Seien $M$, $N$ nichtleere Mengen. Eine Abbildung $f: M \ra N$ heißt \emph{injektiv}, wenn $\forall x,y \in M$ mit $x \ne y : f(x) \ne f(y)$.
Sie heißt \emph{surjektiv}, wenn $\forall z \in n\,\exists x\in $ mit $f(x)=z$.
$f$ heißt \emph{bijektiv}, wenn $f$ injektiv und surjektiv ist, d.h. $\forall z \in N\, \exists !x\in M$ mit $f(x)=z$.
Für bijektive $f: M \ra N$ definiert man die Umkehrabbildung $f^{-1} : N \ra M$ durch $f^{-1}(z) = x$, wenn $f(x) = z$, $z\in N$.
\end{dfn*}
\begin{dfn}\label{dfn:zahlen.maechtigigkeit}
Zwei Mengen $M$, $N$ heißen gleichmächtig, wenn es ein bijektive Abbildung $f: M \ra N$ gibt. $M$ hat die Mächtigkeit
(Kardinalität) \nat{n}, wenn M und $\{1,2, \dotsc , n\}$ gleichmächtig sind. Wenn dies für kein $n \in \set{N}$ der Fall ist,
so ist $M$ unendlich. Man schreibt dann $\#M = n$ bzw. $\#M = \infty$.
\end{dfn}

\begin{bsp*}
Sei $M = \{A, B, C\}$. Dann ist $f:M \ra \{1,2,3\}$ mit 
$f(A) = 1$, $f(B)=3$, $f(C)=2$ eine bijektive Abbildung $\folgt \#M = 3$.
\end{bsp*}
\paragraph{Beachte:} Wenn $\#M=n$, dann gilt $M = {x_1, \dotsc , x_j}$, wobei $x_j := f^{-1}(j)$ mit $f$
aus Def. \ref{dfn:zahlen.maechtigigkeit} und $j \in \{1, \dotsc, n\}$. Wenn $M$ und $N$ gleichmächtig sind,
dann $\#M=\#N$, da die Verkettung bijektiver Abbildungen bijektiv ist.
\begin{bem*}Gleichmächtigkeit ist eine Äquivalenzrelation.\end{bem*}

\begin{satz}\label{satz:zahlen.maechtigkeit-NQ}
\begin{enumerate}
\item \label{satz:zahlen.maechtigkeit-NQ.a}
Sei $m \in \set{N}$. Dann ist $\#\{j\in\set{N}:j\ge m\} = \infty$. Speziell $\#\set{N} = \infty$
\item \label{satz:zahlen.maechtigkeit-NQ.b}
Seien $a, b \in \set{R}$ mit $b > a$. Dann $\#\{x \in \set{Q}:a < x<b\} = \infty$
\end{enumerate}
\end{satz}
\begin{proof}
\begin{enumerate}
\item Annahme: $\#\{j\in\set{N}:j\ge m\} = n$. Dann $\exists x_1,\dotsc,x_n\in \set{N}$
mit $M:=\{j\in\set{N}:j\ge m\}={x_1, \dotsc, x_n}$. Dann
$y=x_1 + \dotsb  + x_n + 1 \in \set{N}$ und 
\[ y =\begin{cases} m &\folgt y \in M \\ x_j, j \in \{1, \dotsc, n\} & \folgt y \notin M \end{cases}\]
\item Zuerst konstruiert man ein $q \in \set{Q} \cap (a,b)$. Nach Satz~\ref{satz:zahlen.arch-ord} 
$\exists n \in \set{N} : b-a > \frac{1}{n} > 0$, also 
\begin{equation}
nb > 1+na \label{satz:zahlen.maechtigkeit-NQ.eqn}\tag{$*$}
\end{equation}
Sei $a \ge 0$. Dann existiert nach Satz~\ref{satz:zahlen.arch-ord} und Satz~\ref{satz:zahlen.min-in-nat} ein minimales
$k\in\set{N}$ mit $k>na$. Sei $a<0$. Dann erha"lt man genauso ein minimales $l \in \set{N}$ mit $l \ge -na$,
also $-l\le an$. Somit liegt 
\[m:= 
\begin{cases}
	k &,a \ge0 \\
	1-l &, a<0
\end{cases}\]
in $\set{Z}$ und $na < m \le an+1 \nach{<}{(\ref{satz:zahlen.maechtigkeit-NQ.eqn})} nb \folgt
a< \frac{m}{n} < b$, $q:= \frac{m}{n} \in \set{Q}$.
Nach Satz~\ref{satz:zahlen.arch-ord} $\exists j_0 \in \set{N}$ mit $b-q > \frac{1}{j_0} > 0$.
Sei $j \in J := \{k\in\set{N}:k\ge j_0\} \folgt q + \frac{1}{j} \in \set{Q}$ und $a < q + \frac{1}{j}\le q + \frac{1}{j_0} < b$, 
$\forall j \in J$. Die Menge $M = \{q + \frac{1}{j}, j\in J\}$ ist nach \ref{satz:zahlen.maechtigkeit-NQ.a} unendlich da 
$f: J\ra M, f(j) = b + \frac{1}{j}$ bijektiv ist.
\end{enumerate}
\end{proof}

\begin{dfn*}
Seien $A,B \subseteq R$. Dann setzt man 
\begin{align*}
A + B &:= \{x:\exists a \in A, b\in B\text{ mit }x=a+b\}\\
A \cdot B &:= \{x:\exists a \in A, b\in B\text{ mit }x=a\cdot b\}\\
\text{speziell: }y+B&=\{y\}+B = \{x=y+b, b\in B\}\\
y\cdot B&=\{y\}\cdot B = \{x=y\cdot b, b\in B\}
\end{align*}
\end{dfn*}

\begin{bsp*}
$[0;1]+[2;3]=[2;4]$
\end{bsp*}
\begin{proof}
"`$\subset$"' ist klar.
"`$\supset$"' Sei $x \in [2;3]$.\\
Wenn $x \in [2;3]$, dann wähle $a=x-2\in[0;1]$ und $b=2$\\
Wenn $x \in [3;4]$, dann wähle $a=x-3\in[0;1]$ und $b=3$\\
In beiden Fällen: $a+b=x$
\end{proof}

\begin{satz}\label{satz:zahlen.sup-intervalle}
Seien $A$, $B \subseteq \set{R}$ nichtleer.
\begin{enumerate}
\item Seien $A$ und $B$ nach oben beschränkt. Dann:
	\begin{enumerate}
	\item \label{satz:zahlen.sup-intervalle.a}
		Wenn $A \subseteq B$, dann $\sup A \le \sup B$
	\item \label{satz:zahlen.sup-intervalle.b}
		$\sup (A+B) = \sup A + \sup B$
	\item \label{satz:zahlen.sup-intervalle.c}
		Wenn $A$, $B \subseteq (0, \infty)$, dann $\sup(A\cdot B) = \sup A \cdot \sup B$
	\end{enumerate}
\item Seien $A$ und $B$ nach unten beschränkt. Dann gelten
	\ref{satz:zahlen.sup-intervalle.b} und \ref{satz:zahlen.sup-intervalle.a} von 1) auch für das Infimum. Weiter gelten:
	\begin{enumerate} % FIXME Labels a' und d
	\item[a\textsuperscript{$\prime$})] \label{satz:zahlen.sup-intervalle.a2}
	$A \subseteq B \folgt \inf A \supseteq \inf B$
	\setcounter{enumii}{3}
	\item \label{satz:zahlen.sup-intervalle.d}
	$-A$ ist nach oben beschränkt und $\inf A = -\sup(-A)$, wobei $-A:=(-1)\cdot A$.
	\end{enumerate}
\end{enumerate}
\end{satz}
\begin{proof}
\begin{enumerate}
\item Sei $A\subseteq B$. Wenn $z$ eine obere Schranke von $B$ ist, dann auch von $A$. $\folgt$ Beh. \ref{satz:zahlen.sup-intervalle.a}
\item Seien $x=\sup A$ und $y = \sup B$. Dann $x+y\ge a+b\,\forall a \in A$, $b\in B \folgt x+y $ ist obere Schranke von $A+B$. Sei $\varepsilon >0$
gegeben (fest aber beliebig). Setze  $\eta = \frac{\varepsilon}{2} > 0$. Satz \ref{satz:zahlen.sup-schranke} liefert $a_\eta \in A$ und 
$b_\eta \in B$ mit $x-\eta < a_\eta \le x$ bzw. $y-\eta < b_\eta \le y
\folgt x+y-\underbrace{2\eta}_\varepsilon < \underbrace{a_\eta + b_\eta}_{\in A+B} \le x+y \folgtnach{\ref{satz:zahlen.sup-schranke}}$ Beh. $\ref{satz:zahlen.sup-intervalle.b}$
(Rest in Übungen)
\end{enumerate}
\end{proof}

\subsection*{Potenzen mit rationalen Exponenten}
Seien \reell{a, b} mit $a, b > 0$, $r=\frac{m}{n}$, $n\in\set{N}$, $m\in\set{Z}$ gegeben.\\
\emph{Ziel:} Definiere $a^\frac{m}{n}$ und zeige Potenzgesetze. Vorrausgesetzt wird dabei der Fall
<\[a^m =
\begin{cases}
\underbrace{a \cdot a \cdot \dotsm \cdot a}_{m\text{ mal}} &\text{für }m>0\\
1 &\text{für }m=0\\
\frac{1}{a^{\abs{m}}} &\text{für }m<0
\end{cases}\]
Wie verwenden (wobei $a$, $b > 0$)
\begin{equation} 
a<b \gdw a^n <  b^n 
\label{eqn:zahlen.ordn-potenzen}
\end{equation}
\begin{proof}
"`\folgt"' $a<b \folgt a^2 < ab$ und $ab < b^2$ induktiv für alle \nat{n}.\\
"`$\Longleftarrow$"' Sei $a^n < b^n$. Annahme: $a \ge b \folgtwegen{\text{wie oben}} a^n \ge b^n \folgt \blitz$\\
Hauptschritt: Fall $m=1$. Sei $M = \{ x \in \set{R}_+ : x^n \le a \}$. Dann
\begin{enumerate}
\item $M \ne \emptyset$, da $0 \in M$
\item $M$ hat obere Schranke $1+a$, denn Annahme: $1+a$ hat keine 
obere Schranke: $x>1+a$ für $x \in M \folgtnach{(\ref{eqn:zahlen.ordn-potenzen})} 
x^n \ge (1+a)^n\ge (1+a)\cdot 1^{n-1} > a \blitz$
\end{enumerate}
\end{proof}

\begin{equation}
\text{Def.~\ref{dfn:zahlen.reelle-zahlen}} \folgt \exists w = \sup M
\label{eqn:zahlen.exist-wurzel}
\end{equation}

\begin{lem}\label{lem:zahlen.wurzel}
$w$ ist die einzige positive reelele Lösung der Gleichung $y^n = a$
\end{lem}
\begin{proof}
\begin{enumerate}
\item Annahme: $w^n<a$. Sei $\varepsilon \in (0;1]$. Dann $(w+\varepsilon)^n \folgtnach{\ref{vor.binom}} \sum^n_{j=0}\binom{n}{j}w^j\varepsilon^{n-j}
= w^n+ \varepsilon \sum^{n-1}_{j=0}\binom{n}{j}\underbrace{w^j}_{\ge 0}\underbrace{\varepsilon^{n-j-1}}_{\le1}
\le w^n + \varepsilon \sum^n_{j=0}\binom{n}{j}w^i \nach{=}{\ref{vor.binom}} w^n + \varepsilon (1+w)^n$.

Wähle speziell $\varepsilon= \min \left\{1,\frac{a-w^n}{(1+w)^n}\right\} \in (0;1] \folgt (w+\varepsilon)^n \le w^n +\frac{a-w^n}{(1+w)^n} (1+w)^n=a
\folgt w+ \varepsilon \in M \folgt \blitz$ zu $w=\sup  M \folgt w^n \ge a$
\item Ähnlich sieht man $w^n \le a \folgt w^n = a$
\item Es gelte $v^n=a$ für ein $v \in \set{R}_+$. Wenn $v<(>)\;w$, dann $v^n <(>)\;w^n$ nach (\ref{eqn:zahlen.ordn-potenzen}) $\folgt \blitz$
zu $v^n = a = w^n$ \folgt $v=w$
\end{enumerate}
\end{proof}

\paragraph{Folgerung.}
Sei $x \in \set{R}$. Dann ist $y = \sqrt{x^2}$ die einzige positive Lösung von $y^2 = x^2$. Weitere Lösung ist $\abs{x} 
\folgtwegen{Eind.}$
\begin{equation}
\sqrt{x^2} = \abs{x}
\label{eqn:zahlen.betrag-wurzel}
\end{equation}
\begin{dfn}\label{dfn:zahlen.wurzel}
Sei $a \in \set{R}$, $a > 0$, \nat{n}, $m \in \set{Z}$, $q=\frac{m}{n}$, $w$ wie in (\ref{eqn:zahlen.exist-wurzel}).
Dann setzen wir $\sqrt[n]{a} := a^{\frac{1}{n}} := w$ und $a^q:=(a^{\frac{1}{n}})^m$
\end{dfn}

\begin{satz}\label{satz:zahlen.potenzen}
Seien \reell{a,b}, $a$, $b > 0$, $p$, $q \in {Q}$. Dann gelten:
\begin{enumerate}
\item \label{satz:zahlen.potenzen.a}
$a^p b^p = (ab)^p$
\item \label{satz:zahlen.potenzen.b}
$a^p a^q = a^{p+q}$
\item \label{satz:zahlen.potenzen.c}
$(a^p)^q = a^{pq}$
\item \label{satz:zahlen.potenzen.d}\vspace{-0.7\baselineskip}
$a>b>0 \folgt \begin{cases}
a^p>b^p, &p>0\\
a<b^p, &p<0
\end{cases}$
\end{enumerate}
\end{satz}
\begin{proof}
\begin{enumerate}
\item Seien $a$, $b >0$, $p = \frac{m}{n}$, $m \in \set{Z}$, \nat{n}.
Zu zeigen: $a^p b^p = (ab)^p$ Dann: $(a^{\frac{1}{n}}b^{\frac{1}{n}})^n =
\underbrace{a^{\frac{1}{n}}b^{\frac{1}{n}}\dotsm}_{n\text{-mal}} =
(a^{\frac{1}{n}})^n (b^{\frac{1}{n}})^n \nach{=}{\ref{lem:zahlen.wurzel}} ab \folgtnach{\ref{lem:zahlen.wurzel}} $
$a^{\frac{1}{n}}b^{\frac{1}{n}} = (ab)^{\frac{1}{n}}$. $n$-te Potenz liefert Beh. \ref{satz:zahlen.potenzen.a}.
\item ...
\item ...
\item Sei $p = \frac{m}{n} \in \set{Q}$m $a>b>0$. Zu zeigen: 
$\begin{cases}
p>0 &\folgt a^p > b^p\\
p<0 &\folgt a^p < b^p
\end{cases}$\\
Annahme: $a^\frac{1}{n} \le b^\frac{1}{n}$, \nat{n}
$a \nach{=}{Def.} (a^\frac{1}{n})^n \nach{\le}{\ref{eqn:zahlen.ordn-potenzen}} (b^\frac{1}{n})^n \nach{=}{Def.} b 
\blitz \folgt a^\frac{1}{n} > b^\frac{1}{n}$ n-te Potenz, \ref{eqn:zahlen.ordn-potenzen}, Übung 2.5, \ref{satz:zahlen.potenzen.a} 
für $m<0$ liefern \ref{satz:zahlen.potenzen.d}

\end{enumerate}
\end{proof}

\section{Komplexe Zahlen}
\label{sec:zahlen.komplex}
Ausgangspunkt: Löse $x^2 = -1$ Nach Satz \ref{satz:zahlen.oreg} hat diese Gleichung keine Lösung in einem geordneten Körper, insbesondere
keine Lösung in $\set{R}$. Idee: Konstruiere einen nicht geordneten Körper, der $\set{R}$ enthält und in dem $x^2 = -1$ lösbar ist. 
\paragraph{Ansatz.} Auf $\set{R}^2$ gibt es (Vektor-)addition: $\begin{pmatrix} x\\y \end{pmatrix} + \begin{pmatrix} u\\v \end{pmatrix} =
\begin{pmatrix} x+u \\ y+v \end{pmatrix}$\\
Def.: $\begin{pmatrix} x\\ y \end{pmatrix} \cdot \begin{pmatrix} u \\ c \end{pmatrix} := \begin{pmatrix} xu-yv \\ xv+xy \end{pmatrix} \in \set{R}^2$,\\
Bsp.: $\begin{pmatrix} 0 \\ 1 \end{pmatrix}\cdot\begin{pmatrix} 0 \\ 1 \end{pmatrix}=\begin{pmatrix} -1 \\ 0 \end{pmatrix}$\\
Neue Bezeichnungen: $1$ statt $\begin{pmatrix} 1 \\ 0 \end{pmatrix}$, $i$ statt $\begin{pmatrix} 0 \\ 1 \end{pmatrix}$, 
$x+iy = z$ statt $ \begin{pmatrix} x \\ y \end{pmatrix}$ mit \reell{x,y} (also $i^2=-1$) \\

\begin{tikzpicture}
	\draw[->] (-1.5,0) -- (1.9, 0) node[anchor=north] {$\set{R}$};
	\draw[->] (0,-1.5) -- (0, 2.9) node[anchor=east] {$i\set{R}$};
	\draw (0 , 0) -- (1.4, 2.4) node[anchor=west] {$z = x+ iy$};
	
	\draw (-.15, -1) node[anchor=east] {-1} -- (.15,-1);
	\draw (-.15, 1) node[anchor=east] {1} -- (.15,1);
	\draw (-.15, 2) node[anchor=east] {2} -- (.15,2);	

	\draw (-1, -.15) node[anchor=north] {-1} -- (-1, .15);
	\draw (1, -.15) node[anchor=north] {1} -- (1, .15);
	
	\draw[dotted] (0, 2.4) node[anchor=east] {$y$} -- (1.4,2.4) -- (1.4, 0) node[anchor=north] {$x$};
	
	\path node at (-1.6, 2.6) {$\set{C}$};
\end{tikzpicture}

\begin{dfn*}
$\set{C} := \{z=x+iy : x,y \in \set{R}\}$
Fasse $\set{R} = \{z=x+i\cdot 0=x, x\in\set{R}\}$ als Teilmenge von $\set{C}$ auf.
\end{dfn*}

Seien $z=x+iy$, $w=u+iv$ für $x,y,u,v \in \set{R}$.
Dann setzt man
$$ z+w = (x+iy) + (u+iv) := (x+u) + i(y+v) \in \set{C}$$
$$ z\cdot w := (xu-yr) + i(yu+xv) \in \set{C}$$
\paragraph{Beachte.} Auf der rechten Seite der obigen Definition stehen in den Klammern nur reelle Ausdrücke, die somit wohldefiniert sind.
Falls $z=x \in \set{R}$ und $w=u \in \set{R}$, so erhält man wieder die reelen $+, -$
Lineare Algebra: $(\set{C},0,1,+,\cdot)$ ist ein Körper.

\begin{dfn}\label{dfn:zahlen.komplex}
Sei $z=x+iy \in \set{C} $ mit $x,y \in \set{R}$. Dann heißt $x$ der Realteil von $z$, $y$ der Imaginärteil von $z$, 
$\abs{z}_\set{C} := \sqrt{x^2+y^2}$ der Betrag von $z$ und $\bar{z}:=x-iy$ das konjungiert Komplexe von $z$.
Man schreibt $x = \Re{z}$ und $ y = \Im{z}$.

\begin{bem*}
Für $z = x \in \set{R}$ gilt $\abs{x}_\set{C} = \sqrt{x^2} \wegen{=}{\ref{satz:zahlen.bernoulli}} \abs{x}_\set{R}$.
Somit schreiben wir $\abs{z}$ statt $\abs{z}_\set{C}$ für $z \in \set{C}$.
\end{bem*}
Sei $z \in \set{C}$, $r \in \set{R}, r>0$. Dann ist $B(z,r) = \{w \in \set{C}:\abs{z-w}<r\}$
die offene Kreisscheibe in $\set{R}^2$ mit Mittelpunkt $z = \begin{pmatrix} x \\ y \end{pmatrix}$ und Radius
$r$, $\bar{B}(z,r)   = \{w \in \set{C}:\abs{z-w}\le r\}$ die abgeschlossene Kreisscheibe,
$s(z,r)   = \{w \in \set{C}:\abs{z-w}= r\}$ die Kreislinie.\\
Ferner: Sei $z = x \in \set{R}$. Dann $ B(x,r) \cap \set{R} = \{x-r, x+r\}$.
\end{dfn}

\begin{satz}\label{satz:zahlen.kreg}
Für $w$, $z \in \set{C}$ gelten:
\begin{enumerate}
\item \label{satz:zahlen.kreg.a}
$\bar{\bar{z}} = z$, $\abs{z}^2 = z \cdot \bar{z}$ ($\folgt \frac{1}{z} = \frac{\bar{z}}{\abs{z}^2}, z \ne 0$)
\item \label{satz:zahlen.kreg.b}
$\overline{z+w} = \bar{z} + \bar{w}$, $\overline{zw} = \bar{z} \cdot \bar{w}$
\item \label{satz:zahlen.kreg.c}
$\Re{z} = \frac{1}{2}(z + \bar{z}), \Im{z} = \frac{1}{2}(z-\bar{z})$
\item \label{satz:zahlen.kreg.d}
$\abs{\Re{z}} \le \abs{z}, \abs{\Im{z}} \le \abs{z}, \abs{\bar{z}} = \abs{z}$
\item \label{satz:zahlen.kreg.e}
$\abs{z} \ge 0$, $z = 0 \gdw \abs{z} = 0$ 
\item \label{satz:zahlen.kreg.f}
$\abs{zw} = \abs{z} \cdot \abs{w}$
\item \label{satz:zahlen.kreg.g}
$\abs{z+w} \le \abs{z} + \abs{w}$ (Dreiecksungleichung)
\item \label{satz:zahlen.kreg.h}
$\abs{z-w}\ge \abs{\abs{z} - \abs{w}}$
\end{enumerate}
\end{satz}
\begin{proof}
Seien $z = x+iy$, $w = u+iv$ für $x,y,u,v \ in \set{R}$.
\begin{enumerate}
\item[a1)]
$\bar{\bar{z}} = \overline{x+i(-y)} = x - i(-y) = z$
\item[a2)]
$z\bar{z} = (x+iy)(x-iy) = x^2 - ixy + ixy - i^2y^2 = x^2 +y^2 = \abs{z}^2$
\item[b1)]
ist klar
\item[b2)]
$\overline{zw} = \overline{xu - yv + i(xv+yu)} = xu-yv-i(xv-yu)$
$= xu - yv - ixv -iyu = (x-iy)(u-iv) = \bar{z}\bar{w}$
\item[c1)]
$z + \bar{z} = x+ iy + x - iy = 2x \gdw \frac{1}{2}(z + \bar{z} = x)$
\item[c2)]
genauso
\item[d1)]
$\abs{\Re{z}} = \abs{x} \wegen{=}{\ref{dfn:zahlen.gk}} \sqrt{x^2} \wegen{\le}{\ref{satz:zahlen.potenzen}} \sqrt{x^2 +y^2} = \abs{z}$
\item[d2)]
genauso
\item[d3)]
$\abs{\bar{z}} = \sqrt{x^2 + {-y}^2} = \abs{z}$
\item[e1)]
klar
\item[e2)]
$\abs{z} = \sqrt{x^2+y^2} = 0 \gdw x^2 +y^2 = 0 \gdw x=0$, $y=0$
\item[f)]
$\abs{zw}^2 = zw \cdot \overline{zw} = z\bar{z}w\bar{w} \cdot \abs{z}^2 \abs{w}^2$
\item[g)]
$\abs{z+w}^2 = (z+w)(\bar{z}+\bar{w}) = z\bar{z} + z\bar{w} + w\bar{z} + w\bar{w} = 
\abs{z}^2 + \underbrace{z\bar{w} + \overline{w\bar{w}}}_{=2\Re{(z\bar{w})}} + \abs{w}^2$
$\le \abs{z}^2 + 2\underbrace{\abs{z\bar{w}}}_{\abs{z}\cdot\abs{w}} + \abs{w}^2 = (\abs{z} +\abs{w})^2 \folgtwegen{\sqrt{\,\,}}$ Beh.
\end{enumerate}
\end{proof}

\chapter{Konvergenz von Folgen}
\label{cha:konv}

\section{Einfache Eigenschaften}
\label{sec:konv.eigenschaften}
\begin{dfn}
  \label{dev:konv.folge}
  Eine Abbildung $A: \set{N} \ra \set{C}$ heißt \emph{Folge}. Man
  schreibt $a_n$ statt $A(n)$ für \nat{n} und $(a_n)_{n\ge1}$ oder
  $(a_n)$ statt $A$. Wenn $\reell{a_n}\,\forall \nat{n}$, so heißt
  $(a_n)$ \emph{reelle Folge}.
\end{dfn}

\begin{dfn}
  \label{dfn:konv.konv}
  Seien $(a_n)$ eine Folge und \kmplx{a}. $(a_n)$ \emph{konvergiert}
  gegen $a$, wenn es zu jeden $\ep>0$ ein $\nat{N_\ep}$ gibt, sodass
  $\abs{a_n-a}\le\ep$ für alle $n\ge N_\ep$, wobei $\nat{n}$, also
  \[\forall\ep>0\,\exists \nat{N_\ep}\,\forall n\ge
  N_\ep:\abs{a_n-a}\le\ep.\] $a$ heißt dann \emph{Grenzwert} (oder
  \emph{Limes}) von $(a_n)$ und man schreibt "`$a = \lim_{\ninf}a_n$"'
  oder "`$a_n \to a$ für $n \to \infty$"'. Wenn $(a_n)$ keinen
  Grenzwert hat, so heißt $(a_n)$ \emph{divergent} (div.).
\end{dfn}

\begin{bem*}
  $\abs{a_n-a}\le\ep \gdw a_n \in \abg{B}(a,\ep) \gdw $ Abstand von
  $a_n$ und $a$ ist kleiner als $\ep$
\end{bem*}

\begin{bem*}
  Wenn $a_n \to 0$ für $\ninf$, dann heißt $(a_n)$ Nullfolge
  (NF). Somit $a_n \to a$ für $\ninf \gdw
  \left(\abs{a_n-a}\right)_{n\ge1}$ ist Nullfolge.
\end{bem*}

\begin{bsp}
  \label{bsp:konv.konv}
  (Sei stets \nat{n})
  \begin{enumerate}
  \item Sei \kmplx{z} und $a_n = z \,\forall n$.\\
    \emph{Behauptung.} $a_n\to z$ (\ninf).
    \begin{proof} Sei $\ep>0$ beliebig gegeben. Wähle $N_\ep=1$. Sei
      $n \ge N_\ep=1$. Dann $\abs{a_n-z}=0<\ep$. \end{proof}
  \item Sei $p\in\set{Q}$ mit $p>0$ und $a_n=n^{-p}$, also $(a_n)=(1,\frac{1}{2^p},\frac{1}{3^p}, \ldots)$.\\
    \emph{Behauptung.} $a_n\to0$ (\ninf) (speziell für $p=1$:
    $\frac{1}{n}\to0$ (\ninf).
    \begin{proof} Sei $\ep>0$ beliebig gegeben. Wähle $\nat{N_\ep}$
      mit $N_\ep\ge\ep^{-\frac{1}{p}}$ ($N_\ep$ existiert nach
      Satz~\ref{satz:zahlen.arch-ord}). Sei $n\ge N_\ep$. Dann:
      \[\abs{a_n-0}=n^{-p}
      \nach{\le}{\ref{satz:zahlen.potenzen}\ref{satz:zahlen.potenzen.d}}
      N_\ep^{-p}
      \nach{\le}{\ref{satz:zahlen.potenzen}\ref{satz:zahlen.potenzen.d}}
      \left(\ep^{-\frac{1}{p}}\right)^{-p}=\ep.\]
    \end{proof}
  \item Sei $a_n=(-1)^n$.\\
    \emph{Behauptung.} Diese Folge ist divergent.
    \begin{proof} Zu zeigen: $\forall \kmplx{a}\,\exists \ep_a>0 \,\forall \nat{N} \,\exists n=n_{a,N}\ge N: \abs{a_N-a}>\ep_a$.\\
      1. Fall: $a=1$. Wähle $\ep_1=1$. Sei $\nat{N}$ gegeben. Sei $n \ge N$ ungerade. Dann $\abs{a_n-a}=\abs{-1-1}=2>1=\ep_1$.\\
      2. Fall: $a=-1$ genauso.\\
      3. Fall: $\kmplx{a}\backslash\{-1,1\}$. Wähle
      $\ep_a=\frac12\min\{\abs{1-a},\abs{-1-a}\}>0$. Sei $\nat{N}$
      gegeben. Wähle $n=N$. Dann \[\abs{a_n-a}=\begin{cases}
        \abs{1-a}, & \text{wenn }n\text{ gerade}\\ \abs{-1-a}, &
        \text{wenn }n\text{ ungerade} \end{cases}\
      >\ep_a.\] \end{proof}
  \end{enumerate}
\end{bsp}

\begin{satz}
  \label{satz:konv.konv}
  Die Folge $(a_n)$ konvergiere gegen $\kmplx{a}$. Dann gelten:
  \begin{enumerate}
  \item $(a_n)$ ist beschränkt, d.h. $\exists M\ge0: \abs{a_n}\le M,
    \,\forall \nat{n}$. \label{satz:konv.konv.a}
  \item Wenn $a_n \to b$ für $\ninf$ und $\kmplx{b}$, dann
    $a=b$. \label{satz:konv.konv.b}
  \end{enumerate}
\end{satz}
\begin{proof}
  \begin{enumerate}
  \item Wähle $\ep=1$. Nach Def.~\ref{dfn:konv.konv} gibt es $\nat{N}$
    mit $\abs{a_n-a}\le1,\,\forall n\ge N$
    \begin{align*}
      &\folgt \abs{a_n}=\abs{a_n-a+a} \nach{\le}{$\triangle$-Ungl.} \abs{a_n-a}+\abs{a} \le 1+\abs{a},\,\forall n\ge N\\
      &\folgt \abs{a_n} \le \max
      \{1+\abs{a},\abs{a_1},\abs{a_2},\dotsc,\abs{a_{N-1}}\} =:
      M,\,\forall \nat{n}.
    \end{align*}
  \item Sei $\ep>0$ gegeben. Nach Vorraussetzung und
    Def. \ref{dfn:konv.konv} existieren $\nat{N_{\ep,\,a}}$ und
    $\nat{N_{\ep,\,b}}$, sodass $\abs{a_n-a}\le\ep\,\forall n\ge
    N_{\ep,\,a}$ und $\abs{a_n-b}\le\ep\,\forall n\ge
    N_{\ep,\,b}$. Setze $N_\ep=\max
    \{N_{\ep,\,a},\,N_{\ep,\,b}\}$. Dann
    \[0\le\abs{a-b}=\abs{a-a_n+a_n-b}\nach{\le}{$\triangle$-Ungl.}\abs{a-a_n}+\abs{a_n-b}\le2\ep\]
    (nach obiger Abschätzung). Da $\ep>0$ beliebig war, folgt
    $\abs{a-b}=0$, also $a=b$ (siehe
    Satz~\ref{satz:zahlen.arch-ord}\ref{satz:zahlen.arch-ord.c})
  \end{enumerate}
\end{proof}

\begin{bsp}
  \label{bsp:konv.konv.a}
  Sei $p\in\set{Q}$ mit $p>0$ und $a_n=n^p$ für $\nat{n}$.\\
  \emph{Behauptung.} $(a_n)$ ist unbeschränkt, also divergent nach
  Satz~\ref{satz:konv.konv}\ref{satz:konv.konv.a}
  \begin{proof} Ann.: Es existiere ein $M\ge0$ mit $a_n=n^p\le
    M,\,\forall \nat{n}
    \folgtnach{\ref{satz:zahlen.potenzen}\ref{satz:zahlen.potenzen.d}}
    n\le M^{\frac1p}\,\forall \nat{n}$ \folgt $\blitz$
    Satz~\ref{satz:zahlen.arch-ord} \end{proof}
\end{bsp}

\begin{bem}
  \label{bem:konv.konv-c-n0}
  \begin{enumerate}
  \item Sei $(a_n)_{n\ge1}$ eine Folge. Es gebe ein $\kmplx{a}$ und
    eine Konstante $c>0$, sodass:
    \begin{equation}\label{eqn:konv.konv-c-n0.sternchen}
      \forall\ep>0\,\exists \nat{N_\ep}\,\forall n\ge N_\ep:\abs{a_n-a}\le c\ep
      \tag{$*$}
    \end{equation}
    \emph{Behauptung.} Dann $a_n \to a$ für $\ninf$.\\
    \begin{proof} Setze $\eta=c\ep \gdw \ep=\frac \eta c$. Setze
      $N_\eta=N_\ep$. Dann liefert
      (\ref{eqn:konv.konv-c-n0.sternchen}):
      \[\forall\eta>0\,\exists \nat{N_\eta}\,\forall n\ge
      N_\eta:\abs{a_n-a}\le \eta\]
    \end{proof}
    Vorsicht: $c$ darf \emph{nicht} von $n$, $\ep$
    abhängen! \label{bem:konv.konv-c-n0.a}
  \item Für $n_0\in\set{Z}$ setze $J(n_0)=\{n\in\set{Z}:n\ge
    n_0\}$. Eine Abbildung $A: J(n_0)\ra\set{C}$ bezeichnet man auch
    als Folge. Man schreibt wieder $a_n$ statt $A(n)$ und $(a_n)_{n\ge
      n_0}$ statt $A$. Die Konvergenz von $(a_n)_{n\ge n_0}$ definiert
    man wie in Def.~\ref{dfn:konv.konv}, wobei man zusätzlich
    $N_\ep\ge n_0$ fordert. Indem man $b_n:=a_{n+n_0-1}$ für $\nat{n}$
    setzt, erhält man eine Folge $(b_n)_{n\ge1}$ mit Indexbereich
    $J(n_0)$. Offenbar konvergiert $(a_n)_{n\ge n_0}$ genau dann, wenn
    $(b_n)_{n\ge1}$ konvergiert, und die jeweiligen Grenzwerte sind
    gleich. Somit können wir uns weiterhin auf den Fall $n_0=1$
    beschränken. \label{bem:konv.konv-c-n0.b}
  \end{enumerate}
\end{bem}

\begin{satz}
  \label{satz:konv.greg}
  Seien $(a_n)_{n\ge1}$ und $(b_n)_{n\ge1}$ Folgen und $a, b \in
  \set{C}$. Es gelte $a_n \to a$ und $b_n \to b$ für $\ninf$. Dann:
  \begin{enumerate}
  \item $a_n+b_n \to a+b$ für $\ninf$ \label{satz:konv.greg.a}
  \item $a_n \cdot b_n \to ab$ für $\ninf$ (speziell $ab_n \to ab$ für
    $\ninf$)\label{satz:konv.greg.b}
  \item Wenn $a\ne0$, dann existiert ein $\nat{N}$, sodass $a_n\ne0$
    für alle $n \ge N$ und es gilt $\frac{1}{a_n}\to\frac{1}{a}$ für
    $\ninf$ ($n\ge N$). \label{satz:konv.greg.c}
  \end{enumerate}
\end{satz}
\begin{proof}
  Sei $\ep>0$ (beliebig) gegeben. Nach Voraussetzung:
  \begin{equation}\label{eqn:konv.greg.einzel-grenzw} \exists
    \nat{N_{\ep,\,a}}, \nat{N_{\ep,\,b}} \text{, sodass}
    \abs{a_n-a}\le\ep\,\forall n\ge N_{\ep,\,a} \text{ und }
    \abs{b_n-b}\le\ep\,\forall n\ge N_{\ep,\,b} \end{equation}
  Setze $N_\ep = \max{\{N_{\ep,\,a},\,N_{\ep,\,b}\}}$. Sei $n\ge N_\ep$.
  \begin{enumerate}
  \item $\abs{a_n+b_n-(a+b)} \nach{\le}{$\triangle$-Ungl.}
    \abs{a_n-b}+\abs{b_n-b}
    \nach{\le}{(\ref{eqn:konv.greg.einzel-grenzw})} 2\ep,\,\forall
    \nat{n} \folgtnach{Bem~\ref{bem:konv.konv-c-n0}}$ Beh. a)
  \item \begin{align*} \abs{a_nb_n-ab} &= \abs{(a_n-a)b+a(b_n-b)}
      \nach{\le}{$\triangle$-Ungl., \ref{satz:zahlen.kreg}}
      \abs{a_n-a}\cdot\underbrace{\abs{b_n}}_{\le M \text{ nach \ref{satz:konv.konv}}}+\abs{a}\cdot\abs{b_n-b}\\
      &\nach{\le}{(\ref{eqn:konv.greg.einzel-grenzw})}
      (M+\abs{a})\cdot\ep\quad\forall n\ge N_\ep
      \folgtnach{Bem~\ref{bem:konv.konv-c-n0}} \text{Beh. b)}
    \end{align*}
  \item Sei $\ep_0=\frac{\abs{a}}{2}>0$ (da $a\ne0$). Sei $\nat{N=N_{\ep_0,\,a}}$ aus (\ref{eqn:konv.greg.einzel-grenzw}). Dann gilt für $n\ge N$: $\abs{a_n}=\abs{a+a_n-a} \nach{\ge}{\ref{satz:zahlen.kreg}\ref{satz:zahlen.kreg.h}} \abs{a}-\abs{a_n-a} \nach{\ge}{(\ref{eqn:konv.greg.einzel-grenzw})} \abs{a}-\ep_0 = \frac{\abs{a}}{2} > 0 \folgt$ erste Beh.\\
    Setze $\tilde{N_\ep} = \max{\{N_\ep, N\}}$. Sei
    $n\ge\tilde{N_\ep}$. Dann
    \[\abs{\frac{1}{\ep_n}-\frac{1}{a}} = \abs{\frac{a-a_n}{a_n}}
    \nach{=}{\ref{satz:zahlen.kreg}\ref{satz:zahlen.kreg.h}}
    \frac{\abs{a-a_n}}{\abs{a}-\abs{a_n}}
    \nach{\le}{(\ref{eqn:konv.greg.einzel-grenzw})}
    \frac{\ep}{\abs{a}\cdot\frac{\abs{a}}{2}} \quad(\forall
    n\ge\tilde{N_\ep})\]
  \end{enumerate}
\end{proof}

\begin{bsp}
  \label{bsp:konv.greg}
  \[a_n = \frac{3n^2+2n}{5n^2+4n+i}\] \emph{Behauptung.}$\
  a_n\to\frac35$ für $\ninf$
  \begin{proof}
    \[a_n=\frac{3+\frac{2}{n}}{5+\frac{4}{n}+\frac{i}{n^2}},\quad\nat{n}.\]
    Nach Bsp.~\ref{bsp:konv.konv}: $3\to3$, $5\to5$,
    $\frac{1}{n}\to0$, $\frac{1}{n^2}\to0$
    ($\ninf$). Satz~\ref{satz:konv.greg}: Zähler $\to 3+2\cdot0=3$,
    Nenner $\to 5\ne0$ ($\ninf$)
    \folgtnach{Satz~\ref{satz:konv.greg}\ref{satz:konv.greg.c}}
    $a_n\to\frac35$ für $\ninf$
  \end{proof}
\end{bsp}

\begin{satz}
  \label{satz:konv.grenzw-ordn}
  Seien $(a_n)_{n\ge1}$, $(b_n)_{n\ge1}$, $(c_n)_{n\ge1}$ reelle
  Folgen mit $a_n \to a$ und $b_n \to b$ für $\ninf$. Dabei sei $a,
  \reell{b}$ (dies gilt stets gemäß
  Satz~\ref{satz:konv.grenzw-komplex}). Sei $\nat{n_0}$.
  \begin{enumerate}
  \item Wenn $a_n\le b_n$ für $n\ge n_0$, dann $a\le
    b$. \label{satz:konv.grenzw-ordn.a}
  \item Wenn $a_n\le c_n\le b_n$ für $n\ge n_0$ und $a=b$, dann
    $c_n\to a$ für $\ninf$
    ("`Sandwichprinzip"'). \label{satz:konv.grenzw-ordn.b}
  \end{enumerate}
\end{satz}
\begin{proof}
  Sei $\ep>0$ gegeben. Wie in (\ref{eqn:konv.greg.einzel-grenzw})
  existiert ein $\nat{N_\ep}$, sodass
  \begin{equation}\abs{a_n-a}\le\ep\text{, }\abs{b_n-b}\le\ep\text{
      für alle }n\ge N_\ep \label{eqn:konv.sternchen}\tag
    {$*$} \end{equation}
  Sei $n\ge\max{\{N_\ep, n_o\}}$.
  \begin{enumerate}
  \item \[a-b = a-a_n+\underbrace{a_n-b_n}_{\le0\text{ (n.V.)}}+b_n-b
    \le \abs{a-a_n}+\abs{b-b_n} \nach{\le}{(\ref{eqn:konv.sternchen})}
    2\ep.\] Da $\ep>0$ beliebig ist, folgt $a-b\le0$ (Wenn $a-b>0$
    wäre, dann folgte $\blitz$ mit
    Satz~\ref{satz:zahlen.arch-ord}\ref{satz:zahlen.arch-ord.c})
    \folgt $a\le b$.
  \item
    \begin{align*}
      \abs{c_n-a}&=\begin{cases} c_n-a \nach{\le}{n.V.} b_n-a \le \abs{b_n-a} & \text{für }c_n\ge a\\a-c_n \nach{\le}{n.V.} a-a_n \le \abs{a_n-a} & \text{für }c_n<a \end{cases}\\
      &\negthickspace\nach{\le}{(\ref{eqn:konv.sternchen})}\ep \text{
        für }n\ge\max{\{N_\ep,n_0\}}\text{, da }a=b \quad\folgt c_n\to
      a\text{ für }\ninf
    \end{align*}
  \end{enumerate}
\end{proof}

\begin{bsp}
  \label{bsp:konv.kte-wurzel}
  \emph{Behauptung.}  Sei $q>0$. Dann $a_n := q^{\frac{1}{n}}\to1$ für
  $\ninf$.\\$(a_n)=(a, \sqrt{a}, \sqrt[3]{a}, \sqrt[4]{a}, \dotsc)$
  \begin{proof}
    \begin{enumerate}
    \item Sei zuerst $q\ge1$. Dann $a_n\ge$ nach
      Satz~\ref{satz:zahlen.potenzen}\ref{satz:zahlen.potenzen.d}. Weiter:
      \begin{gather*}
        q=a_n^n=\Big(1+\underbrace{\left(a_n+1\right)}_{>-1}\Big)^n \nach{\ge}{Bernoulli-U.} 1+n\left(a_n-1\right)\\
        \folgt 0\le a_n-1 \le \frac{q-1}{n} \to 0\text{ für }\ninf
      \end{gather*}
      (nach Bsp.~\ref{bsp:konv.konv}, Satz~\ref{satz:konv.greg}) \folgt nach Satz~\ref{satz:konv.grenzw-ordn}\ref{satz:konv.grenzw-ordn.b} $a_n-1\to0 \folgt a_n\to1$ für $\ninf$.\\
    \item Sei nun $0<q<1$. Dann $\frac1q>1$ und
      $\frac{1}{a_n}=\left(\frac1a\right)^\frac1n\to1$ nach Teil
      a). Nach Satz~\ref{satz:konv.greg}\ref{satz:konv.greg.c} \folgt
      $a_n=\left(\frac{1}{a_n}\right)^{-1}\to1$ für $\ninf$.
    \end{enumerate}
  \end{proof}
\end{bsp}

\begin{satz}
  \label{satz:konv.grenzw-komplex}
  Sei $(a_n)$ eine Folge. Dann:
  \begin{enumerate}
  \item Sei zusätzlich $a_n\to a$ für $\ninf$. Dann gelten
    $\overline{a_n}\to\overline{a}$, $\Re{a_n}\to\Re{a}$,
    $\Im{a_n}\to\Im{a}$, $\abs{a_n}\to\abs{a}$ (jeweils für
    $\ninf$). Wenn zusätzlich $(a_n)$ reell ist, dann ist $\reell{a}$.
  \item Es gelte $\Re{a_n}\to b$ und $\Im{a_n}\to c$ für $\ninf$. Dann
    $a_n\to b+ic$ für $\ninf$.
  \end{enumerate}
\end{satz}
\begin{proof}
  \begin{enumerate}
  \item $0\le\abs{\overline{a_n}-\overline{a}}
    \nach{=}{\ref{satz:zahlen.kreg}} \abs{\overline{a_n-a}}
    \nach{=}{\ref{satz:zahlen.kreg}} \abs{a_n-a}\to0$ für
    $\ninf$. Satz~\ref{satz:konv.grenzw-ordn}\ref{satz:konv.grenzw-ordn.b}
    \folgt $\abs{\overline{a_n}-\overline{a}}\to0$ \folgt
    $\overline{a_n}\to\overline{a}$ für $\ninf$. \folgt $\Re{a_n}
    \nach{=}{\ref{satz:zahlen.kreg}}
    \frac12(a_n+\overline{a_n})\to\frac12(a+\overline{a})=\Re{a}$ für
    $\ninf$. Entsprechend $\Im{a_n}\to\Im{a}$ (verwende in beiden
    Fällen Satz~\ref{satz:konv.greg}). Ferner $\abs{\abs{a_n}-\abs{a}}
    \nach{\le}{\ref{satz:zahlen.kreg}} |a_n-a| \nach{\ra}{n.V.} 0$
    ($\ninf$). Satz~\ref{satz:konv.grenzw-ordn}\ref{satz:konv.grenzw-ordn.b}
    \folgt $\abs{a_n}\to\abs{a}$ für $\ninf$. Wenn $\reell{a_n}$, dann
    $\Im{a_n}=0$ \folgt $\Im{a}=0$.
  \item $0\le\abs{a_n-(b+ic)}=\abs{(\Re{a_n}-b)+i(\Im{a_n}-c)}
    \nach{\le}{$\triangle$-Ungl.}
    \abs{\Re{a_n}-b}+\abs{\Im{a_n}-c}\to0$,
    n.\,V. ($\ninf$). Satz~\ref{satz:konv.grenzw-ordn}\ref{satz:konv.grenzw-ordn.b}
    \folgt Beh. b)
  \end{enumerate}
\end{proof}

\section{Monotone Folgen}
\label{sec:konv.folge-monoton}

\begin{dfn}
  \label{dfn:konv.monotonie}
  Sei $(a_n)_{n\ge1}$ eine reelle Folge.
  \begin{enumerate}
  \item $(a_n)$ \emph{wächst} (\emph{strikt}), wenn $a_{n+1}\ge a_n$
    ($a_{n+1}>a_n$) für alle \nat{n}.
  \item $(a_n)$ \emph{fällt} (\emph{strikt}), wenn $a_{n+1}\le a_n$
    ($a_{n+1}<a_n$) für alle \nat{n}.
  \item $(a_n)$ ist (\emph{strikt}) \emph{monoton}, wenn $(a_n)$
    (strikt) wächst oder (strikt) fällt.
  \end{enumerate}
  \begin{bem*}
    $(a_n)$ wächst (strikt) \gdw $\left(-a_n\right)$ fällt (strikt)
  \end{bem*}
\end{dfn}

\begin{bsp}
  \label{bsp:konv.monotonie}
  \begin{enumerate}
  \item Sei $0<p\in\set{Q}$. Dann fällt $a_n=n^{-p}$ (\nat{n}) strikt,
    da $(n+1)^{-p}<n^{-p}$ nach
    Satz~\ref{satz:zahlen.potenzen}\ref{satz:zahlen.potenzen.d}.
  \item $a_n = \frac{n}{2n+1} = \frac12\cdot\frac{2n+1-1}{2n+1} =
    \frac12 - \frac12\cdot\frac{1}{2n+1}$ wächst strikt, da
    $\frac{1}{2n+1}$ strikt fällt (vgl. a)).
  \item $a_n=(-1)^n$ ist nicht monoton, da $a_{n+1}=1>-1=a_n$ für
    ungerade $n$ und $a_{n+1}=-1<1=a_n$ für gerade $n$.
  \end{enumerate}
\end{bsp}

\noindent Standardbsp. für divergente Folgen:
\begin{enumerate}
\item $a_n=(-1)^n$ nicht monoton, aber beschränkt
\item $a_n=n$ monoton, aber nicht beschränkt
\end{enumerate}

\begin{thm}
  \label{thm:konv.mon-beschr}
  Sei $(a_n)_{n\ge1}$ eine reelle Folge. Dann gelten:
  \begin{enumerate}
  \item Wenn $(a_n)$ wächst und nach oben beschränkt ist, dann
    existiert
    \[\lim_{\ninf}{a_n}=\sup_{n\ge1}{a_n}:=\sup{\{a_n:\nat{n}\}}\]
    \label{thm:konv.mon-beschr.a}
  \item Wenn $(a_n)$ fällt und nach unten beschränkt ist, dann
    existiert
    \[\lim_{\ninf}{a_n}=\inf_{n\ge1}{a_n}:=\inf{\{a_n:\nat{n}\}}\]
    \label{thm:konv.mon-beschr.b}
  \end{enumerate}
\end{thm}
\begin{proof}
  \begin{enumerate}
  \item n. V. $\exists a:=\sup_{n\ge1}{a_n}$. Sei $\ep>0$ beliebig
    gegeben. Satz~\ref{satz:zahlen.sup-schranke} \folgt $\exists
    \nat{N_\ep}$ mit $a-\ep < a_{N_\ep}\le a$. Sei $n\ge N_\ep$. Da
    $(a_n)$ wächst und $a=\sup{a_n}$ gilt:
    \[a-\ep \le a_{N_\ep} \le a_n \folgt a_n-a \le \ep \quad \forall
    n\ge N_\ep\]
  \item Betrachte $-a_n$ und verwende Teil a) und
    Satz~\ref{satz:zahlen.sup-intervalle}\ref{satz:zahlen.sup-intervalle.d}
  \end{enumerate}
\end{proof}

\begin{bsp}[Heron-Verfahren zur Quadratwurzelbestimmung]
  \label{bsp:konv.heron}
  Sei $x>0$ gegeben. Definiere rekursiv $a_1=1$ und
  $a_{n+1}=\frac12(a_n+\frac{x}{a_n})$ für \nat{n}. (Beachte:
  $a_1>0$. Wenn $a_n>0$, dann $a_{n+1}>0$ \folgtnach{Indukt.} $a_k>0$
  für alle \nat{k}.  \\\emph{Behauptung.} $a_n\to\sqrt{x}$ (\ninf)
  \begin{proof}
    1. Schritt: Zeige Konvergenz mit Thm.~\ref{thm:konv.mon-beschr}.\\
    Sei \nat{n}. Dann
    \begin{equation} \label{eqn:konv.heron.stern} a_{n+1}-a_n
      \nach{=}{Def.} \frac{a_n}{2} + \frac{x}{2a_n}-a_n =
      \underbrace{\frac{1}{2a_n}}_{>0}
      \underbrace{\left(x-a_n^2\right)}_{\text{Vorzeichen?}}
      \tag{$*$} \end{equation} Sei $n\ge2$. Dann
    \begin{multline*} \label{eqn:konv.heron.doppelstern} a_n^2 - x
      \nach{=}{Def.} \frac14 \left( a_{n-1} + \frac{x}{a_{n-1}}
      \right)^2 - x = \frac14 \left( a_{n-1}^2 + 2x +
        \frac{x^2}{a_{n-1}^2} - 4x\right) - y \\= \frac14 \left(
        a_{n+1} - \frac{x}{a_{n-1}} \right)^2 \ge 0
      \tag{$**$} \end{multline*}
    $\xLongrightarrow[\text{(\ref{eqn:konv.heron.doppelstern})}]{\text{(\ref{eqn:konv.heron.stern})}}$ $a_{n+1}-a_n \le 0$ und $a_n^2 \ge x$ \folgtnach{\ref{satz:zahlen.potenzen}} $a_n \ge \sqrt{x}$ (für $n\ge2$).\\Thm. \ref{thm:konv.mon-beschr} \folgt $\exists\,a:=\lim_{\ninf}{a_n}$.\\
    2. Schritt: Berechne $a$ mit Hilfe der Rekursion. Satz
    \ref{satz:konv.grenzw-ordn}: $a \ge \sqrt{x} > 0$.\\Ferner:
    $\underbrace{a_{n+1}}_{\to
      a}=\underbrace{\frac12\left(a_n+\frac{x}{a_n}\right)}_{\to
      \frac12\left(a+\frac{x}{a}\right)}$ für \ninf (nach Satz
    \ref{satz:konv.greg}, $a\ne0$). Nach Satz \ref{satz:konv.konv}:
    \[a = \frac12\left(a+\frac{x}{a}\right) \gdw a = \frac{x}{a} \gdw
    x = a^2 \gdwwegen{a>0} a = \sqrt{x}\]
  \end{proof}
\end{bsp}

\begin{bsp}[Die \textsc{Euler}sche Zahl $e$]
  \label{bsp:konv.e}
  Sei \nat{x}, \[a_n = \left(1+\frac1n\right)^n \text{, } b_n=\sum_{j=0}^{n}{\frac{1}{j!}} = 1+1+\frac12+\frac16+\dotsb+\frac{1}{n!}\]\\
  \emph{Behauptung.}
  $\exists\,\lim_{\ninf}{a_n}=\lim_{\ninf}{b_n}=:e\approx2,71828\dotsc$
  \begin{proof}
    Überblick:
    \begin{enumerate}
    \item[Beh. a)] $(a_n)$ wächst strikt
    \item[Beh. b)] $a_n \le b_n < 3 \,\forall\nat{n}$
      \begin{equation*} \label{eqn:konv.e.stern} (*) \begin{cases}
          \text{a)}+\text{b)}+\text{Thm. \ref{thm:konv.mon-beschr}}
          &\folgt \,\exists\,\lim_{\ninf}{a_n}=:a,
          \lim_{\ninf}{b_n}=:b \\ \text{b)}+\text{Satz
            \ref{satz:konv.grenzw-ordn}\,\ref{satz:konv.grenzw-ordn.a}}
          &\folgt a \le b \end{cases}
      \end{equation*}
    \item[Beh. c)] $a \ge b$
    \end{enumerate}
    Beachte: $(b_n)$ wächst strikt.\\\folgt Beh.
    \begin{enumerate}
    \item Sei \nat{n}. Dann
      \begin{gather*}
        \frac{a_{n+1}}{a_n} = \left(1+\frac{1}{n+1}\right)
        \frac{\left(1+\frac{1}{n+1}\right)^n}{\left(1+\frac1n\right)^n}
        = \frac{n+2}{n+1} \left( \frac{\frac{n+2}{n+1}}{\frac{n+1}{n}}
        \right)^n = \frac{n+2}{n+1} \left(
          \frac{\left(n+1\right)^2-1}{\left(n+1\right)^2} \right)^n
        \\= \underbrace{\frac{n+2}{n+1}}_{>0}
        \Big(1-\underbrace{\frac{1}{\left(1+n\right)^2}}_{>-1} \Big)^n
        \nach{\ge}{\ref{satz:zahlen.bernoulli}} \frac{n+2}{n+1}
        \left(1-\frac{n}{\left(1+n\right)^2} \right) = \frac{n+2}{n+1}
        \cdot \frac{1+n+n^2}{\left(1+n\right)} \\=
        \frac{n^3+3n^2+3n+2}{\left(1+n\right)^3}
        \nach{=}{\ref{bsp:vor.binom}}
        \frac{\left(n+1\right)^3+1}{\left(n+1\right)^3} > 1
      \end{gather*}
      $(b_n)$ wächst offensichtlich
    \item \[a_n \nach{=}{\ref{bsp:vor.binom}}
      \sum_{j=0}^{n}\binom{n}{j}\left(\frac1n\right)^j\] Für $1\le j
      \le n$ ist
      \begin{equation} \label{eqn:konv.e.plus}
        \binom{n}{j}\frac{1}{n^j} =
        \frac{1}{j!}\cdot\frac{n!}{(n-j)!}\cdot\frac{1}{n^j} =
        \frac{1}{j!}\cdot\underbrace{\frac{n}{n}}_{=1}\cdot\underbrace{\frac{n-1}{n}}_{\in(0,1)}\cdot\underbrace{\frac{n-2}{n}}_{\in(0,1)}\dotsm\underbrace{\frac{n-j+1}{n}}_{\in(0,1)}
        \le \frac{1}{j!} \le \frac{1}{2^{j-1}}
        \tag{$+$} \end{equation} \emph{Behauptung.} $2^{n-1}\le
      n!\,\forall\,\nat{}$
      \begin{proof} (per vollst. Ind.)\\
        IA: $n=1$ ist klar.\\
        IS: Beh. gelte für ein \nat{n} (IV).\\
        \folgt $2^n \nach{\le}{IV} 2n! \le (n+1)n! = (n+1)!$
      \end{proof}
      \begin{multline*}
        \folgt a_n=1+\sum_{j=1}^{n}\binom{n}{j}\frac{1}{n^j} \nach{\le}{(\ref{eqn:konv.e.plus})} 1 + \sum_{j=1}^{n} \frac{1}{j!} = b_n\\
        \le 1+\sum_{j=1}^{n} \frac{1}{2^{j-1}} \wegen{=}{k:=j-1} 1 +
        \sum_{k=0}^{n-1} \left(\frac12\right)^k
        \nach{=}{\ref{eqn:vor.np1-ueber-j}}
        1+\frac{1-\left(\frac12\right)^n}{1-\frac12} < 1 +
        \frac{1}{\frac12} = 3
      \end{multline*}
    \item Sei \nat{m} und $n\ge m$, $m$ fest. Wie in b):
      \begin{multline*} \label{eqn:konv.e.doppelplus}
        a_n=1+\sum_{j=1}^{n}\underbrace{\frac{1}{j!}\cdot\frac{n}{n}\cdot\frac{n-1}{n}\cdot\frac{n-2}{n}\dotsm\frac{n-j+1}{n}}_{>0}\\\ge
        1+\sum_{j=1}^{n}\frac{1}{j!}\underbrace{\left(1-\frac1n\right)}_{\to1(\ninf)}\dotsm\underbrace{\left(1-\frac{j-1}{n}\right)}_{\to1(\ninf)}=:c_{mn}
        \tag{$++$}
      \end{multline*}
      nach Bsp. \ref{bsp:konv.konv}, Satz \ref{satz:konv.greg} \folgt
      $c_{mn}\to 1+\sum_{j=1}^{m} \frac{1}{j!}=b_m$ für \ninf, $m$
      fest. Lasse \ninf gehen in (\ref{eqn:konv.e.doppelplus}). Dann
      liefern (\ref{eqn:konv.e.stern}) und Satz
      \ref{satz:konv.grenzw-ordn}, dass $a\ge b_m$ für \nat{m}. Mit
      $m\to\infty$, (\ref{eqn:konv.e.stern}), Satz
      \ref{satz:konv.grenzw-ordn} folgt $a\ge b$.
    \end{enumerate}
  \end{proof}
\end{bsp}

\section{Teilfolgen und Vollständigkeit}
\label{sec:konv.teilfolgen}

\paragraph{Motivation:} $(a_n) = ((-1)^n) = (-1,1,-1,1,\dotsc)$ ist
divergent, enthält aber konvergente "`Teile"'.

\begin{dfn}
  \label{dfn:konv.teilfolge}
  Sei $(a_n)_{n\ge1}$ eine Folge und $\varphi: \set{N} \rightarrow
  \set{N}$ eine strikt wachsende Funktion
  (d.h. $\varphi(n+1)>\varphi(n)\,\forall\nat{n}$). Setze
  $b_j=a_{\varphi(j)}, \nat{j}$. Dann heißt die Folge $(b_j)j_{j\ge1}$
  \emph{Teilfolge} von $(a_n)_{n\ge1}$ (TF). Man schreibt meist
  $\left(a_{n_j}\right)_{j\ge1}$ statt $(b_j)_{j\ge1}$.
\end{dfn}

\begin{bsp*}
  \begin{enumerate}
  \item $(a_n)$ ist Teilfolge von sich selbst, wähle
    $\varphi(j)=j\,\forall\nat{j}$
  \item Sei $a_n=(-1)^n$. Wähle $\varphi(j)=2j$ für \nat{j}. Dann ist
    $b_j:=a_{2j}=1 \,\forall\nat{j}$.
  \item \[a_n = \begin{cases} \frac{1}{n^2} & \text{wenn } n \text{
        Primzahl} \\ 1 & \text{sonst} \end{cases} \text{, } \nat{n}
    \text{. } (a_n) = (1, \frac14, \frac19, 1, \frac{1}{25}, 1,
    \dotsm)\] Setze $\varphi(j)=j$-te Primzahl, \nat{j}. \folgt $(b_j)
    = (a_{\varphi(j)}) = (\frac14, \frac19, \frac{1}{25},\dotsm)$
  \end{enumerate}
\end{bsp*}

\begin{bem*}
  $a_n\to a$ (\ninf) \folgt $a_{n_j}\to a$ ($j\to\infty$) für jede
  Teilfolge.
\end{bem*}

\begin{dfn}
  \label{dfn:konv.hp}
  Sei $(a_n)$ eine Folge und \kmplx{a}. Dann heißt $a$
  \emph{Häufungspunkt} (HP) von $(a_n)$, wenn für jedes $\ep>0$ für
  unendlich viele $n$ die Ungleichung $\abs{a-a_n}\le\ep$ gilt.
\end{dfn}
\begin{bsp*}
  \begin{enumerate}
  \item $(-1)^n$ hat HP $+1$ und $-1$, da $a_n\in\bar{B}(1,\ep)$ für
    alle $\ep>0$ und alle geraden \nat{n}, sowie
    $a_n\in\bar{B}(-1,\ep)$ für alle $\ep>0$ und alle ungeraden
    \nat{n}.
  \item Die Folge $a_n=n$ hat keinen HP, da $\abs{a_n-a_m}\ge1$, $n\ne
    m$. Also liegt in einer Kugel $\bar{B}(a, \frac13)$ höchstens ein
    $a_n$.
  \end{enumerate}
\end{bsp*}

\begin{satz}
  \label{satz:konv.tf-hp}
  Sei $(a_n)$ eine Folge und \kmplx{a}. Dann:
  \[a \text{ ist HP} \gdw \exists \text{ TF mit } a_{n_j} \to a \
  (j\to\infty)\]
\end{satz}
\begin{proof}
  \begin{enumerate}
  \item["`$\Rightarrow$"'] Sei $a$ HP. Wir definieren rekursiv eine TF
    $(a_{n_j})$ mit
    $\abs{a-a_{n_j}}\le\frac1j\,\forall\nat{j}$. \folgt $a_{n_j}\to a$
    nach Satz \ref{satz:konv.grenzw-ordn} (für $j\to\infty$). Wähle
    \nat{n_1} mit $\abs{a_{n_1}-a}\le1$ (verwende Voraussetzung mit
    $\ep=1$). Sei $n_{j-1}$ mit $n_{j-1}>n_{j-2}$ und
    $\abs{a_{n_{j-1}}-a}\le\frac{1}{j-1}$ gewählt. Nach Voraussetzung
    gibt es unendlich viele $a_n$ in $\bar{B}(a,\frac1j)$. Da
    $\{1,\dotsc,n_{j-1}\}$ endlich ist, existiert ein $n_j>n_{j-1}$
    mit $\abs{a_{n_j}-a}\le\frac1j$. Induktionsprinzip liefert
    gewünschte TF $a_{n_j}\to a$.
  \item["`$\Leftarrow$"'] Sei $a_{n_j}\to a$ ($j\to\infty$). Sei
    $\ep>0$ beliebig gegeben. Dann $\exists \nat{J_\ep}:\,\forall j\ge
    J_\ep:\abs{a_{n_j}-a}\le\ep$. Also $\#\{a_{n_j}:j\ge J_\ep\} =
    \#\{\nat{j}:j\ge J_\ep\}=\infty$ nach Satz
    \ref{satz:zahlen.maechtigkeit-NQ}.
  \end{enumerate}
\end{proof}

\begin{kor}
  \label{kor:konv.lim-hp}
  Wenn $\exists\lim\limits_{\ninf}{a_n}=a$, dann ist $a$ der einzige
  HP von $(a_n)$.
\end{kor}
\begin{proof}
  Satz \ref{satz:konv.tf-hp} \folgt $a$ ist HP, da es der Limes
  ist. Sei $b$ ein weiterer HP von $(a_n)$. Nach Satz
  \ref{satz:konv.tf-hp} $\exists$ TF $a_{n_j}\to b$
  ($j\to\infty$). Dann gilt aber auch $a_{n_j}\to a$
  ($j\to\infty$). Satz \ref{satz:konv.konv} \folgt $a=b$.
\end{proof}

\noindent Sei $(a_n)$ eine reelle beschränkte Folge. Setze
$A_n=\{a_j:j\ge n\}$ für \nat{n}. Beachte $A_{n+1}\subset A_n$, $A_n$
ist beschränkt für alle \nat{n}.
\begin{equation} \label{eqn:konv.infsup-folge} \folgt \exists
  b_n:=\sup{A_n}, c_n:=\inf{A_n}\text{, wobei } b_n \ge a_j \ge c_n \
  \forall j \ge n
\end{equation}
Satz
\ref{satz:zahlen.sup-intervalle}\,\ref{satz:zahlen.sup-intervalle.a}
liefert $b_1 \ge b_n \ge b_{n+1} \ge c_{n+1} \ge c_n \ge c_1$
($\forall\nat{n}$. Nach Thm. \ref{thm:konv.mon-beschr} existieren
\begin{equation} \begin{split} \label{eqn:konv.def-limsupinf}
    \lim_{\ninf}{b_n} = \inf_{\nat{n}}{b_n} = \inf_{\nat{n}}{\sup_{j\ge n}{a_j}} &=: \ls_{\ninf}{a_n} = \limsup_{\ninf}{a_n}\;\text{("`Limes superior"')}\\
    \text{und}\;
    \lim_{\ninf}{c_n} = \sup_{\nat{n}}{c_n} = \sup_{\nat{n}}{\inf_{j\ge n}{a_j}} &=: \li_{\ninf}{a_n} = \liminf_{\ninf}{a_n}\;\text{("`Limes inferior"')}\\
  \end{split} \end{equation} (\ref{eqn:konv.infsup-folge}), Satz
\ref{satz:konv.grenzw-ordn}
\folgt \begin{equation} \label{eqn:konv.ordn-limsupinf}
  \li_{\ninf}{a_n}\le\ls_{\ninf}{a_n} \end{equation}
\begin{bsp*}
  $\ls\limits_{\ninf}{(-1)^n}=1$, $\li\limits_{\ninf}{(-1)^n}=-1$, da
  in $A_n$ nur $+1$ und $-1$ stehen.
\end{bsp*}

\begin{thm}[Satz von \textsc{Bolzano}-\textsc{Weierstraß}]
  \label{thm:konv.bw}
  Jede beschränkte Folge $(a_n)_{n\ge1}$ hat eine konvergente
  Teilfolge und damit einen Häufungspunkt. Wenn die Folge außerdem
  reell ist, dann ist $\ls_{\ninf}{a_n}$ das Maximum aller
  Häufungspunkte und $\li_{\ninf}{a_n}$ das Minimum aller
  Häufungspunkte.
\end{thm}
\begin{proof}
  \begin{enumerate}
  \item Sei $(a_n)$ reell und beschränkt. Setze $\bar{a} = \ls_{\ninf}{a_n}$. Suche TF $a_{n_j}\to\bar{a}$ ($j\to\infty$). Wir wissen aus (\ref{eqn:konv.infsup-folge}) und (\ref{eqn:konv.def-limsupinf}): $b_n=\sup_{j\ge n}{a_j}$ konvergiert gegen $\bar{a}$ für \ninf. $b_n$ muss nicht ein Folgeglied sein.\\
    Definiere rekursiv die gewünschte TF $(a_{n_j})$: wähle \nat{N_1}
    mit $\abs{\bar{a}-b_{N_1}}\le\frac12$. Da $b_{N_1} = \sup_{j\ge
      N_1}{a_j}$ ist, existiert nach Satz
    \ref{satz:zahlen.sup-schranke} ein $n_1>N_1$ mit
    $\abs{b_{N_1}-a_{n_1}}\le\frac12$ \folgt $\abs{\bar{a}-a_{n_1}}
    \le \abs{\bar{a}-b_{N_1}} + \abs{b_{N_1}-a_{n_1}} \le 1$. Es
    $n_{j-1}>n_{j-2}$ konstruiert mit
    $\abs{\bar{a}-a_{n_{j-1}}}\le\frac{1}{j-1}$. Wähle $N_j>n_{j-1}$
    mit $\abs{\bar{a}-b_{N_j}}\le\frac{1}{2j}$ (verwende
    (\ref{eqn:konv.def-limsupinf})). Da $b_{N_j} = \sup_{k\ge
      N_j}{a_k}$ existiert nach Satz \ref{satz:zahlen.sup-schranke}
    ein $n_j\ge N_j>n_{j-1}$ mit
    $\abs{b_{N_j}-a_{n_j}}\le\frac{1}{2j}$ \folgt
    $\abs{\bar{a}-a_{n_j}} \le \abs{\bar{a}-b_{N_j}} +
    \abs{b_{N_j}-a_{n_j}} \le \frac1j$. Erhalten induktiv TF
    $a_{n_j}\to\bar{a}$. Insbesondere ist $\bar{a}$ ein HP von $(a_n)$
    nach Satz \ref{satz:konv.tf-hp}. Entsprechend sieht man, dass
    $\li_{\ninf}{a_n}$ ist ein HP von $(a_n)$. Sei $(a_{n_l})$ eine
    weitere TF mit Grenzwert $a$.
    \[\xLongrightarrow[\ref{satz:konv.tf-hp}]{(\ref{eqn:konv.infsup-folge})}
    \underbrace{c_{n_l}}_{\to\li\limits_{\ninf}{a_n}} \le
    \underbrace{a_{n_l}}_{\to a} \le
    \underbrace{b_{n_l}}_{\ls\limits_{\ninf}{a_n}}\]
  \item Sei $(a_n)$ eine beschränkte Folge (in $\set{C}$). Sei
    $x_n=\Re{a_n}$, $y_n=\Im{a_n}$. Dann ist (nach Satz
    \ref{satz:zahlen.kreg}) $(x_n)_n$ beschränkt \folgtnach{a)}
    $\exists$ TF $x_{n_l}\to\reell{x}$ ($l\to\infty$). Weiter ist
    $(y_{n_l})_l$ beschränkt \folgtnach{a)} $\exists$ TF
    $y_{n_{l_j}}\to\reell{y}$ ($j\to\infty$). Damit gilt:
    \[a_{n_{l_j}} = x_{n_{l_j}} + iy_{n_{l_j}} \to x+iy \text{ (}
    j\to\infty \text{)}\]
  \end{enumerate}
\end{proof}

\begin{lem}
  \label{lem:konv.alle-hp}
  Sei $(a_n)$ eine Folge mit den HP $\alpha_1,\dotsc,\alpha_m$ und den
  zugehörigen TF
  $a_{\varphi_1(j)}\to\alpha_1,\dotsc,a_{\varphi_m(j)}\to\alpha_m$
  ($j\to\infty$). Jedes $a_n$ liege in (mindestens) einer TF. Dann hat
  $(a_n)$ keine weiteren HP.
\end{lem}
\begin{proof}
  Annahme: Sei \kmplx{\alpha} ein weiterer HP. Satz
  \ref{satz:konv.tf-hp} \folgt $\exists$ TF $a_{n_l}\to\alpha$
  ($l\to\infty$). Sei $\ep_0 = \frac13 \min{\{\abs{\alpha-\alpha_1},
    \abs{\alpha-\alpha_2}, \dotsc, \abs{\alpha-\alpha_m}\}} >
  0$. Ferner existiert \nat{L} mit
  $\abs{a_{n_l}-\alpha}\le\ep_0\,\forall l\ge L$. \folgt Für $l\ge L$,
  $j\in\{1,\dotsc,m\}$ gilt $\abs{a_{n_l}-\alpha_j} \ge
  \abs{\alpha_j-\alpha}-\abs{\alpha-a_{n_l}} \ge 3\ep_0-\ep_0 =
  2\ep_0$ \folgt $a_{n_l} \not\in B(\alpha_j, \ep_0)\,\forall l\ge L,
  j\in\{1,\dotsc,m\}$. Andererseits liegen die $a_{n_l}$ in mindestens
  einer TF die gegen ein $\alpha_j$ konvergiert \folgt $\blitz$
\end{proof}

\begin{bsp}
  \label{bsp:konv.bw}
  \[a_n = \begin{cases}
    \left(-1\right)^{\frac{n}{2}}\frac1n &\text{, } n \text{ gerade} \\\;\\
    \left(-1\right)^{\frac{n+1}{2}}\frac{2n^2+3}{3n^2-1} &\text{, } n
    \text{ ungerade} \end{cases} \] $\exists$ konv. TF:
  \begin{align*}
    b_k &= a_{2k} = (-1)^k \cdot \frac{1}{2k} \to 0 \; (\kinf) \\
    c_k &= a_{4k+1} = \underbrace{(-1)^{2k+1}}_{=-1} \cdot \frac{2(4k+1)^2+3}{3(4k+1)^2-1} \to -\frac23 \; (\kinf) \\
    d_k &= a_{4k+3} = \underbrace{(-1)^{2k+2}}_{=1} \cdot
    \frac{2(4k+3)^2+3}{3(4k+3)^2-1} \to \frac23 \; (\kinf)
  \end{align*}
  \folgt $\exists$ HP $-\frac23$, $0$, $\frac23$. Nach Lemma \ref{lem:konv.alle-hp} sind das alle HP der Folge.\\
  \folgtnach{\ref{thm:konv.bw}} $\ls\limits_{\ninf}{a_n}=\frac23$,
  $\li\limits_{\ninf}{a_n}=-\frac23$.
\end{bsp}

\begin{kor}
  \label{kor:konv.namevergessen}
  Sei $(a_n)$ beschränkt und \kmplx{a}. Dann gelten:
  \begin{enumerate}
  \item \label{kor:konv.namevergessen.a} $a_n\to a\ (\ninf)$ \gdw
    $(a_n)$ besitzt genau einen HP und dieser ist $a$
  \item \label{kor:konv.namevergessen.b} Sei $(a_n)$ reell. Dann
    konvergiert $(a_n)$ genau dann, wenn
    $\li_{\ninf}{a_n}=\ls_{\ninf}{a_n}$. In diesem Fall gilt
    $\lim_{\ninf}{a_n}=\li_{\ninf}{a_n}=\ls_{\ninf}{a_n}$.
  \end{enumerate}
\end{kor}
\begin{proof}
  \begin{enumerate}
  \item
    \begin{enumerate}
    \item["`$\Rightarrow$"'] Kor. \ref{kor:konv.lim-hp}
    \item["`$\Leftarrow$"'] Sei $a$ der einzige HP von
      $(a_n)$. Annahme: $a_n\not\to a$ (\ninf). Das heißt $\exists
      \ep_0>0:\,\forall \nat{N}:\,\exists n\ge N:
      \abs{a_n-a}>\ep_0$. Wir erhalten induktiv eine TF $(a_{n_l})_l$
      mit $\abs{a_{n_l}-a}>\ep_0\,\forall\nat{l}$ (vgl. Beweis von
      Satz \ref{satz:konv.tf-hp}). Andererseits: Da $(a_{n_l})_l$
      beschränkt ist, liefert Thm. \ref{thm:konv.bw} eine konvergente
      TF $(a_{n_{l_j}})_j$. Nach Satz \ref{satz:konv.tf-hp} und der
      Voraussetzung gilt $a_{n_{l_j}}\to a$ $\blitz$
    \end{enumerate}
  \item Sei nun $(a_n)$ reell. Dann zeigt Thm. \ref{thm:konv.bw}
    $\exists!$ HP von $(a_n)$ \gdw $\li_{\ninf}{a_n}=\ls_{\ninf}{a_n}$
    \folgtnach{a)} Beh.
  \end{enumerate}
\end{proof}

\begin{bem*}
  \[ a_n = 
  \begin{cases}
    1&\text{, } n \text{ gerade} \\
    n&\text{, } n \text{ ungerade}
  \end{cases}
  \] hat genau einen HP ($=1$), ist aber unbeschränkt, also
  divergent. \folgt in \ref{kor:konv.namevergessen} muss man Beschränktheit voraussetzen.
\end{bem*}

\begin{dfn}
  \label{dfn:konv.cauchyfolge}
  Eine Folge $(a_n)$ heißt \emph{Cauchy-Folge} (CF), wenn es für jedes
  $\ep>0$ ein \nat{N_\ep} gibt, sodass $\abs{a_n-a_m}\le\ep$ für alle
  $n, m \ge N_\ep$, d.h.
  \[\forall \ep>0\, \exists \nat{N_\ep}\, \forall n, m \ge N_\ep: \abs{a_n-a_m}\le\ep\]
\end{dfn}

\begin{thm}
  \label{thm:konv.cauchyfolge}
  Eine Folge $(a_n)$ konvergiert genau dann, wenn sie eine
  Cauchy-Folge ist. (Man sagt, dass $\set{C}$ (und damit $\set{R}$)
  vollständig sind.)
\end{thm}
\begin{proof}
  \begin{enumerate}
  \item["`$\Rightarrow$"'] Sei $a_n\to a$ (\ninf). Für $\ep>0$
    existiert also ein \nat{N_\ep} mit $\abs{a_k-a}\le\ep$ für alle
    $k\ge N_\ep$. Damit $\abs{a_n-a_m} \le \abs{a_n-a}+\abs{a-a_m} \le
    2\ep$ für alle $n, m \ge N_\ep$.
  \item["`$\Leftarrow$"'] Sei $(a_n)$ eine CF. Nach
    Def. \ref{dfn:konv.cauchyfolge} mit $\ep=1$ existiert ein
    \nat{N_1} mit $\abs{a_n-a_{N_1}}\le1$ für alle $n\ge N_1$. \folgt
    $\abs{a_n} \le \abs{a_n-a_{N_1}}+\abs{a_{N_1}} \le
    1+\abs{a_{N_1}}$ ($\forall n\ge N_1$) \folgt $(a_n)$ ist
    beschränkt. Thm \ref{thm:konv.bw} \folgt existiert TF $a_{n_j}\to
    a$ ($j\to\infty$). Sei $\ep>0$ gegeben. Dann existiert ein
    \nat{J_\ep} mit
    \begin{equation} \label{eqn:konv.cauchyfolge.stern}
      \abs{a_{n_j}-a} \le \ep \ \forall j \ge J_\ep \tag{*}
    \end{equation}
    Sei ferner $N_\ep$ aus Def. \ref{dfn:konv.cauchyfolge}. Wähle
    $n\ge N_\ep$ Dann existiert ein $n_j\ge N_\ep$ mit $j\ge
    J_\ep$. Somit $\abs{a_n-a} \le \abs{a_n-a_{n_j}}+\abs{a_{n_j}-a}
    \nach{\le}{\ref{dfn:konv.cauchyfolge}, (\ref{eqn:konv.cauchyfolge.stern})} \ep+\ep = 2\ep$
  \end{enumerate}
\end{proof}

\begin{bem*}
  \begin{enumerate}
  \item Cauchyfolgen haben also (in $\set{R}$ und $\set{C}$) dieselben
    Eigenschaften wie konvergente Folgen (kann man auch direkt zeigen).
  \item In Bsp. \ref{bsp:konv.heron} mit $x=2$ und $a_1=1$ ist
    $a_{n+1}=\frac12\left(a_n+\frac{2}{a_n}\right) \in \set{Q}$
    (Beweis per Induktion). Ferner gilt $a_n\to\sqrt{2}$. Nach Bsp
    \ref{bsp:zahlen.sup-wurzel} gilt $\sqrt{2}\not\in\set{Q}$ \folgt
    $\set{Q}$ ist nicht vollständig
  \item Bsp. $a_n=\sqrt{a_n}$. Folge ist unbeschränkt \folgt divergent
    \folgt keine CF. Andererseits: $0 \le \sqrt{n+1} - \sqrt{n} =
    \frac{1}{\sqrt{n+1}+\sqrt{n}} \to 0$ (\ninf). Also: Def
    \ref{def:konv.cauchyfolge} gilt für $m=n+1$, aber $(a_n)$ ist eine CF.
  \end{enumerate}
\end{bem*}

\begin{lem}
  \label{lem:konv.lim-ep}
  Sei $(a_n)$ eine beschränkte und reelle Folge und $\ep>0$. Dann
  \[\exists \nat{J_\ep} \text{ mit } -\ep+\li_{\ninf}{a_n} \le a_j \le
  \ep+\ls_{\ninf}{a_n}\ \forall j \ge J_\ep \]
\end{lem}
\begin{proof}
  Nach Satz \ref{satz:zahlen.sup-schranke} $\exists \nat{\overline{J_\ep}}$
  mit \[\ep+\ls_{\ninf}{a_n} \nach{=}{\ref{eqn:konv.def-limsupinf}} \ep
  + \inf_{\nat{n}}{\sup_{j\ge n}{a_j}}
  \nach{\ge}{\ref{satz:zahlen.sup-schranke}}
  \sup_{j\ge\overline{J_\ep}}{a_j} \ge a_j \ \forall
  j\ge\overline{J_\ep}.\]
  Entprechend: $\exists \nat{\underline{J_\ep}}$ mit $a_j \ge -\ep +
  \li_{\ninf}{a_n} \ \forall j\ge\underline{J_\ep}$. \folgt Beh. mit $J_\ep = \max{\{\overline{J_\ep}, \underline{J_\ep}\}}$.
\end{proof}

\begin{satz}
  \label{satz:konv.limsupinf-reg}
  Seien $(a_n)$, $(b_n)$ beschränkte reelle Folgen. Dann gelten:
  \begin{enumerate}
  \item \label{satz:konv.limsupinf-reg.a}
    \[ \li_{\ninf}{a_n} = - \ls_{\ninf}{(-a_n)} \]
  \item \label{satz:konv.limsupinf-reg.b}
    Wenn $a_n\le b_n$ für alle \nat{n}, dann
    \[ \li_{\ninf}{a_n}\le\li_{\ninf}{b_n} \text{, }
    \ls_{\ninf}{a_n}\le\ls_{\ninf}{b_n} \]
  \item \label{satz:konv.limsupinf-reg.c}
    \begin{align*}
      \li_{\ninf}{(a_n+b_n)}&\ge\li_{\ninf}{a_n}+\li_{\ninf}{b_n}\\
      \ls_{\ninf}{(a_n+b_n)}&\le\ls_{\ninf}{a_n}+\ls_{\ninf}{b_n}
    \end{align*}
  \item \label{satz:konv.limsupinf-reg.d}
    Seien $a_n, b_n \ge 0$ für alle \nat{n}. Dann:
    \begin{align*}
      \ls_{\ninf}{(a_n\cdot b_n)} &\le
      \ls_{\ninf}{a_n}\cdot\ls_{\ninf}{b_n}\\
      \li_{\ninf}{(a_n\cdot b_n)} &\ge
      \li_{\ninf}{a_n}\cdot\li_{\ninf}{b_n}
    \end{align*}
  \item \label{satz:konv.limsupinf-reg.e}
    Wenn in \ref{satz:konv.limsupinf-reg.c} oder
    \ref{satz:konv.limsupinf-reg.d} eine der beiden Folgen
    konvergiert, dann gilt "`$=$"' in den Aussagen.
  \end{enumerate}
\end{satz}
\begin{bem*}
  In \ref{satz:konv.limsupinf-reg.c} oder
  \ref{satz:konv.limsupinf-reg.d} kann "`$<$"' bzw. "`$>$"'
  gelten. Bsp.: $a_n=(-1)^n$, $b_n=(-1)^{n+1}$ \folgt $a_n+b_n=0$
  \folgt $\ls\limits_{\ninf}{a_n+b_n}=0$,
  $\ls\limits_{\ninf}{a_n}=\ls\limits_{\ninf}{b_n}=1$.
\end{bem*}
\begin{proof}
  \begin{enumerate}
  \item \[ \li_{\ninf}{a_n} \nach{=}{(\ref{eqn:konv.def-limsupinf})}
    \sup_{\nat{n}}{\inf_{j\ge n}{a_j}}
    \nach{=}{\ref{satz:zahlen.sup-intervalle}}
    \sup_{\nat{n}}{(-\sup_{j\ge n}{(-a_j)})}
    \nach{=}{\ref{satz:zahlen.sup-intervalle}}
    -\inf_{\nat{n}}{\sup_{j\ge n}{(-a_j)}}
    \nach{=}{(\ref{eqn:konv.def-limsupinf})}
    -\ls_{\ninf}{(-a_n)} \]
  \item Sei $a_j\le b_j\,\forall j$. Nach
    Def. \ref{dfn:zahlen.sup-inf} des Supremums $\sup_{j\ge n}{a_j}
    \le \sup_{j\ge n}{n_j}\,\forall\nat{n}$. Def. des Infimums liefert
    \[\underbrace{\inf_{\nat{n}}{\sup_{j\ge n}{a_j}}}_{=\ls\limits_{\ninf}{a_n}}
    \le \underbrace{\inf_{\nat{n}}{\sup_{j\ge n}{b_j}}}_{=\ls\limits_{\ninf}{b_n}} \]
  \item Sei $\ep>0$. Nach Lemma \ref{satz:zahlen.sup-schranke}
    $\exists \nat{N_\ep}$, sodass
    \begin{gather*}
      a_j\le\ep+\ls_{\ninf}{a_n} \text{, }
      b_j\le\ep+\ls_{\ninf}{b_n}\ \forall j\ge N_\ep.\\
      \folgt \ls_{\ninf}{(a_n+b_n)} \nach{\le}{Def.} \sup_{j\ge
        N_\ep}{(a_j+b_j)} \ge 2\ep + \ls_{\ninf}{a_n} +
      \ls_{\ninf}{b_n}.
    \end{gather*}
    Da $\ep>0$ beliebig ist, folgt Beh. c1).
  \end{enumerate}
  Andere Behauptungen zeigt man ähnlich.
\end{proof}


\chapter{Reihen}
...
\section{Konvergenzkriterien}
...

\section{Einige Vertiefungen/Vermischtes}
...

\subsection*{Umordnung von Reihen}
\begin{bsp}\label{bsp:reihen.umordn-altreihe}
Nach Bsp.~\ref{bsp:reihen.alt-reihe} konvergiert $\sum_{k=1}^\infty (-1)^{k+1} \frac{1}{k} = 1 - \frac{1}{2} + \frac{1}{3} - \frac{1}{4} \dotsb$.
Definiere rekursiv eine "`Umordnung"' $(b_k)_{k\ge 1}$ von $a_k = (-1)^{k+1}\frac{1}{k}$, \nat{k}.\\

\noindent Setze: $m = 1$: $b_1 \da 1$, $b_2 \da -\frac{1}{2}$ \folgt $b_1 + b_2 \ge \frac{1}{4}$\\
$m = 2$: $b_3 \da \frac{1}{3}$, $b_4 \da \frac{1}{5}$, $b_5 \da -\frac{1}{4} \folgt b_3 + b_4 + b_5 \ge \frac{1}{2}$\\

\noindent Definiert seien $b_{n_m} = -\frac{1}{2m}$ für ein \nat{m} mit $m\ge 2$, sowie 
\[b_{n_{m-1}+1} = \frac{1}{2 l_{m+1} +1}, \dotsc, b_{n_{m}-1} = \frac{1}{2l_m -1}\] 
für ein \nat{l_m}. Da $\sum_{k\ge l_m} \frac{1}{2k+1}$ divergiert (Übung) finden wir ein \nat{j}
\[b_{n_m +1} = \frac{1}{2l_m +1}, \dotsc, b_{n_m + j} = \frac{1}{2l+j},\] 
sodass: $b_{n_m+1} + \dotsb + b_{n_m + j} \ge \frac{1}{4} + \frac{1}{2m+2}.$\\

\noindent Setze $n_{m+1} = n_m + j +1$ und $b_{n_{m+1}} = -\frac{1}{2m+2}$
\folgt erhalten rekursiv $(b_k)_{k\ge 1}$ mit \[\sum_{k=1}^{n_m +1} b_n \ge (m+1)\frac{1}{4} \ra \infty,\quad (m\ra\infty)\]
\end{bsp}

\paragraph{Fazit.} $\sum_{k\ge 0}a_k$ divergiert, \emph{obwohl} die Reihe $\sum_{k=1}^\infty a_k$ mit
den gleichen Summanden konvergiert! Also: Hier gilt kein "`unendliches Kommutativgesetz"'.

\begin{dfn}\label{dfn:reihen.umordnung}
Sei $\sum_{k\ge 0}a_k$ eine Reihe und $\varphi: \set{N}_0 \ra \set{N}_0$ 
eine Bijektion. Setze $b_k = a_{\varphi(k)}$ für $k \in \set{N}_0$. Die Reihe $\sum_k b_k$
heißt Umordnung von $\sum_k a_k$.
\end{dfn}

\begin{satz}\label{satz:reihen.umordn-konv}
Sei $\sum_k a_k$ eine absolut konvergente Reihe. Dann konvergiert \emph{jede} Umordnung von $\sum_k a_k$
gegen den Wert $\sum^\infty_k a_k$.
\end{satz}
\begin{proof}
Sei $\ep > 0$ gegeben. Da $\sum \abs{a_k}$ konvergiert, gilt:
\begin{equation}\label{eqn:reihen.umordn-konv.a}\tag{$*$}
\exists N_\ep \in \set{N}: \forall n\ge N_\ep: \sum^n_{J = N_\ep+1} \abs{a_j} \le \ep\quad \text{nach 
Satz~\ref{satz:reihen.cauchy-krit}}
\end{equation}

\noindent Sei $\varphi: \set{N}_0 \ra \set{N}_0$ bijektiv. Sei $M_\ep = \max \left\{\varphi^{-1}(0), \dotsc, \varphi^{-1}(N_\ep)\right\}
\folgt \left\{0, \dotsc, N_\ep\right\} \subseteq \left\{\varphi(0), \dotsc, \varphi(M_\ep)\right\}$.

\bigskip 

\noindent Seien $n \ge N_\ep$, $m\ge M_\ep$. Setze \[D_{m,n} = \sum_{j=0}^m a_{\varphi(j)} + \sum_{j=0}^n (-a_j).\]
Als Summanden treten in $D_{m,n}$ nur $\pm a_k$ auf mit $k > N_\ep$. (alle $a_k$ mit $k \le N_\ep$ treten doppelt
auf und kürzen sich).

\[\folgt \abs{D_{m,n}} \le \sum_{k = N_\ep +1}^\infty \abs{a_k} \nach{\le}{(\ref{eqn:reihen.umordn-konv.a})} \ep
\quad \forall n\ge N_\ep, m\ge M_\ep\]

Da $\sum_{j=0}^\infty a_j$ existiert, folgt mit \ninf{} und Satz \ref{satz:konv.grenzw-ordn}, dass:
\[\exists\lim_{\ninf}\abs{D_{m,n}} = \abs{\sum_{j=0}^m a_{\varphi(j)} - \sum_{j=0}^\infty a_j} \le \ep, \forall m\ge M_\ep\]
Das ist die Behauptung.
\end{proof}

\subsection*{Cauchyprodukte}
Frage: Wie multipliziert man Reihen?

\begin{equation}\label{eqn:reihen.mult-rwert}
\left(\sum_{j=0}^\infty a_j\right)\left(\sum_{k=0}^\infty a_k\right) = \lim_{\ninf}\left(\sum_{j=0}^n a_j\right) 
\cdot \lim_{\ninf}\left(\sum_{k=0}^n a_k\right)
\end{equation}
...

Schema für Summanden $a_jb_k$:
...

Setze $Q_n = \{0, \dotsc, n\}^n$, $D_n = \{(j, k) \in Q_n, k+j \le n\}$.
Summiere $A_nB_n$ "`diagonal"', das heißt bilde zuerst

\begin{equation}\label{eqn:reihen.diag-summe}
c_n = \sum_{l=0}^n a_lb_{n-l}, n \in \set{N}
\end{equation}

$c_n = $ "`Summe über $a_jb_k$ mit $j+k = n$.

\begin{satz}\label{satz:reihen.cauchy-prod}
Seien $\sum_k a_k$, $\sum_k b_k$ absolut konvergente Reihen. Seien $c_n$ (\nat{n}) in (\ref{eqn:reihen.diag-summe}) definiert.
Dann konvergiert $\sum_{n\ge0} c_n$ absolut und es gilt:

\begin{equation}\label{eqn:reihen.cauchy-prod}
\left(\sum_{j=0}^\infty a_j\right)\left(\sum_{k=0}^\infty a_k\right) = \sum_{n=0}^\infty c_n = \sum_{n=0}^\infty \sum_{j=0}^n a_jb_{n-j}
\end{equation}
\end{satz}

\begin{bem*}
Satz ist (im Allgemeinen) falsch für konvergente, nicht absolut konvergente Reihen (siehe Übung).
\end{bem*}

\begin{proof}
Seien $A_n$, $B_n$ aus (\ref{eqn:reihen.mult-rwert}), $A_n^* = \sum_{j=0}^n \abs{a_j}$, $B_n^* = \sum_{k=0}^n \abs{b_k}$,
$C_n = \sum_{j=0}^n c_j$. Nach Vorraussetzung $\exists A^* = \sum_{j=0}^\infty \abs{a_j}, B^* = \sum_{k=0}^\infty \abs{b_k}$.
Dann:

\begin{align*}
\abs{A_nB_n - C_n} = \Bigg\lvert\sum_{(j,k)\in Q_n\backslash D_n} a_jb_k\Bigg\lvert &\le 
	\sum_{(j,k)\in Q_n\backslash Q_{(\frac{n}{2})}} \abs{a_j}\abs{b_k} \\
	&= \sum_{(j,k)\in Q_n} \abs{a_j}\abs{b_k} - \sum_{(j,k)\in Q_{(\frac{n}{2})}} \abs{a_j}\abs{b_k}\\
	&= A_n^*B_n^* - A_{(\frac{n}{2})}^*B_{(\frac{n}{2})}^*
	%& \ra A^*B^*\text{ nach Satz~\ref{satz:konv.greg}} &\ra A^*B^*\quad(\ninf) %TODO
\end{align*}
\[\folgt \exists\lim_{\ninf} \abs{A_nB_n - C_n} = 0\]

Da $A_nB_n \ra AB (\ninf)$, folgt $\exists\sum_{n=0}^\infty C_n - AB \folgt$ (\ref{eqn:reihen.cauchy-prod}).
Ferner: \[\sum_{n=0}^N \abs{c_n} \nach{\le}{(\ref{eqn:reihen.diag-summe})}
\sum_{n=0}^N\sum_{j=0}^n \abs{a_j}\abs{b_{n-j}} \le %s.o.
A_N^*B_N^* \le A^*B^*\] für alle \nat{N}. Nach Satz \ref{satz:reihen.beschr-gwert} folgt die absolute Konvergenz von $\sum c_n$
\end{proof}

\begin{bsp}[Exponentialreihe]\label{bsp:reihen.expon-reihe}
Sei $z$, \kmplx{w}, $\exp(z) \da \sum_{k=0}^\infty \frac{z^k}{k!}$.
Die Reihe konvergiert absolut nach Bsp.~\ref{bsp:reihen.quot-krit} ($\forall \kmplx{z}$).
Beachte: $\exp(0) = 1$, $\exp(1) = \mathrm{e}$ (Bsp.~\ref{bsp:reihen.wurzel-krit})\\

\noindent\emph{Behauptung:}
\begin{enumerate}
\item $\exp(z+w) = \exp(z)\exp(w)$\label{bsp:reihen.expon-reihe.a}
\item $\exp(z) \neq 0, \exp(-z) = \frac{1}{\exp(z)}$\label{bsp:reihen.expon-reihe.b}
\item Sei $p\in\set{Q}\colon\exp(p) = \mathrm{e}^p$\label{bsp:reihen.expon-reihe.c}
\end{enumerate}
\end{bsp}
\begin{proof}
\begin{enumerate}
\item \[\exp(z)\exp(w) \nach{=}{Satz \ref{satz:reihen.cauchy-prod}} \sum_{n=0}^\infty\sum_{j=0}^\infty \frac{z^j}{j!} \frac{w^{n-j}}{(n-j)!} \frac{n!}{n!}
= \sum_{n=0}^\infty \frac{1}{n!} \underbrace{\sum_{j=0}^n \binom{n}{j}z^j w^{n-j}}_{\text{= Bsp.~\ref{bsp:vor.binom}: $(z+w)^n$}} 
= \exp(z+w)\]
\item $1 = \exp(0) = \exp(z-z) \nach{=}{a)} \exp(z)\exp(-z) \folgt$ b)
\item Sei $p = \frac{m}{n}$, $m\in\set{Z}$. \nat{n}. Dann gilt für $m > 0$ 
\[\exp(p)^n = \underbrace{\exp(p)\cdot\dotsm\cdot\exp(p)}_{\text{$n$-mal}} \nach{=}{a)} \exp(\underbrace{np}_m)
= \exp(\underbrace{1+\dotsb+1}_{\text{$m$-mal}}) = \exp(1)^m =\mathrm{e}^m\]
$\folgt \exp(p) = \mathrm{e}^{\frac{m}{n}}$. Fall $m < 0$ mit b).
\end{enumerate}
\end{proof}

\section{Potenzreihe}
\label{sec:reihen.potenzreihe}
\begin{dfn}\label{dfn:reihen.potenzreihe}
Es sei $(a_k)_{k\ge 0}$ gegeben. Für \kmplx{z} heißt $\sum_{k\ge 0} a_kz^k$ \emph{Potenzreihe}.
\end{dfn}

\begin{bem*}
Sei $D$ die Menge der \kmplx{z}, sodass die Potenzreihe konvergiert, dann ist 
$f: D \ra \set{C}$, $f(z) = \sum_{k=0}^\infty a_kz^k$ eine Funktion. Es gilt stets
$0\in D$, $f(0)=a_0$. (Man setzt $0^0 \da 1$)
\end{bem*}

\begin{dfn}\label{dfn:reihen.konvradius}
Der \emph{Konvergenzradius} $\varrho$ von $\sum a_kz^k$ ist gegeben durch:

\[\varrho = 
\begin{cases}
	\frac{1}{\lim\limits_{\kinf}\sqrt[k]{\abs{a_k}}},	& \text{wenn } \left(\sqrt[k]{\abs{a_k}}\right) \text{beschränkt und keine NF},\\
	0,											& \text{wenn } \left(\sqrt[k]{\abs{a_k}}\right) \text{unbeschränkt},\\
	\infty,										& \text{wenn } \left(\sqrt[k]{\abs{a_k}}\right) \text{NF}.
\end{cases}\]
\end{dfn}

\begin{thm}\label{thm:reihen.konvradius}
Sei $\varrho$ der Konvergenzradius von $\sum_{k \ge 0} a_kz^k$. Dann gelten:
\begin{enumerate}
\item $0 < \varrho < \infty$, dann konvergiert $\sum a_kz^k$ absolut für $\abs{z} < \varrho$
und divergiert für $\abs{z} > \varrho$, wobei \kmplx{z}.\label{thm:reihen.konvradius.a}
\item Wenn $\varrho = 0$, dann divergiert $\sum a_kz^k$ für alle $\kmplx{z}\backslash\{0\}$\label{thm:reihen.konvradius.b}
\item Wenn $\varrho = \infty$, dann konvergiert $\sum a_kz^k$ absolut für alle \kmplx{z}\label{thm:reihen.konvradius.c}
\end{enumerate}
Also: $\varrho = \sup \left\{r \ge 0\colon \sum a_kz^k\text{ konvergiert }\forall \kmplx{z}\text{ mit }\abs{z} \le r\right\}$ 
(dabei ist $\sup \set{R}_+ \da \infty$).
\end{thm}
\begin{proof}
Es gilt $\sqrt[k]{\abs{a_kz^k}} = \left(\abs{a_k}\abs{z}^k\right)^{\frac{1}{k}} = \abs{z}\sqrt[k]{\abs{a_k}} \ad b_k$
\begin{enumerate}
\item $\ls\limits_{\kinf} b_k \nach{=}{\ref{satz:konv.limsupinf-reg.e}} \abs{z}\ls\limits_{\kinf}\sqrt[k]{\abs{a_k}}$.
Nach Wurzelkriterium: \[\folgt \begin{cases}
	\abs{z} < \varrho \gdw \ls b_k < 1 \folgt \sum a_kz^k\text{ konvergiert absolut}\\
	\abs{z} > \varrho \gdw \ls b_k > 1 \folgt \sum a_kz^k\text{ divergiert}
\end{cases}\]
% FIXME: erst c, dann b
\setcounter{enumi}{2}
\item $\ls_{\kinf} b_k = \lim_{\kinf} b_k = 0 \folgt \sum a_kz^k$ konvergiert absolut 
$\forall \kmplx{z}$ nach Wurzelkriterium
\setcounter{enumi}{1}
\item Falls $\abs{z} \neq 0$, dann ist $(b_k)$ unbeschränkt $\folgt \left(b_k^k\right)$ ist unbeschränkt
$\folgt (a_kz^k)$ ist keine NF. Nach Kor.~\ref{kor:reihen.konv-nf} $\folgt \sum a_kz^k$ divergiert
\end{enumerate}
\end{proof}

\begin{bsp}\label{bsp:reihen.konvradius}
\begin{enumerate}
\item Polynome $p(z) = a_0 + a_1z + \dotsb + a_nz^n$ (\kmplx{z}), wobei $a_1, \dotsc, a_n$ gegeben.
Setze $a_j = 0$ für $j > n$ $\folgt \varrho = \infty \folgt$ konvergiert $\forall \kmplx{z}$

\item $\exp(z) = \sum_{k=0}^\infty \frac{z^k}{k!}$ konvergiert $\forall \kmplx{z}$ nach Bsp.~\ref{bsp:reihen.quot-krit}.
Nach Thm.~\ref{thm:reihen.konvradius} gilt: 
\begin{equation}\label{eqn:reihen.wurzelkrit-exp}
0 = \lim_{\kinf} \sqrt[k]{\frac{1}{k!}} = \lim_{\kinf} \frac{1}{\sqrt[k]{k!}}
\end{equation}
da $\varrho = \infty$ und $a_k = \frac{1}{k!}$

\item Geometrische Reihe $\sum_{k \ge 0} z^k$. Hier ist $a_k = 1 \folgt \varrho = 1$. Genauer: Bsp.~\ref{bsp:reihen.reihe}
liefert $\exists \sum_{k=0}^\infty z^k = \frac{1}{1-z}$ für $\abs{z} < 1$. Bsp.~\ref{bsp:reihen.konv-nf} $\folgt$ Divergenz
wenn $\abs{z} \ge 1$.

\item Sei $a_k = k!$. Nach (\ref{eqn:reihen.wurzelkrit-exp}) $\forall \nat{n}\,\exists \nat{K_n}\colon \frac{1}{\sqrt[k]{k!}} \le \frac{1}{n}$ 
$(\forall k \ge K_n)$ \folgt $n \le \sqrt[k]{k!}$ $(\forall k \ge K_n)$ \folgt $(\sqrt[k]{k!})_k$ ist unbeschränkt. 
Thm.~\ref{thm:reihen.konvradius} \folgt $\sum_k k! z^k$ konvergiert \emph{nur} für $z=0$, da $\varrho=0$.

\item Betrachte $\sum_{k \ge 1} \frac{1}{k} (2z)^k$, d.\,h. $a_k = \frac{2^k}{k}$. 
Damit $\sqrt[k]{\abs{a_k}} = \frac{2}{\sqrt[k]{k}} \ra 2$ (\kinf, Üb.)
$\folgt \varrho = \frac{1}{2}$. Also absolute Konvergenz für $\abs{z} < \frac{1}{2}$, Divergenz für $\abs{z} > \frac{1}{2}$.
Hier gilt Konvergenz für $z = -\frac{1}{2}$, Divergenz für $z = \frac{1}{2}$ (nach Bsp.~\ref{bsp:reihen.alt-reihe} und 
\ref{bsp:reihen.reihe})
\end{enumerate}
\end{bsp}

\begin{bem*}
Im Fall $\abs{z} = \varrho \in (0, \infty)$ ist keine allgemeine Aussage möglich.
\end{bem*}

\begin{satz}\label{bsp:reihen.potreihe-reg}
Es seien $\sum a_kz^k, \sum b_kz^k$ Potenzreihen mit Konvergenzradius $\varrho_a, \varrho_b > 0$ und $\alpha, \beta \in \set{C}$.
Dann gelten für \kmplx{z} mit $\abs{z} < \min \{\varrho_a, \varrho_b\}$ (wobei $\min \{x, \infty\} = x$ für $\reell{x}$)
\begin{enumerate}
\item $\exists \sum\limits_{k=0}^\infty (\alpha a_k + \beta b_k) z^k = \alpha \sum\limits_{k=0}^\infty a_kz^k + \beta \sum\limits_{k=0}^\infty b_kz^k$\label{bsp:reihen.potreihe-reg.a}

\item $\exists \sum\limits_{n=0}^\infty \biggl(\sum\limits_{j=0}^n a_jb_{n-j}\biggr)z^n = \left(\sum\limits_{k=0}^\infty a_kz^k\right)\left(\sum\limits_{k=0}^\infty b_kz^k\right)$\label{bsp:reihen.potreihe-reg.b}
\end{enumerate}
\end{satz}
\begin{proof}
\begin{enumerate}
\item Thm.~\ref{thm:reihen.konvradius} und Satz \ref{satz:reihen.addit-reihenwerte}
\item Thm.~\ref{thm:reihen.konvradius} und Satz \ref{satz:reihen.cauchy-prod}, wobei in (\ref{eqn:reihen.diag-summe}) gilt:
\[c_n = \sum_{j=0}^n a_jz^jb_{n-j}z^{n-j} = z^n \sum_{j=0}^n a_j b_{n-j}\]
\end{enumerate}
\end{proof}

\begin{bsp}[Sinus und Cosinus]\label{bsp:reihen.sin-cos}
Für \kmplx{z} konvergieren absolut:
\[\sin(z) \da \sum_{k=0}^\infty \frac{(-1)^k}{(2k+1)!} z^{2k+1},\quad
\cos(z) \da \sum_{k=0}^\infty \frac{(-1)^k}{(2k)!} z^{2k}\]

\noindent Das sind Potenzreihen mit Koeffizienten
\[\sin\colon a_n = \begin{cases}
\frac{(-1)^k}{(2k+1)!}, & n= 2k+1 \text{ ungerade}\\
0, & \text{$n$ gerade}
\end{cases},\quad
\cos\colon a_n = \begin{cases}
0, & \text{$n$ ungerade}\\
\frac{(-1)^k}{(2k)!}, & n= 2k \text{ gerade}
\end{cases}\]
\end{bsp}
\begin{proof}
Zeige $\varrho = \infty$. \[\sin: \sqrt[k]{\abs{a_k}} = 
\begin{cases}
0, & n\text{ gerade}\\
\frac{1}{\sqrt[n]{n!}}, & n\text{ ungerade}
\end{cases} \tonach{(\ref{eqn:reihen.wurzelkrit-exp})} 0,\quad\ninf\]
cos genauso.
\end{proof}

\noindent Aus Reihen folgt:
\begin{equation}\label{eqn:reihen.reell-sincos}
\forall \reell{x}\colon \cos x, \sin x \in\set{R}
\end{equation}
\begin{equation}\label{eqn:reihen.sincos-reg}
\forall \kmplx{z}\colon \cos(-z) = \cos z,\quad \sin(-z) = -\sin z
\end{equation}

\begin{satz}\label{satz:reihen.euler}
Sei \kmplx{z}. Dann gelten:
\[\text{Euler: }\exp(iz) = \cos(z) + i\sin(z), \quad (\cos z)^2 + (\sin z)^2 = 1\] % FIXME, Euler nur erste Formel
\begin{equation}\label{eqn:reihen.komplx-sincos}
\cos z = \frac{1}{2}(\exp(iz) + \exp(-iz)),\quad \sin z = \frac{1}{2i}(\exp(iz) - \exp(-iz))
\end{equation}
Für \reell{x} folgt mit (\ref{eqn:reihen.reell-sincos}) $\Re \exp(ix) = \cos x$, $\Im \exp(iz) = \sin x$, 
$\abs{\exp(iz)} = 1$, $\abs{\cos x}$, $\abs{\sin x} \le 1$.
\end{satz}
\begin{proof}
Es gilt: $\exp(iz) = \sum_{n=0}^\infty \frac{(iz)^n}{n!} = \reihe{\frac{(i^2)^kz^{2k}}{(2k)!}} + \reihe{\frac{i(i^2)^kz^{2k+1}}{(2k+1)!}} \wegen{=}{i^2 = -1} \cos z + i\sin z$.
Ferner $1= \exp(iz - iz) \nach{=}{(\ref{bsp:reihen.expon-reihe})} exp(iz) \cdot exp(i(z-z)) \nach{=}{(\ref{eqn:reihen.sincos-reg}), Euler} (\cos z + i\sin z)(\cos z - i\sin z) = (\cos z)^2 + (\sin z)^2$. 
(\ref{eqn:reihen.komplx-sincos}) folgt ähnlich aus Euler, (\ref{eqn:reihen.sincos-reg})
\end{proof}

\begin{kor}\label{kor:reihen.sin-nach-cos}
Seien $z$, $w\in\set{C}$. Dann:
\begin{multline*}
-2\sin\left(\frac{z+w}{2}\right)\cdot\sin\left(\frac{z-w}{2}\right) \nach{=}{\ref{eqn:reihen.komplx-sincos}} \\
\frac{-2}{(2i)^2}\left(\exp\left(\frac{i}{2}(z+w)\right)-\exp\left(-\frac{i}{2}(z+w)\right)\right)\cdot \left(\exp\left(\frac{i}{2}(z+w)\right)\right) - \exp\left(-\frac{i}{2}(z-w)\right)\\
\nach{=}{(\ref{bsp:reihen.expon-reihe})} \frac{1}{2}\left(\exp\left(\frac{i}{2} 2z\right)-\exp\left(\frac{i}{2}2w\right) - \exp\left(\frac{i}{2}(-2w)\right) + \exp\left(\frac{i}{2}(-2z)\right)\right)\\
\nach{=}{(\ref{eqn:reihen.komplx-sincos})} \cos z - \cos w
\end{multline*}
\end{kor}

\chapter{Stetige Funktionen}
\label{cha:fnkt}
Ab jetzt wird (fast) immer in $\set{R}$ gerrechnet, insbesondere $B(x, r) = (x-r, x+r)$, $\bar{B}(x, r) = [x-r, x+r]$.
Stets sei $D \neq \emptyset$.

\section{Grenzwerte stetiger Funktionen}
\label{sec:fnkt.grenzw-stetigk}
\begin{dfn}\label{dfn:fnkt.abschluss}
Sei $D \subseteq \set{R}$. Dann heißt die Menge $\abg{D} \da \{\reell{x}\colon \exists x_n\in D$ (\nat{n})
mit $ x_n \ra x$, $\ninf\}$ der \emph{Abschluss} von $D$. $D$ heißt \emph{abgeschlossen} (abg.) falls $D = \abg{D}$.
\end{dfn}

\begin{bem*}
Es gilt $D \subseteq \abg{D}$ (Betrachte für $x\in D$ die Folge $(x_n)_{n\ge 1} = (x)_{n\ge 1}$)
\end{bem*}

\begin{bsp*}
Sei $D = (0, 1]$, dann ist $\abg{D} = [0, 1]$
\end{bsp*}
\begin{proof}
Es gilt $0\in\abg{D}$, da $\frac{1}{n}\in D$, $\frac{1}{n} \ra 0$ \nat{n} \folgt $[0, 1] \subseteq \abg{D}$. 
Umgekehrt: Sei $x_n \in (0, 1] = D$ mit $x_n \ra x$ für ein \reell{x}. Satz~\ref{satz:konv.grenzw-ordn}: $0\le x\le 1$ \folgt $\abg{D} \subseteq [0,1]$ \folgt Beh.
\end{proof}

\paragraph{Ebenso:}
\begin{enumerate}
\item $\abg{\set{R}\backslash\{0\}} = \set{R}$
\item Abgeschlossene Intervalle im Sinne von Def.~\ref{dfn:zahlen.intervalle} sind abgeschlossen im Sinne 
von Def.~\ref{dfn:fnkt.abschluss}, Bsp: $\abg{[0, 1]} = [0, 1]$.
\end{enumerate}

\begin{dfn}\label{dfn:fnkt.grenzw-fnkt}
Sei $D \subseteq \set{R}$, $x_0 \in \abg{D}$, \reell{y_0}. Eine Funktion $f: D \ra \set{R}$ \emph{konvergiert}
gegen den \emph{Grenzwert} $y_0$, wenn für \emph{jede} Folge $(x_n)_{n \ge 1} \subseteq D$ mit $x_n \ra x_0$ (\ninf) gilt:
$f(x_n) \ra y_0$ (\ninf). Man schreibt dann $y_0 = \lim_{x\ra x_0} f(x)$ oder $f(x) \ra y_0$ für $x \ra x_0$.
Wenn man zusätzlich $x_n < x_0$, bzw. $x_n > x_0$ ($\forall \nat{n}$) fordert, dann spricht man vom \emph{links-}, 
bzw. \emph{rechtsseitigen Grenzwert} und schreibt $y_0 = \lim_{x\ra x_0^-} f(x)$, bzw. $y_0 = \lim_{x\ra x_0^+} f(x)$.
\end{dfn}

\begin{bsp}\label{bsp:fnkt.grenzw-fnkt}
\begin{enumerate}
\item Sei $D = \set{R}$, $f(x) = x^2 +3$, \reell{x_0}. Sei \reell{x_n}, $x_n \ra x_0$. Dann $f(x_n) = x_n^2 + 3 \ra x_0^2 + 3$
(\ninf) nach Satz~\ref{satz:konv.greg} \folgt $\lim_{x\ra x_0} f(x) = x_0^2 + 3$
\item Sei $M \subseteq \set{R}$. Setze \[\charfnkt{M}(x) = 
\begin{cases}
1, & x\in M\\
0, & x\in \set{R}\backslash M
\end{cases} \tag{charakteristische Funktion}\]
\emph{Behauptung.} Sei $D=\set{R}$, $f=\charfnkt{R_+}$. Dann: $\lim_{x\ra 0} f(x)$ existiert nicht.
\begin{proof} Wähle $x_n = (-1)^n\frac{1}{n} \ra 0$, \ninf. Dann \[f(x_n) = \begin{cases}
1, &n\text{ gerade}\\
0, &n\text{ ungerade}\end{cases}\]
Sei $x_n\ra 0$ (\ninf). Wenn $x_n > 0$, dann $f(x_n)=1$. Wenn $x_n < 0$, dann $f(x_n)=0$ \folgt 
$\exists\lim_{x\ra 0^+} f(x) = 1$, $\exists\lim_{x\ra 0^-} f(x) = 0$
\end{proof}
\item Sei $D = \set{R}\backslash\{0\}$, $f(x) = \frac{1}{x}$, $x\in D$. Dann: $\lim_{x\ra 0} f(x)$ existiert nicht, 
da $\frac{1}{n} \ra 0$, aber $f(\frac{1}{n}) = n$ divergiert (\ninf).
\end{enumerate}
\end{bsp}

\begin{satz}[$\ep$-$\delta$-Charakterisierung]\label{satz:fnkt.ep-delta}
Sei $D \subseteq \set{R}$, $x_0\in\abg{D}$, $f: D \ra \set{R}$, \reell{y_0}. Dann sind äquivalent:
\begin{enumerate}
\item $\exists \lim_{x\ra x_0} f(x) = y_0$ \label{satz:fnkt.ep-delta.a}
\item $\forall\ep > 0\,\exists\delta_\ep > 0\,\forall x\in D\cap \bar{B}(x_0, \delta_\ep)$ gilt: $\abs{f(x) - y_0} \le \ep$ \label{satz:fnkt.ep-delta.b}
% FIXME: Bild
\end{enumerate}
\end{satz}
\begin{proof}
\begin{enumerate}
\item Es gelte \ref{satz:fnkt.ep-delta.b}). Sei $x_n\in D$ (\nat{n}) mit $x_n \ra x_0$ beliebig gegeben (\ninf). Sei $\ep > 0$. Wähle $\delta_\ep > 0$
aus \ref{satz:fnkt.ep-delta.b}). Dann $\exists N_\ep \in\set{N}$ mit $\abs{x_n - x_0} \le \delta_\ep$ für alle $n \ge N_\ep$. \ref{satz:fnkt.ep-delta.b}) liefert:
$\abs{f(x_n) - y_0} \le \ep$ ($\forall n \ge N_\ep$) \folgt $f(x_n) \ra y_0$, \ninf{} \folgt \ref{satz:fnkt.ep-delta.a})
\item Es gelte \ref{satz:fnkt.ep-delta.a}). Annahme: \ref{satz:fnkt.ep-delta.b}) sei falsch. Daraus folgt mit $\delta = \frac{1}{n}$:
$\exists \ep_\delta > 0\,\forall \nat{n}\,\exists x_n \in D$ mit $\abs{x_0 - x_n} \le \frac{1}{n}$ und $\abs{f(x) - y_0} > \ep_0$, 
d.\,h. $x_n \ra x_0$ (Satz~\ref{satz:konv.grenzw-ordn}) und $f(x_n) \not\ra y_0$ (\ninf) $\blitz$\ref{satz:fnkt.ep-delta.a}) \folgt \ref{satz:fnkt.ep-delta.b}
\end{enumerate}
\end{proof}

\begin{satz}\label{satz:fnkt.grenzw-reg}
Es seien $D \subseteq \set{R}$, $x_0\in\abg{D}$, $f, g: D\ra\set{R}$, $y_0, \reell{z_0}$, sodass $\exists \lim_{x\ra x_0}f(x) = y_0$,
$\exists \lim_{x\ra x_0}g(x) = z_0$. Dann gelten:
\begin{enumerate}
\item $\exists \lim\limits_{x\ra x_0} (f(x) + g(x)) = y_0 + z_0$\label{satz:fnkt.grenzw-reg.a}
\item $\exists \lim\limits_{x\ra x_0} f(x)g(x) = y_0z_0$\label{satz:fnkt.grenzw-reg.b}
\item $\exists \lim\limits_{x\ra x_0} \abs{f(x)} = \abs{y_0}$\label{satz:fnkt.grenzw-reg.c}
\item Sei zusätzlich $y_0 \neq 0$. Dann $\exists r > 0$, sodass $\abs{f(x)} \ge \frac{\abs{y_0}}{2} > 0$ 
für alle $x\in D$ mit $\abs{x - x_0} \le r$. Ferner $\exists \lim\limits_{x\ra x_0} \frac{1}{f(x)} = \frac{1}{y_0}$\label{satz:fnkt.grenzw-reg.d}
\item Sei zusätzlich $f(x) \le g(x)$ für alle $x\in D$. Dann gilt $x_0 \le z_0$. (Entsprechendes gilt für $\lim\limits_{x\ra {x_0}^\pm}$)\label{satz:fnkt.grenzw-reg.e}
\end{enumerate}
\end{satz}
\begin{proof}
\begin{enumerate}\setcounter{enumi}{2}
\item Sei $x_n \in D$ (\nat{n}) mit $x_n\ra x_0$ (\ninf) beliebig gewählt. \folgtnach{n.\,V.} $f(x_n) \ra y_0$ 
\folgtnach{\ref{satz:konv.grenzw-komplex}} $\abs{f(x_n)} \ra \abs{y_0}$ (\ninf) \folgt Behauptung
\item Wähle $\ep = \frac{\abs{y_0}}{2} > 0$. Nach Teil \ref{satz:fnkt.grenzw-reg.c} und Satz~\ref{satz:fnkt.ep-delta}
$\exists r=\delta_\ep > 0$, sodass für alle $x\in D\cap\abg{B}(x_0, r)$ gilt $\frac{\abs{y_0}}{2} \ge \abs{\abs{f(x)} - \abs{y_0}} \ge \abs{y_0} - \abs{f(x)}$
\gdw $\abs{f(x)} \ge \frac{\abs{y_0}}{2}$. Sei nun $x_n\ra x_0$ (\ninf) mit $x_n \in D \cap \abg{B}(x_0, r)$
\folgtnach{n.\,V.} $f(x_n) \ra y_0$ \folgtnach{\ref{satz:konv.greg}} $\frac{1}{f(x_n)} \ra \frac{1}{y_0}$ (\ninf) \folgt Behauptung
\end{enumerate}

\noindent a), b), e) gehen genauso mit Satz~\ref{satz:konv.greg} und Satz~\ref{satz:konv.grenzw-ordn}.
\end{proof}

\subsection*{Uneigentliche Grenzwerte}
\begin{dfn*}
Erweiterte Zahlengerade $\overline{\set{R}} = \set{R} \cup \{-\infty, +\infty\}$ (man schreibt
oft $\infty$ statt $+\infty$). Ordnung: $-\infty < x < +\infty$ ($\forall \reell{x}$), $\abs{\pm\infty} \da +\infty$
\end{dfn*}

\begin{dfn}\label{dfn:fnkt.uneig-grenzw}
Man schreibt $\lim\limits_{\ninf} x_n = +\infty\; (-\infty)$ für \reell{x_n}, \nat{n}, falls:
\[\forall \nat{K}\,\exists\nat{N_K}\,\forall n\ge N_K\colon x_n \ge K\; (x_n \le -K)\]
Damit $n^2 \ra \infty$, $-n^3 \ra -\infty$ (\ninf). \emph{Beachte:} $\left((-1)^n\right)$ divergiert nach wie vor.
\end{dfn}

\begin{bem}\label{bem:fnkt.uneig-grenzw}
\begin{enumerate}
\item Wenn $x_n\ra\infty$ oder $x_n\ra -\infty$, dann $\frac{1}{x_n}\ra 0$ (\ninf). (Beachte, nach 
Def.~\ref{dfn:fnkt.uneig-grenzw} gilt: $x_n \neq 0$, $n \ge N_1$)\label{bem:fnkt.uneig-grenzw.a}
\item Wenn $x_n \ra 0$ und ein \nat{n_0} existiert mit $x_n > 0$ für alle $n \ge n_0$, dann geht $\frac{1}{x_n}\ra +\infty$\label{bem:fnkt.uneig-grenzw.b}
\item Wenn $x_n \ra 0$, $x_n < 0$ ($\forall n \ge n_0$), dann $\frac{1}{x_n}\ra -\infty$\label{bem:fnkt.uneig-grenzw.c}
\end{enumerate}
\end{bem}
\begin{proof}
\begin{enumerate}
\item Sei $x_n\ra +\infty$ oder $x_n\ra -\infty$ (\ninf). Nach Def.~\ref{dfn:fnkt.uneig-grenzw} gilt
\[\forall \nat{K}\,\exists\nat{N_K}\,\forall n\ge N_K\colon \abs{x_n} \ge K \gdw 0 < \frac{1}{\abs{x_n}} \le \frac{1}{K} \ad \ep,\]
d.\,h. $\frac{1}{x_n} \ra 0$, \ninf.
\end{enumerate}

b), c) zeigt man ähnlich.
\end{proof}

\noindent In Anbetracht von \ref{bem:fnkt.uneig-grenzw}.\ref{bem:fnkt.uneig-grenzw.a}) %FIXME 4.7a) anstatt 4.7.1)
schreibt man 
\begin{equation}\label{eqn:fnkt.x-div-infty}
\frac{x}{\pm\infty} = 0,\;\reell{x}
\end{equation}
(damit gilt $\lim_{\ninf}\frac{1}{x_n} = \frac{1}{\lim_{\ninf} x_n}$ auch in Bem.~\ref{bem:fnkt.uneig-grenzw}\ref{bem:fnkt.uneig-grenzw.a})
Wenn $(x_n)$ nach oben (nach unten) unbeschränkt ist (wobei \reell{x_n}) dann setzt man $\ls_{\ninf} x_n\da\infty$ $\li_{\ninf} x_n \da -\infty$.
Mit identischem Beweis gelten dann Wurzel- und Quotientenkriterium ohne die jeweilige Beschränktheitsvorraussetzung. 
Ferner liefert (\ref{eqn:fnkt.x-div-infty}) und Bem.~\ref{bem:fnkt.uneig-grenzw} in Thm.~\ref{thm:reihen.konvradius}
\[\varrho = \frac{1}{\ls\limits_{\kinf}\sqrt[k]{\abs{a_k}}}\]

\noindent Gilt auch wenn $\sqrt[k]{\abs{a_k}}$ unbeschränkt ($\varrho = \frac{1}{\infty} = 0$) oder wenn $\sqrt[k]{\abs{a_k}} \to 0^+$ (\kinf)
("`$\varrho = \frac{1}{0^+} = +\infty$"'). Weiter schreibt man $\sup D = +\infty$ wenn $D\subseteq\set{R}$ nach oben unbeschränkt
ist, sowie $\inf D= -\infty$, wenn $D$ nach unten unbeschränkt ist.

Sei $f: D\ra\set{R}$, $x_0\in\abg{D}$, $y_0\in\abg{R}$. Dann definiert man $\lim_{n\to x_0} f(x) = y_0$ wie in Def.~\ref{dfn:fnkt.grenzw-fnkt},
d.\,h. für alle $x_n \to x_0$ muss $f(x_n)\to y_0$ in $\abg{\set{R}}$ gelten. Dabei ist $+\infty\in\abg{D}$ wenn $\sup D = \infty$ und
$-\infty \in \abg{D}$, wenn $\inf D = -\infty$.

\begin{bsp*}
Mit Bem.~\ref{bem:fnkt.uneig-grenzw} folgt $\lim_{x\to 0}\frac{1}{x^2} = +\infty$, $\lim_{x\to 0^+}\frac{1}{x} = \infty$,
$\lim_{x\to 0^-}\frac{1}{x} = -\infty$ und $\not\exists\lim_{x\to 0}\frac{1}{x}$.
\end{bsp*}

\section{Eigenschaften stetiger Funktionen}
\label{sec:fnkt.stetige-fnkt}

\begin{dfn}\label{dfn:fnkt.stetigkeit}
Seien $D \subseteq\set{R}$, $f: D\ra\set{R}$, $x_0\in D$. Dann heißt $f$ \emph{stetig in $x_0$}, falls 
$\exists \lim_{x\to x_0} f(x) = f(x_0)$, d.\,h. für \emph{jede} Folge $(x_n) \subseteq D$ mit $x_n \to x_0$
(\ninf) gilt: $f(x_n) \to f(x_0)$ (\ninf). $f$ heißt stetig (auf $D$), wenn $f$ für alle $x_0\in D$ stetig ist.
Man schreibt: $C(D) = \{f: D\ra \set{R}\text{, $f$ stetig}\}$.
\end{dfn}

\begin{bsp}[vgl.~\ref{bsp:fnkt.grenzw-fnkt}]\label{bsp:fnkt.stetigkeit}
\begin{enumerate}
\item Sei $D=\set{R}$ und $c\in\set{R}$ (fest gegeben). Dann sind die Funktionen $f(x)=c$, $g(x)=x$
(\reell{x}) stetig auf $\set{R}$.\label{bsp:fnkt.stetigkeit.a}
\item Sei $D=\set{R}_+$, $x_0$, $x_n\in D$. Übung: Wenn $x_n\to x_0$, dann $\sqrt{x_n}\to\sqrt{x_0}$
(\ninf). Also ist $f(x) = \sqrt{x}$ stetig auf $\set{R}_+$\label{bsp:fnkt.stetigkeit.b}
\item Sei $D=\set{R}$ und $f=\charfnkt{\set{R}_+}$. \folgt $f$ ist stetig für $x_0\in\set{R}\backslash\{0\}$
aber unstetig für $x_0 = 0$, $\not\exists\lim_{x\to 0} f(x)$\label{bsp:fnkt.stetigkeit.c}
\item Sei $D=\set{R}\backslash\{0\}$. Dann ist $f(x) = \frac{1}{x}$, $x\in D$ stetig auf $D$\label{bsp:fnkt.stetigkeit.d}
\item Sei $D=\set{R}$, $f(x) = $ ...% TODO, macht 0^+ Sinn?
\folgt $f$ unstetig in $x_0=0$, da $\not\exists\lim_{x\to 0^+} f(x)$\label{bsp:fnkt.stetigkeit.e}
\end{enumerate}
\end{bsp}

\begin{dfn*}
Seien $f$, $g: D\ra\set{R}$, $\alpha\in\set{R}$. Dann definiere man die Funktion $f+g: D\ra\set{R}$ punktweise durch
$(f+g)(x) \da f(x) + g(x)$ $(x\in D)$. Analog definiere man die Funktionen $\alpha f$, $f\cdot g$, $\abs{f}$ und $\frac{1}{f}$
(soweit $f(x) \ne 0$). Ferner sei $f(D) = \{y\in\set{R}:\exists x\in D: f(x)=y\}$ und $h: f(D)\ra\set{R}$. Dann
definiert man die \emph{Komposition} $h\circ f:D\ra\set{R}$ durch $(h\circ f)(x) = h(f(x))$, $x\in D$.
\end{dfn*}

\begin{satz}\label{satz:fnkt.uebertragung-stetigk}
Seien $D\subseteq\set{R}$, \reell{x_0}, \reell{\alpha}, sowie $f, g: D\ra\set{R}$ stetig in $x_0$ und $h: f(D)\ra\set{R}$ 
stetig in $f(x_0)$. Dann sind die Funktionen $f+g$, $fg$ (speziell $\alpha g$), $\abs{f}$, $h\circ f$ stetig bei $x_0$.
Wenn $f(x_0)\ne0$, dann existiert nach Satz~\ref{satz:fnkt.grenzw-reg} ein $x > 0$ mit $f(x)\ne 0$ für $x\in\abg{B}(x_0, r)\cap D \da \tilde{D}$.
Ferner ist $\frac{1}{f}: \tilde{D}\ra\set{R}$ stetig in $x_0$. (Also: $C(D)$ ist ein Vektorraum).
\end{satz}
\begin{proof}[Beweis (beispielhaft).]
Sei $x_n\in D$ mit $x_n\to x_0$ (\ninf). Dann $f(x_n)\in f(D)$, $f(x_n)\to f(x_0)$ (\ninf) (da $f$ stetig in $x_0$).
Also: $h(f(x_n))\to h(f(x_0))$, da h stetig in $f(x_0)$ (\ninf). Somit ist $h\circ f$ stetig in $x_0$. Der Rest folgt
analog mit Satz~\ref{satz:fnkt.grenzw-reg}.
\end{proof}

\begin{bsp}[Satz~\ref{satz:fnkt.uebertragung-stetigk} liefert:]\label{bsp:fnkt.uebertragung-stetigk}
\begin{enumerate}
\item Polynome sind auf \set{R} stetig, da sie aus $p_0(x)=1$, $p_1(x)=x$ zusammengesetzt sind.\label{bsp:fnkt.uebertragung-stetigk.a}
\item Rationale Funktionen $f=\frac{p}{q}$ sind auf $D=\{x\in\set{R}: q(x)\ne 0\}$ stetig, als Quotient
der Polynome $p$, $q$.\label{bsp:fnkt.uebertragung-stetigk.b}
\item $f(x) = \sqrt{1+3\abs{x}}$ ist stetig auf $D=\set{R}$, da $f=w\cdot g$ mit $w(y)=\sqrt{y}$ und $g(x)=1+3\abs{x}$, $g=1+3\abs{p_1}$.\label{bsp:fnkt.uebertragung-stetigk.c}
\end{enumerate}
\end{bsp}

\begin{thm}\label{thm:fnkt.stetigk-potreihe}
Sei $f(x) = \sum_{n=0}^\infty a_nx^n$ eine Potenzreihe mit Konvergenzradius $\varrho > 0$. Dann ist $f: B(0, \varrho) = (-\varrho, \varrho)\ra\set{R}$
stetig, d.\,h. \[\lim_{x\to x_0} \sum_{n=0}^\infty a_nx^n = \sum_{n=0}^\infty a_nx_0^n\quad (x_0\in B(0, \varrho))\]
\end{thm}
\begin{bsp*}
Stetig auf \set{R} sind $\sin$, $\cos$, $\exp$ sowie $f(x) = \exp(1+2x^2)$ (\reell{x}),
da $f=\exp p$, $p(x) = 1+2x^2$ (Hier sei vorrübergehend $B(0, \infty) = \set{R}$).
\end{bsp*}
\begin{proof}[Beweis des Theorems.]
Sei $x_0$, $x_n \in (-\varrho, \varrho)$ mit $x_n\to x_0$ (\ninf). Setze $d\da \varrho-\abs{x_0} > 0$
\folgt $\exists x_0\in\set{N}: \abs{x_n-x_0} \le \frac{d}{2}$ ($\forall n\ge n_0$)
\begin{equation}\label{...}\tag{$*$}
\folgt \abs{x_n} \le \abs{x_n - x_0} + \abs{x_0} \le ... + \abs{x_0} = \varrho - \frac{d}{2} < \varrho\quad (n\ge n_0)
\end{equation}

Setze $r=p - ...$. Dann (nach Thm.~\ref{thm:reihen.konvradius}) $\exists...$. Sei $\ep >0$ beliebig, fest gegeben.
Dann $\exists J_\ep\in\set{N}$, sodass 
\begin{equation}\label{...}\tag{$**$}
\sum_{j=J_\ep + 1}^\infty \abs{a_j}r^j \le \ep
\end{equation}
\noindent Setze $p_\ep(x) = ...$
\end{proof}

\begin{satz}\label{satz:fnkt.stetigk-aequiv}
Seien $D\subseteq\set{R}$, $f:D\ra\set{R}$, $x_0\in D$. Dann sind äquivalent:
\begin{enumerate}
\item $f$ ist stetig in $x_0$\label{satz:fnkt.stetigk-aequiv.a}
\item $\forall\ep>0\,\exists\delta_\ep > 0\,\forall x\in D\cap\abg{B}(x_0, \delta_\ep): \abs{f(x) - f(x_0)} \le \ep$\label{satz:fnkt.stetigk-aequiv.b}
\item $\forall\ep>0\,\exists\delta_\ep > 0: f(D\cap\abg{B}(x_0, \delta_\ep)): ...$\label{satz:fnkt.stetigk-aequiv.c}
\end{enumerate}
\end{satz}
\begin{proof}
...
\end{proof}

\begin{dfn}\label{dfn:fnkt.glm-stetig}
Sei $D\subseteq\set{R}$ und $f: D\ra\set{R}$. $f$ heißt \emph{gleichmäßig stetig} (glm stetig), wenn
\begin{equation}\label{eqn:fnkt.glm-stetig}
\forall\ep>0\,\exists\delta_\ep > 0\,\forall x, y\in D\text{ mit }\abs{x-y} \le \delta_\ep\text{ gilt} \abs{f(x)-f(y)} \le \ep
\end{equation}
(Im Gegensatz zu \ref{satz:fnkt.stetigk-aequiv}\ref{satz:fnkt.stetigk-aequiv.b} hängt $\delta_\ep$ nicht von $x_0$ ab).
\end{dfn}

\begin{bsp}\label{bsp:fnkt.glm-stetig}
\begin{enumerate}
\item Sei $D = (0, 1]$, $f(x)=\frac{1}{x}$. Sei $\ep_0=1$, sei $\delta > 0$ beliebig. Wähle $x\in(0, 1]$ mit
$x\le 2\delta$, $y=\frac{x}{2}$ \folgt $\abs{x-y} = \frac{x}{2} \le \delta$. ...\label{bsp:fnkt.glm-stetig.a}
\item Sei $D=\set{R}$, $f(x)=x^2$. Sei $\ep_0 = 1$, sei $\delta > 0$ beliebig. Wähle $x=\delta + \frac{1}{\delta}$, $y=\frac{1}{\delta}$
\folgt $\abs{x-y}=\delta$, aber $\abs{f(x)-f(y)} ... > 1 = \ep_0$
\folgt $f$ nicht glm stetig, obwohl $f$ stetig.\label{bsp:fnkt.glm-stetig.b}
\end{enumerate}
\end{bsp}

\section{Hauptsätze über stetige Funktionen}
\label{sec:fnkt.hauptsaetze}

\begin{thm}\label{thm:fnkt.abg-beschr-glm-stetig}
Sei $D\subseteq\set{R}$ abgeschlossen und beschränkt, $f: D\ra\set{R}$ sei stetig. Dann: $f$ ist glm. stetig. (Beispiel: $D=[a, b]$)
\end{thm}
\begin{proof}
Annahme: $f$ sei nicht glm. stetig. (\ref{eqn:fnkt.glm-stetig}) (mit $\delta=\frac{1}{n}$) liefert: 
\begin{equation}\label{eqn:fnkt.abg-beschr-glm-stetig.1}\tag{$*$}
\exists \ep_0 >0\,\forall\nat{n}\,\exists x_n,\,y_n\in D: \abs{x_n - y_n} \le \frac{1}{n},\;\abs{f(x_n)-f(y_n)} > \ep_0
\end{equation}
$D$ beschränkt, Thm.~\ref{thm:konv.bw} (=BW) \folgt $\exists \text{ TF }x_{n_k} \to x$ (\kinf), $y_{n_{k_l}}\to y$ ($l\to\infty$) \folgt
$x, y \in \abg{D} \nach{=}{n.V.} D$. Ferner: 
\[\abs{x-y} \le \abs{x-y_{n_{k_l}}} + \underbrace{\abs{x_{n_{k_l}}-y_{n_{k_l}}}}_{\nach{\le}{(\ref{eqn:fnkt.abg-beschr-glm-stetig.1})} \frac{1}{n_{k_l}}} + \abs{y_{n_{k_l}} - y} \to 0\quad (l\to\infty)\]
\folgtnach{\ref{satz:zahlen.arch-ord}.\ref{satz:zahlen.arch-ord.c}} $x=y$. 
$f$ stetig: $f(x_{n_{k_l}}) - f(y_{n_{k_l}}) \to f(x)-f(y) = f(x) - f(x) = 0$ ($l\to\infty$) $\blitz$ (\ref{eqn:fnkt.abg-beschr-glm-stetig.1})
\end{proof}

\begin{dfn}\label{dfn:fnkt.stetige-forts}
Sei $D\subseteq\set{R}$ nicht abgeschlossen. $x_0,\,y_0\in\set{R}$, $x_0\in\abg{D}\backslash D$. Die Funktion
$\tilde{f}(x) = \begin{cases}
f(x), &x\in D\\
y_0, & x = x_0
\end{cases}$ (definiert auf $\tilde{D} = D\cup\{x_0\}$) heißt \emph{stetige Fortsetzung} von $f$ in $x_0$, wenn $\lim_{x\to x_0} f(x) = y_0$.
\end{dfn}

\begin{bsp}\label{bsp:fnkt.stetige-forts}
\begin{enumerate}
\item Sei $D=\set{R}\backslash\{1\}$, $f(x)=\frac{x^2-1}{x-1}$, $x\in D$, $x_0=1$, $y_0=2$\\
\folgt $\tilde{f}(x)=\begin{cases}
\frac{x^2-1}{x-1} = x+1, &x\ne 1\\
2, & x=1
\end{cases}$, also $\tilde{f}(x) = x+1$, $x\in\tilde{D}=\set{R}.$\\
Da $\tilde{f}$ stetig auf $\set{R}$, ist $f$ in 1 stetig fortsetzbar. (Wenn man $y_0=3$ setzen würde, wäre $\tilde{f}$ keine
stetige Fortsetzung).\label{bsp:fnkt.stetige-forts.a}
\item Sei $D=\set{R}\backslash\{0\}$. Nicht stetig fortsetzbar sind $f(x)=\frac{1}{x}$, $f(x)=\charfnkt{\set{R}_+}(x)$, 
da jeweils $\lim_{x\to 0}f(x)$ nicht existiert. (siehe Bsp.~\ref{bsp:fnkt.grenzw-fnkt})\label{bsp:fnkt.stetige-forts.b}
\end{enumerate}
\end{bsp}

\begin{satz}Sei $f:D\ra \set{R}$ stetig, $x_0\in\abg{D}\backslash D$, \reell{x_0}. Dann:\label{satz:fnkt.glm-stetig-forts}
\begin{enumerate}
\item Wenn $f$ auf $D$ gleichmäßig stetig ist, dann hat $f$ in $x_0$ eine stetige Fortsetzung. \label{satz:fnkt.glm-stetig-forts.a}
\item Wenn $\tilde{D}=D\cup\{x_0\}$ abgeschlossen und beschränkt ist und $f$ in $x_0$ stetig fortsetzbar ist, 
dann ist $f$ mit $D$ gleichmäßig stetig.\label{satz:fnkt.glm-stetig-forts.b}
\end{enumerate}
\end{satz}
\begin{proof}
\begin{enumerate}\setcounter{enumi}{1}
\item Thm.~\ref{thm:fnkt.abg-beschr-glm-stetig}: $\tilde{f}$ ist gleichmäßig stetig auf $\tilde{D}$. \folgt 
$f$ gleichmäßig stetig auf $D$.\setcounter{enumi}{0}
\item Sei $f$ gleichmäßig stetig.
\begin{enumerate}
\item Sei $x_n\in D$ mit $x_n\to x_0$. Sei $\ep>0$ gegeben. Sei $\delta_\ep$ aus (\ref{eqn:fnkt.glm-stetig}). Dann:
$\exists N_\ep: \abs{x_n - x_0} \le \frac{\delta_\ep}{2}\quad (\forall n \ge N_\ep)$
\folgt $\abs{x_n - x_m} \le \abs{x_n - x_0} + \abs{x_0 - x_m} \le \delta_\ep\quad (\forall n,\,m \ge N_\ep)$ 
\folgtnach{(\ref{eqn:fnkt.glm-stetig})} $\abs{f(_n) - f(x_m)} \le \ep$ $(\forall n,\,m \ge N_\ep)$. 
Thm.~\ref{thm:konv.cauchyfolge} \folgt $\exists \lim_{\ninf} f(x_n) \ad y_0$
\item Sei $\tilde{x_n}$ in $D$ eine weitere Folge mit $\tilde{x_n} \to x_0$. Dann $\exists \tilde{N_\ep} \ge N_\ep$
mit $\abs{\tilde{x_n} - x_0} \le \frac{\delta_\ep}{2}$ ($\forall n \ge \tilde{N_\ep}$)
\folgt $\abs{x_n - \tilde{x_n}} \le \abs{x_n - x_0} + \abs{x_0 - \tilde{x_n}} \le \delta_\ep$ ($\forall n \ge \tilde{N_\ep}$)
\folgtnach{(\ref{eqn:fnkt.glm-stetig})} $\abs{f(x_n) - f(\tilde{x_n})} \le \ep$  ($\forall n \ge \tilde{N_\ep}$)
\folgt $\abs{f(\tilde{x_n}) - y_0} \le \abs{f(\tilde{x_n}) - f(x_n)} + \abs{f(x_n) - y_0} \nach{\le}{1)}
\ep + \lim_{m\to\infty}\abs{f(x_n)-f(x_m)} \nach{\le}{1)} 2\ep$ ($\forall n \ge \tilde{N_\ep}$)
\folgt $f(\tilde{x_n}) \to y_0$ \folgt $\exists\lim_{x\to x_0} f(x) = y_0$.
\end{enumerate}
\end{enumerate}
\end{proof}

\begin{thm}[Satz vom Maximum]\label{thm:fnkt.max}
Sei $D\subseteq\set{R}$ abgeschlossen und beschränkt und $f: D\ra\set{R}$ stetig. Dann $\exists x_{\pm}\in D$ mit
$f(x_+) = \max_{x\in D} f(x)$, $f(x_-) = \min_{x\in D} f(x)$. Insbesondere ist $f$ beschränkt, d.\,h. 
$\abs{f(x)} \le M$ $(\da \max\{f(x_+), f(x_-)\})$, $\forall x\in D.$
\end{thm}
\begin{proof}
\begin{enumerate}
\item ... % TODO
\item ...
\end{enumerate}
\end{proof}

\begin{kor}\label{kor:fnkt.max-kor}
Sei $D\subseteq\set{R}$ abgeschlossen und beschränkt, $f: D\ra\set{R}$ stetig, $f(x) > 0 \forall x\in D$.
Dann: $f(x) \ge f(x_-) > 0$ $(\forall x\in D)$, (wobei $x_-\in D$ aus Thm.~\ref{thm:fnkt.max}).
\end{kor}

\begin{bsp*}
Wenn $D$ nicht abgeschlossen oder unbeschränkt, dann sind Thm. und Kor. im Allgemeinen falsch.
\begin{enumerate}
\item $D=(0, 1]$, $f(x) = \frac{1}{x}$. $D=\set{R}_+$, $g(x) = x$. \folgt $f$, $g$ stetig und unbeschränkt.
\item $D = [1, \infty), f(x) = \frac{1}{x} > 0$ $\forall x\ge 1$. Aber $\inf\limits_{x\in D} f(x) = 0$.
\end{enumerate}
\end{bsp*}

\paragraph{Frage.} Wie sieht Bild von $f$ aus? $f(D)$ kann Lücken haben, wenn:
% TODO, bilder

\begin{thm}[Zwischenwertsatz/ZWS]\label{thm:fnkt.zws}
Sei $f: [a, b] \ra \set{R}$ stetig. Dann: $f([a, b]) = \left[\min_{[a, b]} f, \max_{[a, b]} f\right]$.
Also: $\forall y_0 \in [\min f, \max f]\,\exists x_0\in [a, b]$ mit $f(x_0) = y_0$.
\end{thm}
\begin{proof}
...
\end{proof}

\begin{kor}[Nullstellensatz]\label{kor:fnkt.nst}
Sei $f:[a,b]\ra\set{R}$ stetig und $f(a)\cdot f(b) \le 0$. Dann $\exists x_*\in [a, b]: f(x_*) = 0$. 
\end{kor}
\begin{proof}
Nach Vorraussetzung $f(x) \le 0 \le f(b)$, $f(b) \le 0 \le f(a)$ \folgt $0\in f([a, b])$. ZWS \folgt Beh.
\end{proof}

\begin{kor}\label{kor:fnkt.intervallsatz}
Sei $I$ ein Intervall und $f: I\ra\set{R}$ stetig. Dann ist $f(I)$ ein Intervall (Intervallsatz).
\end{kor}
\begin{proof}
Annahme: $f(I)$ sei kein Intervall \folgt $\exists a, b\in I$ mit $y\da f(a) < f(b) \ad z$ und $u\in (y, z)$
mit $u\not\in f(I)$. Sei etwa $a < b$. ZWS \folgt $f([a, b])$ Intervall, $y,\,z\in f([a,b])$ \folgt $u\in f([a,b])$ \folgt $\blitz$
\end{proof}

\begin{bsp}\label{bsp:fnkt.nst-intsatz-bsp}
\begin{enumerate}
\item $D=\set{R}_+$, $f(x)= x^k$ (\nat{k} fest). Dann $f$ stetig, $f(0)=0$, $f(x) \ge 0\;(\forall x\ge 0)$, 
$f(n)\to\infty$ (\ninf). Kor.~\ref{kor:fnkt.intervallsatz}: $f(\set{R}_+) = $ Intervall \folgt $f(\set{R}_+) = \set{R}_+$\label{bsp:fnkt.nst-intsatz-bsp.a}
\item Suche $x_0 \ge 0$: $\exp(-x_0) = x_0$ \gdw $f(x_0) = \exp(-x_0) - x_0 = 0$. Hier $f$ stetig: 
$f(0)=1$, $f(1)=\frac{1}{e} -1 <0$. \ref{kor:fnkt.nst} \folgt $\exists x_0: f(x_0) = 0$.
\label{bsp:fnkt.nst-intsatz-bsp.b}
\end{enumerate}
\end{bsp}

% ----------------- Weihnachtsferien ------
\begin{dfn}\label{dfn:fnkt.} %Def. 4.26

\end{dfn}

\begin{bsp*}
\begin{enumerate}
\item ...
\item ...
\end{enumerate}
\end{bsp*}

\begin{bem}\label{}

\end{bem}
\begin{proof}

\end{proof}

\begin{thm}\label{}

\end{thm}
\begin{proof}

\end{proof}

\begin{bsp}\label{}

\end{bsp}

\section{Exponentialfunktion und ihre Verwandtschaft}
\label{}
...

\begin{dfn}\label{}

\end{dfn}

\begin{dfn}\label{}

\end{dfn}

\begin{bem}\label{}
...
\begin{enumerate}
\item ...
\item ...
\item ...
\item ...
\item ...
\item ...
\end{enumerate}
\end{bem}

\subsection*{Trigonometrische Funktionen}
...

\begin{satz}\label{}

\end{satz}

\begin{dfn*}

\end{dfn*}

...

\begin{dfn}\label{}

\end{dfn}

\begin{dfn}\label{}

\end{dfn}

\begin{bsp}\label{}

\end{bsp}

\chapter{Differentialrechnung}
\label{cha:diff}
Stets sei $I$ ein Intervall das stets mehr als einen Punkt enthält.
\section{Rechenregeln}
\label{sec:diff.rechenregeln}
\emph{Ziel}: Finde beste lineare Approximation für $f$ nahe bei $x_0$.
\emph{Idee}: Betrachte Tangente bei $(x_0, f(x_0))$\\
$t(x) = f(x_0) + m(x-x_0)$,
wobei $m= $ Tangentensteigung in $x_0 =$ Grenzwert der Steigung der Sekante in $x_0, x_1$ also 
$s(x) = f(x_0) + \underbrace{\frac{f(x_1)-f(x_0)}{x-x_0}}_{m(x_1)} (x-x_0)$

\begin{dfn}\label{dfn:diff.rechenregeln.diffbar} $f: I \rightarrow \set{R}$ ist in $x_0 \in \set{R}$ differenzierbar(diff'bar), falls
$\exists\,\lim_{x\to x_0} \frac{f(x) - f(x_0)}{x-x_0} =: f'(x_0) = \frac{\delta f}{\delta x}(x_0)$ %TODO label 5.1
$f'(x_0)$ heißt Ableitung von $f$ an $x$. $f$ heißt diff'bar (auf $I$), wenn $f$ in jedem $x_0 \in I$ diff'bar ist. Damm definiert man
iterativ $f'' = (f')', f^(n) = (f^(n-1))'$ ($n \in \set{N}$) die n-te Ableitung.
Entsprechend def. man die rechts/linksseite Ableitung:

\begin{equation}
\frac{\mathrm{d} \pm f}{\mathrm{d} x}(x_0) = {\lim_{x\to x_0}}_\pm \frac{f(x) - f(x_0)}{x-x_0}
\end{equation}
\end{dfn}

\begin{proof}
\begin{enumerate}
\item Die Fkt. $g(x) = \frac{f(x) - f(x_0)}{x-x_0}$ ist für $I\\\{x_0\}$ definiert
\item Wenn $I=[a,b]$ und $x_0=a,b$, dann stimmen %TODO add refs
überein soweit existent.
\item Sei $f$ ind $x$ diff'bar. Sei $g(x) = f(x_0) + a(x-x_0)$ mit $a \ne f'(x_0)$ eine weitere Gerade durch
$(x_0, f(x_0))$. \emph{Beh.} $\exists \delta > 0: \abs{f(x) - g(x)} > \abs{f(x)}-t(x)|$ für alle 
$x \in I\\\{x_0\} $, $\abs{x-x_0} < \delta$
\begin{proof}
$\abs{\frac{f(x) - g(x)}{x-x_0}} = \abs{\frac{f(x)-f(x_0}{x-x_0} - q} \to \abs{f'(x_0)-a} \ne 0, x\to x_0$
genauso: $\abs{\frac{f(x)-t(x)}{x-x_0}} \to 0$,$x\to x_0$
$\folgt \exists\delta>0:\forall x\in I\\\{x_0\}$ mit $\abs{x-x_0} < \delta:\abs{\frac{f(x)-g(x)}{x-x_0}}\ge
\frac{1}{2}\abs{f'(x_0) -a} > \frac{1}{4}\abs{f'(x_0)-a}\ge\abs{\frac{f(x)-t(x)}{x-x_0}} \folgt$ Beh.
\end{proof}
\item Andere Interpretation:\\
Sei $u(t) \in \set{R}$ eine Größe zur Zeit $t\in\set{R}$ (z.B. Stoffmenge, Ort) und $h>0$.
Dann ist $\frac{1}{h}u(t+h) - u(t)$) der mittlere Zuwachs der Größe im Zeitintervall $[t, t+h]$. Somit ist
$n'(t) = \lim_{h\to 0}\frac{1}{h}(u(t+h)-u(t))$ die momentane Änderungsgeschwindigkeit der Größe. $u''(t)$ ist die Beschleunigung.
\end{enumerate}
\end{proof}

\begin{bsp}\label{}
\begin{enumerate}
\item Seien $a, m \in \set{R}$ fest gegeben. Setzte $f(x) = mx +a$, $x\in\set{R}$. Sei $x_0 \in \set{R}$.
Dann $\frac{f(x)-f(x_0}{x-x_0} = m(\forall{x}\ne x_0). \folgt \exists f'(x_0)=m$
\item $f(x) = \abs{x}$ für $x\in \set{R}$. Dann $\exists f'(x)=\begin{cases}
1 &, x > 0 \\ -1 &, x < 0
\end{cases} $ Ferner $\exists \frac{\mathrm{d}^+f}{\mathrm{d}x}(0)= $

\item ...
\end{enumerate}
\end{bsp}

\begin{satz}\label{}

\end{satz}
\begin{proof}

\end{proof}

\begin{satz}\label{}
...
\begin{enumerate}
\item ...
\item ...
\item ...
\end{enumerate}
\end{satz}
\begin{proof}
\begin{enumerate}
\item ...
\item ...
\item ...
\end{enumerate}
\end{proof}

\begin{kor}\label{}

\end{kor}

\begin{satz}\label{}

\end{satz}
\begin{proof}

\end{proof}

\begin{satz}\label{}

\end{satz}
\begin{bem*}

\end{bem*}
\begin{proof}

\end{proof}

\begin{bsp}\label{}
\begin{enumerate}
\item ...
\item ...
\end{enumerate}
\end{bsp}

\begin{thm}\label{}

\end{thm}
\begin{proof}
\begin{enumerate}
\item ...
\item ...
\end{enumerate}
\end{proof}

\begin{bsp}\label{}
\begin{enumerate}
\item ...
\item ...
\item ...
\item ...
\item ...
\end{enumerate}
\end{bsp}

\begin{bsp}\label{}

\end{bsp}

\begin{dfn}\label{}

\end{dfn}
\begin{bem*}

\end{bem*}

\section{Qualitative Eigenschaften differenzierbarer Funktionen}
\label{}

\begin{dfn}\label{}

\end{dfn}

\begin{satz}\label{}
\begin{enumerate}
\item ...
\item ...
\item ...
\end{enumerate}
\end{satz}
\begin{proof}

\end{proof}
\begin{bem*}

\end{bem*}
\begin{bsp*}

\end{bsp*}
\begin{proof}

\end{proof}

\begin{thm}\label{}

\end{thm}
\begin{proof}

\end{proof}

\begin{satz}\label{}

\end{satz}
\begin{proof}

\end{proof}

\begin{dfn}\label{}

\end{dfn}

\begin{bem}\label{}
\begin{enumerate}
\item ...
\item ...
\item ...
\end{enumerate}
\end{bem}

\begin{kor}\label{}

\end{kor}
\begin{proof}

\end{proof}

\begin{satz}\label{}
\begin{enumerate}
\item ...
\item ...
\end{enumerate}
\end{satz}
\begin{bem*}

\end{bem*}
\begin{proof}

\end{proof}

\begin{bsp}\label{}

\end{bsp}
\begin{proof}

\end{proof}

\begin{kor}\label{}
\begin{enumerate}
\item ...
\item ...
\end{enumerate}
\end{kor}
\begin{bem*}

\end{bem*}
\begin{proof}
\begin{enumerate}
\item ...
\item ...
\end{enumerate}
\end{proof}

\begin{dfn}\label{}

\end{dfn}
\begin{bem*}

\end{bem*}

\begin{satz}\label{}

\end{satz}
\begin{bsp}\label{}
\begin{enumerate}
\item ...
\end{enumerate}
\end{bsp}
\begin{proof}

\end{proof}

\begin{bsp}\label{}
\begin{enumerate}
\item ...
\end{enumerate}
\end{bsp}
\begin{proof}

\end{proof}

\begin{thm}\label{}
\begin{enumerate}
\item ...
\item ...
\end{enumerate}
\end{thm}
\begin{proof}

\end{proof}

\begin{bsp}\label{}
\begin{enumerate}
\item ...
\item ...
\item ...
\item ...
\end{enumerate}
\end{bsp}

\section{Der Satz von Taylor}
\label{}

\begin{thm}\label{}

\end{thm}
\begin{proof}

\end{proof}

\begin{dfn}\label{}

\end{dfn}

\begin{bem}\label{}
\begin{enumerate}
\item ...
\item ...
\end{enumerate}
\end{bem}

\begin{thm}\label{}
\begin{enumerate}
\item ...
\item ...
\item ...
\end{enumerate}
\end{thm}
\begin{bsp*}

\end{bsp*}
\begin{proof}

\end{proof}

\begin{dfn}\label{}

\end{dfn}

\begin{bsp}\label{}
\begin{enumerate}
\item ...
\item ...
\item ...
\end{enumerate}
\end{bsp}

\subsection*{Newton-Verfahren}

\begin{thm}\label{}

\end{thm}
\begin{proof}

\end{proof}

\begin{bsp}\label{}

\end{bsp}

\chapter{Integralrechnung}
\label{cha:int}

\section{Riemann-Integral}
\label{}

\begin{dfn}\label{}

\end{dfn}

\begin{lem}\label{}
\begin{enumerate}
\item ...
\item ...
\end{enumerate}
\end{lem}
\begin{proof}

\end{proof}

\begin{bsp}\label{}

\end{bsp}

\begin{bem}\label{}

\end{bem}

\begin{satz}\label{}

\end{satz}
\begin{proof}

\end{proof}

\begin{dfn*}

\end{dfn*}

\begin{satz}\label{}
\begin{enumerate}
\item ...
\item ...
\item ...
\item ...
\end{enumerate}
\end{satz}
\begin{proof}
\begin{enumerate}
\item ...
\item ...
\item ...
\item ...
\end{enumerate}
\end{proof}

\section{Hauptsatz der Differential- und Integralrechnung}
\label{}

\begin{dfn}\label{}

\end{dfn}

\begin{lem}\label{}

\end{lem}
\begin{proof}

\end{proof}

\begin{thm}\label{}
\begin{enumerate}
\item ...
\item ...
\end{enumerate}
\end{thm}
\begin{proof}
\begin{enumerate}
\item ...
\item ...
\end{enumerate}
\end{proof}

\begin{bem*}

\end{bem*}

\begin{bsp}\label{}
\begin{enumerate}
\item ...
\item ...
\end{enumerate}
\end{bsp}

\begin{bsp*}
\begin{enumerate}
\item ...
\item ...
\item ...
\item ...
\item ...
\end{enumerate}
\end{bsp*}

\begin{satz}\label{}

\end{satz}
\begin{proof}

\end{proof}

\begin{bsp}\label{}
\begin{enumerate}
\item ...
\item ...
\item ...
\item ...
\end{enumerate}
\end{bsp}

\begin{satz}\label{}

\end{satz}
\begin{proof}

\end{proof}

\begin{bsp}\label{}
\begin{enumerate}
\item ...
\item ...
\item ...
\item ...
\end{enumerate}
\end{bsp}

\subsection*{6.15 Integration rationaler Funktionen}
\label{}

\begin{enumerate}
\item ...
\item ...
\item ...
\item \begin{enumerate}
\item ...
\item ...
\item ...
\end{enumerate}
\end{enumerate}

\begin{bsp}\label{}
\begin{enumerate}
\item ...
\item ...
\end{enumerate}
\end{bsp}

\section{Skalare Differentialgleichungen erster Ordnung}
\label{}

\begin{bsp*}

\end{bsp*}

\begin{satz}\label{}

\end{satz}
\begin{proof}

\end{proof}

\begin{bsp}\label{}
\begin{enumerate}
\item ...
\item ...
\item ...
\end{enumerate}
\end{bsp}

\section{Uneigentliche Riemann-Integrale}
\label{}

\begin{dfn}\label{}
\begin{enumerate}
\item ...
\item ...
\end{enumerate}
\end{dfn}

\begin{bem}\label{}
\begin{enumerate}
\item ...
\item ...
\end{enumerate}
\end{bem}

\begin{bsp}\label{}
\begin{enumerate}
\item ...
\item ...
\item ...
\item ...
\end{enumerate}
\end{bsp}

\begin{satz}\label{}
\begin{enumerate}
\item ...
\item ...
\end{enumerate}
\end{satz}
\begin{bem*}

\end{bem*}
\begin{proof}
\begin{enumerate}
\item ...
\item ...
\end{enumerate}
\end{proof}

\begin{bsp}\label{}
\begin{enumerate}
\item ...
\item ...
\item ...
\end{enumerate}
\end{bsp}

\begin{bsp}\label{}

\end{bsp}

\begin{bsp}\label{}

\end{bsp}

\paragraph{Trapezregel.} ...


\end{document}
