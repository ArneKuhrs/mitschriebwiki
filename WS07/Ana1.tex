\documentclass[a4paper,twoside,DIV15,BCOR12mm]{scrbook}
\usepackage{mathe}
\usepackage{saetze-schnaubelt}

\pdfinfo{
	/Author (Die Mitarbeiter von http://mitschriebwiki.nomeata.de/)
	/Title   (Analysis I)
	/Subject (Analysis I)
	/Keywords (Analysis)
}

\author{Die Mitarbeiter von \url{http://mitschriebwiki.nomeata.de/}}
\title{Analysis I}
\makeindex

\begin{document}
\maketitle

\renewcommand{\thechapter}{\Roman{chapter}}
%\chapter{Inhaltsverzeichnis}
\addcontentsline{toc}{chapter}{Inhaltsverzeichnis}
\tableofcontents

\chapter{Vorwort}

\section{Über dieses Skriptum}
Dies ist ein erweiterter Mitschrieb der Vorlesung \glqq Analysis I\grqq\ von Herrn Schnaubelt
Wintersemester 07/08 an der Universität Karlsruhe (TH).
Die Mitschriebe der Vorlesung werden mit ausdrücklicher Genehmigung von Herrn Schnaubelt hier veröffentlicht,
Herr Schnaubelt ist für den Inhalt nicht verantwortlich.

\section{Wer}
Gestartet wurde das Projekt von \ldots, Beteiligt am Mitschrieb sind außer \ldots
noch \ldots

\section{Wo}
Alle Kapitel inklusive \LaTeX-Quellen können unter \url{http://mitschriebwiki.nomeata.de} abgerufen werden.
Dort ist ein \emph{Wiki} eingerichtet und von Joachim Breitner um die \LaTeX-Funktionen erweitert.
Das heißt, jeder kann Fehler nachbessern und sich an der Entwicklung
beteiligen. Auf Wunsch ist auch ein Zugang über \emph{Subversion} möglich.

\chapter{Reelle Zahlen}

Die \begriff{Reellen Zahlen} sind eine Erfindung des menschlichen Geistes, sie haben von Natur aus keine Eigenschaften. Wie Schachfiguren haben sie nur eine Bedeutung im Rahmen der Regeln. Diese Regeln heißen hier Axiome, das sind Forderungen, die wir an etwas stellen, und aus denen wir dann weitere Erkenntnisse erlangen.

Die Grundmenge der Analysis ist $\MdR$, die Menge der reellen Zahlen: Diese Menge führen wir axiomatisch ein, durch die folgenden 15 Axiome.

In $\MdR$ sind zwei Verknüpfungen \glqq +\grqq und \glqq $\cdot$\grqq gegeben, die jedem Paar $a,b \in \MdR$ genau ein $ a+b \in \MdR$ und genau ein $ ab := a \cdot b \in \MdR$ zuordnen.

\ldots

\appendix
\chapter{Satz um Satz (hüpft der Has)}
\listtheorems{satz,wichtigedefinition}

\renewcommand{\indexname}{Stichwortverzeichnis}
\addcontentsline{toc}{chapter}{Stichwortverzeichnis}
\printindex

\input{Ana1Credits}

\end{document}
