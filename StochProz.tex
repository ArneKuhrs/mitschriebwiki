\documentclass[a4paper,twoside,DIV15,BCOR12mm]{scrbook}

\usepackage{mathe}
\usepackage{saetze-baeuerle}
\usepackage{faktor}
\usepackage{enumerate}

\newcommand{\cF}{\mathcal{F}}

\author{Die Mitarbeiter von \url{http://mitschriebwiki.nomeata.de/}}
\title{Stochasitsche Prozesse}
\makeindex

\begin{document}
\maketitle
 

\renewcommand{\thechapter}{\arabic{chapter}}
%\chapter{Inhaltsverzeichnis}
\stepcounter{chapter}
%\renewcommand{\tocname}{bla}
\addcontentsline{toc}{chapter}{\protect\numberline {\thechapter}Inhaltsverzeichnis}
\tableofcontents

 % Vorwort

\chapter{Vorwort}
%\addcontentsline{toc}{chapter}{Vorwort}

\section*{Über dieses Skriptum}
Dies ist ein Mitschrieb der Vorlesung \glqq Stochastische Prozesse\grqq\ von Prof. Dr. Bäuerle im
Sommersemester 08 an der Universität Karlsruhe (TH).
Die Mitschriebe der Vorlesung werden mit ausdrücklicher Genehmigung von Prof Dr. Bäuerle hier veröffentlicht,
Prof. Dr. Bäuerle ist für  den Inhalt nicht verantwortlich.
\section*{Wer}
Gestartet wurde das Projekt von Joachim Breitner.


\section*{Wo}
Alle Kapitel inklusive \LaTeX-Quellen können unter \url{http://mitschriebwiki.nomeata.de} abgerufen werden.
Dort ist ein von Joachim Breitner programmiertes \emph{Wiki}, basierend auf \url{http://latexki.nomeata.de} installiert. 
Das heißt, jeder kann Fehler nachbessern und sich an der Entwicklung
beteiligen. Auf Wunsch ist auch ein Zugang über \emph{Subversion} möglich.

\setcounter{chapter}{0}
% section mit §?
\renewcommand{\thechapter}{\Roman{chapter}}


\chapter{Markov-Ketten mit diskretem Zeitparameter}

\section{Elementare Eigenschaften von Markov-Ketten}

Gegeben sei eine Folge von Zufallsvariablen $(X_n)$ auf dem Wahrscheinlichkeitsraum $(\Omega, \cF, P)$ mit $X_n:\Omega -> S$ wobei $S$ nicht leer, und endlich oder abzählbar unendlich ist.

\begin{definition}
Eine $S\times S$-Matrix $P=(p_{ij})$ heißt \emph{stochastische Matrix}\index{stochastische Matrix}, falls $p_{ij}>0$ ist und für alle $i\in S$ die Zeilensumme $\sum_{j\in S} p_{ij} = 1$ ist.
\end{definition}

\begin{definition}
Sei $P$ eine stochastische Matrix. Eine (endliche oder unendliche) Folge $X_0, X_1, X_2,\ldots$ von $S$-wertigen Zufallsvariablen heißt (homogene\footnote{kurz für zeit-homogen. Die Übergangswahrscheinlichkeiten hängen nicht vom aktuellen Zeitpunkt ab.}) \emph{Markov-Kette}\index{Markov-Kette} mit Übergangsmatrix $P$, falls für alle $n\in \MdN$\footnote{Hier ist $\MdN=1,2,\ldots$} und für alle Zustände $i_k\in S$ mit 
\[
P(X_0=i_0,\ldots,X_n=i_n) >0
\]
gilt
\[
P(X_{n+1} = i_{n+1} \mid X_0 = i_0, \ldots, X_n=i_n) = 
P(X_{n+1} = i_{n+1} \mid X_n=i_n) = p_{i_n i_{n+1}}.
\]

Die $p_{ij}$ heißen Übergangswahrscheinlichkeiten und die \emph{Startverteilung} $\nu$ der Kette ist definiert durch $\nu(i)=P(X_0=i)$ für $i\in S$.
\end{definition}

\begin{bemerkung}
Jede Folge von unabhängigen Zufallsvariablen ist eine Markov-Kette.
\end{bemerkung}

\begin{satz}[Eigenschaften von Markov-Ketten]
$(X_n)$ ist genau dann eine Markovkette mit Übergangsmatrix $P$, falls gilt:
\[
P(X_k = i_k,\, 0\le k\le n) = P(X_0 = i_0)\prod_{k=0}^{n-1} p_{i_k i_{k+1}} \quad \forall n\in \MdN_0 \ \forall i_k\in S
\]
genau dann wenn gilt:
\[
P(X_k = i_k,\, 1\le k\le n \mid X_0=i_0) = \prod_{k=0}^{n-1} p_{i_k i_{k+1}} \quad \forall n\in \MdN_0 \ \forall i_k\in S\text{ mit $P(X_0=i_0)>0$}
\]
genau dann wenn gilt:
\[
P(X_k = i_k,\,m\le k\le m+n) = \prod_{k=m}^{m+n-1} p_{i_k i_{k+1}} \quad \forall m,n\in \MdN_0 \ \forall i_k\in S
\]
\end{satz}

\begin{beweis}
Zur ersten Äquivalenz. Sei $A_k \da [X_k=i_k]$, $k\in\MdN_0$.

„$\Longrightarrow$“ Induktion über $n$: $n=0$ $\checkmark$, $n\curvearrowright n+1:$ 
\begin{align*}
P(A_0A_1\ldots A_nA_{n+1}) &= P(A_0\ldots A_n)\cdot P(A_{n+1}\mid A_0\ldots A_n) \\
&= P(A_0\ldots A_n)\cdot p_{i_ni_{n+1}} && \text{(Markov-Eigenschaft)}\\
&= P(X_0=i_0)\prod_{k=0}^n p_{i_ki_{k+1}} && \text{(I.V.)}
\end{align*}

„$\Longleftarrow$“ 
\begin{align*}
P(A_{n+1}\mid A_0\ldots A_n) &= \frac{P(A_0\ldots A_nA_{n+1})}{P(A_0\ldots A_n)} \\
&= p_{i_ni_{n+1}} && \text{(Vor.)}
\end{align*}
Die weiteren Äquivalenzen sind ähnlich zu beweisen.
\end{beweis}

\paragraph{Konstruktion einer Markov-Kette.} Seien $(Y_n)$ Zufallsvariablen, unabhängig und identisch gleichverteilt (u.i.v.), in $Z$. Weiter ist $g:S\times Z\to S$ eine messbare Abbildung. Definiere die Folge $(X_n)$ mit 
\[X_0=c\in S, \quad X_n = g(X_{n-1},Y_n).\]
Die so konstruierte Folge $(X_n)$ ist eine Markov-Kette mit Werten in $S$ und Übergangsmatrix $P=(p_{ij})$ mit $p_{ij} = P(g(i,X_n),j)$.

\begin{beweis}
Die Variablen $X_0,\ldots,X_n$ hängen nur von $X_0,Y_1,\ldots Y_n$ ab, sind also unabhängig von $Y_{n+1}$.
\begin{align*}
P(X_{n+1} = i_{n+1} \mid X_k = i_k, 0\le k\le n) 
&= \frac{P(X_k = i_k,\, 0\le k\le n+1)}{P(X_k = i_k,\, 0\le k\le n)} \\
&= \frac{P(X_k = i_k,\, 0\le k\le n,\, g(i_n,Y_{n+1})=i_{n+1})}{P(X_k = i_k,\, 0\le k\le n)} \\
&= \frac{P(X_k = i_k,\, 0\le k\le n)\cdot P(g(i_n,Y_{n+1})=i_{n+1})}{P(X_k = i_k,\, 0\le k\le n)} \\
&= P(g(i_n,Y_{n+1})=i_{n+1}) \\
&= \frac{P(g(i_n,Y_{n+1})=i_{n+1})\cdot P(X_n=i_n)}{P(X_n =i_n)} \\
&= \frac{P(g(i_n,Y_{n+1})=i_{n+1}, X_n=i_n)}{P(X_n =i_n)} \\
&= P(g(i_n,Y_{n+1})=i_{n+1}\mid X_n=i_n) \\
&= P(X_{n+1} = i_{n+1} \mid X_n=i_n)
\end{align*}
\end{beweis}

\begin{bemerkung}
Umgekehrt kann zu jeder stochastischen Matrix $P$ eine Markovkette $(X_n)$ konstruiert werden mit $X_n=g(X_{n-1},Y_n)$, wobei $(Y_n)$ u.i.v. und o.B.d.A $Y_n \sim U[0,1]$.
\end{bemerkung}

\begin{beispiel}[Lagerhaltung]
Sei $Y_n$ die Nachfrage nach einem gelagerten Produkt im Zeitintervall $(n-1,n]$. $(Y_n)$ sei u.i.v. und $Y_n\in \MdN_0$. Die Auffüll-Politik sei eine $(z,Z)$-Politik mit $z\le Z$, $z,Z\in \MdN$, die wie folgt funktiert: Falls der der Lagerbestand zur Zeit $n\le z$ ist, dann fülle auf $Z$ auf, sonst tue nichts.

Sei $X_n$ der Lagerbestand zum Zeitpunkt $n$, $S=\MdN_0$. Es gilt
\[
X_n =
\begin{cases}
(Z - Y_n)^+, & X_{n-1} \le z \\
(X_{n+1} - Y_n)^+, & X_{n-1} > z
\end{cases}
\]
Also ist $(X_n)$ eine Markov-Kette mit Übergangsmatrix $P=(p_{ij})$ und
\[
p_{ij} = 
\begin{cases}
P( (Z-Y_n)^+ = j), & i\le z \\
P( (i-Y_n)^+ = j), & i > z
\end{cases}
\]
\end{beispiel}

\begin{beispiel}[Ruinspiel]
% später
\end{beispiel}


\chapter{Satz um Satz (hüpft der Has)}
\listtheorems{satz,wichtigedefinition}

\renewcommand{\indexname}{Stichwortverzeichnis}
\addtocounter{chapter}{1}
\addcontentsline{toc}{chapter}{\protect\numberline {\thechapter}Stichwortverzeichnis}
\printindex
\end{document}
