\documentclass{article}
\newcounter{chapter}
\setcounter{chapter}{7}
\usepackage{ana}
\def\gdw{\equizu}
\def\Arg{\text{Arg}}
\def\MdD{\mathbb{D}}
\def\Log{\text{Log}}
\title{Der komplexe Logarithmus}
\author{Ferdinand Szekeresch}
% Wer nennenswerte Änderungen macht, schreibt euch bei \author dazu

\begin{document}
\maketitle

\begin{definition}
Sei $w \in \MdC \backslash\{0\}$ Jedes $z \in \MdC$ mit $e^z = w$ heißt \begriff{ein Logarithmus von $w$}. Man schreibt in diesem Fall (ungenau): $z = \log w$.
\end{definition}

\begin{satz} %7.1
Sei $w \in \MdC\backslash\{0\}, w = |w|e^{i\Arg w} (\Arg w \in (-\pi , \pi])$ \\
Für $z \in \MdC$ gilt: $e^z = w \gdw \exists k \in \MdZ : z = \log |w| + i\Arg w + 2k\pi i$
($\log |w|$ ist der reelle $\Log$)
\end{satz}

\begin{beweis}
$"\Longleftarrow ": e^z = \underbrace{e^{\log|w|}}_{|w|} e^{i\Arg w}\underbrace{e^{2k\pi i}}_{1} = |w|e^{i\Arg w} = w$ \\
$"\folgt "$ Sei $z = x + iy (x,y \in \MdR)$ und $e^z = w$. Dann: $|w| = |e^z| = e^x \folgt x = \log|w|$ \\
$|w|e^{i\Arg w} = w = e^z = e^xe^{iy} = |w| e^{iy}$ \\
$\folgt e^{iy} = e^{i\Arg w} \folgt e^{i(y-\Arg w)} = 1 \stackrel{6.3}{\folgt} \exists k \in \MdZ : iy - i\Arg w = 2k\pi i$ \\
$\folgt z = \log |w| + i\Arg w + 2k\pi i$
\end{beweis}

\begin{definition}
Die Funktion $\Log : \MdC \backslash\{0\}\rightarrow \MdC$ def. durch $\Log w := \log |w| + i\Arg w$ heißt der \begriff{Hauptzweig des Logarithmus}.
\end{definition}

\begin{beispiele}
\item Alle $\Log$ von $w = 1 : 2k\pi i (k \in \MdZ)$ \\
$\Log \ 1 = 0$
\item $\Log (-1) = i\pi$
\item $w = 1+i, \; |w| = \sqrt 2, \; \Arg w = \frac{\pi}{4}$ \\
$\Log (1+i) = \log \sqrt 2 + i\frac{\pi}{4}$
\end{beispiele}

\newpage

\begin{satz} %7.2
Sei $A = \{z \in \MdC : -\pi < \Im z \leq \pi\}$ \\
$f := \exp _{|A}$
\begin{liste}
\item $f$ ist auf $A$ injektiv.
\item $f(A) = \MdC\backslash\{0\}$
\item $f^{-1}(w) = \Log $ $w (w \in \MdC\backslash\{0\})$
\item Die Funktion $\Log$ ist unstetig in jedem $w \in \MdR$ und $w < 0$
\end{liste}
\end{satz}

\begin{beweis}
\begin{liste}
\item 6.3, 7.1
\item 6.3, 7.1
\item 6.3, 7.1
\item §3 Beispiel: $w \rightarrow \Arg$ $ w$ ist in $w < 0$ unstetig.
\end{liste}
\end{beweis}


%Ab hier Bernhard Konrad und Matej Belica, Mi 17.05.2006

\begin{definition}
$\MdC\_ := \MdC \backslash \{ t \in \MdR: t \leq 0 \} \; (\subseteq \MdC \backslash \{0\})$\\
\\
F"ur $w \in \MdC\_$ ist Arg$w \in (-\pi,\pi)$.
\end{definition}

%Satz 7.3
\begin{satz}
$\Log \in C(\MdC\_)$
\end{satz}

\begin{beweis}
Sei $w_0 \in \MdC\_ ,$
$z_0 := \Log $ $w_0, \, x_0 := \Re z_0, \, y_0:=\Im z_0;$ also: $x_0 = \log |w_0|,$
$y_0 = \Arg w_0 \in (-\pi,\pi). \; R:= \{ z= x+iy: x,y \in \MdR, |x-x_0| \leq \log 2, |y| \leq  \pi \}.$\\
Sei $\varepsilon > 0$ so klein, dass $K:= R \cap \left( \MdC \backslash U_{\varepsilon}(z_0)\right) \not= \emptyset.\; $ Klar: $K$ ist kompakt, $z_0 \notin K.$\\
Definiere $\varphi: K \rightarrow \MdR$ durch $\varphi(z):= |e^z - w_0| = |e^z - e^{z_0}|$.\\
Dann: $\varphi \in C(K)$. 3.3 $\Rightarrow \exists \varrho := \min \varphi(K).$ \\
Annahme: $\varrho = 0$. Also existiert ein $z \in K: e^z = e^{z_0} \Rightarrow e^{z-z_0}=1.$ \;6.3 $\Rightarrow \exists j \in \MdZ: z-z_0 = 2j\pi i \Rightarrow 2j\pi = \Im(z-z_0) = \Im z - \Im z_0 \Rightarrow 2|j|\pi = |\Im z - \Im z_0| \leq \underbrace{|\Im z|}_{\leq \pi} + \underbrace{|\Im z_0|}_{< \pi} < 2\pi \Rightarrow j = 0 \Rightarrow z_0 = z \in K.$ Wid!\\
Also: $\varrho > 0$\\
$\delta := \min\{ \varrho, \frac1{2}e^{x_0}\}.$ Sei $w \in \MdC\_$ und $|w-w_0| < \delta; \; z:=\log w.$ \; Z.z: $|z-z_0|<\varepsilon.$ \\
Sei $z=x+iy \, (x,y \in \MdR); \, y = \Arg w \in (-\pi,\pi)$, also: $|y| \leq \pi$. \\
Annahme: $x>x_0 + \log 2 \;$. Dann:\\
\[
\frac{1}{2} e^{x_0} \geq \delta > |w-w_0| = |e^z - e^{z_0}| \geq | |e^z| - |e^{z_0}| | = |e^x - e^{x_0}| \geq e^x - e^{x_0} > e^{x_0 + \log 2} - e^{x_0} = e^{x_0} \quad \mbox{Wid!}
\]
Also: $x \leq x_0 + \log 2.$ Analog: $x \geq x_0 - \log 2.$ \\
Fazit: $z\in \MdR.$\\
Annahme: $|z-z_0| \geq \varepsilon \, \Rightarrow z\in K \Rightarrow \delta \leq \varrho \leq \varphi(z) = |e^z - e^{z_0}| = $\\
$|w-w_0| < \delta. \quad \mbox{Wid!}$
\end{beweis}

%Satz 7.4
\begin{satz}
$\Log \in H(\MdC\_)$ und $\Log'w= \frac1w \; \forall w \in \MdC\_$
\end{satz}

\begin{beweis}
Sei $w_0 \in \MdC\_$; $(w_n)$ eine Folge in $\MdC\_$ mit: $w_n \not= w_0 \ \forall n \in \MdN$ und $w_n \rightarrow w_0, \, z_0 := \log w_0, \, z_n:=\log w_n. \; 7.3 \Rightarrow z_n \rightarrow z_0.$ Dann:\\
\[
\frac{\Log w_n - \Log w_0}{w_n - w_0} = \frac{z_n-z_0}{e^{z_n}-e^{z_0}} = \left(\frac{e^{z_n}-e^{z_0}}{z_n -z_0}\right)^{-1} \rightarrow \frac1{e^{z_0}} = \frac1{w_0}
\]
D.h. $\Log$ ist in $w_0$ komplex differenzierbar und $\log'w_0 = \frac1{w_0}$
\end{beweis}

\textbf{Bezeichnung:} $\; \MdD := \{ z \in \MdC : |z| < 1 \} = U_1(0)$ \\
Beachte: F"ur $z \in \MdD$ ist $1-z \in \MdC\_$

%Satz 7.5
\begin{satz}
F"ur alle $ z \in \MdD$ gilt:
\[
\log (1+z) = \sum_{n=1}^{\infty} (-1)^{n+1} \frac{z^n}{n} 
\]
\end{satz}

\begin{beweis}
7.4, 5.4 $\Rightarrow f(z) := \log(1+z) - \sum_{n=1}^{\infty} (-1)^{n+1} \frac{z^n}{n}$ ist auf $\MdD$ holomorph und \\
$f'(z) = \frac1{1+z} - \sum_{n=1}^{\infty} (-1)^{n+1} z^{n-1} = \frac1{1+z} - \sum_{n=1}^{\infty} (-z)^{n-1} = \frac1{1+z} - \frac1{1-(-z)} = 0 \ \forall z \in \MdD$\\
$\MdD$ ist ein Gebiet $\stackrel{4.2}{\Rightarrow} f$ ist auf $\MdD$ konstant. $f(0) = 0 \Rightarrow$ Beh.
\end{beweis}

\begin{definition}
Sei $w \in \MdC \backslash \{0\}$ und $a \in \MdC. $\\
$w^a := e^{a \Log w} \quad $ (\begriff{Hauptzweig der allgemeinen Potenz})
\end{definition}

\begin{beispiele}
\item F"ur $a= k \in \MdZ$ ist obige Definition die fr"uhere Potenz von $w$. Denn: $k \in \MdN:$ \\
$e^{k \Log w} = e^{\Log w + \Log w + \dots + \Log w} = \left(e^{\Log w}\right)^k = w^k$\\
$e^{-k \Log w} = \frac1{e^{k \log w}} \stackrel{s.o.}{=} \frac1{w^k} = w^{-k}$
\item $w=a=i, \, \log|w| = 0, \, \Arg w = \frac{\pi}2, \, \Log w = i \frac{\pi}{2} \, \Rightarrow i^i = e^{i\cdot i \frac{\pi}{2}} = e^{-\frac{\pi}{2}} \in \MdR$
\end{beispiele}

%Satz 7.6
\begin{satz}
Sei $a \in \MdC$ und $f: \MdC\_ \rightarrow \MdC$ definiert durch $f(w) := w^a$. Dann: \\
$f \in H(\MdC\_)$ und $f'(w) = aw^{a-1} \ \forall w \in \MdC\_$
\end{satz}

\begin{beweis}
$7.4, 4.4 \Rightarrow f \in H(\MdC\_)$ und $f'(w) = e^{a \Log w}(a \Log w)' = ae^{a \Log w}\frac1w \stackrel{Bsp(1)}{=} ae^{a \Log w}e^{- \Log w} = ae^{(a-1) \Log w} = aw^{a-1}$
\end{beweis}

\end{document}
