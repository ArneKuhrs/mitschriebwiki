\documentclass{article}
\usepackage[utf8]{inputenc}
\usepackage{amsmath}
\usepackage{amsfonts}
\usepackage{amssymb}
\usepackage{amsthm}
\usepackage{mathrsfs}
\usepackage{german}
\usepackage{enumerate}
\usepackage{stmaryrd}
\title{1. Algebra Übung}
\author{Ferdinand Szekeresch}
\begin{document}
\maketitle
\textbf{Aufgabe 1}\\
$(M,\cdot )$ Magma, $\mathscr{P}(M)$ Potenzmenge.\\
Für $A,B\subseteq M: A*B := \{a,b : a\in A, b\in B\}$, d.h. $x\in A*B \Leftrightarrow \exists a\in A,b\in B: x = ab$.\\
dabei auch: $x\in A*\emptyset \Leftrightarrow\exists a\in A,\underbrace{b\in\emptyset}_\lightning : x=ab\\
\Rightarrow x\in A*\emptyset$ existiert nicht. $\Rightarrow A*\emptyset = \emptyset$
\begin{enumerate}[a)]
\item Sei $M$ kommutativ. Für alle $A,B\subseteq M : A*B = \{a\cdot b : a\in A, b\in B\} = \{b\cdot a: b\in B, a\in A\} = B*A \\
\Rightarrow \mathscr{P}(M)$ ist kommutativ.
\item Sei $M$ eine Halbgruppe, d.h. assoziativ. Für alle $A,B, C \subseteq M: \\
(A*B)*C = \{x\cdot c : x\in A*B, c\in C\} = \{(a\cdot b)\cdot c : a\in A, b\in B, c\in C\} = \{a\cdot(b\cdot c) : a\in A, b\in B, c\in C\}\\
=\{a\cdot y : a\in A, y\in B*C\} = A*(B*C)\\
\Rightarrow \mathscr{P}(M)$ ist assoziativ (Halbgruppe).\\
\item Sei $M$ ein Monoid mit neutralem Element $1$.\\
Zeige: $\{1\}$ ist neutrales Element von $\mathscr{P}(M)$.\\
Für alle $A\subseteq M : A*\{1\} = \{a\cdot 1 : a\in A\} = A = \{1\cdot a : a\in A\} = \{1\}*A\\
\Rightarrow \mathscr{P}(M)$ ist Monoid mit neutralem Element $\{1\}$ (Assoziativität nach b)).\\
Welche $A\subseteq M$ sind invertierbar?\\
\begin{enumerate}[1.{Fall}]
\item $A=\emptyset \\
\Rightarrow \forall B\subseteq M : \emptyset*B = B*\emptyset = \emptyset\neq \{1\}\\
\Rightarrow\emptyset$ ist nicht invertierbar.
\item $A=\{a_0\}$ einelementig\\
Behauptung: $\{a_0\} \in \mathscr{P}(M)^\times \Leftrightarrow a_0 \in M^\times$
\begin{itemize}
\item[$\glqq\Rightarrow\grqq$ ]$\{a_0\}$ invertierbar $\Rightarrow\exists B\subseteq M, B\neq\emptyset : \{a_0\}*B = B*\{a_0\} = \{1\}\\
b\in B \Rightarrow a_0b = 1 = ba_0\Rightarrow a_0$ invertierbar in $M$ und $a_0^{-1} = b \Rightarrow B = \{a_0\}^{-1} = \{a_0^{-1}\}$
\item[$\glqq\Leftarrow\grqq$ ] $a_0 \in M^\times\\
\Rightarrow\{a_0\}*\{a_0^{-1}\} = \{a_0a_0^{-1}\} = \{1\} = \{a_0^{-1}\}*\{a_0\}\\
\Rightarrow \{a_0\} \in\mathscr{P}(M)^\times$ und $\{a_0\}^{-1} = \{a_0^{-1}\}$.
\end{itemize}
\item $|A| \geq 2; a_1,a_2\in A; a_1\neq a_2$.\\
Annahme: $\exists B\subseteq M, B\neq\emptyset : A*B = B*A = \{1\}$\\
Sei $b\in B \Rightarrow a_1b = 1 = ba_1 \Rightarrow a_1 = b^{-1}\\
a_2b = 1 = ba_2 \Rightarrow a_2 = b^{-1} \Rightarrow a_1 = a_2\quad\lightning$\\
\end{enumerate}
Fazit: $A\subseteq M$ ist invertierber $\Leftrightarrow A = \{a_1\}$ für ein $a_1\in M^\times$. In diesem Fall ist $A^{-1} = \{a_1^{-1}\}$.
\item Sei $M$ eine Gruppe.\\
nach c): $\mathscr{P}(M)$ ist ein Monoid mit neutralem Element $\{1\}$.\\
$\mathscr{P}(M)^\times = \{\{a\}:a \in M\}$\\
$\Rightarrow \emptyset$ ist nicht invertierbar $\Rightarrow \mathscr{P}(M)$ ist keine Gruppe.
\end{enumerate}

\textbf{Aufgabe 2}\\
$(M,\cdot )$ Monoid mit neutralem Element 1. Voraussetzung: $\forall x\in M: x^2 = 1$.\\
\begin{itemize}
\item Zeige $M$ ist Gruppe, also jedes $x\in M$ ist invertierbar. Wähle $y=x$.
\item $M$ ist abelsch. Seien $x,y\in M \Rightarrow (xy)^2 = 1 \Rightarrow xyxy = 1 \Rightarrow xyxyyx = yx \Rightarrow xyx^2 = yx \\
\Rightarrow xy = yx$.\\
\end{itemize}

\textbf{Aufgabe 3}\\
\begin{enumerate}[a)]
\item $f=X^3 + aX^2 + bX + c, \\
Y:=X + \frac a3$ bzw. $X = Y - \frac a3 \\
\Rightarrow f = (Y-\frac a3)^3 + a(Y-\frac a3)^2 + b(Y-\frac a3) + c\\
=(Y^3 - aY^2 + \frac{a^2}3Y - \frac{a^3}{27}) + a(Y^2 - \frac{2a}3Y + \frac{a^2}9) + b(Y-\frac a3) + c = Y^3 + \underbrace{(b-\frac{a^3}3)}_{=:p} Y + \underbrace{(\frac{2a^3}{27} - \frac{ab}{3} + c)}_{=:q}$
\item $g = Y^3 + pY + q, D:= - 4p^3 - 27q^2, D<0\\
\zeta := e^{\frac{2\pi i}3} = \cos(\frac{2\pi}3) + i\sin(\frac{2\pi}3) = -\frac12 + i\frac{\sqrt3}2$ dritte Einheitswurzel \\
$\zeta^2 = e^{\frac{4\pi i}3} = \cos(\frac{4\pi}3) + i\sin(\frac{4\pi}3) = -\frac12 - i\frac{\sqrt3}2 \\
\zeta^3 = 1\\
1+\zeta+\zeta^2 = 1 + (-\frac12 + i\frac{\sqrt3}2) + (-\frac12 - i\frac{\sqrt3}2) = 0$.\\
damit:\\
$\big(Y-(u+v)\big)\big(Y-(\zeta u + \zeta^2v)\big)\big(Y-(\zeta^2u + \zeta v)\big) = \\
(Y-u-v)(Y-\zeta u-\zeta^2v)(Y-\zeta^2u-\zeta v) = \ldots = Y^3 + \big(-u\underbrace{(1+\zeta+\zeta^2)}_{=0}-v\underbrace{(1+\zeta+\zeta^2)}_{=0}\big)Y^2 \\
+\big(-u^2\underbrace{(1+\zeta+\zeta^2)}_{=0}+v^2\underbrace{(1+\zeta+\zeta^2)}_{=0}+3uv\underbrace{(\zeta+\zeta^2)}_{=-1}\big)Y+\big(-u^3-v^3-uv^2\underbrace{(1+\zeta+\zeta^2)}_{=0}-u^2v\underbrace{(1+\zeta+\zeta^2)}_{=0}\big) \\
=Y^3 - 3uvY - (u^3+v^3)$\\
\begin{itemize}
\item $-3uv = -3\big(\frac12(-q + \frac19\sqrt{-3D})\cdot(-q-\frac19\sqrt{-3D})\big)^{\frac13}=-3\Big(\frac14\big(q^2-\frac1{81}(-3D)\big)\Big)^{\frac13}=\\
-3\Big(\frac14\big(q^2+\frac1{27}(-4p^3-27q^2)\big)\Big)^\frac13 = -3(-\frac1{27}p^3)^\frac13 = p$
\item $-u^3-v^3 = -\frac12(-q+\frac19\sqrt{-3D})-\frac12(-q-\frac19\sqrt{-3D}) = q$
\end{itemize}
$\Rightarrow\big(Y-(u+v)\big)\big(Y-(\zeta u + \zeta^2v)\big)\big(Y-(\zeta^2u + \zeta v)\big) = Y^3 - 3uvY - (u^3+v^3) = Y^3 + pY + q \\
\blacksquare$
\item[Beispiel: ]$f = X^3+X^2-X-1,\quad Y=X+\frac13$\\
damit: $f=Y^3\underbrace{-\frac43}_{=:p}Y\underbrace{-\frac{16}{27}}_{=:q}, D=0 \Rightarrow u=v=\frac23\\
\Rightarrow f=(Y-\frac43)\big(Y-\frac23(\zeta+\zeta^2)\big)^2 = (Y-\frac43)(Y+\frac23)^2\\
\stackrel{Y=X+\frac13}{=}(X-1)(X+1)^2$. 
\end{enumerate}
\end{document}
