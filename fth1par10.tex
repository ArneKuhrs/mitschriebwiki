\documentclass{article}

\newcounter{chapter}

\setcounter{chapter}{10}

\usepackage{ana}

\def\gdw{\equizu}

\def\Arg{\text{Arg}}

\def\MdD{\mathbb{D}}

\def\Log{\text{Log}}

\def\Tr{\text{Tr}}

\def\abnC{\ensuremath{[a,b]\to\MdC}}

\def\wegint{\ensuremath{\int\limits_\gamma}}

%iso-int

\def\iint{\ensuremath{\int\limits}}

\title{Folgerungen aus den Integralformeln}
\author{Ferdinand Szekeresch und Dennis Prill}
% Wer nennenswerte �nderungen macht, schreibt euch bei \author dazu

\begin{document}
\maketitle

%Satz 10.1
\begin{satz}[Cauchysche Absch�tzungen]
Sei $z_0 \in \MdC, r>0, f \in H(U_r(z_0))$ und $f$ sei auf $U_r(z_0)$ beschr�nkt mit $M := \sup\limits _{U_r(z_0)} |f(z)|$.\\
Dann: $|f^{(n)}(z_0)| \leq \frac{Mn!}{r^n} \,\forall n \in \MdN _0$.
\end{satz}

\begin{beweis}
Sei $0<\rho < r, \gamma (t) := z_0 + \rho e^{it} (t \in [0,2\pi ]).\; 9.6 \folgt f^{(n)}(z_0) = \frac{n!}{2\pi i} \wegint \frac{f(w)}{(w - z_0)^{n+1}}dw$.\\
F�r $w \in \Tr (\gamma ) : |w - z_0| = \rho$, also $\frac{|f(w)|}{|w - z_0|^{n+1}} \leq \frac{M}{\rho ^ {n+1}}$ \\
$\stackrel{8.4}{\folgt} |f^{(n)}(z_0)| \leq \frac{n!}{2\pi} \cdot \frac{M}{\rho^{n+1}}2\pi\rho = \frac{Mn!}{\rho ^n}\stackrel{\rho \to r}{\folgt}$ Beh.
\end{beweis}

%Satz 10.2
\begin{satz}[Satz von Liouville]
Ist $f \in H(\MdC )$ auf $\MdC$ beschr�nkt, so ist $f$ konstant.
\end{satz}

\begin{beweis}
Sei $z_0 \in \MdC$ und $r > 0$. 10.1 $\folgt |f'(z_0)| \leq \frac{M}{r}; r>0$ beliebig. \\
$\stackrel{r \to \infty}{\folgt} f'(z_0) = 0, z_0 \in \MdC$ beliebig $\folgt f' = 0$ auf $\MdC. 4.2 \folgt$ Beh.
\end{beweis}

\begin{bemerkung}
10.2 ist in $\MdR$ falsch. Z.B. ist $x \to \cos x$ auf $\MdR$ beschr�nkt aber nicht konstant. F�r $t \in \MdR: \cos (it) = \frac{1}{2}(e^{i(it)} + e^{i(-it)}) = \frac{1}{2}(e^t + e^{-t})$\\ 
$= \cosh t \to \infty (t \to \pm\infty )$
\end{bemerkung}

\textbf{Hilfssatz}\\
Sei $n \in \MdN. a_0,\ldots ,a_n \in \MdC, a_n \neq 0$ und $p(z):=a_0 + a_1z + \ldots + a_nz^n$. \\
Dann ex. ein $R>0: |p(z)| \geq 1 \forall z \in \MdC$ mit $|z| > R$.

\begin{beweis}
F�r $z \neq 0: \varphi (z) := \frac{|a_0|}{|z^n|} + \frac{|a_1|}{|z^{n-1}|} + \ldots + \frac{|a_{n-1}|}{|z|} + |a_n|$.\\
$\folgt \varphi (z) \to \underbrace{|a_n|}_{\neq 0} (|z| \to \infty) \folgt |p(z)| = |z|^n |\varphi(z)| \to \infty (|z| \to \infty) \folgt$ Beh.
\end{beweis}

%Satz 10.3
\begin{satz}[Fundamentalsatz der Algebra]
Sei $p$ wie in obigem Hilfssatz. Dann ex. ein $z_0 \in \MdC: p(z_0) = 0$
\end{satz}

\begin{beweis}
Hilfssatz $\folgt \exists R>0 : |p(z)|\geq 1 \forall z \in \MdC$ mit $|z|>R$.\\
Annahme: $p(z) \neq 0 \forall z \in \MdC$. Dann $q:=\frac{1}{p} \in H(\MdC)$ und $|q(z)|\leq 1$ f�r $z\in \MdC$ mit $|z| > R$. \\
$q$ ist stetig $\folgt q$ ist beschr�nkt auf $\overline{U_R(0)} \folgt q$ ist auf $\MdC$ beschr�nkt. \\
$10.2 \folgt q$ ist konstant $\folgt p$ ist konstant, Wid!
\end{beweis}

%Satz 10.4
\begin{satz}[Potenzreihenentwicklung]
Sei $D \subseteq \MdC$ offen, $f \in H(D), z_0 \in D$ und $r > 0$ so, dass $U_r(z_0) \subseteq D$. Dann:
$$(*)f(z) = \sum _{n=0}^{\infty}a_n(z-z_0)^n \quad \forall z \in U_r(z_0)$$
wobei
$$(**) a_n = \frac{f^{(n)}(z_0)}{n!} = \frac{1}{2\pi i}\wegint\frac{f(w)}{(w-z_0)^{n+1}}dw$$
mit $\gamma (t) = z_0 + \rho e^{it}, t \in [0,2\pi], 0<\rho <r$
\end{satz}

\begin{beweis}
(**) folgt aus (*), 5.4 und 9.6. O.B.d.A.:$\;z_0 = 0$. \\
Sei $z \in U_r(0)$ und sei $R>0$ so, dass $|z|<R<r$;\\
$\gamma _0(t) := z_0 + R\cdot e^{it} \;(t \in [0,2\pi])$. \\
Sei $w \in \Tr (\gamma _0)$. Dann $\frac{|z|}{|w|} = \frac{|z|}{R} < 1$, also $\frac{f(w)}{w-z} = \frac{f(w)}{w}\cdot \frac{1}{1-\frac{z}{w}} = \frac{f(w)}{w}\sum\limits _{n=0}^{\infty}\frac{f(w)}{w^{n+1}}z^n$.\\
Also $\underbrace{\int\limits _{\gamma _0} \frac{f(w)}{w-z} dw}_{\stackrel{9.4}{=}2\pi if(z)} = \int\limits _{\gamma _0}(\sum\limits _{n=0}^{\infty}\frac{f(w)}{w^{n+1}}z^n)dw $\\
$\stackrel{8.4}{=}\sum\limits _{n=0}^{\infty}(\underbrace{\int\limits _{\gamma _0}\frac{f(w)}{w^{n+1}}dw}_{\stackrel{9.6}{=}2\pi i\cdot\frac{f^{(n)}(0)}{n!}})z^n \folgt f(z) = \sum\limits _{n=0}^{\infty}\frac{f^{(n)}(0)}{n!})z^n$
\end{beweis} 

\begin{bemerkungen}
\item 10.4 ist in $\MdR$ falsch. \\Bekannt aus der Analysis: Die Funktion
$$f(x) := \begin{cases}e^{-1/x^2} &, x\in\MdR\backslash\{0\} \\ 0 &, x=0\end{cases}$$ ist auf $\MdR$ bel. oft db und $f^{(n)}(0) = 0 \,\forall n \in \MdN _0.$ \\
Also: $\sum\limits _{n=0}^{\infty}\frac{f^{(n)}(0)}{n!}x^n \equiv 0$ auf $\MdR$.
\item Die Entwicklung (*) gilt in der gr��ten offenen Kreisscheibe um $z_0$, die noch ganz in $D$ liegt. Sei $r_0$ der Radius dieser Kreisscheibe (ist $D = \MdC$, so ist $r = \infty)$. Sei R der KR der PR in (*). Also : $R \geq r_0$.
\end{bemerkungen}

%\begin{beispiel}
%Fehlt noch, kommt aber
%\end{beispiel}
\begin{satz}[Konvergenzsatz von Weierstra�]
$D \subseteq \MdC$ sei offen, $(f_n)$ sei eine Folge in $H(D)$ und $(f_n)$ konvergiere auf $D$ \begriff{lokal gleichm��ig} gegen eine Funktion $f: D \rightarrow \MdC$.
\begin{liste}
\item $f \in H(D)$
\item $(f_n')$ konvergiere auf $D$ lokal gleichm��ig gegen $f'$.
\end{liste}
\end{satz}

\begin{beweis}
\begin{liste}
\item 5.1 $\Rightarrow f \in C(D)$. Sei $\Delta \subseteq D$ ein Dreieck. $(f_n)$ konvergiere auf $\partial\Delta$ \\
gleichm��ig $\Rightarrow$ $\int_{\partial\Delta} f(z) dz \stackrel{8.4}{=} \lim_{n\to\infty}\underbrace{\int_{\partial\Delta}f_n(z) dz}_{\stackrel{9.1}{=}}= 0$.\\
9.7 $\Rightarrow f \in H(D)$.
\item O.B.d.A. $f = 0$ auf $D$ (ansonsten betrachte $f_n - f$). Sei $z_0 \in D$ und $r > 0$ \\
so, da� $\overline{U_r(z_0)} \subseteq D$. Es gen�gt zu zeigen: \\
$(f_n')$ konvergiert auf $\overline{U_{\frac{r}{2}}(z_0)}$ gleichm��ig gegen $0$.\\
$\gamma (t) := z_0 + r \cdot e^{it} (t \in [0,2\pi])$. $M_n := \max\limits _{w \in \Tr(\gamma)} |f_n(w)|$\\
Vor $\Rightarrow M_n \to 0$.\\
Sei $z \in \overline{U_{\frac{r}{2}}(z_0)}. f_n (z) \stackrel{9.6}{=} \frac{1}{2 \pi i} \int_{\gamma} \frac{f_n(w)}{(w-z)^2}dw$\\
$w \in \Tr(\gamma) : |w-z| \geq \frac{r}{2} \Rightarrow \frac{|f_n(w)|}{|w-z|^2} \leq \frac{4 M_n}{r^2}$\\
$\Rightarrow |f_n ' (z)| \leq \frac{1}{2\pi} \frac{4 M_n}{r^2} 2 \pi r = \frac{4 M_n}{r}$\\
Also: $|f_n'(z)| \leq \frac{4 M_n}{r} \forall z \in \overline{U_{\frac{r}{2}}(z_0)} \forall n \in \MdN$ und $M_n \to 0$.
\end{liste}
\end{beweis}

\end{document}
